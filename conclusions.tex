\chapter*{Conclusions}
\addcontentsline{toc}{chapter}{Conclusions} 

The contents presented in this thesis come from different areas that are seemingly unrelated but connected through the application to random matrix theory. The thesis has no pretensions of arriving at new results through the connections of these topics, but a deeper work in this direction can lead to mutual enrichment. The study of convolution from the stochastic processes point of view could allow us to get new results. Also, the use of algebraic and combinatorics tools of Free Probability Theory is still not very well explored and could be a starter for new ways to explore these objects.

Two of the most important results in this work are Theorem \ref{thm:dyson_is_hermite} in conjunction with Proposition \ref{prop:convolution_of_hermites} and Theorem \ref{thm:wishart_is_laguerre} together with \ref{prop:convolution_of_laguerres}. Each pair of these results tells us that, for the specific case of Hermite and Laguerre polynomials, we can relate the symmetric (correspondingly asymmetric) additive convolution to the evolution of an Itô process over time. This fact has the advantage that allows us to use the several tools of stochastic calculus applied to a different topic. As an example, the proof of Proposition \ref{prop:convolution_of_hermites}  could be made much simpler using the Lévy process properties of the Brownian motion. A slightly less trivial example is that, by applying \ref{thm:hermite_minimal_grows} with the former theorem and proposition relating the convolution to the evolution of the process, we can conclude the original result of \cite{article:roots_grow_polya} that the minimal distance of the roots in a polynomial always grows when doing symmetric additive convolution with another polynomial. In the same line, an implication of Theorem \ref{thm:laguerre_does_not_grow} is that the asymmetric additive convolution does not satisfy the same property, i.e. we have found an example of an asymmetric additive convolution of polynomials in which the mesh (minimal distance between roots) does not grow.

Given the existence of several algorithms in Machine Learning and multivariate statistics that are modeled by the transformation of a random matrix \cite{pmlr-v70-pennington17a,srivastava2002methods}, it is natural to expect that stochastic calculus could be applied to deduce results about the behavior of the system over time. This approach has already been proved useful in the work of Bru to study the behavior of Principal Analysis Components under a random perturbation \cite{bru1989diffusions}.

On the other hand, the proofs of some of the results about the matrix-valued processes are greatly simplified with the tools of Finite Free Probability Theory. As an example, the relationship of Jacobi polynomials to the Jacobi matrix distribution is much shorter using the properties of convolution than the one found in \cite{article:aomoto1987jacobi_selberg_integrals}. Similar ideas could be applied to get analogous results for other matrix-valued stochastic processes. 

There are several problems related to the relationship between the topics in the thesis that have not been covered or even mentioned. In the first place, most of the work was focused on symmetric additive convolution. While this is arguably the notion of convolution that can be the most easily related to stochastic processes, similar results could be obtained for the other two notions and they could be applied to processes with multiplicative increments, as an example. The exploration of more matrix-valued stochastic models and how they can be studied from stochastic analysis and finite free probability points of view is still a pending topic. 

Finally, some questions arose during the realization of this work that were left behind because they surpassed the goals and time limitations of the thesis. As an example, it would be of interest what is the behavior of the stochastic expected polynomial of a matrix-valued diffusion process. This more general approach would deal not only with expected values but also with more general properties of the eigenvalues and maybe it could be related to Finite Free Probability Theory using convolution of random polynomials.  A natural question concerning the last section would be to find an explicit representation for the Jacobi polynomials considering different starting positions, prove that their roots converge to the stable distribution, and evaluate the velocity of convergence.

In sum, this master's thesis serves as an exploration of how two different branches of probability can be related through the study of a common object of study. While some of the results presented here, can be interesting, there are a lot of questions and opportunities to explore deeper connections between these areas.