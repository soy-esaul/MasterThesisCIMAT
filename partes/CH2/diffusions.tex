\section{Generalization for matrix-valued diffusion processes} \label{sec:matrix_difusions}

\begin{theorem} \label{thm:diffusion_real}
    Let $B = (B(t), t\ge 0)$ be a Brownian motion in $\M_{p,p}(\R)$ and $X(t)$ be a symmetric $p\times p$ matrix-valued stochastic process satisfying the stochastic differential equation

    \begin{equation}
        \d X(t) = g(X(t)) \d B(t) h(X(t)) + h(X(t)) \d \trans{B(t)} g(X(t)) + b(X(t))\d t, \label{eq:matrix_diffusion}
    \end{equation}

    where $g,h,b$ are real functions acting spectrally, and $X(0)$ is a symmetric $p\times p$ matrix with $p$ different eigenvalues. 

    Let $G(x,y) = g^2(x)h^2(y) + g^2(y)h^2(x)$, and
    
    \begin{equation}
        \tau = \inf\{ t: \lambda_i(t) = \lambda_j(t) \text{ for some } i\neq j \}. \label{eq:collision_time}
    \end{equation} 
    
    Then, for $t < \tau$ the eigenvalue process $\Lambda(t)$ verifies the following stochastic differential equations:

    \begin{equation}
        \d \lambda_i = 2g(\lambda_i)h(\lambda_i)\d W_i + \biggl( b(\lambda_i) + \sum_{k\neq i} \frac{G(\lambda_i,\lambda_k)}{\lambda_i - \lambda_k} \biggr)\d t, \label{eq:gen_dyson}
    \end{equation}

    \noindent where $(W_i)_{i}$ are independent Brownian motions.
\end{theorem}

\begin{proof}
    Recall that for every $t$, the process $X(t)$ admits a decomposition of the form 

    \[ X(t) = H \Lambda H^T, \]

    \noindent where both $\Lambda$ and $H$ are matrix-valued stochastic processes, $\Lambda = \mathrm{diag}(\lambda_1, \dots, \lambda_p)$ is the diagonal matrix of ordered eigenvalues of $X(t)$ and $H$ is the corresponding matrix of eigenvectors.

    Let us define the stochastic logarith of $H$ as

    \begin{equation*}
        \d A \coloneqq H^{-1} \partial H = H^T \partial H = H^T \d H + \frac12(\d H^T)\d H.
    \end{equation*}

    By using Itô's formula on $I = H^T H$ we find

    \begin{align*}
        0 = \d I = \d(H^T H) = H^T\d H + (\d H)^T H + (\d H)^T \d H = H^T\partial H + (\partial H)^T H = A + A^T.
    \end{align*}

    Which means $A$ is skew symmetric. Using that $H^T H = I$, we have $\Lambda = H^T H \Lambda H^T H = H^T X H$, by the matrix Itô formula, we find \todo{Revisar bien esta cuenta}


    \begin{align*} 
        \d \Lambda & = \d(H^TXH) = (\partial H^TX)H + H^TX\partial H,\\ 
        & = (\partial H)^T XH + H^T(\partial X)H + H^T X \partial H,\\
        & = (\partial H)^T H\Lambda + H^T(\partial X) H + \Lambda H^T \partial H,\\
        & = (\partial A)^T \Lambda + H^T(\partial X) H + \Lambda \partial A,\\
        & = H^T(\partial X) H -  (\partial A)\Lambda + \Lambda \partial A.
    \end{align*}
    The entries in the diagonals of $(\partial A)\Lambda$ and $\Lambda\partial A$ coincide, and thus the diagonal of $\Lambda\partial A-(\partial A)\Lambda$ is zero. Let us denote $\d N = H^T(\partial X) H$, then

    \[ \d \lambda_j = \d N_{jj}, \]

    \noindent and, using that $\Lambda$ is a diagonal matrix, if $i\neq j$,

    \[ 0 = \d N_{i,j} + (\lambda_i - \lambda_j)\d A_{ij}. \]

    This leads to the following representation for $A_{ij}$,

    \begin{equation} \label{eq:eigenAN}
        \d A_{i,j} = \frac{\d N_{i,j}}{\lambda_j - \lambda_i}, \qquad i \neq j.
    \end{equation}

    From \eqref{eq:matrix_diffusion} we compute the quadratic covariation $\d X_{ij}\d X_{km}$,

    \begin{align*}
        \d X_{ij}\d X_{km} &= \d \bigl< (g(X(t))\d B(t)h(X(t)))_{ij} + (h(X_t)\d \trans{B}(t)(g(X(t))))_{ij},\\
        &(g(X(t))\d B(t)h(X(t)))_{km} + (h(X_t)\d \trans{B}(t)(g(X(t))))_{km} \bigr>,\\
        &= \d \bigl< (g(X(t))\d B(t)h(X(t)))_{ij}, (g(X(t))\d B(t)h(X(t)))_{km} \bigr> \\
        & + \d \bigl< (g(X(t))\d B(t)h(X(t)))_{ij}, (h(X_t)\d \trans{B}(t)(g(X(t))))_{km} \bigr> \\
        & + \d \bigl< (h(X_t)\d \trans{B}(t)(g(X(t))))_{ij} , (h(X_t)\d \trans{B}(t)(g(X(t))))_{km} \bigr> \\
        & + \d \bigl< (h(X_t)\d \trans{B}(t)(g(X(t))))_{ij}, (g(X(t))\d B(t)h(X(t)))_{km} \bigr>
    \end{align*}

        Let us first find $\d \bigl< (g(X(t))\d B(t)h(X(t)))_{ij}, (g(X(t))\d B(t)h(X(t)))_{km} \bigr>$, the other summands are analogous,

    \begin{align*}
        \d \bigl< (g(X(t))&\d B(t)h(X(t)))_{ij}, (g(X(t))\d B(t)h(X(t)))_{km} \bigr>, \\
        &=\d \biggl< \sum_{p,q} g(X(t))_{ip}\d B(t)_{pq}h(X(t))_{qj}, \sum_{r,s} g(X(t))_{kr}\d B(t)_{rs}h(X(t))_{sm} \biggr> 
        \intertext{using the independence between the entries in the brownian matrix,}
        &= \sum_{p,q} \d\bigl< g(X(t))_{ip}\d B(t)_{pq}h(X(t))_{qj} , g(X(t))_{kp}\d B(t)_{pq}h(X(t))_{qm} \bigr> \\
        &= \sum_{pq} g(X(t))_{ip}h(X(t))_{qj}, g(X(t))_{kp}h(X(t))_{qm}\d t,\\
        &= \biggl( \sum_p g(X(t))_{ip}g(X(t))_{kp}\biggr)\biggl(\sum_q h(X(t))_{qj}h(X(t))_{qm}\biggr) \d t, \\
        &= \bigl(g(X(t))\trans{g(X(t))}\bigr)_{ik}\bigl( \trans{h(X(t))}h(X(t))\bigr)_{jm}\d t, \\
        &= \left( Hg(\Lambda)\trans H Hg(\Lambda)\trans H \right)_{ik}\left( Hh(\Lambda)\trans H Hh(\Lambda)\trans H \right)_{jm} \d t,\\
        &= \left( Hg^2(\Lambda)\trans H \right)_{ik}\left( Hh^2(\Lambda)\trans H \right)_{jm} \d t = g^2(X)_{ik} h^2(X)_{jm} \d t.
    \end{align*}

    Proceeding similarly with the other four summands we find

    \begin{align*}
        \d X_{ij} \d X_{km} = (g^2(X)_{ik}h^2(X)_{jm} + g^2(X)_{im}h^2(X)_{jk} + g^2(X)_{jk}h^2(X)_{im} + g^2(X)_{jm}h^2(X)_{ik})\d t.
    \end{align*}

    Since $\d N = H^T(\partial X)H$ only differs in a finite variation part of $H^T(\d X) H$, the martingale part of both processes coincide and then the quadratic covariation of the entries of $N$ is

    \begin{align*}
        \d N_{ij}\d N_{km} &= \d \bigl< (\trans H\d X H)_{ij},(\trans H \d X H)_{km} \bigr> = \sum_{pqrs} \d\bigl< \trans{H}_{ip}\d X_{pq} H_{qj}, \trans{H}_{kr}\d X_{rs} H_{sm} \bigr>, \\
        &= \sum_{pqrs} \trans{H}_{ip}H_{qj}\trans{H}_{kr}H_{sm}\d X_{pq} \d X_{rs},\\ 
        &= \sum_{pqrs} \trans{H}_{ip}H_{qj}\trans{H}_{kr}H_{sm}\bigl(g^2(X)_{pr}h^2(X)_{qs} + g^2(X)_{ps}h^2(X)_{qr} + g^2(X)_{qs}h^2(X)_{pr} \\ 
        &+ g^2(X)_{qr}h^2(X)_{ps}\bigr)\d t.
    \end{align*}

    We find first $\sum_{pqrs} \trans{H}_{ip}H_{qj}\trans{H}_{kr}H_{sm}g^2(X)_{pr}h^2(X)_{qs}$ and the other terms are similar,

    \begin{align*}
        \sum_{pqrs} \trans{H}_{ip}H_{qj}\trans{H}_{kr}H_{sm}g^2(X)_{pr}h^2(X)_{qs} &= \biggl(\sum_{pr}\trans{H}_{ip}g^2(X)_{pr}H_{rk}\biggr)\biggl(\sum_{qs}\trans{H}_{jq}h^2(X)_{qs}H_{sm}\biggr),\\
        &= \bigl( \trans{H}Hg^2(\Lambda)\trans{H}H\bigr)_{ik}\bigl(\trans{H}Hh^2(\Lambda)\trans{H}H\bigr)_{jm} = g^2(\Lambda)_{ik}h^2(\Lambda)_{jm}.
    \end{align*}

    Repeating the analogous procedure with all of the terms we find that the covariation is

    \begin{equation*} \label{eq:quadvarN}
        \d N_{ij}\d N_{km} = (g^2(\Lambda)_{ik}h^2(\Lambda)_{jm} + g^2(\Lambda)_{im}h^2(\Lambda)_{jk} + g^2(\Lambda)_{jk}h^2(\Lambda)_{im} + g^2(\Lambda)_{jm}h^2(\Lambda)_{ik})\d t.
    \end{equation*}

    It follows that the quadratic variation in the diagonal is

    \begin{equation*}
        \d N_{ii} \d N_{jj} = 4\delta_{ij}g^2(\lambda_i)h^2(\lambda_j)\d t.
    \end{equation*}

    Now, in order to compute $F$, the finite variation part of $N$, we use \eqref{eq:matrix_diffusion},

    \begin{align*}
        \d F &= H^Tb(X)H\d t + \frac12(\d H^T \d X H + H^T \d X \d H),\\
        &= b(\Lambda)\d t + \frac12 \left( (\d H^T H)(H^T\d X H) + (H^T \d X H)(H^T \d H) \right),\\
        \intertext{using that the martingale part of $H^T\d H$ and $H^T\partial H$ coincide and the same with $H^T(\partial X)H$ and $H^T (\d X) H$,}
        &= b(\Lambda)\d t + \frac12( (\d N \d A)^T + \d N \d A).
    \end{align*}

    Now we can use \eqref{eq:eigenAN} and \eqref{eq:quadvarN} to find $\d N \d A$,

    \begin{align*}
        (\d N \d A)_{ij} &= \sum_{k \neq j} \d N_{ik}\d A_{kj} = \sum_{k\neq j} \frac{\d N_{ik}\d N_{kj}}{\lambda_j - \lambda_k} = \delta_{ij}\sum_{k\neq j} \frac{ g^2(\lambda_i)h^2(\lambda_k) + g^2(\lambda_k)h^2(\lambda_i) }{\lambda_i - \lambda_k} \d t.
    \end{align*}

    Recalling that $G(x,y) = g^2(x)h^2(x) + g^2(y)h^2(y)$, we have that

    \begin{equation*}
        (\d N \d A)_{ij} = \delta_{ij}\sum_{k\neq j} \frac{ G(\lambda_i,\lambda_k) }{\lambda_i - \lambda_k} \d t.
    \end{equation*}

    From \eqref{eq:quadvarN} we have that the martingale part of $N_{ii}$ has the form
    $2g(\lambda_i)h(\lambda_i)\d W_i$ for some Brownian motion $W_i$. Putting together the martingale and finite variation parts of $N$ we have that

    \begin{equation*}
        \d N_{ii} = 2g(\lambda_i)h(\lambda_i)\d W_i + \sum_{k\neq j} \frac{ G(\lambda_i,\lambda_k) }{\lambda_i - \lambda_k} \d t.
    \end{equation*}

    Since $\d \lambda_i = \d N_{ii}$, this finishes the proof.
\end{proof}





\begin{theorem} \label{thm:diffusion_complejo}
    Let $W(t)$ be a complex $p\times p$ Brownian matrix. Suppose that $X = (X(t), t \ge 0)$ is a matrix-valued process taking values in the group of self adjoint matrices and it satisfies the following matrix stochastic differential equation:

    \begin{equation}\label{eq:complex_diff}
        \d X(t) = g(X(t))\d  W(t) h(X(t)) + h(X(t))\hermit{\d W(t)} g(X(t)) + b(X(t))\d t,
    \end{equation}

    \noindent with $g,h,b: \R \to \R$ and $X_0$ is a hermitian $p\times p$ random matrix with $p$ different eigenvalues.

    Let $G(x,y) = g^2(x)h^2(y) + g^2(y)h^2(x)$, and
    
    \[ \tau = \inf\{ t: \lambda_i(t) = \lambda_j(t) \text{ for some } i\neq j \}. \]
    
    Then, for $t < \tau$ the eigenvalue process $\Lambda_t$ verifies the following stochastic differential equations:

    \begin{equation}
        \d \lambda_i = 2g(\lambda_i)h(\lambda_i)\d W_i + \biggl( b(\lambda_i) + 2\sum_{k\neq i} \frac{G(\lambda_i,\lambda_k)}{\lambda_i - \lambda_k} \biggr)\d t,
    \end{equation}

    \noindent where $(W_i)_{i}$ are independent Brownian motions.
\end{theorem}

\begin{proof} 
    Recall that for a complex Brownian motion $Z$ we have that

    \[\d\langle Z,Z\rangle(t) = 0, \qquad \d \langle Z ,\overline Z\rangle(t) = 2\d t.\]

    
    Then we can compute the quadratic covariation $\d X_{ij}\d X_{kl}$ using \eqref{eq:complex_diff},

    \begin{align}
        \d X_{ij}\d X_{kl} &=  \d\langle X_{ij},X_{kl} \rangle(t), \\
        &= \d\langle \left( g(X)\d  W h(X) + h(X)\d \hermit{W} g(X) \right)_{ij}, \left(g(X)\d W h(X) + h(X)\d \hermit{W} g(X)\right)_{kl} \rangle(t),\\
        &= \d\langle \left(g(X)\d  W h(X)\right)_{ij}, \left(h(X)\d \hermit{W} g(X)\right)_{kl}\rangle + \d\langle \left(g(X)\d  W h(X)\right)_{kl}, \left(h(X)\d \hermit{W} g(X)\right)_{ij}\rangle(t),\\
        &= 2g^2(X)_{il}h^2(X)_{jk}\d t + 2g^2(X)_{kj}h^2(X)_{li}\d t,\\
        &= 2\left( g^2(X)_{il}h^2(X)_{kj} + g^2(X)_{jk}h^2(X)_{il} \right)\d t.
    \end{align}


    Analogously to the real case, we define $A$, the stochastic logarithm of $H$, as

    \begin{equation*}
        A \coloneqq H^{-1}\partial H = H^* \partial H.
    \end{equation*}

    By using Itô's formula we find,

    \begin{equation*}
        0 = \d I = \d(H^*H) = H \partial H^* + (\partial H) H^* = A^* + A,
    \end{equation*}

    \noindent which means $A$ is skew-Hermitian. This implies that the real parto of the terms in the diagonal of $A$ is zero. Let us now apply Itô's formula to $\Lambda = H^* X H$,

    \begin{align*} % Mejorar esto.
        \d \Lambda &= \d(H^* X H) = H^*(\d(XH)) + (\d H^*) XH + \d(H^*)\d (XH),\\
        &= H^*(\d X) H + H^* X \d H + H^*(\d X\d H) + (\d H^*)X H + \d(H^*)(\d X)H + \d(H^*)X\d H + \d H^*\d X\d H,\\
        &= H^*(\partial X)H + H^*X\partial H + (\partial H^*)XH = H^*(\partial X)H + \Lambda H^* \partial H + (\partial H^*)H\Lambda,\\
        &= H^*(\partial X)H + \Lambda \partial A + \trans{\partial A}\Lambda = H^*(\partial X)H + \Lambda \partial A - \partial A \Lambda.
    \end{align*}

    By the relationship between Itô's and Stratanovich's integrals,

    \begin{equation*}
        H^*(\partial X)H = H^*(\d X)H + \frac12(\d H^*(\d X)H + H^*\d X\d H),
    \end{equation*}

    \noindent so using that $X$ is hermitian, we have that $H^*(\partial X)H$ is also hermitian and its diagonal elements are real. The process $\Lambda \partial A - (\partial A)\Lambda$ is zero in the diagonal and thus $\d \lambda_i = (H^*(\partial X)H)_{ii}$.  If $i \neq j$, we have

    \[ 0 = (H^*(\partial X)H)_{ij} + \lambda_i\partial A_{ij} - \lambda_j \partial A_{ji} = (H^*(\partial X)H)_{ij} + (\lambda_i - \lambda_j)\partial A_{ij}. \]

    The last part implies $\partial A_{ij} = \frac{(H^*(\partial X)H)_{ij}}{\lambda_j - \lambda_i}$, whenever $i\neq j$.

    Define $\d N = \d H^*(\partial X)H$. The martingale part of $N$ and $H^*(\d X)H$ is the same, since they differ only in a finite variation term. We can find $\d N_{ij}\d N_{kl}$ using $\d X_{ij} \d X_{kl}$,

    \begin{equation*}
        \d N_{ij}\d N{kl} = 2(g^2(\Lambda)_{il}h^2(\Lambda)_{jk} + g^2(\Lambda)_{jk}h^2(\Lambda)_{il})\d t.
    \end{equation*}

    Then, for the elements in the diagonal we have

    \begin{equation}
        \d N_{ii}\d N_{jj} = 4\delta_{ij}(g^2(\lambda_i)h^2(\lambda_i))\d t.
    \end{equation}

    Now we compute the finite variation part of $\d N$ from \eqref{eq:complex_diff}. Let us denote it as $\d F$.

    \begin{align*}
        \d F &= H^*b(X)H\d t + \frac12(\d H^*(\d X)H + H^*\d X\d H),\\
             &= b(\Lambda)\d t + \frac12 \bigl((\d H^*H) (H^*\d XH) + (H^*\d X H)(H^*\d H)\bigr),\\
             &= b(\Lambda)\d t + \frac12\bigl((\d N \d A)^* + \d N \d A \bigr).
    \end{align*}

    Using the quadratic variation of $\d N$ and $\d A$ we find their covariation,

    \begin{align*}
        (\d N\d A)_{ij} &= \sum_{k} (\d N)_{ik}(\d A)_{kj} = \sum_k \frac{(\d N)_{ik}(\d N)_{kj}}{\lambda_j - \lambda_i}, \\
        &= 2\delta_{ij}\sum_{k\neq j} \frac{ g^2(\lambda_i)h^2(\lambda_k) + g^2(\lambda_k)h^2(\lambda_j)}{\lambda_j - \lambda_k} + \d N_{ij}\d A_{jj}.
    \end{align*}

    By the properties shown above for $\d N$ and $\d A$, if$i=j$, $\d N_{jj}$ is real and $\d A_{jj}$ is purely imaginary. By independence of the real and imaginary parts of the complex Brownian motion, this implies that $\d N_{jj}\d A_{jj}=0$. We have

    \begin{equation*}
        \d F_{ii} = \left(b(\lambda_i) + 2 \sum_{k\neq i} \frac{G(\lambda_i,\lambda_k)}{\lambda_i - \lambda_j}\right)\d t,
    \end{equation*}

    \noindent where $G(x,y)=g^2(x)h^2(y) + g^2(y)h^2(x)$.

    Using the quadratic variation of $\d N$, we find that the martingale part of $\d N_{ii}$ is 

    \begin{equation*}
        \d M_{ii} = 2g(\lambda_i)h(\lambda_i)\d W_i,
    \end{equation*}

    \noindent for some Brownian motion. Recall that $\d \lambda_i = \d N_{ii}$, then we have that there exist $W_1,\dots,W_p$ independent Brownian motions such that

    \begin{equation*}
        \d \lambda_i = \d N_{ii} = 2g(\lambda_i)h(\lambda_i)\d W_i + \left(b(\lambda_i) + 2 \sum_{k\neq i} \frac{G(\lambda_i,\lambda_k)}{\lambda_i - \lambda_j}\right)\d t.
    \end{equation*}

    This ends the proof.
\end{proof}


\begin{theorem}[Multidimensional Yamada-Watanabe Theorem \cite{article:multiyamada}]%First part of Multidimensional Yamada-Watanabe] 
    \label{thm:mult_yamada_watanabe}
    Let $p\in \N$ and %,q,r\in \N$ and 

    \begin{align*}
        b_i&: \R^p \to \R, \qquad i = 1, \dots, p,%\\
        %c_k&: \R^{p+r} \to \R, \qquad k = p+1, \dots, p + q,\\
        %d_j&: \R^{p+r} \to \R, \qquad j = p+1, \dots, p + r, 
    \end{align*}

    \noindent be real-valued continuous functions satisfying the following Lipschitz conditions for $C>0$,

    \begin{align*}
        \abs{b_i(y_1) - b_i(y_2)} &\le C \norm{y_1 - y_2}, \quad i = 1, \dots, p,%\\
        %\abs{c_k(y_1,z_1) - c_k(y_2,z_2)} &\le C\norm{(y_1,z_1) - (y_2,z_2)}, \quad k = p+1, \dots, p + q,\\
        %\abs{d_j(y_1,z_1) - d_j(y_2,z_2)} &\le C\norm{(y_1,z_1) - (y_2,z_2)}, \quad j = p+1, \dots, p + r,
    \end{align*}

    \noindent for every $y_1, y_2 \in \R^p$. %and $z_1,z_2\in \R^r$. 

    Further, let $\sigma_i:\R \to \R, i =1, \dots, p$ be a set of measurable functions such that

    \[ \abs{\sigma_i(x) - \sigma_i(y)}^2 \le \rho_i(\abs{x-y}), \quad x,y \in \R, \]

    \noindent where $\rho_i:(0,\infty)\to(0,\infty)$ are measurable functions such that 

    \[ \int_{0^+} \rho_i^{-1}(x)~\d x = \infty.\]

    Then the pathwise uniqueness holds for the following system of stochastic differential equations

    \begin{align}
        \d Y_i &= \sigma_i(Y_i)\d B_i + b_i(Y)\d t, \qquad i = 1, \dots, p,%\\
        %\d Z_j &= \sum_{k=p+1}^{p+q} c_k(Y,Z) \d B_k + d_j(Y,Z)\d t, \quad j = p + 1, \dots, p+r,
    \end{align}

    \noindent where $B_1, \dots, B_{p}$ are independent Brownian motions.
\end{theorem}

\begin{proof}
    Let $Y$ and $\hat Y$ be two solutions with respect to the same multidimensional Brownian motion $B = (B_i)_{i \le p}$ such that $Y(0) = \hat Y(0)$ %and $Z(0) = \hat Z(0)$ 
    a.s., for $i\le p$ we have

    \begin{equation}
        Y_i(t) - \hat Y_i(t) = \int_0^t \sigma_i(Y_i) - \sigma_i(\hat Y_i) ~\d B_i(s) + \int_0^t b_i(Y_i) - b_i(\hat Y_i) ~\d s.
    \end{equation}

    We can then see that

    \begin{equation*}
        \int_0^t \frac{\mathds{1}_{\{Y_i(s) > \hat Y_i(s)\}}}{\rho_i(Y_i(s)- \hat Y_i(s))} \d \langle Y_i - \hat Y_i, Y_i - \hat Y_i \rangle = \int_0^t \frac{( \sigma_i (Y_i(s)) - \sigma_i (\hat Y_i(s)) )^2}{\rho_i(Y_i(s) - \hat Y_i(s))} \mathds 1_{\{Y_i(s) > \hat Y_i(s)\}} ~\d s \le t.
    \end{equation*}
    
    Applying Theorem \ref{thm:local_zero} we have that the local time of $Y_i - \hat Y_i$ at 0 is 0. Then, we can use the Tanaka formula to find

    \begin{align*}
        \abs{ Y_i(t) - \hat Y_i(t) } &= \int_0^t \mathrm{sgn}(Y_i(s) - \hat Y_i(s)) ~\d( Y_i(s) - \hat Y_i(s) ),\\
        &= \int_0^t \mathrm{sgn}(Y_i(t) - \hat Y_i(t))(\sigma_i(Y_i) - \sigma_i(\hat Y_i)) ~\d B_i(s)\\
        &\phantom{espacioteeeee}+ \int_0^t \mathrm{sgn}(Y_i(s) - \hat Y_i(s))(b_i(Y_i(s)) - b_i(\hat Y_i(s))) ~ \d s.
    \end{align*}

    Since $\sigma_i$ is bounded, we have that $\sgn(Y_i(t) - \hat Y_i(t))(\sigma_i(Y_i(t)) - \sigma_i(\hat Y_i(t)))$ is bounded and therefore the first integral in the last expression is a martingale with mean 0, which in turns implies that

    \begin{equation*}
        \abs{ Y_i(t) - \hat Y_i(t) } - \int_0^t \sgn(Y_i(s) - \hat Y_i(s))(b_i(Y_i(s)) - b_i(\hat Y_i(s)))~\d t,
    \end{equation*}

    \noindent is a zero-mean martingale. Then, by using the Lipschitz properties of $b_i$ we have

    \begin{align*}
        \E{\abs{Y_i(t) - \hat Y_i(t)}} &= \E{  \int_0^t \sgn(Y_i(s) - \hat Y_i(s))(b_i(Y_i(s)) - b_i(\hat Y_i(s)))~\d s },\\
        &\le \E{ \int_0^t \abs{b_i(Y_i(s)) - b_i(\hat Y_i(s))}~\d s},\\
        &= \int_0^t \E{ \abs{ b_i(Y_i(s)) - b_i(\hat Y_i(s)) } }~\d s 
        \le  C\int_0^t \E{\abs{Y_i(s) - \hat Y_i(s)}}~\d s.
    \end{align*}

    Summing for every $i$ we get

    \begin{equation*}
        \E{\abs{Y(t) - \hat Y(t)}} \le  Cp\int_0^t \E{\abs{Y(t) - \hat Y(s)}}~\d s.
    \end{equation*}

    Using Gronwall's lemma (\ref{lemma:gronwall}) we get that

    \begin{equation*}
        \E{\abs{Y(t) - \hat Y(t)}} = 0, 
    \end{equation*}

    \noindent which implies $Y(t) = \hat Y(t)$ a.s. for every $t>0$, ending the proof.
\end{proof}




\begin{theorem}[Spectral matrix Yamada-Watanabe theorem] \label{thm:spectral_yamadawatanabe}
    Let $X(t)$ be a $p\times p$ symmetric matrix-valued process satisfying the equation \eqref{eq:matrix_diffusion} with initial condition $X(0)$ that is a symmetric $p\times p$ matrix with $p$ different eigenvalues. Suppose further that

    \begin{equation}
        \abs{ g(x)h(x) - g(y)h(y) }^2 \le \rho(\abs{x-y}), \qquad x,y \in \R,
    \end{equation}

    \noindent with $\rho:(0,\infty)\to(0,\infty)$ a measurable function satisfying

    \[ \int_{0^+} \rho^{-1}(x) \d x = \infty, \]

    \noindent that $G(x,y) \coloneqq g^2(x)h^2(y) + g^2(y)h^2(x)$ is locally Lipschitz and strictly positive on the set $\{ x \neq y\}$ and that $b$ is locally Lipschitz. Then if $\tau$ is defined as in \eqref{eq:collision_time}, for $t < \tau$, the process of eigenvalues satisfying \eqref{eq:gen_dyson} has a pathwise unique solution.
\end{theorem}

\begin{proof}
    Let $PN \trans{P}$ be a diagonalization for $X(0)$. We need to show that a unique strong solution exists for \eqref{eq:gen_dyson} when $\Lambda(0)=N$. The functions
    
    \[ b_i(\lambda_1, \dots,\lambda_p) = b(\lambda_i) + \sum_{k \neq i} \frac{G(\lambda_i,\lambda_k)}{\lambda_i - \lambda_k}, \]

    \noindent are locally Lipschitz continuous on $\Delta_p$ so they can be extended from the compact sets 

    \[ D_m = \{ 0 \le \lambda_1 < \lambda_2 < \cdots < \lambda_p < m, \lambda_{i+1} - \lambda_i \ge 1/m \}, \]

    \noindent to bounded Lipschitz functions on $\R^p$. Let $b_i^m$ denote such extension for $m \in \N \setminus \{0\}$. For $i = 1, \dots, p$, we consider the system of SDEs,

    \[ \d \lambda_i = 2g(\lambda_i^m)h(\lambda_i^m)\d W_i + b_i^m(\Lambda^m) \d t. \]

    We have that $\abs{g(x)h(x) - g(y)h(y)}^2 \le \rho(\abs{x-y})$ and $\int_{0^+} \rho(x)^{-1}\d x=\infty$, and using Theorem \ref{thm:mult_yamada_watanabe} we get that there is a unique strong solution for the system of SDEs. Since $D_m \subset D_{m+1}$, we have that $\lim_{m\to\infty} D_m = \Delta_p$, so there exists a unique strong solution $\Lambda(t)$ for the SDEs system up to the first exit time from $\Delta_p$. This time is $\tau$, the first collision time of the eigenvalues.

\end{proof}

\begin{corollary}
    Suppose that $b, g^2, h^2$ are Lipschitz continuous, $g^2h^2$ is convex or or continuously differentiable with derivative uniformly Lipschitz on $\R$ and that $G(x,y) \coloneqq g^2(x)h^2(y) + g^2(y)h^2(x)$ is strictly positive on $\{x \neq y\}$. Then the system of SDEs \eqref{eq:gen_dyson} for the eigenvalue process satisfying \eqref{eq:matrix_diffusion} has a unique strong solution on $[0,\infty)$.
\end{corollary}

\begin{proof}
    Recall that if $f$ is a non-negative Lipschitz continuous function, then $\sqrt f$ is $1/2$-Hölder continuous. Since $g^2$ and $h^2$ are Lipschitz continuous, then $g^2h^2$ is locally Lipschitz continuous and $gh$ is $1/2$-Hölder continuous. Then

    \begin{align*}
        \abs{g(x)h(x) - g(y)h(y)}^2 &\le \left( K\abs{x-y}^{\frac12} \right)^2 = K^2 \abs{x-y}.
    \end{align*}

    Taking $\rho(|x-y|) = K^2\abs{x-y}$ we see that the conditions of Theorem \ref{thm:spectral_yamadawatanabe} are satisfied and then the uniqueness and existence of the strong solution applies on $[0,\tau)$. By Theorem \ref{thm:collision} we have that $\tau=\infty$ a.s., and thus the existence and uniqueness is satisfied on $[0,\infty)$.
\end{proof}



\subsection{Wishart process}


The Wishart process is a dynamical version of the Wishart matrix, which was first described by John Wishart \cite{article:wishart}. If we assume a data population to consist of $n$ features observed in $k$ individuals, then we can form a rectangular $n\times k$ array with this data, let us name $X$ to such array. If we further assume that the variables and individualsa re totally non-correlated and they individual data points ($i$th feature of the $j$th individual) follow a standard normal distribution, then $X$ is a standard independent Gaussian matrix of size $n\times k$. The matrix $X$ can be thought as a size $k$ sample observation of an independent Normal vector $\vec v$ in $\R^k$. It is a well known fact in statistics that an estimator for the covariance matrix of $\vec v$ is $W \coloneqq \trans{X}X$. Under the former assumptions, $W$ follows a Wishart distribution.

Besides estimating a covariance matrix, the Wishart matrix distribution has some other uses in multivariate statistics. A notable one is its use for Principal Component Analysis (PCA). In \cite{book:hastie_tibshirani}, Principal Components are described as ``a sequence of projections of the data, mutually uncorrelated and ordered in variance''. The key idea is that if we find the eigenvalues of a covariance matrix and order them, then we can know what are the most influential features in the data variance, i.e.\ the eigenvector associated to the largest eigenvalue carries out the most variance of the data. Every eigenvector is a ``projection of the data'' and uncorrelated to any other eigenvector (as they form an orthogonal basis). 

A natural question that comes to mind when doing PCA is what would happen to the Principal Components if we add a mild perturbation (e.g. Gaussian noise). Bru \cite{bru1989diffusions} tackled this problem by considering adding a Brownian motion as a noise. With this, she was able to use stochastic calculus techniques to study the behaviour of the Principal Components when the added variance fluctuated. 


In this subsection, we use the Theorems proven previously to replicate Bru's results. It is worthmentioning that the techniques in this section taken from \cite{article:multiyamada} were originally inspired by the work of Bru in \cite{bru1989diffusions}.


Once we have Theorems \ref{thm:diffusion_real} and \ref{thm:diffusion_complejo}, proving the form of the eigenvalues in a Wishart process is quite straight-forward.



\begin{corollary}
    Let $\tilde B = (\tilde B(t), t\ge 0)$ be a Brownian motion in $\mathcal M_{n,m}(\R)$ with $n\ge m$ and define $X = \trans{\tilde B}\tilde B$. Then the eigenvalues of $X$, $\lambda_1 > \lambda_2 > ... > \lambda_n$ are given by the unique strong solution to the following system of stochastic differential equations

    \begin{equation*}
        \d \lambda_i = 2 \sqrt{\lambda_i} \d W_i + \left(m + \sum_{k\neq i} \frac{\abs{\lambda_i} + \abs{\lambda_k}}{ \lambda_i - \lambda_k} \right) \d t.
    \end{equation*}

    Morover, if $Y$ is any matrix-valued stochastic process satisfying the stochastic differential equation

    \begin{equation*}
        \d Y(t) = \sqrt{Y(t)} \d B(t) + \d \trans{B}(t) \sqrt{Y(t)} + \alpha I \d t,
    \end{equation*}

    with respect to $B = (B(t), t \ge 0)$ a Brownian motion in $\M_{n\times n}(\R)$ and $n\ge p-1$, then its eigenvalues are the unique strong solution to the system of stochastic differential equations

    \begin{equation} \label{eq:wishart}
        \d \lambda_i = 2 \sqrt{\lambda_i} \d W_i + \left(\alpha + \sum_{k\neq i} \frac{\abs{\lambda_i} + \abs{\lambda_k}}{ \lambda_i - \lambda_k} \right) \d t.
    \end{equation}

\end{corollary}

\begin{proof}

    We prove first that $X$ satisfies

    \begin{equation*}
        \d X(t) = \sqrt{X(t)} \d B(t) + \d \trans{B}(t) \sqrt{X(t)} + \alpha I \d t,
    \end{equation*}

    for an $n\times n$ matrix-valued Brownian motion, and then use Theorem \ref{thm:diffusion_real}. 

    By the matrix Itô formula we have for $X$,

    \begin{equation*}
        \d X(t) = (\d \trans{\tilde B}(t)) \tilde B(t) + (\trans{\tilde B}(t)) \d B(t) + \d \langle \trans{\tilde B}, \tilde B \rangle(t).
    \end{equation*}

    For the covariation term we can find

    \begin{align*}
        \d \langle \trans{\tilde B}, \tilde B \rangle(t)_{ij} &= \sum_{k=1}^m \d \langle \tilde B_{ki}, B_{kj} \rangle(t) = m\delta_{ij} \d t.
    \end{align*}

    This means $\d \langle \trans{\tilde B}, \tilde B \rangle(t) = m I \d t$.

    For the remaining terms, we find the covariation,

    \begin{equation*}
        (\trans {\tilde B} \d B)_{ij} (\trans{(\d \tilde B)}B)_{kl} = X_{il}\delta_{jk} \d t,
    \end{equation*}

    \noindent which in total accounts for 

    \begin{equation*}
        \d \langle X_{ij}, X_{kl} \rangle(t) = ( X_{ik}\delta_{jl} + X_{il}\delta_{jk} + X_{jk}\delta_{il} + X_{jl}\delta_{ik} )\d t.
    \end{equation*}

    With this, we can find the quadratic variation for the diagonal and off-diagonal entries of $X$.

    \begin{align*}
        \d \langle X_{ij}, X_{ij} \rangle (t) &= \left\{ \begin{array}{cc}
            (X_{ii} + X_{jj})\d t & \text{if $i\neq j$},\\
            4 X_{ii} \d t & \text{if $i=j$}.
        \end{array} \right.
    \end{align*}

    With this covariations, we find that the entries of $\d X$ coincide with those of $\sqrt{X(t)} \d B(t) + \d \trans{B}(t) \sqrt{X(t)} + \alpha I \d t$. Now, in Theorem \ref{thm:diffusion_real} substitute $g(x) = \sqrt{x}, h(x) \equiv 1$ and $b(x) \equiv \alpha$ to find that \eqref{eq:wishart} is satisfied.
\end{proof}

Using Brownian motions in $\M_{n,m}(\C)$ instead and repeating all the steps we find the corresponding equation for the complex Wishart process. Also, the matrix can be rescaled as in the Dyson Brownian motion case to find a version where the $\beta$ parameter affects the martingale part.

\subsection{Jacobi process}

Similarly to the Wishart case, the Jacobi process is a dynamical generalization of a random matrix used in statistics. There are basically two contexts in where the matrix jacobi process appears, one of them is in analysis of variance \cite{book:multivariate_statistics} and there it is defined as the ``quotient'' of a Wishart matrix and its sum to an independent Wishart matrix, i.e.

\begin{equation*}
    J \coloneqq (W_1 + W_2)^{-1}W_1,
\end{equation*}

\noindent where $W_1, W_2$ are indepenent Wishart matrices. In this context, the Jacobi matrix (known as MANOVA matrix) is something as a generalization of an $F$ distribution in the context of univariate ANOVA.

The second context, and the one we are interested in is the Generalized Singular Value Decomposition (GSVD) algorithm \cite{article:GSVD_van_loan} which is used for a block matrix. The construction we give here is found in \cite{article:marcus_finite_free_point_processes} and it coincides with the construction of a Beta-matrix in \cite{doumerc2005matrices}. 

Let $M$ be an independent Gaussian matrix in $\M_{m,n}(\R)$ with $n \le m$. We can decompose $M$ as a block matrix in the way

\begin{equation*}
    M = \begin{bmatrix} M_1 \\ M_2 \end{bmatrix} \begin{matrix} n_1 \\ n_2 \end{matrix}.
\end{equation*}

So $M_1 \in \M_{n_1,n}$ and $M_2 \in \M_{n_2,n}$ are independent Gaussian matrices in its respective spaces, and $n_1 + n_2 = m$. With the GSVD algorithm we can find simultaneously singular value decompositions for $M_1$ and $M_2$ such that $M_1 = U_1 CH, M_2 = U_2 S H$ where $U_1 \in \M_{n_1,n_1}(\R), U_2 \in \M_{n_2,n_2}(\R)$ are orthogonal matrices, $C\in \M_{n_1,n}, S \in \M_{n_2,n}(\R)$ are pseudo diagonals satisfying $\trans{C}C + \trans{S}S = I_{n\times n}$, and $H \in \M_{n,n}$ is invertible. Although the decomposition is not unique, it can be taken so that $U_1,U_2,H$ are Haar distributed, mutually independent and independent from $C,S$. 

Then we take $W_1 = \trans{M_1}M_1$ and $W_2 = \trans{M_1}M_1$, we have that $W_1$ and $W_2$ are $n\times n$ Wishart matrices with shape parameters $n_1$ and $n_2$, respectively. Our matrix of interest is

\begin{equation*}
    J \coloneqq (W_1 + W_2)^{-\frac12} W_1 (W_1 + W_2)^{-\frac12}.
\end{equation*}

    With the singular values decomposition of $M_1$ and $M_2$, we notice that 

\begin{align*}
    \det[J] &= \det[(W_1 + W_2)^{-\frac12} W_1 (W_1 + W_2)^{-\frac12}],\\ 
    &= \det[(\trans H \trans C C H + \trans H \trans SS H)^{-\frac12}\trans H \trans C C H(\trans H \trans C C H + \trans H \trans SS H)^{-\frac12}],\\
    &= \det[(\trans C C + \trans S S)^{\frac12}\trans CC (\trans C C + \trans S S)^{\frac12}],\\
    &= \det{\trans CC}.
\end{align*}

So $J$ has the same eignevalues of $\trans CC$. If we give a singular values decomposition for $M$, $M = VDU$ with $D \in \M_{m,n}(\R)$ pseudodiagonal $D = (\Delta, 0)^T$, $\Delta$ diagonal, and $U \in \M_{n,n}(\R),V \in \M_{m,m}(\R)$ Haar unitaries and independent. Then $\trans M M = \trans U \Delta^2 U$, and

\begin{equation*}
    (W_1 + W_2)^{\frac12} = (\trans U \Delta U)^{\frac12}.
\end{equation*}

Letting $X$ be the $m\times n_1$ upper left corner of $V$ we have that $M_1 = U\Delta X$, and then

\begin{equation*}
    \trans{M_1}M_1 = \trans U \Delta \trans X X \Delta U = (\trans M M)^{\frac12} (\trans U \trans X X U)  (\trans M M)^{\frac12}.
\end{equation*}

Substituting this in our previous definition for $J$ we have

\begin{align*}
    J &= (W_1 + W_2)^{-\frac12} W_1 (W_1 + W_2)^{-\frac12},\\ &= (\trans M M)^{-\frac12} (\trans M M)^{\frac12} (\trans U \trans X X U)  (\trans M M)^{\frac12} (\trans M M)^{-\frac12} = \trans{(XU)}XU.
\end{align*}

Since $X$ is invarant under multiplication by orthogonal matrices, and $U,X$ are indeopndent, then law of $J$ is equal to the law of $\trans X X$. The conclusion is that we can build the Jacobi matrix in two different ways, one using the standard definition as in the MANOVA case and another as the square of the upper left corner in a Haar distributed random matrix. This was discovered by Collins \cite{thesis:collins} and is used by Doumerc to construct the dynamical version of this matrix \cite{doumerc2005matrices}.

For the stochastic process case, it is shown in \cite{doumerc2005matrices} that if $X$ is the upper corner of a Haar unitary Brownian motion, then $J \coloneqq \trans X X$ satisfies the following stochastic differential equation

\begin{equation*}
    \d J (t) = \sqrt{J(t)} \d B(t) \sqrt{I_n - J(t)} + \sqrt{I_n - J(t)} \d B(t) \sqrt{J(t)} + (n_2 I_n - (n_1 + n_2) J(t) ) \d t.
\end{equation*}

With this differential equation, it is easy again to use Theorem \ref{thm:diffusion_real} to conclude the next corollary: 

\begin{corollary}
    Let $X$ be an $n\times n$ matrix valued process satisfying the following stochastic differential equation

    \begin{equation*}
        \d X (t) = \sqrt{X(t)} \d B(t) \sqrt{I_n - X(t)} + \sqrt{I_n - X(t)} \d B(t) \sqrt{X(t)} + (n_2 I_n - (n_1 + n_2) X(t) ) \d t.
    \end{equation*}

    Then its eigenvalues are the unique strong solution to the system of stochastic differential equations

    \begin{equation} \label{eq:jacobi}
        \d \lambda_i = 2 \sqrt{\lambda_i(1-\lambda_i)} \d W_i + \left( n_2 -(n_1 + n_2)\lambda_i + \sum_{k\neq i} \frac{\lambda_i(1-\lambda_k) + \lambda_k(1-\lambda_i)}{\lambda_i - \lambda_k} \right) \d t,
    \end{equation}

    \noindent where $\left\{ W_{i} \right\}_{i=1}^n$ are $n$ independent standard Brownian motions.
\end{corollary}


\begin{proof}
    Let $g(x) = \sqrt{\abs{x}}, h(x) = \sqrt{\abs{1-x}}$ and $b(x) = n_2 - (n_1 + n_2)x$ in \ref{thm:diffusion_real}. \todo{Mencionar en estas pruebas también cómo se da la existencia y unicidad de las soluciones fuertes (por los teoremas probados anteriormente.)}
\end{proof}


\section{Path simulations}

A standard technique for the simulation of solutions to stochastic differential equations is the Euler-Maruyama method. Details about the method can be found in \cite{book:asmussen}[Chapter 10]. The code used for the following figures can be found in the Appendix \ref{appendix:codes}. 

These path simulations have the purpose of visualizing the behavior of the eigenvalue process and how the presence of the Brownian motion term affects the trajectories. Comparing with the simulations of the deterministic processes in Chapter \ref{ch:determinist} will allow us to appreciate how mcuh the finite-variation part of the process ``determines'' its evolution.

In Figure \ref{fig:dyson_comparison}, we can see the path of a nine-dimensional Dyson Brownian motion compared to the finite-variation term. The color of each line represents the correspondence between the drift and the stochastic visualization. We can notice that both the random and the deterministic version do not collide. In the deterministic version, the values separate over time, while in the stochastic one, they ar affected by a noise that causes them to deviate slightly.

\begin{figure}[h!] 
    %% Creator: Matplotlib, PGF backend
%%
%% To include the figure in your LaTeX document, write
%%   \input{<filename>.pgf}
%%
%% Make sure the required packages are loaded in your preamble
%%   \usepackage{pgf}
%%
%% Also ensure that all the required font packages are loaded; for instance,
%% the lmodern package is sometimes necessary when using math font.
%%   \usepackage{lmodern}
%%
%% Figures using additional raster images can only be included by \input if
%% they are in the same directory as the main LaTeX file. For loading figures
%% from other directories you can use the `import` package
%%   \usepackage{import}
%%
%% and then include the figures with
%%   \import{<path to file>}{<filename>.pgf}
%%
%% Matplotlib used the following preamble
%%   
%%   \makeatletter\@ifpackageloaded{underscore}{}{\usepackage[strings]{underscore}}\makeatother
%%
\begingroup%
\makeatletter%
\begin{pgfpicture}%
\pgfpathrectangle{\pgfpointorigin}{\pgfqpoint{6.000000in}{2.500000in}}%
\pgfusepath{use as bounding box, clip}%
\begin{pgfscope}%
\pgfsetbuttcap%
\pgfsetmiterjoin%
\definecolor{currentfill}{rgb}{1.000000,1.000000,1.000000}%
\pgfsetfillcolor{currentfill}%
\pgfsetlinewidth{0.000000pt}%
\definecolor{currentstroke}{rgb}{1.000000,1.000000,1.000000}%
\pgfsetstrokecolor{currentstroke}%
\pgfsetdash{}{0pt}%
\pgfpathmoveto{\pgfqpoint{0.000000in}{0.000000in}}%
\pgfpathlineto{\pgfqpoint{6.000000in}{0.000000in}}%
\pgfpathlineto{\pgfqpoint{6.000000in}{2.500000in}}%
\pgfpathlineto{\pgfqpoint{0.000000in}{2.500000in}}%
\pgfpathlineto{\pgfqpoint{0.000000in}{0.000000in}}%
\pgfpathclose%
\pgfusepath{fill}%
\end{pgfscope}%
\begin{pgfscope}%
\pgfsetbuttcap%
\pgfsetmiterjoin%
\definecolor{currentfill}{rgb}{0.917647,0.917647,0.949020}%
\pgfsetfillcolor{currentfill}%
\pgfsetlinewidth{0.000000pt}%
\definecolor{currentstroke}{rgb}{0.000000,0.000000,0.000000}%
\pgfsetstrokecolor{currentstroke}%
\pgfsetstrokeopacity{0.000000}%
\pgfsetdash{}{0pt}%
\pgfpathmoveto{\pgfqpoint{0.750000in}{0.275000in}}%
\pgfpathlineto{\pgfqpoint{5.400000in}{0.275000in}}%
\pgfpathlineto{\pgfqpoint{5.400000in}{2.200000in}}%
\pgfpathlineto{\pgfqpoint{0.750000in}{2.200000in}}%
\pgfpathlineto{\pgfqpoint{0.750000in}{0.275000in}}%
\pgfpathclose%
\pgfusepath{fill}%
\end{pgfscope}%
\begin{pgfscope}%
\pgfpathrectangle{\pgfqpoint{0.750000in}{0.275000in}}{\pgfqpoint{4.650000in}{1.925000in}}%
\pgfusepath{clip}%
\pgfsetroundcap%
\pgfsetroundjoin%
\pgfsetlinewidth{1.003750pt}%
\definecolor{currentstroke}{rgb}{1.000000,1.000000,1.000000}%
\pgfsetstrokecolor{currentstroke}%
\pgfsetdash{}{0pt}%
\pgfpathmoveto{\pgfqpoint{0.961364in}{0.275000in}}%
\pgfpathlineto{\pgfqpoint{0.961364in}{2.200000in}}%
\pgfusepath{stroke}%
\end{pgfscope}%
\begin{pgfscope}%
\definecolor{textcolor}{rgb}{0.150000,0.150000,0.150000}%
\pgfsetstrokecolor{textcolor}%
\pgfsetfillcolor{textcolor}%
\pgftext[x=0.961364in,y=0.177778in,,top]{\color{textcolor}\rmfamily\fontsize{10.000000}{12.000000}\selectfont \(\displaystyle {0.0}\)}%
\end{pgfscope}%
\begin{pgfscope}%
\pgfpathrectangle{\pgfqpoint{0.750000in}{0.275000in}}{\pgfqpoint{4.650000in}{1.925000in}}%
\pgfusepath{clip}%
\pgfsetroundcap%
\pgfsetroundjoin%
\pgfsetlinewidth{1.003750pt}%
\definecolor{currentstroke}{rgb}{1.000000,1.000000,1.000000}%
\pgfsetstrokecolor{currentstroke}%
\pgfsetdash{}{0pt}%
\pgfpathmoveto{\pgfqpoint{1.489773in}{0.275000in}}%
\pgfpathlineto{\pgfqpoint{1.489773in}{2.200000in}}%
\pgfusepath{stroke}%
\end{pgfscope}%
\begin{pgfscope}%
\definecolor{textcolor}{rgb}{0.150000,0.150000,0.150000}%
\pgfsetstrokecolor{textcolor}%
\pgfsetfillcolor{textcolor}%
\pgftext[x=1.489773in,y=0.177778in,,top]{\color{textcolor}\rmfamily\fontsize{10.000000}{12.000000}\selectfont \(\displaystyle {2.5}\)}%
\end{pgfscope}%
\begin{pgfscope}%
\pgfpathrectangle{\pgfqpoint{0.750000in}{0.275000in}}{\pgfqpoint{4.650000in}{1.925000in}}%
\pgfusepath{clip}%
\pgfsetroundcap%
\pgfsetroundjoin%
\pgfsetlinewidth{1.003750pt}%
\definecolor{currentstroke}{rgb}{1.000000,1.000000,1.000000}%
\pgfsetstrokecolor{currentstroke}%
\pgfsetdash{}{0pt}%
\pgfpathmoveto{\pgfqpoint{2.018182in}{0.275000in}}%
\pgfpathlineto{\pgfqpoint{2.018182in}{2.200000in}}%
\pgfusepath{stroke}%
\end{pgfscope}%
\begin{pgfscope}%
\definecolor{textcolor}{rgb}{0.150000,0.150000,0.150000}%
\pgfsetstrokecolor{textcolor}%
\pgfsetfillcolor{textcolor}%
\pgftext[x=2.018182in,y=0.177778in,,top]{\color{textcolor}\rmfamily\fontsize{10.000000}{12.000000}\selectfont \(\displaystyle {5.0}\)}%
\end{pgfscope}%
\begin{pgfscope}%
\pgfpathrectangle{\pgfqpoint{0.750000in}{0.275000in}}{\pgfqpoint{4.650000in}{1.925000in}}%
\pgfusepath{clip}%
\pgfsetroundcap%
\pgfsetroundjoin%
\pgfsetlinewidth{1.003750pt}%
\definecolor{currentstroke}{rgb}{1.000000,1.000000,1.000000}%
\pgfsetstrokecolor{currentstroke}%
\pgfsetdash{}{0pt}%
\pgfpathmoveto{\pgfqpoint{2.546591in}{0.275000in}}%
\pgfpathlineto{\pgfqpoint{2.546591in}{2.200000in}}%
\pgfusepath{stroke}%
\end{pgfscope}%
\begin{pgfscope}%
\definecolor{textcolor}{rgb}{0.150000,0.150000,0.150000}%
\pgfsetstrokecolor{textcolor}%
\pgfsetfillcolor{textcolor}%
\pgftext[x=2.546591in,y=0.177778in,,top]{\color{textcolor}\rmfamily\fontsize{10.000000}{12.000000}\selectfont \(\displaystyle {7.5}\)}%
\end{pgfscope}%
\begin{pgfscope}%
\pgfpathrectangle{\pgfqpoint{0.750000in}{0.275000in}}{\pgfqpoint{4.650000in}{1.925000in}}%
\pgfusepath{clip}%
\pgfsetroundcap%
\pgfsetroundjoin%
\pgfsetlinewidth{1.003750pt}%
\definecolor{currentstroke}{rgb}{1.000000,1.000000,1.000000}%
\pgfsetstrokecolor{currentstroke}%
\pgfsetdash{}{0pt}%
\pgfpathmoveto{\pgfqpoint{3.075000in}{0.275000in}}%
\pgfpathlineto{\pgfqpoint{3.075000in}{2.200000in}}%
\pgfusepath{stroke}%
\end{pgfscope}%
\begin{pgfscope}%
\definecolor{textcolor}{rgb}{0.150000,0.150000,0.150000}%
\pgfsetstrokecolor{textcolor}%
\pgfsetfillcolor{textcolor}%
\pgftext[x=3.075000in,y=0.177778in,,top]{\color{textcolor}\rmfamily\fontsize{10.000000}{12.000000}\selectfont \(\displaystyle {10.0}\)}%
\end{pgfscope}%
\begin{pgfscope}%
\pgfpathrectangle{\pgfqpoint{0.750000in}{0.275000in}}{\pgfqpoint{4.650000in}{1.925000in}}%
\pgfusepath{clip}%
\pgfsetroundcap%
\pgfsetroundjoin%
\pgfsetlinewidth{1.003750pt}%
\definecolor{currentstroke}{rgb}{1.000000,1.000000,1.000000}%
\pgfsetstrokecolor{currentstroke}%
\pgfsetdash{}{0pt}%
\pgfpathmoveto{\pgfqpoint{3.603409in}{0.275000in}}%
\pgfpathlineto{\pgfqpoint{3.603409in}{2.200000in}}%
\pgfusepath{stroke}%
\end{pgfscope}%
\begin{pgfscope}%
\definecolor{textcolor}{rgb}{0.150000,0.150000,0.150000}%
\pgfsetstrokecolor{textcolor}%
\pgfsetfillcolor{textcolor}%
\pgftext[x=3.603409in,y=0.177778in,,top]{\color{textcolor}\rmfamily\fontsize{10.000000}{12.000000}\selectfont \(\displaystyle {12.5}\)}%
\end{pgfscope}%
\begin{pgfscope}%
\pgfpathrectangle{\pgfqpoint{0.750000in}{0.275000in}}{\pgfqpoint{4.650000in}{1.925000in}}%
\pgfusepath{clip}%
\pgfsetroundcap%
\pgfsetroundjoin%
\pgfsetlinewidth{1.003750pt}%
\definecolor{currentstroke}{rgb}{1.000000,1.000000,1.000000}%
\pgfsetstrokecolor{currentstroke}%
\pgfsetdash{}{0pt}%
\pgfpathmoveto{\pgfqpoint{4.131818in}{0.275000in}}%
\pgfpathlineto{\pgfqpoint{4.131818in}{2.200000in}}%
\pgfusepath{stroke}%
\end{pgfscope}%
\begin{pgfscope}%
\definecolor{textcolor}{rgb}{0.150000,0.150000,0.150000}%
\pgfsetstrokecolor{textcolor}%
\pgfsetfillcolor{textcolor}%
\pgftext[x=4.131818in,y=0.177778in,,top]{\color{textcolor}\rmfamily\fontsize{10.000000}{12.000000}\selectfont \(\displaystyle {15.0}\)}%
\end{pgfscope}%
\begin{pgfscope}%
\pgfpathrectangle{\pgfqpoint{0.750000in}{0.275000in}}{\pgfqpoint{4.650000in}{1.925000in}}%
\pgfusepath{clip}%
\pgfsetroundcap%
\pgfsetroundjoin%
\pgfsetlinewidth{1.003750pt}%
\definecolor{currentstroke}{rgb}{1.000000,1.000000,1.000000}%
\pgfsetstrokecolor{currentstroke}%
\pgfsetdash{}{0pt}%
\pgfpathmoveto{\pgfqpoint{4.660227in}{0.275000in}}%
\pgfpathlineto{\pgfqpoint{4.660227in}{2.200000in}}%
\pgfusepath{stroke}%
\end{pgfscope}%
\begin{pgfscope}%
\definecolor{textcolor}{rgb}{0.150000,0.150000,0.150000}%
\pgfsetstrokecolor{textcolor}%
\pgfsetfillcolor{textcolor}%
\pgftext[x=4.660227in,y=0.177778in,,top]{\color{textcolor}\rmfamily\fontsize{10.000000}{12.000000}\selectfont \(\displaystyle {17.5}\)}%
\end{pgfscope}%
\begin{pgfscope}%
\pgfpathrectangle{\pgfqpoint{0.750000in}{0.275000in}}{\pgfqpoint{4.650000in}{1.925000in}}%
\pgfusepath{clip}%
\pgfsetroundcap%
\pgfsetroundjoin%
\pgfsetlinewidth{1.003750pt}%
\definecolor{currentstroke}{rgb}{1.000000,1.000000,1.000000}%
\pgfsetstrokecolor{currentstroke}%
\pgfsetdash{}{0pt}%
\pgfpathmoveto{\pgfqpoint{5.188636in}{0.275000in}}%
\pgfpathlineto{\pgfqpoint{5.188636in}{2.200000in}}%
\pgfusepath{stroke}%
\end{pgfscope}%
\begin{pgfscope}%
\definecolor{textcolor}{rgb}{0.150000,0.150000,0.150000}%
\pgfsetstrokecolor{textcolor}%
\pgfsetfillcolor{textcolor}%
\pgftext[x=5.188636in,y=0.177778in,,top]{\color{textcolor}\rmfamily\fontsize{10.000000}{12.000000}\selectfont \(\displaystyle {20.0}\)}%
\end{pgfscope}%
\begin{pgfscope}%
\definecolor{textcolor}{rgb}{0.150000,0.150000,0.150000}%
\pgfsetstrokecolor{textcolor}%
\pgfsetfillcolor{textcolor}%
\pgftext[x=3.075000in,y=-0.001234in,,top]{\color{textcolor}\rmfamily\fontsize{11.000000}{13.200000}\selectfont Time (\(\displaystyle t\))}%
\end{pgfscope}%
\begin{pgfscope}%
\pgfpathrectangle{\pgfqpoint{0.750000in}{0.275000in}}{\pgfqpoint{4.650000in}{1.925000in}}%
\pgfusepath{clip}%
\pgfsetroundcap%
\pgfsetroundjoin%
\pgfsetlinewidth{1.003750pt}%
\definecolor{currentstroke}{rgb}{1.000000,1.000000,1.000000}%
\pgfsetstrokecolor{currentstroke}%
\pgfsetdash{}{0pt}%
\pgfpathmoveto{\pgfqpoint{0.750000in}{0.417360in}}%
\pgfpathlineto{\pgfqpoint{5.400000in}{0.417360in}}%
\pgfusepath{stroke}%
\end{pgfscope}%
\begin{pgfscope}%
\definecolor{textcolor}{rgb}{0.150000,0.150000,0.150000}%
\pgfsetstrokecolor{textcolor}%
\pgfsetfillcolor{textcolor}%
\pgftext[x=0.405863in, y=0.369135in, left, base]{\color{textcolor}\rmfamily\fontsize{10.000000}{12.000000}\selectfont \(\displaystyle {\ensuremath{-}20}\)}%
\end{pgfscope}%
\begin{pgfscope}%
\pgfpathrectangle{\pgfqpoint{0.750000in}{0.275000in}}{\pgfqpoint{4.650000in}{1.925000in}}%
\pgfusepath{clip}%
\pgfsetroundcap%
\pgfsetroundjoin%
\pgfsetlinewidth{1.003750pt}%
\definecolor{currentstroke}{rgb}{1.000000,1.000000,1.000000}%
\pgfsetstrokecolor{currentstroke}%
\pgfsetdash{}{0pt}%
\pgfpathmoveto{\pgfqpoint{0.750000in}{0.828827in}}%
\pgfpathlineto{\pgfqpoint{5.400000in}{0.828827in}}%
\pgfusepath{stroke}%
\end{pgfscope}%
\begin{pgfscope}%
\definecolor{textcolor}{rgb}{0.150000,0.150000,0.150000}%
\pgfsetstrokecolor{textcolor}%
\pgfsetfillcolor{textcolor}%
\pgftext[x=0.405863in, y=0.780602in, left, base]{\color{textcolor}\rmfamily\fontsize{10.000000}{12.000000}\selectfont \(\displaystyle {\ensuremath{-}10}\)}%
\end{pgfscope}%
\begin{pgfscope}%
\pgfpathrectangle{\pgfqpoint{0.750000in}{0.275000in}}{\pgfqpoint{4.650000in}{1.925000in}}%
\pgfusepath{clip}%
\pgfsetroundcap%
\pgfsetroundjoin%
\pgfsetlinewidth{1.003750pt}%
\definecolor{currentstroke}{rgb}{1.000000,1.000000,1.000000}%
\pgfsetstrokecolor{currentstroke}%
\pgfsetdash{}{0pt}%
\pgfpathmoveto{\pgfqpoint{0.750000in}{1.240293in}}%
\pgfpathlineto{\pgfqpoint{5.400000in}{1.240293in}}%
\pgfusepath{stroke}%
\end{pgfscope}%
\begin{pgfscope}%
\definecolor{textcolor}{rgb}{0.150000,0.150000,0.150000}%
\pgfsetstrokecolor{textcolor}%
\pgfsetfillcolor{textcolor}%
\pgftext[x=0.583333in, y=1.192068in, left, base]{\color{textcolor}\rmfamily\fontsize{10.000000}{12.000000}\selectfont \(\displaystyle {0}\)}%
\end{pgfscope}%
\begin{pgfscope}%
\pgfpathrectangle{\pgfqpoint{0.750000in}{0.275000in}}{\pgfqpoint{4.650000in}{1.925000in}}%
\pgfusepath{clip}%
\pgfsetroundcap%
\pgfsetroundjoin%
\pgfsetlinewidth{1.003750pt}%
\definecolor{currentstroke}{rgb}{1.000000,1.000000,1.000000}%
\pgfsetstrokecolor{currentstroke}%
\pgfsetdash{}{0pt}%
\pgfpathmoveto{\pgfqpoint{0.750000in}{1.651760in}}%
\pgfpathlineto{\pgfqpoint{5.400000in}{1.651760in}}%
\pgfusepath{stroke}%
\end{pgfscope}%
\begin{pgfscope}%
\definecolor{textcolor}{rgb}{0.150000,0.150000,0.150000}%
\pgfsetstrokecolor{textcolor}%
\pgfsetfillcolor{textcolor}%
\pgftext[x=0.513888in, y=1.603535in, left, base]{\color{textcolor}\rmfamily\fontsize{10.000000}{12.000000}\selectfont \(\displaystyle {10}\)}%
\end{pgfscope}%
\begin{pgfscope}%
\pgfpathrectangle{\pgfqpoint{0.750000in}{0.275000in}}{\pgfqpoint{4.650000in}{1.925000in}}%
\pgfusepath{clip}%
\pgfsetroundcap%
\pgfsetroundjoin%
\pgfsetlinewidth{1.003750pt}%
\definecolor{currentstroke}{rgb}{1.000000,1.000000,1.000000}%
\pgfsetstrokecolor{currentstroke}%
\pgfsetdash{}{0pt}%
\pgfpathmoveto{\pgfqpoint{0.750000in}{2.063226in}}%
\pgfpathlineto{\pgfqpoint{5.400000in}{2.063226in}}%
\pgfusepath{stroke}%
\end{pgfscope}%
\begin{pgfscope}%
\definecolor{textcolor}{rgb}{0.150000,0.150000,0.150000}%
\pgfsetstrokecolor{textcolor}%
\pgfsetfillcolor{textcolor}%
\pgftext[x=0.513888in, y=2.015001in, left, base]{\color{textcolor}\rmfamily\fontsize{10.000000}{12.000000}\selectfont \(\displaystyle {20}\)}%
\end{pgfscope}%
\begin{pgfscope}%
\definecolor{textcolor}{rgb}{0.150000,0.150000,0.150000}%
\pgfsetstrokecolor{textcolor}%
\pgfsetfillcolor{textcolor}%
\pgftext[x=0.350308in,y=1.237500in,,bottom,rotate=90.000000]{\color{textcolor}\rmfamily\fontsize{11.000000}{13.200000}\selectfont Position in space (\(\displaystyle \lambda_i\))}%
\end{pgfscope}%
\begin{pgfscope}%
\pgfpathrectangle{\pgfqpoint{0.750000in}{0.275000in}}{\pgfqpoint{4.650000in}{1.925000in}}%
\pgfusepath{clip}%
\pgfsetroundcap%
\pgfsetroundjoin%
\pgfsetlinewidth{0.803000pt}%
\definecolor{currentstroke}{rgb}{0.215686,0.494118,0.721569}%
\pgfsetstrokecolor{currentstroke}%
\pgfsetdash{}{0pt}%
\pgfpathmoveto{\pgfqpoint{0.961364in}{1.116853in}}%
\pgfpathlineto{\pgfqpoint{0.986740in}{1.100249in}}%
\pgfpathlineto{\pgfqpoint{1.018460in}{1.082305in}}%
\pgfpathlineto{\pgfqpoint{1.056525in}{1.063377in}}%
\pgfpathlineto{\pgfqpoint{1.103048in}{1.042833in}}%
\pgfpathlineto{\pgfqpoint{1.158030in}{1.021115in}}%
\pgfpathlineto{\pgfqpoint{1.221471in}{0.998529in}}%
\pgfpathlineto{\pgfqpoint{1.295485in}{0.974636in}}%
\pgfpathlineto{\pgfqpoint{1.380073in}{0.949755in}}%
\pgfpathlineto{\pgfqpoint{1.475234in}{0.924117in}}%
\pgfpathlineto{\pgfqpoint{1.583084in}{0.897385in}}%
\pgfpathlineto{\pgfqpoint{1.705736in}{0.869330in}}%
\pgfpathlineto{\pgfqpoint{1.843191in}{0.840228in}}%
\pgfpathlineto{\pgfqpoint{1.995449in}{0.810293in}}%
\pgfpathlineto{\pgfqpoint{2.166739in}{0.778937in}}%
\pgfpathlineto{\pgfqpoint{2.354947in}{0.746780in}}%
\pgfpathlineto{\pgfqpoint{2.564301in}{0.713314in}}%
\pgfpathlineto{\pgfqpoint{2.794803in}{0.678772in}}%
\pgfpathlineto{\pgfqpoint{3.046452in}{0.643342in}}%
\pgfpathlineto{\pgfqpoint{3.323477in}{0.606627in}}%
\pgfpathlineto{\pgfqpoint{3.625878in}{0.568839in}}%
\pgfpathlineto{\pgfqpoint{3.953655in}{0.530148in}}%
\pgfpathlineto{\pgfqpoint{4.311038in}{0.490235in}}%
\pgfpathlineto{\pgfqpoint{4.698027in}{0.449285in}}%
\pgfpathlineto{\pgfqpoint{5.116737in}{0.407243in}}%
\pgfpathlineto{\pgfqpoint{5.188636in}{0.400235in}}%
\pgfpathlineto{\pgfqpoint{5.188636in}{0.400235in}}%
\pgfusepath{stroke}%
\end{pgfscope}%
\begin{pgfscope}%
\pgfpathrectangle{\pgfqpoint{0.750000in}{0.275000in}}{\pgfqpoint{4.650000in}{1.925000in}}%
\pgfusepath{clip}%
\pgfsetroundcap%
\pgfsetroundjoin%
\pgfsetlinewidth{0.803000pt}%
\definecolor{currentstroke}{rgb}{1.000000,0.498039,0.000000}%
\pgfsetstrokecolor{currentstroke}%
\pgfsetdash{}{0pt}%
\pgfpathmoveto{\pgfqpoint{0.961364in}{1.147713in}}%
\pgfpathlineto{\pgfqpoint{1.007887in}{1.129589in}}%
\pgfpathlineto{\pgfqpoint{1.058640in}{1.112340in}}%
\pgfpathlineto{\pgfqpoint{1.117851in}{1.094656in}}%
\pgfpathlineto{\pgfqpoint{1.187636in}{1.076187in}}%
\pgfpathlineto{\pgfqpoint{1.270109in}{1.056711in}}%
\pgfpathlineto{\pgfqpoint{1.367385in}{1.036095in}}%
\pgfpathlineto{\pgfqpoint{1.479464in}{1.014655in}}%
\pgfpathlineto{\pgfqpoint{1.610575in}{0.991907in}}%
\pgfpathlineto{\pgfqpoint{1.760718in}{0.968178in}}%
\pgfpathlineto{\pgfqpoint{1.932008in}{0.943412in}}%
\pgfpathlineto{\pgfqpoint{2.126560in}{0.917579in}}%
\pgfpathlineto{\pgfqpoint{2.346488in}{0.890667in}}%
\pgfpathlineto{\pgfqpoint{2.593907in}{0.862676in}}%
\pgfpathlineto{\pgfqpoint{2.870932in}{0.833616in}}%
\pgfpathlineto{\pgfqpoint{3.179677in}{0.803500in}}%
\pgfpathlineto{\pgfqpoint{3.522258in}{0.772348in}}%
\pgfpathlineto{\pgfqpoint{3.902903in}{0.740004in}}%
\pgfpathlineto{\pgfqpoint{4.323727in}{0.706523in}}%
\pgfpathlineto{\pgfqpoint{4.786845in}{0.671952in}}%
\pgfpathlineto{\pgfqpoint{5.188636in}{0.643580in}}%
\pgfpathlineto{\pgfqpoint{5.188636in}{0.643580in}}%
\pgfusepath{stroke}%
\end{pgfscope}%
\begin{pgfscope}%
\pgfpathrectangle{\pgfqpoint{0.750000in}{0.275000in}}{\pgfqpoint{4.650000in}{1.925000in}}%
\pgfusepath{clip}%
\pgfsetroundcap%
\pgfsetroundjoin%
\pgfsetlinewidth{0.803000pt}%
\definecolor{currentstroke}{rgb}{0.301961,0.686275,0.290196}%
\pgfsetstrokecolor{currentstroke}%
\pgfsetdash{}{0pt}%
\pgfpathmoveto{\pgfqpoint{0.961364in}{1.178573in}}%
\pgfpathlineto{\pgfqpoint{1.033263in}{1.161992in}}%
\pgfpathlineto{\pgfqpoint{1.107278in}{1.147415in}}%
\pgfpathlineto{\pgfqpoint{1.196095in}{1.132317in}}%
\pgfpathlineto{\pgfqpoint{1.303944in}{1.116352in}}%
\pgfpathlineto{\pgfqpoint{1.432940in}{1.099573in}}%
\pgfpathlineto{\pgfqpoint{1.587313in}{1.081787in}}%
\pgfpathlineto{\pgfqpoint{1.771291in}{1.062891in}}%
\pgfpathlineto{\pgfqpoint{1.986990in}{1.043020in}}%
\pgfpathlineto{\pgfqpoint{2.240753in}{1.021941in}}%
\pgfpathlineto{\pgfqpoint{2.534696in}{0.999820in}}%
\pgfpathlineto{\pgfqpoint{2.873047in}{0.976643in}}%
\pgfpathlineto{\pgfqpoint{3.260036in}{0.952412in}}%
\pgfpathlineto{\pgfqpoint{3.699892in}{0.927139in}}%
\pgfpathlineto{\pgfqpoint{4.198960in}{0.900733in}}%
\pgfpathlineto{\pgfqpoint{4.761468in}{0.873241in}}%
\pgfpathlineto{\pgfqpoint{5.188636in}{0.853667in}}%
\pgfpathlineto{\pgfqpoint{5.188636in}{0.853667in}}%
\pgfusepath{stroke}%
\end{pgfscope}%
\begin{pgfscope}%
\pgfpathrectangle{\pgfqpoint{0.750000in}{0.275000in}}{\pgfqpoint{4.650000in}{1.925000in}}%
\pgfusepath{clip}%
\pgfsetroundcap%
\pgfsetroundjoin%
\pgfsetlinewidth{0.803000pt}%
\definecolor{currentstroke}{rgb}{0.968627,0.505882,0.749020}%
\pgfsetstrokecolor{currentstroke}%
\pgfsetdash{}{0pt}%
\pgfpathmoveto{\pgfqpoint{0.961364in}{1.209433in}}%
\pgfpathlineto{\pgfqpoint{1.069213in}{1.197923in}}%
\pgfpathlineto{\pgfqpoint{1.189751in}{1.187509in}}%
\pgfpathlineto{\pgfqpoint{1.346238in}{1.176356in}}%
\pgfpathlineto{\pgfqpoint{1.549248in}{1.164239in}}%
\pgfpathlineto{\pgfqpoint{1.807241in}{1.151166in}}%
\pgfpathlineto{\pgfqpoint{2.130789in}{1.137074in}}%
\pgfpathlineto{\pgfqpoint{2.532581in}{1.121881in}}%
\pgfpathlineto{\pgfqpoint{3.023190in}{1.105632in}}%
\pgfpathlineto{\pgfqpoint{3.615304in}{1.088315in}}%
\pgfpathlineto{\pgfqpoint{4.321612in}{1.069946in}}%
\pgfpathlineto{\pgfqpoint{5.156916in}{1.050502in}}%
\pgfpathlineto{\pgfqpoint{5.188636in}{1.049803in}}%
\pgfpathlineto{\pgfqpoint{5.188636in}{1.049803in}}%
\pgfusepath{stroke}%
\end{pgfscope}%
\begin{pgfscope}%
\pgfpathrectangle{\pgfqpoint{0.750000in}{0.275000in}}{\pgfqpoint{4.650000in}{1.925000in}}%
\pgfusepath{clip}%
\pgfsetroundcap%
\pgfsetroundjoin%
\pgfsetlinewidth{0.803000pt}%
\definecolor{currentstroke}{rgb}{0.650980,0.337255,0.156863}%
\pgfsetstrokecolor{currentstroke}%
\pgfsetdash{}{0pt}%
\pgfpathmoveto{\pgfqpoint{0.961364in}{1.240293in}}%
\pgfpathlineto{\pgfqpoint{5.188636in}{1.240293in}}%
\pgfpathlineto{\pgfqpoint{5.188636in}{1.240293in}}%
\pgfusepath{stroke}%
\end{pgfscope}%
\begin{pgfscope}%
\pgfpathrectangle{\pgfqpoint{0.750000in}{0.275000in}}{\pgfqpoint{4.650000in}{1.925000in}}%
\pgfusepath{clip}%
\pgfsetroundcap%
\pgfsetroundjoin%
\pgfsetlinewidth{0.803000pt}%
\definecolor{currentstroke}{rgb}{0.596078,0.305882,0.639216}%
\pgfsetstrokecolor{currentstroke}%
\pgfsetdash{}{0pt}%
\pgfpathmoveto{\pgfqpoint{0.961364in}{1.271153in}}%
\pgfpathlineto{\pgfqpoint{1.069213in}{1.282664in}}%
\pgfpathlineto{\pgfqpoint{1.189751in}{1.293077in}}%
\pgfpathlineto{\pgfqpoint{1.346238in}{1.304231in}}%
\pgfpathlineto{\pgfqpoint{1.549248in}{1.316347in}}%
\pgfpathlineto{\pgfqpoint{1.807241in}{1.329421in}}%
\pgfpathlineto{\pgfqpoint{2.130789in}{1.343513in}}%
\pgfpathlineto{\pgfqpoint{2.532581in}{1.358706in}}%
\pgfpathlineto{\pgfqpoint{3.023190in}{1.374955in}}%
\pgfpathlineto{\pgfqpoint{3.615304in}{1.392271in}}%
\pgfpathlineto{\pgfqpoint{4.321612in}{1.410641in}}%
\pgfpathlineto{\pgfqpoint{5.156916in}{1.430085in}}%
\pgfpathlineto{\pgfqpoint{5.188636in}{1.430784in}}%
\pgfpathlineto{\pgfqpoint{5.188636in}{1.430784in}}%
\pgfusepath{stroke}%
\end{pgfscope}%
\begin{pgfscope}%
\pgfpathrectangle{\pgfqpoint{0.750000in}{0.275000in}}{\pgfqpoint{4.650000in}{1.925000in}}%
\pgfusepath{clip}%
\pgfsetroundcap%
\pgfsetroundjoin%
\pgfsetlinewidth{0.803000pt}%
\definecolor{currentstroke}{rgb}{0.600000,0.600000,0.600000}%
\pgfsetstrokecolor{currentstroke}%
\pgfsetdash{}{0pt}%
\pgfpathmoveto{\pgfqpoint{0.961364in}{1.302013in}}%
\pgfpathlineto{\pgfqpoint{1.033263in}{1.318595in}}%
\pgfpathlineto{\pgfqpoint{1.107278in}{1.333172in}}%
\pgfpathlineto{\pgfqpoint{1.196095in}{1.348269in}}%
\pgfpathlineto{\pgfqpoint{1.303944in}{1.364235in}}%
\pgfpathlineto{\pgfqpoint{1.432940in}{1.381014in}}%
\pgfpathlineto{\pgfqpoint{1.587313in}{1.398800in}}%
\pgfpathlineto{\pgfqpoint{1.771291in}{1.417696in}}%
\pgfpathlineto{\pgfqpoint{1.986990in}{1.437567in}}%
\pgfpathlineto{\pgfqpoint{2.240753in}{1.458646in}}%
\pgfpathlineto{\pgfqpoint{2.534696in}{1.480767in}}%
\pgfpathlineto{\pgfqpoint{2.873047in}{1.503943in}}%
\pgfpathlineto{\pgfqpoint{3.260036in}{1.528175in}}%
\pgfpathlineto{\pgfqpoint{3.699892in}{1.553448in}}%
\pgfpathlineto{\pgfqpoint{4.198960in}{1.579853in}}%
\pgfpathlineto{\pgfqpoint{4.761468in}{1.607346in}}%
\pgfpathlineto{\pgfqpoint{5.188636in}{1.626919in}}%
\pgfpathlineto{\pgfqpoint{5.188636in}{1.626919in}}%
\pgfusepath{stroke}%
\end{pgfscope}%
\begin{pgfscope}%
\pgfpathrectangle{\pgfqpoint{0.750000in}{0.275000in}}{\pgfqpoint{4.650000in}{1.925000in}}%
\pgfusepath{clip}%
\pgfsetroundcap%
\pgfsetroundjoin%
\pgfsetlinewidth{0.803000pt}%
\definecolor{currentstroke}{rgb}{0.894118,0.101961,0.109804}%
\pgfsetstrokecolor{currentstroke}%
\pgfsetdash{}{0pt}%
\pgfpathmoveto{\pgfqpoint{0.961364in}{1.332873in}}%
\pgfpathlineto{\pgfqpoint{1.007887in}{1.350997in}}%
\pgfpathlineto{\pgfqpoint{1.058640in}{1.368247in}}%
\pgfpathlineto{\pgfqpoint{1.117851in}{1.385931in}}%
\pgfpathlineto{\pgfqpoint{1.187636in}{1.404400in}}%
\pgfpathlineto{\pgfqpoint{1.270109in}{1.423876in}}%
\pgfpathlineto{\pgfqpoint{1.367385in}{1.444492in}}%
\pgfpathlineto{\pgfqpoint{1.479464in}{1.465931in}}%
\pgfpathlineto{\pgfqpoint{1.610575in}{1.488680in}}%
\pgfpathlineto{\pgfqpoint{1.760718in}{1.512409in}}%
\pgfpathlineto{\pgfqpoint{1.932008in}{1.537175in}}%
\pgfpathlineto{\pgfqpoint{2.126560in}{1.563008in}}%
\pgfpathlineto{\pgfqpoint{2.346488in}{1.589920in}}%
\pgfpathlineto{\pgfqpoint{2.593907in}{1.617911in}}%
\pgfpathlineto{\pgfqpoint{2.870932in}{1.646971in}}%
\pgfpathlineto{\pgfqpoint{3.179677in}{1.677087in}}%
\pgfpathlineto{\pgfqpoint{3.522258in}{1.708239in}}%
\pgfpathlineto{\pgfqpoint{3.902903in}{1.740583in}}%
\pgfpathlineto{\pgfqpoint{4.323727in}{1.774064in}}%
\pgfpathlineto{\pgfqpoint{4.786845in}{1.808635in}}%
\pgfpathlineto{\pgfqpoint{5.188636in}{1.837007in}}%
\pgfpathlineto{\pgfqpoint{5.188636in}{1.837007in}}%
\pgfusepath{stroke}%
\end{pgfscope}%
\begin{pgfscope}%
\pgfpathrectangle{\pgfqpoint{0.750000in}{0.275000in}}{\pgfqpoint{4.650000in}{1.925000in}}%
\pgfusepath{clip}%
\pgfsetroundcap%
\pgfsetroundjoin%
\pgfsetlinewidth{0.803000pt}%
\definecolor{currentstroke}{rgb}{0.870588,0.870588,0.000000}%
\pgfsetstrokecolor{currentstroke}%
\pgfsetdash{}{0pt}%
\pgfpathmoveto{\pgfqpoint{0.961364in}{1.363733in}}%
\pgfpathlineto{\pgfqpoint{0.986740in}{1.380338in}}%
\pgfpathlineto{\pgfqpoint{1.018460in}{1.398282in}}%
\pgfpathlineto{\pgfqpoint{1.056525in}{1.417209in}}%
\pgfpathlineto{\pgfqpoint{1.103048in}{1.437753in}}%
\pgfpathlineto{\pgfqpoint{1.158030in}{1.459472in}}%
\pgfpathlineto{\pgfqpoint{1.221471in}{1.482058in}}%
\pgfpathlineto{\pgfqpoint{1.295485in}{1.505951in}}%
\pgfpathlineto{\pgfqpoint{1.380073in}{1.530832in}}%
\pgfpathlineto{\pgfqpoint{1.475234in}{1.556470in}}%
\pgfpathlineto{\pgfqpoint{1.583084in}{1.583201in}}%
\pgfpathlineto{\pgfqpoint{1.705736in}{1.611257in}}%
\pgfpathlineto{\pgfqpoint{1.843191in}{1.640358in}}%
\pgfpathlineto{\pgfqpoint{1.995449in}{1.670294in}}%
\pgfpathlineto{\pgfqpoint{2.166739in}{1.701650in}}%
\pgfpathlineto{\pgfqpoint{2.354947in}{1.733807in}}%
\pgfpathlineto{\pgfqpoint{2.564301in}{1.767273in}}%
\pgfpathlineto{\pgfqpoint{2.794803in}{1.801815in}}%
\pgfpathlineto{\pgfqpoint{3.046452in}{1.837245in}}%
\pgfpathlineto{\pgfqpoint{3.323477in}{1.873960in}}%
\pgfpathlineto{\pgfqpoint{3.625878in}{1.911748in}}%
\pgfpathlineto{\pgfqpoint{3.953655in}{1.950438in}}%
\pgfpathlineto{\pgfqpoint{4.311038in}{1.990352in}}%
\pgfpathlineto{\pgfqpoint{4.698027in}{2.031302in}}%
\pgfpathlineto{\pgfqpoint{5.116737in}{2.073344in}}%
\pgfpathlineto{\pgfqpoint{5.188636in}{2.080352in}}%
\pgfpathlineto{\pgfqpoint{5.188636in}{2.080352in}}%
\pgfusepath{stroke}%
\end{pgfscope}%
\begin{pgfscope}%
\pgfpathrectangle{\pgfqpoint{0.750000in}{0.275000in}}{\pgfqpoint{4.650000in}{1.925000in}}%
\pgfusepath{clip}%
\pgfsetroundcap%
\pgfsetroundjoin%
\pgfsetlinewidth{1.003750pt}%
\definecolor{currentstroke}{rgb}{0.215686,0.494118,0.721569}%
\pgfsetstrokecolor{currentstroke}%
\pgfsetdash{}{0pt}%
\pgfpathmoveto{\pgfqpoint{0.961364in}{1.116853in}}%
\pgfpathlineto{\pgfqpoint{0.963478in}{1.114823in}}%
\pgfpathlineto{\pgfqpoint{0.965593in}{1.106375in}}%
\pgfpathlineto{\pgfqpoint{0.967708in}{1.103611in}}%
\pgfpathlineto{\pgfqpoint{0.969822in}{1.107093in}}%
\pgfpathlineto{\pgfqpoint{0.971937in}{1.108043in}}%
\pgfpathlineto{\pgfqpoint{0.974052in}{1.101378in}}%
\pgfpathlineto{\pgfqpoint{0.976166in}{1.107562in}}%
\pgfpathlineto{\pgfqpoint{0.978281in}{1.107949in}}%
\pgfpathlineto{\pgfqpoint{0.982511in}{1.103682in}}%
\pgfpathlineto{\pgfqpoint{0.984625in}{1.108833in}}%
\pgfpathlineto{\pgfqpoint{0.986740in}{1.117950in}}%
\pgfpathlineto{\pgfqpoint{0.993084in}{1.093323in}}%
\pgfpathlineto{\pgfqpoint{0.995199in}{1.099052in}}%
\pgfpathlineto{\pgfqpoint{0.997313in}{1.100516in}}%
\pgfpathlineto{\pgfqpoint{1.005772in}{1.089281in}}%
\pgfpathlineto{\pgfqpoint{1.010002in}{1.101414in}}%
\pgfpathlineto{\pgfqpoint{1.012116in}{1.094878in}}%
\pgfpathlineto{\pgfqpoint{1.016346in}{1.106213in}}%
\pgfpathlineto{\pgfqpoint{1.018460in}{1.089428in}}%
\pgfpathlineto{\pgfqpoint{1.022690in}{1.094038in}}%
\pgfpathlineto{\pgfqpoint{1.029034in}{1.081796in}}%
\pgfpathlineto{\pgfqpoint{1.033263in}{1.069305in}}%
\pgfpathlineto{\pgfqpoint{1.035378in}{1.068874in}}%
\pgfpathlineto{\pgfqpoint{1.037493in}{1.062342in}}%
\pgfpathlineto{\pgfqpoint{1.039607in}{1.059890in}}%
\pgfpathlineto{\pgfqpoint{1.041722in}{1.059763in}}%
\pgfpathlineto{\pgfqpoint{1.043837in}{1.064409in}}%
\pgfpathlineto{\pgfqpoint{1.045951in}{1.061278in}}%
\pgfpathlineto{\pgfqpoint{1.050181in}{1.065484in}}%
\pgfpathlineto{\pgfqpoint{1.052295in}{1.068245in}}%
\pgfpathlineto{\pgfqpoint{1.054410in}{1.060457in}}%
\pgfpathlineto{\pgfqpoint{1.056525in}{1.056959in}}%
\pgfpathlineto{\pgfqpoint{1.058640in}{1.049803in}}%
\pgfpathlineto{\pgfqpoint{1.064984in}{1.044921in}}%
\pgfpathlineto{\pgfqpoint{1.067098in}{1.039425in}}%
\pgfpathlineto{\pgfqpoint{1.069213in}{1.040845in}}%
\pgfpathlineto{\pgfqpoint{1.071328in}{1.035294in}}%
\pgfpathlineto{\pgfqpoint{1.073442in}{1.044554in}}%
\pgfpathlineto{\pgfqpoint{1.077672in}{1.033020in}}%
\pgfpathlineto{\pgfqpoint{1.084016in}{1.026808in}}%
\pgfpathlineto{\pgfqpoint{1.086131in}{1.022615in}}%
\pgfpathlineto{\pgfqpoint{1.092475in}{1.017246in}}%
\pgfpathlineto{\pgfqpoint{1.096704in}{1.028390in}}%
\pgfpathlineto{\pgfqpoint{1.098819in}{1.030204in}}%
\pgfpathlineto{\pgfqpoint{1.100933in}{1.025959in}}%
\pgfpathlineto{\pgfqpoint{1.103048in}{1.026567in}}%
\pgfpathlineto{\pgfqpoint{1.105163in}{1.035304in}}%
\pgfpathlineto{\pgfqpoint{1.107278in}{1.035255in}}%
\pgfpathlineto{\pgfqpoint{1.109392in}{1.031226in}}%
\pgfpathlineto{\pgfqpoint{1.111507in}{1.035600in}}%
\pgfpathlineto{\pgfqpoint{1.115736in}{1.021913in}}%
\pgfpathlineto{\pgfqpoint{1.117851in}{1.022873in}}%
\pgfpathlineto{\pgfqpoint{1.119966in}{1.026646in}}%
\pgfpathlineto{\pgfqpoint{1.124195in}{1.021189in}}%
\pgfpathlineto{\pgfqpoint{1.128424in}{1.005823in}}%
\pgfpathlineto{\pgfqpoint{1.130539in}{1.010088in}}%
\pgfpathlineto{\pgfqpoint{1.132654in}{1.006207in}}%
\pgfpathlineto{\pgfqpoint{1.141113in}{1.019833in}}%
\pgfpathlineto{\pgfqpoint{1.145342in}{1.005183in}}%
\pgfpathlineto{\pgfqpoint{1.147457in}{1.003652in}}%
\pgfpathlineto{\pgfqpoint{1.149571in}{1.003674in}}%
\pgfpathlineto{\pgfqpoint{1.153801in}{0.995877in}}%
\pgfpathlineto{\pgfqpoint{1.155915in}{0.997250in}}%
\pgfpathlineto{\pgfqpoint{1.158030in}{0.999987in}}%
\pgfpathlineto{\pgfqpoint{1.160145in}{0.999563in}}%
\pgfpathlineto{\pgfqpoint{1.162260in}{1.002873in}}%
\pgfpathlineto{\pgfqpoint{1.170718in}{1.003623in}}%
\pgfpathlineto{\pgfqpoint{1.172833in}{1.008324in}}%
\pgfpathlineto{\pgfqpoint{1.174948in}{1.008206in}}%
\pgfpathlineto{\pgfqpoint{1.177062in}{1.000953in}}%
\pgfpathlineto{\pgfqpoint{1.179177in}{0.999470in}}%
\pgfpathlineto{\pgfqpoint{1.181292in}{1.000184in}}%
\pgfpathlineto{\pgfqpoint{1.183406in}{0.998007in}}%
\pgfpathlineto{\pgfqpoint{1.185521in}{0.998402in}}%
\pgfpathlineto{\pgfqpoint{1.189751in}{1.008475in}}%
\pgfpathlineto{\pgfqpoint{1.193980in}{0.999020in}}%
\pgfpathlineto{\pgfqpoint{1.196095in}{1.003759in}}%
\pgfpathlineto{\pgfqpoint{1.198209in}{0.997795in}}%
\pgfpathlineto{\pgfqpoint{1.200324in}{0.997392in}}%
\pgfpathlineto{\pgfqpoint{1.202439in}{0.989884in}}%
\pgfpathlineto{\pgfqpoint{1.204553in}{0.990229in}}%
\pgfpathlineto{\pgfqpoint{1.206668in}{0.995590in}}%
\pgfpathlineto{\pgfqpoint{1.208783in}{0.991164in}}%
\pgfpathlineto{\pgfqpoint{1.210897in}{0.992754in}}%
\pgfpathlineto{\pgfqpoint{1.213012in}{0.986398in}}%
\pgfpathlineto{\pgfqpoint{1.215127in}{0.987242in}}%
\pgfpathlineto{\pgfqpoint{1.219356in}{0.985233in}}%
\pgfpathlineto{\pgfqpoint{1.225700in}{0.985557in}}%
\pgfpathlineto{\pgfqpoint{1.227815in}{0.982874in}}%
\pgfpathlineto{\pgfqpoint{1.229930in}{0.983655in}}%
\pgfpathlineto{\pgfqpoint{1.232044in}{0.978412in}}%
\pgfpathlineto{\pgfqpoint{1.234159in}{0.977586in}}%
\pgfpathlineto{\pgfqpoint{1.236274in}{0.980123in}}%
\pgfpathlineto{\pgfqpoint{1.242618in}{0.996523in}}%
\pgfpathlineto{\pgfqpoint{1.244733in}{0.993663in}}%
\pgfpathlineto{\pgfqpoint{1.246847in}{0.994219in}}%
\pgfpathlineto{\pgfqpoint{1.248962in}{0.993334in}}%
\pgfpathlineto{\pgfqpoint{1.251077in}{0.997175in}}%
\pgfpathlineto{\pgfqpoint{1.253191in}{0.996444in}}%
\pgfpathlineto{\pgfqpoint{1.257421in}{0.997651in}}%
\pgfpathlineto{\pgfqpoint{1.261650in}{0.991029in}}%
\pgfpathlineto{\pgfqpoint{1.272224in}{1.005204in}}%
\pgfpathlineto{\pgfqpoint{1.274338in}{1.012644in}}%
\pgfpathlineto{\pgfqpoint{1.276453in}{1.012445in}}%
\pgfpathlineto{\pgfqpoint{1.278568in}{1.002623in}}%
\pgfpathlineto{\pgfqpoint{1.280682in}{1.006769in}}%
\pgfpathlineto{\pgfqpoint{1.282797in}{1.001760in}}%
\pgfpathlineto{\pgfqpoint{1.284912in}{1.001406in}}%
\pgfpathlineto{\pgfqpoint{1.287026in}{1.004458in}}%
\pgfpathlineto{\pgfqpoint{1.289141in}{1.001549in}}%
\pgfpathlineto{\pgfqpoint{1.291256in}{1.000893in}}%
\pgfpathlineto{\pgfqpoint{1.293371in}{1.007188in}}%
\pgfpathlineto{\pgfqpoint{1.295485in}{0.995481in}}%
\pgfpathlineto{\pgfqpoint{1.297600in}{0.994940in}}%
\pgfpathlineto{\pgfqpoint{1.299715in}{0.990570in}}%
\pgfpathlineto{\pgfqpoint{1.301829in}{0.989080in}}%
\pgfpathlineto{\pgfqpoint{1.306059in}{0.992855in}}%
\pgfpathlineto{\pgfqpoint{1.310288in}{0.985876in}}%
\pgfpathlineto{\pgfqpoint{1.312403in}{0.985303in}}%
\pgfpathlineto{\pgfqpoint{1.314517in}{0.979531in}}%
\pgfpathlineto{\pgfqpoint{1.316632in}{0.989373in}}%
\pgfpathlineto{\pgfqpoint{1.320862in}{0.983215in}}%
\pgfpathlineto{\pgfqpoint{1.322976in}{0.981665in}}%
\pgfpathlineto{\pgfqpoint{1.327206in}{0.970150in}}%
\pgfpathlineto{\pgfqpoint{1.331435in}{0.969009in}}%
\pgfpathlineto{\pgfqpoint{1.333550in}{0.972387in}}%
\pgfpathlineto{\pgfqpoint{1.335664in}{0.970595in}}%
\pgfpathlineto{\pgfqpoint{1.337779in}{0.974129in}}%
\pgfpathlineto{\pgfqpoint{1.342009in}{0.967795in}}%
\pgfpathlineto{\pgfqpoint{1.344123in}{0.967952in}}%
\pgfpathlineto{\pgfqpoint{1.346238in}{0.961828in}}%
\pgfpathlineto{\pgfqpoint{1.350467in}{0.967374in}}%
\pgfpathlineto{\pgfqpoint{1.352582in}{0.959824in}}%
\pgfpathlineto{\pgfqpoint{1.354697in}{0.958837in}}%
\pgfpathlineto{\pgfqpoint{1.356811in}{0.956014in}}%
\pgfpathlineto{\pgfqpoint{1.358926in}{0.947674in}}%
\pgfpathlineto{\pgfqpoint{1.361041in}{0.946041in}}%
\pgfpathlineto{\pgfqpoint{1.363155in}{0.945902in}}%
\pgfpathlineto{\pgfqpoint{1.365270in}{0.942321in}}%
\pgfpathlineto{\pgfqpoint{1.367385in}{0.941602in}}%
\pgfpathlineto{\pgfqpoint{1.369500in}{0.947276in}}%
\pgfpathlineto{\pgfqpoint{1.373729in}{0.937875in}}%
\pgfpathlineto{\pgfqpoint{1.375844in}{0.934813in}}%
\pgfpathlineto{\pgfqpoint{1.380073in}{0.936779in}}%
\pgfpathlineto{\pgfqpoint{1.382188in}{0.944622in}}%
\pgfpathlineto{\pgfqpoint{1.384302in}{0.942185in}}%
\pgfpathlineto{\pgfqpoint{1.386417in}{0.941698in}}%
\pgfpathlineto{\pgfqpoint{1.388532in}{0.945916in}}%
\pgfpathlineto{\pgfqpoint{1.390646in}{0.945488in}}%
\pgfpathlineto{\pgfqpoint{1.394876in}{0.948090in}}%
\pgfpathlineto{\pgfqpoint{1.396991in}{0.954621in}}%
\pgfpathlineto{\pgfqpoint{1.399105in}{0.954789in}}%
\pgfpathlineto{\pgfqpoint{1.401220in}{0.945735in}}%
\pgfpathlineto{\pgfqpoint{1.405449in}{0.948878in}}%
\pgfpathlineto{\pgfqpoint{1.407564in}{0.945950in}}%
\pgfpathlineto{\pgfqpoint{1.416023in}{0.920179in}}%
\pgfpathlineto{\pgfqpoint{1.418137in}{0.920045in}}%
\pgfpathlineto{\pgfqpoint{1.422367in}{0.914411in}}%
\pgfpathlineto{\pgfqpoint{1.424482in}{0.914070in}}%
\pgfpathlineto{\pgfqpoint{1.426596in}{0.910870in}}%
\pgfpathlineto{\pgfqpoint{1.430826in}{0.909338in}}%
\pgfpathlineto{\pgfqpoint{1.432940in}{0.912631in}}%
\pgfpathlineto{\pgfqpoint{1.435055in}{0.918198in}}%
\pgfpathlineto{\pgfqpoint{1.439284in}{0.921172in}}%
\pgfpathlineto{\pgfqpoint{1.445628in}{0.934686in}}%
\pgfpathlineto{\pgfqpoint{1.449858in}{0.940414in}}%
\pgfpathlineto{\pgfqpoint{1.451973in}{0.940431in}}%
\pgfpathlineto{\pgfqpoint{1.454087in}{0.937569in}}%
\pgfpathlineto{\pgfqpoint{1.456202in}{0.937535in}}%
\pgfpathlineto{\pgfqpoint{1.458317in}{0.941293in}}%
\pgfpathlineto{\pgfqpoint{1.460431in}{0.940073in}}%
\pgfpathlineto{\pgfqpoint{1.462546in}{0.941269in}}%
\pgfpathlineto{\pgfqpoint{1.464661in}{0.946504in}}%
\pgfpathlineto{\pgfqpoint{1.468890in}{0.943436in}}%
\pgfpathlineto{\pgfqpoint{1.471005in}{0.950202in}}%
\pgfpathlineto{\pgfqpoint{1.473120in}{0.941709in}}%
\pgfpathlineto{\pgfqpoint{1.475234in}{0.942348in}}%
\pgfpathlineto{\pgfqpoint{1.479464in}{0.932320in}}%
\pgfpathlineto{\pgfqpoint{1.483693in}{0.943537in}}%
\pgfpathlineto{\pgfqpoint{1.485808in}{0.938223in}}%
\pgfpathlineto{\pgfqpoint{1.490037in}{0.940582in}}%
\pgfpathlineto{\pgfqpoint{1.492152in}{0.937487in}}%
\pgfpathlineto{\pgfqpoint{1.496381in}{0.945403in}}%
\pgfpathlineto{\pgfqpoint{1.498496in}{0.947707in}}%
\pgfpathlineto{\pgfqpoint{1.500611in}{0.946204in}}%
\pgfpathlineto{\pgfqpoint{1.502725in}{0.939913in}}%
\pgfpathlineto{\pgfqpoint{1.504840in}{0.940240in}}%
\pgfpathlineto{\pgfqpoint{1.513299in}{0.953071in}}%
\pgfpathlineto{\pgfqpoint{1.517528in}{0.952529in}}%
\pgfpathlineto{\pgfqpoint{1.519643in}{0.953044in}}%
\pgfpathlineto{\pgfqpoint{1.523872in}{0.942518in}}%
\pgfpathlineto{\pgfqpoint{1.525987in}{0.943646in}}%
\pgfpathlineto{\pgfqpoint{1.528102in}{0.946554in}}%
\pgfpathlineto{\pgfqpoint{1.530216in}{0.944509in}}%
\pgfpathlineto{\pgfqpoint{1.532331in}{0.938460in}}%
\pgfpathlineto{\pgfqpoint{1.534446in}{0.936551in}}%
\pgfpathlineto{\pgfqpoint{1.536560in}{0.940298in}}%
\pgfpathlineto{\pgfqpoint{1.538675in}{0.930525in}}%
\pgfpathlineto{\pgfqpoint{1.540790in}{0.932731in}}%
\pgfpathlineto{\pgfqpoint{1.542904in}{0.937395in}}%
\pgfpathlineto{\pgfqpoint{1.545019in}{0.932930in}}%
\pgfpathlineto{\pgfqpoint{1.549248in}{0.945604in}}%
\pgfpathlineto{\pgfqpoint{1.551363in}{0.942178in}}%
\pgfpathlineto{\pgfqpoint{1.557707in}{0.944218in}}%
\pgfpathlineto{\pgfqpoint{1.559822in}{0.946038in}}%
\pgfpathlineto{\pgfqpoint{1.561937in}{0.944366in}}%
\pgfpathlineto{\pgfqpoint{1.564051in}{0.938878in}}%
\pgfpathlineto{\pgfqpoint{1.566166in}{0.941982in}}%
\pgfpathlineto{\pgfqpoint{1.568281in}{0.941809in}}%
\pgfpathlineto{\pgfqpoint{1.570395in}{0.945237in}}%
\pgfpathlineto{\pgfqpoint{1.572510in}{0.952214in}}%
\pgfpathlineto{\pgfqpoint{1.574625in}{0.952961in}}%
\pgfpathlineto{\pgfqpoint{1.576740in}{0.947469in}}%
\pgfpathlineto{\pgfqpoint{1.578854in}{0.950829in}}%
\pgfpathlineto{\pgfqpoint{1.580969in}{0.949733in}}%
\pgfpathlineto{\pgfqpoint{1.583084in}{0.951502in}}%
\pgfpathlineto{\pgfqpoint{1.585198in}{0.950222in}}%
\pgfpathlineto{\pgfqpoint{1.589428in}{0.955797in}}%
\pgfpathlineto{\pgfqpoint{1.595772in}{0.945022in}}%
\pgfpathlineto{\pgfqpoint{1.597886in}{0.948738in}}%
\pgfpathlineto{\pgfqpoint{1.600001in}{0.948775in}}%
\pgfpathlineto{\pgfqpoint{1.602116in}{0.950236in}}%
\pgfpathlineto{\pgfqpoint{1.604231in}{0.953605in}}%
\pgfpathlineto{\pgfqpoint{1.606345in}{0.948244in}}%
\pgfpathlineto{\pgfqpoint{1.608460in}{0.949467in}}%
\pgfpathlineto{\pgfqpoint{1.610575in}{0.948021in}}%
\pgfpathlineto{\pgfqpoint{1.612689in}{0.949174in}}%
\pgfpathlineto{\pgfqpoint{1.614804in}{0.944063in}}%
\pgfpathlineto{\pgfqpoint{1.616919in}{0.949048in}}%
\pgfpathlineto{\pgfqpoint{1.621148in}{0.938237in}}%
\pgfpathlineto{\pgfqpoint{1.623263in}{0.935686in}}%
\pgfpathlineto{\pgfqpoint{1.625377in}{0.924408in}}%
\pgfpathlineto{\pgfqpoint{1.627492in}{0.925356in}}%
\pgfpathlineto{\pgfqpoint{1.629607in}{0.921498in}}%
\pgfpathlineto{\pgfqpoint{1.631722in}{0.920712in}}%
\pgfpathlineto{\pgfqpoint{1.633836in}{0.917062in}}%
\pgfpathlineto{\pgfqpoint{1.635951in}{0.922788in}}%
\pgfpathlineto{\pgfqpoint{1.640180in}{0.917343in}}%
\pgfpathlineto{\pgfqpoint{1.642295in}{0.919170in}}%
\pgfpathlineto{\pgfqpoint{1.644410in}{0.911699in}}%
\pgfpathlineto{\pgfqpoint{1.646524in}{0.910960in}}%
\pgfpathlineto{\pgfqpoint{1.648639in}{0.912777in}}%
\pgfpathlineto{\pgfqpoint{1.659213in}{0.895872in}}%
\pgfpathlineto{\pgfqpoint{1.663442in}{0.899763in}}%
\pgfpathlineto{\pgfqpoint{1.665557in}{0.905312in}}%
\pgfpathlineto{\pgfqpoint{1.667671in}{0.904571in}}%
\pgfpathlineto{\pgfqpoint{1.669786in}{0.896676in}}%
\pgfpathlineto{\pgfqpoint{1.674015in}{0.891718in}}%
\pgfpathlineto{\pgfqpoint{1.678245in}{0.882435in}}%
\pgfpathlineto{\pgfqpoint{1.680359in}{0.872016in}}%
\pgfpathlineto{\pgfqpoint{1.682474in}{0.872804in}}%
\pgfpathlineto{\pgfqpoint{1.684589in}{0.876096in}}%
\pgfpathlineto{\pgfqpoint{1.686704in}{0.876178in}}%
\pgfpathlineto{\pgfqpoint{1.688818in}{0.877561in}}%
\pgfpathlineto{\pgfqpoint{1.690933in}{0.872037in}}%
\pgfpathlineto{\pgfqpoint{1.695162in}{0.880358in}}%
\pgfpathlineto{\pgfqpoint{1.697277in}{0.877120in}}%
\pgfpathlineto{\pgfqpoint{1.699392in}{0.869577in}}%
\pgfpathlineto{\pgfqpoint{1.701506in}{0.876075in}}%
\pgfpathlineto{\pgfqpoint{1.705736in}{0.876394in}}%
\pgfpathlineto{\pgfqpoint{1.707851in}{0.878457in}}%
\pgfpathlineto{\pgfqpoint{1.709965in}{0.877359in}}%
\pgfpathlineto{\pgfqpoint{1.712080in}{0.872798in}}%
\pgfpathlineto{\pgfqpoint{1.714195in}{0.864107in}}%
\pgfpathlineto{\pgfqpoint{1.716309in}{0.872261in}}%
\pgfpathlineto{\pgfqpoint{1.718424in}{0.869600in}}%
\pgfpathlineto{\pgfqpoint{1.720539in}{0.870231in}}%
\pgfpathlineto{\pgfqpoint{1.722653in}{0.868351in}}%
\pgfpathlineto{\pgfqpoint{1.724768in}{0.864968in}}%
\pgfpathlineto{\pgfqpoint{1.726883in}{0.865638in}}%
\pgfpathlineto{\pgfqpoint{1.731112in}{0.871473in}}%
\pgfpathlineto{\pgfqpoint{1.733227in}{0.867867in}}%
\pgfpathlineto{\pgfqpoint{1.737456in}{0.867447in}}%
\pgfpathlineto{\pgfqpoint{1.739571in}{0.866768in}}%
\pgfpathlineto{\pgfqpoint{1.741686in}{0.870705in}}%
\pgfpathlineto{\pgfqpoint{1.745915in}{0.868418in}}%
\pgfpathlineto{\pgfqpoint{1.748030in}{0.868012in}}%
\pgfpathlineto{\pgfqpoint{1.750144in}{0.870201in}}%
\pgfpathlineto{\pgfqpoint{1.752259in}{0.863027in}}%
\pgfpathlineto{\pgfqpoint{1.756488in}{0.860011in}}%
\pgfpathlineto{\pgfqpoint{1.758603in}{0.860740in}}%
\pgfpathlineto{\pgfqpoint{1.764947in}{0.845831in}}%
\pgfpathlineto{\pgfqpoint{1.773406in}{0.854205in}}%
\pgfpathlineto{\pgfqpoint{1.775521in}{0.854803in}}%
\pgfpathlineto{\pgfqpoint{1.777635in}{0.857697in}}%
\pgfpathlineto{\pgfqpoint{1.779750in}{0.855578in}}%
\pgfpathlineto{\pgfqpoint{1.783979in}{0.843003in}}%
\pgfpathlineto{\pgfqpoint{1.786094in}{0.842007in}}%
\pgfpathlineto{\pgfqpoint{1.788209in}{0.848645in}}%
\pgfpathlineto{\pgfqpoint{1.796668in}{0.841997in}}%
\pgfpathlineto{\pgfqpoint{1.798782in}{0.839722in}}%
\pgfpathlineto{\pgfqpoint{1.800897in}{0.843499in}}%
\pgfpathlineto{\pgfqpoint{1.805126in}{0.857702in}}%
\pgfpathlineto{\pgfqpoint{1.811471in}{0.860086in}}%
\pgfpathlineto{\pgfqpoint{1.817815in}{0.859812in}}%
\pgfpathlineto{\pgfqpoint{1.819929in}{0.865134in}}%
\pgfpathlineto{\pgfqpoint{1.822044in}{0.860036in}}%
\pgfpathlineto{\pgfqpoint{1.824159in}{0.859915in}}%
\pgfpathlineto{\pgfqpoint{1.826273in}{0.857780in}}%
\pgfpathlineto{\pgfqpoint{1.832617in}{0.865918in}}%
\pgfpathlineto{\pgfqpoint{1.836847in}{0.861669in}}%
\pgfpathlineto{\pgfqpoint{1.838962in}{0.852808in}}%
\pgfpathlineto{\pgfqpoint{1.841076in}{0.850658in}}%
\pgfpathlineto{\pgfqpoint{1.843191in}{0.845740in}}%
\pgfpathlineto{\pgfqpoint{1.845306in}{0.846105in}}%
\pgfpathlineto{\pgfqpoint{1.847420in}{0.838235in}}%
\pgfpathlineto{\pgfqpoint{1.849535in}{0.839928in}}%
\pgfpathlineto{\pgfqpoint{1.853764in}{0.830515in}}%
\pgfpathlineto{\pgfqpoint{1.857994in}{0.821583in}}%
\pgfpathlineto{\pgfqpoint{1.860108in}{0.833076in}}%
\pgfpathlineto{\pgfqpoint{1.862223in}{0.828363in}}%
\pgfpathlineto{\pgfqpoint{1.866453in}{0.831280in}}%
\pgfpathlineto{\pgfqpoint{1.868567in}{0.836796in}}%
\pgfpathlineto{\pgfqpoint{1.872797in}{0.830468in}}%
\pgfpathlineto{\pgfqpoint{1.874911in}{0.831444in}}%
\pgfpathlineto{\pgfqpoint{1.877026in}{0.830216in}}%
\pgfpathlineto{\pgfqpoint{1.883370in}{0.824221in}}%
\pgfpathlineto{\pgfqpoint{1.885485in}{0.825230in}}%
\pgfpathlineto{\pgfqpoint{1.887599in}{0.821191in}}%
\pgfpathlineto{\pgfqpoint{1.889714in}{0.821648in}}%
\pgfpathlineto{\pgfqpoint{1.891829in}{0.816742in}}%
\pgfpathlineto{\pgfqpoint{1.893944in}{0.819174in}}%
\pgfpathlineto{\pgfqpoint{1.896058in}{0.825342in}}%
\pgfpathlineto{\pgfqpoint{1.898173in}{0.826460in}}%
\pgfpathlineto{\pgfqpoint{1.900288in}{0.826254in}}%
\pgfpathlineto{\pgfqpoint{1.902402in}{0.828376in}}%
\pgfpathlineto{\pgfqpoint{1.904517in}{0.827387in}}%
\pgfpathlineto{\pgfqpoint{1.908746in}{0.830935in}}%
\pgfpathlineto{\pgfqpoint{1.910861in}{0.827462in}}%
\pgfpathlineto{\pgfqpoint{1.912976in}{0.820641in}}%
\pgfpathlineto{\pgfqpoint{1.915090in}{0.821085in}}%
\pgfpathlineto{\pgfqpoint{1.917205in}{0.827425in}}%
\pgfpathlineto{\pgfqpoint{1.919320in}{0.827917in}}%
\pgfpathlineto{\pgfqpoint{1.921435in}{0.834267in}}%
\pgfpathlineto{\pgfqpoint{1.923549in}{0.834089in}}%
\pgfpathlineto{\pgfqpoint{1.925664in}{0.836412in}}%
\pgfpathlineto{\pgfqpoint{1.929893in}{0.826827in}}%
\pgfpathlineto{\pgfqpoint{1.932008in}{0.837767in}}%
\pgfpathlineto{\pgfqpoint{1.936237in}{0.847844in}}%
\pgfpathlineto{\pgfqpoint{1.940467in}{0.836770in}}%
\pgfpathlineto{\pgfqpoint{1.944696in}{0.834521in}}%
\pgfpathlineto{\pgfqpoint{1.946811in}{0.829285in}}%
\pgfpathlineto{\pgfqpoint{1.948926in}{0.828333in}}%
\pgfpathlineto{\pgfqpoint{1.953155in}{0.837467in}}%
\pgfpathlineto{\pgfqpoint{1.955270in}{0.833375in}}%
\pgfpathlineto{\pgfqpoint{1.961614in}{0.841903in}}%
\pgfpathlineto{\pgfqpoint{1.963728in}{0.839241in}}%
\pgfpathlineto{\pgfqpoint{1.965843in}{0.840370in}}%
\pgfpathlineto{\pgfqpoint{1.967958in}{0.836785in}}%
\pgfpathlineto{\pgfqpoint{1.970073in}{0.843174in}}%
\pgfpathlineto{\pgfqpoint{1.972187in}{0.843715in}}%
\pgfpathlineto{\pgfqpoint{1.974302in}{0.838836in}}%
\pgfpathlineto{\pgfqpoint{1.976417in}{0.838340in}}%
\pgfpathlineto{\pgfqpoint{1.978531in}{0.833300in}}%
\pgfpathlineto{\pgfqpoint{1.980646in}{0.837701in}}%
\pgfpathlineto{\pgfqpoint{1.984875in}{0.840207in}}%
\pgfpathlineto{\pgfqpoint{1.986990in}{0.853122in}}%
\pgfpathlineto{\pgfqpoint{1.991219in}{0.855769in}}%
\pgfpathlineto{\pgfqpoint{1.993334in}{0.850063in}}%
\pgfpathlineto{\pgfqpoint{1.997564in}{0.845805in}}%
\pgfpathlineto{\pgfqpoint{1.999678in}{0.846003in}}%
\pgfpathlineto{\pgfqpoint{2.003908in}{0.837793in}}%
\pgfpathlineto{\pgfqpoint{2.008137in}{0.842031in}}%
\pgfpathlineto{\pgfqpoint{2.012366in}{0.852057in}}%
\pgfpathlineto{\pgfqpoint{2.014481in}{0.849816in}}%
\pgfpathlineto{\pgfqpoint{2.016596in}{0.850676in}}%
\pgfpathlineto{\pgfqpoint{2.020825in}{0.845475in}}%
\pgfpathlineto{\pgfqpoint{2.022940in}{0.851376in}}%
\pgfpathlineto{\pgfqpoint{2.027169in}{0.843499in}}%
\pgfpathlineto{\pgfqpoint{2.031399in}{0.841013in}}%
\pgfpathlineto{\pgfqpoint{2.037743in}{0.828976in}}%
\pgfpathlineto{\pgfqpoint{2.039857in}{0.825386in}}%
\pgfpathlineto{\pgfqpoint{2.044087in}{0.814181in}}%
\pgfpathlineto{\pgfqpoint{2.050431in}{0.821196in}}%
\pgfpathlineto{\pgfqpoint{2.052546in}{0.817740in}}%
\pgfpathlineto{\pgfqpoint{2.054660in}{0.822924in}}%
\pgfpathlineto{\pgfqpoint{2.058890in}{0.821197in}}%
\pgfpathlineto{\pgfqpoint{2.061004in}{0.815872in}}%
\pgfpathlineto{\pgfqpoint{2.063119in}{0.818657in}}%
\pgfpathlineto{\pgfqpoint{2.065234in}{0.807578in}}%
\pgfpathlineto{\pgfqpoint{2.067348in}{0.808310in}}%
\pgfpathlineto{\pgfqpoint{2.073693in}{0.794963in}}%
\pgfpathlineto{\pgfqpoint{2.077922in}{0.793065in}}%
\pgfpathlineto{\pgfqpoint{2.080037in}{0.798240in}}%
\pgfpathlineto{\pgfqpoint{2.084266in}{0.790595in}}%
\pgfpathlineto{\pgfqpoint{2.086381in}{0.791063in}}%
\pgfpathlineto{\pgfqpoint{2.088495in}{0.785615in}}%
\pgfpathlineto{\pgfqpoint{2.092725in}{0.789061in}}%
\pgfpathlineto{\pgfqpoint{2.094839in}{0.784398in}}%
\pgfpathlineto{\pgfqpoint{2.096954in}{0.788708in}}%
\pgfpathlineto{\pgfqpoint{2.099069in}{0.785862in}}%
\pgfpathlineto{\pgfqpoint{2.101184in}{0.780579in}}%
\pgfpathlineto{\pgfqpoint{2.103298in}{0.784005in}}%
\pgfpathlineto{\pgfqpoint{2.107528in}{0.766618in}}%
\pgfpathlineto{\pgfqpoint{2.109642in}{0.766236in}}%
\pgfpathlineto{\pgfqpoint{2.111757in}{0.767035in}}%
\pgfpathlineto{\pgfqpoint{2.113872in}{0.764773in}}%
\pgfpathlineto{\pgfqpoint{2.115986in}{0.753964in}}%
\pgfpathlineto{\pgfqpoint{2.118101in}{0.752228in}}%
\pgfpathlineto{\pgfqpoint{2.122330in}{0.743839in}}%
\pgfpathlineto{\pgfqpoint{2.126560in}{0.735072in}}%
\pgfpathlineto{\pgfqpoint{2.128675in}{0.737979in}}%
\pgfpathlineto{\pgfqpoint{2.130789in}{0.734426in}}%
\pgfpathlineto{\pgfqpoint{2.135019in}{0.724778in}}%
\pgfpathlineto{\pgfqpoint{2.137133in}{0.726347in}}%
\pgfpathlineto{\pgfqpoint{2.139248in}{0.731870in}}%
\pgfpathlineto{\pgfqpoint{2.141363in}{0.727795in}}%
\pgfpathlineto{\pgfqpoint{2.143477in}{0.729414in}}%
\pgfpathlineto{\pgfqpoint{2.147707in}{0.735937in}}%
\pgfpathlineto{\pgfqpoint{2.149822in}{0.732720in}}%
\pgfpathlineto{\pgfqpoint{2.151936in}{0.722664in}}%
\pgfpathlineto{\pgfqpoint{2.160395in}{0.730433in}}%
\pgfpathlineto{\pgfqpoint{2.162510in}{0.723339in}}%
\pgfpathlineto{\pgfqpoint{2.164624in}{0.729119in}}%
\pgfpathlineto{\pgfqpoint{2.166739in}{0.726192in}}%
\pgfpathlineto{\pgfqpoint{2.168854in}{0.728692in}}%
\pgfpathlineto{\pgfqpoint{2.173083in}{0.736090in}}%
\pgfpathlineto{\pgfqpoint{2.175198in}{0.731028in}}%
\pgfpathlineto{\pgfqpoint{2.177313in}{0.731739in}}%
\pgfpathlineto{\pgfqpoint{2.179427in}{0.729040in}}%
\pgfpathlineto{\pgfqpoint{2.181542in}{0.724454in}}%
\pgfpathlineto{\pgfqpoint{2.183657in}{0.725592in}}%
\pgfpathlineto{\pgfqpoint{2.185771in}{0.722085in}}%
\pgfpathlineto{\pgfqpoint{2.187886in}{0.728269in}}%
\pgfpathlineto{\pgfqpoint{2.190001in}{0.727901in}}%
\pgfpathlineto{\pgfqpoint{2.192115in}{0.731567in}}%
\pgfpathlineto{\pgfqpoint{2.194230in}{0.731107in}}%
\pgfpathlineto{\pgfqpoint{2.196345in}{0.736339in}}%
\pgfpathlineto{\pgfqpoint{2.198459in}{0.733647in}}%
\pgfpathlineto{\pgfqpoint{2.202689in}{0.745657in}}%
\pgfpathlineto{\pgfqpoint{2.204804in}{0.741759in}}%
\pgfpathlineto{\pgfqpoint{2.206918in}{0.747151in}}%
\pgfpathlineto{\pgfqpoint{2.213262in}{0.728370in}}%
\pgfpathlineto{\pgfqpoint{2.219606in}{0.744853in}}%
\pgfpathlineto{\pgfqpoint{2.223836in}{0.737404in}}%
\pgfpathlineto{\pgfqpoint{2.225950in}{0.743535in}}%
\pgfpathlineto{\pgfqpoint{2.228065in}{0.742905in}}%
\pgfpathlineto{\pgfqpoint{2.234409in}{0.731496in}}%
\pgfpathlineto{\pgfqpoint{2.240753in}{0.749649in}}%
\pgfpathlineto{\pgfqpoint{2.247097in}{0.750421in}}%
\pgfpathlineto{\pgfqpoint{2.251327in}{0.748862in}}%
\pgfpathlineto{\pgfqpoint{2.253441in}{0.741835in}}%
\pgfpathlineto{\pgfqpoint{2.255556in}{0.740040in}}%
\pgfpathlineto{\pgfqpoint{2.257671in}{0.741029in}}%
\pgfpathlineto{\pgfqpoint{2.264015in}{0.731053in}}%
\pgfpathlineto{\pgfqpoint{2.266130in}{0.730246in}}%
\pgfpathlineto{\pgfqpoint{2.268244in}{0.727206in}}%
\pgfpathlineto{\pgfqpoint{2.270359in}{0.726200in}}%
\pgfpathlineto{\pgfqpoint{2.274588in}{0.717359in}}%
\pgfpathlineto{\pgfqpoint{2.278818in}{0.714472in}}%
\pgfpathlineto{\pgfqpoint{2.283047in}{0.705638in}}%
\pgfpathlineto{\pgfqpoint{2.285162in}{0.704211in}}%
\pgfpathlineto{\pgfqpoint{2.289391in}{0.705859in}}%
\pgfpathlineto{\pgfqpoint{2.291506in}{0.703350in}}%
\pgfpathlineto{\pgfqpoint{2.293621in}{0.705420in}}%
\pgfpathlineto{\pgfqpoint{2.295735in}{0.700607in}}%
\pgfpathlineto{\pgfqpoint{2.297850in}{0.702765in}}%
\pgfpathlineto{\pgfqpoint{2.299965in}{0.694491in}}%
\pgfpathlineto{\pgfqpoint{2.304194in}{0.693609in}}%
\pgfpathlineto{\pgfqpoint{2.306309in}{0.694483in}}%
\pgfpathlineto{\pgfqpoint{2.308424in}{0.697000in}}%
\pgfpathlineto{\pgfqpoint{2.310538in}{0.690509in}}%
\pgfpathlineto{\pgfqpoint{2.312653in}{0.677933in}}%
\pgfpathlineto{\pgfqpoint{2.314768in}{0.673982in}}%
\pgfpathlineto{\pgfqpoint{2.316882in}{0.676193in}}%
\pgfpathlineto{\pgfqpoint{2.318997in}{0.681359in}}%
\pgfpathlineto{\pgfqpoint{2.321112in}{0.678051in}}%
\pgfpathlineto{\pgfqpoint{2.325341in}{0.667812in}}%
\pgfpathlineto{\pgfqpoint{2.327456in}{0.666821in}}%
\pgfpathlineto{\pgfqpoint{2.329570in}{0.669879in}}%
\pgfpathlineto{\pgfqpoint{2.331685in}{0.667027in}}%
\pgfpathlineto{\pgfqpoint{2.333800in}{0.661275in}}%
\pgfpathlineto{\pgfqpoint{2.335915in}{0.662598in}}%
\pgfpathlineto{\pgfqpoint{2.338029in}{0.673558in}}%
\pgfpathlineto{\pgfqpoint{2.340144in}{0.672706in}}%
\pgfpathlineto{\pgfqpoint{2.350717in}{0.640727in}}%
\pgfpathlineto{\pgfqpoint{2.354947in}{0.639025in}}%
\pgfpathlineto{\pgfqpoint{2.357061in}{0.634538in}}%
\pgfpathlineto{\pgfqpoint{2.359176in}{0.642303in}}%
\pgfpathlineto{\pgfqpoint{2.361291in}{0.645595in}}%
\pgfpathlineto{\pgfqpoint{2.363406in}{0.645109in}}%
\pgfpathlineto{\pgfqpoint{2.367635in}{0.649574in}}%
\pgfpathlineto{\pgfqpoint{2.369750in}{0.641674in}}%
\pgfpathlineto{\pgfqpoint{2.371864in}{0.645730in}}%
\pgfpathlineto{\pgfqpoint{2.376094in}{0.636818in}}%
\pgfpathlineto{\pgfqpoint{2.378208in}{0.640439in}}%
\pgfpathlineto{\pgfqpoint{2.380323in}{0.639619in}}%
\pgfpathlineto{\pgfqpoint{2.382438in}{0.644910in}}%
\pgfpathlineto{\pgfqpoint{2.384553in}{0.638098in}}%
\pgfpathlineto{\pgfqpoint{2.386667in}{0.642051in}}%
\pgfpathlineto{\pgfqpoint{2.388782in}{0.641794in}}%
\pgfpathlineto{\pgfqpoint{2.390897in}{0.643919in}}%
\pgfpathlineto{\pgfqpoint{2.393011in}{0.636650in}}%
\pgfpathlineto{\pgfqpoint{2.397241in}{0.654940in}}%
\pgfpathlineto{\pgfqpoint{2.399355in}{0.657453in}}%
\pgfpathlineto{\pgfqpoint{2.401470in}{0.656053in}}%
\pgfpathlineto{\pgfqpoint{2.403585in}{0.652102in}}%
\pgfpathlineto{\pgfqpoint{2.405699in}{0.653344in}}%
\pgfpathlineto{\pgfqpoint{2.409929in}{0.648760in}}%
\pgfpathlineto{\pgfqpoint{2.412044in}{0.640345in}}%
\pgfpathlineto{\pgfqpoint{2.414158in}{0.643451in}}%
\pgfpathlineto{\pgfqpoint{2.416273in}{0.637768in}}%
\pgfpathlineto{\pgfqpoint{2.418388in}{0.637774in}}%
\pgfpathlineto{\pgfqpoint{2.420502in}{0.639530in}}%
\pgfpathlineto{\pgfqpoint{2.422617in}{0.637318in}}%
\pgfpathlineto{\pgfqpoint{2.424732in}{0.640652in}}%
\pgfpathlineto{\pgfqpoint{2.428961in}{0.640057in}}%
\pgfpathlineto{\pgfqpoint{2.431076in}{0.645784in}}%
\pgfpathlineto{\pgfqpoint{2.433190in}{0.642810in}}%
\pgfpathlineto{\pgfqpoint{2.435305in}{0.642528in}}%
\pgfpathlineto{\pgfqpoint{2.437420in}{0.633987in}}%
\pgfpathlineto{\pgfqpoint{2.439535in}{0.632017in}}%
\pgfpathlineto{\pgfqpoint{2.441649in}{0.632895in}}%
\pgfpathlineto{\pgfqpoint{2.443764in}{0.635807in}}%
\pgfpathlineto{\pgfqpoint{2.445879in}{0.635456in}}%
\pgfpathlineto{\pgfqpoint{2.450108in}{0.626705in}}%
\pgfpathlineto{\pgfqpoint{2.452223in}{0.628165in}}%
\pgfpathlineto{\pgfqpoint{2.456452in}{0.639013in}}%
\pgfpathlineto{\pgfqpoint{2.458567in}{0.637621in}}%
\pgfpathlineto{\pgfqpoint{2.460681in}{0.626672in}}%
\pgfpathlineto{\pgfqpoint{2.462796in}{0.629090in}}%
\pgfpathlineto{\pgfqpoint{2.467026in}{0.624418in}}%
\pgfpathlineto{\pgfqpoint{2.469140in}{0.620269in}}%
\pgfpathlineto{\pgfqpoint{2.475484in}{0.626234in}}%
\pgfpathlineto{\pgfqpoint{2.477599in}{0.623450in}}%
\pgfpathlineto{\pgfqpoint{2.481828in}{0.633180in}}%
\pgfpathlineto{\pgfqpoint{2.483943in}{0.628428in}}%
\pgfpathlineto{\pgfqpoint{2.486058in}{0.628383in}}%
\pgfpathlineto{\pgfqpoint{2.488172in}{0.626493in}}%
\pgfpathlineto{\pgfqpoint{2.494517in}{0.612536in}}%
\pgfpathlineto{\pgfqpoint{2.496631in}{0.612366in}}%
\pgfpathlineto{\pgfqpoint{2.498746in}{0.616452in}}%
\pgfpathlineto{\pgfqpoint{2.500861in}{0.617299in}}%
\pgfpathlineto{\pgfqpoint{2.502975in}{0.614501in}}%
\pgfpathlineto{\pgfqpoint{2.505090in}{0.621443in}}%
\pgfpathlineto{\pgfqpoint{2.507205in}{0.619715in}}%
\pgfpathlineto{\pgfqpoint{2.509319in}{0.624319in}}%
\pgfpathlineto{\pgfqpoint{2.511434in}{0.624349in}}%
\pgfpathlineto{\pgfqpoint{2.513549in}{0.626249in}}%
\pgfpathlineto{\pgfqpoint{2.515664in}{0.621525in}}%
\pgfpathlineto{\pgfqpoint{2.517778in}{0.629189in}}%
\pgfpathlineto{\pgfqpoint{2.524122in}{0.620552in}}%
\pgfpathlineto{\pgfqpoint{2.526237in}{0.624675in}}%
\pgfpathlineto{\pgfqpoint{2.528352in}{0.625096in}}%
\pgfpathlineto{\pgfqpoint{2.530466in}{0.627163in}}%
\pgfpathlineto{\pgfqpoint{2.534696in}{0.626132in}}%
\pgfpathlineto{\pgfqpoint{2.538925in}{0.621892in}}%
\pgfpathlineto{\pgfqpoint{2.541040in}{0.623040in}}%
\pgfpathlineto{\pgfqpoint{2.551613in}{0.610437in}}%
\pgfpathlineto{\pgfqpoint{2.553728in}{0.615706in}}%
\pgfpathlineto{\pgfqpoint{2.555843in}{0.611828in}}%
\pgfpathlineto{\pgfqpoint{2.557957in}{0.610472in}}%
\pgfpathlineto{\pgfqpoint{2.564301in}{0.610614in}}%
\pgfpathlineto{\pgfqpoint{2.566416in}{0.616590in}}%
\pgfpathlineto{\pgfqpoint{2.568531in}{0.609590in}}%
\pgfpathlineto{\pgfqpoint{2.570646in}{0.611621in}}%
\pgfpathlineto{\pgfqpoint{2.574875in}{0.612446in}}%
\pgfpathlineto{\pgfqpoint{2.576990in}{0.619264in}}%
\pgfpathlineto{\pgfqpoint{2.579104in}{0.618558in}}%
\pgfpathlineto{\pgfqpoint{2.581219in}{0.612988in}}%
\pgfpathlineto{\pgfqpoint{2.583334in}{0.613568in}}%
\pgfpathlineto{\pgfqpoint{2.585448in}{0.619389in}}%
\pgfpathlineto{\pgfqpoint{2.587563in}{0.619540in}}%
\pgfpathlineto{\pgfqpoint{2.589678in}{0.632236in}}%
\pgfpathlineto{\pgfqpoint{2.591792in}{0.624518in}}%
\pgfpathlineto{\pgfqpoint{2.593907in}{0.624494in}}%
\pgfpathlineto{\pgfqpoint{2.596022in}{0.628771in}}%
\pgfpathlineto{\pgfqpoint{2.598137in}{0.621977in}}%
\pgfpathlineto{\pgfqpoint{2.600251in}{0.619358in}}%
\pgfpathlineto{\pgfqpoint{2.602366in}{0.624183in}}%
\pgfpathlineto{\pgfqpoint{2.604481in}{0.622987in}}%
\pgfpathlineto{\pgfqpoint{2.606595in}{0.628659in}}%
\pgfpathlineto{\pgfqpoint{2.608710in}{0.618944in}}%
\pgfpathlineto{\pgfqpoint{2.610825in}{0.622241in}}%
\pgfpathlineto{\pgfqpoint{2.615054in}{0.622519in}}%
\pgfpathlineto{\pgfqpoint{2.617169in}{0.620256in}}%
\pgfpathlineto{\pgfqpoint{2.619284in}{0.621880in}}%
\pgfpathlineto{\pgfqpoint{2.621398in}{0.619831in}}%
\pgfpathlineto{\pgfqpoint{2.623513in}{0.621663in}}%
\pgfpathlineto{\pgfqpoint{2.625628in}{0.619206in}}%
\pgfpathlineto{\pgfqpoint{2.627742in}{0.622081in}}%
\pgfpathlineto{\pgfqpoint{2.629857in}{0.622511in}}%
\pgfpathlineto{\pgfqpoint{2.631972in}{0.625140in}}%
\pgfpathlineto{\pgfqpoint{2.634086in}{0.625565in}}%
\pgfpathlineto{\pgfqpoint{2.636201in}{0.630597in}}%
\pgfpathlineto{\pgfqpoint{2.638316in}{0.632464in}}%
\pgfpathlineto{\pgfqpoint{2.640430in}{0.640593in}}%
\pgfpathlineto{\pgfqpoint{2.642545in}{0.638822in}}%
\pgfpathlineto{\pgfqpoint{2.644660in}{0.638971in}}%
\pgfpathlineto{\pgfqpoint{2.646775in}{0.634583in}}%
\pgfpathlineto{\pgfqpoint{2.648889in}{0.638765in}}%
\pgfpathlineto{\pgfqpoint{2.653119in}{0.626287in}}%
\pgfpathlineto{\pgfqpoint{2.657348in}{0.626381in}}%
\pgfpathlineto{\pgfqpoint{2.659463in}{0.622140in}}%
\pgfpathlineto{\pgfqpoint{2.661577in}{0.621859in}}%
\pgfpathlineto{\pgfqpoint{2.663692in}{0.619130in}}%
\pgfpathlineto{\pgfqpoint{2.665807in}{0.618517in}}%
\pgfpathlineto{\pgfqpoint{2.667921in}{0.622686in}}%
\pgfpathlineto{\pgfqpoint{2.670036in}{0.620689in}}%
\pgfpathlineto{\pgfqpoint{2.674266in}{0.614589in}}%
\pgfpathlineto{\pgfqpoint{2.676380in}{0.616561in}}%
\pgfpathlineto{\pgfqpoint{2.678495in}{0.614780in}}%
\pgfpathlineto{\pgfqpoint{2.680610in}{0.620841in}}%
\pgfpathlineto{\pgfqpoint{2.684839in}{0.622258in}}%
\pgfpathlineto{\pgfqpoint{2.686954in}{0.620136in}}%
\pgfpathlineto{\pgfqpoint{2.691183in}{0.629659in}}%
\pgfpathlineto{\pgfqpoint{2.695412in}{0.629661in}}%
\pgfpathlineto{\pgfqpoint{2.699642in}{0.649137in}}%
\pgfpathlineto{\pgfqpoint{2.701757in}{0.643927in}}%
\pgfpathlineto{\pgfqpoint{2.703871in}{0.644320in}}%
\pgfpathlineto{\pgfqpoint{2.705986in}{0.646100in}}%
\pgfpathlineto{\pgfqpoint{2.708101in}{0.645930in}}%
\pgfpathlineto{\pgfqpoint{2.710215in}{0.641909in}}%
\pgfpathlineto{\pgfqpoint{2.712330in}{0.644130in}}%
\pgfpathlineto{\pgfqpoint{2.714445in}{0.651592in}}%
\pgfpathlineto{\pgfqpoint{2.716559in}{0.649797in}}%
\pgfpathlineto{\pgfqpoint{2.718674in}{0.650854in}}%
\pgfpathlineto{\pgfqpoint{2.720789in}{0.654907in}}%
\pgfpathlineto{\pgfqpoint{2.725018in}{0.656647in}}%
\pgfpathlineto{\pgfqpoint{2.727133in}{0.668854in}}%
\pgfpathlineto{\pgfqpoint{2.729248in}{0.667058in}}%
\pgfpathlineto{\pgfqpoint{2.731362in}{0.661444in}}%
\pgfpathlineto{\pgfqpoint{2.733477in}{0.664084in}}%
\pgfpathlineto{\pgfqpoint{2.737706in}{0.652061in}}%
\pgfpathlineto{\pgfqpoint{2.739821in}{0.651081in}}%
\pgfpathlineto{\pgfqpoint{2.741936in}{0.648389in}}%
\pgfpathlineto{\pgfqpoint{2.744050in}{0.652232in}}%
\pgfpathlineto{\pgfqpoint{2.746165in}{0.653042in}}%
\pgfpathlineto{\pgfqpoint{2.748280in}{0.657810in}}%
\pgfpathlineto{\pgfqpoint{2.750395in}{0.655212in}}%
\pgfpathlineto{\pgfqpoint{2.752509in}{0.655470in}}%
\pgfpathlineto{\pgfqpoint{2.754624in}{0.654291in}}%
\pgfpathlineto{\pgfqpoint{2.758853in}{0.648407in}}%
\pgfpathlineto{\pgfqpoint{2.760968in}{0.647983in}}%
\pgfpathlineto{\pgfqpoint{2.763083in}{0.645639in}}%
\pgfpathlineto{\pgfqpoint{2.765197in}{0.651866in}}%
\pgfpathlineto{\pgfqpoint{2.767312in}{0.650312in}}%
\pgfpathlineto{\pgfqpoint{2.771541in}{0.638522in}}%
\pgfpathlineto{\pgfqpoint{2.773656in}{0.634456in}}%
\pgfpathlineto{\pgfqpoint{2.775771in}{0.637160in}}%
\pgfpathlineto{\pgfqpoint{2.777886in}{0.632954in}}%
\pgfpathlineto{\pgfqpoint{2.780000in}{0.634123in}}%
\pgfpathlineto{\pgfqpoint{2.782115in}{0.640429in}}%
\pgfpathlineto{\pgfqpoint{2.788459in}{0.626273in}}%
\pgfpathlineto{\pgfqpoint{2.790574in}{0.623057in}}%
\pgfpathlineto{\pgfqpoint{2.792688in}{0.627010in}}%
\pgfpathlineto{\pgfqpoint{2.799032in}{0.628675in}}%
\pgfpathlineto{\pgfqpoint{2.801147in}{0.620376in}}%
\pgfpathlineto{\pgfqpoint{2.805377in}{0.621961in}}%
\pgfpathlineto{\pgfqpoint{2.809606in}{0.614293in}}%
\pgfpathlineto{\pgfqpoint{2.811721in}{0.614260in}}%
\pgfpathlineto{\pgfqpoint{2.813835in}{0.616956in}}%
\pgfpathlineto{\pgfqpoint{2.815950in}{0.622544in}}%
\pgfpathlineto{\pgfqpoint{2.820179in}{0.626234in}}%
\pgfpathlineto{\pgfqpoint{2.822294in}{0.632672in}}%
\pgfpathlineto{\pgfqpoint{2.824409in}{0.627753in}}%
\pgfpathlineto{\pgfqpoint{2.826523in}{0.627064in}}%
\pgfpathlineto{\pgfqpoint{2.830753in}{0.618164in}}%
\pgfpathlineto{\pgfqpoint{2.832868in}{0.626200in}}%
\pgfpathlineto{\pgfqpoint{2.837097in}{0.627717in}}%
\pgfpathlineto{\pgfqpoint{2.839212in}{0.624783in}}%
\pgfpathlineto{\pgfqpoint{2.841326in}{0.634162in}}%
\pgfpathlineto{\pgfqpoint{2.843441in}{0.635442in}}%
\pgfpathlineto{\pgfqpoint{2.845556in}{0.632058in}}%
\pgfpathlineto{\pgfqpoint{2.849785in}{0.642682in}}%
\pgfpathlineto{\pgfqpoint{2.854015in}{0.645669in}}%
\pgfpathlineto{\pgfqpoint{2.856129in}{0.646285in}}%
\pgfpathlineto{\pgfqpoint{2.858244in}{0.651839in}}%
\pgfpathlineto{\pgfqpoint{2.862473in}{0.645664in}}%
\pgfpathlineto{\pgfqpoint{2.866703in}{0.643260in}}%
\pgfpathlineto{\pgfqpoint{2.868817in}{0.644549in}}%
\pgfpathlineto{\pgfqpoint{2.870932in}{0.651385in}}%
\pgfpathlineto{\pgfqpoint{2.873047in}{0.649369in}}%
\pgfpathlineto{\pgfqpoint{2.875161in}{0.650981in}}%
\pgfpathlineto{\pgfqpoint{2.877276in}{0.657098in}}%
\pgfpathlineto{\pgfqpoint{2.879391in}{0.657547in}}%
\pgfpathlineto{\pgfqpoint{2.883620in}{0.654218in}}%
\pgfpathlineto{\pgfqpoint{2.885735in}{0.657749in}}%
\pgfpathlineto{\pgfqpoint{2.889964in}{0.650728in}}%
\pgfpathlineto{\pgfqpoint{2.892079in}{0.655765in}}%
\pgfpathlineto{\pgfqpoint{2.896308in}{0.653063in}}%
\pgfpathlineto{\pgfqpoint{2.898423in}{0.655048in}}%
\pgfpathlineto{\pgfqpoint{2.904767in}{0.646197in}}%
\pgfpathlineto{\pgfqpoint{2.906882in}{0.649175in}}%
\pgfpathlineto{\pgfqpoint{2.908997in}{0.645658in}}%
\pgfpathlineto{\pgfqpoint{2.915341in}{0.655889in}}%
\pgfpathlineto{\pgfqpoint{2.917455in}{0.656331in}}%
\pgfpathlineto{\pgfqpoint{2.919570in}{0.659851in}}%
\pgfpathlineto{\pgfqpoint{2.921685in}{0.655668in}}%
\pgfpathlineto{\pgfqpoint{2.923799in}{0.656765in}}%
\pgfpathlineto{\pgfqpoint{2.925914in}{0.662279in}}%
\pgfpathlineto{\pgfqpoint{2.928029in}{0.661557in}}%
\pgfpathlineto{\pgfqpoint{2.930143in}{0.665073in}}%
\pgfpathlineto{\pgfqpoint{2.932258in}{0.658600in}}%
\pgfpathlineto{\pgfqpoint{2.934373in}{0.658761in}}%
\pgfpathlineto{\pgfqpoint{2.936488in}{0.652138in}}%
\pgfpathlineto{\pgfqpoint{2.938602in}{0.650264in}}%
\pgfpathlineto{\pgfqpoint{2.942832in}{0.653291in}}%
\pgfpathlineto{\pgfqpoint{2.947061in}{0.647735in}}%
\pgfpathlineto{\pgfqpoint{2.949176in}{0.647794in}}%
\pgfpathlineto{\pgfqpoint{2.951290in}{0.644339in}}%
\pgfpathlineto{\pgfqpoint{2.953405in}{0.650926in}}%
\pgfpathlineto{\pgfqpoint{2.955520in}{0.643979in}}%
\pgfpathlineto{\pgfqpoint{2.957634in}{0.646426in}}%
\pgfpathlineto{\pgfqpoint{2.959749in}{0.651142in}}%
\pgfpathlineto{\pgfqpoint{2.966093in}{0.643814in}}%
\pgfpathlineto{\pgfqpoint{2.968208in}{0.646743in}}%
\pgfpathlineto{\pgfqpoint{2.972437in}{0.643239in}}%
\pgfpathlineto{\pgfqpoint{2.974552in}{0.646081in}}%
\pgfpathlineto{\pgfqpoint{2.976667in}{0.644384in}}%
\pgfpathlineto{\pgfqpoint{2.980896in}{0.650871in}}%
\pgfpathlineto{\pgfqpoint{2.987240in}{0.665469in}}%
\pgfpathlineto{\pgfqpoint{2.989355in}{0.661902in}}%
\pgfpathlineto{\pgfqpoint{2.991470in}{0.662634in}}%
\pgfpathlineto{\pgfqpoint{2.993584in}{0.661489in}}%
\pgfpathlineto{\pgfqpoint{2.995699in}{0.662579in}}%
\pgfpathlineto{\pgfqpoint{2.999928in}{0.647752in}}%
\pgfpathlineto{\pgfqpoint{3.002043in}{0.648336in}}%
\pgfpathlineto{\pgfqpoint{3.004158in}{0.645599in}}%
\pgfpathlineto{\pgfqpoint{3.006272in}{0.644926in}}%
\pgfpathlineto{\pgfqpoint{3.008387in}{0.637892in}}%
\pgfpathlineto{\pgfqpoint{3.010502in}{0.645771in}}%
\pgfpathlineto{\pgfqpoint{3.012617in}{0.647901in}}%
\pgfpathlineto{\pgfqpoint{3.014731in}{0.652741in}}%
\pgfpathlineto{\pgfqpoint{3.016846in}{0.648655in}}%
\pgfpathlineto{\pgfqpoint{3.018961in}{0.656373in}}%
\pgfpathlineto{\pgfqpoint{3.021075in}{0.654957in}}%
\pgfpathlineto{\pgfqpoint{3.023190in}{0.656543in}}%
\pgfpathlineto{\pgfqpoint{3.027419in}{0.655184in}}%
\pgfpathlineto{\pgfqpoint{3.029534in}{0.653042in}}%
\pgfpathlineto{\pgfqpoint{3.031649in}{0.655750in}}%
\pgfpathlineto{\pgfqpoint{3.042222in}{0.639165in}}%
\pgfpathlineto{\pgfqpoint{3.044337in}{0.642878in}}%
\pgfpathlineto{\pgfqpoint{3.046452in}{0.644175in}}%
\pgfpathlineto{\pgfqpoint{3.048566in}{0.642400in}}%
\pgfpathlineto{\pgfqpoint{3.054910in}{0.648863in}}%
\pgfpathlineto{\pgfqpoint{3.057025in}{0.648395in}}%
\pgfpathlineto{\pgfqpoint{3.059140in}{0.644109in}}%
\pgfpathlineto{\pgfqpoint{3.061254in}{0.645369in}}%
\pgfpathlineto{\pgfqpoint{3.063369in}{0.649316in}}%
\pgfpathlineto{\pgfqpoint{3.065484in}{0.648805in}}%
\pgfpathlineto{\pgfqpoint{3.067599in}{0.656114in}}%
\pgfpathlineto{\pgfqpoint{3.069713in}{0.653846in}}%
\pgfpathlineto{\pgfqpoint{3.071828in}{0.646632in}}%
\pgfpathlineto{\pgfqpoint{3.073943in}{0.648426in}}%
\pgfpathlineto{\pgfqpoint{3.076057in}{0.641825in}}%
\pgfpathlineto{\pgfqpoint{3.078172in}{0.646349in}}%
\pgfpathlineto{\pgfqpoint{3.088746in}{0.621932in}}%
\pgfpathlineto{\pgfqpoint{3.090860in}{0.631589in}}%
\pgfpathlineto{\pgfqpoint{3.092975in}{0.634476in}}%
\pgfpathlineto{\pgfqpoint{3.099319in}{0.634956in}}%
\pgfpathlineto{\pgfqpoint{3.101434in}{0.632716in}}%
\pgfpathlineto{\pgfqpoint{3.103548in}{0.635966in}}%
\pgfpathlineto{\pgfqpoint{3.105663in}{0.633799in}}%
\pgfpathlineto{\pgfqpoint{3.107778in}{0.642059in}}%
\pgfpathlineto{\pgfqpoint{3.109892in}{0.639039in}}%
\pgfpathlineto{\pgfqpoint{3.114122in}{0.641560in}}%
\pgfpathlineto{\pgfqpoint{3.116237in}{0.631482in}}%
\pgfpathlineto{\pgfqpoint{3.118351in}{0.629116in}}%
\pgfpathlineto{\pgfqpoint{3.122581in}{0.619260in}}%
\pgfpathlineto{\pgfqpoint{3.124695in}{0.618797in}}%
\pgfpathlineto{\pgfqpoint{3.128925in}{0.615656in}}%
\pgfpathlineto{\pgfqpoint{3.131039in}{0.621748in}}%
\pgfpathlineto{\pgfqpoint{3.133154in}{0.623746in}}%
\pgfpathlineto{\pgfqpoint{3.139498in}{0.624148in}}%
\pgfpathlineto{\pgfqpoint{3.141613in}{0.617721in}}%
\pgfpathlineto{\pgfqpoint{3.143728in}{0.620065in}}%
\pgfpathlineto{\pgfqpoint{3.145842in}{0.615901in}}%
\pgfpathlineto{\pgfqpoint{3.150072in}{0.622515in}}%
\pgfpathlineto{\pgfqpoint{3.152186in}{0.616838in}}%
\pgfpathlineto{\pgfqpoint{3.154301in}{0.619841in}}%
\pgfpathlineto{\pgfqpoint{3.158530in}{0.616455in}}%
\pgfpathlineto{\pgfqpoint{3.160645in}{0.613436in}}%
\pgfpathlineto{\pgfqpoint{3.162760in}{0.614715in}}%
\pgfpathlineto{\pgfqpoint{3.164874in}{0.608035in}}%
\pgfpathlineto{\pgfqpoint{3.166989in}{0.617535in}}%
\pgfpathlineto{\pgfqpoint{3.169104in}{0.618044in}}%
\pgfpathlineto{\pgfqpoint{3.171219in}{0.614224in}}%
\pgfpathlineto{\pgfqpoint{3.173333in}{0.613340in}}%
\pgfpathlineto{\pgfqpoint{3.175448in}{0.618019in}}%
\pgfpathlineto{\pgfqpoint{3.177563in}{0.619033in}}%
\pgfpathlineto{\pgfqpoint{3.181792in}{0.624889in}}%
\pgfpathlineto{\pgfqpoint{3.188136in}{0.616795in}}%
\pgfpathlineto{\pgfqpoint{3.190251in}{0.620430in}}%
\pgfpathlineto{\pgfqpoint{3.192366in}{0.620586in}}%
\pgfpathlineto{\pgfqpoint{3.198710in}{0.629154in}}%
\pgfpathlineto{\pgfqpoint{3.200824in}{0.628658in}}%
\pgfpathlineto{\pgfqpoint{3.202939in}{0.623491in}}%
\pgfpathlineto{\pgfqpoint{3.207168in}{0.636043in}}%
\pgfpathlineto{\pgfqpoint{3.209283in}{0.628396in}}%
\pgfpathlineto{\pgfqpoint{3.211398in}{0.627362in}}%
\pgfpathlineto{\pgfqpoint{3.215627in}{0.635611in}}%
\pgfpathlineto{\pgfqpoint{3.217742in}{0.637548in}}%
\pgfpathlineto{\pgfqpoint{3.219857in}{0.632676in}}%
\pgfpathlineto{\pgfqpoint{3.221971in}{0.623341in}}%
\pgfpathlineto{\pgfqpoint{3.224086in}{0.628442in}}%
\pgfpathlineto{\pgfqpoint{3.228315in}{0.613512in}}%
\pgfpathlineto{\pgfqpoint{3.230430in}{0.610999in}}%
\pgfpathlineto{\pgfqpoint{3.232545in}{0.614676in}}%
\pgfpathlineto{\pgfqpoint{3.236774in}{0.602594in}}%
\pgfpathlineto{\pgfqpoint{3.238889in}{0.604337in}}%
\pgfpathlineto{\pgfqpoint{3.243118in}{0.604043in}}%
\pgfpathlineto{\pgfqpoint{3.245233in}{0.599036in}}%
\pgfpathlineto{\pgfqpoint{3.249462in}{0.605837in}}%
\pgfpathlineto{\pgfqpoint{3.251577in}{0.606533in}}%
\pgfpathlineto{\pgfqpoint{3.253692in}{0.601980in}}%
\pgfpathlineto{\pgfqpoint{3.255806in}{0.604087in}}%
\pgfpathlineto{\pgfqpoint{3.257921in}{0.602355in}}%
\pgfpathlineto{\pgfqpoint{3.260036in}{0.598490in}}%
\pgfpathlineto{\pgfqpoint{3.262150in}{0.603102in}}%
\pgfpathlineto{\pgfqpoint{3.264265in}{0.593091in}}%
\pgfpathlineto{\pgfqpoint{3.266380in}{0.597262in}}%
\pgfpathlineto{\pgfqpoint{3.268494in}{0.591411in}}%
\pgfpathlineto{\pgfqpoint{3.272724in}{0.601904in}}%
\pgfpathlineto{\pgfqpoint{3.276953in}{0.595820in}}%
\pgfpathlineto{\pgfqpoint{3.279068in}{0.598368in}}%
\pgfpathlineto{\pgfqpoint{3.281183in}{0.594465in}}%
\pgfpathlineto{\pgfqpoint{3.283297in}{0.598925in}}%
\pgfpathlineto{\pgfqpoint{3.285412in}{0.598314in}}%
\pgfpathlineto{\pgfqpoint{3.287527in}{0.601363in}}%
\pgfpathlineto{\pgfqpoint{3.291756in}{0.586453in}}%
\pgfpathlineto{\pgfqpoint{3.293871in}{0.591832in}}%
\pgfpathlineto{\pgfqpoint{3.300215in}{0.564742in}}%
\pgfpathlineto{\pgfqpoint{3.302330in}{0.573053in}}%
\pgfpathlineto{\pgfqpoint{3.308674in}{0.571784in}}%
\pgfpathlineto{\pgfqpoint{3.312903in}{0.575153in}}%
\pgfpathlineto{\pgfqpoint{3.315018in}{0.579491in}}%
\pgfpathlineto{\pgfqpoint{3.317132in}{0.577226in}}%
\pgfpathlineto{\pgfqpoint{3.321362in}{0.567566in}}%
\pgfpathlineto{\pgfqpoint{3.323477in}{0.568105in}}%
\pgfpathlineto{\pgfqpoint{3.327706in}{0.565847in}}%
\pgfpathlineto{\pgfqpoint{3.329821in}{0.561647in}}%
\pgfpathlineto{\pgfqpoint{3.331935in}{0.560795in}}%
\pgfpathlineto{\pgfqpoint{3.334050in}{0.551077in}}%
\pgfpathlineto{\pgfqpoint{3.336165in}{0.554558in}}%
\pgfpathlineto{\pgfqpoint{3.338279in}{0.554095in}}%
\pgfpathlineto{\pgfqpoint{3.340394in}{0.552101in}}%
\pgfpathlineto{\pgfqpoint{3.344623in}{0.533831in}}%
\pgfpathlineto{\pgfqpoint{3.346738in}{0.542375in}}%
\pgfpathlineto{\pgfqpoint{3.348853in}{0.536997in}}%
\pgfpathlineto{\pgfqpoint{3.350968in}{0.537505in}}%
\pgfpathlineto{\pgfqpoint{3.353082in}{0.534283in}}%
\pgfpathlineto{\pgfqpoint{3.355197in}{0.539650in}}%
\pgfpathlineto{\pgfqpoint{3.357312in}{0.535856in}}%
\pgfpathlineto{\pgfqpoint{3.359426in}{0.535397in}}%
\pgfpathlineto{\pgfqpoint{3.361541in}{0.533301in}}%
\pgfpathlineto{\pgfqpoint{3.363656in}{0.540443in}}%
\pgfpathlineto{\pgfqpoint{3.365770in}{0.534589in}}%
\pgfpathlineto{\pgfqpoint{3.367885in}{0.534021in}}%
\pgfpathlineto{\pgfqpoint{3.370000in}{0.534925in}}%
\pgfpathlineto{\pgfqpoint{3.372114in}{0.529920in}}%
\pgfpathlineto{\pgfqpoint{3.374229in}{0.531601in}}%
\pgfpathlineto{\pgfqpoint{3.378459in}{0.526513in}}%
\pgfpathlineto{\pgfqpoint{3.384803in}{0.520547in}}%
\pgfpathlineto{\pgfqpoint{3.386917in}{0.524196in}}%
\pgfpathlineto{\pgfqpoint{3.389032in}{0.523153in}}%
\pgfpathlineto{\pgfqpoint{3.391147in}{0.525259in}}%
\pgfpathlineto{\pgfqpoint{3.395376in}{0.517780in}}%
\pgfpathlineto{\pgfqpoint{3.397491in}{0.521322in}}%
\pgfpathlineto{\pgfqpoint{3.399605in}{0.518218in}}%
\pgfpathlineto{\pgfqpoint{3.401720in}{0.519401in}}%
\pgfpathlineto{\pgfqpoint{3.405950in}{0.514149in}}%
\pgfpathlineto{\pgfqpoint{3.408064in}{0.509549in}}%
\pgfpathlineto{\pgfqpoint{3.410179in}{0.513143in}}%
\pgfpathlineto{\pgfqpoint{3.412294in}{0.510888in}}%
\pgfpathlineto{\pgfqpoint{3.414408in}{0.512025in}}%
\pgfpathlineto{\pgfqpoint{3.416523in}{0.514857in}}%
\pgfpathlineto{\pgfqpoint{3.420752in}{0.513236in}}%
\pgfpathlineto{\pgfqpoint{3.422867in}{0.520271in}}%
\pgfpathlineto{\pgfqpoint{3.427097in}{0.524285in}}%
\pgfpathlineto{\pgfqpoint{3.429211in}{0.519176in}}%
\pgfpathlineto{\pgfqpoint{3.431326in}{0.523209in}}%
\pgfpathlineto{\pgfqpoint{3.437670in}{0.509364in}}%
\pgfpathlineto{\pgfqpoint{3.441899in}{0.522748in}}%
\pgfpathlineto{\pgfqpoint{3.446129in}{0.521180in}}%
\pgfpathlineto{\pgfqpoint{3.450358in}{0.520226in}}%
\pgfpathlineto{\pgfqpoint{3.454588in}{0.518737in}}%
\pgfpathlineto{\pgfqpoint{3.460932in}{0.516241in}}%
\pgfpathlineto{\pgfqpoint{3.463046in}{0.509609in}}%
\pgfpathlineto{\pgfqpoint{3.469390in}{0.518095in}}%
\pgfpathlineto{\pgfqpoint{3.471505in}{0.517807in}}%
\pgfpathlineto{\pgfqpoint{3.475734in}{0.511995in}}%
\pgfpathlineto{\pgfqpoint{3.477849in}{0.514874in}}%
\pgfpathlineto{\pgfqpoint{3.479964in}{0.512464in}}%
\pgfpathlineto{\pgfqpoint{3.484193in}{0.511752in}}%
\pgfpathlineto{\pgfqpoint{3.486308in}{0.514650in}}%
\pgfpathlineto{\pgfqpoint{3.488423in}{0.514370in}}%
\pgfpathlineto{\pgfqpoint{3.490537in}{0.521827in}}%
\pgfpathlineto{\pgfqpoint{3.498996in}{0.522562in}}%
\pgfpathlineto{\pgfqpoint{3.501111in}{0.516167in}}%
\pgfpathlineto{\pgfqpoint{3.503225in}{0.522568in}}%
\pgfpathlineto{\pgfqpoint{3.505340in}{0.524624in}}%
\pgfpathlineto{\pgfqpoint{3.507455in}{0.532328in}}%
\pgfpathlineto{\pgfqpoint{3.509570in}{0.532070in}}%
\pgfpathlineto{\pgfqpoint{3.511684in}{0.535232in}}%
\pgfpathlineto{\pgfqpoint{3.513799in}{0.534690in}}%
\pgfpathlineto{\pgfqpoint{3.515914in}{0.524709in}}%
\pgfpathlineto{\pgfqpoint{3.522258in}{0.514409in}}%
\pgfpathlineto{\pgfqpoint{3.524372in}{0.519514in}}%
\pgfpathlineto{\pgfqpoint{3.526487in}{0.518793in}}%
\pgfpathlineto{\pgfqpoint{3.528602in}{0.516253in}}%
\pgfpathlineto{\pgfqpoint{3.532831in}{0.530417in}}%
\pgfpathlineto{\pgfqpoint{3.534946in}{0.527580in}}%
\pgfpathlineto{\pgfqpoint{3.537061in}{0.531473in}}%
\pgfpathlineto{\pgfqpoint{3.539175in}{0.532694in}}%
\pgfpathlineto{\pgfqpoint{3.541290in}{0.537721in}}%
\pgfpathlineto{\pgfqpoint{3.543405in}{0.532087in}}%
\pgfpathlineto{\pgfqpoint{3.545519in}{0.533439in}}%
\pgfpathlineto{\pgfqpoint{3.551863in}{0.529394in}}%
\pgfpathlineto{\pgfqpoint{3.553978in}{0.535519in}}%
\pgfpathlineto{\pgfqpoint{3.556093in}{0.528731in}}%
\pgfpathlineto{\pgfqpoint{3.558208in}{0.532359in}}%
\pgfpathlineto{\pgfqpoint{3.560322in}{0.531935in}}%
\pgfpathlineto{\pgfqpoint{3.562437in}{0.535125in}}%
\pgfpathlineto{\pgfqpoint{3.564552in}{0.534884in}}%
\pgfpathlineto{\pgfqpoint{3.566666in}{0.536379in}}%
\pgfpathlineto{\pgfqpoint{3.568781in}{0.535343in}}%
\pgfpathlineto{\pgfqpoint{3.573010in}{0.528294in}}%
\pgfpathlineto{\pgfqpoint{3.575125in}{0.530438in}}%
\pgfpathlineto{\pgfqpoint{3.577240in}{0.528651in}}%
\pgfpathlineto{\pgfqpoint{3.579354in}{0.519839in}}%
\pgfpathlineto{\pgfqpoint{3.585699in}{0.514575in}}%
\pgfpathlineto{\pgfqpoint{3.587813in}{0.514602in}}%
\pgfpathlineto{\pgfqpoint{3.589928in}{0.512830in}}%
\pgfpathlineto{\pgfqpoint{3.592043in}{0.512781in}}%
\pgfpathlineto{\pgfqpoint{3.594157in}{0.514289in}}%
\pgfpathlineto{\pgfqpoint{3.598387in}{0.506869in}}%
\pgfpathlineto{\pgfqpoint{3.600501in}{0.507334in}}%
\pgfpathlineto{\pgfqpoint{3.602616in}{0.503204in}}%
\pgfpathlineto{\pgfqpoint{3.604731in}{0.504970in}}%
\pgfpathlineto{\pgfqpoint{3.608960in}{0.497668in}}%
\pgfpathlineto{\pgfqpoint{3.611075in}{0.491922in}}%
\pgfpathlineto{\pgfqpoint{3.613190in}{0.505572in}}%
\pgfpathlineto{\pgfqpoint{3.619534in}{0.498125in}}%
\pgfpathlineto{\pgfqpoint{3.623763in}{0.508627in}}%
\pgfpathlineto{\pgfqpoint{3.627992in}{0.505964in}}%
\pgfpathlineto{\pgfqpoint{3.630107in}{0.503646in}}%
\pgfpathlineto{\pgfqpoint{3.634336in}{0.504954in}}%
\pgfpathlineto{\pgfqpoint{3.636451in}{0.495360in}}%
\pgfpathlineto{\pgfqpoint{3.638566in}{0.498496in}}%
\pgfpathlineto{\pgfqpoint{3.640681in}{0.497541in}}%
\pgfpathlineto{\pgfqpoint{3.642795in}{0.502111in}}%
\pgfpathlineto{\pgfqpoint{3.647025in}{0.493285in}}%
\pgfpathlineto{\pgfqpoint{3.649139in}{0.498919in}}%
\pgfpathlineto{\pgfqpoint{3.651254in}{0.500162in}}%
\pgfpathlineto{\pgfqpoint{3.653369in}{0.509443in}}%
\pgfpathlineto{\pgfqpoint{3.655483in}{0.511250in}}%
\pgfpathlineto{\pgfqpoint{3.659713in}{0.501032in}}%
\pgfpathlineto{\pgfqpoint{3.661828in}{0.500525in}}%
\pgfpathlineto{\pgfqpoint{3.663942in}{0.496261in}}%
\pgfpathlineto{\pgfqpoint{3.666057in}{0.487739in}}%
\pgfpathlineto{\pgfqpoint{3.668172in}{0.487264in}}%
\pgfpathlineto{\pgfqpoint{3.670286in}{0.480991in}}%
\pgfpathlineto{\pgfqpoint{3.672401in}{0.486972in}}%
\pgfpathlineto{\pgfqpoint{3.676630in}{0.485671in}}%
\pgfpathlineto{\pgfqpoint{3.678745in}{0.476613in}}%
\pgfpathlineto{\pgfqpoint{3.680860in}{0.476780in}}%
\pgfpathlineto{\pgfqpoint{3.682974in}{0.479372in}}%
\pgfpathlineto{\pgfqpoint{3.687204in}{0.466914in}}%
\pgfpathlineto{\pgfqpoint{3.691433in}{0.469487in}}%
\pgfpathlineto{\pgfqpoint{3.693548in}{0.465094in}}%
\pgfpathlineto{\pgfqpoint{3.695663in}{0.464740in}}%
\pgfpathlineto{\pgfqpoint{3.699892in}{0.467714in}}%
\pgfpathlineto{\pgfqpoint{3.702007in}{0.466431in}}%
\pgfpathlineto{\pgfqpoint{3.704121in}{0.468502in}}%
\pgfpathlineto{\pgfqpoint{3.706236in}{0.459689in}}%
\pgfpathlineto{\pgfqpoint{3.708351in}{0.462829in}}%
\pgfpathlineto{\pgfqpoint{3.710465in}{0.454295in}}%
\pgfpathlineto{\pgfqpoint{3.712580in}{0.457334in}}%
\pgfpathlineto{\pgfqpoint{3.714695in}{0.456275in}}%
\pgfpathlineto{\pgfqpoint{3.716810in}{0.451327in}}%
\pgfpathlineto{\pgfqpoint{3.718924in}{0.441787in}}%
\pgfpathlineto{\pgfqpoint{3.721039in}{0.442510in}}%
\pgfpathlineto{\pgfqpoint{3.725268in}{0.437774in}}%
\pgfpathlineto{\pgfqpoint{3.727383in}{0.438230in}}%
\pgfpathlineto{\pgfqpoint{3.729498in}{0.442287in}}%
\pgfpathlineto{\pgfqpoint{3.733727in}{0.454286in}}%
\pgfpathlineto{\pgfqpoint{3.735842in}{0.449882in}}%
\pgfpathlineto{\pgfqpoint{3.740071in}{0.464413in}}%
\pgfpathlineto{\pgfqpoint{3.742186in}{0.464200in}}%
\pgfpathlineto{\pgfqpoint{3.744301in}{0.468241in}}%
\pgfpathlineto{\pgfqpoint{3.748530in}{0.466433in}}%
\pgfpathlineto{\pgfqpoint{3.756989in}{0.481167in}}%
\pgfpathlineto{\pgfqpoint{3.759103in}{0.478972in}}%
\pgfpathlineto{\pgfqpoint{3.761218in}{0.479021in}}%
\pgfpathlineto{\pgfqpoint{3.763333in}{0.472797in}}%
\pgfpathlineto{\pgfqpoint{3.765447in}{0.478209in}}%
\pgfpathlineto{\pgfqpoint{3.767562in}{0.471831in}}%
\pgfpathlineto{\pgfqpoint{3.769677in}{0.474206in}}%
\pgfpathlineto{\pgfqpoint{3.771792in}{0.472300in}}%
\pgfpathlineto{\pgfqpoint{3.773906in}{0.476536in}}%
\pgfpathlineto{\pgfqpoint{3.778136in}{0.467754in}}%
\pgfpathlineto{\pgfqpoint{3.780250in}{0.471432in}}%
\pgfpathlineto{\pgfqpoint{3.782365in}{0.466345in}}%
\pgfpathlineto{\pgfqpoint{3.784480in}{0.466837in}}%
\pgfpathlineto{\pgfqpoint{3.786594in}{0.468798in}}%
\pgfpathlineto{\pgfqpoint{3.788709in}{0.467588in}}%
\pgfpathlineto{\pgfqpoint{3.790824in}{0.471990in}}%
\pgfpathlineto{\pgfqpoint{3.795053in}{0.470197in}}%
\pgfpathlineto{\pgfqpoint{3.799283in}{0.474596in}}%
\pgfpathlineto{\pgfqpoint{3.803512in}{0.470942in}}%
\pgfpathlineto{\pgfqpoint{3.807741in}{0.480525in}}%
\pgfpathlineto{\pgfqpoint{3.809856in}{0.476987in}}%
\pgfpathlineto{\pgfqpoint{3.814085in}{0.465851in}}%
\pgfpathlineto{\pgfqpoint{3.818315in}{0.468298in}}%
\pgfpathlineto{\pgfqpoint{3.820430in}{0.472206in}}%
\pgfpathlineto{\pgfqpoint{3.822544in}{0.471252in}}%
\pgfpathlineto{\pgfqpoint{3.824659in}{0.474041in}}%
\pgfpathlineto{\pgfqpoint{3.826774in}{0.470157in}}%
\pgfpathlineto{\pgfqpoint{3.828888in}{0.476013in}}%
\pgfpathlineto{\pgfqpoint{3.831003in}{0.475556in}}%
\pgfpathlineto{\pgfqpoint{3.835232in}{0.471343in}}%
\pgfpathlineto{\pgfqpoint{3.837347in}{0.466900in}}%
\pgfpathlineto{\pgfqpoint{3.839462in}{0.467138in}}%
\pgfpathlineto{\pgfqpoint{3.841576in}{0.462031in}}%
\pgfpathlineto{\pgfqpoint{3.843691in}{0.463133in}}%
\pgfpathlineto{\pgfqpoint{3.847921in}{0.458912in}}%
\pgfpathlineto{\pgfqpoint{3.850035in}{0.459674in}}%
\pgfpathlineto{\pgfqpoint{3.852150in}{0.456852in}}%
\pgfpathlineto{\pgfqpoint{3.854265in}{0.462900in}}%
\pgfpathlineto{\pgfqpoint{3.856379in}{0.463291in}}%
\pgfpathlineto{\pgfqpoint{3.858494in}{0.461537in}}%
\pgfpathlineto{\pgfqpoint{3.862723in}{0.454416in}}%
\pgfpathlineto{\pgfqpoint{3.864838in}{0.455453in}}%
\pgfpathlineto{\pgfqpoint{3.866953in}{0.459605in}}%
\pgfpathlineto{\pgfqpoint{3.871182in}{0.452802in}}%
\pgfpathlineto{\pgfqpoint{3.873297in}{0.454280in}}%
\pgfpathlineto{\pgfqpoint{3.875412in}{0.451538in}}%
\pgfpathlineto{\pgfqpoint{3.877526in}{0.453796in}}%
\pgfpathlineto{\pgfqpoint{3.879641in}{0.447794in}}%
\pgfpathlineto{\pgfqpoint{3.881756in}{0.453336in}}%
\pgfpathlineto{\pgfqpoint{3.883870in}{0.453215in}}%
\pgfpathlineto{\pgfqpoint{3.885985in}{0.456791in}}%
\pgfpathlineto{\pgfqpoint{3.890214in}{0.468379in}}%
\pgfpathlineto{\pgfqpoint{3.892329in}{0.470767in}}%
\pgfpathlineto{\pgfqpoint{3.894444in}{0.475309in}}%
\pgfpathlineto{\pgfqpoint{3.896559in}{0.476576in}}%
\pgfpathlineto{\pgfqpoint{3.898673in}{0.484657in}}%
\pgfpathlineto{\pgfqpoint{3.900788in}{0.487948in}}%
\pgfpathlineto{\pgfqpoint{3.905017in}{0.468870in}}%
\pgfpathlineto{\pgfqpoint{3.907132in}{0.463521in}}%
\pgfpathlineto{\pgfqpoint{3.911361in}{0.464879in}}%
\pgfpathlineto{\pgfqpoint{3.915591in}{0.460678in}}%
\pgfpathlineto{\pgfqpoint{3.919820in}{0.457227in}}%
\pgfpathlineto{\pgfqpoint{3.921935in}{0.457876in}}%
\pgfpathlineto{\pgfqpoint{3.926164in}{0.450536in}}%
\pgfpathlineto{\pgfqpoint{3.930394in}{0.450816in}}%
\pgfpathlineto{\pgfqpoint{3.932508in}{0.445454in}}%
\pgfpathlineto{\pgfqpoint{3.934623in}{0.450031in}}%
\pgfpathlineto{\pgfqpoint{3.936738in}{0.447200in}}%
\pgfpathlineto{\pgfqpoint{3.938852in}{0.446728in}}%
\pgfpathlineto{\pgfqpoint{3.945196in}{0.454175in}}%
\pgfpathlineto{\pgfqpoint{3.947311in}{0.449610in}}%
\pgfpathlineto{\pgfqpoint{3.949426in}{0.451029in}}%
\pgfpathlineto{\pgfqpoint{3.951541in}{0.456026in}}%
\pgfpathlineto{\pgfqpoint{3.953655in}{0.457985in}}%
\pgfpathlineto{\pgfqpoint{3.955770in}{0.454769in}}%
\pgfpathlineto{\pgfqpoint{3.957885in}{0.456848in}}%
\pgfpathlineto{\pgfqpoint{3.964229in}{0.455190in}}%
\pgfpathlineto{\pgfqpoint{3.966343in}{0.457937in}}%
\pgfpathlineto{\pgfqpoint{3.968458in}{0.464816in}}%
\pgfpathlineto{\pgfqpoint{3.970573in}{0.461067in}}%
\pgfpathlineto{\pgfqpoint{3.972687in}{0.453418in}}%
\pgfpathlineto{\pgfqpoint{3.974802in}{0.455191in}}%
\pgfpathlineto{\pgfqpoint{3.976917in}{0.447424in}}%
\pgfpathlineto{\pgfqpoint{3.979032in}{0.455087in}}%
\pgfpathlineto{\pgfqpoint{3.981146in}{0.454250in}}%
\pgfpathlineto{\pgfqpoint{3.985376in}{0.444442in}}%
\pgfpathlineto{\pgfqpoint{3.987490in}{0.441853in}}%
\pgfpathlineto{\pgfqpoint{3.991720in}{0.445815in}}%
\pgfpathlineto{\pgfqpoint{3.993834in}{0.440104in}}%
\pgfpathlineto{\pgfqpoint{3.995949in}{0.442956in}}%
\pgfpathlineto{\pgfqpoint{3.998064in}{0.441997in}}%
\pgfpathlineto{\pgfqpoint{4.004408in}{0.429445in}}%
\pgfpathlineto{\pgfqpoint{4.006523in}{0.431923in}}%
\pgfpathlineto{\pgfqpoint{4.008637in}{0.429830in}}%
\pgfpathlineto{\pgfqpoint{4.010752in}{0.423673in}}%
\pgfpathlineto{\pgfqpoint{4.012867in}{0.427628in}}%
\pgfpathlineto{\pgfqpoint{4.014981in}{0.426700in}}%
\pgfpathlineto{\pgfqpoint{4.019211in}{0.418155in}}%
\pgfpathlineto{\pgfqpoint{4.023440in}{0.425160in}}%
\pgfpathlineto{\pgfqpoint{4.025555in}{0.429629in}}%
\pgfpathlineto{\pgfqpoint{4.027670in}{0.427755in}}%
\pgfpathlineto{\pgfqpoint{4.029784in}{0.433547in}}%
\pgfpathlineto{\pgfqpoint{4.034014in}{0.435820in}}%
\pgfpathlineto{\pgfqpoint{4.036128in}{0.434918in}}%
\pgfpathlineto{\pgfqpoint{4.040358in}{0.438184in}}%
\pgfpathlineto{\pgfqpoint{4.042472in}{0.445604in}}%
\pgfpathlineto{\pgfqpoint{4.044587in}{0.446557in}}%
\pgfpathlineto{\pgfqpoint{4.046702in}{0.444629in}}%
\pgfpathlineto{\pgfqpoint{4.048816in}{0.441056in}}%
\pgfpathlineto{\pgfqpoint{4.055161in}{0.455059in}}%
\pgfpathlineto{\pgfqpoint{4.057275in}{0.450546in}}%
\pgfpathlineto{\pgfqpoint{4.061505in}{0.454860in}}%
\pgfpathlineto{\pgfqpoint{4.072078in}{0.474856in}}%
\pgfpathlineto{\pgfqpoint{4.074193in}{0.477680in}}%
\pgfpathlineto{\pgfqpoint{4.076307in}{0.471593in}}%
\pgfpathlineto{\pgfqpoint{4.078422in}{0.470368in}}%
\pgfpathlineto{\pgfqpoint{4.080537in}{0.466598in}}%
\pgfpathlineto{\pgfqpoint{4.082652in}{0.466087in}}%
\pgfpathlineto{\pgfqpoint{4.084766in}{0.468022in}}%
\pgfpathlineto{\pgfqpoint{4.086881in}{0.463043in}}%
\pgfpathlineto{\pgfqpoint{4.088996in}{0.464390in}}%
\pgfpathlineto{\pgfqpoint{4.091110in}{0.467566in}}%
\pgfpathlineto{\pgfqpoint{4.093225in}{0.475106in}}%
\pgfpathlineto{\pgfqpoint{4.097454in}{0.473281in}}%
\pgfpathlineto{\pgfqpoint{4.099569in}{0.472377in}}%
\pgfpathlineto{\pgfqpoint{4.101684in}{0.481197in}}%
\pgfpathlineto{\pgfqpoint{4.108028in}{0.465200in}}%
\pgfpathlineto{\pgfqpoint{4.110143in}{0.460709in}}%
\pgfpathlineto{\pgfqpoint{4.112257in}{0.464369in}}%
\pgfpathlineto{\pgfqpoint{4.116487in}{0.474900in}}%
\pgfpathlineto{\pgfqpoint{4.118601in}{0.466436in}}%
\pgfpathlineto{\pgfqpoint{4.120716in}{0.467756in}}%
\pgfpathlineto{\pgfqpoint{4.122831in}{0.464325in}}%
\pgfpathlineto{\pgfqpoint{4.124945in}{0.455487in}}%
\pgfpathlineto{\pgfqpoint{4.129175in}{0.451963in}}%
\pgfpathlineto{\pgfqpoint{4.131290in}{0.458545in}}%
\pgfpathlineto{\pgfqpoint{4.133404in}{0.458186in}}%
\pgfpathlineto{\pgfqpoint{4.135519in}{0.455740in}}%
\pgfpathlineto{\pgfqpoint{4.137634in}{0.458416in}}%
\pgfpathlineto{\pgfqpoint{4.139748in}{0.455520in}}%
\pgfpathlineto{\pgfqpoint{4.141863in}{0.456875in}}%
\pgfpathlineto{\pgfqpoint{4.143978in}{0.447748in}}%
\pgfpathlineto{\pgfqpoint{4.146092in}{0.449048in}}%
\pgfpathlineto{\pgfqpoint{4.148207in}{0.458043in}}%
\pgfpathlineto{\pgfqpoint{4.150322in}{0.454100in}}%
\pgfpathlineto{\pgfqpoint{4.154551in}{0.452927in}}%
\pgfpathlineto{\pgfqpoint{4.158781in}{0.438804in}}%
\pgfpathlineto{\pgfqpoint{4.160895in}{0.438807in}}%
\pgfpathlineto{\pgfqpoint{4.167239in}{0.420348in}}%
\pgfpathlineto{\pgfqpoint{4.175698in}{0.427713in}}%
\pgfpathlineto{\pgfqpoint{4.177813in}{0.426218in}}%
\pgfpathlineto{\pgfqpoint{4.179927in}{0.429799in}}%
\pgfpathlineto{\pgfqpoint{4.186272in}{0.422656in}}%
\pgfpathlineto{\pgfqpoint{4.190501in}{0.420332in}}%
\pgfpathlineto{\pgfqpoint{4.192616in}{0.415591in}}%
\pgfpathlineto{\pgfqpoint{4.194730in}{0.414430in}}%
\pgfpathlineto{\pgfqpoint{4.196845in}{0.411357in}}%
\pgfpathlineto{\pgfqpoint{4.198960in}{0.411160in}}%
\pgfpathlineto{\pgfqpoint{4.201074in}{0.405702in}}%
\pgfpathlineto{\pgfqpoint{4.203189in}{0.406968in}}%
\pgfpathlineto{\pgfqpoint{4.205304in}{0.416257in}}%
\pgfpathlineto{\pgfqpoint{4.207418in}{0.417145in}}%
\pgfpathlineto{\pgfqpoint{4.209533in}{0.408817in}}%
\pgfpathlineto{\pgfqpoint{4.211648in}{0.414839in}}%
\pgfpathlineto{\pgfqpoint{4.213763in}{0.411846in}}%
\pgfpathlineto{\pgfqpoint{4.217992in}{0.409807in}}%
\pgfpathlineto{\pgfqpoint{4.220107in}{0.415180in}}%
\pgfpathlineto{\pgfqpoint{4.222221in}{0.417418in}}%
\pgfpathlineto{\pgfqpoint{4.224336in}{0.412158in}}%
\pgfpathlineto{\pgfqpoint{4.226451in}{0.412585in}}%
\pgfpathlineto{\pgfqpoint{4.228565in}{0.406228in}}%
\pgfpathlineto{\pgfqpoint{4.232795in}{0.415978in}}%
\pgfpathlineto{\pgfqpoint{4.234910in}{0.416061in}}%
\pgfpathlineto{\pgfqpoint{4.237024in}{0.412186in}}%
\pgfpathlineto{\pgfqpoint{4.239139in}{0.410810in}}%
\pgfpathlineto{\pgfqpoint{4.243368in}{0.411605in}}%
\pgfpathlineto{\pgfqpoint{4.247598in}{0.400638in}}%
\pgfpathlineto{\pgfqpoint{4.251827in}{0.402683in}}%
\pgfpathlineto{\pgfqpoint{4.258171in}{0.412653in}}%
\pgfpathlineto{\pgfqpoint{4.260286in}{0.409659in}}%
\pgfpathlineto{\pgfqpoint{4.262401in}{0.416233in}}%
\pgfpathlineto{\pgfqpoint{4.264515in}{0.412731in}}%
\pgfpathlineto{\pgfqpoint{4.266630in}{0.414367in}}%
\pgfpathlineto{\pgfqpoint{4.268745in}{0.422397in}}%
\pgfpathlineto{\pgfqpoint{4.272974in}{0.414379in}}%
\pgfpathlineto{\pgfqpoint{4.275089in}{0.414806in}}%
\pgfpathlineto{\pgfqpoint{4.279318in}{0.406804in}}%
\pgfpathlineto{\pgfqpoint{4.283547in}{0.404485in}}%
\pgfpathlineto{\pgfqpoint{4.285662in}{0.408333in}}%
\pgfpathlineto{\pgfqpoint{4.287777in}{0.405863in}}%
\pgfpathlineto{\pgfqpoint{4.289892in}{0.411396in}}%
\pgfpathlineto{\pgfqpoint{4.292006in}{0.411349in}}%
\pgfpathlineto{\pgfqpoint{4.294121in}{0.405978in}}%
\pgfpathlineto{\pgfqpoint{4.296236in}{0.412123in}}%
\pgfpathlineto{\pgfqpoint{4.298350in}{0.413654in}}%
\pgfpathlineto{\pgfqpoint{4.302580in}{0.411962in}}%
\pgfpathlineto{\pgfqpoint{4.304694in}{0.412987in}}%
\pgfpathlineto{\pgfqpoint{4.306809in}{0.411221in}}%
\pgfpathlineto{\pgfqpoint{4.313153in}{0.382050in}}%
\pgfpathlineto{\pgfqpoint{4.315268in}{0.381673in}}%
\pgfpathlineto{\pgfqpoint{4.317383in}{0.379826in}}%
\pgfpathlineto{\pgfqpoint{4.319497in}{0.384938in}}%
\pgfpathlineto{\pgfqpoint{4.321612in}{0.385580in}}%
\pgfpathlineto{\pgfqpoint{4.323727in}{0.383568in}}%
\pgfpathlineto{\pgfqpoint{4.327956in}{0.392261in}}%
\pgfpathlineto{\pgfqpoint{4.330071in}{0.398907in}}%
\pgfpathlineto{\pgfqpoint{4.332185in}{0.399949in}}%
\pgfpathlineto{\pgfqpoint{4.334300in}{0.394291in}}%
\pgfpathlineto{\pgfqpoint{4.336415in}{0.396816in}}%
\pgfpathlineto{\pgfqpoint{4.338529in}{0.391085in}}%
\pgfpathlineto{\pgfqpoint{4.342759in}{0.394565in}}%
\pgfpathlineto{\pgfqpoint{4.351218in}{0.380033in}}%
\pgfpathlineto{\pgfqpoint{4.355447in}{0.387776in}}%
\pgfpathlineto{\pgfqpoint{4.357562in}{0.386978in}}%
\pgfpathlineto{\pgfqpoint{4.359676in}{0.388172in}}%
\pgfpathlineto{\pgfqpoint{4.361791in}{0.382045in}}%
\pgfpathlineto{\pgfqpoint{4.366021in}{0.386687in}}%
\pgfpathlineto{\pgfqpoint{4.370250in}{0.384603in}}%
\pgfpathlineto{\pgfqpoint{4.372365in}{0.379067in}}%
\pgfpathlineto{\pgfqpoint{4.376594in}{0.381654in}}%
\pgfpathlineto{\pgfqpoint{4.380823in}{0.383460in}}%
\pgfpathlineto{\pgfqpoint{4.382938in}{0.385343in}}%
\pgfpathlineto{\pgfqpoint{4.385053in}{0.390627in}}%
\pgfpathlineto{\pgfqpoint{4.387167in}{0.388193in}}%
\pgfpathlineto{\pgfqpoint{4.389282in}{0.391006in}}%
\pgfpathlineto{\pgfqpoint{4.391397in}{0.396754in}}%
\pgfpathlineto{\pgfqpoint{4.393512in}{0.390438in}}%
\pgfpathlineto{\pgfqpoint{4.397741in}{0.386489in}}%
\pgfpathlineto{\pgfqpoint{4.399856in}{0.382309in}}%
\pgfpathlineto{\pgfqpoint{4.401970in}{0.374680in}}%
\pgfpathlineto{\pgfqpoint{4.404085in}{0.374445in}}%
\pgfpathlineto{\pgfqpoint{4.406200in}{0.376570in}}%
\pgfpathlineto{\pgfqpoint{4.408314in}{0.382791in}}%
\pgfpathlineto{\pgfqpoint{4.410429in}{0.376454in}}%
\pgfpathlineto{\pgfqpoint{4.412544in}{0.383692in}}%
\pgfpathlineto{\pgfqpoint{4.414658in}{0.381247in}}%
\pgfpathlineto{\pgfqpoint{4.418888in}{0.383285in}}%
\pgfpathlineto{\pgfqpoint{4.421003in}{0.388461in}}%
\pgfpathlineto{\pgfqpoint{4.423117in}{0.388544in}}%
\pgfpathlineto{\pgfqpoint{4.425232in}{0.386499in}}%
\pgfpathlineto{\pgfqpoint{4.427347in}{0.387246in}}%
\pgfpathlineto{\pgfqpoint{4.429461in}{0.384523in}}%
\pgfpathlineto{\pgfqpoint{4.431576in}{0.386750in}}%
\pgfpathlineto{\pgfqpoint{4.433691in}{0.379226in}}%
\pgfpathlineto{\pgfqpoint{4.435805in}{0.376271in}}%
\pgfpathlineto{\pgfqpoint{4.437920in}{0.376157in}}%
\pgfpathlineto{\pgfqpoint{4.440035in}{0.372482in}}%
\pgfpathlineto{\pgfqpoint{4.442149in}{0.372442in}}%
\pgfpathlineto{\pgfqpoint{4.444264in}{0.363904in}}%
\pgfpathlineto{\pgfqpoint{4.446379in}{0.362500in}}%
\pgfpathlineto{\pgfqpoint{4.448494in}{0.365691in}}%
\pgfpathlineto{\pgfqpoint{4.450608in}{0.365005in}}%
\pgfpathlineto{\pgfqpoint{4.452723in}{0.362526in}}%
\pgfpathlineto{\pgfqpoint{4.454838in}{0.372355in}}%
\pgfpathlineto{\pgfqpoint{4.459067in}{0.377384in}}%
\pgfpathlineto{\pgfqpoint{4.461182in}{0.385115in}}%
\pgfpathlineto{\pgfqpoint{4.463296in}{0.383029in}}%
\pgfpathlineto{\pgfqpoint{4.465411in}{0.385973in}}%
\pgfpathlineto{\pgfqpoint{4.467526in}{0.385791in}}%
\pgfpathlineto{\pgfqpoint{4.471755in}{0.396233in}}%
\pgfpathlineto{\pgfqpoint{4.475985in}{0.398502in}}%
\pgfpathlineto{\pgfqpoint{4.478099in}{0.408175in}}%
\pgfpathlineto{\pgfqpoint{4.480214in}{0.412397in}}%
\pgfpathlineto{\pgfqpoint{4.484443in}{0.405733in}}%
\pgfpathlineto{\pgfqpoint{4.486558in}{0.408213in}}%
\pgfpathlineto{\pgfqpoint{4.488673in}{0.405832in}}%
\pgfpathlineto{\pgfqpoint{4.490787in}{0.399906in}}%
\pgfpathlineto{\pgfqpoint{4.495017in}{0.405526in}}%
\pgfpathlineto{\pgfqpoint{4.497132in}{0.400680in}}%
\pgfpathlineto{\pgfqpoint{4.499246in}{0.402151in}}%
\pgfpathlineto{\pgfqpoint{4.501361in}{0.408435in}}%
\pgfpathlineto{\pgfqpoint{4.503476in}{0.405209in}}%
\pgfpathlineto{\pgfqpoint{4.511934in}{0.404966in}}%
\pgfpathlineto{\pgfqpoint{4.516164in}{0.414697in}}%
\pgfpathlineto{\pgfqpoint{4.518278in}{0.410802in}}%
\pgfpathlineto{\pgfqpoint{4.520393in}{0.413223in}}%
\pgfpathlineto{\pgfqpoint{4.522508in}{0.410615in}}%
\pgfpathlineto{\pgfqpoint{4.524623in}{0.411395in}}%
\pgfpathlineto{\pgfqpoint{4.528852in}{0.407221in}}%
\pgfpathlineto{\pgfqpoint{4.530967in}{0.412176in}}%
\pgfpathlineto{\pgfqpoint{4.535196in}{0.413357in}}%
\pgfpathlineto{\pgfqpoint{4.537311in}{0.412545in}}%
\pgfpathlineto{\pgfqpoint{4.539425in}{0.415426in}}%
\pgfpathlineto{\pgfqpoint{4.541540in}{0.421583in}}%
\pgfpathlineto{\pgfqpoint{4.545769in}{0.408458in}}%
\pgfpathlineto{\pgfqpoint{4.547884in}{0.408009in}}%
\pgfpathlineto{\pgfqpoint{4.549999in}{0.411150in}}%
\pgfpathlineto{\pgfqpoint{4.552114in}{0.411062in}}%
\pgfpathlineto{\pgfqpoint{4.554228in}{0.423545in}}%
\pgfpathlineto{\pgfqpoint{4.562687in}{0.417601in}}%
\pgfpathlineto{\pgfqpoint{4.564802in}{0.421999in}}%
\pgfpathlineto{\pgfqpoint{4.566916in}{0.419257in}}%
\pgfpathlineto{\pgfqpoint{4.569031in}{0.421309in}}%
\pgfpathlineto{\pgfqpoint{4.571146in}{0.420498in}}%
\pgfpathlineto{\pgfqpoint{4.573260in}{0.413876in}}%
\pgfpathlineto{\pgfqpoint{4.575375in}{0.414937in}}%
\pgfpathlineto{\pgfqpoint{4.577490in}{0.417501in}}%
\pgfpathlineto{\pgfqpoint{4.579605in}{0.413745in}}%
\pgfpathlineto{\pgfqpoint{4.583834in}{0.416043in}}%
\pgfpathlineto{\pgfqpoint{4.585949in}{0.416824in}}%
\pgfpathlineto{\pgfqpoint{4.588063in}{0.414528in}}%
\pgfpathlineto{\pgfqpoint{4.590178in}{0.416343in}}%
\pgfpathlineto{\pgfqpoint{4.592293in}{0.416163in}}%
\pgfpathlineto{\pgfqpoint{4.596522in}{0.410970in}}%
\pgfpathlineto{\pgfqpoint{4.598637in}{0.415061in}}%
\pgfpathlineto{\pgfqpoint{4.602866in}{0.416461in}}%
\pgfpathlineto{\pgfqpoint{4.604981in}{0.415761in}}%
\pgfpathlineto{\pgfqpoint{4.607096in}{0.418592in}}%
\pgfpathlineto{\pgfqpoint{4.609210in}{0.414690in}}%
\pgfpathlineto{\pgfqpoint{4.611325in}{0.423896in}}%
\pgfpathlineto{\pgfqpoint{4.613440in}{0.424026in}}%
\pgfpathlineto{\pgfqpoint{4.615554in}{0.416534in}}%
\pgfpathlineto{\pgfqpoint{4.617669in}{0.415219in}}%
\pgfpathlineto{\pgfqpoint{4.619784in}{0.415431in}}%
\pgfpathlineto{\pgfqpoint{4.621898in}{0.412685in}}%
\pgfpathlineto{\pgfqpoint{4.626128in}{0.404455in}}%
\pgfpathlineto{\pgfqpoint{4.628243in}{0.411188in}}%
\pgfpathlineto{\pgfqpoint{4.630357in}{0.408864in}}%
\pgfpathlineto{\pgfqpoint{4.632472in}{0.411229in}}%
\pgfpathlineto{\pgfqpoint{4.636701in}{0.406809in}}%
\pgfpathlineto{\pgfqpoint{4.638816in}{0.412101in}}%
\pgfpathlineto{\pgfqpoint{4.640931in}{0.421088in}}%
\pgfpathlineto{\pgfqpoint{4.643045in}{0.422201in}}%
\pgfpathlineto{\pgfqpoint{4.645160in}{0.429382in}}%
\pgfpathlineto{\pgfqpoint{4.649389in}{0.416859in}}%
\pgfpathlineto{\pgfqpoint{4.651504in}{0.417219in}}%
\pgfpathlineto{\pgfqpoint{4.653619in}{0.416111in}}%
\pgfpathlineto{\pgfqpoint{4.655734in}{0.419737in}}%
\pgfpathlineto{\pgfqpoint{4.657848in}{0.425623in}}%
\pgfpathlineto{\pgfqpoint{4.659963in}{0.424828in}}%
\pgfpathlineto{\pgfqpoint{4.662078in}{0.428453in}}%
\pgfpathlineto{\pgfqpoint{4.666307in}{0.417779in}}%
\pgfpathlineto{\pgfqpoint{4.668422in}{0.407359in}}%
\pgfpathlineto{\pgfqpoint{4.670536in}{0.406191in}}%
\pgfpathlineto{\pgfqpoint{4.672651in}{0.408621in}}%
\pgfpathlineto{\pgfqpoint{4.674766in}{0.403363in}}%
\pgfpathlineto{\pgfqpoint{4.676880in}{0.407074in}}%
\pgfpathlineto{\pgfqpoint{4.678995in}{0.407139in}}%
\pgfpathlineto{\pgfqpoint{4.681110in}{0.405523in}}%
\pgfpathlineto{\pgfqpoint{4.685339in}{0.410555in}}%
\pgfpathlineto{\pgfqpoint{4.687454in}{0.414163in}}%
\pgfpathlineto{\pgfqpoint{4.691683in}{0.406908in}}%
\pgfpathlineto{\pgfqpoint{4.695913in}{0.405190in}}%
\pgfpathlineto{\pgfqpoint{4.698027in}{0.397418in}}%
\pgfpathlineto{\pgfqpoint{4.700142in}{0.404577in}}%
\pgfpathlineto{\pgfqpoint{4.702257in}{0.404887in}}%
\pgfpathlineto{\pgfqpoint{4.706486in}{0.409072in}}%
\pgfpathlineto{\pgfqpoint{4.710716in}{0.416920in}}%
\pgfpathlineto{\pgfqpoint{4.712830in}{0.413921in}}%
\pgfpathlineto{\pgfqpoint{4.721289in}{0.395504in}}%
\pgfpathlineto{\pgfqpoint{4.723404in}{0.399143in}}%
\pgfpathlineto{\pgfqpoint{4.725518in}{0.398952in}}%
\pgfpathlineto{\pgfqpoint{4.729748in}{0.395715in}}%
\pgfpathlineto{\pgfqpoint{4.731863in}{0.396676in}}%
\pgfpathlineto{\pgfqpoint{4.733977in}{0.396117in}}%
\pgfpathlineto{\pgfqpoint{4.740321in}{0.406426in}}%
\pgfpathlineto{\pgfqpoint{4.742436in}{0.406567in}}%
\pgfpathlineto{\pgfqpoint{4.744551in}{0.405160in}}%
\pgfpathlineto{\pgfqpoint{4.746665in}{0.402097in}}%
\pgfpathlineto{\pgfqpoint{4.750895in}{0.401680in}}%
\pgfpathlineto{\pgfqpoint{4.755124in}{0.413383in}}%
\pgfpathlineto{\pgfqpoint{4.759354in}{0.418031in}}%
\pgfpathlineto{\pgfqpoint{4.761468in}{0.411013in}}%
\pgfpathlineto{\pgfqpoint{4.763583in}{0.415681in}}%
\pgfpathlineto{\pgfqpoint{4.765698in}{0.413936in}}%
\pgfpathlineto{\pgfqpoint{4.767812in}{0.414341in}}%
\pgfpathlineto{\pgfqpoint{4.772042in}{0.410099in}}%
\pgfpathlineto{\pgfqpoint{4.776271in}{0.412588in}}%
\pgfpathlineto{\pgfqpoint{4.782615in}{0.406416in}}%
\pgfpathlineto{\pgfqpoint{4.788959in}{0.409375in}}%
\pgfpathlineto{\pgfqpoint{4.791074in}{0.412427in}}%
\pgfpathlineto{\pgfqpoint{4.797418in}{0.398004in}}%
\pgfpathlineto{\pgfqpoint{4.799533in}{0.394663in}}%
\pgfpathlineto{\pgfqpoint{4.801647in}{0.389198in}}%
\pgfpathlineto{\pgfqpoint{4.805877in}{0.394733in}}%
\pgfpathlineto{\pgfqpoint{4.810106in}{0.402129in}}%
\pgfpathlineto{\pgfqpoint{4.812221in}{0.397554in}}%
\pgfpathlineto{\pgfqpoint{4.814336in}{0.396881in}}%
\pgfpathlineto{\pgfqpoint{4.816450in}{0.387640in}}%
\pgfpathlineto{\pgfqpoint{4.818565in}{0.385912in}}%
\pgfpathlineto{\pgfqpoint{4.820680in}{0.387705in}}%
\pgfpathlineto{\pgfqpoint{4.824909in}{0.394403in}}%
\pgfpathlineto{\pgfqpoint{4.827024in}{0.389782in}}%
\pgfpathlineto{\pgfqpoint{4.831253in}{0.394059in}}%
\pgfpathlineto{\pgfqpoint{4.833368in}{0.396076in}}%
\pgfpathlineto{\pgfqpoint{4.835483in}{0.401794in}}%
\pgfpathlineto{\pgfqpoint{4.839712in}{0.398757in}}%
\pgfpathlineto{\pgfqpoint{4.843941in}{0.393877in}}%
\pgfpathlineto{\pgfqpoint{4.846056in}{0.397811in}}%
\pgfpathlineto{\pgfqpoint{4.848171in}{0.397261in}}%
\pgfpathlineto{\pgfqpoint{4.852400in}{0.403041in}}%
\pgfpathlineto{\pgfqpoint{4.854515in}{0.404978in}}%
\pgfpathlineto{\pgfqpoint{4.856629in}{0.400068in}}%
\pgfpathlineto{\pgfqpoint{4.860859in}{0.397731in}}%
\pgfpathlineto{\pgfqpoint{4.862974in}{0.398630in}}%
\pgfpathlineto{\pgfqpoint{4.865088in}{0.402228in}}%
\pgfpathlineto{\pgfqpoint{4.867203in}{0.413929in}}%
\pgfpathlineto{\pgfqpoint{4.871432in}{0.408181in}}%
\pgfpathlineto{\pgfqpoint{4.873547in}{0.412291in}}%
\pgfpathlineto{\pgfqpoint{4.875662in}{0.411517in}}%
\pgfpathlineto{\pgfqpoint{4.877776in}{0.403498in}}%
\pgfpathlineto{\pgfqpoint{4.879891in}{0.414920in}}%
\pgfpathlineto{\pgfqpoint{4.882006in}{0.416130in}}%
\pgfpathlineto{\pgfqpoint{4.884120in}{0.405335in}}%
\pgfpathlineto{\pgfqpoint{4.886235in}{0.405487in}}%
\pgfpathlineto{\pgfqpoint{4.888350in}{0.414295in}}%
\pgfpathlineto{\pgfqpoint{4.890465in}{0.412436in}}%
\pgfpathlineto{\pgfqpoint{4.892579in}{0.405731in}}%
\pgfpathlineto{\pgfqpoint{4.896809in}{0.405904in}}%
\pgfpathlineto{\pgfqpoint{4.898923in}{0.406111in}}%
\pgfpathlineto{\pgfqpoint{4.901038in}{0.407642in}}%
\pgfpathlineto{\pgfqpoint{4.907382in}{0.418017in}}%
\pgfpathlineto{\pgfqpoint{4.909497in}{0.424200in}}%
\pgfpathlineto{\pgfqpoint{4.911611in}{0.425974in}}%
\pgfpathlineto{\pgfqpoint{4.917956in}{0.414891in}}%
\pgfpathlineto{\pgfqpoint{4.920070in}{0.421283in}}%
\pgfpathlineto{\pgfqpoint{4.922185in}{0.420498in}}%
\pgfpathlineto{\pgfqpoint{4.926414in}{0.427021in}}%
\pgfpathlineto{\pgfqpoint{4.928529in}{0.422621in}}%
\pgfpathlineto{\pgfqpoint{4.930644in}{0.426007in}}%
\pgfpathlineto{\pgfqpoint{4.932758in}{0.419613in}}%
\pgfpathlineto{\pgfqpoint{4.934873in}{0.420176in}}%
\pgfpathlineto{\pgfqpoint{4.936988in}{0.416206in}}%
\pgfpathlineto{\pgfqpoint{4.941217in}{0.400824in}}%
\pgfpathlineto{\pgfqpoint{4.943332in}{0.405745in}}%
\pgfpathlineto{\pgfqpoint{4.945447in}{0.403915in}}%
\pgfpathlineto{\pgfqpoint{4.947561in}{0.397945in}}%
\pgfpathlineto{\pgfqpoint{4.949676in}{0.399753in}}%
\pgfpathlineto{\pgfqpoint{4.951791in}{0.397569in}}%
\pgfpathlineto{\pgfqpoint{4.953905in}{0.397692in}}%
\pgfpathlineto{\pgfqpoint{4.956020in}{0.400831in}}%
\pgfpathlineto{\pgfqpoint{4.958135in}{0.401268in}}%
\pgfpathlineto{\pgfqpoint{4.960249in}{0.396219in}}%
\pgfpathlineto{\pgfqpoint{4.962364in}{0.398313in}}%
\pgfpathlineto{\pgfqpoint{4.964479in}{0.397636in}}%
\pgfpathlineto{\pgfqpoint{4.966594in}{0.402119in}}%
\pgfpathlineto{\pgfqpoint{4.968708in}{0.402410in}}%
\pgfpathlineto{\pgfqpoint{4.970823in}{0.404213in}}%
\pgfpathlineto{\pgfqpoint{4.972938in}{0.397270in}}%
\pgfpathlineto{\pgfqpoint{4.975052in}{0.396079in}}%
\pgfpathlineto{\pgfqpoint{4.977167in}{0.396813in}}%
\pgfpathlineto{\pgfqpoint{4.979282in}{0.395692in}}%
\pgfpathlineto{\pgfqpoint{4.983511in}{0.402847in}}%
\pgfpathlineto{\pgfqpoint{4.985626in}{0.398514in}}%
\pgfpathlineto{\pgfqpoint{4.987740in}{0.389934in}}%
\pgfpathlineto{\pgfqpoint{4.991970in}{0.394487in}}%
\pgfpathlineto{\pgfqpoint{4.994085in}{0.392988in}}%
\pgfpathlineto{\pgfqpoint{4.996199in}{0.396722in}}%
\pgfpathlineto{\pgfqpoint{5.002543in}{0.384576in}}%
\pgfpathlineto{\pgfqpoint{5.006773in}{0.393477in}}%
\pgfpathlineto{\pgfqpoint{5.008887in}{0.390955in}}%
\pgfpathlineto{\pgfqpoint{5.011002in}{0.396562in}}%
\pgfpathlineto{\pgfqpoint{5.013117in}{0.397783in}}%
\pgfpathlineto{\pgfqpoint{5.015231in}{0.395598in}}%
\pgfpathlineto{\pgfqpoint{5.019461in}{0.397427in}}%
\pgfpathlineto{\pgfqpoint{5.021576in}{0.401463in}}%
\pgfpathlineto{\pgfqpoint{5.023690in}{0.408588in}}%
\pgfpathlineto{\pgfqpoint{5.025805in}{0.402141in}}%
\pgfpathlineto{\pgfqpoint{5.027920in}{0.409502in}}%
\pgfpathlineto{\pgfqpoint{5.030034in}{0.403567in}}%
\pgfpathlineto{\pgfqpoint{5.032149in}{0.404196in}}%
\pgfpathlineto{\pgfqpoint{5.040608in}{0.420228in}}%
\pgfpathlineto{\pgfqpoint{5.042722in}{0.410266in}}%
\pgfpathlineto{\pgfqpoint{5.044837in}{0.414479in}}%
\pgfpathlineto{\pgfqpoint{5.046952in}{0.415922in}}%
\pgfpathlineto{\pgfqpoint{5.049067in}{0.412685in}}%
\pgfpathlineto{\pgfqpoint{5.051181in}{0.413244in}}%
\pgfpathlineto{\pgfqpoint{5.053296in}{0.417219in}}%
\pgfpathlineto{\pgfqpoint{5.055411in}{0.414858in}}%
\pgfpathlineto{\pgfqpoint{5.057525in}{0.418935in}}%
\pgfpathlineto{\pgfqpoint{5.059640in}{0.425702in}}%
\pgfpathlineto{\pgfqpoint{5.063869in}{0.426938in}}%
\pgfpathlineto{\pgfqpoint{5.065984in}{0.424929in}}%
\pgfpathlineto{\pgfqpoint{5.068099in}{0.430277in}}%
\pgfpathlineto{\pgfqpoint{5.070214in}{0.431978in}}%
\pgfpathlineto{\pgfqpoint{5.072328in}{0.443048in}}%
\pgfpathlineto{\pgfqpoint{5.076558in}{0.446334in}}%
\pgfpathlineto{\pgfqpoint{5.078672in}{0.441609in}}%
\pgfpathlineto{\pgfqpoint{5.080787in}{0.444824in}}%
\pgfpathlineto{\pgfqpoint{5.082902in}{0.440877in}}%
\pgfpathlineto{\pgfqpoint{5.085016in}{0.431096in}}%
\pgfpathlineto{\pgfqpoint{5.087131in}{0.433290in}}%
\pgfpathlineto{\pgfqpoint{5.089246in}{0.429773in}}%
\pgfpathlineto{\pgfqpoint{5.091360in}{0.430593in}}%
\pgfpathlineto{\pgfqpoint{5.093475in}{0.433459in}}%
\pgfpathlineto{\pgfqpoint{5.095590in}{0.427998in}}%
\pgfpathlineto{\pgfqpoint{5.097705in}{0.431276in}}%
\pgfpathlineto{\pgfqpoint{5.099819in}{0.430919in}}%
\pgfpathlineto{\pgfqpoint{5.101934in}{0.421864in}}%
\pgfpathlineto{\pgfqpoint{5.104049in}{0.422587in}}%
\pgfpathlineto{\pgfqpoint{5.106163in}{0.425515in}}%
\pgfpathlineto{\pgfqpoint{5.108278in}{0.425026in}}%
\pgfpathlineto{\pgfqpoint{5.110393in}{0.426788in}}%
\pgfpathlineto{\pgfqpoint{5.112507in}{0.432575in}}%
\pgfpathlineto{\pgfqpoint{5.114622in}{0.434508in}}%
\pgfpathlineto{\pgfqpoint{5.116737in}{0.432225in}}%
\pgfpathlineto{\pgfqpoint{5.120966in}{0.419808in}}%
\pgfpathlineto{\pgfqpoint{5.123081in}{0.418422in}}%
\pgfpathlineto{\pgfqpoint{5.125196in}{0.421442in}}%
\pgfpathlineto{\pgfqpoint{5.127310in}{0.420626in}}%
\pgfpathlineto{\pgfqpoint{5.129425in}{0.421713in}}%
\pgfpathlineto{\pgfqpoint{5.131540in}{0.417904in}}%
\pgfpathlineto{\pgfqpoint{5.133654in}{0.417617in}}%
\pgfpathlineto{\pgfqpoint{5.135769in}{0.420249in}}%
\pgfpathlineto{\pgfqpoint{5.139998in}{0.421198in}}%
\pgfpathlineto{\pgfqpoint{5.142113in}{0.432691in}}%
\pgfpathlineto{\pgfqpoint{5.144228in}{0.430675in}}%
\pgfpathlineto{\pgfqpoint{5.148457in}{0.437244in}}%
\pgfpathlineto{\pgfqpoint{5.150572in}{0.438336in}}%
\pgfpathlineto{\pgfqpoint{5.152687in}{0.441224in}}%
\pgfpathlineto{\pgfqpoint{5.154801in}{0.441749in}}%
\pgfpathlineto{\pgfqpoint{5.156916in}{0.431388in}}%
\pgfpathlineto{\pgfqpoint{5.159031in}{0.432535in}}%
\pgfpathlineto{\pgfqpoint{5.161145in}{0.429498in}}%
\pgfpathlineto{\pgfqpoint{5.163260in}{0.432506in}}%
\pgfpathlineto{\pgfqpoint{5.169604in}{0.429832in}}%
\pgfpathlineto{\pgfqpoint{5.171719in}{0.436125in}}%
\pgfpathlineto{\pgfqpoint{5.173834in}{0.435896in}}%
\pgfpathlineto{\pgfqpoint{5.175948in}{0.432305in}}%
\pgfpathlineto{\pgfqpoint{5.178063in}{0.443900in}}%
\pgfpathlineto{\pgfqpoint{5.180178in}{0.444208in}}%
\pgfpathlineto{\pgfqpoint{5.182292in}{0.446578in}}%
\pgfpathlineto{\pgfqpoint{5.184407in}{0.446120in}}%
\pgfpathlineto{\pgfqpoint{5.186522in}{0.447827in}}%
\pgfpathlineto{\pgfqpoint{5.188636in}{0.454671in}}%
\pgfpathlineto{\pgfqpoint{5.188636in}{0.454671in}}%
\pgfusepath{stroke}%
\end{pgfscope}%
\begin{pgfscope}%
\pgfpathrectangle{\pgfqpoint{0.750000in}{0.275000in}}{\pgfqpoint{4.650000in}{1.925000in}}%
\pgfusepath{clip}%
\pgfsetroundcap%
\pgfsetroundjoin%
\pgfsetlinewidth{1.003750pt}%
\definecolor{currentstroke}{rgb}{1.000000,0.498039,0.000000}%
\pgfsetstrokecolor{currentstroke}%
\pgfsetdash{}{0pt}%
\pgfpathmoveto{\pgfqpoint{0.961364in}{1.147713in}}%
\pgfpathlineto{\pgfqpoint{0.969822in}{1.142280in}}%
\pgfpathlineto{\pgfqpoint{0.971937in}{1.136725in}}%
\pgfpathlineto{\pgfqpoint{0.974052in}{1.135744in}}%
\pgfpathlineto{\pgfqpoint{0.976166in}{1.139466in}}%
\pgfpathlineto{\pgfqpoint{0.978281in}{1.133516in}}%
\pgfpathlineto{\pgfqpoint{0.980396in}{1.136857in}}%
\pgfpathlineto{\pgfqpoint{0.984625in}{1.131359in}}%
\pgfpathlineto{\pgfqpoint{0.986740in}{1.138094in}}%
\pgfpathlineto{\pgfqpoint{0.990969in}{1.124323in}}%
\pgfpathlineto{\pgfqpoint{0.993084in}{1.122629in}}%
\pgfpathlineto{\pgfqpoint{0.997313in}{1.125909in}}%
\pgfpathlineto{\pgfqpoint{1.005772in}{1.108492in}}%
\pgfpathlineto{\pgfqpoint{1.007887in}{1.108513in}}%
\pgfpathlineto{\pgfqpoint{1.010002in}{1.106567in}}%
\pgfpathlineto{\pgfqpoint{1.012116in}{1.110090in}}%
\pgfpathlineto{\pgfqpoint{1.016346in}{1.107120in}}%
\pgfpathlineto{\pgfqpoint{1.018460in}{1.123771in}}%
\pgfpathlineto{\pgfqpoint{1.020575in}{1.125994in}}%
\pgfpathlineto{\pgfqpoint{1.022690in}{1.114645in}}%
\pgfpathlineto{\pgfqpoint{1.024804in}{1.118597in}}%
\pgfpathlineto{\pgfqpoint{1.026919in}{1.115101in}}%
\pgfpathlineto{\pgfqpoint{1.029034in}{1.118481in}}%
\pgfpathlineto{\pgfqpoint{1.031149in}{1.113744in}}%
\pgfpathlineto{\pgfqpoint{1.037493in}{1.107273in}}%
\pgfpathlineto{\pgfqpoint{1.039607in}{1.100492in}}%
\pgfpathlineto{\pgfqpoint{1.043837in}{1.097452in}}%
\pgfpathlineto{\pgfqpoint{1.045951in}{1.096241in}}%
\pgfpathlineto{\pgfqpoint{1.048066in}{1.099196in}}%
\pgfpathlineto{\pgfqpoint{1.050181in}{1.096145in}}%
\pgfpathlineto{\pgfqpoint{1.052295in}{1.087741in}}%
\pgfpathlineto{\pgfqpoint{1.054410in}{1.096112in}}%
\pgfpathlineto{\pgfqpoint{1.056525in}{1.089479in}}%
\pgfpathlineto{\pgfqpoint{1.058640in}{1.088324in}}%
\pgfpathlineto{\pgfqpoint{1.060754in}{1.085889in}}%
\pgfpathlineto{\pgfqpoint{1.062869in}{1.087543in}}%
\pgfpathlineto{\pgfqpoint{1.067098in}{1.099971in}}%
\pgfpathlineto{\pgfqpoint{1.071328in}{1.091244in}}%
\pgfpathlineto{\pgfqpoint{1.073442in}{1.096272in}}%
\pgfpathlineto{\pgfqpoint{1.075557in}{1.095062in}}%
\pgfpathlineto{\pgfqpoint{1.077672in}{1.096898in}}%
\pgfpathlineto{\pgfqpoint{1.081901in}{1.087873in}}%
\pgfpathlineto{\pgfqpoint{1.084016in}{1.095809in}}%
\pgfpathlineto{\pgfqpoint{1.088245in}{1.085246in}}%
\pgfpathlineto{\pgfqpoint{1.090360in}{1.085411in}}%
\pgfpathlineto{\pgfqpoint{1.094589in}{1.081143in}}%
\pgfpathlineto{\pgfqpoint{1.096704in}{1.083014in}}%
\pgfpathlineto{\pgfqpoint{1.098819in}{1.082833in}}%
\pgfpathlineto{\pgfqpoint{1.105163in}{1.071377in}}%
\pgfpathlineto{\pgfqpoint{1.107278in}{1.070028in}}%
\pgfpathlineto{\pgfqpoint{1.111507in}{1.084913in}}%
\pgfpathlineto{\pgfqpoint{1.113622in}{1.085068in}}%
\pgfpathlineto{\pgfqpoint{1.115736in}{1.080006in}}%
\pgfpathlineto{\pgfqpoint{1.117851in}{1.081308in}}%
\pgfpathlineto{\pgfqpoint{1.119966in}{1.077440in}}%
\pgfpathlineto{\pgfqpoint{1.122080in}{1.081395in}}%
\pgfpathlineto{\pgfqpoint{1.124195in}{1.077979in}}%
\pgfpathlineto{\pgfqpoint{1.128424in}{1.085901in}}%
\pgfpathlineto{\pgfqpoint{1.132654in}{1.080277in}}%
\pgfpathlineto{\pgfqpoint{1.134769in}{1.078114in}}%
\pgfpathlineto{\pgfqpoint{1.136883in}{1.087557in}}%
\pgfpathlineto{\pgfqpoint{1.141113in}{1.077632in}}%
\pgfpathlineto{\pgfqpoint{1.145342in}{1.077537in}}%
\pgfpathlineto{\pgfqpoint{1.149571in}{1.070854in}}%
\pgfpathlineto{\pgfqpoint{1.151686in}{1.072311in}}%
\pgfpathlineto{\pgfqpoint{1.153801in}{1.076087in}}%
\pgfpathlineto{\pgfqpoint{1.155915in}{1.073907in}}%
\pgfpathlineto{\pgfqpoint{1.158030in}{1.077280in}}%
\pgfpathlineto{\pgfqpoint{1.160145in}{1.073840in}}%
\pgfpathlineto{\pgfqpoint{1.162260in}{1.074467in}}%
\pgfpathlineto{\pgfqpoint{1.164374in}{1.078757in}}%
\pgfpathlineto{\pgfqpoint{1.166489in}{1.075386in}}%
\pgfpathlineto{\pgfqpoint{1.168604in}{1.068330in}}%
\pgfpathlineto{\pgfqpoint{1.170718in}{1.070558in}}%
\pgfpathlineto{\pgfqpoint{1.172833in}{1.065114in}}%
\pgfpathlineto{\pgfqpoint{1.174948in}{1.070949in}}%
\pgfpathlineto{\pgfqpoint{1.177062in}{1.065969in}}%
\pgfpathlineto{\pgfqpoint{1.179177in}{1.071474in}}%
\pgfpathlineto{\pgfqpoint{1.181292in}{1.072571in}}%
\pgfpathlineto{\pgfqpoint{1.183406in}{1.071507in}}%
\pgfpathlineto{\pgfqpoint{1.187636in}{1.075760in}}%
\pgfpathlineto{\pgfqpoint{1.189751in}{1.075600in}}%
\pgfpathlineto{\pgfqpoint{1.191865in}{1.070686in}}%
\pgfpathlineto{\pgfqpoint{1.193980in}{1.074219in}}%
\pgfpathlineto{\pgfqpoint{1.196095in}{1.070629in}}%
\pgfpathlineto{\pgfqpoint{1.200324in}{1.072434in}}%
\pgfpathlineto{\pgfqpoint{1.202439in}{1.071002in}}%
\pgfpathlineto{\pgfqpoint{1.206668in}{1.071051in}}%
\pgfpathlineto{\pgfqpoint{1.208783in}{1.063685in}}%
\pgfpathlineto{\pgfqpoint{1.210897in}{1.063379in}}%
\pgfpathlineto{\pgfqpoint{1.215127in}{1.048807in}}%
\pgfpathlineto{\pgfqpoint{1.217242in}{1.047023in}}%
\pgfpathlineto{\pgfqpoint{1.219356in}{1.049388in}}%
\pgfpathlineto{\pgfqpoint{1.221471in}{1.045259in}}%
\pgfpathlineto{\pgfqpoint{1.223586in}{1.046160in}}%
\pgfpathlineto{\pgfqpoint{1.225700in}{1.042028in}}%
\pgfpathlineto{\pgfqpoint{1.229930in}{1.041634in}}%
\pgfpathlineto{\pgfqpoint{1.232044in}{1.034011in}}%
\pgfpathlineto{\pgfqpoint{1.234159in}{1.039328in}}%
\pgfpathlineto{\pgfqpoint{1.236274in}{1.039938in}}%
\pgfpathlineto{\pgfqpoint{1.238389in}{1.043001in}}%
\pgfpathlineto{\pgfqpoint{1.242618in}{1.039095in}}%
\pgfpathlineto{\pgfqpoint{1.251077in}{1.045605in}}%
\pgfpathlineto{\pgfqpoint{1.255306in}{1.038974in}}%
\pgfpathlineto{\pgfqpoint{1.257421in}{1.025168in}}%
\pgfpathlineto{\pgfqpoint{1.259535in}{1.018827in}}%
\pgfpathlineto{\pgfqpoint{1.261650in}{1.023734in}}%
\pgfpathlineto{\pgfqpoint{1.263765in}{1.033167in}}%
\pgfpathlineto{\pgfqpoint{1.265880in}{1.032119in}}%
\pgfpathlineto{\pgfqpoint{1.267994in}{1.029170in}}%
\pgfpathlineto{\pgfqpoint{1.270109in}{1.024247in}}%
\pgfpathlineto{\pgfqpoint{1.274338in}{1.033210in}}%
\pgfpathlineto{\pgfqpoint{1.278568in}{1.030285in}}%
\pgfpathlineto{\pgfqpoint{1.280682in}{1.028195in}}%
\pgfpathlineto{\pgfqpoint{1.282797in}{1.027814in}}%
\pgfpathlineto{\pgfqpoint{1.284912in}{1.021643in}}%
\pgfpathlineto{\pgfqpoint{1.287026in}{1.023088in}}%
\pgfpathlineto{\pgfqpoint{1.289141in}{1.018838in}}%
\pgfpathlineto{\pgfqpoint{1.291256in}{1.018772in}}%
\pgfpathlineto{\pgfqpoint{1.293371in}{1.009664in}}%
\pgfpathlineto{\pgfqpoint{1.295485in}{1.018791in}}%
\pgfpathlineto{\pgfqpoint{1.297600in}{1.016845in}}%
\pgfpathlineto{\pgfqpoint{1.301829in}{1.018333in}}%
\pgfpathlineto{\pgfqpoint{1.303944in}{1.014012in}}%
\pgfpathlineto{\pgfqpoint{1.306059in}{1.006846in}}%
\pgfpathlineto{\pgfqpoint{1.308173in}{1.006432in}}%
\pgfpathlineto{\pgfqpoint{1.310288in}{1.003522in}}%
\pgfpathlineto{\pgfqpoint{1.312403in}{1.005071in}}%
\pgfpathlineto{\pgfqpoint{1.316632in}{0.998091in}}%
\pgfpathlineto{\pgfqpoint{1.318747in}{1.003450in}}%
\pgfpathlineto{\pgfqpoint{1.322976in}{0.993743in}}%
\pgfpathlineto{\pgfqpoint{1.325091in}{0.999925in}}%
\pgfpathlineto{\pgfqpoint{1.329320in}{0.989327in}}%
\pgfpathlineto{\pgfqpoint{1.331435in}{0.991155in}}%
\pgfpathlineto{\pgfqpoint{1.333550in}{0.994591in}}%
\pgfpathlineto{\pgfqpoint{1.335664in}{0.993886in}}%
\pgfpathlineto{\pgfqpoint{1.337779in}{0.998149in}}%
\pgfpathlineto{\pgfqpoint{1.339894in}{0.992838in}}%
\pgfpathlineto{\pgfqpoint{1.342009in}{0.991941in}}%
\pgfpathlineto{\pgfqpoint{1.346238in}{0.999209in}}%
\pgfpathlineto{\pgfqpoint{1.348353in}{0.995682in}}%
\pgfpathlineto{\pgfqpoint{1.350467in}{1.005391in}}%
\pgfpathlineto{\pgfqpoint{1.352582in}{1.006332in}}%
\pgfpathlineto{\pgfqpoint{1.354697in}{1.004971in}}%
\pgfpathlineto{\pgfqpoint{1.356811in}{1.009820in}}%
\pgfpathlineto{\pgfqpoint{1.358926in}{1.007639in}}%
\pgfpathlineto{\pgfqpoint{1.361041in}{1.007357in}}%
\pgfpathlineto{\pgfqpoint{1.365270in}{1.014238in}}%
\pgfpathlineto{\pgfqpoint{1.369500in}{1.009339in}}%
\pgfpathlineto{\pgfqpoint{1.371614in}{1.009720in}}%
\pgfpathlineto{\pgfqpoint{1.373729in}{1.011523in}}%
\pgfpathlineto{\pgfqpoint{1.375844in}{1.011647in}}%
\pgfpathlineto{\pgfqpoint{1.377958in}{1.017426in}}%
\pgfpathlineto{\pgfqpoint{1.380073in}{1.011780in}}%
\pgfpathlineto{\pgfqpoint{1.382188in}{1.023644in}}%
\pgfpathlineto{\pgfqpoint{1.384302in}{1.023873in}}%
\pgfpathlineto{\pgfqpoint{1.386417in}{1.031091in}}%
\pgfpathlineto{\pgfqpoint{1.388532in}{1.026204in}}%
\pgfpathlineto{\pgfqpoint{1.390646in}{1.032072in}}%
\pgfpathlineto{\pgfqpoint{1.392761in}{1.031242in}}%
\pgfpathlineto{\pgfqpoint{1.394876in}{1.037040in}}%
\pgfpathlineto{\pgfqpoint{1.396991in}{1.030064in}}%
\pgfpathlineto{\pgfqpoint{1.399105in}{1.036557in}}%
\pgfpathlineto{\pgfqpoint{1.401220in}{1.032075in}}%
\pgfpathlineto{\pgfqpoint{1.403335in}{1.032016in}}%
\pgfpathlineto{\pgfqpoint{1.407564in}{1.041465in}}%
\pgfpathlineto{\pgfqpoint{1.409679in}{1.030316in}}%
\pgfpathlineto{\pgfqpoint{1.413908in}{1.030022in}}%
\pgfpathlineto{\pgfqpoint{1.416023in}{1.026510in}}%
\pgfpathlineto{\pgfqpoint{1.418137in}{1.025834in}}%
\pgfpathlineto{\pgfqpoint{1.422367in}{1.013831in}}%
\pgfpathlineto{\pgfqpoint{1.424482in}{1.017698in}}%
\pgfpathlineto{\pgfqpoint{1.426596in}{1.014953in}}%
\pgfpathlineto{\pgfqpoint{1.430826in}{1.020284in}}%
\pgfpathlineto{\pgfqpoint{1.432940in}{1.019179in}}%
\pgfpathlineto{\pgfqpoint{1.437170in}{1.020497in}}%
\pgfpathlineto{\pgfqpoint{1.439284in}{1.021396in}}%
\pgfpathlineto{\pgfqpoint{1.441399in}{1.023720in}}%
\pgfpathlineto{\pgfqpoint{1.443514in}{1.022454in}}%
\pgfpathlineto{\pgfqpoint{1.445628in}{1.014722in}}%
\pgfpathlineto{\pgfqpoint{1.447743in}{1.002472in}}%
\pgfpathlineto{\pgfqpoint{1.449858in}{1.000252in}}%
\pgfpathlineto{\pgfqpoint{1.451973in}{1.005593in}}%
\pgfpathlineto{\pgfqpoint{1.454087in}{1.004611in}}%
\pgfpathlineto{\pgfqpoint{1.456202in}{1.000341in}}%
\pgfpathlineto{\pgfqpoint{1.458317in}{1.004454in}}%
\pgfpathlineto{\pgfqpoint{1.460431in}{1.003669in}}%
\pgfpathlineto{\pgfqpoint{1.462546in}{1.009687in}}%
\pgfpathlineto{\pgfqpoint{1.464661in}{1.003111in}}%
\pgfpathlineto{\pgfqpoint{1.468890in}{1.013925in}}%
\pgfpathlineto{\pgfqpoint{1.473120in}{1.013049in}}%
\pgfpathlineto{\pgfqpoint{1.477349in}{1.006368in}}%
\pgfpathlineto{\pgfqpoint{1.479464in}{0.997347in}}%
\pgfpathlineto{\pgfqpoint{1.483693in}{1.001555in}}%
\pgfpathlineto{\pgfqpoint{1.485808in}{0.996351in}}%
\pgfpathlineto{\pgfqpoint{1.490037in}{0.992553in}}%
\pgfpathlineto{\pgfqpoint{1.492152in}{0.989669in}}%
\pgfpathlineto{\pgfqpoint{1.494266in}{0.978859in}}%
\pgfpathlineto{\pgfqpoint{1.496381in}{0.986481in}}%
\pgfpathlineto{\pgfqpoint{1.500611in}{0.986201in}}%
\pgfpathlineto{\pgfqpoint{1.502725in}{0.984563in}}%
\pgfpathlineto{\pgfqpoint{1.504840in}{0.984840in}}%
\pgfpathlineto{\pgfqpoint{1.506955in}{0.991399in}}%
\pgfpathlineto{\pgfqpoint{1.509069in}{0.988609in}}%
\pgfpathlineto{\pgfqpoint{1.511184in}{0.983695in}}%
\pgfpathlineto{\pgfqpoint{1.515413in}{0.980309in}}%
\pgfpathlineto{\pgfqpoint{1.519643in}{0.996179in}}%
\pgfpathlineto{\pgfqpoint{1.521757in}{0.992353in}}%
\pgfpathlineto{\pgfqpoint{1.523872in}{0.990901in}}%
\pgfpathlineto{\pgfqpoint{1.525987in}{0.993114in}}%
\pgfpathlineto{\pgfqpoint{1.528102in}{1.001578in}}%
\pgfpathlineto{\pgfqpoint{1.530216in}{0.998706in}}%
\pgfpathlineto{\pgfqpoint{1.532331in}{1.003262in}}%
\pgfpathlineto{\pgfqpoint{1.534446in}{0.999089in}}%
\pgfpathlineto{\pgfqpoint{1.538675in}{0.998548in}}%
\pgfpathlineto{\pgfqpoint{1.540790in}{0.991228in}}%
\pgfpathlineto{\pgfqpoint{1.542904in}{0.992446in}}%
\pgfpathlineto{\pgfqpoint{1.545019in}{0.990413in}}%
\pgfpathlineto{\pgfqpoint{1.547134in}{0.991794in}}%
\pgfpathlineto{\pgfqpoint{1.549248in}{0.990374in}}%
\pgfpathlineto{\pgfqpoint{1.553478in}{0.977047in}}%
\pgfpathlineto{\pgfqpoint{1.557707in}{0.968512in}}%
\pgfpathlineto{\pgfqpoint{1.559822in}{0.967509in}}%
\pgfpathlineto{\pgfqpoint{1.561937in}{0.973959in}}%
\pgfpathlineto{\pgfqpoint{1.564051in}{0.976234in}}%
\pgfpathlineto{\pgfqpoint{1.566166in}{0.974550in}}%
\pgfpathlineto{\pgfqpoint{1.568281in}{0.974695in}}%
\pgfpathlineto{\pgfqpoint{1.570395in}{0.969653in}}%
\pgfpathlineto{\pgfqpoint{1.572510in}{0.976975in}}%
\pgfpathlineto{\pgfqpoint{1.574625in}{0.975996in}}%
\pgfpathlineto{\pgfqpoint{1.578854in}{0.983900in}}%
\pgfpathlineto{\pgfqpoint{1.580969in}{0.983404in}}%
\pgfpathlineto{\pgfqpoint{1.583084in}{0.989244in}}%
\pgfpathlineto{\pgfqpoint{1.589428in}{0.987706in}}%
\pgfpathlineto{\pgfqpoint{1.595772in}{0.994510in}}%
\pgfpathlineto{\pgfqpoint{1.597886in}{0.994721in}}%
\pgfpathlineto{\pgfqpoint{1.600001in}{0.992497in}}%
\pgfpathlineto{\pgfqpoint{1.602116in}{0.988605in}}%
\pgfpathlineto{\pgfqpoint{1.604231in}{0.990761in}}%
\pgfpathlineto{\pgfqpoint{1.608460in}{1.001096in}}%
\pgfpathlineto{\pgfqpoint{1.610575in}{1.006607in}}%
\pgfpathlineto{\pgfqpoint{1.612689in}{1.007249in}}%
\pgfpathlineto{\pgfqpoint{1.619033in}{1.020642in}}%
\pgfpathlineto{\pgfqpoint{1.621148in}{1.014126in}}%
\pgfpathlineto{\pgfqpoint{1.623263in}{1.013922in}}%
\pgfpathlineto{\pgfqpoint{1.625377in}{1.018264in}}%
\pgfpathlineto{\pgfqpoint{1.627492in}{1.015709in}}%
\pgfpathlineto{\pgfqpoint{1.629607in}{1.009155in}}%
\pgfpathlineto{\pgfqpoint{1.631722in}{0.997995in}}%
\pgfpathlineto{\pgfqpoint{1.633836in}{0.995688in}}%
\pgfpathlineto{\pgfqpoint{1.638066in}{0.987050in}}%
\pgfpathlineto{\pgfqpoint{1.642295in}{0.986025in}}%
\pgfpathlineto{\pgfqpoint{1.644410in}{0.977603in}}%
\pgfpathlineto{\pgfqpoint{1.646524in}{0.979812in}}%
\pgfpathlineto{\pgfqpoint{1.648639in}{0.978856in}}%
\pgfpathlineto{\pgfqpoint{1.650754in}{0.974361in}}%
\pgfpathlineto{\pgfqpoint{1.654983in}{0.979719in}}%
\pgfpathlineto{\pgfqpoint{1.657098in}{0.975391in}}%
\pgfpathlineto{\pgfqpoint{1.661327in}{0.976981in}}%
\pgfpathlineto{\pgfqpoint{1.663442in}{0.976727in}}%
\pgfpathlineto{\pgfqpoint{1.665557in}{0.972187in}}%
\pgfpathlineto{\pgfqpoint{1.667671in}{0.975139in}}%
\pgfpathlineto{\pgfqpoint{1.669786in}{0.972860in}}%
\pgfpathlineto{\pgfqpoint{1.671901in}{0.974506in}}%
\pgfpathlineto{\pgfqpoint{1.674015in}{0.973482in}}%
\pgfpathlineto{\pgfqpoint{1.678245in}{0.977804in}}%
\pgfpathlineto{\pgfqpoint{1.680359in}{0.974287in}}%
\pgfpathlineto{\pgfqpoint{1.682474in}{0.974072in}}%
\pgfpathlineto{\pgfqpoint{1.686704in}{0.968009in}}%
\pgfpathlineto{\pgfqpoint{1.688818in}{0.973325in}}%
\pgfpathlineto{\pgfqpoint{1.690933in}{0.971989in}}%
\pgfpathlineto{\pgfqpoint{1.693048in}{0.972030in}}%
\pgfpathlineto{\pgfqpoint{1.695162in}{0.965009in}}%
\pgfpathlineto{\pgfqpoint{1.697277in}{0.974064in}}%
\pgfpathlineto{\pgfqpoint{1.699392in}{0.966070in}}%
\pgfpathlineto{\pgfqpoint{1.703621in}{0.966525in}}%
\pgfpathlineto{\pgfqpoint{1.705736in}{0.964487in}}%
\pgfpathlineto{\pgfqpoint{1.707851in}{0.970853in}}%
\pgfpathlineto{\pgfqpoint{1.709965in}{0.967859in}}%
\pgfpathlineto{\pgfqpoint{1.712080in}{0.971084in}}%
\pgfpathlineto{\pgfqpoint{1.714195in}{0.979628in}}%
\pgfpathlineto{\pgfqpoint{1.716309in}{0.972608in}}%
\pgfpathlineto{\pgfqpoint{1.718424in}{0.979315in}}%
\pgfpathlineto{\pgfqpoint{1.724768in}{0.980654in}}%
\pgfpathlineto{\pgfqpoint{1.726883in}{0.979271in}}%
\pgfpathlineto{\pgfqpoint{1.728997in}{0.975960in}}%
\pgfpathlineto{\pgfqpoint{1.731112in}{0.969351in}}%
\pgfpathlineto{\pgfqpoint{1.733227in}{0.970589in}}%
\pgfpathlineto{\pgfqpoint{1.735342in}{0.965713in}}%
\pgfpathlineto{\pgfqpoint{1.741686in}{0.962802in}}%
\pgfpathlineto{\pgfqpoint{1.743800in}{0.960655in}}%
\pgfpathlineto{\pgfqpoint{1.745915in}{0.961016in}}%
\pgfpathlineto{\pgfqpoint{1.750144in}{0.955473in}}%
\pgfpathlineto{\pgfqpoint{1.754374in}{0.965436in}}%
\pgfpathlineto{\pgfqpoint{1.756488in}{0.963691in}}%
\pgfpathlineto{\pgfqpoint{1.760718in}{0.973095in}}%
\pgfpathlineto{\pgfqpoint{1.764947in}{0.972286in}}%
\pgfpathlineto{\pgfqpoint{1.767062in}{0.975944in}}%
\pgfpathlineto{\pgfqpoint{1.769177in}{0.975357in}}%
\pgfpathlineto{\pgfqpoint{1.771291in}{0.976086in}}%
\pgfpathlineto{\pgfqpoint{1.775521in}{0.972505in}}%
\pgfpathlineto{\pgfqpoint{1.777635in}{0.971210in}}%
\pgfpathlineto{\pgfqpoint{1.779750in}{0.971813in}}%
\pgfpathlineto{\pgfqpoint{1.781865in}{0.978554in}}%
\pgfpathlineto{\pgfqpoint{1.783979in}{0.976870in}}%
\pgfpathlineto{\pgfqpoint{1.786094in}{0.971505in}}%
\pgfpathlineto{\pgfqpoint{1.788209in}{0.973623in}}%
\pgfpathlineto{\pgfqpoint{1.790324in}{0.972465in}}%
\pgfpathlineto{\pgfqpoint{1.794553in}{0.974509in}}%
\pgfpathlineto{\pgfqpoint{1.796668in}{0.976798in}}%
\pgfpathlineto{\pgfqpoint{1.798782in}{0.975790in}}%
\pgfpathlineto{\pgfqpoint{1.800897in}{0.976118in}}%
\pgfpathlineto{\pgfqpoint{1.805126in}{0.971477in}}%
\pgfpathlineto{\pgfqpoint{1.807241in}{0.966062in}}%
\pgfpathlineto{\pgfqpoint{1.809356in}{0.967903in}}%
\pgfpathlineto{\pgfqpoint{1.811471in}{0.957788in}}%
\pgfpathlineto{\pgfqpoint{1.813585in}{0.959021in}}%
\pgfpathlineto{\pgfqpoint{1.815700in}{0.957613in}}%
\pgfpathlineto{\pgfqpoint{1.817815in}{0.961998in}}%
\pgfpathlineto{\pgfqpoint{1.819929in}{0.960722in}}%
\pgfpathlineto{\pgfqpoint{1.822044in}{0.961720in}}%
\pgfpathlineto{\pgfqpoint{1.828388in}{0.943630in}}%
\pgfpathlineto{\pgfqpoint{1.830503in}{0.945805in}}%
\pgfpathlineto{\pgfqpoint{1.834732in}{0.958942in}}%
\pgfpathlineto{\pgfqpoint{1.836847in}{0.961462in}}%
\pgfpathlineto{\pgfqpoint{1.838962in}{0.956323in}}%
\pgfpathlineto{\pgfqpoint{1.841076in}{0.958173in}}%
\pgfpathlineto{\pgfqpoint{1.845306in}{0.951674in}}%
\pgfpathlineto{\pgfqpoint{1.847420in}{0.955995in}}%
\pgfpathlineto{\pgfqpoint{1.849535in}{0.956022in}}%
\pgfpathlineto{\pgfqpoint{1.851650in}{0.953952in}}%
\pgfpathlineto{\pgfqpoint{1.855879in}{0.957857in}}%
\pgfpathlineto{\pgfqpoint{1.857994in}{0.950952in}}%
\pgfpathlineto{\pgfqpoint{1.860108in}{0.947981in}}%
\pgfpathlineto{\pgfqpoint{1.862223in}{0.948497in}}%
\pgfpathlineto{\pgfqpoint{1.866453in}{0.953173in}}%
\pgfpathlineto{\pgfqpoint{1.868567in}{0.952888in}}%
\pgfpathlineto{\pgfqpoint{1.870682in}{0.957810in}}%
\pgfpathlineto{\pgfqpoint{1.874911in}{0.951781in}}%
\pgfpathlineto{\pgfqpoint{1.877026in}{0.956188in}}%
\pgfpathlineto{\pgfqpoint{1.881255in}{0.954794in}}%
\pgfpathlineto{\pgfqpoint{1.883370in}{0.957580in}}%
\pgfpathlineto{\pgfqpoint{1.885485in}{0.953287in}}%
\pgfpathlineto{\pgfqpoint{1.891829in}{0.962698in}}%
\pgfpathlineto{\pgfqpoint{1.896058in}{0.955791in}}%
\pgfpathlineto{\pgfqpoint{1.900288in}{0.959048in}}%
\pgfpathlineto{\pgfqpoint{1.902402in}{0.949995in}}%
\pgfpathlineto{\pgfqpoint{1.904517in}{0.949910in}}%
\pgfpathlineto{\pgfqpoint{1.910861in}{0.957253in}}%
\pgfpathlineto{\pgfqpoint{1.912976in}{0.956217in}}%
\pgfpathlineto{\pgfqpoint{1.917205in}{0.944887in}}%
\pgfpathlineto{\pgfqpoint{1.919320in}{0.950939in}}%
\pgfpathlineto{\pgfqpoint{1.921435in}{0.947023in}}%
\pgfpathlineto{\pgfqpoint{1.923549in}{0.946176in}}%
\pgfpathlineto{\pgfqpoint{1.925664in}{0.954006in}}%
\pgfpathlineto{\pgfqpoint{1.927779in}{0.956353in}}%
\pgfpathlineto{\pgfqpoint{1.929893in}{0.952254in}}%
\pgfpathlineto{\pgfqpoint{1.932008in}{0.951332in}}%
\pgfpathlineto{\pgfqpoint{1.934123in}{0.952854in}}%
\pgfpathlineto{\pgfqpoint{1.936237in}{0.951002in}}%
\pgfpathlineto{\pgfqpoint{1.938352in}{0.952236in}}%
\pgfpathlineto{\pgfqpoint{1.940467in}{0.956700in}}%
\pgfpathlineto{\pgfqpoint{1.942582in}{0.957922in}}%
\pgfpathlineto{\pgfqpoint{1.944696in}{0.955567in}}%
\pgfpathlineto{\pgfqpoint{1.946811in}{0.963892in}}%
\pgfpathlineto{\pgfqpoint{1.951040in}{0.966767in}}%
\pgfpathlineto{\pgfqpoint{1.953155in}{0.964842in}}%
\pgfpathlineto{\pgfqpoint{1.955270in}{0.968049in}}%
\pgfpathlineto{\pgfqpoint{1.961614in}{0.967297in}}%
\pgfpathlineto{\pgfqpoint{1.963728in}{0.967944in}}%
\pgfpathlineto{\pgfqpoint{1.965843in}{0.965589in}}%
\pgfpathlineto{\pgfqpoint{1.967958in}{0.967166in}}%
\pgfpathlineto{\pgfqpoint{1.970073in}{0.964094in}}%
\pgfpathlineto{\pgfqpoint{1.972187in}{0.966740in}}%
\pgfpathlineto{\pgfqpoint{1.976417in}{0.964900in}}%
\pgfpathlineto{\pgfqpoint{1.980646in}{0.969887in}}%
\pgfpathlineto{\pgfqpoint{1.982761in}{0.968895in}}%
\pgfpathlineto{\pgfqpoint{1.984875in}{0.973188in}}%
\pgfpathlineto{\pgfqpoint{1.986990in}{0.973306in}}%
\pgfpathlineto{\pgfqpoint{1.989105in}{0.975520in}}%
\pgfpathlineto{\pgfqpoint{1.991219in}{0.969037in}}%
\pgfpathlineto{\pgfqpoint{1.993334in}{0.966260in}}%
\pgfpathlineto{\pgfqpoint{1.995449in}{0.967948in}}%
\pgfpathlineto{\pgfqpoint{1.997564in}{0.973587in}}%
\pgfpathlineto{\pgfqpoint{1.999678in}{0.976057in}}%
\pgfpathlineto{\pgfqpoint{2.001793in}{0.975897in}}%
\pgfpathlineto{\pgfqpoint{2.003908in}{0.974619in}}%
\pgfpathlineto{\pgfqpoint{2.006022in}{0.977678in}}%
\pgfpathlineto{\pgfqpoint{2.008137in}{0.974219in}}%
\pgfpathlineto{\pgfqpoint{2.010252in}{0.974488in}}%
\pgfpathlineto{\pgfqpoint{2.012366in}{0.971753in}}%
\pgfpathlineto{\pgfqpoint{2.016596in}{0.977365in}}%
\pgfpathlineto{\pgfqpoint{2.018710in}{0.974432in}}%
\pgfpathlineto{\pgfqpoint{2.020825in}{0.980506in}}%
\pgfpathlineto{\pgfqpoint{2.022940in}{0.980794in}}%
\pgfpathlineto{\pgfqpoint{2.025055in}{0.978743in}}%
\pgfpathlineto{\pgfqpoint{2.029284in}{0.968446in}}%
\pgfpathlineto{\pgfqpoint{2.031399in}{0.967838in}}%
\pgfpathlineto{\pgfqpoint{2.033513in}{0.969597in}}%
\pgfpathlineto{\pgfqpoint{2.037743in}{0.964539in}}%
\pgfpathlineto{\pgfqpoint{2.041972in}{0.968891in}}%
\pgfpathlineto{\pgfqpoint{2.044087in}{0.969782in}}%
\pgfpathlineto{\pgfqpoint{2.046202in}{0.964401in}}%
\pgfpathlineto{\pgfqpoint{2.050431in}{0.961788in}}%
\pgfpathlineto{\pgfqpoint{2.054660in}{0.967369in}}%
\pgfpathlineto{\pgfqpoint{2.056775in}{0.973813in}}%
\pgfpathlineto{\pgfqpoint{2.058890in}{0.972762in}}%
\pgfpathlineto{\pgfqpoint{2.061004in}{0.974179in}}%
\pgfpathlineto{\pgfqpoint{2.063119in}{0.965562in}}%
\pgfpathlineto{\pgfqpoint{2.067348in}{0.972710in}}%
\pgfpathlineto{\pgfqpoint{2.069463in}{0.971802in}}%
\pgfpathlineto{\pgfqpoint{2.071578in}{0.974159in}}%
\pgfpathlineto{\pgfqpoint{2.075807in}{0.972722in}}%
\pgfpathlineto{\pgfqpoint{2.077922in}{0.974006in}}%
\pgfpathlineto{\pgfqpoint{2.080037in}{0.973169in}}%
\pgfpathlineto{\pgfqpoint{2.082151in}{0.975355in}}%
\pgfpathlineto{\pgfqpoint{2.084266in}{0.970925in}}%
\pgfpathlineto{\pgfqpoint{2.086381in}{0.963626in}}%
\pgfpathlineto{\pgfqpoint{2.088495in}{0.964177in}}%
\pgfpathlineto{\pgfqpoint{2.092725in}{0.970754in}}%
\pgfpathlineto{\pgfqpoint{2.094839in}{0.973836in}}%
\pgfpathlineto{\pgfqpoint{2.096954in}{0.980094in}}%
\pgfpathlineto{\pgfqpoint{2.101184in}{0.983871in}}%
\pgfpathlineto{\pgfqpoint{2.105413in}{0.981390in}}%
\pgfpathlineto{\pgfqpoint{2.107528in}{0.978428in}}%
\pgfpathlineto{\pgfqpoint{2.111757in}{0.985726in}}%
\pgfpathlineto{\pgfqpoint{2.115986in}{0.983898in}}%
\pgfpathlineto{\pgfqpoint{2.118101in}{0.980374in}}%
\pgfpathlineto{\pgfqpoint{2.120216in}{0.989605in}}%
\pgfpathlineto{\pgfqpoint{2.122330in}{0.989868in}}%
\pgfpathlineto{\pgfqpoint{2.124445in}{0.992309in}}%
\pgfpathlineto{\pgfqpoint{2.126560in}{0.988040in}}%
\pgfpathlineto{\pgfqpoint{2.128675in}{0.989339in}}%
\pgfpathlineto{\pgfqpoint{2.130789in}{0.994778in}}%
\pgfpathlineto{\pgfqpoint{2.132904in}{0.989969in}}%
\pgfpathlineto{\pgfqpoint{2.135019in}{0.988779in}}%
\pgfpathlineto{\pgfqpoint{2.137133in}{0.982380in}}%
\pgfpathlineto{\pgfqpoint{2.139248in}{0.983457in}}%
\pgfpathlineto{\pgfqpoint{2.141363in}{0.971513in}}%
\pgfpathlineto{\pgfqpoint{2.143477in}{0.967352in}}%
\pgfpathlineto{\pgfqpoint{2.145592in}{0.966678in}}%
\pgfpathlineto{\pgfqpoint{2.147707in}{0.964733in}}%
\pgfpathlineto{\pgfqpoint{2.149822in}{0.967744in}}%
\pgfpathlineto{\pgfqpoint{2.151936in}{0.976961in}}%
\pgfpathlineto{\pgfqpoint{2.154051in}{0.969002in}}%
\pgfpathlineto{\pgfqpoint{2.156166in}{0.970680in}}%
\pgfpathlineto{\pgfqpoint{2.158280in}{0.966892in}}%
\pgfpathlineto{\pgfqpoint{2.160395in}{0.968009in}}%
\pgfpathlineto{\pgfqpoint{2.162510in}{0.967287in}}%
\pgfpathlineto{\pgfqpoint{2.166739in}{0.953546in}}%
\pgfpathlineto{\pgfqpoint{2.168854in}{0.943893in}}%
\pgfpathlineto{\pgfqpoint{2.170968in}{0.942327in}}%
\pgfpathlineto{\pgfqpoint{2.173083in}{0.945260in}}%
\pgfpathlineto{\pgfqpoint{2.177313in}{0.943886in}}%
\pgfpathlineto{\pgfqpoint{2.181542in}{0.945531in}}%
\pgfpathlineto{\pgfqpoint{2.185771in}{0.952747in}}%
\pgfpathlineto{\pgfqpoint{2.190001in}{0.951730in}}%
\pgfpathlineto{\pgfqpoint{2.192115in}{0.948136in}}%
\pgfpathlineto{\pgfqpoint{2.194230in}{0.951616in}}%
\pgfpathlineto{\pgfqpoint{2.196345in}{0.946641in}}%
\pgfpathlineto{\pgfqpoint{2.198459in}{0.944507in}}%
\pgfpathlineto{\pgfqpoint{2.200574in}{0.944596in}}%
\pgfpathlineto{\pgfqpoint{2.202689in}{0.949133in}}%
\pgfpathlineto{\pgfqpoint{2.204804in}{0.950439in}}%
\pgfpathlineto{\pgfqpoint{2.211148in}{0.929868in}}%
\pgfpathlineto{\pgfqpoint{2.213262in}{0.931846in}}%
\pgfpathlineto{\pgfqpoint{2.215377in}{0.935484in}}%
\pgfpathlineto{\pgfqpoint{2.217492in}{0.934854in}}%
\pgfpathlineto{\pgfqpoint{2.219606in}{0.930880in}}%
\pgfpathlineto{\pgfqpoint{2.221721in}{0.934682in}}%
\pgfpathlineto{\pgfqpoint{2.223836in}{0.943340in}}%
\pgfpathlineto{\pgfqpoint{2.225950in}{0.943083in}}%
\pgfpathlineto{\pgfqpoint{2.228065in}{0.939221in}}%
\pgfpathlineto{\pgfqpoint{2.230180in}{0.941862in}}%
\pgfpathlineto{\pgfqpoint{2.232295in}{0.932552in}}%
\pgfpathlineto{\pgfqpoint{2.236524in}{0.933227in}}%
\pgfpathlineto{\pgfqpoint{2.238639in}{0.936196in}}%
\pgfpathlineto{\pgfqpoint{2.244983in}{0.916744in}}%
\pgfpathlineto{\pgfqpoint{2.249212in}{0.923791in}}%
\pgfpathlineto{\pgfqpoint{2.251327in}{0.922694in}}%
\pgfpathlineto{\pgfqpoint{2.253441in}{0.919176in}}%
\pgfpathlineto{\pgfqpoint{2.255556in}{0.920384in}}%
\pgfpathlineto{\pgfqpoint{2.259786in}{0.927899in}}%
\pgfpathlineto{\pgfqpoint{2.264015in}{0.925597in}}%
\pgfpathlineto{\pgfqpoint{2.266130in}{0.918443in}}%
\pgfpathlineto{\pgfqpoint{2.270359in}{0.914803in}}%
\pgfpathlineto{\pgfqpoint{2.272474in}{0.922089in}}%
\pgfpathlineto{\pgfqpoint{2.274588in}{0.924053in}}%
\pgfpathlineto{\pgfqpoint{2.276703in}{0.928176in}}%
\pgfpathlineto{\pgfqpoint{2.278818in}{0.928678in}}%
\pgfpathlineto{\pgfqpoint{2.280933in}{0.927311in}}%
\pgfpathlineto{\pgfqpoint{2.287277in}{0.945026in}}%
\pgfpathlineto{\pgfqpoint{2.291506in}{0.934688in}}%
\pgfpathlineto{\pgfqpoint{2.293621in}{0.943327in}}%
\pgfpathlineto{\pgfqpoint{2.297850in}{0.939932in}}%
\pgfpathlineto{\pgfqpoint{2.299965in}{0.941523in}}%
\pgfpathlineto{\pgfqpoint{2.302079in}{0.946428in}}%
\pgfpathlineto{\pgfqpoint{2.306309in}{0.947720in}}%
\pgfpathlineto{\pgfqpoint{2.312653in}{0.933420in}}%
\pgfpathlineto{\pgfqpoint{2.314768in}{0.940010in}}%
\pgfpathlineto{\pgfqpoint{2.316882in}{0.942096in}}%
\pgfpathlineto{\pgfqpoint{2.318997in}{0.941599in}}%
\pgfpathlineto{\pgfqpoint{2.321112in}{0.933116in}}%
\pgfpathlineto{\pgfqpoint{2.323226in}{0.935188in}}%
\pgfpathlineto{\pgfqpoint{2.325341in}{0.932212in}}%
\pgfpathlineto{\pgfqpoint{2.327456in}{0.936708in}}%
\pgfpathlineto{\pgfqpoint{2.329570in}{0.937755in}}%
\pgfpathlineto{\pgfqpoint{2.331685in}{0.934178in}}%
\pgfpathlineto{\pgfqpoint{2.333800in}{0.941118in}}%
\pgfpathlineto{\pgfqpoint{2.335915in}{0.938025in}}%
\pgfpathlineto{\pgfqpoint{2.344373in}{0.917599in}}%
\pgfpathlineto{\pgfqpoint{2.346488in}{0.925366in}}%
\pgfpathlineto{\pgfqpoint{2.350717in}{0.908519in}}%
\pgfpathlineto{\pgfqpoint{2.354947in}{0.902765in}}%
\pgfpathlineto{\pgfqpoint{2.357061in}{0.904769in}}%
\pgfpathlineto{\pgfqpoint{2.361291in}{0.901309in}}%
\pgfpathlineto{\pgfqpoint{2.363406in}{0.901428in}}%
\pgfpathlineto{\pgfqpoint{2.367635in}{0.905895in}}%
\pgfpathlineto{\pgfqpoint{2.369750in}{0.902618in}}%
\pgfpathlineto{\pgfqpoint{2.373979in}{0.905644in}}%
\pgfpathlineto{\pgfqpoint{2.376094in}{0.904167in}}%
\pgfpathlineto{\pgfqpoint{2.378208in}{0.907341in}}%
\pgfpathlineto{\pgfqpoint{2.382438in}{0.922815in}}%
\pgfpathlineto{\pgfqpoint{2.384553in}{0.923561in}}%
\pgfpathlineto{\pgfqpoint{2.386667in}{0.918395in}}%
\pgfpathlineto{\pgfqpoint{2.393011in}{0.929573in}}%
\pgfpathlineto{\pgfqpoint{2.399355in}{0.930714in}}%
\pgfpathlineto{\pgfqpoint{2.403585in}{0.924525in}}%
\pgfpathlineto{\pgfqpoint{2.405699in}{0.927261in}}%
\pgfpathlineto{\pgfqpoint{2.407814in}{0.924807in}}%
\pgfpathlineto{\pgfqpoint{2.409929in}{0.929417in}}%
\pgfpathlineto{\pgfqpoint{2.414158in}{0.926628in}}%
\pgfpathlineto{\pgfqpoint{2.416273in}{0.927780in}}%
\pgfpathlineto{\pgfqpoint{2.418388in}{0.921467in}}%
\pgfpathlineto{\pgfqpoint{2.420502in}{0.921003in}}%
\pgfpathlineto{\pgfqpoint{2.424732in}{0.916192in}}%
\pgfpathlineto{\pgfqpoint{2.428961in}{0.911685in}}%
\pgfpathlineto{\pgfqpoint{2.431076in}{0.912512in}}%
\pgfpathlineto{\pgfqpoint{2.433190in}{0.915141in}}%
\pgfpathlineto{\pgfqpoint{2.435305in}{0.907560in}}%
\pgfpathlineto{\pgfqpoint{2.437420in}{0.912218in}}%
\pgfpathlineto{\pgfqpoint{2.439535in}{0.921391in}}%
\pgfpathlineto{\pgfqpoint{2.445879in}{0.910863in}}%
\pgfpathlineto{\pgfqpoint{2.447993in}{0.914986in}}%
\pgfpathlineto{\pgfqpoint{2.452223in}{0.901716in}}%
\pgfpathlineto{\pgfqpoint{2.454337in}{0.905844in}}%
\pgfpathlineto{\pgfqpoint{2.456452in}{0.898969in}}%
\pgfpathlineto{\pgfqpoint{2.458567in}{0.900204in}}%
\pgfpathlineto{\pgfqpoint{2.462796in}{0.897089in}}%
\pgfpathlineto{\pgfqpoint{2.464911in}{0.898107in}}%
\pgfpathlineto{\pgfqpoint{2.467026in}{0.902205in}}%
\pgfpathlineto{\pgfqpoint{2.469140in}{0.900140in}}%
\pgfpathlineto{\pgfqpoint{2.471255in}{0.901492in}}%
\pgfpathlineto{\pgfqpoint{2.475484in}{0.888062in}}%
\pgfpathlineto{\pgfqpoint{2.479714in}{0.881328in}}%
\pgfpathlineto{\pgfqpoint{2.481828in}{0.881321in}}%
\pgfpathlineto{\pgfqpoint{2.483943in}{0.884870in}}%
\pgfpathlineto{\pgfqpoint{2.490287in}{0.872600in}}%
\pgfpathlineto{\pgfqpoint{2.492402in}{0.878617in}}%
\pgfpathlineto{\pgfqpoint{2.494517in}{0.870128in}}%
\pgfpathlineto{\pgfqpoint{2.496631in}{0.869397in}}%
\pgfpathlineto{\pgfqpoint{2.502975in}{0.874222in}}%
\pgfpathlineto{\pgfqpoint{2.505090in}{0.871462in}}%
\pgfpathlineto{\pgfqpoint{2.507205in}{0.865171in}}%
\pgfpathlineto{\pgfqpoint{2.513549in}{0.862632in}}%
\pgfpathlineto{\pgfqpoint{2.517778in}{0.863598in}}%
\pgfpathlineto{\pgfqpoint{2.519893in}{0.859699in}}%
\pgfpathlineto{\pgfqpoint{2.522008in}{0.859392in}}%
\pgfpathlineto{\pgfqpoint{2.526237in}{0.872430in}}%
\pgfpathlineto{\pgfqpoint{2.528352in}{0.869687in}}%
\pgfpathlineto{\pgfqpoint{2.530466in}{0.872739in}}%
\pgfpathlineto{\pgfqpoint{2.534696in}{0.872736in}}%
\pgfpathlineto{\pgfqpoint{2.538925in}{0.860776in}}%
\pgfpathlineto{\pgfqpoint{2.541040in}{0.866441in}}%
\pgfpathlineto{\pgfqpoint{2.545269in}{0.863744in}}%
\pgfpathlineto{\pgfqpoint{2.549499in}{0.875090in}}%
\pgfpathlineto{\pgfqpoint{2.551613in}{0.874678in}}%
\pgfpathlineto{\pgfqpoint{2.555843in}{0.877751in}}%
\pgfpathlineto{\pgfqpoint{2.557957in}{0.875771in}}%
\pgfpathlineto{\pgfqpoint{2.560072in}{0.864990in}}%
\pgfpathlineto{\pgfqpoint{2.562187in}{0.863702in}}%
\pgfpathlineto{\pgfqpoint{2.564301in}{0.865418in}}%
\pgfpathlineto{\pgfqpoint{2.566416in}{0.864681in}}%
\pgfpathlineto{\pgfqpoint{2.576990in}{0.852638in}}%
\pgfpathlineto{\pgfqpoint{2.579104in}{0.853868in}}%
\pgfpathlineto{\pgfqpoint{2.581219in}{0.853809in}}%
\pgfpathlineto{\pgfqpoint{2.583334in}{0.862080in}}%
\pgfpathlineto{\pgfqpoint{2.585448in}{0.861816in}}%
\pgfpathlineto{\pgfqpoint{2.593907in}{0.849875in}}%
\pgfpathlineto{\pgfqpoint{2.596022in}{0.848616in}}%
\pgfpathlineto{\pgfqpoint{2.598137in}{0.843715in}}%
\pgfpathlineto{\pgfqpoint{2.602366in}{0.852050in}}%
\pgfpathlineto{\pgfqpoint{2.604481in}{0.852993in}}%
\pgfpathlineto{\pgfqpoint{2.606595in}{0.856035in}}%
\pgfpathlineto{\pgfqpoint{2.608710in}{0.853702in}}%
\pgfpathlineto{\pgfqpoint{2.610825in}{0.856409in}}%
\pgfpathlineto{\pgfqpoint{2.612939in}{0.856036in}}%
\pgfpathlineto{\pgfqpoint{2.617169in}{0.846682in}}%
\pgfpathlineto{\pgfqpoint{2.619284in}{0.848146in}}%
\pgfpathlineto{\pgfqpoint{2.623513in}{0.842527in}}%
\pgfpathlineto{\pgfqpoint{2.627742in}{0.853181in}}%
\pgfpathlineto{\pgfqpoint{2.631972in}{0.854054in}}%
\pgfpathlineto{\pgfqpoint{2.634086in}{0.851665in}}%
\pgfpathlineto{\pgfqpoint{2.638316in}{0.840527in}}%
\pgfpathlineto{\pgfqpoint{2.640430in}{0.841341in}}%
\pgfpathlineto{\pgfqpoint{2.646775in}{0.851913in}}%
\pgfpathlineto{\pgfqpoint{2.648889in}{0.851457in}}%
\pgfpathlineto{\pgfqpoint{2.651004in}{0.858511in}}%
\pgfpathlineto{\pgfqpoint{2.653119in}{0.852814in}}%
\pgfpathlineto{\pgfqpoint{2.655233in}{0.852579in}}%
\pgfpathlineto{\pgfqpoint{2.661577in}{0.846607in}}%
\pgfpathlineto{\pgfqpoint{2.663692in}{0.848950in}}%
\pgfpathlineto{\pgfqpoint{2.665807in}{0.854902in}}%
\pgfpathlineto{\pgfqpoint{2.667921in}{0.851842in}}%
\pgfpathlineto{\pgfqpoint{2.670036in}{0.855284in}}%
\pgfpathlineto{\pgfqpoint{2.674266in}{0.853992in}}%
\pgfpathlineto{\pgfqpoint{2.676380in}{0.861234in}}%
\pgfpathlineto{\pgfqpoint{2.678495in}{0.863252in}}%
\pgfpathlineto{\pgfqpoint{2.680610in}{0.854116in}}%
\pgfpathlineto{\pgfqpoint{2.682724in}{0.855295in}}%
\pgfpathlineto{\pgfqpoint{2.684839in}{0.850122in}}%
\pgfpathlineto{\pgfqpoint{2.686954in}{0.853653in}}%
\pgfpathlineto{\pgfqpoint{2.689068in}{0.851892in}}%
\pgfpathlineto{\pgfqpoint{2.691183in}{0.854080in}}%
\pgfpathlineto{\pgfqpoint{2.693298in}{0.848621in}}%
\pgfpathlineto{\pgfqpoint{2.695412in}{0.848938in}}%
\pgfpathlineto{\pgfqpoint{2.699642in}{0.845986in}}%
\pgfpathlineto{\pgfqpoint{2.701757in}{0.848964in}}%
\pgfpathlineto{\pgfqpoint{2.703871in}{0.846233in}}%
\pgfpathlineto{\pgfqpoint{2.705986in}{0.847784in}}%
\pgfpathlineto{\pgfqpoint{2.708101in}{0.847676in}}%
\pgfpathlineto{\pgfqpoint{2.712330in}{0.843738in}}%
\pgfpathlineto{\pgfqpoint{2.714445in}{0.846446in}}%
\pgfpathlineto{\pgfqpoint{2.720789in}{0.832248in}}%
\pgfpathlineto{\pgfqpoint{2.722903in}{0.834820in}}%
\pgfpathlineto{\pgfqpoint{2.725018in}{0.834570in}}%
\pgfpathlineto{\pgfqpoint{2.729248in}{0.831006in}}%
\pgfpathlineto{\pgfqpoint{2.731362in}{0.831074in}}%
\pgfpathlineto{\pgfqpoint{2.733477in}{0.826282in}}%
\pgfpathlineto{\pgfqpoint{2.737706in}{0.837938in}}%
\pgfpathlineto{\pgfqpoint{2.741936in}{0.836284in}}%
\pgfpathlineto{\pgfqpoint{2.744050in}{0.837850in}}%
\pgfpathlineto{\pgfqpoint{2.746165in}{0.835574in}}%
\pgfpathlineto{\pgfqpoint{2.750395in}{0.845648in}}%
\pgfpathlineto{\pgfqpoint{2.752509in}{0.841642in}}%
\pgfpathlineto{\pgfqpoint{2.754624in}{0.847190in}}%
\pgfpathlineto{\pgfqpoint{2.758853in}{0.842378in}}%
\pgfpathlineto{\pgfqpoint{2.760968in}{0.834786in}}%
\pgfpathlineto{\pgfqpoint{2.763083in}{0.836913in}}%
\pgfpathlineto{\pgfqpoint{2.765197in}{0.835701in}}%
\pgfpathlineto{\pgfqpoint{2.767312in}{0.828467in}}%
\pgfpathlineto{\pgfqpoint{2.769427in}{0.830814in}}%
\pgfpathlineto{\pgfqpoint{2.771541in}{0.826459in}}%
\pgfpathlineto{\pgfqpoint{2.773656in}{0.826953in}}%
\pgfpathlineto{\pgfqpoint{2.780000in}{0.833540in}}%
\pgfpathlineto{\pgfqpoint{2.782115in}{0.828847in}}%
\pgfpathlineto{\pgfqpoint{2.784230in}{0.831995in}}%
\pgfpathlineto{\pgfqpoint{2.788459in}{0.841659in}}%
\pgfpathlineto{\pgfqpoint{2.792688in}{0.837346in}}%
\pgfpathlineto{\pgfqpoint{2.794803in}{0.835814in}}%
\pgfpathlineto{\pgfqpoint{2.796918in}{0.838598in}}%
\pgfpathlineto{\pgfqpoint{2.799032in}{0.831256in}}%
\pgfpathlineto{\pgfqpoint{2.803262in}{0.835898in}}%
\pgfpathlineto{\pgfqpoint{2.805377in}{0.832214in}}%
\pgfpathlineto{\pgfqpoint{2.807491in}{0.834491in}}%
\pgfpathlineto{\pgfqpoint{2.809606in}{0.833068in}}%
\pgfpathlineto{\pgfqpoint{2.813835in}{0.846023in}}%
\pgfpathlineto{\pgfqpoint{2.820179in}{0.833881in}}%
\pgfpathlineto{\pgfqpoint{2.822294in}{0.835345in}}%
\pgfpathlineto{\pgfqpoint{2.824409in}{0.834256in}}%
\pgfpathlineto{\pgfqpoint{2.832868in}{0.820671in}}%
\pgfpathlineto{\pgfqpoint{2.834982in}{0.820729in}}%
\pgfpathlineto{\pgfqpoint{2.837097in}{0.819499in}}%
\pgfpathlineto{\pgfqpoint{2.845556in}{0.808049in}}%
\pgfpathlineto{\pgfqpoint{2.847670in}{0.810348in}}%
\pgfpathlineto{\pgfqpoint{2.849785in}{0.818464in}}%
\pgfpathlineto{\pgfqpoint{2.851900in}{0.820174in}}%
\pgfpathlineto{\pgfqpoint{2.858244in}{0.810011in}}%
\pgfpathlineto{\pgfqpoint{2.860359in}{0.811095in}}%
\pgfpathlineto{\pgfqpoint{2.862473in}{0.802162in}}%
\pgfpathlineto{\pgfqpoint{2.864588in}{0.802230in}}%
\pgfpathlineto{\pgfqpoint{2.866703in}{0.798014in}}%
\pgfpathlineto{\pgfqpoint{2.868817in}{0.798791in}}%
\pgfpathlineto{\pgfqpoint{2.870932in}{0.796039in}}%
\pgfpathlineto{\pgfqpoint{2.875161in}{0.795384in}}%
\pgfpathlineto{\pgfqpoint{2.877276in}{0.792718in}}%
\pgfpathlineto{\pgfqpoint{2.879391in}{0.793059in}}%
\pgfpathlineto{\pgfqpoint{2.881506in}{0.795296in}}%
\pgfpathlineto{\pgfqpoint{2.883620in}{0.794439in}}%
\pgfpathlineto{\pgfqpoint{2.887850in}{0.787993in}}%
\pgfpathlineto{\pgfqpoint{2.889964in}{0.783787in}}%
\pgfpathlineto{\pgfqpoint{2.892079in}{0.783881in}}%
\pgfpathlineto{\pgfqpoint{2.894194in}{0.786782in}}%
\pgfpathlineto{\pgfqpoint{2.896308in}{0.783795in}}%
\pgfpathlineto{\pgfqpoint{2.898423in}{0.784042in}}%
\pgfpathlineto{\pgfqpoint{2.900538in}{0.785452in}}%
\pgfpathlineto{\pgfqpoint{2.902652in}{0.791087in}}%
\pgfpathlineto{\pgfqpoint{2.904767in}{0.785692in}}%
\pgfpathlineto{\pgfqpoint{2.906882in}{0.794653in}}%
\pgfpathlineto{\pgfqpoint{2.911111in}{0.794729in}}%
\pgfpathlineto{\pgfqpoint{2.913226in}{0.804219in}}%
\pgfpathlineto{\pgfqpoint{2.915341in}{0.803819in}}%
\pgfpathlineto{\pgfqpoint{2.917455in}{0.805604in}}%
\pgfpathlineto{\pgfqpoint{2.919570in}{0.810094in}}%
\pgfpathlineto{\pgfqpoint{2.921685in}{0.806280in}}%
\pgfpathlineto{\pgfqpoint{2.923799in}{0.807540in}}%
\pgfpathlineto{\pgfqpoint{2.925914in}{0.805926in}}%
\pgfpathlineto{\pgfqpoint{2.928029in}{0.801364in}}%
\pgfpathlineto{\pgfqpoint{2.930143in}{0.801315in}}%
\pgfpathlineto{\pgfqpoint{2.934373in}{0.795413in}}%
\pgfpathlineto{\pgfqpoint{2.936488in}{0.786333in}}%
\pgfpathlineto{\pgfqpoint{2.938602in}{0.795421in}}%
\pgfpathlineto{\pgfqpoint{2.942832in}{0.790541in}}%
\pgfpathlineto{\pgfqpoint{2.944946in}{0.787674in}}%
\pgfpathlineto{\pgfqpoint{2.947061in}{0.788284in}}%
\pgfpathlineto{\pgfqpoint{2.949176in}{0.781882in}}%
\pgfpathlineto{\pgfqpoint{2.951290in}{0.780166in}}%
\pgfpathlineto{\pgfqpoint{2.957634in}{0.791212in}}%
\pgfpathlineto{\pgfqpoint{2.959749in}{0.790009in}}%
\pgfpathlineto{\pgfqpoint{2.963979in}{0.774768in}}%
\pgfpathlineto{\pgfqpoint{2.966093in}{0.769593in}}%
\pgfpathlineto{\pgfqpoint{2.968208in}{0.772078in}}%
\pgfpathlineto{\pgfqpoint{2.970323in}{0.772721in}}%
\pgfpathlineto{\pgfqpoint{2.974552in}{0.762145in}}%
\pgfpathlineto{\pgfqpoint{2.976667in}{0.764162in}}%
\pgfpathlineto{\pgfqpoint{2.978781in}{0.755538in}}%
\pgfpathlineto{\pgfqpoint{2.980896in}{0.758171in}}%
\pgfpathlineto{\pgfqpoint{2.985126in}{0.767071in}}%
\pgfpathlineto{\pgfqpoint{2.989355in}{0.773773in}}%
\pgfpathlineto{\pgfqpoint{2.991470in}{0.771511in}}%
\pgfpathlineto{\pgfqpoint{2.993584in}{0.778522in}}%
\pgfpathlineto{\pgfqpoint{2.997814in}{0.781485in}}%
\pgfpathlineto{\pgfqpoint{3.002043in}{0.793611in}}%
\pgfpathlineto{\pgfqpoint{3.006272in}{0.785228in}}%
\pgfpathlineto{\pgfqpoint{3.008387in}{0.786042in}}%
\pgfpathlineto{\pgfqpoint{3.010502in}{0.782459in}}%
\pgfpathlineto{\pgfqpoint{3.012617in}{0.786023in}}%
\pgfpathlineto{\pgfqpoint{3.014731in}{0.784762in}}%
\pgfpathlineto{\pgfqpoint{3.021075in}{0.768769in}}%
\pgfpathlineto{\pgfqpoint{3.023190in}{0.766724in}}%
\pgfpathlineto{\pgfqpoint{3.025305in}{0.769574in}}%
\pgfpathlineto{\pgfqpoint{3.027419in}{0.769854in}}%
\pgfpathlineto{\pgfqpoint{3.029534in}{0.767176in}}%
\pgfpathlineto{\pgfqpoint{3.031649in}{0.767981in}}%
\pgfpathlineto{\pgfqpoint{3.033763in}{0.766872in}}%
\pgfpathlineto{\pgfqpoint{3.035878in}{0.768808in}}%
\pgfpathlineto{\pgfqpoint{3.037993in}{0.768120in}}%
\pgfpathlineto{\pgfqpoint{3.040108in}{0.769690in}}%
\pgfpathlineto{\pgfqpoint{3.042222in}{0.768970in}}%
\pgfpathlineto{\pgfqpoint{3.046452in}{0.778549in}}%
\pgfpathlineto{\pgfqpoint{3.048566in}{0.776000in}}%
\pgfpathlineto{\pgfqpoint{3.052796in}{0.765706in}}%
\pgfpathlineto{\pgfqpoint{3.054910in}{0.772397in}}%
\pgfpathlineto{\pgfqpoint{3.057025in}{0.772656in}}%
\pgfpathlineto{\pgfqpoint{3.061254in}{0.776442in}}%
\pgfpathlineto{\pgfqpoint{3.063369in}{0.777795in}}%
\pgfpathlineto{\pgfqpoint{3.067599in}{0.783364in}}%
\pgfpathlineto{\pgfqpoint{3.069713in}{0.783867in}}%
\pgfpathlineto{\pgfqpoint{3.071828in}{0.781539in}}%
\pgfpathlineto{\pgfqpoint{3.076057in}{0.796292in}}%
\pgfpathlineto{\pgfqpoint{3.078172in}{0.795196in}}%
\pgfpathlineto{\pgfqpoint{3.080287in}{0.798238in}}%
\pgfpathlineto{\pgfqpoint{3.082401in}{0.796293in}}%
\pgfpathlineto{\pgfqpoint{3.084516in}{0.796321in}}%
\pgfpathlineto{\pgfqpoint{3.086631in}{0.802424in}}%
\pgfpathlineto{\pgfqpoint{3.088746in}{0.795474in}}%
\pgfpathlineto{\pgfqpoint{3.090860in}{0.797186in}}%
\pgfpathlineto{\pgfqpoint{3.095090in}{0.786420in}}%
\pgfpathlineto{\pgfqpoint{3.097204in}{0.786663in}}%
\pgfpathlineto{\pgfqpoint{3.099319in}{0.793799in}}%
\pgfpathlineto{\pgfqpoint{3.103548in}{0.792348in}}%
\pgfpathlineto{\pgfqpoint{3.105663in}{0.794461in}}%
\pgfpathlineto{\pgfqpoint{3.107778in}{0.801329in}}%
\pgfpathlineto{\pgfqpoint{3.112007in}{0.804307in}}%
\pgfpathlineto{\pgfqpoint{3.116237in}{0.813609in}}%
\pgfpathlineto{\pgfqpoint{3.120466in}{0.810202in}}%
\pgfpathlineto{\pgfqpoint{3.122581in}{0.810254in}}%
\pgfpathlineto{\pgfqpoint{3.124695in}{0.811566in}}%
\pgfpathlineto{\pgfqpoint{3.126810in}{0.809315in}}%
\pgfpathlineto{\pgfqpoint{3.131039in}{0.812495in}}%
\pgfpathlineto{\pgfqpoint{3.133154in}{0.810033in}}%
\pgfpathlineto{\pgfqpoint{3.135269in}{0.812704in}}%
\pgfpathlineto{\pgfqpoint{3.137383in}{0.810037in}}%
\pgfpathlineto{\pgfqpoint{3.141613in}{0.816134in}}%
\pgfpathlineto{\pgfqpoint{3.143728in}{0.822204in}}%
\pgfpathlineto{\pgfqpoint{3.145842in}{0.820762in}}%
\pgfpathlineto{\pgfqpoint{3.150072in}{0.825869in}}%
\pgfpathlineto{\pgfqpoint{3.152186in}{0.831030in}}%
\pgfpathlineto{\pgfqpoint{3.156416in}{0.815769in}}%
\pgfpathlineto{\pgfqpoint{3.158530in}{0.820883in}}%
\pgfpathlineto{\pgfqpoint{3.160645in}{0.820174in}}%
\pgfpathlineto{\pgfqpoint{3.162760in}{0.821148in}}%
\pgfpathlineto{\pgfqpoint{3.164874in}{0.818803in}}%
\pgfpathlineto{\pgfqpoint{3.166989in}{0.823104in}}%
\pgfpathlineto{\pgfqpoint{3.169104in}{0.821072in}}%
\pgfpathlineto{\pgfqpoint{3.173333in}{0.822891in}}%
\pgfpathlineto{\pgfqpoint{3.175448in}{0.818604in}}%
\pgfpathlineto{\pgfqpoint{3.177563in}{0.820431in}}%
\pgfpathlineto{\pgfqpoint{3.186021in}{0.808758in}}%
\pgfpathlineto{\pgfqpoint{3.188136in}{0.808390in}}%
\pgfpathlineto{\pgfqpoint{3.190251in}{0.815823in}}%
\pgfpathlineto{\pgfqpoint{3.192366in}{0.818534in}}%
\pgfpathlineto{\pgfqpoint{3.196595in}{0.812875in}}%
\pgfpathlineto{\pgfqpoint{3.198710in}{0.805374in}}%
\pgfpathlineto{\pgfqpoint{3.200824in}{0.806781in}}%
\pgfpathlineto{\pgfqpoint{3.202939in}{0.806765in}}%
\pgfpathlineto{\pgfqpoint{3.205054in}{0.803529in}}%
\pgfpathlineto{\pgfqpoint{3.207168in}{0.809319in}}%
\pgfpathlineto{\pgfqpoint{3.209283in}{0.803621in}}%
\pgfpathlineto{\pgfqpoint{3.211398in}{0.806705in}}%
\pgfpathlineto{\pgfqpoint{3.213512in}{0.806703in}}%
\pgfpathlineto{\pgfqpoint{3.215627in}{0.801938in}}%
\pgfpathlineto{\pgfqpoint{3.217742in}{0.806158in}}%
\pgfpathlineto{\pgfqpoint{3.219857in}{0.797867in}}%
\pgfpathlineto{\pgfqpoint{3.221971in}{0.798624in}}%
\pgfpathlineto{\pgfqpoint{3.224086in}{0.796107in}}%
\pgfpathlineto{\pgfqpoint{3.226201in}{0.801199in}}%
\pgfpathlineto{\pgfqpoint{3.230430in}{0.802062in}}%
\pgfpathlineto{\pgfqpoint{3.232545in}{0.801532in}}%
\pgfpathlineto{\pgfqpoint{3.234659in}{0.806393in}}%
\pgfpathlineto{\pgfqpoint{3.236774in}{0.807288in}}%
\pgfpathlineto{\pgfqpoint{3.238889in}{0.797903in}}%
\pgfpathlineto{\pgfqpoint{3.241003in}{0.801103in}}%
\pgfpathlineto{\pgfqpoint{3.243118in}{0.801257in}}%
\pgfpathlineto{\pgfqpoint{3.247348in}{0.808451in}}%
\pgfpathlineto{\pgfqpoint{3.249462in}{0.811807in}}%
\pgfpathlineto{\pgfqpoint{3.251577in}{0.809131in}}%
\pgfpathlineto{\pgfqpoint{3.253692in}{0.810058in}}%
\pgfpathlineto{\pgfqpoint{3.255806in}{0.809648in}}%
\pgfpathlineto{\pgfqpoint{3.262150in}{0.817980in}}%
\pgfpathlineto{\pgfqpoint{3.264265in}{0.813939in}}%
\pgfpathlineto{\pgfqpoint{3.268494in}{0.820313in}}%
\pgfpathlineto{\pgfqpoint{3.270609in}{0.818899in}}%
\pgfpathlineto{\pgfqpoint{3.272724in}{0.813825in}}%
\pgfpathlineto{\pgfqpoint{3.274839in}{0.812382in}}%
\pgfpathlineto{\pgfqpoint{3.276953in}{0.813405in}}%
\pgfpathlineto{\pgfqpoint{3.279068in}{0.817527in}}%
\pgfpathlineto{\pgfqpoint{3.281183in}{0.812514in}}%
\pgfpathlineto{\pgfqpoint{3.283297in}{0.811939in}}%
\pgfpathlineto{\pgfqpoint{3.285412in}{0.812791in}}%
\pgfpathlineto{\pgfqpoint{3.289641in}{0.820655in}}%
\pgfpathlineto{\pgfqpoint{3.291756in}{0.817817in}}%
\pgfpathlineto{\pgfqpoint{3.298100in}{0.817883in}}%
\pgfpathlineto{\pgfqpoint{3.300215in}{0.814026in}}%
\pgfpathlineto{\pgfqpoint{3.306559in}{0.808491in}}%
\pgfpathlineto{\pgfqpoint{3.308674in}{0.800830in}}%
\pgfpathlineto{\pgfqpoint{3.310788in}{0.801163in}}%
\pgfpathlineto{\pgfqpoint{3.315018in}{0.795555in}}%
\pgfpathlineto{\pgfqpoint{3.317132in}{0.795244in}}%
\pgfpathlineto{\pgfqpoint{3.319247in}{0.790951in}}%
\pgfpathlineto{\pgfqpoint{3.321362in}{0.790505in}}%
\pgfpathlineto{\pgfqpoint{3.325591in}{0.791172in}}%
\pgfpathlineto{\pgfqpoint{3.327706in}{0.787373in}}%
\pgfpathlineto{\pgfqpoint{3.331935in}{0.785135in}}%
\pgfpathlineto{\pgfqpoint{3.336165in}{0.785042in}}%
\pgfpathlineto{\pgfqpoint{3.338279in}{0.783840in}}%
\pgfpathlineto{\pgfqpoint{3.342509in}{0.790834in}}%
\pgfpathlineto{\pgfqpoint{3.344623in}{0.792338in}}%
\pgfpathlineto{\pgfqpoint{3.346738in}{0.791217in}}%
\pgfpathlineto{\pgfqpoint{3.350968in}{0.791848in}}%
\pgfpathlineto{\pgfqpoint{3.353082in}{0.789626in}}%
\pgfpathlineto{\pgfqpoint{3.355197in}{0.785243in}}%
\pgfpathlineto{\pgfqpoint{3.357312in}{0.783565in}}%
\pgfpathlineto{\pgfqpoint{3.359426in}{0.778007in}}%
\pgfpathlineto{\pgfqpoint{3.361541in}{0.778487in}}%
\pgfpathlineto{\pgfqpoint{3.363656in}{0.775110in}}%
\pgfpathlineto{\pgfqpoint{3.365770in}{0.779412in}}%
\pgfpathlineto{\pgfqpoint{3.367885in}{0.791658in}}%
\pgfpathlineto{\pgfqpoint{3.370000in}{0.791015in}}%
\pgfpathlineto{\pgfqpoint{3.372114in}{0.793304in}}%
\pgfpathlineto{\pgfqpoint{3.378459in}{0.814092in}}%
\pgfpathlineto{\pgfqpoint{3.380573in}{0.815773in}}%
\pgfpathlineto{\pgfqpoint{3.382688in}{0.815610in}}%
\pgfpathlineto{\pgfqpoint{3.386917in}{0.806072in}}%
\pgfpathlineto{\pgfqpoint{3.393261in}{0.816998in}}%
\pgfpathlineto{\pgfqpoint{3.395376in}{0.817834in}}%
\pgfpathlineto{\pgfqpoint{3.399605in}{0.816645in}}%
\pgfpathlineto{\pgfqpoint{3.401720in}{0.816466in}}%
\pgfpathlineto{\pgfqpoint{3.403835in}{0.819383in}}%
\pgfpathlineto{\pgfqpoint{3.405950in}{0.826655in}}%
\pgfpathlineto{\pgfqpoint{3.408064in}{0.825998in}}%
\pgfpathlineto{\pgfqpoint{3.410179in}{0.827254in}}%
\pgfpathlineto{\pgfqpoint{3.412294in}{0.837905in}}%
\pgfpathlineto{\pgfqpoint{3.414408in}{0.838501in}}%
\pgfpathlineto{\pgfqpoint{3.416523in}{0.851648in}}%
\pgfpathlineto{\pgfqpoint{3.418638in}{0.846992in}}%
\pgfpathlineto{\pgfqpoint{3.420752in}{0.848196in}}%
\pgfpathlineto{\pgfqpoint{3.422867in}{0.847687in}}%
\pgfpathlineto{\pgfqpoint{3.427097in}{0.843618in}}%
\pgfpathlineto{\pgfqpoint{3.429211in}{0.844007in}}%
\pgfpathlineto{\pgfqpoint{3.431326in}{0.850931in}}%
\pgfpathlineto{\pgfqpoint{3.433441in}{0.848869in}}%
\pgfpathlineto{\pgfqpoint{3.437670in}{0.858146in}}%
\pgfpathlineto{\pgfqpoint{3.439785in}{0.858721in}}%
\pgfpathlineto{\pgfqpoint{3.444014in}{0.858060in}}%
\pgfpathlineto{\pgfqpoint{3.448243in}{0.862239in}}%
\pgfpathlineto{\pgfqpoint{3.452473in}{0.854539in}}%
\pgfpathlineto{\pgfqpoint{3.454588in}{0.854339in}}%
\pgfpathlineto{\pgfqpoint{3.456702in}{0.847736in}}%
\pgfpathlineto{\pgfqpoint{3.463046in}{0.849612in}}%
\pgfpathlineto{\pgfqpoint{3.467276in}{0.846758in}}%
\pgfpathlineto{\pgfqpoint{3.469390in}{0.850858in}}%
\pgfpathlineto{\pgfqpoint{3.473620in}{0.842907in}}%
\pgfpathlineto{\pgfqpoint{3.475734in}{0.834882in}}%
\pgfpathlineto{\pgfqpoint{3.477849in}{0.839081in}}%
\pgfpathlineto{\pgfqpoint{3.479964in}{0.835807in}}%
\pgfpathlineto{\pgfqpoint{3.482079in}{0.829824in}}%
\pgfpathlineto{\pgfqpoint{3.484193in}{0.829895in}}%
\pgfpathlineto{\pgfqpoint{3.486308in}{0.831640in}}%
\pgfpathlineto{\pgfqpoint{3.490537in}{0.831975in}}%
\pgfpathlineto{\pgfqpoint{3.492652in}{0.833734in}}%
\pgfpathlineto{\pgfqpoint{3.494767in}{0.830954in}}%
\pgfpathlineto{\pgfqpoint{3.496881in}{0.822829in}}%
\pgfpathlineto{\pgfqpoint{3.498996in}{0.819755in}}%
\pgfpathlineto{\pgfqpoint{3.501111in}{0.821295in}}%
\pgfpathlineto{\pgfqpoint{3.505340in}{0.832068in}}%
\pgfpathlineto{\pgfqpoint{3.509570in}{0.829693in}}%
\pgfpathlineto{\pgfqpoint{3.511684in}{0.829871in}}%
\pgfpathlineto{\pgfqpoint{3.513799in}{0.819714in}}%
\pgfpathlineto{\pgfqpoint{3.515914in}{0.824709in}}%
\pgfpathlineto{\pgfqpoint{3.520143in}{0.819566in}}%
\pgfpathlineto{\pgfqpoint{3.522258in}{0.819257in}}%
\pgfpathlineto{\pgfqpoint{3.524372in}{0.820701in}}%
\pgfpathlineto{\pgfqpoint{3.526487in}{0.816721in}}%
\pgfpathlineto{\pgfqpoint{3.530716in}{0.823553in}}%
\pgfpathlineto{\pgfqpoint{3.532831in}{0.825226in}}%
\pgfpathlineto{\pgfqpoint{3.534946in}{0.819734in}}%
\pgfpathlineto{\pgfqpoint{3.539175in}{0.822643in}}%
\pgfpathlineto{\pgfqpoint{3.541290in}{0.816904in}}%
\pgfpathlineto{\pgfqpoint{3.543405in}{0.815801in}}%
\pgfpathlineto{\pgfqpoint{3.545519in}{0.817184in}}%
\pgfpathlineto{\pgfqpoint{3.547634in}{0.816905in}}%
\pgfpathlineto{\pgfqpoint{3.549749in}{0.815097in}}%
\pgfpathlineto{\pgfqpoint{3.551863in}{0.814762in}}%
\pgfpathlineto{\pgfqpoint{3.553978in}{0.808068in}}%
\pgfpathlineto{\pgfqpoint{3.556093in}{0.805915in}}%
\pgfpathlineto{\pgfqpoint{3.560322in}{0.809908in}}%
\pgfpathlineto{\pgfqpoint{3.562437in}{0.815349in}}%
\pgfpathlineto{\pgfqpoint{3.566666in}{0.814392in}}%
\pgfpathlineto{\pgfqpoint{3.570896in}{0.807233in}}%
\pgfpathlineto{\pgfqpoint{3.575125in}{0.812575in}}%
\pgfpathlineto{\pgfqpoint{3.577240in}{0.808269in}}%
\pgfpathlineto{\pgfqpoint{3.579354in}{0.806887in}}%
\pgfpathlineto{\pgfqpoint{3.581469in}{0.808261in}}%
\pgfpathlineto{\pgfqpoint{3.583584in}{0.811843in}}%
\pgfpathlineto{\pgfqpoint{3.587813in}{0.803201in}}%
\pgfpathlineto{\pgfqpoint{3.589928in}{0.806472in}}%
\pgfpathlineto{\pgfqpoint{3.592043in}{0.807414in}}%
\pgfpathlineto{\pgfqpoint{3.594157in}{0.814764in}}%
\pgfpathlineto{\pgfqpoint{3.598387in}{0.817728in}}%
\pgfpathlineto{\pgfqpoint{3.600501in}{0.816797in}}%
\pgfpathlineto{\pgfqpoint{3.604731in}{0.804620in}}%
\pgfpathlineto{\pgfqpoint{3.606845in}{0.809249in}}%
\pgfpathlineto{\pgfqpoint{3.611075in}{0.806975in}}%
\pgfpathlineto{\pgfqpoint{3.617419in}{0.815703in}}%
\pgfpathlineto{\pgfqpoint{3.619534in}{0.817416in}}%
\pgfpathlineto{\pgfqpoint{3.621648in}{0.815723in}}%
\pgfpathlineto{\pgfqpoint{3.623763in}{0.815825in}}%
\pgfpathlineto{\pgfqpoint{3.625878in}{0.817897in}}%
\pgfpathlineto{\pgfqpoint{3.627992in}{0.810845in}}%
\pgfpathlineto{\pgfqpoint{3.630107in}{0.808020in}}%
\pgfpathlineto{\pgfqpoint{3.632222in}{0.807789in}}%
\pgfpathlineto{\pgfqpoint{3.634336in}{0.805175in}}%
\pgfpathlineto{\pgfqpoint{3.640681in}{0.820235in}}%
\pgfpathlineto{\pgfqpoint{3.642795in}{0.820238in}}%
\pgfpathlineto{\pgfqpoint{3.644910in}{0.817775in}}%
\pgfpathlineto{\pgfqpoint{3.647025in}{0.820314in}}%
\pgfpathlineto{\pgfqpoint{3.649139in}{0.811913in}}%
\pgfpathlineto{\pgfqpoint{3.653369in}{0.804960in}}%
\pgfpathlineto{\pgfqpoint{3.655483in}{0.812454in}}%
\pgfpathlineto{\pgfqpoint{3.657598in}{0.807777in}}%
\pgfpathlineto{\pgfqpoint{3.659713in}{0.807936in}}%
\pgfpathlineto{\pgfqpoint{3.661828in}{0.805275in}}%
\pgfpathlineto{\pgfqpoint{3.663942in}{0.805300in}}%
\pgfpathlineto{\pgfqpoint{3.668172in}{0.811987in}}%
\pgfpathlineto{\pgfqpoint{3.670286in}{0.811495in}}%
\pgfpathlineto{\pgfqpoint{3.672401in}{0.813643in}}%
\pgfpathlineto{\pgfqpoint{3.676630in}{0.821365in}}%
\pgfpathlineto{\pgfqpoint{3.678745in}{0.822541in}}%
\pgfpathlineto{\pgfqpoint{3.680860in}{0.822309in}}%
\pgfpathlineto{\pgfqpoint{3.682974in}{0.824609in}}%
\pgfpathlineto{\pgfqpoint{3.687204in}{0.824919in}}%
\pgfpathlineto{\pgfqpoint{3.689319in}{0.827347in}}%
\pgfpathlineto{\pgfqpoint{3.691433in}{0.827746in}}%
\pgfpathlineto{\pgfqpoint{3.695663in}{0.832648in}}%
\pgfpathlineto{\pgfqpoint{3.697777in}{0.832448in}}%
\pgfpathlineto{\pgfqpoint{3.704121in}{0.815802in}}%
\pgfpathlineto{\pgfqpoint{3.710465in}{0.818215in}}%
\pgfpathlineto{\pgfqpoint{3.712580in}{0.822595in}}%
\pgfpathlineto{\pgfqpoint{3.714695in}{0.824111in}}%
\pgfpathlineto{\pgfqpoint{3.718924in}{0.820777in}}%
\pgfpathlineto{\pgfqpoint{3.721039in}{0.832771in}}%
\pgfpathlineto{\pgfqpoint{3.723154in}{0.838048in}}%
\pgfpathlineto{\pgfqpoint{3.725268in}{0.838650in}}%
\pgfpathlineto{\pgfqpoint{3.727383in}{0.845930in}}%
\pgfpathlineto{\pgfqpoint{3.729498in}{0.841595in}}%
\pgfpathlineto{\pgfqpoint{3.731612in}{0.840456in}}%
\pgfpathlineto{\pgfqpoint{3.733727in}{0.841714in}}%
\pgfpathlineto{\pgfqpoint{3.735842in}{0.839155in}}%
\pgfpathlineto{\pgfqpoint{3.737956in}{0.839866in}}%
\pgfpathlineto{\pgfqpoint{3.740071in}{0.838232in}}%
\pgfpathlineto{\pgfqpoint{3.744301in}{0.849156in}}%
\pgfpathlineto{\pgfqpoint{3.746415in}{0.849554in}}%
\pgfpathlineto{\pgfqpoint{3.748530in}{0.845496in}}%
\pgfpathlineto{\pgfqpoint{3.750645in}{0.848969in}}%
\pgfpathlineto{\pgfqpoint{3.752759in}{0.846815in}}%
\pgfpathlineto{\pgfqpoint{3.756989in}{0.837027in}}%
\pgfpathlineto{\pgfqpoint{3.759103in}{0.840203in}}%
\pgfpathlineto{\pgfqpoint{3.761218in}{0.838796in}}%
\pgfpathlineto{\pgfqpoint{3.763333in}{0.845622in}}%
\pgfpathlineto{\pgfqpoint{3.765447in}{0.845713in}}%
\pgfpathlineto{\pgfqpoint{3.769677in}{0.841387in}}%
\pgfpathlineto{\pgfqpoint{3.771792in}{0.836583in}}%
\pgfpathlineto{\pgfqpoint{3.776021in}{0.849547in}}%
\pgfpathlineto{\pgfqpoint{3.778136in}{0.849736in}}%
\pgfpathlineto{\pgfqpoint{3.780250in}{0.856608in}}%
\pgfpathlineto{\pgfqpoint{3.782365in}{0.854822in}}%
\pgfpathlineto{\pgfqpoint{3.784480in}{0.856984in}}%
\pgfpathlineto{\pgfqpoint{3.786594in}{0.853941in}}%
\pgfpathlineto{\pgfqpoint{3.788709in}{0.855275in}}%
\pgfpathlineto{\pgfqpoint{3.795053in}{0.840592in}}%
\pgfpathlineto{\pgfqpoint{3.799283in}{0.839592in}}%
\pgfpathlineto{\pgfqpoint{3.801397in}{0.836623in}}%
\pgfpathlineto{\pgfqpoint{3.803512in}{0.838636in}}%
\pgfpathlineto{\pgfqpoint{3.805627in}{0.838217in}}%
\pgfpathlineto{\pgfqpoint{3.807741in}{0.829716in}}%
\pgfpathlineto{\pgfqpoint{3.814085in}{0.829499in}}%
\pgfpathlineto{\pgfqpoint{3.816200in}{0.825606in}}%
\pgfpathlineto{\pgfqpoint{3.820430in}{0.833942in}}%
\pgfpathlineto{\pgfqpoint{3.822544in}{0.837922in}}%
\pgfpathlineto{\pgfqpoint{3.824659in}{0.836517in}}%
\pgfpathlineto{\pgfqpoint{3.826774in}{0.833329in}}%
\pgfpathlineto{\pgfqpoint{3.828888in}{0.842278in}}%
\pgfpathlineto{\pgfqpoint{3.831003in}{0.843764in}}%
\pgfpathlineto{\pgfqpoint{3.833118in}{0.843712in}}%
\pgfpathlineto{\pgfqpoint{3.839462in}{0.850208in}}%
\pgfpathlineto{\pgfqpoint{3.843691in}{0.843297in}}%
\pgfpathlineto{\pgfqpoint{3.845806in}{0.841413in}}%
\pgfpathlineto{\pgfqpoint{3.850035in}{0.841711in}}%
\pgfpathlineto{\pgfqpoint{3.852150in}{0.835680in}}%
\pgfpathlineto{\pgfqpoint{3.854265in}{0.833864in}}%
\pgfpathlineto{\pgfqpoint{3.856379in}{0.828647in}}%
\pgfpathlineto{\pgfqpoint{3.858494in}{0.831180in}}%
\pgfpathlineto{\pgfqpoint{3.860609in}{0.821514in}}%
\pgfpathlineto{\pgfqpoint{3.864838in}{0.835724in}}%
\pgfpathlineto{\pgfqpoint{3.866953in}{0.836494in}}%
\pgfpathlineto{\pgfqpoint{3.869067in}{0.830168in}}%
\pgfpathlineto{\pgfqpoint{3.871182in}{0.833231in}}%
\pgfpathlineto{\pgfqpoint{3.873297in}{0.828960in}}%
\pgfpathlineto{\pgfqpoint{3.875412in}{0.832656in}}%
\pgfpathlineto{\pgfqpoint{3.877526in}{0.830615in}}%
\pgfpathlineto{\pgfqpoint{3.879641in}{0.835022in}}%
\pgfpathlineto{\pgfqpoint{3.881756in}{0.829373in}}%
\pgfpathlineto{\pgfqpoint{3.883870in}{0.828458in}}%
\pgfpathlineto{\pgfqpoint{3.885985in}{0.833071in}}%
\pgfpathlineto{\pgfqpoint{3.890214in}{0.831929in}}%
\pgfpathlineto{\pgfqpoint{3.892329in}{0.833446in}}%
\pgfpathlineto{\pgfqpoint{3.894444in}{0.831058in}}%
\pgfpathlineto{\pgfqpoint{3.898673in}{0.840038in}}%
\pgfpathlineto{\pgfqpoint{3.900788in}{0.844320in}}%
\pgfpathlineto{\pgfqpoint{3.902903in}{0.844951in}}%
\pgfpathlineto{\pgfqpoint{3.905017in}{0.843349in}}%
\pgfpathlineto{\pgfqpoint{3.907132in}{0.847435in}}%
\pgfpathlineto{\pgfqpoint{3.909247in}{0.857008in}}%
\pgfpathlineto{\pgfqpoint{3.911361in}{0.859984in}}%
\pgfpathlineto{\pgfqpoint{3.913476in}{0.857005in}}%
\pgfpathlineto{\pgfqpoint{3.915591in}{0.858264in}}%
\pgfpathlineto{\pgfqpoint{3.917705in}{0.856218in}}%
\pgfpathlineto{\pgfqpoint{3.921935in}{0.844623in}}%
\pgfpathlineto{\pgfqpoint{3.926164in}{0.848613in}}%
\pgfpathlineto{\pgfqpoint{3.928279in}{0.847179in}}%
\pgfpathlineto{\pgfqpoint{3.930394in}{0.842126in}}%
\pgfpathlineto{\pgfqpoint{3.932508in}{0.842075in}}%
\pgfpathlineto{\pgfqpoint{3.934623in}{0.845618in}}%
\pgfpathlineto{\pgfqpoint{3.936738in}{0.845090in}}%
\pgfpathlineto{\pgfqpoint{3.938852in}{0.855674in}}%
\pgfpathlineto{\pgfqpoint{3.940967in}{0.857725in}}%
\pgfpathlineto{\pgfqpoint{3.943082in}{0.862531in}}%
\pgfpathlineto{\pgfqpoint{3.945196in}{0.856649in}}%
\pgfpathlineto{\pgfqpoint{3.949426in}{0.875037in}}%
\pgfpathlineto{\pgfqpoint{3.951541in}{0.877046in}}%
\pgfpathlineto{\pgfqpoint{3.953655in}{0.876322in}}%
\pgfpathlineto{\pgfqpoint{3.955770in}{0.881037in}}%
\pgfpathlineto{\pgfqpoint{3.957885in}{0.877678in}}%
\pgfpathlineto{\pgfqpoint{3.959999in}{0.879791in}}%
\pgfpathlineto{\pgfqpoint{3.964229in}{0.877063in}}%
\pgfpathlineto{\pgfqpoint{3.966343in}{0.871498in}}%
\pgfpathlineto{\pgfqpoint{3.968458in}{0.871526in}}%
\pgfpathlineto{\pgfqpoint{3.970573in}{0.875608in}}%
\pgfpathlineto{\pgfqpoint{3.976917in}{0.854008in}}%
\pgfpathlineto{\pgfqpoint{3.979032in}{0.852238in}}%
\pgfpathlineto{\pgfqpoint{3.981146in}{0.856862in}}%
\pgfpathlineto{\pgfqpoint{3.983261in}{0.857766in}}%
\pgfpathlineto{\pgfqpoint{3.985376in}{0.854568in}}%
\pgfpathlineto{\pgfqpoint{3.987490in}{0.855080in}}%
\pgfpathlineto{\pgfqpoint{3.989605in}{0.857267in}}%
\pgfpathlineto{\pgfqpoint{3.991720in}{0.854508in}}%
\pgfpathlineto{\pgfqpoint{3.995949in}{0.859570in}}%
\pgfpathlineto{\pgfqpoint{4.002293in}{0.848594in}}%
\pgfpathlineto{\pgfqpoint{4.006523in}{0.854437in}}%
\pgfpathlineto{\pgfqpoint{4.010752in}{0.849859in}}%
\pgfpathlineto{\pgfqpoint{4.012867in}{0.840438in}}%
\pgfpathlineto{\pgfqpoint{4.014981in}{0.842668in}}%
\pgfpathlineto{\pgfqpoint{4.017096in}{0.842059in}}%
\pgfpathlineto{\pgfqpoint{4.019211in}{0.847497in}}%
\pgfpathlineto{\pgfqpoint{4.021325in}{0.849116in}}%
\pgfpathlineto{\pgfqpoint{4.023440in}{0.855309in}}%
\pgfpathlineto{\pgfqpoint{4.025555in}{0.854634in}}%
\pgfpathlineto{\pgfqpoint{4.027670in}{0.863356in}}%
\pgfpathlineto{\pgfqpoint{4.034014in}{0.862694in}}%
\pgfpathlineto{\pgfqpoint{4.036128in}{0.857574in}}%
\pgfpathlineto{\pgfqpoint{4.038243in}{0.861035in}}%
\pgfpathlineto{\pgfqpoint{4.040358in}{0.858202in}}%
\pgfpathlineto{\pgfqpoint{4.042472in}{0.864646in}}%
\pgfpathlineto{\pgfqpoint{4.044587in}{0.867484in}}%
\pgfpathlineto{\pgfqpoint{4.048816in}{0.859927in}}%
\pgfpathlineto{\pgfqpoint{4.050931in}{0.861797in}}%
\pgfpathlineto{\pgfqpoint{4.053046in}{0.856565in}}%
\pgfpathlineto{\pgfqpoint{4.055161in}{0.859822in}}%
\pgfpathlineto{\pgfqpoint{4.057275in}{0.858487in}}%
\pgfpathlineto{\pgfqpoint{4.059390in}{0.865787in}}%
\pgfpathlineto{\pgfqpoint{4.061505in}{0.865659in}}%
\pgfpathlineto{\pgfqpoint{4.063619in}{0.861923in}}%
\pgfpathlineto{\pgfqpoint{4.065734in}{0.863425in}}%
\pgfpathlineto{\pgfqpoint{4.067849in}{0.858064in}}%
\pgfpathlineto{\pgfqpoint{4.069963in}{0.856695in}}%
\pgfpathlineto{\pgfqpoint{4.072078in}{0.862532in}}%
\pgfpathlineto{\pgfqpoint{4.078422in}{0.859503in}}%
\pgfpathlineto{\pgfqpoint{4.080537in}{0.855922in}}%
\pgfpathlineto{\pgfqpoint{4.082652in}{0.857837in}}%
\pgfpathlineto{\pgfqpoint{4.084766in}{0.856990in}}%
\pgfpathlineto{\pgfqpoint{4.086881in}{0.848293in}}%
\pgfpathlineto{\pgfqpoint{4.088996in}{0.847470in}}%
\pgfpathlineto{\pgfqpoint{4.091110in}{0.843517in}}%
\pgfpathlineto{\pgfqpoint{4.093225in}{0.842174in}}%
\pgfpathlineto{\pgfqpoint{4.095340in}{0.838244in}}%
\pgfpathlineto{\pgfqpoint{4.097454in}{0.838769in}}%
\pgfpathlineto{\pgfqpoint{4.099569in}{0.828506in}}%
\pgfpathlineto{\pgfqpoint{4.101684in}{0.835778in}}%
\pgfpathlineto{\pgfqpoint{4.105913in}{0.830795in}}%
\pgfpathlineto{\pgfqpoint{4.108028in}{0.830946in}}%
\pgfpathlineto{\pgfqpoint{4.110143in}{0.833005in}}%
\pgfpathlineto{\pgfqpoint{4.116487in}{0.814880in}}%
\pgfpathlineto{\pgfqpoint{4.118601in}{0.811757in}}%
\pgfpathlineto{\pgfqpoint{4.120716in}{0.813218in}}%
\pgfpathlineto{\pgfqpoint{4.122831in}{0.807546in}}%
\pgfpathlineto{\pgfqpoint{4.124945in}{0.812255in}}%
\pgfpathlineto{\pgfqpoint{4.127060in}{0.809149in}}%
\pgfpathlineto{\pgfqpoint{4.129175in}{0.812036in}}%
\pgfpathlineto{\pgfqpoint{4.133404in}{0.812352in}}%
\pgfpathlineto{\pgfqpoint{4.135519in}{0.815256in}}%
\pgfpathlineto{\pgfqpoint{4.137634in}{0.820302in}}%
\pgfpathlineto{\pgfqpoint{4.139748in}{0.812928in}}%
\pgfpathlineto{\pgfqpoint{4.141863in}{0.819081in}}%
\pgfpathlineto{\pgfqpoint{4.143978in}{0.810226in}}%
\pgfpathlineto{\pgfqpoint{4.146092in}{0.807703in}}%
\pgfpathlineto{\pgfqpoint{4.148207in}{0.802552in}}%
\pgfpathlineto{\pgfqpoint{4.150322in}{0.802440in}}%
\pgfpathlineto{\pgfqpoint{4.154551in}{0.792283in}}%
\pgfpathlineto{\pgfqpoint{4.158781in}{0.796997in}}%
\pgfpathlineto{\pgfqpoint{4.160895in}{0.795250in}}%
\pgfpathlineto{\pgfqpoint{4.163010in}{0.800049in}}%
\pgfpathlineto{\pgfqpoint{4.165125in}{0.798210in}}%
\pgfpathlineto{\pgfqpoint{4.167239in}{0.800753in}}%
\pgfpathlineto{\pgfqpoint{4.169354in}{0.791469in}}%
\pgfpathlineto{\pgfqpoint{4.171469in}{0.801371in}}%
\pgfpathlineto{\pgfqpoint{4.173583in}{0.796354in}}%
\pgfpathlineto{\pgfqpoint{4.175698in}{0.802782in}}%
\pgfpathlineto{\pgfqpoint{4.179927in}{0.800979in}}%
\pgfpathlineto{\pgfqpoint{4.182042in}{0.801613in}}%
\pgfpathlineto{\pgfqpoint{4.186272in}{0.805322in}}%
\pgfpathlineto{\pgfqpoint{4.188386in}{0.802715in}}%
\pgfpathlineto{\pgfqpoint{4.192616in}{0.810938in}}%
\pgfpathlineto{\pgfqpoint{4.194730in}{0.807690in}}%
\pgfpathlineto{\pgfqpoint{4.198960in}{0.809357in}}%
\pgfpathlineto{\pgfqpoint{4.201074in}{0.804286in}}%
\pgfpathlineto{\pgfqpoint{4.203189in}{0.807410in}}%
\pgfpathlineto{\pgfqpoint{4.205304in}{0.805939in}}%
\pgfpathlineto{\pgfqpoint{4.209533in}{0.810377in}}%
\pgfpathlineto{\pgfqpoint{4.211648in}{0.811451in}}%
\pgfpathlineto{\pgfqpoint{4.213763in}{0.817955in}}%
\pgfpathlineto{\pgfqpoint{4.215877in}{0.820398in}}%
\pgfpathlineto{\pgfqpoint{4.220107in}{0.826732in}}%
\pgfpathlineto{\pgfqpoint{4.222221in}{0.828802in}}%
\pgfpathlineto{\pgfqpoint{4.224336in}{0.825950in}}%
\pgfpathlineto{\pgfqpoint{4.226451in}{0.831049in}}%
\pgfpathlineto{\pgfqpoint{4.228565in}{0.829280in}}%
\pgfpathlineto{\pgfqpoint{4.230680in}{0.830036in}}%
\pgfpathlineto{\pgfqpoint{4.232795in}{0.833956in}}%
\pgfpathlineto{\pgfqpoint{4.245483in}{0.820809in}}%
\pgfpathlineto{\pgfqpoint{4.249712in}{0.813446in}}%
\pgfpathlineto{\pgfqpoint{4.251827in}{0.815360in}}%
\pgfpathlineto{\pgfqpoint{4.253942in}{0.815111in}}%
\pgfpathlineto{\pgfqpoint{4.256056in}{0.810189in}}%
\pgfpathlineto{\pgfqpoint{4.258171in}{0.818914in}}%
\pgfpathlineto{\pgfqpoint{4.266630in}{0.812107in}}%
\pgfpathlineto{\pgfqpoint{4.268745in}{0.812510in}}%
\pgfpathlineto{\pgfqpoint{4.270859in}{0.804641in}}%
\pgfpathlineto{\pgfqpoint{4.272974in}{0.809686in}}%
\pgfpathlineto{\pgfqpoint{4.275089in}{0.807780in}}%
\pgfpathlineto{\pgfqpoint{4.277203in}{0.810617in}}%
\pgfpathlineto{\pgfqpoint{4.279318in}{0.816005in}}%
\pgfpathlineto{\pgfqpoint{4.281433in}{0.814554in}}%
\pgfpathlineto{\pgfqpoint{4.283547in}{0.811306in}}%
\pgfpathlineto{\pgfqpoint{4.287777in}{0.809145in}}%
\pgfpathlineto{\pgfqpoint{4.292006in}{0.807707in}}%
\pgfpathlineto{\pgfqpoint{4.294121in}{0.812144in}}%
\pgfpathlineto{\pgfqpoint{4.296236in}{0.808313in}}%
\pgfpathlineto{\pgfqpoint{4.298350in}{0.808557in}}%
\pgfpathlineto{\pgfqpoint{4.300465in}{0.811994in}}%
\pgfpathlineto{\pgfqpoint{4.304694in}{0.814720in}}%
\pgfpathlineto{\pgfqpoint{4.308924in}{0.813545in}}%
\pgfpathlineto{\pgfqpoint{4.313153in}{0.807476in}}%
\pgfpathlineto{\pgfqpoint{4.315268in}{0.803250in}}%
\pgfpathlineto{\pgfqpoint{4.317383in}{0.796257in}}%
\pgfpathlineto{\pgfqpoint{4.319497in}{0.797745in}}%
\pgfpathlineto{\pgfqpoint{4.321612in}{0.795319in}}%
\pgfpathlineto{\pgfqpoint{4.323727in}{0.802296in}}%
\pgfpathlineto{\pgfqpoint{4.325841in}{0.797447in}}%
\pgfpathlineto{\pgfqpoint{4.330071in}{0.801829in}}%
\pgfpathlineto{\pgfqpoint{4.334300in}{0.784050in}}%
\pgfpathlineto{\pgfqpoint{4.336415in}{0.787665in}}%
\pgfpathlineto{\pgfqpoint{4.338529in}{0.782040in}}%
\pgfpathlineto{\pgfqpoint{4.340644in}{0.782835in}}%
\pgfpathlineto{\pgfqpoint{4.342759in}{0.779811in}}%
\pgfpathlineto{\pgfqpoint{4.344874in}{0.782919in}}%
\pgfpathlineto{\pgfqpoint{4.346988in}{0.779901in}}%
\pgfpathlineto{\pgfqpoint{4.349103in}{0.779506in}}%
\pgfpathlineto{\pgfqpoint{4.353332in}{0.772830in}}%
\pgfpathlineto{\pgfqpoint{4.355447in}{0.774496in}}%
\pgfpathlineto{\pgfqpoint{4.357562in}{0.778330in}}%
\pgfpathlineto{\pgfqpoint{4.359676in}{0.776198in}}%
\pgfpathlineto{\pgfqpoint{4.361791in}{0.777997in}}%
\pgfpathlineto{\pgfqpoint{4.363906in}{0.771350in}}%
\pgfpathlineto{\pgfqpoint{4.366021in}{0.772359in}}%
\pgfpathlineto{\pgfqpoint{4.368135in}{0.766909in}}%
\pgfpathlineto{\pgfqpoint{4.370250in}{0.775375in}}%
\pgfpathlineto{\pgfqpoint{4.372365in}{0.778692in}}%
\pgfpathlineto{\pgfqpoint{4.374479in}{0.778332in}}%
\pgfpathlineto{\pgfqpoint{4.376594in}{0.771321in}}%
\pgfpathlineto{\pgfqpoint{4.385053in}{0.788830in}}%
\pgfpathlineto{\pgfqpoint{4.387167in}{0.780419in}}%
\pgfpathlineto{\pgfqpoint{4.389282in}{0.776624in}}%
\pgfpathlineto{\pgfqpoint{4.391397in}{0.778696in}}%
\pgfpathlineto{\pgfqpoint{4.393512in}{0.783962in}}%
\pgfpathlineto{\pgfqpoint{4.395626in}{0.784231in}}%
\pgfpathlineto{\pgfqpoint{4.397741in}{0.786107in}}%
\pgfpathlineto{\pgfqpoint{4.399856in}{0.792574in}}%
\pgfpathlineto{\pgfqpoint{4.401970in}{0.795129in}}%
\pgfpathlineto{\pgfqpoint{4.404085in}{0.795176in}}%
\pgfpathlineto{\pgfqpoint{4.406200in}{0.796583in}}%
\pgfpathlineto{\pgfqpoint{4.408314in}{0.788925in}}%
\pgfpathlineto{\pgfqpoint{4.410429in}{0.789975in}}%
\pgfpathlineto{\pgfqpoint{4.412544in}{0.787182in}}%
\pgfpathlineto{\pgfqpoint{4.414658in}{0.777942in}}%
\pgfpathlineto{\pgfqpoint{4.416773in}{0.780002in}}%
\pgfpathlineto{\pgfqpoint{4.421003in}{0.764631in}}%
\pgfpathlineto{\pgfqpoint{4.423117in}{0.761745in}}%
\pgfpathlineto{\pgfqpoint{4.425232in}{0.753853in}}%
\pgfpathlineto{\pgfqpoint{4.429461in}{0.757965in}}%
\pgfpathlineto{\pgfqpoint{4.431576in}{0.752325in}}%
\pgfpathlineto{\pgfqpoint{4.433691in}{0.750674in}}%
\pgfpathlineto{\pgfqpoint{4.435805in}{0.751118in}}%
\pgfpathlineto{\pgfqpoint{4.437920in}{0.747828in}}%
\pgfpathlineto{\pgfqpoint{4.442149in}{0.756903in}}%
\pgfpathlineto{\pgfqpoint{4.444264in}{0.758812in}}%
\pgfpathlineto{\pgfqpoint{4.448494in}{0.759197in}}%
\pgfpathlineto{\pgfqpoint{4.450608in}{0.754482in}}%
\pgfpathlineto{\pgfqpoint{4.452723in}{0.757247in}}%
\pgfpathlineto{\pgfqpoint{4.454838in}{0.750981in}}%
\pgfpathlineto{\pgfqpoint{4.456952in}{0.757259in}}%
\pgfpathlineto{\pgfqpoint{4.463296in}{0.739693in}}%
\pgfpathlineto{\pgfqpoint{4.465411in}{0.740070in}}%
\pgfpathlineto{\pgfqpoint{4.467526in}{0.738061in}}%
\pgfpathlineto{\pgfqpoint{4.471755in}{0.736841in}}%
\pgfpathlineto{\pgfqpoint{4.473870in}{0.734875in}}%
\pgfpathlineto{\pgfqpoint{4.478099in}{0.727210in}}%
\pgfpathlineto{\pgfqpoint{4.480214in}{0.727360in}}%
\pgfpathlineto{\pgfqpoint{4.484443in}{0.720137in}}%
\pgfpathlineto{\pgfqpoint{4.486558in}{0.726876in}}%
\pgfpathlineto{\pgfqpoint{4.490787in}{0.724794in}}%
\pgfpathlineto{\pgfqpoint{4.492902in}{0.718569in}}%
\pgfpathlineto{\pgfqpoint{4.495017in}{0.723182in}}%
\pgfpathlineto{\pgfqpoint{4.497132in}{0.716403in}}%
\pgfpathlineto{\pgfqpoint{4.499246in}{0.717787in}}%
\pgfpathlineto{\pgfqpoint{4.501361in}{0.716654in}}%
\pgfpathlineto{\pgfqpoint{4.503476in}{0.712429in}}%
\pgfpathlineto{\pgfqpoint{4.505590in}{0.714540in}}%
\pgfpathlineto{\pgfqpoint{4.507705in}{0.712673in}}%
\pgfpathlineto{\pgfqpoint{4.509820in}{0.713624in}}%
\pgfpathlineto{\pgfqpoint{4.511934in}{0.716828in}}%
\pgfpathlineto{\pgfqpoint{4.516164in}{0.704963in}}%
\pgfpathlineto{\pgfqpoint{4.518278in}{0.707004in}}%
\pgfpathlineto{\pgfqpoint{4.522508in}{0.714519in}}%
\pgfpathlineto{\pgfqpoint{4.524623in}{0.708489in}}%
\pgfpathlineto{\pgfqpoint{4.526737in}{0.706749in}}%
\pgfpathlineto{\pgfqpoint{4.533081in}{0.709911in}}%
\pgfpathlineto{\pgfqpoint{4.535196in}{0.716524in}}%
\pgfpathlineto{\pgfqpoint{4.537311in}{0.711932in}}%
\pgfpathlineto{\pgfqpoint{4.539425in}{0.713891in}}%
\pgfpathlineto{\pgfqpoint{4.541540in}{0.704804in}}%
\pgfpathlineto{\pgfqpoint{4.545769in}{0.711773in}}%
\pgfpathlineto{\pgfqpoint{4.547884in}{0.704300in}}%
\pgfpathlineto{\pgfqpoint{4.549999in}{0.704461in}}%
\pgfpathlineto{\pgfqpoint{4.552114in}{0.697365in}}%
\pgfpathlineto{\pgfqpoint{4.554228in}{0.697852in}}%
\pgfpathlineto{\pgfqpoint{4.556343in}{0.689252in}}%
\pgfpathlineto{\pgfqpoint{4.558458in}{0.690361in}}%
\pgfpathlineto{\pgfqpoint{4.562687in}{0.684816in}}%
\pgfpathlineto{\pgfqpoint{4.564802in}{0.689072in}}%
\pgfpathlineto{\pgfqpoint{4.569031in}{0.681553in}}%
\pgfpathlineto{\pgfqpoint{4.573260in}{0.688462in}}%
\pgfpathlineto{\pgfqpoint{4.577490in}{0.680437in}}%
\pgfpathlineto{\pgfqpoint{4.581719in}{0.684739in}}%
\pgfpathlineto{\pgfqpoint{4.583834in}{0.680516in}}%
\pgfpathlineto{\pgfqpoint{4.588063in}{0.696414in}}%
\pgfpathlineto{\pgfqpoint{4.590178in}{0.683391in}}%
\pgfpathlineto{\pgfqpoint{4.594407in}{0.681014in}}%
\pgfpathlineto{\pgfqpoint{4.596522in}{0.681332in}}%
\pgfpathlineto{\pgfqpoint{4.598637in}{0.684749in}}%
\pgfpathlineto{\pgfqpoint{4.600752in}{0.681343in}}%
\pgfpathlineto{\pgfqpoint{4.602866in}{0.671489in}}%
\pgfpathlineto{\pgfqpoint{4.604981in}{0.677339in}}%
\pgfpathlineto{\pgfqpoint{4.607096in}{0.674186in}}%
\pgfpathlineto{\pgfqpoint{4.611325in}{0.675780in}}%
\pgfpathlineto{\pgfqpoint{4.613440in}{0.677753in}}%
\pgfpathlineto{\pgfqpoint{4.615554in}{0.676167in}}%
\pgfpathlineto{\pgfqpoint{4.617669in}{0.670831in}}%
\pgfpathlineto{\pgfqpoint{4.619784in}{0.670613in}}%
\pgfpathlineto{\pgfqpoint{4.621898in}{0.664257in}}%
\pgfpathlineto{\pgfqpoint{4.632472in}{0.653774in}}%
\pgfpathlineto{\pgfqpoint{4.640931in}{0.671238in}}%
\pgfpathlineto{\pgfqpoint{4.643045in}{0.669805in}}%
\pgfpathlineto{\pgfqpoint{4.645160in}{0.670412in}}%
\pgfpathlineto{\pgfqpoint{4.647275in}{0.672245in}}%
\pgfpathlineto{\pgfqpoint{4.649389in}{0.675785in}}%
\pgfpathlineto{\pgfqpoint{4.651504in}{0.674730in}}%
\pgfpathlineto{\pgfqpoint{4.653619in}{0.675197in}}%
\pgfpathlineto{\pgfqpoint{4.657848in}{0.668320in}}%
\pgfpathlineto{\pgfqpoint{4.659963in}{0.671603in}}%
\pgfpathlineto{\pgfqpoint{4.666307in}{0.664427in}}%
\pgfpathlineto{\pgfqpoint{4.668422in}{0.659516in}}%
\pgfpathlineto{\pgfqpoint{4.676880in}{0.653930in}}%
\pgfpathlineto{\pgfqpoint{4.678995in}{0.654666in}}%
\pgfpathlineto{\pgfqpoint{4.681110in}{0.656679in}}%
\pgfpathlineto{\pgfqpoint{4.683225in}{0.650711in}}%
\pgfpathlineto{\pgfqpoint{4.685339in}{0.649859in}}%
\pgfpathlineto{\pgfqpoint{4.689569in}{0.642556in}}%
\pgfpathlineto{\pgfqpoint{4.693798in}{0.632599in}}%
\pgfpathlineto{\pgfqpoint{4.695913in}{0.634376in}}%
\pgfpathlineto{\pgfqpoint{4.698027in}{0.639766in}}%
\pgfpathlineto{\pgfqpoint{4.700142in}{0.638090in}}%
\pgfpathlineto{\pgfqpoint{4.702257in}{0.638544in}}%
\pgfpathlineto{\pgfqpoint{4.704372in}{0.636934in}}%
\pgfpathlineto{\pgfqpoint{4.706486in}{0.629125in}}%
\pgfpathlineto{\pgfqpoint{4.708601in}{0.632581in}}%
\pgfpathlineto{\pgfqpoint{4.710716in}{0.630465in}}%
\pgfpathlineto{\pgfqpoint{4.712830in}{0.635221in}}%
\pgfpathlineto{\pgfqpoint{4.714945in}{0.633361in}}%
\pgfpathlineto{\pgfqpoint{4.717060in}{0.633177in}}%
\pgfpathlineto{\pgfqpoint{4.721289in}{0.629406in}}%
\pgfpathlineto{\pgfqpoint{4.723404in}{0.634401in}}%
\pgfpathlineto{\pgfqpoint{4.729748in}{0.638997in}}%
\pgfpathlineto{\pgfqpoint{4.731863in}{0.640176in}}%
\pgfpathlineto{\pgfqpoint{4.733977in}{0.636315in}}%
\pgfpathlineto{\pgfqpoint{4.738207in}{0.647318in}}%
\pgfpathlineto{\pgfqpoint{4.740321in}{0.657960in}}%
\pgfpathlineto{\pgfqpoint{4.742436in}{0.661373in}}%
\pgfpathlineto{\pgfqpoint{4.744551in}{0.659639in}}%
\pgfpathlineto{\pgfqpoint{4.746665in}{0.661632in}}%
\pgfpathlineto{\pgfqpoint{4.748780in}{0.654299in}}%
\pgfpathlineto{\pgfqpoint{4.750895in}{0.655592in}}%
\pgfpathlineto{\pgfqpoint{4.753009in}{0.662396in}}%
\pgfpathlineto{\pgfqpoint{4.759354in}{0.658665in}}%
\pgfpathlineto{\pgfqpoint{4.761468in}{0.654667in}}%
\pgfpathlineto{\pgfqpoint{4.765698in}{0.654150in}}%
\pgfpathlineto{\pgfqpoint{4.767812in}{0.660901in}}%
\pgfpathlineto{\pgfqpoint{4.769927in}{0.663208in}}%
\pgfpathlineto{\pgfqpoint{4.772042in}{0.662215in}}%
\pgfpathlineto{\pgfqpoint{4.774156in}{0.659062in}}%
\pgfpathlineto{\pgfqpoint{4.776271in}{0.660038in}}%
\pgfpathlineto{\pgfqpoint{4.780500in}{0.659113in}}%
\pgfpathlineto{\pgfqpoint{4.782615in}{0.662293in}}%
\pgfpathlineto{\pgfqpoint{4.784730in}{0.660760in}}%
\pgfpathlineto{\pgfqpoint{4.786845in}{0.660628in}}%
\pgfpathlineto{\pgfqpoint{4.788959in}{0.663166in}}%
\pgfpathlineto{\pgfqpoint{4.791074in}{0.670140in}}%
\pgfpathlineto{\pgfqpoint{4.793189in}{0.662705in}}%
\pgfpathlineto{\pgfqpoint{4.795303in}{0.666201in}}%
\pgfpathlineto{\pgfqpoint{4.797418in}{0.666283in}}%
\pgfpathlineto{\pgfqpoint{4.799533in}{0.662272in}}%
\pgfpathlineto{\pgfqpoint{4.801647in}{0.664866in}}%
\pgfpathlineto{\pgfqpoint{4.803762in}{0.662862in}}%
\pgfpathlineto{\pgfqpoint{4.805877in}{0.664629in}}%
\pgfpathlineto{\pgfqpoint{4.807991in}{0.660822in}}%
\pgfpathlineto{\pgfqpoint{4.812221in}{0.664018in}}%
\pgfpathlineto{\pgfqpoint{4.814336in}{0.659605in}}%
\pgfpathlineto{\pgfqpoint{4.816450in}{0.670175in}}%
\pgfpathlineto{\pgfqpoint{4.818565in}{0.674876in}}%
\pgfpathlineto{\pgfqpoint{4.820680in}{0.673880in}}%
\pgfpathlineto{\pgfqpoint{4.824909in}{0.681380in}}%
\pgfpathlineto{\pgfqpoint{4.827024in}{0.675654in}}%
\pgfpathlineto{\pgfqpoint{4.829138in}{0.666634in}}%
\pgfpathlineto{\pgfqpoint{4.839712in}{0.656315in}}%
\pgfpathlineto{\pgfqpoint{4.843941in}{0.660862in}}%
\pgfpathlineto{\pgfqpoint{4.846056in}{0.666941in}}%
\pgfpathlineto{\pgfqpoint{4.852400in}{0.655913in}}%
\pgfpathlineto{\pgfqpoint{4.854515in}{0.658629in}}%
\pgfpathlineto{\pgfqpoint{4.856629in}{0.650869in}}%
\pgfpathlineto{\pgfqpoint{4.858744in}{0.654089in}}%
\pgfpathlineto{\pgfqpoint{4.860859in}{0.648788in}}%
\pgfpathlineto{\pgfqpoint{4.862974in}{0.648604in}}%
\pgfpathlineto{\pgfqpoint{4.867203in}{0.646749in}}%
\pgfpathlineto{\pgfqpoint{4.869318in}{0.640586in}}%
\pgfpathlineto{\pgfqpoint{4.871432in}{0.639678in}}%
\pgfpathlineto{\pgfqpoint{4.873547in}{0.640283in}}%
\pgfpathlineto{\pgfqpoint{4.877776in}{0.649827in}}%
\pgfpathlineto{\pgfqpoint{4.879891in}{0.651943in}}%
\pgfpathlineto{\pgfqpoint{4.884120in}{0.644545in}}%
\pgfpathlineto{\pgfqpoint{4.886235in}{0.644954in}}%
\pgfpathlineto{\pgfqpoint{4.888350in}{0.649984in}}%
\pgfpathlineto{\pgfqpoint{4.890465in}{0.647268in}}%
\pgfpathlineto{\pgfqpoint{4.892579in}{0.642224in}}%
\pgfpathlineto{\pgfqpoint{4.894694in}{0.641953in}}%
\pgfpathlineto{\pgfqpoint{4.896809in}{0.647553in}}%
\pgfpathlineto{\pgfqpoint{4.898923in}{0.649584in}}%
\pgfpathlineto{\pgfqpoint{4.903153in}{0.650190in}}%
\pgfpathlineto{\pgfqpoint{4.905267in}{0.645685in}}%
\pgfpathlineto{\pgfqpoint{4.909497in}{0.657709in}}%
\pgfpathlineto{\pgfqpoint{4.911611in}{0.658463in}}%
\pgfpathlineto{\pgfqpoint{4.917956in}{0.646700in}}%
\pgfpathlineto{\pgfqpoint{4.920070in}{0.646543in}}%
\pgfpathlineto{\pgfqpoint{4.924300in}{0.638659in}}%
\pgfpathlineto{\pgfqpoint{4.926414in}{0.635807in}}%
\pgfpathlineto{\pgfqpoint{4.930644in}{0.639063in}}%
\pgfpathlineto{\pgfqpoint{4.932758in}{0.636372in}}%
\pgfpathlineto{\pgfqpoint{4.936988in}{0.643377in}}%
\pgfpathlineto{\pgfqpoint{4.939103in}{0.644488in}}%
\pgfpathlineto{\pgfqpoint{4.941217in}{0.640674in}}%
\pgfpathlineto{\pgfqpoint{4.947561in}{0.657851in}}%
\pgfpathlineto{\pgfqpoint{4.949676in}{0.659271in}}%
\pgfpathlineto{\pgfqpoint{4.951791in}{0.657859in}}%
\pgfpathlineto{\pgfqpoint{4.953905in}{0.654944in}}%
\pgfpathlineto{\pgfqpoint{4.958135in}{0.660507in}}%
\pgfpathlineto{\pgfqpoint{4.960249in}{0.658214in}}%
\pgfpathlineto{\pgfqpoint{4.962364in}{0.654023in}}%
\pgfpathlineto{\pgfqpoint{4.964479in}{0.653118in}}%
\pgfpathlineto{\pgfqpoint{4.966594in}{0.649741in}}%
\pgfpathlineto{\pgfqpoint{4.968708in}{0.654801in}}%
\pgfpathlineto{\pgfqpoint{4.970823in}{0.656547in}}%
\pgfpathlineto{\pgfqpoint{4.972938in}{0.651048in}}%
\pgfpathlineto{\pgfqpoint{4.975052in}{0.648711in}}%
\pgfpathlineto{\pgfqpoint{4.977167in}{0.644627in}}%
\pgfpathlineto{\pgfqpoint{4.979282in}{0.650098in}}%
\pgfpathlineto{\pgfqpoint{4.983511in}{0.638988in}}%
\pgfpathlineto{\pgfqpoint{4.987740in}{0.642997in}}%
\pgfpathlineto{\pgfqpoint{4.989855in}{0.650975in}}%
\pgfpathlineto{\pgfqpoint{4.994085in}{0.641162in}}%
\pgfpathlineto{\pgfqpoint{4.996199in}{0.645853in}}%
\pgfpathlineto{\pgfqpoint{5.000429in}{0.643797in}}%
\pgfpathlineto{\pgfqpoint{5.002543in}{0.642814in}}%
\pgfpathlineto{\pgfqpoint{5.006773in}{0.637794in}}%
\pgfpathlineto{\pgfqpoint{5.008887in}{0.640769in}}%
\pgfpathlineto{\pgfqpoint{5.011002in}{0.645856in}}%
\pgfpathlineto{\pgfqpoint{5.013117in}{0.645653in}}%
\pgfpathlineto{\pgfqpoint{5.015231in}{0.653512in}}%
\pgfpathlineto{\pgfqpoint{5.017346in}{0.655757in}}%
\pgfpathlineto{\pgfqpoint{5.021576in}{0.644606in}}%
\pgfpathlineto{\pgfqpoint{5.025805in}{0.632916in}}%
\pgfpathlineto{\pgfqpoint{5.027920in}{0.632608in}}%
\pgfpathlineto{\pgfqpoint{5.030034in}{0.622284in}}%
\pgfpathlineto{\pgfqpoint{5.032149in}{0.623249in}}%
\pgfpathlineto{\pgfqpoint{5.036378in}{0.621996in}}%
\pgfpathlineto{\pgfqpoint{5.038493in}{0.623178in}}%
\pgfpathlineto{\pgfqpoint{5.042722in}{0.615721in}}%
\pgfpathlineto{\pgfqpoint{5.046952in}{0.615696in}}%
\pgfpathlineto{\pgfqpoint{5.049067in}{0.608089in}}%
\pgfpathlineto{\pgfqpoint{5.051181in}{0.611797in}}%
\pgfpathlineto{\pgfqpoint{5.053296in}{0.611344in}}%
\pgfpathlineto{\pgfqpoint{5.055411in}{0.616153in}}%
\pgfpathlineto{\pgfqpoint{5.057525in}{0.616086in}}%
\pgfpathlineto{\pgfqpoint{5.059640in}{0.604295in}}%
\pgfpathlineto{\pgfqpoint{5.063869in}{0.608386in}}%
\pgfpathlineto{\pgfqpoint{5.065984in}{0.605539in}}%
\pgfpathlineto{\pgfqpoint{5.068099in}{0.600509in}}%
\pgfpathlineto{\pgfqpoint{5.070214in}{0.605044in}}%
\pgfpathlineto{\pgfqpoint{5.072328in}{0.603375in}}%
\pgfpathlineto{\pgfqpoint{5.076558in}{0.604908in}}%
\pgfpathlineto{\pgfqpoint{5.078672in}{0.601903in}}%
\pgfpathlineto{\pgfqpoint{5.080787in}{0.604610in}}%
\pgfpathlineto{\pgfqpoint{5.082902in}{0.594694in}}%
\pgfpathlineto{\pgfqpoint{5.085016in}{0.593060in}}%
\pgfpathlineto{\pgfqpoint{5.087131in}{0.600793in}}%
\pgfpathlineto{\pgfqpoint{5.089246in}{0.603334in}}%
\pgfpathlineto{\pgfqpoint{5.091360in}{0.600599in}}%
\pgfpathlineto{\pgfqpoint{5.095590in}{0.580500in}}%
\pgfpathlineto{\pgfqpoint{5.097705in}{0.576351in}}%
\pgfpathlineto{\pgfqpoint{5.099819in}{0.578086in}}%
\pgfpathlineto{\pgfqpoint{5.101934in}{0.577938in}}%
\pgfpathlineto{\pgfqpoint{5.104049in}{0.581935in}}%
\pgfpathlineto{\pgfqpoint{5.106163in}{0.578624in}}%
\pgfpathlineto{\pgfqpoint{5.108278in}{0.585948in}}%
\pgfpathlineto{\pgfqpoint{5.112507in}{0.587308in}}%
\pgfpathlineto{\pgfqpoint{5.114622in}{0.593524in}}%
\pgfpathlineto{\pgfqpoint{5.116737in}{0.592899in}}%
\pgfpathlineto{\pgfqpoint{5.120966in}{0.590037in}}%
\pgfpathlineto{\pgfqpoint{5.123081in}{0.592629in}}%
\pgfpathlineto{\pgfqpoint{5.133654in}{0.578265in}}%
\pgfpathlineto{\pgfqpoint{5.135769in}{0.583570in}}%
\pgfpathlineto{\pgfqpoint{5.137884in}{0.584058in}}%
\pgfpathlineto{\pgfqpoint{5.142113in}{0.579344in}}%
\pgfpathlineto{\pgfqpoint{5.146342in}{0.582517in}}%
\pgfpathlineto{\pgfqpoint{5.152687in}{0.601398in}}%
\pgfpathlineto{\pgfqpoint{5.156916in}{0.596305in}}%
\pgfpathlineto{\pgfqpoint{5.159031in}{0.603041in}}%
\pgfpathlineto{\pgfqpoint{5.163260in}{0.603926in}}%
\pgfpathlineto{\pgfqpoint{5.165375in}{0.601590in}}%
\pgfpathlineto{\pgfqpoint{5.169604in}{0.592204in}}%
\pgfpathlineto{\pgfqpoint{5.171719in}{0.594131in}}%
\pgfpathlineto{\pgfqpoint{5.173834in}{0.598711in}}%
\pgfpathlineto{\pgfqpoint{5.175948in}{0.599523in}}%
\pgfpathlineto{\pgfqpoint{5.178063in}{0.593858in}}%
\pgfpathlineto{\pgfqpoint{5.180178in}{0.594807in}}%
\pgfpathlineto{\pgfqpoint{5.184407in}{0.590899in}}%
\pgfpathlineto{\pgfqpoint{5.186522in}{0.594507in}}%
\pgfpathlineto{\pgfqpoint{5.188636in}{0.593279in}}%
\pgfpathlineto{\pgfqpoint{5.188636in}{0.593279in}}%
\pgfusepath{stroke}%
\end{pgfscope}%
\begin{pgfscope}%
\pgfpathrectangle{\pgfqpoint{0.750000in}{0.275000in}}{\pgfqpoint{4.650000in}{1.925000in}}%
\pgfusepath{clip}%
\pgfsetroundcap%
\pgfsetroundjoin%
\pgfsetlinewidth{1.003750pt}%
\definecolor{currentstroke}{rgb}{0.301961,0.686275,0.290196}%
\pgfsetstrokecolor{currentstroke}%
\pgfsetdash{}{0pt}%
\pgfpathmoveto{\pgfqpoint{0.961364in}{1.178573in}}%
\pgfpathlineto{\pgfqpoint{0.965593in}{1.164920in}}%
\pgfpathlineto{\pgfqpoint{0.967708in}{1.173136in}}%
\pgfpathlineto{\pgfqpoint{0.974052in}{1.157524in}}%
\pgfpathlineto{\pgfqpoint{0.976166in}{1.158799in}}%
\pgfpathlineto{\pgfqpoint{0.978281in}{1.162307in}}%
\pgfpathlineto{\pgfqpoint{0.980396in}{1.159705in}}%
\pgfpathlineto{\pgfqpoint{0.982511in}{1.160737in}}%
\pgfpathlineto{\pgfqpoint{0.988855in}{1.154882in}}%
\pgfpathlineto{\pgfqpoint{0.990969in}{1.145695in}}%
\pgfpathlineto{\pgfqpoint{0.993084in}{1.149716in}}%
\pgfpathlineto{\pgfqpoint{0.995199in}{1.156926in}}%
\pgfpathlineto{\pgfqpoint{0.997313in}{1.153407in}}%
\pgfpathlineto{\pgfqpoint{0.999428in}{1.147664in}}%
\pgfpathlineto{\pgfqpoint{1.001543in}{1.148066in}}%
\pgfpathlineto{\pgfqpoint{1.003658in}{1.151245in}}%
\pgfpathlineto{\pgfqpoint{1.005772in}{1.152055in}}%
\pgfpathlineto{\pgfqpoint{1.007887in}{1.151265in}}%
\pgfpathlineto{\pgfqpoint{1.010002in}{1.154241in}}%
\pgfpathlineto{\pgfqpoint{1.014231in}{1.141136in}}%
\pgfpathlineto{\pgfqpoint{1.018460in}{1.143584in}}%
\pgfpathlineto{\pgfqpoint{1.020575in}{1.137933in}}%
\pgfpathlineto{\pgfqpoint{1.022690in}{1.145151in}}%
\pgfpathlineto{\pgfqpoint{1.024804in}{1.144060in}}%
\pgfpathlineto{\pgfqpoint{1.029034in}{1.148084in}}%
\pgfpathlineto{\pgfqpoint{1.033263in}{1.129382in}}%
\pgfpathlineto{\pgfqpoint{1.039607in}{1.118698in}}%
\pgfpathlineto{\pgfqpoint{1.041722in}{1.128821in}}%
\pgfpathlineto{\pgfqpoint{1.043837in}{1.132324in}}%
\pgfpathlineto{\pgfqpoint{1.045951in}{1.131432in}}%
\pgfpathlineto{\pgfqpoint{1.050181in}{1.121989in}}%
\pgfpathlineto{\pgfqpoint{1.052295in}{1.125916in}}%
\pgfpathlineto{\pgfqpoint{1.054410in}{1.122549in}}%
\pgfpathlineto{\pgfqpoint{1.058640in}{1.129589in}}%
\pgfpathlineto{\pgfqpoint{1.060754in}{1.128295in}}%
\pgfpathlineto{\pgfqpoint{1.064984in}{1.121907in}}%
\pgfpathlineto{\pgfqpoint{1.067098in}{1.132499in}}%
\pgfpathlineto{\pgfqpoint{1.071328in}{1.136340in}}%
\pgfpathlineto{\pgfqpoint{1.073442in}{1.134270in}}%
\pgfpathlineto{\pgfqpoint{1.077672in}{1.127172in}}%
\pgfpathlineto{\pgfqpoint{1.079786in}{1.127579in}}%
\pgfpathlineto{\pgfqpoint{1.084016in}{1.132630in}}%
\pgfpathlineto{\pgfqpoint{1.086131in}{1.132379in}}%
\pgfpathlineto{\pgfqpoint{1.088245in}{1.124241in}}%
\pgfpathlineto{\pgfqpoint{1.090360in}{1.123014in}}%
\pgfpathlineto{\pgfqpoint{1.094589in}{1.109767in}}%
\pgfpathlineto{\pgfqpoint{1.098819in}{1.121463in}}%
\pgfpathlineto{\pgfqpoint{1.100933in}{1.119535in}}%
\pgfpathlineto{\pgfqpoint{1.103048in}{1.122744in}}%
\pgfpathlineto{\pgfqpoint{1.105163in}{1.118146in}}%
\pgfpathlineto{\pgfqpoint{1.107278in}{1.118455in}}%
\pgfpathlineto{\pgfqpoint{1.111507in}{1.129584in}}%
\pgfpathlineto{\pgfqpoint{1.113622in}{1.127513in}}%
\pgfpathlineto{\pgfqpoint{1.117851in}{1.118866in}}%
\pgfpathlineto{\pgfqpoint{1.119966in}{1.120982in}}%
\pgfpathlineto{\pgfqpoint{1.122080in}{1.118731in}}%
\pgfpathlineto{\pgfqpoint{1.124195in}{1.122656in}}%
\pgfpathlineto{\pgfqpoint{1.126310in}{1.118170in}}%
\pgfpathlineto{\pgfqpoint{1.128424in}{1.120996in}}%
\pgfpathlineto{\pgfqpoint{1.130539in}{1.118389in}}%
\pgfpathlineto{\pgfqpoint{1.132654in}{1.119497in}}%
\pgfpathlineto{\pgfqpoint{1.134769in}{1.114281in}}%
\pgfpathlineto{\pgfqpoint{1.136883in}{1.114701in}}%
\pgfpathlineto{\pgfqpoint{1.138998in}{1.110732in}}%
\pgfpathlineto{\pgfqpoint{1.141113in}{1.104285in}}%
\pgfpathlineto{\pgfqpoint{1.143227in}{1.107528in}}%
\pgfpathlineto{\pgfqpoint{1.145342in}{1.103758in}}%
\pgfpathlineto{\pgfqpoint{1.147457in}{1.107841in}}%
\pgfpathlineto{\pgfqpoint{1.149571in}{1.109065in}}%
\pgfpathlineto{\pgfqpoint{1.151686in}{1.113988in}}%
\pgfpathlineto{\pgfqpoint{1.153801in}{1.108903in}}%
\pgfpathlineto{\pgfqpoint{1.155915in}{1.109014in}}%
\pgfpathlineto{\pgfqpoint{1.158030in}{1.102844in}}%
\pgfpathlineto{\pgfqpoint{1.160145in}{1.106087in}}%
\pgfpathlineto{\pgfqpoint{1.164374in}{1.094430in}}%
\pgfpathlineto{\pgfqpoint{1.168604in}{1.114524in}}%
\pgfpathlineto{\pgfqpoint{1.170718in}{1.112206in}}%
\pgfpathlineto{\pgfqpoint{1.172833in}{1.108199in}}%
\pgfpathlineto{\pgfqpoint{1.174948in}{1.100250in}}%
\pgfpathlineto{\pgfqpoint{1.177062in}{1.101359in}}%
\pgfpathlineto{\pgfqpoint{1.179177in}{1.099840in}}%
\pgfpathlineto{\pgfqpoint{1.185521in}{1.109623in}}%
\pgfpathlineto{\pgfqpoint{1.187636in}{1.108952in}}%
\pgfpathlineto{\pgfqpoint{1.189751in}{1.110483in}}%
\pgfpathlineto{\pgfqpoint{1.191865in}{1.115704in}}%
\pgfpathlineto{\pgfqpoint{1.193980in}{1.114372in}}%
\pgfpathlineto{\pgfqpoint{1.196095in}{1.118839in}}%
\pgfpathlineto{\pgfqpoint{1.198209in}{1.114894in}}%
\pgfpathlineto{\pgfqpoint{1.200324in}{1.116479in}}%
\pgfpathlineto{\pgfqpoint{1.202439in}{1.114563in}}%
\pgfpathlineto{\pgfqpoint{1.204553in}{1.107267in}}%
\pgfpathlineto{\pgfqpoint{1.206668in}{1.104312in}}%
\pgfpathlineto{\pgfqpoint{1.208783in}{1.104834in}}%
\pgfpathlineto{\pgfqpoint{1.210897in}{1.101272in}}%
\pgfpathlineto{\pgfqpoint{1.217242in}{1.111315in}}%
\pgfpathlineto{\pgfqpoint{1.219356in}{1.109120in}}%
\pgfpathlineto{\pgfqpoint{1.223586in}{1.092580in}}%
\pgfpathlineto{\pgfqpoint{1.225700in}{1.087924in}}%
\pgfpathlineto{\pgfqpoint{1.227815in}{1.099378in}}%
\pgfpathlineto{\pgfqpoint{1.229930in}{1.098401in}}%
\pgfpathlineto{\pgfqpoint{1.232044in}{1.100675in}}%
\pgfpathlineto{\pgfqpoint{1.234159in}{1.099904in}}%
\pgfpathlineto{\pgfqpoint{1.236274in}{1.096265in}}%
\pgfpathlineto{\pgfqpoint{1.238389in}{1.101544in}}%
\pgfpathlineto{\pgfqpoint{1.240503in}{1.103525in}}%
\pgfpathlineto{\pgfqpoint{1.244733in}{1.112272in}}%
\pgfpathlineto{\pgfqpoint{1.246847in}{1.111279in}}%
\pgfpathlineto{\pgfqpoint{1.248962in}{1.113140in}}%
\pgfpathlineto{\pgfqpoint{1.251077in}{1.119445in}}%
\pgfpathlineto{\pgfqpoint{1.253191in}{1.112572in}}%
\pgfpathlineto{\pgfqpoint{1.255306in}{1.110837in}}%
\pgfpathlineto{\pgfqpoint{1.259535in}{1.115069in}}%
\pgfpathlineto{\pgfqpoint{1.263765in}{1.117554in}}%
\pgfpathlineto{\pgfqpoint{1.265880in}{1.110103in}}%
\pgfpathlineto{\pgfqpoint{1.267994in}{1.113019in}}%
\pgfpathlineto{\pgfqpoint{1.272224in}{1.103372in}}%
\pgfpathlineto{\pgfqpoint{1.274338in}{1.102605in}}%
\pgfpathlineto{\pgfqpoint{1.278568in}{1.095233in}}%
\pgfpathlineto{\pgfqpoint{1.280682in}{1.096329in}}%
\pgfpathlineto{\pgfqpoint{1.284912in}{1.100830in}}%
\pgfpathlineto{\pgfqpoint{1.291256in}{1.100922in}}%
\pgfpathlineto{\pgfqpoint{1.299715in}{1.096083in}}%
\pgfpathlineto{\pgfqpoint{1.301829in}{1.098281in}}%
\pgfpathlineto{\pgfqpoint{1.308173in}{1.099342in}}%
\pgfpathlineto{\pgfqpoint{1.314517in}{1.092491in}}%
\pgfpathlineto{\pgfqpoint{1.316632in}{1.098976in}}%
\pgfpathlineto{\pgfqpoint{1.318747in}{1.101470in}}%
\pgfpathlineto{\pgfqpoint{1.320862in}{1.098155in}}%
\pgfpathlineto{\pgfqpoint{1.322976in}{1.097032in}}%
\pgfpathlineto{\pgfqpoint{1.327206in}{1.087552in}}%
\pgfpathlineto{\pgfqpoint{1.329320in}{1.083488in}}%
\pgfpathlineto{\pgfqpoint{1.331435in}{1.075349in}}%
\pgfpathlineto{\pgfqpoint{1.333550in}{1.078056in}}%
\pgfpathlineto{\pgfqpoint{1.335664in}{1.077850in}}%
\pgfpathlineto{\pgfqpoint{1.339894in}{1.084295in}}%
\pgfpathlineto{\pgfqpoint{1.342009in}{1.075694in}}%
\pgfpathlineto{\pgfqpoint{1.344123in}{1.077991in}}%
\pgfpathlineto{\pgfqpoint{1.346238in}{1.073697in}}%
\pgfpathlineto{\pgfqpoint{1.348353in}{1.072348in}}%
\pgfpathlineto{\pgfqpoint{1.350467in}{1.074260in}}%
\pgfpathlineto{\pgfqpoint{1.354697in}{1.060413in}}%
\pgfpathlineto{\pgfqpoint{1.356811in}{1.063681in}}%
\pgfpathlineto{\pgfqpoint{1.358926in}{1.064803in}}%
\pgfpathlineto{\pgfqpoint{1.361041in}{1.060545in}}%
\pgfpathlineto{\pgfqpoint{1.363155in}{1.059621in}}%
\pgfpathlineto{\pgfqpoint{1.365270in}{1.063220in}}%
\pgfpathlineto{\pgfqpoint{1.367385in}{1.063625in}}%
\pgfpathlineto{\pgfqpoint{1.371614in}{1.060685in}}%
\pgfpathlineto{\pgfqpoint{1.373729in}{1.067113in}}%
\pgfpathlineto{\pgfqpoint{1.375844in}{1.059929in}}%
\pgfpathlineto{\pgfqpoint{1.377958in}{1.062286in}}%
\pgfpathlineto{\pgfqpoint{1.380073in}{1.062108in}}%
\pgfpathlineto{\pgfqpoint{1.382188in}{1.049985in}}%
\pgfpathlineto{\pgfqpoint{1.384302in}{1.049477in}}%
\pgfpathlineto{\pgfqpoint{1.386417in}{1.045269in}}%
\pgfpathlineto{\pgfqpoint{1.388532in}{1.045358in}}%
\pgfpathlineto{\pgfqpoint{1.390646in}{1.043546in}}%
\pgfpathlineto{\pgfqpoint{1.392761in}{1.038616in}}%
\pgfpathlineto{\pgfqpoint{1.394876in}{1.042891in}}%
\pgfpathlineto{\pgfqpoint{1.396991in}{1.049858in}}%
\pgfpathlineto{\pgfqpoint{1.399105in}{1.047800in}}%
\pgfpathlineto{\pgfqpoint{1.401220in}{1.048992in}}%
\pgfpathlineto{\pgfqpoint{1.405449in}{1.053739in}}%
\pgfpathlineto{\pgfqpoint{1.407564in}{1.043211in}}%
\pgfpathlineto{\pgfqpoint{1.409679in}{1.050664in}}%
\pgfpathlineto{\pgfqpoint{1.411793in}{1.051453in}}%
\pgfpathlineto{\pgfqpoint{1.416023in}{1.056289in}}%
\pgfpathlineto{\pgfqpoint{1.418137in}{1.057369in}}%
\pgfpathlineto{\pgfqpoint{1.420252in}{1.064032in}}%
\pgfpathlineto{\pgfqpoint{1.422367in}{1.064674in}}%
\pgfpathlineto{\pgfqpoint{1.424482in}{1.071486in}}%
\pgfpathlineto{\pgfqpoint{1.428711in}{1.072802in}}%
\pgfpathlineto{\pgfqpoint{1.435055in}{1.082530in}}%
\pgfpathlineto{\pgfqpoint{1.439284in}{1.071930in}}%
\pgfpathlineto{\pgfqpoint{1.445628in}{1.080107in}}%
\pgfpathlineto{\pgfqpoint{1.447743in}{1.088769in}}%
\pgfpathlineto{\pgfqpoint{1.449858in}{1.081210in}}%
\pgfpathlineto{\pgfqpoint{1.451973in}{1.085573in}}%
\pgfpathlineto{\pgfqpoint{1.456202in}{1.079059in}}%
\pgfpathlineto{\pgfqpoint{1.458317in}{1.085131in}}%
\pgfpathlineto{\pgfqpoint{1.460431in}{1.080918in}}%
\pgfpathlineto{\pgfqpoint{1.462546in}{1.082599in}}%
\pgfpathlineto{\pgfqpoint{1.464661in}{1.078697in}}%
\pgfpathlineto{\pgfqpoint{1.466775in}{1.082714in}}%
\pgfpathlineto{\pgfqpoint{1.468890in}{1.090946in}}%
\pgfpathlineto{\pgfqpoint{1.471005in}{1.086845in}}%
\pgfpathlineto{\pgfqpoint{1.473120in}{1.085906in}}%
\pgfpathlineto{\pgfqpoint{1.477349in}{1.070658in}}%
\pgfpathlineto{\pgfqpoint{1.479464in}{1.066159in}}%
\pgfpathlineto{\pgfqpoint{1.481578in}{1.065576in}}%
\pgfpathlineto{\pgfqpoint{1.490037in}{1.051799in}}%
\pgfpathlineto{\pgfqpoint{1.492152in}{1.051430in}}%
\pgfpathlineto{\pgfqpoint{1.496381in}{1.066119in}}%
\pgfpathlineto{\pgfqpoint{1.498496in}{1.063091in}}%
\pgfpathlineto{\pgfqpoint{1.500611in}{1.064979in}}%
\pgfpathlineto{\pgfqpoint{1.502725in}{1.069190in}}%
\pgfpathlineto{\pgfqpoint{1.506955in}{1.063244in}}%
\pgfpathlineto{\pgfqpoint{1.509069in}{1.063875in}}%
\pgfpathlineto{\pgfqpoint{1.511184in}{1.060487in}}%
\pgfpathlineto{\pgfqpoint{1.515413in}{1.067332in}}%
\pgfpathlineto{\pgfqpoint{1.517528in}{1.064988in}}%
\pgfpathlineto{\pgfqpoint{1.519643in}{1.064867in}}%
\pgfpathlineto{\pgfqpoint{1.521757in}{1.061400in}}%
\pgfpathlineto{\pgfqpoint{1.528102in}{1.075690in}}%
\pgfpathlineto{\pgfqpoint{1.530216in}{1.076571in}}%
\pgfpathlineto{\pgfqpoint{1.534446in}{1.075583in}}%
\pgfpathlineto{\pgfqpoint{1.536560in}{1.083915in}}%
\pgfpathlineto{\pgfqpoint{1.538675in}{1.080930in}}%
\pgfpathlineto{\pgfqpoint{1.540790in}{1.092393in}}%
\pgfpathlineto{\pgfqpoint{1.542904in}{1.091759in}}%
\pgfpathlineto{\pgfqpoint{1.545019in}{1.095657in}}%
\pgfpathlineto{\pgfqpoint{1.547134in}{1.088872in}}%
\pgfpathlineto{\pgfqpoint{1.549248in}{1.100729in}}%
\pgfpathlineto{\pgfqpoint{1.551363in}{1.100128in}}%
\pgfpathlineto{\pgfqpoint{1.555593in}{1.107369in}}%
\pgfpathlineto{\pgfqpoint{1.557707in}{1.111320in}}%
\pgfpathlineto{\pgfqpoint{1.559822in}{1.107975in}}%
\pgfpathlineto{\pgfqpoint{1.561937in}{1.109528in}}%
\pgfpathlineto{\pgfqpoint{1.564051in}{1.108485in}}%
\pgfpathlineto{\pgfqpoint{1.566166in}{1.105681in}}%
\pgfpathlineto{\pgfqpoint{1.570395in}{1.108657in}}%
\pgfpathlineto{\pgfqpoint{1.574625in}{1.100524in}}%
\pgfpathlineto{\pgfqpoint{1.576740in}{1.109899in}}%
\pgfpathlineto{\pgfqpoint{1.578854in}{1.105723in}}%
\pgfpathlineto{\pgfqpoint{1.580969in}{1.108048in}}%
\pgfpathlineto{\pgfqpoint{1.583084in}{1.101965in}}%
\pgfpathlineto{\pgfqpoint{1.585198in}{1.103728in}}%
\pgfpathlineto{\pgfqpoint{1.589428in}{1.100044in}}%
\pgfpathlineto{\pgfqpoint{1.593657in}{1.103787in}}%
\pgfpathlineto{\pgfqpoint{1.595772in}{1.104396in}}%
\pgfpathlineto{\pgfqpoint{1.600001in}{1.112479in}}%
\pgfpathlineto{\pgfqpoint{1.602116in}{1.107604in}}%
\pgfpathlineto{\pgfqpoint{1.604231in}{1.108250in}}%
\pgfpathlineto{\pgfqpoint{1.606345in}{1.107292in}}%
\pgfpathlineto{\pgfqpoint{1.608460in}{1.109144in}}%
\pgfpathlineto{\pgfqpoint{1.612689in}{1.108895in}}%
\pgfpathlineto{\pgfqpoint{1.614804in}{1.109496in}}%
\pgfpathlineto{\pgfqpoint{1.616919in}{1.111646in}}%
\pgfpathlineto{\pgfqpoint{1.619033in}{1.110036in}}%
\pgfpathlineto{\pgfqpoint{1.623263in}{1.121503in}}%
\pgfpathlineto{\pgfqpoint{1.629607in}{1.112193in}}%
\pgfpathlineto{\pgfqpoint{1.631722in}{1.109946in}}%
\pgfpathlineto{\pgfqpoint{1.633836in}{1.115801in}}%
\pgfpathlineto{\pgfqpoint{1.635951in}{1.111131in}}%
\pgfpathlineto{\pgfqpoint{1.638066in}{1.112594in}}%
\pgfpathlineto{\pgfqpoint{1.640180in}{1.118528in}}%
\pgfpathlineto{\pgfqpoint{1.642295in}{1.119583in}}%
\pgfpathlineto{\pgfqpoint{1.644410in}{1.123827in}}%
\pgfpathlineto{\pgfqpoint{1.646524in}{1.118390in}}%
\pgfpathlineto{\pgfqpoint{1.648639in}{1.117933in}}%
\pgfpathlineto{\pgfqpoint{1.650754in}{1.118851in}}%
\pgfpathlineto{\pgfqpoint{1.652868in}{1.114610in}}%
\pgfpathlineto{\pgfqpoint{1.657098in}{1.126124in}}%
\pgfpathlineto{\pgfqpoint{1.661327in}{1.120346in}}%
\pgfpathlineto{\pgfqpoint{1.663442in}{1.116908in}}%
\pgfpathlineto{\pgfqpoint{1.665557in}{1.117715in}}%
\pgfpathlineto{\pgfqpoint{1.667671in}{1.123401in}}%
\pgfpathlineto{\pgfqpoint{1.669786in}{1.118184in}}%
\pgfpathlineto{\pgfqpoint{1.671901in}{1.116976in}}%
\pgfpathlineto{\pgfqpoint{1.674015in}{1.127725in}}%
\pgfpathlineto{\pgfqpoint{1.676130in}{1.124593in}}%
\pgfpathlineto{\pgfqpoint{1.678245in}{1.133279in}}%
\pgfpathlineto{\pgfqpoint{1.682474in}{1.126503in}}%
\pgfpathlineto{\pgfqpoint{1.684589in}{1.128349in}}%
\pgfpathlineto{\pgfqpoint{1.686704in}{1.126647in}}%
\pgfpathlineto{\pgfqpoint{1.688818in}{1.127039in}}%
\pgfpathlineto{\pgfqpoint{1.690933in}{1.122422in}}%
\pgfpathlineto{\pgfqpoint{1.693048in}{1.128458in}}%
\pgfpathlineto{\pgfqpoint{1.695162in}{1.129837in}}%
\pgfpathlineto{\pgfqpoint{1.697277in}{1.127541in}}%
\pgfpathlineto{\pgfqpoint{1.699392in}{1.136014in}}%
\pgfpathlineto{\pgfqpoint{1.701506in}{1.131019in}}%
\pgfpathlineto{\pgfqpoint{1.703621in}{1.132395in}}%
\pgfpathlineto{\pgfqpoint{1.705736in}{1.130086in}}%
\pgfpathlineto{\pgfqpoint{1.707851in}{1.132681in}}%
\pgfpathlineto{\pgfqpoint{1.712080in}{1.120004in}}%
\pgfpathlineto{\pgfqpoint{1.716309in}{1.127098in}}%
\pgfpathlineto{\pgfqpoint{1.718424in}{1.120978in}}%
\pgfpathlineto{\pgfqpoint{1.720539in}{1.126450in}}%
\pgfpathlineto{\pgfqpoint{1.724768in}{1.119055in}}%
\pgfpathlineto{\pgfqpoint{1.726883in}{1.123105in}}%
\pgfpathlineto{\pgfqpoint{1.728997in}{1.123738in}}%
\pgfpathlineto{\pgfqpoint{1.731112in}{1.122138in}}%
\pgfpathlineto{\pgfqpoint{1.733227in}{1.115815in}}%
\pgfpathlineto{\pgfqpoint{1.735342in}{1.117940in}}%
\pgfpathlineto{\pgfqpoint{1.739571in}{1.111093in}}%
\pgfpathlineto{\pgfqpoint{1.741686in}{1.111304in}}%
\pgfpathlineto{\pgfqpoint{1.743800in}{1.109257in}}%
\pgfpathlineto{\pgfqpoint{1.745915in}{1.110812in}}%
\pgfpathlineto{\pgfqpoint{1.748030in}{1.109639in}}%
\pgfpathlineto{\pgfqpoint{1.754374in}{1.121868in}}%
\pgfpathlineto{\pgfqpoint{1.756488in}{1.119719in}}%
\pgfpathlineto{\pgfqpoint{1.758603in}{1.123855in}}%
\pgfpathlineto{\pgfqpoint{1.760718in}{1.121153in}}%
\pgfpathlineto{\pgfqpoint{1.762833in}{1.120268in}}%
\pgfpathlineto{\pgfqpoint{1.764947in}{1.128603in}}%
\pgfpathlineto{\pgfqpoint{1.767062in}{1.129718in}}%
\pgfpathlineto{\pgfqpoint{1.769177in}{1.127841in}}%
\pgfpathlineto{\pgfqpoint{1.771291in}{1.128630in}}%
\pgfpathlineto{\pgfqpoint{1.773406in}{1.128225in}}%
\pgfpathlineto{\pgfqpoint{1.779750in}{1.140851in}}%
\pgfpathlineto{\pgfqpoint{1.781865in}{1.137742in}}%
\pgfpathlineto{\pgfqpoint{1.783979in}{1.139491in}}%
\pgfpathlineto{\pgfqpoint{1.786094in}{1.136872in}}%
\pgfpathlineto{\pgfqpoint{1.788209in}{1.136073in}}%
\pgfpathlineto{\pgfqpoint{1.790324in}{1.133225in}}%
\pgfpathlineto{\pgfqpoint{1.794553in}{1.126324in}}%
\pgfpathlineto{\pgfqpoint{1.796668in}{1.115506in}}%
\pgfpathlineto{\pgfqpoint{1.805126in}{1.113797in}}%
\pgfpathlineto{\pgfqpoint{1.807241in}{1.117090in}}%
\pgfpathlineto{\pgfqpoint{1.809356in}{1.111725in}}%
\pgfpathlineto{\pgfqpoint{1.813585in}{1.109370in}}%
\pgfpathlineto{\pgfqpoint{1.815700in}{1.108099in}}%
\pgfpathlineto{\pgfqpoint{1.817815in}{1.108593in}}%
\pgfpathlineto{\pgfqpoint{1.822044in}{1.117173in}}%
\pgfpathlineto{\pgfqpoint{1.826273in}{1.110961in}}%
\pgfpathlineto{\pgfqpoint{1.828388in}{1.113235in}}%
\pgfpathlineto{\pgfqpoint{1.830503in}{1.111274in}}%
\pgfpathlineto{\pgfqpoint{1.832617in}{1.116491in}}%
\pgfpathlineto{\pgfqpoint{1.836847in}{1.105890in}}%
\pgfpathlineto{\pgfqpoint{1.838962in}{1.108111in}}%
\pgfpathlineto{\pgfqpoint{1.841076in}{1.102953in}}%
\pgfpathlineto{\pgfqpoint{1.843191in}{1.105927in}}%
\pgfpathlineto{\pgfqpoint{1.847420in}{1.095653in}}%
\pgfpathlineto{\pgfqpoint{1.853764in}{1.086801in}}%
\pgfpathlineto{\pgfqpoint{1.855879in}{1.088115in}}%
\pgfpathlineto{\pgfqpoint{1.857994in}{1.095361in}}%
\pgfpathlineto{\pgfqpoint{1.860108in}{1.095286in}}%
\pgfpathlineto{\pgfqpoint{1.862223in}{1.096327in}}%
\pgfpathlineto{\pgfqpoint{1.868567in}{1.105717in}}%
\pgfpathlineto{\pgfqpoint{1.870682in}{1.104958in}}%
\pgfpathlineto{\pgfqpoint{1.874911in}{1.098598in}}%
\pgfpathlineto{\pgfqpoint{1.877026in}{1.090594in}}%
\pgfpathlineto{\pgfqpoint{1.883370in}{1.082635in}}%
\pgfpathlineto{\pgfqpoint{1.885485in}{1.075969in}}%
\pgfpathlineto{\pgfqpoint{1.887599in}{1.085778in}}%
\pgfpathlineto{\pgfqpoint{1.889714in}{1.086521in}}%
\pgfpathlineto{\pgfqpoint{1.891829in}{1.094654in}}%
\pgfpathlineto{\pgfqpoint{1.893944in}{1.093602in}}%
\pgfpathlineto{\pgfqpoint{1.896058in}{1.090612in}}%
\pgfpathlineto{\pgfqpoint{1.900288in}{1.080399in}}%
\pgfpathlineto{\pgfqpoint{1.902402in}{1.081096in}}%
\pgfpathlineto{\pgfqpoint{1.904517in}{1.076750in}}%
\pgfpathlineto{\pgfqpoint{1.906632in}{1.076903in}}%
\pgfpathlineto{\pgfqpoint{1.908746in}{1.073433in}}%
\pgfpathlineto{\pgfqpoint{1.910861in}{1.075157in}}%
\pgfpathlineto{\pgfqpoint{1.912976in}{1.071706in}}%
\pgfpathlineto{\pgfqpoint{1.915090in}{1.076755in}}%
\pgfpathlineto{\pgfqpoint{1.917205in}{1.074281in}}%
\pgfpathlineto{\pgfqpoint{1.921435in}{1.081170in}}%
\pgfpathlineto{\pgfqpoint{1.925664in}{1.081715in}}%
\pgfpathlineto{\pgfqpoint{1.927779in}{1.085535in}}%
\pgfpathlineto{\pgfqpoint{1.932008in}{1.088658in}}%
\pgfpathlineto{\pgfqpoint{1.934123in}{1.094323in}}%
\pgfpathlineto{\pgfqpoint{1.936237in}{1.093136in}}%
\pgfpathlineto{\pgfqpoint{1.938352in}{1.095426in}}%
\pgfpathlineto{\pgfqpoint{1.940467in}{1.096070in}}%
\pgfpathlineto{\pgfqpoint{1.942582in}{1.098441in}}%
\pgfpathlineto{\pgfqpoint{1.946811in}{1.097617in}}%
\pgfpathlineto{\pgfqpoint{1.948926in}{1.095106in}}%
\pgfpathlineto{\pgfqpoint{1.951040in}{1.090147in}}%
\pgfpathlineto{\pgfqpoint{1.953155in}{1.089651in}}%
\pgfpathlineto{\pgfqpoint{1.955270in}{1.093746in}}%
\pgfpathlineto{\pgfqpoint{1.957384in}{1.092567in}}%
\pgfpathlineto{\pgfqpoint{1.959499in}{1.082987in}}%
\pgfpathlineto{\pgfqpoint{1.961614in}{1.084347in}}%
\pgfpathlineto{\pgfqpoint{1.963728in}{1.083325in}}%
\pgfpathlineto{\pgfqpoint{1.965843in}{1.090021in}}%
\pgfpathlineto{\pgfqpoint{1.970073in}{1.091670in}}%
\pgfpathlineto{\pgfqpoint{1.972187in}{1.090034in}}%
\pgfpathlineto{\pgfqpoint{1.976417in}{1.080459in}}%
\pgfpathlineto{\pgfqpoint{1.978531in}{1.080773in}}%
\pgfpathlineto{\pgfqpoint{1.980646in}{1.074565in}}%
\pgfpathlineto{\pgfqpoint{1.982761in}{1.081874in}}%
\pgfpathlineto{\pgfqpoint{1.984875in}{1.081713in}}%
\pgfpathlineto{\pgfqpoint{1.986990in}{1.085497in}}%
\pgfpathlineto{\pgfqpoint{1.989105in}{1.076892in}}%
\pgfpathlineto{\pgfqpoint{1.991219in}{1.081200in}}%
\pgfpathlineto{\pgfqpoint{1.995449in}{1.073174in}}%
\pgfpathlineto{\pgfqpoint{1.997564in}{1.073181in}}%
\pgfpathlineto{\pgfqpoint{1.999678in}{1.079373in}}%
\pgfpathlineto{\pgfqpoint{2.001793in}{1.078507in}}%
\pgfpathlineto{\pgfqpoint{2.003908in}{1.080335in}}%
\pgfpathlineto{\pgfqpoint{2.006022in}{1.074690in}}%
\pgfpathlineto{\pgfqpoint{2.008137in}{1.076572in}}%
\pgfpathlineto{\pgfqpoint{2.010252in}{1.063599in}}%
\pgfpathlineto{\pgfqpoint{2.016596in}{1.055340in}}%
\pgfpathlineto{\pgfqpoint{2.018710in}{1.057339in}}%
\pgfpathlineto{\pgfqpoint{2.020825in}{1.057169in}}%
\pgfpathlineto{\pgfqpoint{2.025055in}{1.060029in}}%
\pgfpathlineto{\pgfqpoint{2.029284in}{1.069570in}}%
\pgfpathlineto{\pgfqpoint{2.033513in}{1.072049in}}%
\pgfpathlineto{\pgfqpoint{2.035628in}{1.076128in}}%
\pgfpathlineto{\pgfqpoint{2.041972in}{1.068172in}}%
\pgfpathlineto{\pgfqpoint{2.044087in}{1.071170in}}%
\pgfpathlineto{\pgfqpoint{2.046202in}{1.064925in}}%
\pgfpathlineto{\pgfqpoint{2.048316in}{1.069013in}}%
\pgfpathlineto{\pgfqpoint{2.050431in}{1.068793in}}%
\pgfpathlineto{\pgfqpoint{2.052546in}{1.066833in}}%
\pgfpathlineto{\pgfqpoint{2.054660in}{1.068963in}}%
\pgfpathlineto{\pgfqpoint{2.056775in}{1.068110in}}%
\pgfpathlineto{\pgfqpoint{2.061004in}{1.076101in}}%
\pgfpathlineto{\pgfqpoint{2.063119in}{1.066510in}}%
\pgfpathlineto{\pgfqpoint{2.065234in}{1.071978in}}%
\pgfpathlineto{\pgfqpoint{2.071578in}{1.077448in}}%
\pgfpathlineto{\pgfqpoint{2.075807in}{1.065642in}}%
\pgfpathlineto{\pgfqpoint{2.077922in}{1.070199in}}%
\pgfpathlineto{\pgfqpoint{2.080037in}{1.066371in}}%
\pgfpathlineto{\pgfqpoint{2.082151in}{1.069836in}}%
\pgfpathlineto{\pgfqpoint{2.084266in}{1.067256in}}%
\pgfpathlineto{\pgfqpoint{2.086381in}{1.069313in}}%
\pgfpathlineto{\pgfqpoint{2.090610in}{1.062695in}}%
\pgfpathlineto{\pgfqpoint{2.094839in}{1.067482in}}%
\pgfpathlineto{\pgfqpoint{2.096954in}{1.062109in}}%
\pgfpathlineto{\pgfqpoint{2.101184in}{1.062376in}}%
\pgfpathlineto{\pgfqpoint{2.105413in}{1.074907in}}%
\pgfpathlineto{\pgfqpoint{2.107528in}{1.075034in}}%
\pgfpathlineto{\pgfqpoint{2.109642in}{1.069205in}}%
\pgfpathlineto{\pgfqpoint{2.111757in}{1.067087in}}%
\pgfpathlineto{\pgfqpoint{2.113872in}{1.057422in}}%
\pgfpathlineto{\pgfqpoint{2.118101in}{1.051622in}}%
\pgfpathlineto{\pgfqpoint{2.122330in}{1.041187in}}%
\pgfpathlineto{\pgfqpoint{2.124445in}{1.032238in}}%
\pgfpathlineto{\pgfqpoint{2.130789in}{1.043083in}}%
\pgfpathlineto{\pgfqpoint{2.137133in}{1.030102in}}%
\pgfpathlineto{\pgfqpoint{2.139248in}{1.027416in}}%
\pgfpathlineto{\pgfqpoint{2.143477in}{1.027290in}}%
\pgfpathlineto{\pgfqpoint{2.145592in}{1.025777in}}%
\pgfpathlineto{\pgfqpoint{2.147707in}{1.027331in}}%
\pgfpathlineto{\pgfqpoint{2.149822in}{1.031441in}}%
\pgfpathlineto{\pgfqpoint{2.151936in}{1.039806in}}%
\pgfpathlineto{\pgfqpoint{2.154051in}{1.038990in}}%
\pgfpathlineto{\pgfqpoint{2.156166in}{1.032219in}}%
\pgfpathlineto{\pgfqpoint{2.158280in}{1.032983in}}%
\pgfpathlineto{\pgfqpoint{2.164624in}{1.050196in}}%
\pgfpathlineto{\pgfqpoint{2.166739in}{1.048473in}}%
\pgfpathlineto{\pgfqpoint{2.170968in}{1.054877in}}%
\pgfpathlineto{\pgfqpoint{2.173083in}{1.056735in}}%
\pgfpathlineto{\pgfqpoint{2.177313in}{1.065707in}}%
\pgfpathlineto{\pgfqpoint{2.179427in}{1.065593in}}%
\pgfpathlineto{\pgfqpoint{2.183657in}{1.063284in}}%
\pgfpathlineto{\pgfqpoint{2.185771in}{1.062962in}}%
\pgfpathlineto{\pgfqpoint{2.190001in}{1.076169in}}%
\pgfpathlineto{\pgfqpoint{2.192115in}{1.067487in}}%
\pgfpathlineto{\pgfqpoint{2.194230in}{1.065631in}}%
\pgfpathlineto{\pgfqpoint{2.196345in}{1.070948in}}%
\pgfpathlineto{\pgfqpoint{2.198459in}{1.071440in}}%
\pgfpathlineto{\pgfqpoint{2.200574in}{1.062451in}}%
\pgfpathlineto{\pgfqpoint{2.202689in}{1.065625in}}%
\pgfpathlineto{\pgfqpoint{2.204804in}{1.066194in}}%
\pgfpathlineto{\pgfqpoint{2.209033in}{1.059700in}}%
\pgfpathlineto{\pgfqpoint{2.213262in}{1.057639in}}%
\pgfpathlineto{\pgfqpoint{2.215377in}{1.059121in}}%
\pgfpathlineto{\pgfqpoint{2.217492in}{1.056816in}}%
\pgfpathlineto{\pgfqpoint{2.219606in}{1.052783in}}%
\pgfpathlineto{\pgfqpoint{2.221721in}{1.054069in}}%
\pgfpathlineto{\pgfqpoint{2.223836in}{1.054007in}}%
\pgfpathlineto{\pgfqpoint{2.228065in}{1.037002in}}%
\pgfpathlineto{\pgfqpoint{2.230180in}{1.041151in}}%
\pgfpathlineto{\pgfqpoint{2.234409in}{1.050984in}}%
\pgfpathlineto{\pgfqpoint{2.236524in}{1.056373in}}%
\pgfpathlineto{\pgfqpoint{2.242868in}{1.043227in}}%
\pgfpathlineto{\pgfqpoint{2.247097in}{1.046619in}}%
\pgfpathlineto{\pgfqpoint{2.251327in}{1.040332in}}%
\pgfpathlineto{\pgfqpoint{2.253441in}{1.037315in}}%
\pgfpathlineto{\pgfqpoint{2.255556in}{1.038792in}}%
\pgfpathlineto{\pgfqpoint{2.257671in}{1.038780in}}%
\pgfpathlineto{\pgfqpoint{2.261900in}{1.034528in}}%
\pgfpathlineto{\pgfqpoint{2.266130in}{1.040340in}}%
\pgfpathlineto{\pgfqpoint{2.268244in}{1.038989in}}%
\pgfpathlineto{\pgfqpoint{2.274588in}{1.044129in}}%
\pgfpathlineto{\pgfqpoint{2.278818in}{1.040764in}}%
\pgfpathlineto{\pgfqpoint{2.280933in}{1.043107in}}%
\pgfpathlineto{\pgfqpoint{2.283047in}{1.049271in}}%
\pgfpathlineto{\pgfqpoint{2.285162in}{1.046256in}}%
\pgfpathlineto{\pgfqpoint{2.287277in}{1.047619in}}%
\pgfpathlineto{\pgfqpoint{2.289391in}{1.051702in}}%
\pgfpathlineto{\pgfqpoint{2.293621in}{1.050323in}}%
\pgfpathlineto{\pgfqpoint{2.295735in}{1.053369in}}%
\pgfpathlineto{\pgfqpoint{2.302079in}{1.046205in}}%
\pgfpathlineto{\pgfqpoint{2.304194in}{1.055242in}}%
\pgfpathlineto{\pgfqpoint{2.308424in}{1.060623in}}%
\pgfpathlineto{\pgfqpoint{2.310538in}{1.061481in}}%
\pgfpathlineto{\pgfqpoint{2.314768in}{1.078254in}}%
\pgfpathlineto{\pgfqpoint{2.318997in}{1.082605in}}%
\pgfpathlineto{\pgfqpoint{2.321112in}{1.080901in}}%
\pgfpathlineto{\pgfqpoint{2.323226in}{1.074498in}}%
\pgfpathlineto{\pgfqpoint{2.325341in}{1.072916in}}%
\pgfpathlineto{\pgfqpoint{2.327456in}{1.075674in}}%
\pgfpathlineto{\pgfqpoint{2.331685in}{1.069780in}}%
\pgfpathlineto{\pgfqpoint{2.333800in}{1.072141in}}%
\pgfpathlineto{\pgfqpoint{2.335915in}{1.067783in}}%
\pgfpathlineto{\pgfqpoint{2.338029in}{1.070279in}}%
\pgfpathlineto{\pgfqpoint{2.340144in}{1.065420in}}%
\pgfpathlineto{\pgfqpoint{2.342259in}{1.057201in}}%
\pgfpathlineto{\pgfqpoint{2.346488in}{1.064272in}}%
\pgfpathlineto{\pgfqpoint{2.348603in}{1.069576in}}%
\pgfpathlineto{\pgfqpoint{2.352832in}{1.059859in}}%
\pgfpathlineto{\pgfqpoint{2.354947in}{1.059448in}}%
\pgfpathlineto{\pgfqpoint{2.357061in}{1.060472in}}%
\pgfpathlineto{\pgfqpoint{2.359176in}{1.058573in}}%
\pgfpathlineto{\pgfqpoint{2.361291in}{1.052095in}}%
\pgfpathlineto{\pgfqpoint{2.363406in}{1.057860in}}%
\pgfpathlineto{\pgfqpoint{2.365520in}{1.058063in}}%
\pgfpathlineto{\pgfqpoint{2.367635in}{1.055029in}}%
\pgfpathlineto{\pgfqpoint{2.371864in}{1.061294in}}%
\pgfpathlineto{\pgfqpoint{2.373979in}{1.053590in}}%
\pgfpathlineto{\pgfqpoint{2.376094in}{1.056475in}}%
\pgfpathlineto{\pgfqpoint{2.380323in}{1.053822in}}%
\pgfpathlineto{\pgfqpoint{2.384553in}{1.058818in}}%
\pgfpathlineto{\pgfqpoint{2.386667in}{1.052069in}}%
\pgfpathlineto{\pgfqpoint{2.390897in}{1.048333in}}%
\pgfpathlineto{\pgfqpoint{2.395126in}{1.045727in}}%
\pgfpathlineto{\pgfqpoint{2.397241in}{1.041218in}}%
\pgfpathlineto{\pgfqpoint{2.401470in}{1.042450in}}%
\pgfpathlineto{\pgfqpoint{2.403585in}{1.049840in}}%
\pgfpathlineto{\pgfqpoint{2.405699in}{1.042065in}}%
\pgfpathlineto{\pgfqpoint{2.412044in}{1.037061in}}%
\pgfpathlineto{\pgfqpoint{2.414158in}{1.039248in}}%
\pgfpathlineto{\pgfqpoint{2.416273in}{1.035942in}}%
\pgfpathlineto{\pgfqpoint{2.418388in}{1.035707in}}%
\pgfpathlineto{\pgfqpoint{2.420502in}{1.038581in}}%
\pgfpathlineto{\pgfqpoint{2.422617in}{1.035439in}}%
\pgfpathlineto{\pgfqpoint{2.424732in}{1.026179in}}%
\pgfpathlineto{\pgfqpoint{2.426846in}{1.028000in}}%
\pgfpathlineto{\pgfqpoint{2.428961in}{1.026175in}}%
\pgfpathlineto{\pgfqpoint{2.431076in}{1.020161in}}%
\pgfpathlineto{\pgfqpoint{2.433190in}{1.023873in}}%
\pgfpathlineto{\pgfqpoint{2.435305in}{1.016139in}}%
\pgfpathlineto{\pgfqpoint{2.441649in}{1.020826in}}%
\pgfpathlineto{\pgfqpoint{2.443764in}{1.017489in}}%
\pgfpathlineto{\pgfqpoint{2.445879in}{1.017335in}}%
\pgfpathlineto{\pgfqpoint{2.447993in}{1.022007in}}%
\pgfpathlineto{\pgfqpoint{2.450108in}{1.023082in}}%
\pgfpathlineto{\pgfqpoint{2.454337in}{1.014490in}}%
\pgfpathlineto{\pgfqpoint{2.456452in}{1.008500in}}%
\pgfpathlineto{\pgfqpoint{2.460681in}{1.004306in}}%
\pgfpathlineto{\pgfqpoint{2.462796in}{1.004612in}}%
\pgfpathlineto{\pgfqpoint{2.464911in}{1.008996in}}%
\pgfpathlineto{\pgfqpoint{2.469140in}{1.007348in}}%
\pgfpathlineto{\pgfqpoint{2.475484in}{1.023469in}}%
\pgfpathlineto{\pgfqpoint{2.479714in}{1.027186in}}%
\pgfpathlineto{\pgfqpoint{2.483943in}{1.026384in}}%
\pgfpathlineto{\pgfqpoint{2.486058in}{1.027516in}}%
\pgfpathlineto{\pgfqpoint{2.490287in}{1.025535in}}%
\pgfpathlineto{\pgfqpoint{2.492402in}{1.027393in}}%
\pgfpathlineto{\pgfqpoint{2.494517in}{1.024308in}}%
\pgfpathlineto{\pgfqpoint{2.496631in}{1.023318in}}%
\pgfpathlineto{\pgfqpoint{2.500861in}{1.031499in}}%
\pgfpathlineto{\pgfqpoint{2.502975in}{1.030226in}}%
\pgfpathlineto{\pgfqpoint{2.507205in}{1.023342in}}%
\pgfpathlineto{\pgfqpoint{2.509319in}{1.027418in}}%
\pgfpathlineto{\pgfqpoint{2.513549in}{1.025599in}}%
\pgfpathlineto{\pgfqpoint{2.515664in}{1.030994in}}%
\pgfpathlineto{\pgfqpoint{2.517778in}{1.027403in}}%
\pgfpathlineto{\pgfqpoint{2.519893in}{1.031107in}}%
\pgfpathlineto{\pgfqpoint{2.522008in}{1.029655in}}%
\pgfpathlineto{\pgfqpoint{2.526237in}{1.032568in}}%
\pgfpathlineto{\pgfqpoint{2.528352in}{1.038312in}}%
\pgfpathlineto{\pgfqpoint{2.530466in}{1.037425in}}%
\pgfpathlineto{\pgfqpoint{2.532581in}{1.026675in}}%
\pgfpathlineto{\pgfqpoint{2.534696in}{1.025738in}}%
\pgfpathlineto{\pgfqpoint{2.538925in}{1.033181in}}%
\pgfpathlineto{\pgfqpoint{2.545269in}{1.036023in}}%
\pgfpathlineto{\pgfqpoint{2.547384in}{1.035038in}}%
\pgfpathlineto{\pgfqpoint{2.549499in}{1.037529in}}%
\pgfpathlineto{\pgfqpoint{2.551613in}{1.047183in}}%
\pgfpathlineto{\pgfqpoint{2.553728in}{1.041669in}}%
\pgfpathlineto{\pgfqpoint{2.555843in}{1.046561in}}%
\pgfpathlineto{\pgfqpoint{2.557957in}{1.045908in}}%
\pgfpathlineto{\pgfqpoint{2.564301in}{1.052110in}}%
\pgfpathlineto{\pgfqpoint{2.566416in}{1.051541in}}%
\pgfpathlineto{\pgfqpoint{2.568531in}{1.056835in}}%
\pgfpathlineto{\pgfqpoint{2.570646in}{1.057663in}}%
\pgfpathlineto{\pgfqpoint{2.574875in}{1.049326in}}%
\pgfpathlineto{\pgfqpoint{2.576990in}{1.049568in}}%
\pgfpathlineto{\pgfqpoint{2.585448in}{1.035221in}}%
\pgfpathlineto{\pgfqpoint{2.587563in}{1.038559in}}%
\pgfpathlineto{\pgfqpoint{2.589678in}{1.038548in}}%
\pgfpathlineto{\pgfqpoint{2.591792in}{1.035851in}}%
\pgfpathlineto{\pgfqpoint{2.593907in}{1.028339in}}%
\pgfpathlineto{\pgfqpoint{2.600251in}{1.042521in}}%
\pgfpathlineto{\pgfqpoint{2.604481in}{1.037003in}}%
\pgfpathlineto{\pgfqpoint{2.606595in}{1.037010in}}%
\pgfpathlineto{\pgfqpoint{2.608710in}{1.038400in}}%
\pgfpathlineto{\pgfqpoint{2.617169in}{1.037640in}}%
\pgfpathlineto{\pgfqpoint{2.619284in}{1.033943in}}%
\pgfpathlineto{\pgfqpoint{2.621398in}{1.034097in}}%
\pgfpathlineto{\pgfqpoint{2.623513in}{1.037243in}}%
\pgfpathlineto{\pgfqpoint{2.625628in}{1.029717in}}%
\pgfpathlineto{\pgfqpoint{2.627742in}{1.032046in}}%
\pgfpathlineto{\pgfqpoint{2.631972in}{1.042628in}}%
\pgfpathlineto{\pgfqpoint{2.634086in}{1.038620in}}%
\pgfpathlineto{\pgfqpoint{2.636201in}{1.042128in}}%
\pgfpathlineto{\pgfqpoint{2.640430in}{1.031846in}}%
\pgfpathlineto{\pgfqpoint{2.642545in}{1.034105in}}%
\pgfpathlineto{\pgfqpoint{2.644660in}{1.039491in}}%
\pgfpathlineto{\pgfqpoint{2.646775in}{1.036876in}}%
\pgfpathlineto{\pgfqpoint{2.648889in}{1.029612in}}%
\pgfpathlineto{\pgfqpoint{2.651004in}{1.032388in}}%
\pgfpathlineto{\pgfqpoint{2.653119in}{1.030562in}}%
\pgfpathlineto{\pgfqpoint{2.655233in}{1.034787in}}%
\pgfpathlineto{\pgfqpoint{2.657348in}{1.043578in}}%
\pgfpathlineto{\pgfqpoint{2.659463in}{1.033981in}}%
\pgfpathlineto{\pgfqpoint{2.661577in}{1.035304in}}%
\pgfpathlineto{\pgfqpoint{2.667921in}{1.049370in}}%
\pgfpathlineto{\pgfqpoint{2.670036in}{1.050026in}}%
\pgfpathlineto{\pgfqpoint{2.672151in}{1.045101in}}%
\pgfpathlineto{\pgfqpoint{2.674266in}{1.044486in}}%
\pgfpathlineto{\pgfqpoint{2.676380in}{1.050196in}}%
\pgfpathlineto{\pgfqpoint{2.678495in}{1.044005in}}%
\pgfpathlineto{\pgfqpoint{2.680610in}{1.043994in}}%
\pgfpathlineto{\pgfqpoint{2.682724in}{1.040374in}}%
\pgfpathlineto{\pgfqpoint{2.684839in}{1.045415in}}%
\pgfpathlineto{\pgfqpoint{2.686954in}{1.045121in}}%
\pgfpathlineto{\pgfqpoint{2.689068in}{1.048623in}}%
\pgfpathlineto{\pgfqpoint{2.691183in}{1.060498in}}%
\pgfpathlineto{\pgfqpoint{2.693298in}{1.059952in}}%
\pgfpathlineto{\pgfqpoint{2.695412in}{1.056281in}}%
\pgfpathlineto{\pgfqpoint{2.697527in}{1.059628in}}%
\pgfpathlineto{\pgfqpoint{2.699642in}{1.052235in}}%
\pgfpathlineto{\pgfqpoint{2.701757in}{1.050398in}}%
\pgfpathlineto{\pgfqpoint{2.703871in}{1.058738in}}%
\pgfpathlineto{\pgfqpoint{2.705986in}{1.056407in}}%
\pgfpathlineto{\pgfqpoint{2.710215in}{1.065965in}}%
\pgfpathlineto{\pgfqpoint{2.714445in}{1.063892in}}%
\pgfpathlineto{\pgfqpoint{2.716559in}{1.062073in}}%
\pgfpathlineto{\pgfqpoint{2.718674in}{1.056685in}}%
\pgfpathlineto{\pgfqpoint{2.720789in}{1.064360in}}%
\pgfpathlineto{\pgfqpoint{2.722903in}{1.060694in}}%
\pgfpathlineto{\pgfqpoint{2.727133in}{1.064170in}}%
\pgfpathlineto{\pgfqpoint{2.729248in}{1.068552in}}%
\pgfpathlineto{\pgfqpoint{2.731362in}{1.066406in}}%
\pgfpathlineto{\pgfqpoint{2.733477in}{1.058175in}}%
\pgfpathlineto{\pgfqpoint{2.735592in}{1.057002in}}%
\pgfpathlineto{\pgfqpoint{2.737706in}{1.048244in}}%
\pgfpathlineto{\pgfqpoint{2.739821in}{1.052207in}}%
\pgfpathlineto{\pgfqpoint{2.744050in}{1.051372in}}%
\pgfpathlineto{\pgfqpoint{2.748280in}{1.045029in}}%
\pgfpathlineto{\pgfqpoint{2.752509in}{1.047809in}}%
\pgfpathlineto{\pgfqpoint{2.756739in}{1.050744in}}%
\pgfpathlineto{\pgfqpoint{2.758853in}{1.050795in}}%
\pgfpathlineto{\pgfqpoint{2.769427in}{1.033432in}}%
\pgfpathlineto{\pgfqpoint{2.771541in}{1.023155in}}%
\pgfpathlineto{\pgfqpoint{2.773656in}{1.021525in}}%
\pgfpathlineto{\pgfqpoint{2.777886in}{1.020479in}}%
\pgfpathlineto{\pgfqpoint{2.780000in}{1.025064in}}%
\pgfpathlineto{\pgfqpoint{2.782115in}{1.025980in}}%
\pgfpathlineto{\pgfqpoint{2.784230in}{1.020923in}}%
\pgfpathlineto{\pgfqpoint{2.786344in}{1.019907in}}%
\pgfpathlineto{\pgfqpoint{2.788459in}{1.014451in}}%
\pgfpathlineto{\pgfqpoint{2.790574in}{1.014498in}}%
\pgfpathlineto{\pgfqpoint{2.792688in}{1.010639in}}%
\pgfpathlineto{\pgfqpoint{2.794803in}{1.020499in}}%
\pgfpathlineto{\pgfqpoint{2.796918in}{1.023329in}}%
\pgfpathlineto{\pgfqpoint{2.801147in}{1.017008in}}%
\pgfpathlineto{\pgfqpoint{2.805377in}{1.021882in}}%
\pgfpathlineto{\pgfqpoint{2.807491in}{1.021597in}}%
\pgfpathlineto{\pgfqpoint{2.811721in}{1.015420in}}%
\pgfpathlineto{\pgfqpoint{2.813835in}{1.013745in}}%
\pgfpathlineto{\pgfqpoint{2.815950in}{1.015898in}}%
\pgfpathlineto{\pgfqpoint{2.818065in}{1.019823in}}%
\pgfpathlineto{\pgfqpoint{2.820179in}{1.015535in}}%
\pgfpathlineto{\pgfqpoint{2.822294in}{1.015921in}}%
\pgfpathlineto{\pgfqpoint{2.824409in}{1.014660in}}%
\pgfpathlineto{\pgfqpoint{2.828638in}{1.007575in}}%
\pgfpathlineto{\pgfqpoint{2.830753in}{1.007901in}}%
\pgfpathlineto{\pgfqpoint{2.832868in}{1.002171in}}%
\pgfpathlineto{\pgfqpoint{2.834982in}{1.001389in}}%
\pgfpathlineto{\pgfqpoint{2.841326in}{1.014118in}}%
\pgfpathlineto{\pgfqpoint{2.843441in}{1.011815in}}%
\pgfpathlineto{\pgfqpoint{2.845556in}{1.016561in}}%
\pgfpathlineto{\pgfqpoint{2.847670in}{1.011483in}}%
\pgfpathlineto{\pgfqpoint{2.849785in}{1.013021in}}%
\pgfpathlineto{\pgfqpoint{2.851900in}{1.016494in}}%
\pgfpathlineto{\pgfqpoint{2.854015in}{1.017054in}}%
\pgfpathlineto{\pgfqpoint{2.856129in}{1.016278in}}%
\pgfpathlineto{\pgfqpoint{2.858244in}{1.018869in}}%
\pgfpathlineto{\pgfqpoint{2.862473in}{1.018483in}}%
\pgfpathlineto{\pgfqpoint{2.864588in}{1.016260in}}%
\pgfpathlineto{\pgfqpoint{2.866703in}{1.017813in}}%
\pgfpathlineto{\pgfqpoint{2.868817in}{1.022752in}}%
\pgfpathlineto{\pgfqpoint{2.870932in}{1.021226in}}%
\pgfpathlineto{\pgfqpoint{2.873047in}{1.021855in}}%
\pgfpathlineto{\pgfqpoint{2.875161in}{1.028657in}}%
\pgfpathlineto{\pgfqpoint{2.877276in}{1.024168in}}%
\pgfpathlineto{\pgfqpoint{2.879391in}{1.028711in}}%
\pgfpathlineto{\pgfqpoint{2.881506in}{1.028186in}}%
\pgfpathlineto{\pgfqpoint{2.883620in}{1.030353in}}%
\pgfpathlineto{\pgfqpoint{2.885735in}{1.029918in}}%
\pgfpathlineto{\pgfqpoint{2.887850in}{1.033728in}}%
\pgfpathlineto{\pgfqpoint{2.889964in}{1.025587in}}%
\pgfpathlineto{\pgfqpoint{2.894194in}{1.027836in}}%
\pgfpathlineto{\pgfqpoint{2.896308in}{1.027550in}}%
\pgfpathlineto{\pgfqpoint{2.898423in}{1.022459in}}%
\pgfpathlineto{\pgfqpoint{2.900538in}{1.023201in}}%
\pgfpathlineto{\pgfqpoint{2.908997in}{1.030513in}}%
\pgfpathlineto{\pgfqpoint{2.913226in}{1.025451in}}%
\pgfpathlineto{\pgfqpoint{2.917455in}{1.033317in}}%
\pgfpathlineto{\pgfqpoint{2.923799in}{1.036123in}}%
\pgfpathlineto{\pgfqpoint{2.925914in}{1.032137in}}%
\pgfpathlineto{\pgfqpoint{2.928029in}{1.020639in}}%
\pgfpathlineto{\pgfqpoint{2.934373in}{1.018771in}}%
\pgfpathlineto{\pgfqpoint{2.936488in}{1.026935in}}%
\pgfpathlineto{\pgfqpoint{2.938602in}{1.024084in}}%
\pgfpathlineto{\pgfqpoint{2.947061in}{1.036670in}}%
\pgfpathlineto{\pgfqpoint{2.949176in}{1.037782in}}%
\pgfpathlineto{\pgfqpoint{2.951290in}{1.036505in}}%
\pgfpathlineto{\pgfqpoint{2.953405in}{1.028361in}}%
\pgfpathlineto{\pgfqpoint{2.959749in}{1.020508in}}%
\pgfpathlineto{\pgfqpoint{2.961864in}{1.021265in}}%
\pgfpathlineto{\pgfqpoint{2.966093in}{1.016536in}}%
\pgfpathlineto{\pgfqpoint{2.968208in}{1.017240in}}%
\pgfpathlineto{\pgfqpoint{2.970323in}{1.016559in}}%
\pgfpathlineto{\pgfqpoint{2.972437in}{1.017619in}}%
\pgfpathlineto{\pgfqpoint{2.974552in}{1.023967in}}%
\pgfpathlineto{\pgfqpoint{2.976667in}{1.024384in}}%
\pgfpathlineto{\pgfqpoint{2.978781in}{1.021023in}}%
\pgfpathlineto{\pgfqpoint{2.980896in}{1.022926in}}%
\pgfpathlineto{\pgfqpoint{2.983011in}{1.022301in}}%
\pgfpathlineto{\pgfqpoint{2.985126in}{1.028715in}}%
\pgfpathlineto{\pgfqpoint{2.987240in}{1.022460in}}%
\pgfpathlineto{\pgfqpoint{2.989355in}{1.025120in}}%
\pgfpathlineto{\pgfqpoint{2.991470in}{1.032606in}}%
\pgfpathlineto{\pgfqpoint{2.993584in}{1.031280in}}%
\pgfpathlineto{\pgfqpoint{2.995699in}{1.033839in}}%
\pgfpathlineto{\pgfqpoint{2.997814in}{1.028291in}}%
\pgfpathlineto{\pgfqpoint{2.999928in}{1.027003in}}%
\pgfpathlineto{\pgfqpoint{3.002043in}{1.028879in}}%
\pgfpathlineto{\pgfqpoint{3.004158in}{1.022860in}}%
\pgfpathlineto{\pgfqpoint{3.006272in}{1.025964in}}%
\pgfpathlineto{\pgfqpoint{3.008387in}{1.024152in}}%
\pgfpathlineto{\pgfqpoint{3.012617in}{1.014385in}}%
\pgfpathlineto{\pgfqpoint{3.014731in}{1.010352in}}%
\pgfpathlineto{\pgfqpoint{3.016846in}{1.001318in}}%
\pgfpathlineto{\pgfqpoint{3.021075in}{1.001919in}}%
\pgfpathlineto{\pgfqpoint{3.025305in}{1.009978in}}%
\pgfpathlineto{\pgfqpoint{3.027419in}{1.004519in}}%
\pgfpathlineto{\pgfqpoint{3.029534in}{1.004129in}}%
\pgfpathlineto{\pgfqpoint{3.031649in}{0.998530in}}%
\pgfpathlineto{\pgfqpoint{3.033763in}{0.998265in}}%
\pgfpathlineto{\pgfqpoint{3.037993in}{0.987590in}}%
\pgfpathlineto{\pgfqpoint{3.040108in}{0.988202in}}%
\pgfpathlineto{\pgfqpoint{3.042222in}{0.987091in}}%
\pgfpathlineto{\pgfqpoint{3.044337in}{0.984211in}}%
\pgfpathlineto{\pgfqpoint{3.048566in}{0.989590in}}%
\pgfpathlineto{\pgfqpoint{3.052796in}{0.986256in}}%
\pgfpathlineto{\pgfqpoint{3.057025in}{0.986540in}}%
\pgfpathlineto{\pgfqpoint{3.059140in}{0.981838in}}%
\pgfpathlineto{\pgfqpoint{3.061254in}{0.981579in}}%
\pgfpathlineto{\pgfqpoint{3.063369in}{0.988750in}}%
\pgfpathlineto{\pgfqpoint{3.065484in}{0.990836in}}%
\pgfpathlineto{\pgfqpoint{3.067599in}{0.990056in}}%
\pgfpathlineto{\pgfqpoint{3.071828in}{0.999410in}}%
\pgfpathlineto{\pgfqpoint{3.076057in}{0.993642in}}%
\pgfpathlineto{\pgfqpoint{3.078172in}{0.996829in}}%
\pgfpathlineto{\pgfqpoint{3.080287in}{1.003756in}}%
\pgfpathlineto{\pgfqpoint{3.082401in}{0.991705in}}%
\pgfpathlineto{\pgfqpoint{3.084516in}{0.991292in}}%
\pgfpathlineto{\pgfqpoint{3.086631in}{0.993699in}}%
\pgfpathlineto{\pgfqpoint{3.088746in}{1.002531in}}%
\pgfpathlineto{\pgfqpoint{3.092975in}{0.996316in}}%
\pgfpathlineto{\pgfqpoint{3.095090in}{1.000186in}}%
\pgfpathlineto{\pgfqpoint{3.097204in}{0.995336in}}%
\pgfpathlineto{\pgfqpoint{3.099319in}{0.984100in}}%
\pgfpathlineto{\pgfqpoint{3.103548in}{0.987196in}}%
\pgfpathlineto{\pgfqpoint{3.107778in}{0.981063in}}%
\pgfpathlineto{\pgfqpoint{3.114122in}{0.979270in}}%
\pgfpathlineto{\pgfqpoint{3.116237in}{0.982585in}}%
\pgfpathlineto{\pgfqpoint{3.118351in}{0.979713in}}%
\pgfpathlineto{\pgfqpoint{3.120466in}{0.981330in}}%
\pgfpathlineto{\pgfqpoint{3.122581in}{0.976904in}}%
\pgfpathlineto{\pgfqpoint{3.124695in}{0.976763in}}%
\pgfpathlineto{\pgfqpoint{3.126810in}{0.974344in}}%
\pgfpathlineto{\pgfqpoint{3.128925in}{0.980329in}}%
\pgfpathlineto{\pgfqpoint{3.131039in}{0.978476in}}%
\pgfpathlineto{\pgfqpoint{3.135269in}{0.985013in}}%
\pgfpathlineto{\pgfqpoint{3.139498in}{0.993866in}}%
\pgfpathlineto{\pgfqpoint{3.145842in}{0.977065in}}%
\pgfpathlineto{\pgfqpoint{3.147957in}{0.977376in}}%
\pgfpathlineto{\pgfqpoint{3.154301in}{0.968447in}}%
\pgfpathlineto{\pgfqpoint{3.156416in}{0.970490in}}%
\pgfpathlineto{\pgfqpoint{3.160645in}{0.967855in}}%
\pgfpathlineto{\pgfqpoint{3.162760in}{0.962324in}}%
\pgfpathlineto{\pgfqpoint{3.164874in}{0.962008in}}%
\pgfpathlineto{\pgfqpoint{3.166989in}{0.963461in}}%
\pgfpathlineto{\pgfqpoint{3.169104in}{0.968671in}}%
\pgfpathlineto{\pgfqpoint{3.171219in}{0.967628in}}%
\pgfpathlineto{\pgfqpoint{3.175448in}{0.974030in}}%
\pgfpathlineto{\pgfqpoint{3.179677in}{0.963113in}}%
\pgfpathlineto{\pgfqpoint{3.181792in}{0.966290in}}%
\pgfpathlineto{\pgfqpoint{3.183907in}{0.967084in}}%
\pgfpathlineto{\pgfqpoint{3.186021in}{0.964487in}}%
\pgfpathlineto{\pgfqpoint{3.188136in}{0.959822in}}%
\pgfpathlineto{\pgfqpoint{3.190251in}{0.968295in}}%
\pgfpathlineto{\pgfqpoint{3.192366in}{0.965808in}}%
\pgfpathlineto{\pgfqpoint{3.194480in}{0.970735in}}%
\pgfpathlineto{\pgfqpoint{3.196595in}{0.972191in}}%
\pgfpathlineto{\pgfqpoint{3.198710in}{0.968928in}}%
\pgfpathlineto{\pgfqpoint{3.207168in}{0.990895in}}%
\pgfpathlineto{\pgfqpoint{3.209283in}{0.983793in}}%
\pgfpathlineto{\pgfqpoint{3.211398in}{0.980768in}}%
\pgfpathlineto{\pgfqpoint{3.213512in}{0.973380in}}%
\pgfpathlineto{\pgfqpoint{3.215627in}{0.979966in}}%
\pgfpathlineto{\pgfqpoint{3.217742in}{0.976258in}}%
\pgfpathlineto{\pgfqpoint{3.219857in}{0.976253in}}%
\pgfpathlineto{\pgfqpoint{3.221971in}{0.980132in}}%
\pgfpathlineto{\pgfqpoint{3.224086in}{0.989081in}}%
\pgfpathlineto{\pgfqpoint{3.226201in}{0.985565in}}%
\pgfpathlineto{\pgfqpoint{3.228315in}{0.993793in}}%
\pgfpathlineto{\pgfqpoint{3.230430in}{0.995327in}}%
\pgfpathlineto{\pgfqpoint{3.232545in}{0.993396in}}%
\pgfpathlineto{\pgfqpoint{3.234659in}{1.004063in}}%
\pgfpathlineto{\pgfqpoint{3.236774in}{1.005988in}}%
\pgfpathlineto{\pgfqpoint{3.238889in}{1.011811in}}%
\pgfpathlineto{\pgfqpoint{3.241003in}{1.003294in}}%
\pgfpathlineto{\pgfqpoint{3.243118in}{1.005478in}}%
\pgfpathlineto{\pgfqpoint{3.245233in}{1.002929in}}%
\pgfpathlineto{\pgfqpoint{3.247348in}{0.995703in}}%
\pgfpathlineto{\pgfqpoint{3.251577in}{1.003131in}}%
\pgfpathlineto{\pgfqpoint{3.253692in}{1.000271in}}%
\pgfpathlineto{\pgfqpoint{3.255806in}{1.000712in}}%
\pgfpathlineto{\pgfqpoint{3.257921in}{0.998657in}}%
\pgfpathlineto{\pgfqpoint{3.260036in}{1.003023in}}%
\pgfpathlineto{\pgfqpoint{3.262150in}{0.996767in}}%
\pgfpathlineto{\pgfqpoint{3.264265in}{0.998831in}}%
\pgfpathlineto{\pgfqpoint{3.266380in}{1.010678in}}%
\pgfpathlineto{\pgfqpoint{3.270609in}{1.010341in}}%
\pgfpathlineto{\pgfqpoint{3.272724in}{1.003207in}}%
\pgfpathlineto{\pgfqpoint{3.276953in}{1.005551in}}%
\pgfpathlineto{\pgfqpoint{3.279068in}{1.012601in}}%
\pgfpathlineto{\pgfqpoint{3.281183in}{1.011682in}}%
\pgfpathlineto{\pgfqpoint{3.283297in}{1.014041in}}%
\pgfpathlineto{\pgfqpoint{3.285412in}{1.018878in}}%
\pgfpathlineto{\pgfqpoint{3.287527in}{1.015731in}}%
\pgfpathlineto{\pgfqpoint{3.289641in}{1.015400in}}%
\pgfpathlineto{\pgfqpoint{3.291756in}{1.013272in}}%
\pgfpathlineto{\pgfqpoint{3.293871in}{1.013726in}}%
\pgfpathlineto{\pgfqpoint{3.295985in}{1.022757in}}%
\pgfpathlineto{\pgfqpoint{3.298100in}{1.018361in}}%
\pgfpathlineto{\pgfqpoint{3.302330in}{1.021169in}}%
\pgfpathlineto{\pgfqpoint{3.304444in}{1.019036in}}%
\pgfpathlineto{\pgfqpoint{3.306559in}{1.014250in}}%
\pgfpathlineto{\pgfqpoint{3.308674in}{1.015716in}}%
\pgfpathlineto{\pgfqpoint{3.312903in}{1.015664in}}%
\pgfpathlineto{\pgfqpoint{3.315018in}{1.019179in}}%
\pgfpathlineto{\pgfqpoint{3.319247in}{1.018523in}}%
\pgfpathlineto{\pgfqpoint{3.321362in}{1.027306in}}%
\pgfpathlineto{\pgfqpoint{3.325591in}{1.016734in}}%
\pgfpathlineto{\pgfqpoint{3.327706in}{1.020281in}}%
\pgfpathlineto{\pgfqpoint{3.329821in}{1.017306in}}%
\pgfpathlineto{\pgfqpoint{3.331935in}{1.011701in}}%
\pgfpathlineto{\pgfqpoint{3.334050in}{1.013372in}}%
\pgfpathlineto{\pgfqpoint{3.342509in}{1.027007in}}%
\pgfpathlineto{\pgfqpoint{3.344623in}{1.018015in}}%
\pgfpathlineto{\pgfqpoint{3.346738in}{1.014527in}}%
\pgfpathlineto{\pgfqpoint{3.348853in}{1.014783in}}%
\pgfpathlineto{\pgfqpoint{3.350968in}{1.023833in}}%
\pgfpathlineto{\pgfqpoint{3.353082in}{1.022935in}}%
\pgfpathlineto{\pgfqpoint{3.355197in}{1.025938in}}%
\pgfpathlineto{\pgfqpoint{3.357312in}{1.026610in}}%
\pgfpathlineto{\pgfqpoint{3.361541in}{1.018804in}}%
\pgfpathlineto{\pgfqpoint{3.365770in}{1.021207in}}%
\pgfpathlineto{\pgfqpoint{3.367885in}{1.022531in}}%
\pgfpathlineto{\pgfqpoint{3.370000in}{1.017283in}}%
\pgfpathlineto{\pgfqpoint{3.374229in}{1.021312in}}%
\pgfpathlineto{\pgfqpoint{3.378459in}{1.015824in}}%
\pgfpathlineto{\pgfqpoint{3.380573in}{1.015269in}}%
\pgfpathlineto{\pgfqpoint{3.384803in}{1.024420in}}%
\pgfpathlineto{\pgfqpoint{3.386917in}{1.021393in}}%
\pgfpathlineto{\pgfqpoint{3.389032in}{1.022743in}}%
\pgfpathlineto{\pgfqpoint{3.391147in}{1.019872in}}%
\pgfpathlineto{\pgfqpoint{3.393261in}{1.023046in}}%
\pgfpathlineto{\pgfqpoint{3.395376in}{1.021155in}}%
\pgfpathlineto{\pgfqpoint{3.397491in}{1.020891in}}%
\pgfpathlineto{\pgfqpoint{3.399605in}{1.028691in}}%
\pgfpathlineto{\pgfqpoint{3.401720in}{1.020590in}}%
\pgfpathlineto{\pgfqpoint{3.403835in}{1.017459in}}%
\pgfpathlineto{\pgfqpoint{3.405950in}{1.017490in}}%
\pgfpathlineto{\pgfqpoint{3.412294in}{1.010535in}}%
\pgfpathlineto{\pgfqpoint{3.414408in}{1.010158in}}%
\pgfpathlineto{\pgfqpoint{3.418638in}{1.012416in}}%
\pgfpathlineto{\pgfqpoint{3.420752in}{1.010240in}}%
\pgfpathlineto{\pgfqpoint{3.422867in}{1.003120in}}%
\pgfpathlineto{\pgfqpoint{3.424982in}{1.009038in}}%
\pgfpathlineto{\pgfqpoint{3.427097in}{1.004580in}}%
\pgfpathlineto{\pgfqpoint{3.429211in}{1.012184in}}%
\pgfpathlineto{\pgfqpoint{3.431326in}{1.013984in}}%
\pgfpathlineto{\pgfqpoint{3.433441in}{1.013210in}}%
\pgfpathlineto{\pgfqpoint{3.439785in}{1.024592in}}%
\pgfpathlineto{\pgfqpoint{3.441899in}{1.019621in}}%
\pgfpathlineto{\pgfqpoint{3.444014in}{1.018164in}}%
\pgfpathlineto{\pgfqpoint{3.448243in}{1.031687in}}%
\pgfpathlineto{\pgfqpoint{3.450358in}{1.031475in}}%
\pgfpathlineto{\pgfqpoint{3.454588in}{1.025674in}}%
\pgfpathlineto{\pgfqpoint{3.456702in}{1.022439in}}%
\pgfpathlineto{\pgfqpoint{3.458817in}{1.009330in}}%
\pgfpathlineto{\pgfqpoint{3.460932in}{1.012422in}}%
\pgfpathlineto{\pgfqpoint{3.465161in}{1.011498in}}%
\pgfpathlineto{\pgfqpoint{3.467276in}{1.005376in}}%
\pgfpathlineto{\pgfqpoint{3.471505in}{1.003646in}}%
\pgfpathlineto{\pgfqpoint{3.473620in}{1.007895in}}%
\pgfpathlineto{\pgfqpoint{3.477849in}{1.011921in}}%
\pgfpathlineto{\pgfqpoint{3.479964in}{1.016533in}}%
\pgfpathlineto{\pgfqpoint{3.482079in}{1.017645in}}%
\pgfpathlineto{\pgfqpoint{3.484193in}{1.013768in}}%
\pgfpathlineto{\pgfqpoint{3.486308in}{1.017873in}}%
\pgfpathlineto{\pgfqpoint{3.488423in}{1.016791in}}%
\pgfpathlineto{\pgfqpoint{3.492652in}{1.005617in}}%
\pgfpathlineto{\pgfqpoint{3.494767in}{1.006247in}}%
\pgfpathlineto{\pgfqpoint{3.498996in}{1.019858in}}%
\pgfpathlineto{\pgfqpoint{3.501111in}{1.020682in}}%
\pgfpathlineto{\pgfqpoint{3.503225in}{1.019514in}}%
\pgfpathlineto{\pgfqpoint{3.509570in}{1.003351in}}%
\pgfpathlineto{\pgfqpoint{3.511684in}{1.008971in}}%
\pgfpathlineto{\pgfqpoint{3.513799in}{1.010223in}}%
\pgfpathlineto{\pgfqpoint{3.515914in}{1.001492in}}%
\pgfpathlineto{\pgfqpoint{3.518028in}{1.003429in}}%
\pgfpathlineto{\pgfqpoint{3.520143in}{1.000804in}}%
\pgfpathlineto{\pgfqpoint{3.522258in}{1.005878in}}%
\pgfpathlineto{\pgfqpoint{3.524372in}{1.004721in}}%
\pgfpathlineto{\pgfqpoint{3.528602in}{0.994313in}}%
\pgfpathlineto{\pgfqpoint{3.534946in}{0.988341in}}%
\pgfpathlineto{\pgfqpoint{3.537061in}{0.994106in}}%
\pgfpathlineto{\pgfqpoint{3.539175in}{0.993449in}}%
\pgfpathlineto{\pgfqpoint{3.541290in}{0.987920in}}%
\pgfpathlineto{\pgfqpoint{3.543405in}{0.985609in}}%
\pgfpathlineto{\pgfqpoint{3.545519in}{0.977710in}}%
\pgfpathlineto{\pgfqpoint{3.547634in}{0.977670in}}%
\pgfpathlineto{\pgfqpoint{3.549749in}{0.974229in}}%
\pgfpathlineto{\pgfqpoint{3.551863in}{0.975371in}}%
\pgfpathlineto{\pgfqpoint{3.553978in}{0.970769in}}%
\pgfpathlineto{\pgfqpoint{3.556093in}{0.976385in}}%
\pgfpathlineto{\pgfqpoint{3.558208in}{0.976745in}}%
\pgfpathlineto{\pgfqpoint{3.560322in}{0.979590in}}%
\pgfpathlineto{\pgfqpoint{3.562437in}{0.972797in}}%
\pgfpathlineto{\pgfqpoint{3.564552in}{0.969884in}}%
\pgfpathlineto{\pgfqpoint{3.566666in}{0.970447in}}%
\pgfpathlineto{\pgfqpoint{3.570896in}{0.979781in}}%
\pgfpathlineto{\pgfqpoint{3.573010in}{0.979399in}}%
\pgfpathlineto{\pgfqpoint{3.575125in}{0.972095in}}%
\pgfpathlineto{\pgfqpoint{3.577240in}{0.973059in}}%
\pgfpathlineto{\pgfqpoint{3.579354in}{0.968782in}}%
\pgfpathlineto{\pgfqpoint{3.581469in}{0.968835in}}%
\pgfpathlineto{\pgfqpoint{3.585699in}{0.971686in}}%
\pgfpathlineto{\pgfqpoint{3.587813in}{0.973416in}}%
\pgfpathlineto{\pgfqpoint{3.589928in}{0.968685in}}%
\pgfpathlineto{\pgfqpoint{3.594157in}{0.976868in}}%
\pgfpathlineto{\pgfqpoint{3.596272in}{0.974449in}}%
\pgfpathlineto{\pgfqpoint{3.600501in}{0.978052in}}%
\pgfpathlineto{\pgfqpoint{3.602616in}{0.984122in}}%
\pgfpathlineto{\pgfqpoint{3.604731in}{0.983803in}}%
\pgfpathlineto{\pgfqpoint{3.606845in}{0.985303in}}%
\pgfpathlineto{\pgfqpoint{3.608960in}{0.991843in}}%
\pgfpathlineto{\pgfqpoint{3.611075in}{0.987682in}}%
\pgfpathlineto{\pgfqpoint{3.613190in}{0.987069in}}%
\pgfpathlineto{\pgfqpoint{3.615304in}{0.990472in}}%
\pgfpathlineto{\pgfqpoint{3.617419in}{0.990968in}}%
\pgfpathlineto{\pgfqpoint{3.619534in}{0.989971in}}%
\pgfpathlineto{\pgfqpoint{3.621648in}{0.990572in}}%
\pgfpathlineto{\pgfqpoint{3.623763in}{0.998157in}}%
\pgfpathlineto{\pgfqpoint{3.625878in}{0.995703in}}%
\pgfpathlineto{\pgfqpoint{3.627992in}{1.002349in}}%
\pgfpathlineto{\pgfqpoint{3.630107in}{1.002299in}}%
\pgfpathlineto{\pgfqpoint{3.634336in}{1.004397in}}%
\pgfpathlineto{\pgfqpoint{3.636451in}{1.001414in}}%
\pgfpathlineto{\pgfqpoint{3.640681in}{0.987943in}}%
\pgfpathlineto{\pgfqpoint{3.649139in}{0.981051in}}%
\pgfpathlineto{\pgfqpoint{3.651254in}{0.982730in}}%
\pgfpathlineto{\pgfqpoint{3.653369in}{0.981861in}}%
\pgfpathlineto{\pgfqpoint{3.657598in}{0.991755in}}%
\pgfpathlineto{\pgfqpoint{3.659713in}{0.990529in}}%
\pgfpathlineto{\pgfqpoint{3.661828in}{0.996449in}}%
\pgfpathlineto{\pgfqpoint{3.663942in}{0.997306in}}%
\pgfpathlineto{\pgfqpoint{3.668172in}{0.993014in}}%
\pgfpathlineto{\pgfqpoint{3.670286in}{0.994414in}}%
\pgfpathlineto{\pgfqpoint{3.672401in}{0.999187in}}%
\pgfpathlineto{\pgfqpoint{3.674516in}{1.001110in}}%
\pgfpathlineto{\pgfqpoint{3.678745in}{1.010551in}}%
\pgfpathlineto{\pgfqpoint{3.680860in}{1.006888in}}%
\pgfpathlineto{\pgfqpoint{3.685089in}{1.004723in}}%
\pgfpathlineto{\pgfqpoint{3.689319in}{1.000095in}}%
\pgfpathlineto{\pgfqpoint{3.691433in}{0.990795in}}%
\pgfpathlineto{\pgfqpoint{3.693548in}{0.988433in}}%
\pgfpathlineto{\pgfqpoint{3.695663in}{0.988316in}}%
\pgfpathlineto{\pgfqpoint{3.697777in}{0.986831in}}%
\pgfpathlineto{\pgfqpoint{3.699892in}{0.987792in}}%
\pgfpathlineto{\pgfqpoint{3.702007in}{0.987184in}}%
\pgfpathlineto{\pgfqpoint{3.706236in}{0.993616in}}%
\pgfpathlineto{\pgfqpoint{3.710465in}{0.993594in}}%
\pgfpathlineto{\pgfqpoint{3.712580in}{0.992443in}}%
\pgfpathlineto{\pgfqpoint{3.716810in}{0.995699in}}%
\pgfpathlineto{\pgfqpoint{3.718924in}{0.989656in}}%
\pgfpathlineto{\pgfqpoint{3.721039in}{0.995482in}}%
\pgfpathlineto{\pgfqpoint{3.723154in}{0.995877in}}%
\pgfpathlineto{\pgfqpoint{3.725268in}{0.993946in}}%
\pgfpathlineto{\pgfqpoint{3.727383in}{0.983036in}}%
\pgfpathlineto{\pgfqpoint{3.733727in}{0.997775in}}%
\pgfpathlineto{\pgfqpoint{3.735842in}{1.000015in}}%
\pgfpathlineto{\pgfqpoint{3.737956in}{1.011217in}}%
\pgfpathlineto{\pgfqpoint{3.740071in}{1.004169in}}%
\pgfpathlineto{\pgfqpoint{3.744301in}{1.004788in}}%
\pgfpathlineto{\pgfqpoint{3.746415in}{1.009954in}}%
\pgfpathlineto{\pgfqpoint{3.748530in}{1.002404in}}%
\pgfpathlineto{\pgfqpoint{3.750645in}{1.003849in}}%
\pgfpathlineto{\pgfqpoint{3.752759in}{1.002240in}}%
\pgfpathlineto{\pgfqpoint{3.754874in}{1.005072in}}%
\pgfpathlineto{\pgfqpoint{3.756989in}{1.005186in}}%
\pgfpathlineto{\pgfqpoint{3.759103in}{1.003741in}}%
\pgfpathlineto{\pgfqpoint{3.761218in}{1.006891in}}%
\pgfpathlineto{\pgfqpoint{3.763333in}{1.013064in}}%
\pgfpathlineto{\pgfqpoint{3.765447in}{1.023631in}}%
\pgfpathlineto{\pgfqpoint{3.767562in}{1.025921in}}%
\pgfpathlineto{\pgfqpoint{3.769677in}{1.020235in}}%
\pgfpathlineto{\pgfqpoint{3.771792in}{1.022830in}}%
\pgfpathlineto{\pgfqpoint{3.776021in}{1.033653in}}%
\pgfpathlineto{\pgfqpoint{3.778136in}{1.035798in}}%
\pgfpathlineto{\pgfqpoint{3.780250in}{1.030464in}}%
\pgfpathlineto{\pgfqpoint{3.782365in}{1.032501in}}%
\pgfpathlineto{\pgfqpoint{3.784480in}{1.037561in}}%
\pgfpathlineto{\pgfqpoint{3.786594in}{1.038231in}}%
\pgfpathlineto{\pgfqpoint{3.790824in}{1.044716in}}%
\pgfpathlineto{\pgfqpoint{3.795053in}{1.040541in}}%
\pgfpathlineto{\pgfqpoint{3.797168in}{1.038975in}}%
\pgfpathlineto{\pgfqpoint{3.799283in}{1.038957in}}%
\pgfpathlineto{\pgfqpoint{3.801397in}{1.035366in}}%
\pgfpathlineto{\pgfqpoint{3.803512in}{1.042985in}}%
\pgfpathlineto{\pgfqpoint{3.805627in}{1.042250in}}%
\pgfpathlineto{\pgfqpoint{3.807741in}{1.038935in}}%
\pgfpathlineto{\pgfqpoint{3.809856in}{1.039672in}}%
\pgfpathlineto{\pgfqpoint{3.811971in}{1.043274in}}%
\pgfpathlineto{\pgfqpoint{3.814085in}{1.042757in}}%
\pgfpathlineto{\pgfqpoint{3.818315in}{1.036672in}}%
\pgfpathlineto{\pgfqpoint{3.820430in}{1.040222in}}%
\pgfpathlineto{\pgfqpoint{3.822544in}{1.039237in}}%
\pgfpathlineto{\pgfqpoint{3.826774in}{1.043109in}}%
\pgfpathlineto{\pgfqpoint{3.831003in}{1.043142in}}%
\pgfpathlineto{\pgfqpoint{3.833118in}{1.039896in}}%
\pgfpathlineto{\pgfqpoint{3.835232in}{1.039250in}}%
\pgfpathlineto{\pgfqpoint{3.837347in}{1.041807in}}%
\pgfpathlineto{\pgfqpoint{3.839462in}{1.034903in}}%
\pgfpathlineto{\pgfqpoint{3.843691in}{1.047071in}}%
\pgfpathlineto{\pgfqpoint{3.845806in}{1.051222in}}%
\pgfpathlineto{\pgfqpoint{3.847921in}{1.050206in}}%
\pgfpathlineto{\pgfqpoint{3.852150in}{1.044733in}}%
\pgfpathlineto{\pgfqpoint{3.854265in}{1.048551in}}%
\pgfpathlineto{\pgfqpoint{3.856379in}{1.042775in}}%
\pgfpathlineto{\pgfqpoint{3.858494in}{1.045421in}}%
\pgfpathlineto{\pgfqpoint{3.860609in}{1.045784in}}%
\pgfpathlineto{\pgfqpoint{3.862723in}{1.051578in}}%
\pgfpathlineto{\pgfqpoint{3.864838in}{1.050334in}}%
\pgfpathlineto{\pgfqpoint{3.866953in}{1.054531in}}%
\pgfpathlineto{\pgfqpoint{3.869067in}{1.054110in}}%
\pgfpathlineto{\pgfqpoint{3.873297in}{1.057696in}}%
\pgfpathlineto{\pgfqpoint{3.875412in}{1.067143in}}%
\pgfpathlineto{\pgfqpoint{3.877526in}{1.062434in}}%
\pgfpathlineto{\pgfqpoint{3.879641in}{1.062058in}}%
\pgfpathlineto{\pgfqpoint{3.881756in}{1.057030in}}%
\pgfpathlineto{\pgfqpoint{3.883870in}{1.063133in}}%
\pgfpathlineto{\pgfqpoint{3.885985in}{1.061649in}}%
\pgfpathlineto{\pgfqpoint{3.888100in}{1.058000in}}%
\pgfpathlineto{\pgfqpoint{3.890214in}{1.057484in}}%
\pgfpathlineto{\pgfqpoint{3.894444in}{1.053587in}}%
\pgfpathlineto{\pgfqpoint{3.896559in}{1.054574in}}%
\pgfpathlineto{\pgfqpoint{3.898673in}{1.050065in}}%
\pgfpathlineto{\pgfqpoint{3.900788in}{1.048205in}}%
\pgfpathlineto{\pgfqpoint{3.905017in}{1.042624in}}%
\pgfpathlineto{\pgfqpoint{3.907132in}{1.041199in}}%
\pgfpathlineto{\pgfqpoint{3.915591in}{1.042563in}}%
\pgfpathlineto{\pgfqpoint{3.919820in}{1.056262in}}%
\pgfpathlineto{\pgfqpoint{3.924050in}{1.054346in}}%
\pgfpathlineto{\pgfqpoint{3.930394in}{1.046544in}}%
\pgfpathlineto{\pgfqpoint{3.934623in}{1.050065in}}%
\pgfpathlineto{\pgfqpoint{3.940967in}{1.035841in}}%
\pgfpathlineto{\pgfqpoint{3.943082in}{1.036596in}}%
\pgfpathlineto{\pgfqpoint{3.945196in}{1.041382in}}%
\pgfpathlineto{\pgfqpoint{3.947311in}{1.034214in}}%
\pgfpathlineto{\pgfqpoint{3.949426in}{1.038915in}}%
\pgfpathlineto{\pgfqpoint{3.951541in}{1.035901in}}%
\pgfpathlineto{\pgfqpoint{3.955770in}{1.035997in}}%
\pgfpathlineto{\pgfqpoint{3.959999in}{1.040647in}}%
\pgfpathlineto{\pgfqpoint{3.962114in}{1.048966in}}%
\pgfpathlineto{\pgfqpoint{3.966343in}{1.050229in}}%
\pgfpathlineto{\pgfqpoint{3.968458in}{1.053904in}}%
\pgfpathlineto{\pgfqpoint{3.970573in}{1.053866in}}%
\pgfpathlineto{\pgfqpoint{3.972687in}{1.051480in}}%
\pgfpathlineto{\pgfqpoint{3.974802in}{1.054411in}}%
\pgfpathlineto{\pgfqpoint{3.976917in}{1.053722in}}%
\pgfpathlineto{\pgfqpoint{3.983261in}{1.067410in}}%
\pgfpathlineto{\pgfqpoint{3.985376in}{1.064292in}}%
\pgfpathlineto{\pgfqpoint{3.987490in}{1.067065in}}%
\pgfpathlineto{\pgfqpoint{3.989605in}{1.065366in}}%
\pgfpathlineto{\pgfqpoint{3.991720in}{1.062024in}}%
\pgfpathlineto{\pgfqpoint{3.993834in}{1.064802in}}%
\pgfpathlineto{\pgfqpoint{3.995949in}{1.063986in}}%
\pgfpathlineto{\pgfqpoint{4.000178in}{1.055023in}}%
\pgfpathlineto{\pgfqpoint{4.002293in}{1.056727in}}%
\pgfpathlineto{\pgfqpoint{4.004408in}{1.050230in}}%
\pgfpathlineto{\pgfqpoint{4.006523in}{1.054867in}}%
\pgfpathlineto{\pgfqpoint{4.008637in}{1.049098in}}%
\pgfpathlineto{\pgfqpoint{4.010752in}{1.052890in}}%
\pgfpathlineto{\pgfqpoint{4.014981in}{1.050528in}}%
\pgfpathlineto{\pgfqpoint{4.017096in}{1.051835in}}%
\pgfpathlineto{\pgfqpoint{4.019211in}{1.045759in}}%
\pgfpathlineto{\pgfqpoint{4.021325in}{1.049507in}}%
\pgfpathlineto{\pgfqpoint{4.023440in}{1.049399in}}%
\pgfpathlineto{\pgfqpoint{4.025555in}{1.047534in}}%
\pgfpathlineto{\pgfqpoint{4.027670in}{1.048198in}}%
\pgfpathlineto{\pgfqpoint{4.031899in}{1.038798in}}%
\pgfpathlineto{\pgfqpoint{4.034014in}{1.042471in}}%
\pgfpathlineto{\pgfqpoint{4.040358in}{1.039024in}}%
\pgfpathlineto{\pgfqpoint{4.042472in}{1.041545in}}%
\pgfpathlineto{\pgfqpoint{4.044587in}{1.048747in}}%
\pgfpathlineto{\pgfqpoint{4.046702in}{1.036688in}}%
\pgfpathlineto{\pgfqpoint{4.048816in}{1.037384in}}%
\pgfpathlineto{\pgfqpoint{4.053046in}{1.028959in}}%
\pgfpathlineto{\pgfqpoint{4.057275in}{1.022445in}}%
\pgfpathlineto{\pgfqpoint{4.061505in}{1.024037in}}%
\pgfpathlineto{\pgfqpoint{4.063619in}{1.020740in}}%
\pgfpathlineto{\pgfqpoint{4.065734in}{1.020997in}}%
\pgfpathlineto{\pgfqpoint{4.067849in}{1.028467in}}%
\pgfpathlineto{\pgfqpoint{4.069963in}{1.026972in}}%
\pgfpathlineto{\pgfqpoint{4.072078in}{1.029130in}}%
\pgfpathlineto{\pgfqpoint{4.076307in}{1.016116in}}%
\pgfpathlineto{\pgfqpoint{4.078422in}{1.017135in}}%
\pgfpathlineto{\pgfqpoint{4.080537in}{1.016523in}}%
\pgfpathlineto{\pgfqpoint{4.082652in}{1.014634in}}%
\pgfpathlineto{\pgfqpoint{4.084766in}{1.017895in}}%
\pgfpathlineto{\pgfqpoint{4.091110in}{1.005471in}}%
\pgfpathlineto{\pgfqpoint{4.093225in}{1.009840in}}%
\pgfpathlineto{\pgfqpoint{4.095340in}{1.000035in}}%
\pgfpathlineto{\pgfqpoint{4.097454in}{1.003851in}}%
\pgfpathlineto{\pgfqpoint{4.101684in}{1.002917in}}%
\pgfpathlineto{\pgfqpoint{4.103798in}{1.010676in}}%
\pgfpathlineto{\pgfqpoint{4.105913in}{1.007495in}}%
\pgfpathlineto{\pgfqpoint{4.108028in}{0.999181in}}%
\pgfpathlineto{\pgfqpoint{4.110143in}{1.001425in}}%
\pgfpathlineto{\pgfqpoint{4.112257in}{0.998053in}}%
\pgfpathlineto{\pgfqpoint{4.114372in}{0.997995in}}%
\pgfpathlineto{\pgfqpoint{4.118601in}{1.002063in}}%
\pgfpathlineto{\pgfqpoint{4.120716in}{0.998328in}}%
\pgfpathlineto{\pgfqpoint{4.124945in}{1.008577in}}%
\pgfpathlineto{\pgfqpoint{4.127060in}{1.016113in}}%
\pgfpathlineto{\pgfqpoint{4.129175in}{1.018198in}}%
\pgfpathlineto{\pgfqpoint{4.131290in}{1.018092in}}%
\pgfpathlineto{\pgfqpoint{4.135519in}{1.015238in}}%
\pgfpathlineto{\pgfqpoint{4.139748in}{1.014572in}}%
\pgfpathlineto{\pgfqpoint{4.141863in}{1.028079in}}%
\pgfpathlineto{\pgfqpoint{4.146092in}{1.024774in}}%
\pgfpathlineto{\pgfqpoint{4.148207in}{1.028261in}}%
\pgfpathlineto{\pgfqpoint{4.150322in}{1.028945in}}%
\pgfpathlineto{\pgfqpoint{4.152436in}{1.027520in}}%
\pgfpathlineto{\pgfqpoint{4.154551in}{1.022580in}}%
\pgfpathlineto{\pgfqpoint{4.158781in}{1.021135in}}%
\pgfpathlineto{\pgfqpoint{4.160895in}{1.028094in}}%
\pgfpathlineto{\pgfqpoint{4.163010in}{1.026912in}}%
\pgfpathlineto{\pgfqpoint{4.165125in}{1.027551in}}%
\pgfpathlineto{\pgfqpoint{4.167239in}{1.019537in}}%
\pgfpathlineto{\pgfqpoint{4.171469in}{1.020459in}}%
\pgfpathlineto{\pgfqpoint{4.173583in}{1.019772in}}%
\pgfpathlineto{\pgfqpoint{4.179927in}{1.008293in}}%
\pgfpathlineto{\pgfqpoint{4.182042in}{1.011601in}}%
\pgfpathlineto{\pgfqpoint{4.184157in}{1.010122in}}%
\pgfpathlineto{\pgfqpoint{4.186272in}{1.007172in}}%
\pgfpathlineto{\pgfqpoint{4.188386in}{1.007762in}}%
\pgfpathlineto{\pgfqpoint{4.190501in}{1.006985in}}%
\pgfpathlineto{\pgfqpoint{4.192616in}{1.009409in}}%
\pgfpathlineto{\pgfqpoint{4.194730in}{1.006522in}}%
\pgfpathlineto{\pgfqpoint{4.196845in}{1.001576in}}%
\pgfpathlineto{\pgfqpoint{4.201074in}{1.003092in}}%
\pgfpathlineto{\pgfqpoint{4.205304in}{0.993347in}}%
\pgfpathlineto{\pgfqpoint{4.207418in}{0.991929in}}%
\pgfpathlineto{\pgfqpoint{4.209533in}{0.992704in}}%
\pgfpathlineto{\pgfqpoint{4.211648in}{0.990890in}}%
\pgfpathlineto{\pgfqpoint{4.215877in}{0.983633in}}%
\pgfpathlineto{\pgfqpoint{4.217992in}{0.986577in}}%
\pgfpathlineto{\pgfqpoint{4.220107in}{0.986583in}}%
\pgfpathlineto{\pgfqpoint{4.222221in}{0.989886in}}%
\pgfpathlineto{\pgfqpoint{4.224336in}{0.990697in}}%
\pgfpathlineto{\pgfqpoint{4.226451in}{0.986382in}}%
\pgfpathlineto{\pgfqpoint{4.228565in}{0.984780in}}%
\pgfpathlineto{\pgfqpoint{4.232795in}{0.991088in}}%
\pgfpathlineto{\pgfqpoint{4.234910in}{0.997993in}}%
\pgfpathlineto{\pgfqpoint{4.237024in}{0.993883in}}%
\pgfpathlineto{\pgfqpoint{4.239139in}{0.999535in}}%
\pgfpathlineto{\pgfqpoint{4.241254in}{0.999178in}}%
\pgfpathlineto{\pgfqpoint{4.243368in}{1.003439in}}%
\pgfpathlineto{\pgfqpoint{4.245483in}{1.004702in}}%
\pgfpathlineto{\pgfqpoint{4.247598in}{0.993346in}}%
\pgfpathlineto{\pgfqpoint{4.249712in}{0.994433in}}%
\pgfpathlineto{\pgfqpoint{4.253942in}{0.986882in}}%
\pgfpathlineto{\pgfqpoint{4.258171in}{0.985009in}}%
\pgfpathlineto{\pgfqpoint{4.260286in}{0.985612in}}%
\pgfpathlineto{\pgfqpoint{4.262401in}{0.982227in}}%
\pgfpathlineto{\pgfqpoint{4.264515in}{0.984799in}}%
\pgfpathlineto{\pgfqpoint{4.266630in}{0.990906in}}%
\pgfpathlineto{\pgfqpoint{4.268745in}{0.991213in}}%
\pgfpathlineto{\pgfqpoint{4.270859in}{0.988940in}}%
\pgfpathlineto{\pgfqpoint{4.272974in}{0.996234in}}%
\pgfpathlineto{\pgfqpoint{4.275089in}{0.992327in}}%
\pgfpathlineto{\pgfqpoint{4.281433in}{1.002299in}}%
\pgfpathlineto{\pgfqpoint{4.283547in}{1.000503in}}%
\pgfpathlineto{\pgfqpoint{4.285662in}{1.001610in}}%
\pgfpathlineto{\pgfqpoint{4.289892in}{0.993043in}}%
\pgfpathlineto{\pgfqpoint{4.292006in}{0.995886in}}%
\pgfpathlineto{\pgfqpoint{4.296236in}{0.995552in}}%
\pgfpathlineto{\pgfqpoint{4.298350in}{0.990562in}}%
\pgfpathlineto{\pgfqpoint{4.300465in}{0.992479in}}%
\pgfpathlineto{\pgfqpoint{4.304694in}{0.986073in}}%
\pgfpathlineto{\pgfqpoint{4.306809in}{0.987678in}}%
\pgfpathlineto{\pgfqpoint{4.308924in}{0.983358in}}%
\pgfpathlineto{\pgfqpoint{4.311038in}{0.993678in}}%
\pgfpathlineto{\pgfqpoint{4.313153in}{0.991116in}}%
\pgfpathlineto{\pgfqpoint{4.315268in}{0.992148in}}%
\pgfpathlineto{\pgfqpoint{4.317383in}{0.994421in}}%
\pgfpathlineto{\pgfqpoint{4.319497in}{0.985456in}}%
\pgfpathlineto{\pgfqpoint{4.323727in}{0.990708in}}%
\pgfpathlineto{\pgfqpoint{4.325841in}{0.990971in}}%
\pgfpathlineto{\pgfqpoint{4.327956in}{0.993314in}}%
\pgfpathlineto{\pgfqpoint{4.330071in}{0.989314in}}%
\pgfpathlineto{\pgfqpoint{4.332185in}{0.991425in}}%
\pgfpathlineto{\pgfqpoint{4.334300in}{0.999155in}}%
\pgfpathlineto{\pgfqpoint{4.336415in}{0.999367in}}%
\pgfpathlineto{\pgfqpoint{4.338529in}{0.998038in}}%
\pgfpathlineto{\pgfqpoint{4.340644in}{1.002738in}}%
\pgfpathlineto{\pgfqpoint{4.342759in}{0.995655in}}%
\pgfpathlineto{\pgfqpoint{4.346988in}{0.988331in}}%
\pgfpathlineto{\pgfqpoint{4.349103in}{0.981234in}}%
\pgfpathlineto{\pgfqpoint{4.351218in}{0.985856in}}%
\pgfpathlineto{\pgfqpoint{4.353332in}{0.982691in}}%
\pgfpathlineto{\pgfqpoint{4.355447in}{0.983567in}}%
\pgfpathlineto{\pgfqpoint{4.359676in}{0.990770in}}%
\pgfpathlineto{\pgfqpoint{4.361791in}{0.987143in}}%
\pgfpathlineto{\pgfqpoint{4.363906in}{0.986612in}}%
\pgfpathlineto{\pgfqpoint{4.368135in}{0.992593in}}%
\pgfpathlineto{\pgfqpoint{4.370250in}{0.990889in}}%
\pgfpathlineto{\pgfqpoint{4.372365in}{0.991355in}}%
\pgfpathlineto{\pgfqpoint{4.378709in}{1.004294in}}%
\pgfpathlineto{\pgfqpoint{4.380823in}{1.010993in}}%
\pgfpathlineto{\pgfqpoint{4.382938in}{1.007845in}}%
\pgfpathlineto{\pgfqpoint{4.385053in}{1.010387in}}%
\pgfpathlineto{\pgfqpoint{4.387167in}{1.011167in}}%
\pgfpathlineto{\pgfqpoint{4.389282in}{1.013489in}}%
\pgfpathlineto{\pgfqpoint{4.391397in}{1.022029in}}%
\pgfpathlineto{\pgfqpoint{4.393512in}{1.024696in}}%
\pgfpathlineto{\pgfqpoint{4.395626in}{1.013408in}}%
\pgfpathlineto{\pgfqpoint{4.397741in}{1.016022in}}%
\pgfpathlineto{\pgfqpoint{4.404085in}{1.016812in}}%
\pgfpathlineto{\pgfqpoint{4.410429in}{1.002156in}}%
\pgfpathlineto{\pgfqpoint{4.412544in}{1.001365in}}%
\pgfpathlineto{\pgfqpoint{4.416773in}{1.005861in}}%
\pgfpathlineto{\pgfqpoint{4.418888in}{1.007487in}}%
\pgfpathlineto{\pgfqpoint{4.423117in}{1.012701in}}%
\pgfpathlineto{\pgfqpoint{4.425232in}{1.013160in}}%
\pgfpathlineto{\pgfqpoint{4.427347in}{1.010421in}}%
\pgfpathlineto{\pgfqpoint{4.431576in}{1.018727in}}%
\pgfpathlineto{\pgfqpoint{4.433691in}{1.010912in}}%
\pgfpathlineto{\pgfqpoint{4.435805in}{1.007708in}}%
\pgfpathlineto{\pgfqpoint{4.437920in}{1.011005in}}%
\pgfpathlineto{\pgfqpoint{4.440035in}{1.011046in}}%
\pgfpathlineto{\pgfqpoint{4.442149in}{1.007693in}}%
\pgfpathlineto{\pgfqpoint{4.444264in}{1.008662in}}%
\pgfpathlineto{\pgfqpoint{4.448494in}{1.016743in}}%
\pgfpathlineto{\pgfqpoint{4.450608in}{1.015068in}}%
\pgfpathlineto{\pgfqpoint{4.454838in}{1.021072in}}%
\pgfpathlineto{\pgfqpoint{4.456952in}{1.021888in}}%
\pgfpathlineto{\pgfqpoint{4.459067in}{1.028410in}}%
\pgfpathlineto{\pgfqpoint{4.461182in}{1.018031in}}%
\pgfpathlineto{\pgfqpoint{4.463296in}{1.015175in}}%
\pgfpathlineto{\pgfqpoint{4.465411in}{1.014452in}}%
\pgfpathlineto{\pgfqpoint{4.469641in}{1.017004in}}%
\pgfpathlineto{\pgfqpoint{4.484443in}{1.000996in}}%
\pgfpathlineto{\pgfqpoint{4.488673in}{1.000657in}}%
\pgfpathlineto{\pgfqpoint{4.490787in}{1.003402in}}%
\pgfpathlineto{\pgfqpoint{4.492902in}{1.001694in}}%
\pgfpathlineto{\pgfqpoint{4.495017in}{1.004909in}}%
\pgfpathlineto{\pgfqpoint{4.499246in}{1.016119in}}%
\pgfpathlineto{\pgfqpoint{4.503476in}{1.007938in}}%
\pgfpathlineto{\pgfqpoint{4.507705in}{1.004750in}}%
\pgfpathlineto{\pgfqpoint{4.509820in}{1.003384in}}%
\pgfpathlineto{\pgfqpoint{4.511934in}{1.005885in}}%
\pgfpathlineto{\pgfqpoint{4.516164in}{0.998133in}}%
\pgfpathlineto{\pgfqpoint{4.518278in}{1.004337in}}%
\pgfpathlineto{\pgfqpoint{4.520393in}{1.004160in}}%
\pgfpathlineto{\pgfqpoint{4.522508in}{1.002164in}}%
\pgfpathlineto{\pgfqpoint{4.526737in}{1.009114in}}%
\pgfpathlineto{\pgfqpoint{4.537311in}{0.989706in}}%
\pgfpathlineto{\pgfqpoint{4.539425in}{0.990690in}}%
\pgfpathlineto{\pgfqpoint{4.543655in}{1.000920in}}%
\pgfpathlineto{\pgfqpoint{4.547884in}{1.004914in}}%
\pgfpathlineto{\pgfqpoint{4.549999in}{1.004003in}}%
\pgfpathlineto{\pgfqpoint{4.556343in}{1.008620in}}%
\pgfpathlineto{\pgfqpoint{4.560572in}{1.003168in}}%
\pgfpathlineto{\pgfqpoint{4.562687in}{1.009799in}}%
\pgfpathlineto{\pgfqpoint{4.564802in}{1.010448in}}%
\pgfpathlineto{\pgfqpoint{4.569031in}{1.005444in}}%
\pgfpathlineto{\pgfqpoint{4.573260in}{1.005426in}}%
\pgfpathlineto{\pgfqpoint{4.575375in}{1.007838in}}%
\pgfpathlineto{\pgfqpoint{4.581719in}{1.022253in}}%
\pgfpathlineto{\pgfqpoint{4.583834in}{1.021593in}}%
\pgfpathlineto{\pgfqpoint{4.585949in}{1.023694in}}%
\pgfpathlineto{\pgfqpoint{4.588063in}{1.020541in}}%
\pgfpathlineto{\pgfqpoint{4.590178in}{1.028801in}}%
\pgfpathlineto{\pgfqpoint{4.592293in}{1.026303in}}%
\pgfpathlineto{\pgfqpoint{4.596522in}{1.029827in}}%
\pgfpathlineto{\pgfqpoint{4.598637in}{1.031190in}}%
\pgfpathlineto{\pgfqpoint{4.600752in}{1.027751in}}%
\pgfpathlineto{\pgfqpoint{4.607096in}{1.027174in}}%
\pgfpathlineto{\pgfqpoint{4.609210in}{1.028522in}}%
\pgfpathlineto{\pgfqpoint{4.611325in}{1.028320in}}%
\pgfpathlineto{\pgfqpoint{4.615554in}{1.022484in}}%
\pgfpathlineto{\pgfqpoint{4.617669in}{1.020608in}}%
\pgfpathlineto{\pgfqpoint{4.621898in}{1.013311in}}%
\pgfpathlineto{\pgfqpoint{4.624013in}{1.013304in}}%
\pgfpathlineto{\pgfqpoint{4.626128in}{1.009157in}}%
\pgfpathlineto{\pgfqpoint{4.630357in}{1.012854in}}%
\pgfpathlineto{\pgfqpoint{4.632472in}{1.004817in}}%
\pgfpathlineto{\pgfqpoint{4.634587in}{1.002778in}}%
\pgfpathlineto{\pgfqpoint{4.636701in}{0.997846in}}%
\pgfpathlineto{\pgfqpoint{4.640931in}{1.008159in}}%
\pgfpathlineto{\pgfqpoint{4.643045in}{1.003790in}}%
\pgfpathlineto{\pgfqpoint{4.645160in}{1.002065in}}%
\pgfpathlineto{\pgfqpoint{4.647275in}{1.004927in}}%
\pgfpathlineto{\pgfqpoint{4.649389in}{1.017424in}}%
\pgfpathlineto{\pgfqpoint{4.655734in}{1.021265in}}%
\pgfpathlineto{\pgfqpoint{4.657848in}{1.021849in}}%
\pgfpathlineto{\pgfqpoint{4.662078in}{1.028441in}}%
\pgfpathlineto{\pgfqpoint{4.664192in}{1.025979in}}%
\pgfpathlineto{\pgfqpoint{4.666307in}{1.030164in}}%
\pgfpathlineto{\pgfqpoint{4.668422in}{1.031140in}}%
\pgfpathlineto{\pgfqpoint{4.670536in}{1.029707in}}%
\pgfpathlineto{\pgfqpoint{4.672651in}{1.031589in}}%
\pgfpathlineto{\pgfqpoint{4.674766in}{1.031768in}}%
\pgfpathlineto{\pgfqpoint{4.676880in}{1.027215in}}%
\pgfpathlineto{\pgfqpoint{4.678995in}{1.029883in}}%
\pgfpathlineto{\pgfqpoint{4.687454in}{1.017711in}}%
\pgfpathlineto{\pgfqpoint{4.689569in}{1.021861in}}%
\pgfpathlineto{\pgfqpoint{4.691683in}{1.020415in}}%
\pgfpathlineto{\pgfqpoint{4.693798in}{1.025282in}}%
\pgfpathlineto{\pgfqpoint{4.695913in}{1.024120in}}%
\pgfpathlineto{\pgfqpoint{4.698027in}{1.026347in}}%
\pgfpathlineto{\pgfqpoint{4.700142in}{1.025825in}}%
\pgfpathlineto{\pgfqpoint{4.702257in}{1.031395in}}%
\pgfpathlineto{\pgfqpoint{4.704372in}{1.024217in}}%
\pgfpathlineto{\pgfqpoint{4.706486in}{1.021734in}}%
\pgfpathlineto{\pgfqpoint{4.708601in}{1.024713in}}%
\pgfpathlineto{\pgfqpoint{4.710716in}{1.031307in}}%
\pgfpathlineto{\pgfqpoint{4.712830in}{1.027994in}}%
\pgfpathlineto{\pgfqpoint{4.714945in}{1.032224in}}%
\pgfpathlineto{\pgfqpoint{4.717060in}{1.025390in}}%
\pgfpathlineto{\pgfqpoint{4.721289in}{1.031386in}}%
\pgfpathlineto{\pgfqpoint{4.723404in}{1.037270in}}%
\pgfpathlineto{\pgfqpoint{4.725518in}{1.036588in}}%
\pgfpathlineto{\pgfqpoint{4.727633in}{1.039188in}}%
\pgfpathlineto{\pgfqpoint{4.733977in}{1.034338in}}%
\pgfpathlineto{\pgfqpoint{4.736092in}{1.037433in}}%
\pgfpathlineto{\pgfqpoint{4.738207in}{1.035136in}}%
\pgfpathlineto{\pgfqpoint{4.740321in}{1.042036in}}%
\pgfpathlineto{\pgfqpoint{4.742436in}{1.038390in}}%
\pgfpathlineto{\pgfqpoint{4.744551in}{1.041797in}}%
\pgfpathlineto{\pgfqpoint{4.750895in}{1.039062in}}%
\pgfpathlineto{\pgfqpoint{4.755124in}{1.045675in}}%
\pgfpathlineto{\pgfqpoint{4.757239in}{1.051804in}}%
\pgfpathlineto{\pgfqpoint{4.759354in}{1.044753in}}%
\pgfpathlineto{\pgfqpoint{4.761468in}{1.049761in}}%
\pgfpathlineto{\pgfqpoint{4.763583in}{1.044591in}}%
\pgfpathlineto{\pgfqpoint{4.765698in}{1.044332in}}%
\pgfpathlineto{\pgfqpoint{4.767812in}{1.048790in}}%
\pgfpathlineto{\pgfqpoint{4.772042in}{1.046657in}}%
\pgfpathlineto{\pgfqpoint{4.774156in}{1.038297in}}%
\pgfpathlineto{\pgfqpoint{4.776271in}{1.036612in}}%
\pgfpathlineto{\pgfqpoint{4.778386in}{1.040569in}}%
\pgfpathlineto{\pgfqpoint{4.782615in}{1.029603in}}%
\pgfpathlineto{\pgfqpoint{4.784730in}{1.036542in}}%
\pgfpathlineto{\pgfqpoint{4.786845in}{1.035220in}}%
\pgfpathlineto{\pgfqpoint{4.791074in}{1.028195in}}%
\pgfpathlineto{\pgfqpoint{4.793189in}{1.029918in}}%
\pgfpathlineto{\pgfqpoint{4.795303in}{1.029282in}}%
\pgfpathlineto{\pgfqpoint{4.799533in}{1.026220in}}%
\pgfpathlineto{\pgfqpoint{4.803762in}{1.026034in}}%
\pgfpathlineto{\pgfqpoint{4.805877in}{1.028751in}}%
\pgfpathlineto{\pgfqpoint{4.810106in}{1.015452in}}%
\pgfpathlineto{\pgfqpoint{4.812221in}{1.015526in}}%
\pgfpathlineto{\pgfqpoint{4.814336in}{1.024606in}}%
\pgfpathlineto{\pgfqpoint{4.816450in}{1.028648in}}%
\pgfpathlineto{\pgfqpoint{4.818565in}{1.029022in}}%
\pgfpathlineto{\pgfqpoint{4.824909in}{1.019012in}}%
\pgfpathlineto{\pgfqpoint{4.833368in}{1.045384in}}%
\pgfpathlineto{\pgfqpoint{4.835483in}{1.045631in}}%
\pgfpathlineto{\pgfqpoint{4.837597in}{1.038111in}}%
\pgfpathlineto{\pgfqpoint{4.839712in}{1.040025in}}%
\pgfpathlineto{\pgfqpoint{4.843941in}{1.035002in}}%
\pgfpathlineto{\pgfqpoint{4.846056in}{1.032889in}}%
\pgfpathlineto{\pgfqpoint{4.848171in}{1.033416in}}%
\pgfpathlineto{\pgfqpoint{4.850285in}{1.040158in}}%
\pgfpathlineto{\pgfqpoint{4.854515in}{1.038914in}}%
\pgfpathlineto{\pgfqpoint{4.856629in}{1.036556in}}%
\pgfpathlineto{\pgfqpoint{4.858744in}{1.038374in}}%
\pgfpathlineto{\pgfqpoint{4.860859in}{1.037093in}}%
\pgfpathlineto{\pgfqpoint{4.865088in}{1.025449in}}%
\pgfpathlineto{\pgfqpoint{4.869318in}{1.033114in}}%
\pgfpathlineto{\pgfqpoint{4.871432in}{1.040737in}}%
\pgfpathlineto{\pgfqpoint{4.873547in}{1.036301in}}%
\pgfpathlineto{\pgfqpoint{4.875662in}{1.035608in}}%
\pgfpathlineto{\pgfqpoint{4.877776in}{1.044016in}}%
\pgfpathlineto{\pgfqpoint{4.882006in}{1.050905in}}%
\pgfpathlineto{\pgfqpoint{4.884120in}{1.048853in}}%
\pgfpathlineto{\pgfqpoint{4.886235in}{1.040450in}}%
\pgfpathlineto{\pgfqpoint{4.888350in}{1.038178in}}%
\pgfpathlineto{\pgfqpoint{4.892579in}{1.037701in}}%
\pgfpathlineto{\pgfqpoint{4.894694in}{1.033466in}}%
\pgfpathlineto{\pgfqpoint{4.896809in}{1.035230in}}%
\pgfpathlineto{\pgfqpoint{4.901038in}{1.030920in}}%
\pgfpathlineto{\pgfqpoint{4.903153in}{1.022438in}}%
\pgfpathlineto{\pgfqpoint{4.907382in}{1.029781in}}%
\pgfpathlineto{\pgfqpoint{4.909497in}{1.028226in}}%
\pgfpathlineto{\pgfqpoint{4.913726in}{1.015550in}}%
\pgfpathlineto{\pgfqpoint{4.917956in}{1.027998in}}%
\pgfpathlineto{\pgfqpoint{4.926414in}{1.037572in}}%
\pgfpathlineto{\pgfqpoint{4.928529in}{1.031006in}}%
\pgfpathlineto{\pgfqpoint{4.930644in}{1.036692in}}%
\pgfpathlineto{\pgfqpoint{4.932758in}{1.035760in}}%
\pgfpathlineto{\pgfqpoint{4.934873in}{1.038386in}}%
\pgfpathlineto{\pgfqpoint{4.936988in}{1.036116in}}%
\pgfpathlineto{\pgfqpoint{4.939103in}{1.040887in}}%
\pgfpathlineto{\pgfqpoint{4.941217in}{1.040611in}}%
\pgfpathlineto{\pgfqpoint{4.943332in}{1.043654in}}%
\pgfpathlineto{\pgfqpoint{4.945447in}{1.043596in}}%
\pgfpathlineto{\pgfqpoint{4.947561in}{1.047780in}}%
\pgfpathlineto{\pgfqpoint{4.949676in}{1.049045in}}%
\pgfpathlineto{\pgfqpoint{4.951791in}{1.045461in}}%
\pgfpathlineto{\pgfqpoint{4.953905in}{1.048323in}}%
\pgfpathlineto{\pgfqpoint{4.956020in}{1.044471in}}%
\pgfpathlineto{\pgfqpoint{4.958135in}{1.045910in}}%
\pgfpathlineto{\pgfqpoint{4.962364in}{1.057392in}}%
\pgfpathlineto{\pgfqpoint{4.964479in}{1.052538in}}%
\pgfpathlineto{\pgfqpoint{4.966594in}{1.051112in}}%
\pgfpathlineto{\pgfqpoint{4.968708in}{1.047535in}}%
\pgfpathlineto{\pgfqpoint{4.970823in}{1.046697in}}%
\pgfpathlineto{\pgfqpoint{4.972938in}{1.052002in}}%
\pgfpathlineto{\pgfqpoint{4.975052in}{1.053708in}}%
\pgfpathlineto{\pgfqpoint{4.977167in}{1.049915in}}%
\pgfpathlineto{\pgfqpoint{4.979282in}{1.052390in}}%
\pgfpathlineto{\pgfqpoint{4.981396in}{1.044566in}}%
\pgfpathlineto{\pgfqpoint{4.983511in}{1.044070in}}%
\pgfpathlineto{\pgfqpoint{4.985626in}{1.046836in}}%
\pgfpathlineto{\pgfqpoint{4.987740in}{1.046009in}}%
\pgfpathlineto{\pgfqpoint{4.989855in}{1.052813in}}%
\pgfpathlineto{\pgfqpoint{4.994085in}{1.046053in}}%
\pgfpathlineto{\pgfqpoint{4.996199in}{1.049822in}}%
\pgfpathlineto{\pgfqpoint{4.998314in}{1.043310in}}%
\pgfpathlineto{\pgfqpoint{5.002543in}{1.042291in}}%
\pgfpathlineto{\pgfqpoint{5.004658in}{1.046734in}}%
\pgfpathlineto{\pgfqpoint{5.006773in}{1.043668in}}%
\pgfpathlineto{\pgfqpoint{5.008887in}{1.036598in}}%
\pgfpathlineto{\pgfqpoint{5.011002in}{1.037363in}}%
\pgfpathlineto{\pgfqpoint{5.015231in}{1.027526in}}%
\pgfpathlineto{\pgfqpoint{5.017346in}{1.026676in}}%
\pgfpathlineto{\pgfqpoint{5.021576in}{1.013582in}}%
\pgfpathlineto{\pgfqpoint{5.023690in}{1.012276in}}%
\pgfpathlineto{\pgfqpoint{5.025805in}{1.005866in}}%
\pgfpathlineto{\pgfqpoint{5.030034in}{1.003159in}}%
\pgfpathlineto{\pgfqpoint{5.032149in}{1.002608in}}%
\pgfpathlineto{\pgfqpoint{5.036378in}{0.993455in}}%
\pgfpathlineto{\pgfqpoint{5.038493in}{0.991269in}}%
\pgfpathlineto{\pgfqpoint{5.040608in}{0.994506in}}%
\pgfpathlineto{\pgfqpoint{5.042722in}{0.989746in}}%
\pgfpathlineto{\pgfqpoint{5.044837in}{0.988991in}}%
\pgfpathlineto{\pgfqpoint{5.049067in}{0.979823in}}%
\pgfpathlineto{\pgfqpoint{5.051181in}{0.978052in}}%
\pgfpathlineto{\pgfqpoint{5.055411in}{0.985825in}}%
\pgfpathlineto{\pgfqpoint{5.057525in}{0.982555in}}%
\pgfpathlineto{\pgfqpoint{5.061755in}{0.992852in}}%
\pgfpathlineto{\pgfqpoint{5.063869in}{0.991264in}}%
\pgfpathlineto{\pgfqpoint{5.065984in}{0.988255in}}%
\pgfpathlineto{\pgfqpoint{5.070214in}{0.989274in}}%
\pgfpathlineto{\pgfqpoint{5.074443in}{0.996893in}}%
\pgfpathlineto{\pgfqpoint{5.076558in}{0.997976in}}%
\pgfpathlineto{\pgfqpoint{5.087131in}{0.977687in}}%
\pgfpathlineto{\pgfqpoint{5.089246in}{0.977121in}}%
\pgfpathlineto{\pgfqpoint{5.091360in}{0.979790in}}%
\pgfpathlineto{\pgfqpoint{5.093475in}{0.979368in}}%
\pgfpathlineto{\pgfqpoint{5.095590in}{0.976136in}}%
\pgfpathlineto{\pgfqpoint{5.101934in}{0.954233in}}%
\pgfpathlineto{\pgfqpoint{5.104049in}{0.951717in}}%
\pgfpathlineto{\pgfqpoint{5.108278in}{0.957583in}}%
\pgfpathlineto{\pgfqpoint{5.110393in}{0.953160in}}%
\pgfpathlineto{\pgfqpoint{5.112507in}{0.953908in}}%
\pgfpathlineto{\pgfqpoint{5.114622in}{0.957320in}}%
\pgfpathlineto{\pgfqpoint{5.116737in}{0.953653in}}%
\pgfpathlineto{\pgfqpoint{5.118851in}{0.952818in}}%
\pgfpathlineto{\pgfqpoint{5.120966in}{0.954342in}}%
\pgfpathlineto{\pgfqpoint{5.123081in}{0.947652in}}%
\pgfpathlineto{\pgfqpoint{5.125196in}{0.946770in}}%
\pgfpathlineto{\pgfqpoint{5.127310in}{0.943810in}}%
\pgfpathlineto{\pgfqpoint{5.129425in}{0.945269in}}%
\pgfpathlineto{\pgfqpoint{5.131540in}{0.948611in}}%
\pgfpathlineto{\pgfqpoint{5.135769in}{0.938463in}}%
\pgfpathlineto{\pgfqpoint{5.139998in}{0.941014in}}%
\pgfpathlineto{\pgfqpoint{5.142113in}{0.950985in}}%
\pgfpathlineto{\pgfqpoint{5.144228in}{0.951799in}}%
\pgfpathlineto{\pgfqpoint{5.146342in}{0.954046in}}%
\pgfpathlineto{\pgfqpoint{5.148457in}{0.968181in}}%
\pgfpathlineto{\pgfqpoint{5.150572in}{0.971337in}}%
\pgfpathlineto{\pgfqpoint{5.152687in}{0.971530in}}%
\pgfpathlineto{\pgfqpoint{5.154801in}{0.972988in}}%
\pgfpathlineto{\pgfqpoint{5.156916in}{0.977403in}}%
\pgfpathlineto{\pgfqpoint{5.159031in}{0.977748in}}%
\pgfpathlineto{\pgfqpoint{5.165375in}{0.972416in}}%
\pgfpathlineto{\pgfqpoint{5.167489in}{0.975033in}}%
\pgfpathlineto{\pgfqpoint{5.169604in}{0.974881in}}%
\pgfpathlineto{\pgfqpoint{5.171719in}{0.979984in}}%
\pgfpathlineto{\pgfqpoint{5.173834in}{0.981820in}}%
\pgfpathlineto{\pgfqpoint{5.186522in}{0.971273in}}%
\pgfpathlineto{\pgfqpoint{5.188636in}{0.967767in}}%
\pgfpathlineto{\pgfqpoint{5.188636in}{0.967767in}}%
\pgfusepath{stroke}%
\end{pgfscope}%
\begin{pgfscope}%
\pgfpathrectangle{\pgfqpoint{0.750000in}{0.275000in}}{\pgfqpoint{4.650000in}{1.925000in}}%
\pgfusepath{clip}%
\pgfsetroundcap%
\pgfsetroundjoin%
\pgfsetlinewidth{1.003750pt}%
\definecolor{currentstroke}{rgb}{0.968627,0.505882,0.749020}%
\pgfsetstrokecolor{currentstroke}%
\pgfsetdash{}{0pt}%
\pgfpathmoveto{\pgfqpoint{0.961364in}{1.209433in}}%
\pgfpathlineto{\pgfqpoint{0.967708in}{1.199981in}}%
\pgfpathlineto{\pgfqpoint{0.971937in}{1.187131in}}%
\pgfpathlineto{\pgfqpoint{0.974052in}{1.189790in}}%
\pgfpathlineto{\pgfqpoint{0.976166in}{1.186906in}}%
\pgfpathlineto{\pgfqpoint{0.978281in}{1.186339in}}%
\pgfpathlineto{\pgfqpoint{0.980396in}{1.183264in}}%
\pgfpathlineto{\pgfqpoint{0.982511in}{1.187214in}}%
\pgfpathlineto{\pgfqpoint{0.984625in}{1.178615in}}%
\pgfpathlineto{\pgfqpoint{0.986740in}{1.176737in}}%
\pgfpathlineto{\pgfqpoint{0.988855in}{1.185747in}}%
\pgfpathlineto{\pgfqpoint{0.990969in}{1.177286in}}%
\pgfpathlineto{\pgfqpoint{0.997313in}{1.174841in}}%
\pgfpathlineto{\pgfqpoint{1.001543in}{1.166699in}}%
\pgfpathlineto{\pgfqpoint{1.005772in}{1.172403in}}%
\pgfpathlineto{\pgfqpoint{1.007887in}{1.169484in}}%
\pgfpathlineto{\pgfqpoint{1.010002in}{1.170773in}}%
\pgfpathlineto{\pgfqpoint{1.012116in}{1.175013in}}%
\pgfpathlineto{\pgfqpoint{1.014231in}{1.173693in}}%
\pgfpathlineto{\pgfqpoint{1.016346in}{1.174107in}}%
\pgfpathlineto{\pgfqpoint{1.020575in}{1.179416in}}%
\pgfpathlineto{\pgfqpoint{1.022690in}{1.179820in}}%
\pgfpathlineto{\pgfqpoint{1.024804in}{1.177710in}}%
\pgfpathlineto{\pgfqpoint{1.029034in}{1.171986in}}%
\pgfpathlineto{\pgfqpoint{1.031149in}{1.173719in}}%
\pgfpathlineto{\pgfqpoint{1.033263in}{1.169859in}}%
\pgfpathlineto{\pgfqpoint{1.037493in}{1.180602in}}%
\pgfpathlineto{\pgfqpoint{1.039607in}{1.179229in}}%
\pgfpathlineto{\pgfqpoint{1.041722in}{1.176363in}}%
\pgfpathlineto{\pgfqpoint{1.043837in}{1.171395in}}%
\pgfpathlineto{\pgfqpoint{1.048066in}{1.187153in}}%
\pgfpathlineto{\pgfqpoint{1.050181in}{1.190741in}}%
\pgfpathlineto{\pgfqpoint{1.052295in}{1.191073in}}%
\pgfpathlineto{\pgfqpoint{1.054410in}{1.197281in}}%
\pgfpathlineto{\pgfqpoint{1.056525in}{1.197124in}}%
\pgfpathlineto{\pgfqpoint{1.060754in}{1.191547in}}%
\pgfpathlineto{\pgfqpoint{1.067098in}{1.189831in}}%
\pgfpathlineto{\pgfqpoint{1.071328in}{1.205243in}}%
\pgfpathlineto{\pgfqpoint{1.073442in}{1.212086in}}%
\pgfpathlineto{\pgfqpoint{1.077672in}{1.201392in}}%
\pgfpathlineto{\pgfqpoint{1.079786in}{1.200982in}}%
\pgfpathlineto{\pgfqpoint{1.084016in}{1.198011in}}%
\pgfpathlineto{\pgfqpoint{1.086131in}{1.204328in}}%
\pgfpathlineto{\pgfqpoint{1.088245in}{1.203159in}}%
\pgfpathlineto{\pgfqpoint{1.090360in}{1.205479in}}%
\pgfpathlineto{\pgfqpoint{1.092475in}{1.205038in}}%
\pgfpathlineto{\pgfqpoint{1.094589in}{1.212911in}}%
\pgfpathlineto{\pgfqpoint{1.096704in}{1.213061in}}%
\pgfpathlineto{\pgfqpoint{1.098819in}{1.219782in}}%
\pgfpathlineto{\pgfqpoint{1.103048in}{1.221659in}}%
\pgfpathlineto{\pgfqpoint{1.105163in}{1.215745in}}%
\pgfpathlineto{\pgfqpoint{1.107278in}{1.213282in}}%
\pgfpathlineto{\pgfqpoint{1.111507in}{1.202993in}}%
\pgfpathlineto{\pgfqpoint{1.113622in}{1.204547in}}%
\pgfpathlineto{\pgfqpoint{1.115736in}{1.203693in}}%
\pgfpathlineto{\pgfqpoint{1.117851in}{1.210498in}}%
\pgfpathlineto{\pgfqpoint{1.119966in}{1.203385in}}%
\pgfpathlineto{\pgfqpoint{1.122080in}{1.209056in}}%
\pgfpathlineto{\pgfqpoint{1.124195in}{1.206602in}}%
\pgfpathlineto{\pgfqpoint{1.128424in}{1.199240in}}%
\pgfpathlineto{\pgfqpoint{1.130539in}{1.201003in}}%
\pgfpathlineto{\pgfqpoint{1.132654in}{1.199886in}}%
\pgfpathlineto{\pgfqpoint{1.134769in}{1.200501in}}%
\pgfpathlineto{\pgfqpoint{1.138998in}{1.197475in}}%
\pgfpathlineto{\pgfqpoint{1.145342in}{1.188484in}}%
\pgfpathlineto{\pgfqpoint{1.149571in}{1.198092in}}%
\pgfpathlineto{\pgfqpoint{1.151686in}{1.198748in}}%
\pgfpathlineto{\pgfqpoint{1.153801in}{1.205362in}}%
\pgfpathlineto{\pgfqpoint{1.155915in}{1.207388in}}%
\pgfpathlineto{\pgfqpoint{1.158030in}{1.213114in}}%
\pgfpathlineto{\pgfqpoint{1.160145in}{1.215604in}}%
\pgfpathlineto{\pgfqpoint{1.162260in}{1.215398in}}%
\pgfpathlineto{\pgfqpoint{1.164374in}{1.207423in}}%
\pgfpathlineto{\pgfqpoint{1.166489in}{1.214924in}}%
\pgfpathlineto{\pgfqpoint{1.170718in}{1.200858in}}%
\pgfpathlineto{\pgfqpoint{1.174948in}{1.182602in}}%
\pgfpathlineto{\pgfqpoint{1.181292in}{1.178253in}}%
\pgfpathlineto{\pgfqpoint{1.183406in}{1.174224in}}%
\pgfpathlineto{\pgfqpoint{1.185521in}{1.175803in}}%
\pgfpathlineto{\pgfqpoint{1.187636in}{1.173168in}}%
\pgfpathlineto{\pgfqpoint{1.196095in}{1.175765in}}%
\pgfpathlineto{\pgfqpoint{1.198209in}{1.171371in}}%
\pgfpathlineto{\pgfqpoint{1.200324in}{1.171908in}}%
\pgfpathlineto{\pgfqpoint{1.202439in}{1.175402in}}%
\pgfpathlineto{\pgfqpoint{1.206668in}{1.166294in}}%
\pgfpathlineto{\pgfqpoint{1.208783in}{1.170330in}}%
\pgfpathlineto{\pgfqpoint{1.215127in}{1.162848in}}%
\pgfpathlineto{\pgfqpoint{1.217242in}{1.164073in}}%
\pgfpathlineto{\pgfqpoint{1.219356in}{1.162155in}}%
\pgfpathlineto{\pgfqpoint{1.223586in}{1.154688in}}%
\pgfpathlineto{\pgfqpoint{1.225700in}{1.155505in}}%
\pgfpathlineto{\pgfqpoint{1.229930in}{1.160188in}}%
\pgfpathlineto{\pgfqpoint{1.236274in}{1.159927in}}%
\pgfpathlineto{\pgfqpoint{1.240503in}{1.168305in}}%
\pgfpathlineto{\pgfqpoint{1.242618in}{1.167737in}}%
\pgfpathlineto{\pgfqpoint{1.244733in}{1.159274in}}%
\pgfpathlineto{\pgfqpoint{1.246847in}{1.162952in}}%
\pgfpathlineto{\pgfqpoint{1.248962in}{1.162970in}}%
\pgfpathlineto{\pgfqpoint{1.251077in}{1.168352in}}%
\pgfpathlineto{\pgfqpoint{1.255306in}{1.161598in}}%
\pgfpathlineto{\pgfqpoint{1.257421in}{1.166129in}}%
\pgfpathlineto{\pgfqpoint{1.259535in}{1.162346in}}%
\pgfpathlineto{\pgfqpoint{1.261650in}{1.163462in}}%
\pgfpathlineto{\pgfqpoint{1.263765in}{1.154984in}}%
\pgfpathlineto{\pgfqpoint{1.270109in}{1.157135in}}%
\pgfpathlineto{\pgfqpoint{1.272224in}{1.161200in}}%
\pgfpathlineto{\pgfqpoint{1.274338in}{1.160687in}}%
\pgfpathlineto{\pgfqpoint{1.278568in}{1.154476in}}%
\pgfpathlineto{\pgfqpoint{1.280682in}{1.153957in}}%
\pgfpathlineto{\pgfqpoint{1.282797in}{1.148986in}}%
\pgfpathlineto{\pgfqpoint{1.287026in}{1.154540in}}%
\pgfpathlineto{\pgfqpoint{1.289141in}{1.149964in}}%
\pgfpathlineto{\pgfqpoint{1.291256in}{1.151828in}}%
\pgfpathlineto{\pgfqpoint{1.297600in}{1.148638in}}%
\pgfpathlineto{\pgfqpoint{1.299715in}{1.148355in}}%
\pgfpathlineto{\pgfqpoint{1.303944in}{1.140058in}}%
\pgfpathlineto{\pgfqpoint{1.306059in}{1.141460in}}%
\pgfpathlineto{\pgfqpoint{1.308173in}{1.136563in}}%
\pgfpathlineto{\pgfqpoint{1.310288in}{1.138255in}}%
\pgfpathlineto{\pgfqpoint{1.314517in}{1.144393in}}%
\pgfpathlineto{\pgfqpoint{1.316632in}{1.145573in}}%
\pgfpathlineto{\pgfqpoint{1.318747in}{1.150044in}}%
\pgfpathlineto{\pgfqpoint{1.320862in}{1.149639in}}%
\pgfpathlineto{\pgfqpoint{1.322976in}{1.150373in}}%
\pgfpathlineto{\pgfqpoint{1.325091in}{1.143410in}}%
\pgfpathlineto{\pgfqpoint{1.327206in}{1.146562in}}%
\pgfpathlineto{\pgfqpoint{1.329320in}{1.156356in}}%
\pgfpathlineto{\pgfqpoint{1.333550in}{1.148731in}}%
\pgfpathlineto{\pgfqpoint{1.337779in}{1.163023in}}%
\pgfpathlineto{\pgfqpoint{1.339894in}{1.163483in}}%
\pgfpathlineto{\pgfqpoint{1.342009in}{1.169447in}}%
\pgfpathlineto{\pgfqpoint{1.346238in}{1.169119in}}%
\pgfpathlineto{\pgfqpoint{1.348353in}{1.167322in}}%
\pgfpathlineto{\pgfqpoint{1.350467in}{1.168653in}}%
\pgfpathlineto{\pgfqpoint{1.352582in}{1.166264in}}%
\pgfpathlineto{\pgfqpoint{1.354697in}{1.168303in}}%
\pgfpathlineto{\pgfqpoint{1.356811in}{1.176728in}}%
\pgfpathlineto{\pgfqpoint{1.361041in}{1.168115in}}%
\pgfpathlineto{\pgfqpoint{1.363155in}{1.166006in}}%
\pgfpathlineto{\pgfqpoint{1.367385in}{1.175441in}}%
\pgfpathlineto{\pgfqpoint{1.371614in}{1.173540in}}%
\pgfpathlineto{\pgfqpoint{1.373729in}{1.176761in}}%
\pgfpathlineto{\pgfqpoint{1.377958in}{1.187453in}}%
\pgfpathlineto{\pgfqpoint{1.380073in}{1.189264in}}%
\pgfpathlineto{\pgfqpoint{1.382188in}{1.188187in}}%
\pgfpathlineto{\pgfqpoint{1.384302in}{1.193698in}}%
\pgfpathlineto{\pgfqpoint{1.386417in}{1.192154in}}%
\pgfpathlineto{\pgfqpoint{1.388532in}{1.193356in}}%
\pgfpathlineto{\pgfqpoint{1.390646in}{1.197901in}}%
\pgfpathlineto{\pgfqpoint{1.392761in}{1.196188in}}%
\pgfpathlineto{\pgfqpoint{1.396991in}{1.198227in}}%
\pgfpathlineto{\pgfqpoint{1.399105in}{1.196273in}}%
\pgfpathlineto{\pgfqpoint{1.401220in}{1.201982in}}%
\pgfpathlineto{\pgfqpoint{1.407564in}{1.191302in}}%
\pgfpathlineto{\pgfqpoint{1.413908in}{1.200779in}}%
\pgfpathlineto{\pgfqpoint{1.422367in}{1.188823in}}%
\pgfpathlineto{\pgfqpoint{1.424482in}{1.189552in}}%
\pgfpathlineto{\pgfqpoint{1.426596in}{1.192249in}}%
\pgfpathlineto{\pgfqpoint{1.428711in}{1.192313in}}%
\pgfpathlineto{\pgfqpoint{1.432940in}{1.188355in}}%
\pgfpathlineto{\pgfqpoint{1.435055in}{1.193187in}}%
\pgfpathlineto{\pgfqpoint{1.439284in}{1.187418in}}%
\pgfpathlineto{\pgfqpoint{1.441399in}{1.191994in}}%
\pgfpathlineto{\pgfqpoint{1.443514in}{1.191428in}}%
\pgfpathlineto{\pgfqpoint{1.445628in}{1.193367in}}%
\pgfpathlineto{\pgfqpoint{1.447743in}{1.192698in}}%
\pgfpathlineto{\pgfqpoint{1.449858in}{1.189335in}}%
\pgfpathlineto{\pgfqpoint{1.454087in}{1.194632in}}%
\pgfpathlineto{\pgfqpoint{1.458317in}{1.190411in}}%
\pgfpathlineto{\pgfqpoint{1.460431in}{1.192736in}}%
\pgfpathlineto{\pgfqpoint{1.462546in}{1.189612in}}%
\pgfpathlineto{\pgfqpoint{1.464661in}{1.196033in}}%
\pgfpathlineto{\pgfqpoint{1.468890in}{1.201390in}}%
\pgfpathlineto{\pgfqpoint{1.473120in}{1.196304in}}%
\pgfpathlineto{\pgfqpoint{1.475234in}{1.207886in}}%
\pgfpathlineto{\pgfqpoint{1.477349in}{1.202444in}}%
\pgfpathlineto{\pgfqpoint{1.481578in}{1.207552in}}%
\pgfpathlineto{\pgfqpoint{1.483693in}{1.210250in}}%
\pgfpathlineto{\pgfqpoint{1.485808in}{1.200486in}}%
\pgfpathlineto{\pgfqpoint{1.487922in}{1.199128in}}%
\pgfpathlineto{\pgfqpoint{1.492152in}{1.190196in}}%
\pgfpathlineto{\pgfqpoint{1.494266in}{1.188528in}}%
\pgfpathlineto{\pgfqpoint{1.496381in}{1.184316in}}%
\pgfpathlineto{\pgfqpoint{1.498496in}{1.186406in}}%
\pgfpathlineto{\pgfqpoint{1.500611in}{1.191528in}}%
\pgfpathlineto{\pgfqpoint{1.502725in}{1.192615in}}%
\pgfpathlineto{\pgfqpoint{1.504840in}{1.186405in}}%
\pgfpathlineto{\pgfqpoint{1.513299in}{1.182737in}}%
\pgfpathlineto{\pgfqpoint{1.515413in}{1.188172in}}%
\pgfpathlineto{\pgfqpoint{1.517528in}{1.188417in}}%
\pgfpathlineto{\pgfqpoint{1.519643in}{1.195570in}}%
\pgfpathlineto{\pgfqpoint{1.521757in}{1.195172in}}%
\pgfpathlineto{\pgfqpoint{1.523872in}{1.185363in}}%
\pgfpathlineto{\pgfqpoint{1.525987in}{1.186224in}}%
\pgfpathlineto{\pgfqpoint{1.530216in}{1.185600in}}%
\pgfpathlineto{\pgfqpoint{1.532331in}{1.189162in}}%
\pgfpathlineto{\pgfqpoint{1.534446in}{1.188115in}}%
\pgfpathlineto{\pgfqpoint{1.536560in}{1.189216in}}%
\pgfpathlineto{\pgfqpoint{1.538675in}{1.186287in}}%
\pgfpathlineto{\pgfqpoint{1.545019in}{1.188317in}}%
\pgfpathlineto{\pgfqpoint{1.547134in}{1.187662in}}%
\pgfpathlineto{\pgfqpoint{1.549248in}{1.193290in}}%
\pgfpathlineto{\pgfqpoint{1.551363in}{1.189959in}}%
\pgfpathlineto{\pgfqpoint{1.555593in}{1.200431in}}%
\pgfpathlineto{\pgfqpoint{1.557707in}{1.201019in}}%
\pgfpathlineto{\pgfqpoint{1.559822in}{1.203471in}}%
\pgfpathlineto{\pgfqpoint{1.561937in}{1.207786in}}%
\pgfpathlineto{\pgfqpoint{1.564051in}{1.205418in}}%
\pgfpathlineto{\pgfqpoint{1.566166in}{1.198421in}}%
\pgfpathlineto{\pgfqpoint{1.568281in}{1.184130in}}%
\pgfpathlineto{\pgfqpoint{1.570395in}{1.184003in}}%
\pgfpathlineto{\pgfqpoint{1.572510in}{1.182584in}}%
\pgfpathlineto{\pgfqpoint{1.574625in}{1.183682in}}%
\pgfpathlineto{\pgfqpoint{1.576740in}{1.183089in}}%
\pgfpathlineto{\pgfqpoint{1.578854in}{1.180373in}}%
\pgfpathlineto{\pgfqpoint{1.580969in}{1.182356in}}%
\pgfpathlineto{\pgfqpoint{1.583084in}{1.178528in}}%
\pgfpathlineto{\pgfqpoint{1.585198in}{1.185399in}}%
\pgfpathlineto{\pgfqpoint{1.587313in}{1.182535in}}%
\pgfpathlineto{\pgfqpoint{1.589428in}{1.175035in}}%
\pgfpathlineto{\pgfqpoint{1.593657in}{1.178607in}}%
\pgfpathlineto{\pgfqpoint{1.595772in}{1.176281in}}%
\pgfpathlineto{\pgfqpoint{1.597886in}{1.176966in}}%
\pgfpathlineto{\pgfqpoint{1.602116in}{1.175061in}}%
\pgfpathlineto{\pgfqpoint{1.604231in}{1.180374in}}%
\pgfpathlineto{\pgfqpoint{1.606345in}{1.178060in}}%
\pgfpathlineto{\pgfqpoint{1.610575in}{1.190805in}}%
\pgfpathlineto{\pgfqpoint{1.614804in}{1.176495in}}%
\pgfpathlineto{\pgfqpoint{1.619033in}{1.183851in}}%
\pgfpathlineto{\pgfqpoint{1.625377in}{1.169991in}}%
\pgfpathlineto{\pgfqpoint{1.627492in}{1.168507in}}%
\pgfpathlineto{\pgfqpoint{1.633836in}{1.148565in}}%
\pgfpathlineto{\pgfqpoint{1.635951in}{1.149140in}}%
\pgfpathlineto{\pgfqpoint{1.638066in}{1.155015in}}%
\pgfpathlineto{\pgfqpoint{1.640180in}{1.156718in}}%
\pgfpathlineto{\pgfqpoint{1.642295in}{1.152160in}}%
\pgfpathlineto{\pgfqpoint{1.644410in}{1.156062in}}%
\pgfpathlineto{\pgfqpoint{1.646524in}{1.148461in}}%
\pgfpathlineto{\pgfqpoint{1.650754in}{1.158014in}}%
\pgfpathlineto{\pgfqpoint{1.654983in}{1.152598in}}%
\pgfpathlineto{\pgfqpoint{1.657098in}{1.152072in}}%
\pgfpathlineto{\pgfqpoint{1.659213in}{1.156724in}}%
\pgfpathlineto{\pgfqpoint{1.661327in}{1.153715in}}%
\pgfpathlineto{\pgfqpoint{1.663442in}{1.154058in}}%
\pgfpathlineto{\pgfqpoint{1.665557in}{1.157498in}}%
\pgfpathlineto{\pgfqpoint{1.669786in}{1.159793in}}%
\pgfpathlineto{\pgfqpoint{1.671901in}{1.166616in}}%
\pgfpathlineto{\pgfqpoint{1.674015in}{1.163706in}}%
\pgfpathlineto{\pgfqpoint{1.676130in}{1.164051in}}%
\pgfpathlineto{\pgfqpoint{1.678245in}{1.162446in}}%
\pgfpathlineto{\pgfqpoint{1.680359in}{1.164703in}}%
\pgfpathlineto{\pgfqpoint{1.682474in}{1.163779in}}%
\pgfpathlineto{\pgfqpoint{1.688818in}{1.180123in}}%
\pgfpathlineto{\pgfqpoint{1.690933in}{1.173852in}}%
\pgfpathlineto{\pgfqpoint{1.693048in}{1.173240in}}%
\pgfpathlineto{\pgfqpoint{1.695162in}{1.174190in}}%
\pgfpathlineto{\pgfqpoint{1.697277in}{1.168063in}}%
\pgfpathlineto{\pgfqpoint{1.699392in}{1.176075in}}%
\pgfpathlineto{\pgfqpoint{1.701506in}{1.175631in}}%
\pgfpathlineto{\pgfqpoint{1.703621in}{1.171958in}}%
\pgfpathlineto{\pgfqpoint{1.705736in}{1.176138in}}%
\pgfpathlineto{\pgfqpoint{1.709965in}{1.191646in}}%
\pgfpathlineto{\pgfqpoint{1.712080in}{1.190443in}}%
\pgfpathlineto{\pgfqpoint{1.716309in}{1.196612in}}%
\pgfpathlineto{\pgfqpoint{1.718424in}{1.196220in}}%
\pgfpathlineto{\pgfqpoint{1.720539in}{1.197243in}}%
\pgfpathlineto{\pgfqpoint{1.722653in}{1.193992in}}%
\pgfpathlineto{\pgfqpoint{1.724768in}{1.184189in}}%
\pgfpathlineto{\pgfqpoint{1.728997in}{1.191094in}}%
\pgfpathlineto{\pgfqpoint{1.731112in}{1.192935in}}%
\pgfpathlineto{\pgfqpoint{1.735342in}{1.180745in}}%
\pgfpathlineto{\pgfqpoint{1.737456in}{1.178728in}}%
\pgfpathlineto{\pgfqpoint{1.739571in}{1.174663in}}%
\pgfpathlineto{\pgfqpoint{1.741686in}{1.175482in}}%
\pgfpathlineto{\pgfqpoint{1.743800in}{1.181651in}}%
\pgfpathlineto{\pgfqpoint{1.745915in}{1.182009in}}%
\pgfpathlineto{\pgfqpoint{1.748030in}{1.174950in}}%
\pgfpathlineto{\pgfqpoint{1.750144in}{1.175769in}}%
\pgfpathlineto{\pgfqpoint{1.752259in}{1.172454in}}%
\pgfpathlineto{\pgfqpoint{1.754374in}{1.173108in}}%
\pgfpathlineto{\pgfqpoint{1.758603in}{1.181540in}}%
\pgfpathlineto{\pgfqpoint{1.760718in}{1.180142in}}%
\pgfpathlineto{\pgfqpoint{1.767062in}{1.188261in}}%
\pgfpathlineto{\pgfqpoint{1.769177in}{1.190316in}}%
\pgfpathlineto{\pgfqpoint{1.771291in}{1.198342in}}%
\pgfpathlineto{\pgfqpoint{1.775521in}{1.194984in}}%
\pgfpathlineto{\pgfqpoint{1.777635in}{1.203304in}}%
\pgfpathlineto{\pgfqpoint{1.781865in}{1.193798in}}%
\pgfpathlineto{\pgfqpoint{1.783979in}{1.195411in}}%
\pgfpathlineto{\pgfqpoint{1.786094in}{1.187640in}}%
\pgfpathlineto{\pgfqpoint{1.788209in}{1.186720in}}%
\pgfpathlineto{\pgfqpoint{1.794553in}{1.193736in}}%
\pgfpathlineto{\pgfqpoint{1.796668in}{1.194795in}}%
\pgfpathlineto{\pgfqpoint{1.798782in}{1.191407in}}%
\pgfpathlineto{\pgfqpoint{1.800897in}{1.183655in}}%
\pgfpathlineto{\pgfqpoint{1.803012in}{1.185209in}}%
\pgfpathlineto{\pgfqpoint{1.807241in}{1.177506in}}%
\pgfpathlineto{\pgfqpoint{1.809356in}{1.176861in}}%
\pgfpathlineto{\pgfqpoint{1.811471in}{1.180856in}}%
\pgfpathlineto{\pgfqpoint{1.815700in}{1.177443in}}%
\pgfpathlineto{\pgfqpoint{1.817815in}{1.180369in}}%
\pgfpathlineto{\pgfqpoint{1.822044in}{1.178320in}}%
\pgfpathlineto{\pgfqpoint{1.824159in}{1.174628in}}%
\pgfpathlineto{\pgfqpoint{1.826273in}{1.173653in}}%
\pgfpathlineto{\pgfqpoint{1.828388in}{1.170175in}}%
\pgfpathlineto{\pgfqpoint{1.830503in}{1.172729in}}%
\pgfpathlineto{\pgfqpoint{1.832617in}{1.177444in}}%
\pgfpathlineto{\pgfqpoint{1.834732in}{1.175337in}}%
\pgfpathlineto{\pgfqpoint{1.836847in}{1.174817in}}%
\pgfpathlineto{\pgfqpoint{1.838962in}{1.171731in}}%
\pgfpathlineto{\pgfqpoint{1.841076in}{1.174555in}}%
\pgfpathlineto{\pgfqpoint{1.845306in}{1.171819in}}%
\pgfpathlineto{\pgfqpoint{1.847420in}{1.169558in}}%
\pgfpathlineto{\pgfqpoint{1.851650in}{1.155072in}}%
\pgfpathlineto{\pgfqpoint{1.853764in}{1.159347in}}%
\pgfpathlineto{\pgfqpoint{1.855879in}{1.160883in}}%
\pgfpathlineto{\pgfqpoint{1.857994in}{1.156733in}}%
\pgfpathlineto{\pgfqpoint{1.862223in}{1.141683in}}%
\pgfpathlineto{\pgfqpoint{1.864338in}{1.140585in}}%
\pgfpathlineto{\pgfqpoint{1.868567in}{1.131582in}}%
\pgfpathlineto{\pgfqpoint{1.870682in}{1.131390in}}%
\pgfpathlineto{\pgfqpoint{1.877026in}{1.139244in}}%
\pgfpathlineto{\pgfqpoint{1.879141in}{1.146495in}}%
\pgfpathlineto{\pgfqpoint{1.881255in}{1.158571in}}%
\pgfpathlineto{\pgfqpoint{1.883370in}{1.161228in}}%
\pgfpathlineto{\pgfqpoint{1.889714in}{1.174315in}}%
\pgfpathlineto{\pgfqpoint{1.893944in}{1.177750in}}%
\pgfpathlineto{\pgfqpoint{1.896058in}{1.180543in}}%
\pgfpathlineto{\pgfqpoint{1.898173in}{1.186919in}}%
\pgfpathlineto{\pgfqpoint{1.900288in}{1.188750in}}%
\pgfpathlineto{\pgfqpoint{1.904517in}{1.187294in}}%
\pgfpathlineto{\pgfqpoint{1.906632in}{1.193483in}}%
\pgfpathlineto{\pgfqpoint{1.912976in}{1.175660in}}%
\pgfpathlineto{\pgfqpoint{1.915090in}{1.179093in}}%
\pgfpathlineto{\pgfqpoint{1.919320in}{1.173256in}}%
\pgfpathlineto{\pgfqpoint{1.921435in}{1.169600in}}%
\pgfpathlineto{\pgfqpoint{1.923549in}{1.171018in}}%
\pgfpathlineto{\pgfqpoint{1.927779in}{1.153861in}}%
\pgfpathlineto{\pgfqpoint{1.929893in}{1.151868in}}%
\pgfpathlineto{\pgfqpoint{1.932008in}{1.154495in}}%
\pgfpathlineto{\pgfqpoint{1.938352in}{1.155347in}}%
\pgfpathlineto{\pgfqpoint{1.940467in}{1.155230in}}%
\pgfpathlineto{\pgfqpoint{1.942582in}{1.162741in}}%
\pgfpathlineto{\pgfqpoint{1.944696in}{1.157431in}}%
\pgfpathlineto{\pgfqpoint{1.946811in}{1.157223in}}%
\pgfpathlineto{\pgfqpoint{1.948926in}{1.159686in}}%
\pgfpathlineto{\pgfqpoint{1.951040in}{1.159771in}}%
\pgfpathlineto{\pgfqpoint{1.953155in}{1.158303in}}%
\pgfpathlineto{\pgfqpoint{1.955270in}{1.158505in}}%
\pgfpathlineto{\pgfqpoint{1.957384in}{1.156646in}}%
\pgfpathlineto{\pgfqpoint{1.959499in}{1.158758in}}%
\pgfpathlineto{\pgfqpoint{1.961614in}{1.163945in}}%
\pgfpathlineto{\pgfqpoint{1.963728in}{1.163062in}}%
\pgfpathlineto{\pgfqpoint{1.965843in}{1.167048in}}%
\pgfpathlineto{\pgfqpoint{1.967958in}{1.164604in}}%
\pgfpathlineto{\pgfqpoint{1.970073in}{1.158037in}}%
\pgfpathlineto{\pgfqpoint{1.972187in}{1.162495in}}%
\pgfpathlineto{\pgfqpoint{1.974302in}{1.164087in}}%
\pgfpathlineto{\pgfqpoint{1.976417in}{1.169135in}}%
\pgfpathlineto{\pgfqpoint{1.982761in}{1.171025in}}%
\pgfpathlineto{\pgfqpoint{1.984875in}{1.169510in}}%
\pgfpathlineto{\pgfqpoint{1.986990in}{1.166485in}}%
\pgfpathlineto{\pgfqpoint{1.989105in}{1.169976in}}%
\pgfpathlineto{\pgfqpoint{1.991219in}{1.177744in}}%
\pgfpathlineto{\pgfqpoint{1.993334in}{1.170323in}}%
\pgfpathlineto{\pgfqpoint{1.995449in}{1.180766in}}%
\pgfpathlineto{\pgfqpoint{1.997564in}{1.178342in}}%
\pgfpathlineto{\pgfqpoint{2.003908in}{1.161636in}}%
\pgfpathlineto{\pgfqpoint{2.006022in}{1.164556in}}%
\pgfpathlineto{\pgfqpoint{2.008137in}{1.161769in}}%
\pgfpathlineto{\pgfqpoint{2.010252in}{1.165095in}}%
\pgfpathlineto{\pgfqpoint{2.012366in}{1.160056in}}%
\pgfpathlineto{\pgfqpoint{2.014481in}{1.158900in}}%
\pgfpathlineto{\pgfqpoint{2.018710in}{1.166195in}}%
\pgfpathlineto{\pgfqpoint{2.020825in}{1.163988in}}%
\pgfpathlineto{\pgfqpoint{2.025055in}{1.152978in}}%
\pgfpathlineto{\pgfqpoint{2.027169in}{1.155238in}}%
\pgfpathlineto{\pgfqpoint{2.029284in}{1.159914in}}%
\pgfpathlineto{\pgfqpoint{2.033513in}{1.173595in}}%
\pgfpathlineto{\pgfqpoint{2.039857in}{1.178534in}}%
\pgfpathlineto{\pgfqpoint{2.041972in}{1.170295in}}%
\pgfpathlineto{\pgfqpoint{2.044087in}{1.168089in}}%
\pgfpathlineto{\pgfqpoint{2.046202in}{1.164031in}}%
\pgfpathlineto{\pgfqpoint{2.048316in}{1.165074in}}%
\pgfpathlineto{\pgfqpoint{2.050431in}{1.168639in}}%
\pgfpathlineto{\pgfqpoint{2.052546in}{1.168918in}}%
\pgfpathlineto{\pgfqpoint{2.054660in}{1.162520in}}%
\pgfpathlineto{\pgfqpoint{2.056775in}{1.161533in}}%
\pgfpathlineto{\pgfqpoint{2.058890in}{1.161896in}}%
\pgfpathlineto{\pgfqpoint{2.061004in}{1.164942in}}%
\pgfpathlineto{\pgfqpoint{2.063119in}{1.159248in}}%
\pgfpathlineto{\pgfqpoint{2.065234in}{1.157540in}}%
\pgfpathlineto{\pgfqpoint{2.067348in}{1.161336in}}%
\pgfpathlineto{\pgfqpoint{2.069463in}{1.153397in}}%
\pgfpathlineto{\pgfqpoint{2.071578in}{1.153520in}}%
\pgfpathlineto{\pgfqpoint{2.075807in}{1.140617in}}%
\pgfpathlineto{\pgfqpoint{2.077922in}{1.150353in}}%
\pgfpathlineto{\pgfqpoint{2.084266in}{1.150242in}}%
\pgfpathlineto{\pgfqpoint{2.086381in}{1.153310in}}%
\pgfpathlineto{\pgfqpoint{2.088495in}{1.150268in}}%
\pgfpathlineto{\pgfqpoint{2.090610in}{1.143536in}}%
\pgfpathlineto{\pgfqpoint{2.092725in}{1.145478in}}%
\pgfpathlineto{\pgfqpoint{2.094839in}{1.149783in}}%
\pgfpathlineto{\pgfqpoint{2.096954in}{1.144683in}}%
\pgfpathlineto{\pgfqpoint{2.099069in}{1.147723in}}%
\pgfpathlineto{\pgfqpoint{2.101184in}{1.145356in}}%
\pgfpathlineto{\pgfqpoint{2.105413in}{1.147448in}}%
\pgfpathlineto{\pgfqpoint{2.107528in}{1.145181in}}%
\pgfpathlineto{\pgfqpoint{2.109642in}{1.140620in}}%
\pgfpathlineto{\pgfqpoint{2.111757in}{1.143281in}}%
\pgfpathlineto{\pgfqpoint{2.113872in}{1.151513in}}%
\pgfpathlineto{\pgfqpoint{2.115986in}{1.150387in}}%
\pgfpathlineto{\pgfqpoint{2.120216in}{1.155516in}}%
\pgfpathlineto{\pgfqpoint{2.124445in}{1.159203in}}%
\pgfpathlineto{\pgfqpoint{2.126560in}{1.158311in}}%
\pgfpathlineto{\pgfqpoint{2.128675in}{1.160468in}}%
\pgfpathlineto{\pgfqpoint{2.130789in}{1.157661in}}%
\pgfpathlineto{\pgfqpoint{2.132904in}{1.160398in}}%
\pgfpathlineto{\pgfqpoint{2.135019in}{1.160711in}}%
\pgfpathlineto{\pgfqpoint{2.137133in}{1.159836in}}%
\pgfpathlineto{\pgfqpoint{2.139248in}{1.149445in}}%
\pgfpathlineto{\pgfqpoint{2.141363in}{1.155907in}}%
\pgfpathlineto{\pgfqpoint{2.143477in}{1.153526in}}%
\pgfpathlineto{\pgfqpoint{2.145592in}{1.149190in}}%
\pgfpathlineto{\pgfqpoint{2.149822in}{1.146955in}}%
\pgfpathlineto{\pgfqpoint{2.151936in}{1.142178in}}%
\pgfpathlineto{\pgfqpoint{2.154051in}{1.140384in}}%
\pgfpathlineto{\pgfqpoint{2.158280in}{1.141868in}}%
\pgfpathlineto{\pgfqpoint{2.162510in}{1.125336in}}%
\pgfpathlineto{\pgfqpoint{2.166739in}{1.136679in}}%
\pgfpathlineto{\pgfqpoint{2.177313in}{1.108673in}}%
\pgfpathlineto{\pgfqpoint{2.179427in}{1.113827in}}%
\pgfpathlineto{\pgfqpoint{2.183657in}{1.117540in}}%
\pgfpathlineto{\pgfqpoint{2.185771in}{1.116526in}}%
\pgfpathlineto{\pgfqpoint{2.187886in}{1.119274in}}%
\pgfpathlineto{\pgfqpoint{2.190001in}{1.113174in}}%
\pgfpathlineto{\pgfqpoint{2.192115in}{1.116869in}}%
\pgfpathlineto{\pgfqpoint{2.196345in}{1.117659in}}%
\pgfpathlineto{\pgfqpoint{2.198459in}{1.119417in}}%
\pgfpathlineto{\pgfqpoint{2.200574in}{1.117284in}}%
\pgfpathlineto{\pgfqpoint{2.202689in}{1.124129in}}%
\pgfpathlineto{\pgfqpoint{2.204804in}{1.122809in}}%
\pgfpathlineto{\pgfqpoint{2.206918in}{1.125740in}}%
\pgfpathlineto{\pgfqpoint{2.211148in}{1.133854in}}%
\pgfpathlineto{\pgfqpoint{2.213262in}{1.136671in}}%
\pgfpathlineto{\pgfqpoint{2.215377in}{1.126649in}}%
\pgfpathlineto{\pgfqpoint{2.217492in}{1.126031in}}%
\pgfpathlineto{\pgfqpoint{2.219606in}{1.135648in}}%
\pgfpathlineto{\pgfqpoint{2.221721in}{1.129494in}}%
\pgfpathlineto{\pgfqpoint{2.223836in}{1.127419in}}%
\pgfpathlineto{\pgfqpoint{2.225950in}{1.127286in}}%
\pgfpathlineto{\pgfqpoint{2.228065in}{1.132930in}}%
\pgfpathlineto{\pgfqpoint{2.230180in}{1.132718in}}%
\pgfpathlineto{\pgfqpoint{2.234409in}{1.126879in}}%
\pgfpathlineto{\pgfqpoint{2.238639in}{1.131771in}}%
\pgfpathlineto{\pgfqpoint{2.242868in}{1.134551in}}%
\pgfpathlineto{\pgfqpoint{2.244983in}{1.139753in}}%
\pgfpathlineto{\pgfqpoint{2.247097in}{1.133649in}}%
\pgfpathlineto{\pgfqpoint{2.249212in}{1.135244in}}%
\pgfpathlineto{\pgfqpoint{2.251327in}{1.140544in}}%
\pgfpathlineto{\pgfqpoint{2.253441in}{1.138841in}}%
\pgfpathlineto{\pgfqpoint{2.255556in}{1.145925in}}%
\pgfpathlineto{\pgfqpoint{2.257671in}{1.146421in}}%
\pgfpathlineto{\pgfqpoint{2.259786in}{1.151276in}}%
\pgfpathlineto{\pgfqpoint{2.261900in}{1.149791in}}%
\pgfpathlineto{\pgfqpoint{2.264015in}{1.145557in}}%
\pgfpathlineto{\pgfqpoint{2.266130in}{1.144561in}}%
\pgfpathlineto{\pgfqpoint{2.270359in}{1.147777in}}%
\pgfpathlineto{\pgfqpoint{2.272474in}{1.149803in}}%
\pgfpathlineto{\pgfqpoint{2.278818in}{1.161178in}}%
\pgfpathlineto{\pgfqpoint{2.280933in}{1.154056in}}%
\pgfpathlineto{\pgfqpoint{2.283047in}{1.155429in}}%
\pgfpathlineto{\pgfqpoint{2.285162in}{1.163647in}}%
\pgfpathlineto{\pgfqpoint{2.287277in}{1.165192in}}%
\pgfpathlineto{\pgfqpoint{2.289391in}{1.174911in}}%
\pgfpathlineto{\pgfqpoint{2.291506in}{1.174935in}}%
\pgfpathlineto{\pgfqpoint{2.293621in}{1.184012in}}%
\pgfpathlineto{\pgfqpoint{2.295735in}{1.180154in}}%
\pgfpathlineto{\pgfqpoint{2.297850in}{1.183600in}}%
\pgfpathlineto{\pgfqpoint{2.302079in}{1.180751in}}%
\pgfpathlineto{\pgfqpoint{2.304194in}{1.181720in}}%
\pgfpathlineto{\pgfqpoint{2.306309in}{1.180833in}}%
\pgfpathlineto{\pgfqpoint{2.310538in}{1.192208in}}%
\pgfpathlineto{\pgfqpoint{2.312653in}{1.190858in}}%
\pgfpathlineto{\pgfqpoint{2.314768in}{1.187363in}}%
\pgfpathlineto{\pgfqpoint{2.323226in}{1.164985in}}%
\pgfpathlineto{\pgfqpoint{2.325341in}{1.168269in}}%
\pgfpathlineto{\pgfqpoint{2.329570in}{1.163583in}}%
\pgfpathlineto{\pgfqpoint{2.331685in}{1.159428in}}%
\pgfpathlineto{\pgfqpoint{2.333800in}{1.149223in}}%
\pgfpathlineto{\pgfqpoint{2.335915in}{1.148366in}}%
\pgfpathlineto{\pgfqpoint{2.338029in}{1.151535in}}%
\pgfpathlineto{\pgfqpoint{2.340144in}{1.149005in}}%
\pgfpathlineto{\pgfqpoint{2.342259in}{1.149530in}}%
\pgfpathlineto{\pgfqpoint{2.344373in}{1.152742in}}%
\pgfpathlineto{\pgfqpoint{2.346488in}{1.153512in}}%
\pgfpathlineto{\pgfqpoint{2.350717in}{1.161885in}}%
\pgfpathlineto{\pgfqpoint{2.354947in}{1.155802in}}%
\pgfpathlineto{\pgfqpoint{2.357061in}{1.155489in}}%
\pgfpathlineto{\pgfqpoint{2.359176in}{1.160331in}}%
\pgfpathlineto{\pgfqpoint{2.361291in}{1.161750in}}%
\pgfpathlineto{\pgfqpoint{2.363406in}{1.154502in}}%
\pgfpathlineto{\pgfqpoint{2.367635in}{1.162255in}}%
\pgfpathlineto{\pgfqpoint{2.369750in}{1.160168in}}%
\pgfpathlineto{\pgfqpoint{2.371864in}{1.162654in}}%
\pgfpathlineto{\pgfqpoint{2.376094in}{1.162181in}}%
\pgfpathlineto{\pgfqpoint{2.378208in}{1.165834in}}%
\pgfpathlineto{\pgfqpoint{2.380323in}{1.161759in}}%
\pgfpathlineto{\pgfqpoint{2.382438in}{1.166733in}}%
\pgfpathlineto{\pgfqpoint{2.388782in}{1.163375in}}%
\pgfpathlineto{\pgfqpoint{2.390897in}{1.162417in}}%
\pgfpathlineto{\pgfqpoint{2.393011in}{1.159781in}}%
\pgfpathlineto{\pgfqpoint{2.397241in}{1.170809in}}%
\pgfpathlineto{\pgfqpoint{2.399355in}{1.171778in}}%
\pgfpathlineto{\pgfqpoint{2.401470in}{1.162045in}}%
\pgfpathlineto{\pgfqpoint{2.403585in}{1.159537in}}%
\pgfpathlineto{\pgfqpoint{2.405699in}{1.163236in}}%
\pgfpathlineto{\pgfqpoint{2.407814in}{1.159330in}}%
\pgfpathlineto{\pgfqpoint{2.409929in}{1.160340in}}%
\pgfpathlineto{\pgfqpoint{2.412044in}{1.159704in}}%
\pgfpathlineto{\pgfqpoint{2.414158in}{1.156613in}}%
\pgfpathlineto{\pgfqpoint{2.416273in}{1.159442in}}%
\pgfpathlineto{\pgfqpoint{2.418388in}{1.159250in}}%
\pgfpathlineto{\pgfqpoint{2.424732in}{1.164811in}}%
\pgfpathlineto{\pgfqpoint{2.426846in}{1.168764in}}%
\pgfpathlineto{\pgfqpoint{2.431076in}{1.162120in}}%
\pgfpathlineto{\pgfqpoint{2.433190in}{1.162894in}}%
\pgfpathlineto{\pgfqpoint{2.435305in}{1.170038in}}%
\pgfpathlineto{\pgfqpoint{2.437420in}{1.170097in}}%
\pgfpathlineto{\pgfqpoint{2.439535in}{1.167809in}}%
\pgfpathlineto{\pgfqpoint{2.441649in}{1.169645in}}%
\pgfpathlineto{\pgfqpoint{2.447993in}{1.165952in}}%
\pgfpathlineto{\pgfqpoint{2.450108in}{1.159324in}}%
\pgfpathlineto{\pgfqpoint{2.454337in}{1.159395in}}%
\pgfpathlineto{\pgfqpoint{2.456452in}{1.156371in}}%
\pgfpathlineto{\pgfqpoint{2.458567in}{1.151174in}}%
\pgfpathlineto{\pgfqpoint{2.460681in}{1.158580in}}%
\pgfpathlineto{\pgfqpoint{2.462796in}{1.159008in}}%
\pgfpathlineto{\pgfqpoint{2.464911in}{1.152044in}}%
\pgfpathlineto{\pgfqpoint{2.467026in}{1.149849in}}%
\pgfpathlineto{\pgfqpoint{2.471255in}{1.162436in}}%
\pgfpathlineto{\pgfqpoint{2.473370in}{1.160940in}}%
\pgfpathlineto{\pgfqpoint{2.475484in}{1.157919in}}%
\pgfpathlineto{\pgfqpoint{2.477599in}{1.161033in}}%
\pgfpathlineto{\pgfqpoint{2.479714in}{1.156390in}}%
\pgfpathlineto{\pgfqpoint{2.481828in}{1.161527in}}%
\pgfpathlineto{\pgfqpoint{2.483943in}{1.161421in}}%
\pgfpathlineto{\pgfqpoint{2.486058in}{1.157972in}}%
\pgfpathlineto{\pgfqpoint{2.488172in}{1.158727in}}%
\pgfpathlineto{\pgfqpoint{2.492402in}{1.171283in}}%
\pgfpathlineto{\pgfqpoint{2.494517in}{1.172441in}}%
\pgfpathlineto{\pgfqpoint{2.502975in}{1.168347in}}%
\pgfpathlineto{\pgfqpoint{2.505090in}{1.170692in}}%
\pgfpathlineto{\pgfqpoint{2.507205in}{1.169901in}}%
\pgfpathlineto{\pgfqpoint{2.509319in}{1.174110in}}%
\pgfpathlineto{\pgfqpoint{2.511434in}{1.171121in}}%
\pgfpathlineto{\pgfqpoint{2.513549in}{1.165579in}}%
\pgfpathlineto{\pgfqpoint{2.517778in}{1.169018in}}%
\pgfpathlineto{\pgfqpoint{2.519893in}{1.175936in}}%
\pgfpathlineto{\pgfqpoint{2.522008in}{1.174881in}}%
\pgfpathlineto{\pgfqpoint{2.524122in}{1.172500in}}%
\pgfpathlineto{\pgfqpoint{2.526237in}{1.173928in}}%
\pgfpathlineto{\pgfqpoint{2.528352in}{1.166470in}}%
\pgfpathlineto{\pgfqpoint{2.530466in}{1.171196in}}%
\pgfpathlineto{\pgfqpoint{2.532581in}{1.168060in}}%
\pgfpathlineto{\pgfqpoint{2.534696in}{1.173666in}}%
\pgfpathlineto{\pgfqpoint{2.538925in}{1.174857in}}%
\pgfpathlineto{\pgfqpoint{2.541040in}{1.177783in}}%
\pgfpathlineto{\pgfqpoint{2.543155in}{1.176340in}}%
\pgfpathlineto{\pgfqpoint{2.545269in}{1.177150in}}%
\pgfpathlineto{\pgfqpoint{2.547384in}{1.179415in}}%
\pgfpathlineto{\pgfqpoint{2.549499in}{1.184557in}}%
\pgfpathlineto{\pgfqpoint{2.551613in}{1.184875in}}%
\pgfpathlineto{\pgfqpoint{2.555843in}{1.180577in}}%
\pgfpathlineto{\pgfqpoint{2.557957in}{1.175477in}}%
\pgfpathlineto{\pgfqpoint{2.560072in}{1.175352in}}%
\pgfpathlineto{\pgfqpoint{2.562187in}{1.177189in}}%
\pgfpathlineto{\pgfqpoint{2.564301in}{1.170412in}}%
\pgfpathlineto{\pgfqpoint{2.568531in}{1.172162in}}%
\pgfpathlineto{\pgfqpoint{2.570646in}{1.172286in}}%
\pgfpathlineto{\pgfqpoint{2.572760in}{1.176858in}}%
\pgfpathlineto{\pgfqpoint{2.576990in}{1.177172in}}%
\pgfpathlineto{\pgfqpoint{2.579104in}{1.178616in}}%
\pgfpathlineto{\pgfqpoint{2.585448in}{1.200286in}}%
\pgfpathlineto{\pgfqpoint{2.587563in}{1.199105in}}%
\pgfpathlineto{\pgfqpoint{2.589678in}{1.195669in}}%
\pgfpathlineto{\pgfqpoint{2.591792in}{1.186043in}}%
\pgfpathlineto{\pgfqpoint{2.596022in}{1.196054in}}%
\pgfpathlineto{\pgfqpoint{2.600251in}{1.191242in}}%
\pgfpathlineto{\pgfqpoint{2.602366in}{1.198839in}}%
\pgfpathlineto{\pgfqpoint{2.604481in}{1.202051in}}%
\pgfpathlineto{\pgfqpoint{2.606595in}{1.202971in}}%
\pgfpathlineto{\pgfqpoint{2.608710in}{1.200748in}}%
\pgfpathlineto{\pgfqpoint{2.610825in}{1.202329in}}%
\pgfpathlineto{\pgfqpoint{2.612939in}{1.198302in}}%
\pgfpathlineto{\pgfqpoint{2.615054in}{1.190111in}}%
\pgfpathlineto{\pgfqpoint{2.619284in}{1.196992in}}%
\pgfpathlineto{\pgfqpoint{2.621398in}{1.192315in}}%
\pgfpathlineto{\pgfqpoint{2.623513in}{1.200195in}}%
\pgfpathlineto{\pgfqpoint{2.625628in}{1.201441in}}%
\pgfpathlineto{\pgfqpoint{2.627742in}{1.196066in}}%
\pgfpathlineto{\pgfqpoint{2.629857in}{1.197109in}}%
\pgfpathlineto{\pgfqpoint{2.636201in}{1.204832in}}%
\pgfpathlineto{\pgfqpoint{2.638316in}{1.206676in}}%
\pgfpathlineto{\pgfqpoint{2.640430in}{1.203612in}}%
\pgfpathlineto{\pgfqpoint{2.642545in}{1.210959in}}%
\pgfpathlineto{\pgfqpoint{2.644660in}{1.207926in}}%
\pgfpathlineto{\pgfqpoint{2.648889in}{1.209876in}}%
\pgfpathlineto{\pgfqpoint{2.651004in}{1.223648in}}%
\pgfpathlineto{\pgfqpoint{2.653119in}{1.221610in}}%
\pgfpathlineto{\pgfqpoint{2.655233in}{1.221707in}}%
\pgfpathlineto{\pgfqpoint{2.657348in}{1.217270in}}%
\pgfpathlineto{\pgfqpoint{2.661577in}{1.227679in}}%
\pgfpathlineto{\pgfqpoint{2.663692in}{1.226643in}}%
\pgfpathlineto{\pgfqpoint{2.667921in}{1.230197in}}%
\pgfpathlineto{\pgfqpoint{2.670036in}{1.231264in}}%
\pgfpathlineto{\pgfqpoint{2.672151in}{1.224774in}}%
\pgfpathlineto{\pgfqpoint{2.674266in}{1.222686in}}%
\pgfpathlineto{\pgfqpoint{2.676380in}{1.223775in}}%
\pgfpathlineto{\pgfqpoint{2.678495in}{1.226577in}}%
\pgfpathlineto{\pgfqpoint{2.680610in}{1.221645in}}%
\pgfpathlineto{\pgfqpoint{2.682724in}{1.224869in}}%
\pgfpathlineto{\pgfqpoint{2.684839in}{1.219832in}}%
\pgfpathlineto{\pgfqpoint{2.686954in}{1.217695in}}%
\pgfpathlineto{\pgfqpoint{2.689068in}{1.217227in}}%
\pgfpathlineto{\pgfqpoint{2.693298in}{1.224105in}}%
\pgfpathlineto{\pgfqpoint{2.695412in}{1.221329in}}%
\pgfpathlineto{\pgfqpoint{2.699642in}{1.212431in}}%
\pgfpathlineto{\pgfqpoint{2.703871in}{1.206744in}}%
\pgfpathlineto{\pgfqpoint{2.708101in}{1.208962in}}%
\pgfpathlineto{\pgfqpoint{2.712330in}{1.218590in}}%
\pgfpathlineto{\pgfqpoint{2.714445in}{1.217236in}}%
\pgfpathlineto{\pgfqpoint{2.716559in}{1.213632in}}%
\pgfpathlineto{\pgfqpoint{2.718674in}{1.213013in}}%
\pgfpathlineto{\pgfqpoint{2.720789in}{1.214101in}}%
\pgfpathlineto{\pgfqpoint{2.727133in}{1.195498in}}%
\pgfpathlineto{\pgfqpoint{2.729248in}{1.200990in}}%
\pgfpathlineto{\pgfqpoint{2.731362in}{1.200152in}}%
\pgfpathlineto{\pgfqpoint{2.733477in}{1.196139in}}%
\pgfpathlineto{\pgfqpoint{2.735592in}{1.195359in}}%
\pgfpathlineto{\pgfqpoint{2.737706in}{1.201996in}}%
\pgfpathlineto{\pgfqpoint{2.744050in}{1.195527in}}%
\pgfpathlineto{\pgfqpoint{2.746165in}{1.192244in}}%
\pgfpathlineto{\pgfqpoint{2.748280in}{1.195491in}}%
\pgfpathlineto{\pgfqpoint{2.750395in}{1.190448in}}%
\pgfpathlineto{\pgfqpoint{2.754624in}{1.187643in}}%
\pgfpathlineto{\pgfqpoint{2.756739in}{1.183633in}}%
\pgfpathlineto{\pgfqpoint{2.758853in}{1.184435in}}%
\pgfpathlineto{\pgfqpoint{2.760968in}{1.188185in}}%
\pgfpathlineto{\pgfqpoint{2.763083in}{1.195086in}}%
\pgfpathlineto{\pgfqpoint{2.765197in}{1.196464in}}%
\pgfpathlineto{\pgfqpoint{2.769427in}{1.207057in}}%
\pgfpathlineto{\pgfqpoint{2.773656in}{1.201121in}}%
\pgfpathlineto{\pgfqpoint{2.786344in}{1.226962in}}%
\pgfpathlineto{\pgfqpoint{2.788459in}{1.229835in}}%
\pgfpathlineto{\pgfqpoint{2.790574in}{1.228074in}}%
\pgfpathlineto{\pgfqpoint{2.792688in}{1.223728in}}%
\pgfpathlineto{\pgfqpoint{2.794803in}{1.225943in}}%
\pgfpathlineto{\pgfqpoint{2.796918in}{1.225793in}}%
\pgfpathlineto{\pgfqpoint{2.801147in}{1.230019in}}%
\pgfpathlineto{\pgfqpoint{2.809606in}{1.225662in}}%
\pgfpathlineto{\pgfqpoint{2.811721in}{1.220509in}}%
\pgfpathlineto{\pgfqpoint{2.815950in}{1.215155in}}%
\pgfpathlineto{\pgfqpoint{2.818065in}{1.205339in}}%
\pgfpathlineto{\pgfqpoint{2.822294in}{1.204263in}}%
\pgfpathlineto{\pgfqpoint{2.824409in}{1.206643in}}%
\pgfpathlineto{\pgfqpoint{2.828638in}{1.201151in}}%
\pgfpathlineto{\pgfqpoint{2.832868in}{1.199148in}}%
\pgfpathlineto{\pgfqpoint{2.837097in}{1.182668in}}%
\pgfpathlineto{\pgfqpoint{2.839212in}{1.181934in}}%
\pgfpathlineto{\pgfqpoint{2.841326in}{1.185472in}}%
\pgfpathlineto{\pgfqpoint{2.843441in}{1.180905in}}%
\pgfpathlineto{\pgfqpoint{2.845556in}{1.182413in}}%
\pgfpathlineto{\pgfqpoint{2.847670in}{1.186122in}}%
\pgfpathlineto{\pgfqpoint{2.851900in}{1.188352in}}%
\pgfpathlineto{\pgfqpoint{2.856129in}{1.193414in}}%
\pgfpathlineto{\pgfqpoint{2.860359in}{1.181431in}}%
\pgfpathlineto{\pgfqpoint{2.866703in}{1.186668in}}%
\pgfpathlineto{\pgfqpoint{2.868817in}{1.177851in}}%
\pgfpathlineto{\pgfqpoint{2.870932in}{1.178920in}}%
\pgfpathlineto{\pgfqpoint{2.873047in}{1.170450in}}%
\pgfpathlineto{\pgfqpoint{2.875161in}{1.172080in}}%
\pgfpathlineto{\pgfqpoint{2.881506in}{1.160108in}}%
\pgfpathlineto{\pgfqpoint{2.885735in}{1.165101in}}%
\pgfpathlineto{\pgfqpoint{2.887850in}{1.165362in}}%
\pgfpathlineto{\pgfqpoint{2.889964in}{1.158917in}}%
\pgfpathlineto{\pgfqpoint{2.894194in}{1.161183in}}%
\pgfpathlineto{\pgfqpoint{2.896308in}{1.157154in}}%
\pgfpathlineto{\pgfqpoint{2.898423in}{1.157442in}}%
\pgfpathlineto{\pgfqpoint{2.902652in}{1.145483in}}%
\pgfpathlineto{\pgfqpoint{2.904767in}{1.152172in}}%
\pgfpathlineto{\pgfqpoint{2.906882in}{1.151230in}}%
\pgfpathlineto{\pgfqpoint{2.917455in}{1.168244in}}%
\pgfpathlineto{\pgfqpoint{2.919570in}{1.169761in}}%
\pgfpathlineto{\pgfqpoint{2.921685in}{1.169824in}}%
\pgfpathlineto{\pgfqpoint{2.923799in}{1.159678in}}%
\pgfpathlineto{\pgfqpoint{2.925914in}{1.159609in}}%
\pgfpathlineto{\pgfqpoint{2.928029in}{1.161641in}}%
\pgfpathlineto{\pgfqpoint{2.930143in}{1.166607in}}%
\pgfpathlineto{\pgfqpoint{2.934373in}{1.163962in}}%
\pgfpathlineto{\pgfqpoint{2.938602in}{1.166400in}}%
\pgfpathlineto{\pgfqpoint{2.942832in}{1.171163in}}%
\pgfpathlineto{\pgfqpoint{2.944946in}{1.165980in}}%
\pgfpathlineto{\pgfqpoint{2.951290in}{1.170106in}}%
\pgfpathlineto{\pgfqpoint{2.955520in}{1.178829in}}%
\pgfpathlineto{\pgfqpoint{2.957634in}{1.186836in}}%
\pgfpathlineto{\pgfqpoint{2.959749in}{1.183666in}}%
\pgfpathlineto{\pgfqpoint{2.961864in}{1.186974in}}%
\pgfpathlineto{\pgfqpoint{2.963979in}{1.187302in}}%
\pgfpathlineto{\pgfqpoint{2.966093in}{1.182637in}}%
\pgfpathlineto{\pgfqpoint{2.968208in}{1.187384in}}%
\pgfpathlineto{\pgfqpoint{2.970323in}{1.175795in}}%
\pgfpathlineto{\pgfqpoint{2.972437in}{1.171426in}}%
\pgfpathlineto{\pgfqpoint{2.974552in}{1.173538in}}%
\pgfpathlineto{\pgfqpoint{2.976667in}{1.167423in}}%
\pgfpathlineto{\pgfqpoint{2.978781in}{1.166489in}}%
\pgfpathlineto{\pgfqpoint{2.980896in}{1.168453in}}%
\pgfpathlineto{\pgfqpoint{2.983011in}{1.172618in}}%
\pgfpathlineto{\pgfqpoint{2.985126in}{1.173555in}}%
\pgfpathlineto{\pgfqpoint{2.987240in}{1.177043in}}%
\pgfpathlineto{\pgfqpoint{2.989355in}{1.176493in}}%
\pgfpathlineto{\pgfqpoint{2.991470in}{1.181199in}}%
\pgfpathlineto{\pgfqpoint{2.993584in}{1.173783in}}%
\pgfpathlineto{\pgfqpoint{2.995699in}{1.175091in}}%
\pgfpathlineto{\pgfqpoint{2.997814in}{1.173107in}}%
\pgfpathlineto{\pgfqpoint{2.999928in}{1.168183in}}%
\pgfpathlineto{\pgfqpoint{3.002043in}{1.171836in}}%
\pgfpathlineto{\pgfqpoint{3.004158in}{1.170921in}}%
\pgfpathlineto{\pgfqpoint{3.006272in}{1.162067in}}%
\pgfpathlineto{\pgfqpoint{3.008387in}{1.162361in}}%
\pgfpathlineto{\pgfqpoint{3.010502in}{1.158914in}}%
\pgfpathlineto{\pgfqpoint{3.016846in}{1.175884in}}%
\pgfpathlineto{\pgfqpoint{3.018961in}{1.176666in}}%
\pgfpathlineto{\pgfqpoint{3.021075in}{1.175529in}}%
\pgfpathlineto{\pgfqpoint{3.025305in}{1.176196in}}%
\pgfpathlineto{\pgfqpoint{3.027419in}{1.175750in}}%
\pgfpathlineto{\pgfqpoint{3.031649in}{1.182168in}}%
\pgfpathlineto{\pgfqpoint{3.035878in}{1.186728in}}%
\pgfpathlineto{\pgfqpoint{3.037993in}{1.190395in}}%
\pgfpathlineto{\pgfqpoint{3.040108in}{1.186550in}}%
\pgfpathlineto{\pgfqpoint{3.042222in}{1.192614in}}%
\pgfpathlineto{\pgfqpoint{3.044337in}{1.192468in}}%
\pgfpathlineto{\pgfqpoint{3.046452in}{1.186048in}}%
\pgfpathlineto{\pgfqpoint{3.048566in}{1.186746in}}%
\pgfpathlineto{\pgfqpoint{3.050681in}{1.189197in}}%
\pgfpathlineto{\pgfqpoint{3.052796in}{1.188394in}}%
\pgfpathlineto{\pgfqpoint{3.054910in}{1.192374in}}%
\pgfpathlineto{\pgfqpoint{3.057025in}{1.191438in}}%
\pgfpathlineto{\pgfqpoint{3.061254in}{1.193945in}}%
\pgfpathlineto{\pgfqpoint{3.069713in}{1.190958in}}%
\pgfpathlineto{\pgfqpoint{3.073943in}{1.196727in}}%
\pgfpathlineto{\pgfqpoint{3.076057in}{1.197648in}}%
\pgfpathlineto{\pgfqpoint{3.078172in}{1.195900in}}%
\pgfpathlineto{\pgfqpoint{3.080287in}{1.196001in}}%
\pgfpathlineto{\pgfqpoint{3.082401in}{1.199917in}}%
\pgfpathlineto{\pgfqpoint{3.084516in}{1.199078in}}%
\pgfpathlineto{\pgfqpoint{3.086631in}{1.194074in}}%
\pgfpathlineto{\pgfqpoint{3.088746in}{1.193836in}}%
\pgfpathlineto{\pgfqpoint{3.092975in}{1.178826in}}%
\pgfpathlineto{\pgfqpoint{3.095090in}{1.180986in}}%
\pgfpathlineto{\pgfqpoint{3.097204in}{1.176689in}}%
\pgfpathlineto{\pgfqpoint{3.101434in}{1.178740in}}%
\pgfpathlineto{\pgfqpoint{3.103548in}{1.177163in}}%
\pgfpathlineto{\pgfqpoint{3.105663in}{1.173845in}}%
\pgfpathlineto{\pgfqpoint{3.107778in}{1.174676in}}%
\pgfpathlineto{\pgfqpoint{3.109892in}{1.182685in}}%
\pgfpathlineto{\pgfqpoint{3.112007in}{1.185130in}}%
\pgfpathlineto{\pgfqpoint{3.114122in}{1.194141in}}%
\pgfpathlineto{\pgfqpoint{3.116237in}{1.188939in}}%
\pgfpathlineto{\pgfqpoint{3.118351in}{1.187365in}}%
\pgfpathlineto{\pgfqpoint{3.120466in}{1.188566in}}%
\pgfpathlineto{\pgfqpoint{3.124695in}{1.195392in}}%
\pgfpathlineto{\pgfqpoint{3.126810in}{1.195955in}}%
\pgfpathlineto{\pgfqpoint{3.131039in}{1.186477in}}%
\pgfpathlineto{\pgfqpoint{3.133154in}{1.185647in}}%
\pgfpathlineto{\pgfqpoint{3.135269in}{1.190547in}}%
\pgfpathlineto{\pgfqpoint{3.139498in}{1.174837in}}%
\pgfpathlineto{\pgfqpoint{3.141613in}{1.179532in}}%
\pgfpathlineto{\pgfqpoint{3.143728in}{1.175869in}}%
\pgfpathlineto{\pgfqpoint{3.147957in}{1.183274in}}%
\pgfpathlineto{\pgfqpoint{3.152186in}{1.188464in}}%
\pgfpathlineto{\pgfqpoint{3.154301in}{1.184526in}}%
\pgfpathlineto{\pgfqpoint{3.156416in}{1.186900in}}%
\pgfpathlineto{\pgfqpoint{3.160645in}{1.170558in}}%
\pgfpathlineto{\pgfqpoint{3.162760in}{1.173277in}}%
\pgfpathlineto{\pgfqpoint{3.166989in}{1.173480in}}%
\pgfpathlineto{\pgfqpoint{3.169104in}{1.175517in}}%
\pgfpathlineto{\pgfqpoint{3.171219in}{1.172710in}}%
\pgfpathlineto{\pgfqpoint{3.177563in}{1.169949in}}%
\pgfpathlineto{\pgfqpoint{3.179677in}{1.168571in}}%
\pgfpathlineto{\pgfqpoint{3.186021in}{1.169614in}}%
\pgfpathlineto{\pgfqpoint{3.188136in}{1.173075in}}%
\pgfpathlineto{\pgfqpoint{3.190251in}{1.170879in}}%
\pgfpathlineto{\pgfqpoint{3.192366in}{1.171245in}}%
\pgfpathlineto{\pgfqpoint{3.194480in}{1.173671in}}%
\pgfpathlineto{\pgfqpoint{3.196595in}{1.170788in}}%
\pgfpathlineto{\pgfqpoint{3.200824in}{1.163300in}}%
\pgfpathlineto{\pgfqpoint{3.202939in}{1.160713in}}%
\pgfpathlineto{\pgfqpoint{3.207168in}{1.165805in}}%
\pgfpathlineto{\pgfqpoint{3.209283in}{1.163200in}}%
\pgfpathlineto{\pgfqpoint{3.211398in}{1.154850in}}%
\pgfpathlineto{\pgfqpoint{3.213512in}{1.155682in}}%
\pgfpathlineto{\pgfqpoint{3.215627in}{1.153744in}}%
\pgfpathlineto{\pgfqpoint{3.217742in}{1.155568in}}%
\pgfpathlineto{\pgfqpoint{3.219857in}{1.152612in}}%
\pgfpathlineto{\pgfqpoint{3.221971in}{1.158791in}}%
\pgfpathlineto{\pgfqpoint{3.224086in}{1.159161in}}%
\pgfpathlineto{\pgfqpoint{3.226201in}{1.164155in}}%
\pgfpathlineto{\pgfqpoint{3.228315in}{1.161830in}}%
\pgfpathlineto{\pgfqpoint{3.232545in}{1.174687in}}%
\pgfpathlineto{\pgfqpoint{3.234659in}{1.175357in}}%
\pgfpathlineto{\pgfqpoint{3.236774in}{1.168258in}}%
\pgfpathlineto{\pgfqpoint{3.241003in}{1.164404in}}%
\pgfpathlineto{\pgfqpoint{3.243118in}{1.157028in}}%
\pgfpathlineto{\pgfqpoint{3.245233in}{1.155757in}}%
\pgfpathlineto{\pgfqpoint{3.247348in}{1.150544in}}%
\pgfpathlineto{\pgfqpoint{3.249462in}{1.150868in}}%
\pgfpathlineto{\pgfqpoint{3.251577in}{1.152813in}}%
\pgfpathlineto{\pgfqpoint{3.253692in}{1.150264in}}%
\pgfpathlineto{\pgfqpoint{3.257921in}{1.156151in}}%
\pgfpathlineto{\pgfqpoint{3.264265in}{1.155989in}}%
\pgfpathlineto{\pgfqpoint{3.266380in}{1.153821in}}%
\pgfpathlineto{\pgfqpoint{3.268494in}{1.154892in}}%
\pgfpathlineto{\pgfqpoint{3.274839in}{1.147255in}}%
\pgfpathlineto{\pgfqpoint{3.276953in}{1.151301in}}%
\pgfpathlineto{\pgfqpoint{3.279068in}{1.151618in}}%
\pgfpathlineto{\pgfqpoint{3.281183in}{1.153327in}}%
\pgfpathlineto{\pgfqpoint{3.283297in}{1.157056in}}%
\pgfpathlineto{\pgfqpoint{3.291756in}{1.147337in}}%
\pgfpathlineto{\pgfqpoint{3.293871in}{1.154628in}}%
\pgfpathlineto{\pgfqpoint{3.295985in}{1.155273in}}%
\pgfpathlineto{\pgfqpoint{3.298100in}{1.157487in}}%
\pgfpathlineto{\pgfqpoint{3.300215in}{1.151356in}}%
\pgfpathlineto{\pgfqpoint{3.302330in}{1.151827in}}%
\pgfpathlineto{\pgfqpoint{3.304444in}{1.162632in}}%
\pgfpathlineto{\pgfqpoint{3.308674in}{1.155181in}}%
\pgfpathlineto{\pgfqpoint{3.312903in}{1.157028in}}%
\pgfpathlineto{\pgfqpoint{3.315018in}{1.160364in}}%
\pgfpathlineto{\pgfqpoint{3.319247in}{1.161212in}}%
\pgfpathlineto{\pgfqpoint{3.321362in}{1.156178in}}%
\pgfpathlineto{\pgfqpoint{3.323477in}{1.155793in}}%
\pgfpathlineto{\pgfqpoint{3.325591in}{1.157967in}}%
\pgfpathlineto{\pgfqpoint{3.329821in}{1.154007in}}%
\pgfpathlineto{\pgfqpoint{3.334050in}{1.155642in}}%
\pgfpathlineto{\pgfqpoint{3.336165in}{1.151519in}}%
\pgfpathlineto{\pgfqpoint{3.338279in}{1.152655in}}%
\pgfpathlineto{\pgfqpoint{3.340394in}{1.161483in}}%
\pgfpathlineto{\pgfqpoint{3.342509in}{1.160131in}}%
\pgfpathlineto{\pgfqpoint{3.344623in}{1.162060in}}%
\pgfpathlineto{\pgfqpoint{3.346738in}{1.158224in}}%
\pgfpathlineto{\pgfqpoint{3.348853in}{1.163505in}}%
\pgfpathlineto{\pgfqpoint{3.350968in}{1.157727in}}%
\pgfpathlineto{\pgfqpoint{3.353082in}{1.161985in}}%
\pgfpathlineto{\pgfqpoint{3.355197in}{1.152052in}}%
\pgfpathlineto{\pgfqpoint{3.357312in}{1.149453in}}%
\pgfpathlineto{\pgfqpoint{3.359426in}{1.153534in}}%
\pgfpathlineto{\pgfqpoint{3.361541in}{1.152320in}}%
\pgfpathlineto{\pgfqpoint{3.363656in}{1.153076in}}%
\pgfpathlineto{\pgfqpoint{3.365770in}{1.146096in}}%
\pgfpathlineto{\pgfqpoint{3.367885in}{1.151712in}}%
\pgfpathlineto{\pgfqpoint{3.372114in}{1.143060in}}%
\pgfpathlineto{\pgfqpoint{3.376344in}{1.130991in}}%
\pgfpathlineto{\pgfqpoint{3.378459in}{1.127934in}}%
\pgfpathlineto{\pgfqpoint{3.380573in}{1.128560in}}%
\pgfpathlineto{\pgfqpoint{3.382688in}{1.124623in}}%
\pgfpathlineto{\pgfqpoint{3.384803in}{1.125011in}}%
\pgfpathlineto{\pgfqpoint{3.389032in}{1.129765in}}%
\pgfpathlineto{\pgfqpoint{3.391147in}{1.124006in}}%
\pgfpathlineto{\pgfqpoint{3.393261in}{1.111127in}}%
\pgfpathlineto{\pgfqpoint{3.395376in}{1.112277in}}%
\pgfpathlineto{\pgfqpoint{3.397491in}{1.108920in}}%
\pgfpathlineto{\pgfqpoint{3.399605in}{1.109149in}}%
\pgfpathlineto{\pgfqpoint{3.403835in}{1.117364in}}%
\pgfpathlineto{\pgfqpoint{3.405950in}{1.119071in}}%
\pgfpathlineto{\pgfqpoint{3.408064in}{1.118257in}}%
\pgfpathlineto{\pgfqpoint{3.410179in}{1.123585in}}%
\pgfpathlineto{\pgfqpoint{3.412294in}{1.119540in}}%
\pgfpathlineto{\pgfqpoint{3.414408in}{1.125290in}}%
\pgfpathlineto{\pgfqpoint{3.416523in}{1.125315in}}%
\pgfpathlineto{\pgfqpoint{3.418638in}{1.121276in}}%
\pgfpathlineto{\pgfqpoint{3.420752in}{1.121354in}}%
\pgfpathlineto{\pgfqpoint{3.424982in}{1.117291in}}%
\pgfpathlineto{\pgfqpoint{3.431326in}{1.116869in}}%
\pgfpathlineto{\pgfqpoint{3.433441in}{1.118517in}}%
\pgfpathlineto{\pgfqpoint{3.437670in}{1.111977in}}%
\pgfpathlineto{\pgfqpoint{3.439785in}{1.111975in}}%
\pgfpathlineto{\pgfqpoint{3.441899in}{1.108178in}}%
\pgfpathlineto{\pgfqpoint{3.444014in}{1.109063in}}%
\pgfpathlineto{\pgfqpoint{3.446129in}{1.106330in}}%
\pgfpathlineto{\pgfqpoint{3.450358in}{1.108452in}}%
\pgfpathlineto{\pgfqpoint{3.452473in}{1.111540in}}%
\pgfpathlineto{\pgfqpoint{3.454588in}{1.104110in}}%
\pgfpathlineto{\pgfqpoint{3.456702in}{1.104062in}}%
\pgfpathlineto{\pgfqpoint{3.458817in}{1.100621in}}%
\pgfpathlineto{\pgfqpoint{3.460932in}{1.105177in}}%
\pgfpathlineto{\pgfqpoint{3.463046in}{1.104465in}}%
\pgfpathlineto{\pgfqpoint{3.465161in}{1.106487in}}%
\pgfpathlineto{\pgfqpoint{3.467276in}{1.099988in}}%
\pgfpathlineto{\pgfqpoint{3.471505in}{1.105323in}}%
\pgfpathlineto{\pgfqpoint{3.473620in}{1.103729in}}%
\pgfpathlineto{\pgfqpoint{3.475734in}{1.099028in}}%
\pgfpathlineto{\pgfqpoint{3.477849in}{1.099975in}}%
\pgfpathlineto{\pgfqpoint{3.484193in}{1.084586in}}%
\pgfpathlineto{\pgfqpoint{3.488423in}{1.085118in}}%
\pgfpathlineto{\pgfqpoint{3.490537in}{1.079167in}}%
\pgfpathlineto{\pgfqpoint{3.492652in}{1.076909in}}%
\pgfpathlineto{\pgfqpoint{3.494767in}{1.085880in}}%
\pgfpathlineto{\pgfqpoint{3.496881in}{1.087184in}}%
\pgfpathlineto{\pgfqpoint{3.498996in}{1.091686in}}%
\pgfpathlineto{\pgfqpoint{3.501111in}{1.084919in}}%
\pgfpathlineto{\pgfqpoint{3.503225in}{1.084489in}}%
\pgfpathlineto{\pgfqpoint{3.505340in}{1.081452in}}%
\pgfpathlineto{\pgfqpoint{3.509570in}{1.093009in}}%
\pgfpathlineto{\pgfqpoint{3.511684in}{1.089297in}}%
\pgfpathlineto{\pgfqpoint{3.513799in}{1.091254in}}%
\pgfpathlineto{\pgfqpoint{3.515914in}{1.090544in}}%
\pgfpathlineto{\pgfqpoint{3.518028in}{1.092131in}}%
\pgfpathlineto{\pgfqpoint{3.520143in}{1.097191in}}%
\pgfpathlineto{\pgfqpoint{3.522258in}{1.098611in}}%
\pgfpathlineto{\pgfqpoint{3.524372in}{1.104571in}}%
\pgfpathlineto{\pgfqpoint{3.528602in}{1.095102in}}%
\pgfpathlineto{\pgfqpoint{3.530716in}{1.104442in}}%
\pgfpathlineto{\pgfqpoint{3.532831in}{1.104259in}}%
\pgfpathlineto{\pgfqpoint{3.537061in}{1.113488in}}%
\pgfpathlineto{\pgfqpoint{3.539175in}{1.114350in}}%
\pgfpathlineto{\pgfqpoint{3.541290in}{1.120427in}}%
\pgfpathlineto{\pgfqpoint{3.543405in}{1.115873in}}%
\pgfpathlineto{\pgfqpoint{3.545519in}{1.120029in}}%
\pgfpathlineto{\pgfqpoint{3.547634in}{1.121065in}}%
\pgfpathlineto{\pgfqpoint{3.549749in}{1.118448in}}%
\pgfpathlineto{\pgfqpoint{3.553978in}{1.127705in}}%
\pgfpathlineto{\pgfqpoint{3.556093in}{1.122922in}}%
\pgfpathlineto{\pgfqpoint{3.558208in}{1.124288in}}%
\pgfpathlineto{\pgfqpoint{3.560322in}{1.122686in}}%
\pgfpathlineto{\pgfqpoint{3.562437in}{1.127482in}}%
\pgfpathlineto{\pgfqpoint{3.564552in}{1.127462in}}%
\pgfpathlineto{\pgfqpoint{3.566666in}{1.131019in}}%
\pgfpathlineto{\pgfqpoint{3.570896in}{1.126109in}}%
\pgfpathlineto{\pgfqpoint{3.573010in}{1.128775in}}%
\pgfpathlineto{\pgfqpoint{3.577240in}{1.131488in}}%
\pgfpathlineto{\pgfqpoint{3.581469in}{1.125031in}}%
\pgfpathlineto{\pgfqpoint{3.587813in}{1.106014in}}%
\pgfpathlineto{\pgfqpoint{3.589928in}{1.112483in}}%
\pgfpathlineto{\pgfqpoint{3.594157in}{1.106293in}}%
\pgfpathlineto{\pgfqpoint{3.596272in}{1.109447in}}%
\pgfpathlineto{\pgfqpoint{3.598387in}{1.106929in}}%
\pgfpathlineto{\pgfqpoint{3.600501in}{1.107018in}}%
\pgfpathlineto{\pgfqpoint{3.602616in}{1.108584in}}%
\pgfpathlineto{\pgfqpoint{3.608960in}{1.099911in}}%
\pgfpathlineto{\pgfqpoint{3.611075in}{1.103412in}}%
\pgfpathlineto{\pgfqpoint{3.613190in}{1.098156in}}%
\pgfpathlineto{\pgfqpoint{3.615304in}{1.096055in}}%
\pgfpathlineto{\pgfqpoint{3.617419in}{1.099028in}}%
\pgfpathlineto{\pgfqpoint{3.619534in}{1.098730in}}%
\pgfpathlineto{\pgfqpoint{3.621648in}{1.091785in}}%
\pgfpathlineto{\pgfqpoint{3.623763in}{1.091109in}}%
\pgfpathlineto{\pgfqpoint{3.627992in}{1.082198in}}%
\pgfpathlineto{\pgfqpoint{3.632222in}{1.083049in}}%
\pgfpathlineto{\pgfqpoint{3.634336in}{1.077202in}}%
\pgfpathlineto{\pgfqpoint{3.636451in}{1.078766in}}%
\pgfpathlineto{\pgfqpoint{3.640681in}{1.087734in}}%
\pgfpathlineto{\pgfqpoint{3.644910in}{1.082597in}}%
\pgfpathlineto{\pgfqpoint{3.647025in}{1.097096in}}%
\pgfpathlineto{\pgfqpoint{3.649139in}{1.098531in}}%
\pgfpathlineto{\pgfqpoint{3.653369in}{1.096282in}}%
\pgfpathlineto{\pgfqpoint{3.657598in}{1.100801in}}%
\pgfpathlineto{\pgfqpoint{3.659713in}{1.093758in}}%
\pgfpathlineto{\pgfqpoint{3.661828in}{1.098515in}}%
\pgfpathlineto{\pgfqpoint{3.663942in}{1.093351in}}%
\pgfpathlineto{\pgfqpoint{3.666057in}{1.099277in}}%
\pgfpathlineto{\pgfqpoint{3.668172in}{1.098832in}}%
\pgfpathlineto{\pgfqpoint{3.670286in}{1.101471in}}%
\pgfpathlineto{\pgfqpoint{3.672401in}{1.099328in}}%
\pgfpathlineto{\pgfqpoint{3.674516in}{1.100556in}}%
\pgfpathlineto{\pgfqpoint{3.676630in}{1.105178in}}%
\pgfpathlineto{\pgfqpoint{3.678745in}{1.103520in}}%
\pgfpathlineto{\pgfqpoint{3.680860in}{1.100233in}}%
\pgfpathlineto{\pgfqpoint{3.682974in}{1.100085in}}%
\pgfpathlineto{\pgfqpoint{3.685089in}{1.101442in}}%
\pgfpathlineto{\pgfqpoint{3.687204in}{1.101038in}}%
\pgfpathlineto{\pgfqpoint{3.689319in}{1.104769in}}%
\pgfpathlineto{\pgfqpoint{3.691433in}{1.105943in}}%
\pgfpathlineto{\pgfqpoint{3.693548in}{1.112882in}}%
\pgfpathlineto{\pgfqpoint{3.695663in}{1.114665in}}%
\pgfpathlineto{\pgfqpoint{3.697777in}{1.124586in}}%
\pgfpathlineto{\pgfqpoint{3.699892in}{1.117830in}}%
\pgfpathlineto{\pgfqpoint{3.704121in}{1.121250in}}%
\pgfpathlineto{\pgfqpoint{3.708351in}{1.117130in}}%
\pgfpathlineto{\pgfqpoint{3.710465in}{1.123428in}}%
\pgfpathlineto{\pgfqpoint{3.712580in}{1.121508in}}%
\pgfpathlineto{\pgfqpoint{3.714695in}{1.114812in}}%
\pgfpathlineto{\pgfqpoint{3.716810in}{1.114295in}}%
\pgfpathlineto{\pgfqpoint{3.718924in}{1.118929in}}%
\pgfpathlineto{\pgfqpoint{3.721039in}{1.114573in}}%
\pgfpathlineto{\pgfqpoint{3.729498in}{1.128138in}}%
\pgfpathlineto{\pgfqpoint{3.731612in}{1.122233in}}%
\pgfpathlineto{\pgfqpoint{3.733727in}{1.120765in}}%
\pgfpathlineto{\pgfqpoint{3.735842in}{1.124564in}}%
\pgfpathlineto{\pgfqpoint{3.737956in}{1.131482in}}%
\pgfpathlineto{\pgfqpoint{3.740071in}{1.127385in}}%
\pgfpathlineto{\pgfqpoint{3.742186in}{1.131263in}}%
\pgfpathlineto{\pgfqpoint{3.744301in}{1.131853in}}%
\pgfpathlineto{\pgfqpoint{3.752759in}{1.139049in}}%
\pgfpathlineto{\pgfqpoint{3.754874in}{1.141087in}}%
\pgfpathlineto{\pgfqpoint{3.756989in}{1.139845in}}%
\pgfpathlineto{\pgfqpoint{3.759103in}{1.135036in}}%
\pgfpathlineto{\pgfqpoint{3.761218in}{1.139569in}}%
\pgfpathlineto{\pgfqpoint{3.767562in}{1.129014in}}%
\pgfpathlineto{\pgfqpoint{3.769677in}{1.130755in}}%
\pgfpathlineto{\pgfqpoint{3.771792in}{1.128646in}}%
\pgfpathlineto{\pgfqpoint{3.776021in}{1.138688in}}%
\pgfpathlineto{\pgfqpoint{3.778136in}{1.137413in}}%
\pgfpathlineto{\pgfqpoint{3.780250in}{1.138512in}}%
\pgfpathlineto{\pgfqpoint{3.786594in}{1.133613in}}%
\pgfpathlineto{\pgfqpoint{3.788709in}{1.132729in}}%
\pgfpathlineto{\pgfqpoint{3.792939in}{1.120033in}}%
\pgfpathlineto{\pgfqpoint{3.795053in}{1.123505in}}%
\pgfpathlineto{\pgfqpoint{3.799283in}{1.125371in}}%
\pgfpathlineto{\pgfqpoint{3.801397in}{1.123437in}}%
\pgfpathlineto{\pgfqpoint{3.803512in}{1.119838in}}%
\pgfpathlineto{\pgfqpoint{3.805627in}{1.123137in}}%
\pgfpathlineto{\pgfqpoint{3.807741in}{1.131711in}}%
\pgfpathlineto{\pgfqpoint{3.809856in}{1.134026in}}%
\pgfpathlineto{\pgfqpoint{3.811971in}{1.131224in}}%
\pgfpathlineto{\pgfqpoint{3.814085in}{1.133253in}}%
\pgfpathlineto{\pgfqpoint{3.818315in}{1.141451in}}%
\pgfpathlineto{\pgfqpoint{3.822544in}{1.135119in}}%
\pgfpathlineto{\pgfqpoint{3.824659in}{1.136798in}}%
\pgfpathlineto{\pgfqpoint{3.826774in}{1.135394in}}%
\pgfpathlineto{\pgfqpoint{3.828888in}{1.139363in}}%
\pgfpathlineto{\pgfqpoint{3.833118in}{1.142465in}}%
\pgfpathlineto{\pgfqpoint{3.835232in}{1.144521in}}%
\pgfpathlineto{\pgfqpoint{3.837347in}{1.137615in}}%
\pgfpathlineto{\pgfqpoint{3.839462in}{1.134632in}}%
\pgfpathlineto{\pgfqpoint{3.841576in}{1.136660in}}%
\pgfpathlineto{\pgfqpoint{3.845806in}{1.127668in}}%
\pgfpathlineto{\pgfqpoint{3.852150in}{1.122465in}}%
\pgfpathlineto{\pgfqpoint{3.854265in}{1.125500in}}%
\pgfpathlineto{\pgfqpoint{3.860609in}{1.139503in}}%
\pgfpathlineto{\pgfqpoint{3.862723in}{1.139604in}}%
\pgfpathlineto{\pgfqpoint{3.864838in}{1.133884in}}%
\pgfpathlineto{\pgfqpoint{3.866953in}{1.135104in}}%
\pgfpathlineto{\pgfqpoint{3.869067in}{1.137880in}}%
\pgfpathlineto{\pgfqpoint{3.871182in}{1.145113in}}%
\pgfpathlineto{\pgfqpoint{3.873297in}{1.139493in}}%
\pgfpathlineto{\pgfqpoint{3.875412in}{1.129962in}}%
\pgfpathlineto{\pgfqpoint{3.877526in}{1.135964in}}%
\pgfpathlineto{\pgfqpoint{3.881756in}{1.132382in}}%
\pgfpathlineto{\pgfqpoint{3.883870in}{1.134278in}}%
\pgfpathlineto{\pgfqpoint{3.885985in}{1.133849in}}%
\pgfpathlineto{\pgfqpoint{3.888100in}{1.126045in}}%
\pgfpathlineto{\pgfqpoint{3.892329in}{1.130119in}}%
\pgfpathlineto{\pgfqpoint{3.894444in}{1.129955in}}%
\pgfpathlineto{\pgfqpoint{3.896559in}{1.134856in}}%
\pgfpathlineto{\pgfqpoint{3.898673in}{1.132548in}}%
\pgfpathlineto{\pgfqpoint{3.900788in}{1.133079in}}%
\pgfpathlineto{\pgfqpoint{3.902903in}{1.132218in}}%
\pgfpathlineto{\pgfqpoint{3.907132in}{1.117618in}}%
\pgfpathlineto{\pgfqpoint{3.911361in}{1.122498in}}%
\pgfpathlineto{\pgfqpoint{3.913476in}{1.120715in}}%
\pgfpathlineto{\pgfqpoint{3.915591in}{1.115573in}}%
\pgfpathlineto{\pgfqpoint{3.921935in}{1.119487in}}%
\pgfpathlineto{\pgfqpoint{3.924050in}{1.113435in}}%
\pgfpathlineto{\pgfqpoint{3.926164in}{1.114225in}}%
\pgfpathlineto{\pgfqpoint{3.930394in}{1.124200in}}%
\pgfpathlineto{\pgfqpoint{3.934623in}{1.129818in}}%
\pgfpathlineto{\pgfqpoint{3.936738in}{1.124825in}}%
\pgfpathlineto{\pgfqpoint{3.938852in}{1.133754in}}%
\pgfpathlineto{\pgfqpoint{3.940967in}{1.136862in}}%
\pgfpathlineto{\pgfqpoint{3.943082in}{1.134600in}}%
\pgfpathlineto{\pgfqpoint{3.945196in}{1.126956in}}%
\pgfpathlineto{\pgfqpoint{3.951541in}{1.130699in}}%
\pgfpathlineto{\pgfqpoint{3.953655in}{1.126369in}}%
\pgfpathlineto{\pgfqpoint{3.955770in}{1.125202in}}%
\pgfpathlineto{\pgfqpoint{3.957885in}{1.128743in}}%
\pgfpathlineto{\pgfqpoint{3.959999in}{1.126749in}}%
\pgfpathlineto{\pgfqpoint{3.962114in}{1.135848in}}%
\pgfpathlineto{\pgfqpoint{3.964229in}{1.139970in}}%
\pgfpathlineto{\pgfqpoint{3.966343in}{1.137482in}}%
\pgfpathlineto{\pgfqpoint{3.968458in}{1.140640in}}%
\pgfpathlineto{\pgfqpoint{3.970573in}{1.132332in}}%
\pgfpathlineto{\pgfqpoint{3.972687in}{1.132646in}}%
\pgfpathlineto{\pgfqpoint{3.981146in}{1.142174in}}%
\pgfpathlineto{\pgfqpoint{3.983261in}{1.149379in}}%
\pgfpathlineto{\pgfqpoint{3.985376in}{1.152186in}}%
\pgfpathlineto{\pgfqpoint{3.987490in}{1.150491in}}%
\pgfpathlineto{\pgfqpoint{3.995949in}{1.155547in}}%
\pgfpathlineto{\pgfqpoint{4.000178in}{1.152593in}}%
\pgfpathlineto{\pgfqpoint{4.002293in}{1.156301in}}%
\pgfpathlineto{\pgfqpoint{4.004408in}{1.156884in}}%
\pgfpathlineto{\pgfqpoint{4.006523in}{1.155948in}}%
\pgfpathlineto{\pgfqpoint{4.008637in}{1.158097in}}%
\pgfpathlineto{\pgfqpoint{4.010752in}{1.152509in}}%
\pgfpathlineto{\pgfqpoint{4.012867in}{1.150451in}}%
\pgfpathlineto{\pgfqpoint{4.014981in}{1.150560in}}%
\pgfpathlineto{\pgfqpoint{4.021325in}{1.137236in}}%
\pgfpathlineto{\pgfqpoint{4.023440in}{1.142150in}}%
\pgfpathlineto{\pgfqpoint{4.025555in}{1.140298in}}%
\pgfpathlineto{\pgfqpoint{4.029784in}{1.144167in}}%
\pgfpathlineto{\pgfqpoint{4.031899in}{1.143796in}}%
\pgfpathlineto{\pgfqpoint{4.036128in}{1.129819in}}%
\pgfpathlineto{\pgfqpoint{4.038243in}{1.132016in}}%
\pgfpathlineto{\pgfqpoint{4.040358in}{1.127177in}}%
\pgfpathlineto{\pgfqpoint{4.042472in}{1.132441in}}%
\pgfpathlineto{\pgfqpoint{4.044587in}{1.129333in}}%
\pgfpathlineto{\pgfqpoint{4.046702in}{1.129330in}}%
\pgfpathlineto{\pgfqpoint{4.048816in}{1.125899in}}%
\pgfpathlineto{\pgfqpoint{4.050931in}{1.127709in}}%
\pgfpathlineto{\pgfqpoint{4.053046in}{1.125753in}}%
\pgfpathlineto{\pgfqpoint{4.055161in}{1.126357in}}%
\pgfpathlineto{\pgfqpoint{4.057275in}{1.123657in}}%
\pgfpathlineto{\pgfqpoint{4.059390in}{1.124912in}}%
\pgfpathlineto{\pgfqpoint{4.061505in}{1.122708in}}%
\pgfpathlineto{\pgfqpoint{4.063619in}{1.122413in}}%
\pgfpathlineto{\pgfqpoint{4.067849in}{1.126023in}}%
\pgfpathlineto{\pgfqpoint{4.069963in}{1.127446in}}%
\pgfpathlineto{\pgfqpoint{4.072078in}{1.127420in}}%
\pgfpathlineto{\pgfqpoint{4.074193in}{1.124332in}}%
\pgfpathlineto{\pgfqpoint{4.080537in}{1.133641in}}%
\pgfpathlineto{\pgfqpoint{4.084766in}{1.125361in}}%
\pgfpathlineto{\pgfqpoint{4.086881in}{1.126915in}}%
\pgfpathlineto{\pgfqpoint{4.091110in}{1.133915in}}%
\pgfpathlineto{\pgfqpoint{4.093225in}{1.139749in}}%
\pgfpathlineto{\pgfqpoint{4.095340in}{1.131696in}}%
\pgfpathlineto{\pgfqpoint{4.097454in}{1.132804in}}%
\pgfpathlineto{\pgfqpoint{4.099569in}{1.130333in}}%
\pgfpathlineto{\pgfqpoint{4.105913in}{1.127710in}}%
\pgfpathlineto{\pgfqpoint{4.108028in}{1.133757in}}%
\pgfpathlineto{\pgfqpoint{4.110143in}{1.131718in}}%
\pgfpathlineto{\pgfqpoint{4.112257in}{1.120527in}}%
\pgfpathlineto{\pgfqpoint{4.116487in}{1.121624in}}%
\pgfpathlineto{\pgfqpoint{4.118601in}{1.126712in}}%
\pgfpathlineto{\pgfqpoint{4.122831in}{1.122510in}}%
\pgfpathlineto{\pgfqpoint{4.124945in}{1.120337in}}%
\pgfpathlineto{\pgfqpoint{4.127060in}{1.116198in}}%
\pgfpathlineto{\pgfqpoint{4.131290in}{1.124437in}}%
\pgfpathlineto{\pgfqpoint{4.133404in}{1.119410in}}%
\pgfpathlineto{\pgfqpoint{4.135519in}{1.117469in}}%
\pgfpathlineto{\pgfqpoint{4.137634in}{1.122745in}}%
\pgfpathlineto{\pgfqpoint{4.139748in}{1.117706in}}%
\pgfpathlineto{\pgfqpoint{4.143978in}{1.120226in}}%
\pgfpathlineto{\pgfqpoint{4.146092in}{1.112121in}}%
\pgfpathlineto{\pgfqpoint{4.148207in}{1.112851in}}%
\pgfpathlineto{\pgfqpoint{4.150322in}{1.116021in}}%
\pgfpathlineto{\pgfqpoint{4.156666in}{1.137359in}}%
\pgfpathlineto{\pgfqpoint{4.158781in}{1.137895in}}%
\pgfpathlineto{\pgfqpoint{4.163010in}{1.142887in}}%
\pgfpathlineto{\pgfqpoint{4.165125in}{1.143732in}}%
\pgfpathlineto{\pgfqpoint{4.167239in}{1.146794in}}%
\pgfpathlineto{\pgfqpoint{4.171469in}{1.157819in}}%
\pgfpathlineto{\pgfqpoint{4.173583in}{1.160580in}}%
\pgfpathlineto{\pgfqpoint{4.175698in}{1.160141in}}%
\pgfpathlineto{\pgfqpoint{4.177813in}{1.161648in}}%
\pgfpathlineto{\pgfqpoint{4.179927in}{1.168974in}}%
\pgfpathlineto{\pgfqpoint{4.182042in}{1.168232in}}%
\pgfpathlineto{\pgfqpoint{4.184157in}{1.173665in}}%
\pgfpathlineto{\pgfqpoint{4.188386in}{1.170870in}}%
\pgfpathlineto{\pgfqpoint{4.192616in}{1.171266in}}%
\pgfpathlineto{\pgfqpoint{4.194730in}{1.178377in}}%
\pgfpathlineto{\pgfqpoint{4.196845in}{1.178605in}}%
\pgfpathlineto{\pgfqpoint{4.201074in}{1.184163in}}%
\pgfpathlineto{\pgfqpoint{4.203189in}{1.184784in}}%
\pgfpathlineto{\pgfqpoint{4.209533in}{1.197020in}}%
\pgfpathlineto{\pgfqpoint{4.211648in}{1.191572in}}%
\pgfpathlineto{\pgfqpoint{4.213763in}{1.193731in}}%
\pgfpathlineto{\pgfqpoint{4.215877in}{1.197654in}}%
\pgfpathlineto{\pgfqpoint{4.220107in}{1.198923in}}%
\pgfpathlineto{\pgfqpoint{4.222221in}{1.195207in}}%
\pgfpathlineto{\pgfqpoint{4.224336in}{1.203227in}}%
\pgfpathlineto{\pgfqpoint{4.226451in}{1.205381in}}%
\pgfpathlineto{\pgfqpoint{4.228565in}{1.212333in}}%
\pgfpathlineto{\pgfqpoint{4.230680in}{1.207868in}}%
\pgfpathlineto{\pgfqpoint{4.234910in}{1.210554in}}%
\pgfpathlineto{\pgfqpoint{4.237024in}{1.203491in}}%
\pgfpathlineto{\pgfqpoint{4.239139in}{1.203126in}}%
\pgfpathlineto{\pgfqpoint{4.241254in}{1.205523in}}%
\pgfpathlineto{\pgfqpoint{4.243368in}{1.206002in}}%
\pgfpathlineto{\pgfqpoint{4.245483in}{1.213928in}}%
\pgfpathlineto{\pgfqpoint{4.247598in}{1.216340in}}%
\pgfpathlineto{\pgfqpoint{4.249712in}{1.215845in}}%
\pgfpathlineto{\pgfqpoint{4.253942in}{1.207285in}}%
\pgfpathlineto{\pgfqpoint{4.258171in}{1.203226in}}%
\pgfpathlineto{\pgfqpoint{4.260286in}{1.204456in}}%
\pgfpathlineto{\pgfqpoint{4.262401in}{1.204354in}}%
\pgfpathlineto{\pgfqpoint{4.264515in}{1.199708in}}%
\pgfpathlineto{\pgfqpoint{4.266630in}{1.190372in}}%
\pgfpathlineto{\pgfqpoint{4.268745in}{1.191194in}}%
\pgfpathlineto{\pgfqpoint{4.270859in}{1.183892in}}%
\pgfpathlineto{\pgfqpoint{4.275089in}{1.180468in}}%
\pgfpathlineto{\pgfqpoint{4.277203in}{1.183947in}}%
\pgfpathlineto{\pgfqpoint{4.279318in}{1.183262in}}%
\pgfpathlineto{\pgfqpoint{4.281433in}{1.184963in}}%
\pgfpathlineto{\pgfqpoint{4.283547in}{1.181829in}}%
\pgfpathlineto{\pgfqpoint{4.285662in}{1.182912in}}%
\pgfpathlineto{\pgfqpoint{4.289892in}{1.190044in}}%
\pgfpathlineto{\pgfqpoint{4.292006in}{1.185467in}}%
\pgfpathlineto{\pgfqpoint{4.294121in}{1.183918in}}%
\pgfpathlineto{\pgfqpoint{4.300465in}{1.169581in}}%
\pgfpathlineto{\pgfqpoint{4.302580in}{1.169133in}}%
\pgfpathlineto{\pgfqpoint{4.306809in}{1.172234in}}%
\pgfpathlineto{\pgfqpoint{4.308924in}{1.166681in}}%
\pgfpathlineto{\pgfqpoint{4.311038in}{1.167463in}}%
\pgfpathlineto{\pgfqpoint{4.313153in}{1.171014in}}%
\pgfpathlineto{\pgfqpoint{4.315268in}{1.171633in}}%
\pgfpathlineto{\pgfqpoint{4.317383in}{1.174911in}}%
\pgfpathlineto{\pgfqpoint{4.319497in}{1.168944in}}%
\pgfpathlineto{\pgfqpoint{4.321612in}{1.171970in}}%
\pgfpathlineto{\pgfqpoint{4.325841in}{1.160727in}}%
\pgfpathlineto{\pgfqpoint{4.327956in}{1.158516in}}%
\pgfpathlineto{\pgfqpoint{4.330071in}{1.153548in}}%
\pgfpathlineto{\pgfqpoint{4.334300in}{1.149975in}}%
\pgfpathlineto{\pgfqpoint{4.336415in}{1.158839in}}%
\pgfpathlineto{\pgfqpoint{4.342759in}{1.168007in}}%
\pgfpathlineto{\pgfqpoint{4.344874in}{1.174710in}}%
\pgfpathlineto{\pgfqpoint{4.349103in}{1.174086in}}%
\pgfpathlineto{\pgfqpoint{4.351218in}{1.179726in}}%
\pgfpathlineto{\pgfqpoint{4.355447in}{1.182794in}}%
\pgfpathlineto{\pgfqpoint{4.361791in}{1.170666in}}%
\pgfpathlineto{\pgfqpoint{4.363906in}{1.171441in}}%
\pgfpathlineto{\pgfqpoint{4.366021in}{1.176033in}}%
\pgfpathlineto{\pgfqpoint{4.368135in}{1.174345in}}%
\pgfpathlineto{\pgfqpoint{4.370250in}{1.177883in}}%
\pgfpathlineto{\pgfqpoint{4.372365in}{1.166688in}}%
\pgfpathlineto{\pgfqpoint{4.374479in}{1.162976in}}%
\pgfpathlineto{\pgfqpoint{4.376594in}{1.163589in}}%
\pgfpathlineto{\pgfqpoint{4.380823in}{1.149786in}}%
\pgfpathlineto{\pgfqpoint{4.382938in}{1.151134in}}%
\pgfpathlineto{\pgfqpoint{4.387167in}{1.146438in}}%
\pgfpathlineto{\pgfqpoint{4.389282in}{1.149719in}}%
\pgfpathlineto{\pgfqpoint{4.391397in}{1.145431in}}%
\pgfpathlineto{\pgfqpoint{4.393512in}{1.147128in}}%
\pgfpathlineto{\pgfqpoint{4.395626in}{1.146420in}}%
\pgfpathlineto{\pgfqpoint{4.397741in}{1.138455in}}%
\pgfpathlineto{\pgfqpoint{4.399856in}{1.136599in}}%
\pgfpathlineto{\pgfqpoint{4.401970in}{1.141897in}}%
\pgfpathlineto{\pgfqpoint{4.408314in}{1.133868in}}%
\pgfpathlineto{\pgfqpoint{4.410429in}{1.137348in}}%
\pgfpathlineto{\pgfqpoint{4.412544in}{1.137466in}}%
\pgfpathlineto{\pgfqpoint{4.414658in}{1.140042in}}%
\pgfpathlineto{\pgfqpoint{4.418888in}{1.150872in}}%
\pgfpathlineto{\pgfqpoint{4.421003in}{1.153095in}}%
\pgfpathlineto{\pgfqpoint{4.423117in}{1.159404in}}%
\pgfpathlineto{\pgfqpoint{4.425232in}{1.160965in}}%
\pgfpathlineto{\pgfqpoint{4.429461in}{1.153785in}}%
\pgfpathlineto{\pgfqpoint{4.433691in}{1.163017in}}%
\pgfpathlineto{\pgfqpoint{4.435805in}{1.157961in}}%
\pgfpathlineto{\pgfqpoint{4.440035in}{1.155210in}}%
\pgfpathlineto{\pgfqpoint{4.442149in}{1.155549in}}%
\pgfpathlineto{\pgfqpoint{4.444264in}{1.150385in}}%
\pgfpathlineto{\pgfqpoint{4.446379in}{1.148415in}}%
\pgfpathlineto{\pgfqpoint{4.448494in}{1.148152in}}%
\pgfpathlineto{\pgfqpoint{4.450608in}{1.146607in}}%
\pgfpathlineto{\pgfqpoint{4.452723in}{1.140069in}}%
\pgfpathlineto{\pgfqpoint{4.454838in}{1.141944in}}%
\pgfpathlineto{\pgfqpoint{4.456952in}{1.145384in}}%
\pgfpathlineto{\pgfqpoint{4.459067in}{1.155638in}}%
\pgfpathlineto{\pgfqpoint{4.463296in}{1.151739in}}%
\pgfpathlineto{\pgfqpoint{4.465411in}{1.152694in}}%
\pgfpathlineto{\pgfqpoint{4.467526in}{1.144529in}}%
\pgfpathlineto{\pgfqpoint{4.471755in}{1.141238in}}%
\pgfpathlineto{\pgfqpoint{4.473870in}{1.140609in}}%
\pgfpathlineto{\pgfqpoint{4.475985in}{1.136087in}}%
\pgfpathlineto{\pgfqpoint{4.482329in}{1.145081in}}%
\pgfpathlineto{\pgfqpoint{4.484443in}{1.144279in}}%
\pgfpathlineto{\pgfqpoint{4.486558in}{1.145528in}}%
\pgfpathlineto{\pgfqpoint{4.488673in}{1.142836in}}%
\pgfpathlineto{\pgfqpoint{4.490787in}{1.151031in}}%
\pgfpathlineto{\pgfqpoint{4.495017in}{1.155339in}}%
\pgfpathlineto{\pgfqpoint{4.501361in}{1.153644in}}%
\pgfpathlineto{\pgfqpoint{4.503476in}{1.157698in}}%
\pgfpathlineto{\pgfqpoint{4.509820in}{1.162247in}}%
\pgfpathlineto{\pgfqpoint{4.514049in}{1.176139in}}%
\pgfpathlineto{\pgfqpoint{4.518278in}{1.174310in}}%
\pgfpathlineto{\pgfqpoint{4.520393in}{1.178652in}}%
\pgfpathlineto{\pgfqpoint{4.522508in}{1.179346in}}%
\pgfpathlineto{\pgfqpoint{4.524623in}{1.175743in}}%
\pgfpathlineto{\pgfqpoint{4.526737in}{1.168665in}}%
\pgfpathlineto{\pgfqpoint{4.528852in}{1.168041in}}%
\pgfpathlineto{\pgfqpoint{4.530967in}{1.173475in}}%
\pgfpathlineto{\pgfqpoint{4.533081in}{1.173852in}}%
\pgfpathlineto{\pgfqpoint{4.535196in}{1.164568in}}%
\pgfpathlineto{\pgfqpoint{4.537311in}{1.165951in}}%
\pgfpathlineto{\pgfqpoint{4.539425in}{1.165802in}}%
\pgfpathlineto{\pgfqpoint{4.541540in}{1.166819in}}%
\pgfpathlineto{\pgfqpoint{4.543655in}{1.165792in}}%
\pgfpathlineto{\pgfqpoint{4.545769in}{1.166213in}}%
\pgfpathlineto{\pgfqpoint{4.547884in}{1.165179in}}%
\pgfpathlineto{\pgfqpoint{4.549999in}{1.170560in}}%
\pgfpathlineto{\pgfqpoint{4.552114in}{1.171692in}}%
\pgfpathlineto{\pgfqpoint{4.554228in}{1.168258in}}%
\pgfpathlineto{\pgfqpoint{4.556343in}{1.168314in}}%
\pgfpathlineto{\pgfqpoint{4.558458in}{1.170121in}}%
\pgfpathlineto{\pgfqpoint{4.562687in}{1.184884in}}%
\pgfpathlineto{\pgfqpoint{4.564802in}{1.187296in}}%
\pgfpathlineto{\pgfqpoint{4.566916in}{1.185907in}}%
\pgfpathlineto{\pgfqpoint{4.569031in}{1.180959in}}%
\pgfpathlineto{\pgfqpoint{4.571146in}{1.182902in}}%
\pgfpathlineto{\pgfqpoint{4.573260in}{1.189688in}}%
\pgfpathlineto{\pgfqpoint{4.577490in}{1.190228in}}%
\pgfpathlineto{\pgfqpoint{4.579605in}{1.192651in}}%
\pgfpathlineto{\pgfqpoint{4.581719in}{1.189858in}}%
\pgfpathlineto{\pgfqpoint{4.585949in}{1.178681in}}%
\pgfpathlineto{\pgfqpoint{4.588063in}{1.179277in}}%
\pgfpathlineto{\pgfqpoint{4.590178in}{1.175411in}}%
\pgfpathlineto{\pgfqpoint{4.592293in}{1.175810in}}%
\pgfpathlineto{\pgfqpoint{4.594407in}{1.186637in}}%
\pgfpathlineto{\pgfqpoint{4.596522in}{1.182557in}}%
\pgfpathlineto{\pgfqpoint{4.598637in}{1.182217in}}%
\pgfpathlineto{\pgfqpoint{4.600752in}{1.178789in}}%
\pgfpathlineto{\pgfqpoint{4.602866in}{1.185516in}}%
\pgfpathlineto{\pgfqpoint{4.607096in}{1.183980in}}%
\pgfpathlineto{\pgfqpoint{4.609210in}{1.185640in}}%
\pgfpathlineto{\pgfqpoint{4.615554in}{1.176289in}}%
\pgfpathlineto{\pgfqpoint{4.617669in}{1.177177in}}%
\pgfpathlineto{\pgfqpoint{4.619784in}{1.180918in}}%
\pgfpathlineto{\pgfqpoint{4.621898in}{1.181563in}}%
\pgfpathlineto{\pgfqpoint{4.624013in}{1.177614in}}%
\pgfpathlineto{\pgfqpoint{4.626128in}{1.176655in}}%
\pgfpathlineto{\pgfqpoint{4.630357in}{1.182732in}}%
\pgfpathlineto{\pgfqpoint{4.632472in}{1.179610in}}%
\pgfpathlineto{\pgfqpoint{4.634587in}{1.185290in}}%
\pgfpathlineto{\pgfqpoint{4.636701in}{1.183269in}}%
\pgfpathlineto{\pgfqpoint{4.638816in}{1.188776in}}%
\pgfpathlineto{\pgfqpoint{4.640931in}{1.189558in}}%
\pgfpathlineto{\pgfqpoint{4.643045in}{1.184952in}}%
\pgfpathlineto{\pgfqpoint{4.645160in}{1.185883in}}%
\pgfpathlineto{\pgfqpoint{4.649389in}{1.189709in}}%
\pgfpathlineto{\pgfqpoint{4.651504in}{1.182245in}}%
\pgfpathlineto{\pgfqpoint{4.653619in}{1.179677in}}%
\pgfpathlineto{\pgfqpoint{4.655734in}{1.164681in}}%
\pgfpathlineto{\pgfqpoint{4.659963in}{1.165105in}}%
\pgfpathlineto{\pgfqpoint{4.662078in}{1.171490in}}%
\pgfpathlineto{\pgfqpoint{4.664192in}{1.171258in}}%
\pgfpathlineto{\pgfqpoint{4.666307in}{1.179334in}}%
\pgfpathlineto{\pgfqpoint{4.670536in}{1.177849in}}%
\pgfpathlineto{\pgfqpoint{4.672651in}{1.173940in}}%
\pgfpathlineto{\pgfqpoint{4.676880in}{1.180066in}}%
\pgfpathlineto{\pgfqpoint{4.681110in}{1.178112in}}%
\pgfpathlineto{\pgfqpoint{4.683225in}{1.183613in}}%
\pgfpathlineto{\pgfqpoint{4.685339in}{1.178201in}}%
\pgfpathlineto{\pgfqpoint{4.687454in}{1.178929in}}%
\pgfpathlineto{\pgfqpoint{4.689569in}{1.174289in}}%
\pgfpathlineto{\pgfqpoint{4.693798in}{1.182190in}}%
\pgfpathlineto{\pgfqpoint{4.695913in}{1.176850in}}%
\pgfpathlineto{\pgfqpoint{4.700142in}{1.188098in}}%
\pgfpathlineto{\pgfqpoint{4.702257in}{1.188802in}}%
\pgfpathlineto{\pgfqpoint{4.704372in}{1.191079in}}%
\pgfpathlineto{\pgfqpoint{4.706486in}{1.188782in}}%
\pgfpathlineto{\pgfqpoint{4.708601in}{1.191009in}}%
\pgfpathlineto{\pgfqpoint{4.712830in}{1.178912in}}%
\pgfpathlineto{\pgfqpoint{4.714945in}{1.181681in}}%
\pgfpathlineto{\pgfqpoint{4.719174in}{1.177261in}}%
\pgfpathlineto{\pgfqpoint{4.721289in}{1.172425in}}%
\pgfpathlineto{\pgfqpoint{4.723404in}{1.164344in}}%
\pgfpathlineto{\pgfqpoint{4.725518in}{1.166511in}}%
\pgfpathlineto{\pgfqpoint{4.729748in}{1.174005in}}%
\pgfpathlineto{\pgfqpoint{4.731863in}{1.169904in}}%
\pgfpathlineto{\pgfqpoint{4.733977in}{1.173520in}}%
\pgfpathlineto{\pgfqpoint{4.736092in}{1.170761in}}%
\pgfpathlineto{\pgfqpoint{4.738207in}{1.176264in}}%
\pgfpathlineto{\pgfqpoint{4.740321in}{1.176128in}}%
\pgfpathlineto{\pgfqpoint{4.746665in}{1.181268in}}%
\pgfpathlineto{\pgfqpoint{4.750895in}{1.177677in}}%
\pgfpathlineto{\pgfqpoint{4.753009in}{1.176640in}}%
\pgfpathlineto{\pgfqpoint{4.759354in}{1.189783in}}%
\pgfpathlineto{\pgfqpoint{4.761468in}{1.185517in}}%
\pgfpathlineto{\pgfqpoint{4.763583in}{1.178311in}}%
\pgfpathlineto{\pgfqpoint{4.767812in}{1.186353in}}%
\pgfpathlineto{\pgfqpoint{4.769927in}{1.187033in}}%
\pgfpathlineto{\pgfqpoint{4.774156in}{1.195103in}}%
\pgfpathlineto{\pgfqpoint{4.780500in}{1.178177in}}%
\pgfpathlineto{\pgfqpoint{4.782615in}{1.177312in}}%
\pgfpathlineto{\pgfqpoint{4.784730in}{1.173602in}}%
\pgfpathlineto{\pgfqpoint{4.786845in}{1.175256in}}%
\pgfpathlineto{\pgfqpoint{4.788959in}{1.180082in}}%
\pgfpathlineto{\pgfqpoint{4.791074in}{1.180182in}}%
\pgfpathlineto{\pgfqpoint{4.793189in}{1.177510in}}%
\pgfpathlineto{\pgfqpoint{4.795303in}{1.178211in}}%
\pgfpathlineto{\pgfqpoint{4.799533in}{1.181447in}}%
\pgfpathlineto{\pgfqpoint{4.801647in}{1.179364in}}%
\pgfpathlineto{\pgfqpoint{4.803762in}{1.181729in}}%
\pgfpathlineto{\pgfqpoint{4.807991in}{1.161283in}}%
\pgfpathlineto{\pgfqpoint{4.810106in}{1.162918in}}%
\pgfpathlineto{\pgfqpoint{4.812221in}{1.159340in}}%
\pgfpathlineto{\pgfqpoint{4.814336in}{1.159471in}}%
\pgfpathlineto{\pgfqpoint{4.816450in}{1.152654in}}%
\pgfpathlineto{\pgfqpoint{4.818565in}{1.156320in}}%
\pgfpathlineto{\pgfqpoint{4.820680in}{1.151437in}}%
\pgfpathlineto{\pgfqpoint{4.822794in}{1.149774in}}%
\pgfpathlineto{\pgfqpoint{4.824909in}{1.144358in}}%
\pgfpathlineto{\pgfqpoint{4.827024in}{1.148116in}}%
\pgfpathlineto{\pgfqpoint{4.829138in}{1.148836in}}%
\pgfpathlineto{\pgfqpoint{4.831253in}{1.147270in}}%
\pgfpathlineto{\pgfqpoint{4.835483in}{1.139483in}}%
\pgfpathlineto{\pgfqpoint{4.837597in}{1.135774in}}%
\pgfpathlineto{\pgfqpoint{4.839712in}{1.137801in}}%
\pgfpathlineto{\pgfqpoint{4.841827in}{1.137672in}}%
\pgfpathlineto{\pgfqpoint{4.843941in}{1.142331in}}%
\pgfpathlineto{\pgfqpoint{4.846056in}{1.149743in}}%
\pgfpathlineto{\pgfqpoint{4.848171in}{1.153095in}}%
\pgfpathlineto{\pgfqpoint{4.850285in}{1.149753in}}%
\pgfpathlineto{\pgfqpoint{4.854515in}{1.159545in}}%
\pgfpathlineto{\pgfqpoint{4.856629in}{1.158876in}}%
\pgfpathlineto{\pgfqpoint{4.862974in}{1.172761in}}%
\pgfpathlineto{\pgfqpoint{4.873547in}{1.161582in}}%
\pgfpathlineto{\pgfqpoint{4.877776in}{1.167993in}}%
\pgfpathlineto{\pgfqpoint{4.879891in}{1.159671in}}%
\pgfpathlineto{\pgfqpoint{4.886235in}{1.158942in}}%
\pgfpathlineto{\pgfqpoint{4.888350in}{1.165735in}}%
\pgfpathlineto{\pgfqpoint{4.890465in}{1.165712in}}%
\pgfpathlineto{\pgfqpoint{4.896809in}{1.152181in}}%
\pgfpathlineto{\pgfqpoint{4.898923in}{1.155799in}}%
\pgfpathlineto{\pgfqpoint{4.903153in}{1.147307in}}%
\pgfpathlineto{\pgfqpoint{4.905267in}{1.146347in}}%
\pgfpathlineto{\pgfqpoint{4.909497in}{1.138034in}}%
\pgfpathlineto{\pgfqpoint{4.911611in}{1.139284in}}%
\pgfpathlineto{\pgfqpoint{4.913726in}{1.136142in}}%
\pgfpathlineto{\pgfqpoint{4.915841in}{1.135483in}}%
\pgfpathlineto{\pgfqpoint{4.917956in}{1.131717in}}%
\pgfpathlineto{\pgfqpoint{4.920070in}{1.132350in}}%
\pgfpathlineto{\pgfqpoint{4.922185in}{1.129905in}}%
\pgfpathlineto{\pgfqpoint{4.924300in}{1.130774in}}%
\pgfpathlineto{\pgfqpoint{4.926414in}{1.128413in}}%
\pgfpathlineto{\pgfqpoint{4.928529in}{1.131482in}}%
\pgfpathlineto{\pgfqpoint{4.932758in}{1.124858in}}%
\pgfpathlineto{\pgfqpoint{4.934873in}{1.124981in}}%
\pgfpathlineto{\pgfqpoint{4.936988in}{1.131673in}}%
\pgfpathlineto{\pgfqpoint{4.939103in}{1.130187in}}%
\pgfpathlineto{\pgfqpoint{4.945447in}{1.138312in}}%
\pgfpathlineto{\pgfqpoint{4.951791in}{1.129657in}}%
\pgfpathlineto{\pgfqpoint{4.953905in}{1.129180in}}%
\pgfpathlineto{\pgfqpoint{4.956020in}{1.133766in}}%
\pgfpathlineto{\pgfqpoint{4.960249in}{1.134976in}}%
\pgfpathlineto{\pgfqpoint{4.962364in}{1.139555in}}%
\pgfpathlineto{\pgfqpoint{4.966594in}{1.137179in}}%
\pgfpathlineto{\pgfqpoint{4.968708in}{1.127373in}}%
\pgfpathlineto{\pgfqpoint{4.970823in}{1.128866in}}%
\pgfpathlineto{\pgfqpoint{4.972938in}{1.127010in}}%
\pgfpathlineto{\pgfqpoint{4.975052in}{1.122983in}}%
\pgfpathlineto{\pgfqpoint{4.979282in}{1.127459in}}%
\pgfpathlineto{\pgfqpoint{4.981396in}{1.123017in}}%
\pgfpathlineto{\pgfqpoint{4.985626in}{1.125961in}}%
\pgfpathlineto{\pgfqpoint{4.989855in}{1.114075in}}%
\pgfpathlineto{\pgfqpoint{4.991970in}{1.122568in}}%
\pgfpathlineto{\pgfqpoint{4.996199in}{1.120703in}}%
\pgfpathlineto{\pgfqpoint{5.000429in}{1.117175in}}%
\pgfpathlineto{\pgfqpoint{5.002543in}{1.119523in}}%
\pgfpathlineto{\pgfqpoint{5.004658in}{1.126223in}}%
\pgfpathlineto{\pgfqpoint{5.006773in}{1.124936in}}%
\pgfpathlineto{\pgfqpoint{5.008887in}{1.128831in}}%
\pgfpathlineto{\pgfqpoint{5.011002in}{1.124645in}}%
\pgfpathlineto{\pgfqpoint{5.013117in}{1.125778in}}%
\pgfpathlineto{\pgfqpoint{5.015231in}{1.122950in}}%
\pgfpathlineto{\pgfqpoint{5.017346in}{1.123992in}}%
\pgfpathlineto{\pgfqpoint{5.019461in}{1.115975in}}%
\pgfpathlineto{\pgfqpoint{5.021576in}{1.119028in}}%
\pgfpathlineto{\pgfqpoint{5.025805in}{1.140547in}}%
\pgfpathlineto{\pgfqpoint{5.032149in}{1.154992in}}%
\pgfpathlineto{\pgfqpoint{5.036378in}{1.151367in}}%
\pgfpathlineto{\pgfqpoint{5.038493in}{1.151679in}}%
\pgfpathlineto{\pgfqpoint{5.042722in}{1.139208in}}%
\pgfpathlineto{\pgfqpoint{5.046952in}{1.147120in}}%
\pgfpathlineto{\pgfqpoint{5.049067in}{1.157616in}}%
\pgfpathlineto{\pgfqpoint{5.053296in}{1.146377in}}%
\pgfpathlineto{\pgfqpoint{5.055411in}{1.143337in}}%
\pgfpathlineto{\pgfqpoint{5.057525in}{1.142893in}}%
\pgfpathlineto{\pgfqpoint{5.059640in}{1.151547in}}%
\pgfpathlineto{\pgfqpoint{5.061755in}{1.145334in}}%
\pgfpathlineto{\pgfqpoint{5.063869in}{1.143262in}}%
\pgfpathlineto{\pgfqpoint{5.068099in}{1.149288in}}%
\pgfpathlineto{\pgfqpoint{5.070214in}{1.145668in}}%
\pgfpathlineto{\pgfqpoint{5.072328in}{1.151450in}}%
\pgfpathlineto{\pgfqpoint{5.076558in}{1.150697in}}%
\pgfpathlineto{\pgfqpoint{5.082902in}{1.139617in}}%
\pgfpathlineto{\pgfqpoint{5.085016in}{1.149070in}}%
\pgfpathlineto{\pgfqpoint{5.087131in}{1.146982in}}%
\pgfpathlineto{\pgfqpoint{5.089246in}{1.146893in}}%
\pgfpathlineto{\pgfqpoint{5.091360in}{1.150651in}}%
\pgfpathlineto{\pgfqpoint{5.097705in}{1.149450in}}%
\pgfpathlineto{\pgfqpoint{5.099819in}{1.154512in}}%
\pgfpathlineto{\pgfqpoint{5.101934in}{1.154836in}}%
\pgfpathlineto{\pgfqpoint{5.104049in}{1.159581in}}%
\pgfpathlineto{\pgfqpoint{5.108278in}{1.151470in}}%
\pgfpathlineto{\pgfqpoint{5.112507in}{1.155235in}}%
\pgfpathlineto{\pgfqpoint{5.114622in}{1.154796in}}%
\pgfpathlineto{\pgfqpoint{5.116737in}{1.157529in}}%
\pgfpathlineto{\pgfqpoint{5.118851in}{1.168395in}}%
\pgfpathlineto{\pgfqpoint{5.123081in}{1.169844in}}%
\pgfpathlineto{\pgfqpoint{5.125196in}{1.170323in}}%
\pgfpathlineto{\pgfqpoint{5.127310in}{1.175901in}}%
\pgfpathlineto{\pgfqpoint{5.129425in}{1.176949in}}%
\pgfpathlineto{\pgfqpoint{5.131540in}{1.181316in}}%
\pgfpathlineto{\pgfqpoint{5.135769in}{1.178345in}}%
\pgfpathlineto{\pgfqpoint{5.137884in}{1.182684in}}%
\pgfpathlineto{\pgfqpoint{5.142113in}{1.183616in}}%
\pgfpathlineto{\pgfqpoint{5.144228in}{1.183656in}}%
\pgfpathlineto{\pgfqpoint{5.146342in}{1.176754in}}%
\pgfpathlineto{\pgfqpoint{5.148457in}{1.175059in}}%
\pgfpathlineto{\pgfqpoint{5.152687in}{1.177828in}}%
\pgfpathlineto{\pgfqpoint{5.154801in}{1.178422in}}%
\pgfpathlineto{\pgfqpoint{5.156916in}{1.176617in}}%
\pgfpathlineto{\pgfqpoint{5.159031in}{1.172722in}}%
\pgfpathlineto{\pgfqpoint{5.163260in}{1.173461in}}%
\pgfpathlineto{\pgfqpoint{5.165375in}{1.172859in}}%
\pgfpathlineto{\pgfqpoint{5.167489in}{1.173788in}}%
\pgfpathlineto{\pgfqpoint{5.169604in}{1.172936in}}%
\pgfpathlineto{\pgfqpoint{5.171719in}{1.170686in}}%
\pgfpathlineto{\pgfqpoint{5.173834in}{1.173744in}}%
\pgfpathlineto{\pgfqpoint{5.175948in}{1.171958in}}%
\pgfpathlineto{\pgfqpoint{5.178063in}{1.178340in}}%
\pgfpathlineto{\pgfqpoint{5.180178in}{1.169409in}}%
\pgfpathlineto{\pgfqpoint{5.182292in}{1.165796in}}%
\pgfpathlineto{\pgfqpoint{5.184407in}{1.169868in}}%
\pgfpathlineto{\pgfqpoint{5.186522in}{1.171330in}}%
\pgfpathlineto{\pgfqpoint{5.188636in}{1.168686in}}%
\pgfpathlineto{\pgfqpoint{5.188636in}{1.168686in}}%
\pgfusepath{stroke}%
\end{pgfscope}%
\begin{pgfscope}%
\pgfpathrectangle{\pgfqpoint{0.750000in}{0.275000in}}{\pgfqpoint{4.650000in}{1.925000in}}%
\pgfusepath{clip}%
\pgfsetroundcap%
\pgfsetroundjoin%
\pgfsetlinewidth{1.003750pt}%
\definecolor{currentstroke}{rgb}{0.650980,0.337255,0.156863}%
\pgfsetstrokecolor{currentstroke}%
\pgfsetdash{}{0pt}%
\pgfpathmoveto{\pgfqpoint{0.961364in}{1.240293in}}%
\pgfpathlineto{\pgfqpoint{0.963478in}{1.246581in}}%
\pgfpathlineto{\pgfqpoint{0.965593in}{1.246013in}}%
\pgfpathlineto{\pgfqpoint{0.967708in}{1.250015in}}%
\pgfpathlineto{\pgfqpoint{0.969822in}{1.249824in}}%
\pgfpathlineto{\pgfqpoint{0.974052in}{1.251341in}}%
\pgfpathlineto{\pgfqpoint{0.976166in}{1.243838in}}%
\pgfpathlineto{\pgfqpoint{0.978281in}{1.242519in}}%
\pgfpathlineto{\pgfqpoint{0.982511in}{1.255960in}}%
\pgfpathlineto{\pgfqpoint{0.984625in}{1.250234in}}%
\pgfpathlineto{\pgfqpoint{0.986740in}{1.249689in}}%
\pgfpathlineto{\pgfqpoint{0.990969in}{1.243624in}}%
\pgfpathlineto{\pgfqpoint{0.993084in}{1.242724in}}%
\pgfpathlineto{\pgfqpoint{0.995199in}{1.248910in}}%
\pgfpathlineto{\pgfqpoint{0.997313in}{1.249006in}}%
\pgfpathlineto{\pgfqpoint{0.999428in}{1.254642in}}%
\pgfpathlineto{\pgfqpoint{1.001543in}{1.253549in}}%
\pgfpathlineto{\pgfqpoint{1.003658in}{1.250138in}}%
\pgfpathlineto{\pgfqpoint{1.010002in}{1.262816in}}%
\pgfpathlineto{\pgfqpoint{1.012116in}{1.266812in}}%
\pgfpathlineto{\pgfqpoint{1.014231in}{1.263638in}}%
\pgfpathlineto{\pgfqpoint{1.016346in}{1.252301in}}%
\pgfpathlineto{\pgfqpoint{1.018460in}{1.249922in}}%
\pgfpathlineto{\pgfqpoint{1.020575in}{1.241237in}}%
\pgfpathlineto{\pgfqpoint{1.022690in}{1.238095in}}%
\pgfpathlineto{\pgfqpoint{1.024804in}{1.246162in}}%
\pgfpathlineto{\pgfqpoint{1.029034in}{1.249281in}}%
\pgfpathlineto{\pgfqpoint{1.031149in}{1.244194in}}%
\pgfpathlineto{\pgfqpoint{1.033263in}{1.242104in}}%
\pgfpathlineto{\pgfqpoint{1.035378in}{1.243773in}}%
\pgfpathlineto{\pgfqpoint{1.039607in}{1.234150in}}%
\pgfpathlineto{\pgfqpoint{1.041722in}{1.243803in}}%
\pgfpathlineto{\pgfqpoint{1.043837in}{1.245319in}}%
\pgfpathlineto{\pgfqpoint{1.045951in}{1.252274in}}%
\pgfpathlineto{\pgfqpoint{1.048066in}{1.248706in}}%
\pgfpathlineto{\pgfqpoint{1.054410in}{1.247538in}}%
\pgfpathlineto{\pgfqpoint{1.058640in}{1.239588in}}%
\pgfpathlineto{\pgfqpoint{1.060754in}{1.243234in}}%
\pgfpathlineto{\pgfqpoint{1.062869in}{1.243101in}}%
\pgfpathlineto{\pgfqpoint{1.064984in}{1.245369in}}%
\pgfpathlineto{\pgfqpoint{1.069213in}{1.236875in}}%
\pgfpathlineto{\pgfqpoint{1.071328in}{1.234111in}}%
\pgfpathlineto{\pgfqpoint{1.073442in}{1.237493in}}%
\pgfpathlineto{\pgfqpoint{1.075557in}{1.237534in}}%
\pgfpathlineto{\pgfqpoint{1.077672in}{1.247299in}}%
\pgfpathlineto{\pgfqpoint{1.079786in}{1.241901in}}%
\pgfpathlineto{\pgfqpoint{1.084016in}{1.237995in}}%
\pgfpathlineto{\pgfqpoint{1.086131in}{1.241320in}}%
\pgfpathlineto{\pgfqpoint{1.090360in}{1.238826in}}%
\pgfpathlineto{\pgfqpoint{1.092475in}{1.243222in}}%
\pgfpathlineto{\pgfqpoint{1.096704in}{1.236271in}}%
\pgfpathlineto{\pgfqpoint{1.098819in}{1.227776in}}%
\pgfpathlineto{\pgfqpoint{1.100933in}{1.225415in}}%
\pgfpathlineto{\pgfqpoint{1.105163in}{1.232180in}}%
\pgfpathlineto{\pgfqpoint{1.109392in}{1.226254in}}%
\pgfpathlineto{\pgfqpoint{1.111507in}{1.227794in}}%
\pgfpathlineto{\pgfqpoint{1.113622in}{1.227181in}}%
\pgfpathlineto{\pgfqpoint{1.115736in}{1.229530in}}%
\pgfpathlineto{\pgfqpoint{1.119966in}{1.225376in}}%
\pgfpathlineto{\pgfqpoint{1.122080in}{1.227613in}}%
\pgfpathlineto{\pgfqpoint{1.124195in}{1.227629in}}%
\pgfpathlineto{\pgfqpoint{1.126310in}{1.222242in}}%
\pgfpathlineto{\pgfqpoint{1.130539in}{1.220710in}}%
\pgfpathlineto{\pgfqpoint{1.134769in}{1.226755in}}%
\pgfpathlineto{\pgfqpoint{1.136883in}{1.233608in}}%
\pgfpathlineto{\pgfqpoint{1.138998in}{1.227538in}}%
\pgfpathlineto{\pgfqpoint{1.143227in}{1.236480in}}%
\pgfpathlineto{\pgfqpoint{1.145342in}{1.234931in}}%
\pgfpathlineto{\pgfqpoint{1.147457in}{1.235002in}}%
\pgfpathlineto{\pgfqpoint{1.151686in}{1.231064in}}%
\pgfpathlineto{\pgfqpoint{1.153801in}{1.229210in}}%
\pgfpathlineto{\pgfqpoint{1.158030in}{1.231346in}}%
\pgfpathlineto{\pgfqpoint{1.160145in}{1.230464in}}%
\pgfpathlineto{\pgfqpoint{1.162260in}{1.234256in}}%
\pgfpathlineto{\pgfqpoint{1.166489in}{1.224630in}}%
\pgfpathlineto{\pgfqpoint{1.170718in}{1.227626in}}%
\pgfpathlineto{\pgfqpoint{1.172833in}{1.227658in}}%
\pgfpathlineto{\pgfqpoint{1.174948in}{1.229192in}}%
\pgfpathlineto{\pgfqpoint{1.177062in}{1.224368in}}%
\pgfpathlineto{\pgfqpoint{1.179177in}{1.228306in}}%
\pgfpathlineto{\pgfqpoint{1.181292in}{1.229182in}}%
\pgfpathlineto{\pgfqpoint{1.183406in}{1.232592in}}%
\pgfpathlineto{\pgfqpoint{1.185521in}{1.233122in}}%
\pgfpathlineto{\pgfqpoint{1.187636in}{1.239038in}}%
\pgfpathlineto{\pgfqpoint{1.189751in}{1.230021in}}%
\pgfpathlineto{\pgfqpoint{1.191865in}{1.233363in}}%
\pgfpathlineto{\pgfqpoint{1.193980in}{1.231320in}}%
\pgfpathlineto{\pgfqpoint{1.196095in}{1.232329in}}%
\pgfpathlineto{\pgfqpoint{1.198209in}{1.227513in}}%
\pgfpathlineto{\pgfqpoint{1.200324in}{1.235490in}}%
\pgfpathlineto{\pgfqpoint{1.206668in}{1.234716in}}%
\pgfpathlineto{\pgfqpoint{1.208783in}{1.236783in}}%
\pgfpathlineto{\pgfqpoint{1.210897in}{1.236222in}}%
\pgfpathlineto{\pgfqpoint{1.213012in}{1.233497in}}%
\pgfpathlineto{\pgfqpoint{1.215127in}{1.244136in}}%
\pgfpathlineto{\pgfqpoint{1.217242in}{1.243661in}}%
\pgfpathlineto{\pgfqpoint{1.219356in}{1.247103in}}%
\pgfpathlineto{\pgfqpoint{1.221471in}{1.245031in}}%
\pgfpathlineto{\pgfqpoint{1.225700in}{1.237030in}}%
\pgfpathlineto{\pgfqpoint{1.227815in}{1.236897in}}%
\pgfpathlineto{\pgfqpoint{1.229930in}{1.235626in}}%
\pgfpathlineto{\pgfqpoint{1.234159in}{1.239285in}}%
\pgfpathlineto{\pgfqpoint{1.236274in}{1.236348in}}%
\pgfpathlineto{\pgfqpoint{1.240503in}{1.226129in}}%
\pgfpathlineto{\pgfqpoint{1.242618in}{1.225954in}}%
\pgfpathlineto{\pgfqpoint{1.246847in}{1.237792in}}%
\pgfpathlineto{\pgfqpoint{1.248962in}{1.237034in}}%
\pgfpathlineto{\pgfqpoint{1.251077in}{1.234463in}}%
\pgfpathlineto{\pgfqpoint{1.255306in}{1.238931in}}%
\pgfpathlineto{\pgfqpoint{1.257421in}{1.232308in}}%
\pgfpathlineto{\pgfqpoint{1.259535in}{1.232241in}}%
\pgfpathlineto{\pgfqpoint{1.263765in}{1.240889in}}%
\pgfpathlineto{\pgfqpoint{1.265880in}{1.241245in}}%
\pgfpathlineto{\pgfqpoint{1.267994in}{1.239055in}}%
\pgfpathlineto{\pgfqpoint{1.270109in}{1.245483in}}%
\pgfpathlineto{\pgfqpoint{1.272224in}{1.243649in}}%
\pgfpathlineto{\pgfqpoint{1.274338in}{1.239376in}}%
\pgfpathlineto{\pgfqpoint{1.276453in}{1.243553in}}%
\pgfpathlineto{\pgfqpoint{1.278568in}{1.243299in}}%
\pgfpathlineto{\pgfqpoint{1.282797in}{1.249926in}}%
\pgfpathlineto{\pgfqpoint{1.287026in}{1.249016in}}%
\pgfpathlineto{\pgfqpoint{1.289141in}{1.252181in}}%
\pgfpathlineto{\pgfqpoint{1.291256in}{1.247892in}}%
\pgfpathlineto{\pgfqpoint{1.293371in}{1.246947in}}%
\pgfpathlineto{\pgfqpoint{1.297600in}{1.258730in}}%
\pgfpathlineto{\pgfqpoint{1.299715in}{1.255625in}}%
\pgfpathlineto{\pgfqpoint{1.301829in}{1.256586in}}%
\pgfpathlineto{\pgfqpoint{1.303944in}{1.258915in}}%
\pgfpathlineto{\pgfqpoint{1.306059in}{1.255317in}}%
\pgfpathlineto{\pgfqpoint{1.308173in}{1.255576in}}%
\pgfpathlineto{\pgfqpoint{1.310288in}{1.250815in}}%
\pgfpathlineto{\pgfqpoint{1.314517in}{1.258991in}}%
\pgfpathlineto{\pgfqpoint{1.316632in}{1.257439in}}%
\pgfpathlineto{\pgfqpoint{1.318747in}{1.262153in}}%
\pgfpathlineto{\pgfqpoint{1.320862in}{1.257019in}}%
\pgfpathlineto{\pgfqpoint{1.322976in}{1.255118in}}%
\pgfpathlineto{\pgfqpoint{1.325091in}{1.255881in}}%
\pgfpathlineto{\pgfqpoint{1.327206in}{1.262484in}}%
\pgfpathlineto{\pgfqpoint{1.329320in}{1.260475in}}%
\pgfpathlineto{\pgfqpoint{1.331435in}{1.264851in}}%
\pgfpathlineto{\pgfqpoint{1.333550in}{1.260735in}}%
\pgfpathlineto{\pgfqpoint{1.335664in}{1.267735in}}%
\pgfpathlineto{\pgfqpoint{1.339894in}{1.267532in}}%
\pgfpathlineto{\pgfqpoint{1.342009in}{1.264599in}}%
\pgfpathlineto{\pgfqpoint{1.344123in}{1.271166in}}%
\pgfpathlineto{\pgfqpoint{1.346238in}{1.265195in}}%
\pgfpathlineto{\pgfqpoint{1.348353in}{1.267135in}}%
\pgfpathlineto{\pgfqpoint{1.350467in}{1.267383in}}%
\pgfpathlineto{\pgfqpoint{1.354697in}{1.276863in}}%
\pgfpathlineto{\pgfqpoint{1.356811in}{1.272476in}}%
\pgfpathlineto{\pgfqpoint{1.358926in}{1.272504in}}%
\pgfpathlineto{\pgfqpoint{1.361041in}{1.279162in}}%
\pgfpathlineto{\pgfqpoint{1.363155in}{1.279999in}}%
\pgfpathlineto{\pgfqpoint{1.365270in}{1.273362in}}%
\pgfpathlineto{\pgfqpoint{1.367385in}{1.272409in}}%
\pgfpathlineto{\pgfqpoint{1.369500in}{1.269711in}}%
\pgfpathlineto{\pgfqpoint{1.371614in}{1.270430in}}%
\pgfpathlineto{\pgfqpoint{1.377958in}{1.262346in}}%
\pgfpathlineto{\pgfqpoint{1.380073in}{1.261027in}}%
\pgfpathlineto{\pgfqpoint{1.386417in}{1.246873in}}%
\pgfpathlineto{\pgfqpoint{1.388532in}{1.249935in}}%
\pgfpathlineto{\pgfqpoint{1.390646in}{1.248646in}}%
\pgfpathlineto{\pgfqpoint{1.392761in}{1.248805in}}%
\pgfpathlineto{\pgfqpoint{1.394876in}{1.253664in}}%
\pgfpathlineto{\pgfqpoint{1.396991in}{1.252023in}}%
\pgfpathlineto{\pgfqpoint{1.401220in}{1.243397in}}%
\pgfpathlineto{\pgfqpoint{1.403335in}{1.242545in}}%
\pgfpathlineto{\pgfqpoint{1.405449in}{1.234306in}}%
\pgfpathlineto{\pgfqpoint{1.407564in}{1.240651in}}%
\pgfpathlineto{\pgfqpoint{1.409679in}{1.243321in}}%
\pgfpathlineto{\pgfqpoint{1.411793in}{1.237117in}}%
\pgfpathlineto{\pgfqpoint{1.413908in}{1.238755in}}%
\pgfpathlineto{\pgfqpoint{1.416023in}{1.243206in}}%
\pgfpathlineto{\pgfqpoint{1.418137in}{1.244356in}}%
\pgfpathlineto{\pgfqpoint{1.420252in}{1.240203in}}%
\pgfpathlineto{\pgfqpoint{1.424482in}{1.228078in}}%
\pgfpathlineto{\pgfqpoint{1.426596in}{1.231724in}}%
\pgfpathlineto{\pgfqpoint{1.428711in}{1.229368in}}%
\pgfpathlineto{\pgfqpoint{1.430826in}{1.231566in}}%
\pgfpathlineto{\pgfqpoint{1.432940in}{1.238917in}}%
\pgfpathlineto{\pgfqpoint{1.435055in}{1.233855in}}%
\pgfpathlineto{\pgfqpoint{1.437170in}{1.232836in}}%
\pgfpathlineto{\pgfqpoint{1.439284in}{1.234555in}}%
\pgfpathlineto{\pgfqpoint{1.443514in}{1.221881in}}%
\pgfpathlineto{\pgfqpoint{1.445628in}{1.217645in}}%
\pgfpathlineto{\pgfqpoint{1.447743in}{1.225036in}}%
\pgfpathlineto{\pgfqpoint{1.449858in}{1.220458in}}%
\pgfpathlineto{\pgfqpoint{1.451973in}{1.221578in}}%
\pgfpathlineto{\pgfqpoint{1.454087in}{1.228005in}}%
\pgfpathlineto{\pgfqpoint{1.456202in}{1.229013in}}%
\pgfpathlineto{\pgfqpoint{1.460431in}{1.228301in}}%
\pgfpathlineto{\pgfqpoint{1.464661in}{1.228725in}}%
\pgfpathlineto{\pgfqpoint{1.466775in}{1.227136in}}%
\pgfpathlineto{\pgfqpoint{1.468890in}{1.220954in}}%
\pgfpathlineto{\pgfqpoint{1.471005in}{1.220009in}}%
\pgfpathlineto{\pgfqpoint{1.475234in}{1.215144in}}%
\pgfpathlineto{\pgfqpoint{1.481578in}{1.223961in}}%
\pgfpathlineto{\pgfqpoint{1.483693in}{1.229829in}}%
\pgfpathlineto{\pgfqpoint{1.485808in}{1.226775in}}%
\pgfpathlineto{\pgfqpoint{1.487922in}{1.228780in}}%
\pgfpathlineto{\pgfqpoint{1.492152in}{1.226280in}}%
\pgfpathlineto{\pgfqpoint{1.494266in}{1.232156in}}%
\pgfpathlineto{\pgfqpoint{1.496381in}{1.221238in}}%
\pgfpathlineto{\pgfqpoint{1.498496in}{1.224263in}}%
\pgfpathlineto{\pgfqpoint{1.500611in}{1.217082in}}%
\pgfpathlineto{\pgfqpoint{1.502725in}{1.220399in}}%
\pgfpathlineto{\pgfqpoint{1.509069in}{1.209912in}}%
\pgfpathlineto{\pgfqpoint{1.511184in}{1.209436in}}%
\pgfpathlineto{\pgfqpoint{1.513299in}{1.210845in}}%
\pgfpathlineto{\pgfqpoint{1.515413in}{1.218655in}}%
\pgfpathlineto{\pgfqpoint{1.517528in}{1.219125in}}%
\pgfpathlineto{\pgfqpoint{1.521757in}{1.222836in}}%
\pgfpathlineto{\pgfqpoint{1.523872in}{1.221937in}}%
\pgfpathlineto{\pgfqpoint{1.525987in}{1.233836in}}%
\pgfpathlineto{\pgfqpoint{1.528102in}{1.231822in}}%
\pgfpathlineto{\pgfqpoint{1.530216in}{1.232832in}}%
\pgfpathlineto{\pgfqpoint{1.532331in}{1.235564in}}%
\pgfpathlineto{\pgfqpoint{1.534446in}{1.240676in}}%
\pgfpathlineto{\pgfqpoint{1.536560in}{1.238866in}}%
\pgfpathlineto{\pgfqpoint{1.538675in}{1.239597in}}%
\pgfpathlineto{\pgfqpoint{1.540790in}{1.246370in}}%
\pgfpathlineto{\pgfqpoint{1.542904in}{1.247549in}}%
\pgfpathlineto{\pgfqpoint{1.545019in}{1.253714in}}%
\pgfpathlineto{\pgfqpoint{1.547134in}{1.253194in}}%
\pgfpathlineto{\pgfqpoint{1.551363in}{1.240302in}}%
\pgfpathlineto{\pgfqpoint{1.553478in}{1.242757in}}%
\pgfpathlineto{\pgfqpoint{1.557707in}{1.241627in}}%
\pgfpathlineto{\pgfqpoint{1.559822in}{1.239168in}}%
\pgfpathlineto{\pgfqpoint{1.561937in}{1.243028in}}%
\pgfpathlineto{\pgfqpoint{1.564051in}{1.236759in}}%
\pgfpathlineto{\pgfqpoint{1.566166in}{1.235496in}}%
\pgfpathlineto{\pgfqpoint{1.568281in}{1.232890in}}%
\pgfpathlineto{\pgfqpoint{1.570395in}{1.234528in}}%
\pgfpathlineto{\pgfqpoint{1.572510in}{1.233255in}}%
\pgfpathlineto{\pgfqpoint{1.576740in}{1.225769in}}%
\pgfpathlineto{\pgfqpoint{1.578854in}{1.224724in}}%
\pgfpathlineto{\pgfqpoint{1.580969in}{1.228694in}}%
\pgfpathlineto{\pgfqpoint{1.583084in}{1.228803in}}%
\pgfpathlineto{\pgfqpoint{1.585198in}{1.233053in}}%
\pgfpathlineto{\pgfqpoint{1.587313in}{1.231048in}}%
\pgfpathlineto{\pgfqpoint{1.589428in}{1.234804in}}%
\pgfpathlineto{\pgfqpoint{1.591542in}{1.225301in}}%
\pgfpathlineto{\pgfqpoint{1.600001in}{1.229799in}}%
\pgfpathlineto{\pgfqpoint{1.602116in}{1.227697in}}%
\pgfpathlineto{\pgfqpoint{1.604231in}{1.230090in}}%
\pgfpathlineto{\pgfqpoint{1.608460in}{1.222611in}}%
\pgfpathlineto{\pgfqpoint{1.614804in}{1.243260in}}%
\pgfpathlineto{\pgfqpoint{1.619033in}{1.233048in}}%
\pgfpathlineto{\pgfqpoint{1.621148in}{1.238942in}}%
\pgfpathlineto{\pgfqpoint{1.623263in}{1.238605in}}%
\pgfpathlineto{\pgfqpoint{1.627492in}{1.226619in}}%
\pgfpathlineto{\pgfqpoint{1.629607in}{1.236308in}}%
\pgfpathlineto{\pgfqpoint{1.631722in}{1.234341in}}%
\pgfpathlineto{\pgfqpoint{1.633836in}{1.234417in}}%
\pgfpathlineto{\pgfqpoint{1.635951in}{1.230816in}}%
\pgfpathlineto{\pgfqpoint{1.644410in}{1.239875in}}%
\pgfpathlineto{\pgfqpoint{1.648639in}{1.238175in}}%
\pgfpathlineto{\pgfqpoint{1.650754in}{1.236027in}}%
\pgfpathlineto{\pgfqpoint{1.652868in}{1.236799in}}%
\pgfpathlineto{\pgfqpoint{1.654983in}{1.234781in}}%
\pgfpathlineto{\pgfqpoint{1.657098in}{1.224755in}}%
\pgfpathlineto{\pgfqpoint{1.659213in}{1.220518in}}%
\pgfpathlineto{\pgfqpoint{1.661327in}{1.220985in}}%
\pgfpathlineto{\pgfqpoint{1.663442in}{1.220198in}}%
\pgfpathlineto{\pgfqpoint{1.665557in}{1.217046in}}%
\pgfpathlineto{\pgfqpoint{1.667671in}{1.218034in}}%
\pgfpathlineto{\pgfqpoint{1.669786in}{1.214597in}}%
\pgfpathlineto{\pgfqpoint{1.671901in}{1.214261in}}%
\pgfpathlineto{\pgfqpoint{1.674015in}{1.215996in}}%
\pgfpathlineto{\pgfqpoint{1.676130in}{1.208321in}}%
\pgfpathlineto{\pgfqpoint{1.678245in}{1.207383in}}%
\pgfpathlineto{\pgfqpoint{1.684589in}{1.202460in}}%
\pgfpathlineto{\pgfqpoint{1.686704in}{1.202684in}}%
\pgfpathlineto{\pgfqpoint{1.688818in}{1.205457in}}%
\pgfpathlineto{\pgfqpoint{1.690933in}{1.202342in}}%
\pgfpathlineto{\pgfqpoint{1.693048in}{1.205286in}}%
\pgfpathlineto{\pgfqpoint{1.695162in}{1.199725in}}%
\pgfpathlineto{\pgfqpoint{1.697277in}{1.201442in}}%
\pgfpathlineto{\pgfqpoint{1.699392in}{1.198369in}}%
\pgfpathlineto{\pgfqpoint{1.701506in}{1.192302in}}%
\pgfpathlineto{\pgfqpoint{1.703621in}{1.196572in}}%
\pgfpathlineto{\pgfqpoint{1.705736in}{1.208529in}}%
\pgfpathlineto{\pgfqpoint{1.707851in}{1.213751in}}%
\pgfpathlineto{\pgfqpoint{1.712080in}{1.211741in}}%
\pgfpathlineto{\pgfqpoint{1.714195in}{1.214568in}}%
\pgfpathlineto{\pgfqpoint{1.718424in}{1.211572in}}%
\pgfpathlineto{\pgfqpoint{1.720539in}{1.207002in}}%
\pgfpathlineto{\pgfqpoint{1.726883in}{1.209151in}}%
\pgfpathlineto{\pgfqpoint{1.728997in}{1.201063in}}%
\pgfpathlineto{\pgfqpoint{1.731112in}{1.205414in}}%
\pgfpathlineto{\pgfqpoint{1.733227in}{1.212572in}}%
\pgfpathlineto{\pgfqpoint{1.735342in}{1.211331in}}%
\pgfpathlineto{\pgfqpoint{1.737456in}{1.206680in}}%
\pgfpathlineto{\pgfqpoint{1.741686in}{1.215641in}}%
\pgfpathlineto{\pgfqpoint{1.745915in}{1.213052in}}%
\pgfpathlineto{\pgfqpoint{1.750144in}{1.228781in}}%
\pgfpathlineto{\pgfqpoint{1.752259in}{1.227326in}}%
\pgfpathlineto{\pgfqpoint{1.756488in}{1.236699in}}%
\pgfpathlineto{\pgfqpoint{1.758603in}{1.238249in}}%
\pgfpathlineto{\pgfqpoint{1.760718in}{1.245282in}}%
\pgfpathlineto{\pgfqpoint{1.762833in}{1.245992in}}%
\pgfpathlineto{\pgfqpoint{1.764947in}{1.251599in}}%
\pgfpathlineto{\pgfqpoint{1.771291in}{1.275995in}}%
\pgfpathlineto{\pgfqpoint{1.773406in}{1.277006in}}%
\pgfpathlineto{\pgfqpoint{1.775521in}{1.269923in}}%
\pgfpathlineto{\pgfqpoint{1.779750in}{1.279344in}}%
\pgfpathlineto{\pgfqpoint{1.781865in}{1.274840in}}%
\pgfpathlineto{\pgfqpoint{1.783979in}{1.273850in}}%
\pgfpathlineto{\pgfqpoint{1.786094in}{1.265829in}}%
\pgfpathlineto{\pgfqpoint{1.788209in}{1.269519in}}%
\pgfpathlineto{\pgfqpoint{1.790324in}{1.270293in}}%
\pgfpathlineto{\pgfqpoint{1.794553in}{1.277833in}}%
\pgfpathlineto{\pgfqpoint{1.796668in}{1.277044in}}%
\pgfpathlineto{\pgfqpoint{1.798782in}{1.275021in}}%
\pgfpathlineto{\pgfqpoint{1.800897in}{1.270398in}}%
\pgfpathlineto{\pgfqpoint{1.803012in}{1.275429in}}%
\pgfpathlineto{\pgfqpoint{1.805126in}{1.276366in}}%
\pgfpathlineto{\pgfqpoint{1.807241in}{1.270444in}}%
\pgfpathlineto{\pgfqpoint{1.813585in}{1.282550in}}%
\pgfpathlineto{\pgfqpoint{1.817815in}{1.281043in}}%
\pgfpathlineto{\pgfqpoint{1.822044in}{1.270347in}}%
\pgfpathlineto{\pgfqpoint{1.824159in}{1.273699in}}%
\pgfpathlineto{\pgfqpoint{1.826273in}{1.268785in}}%
\pgfpathlineto{\pgfqpoint{1.828388in}{1.268433in}}%
\pgfpathlineto{\pgfqpoint{1.830503in}{1.266880in}}%
\pgfpathlineto{\pgfqpoint{1.834732in}{1.268712in}}%
\pgfpathlineto{\pgfqpoint{1.838962in}{1.256992in}}%
\pgfpathlineto{\pgfqpoint{1.841076in}{1.255482in}}%
\pgfpathlineto{\pgfqpoint{1.843191in}{1.252065in}}%
\pgfpathlineto{\pgfqpoint{1.845306in}{1.253384in}}%
\pgfpathlineto{\pgfqpoint{1.847420in}{1.244570in}}%
\pgfpathlineto{\pgfqpoint{1.849535in}{1.246156in}}%
\pgfpathlineto{\pgfqpoint{1.851650in}{1.246253in}}%
\pgfpathlineto{\pgfqpoint{1.853764in}{1.253862in}}%
\pgfpathlineto{\pgfqpoint{1.855879in}{1.256007in}}%
\pgfpathlineto{\pgfqpoint{1.857994in}{1.263826in}}%
\pgfpathlineto{\pgfqpoint{1.868567in}{1.249952in}}%
\pgfpathlineto{\pgfqpoint{1.870682in}{1.250893in}}%
\pgfpathlineto{\pgfqpoint{1.872797in}{1.248898in}}%
\pgfpathlineto{\pgfqpoint{1.874911in}{1.255024in}}%
\pgfpathlineto{\pgfqpoint{1.877026in}{1.248666in}}%
\pgfpathlineto{\pgfqpoint{1.879141in}{1.235770in}}%
\pgfpathlineto{\pgfqpoint{1.881255in}{1.235024in}}%
\pgfpathlineto{\pgfqpoint{1.883370in}{1.239399in}}%
\pgfpathlineto{\pgfqpoint{1.885485in}{1.235744in}}%
\pgfpathlineto{\pgfqpoint{1.887599in}{1.229824in}}%
\pgfpathlineto{\pgfqpoint{1.889714in}{1.230266in}}%
\pgfpathlineto{\pgfqpoint{1.891829in}{1.228537in}}%
\pgfpathlineto{\pgfqpoint{1.893944in}{1.229711in}}%
\pgfpathlineto{\pgfqpoint{1.896058in}{1.232364in}}%
\pgfpathlineto{\pgfqpoint{1.898173in}{1.230390in}}%
\pgfpathlineto{\pgfqpoint{1.900288in}{1.236674in}}%
\pgfpathlineto{\pgfqpoint{1.902402in}{1.231920in}}%
\pgfpathlineto{\pgfqpoint{1.906632in}{1.244419in}}%
\pgfpathlineto{\pgfqpoint{1.908746in}{1.242532in}}%
\pgfpathlineto{\pgfqpoint{1.910861in}{1.242851in}}%
\pgfpathlineto{\pgfqpoint{1.912976in}{1.244629in}}%
\pgfpathlineto{\pgfqpoint{1.915090in}{1.253958in}}%
\pgfpathlineto{\pgfqpoint{1.917205in}{1.256003in}}%
\pgfpathlineto{\pgfqpoint{1.919320in}{1.255144in}}%
\pgfpathlineto{\pgfqpoint{1.921435in}{1.258978in}}%
\pgfpathlineto{\pgfqpoint{1.923549in}{1.260172in}}%
\pgfpathlineto{\pgfqpoint{1.925664in}{1.259452in}}%
\pgfpathlineto{\pgfqpoint{1.929893in}{1.261379in}}%
\pgfpathlineto{\pgfqpoint{1.932008in}{1.264526in}}%
\pgfpathlineto{\pgfqpoint{1.934123in}{1.258470in}}%
\pgfpathlineto{\pgfqpoint{1.936237in}{1.260406in}}%
\pgfpathlineto{\pgfqpoint{1.938352in}{1.249642in}}%
\pgfpathlineto{\pgfqpoint{1.940467in}{1.250430in}}%
\pgfpathlineto{\pgfqpoint{1.942582in}{1.253424in}}%
\pgfpathlineto{\pgfqpoint{1.944696in}{1.259939in}}%
\pgfpathlineto{\pgfqpoint{1.946811in}{1.261620in}}%
\pgfpathlineto{\pgfqpoint{1.953155in}{1.249753in}}%
\pgfpathlineto{\pgfqpoint{1.957384in}{1.258349in}}%
\pgfpathlineto{\pgfqpoint{1.965843in}{1.244175in}}%
\pgfpathlineto{\pgfqpoint{1.967958in}{1.246785in}}%
\pgfpathlineto{\pgfqpoint{1.970073in}{1.245639in}}%
\pgfpathlineto{\pgfqpoint{1.972187in}{1.252219in}}%
\pgfpathlineto{\pgfqpoint{1.974302in}{1.252597in}}%
\pgfpathlineto{\pgfqpoint{1.978531in}{1.249345in}}%
\pgfpathlineto{\pgfqpoint{1.980646in}{1.242939in}}%
\pgfpathlineto{\pgfqpoint{1.982761in}{1.248306in}}%
\pgfpathlineto{\pgfqpoint{1.984875in}{1.247072in}}%
\pgfpathlineto{\pgfqpoint{1.986990in}{1.248911in}}%
\pgfpathlineto{\pgfqpoint{1.991219in}{1.246825in}}%
\pgfpathlineto{\pgfqpoint{1.993334in}{1.250549in}}%
\pgfpathlineto{\pgfqpoint{1.997564in}{1.247994in}}%
\pgfpathlineto{\pgfqpoint{1.999678in}{1.239654in}}%
\pgfpathlineto{\pgfqpoint{2.001793in}{1.239808in}}%
\pgfpathlineto{\pgfqpoint{2.003908in}{1.236414in}}%
\pgfpathlineto{\pgfqpoint{2.006022in}{1.238251in}}%
\pgfpathlineto{\pgfqpoint{2.008137in}{1.244014in}}%
\pgfpathlineto{\pgfqpoint{2.010252in}{1.240379in}}%
\pgfpathlineto{\pgfqpoint{2.012366in}{1.242262in}}%
\pgfpathlineto{\pgfqpoint{2.014481in}{1.239296in}}%
\pgfpathlineto{\pgfqpoint{2.018710in}{1.247196in}}%
\pgfpathlineto{\pgfqpoint{2.020825in}{1.240594in}}%
\pgfpathlineto{\pgfqpoint{2.022940in}{1.240214in}}%
\pgfpathlineto{\pgfqpoint{2.025055in}{1.234334in}}%
\pgfpathlineto{\pgfqpoint{2.027169in}{1.233894in}}%
\pgfpathlineto{\pgfqpoint{2.029284in}{1.234999in}}%
\pgfpathlineto{\pgfqpoint{2.031399in}{1.230440in}}%
\pgfpathlineto{\pgfqpoint{2.033513in}{1.231350in}}%
\pgfpathlineto{\pgfqpoint{2.035628in}{1.228070in}}%
\pgfpathlineto{\pgfqpoint{2.037743in}{1.231936in}}%
\pgfpathlineto{\pgfqpoint{2.039857in}{1.231158in}}%
\pgfpathlineto{\pgfqpoint{2.041972in}{1.232759in}}%
\pgfpathlineto{\pgfqpoint{2.044087in}{1.241553in}}%
\pgfpathlineto{\pgfqpoint{2.046202in}{1.240967in}}%
\pgfpathlineto{\pgfqpoint{2.048316in}{1.255927in}}%
\pgfpathlineto{\pgfqpoint{2.050431in}{1.255474in}}%
\pgfpathlineto{\pgfqpoint{2.052546in}{1.263852in}}%
\pgfpathlineto{\pgfqpoint{2.056775in}{1.270152in}}%
\pgfpathlineto{\pgfqpoint{2.058890in}{1.271134in}}%
\pgfpathlineto{\pgfqpoint{2.061004in}{1.269792in}}%
\pgfpathlineto{\pgfqpoint{2.063119in}{1.272072in}}%
\pgfpathlineto{\pgfqpoint{2.065234in}{1.267539in}}%
\pgfpathlineto{\pgfqpoint{2.067348in}{1.260357in}}%
\pgfpathlineto{\pgfqpoint{2.069463in}{1.262677in}}%
\pgfpathlineto{\pgfqpoint{2.071578in}{1.268830in}}%
\pgfpathlineto{\pgfqpoint{2.073693in}{1.268033in}}%
\pgfpathlineto{\pgfqpoint{2.077922in}{1.268293in}}%
\pgfpathlineto{\pgfqpoint{2.080037in}{1.273457in}}%
\pgfpathlineto{\pgfqpoint{2.082151in}{1.268994in}}%
\pgfpathlineto{\pgfqpoint{2.086381in}{1.269951in}}%
\pgfpathlineto{\pgfqpoint{2.090610in}{1.272405in}}%
\pgfpathlineto{\pgfqpoint{2.092725in}{1.275628in}}%
\pgfpathlineto{\pgfqpoint{2.094839in}{1.275561in}}%
\pgfpathlineto{\pgfqpoint{2.096954in}{1.276763in}}%
\pgfpathlineto{\pgfqpoint{2.101184in}{1.274952in}}%
\pgfpathlineto{\pgfqpoint{2.103298in}{1.270762in}}%
\pgfpathlineto{\pgfqpoint{2.105413in}{1.274480in}}%
\pgfpathlineto{\pgfqpoint{2.107528in}{1.269496in}}%
\pgfpathlineto{\pgfqpoint{2.109642in}{1.272793in}}%
\pgfpathlineto{\pgfqpoint{2.111757in}{1.269983in}}%
\pgfpathlineto{\pgfqpoint{2.113872in}{1.269475in}}%
\pgfpathlineto{\pgfqpoint{2.115986in}{1.270366in}}%
\pgfpathlineto{\pgfqpoint{2.118101in}{1.267474in}}%
\pgfpathlineto{\pgfqpoint{2.120216in}{1.266857in}}%
\pgfpathlineto{\pgfqpoint{2.124445in}{1.269529in}}%
\pgfpathlineto{\pgfqpoint{2.126560in}{1.266909in}}%
\pgfpathlineto{\pgfqpoint{2.128675in}{1.267192in}}%
\pgfpathlineto{\pgfqpoint{2.130789in}{1.269247in}}%
\pgfpathlineto{\pgfqpoint{2.132904in}{1.272966in}}%
\pgfpathlineto{\pgfqpoint{2.137133in}{1.266918in}}%
\pgfpathlineto{\pgfqpoint{2.141363in}{1.275276in}}%
\pgfpathlineto{\pgfqpoint{2.143477in}{1.272623in}}%
\pgfpathlineto{\pgfqpoint{2.145592in}{1.272135in}}%
\pgfpathlineto{\pgfqpoint{2.147707in}{1.268612in}}%
\pgfpathlineto{\pgfqpoint{2.149822in}{1.270544in}}%
\pgfpathlineto{\pgfqpoint{2.151936in}{1.263895in}}%
\pgfpathlineto{\pgfqpoint{2.154051in}{1.265693in}}%
\pgfpathlineto{\pgfqpoint{2.156166in}{1.262574in}}%
\pgfpathlineto{\pgfqpoint{2.158280in}{1.267135in}}%
\pgfpathlineto{\pgfqpoint{2.164624in}{1.250851in}}%
\pgfpathlineto{\pgfqpoint{2.166739in}{1.251742in}}%
\pgfpathlineto{\pgfqpoint{2.168854in}{1.255040in}}%
\pgfpathlineto{\pgfqpoint{2.170968in}{1.246178in}}%
\pgfpathlineto{\pgfqpoint{2.173083in}{1.251116in}}%
\pgfpathlineto{\pgfqpoint{2.179427in}{1.248425in}}%
\pgfpathlineto{\pgfqpoint{2.181542in}{1.250632in}}%
\pgfpathlineto{\pgfqpoint{2.183657in}{1.245251in}}%
\pgfpathlineto{\pgfqpoint{2.187886in}{1.250241in}}%
\pgfpathlineto{\pgfqpoint{2.192115in}{1.257892in}}%
\pgfpathlineto{\pgfqpoint{2.196345in}{1.249533in}}%
\pgfpathlineto{\pgfqpoint{2.198459in}{1.250998in}}%
\pgfpathlineto{\pgfqpoint{2.200574in}{1.248143in}}%
\pgfpathlineto{\pgfqpoint{2.202689in}{1.249894in}}%
\pgfpathlineto{\pgfqpoint{2.204804in}{1.255389in}}%
\pgfpathlineto{\pgfqpoint{2.206918in}{1.251953in}}%
\pgfpathlineto{\pgfqpoint{2.209033in}{1.252220in}}%
\pgfpathlineto{\pgfqpoint{2.211148in}{1.247698in}}%
\pgfpathlineto{\pgfqpoint{2.213262in}{1.253228in}}%
\pgfpathlineto{\pgfqpoint{2.215377in}{1.252200in}}%
\pgfpathlineto{\pgfqpoint{2.217492in}{1.247979in}}%
\pgfpathlineto{\pgfqpoint{2.225950in}{1.248850in}}%
\pgfpathlineto{\pgfqpoint{2.230180in}{1.241518in}}%
\pgfpathlineto{\pgfqpoint{2.234409in}{1.257925in}}%
\pgfpathlineto{\pgfqpoint{2.236524in}{1.256141in}}%
\pgfpathlineto{\pgfqpoint{2.240753in}{1.260859in}}%
\pgfpathlineto{\pgfqpoint{2.242868in}{1.259884in}}%
\pgfpathlineto{\pgfqpoint{2.244983in}{1.263439in}}%
\pgfpathlineto{\pgfqpoint{2.249212in}{1.259637in}}%
\pgfpathlineto{\pgfqpoint{2.251327in}{1.260303in}}%
\pgfpathlineto{\pgfqpoint{2.255556in}{1.257078in}}%
\pgfpathlineto{\pgfqpoint{2.257671in}{1.258406in}}%
\pgfpathlineto{\pgfqpoint{2.264015in}{1.249926in}}%
\pgfpathlineto{\pgfqpoint{2.266130in}{1.254863in}}%
\pgfpathlineto{\pgfqpoint{2.270359in}{1.248233in}}%
\pgfpathlineto{\pgfqpoint{2.272474in}{1.243624in}}%
\pgfpathlineto{\pgfqpoint{2.274588in}{1.251664in}}%
\pgfpathlineto{\pgfqpoint{2.276703in}{1.252554in}}%
\pgfpathlineto{\pgfqpoint{2.278818in}{1.260980in}}%
\pgfpathlineto{\pgfqpoint{2.280933in}{1.259128in}}%
\pgfpathlineto{\pgfqpoint{2.283047in}{1.254844in}}%
\pgfpathlineto{\pgfqpoint{2.285162in}{1.258781in}}%
\pgfpathlineto{\pgfqpoint{2.291506in}{1.282409in}}%
\pgfpathlineto{\pgfqpoint{2.293621in}{1.277867in}}%
\pgfpathlineto{\pgfqpoint{2.295735in}{1.276878in}}%
\pgfpathlineto{\pgfqpoint{2.297850in}{1.269418in}}%
\pgfpathlineto{\pgfqpoint{2.299965in}{1.274931in}}%
\pgfpathlineto{\pgfqpoint{2.308424in}{1.264822in}}%
\pgfpathlineto{\pgfqpoint{2.312653in}{1.271647in}}%
\pgfpathlineto{\pgfqpoint{2.316882in}{1.253260in}}%
\pgfpathlineto{\pgfqpoint{2.318997in}{1.251885in}}%
\pgfpathlineto{\pgfqpoint{2.321112in}{1.258926in}}%
\pgfpathlineto{\pgfqpoint{2.323226in}{1.258926in}}%
\pgfpathlineto{\pgfqpoint{2.327456in}{1.253074in}}%
\pgfpathlineto{\pgfqpoint{2.329570in}{1.252200in}}%
\pgfpathlineto{\pgfqpoint{2.331685in}{1.249674in}}%
\pgfpathlineto{\pgfqpoint{2.333800in}{1.252636in}}%
\pgfpathlineto{\pgfqpoint{2.335915in}{1.253575in}}%
\pgfpathlineto{\pgfqpoint{2.338029in}{1.256787in}}%
\pgfpathlineto{\pgfqpoint{2.340144in}{1.252782in}}%
\pgfpathlineto{\pgfqpoint{2.342259in}{1.252070in}}%
\pgfpathlineto{\pgfqpoint{2.346488in}{1.270290in}}%
\pgfpathlineto{\pgfqpoint{2.348603in}{1.265896in}}%
\pgfpathlineto{\pgfqpoint{2.350717in}{1.258865in}}%
\pgfpathlineto{\pgfqpoint{2.352832in}{1.263762in}}%
\pgfpathlineto{\pgfqpoint{2.357061in}{1.265403in}}%
\pgfpathlineto{\pgfqpoint{2.359176in}{1.262172in}}%
\pgfpathlineto{\pgfqpoint{2.363406in}{1.251631in}}%
\pgfpathlineto{\pgfqpoint{2.365520in}{1.249989in}}%
\pgfpathlineto{\pgfqpoint{2.367635in}{1.251138in}}%
\pgfpathlineto{\pgfqpoint{2.369750in}{1.254361in}}%
\pgfpathlineto{\pgfqpoint{2.371864in}{1.252566in}}%
\pgfpathlineto{\pgfqpoint{2.373979in}{1.252932in}}%
\pgfpathlineto{\pgfqpoint{2.376094in}{1.242044in}}%
\pgfpathlineto{\pgfqpoint{2.378208in}{1.239024in}}%
\pgfpathlineto{\pgfqpoint{2.380323in}{1.240825in}}%
\pgfpathlineto{\pgfqpoint{2.382438in}{1.247270in}}%
\pgfpathlineto{\pgfqpoint{2.386667in}{1.239835in}}%
\pgfpathlineto{\pgfqpoint{2.388782in}{1.241746in}}%
\pgfpathlineto{\pgfqpoint{2.390897in}{1.245741in}}%
\pgfpathlineto{\pgfqpoint{2.395126in}{1.244625in}}%
\pgfpathlineto{\pgfqpoint{2.397241in}{1.251890in}}%
\pgfpathlineto{\pgfqpoint{2.401470in}{1.253943in}}%
\pgfpathlineto{\pgfqpoint{2.403585in}{1.244473in}}%
\pgfpathlineto{\pgfqpoint{2.405699in}{1.245886in}}%
\pgfpathlineto{\pgfqpoint{2.407814in}{1.242402in}}%
\pgfpathlineto{\pgfqpoint{2.409929in}{1.246034in}}%
\pgfpathlineto{\pgfqpoint{2.412044in}{1.241979in}}%
\pgfpathlineto{\pgfqpoint{2.414158in}{1.241149in}}%
\pgfpathlineto{\pgfqpoint{2.416273in}{1.238863in}}%
\pgfpathlineto{\pgfqpoint{2.418388in}{1.239865in}}%
\pgfpathlineto{\pgfqpoint{2.422617in}{1.232941in}}%
\pgfpathlineto{\pgfqpoint{2.424732in}{1.233440in}}%
\pgfpathlineto{\pgfqpoint{2.426846in}{1.239070in}}%
\pgfpathlineto{\pgfqpoint{2.428961in}{1.236069in}}%
\pgfpathlineto{\pgfqpoint{2.431076in}{1.236522in}}%
\pgfpathlineto{\pgfqpoint{2.433190in}{1.241324in}}%
\pgfpathlineto{\pgfqpoint{2.439535in}{1.247320in}}%
\pgfpathlineto{\pgfqpoint{2.441649in}{1.247589in}}%
\pgfpathlineto{\pgfqpoint{2.443764in}{1.244489in}}%
\pgfpathlineto{\pgfqpoint{2.447993in}{1.233809in}}%
\pgfpathlineto{\pgfqpoint{2.450108in}{1.235576in}}%
\pgfpathlineto{\pgfqpoint{2.452223in}{1.233352in}}%
\pgfpathlineto{\pgfqpoint{2.454337in}{1.239977in}}%
\pgfpathlineto{\pgfqpoint{2.458567in}{1.235072in}}%
\pgfpathlineto{\pgfqpoint{2.460681in}{1.227548in}}%
\pgfpathlineto{\pgfqpoint{2.467026in}{1.233589in}}%
\pgfpathlineto{\pgfqpoint{2.469140in}{1.235466in}}%
\pgfpathlineto{\pgfqpoint{2.471255in}{1.233008in}}%
\pgfpathlineto{\pgfqpoint{2.473370in}{1.228792in}}%
\pgfpathlineto{\pgfqpoint{2.477599in}{1.238328in}}%
\pgfpathlineto{\pgfqpoint{2.481828in}{1.233103in}}%
\pgfpathlineto{\pgfqpoint{2.486058in}{1.233545in}}%
\pgfpathlineto{\pgfqpoint{2.488172in}{1.230916in}}%
\pgfpathlineto{\pgfqpoint{2.494517in}{1.238769in}}%
\pgfpathlineto{\pgfqpoint{2.496631in}{1.237769in}}%
\pgfpathlineto{\pgfqpoint{2.500861in}{1.248730in}}%
\pgfpathlineto{\pgfqpoint{2.502975in}{1.246631in}}%
\pgfpathlineto{\pgfqpoint{2.505090in}{1.248153in}}%
\pgfpathlineto{\pgfqpoint{2.507205in}{1.247136in}}%
\pgfpathlineto{\pgfqpoint{2.511434in}{1.258940in}}%
\pgfpathlineto{\pgfqpoint{2.513549in}{1.254237in}}%
\pgfpathlineto{\pgfqpoint{2.515664in}{1.256687in}}%
\pgfpathlineto{\pgfqpoint{2.519893in}{1.250129in}}%
\pgfpathlineto{\pgfqpoint{2.524122in}{1.252306in}}%
\pgfpathlineto{\pgfqpoint{2.528352in}{1.243089in}}%
\pgfpathlineto{\pgfqpoint{2.530466in}{1.250279in}}%
\pgfpathlineto{\pgfqpoint{2.534696in}{1.254384in}}%
\pgfpathlineto{\pgfqpoint{2.541040in}{1.284371in}}%
\pgfpathlineto{\pgfqpoint{2.545269in}{1.275334in}}%
\pgfpathlineto{\pgfqpoint{2.547384in}{1.275042in}}%
\pgfpathlineto{\pgfqpoint{2.549499in}{1.270363in}}%
\pgfpathlineto{\pgfqpoint{2.551613in}{1.270659in}}%
\pgfpathlineto{\pgfqpoint{2.553728in}{1.272677in}}%
\pgfpathlineto{\pgfqpoint{2.557957in}{1.289313in}}%
\pgfpathlineto{\pgfqpoint{2.560072in}{1.287155in}}%
\pgfpathlineto{\pgfqpoint{2.562187in}{1.287102in}}%
\pgfpathlineto{\pgfqpoint{2.564301in}{1.284612in}}%
\pgfpathlineto{\pgfqpoint{2.566416in}{1.284613in}}%
\pgfpathlineto{\pgfqpoint{2.568531in}{1.288747in}}%
\pgfpathlineto{\pgfqpoint{2.570646in}{1.287624in}}%
\pgfpathlineto{\pgfqpoint{2.572760in}{1.277406in}}%
\pgfpathlineto{\pgfqpoint{2.576990in}{1.282807in}}%
\pgfpathlineto{\pgfqpoint{2.579104in}{1.279004in}}%
\pgfpathlineto{\pgfqpoint{2.581219in}{1.282184in}}%
\pgfpathlineto{\pgfqpoint{2.585448in}{1.278529in}}%
\pgfpathlineto{\pgfqpoint{2.587563in}{1.278919in}}%
\pgfpathlineto{\pgfqpoint{2.589678in}{1.280936in}}%
\pgfpathlineto{\pgfqpoint{2.593907in}{1.282128in}}%
\pgfpathlineto{\pgfqpoint{2.596022in}{1.280140in}}%
\pgfpathlineto{\pgfqpoint{2.598137in}{1.283252in}}%
\pgfpathlineto{\pgfqpoint{2.600251in}{1.277549in}}%
\pgfpathlineto{\pgfqpoint{2.602366in}{1.275098in}}%
\pgfpathlineto{\pgfqpoint{2.604481in}{1.274603in}}%
\pgfpathlineto{\pgfqpoint{2.606595in}{1.278101in}}%
\pgfpathlineto{\pgfqpoint{2.608710in}{1.274911in}}%
\pgfpathlineto{\pgfqpoint{2.610825in}{1.277961in}}%
\pgfpathlineto{\pgfqpoint{2.612939in}{1.274064in}}%
\pgfpathlineto{\pgfqpoint{2.615054in}{1.277932in}}%
\pgfpathlineto{\pgfqpoint{2.617169in}{1.275826in}}%
\pgfpathlineto{\pgfqpoint{2.619284in}{1.278448in}}%
\pgfpathlineto{\pgfqpoint{2.623513in}{1.269561in}}%
\pgfpathlineto{\pgfqpoint{2.625628in}{1.269802in}}%
\pgfpathlineto{\pgfqpoint{2.627742in}{1.262302in}}%
\pgfpathlineto{\pgfqpoint{2.629857in}{1.266914in}}%
\pgfpathlineto{\pgfqpoint{2.631972in}{1.263359in}}%
\pgfpathlineto{\pgfqpoint{2.634086in}{1.266263in}}%
\pgfpathlineto{\pgfqpoint{2.640430in}{1.265674in}}%
\pgfpathlineto{\pgfqpoint{2.642545in}{1.265050in}}%
\pgfpathlineto{\pgfqpoint{2.646775in}{1.259423in}}%
\pgfpathlineto{\pgfqpoint{2.651004in}{1.264741in}}%
\pgfpathlineto{\pgfqpoint{2.653119in}{1.263737in}}%
\pgfpathlineto{\pgfqpoint{2.655233in}{1.267571in}}%
\pgfpathlineto{\pgfqpoint{2.659463in}{1.261206in}}%
\pgfpathlineto{\pgfqpoint{2.661577in}{1.265417in}}%
\pgfpathlineto{\pgfqpoint{2.663692in}{1.258718in}}%
\pgfpathlineto{\pgfqpoint{2.665807in}{1.256960in}}%
\pgfpathlineto{\pgfqpoint{2.667921in}{1.251493in}}%
\pgfpathlineto{\pgfqpoint{2.670036in}{1.257015in}}%
\pgfpathlineto{\pgfqpoint{2.672151in}{1.255064in}}%
\pgfpathlineto{\pgfqpoint{2.674266in}{1.262148in}}%
\pgfpathlineto{\pgfqpoint{2.676380in}{1.262528in}}%
\pgfpathlineto{\pgfqpoint{2.678495in}{1.258973in}}%
\pgfpathlineto{\pgfqpoint{2.680610in}{1.258787in}}%
\pgfpathlineto{\pgfqpoint{2.684839in}{1.265321in}}%
\pgfpathlineto{\pgfqpoint{2.686954in}{1.261789in}}%
\pgfpathlineto{\pgfqpoint{2.689068in}{1.264613in}}%
\pgfpathlineto{\pgfqpoint{2.691183in}{1.256769in}}%
\pgfpathlineto{\pgfqpoint{2.693298in}{1.255399in}}%
\pgfpathlineto{\pgfqpoint{2.697527in}{1.266526in}}%
\pgfpathlineto{\pgfqpoint{2.699642in}{1.263616in}}%
\pgfpathlineto{\pgfqpoint{2.701757in}{1.266511in}}%
\pgfpathlineto{\pgfqpoint{2.705986in}{1.267010in}}%
\pgfpathlineto{\pgfqpoint{2.708101in}{1.263901in}}%
\pgfpathlineto{\pgfqpoint{2.710215in}{1.265253in}}%
\pgfpathlineto{\pgfqpoint{2.712330in}{1.268315in}}%
\pgfpathlineto{\pgfqpoint{2.714445in}{1.263475in}}%
\pgfpathlineto{\pgfqpoint{2.716559in}{1.268261in}}%
\pgfpathlineto{\pgfqpoint{2.720789in}{1.261037in}}%
\pgfpathlineto{\pgfqpoint{2.722903in}{1.263556in}}%
\pgfpathlineto{\pgfqpoint{2.725018in}{1.255074in}}%
\pgfpathlineto{\pgfqpoint{2.727133in}{1.257069in}}%
\pgfpathlineto{\pgfqpoint{2.729248in}{1.265797in}}%
\pgfpathlineto{\pgfqpoint{2.731362in}{1.264241in}}%
\pgfpathlineto{\pgfqpoint{2.733477in}{1.266890in}}%
\pgfpathlineto{\pgfqpoint{2.735592in}{1.260553in}}%
\pgfpathlineto{\pgfqpoint{2.739821in}{1.262125in}}%
\pgfpathlineto{\pgfqpoint{2.741936in}{1.260760in}}%
\pgfpathlineto{\pgfqpoint{2.744050in}{1.258058in}}%
\pgfpathlineto{\pgfqpoint{2.748280in}{1.266333in}}%
\pgfpathlineto{\pgfqpoint{2.750395in}{1.276246in}}%
\pgfpathlineto{\pgfqpoint{2.752509in}{1.274424in}}%
\pgfpathlineto{\pgfqpoint{2.754624in}{1.280534in}}%
\pgfpathlineto{\pgfqpoint{2.756739in}{1.276254in}}%
\pgfpathlineto{\pgfqpoint{2.760968in}{1.273640in}}%
\pgfpathlineto{\pgfqpoint{2.767312in}{1.268330in}}%
\pgfpathlineto{\pgfqpoint{2.769427in}{1.272200in}}%
\pgfpathlineto{\pgfqpoint{2.771541in}{1.271285in}}%
\pgfpathlineto{\pgfqpoint{2.773656in}{1.273848in}}%
\pgfpathlineto{\pgfqpoint{2.775771in}{1.274370in}}%
\pgfpathlineto{\pgfqpoint{2.777886in}{1.265817in}}%
\pgfpathlineto{\pgfqpoint{2.784230in}{1.263318in}}%
\pgfpathlineto{\pgfqpoint{2.790574in}{1.251908in}}%
\pgfpathlineto{\pgfqpoint{2.794803in}{1.260769in}}%
\pgfpathlineto{\pgfqpoint{2.796918in}{1.259020in}}%
\pgfpathlineto{\pgfqpoint{2.799032in}{1.254086in}}%
\pgfpathlineto{\pgfqpoint{2.805377in}{1.254901in}}%
\pgfpathlineto{\pgfqpoint{2.807491in}{1.260899in}}%
\pgfpathlineto{\pgfqpoint{2.809606in}{1.263093in}}%
\pgfpathlineto{\pgfqpoint{2.811721in}{1.263560in}}%
\pgfpathlineto{\pgfqpoint{2.813835in}{1.262648in}}%
\pgfpathlineto{\pgfqpoint{2.815950in}{1.264023in}}%
\pgfpathlineto{\pgfqpoint{2.818065in}{1.262143in}}%
\pgfpathlineto{\pgfqpoint{2.820179in}{1.267586in}}%
\pgfpathlineto{\pgfqpoint{2.822294in}{1.261303in}}%
\pgfpathlineto{\pgfqpoint{2.824409in}{1.262943in}}%
\pgfpathlineto{\pgfqpoint{2.826523in}{1.262560in}}%
\pgfpathlineto{\pgfqpoint{2.828638in}{1.266990in}}%
\pgfpathlineto{\pgfqpoint{2.830753in}{1.261535in}}%
\pgfpathlineto{\pgfqpoint{2.832868in}{1.252564in}}%
\pgfpathlineto{\pgfqpoint{2.834982in}{1.256725in}}%
\pgfpathlineto{\pgfqpoint{2.837097in}{1.254305in}}%
\pgfpathlineto{\pgfqpoint{2.839212in}{1.248840in}}%
\pgfpathlineto{\pgfqpoint{2.841326in}{1.247358in}}%
\pgfpathlineto{\pgfqpoint{2.843441in}{1.248381in}}%
\pgfpathlineto{\pgfqpoint{2.847670in}{1.257478in}}%
\pgfpathlineto{\pgfqpoint{2.849785in}{1.254420in}}%
\pgfpathlineto{\pgfqpoint{2.854015in}{1.236346in}}%
\pgfpathlineto{\pgfqpoint{2.858244in}{1.246042in}}%
\pgfpathlineto{\pgfqpoint{2.860359in}{1.239407in}}%
\pgfpathlineto{\pgfqpoint{2.862473in}{1.237684in}}%
\pgfpathlineto{\pgfqpoint{2.864588in}{1.237399in}}%
\pgfpathlineto{\pgfqpoint{2.870932in}{1.228625in}}%
\pgfpathlineto{\pgfqpoint{2.873047in}{1.228074in}}%
\pgfpathlineto{\pgfqpoint{2.875161in}{1.229989in}}%
\pgfpathlineto{\pgfqpoint{2.879391in}{1.240703in}}%
\pgfpathlineto{\pgfqpoint{2.883620in}{1.245308in}}%
\pgfpathlineto{\pgfqpoint{2.885735in}{1.249644in}}%
\pgfpathlineto{\pgfqpoint{2.887850in}{1.250956in}}%
\pgfpathlineto{\pgfqpoint{2.889964in}{1.254583in}}%
\pgfpathlineto{\pgfqpoint{2.896308in}{1.253488in}}%
\pgfpathlineto{\pgfqpoint{2.898423in}{1.261250in}}%
\pgfpathlineto{\pgfqpoint{2.902652in}{1.265623in}}%
\pgfpathlineto{\pgfqpoint{2.904767in}{1.262800in}}%
\pgfpathlineto{\pgfqpoint{2.906882in}{1.265638in}}%
\pgfpathlineto{\pgfqpoint{2.908997in}{1.270858in}}%
\pgfpathlineto{\pgfqpoint{2.913226in}{1.271225in}}%
\pgfpathlineto{\pgfqpoint{2.915341in}{1.274872in}}%
\pgfpathlineto{\pgfqpoint{2.919570in}{1.271510in}}%
\pgfpathlineto{\pgfqpoint{2.921685in}{1.272605in}}%
\pgfpathlineto{\pgfqpoint{2.925914in}{1.270462in}}%
\pgfpathlineto{\pgfqpoint{2.930143in}{1.278787in}}%
\pgfpathlineto{\pgfqpoint{2.932258in}{1.277892in}}%
\pgfpathlineto{\pgfqpoint{2.936488in}{1.282660in}}%
\pgfpathlineto{\pgfqpoint{2.938602in}{1.281011in}}%
\pgfpathlineto{\pgfqpoint{2.942832in}{1.290315in}}%
\pgfpathlineto{\pgfqpoint{2.947061in}{1.281215in}}%
\pgfpathlineto{\pgfqpoint{2.949176in}{1.287927in}}%
\pgfpathlineto{\pgfqpoint{2.951290in}{1.287151in}}%
\pgfpathlineto{\pgfqpoint{2.953405in}{1.285086in}}%
\pgfpathlineto{\pgfqpoint{2.959749in}{1.264316in}}%
\pgfpathlineto{\pgfqpoint{2.961864in}{1.268221in}}%
\pgfpathlineto{\pgfqpoint{2.963979in}{1.266957in}}%
\pgfpathlineto{\pgfqpoint{2.970323in}{1.276327in}}%
\pgfpathlineto{\pgfqpoint{2.972437in}{1.268907in}}%
\pgfpathlineto{\pgfqpoint{2.974552in}{1.268424in}}%
\pgfpathlineto{\pgfqpoint{2.978781in}{1.264705in}}%
\pgfpathlineto{\pgfqpoint{2.980896in}{1.264372in}}%
\pgfpathlineto{\pgfqpoint{2.983011in}{1.260531in}}%
\pgfpathlineto{\pgfqpoint{2.985126in}{1.260342in}}%
\pgfpathlineto{\pgfqpoint{2.987240in}{1.264408in}}%
\pgfpathlineto{\pgfqpoint{2.989355in}{1.255325in}}%
\pgfpathlineto{\pgfqpoint{2.991470in}{1.254584in}}%
\pgfpathlineto{\pgfqpoint{2.993584in}{1.257769in}}%
\pgfpathlineto{\pgfqpoint{2.997814in}{1.258683in}}%
\pgfpathlineto{\pgfqpoint{3.002043in}{1.264011in}}%
\pgfpathlineto{\pgfqpoint{3.004158in}{1.260557in}}%
\pgfpathlineto{\pgfqpoint{3.006272in}{1.263571in}}%
\pgfpathlineto{\pgfqpoint{3.008387in}{1.262450in}}%
\pgfpathlineto{\pgfqpoint{3.010502in}{1.270976in}}%
\pgfpathlineto{\pgfqpoint{3.012617in}{1.270819in}}%
\pgfpathlineto{\pgfqpoint{3.014731in}{1.271918in}}%
\pgfpathlineto{\pgfqpoint{3.016846in}{1.277195in}}%
\pgfpathlineto{\pgfqpoint{3.018961in}{1.277548in}}%
\pgfpathlineto{\pgfqpoint{3.021075in}{1.274381in}}%
\pgfpathlineto{\pgfqpoint{3.023190in}{1.273247in}}%
\pgfpathlineto{\pgfqpoint{3.025305in}{1.274976in}}%
\pgfpathlineto{\pgfqpoint{3.027419in}{1.282057in}}%
\pgfpathlineto{\pgfqpoint{3.029534in}{1.277546in}}%
\pgfpathlineto{\pgfqpoint{3.031649in}{1.278537in}}%
\pgfpathlineto{\pgfqpoint{3.033763in}{1.277462in}}%
\pgfpathlineto{\pgfqpoint{3.035878in}{1.269333in}}%
\pgfpathlineto{\pgfqpoint{3.037993in}{1.275740in}}%
\pgfpathlineto{\pgfqpoint{3.040108in}{1.272827in}}%
\pgfpathlineto{\pgfqpoint{3.042222in}{1.273614in}}%
\pgfpathlineto{\pgfqpoint{3.044337in}{1.271398in}}%
\pgfpathlineto{\pgfqpoint{3.048566in}{1.288506in}}%
\pgfpathlineto{\pgfqpoint{3.050681in}{1.290741in}}%
\pgfpathlineto{\pgfqpoint{3.052796in}{1.289232in}}%
\pgfpathlineto{\pgfqpoint{3.057025in}{1.290212in}}%
\pgfpathlineto{\pgfqpoint{3.063369in}{1.273390in}}%
\pgfpathlineto{\pgfqpoint{3.065484in}{1.274431in}}%
\pgfpathlineto{\pgfqpoint{3.069713in}{1.279714in}}%
\pgfpathlineto{\pgfqpoint{3.073943in}{1.264352in}}%
\pgfpathlineto{\pgfqpoint{3.076057in}{1.274590in}}%
\pgfpathlineto{\pgfqpoint{3.078172in}{1.278838in}}%
\pgfpathlineto{\pgfqpoint{3.084516in}{1.282818in}}%
\pgfpathlineto{\pgfqpoint{3.086631in}{1.285660in}}%
\pgfpathlineto{\pgfqpoint{3.088746in}{1.285654in}}%
\pgfpathlineto{\pgfqpoint{3.090860in}{1.282046in}}%
\pgfpathlineto{\pgfqpoint{3.095090in}{1.287246in}}%
\pgfpathlineto{\pgfqpoint{3.099319in}{1.281272in}}%
\pgfpathlineto{\pgfqpoint{3.101434in}{1.282974in}}%
\pgfpathlineto{\pgfqpoint{3.103548in}{1.290863in}}%
\pgfpathlineto{\pgfqpoint{3.105663in}{1.287765in}}%
\pgfpathlineto{\pgfqpoint{3.107778in}{1.281663in}}%
\pgfpathlineto{\pgfqpoint{3.109892in}{1.279475in}}%
\pgfpathlineto{\pgfqpoint{3.112007in}{1.281873in}}%
\pgfpathlineto{\pgfqpoint{3.114122in}{1.282208in}}%
\pgfpathlineto{\pgfqpoint{3.118351in}{1.288679in}}%
\pgfpathlineto{\pgfqpoint{3.120466in}{1.285270in}}%
\pgfpathlineto{\pgfqpoint{3.124695in}{1.288819in}}%
\pgfpathlineto{\pgfqpoint{3.128925in}{1.287310in}}%
\pgfpathlineto{\pgfqpoint{3.131039in}{1.284533in}}%
\pgfpathlineto{\pgfqpoint{3.133154in}{1.276735in}}%
\pgfpathlineto{\pgfqpoint{3.135269in}{1.273779in}}%
\pgfpathlineto{\pgfqpoint{3.137383in}{1.274276in}}%
\pgfpathlineto{\pgfqpoint{3.139498in}{1.269388in}}%
\pgfpathlineto{\pgfqpoint{3.141613in}{1.270055in}}%
\pgfpathlineto{\pgfqpoint{3.143728in}{1.266792in}}%
\pgfpathlineto{\pgfqpoint{3.145842in}{1.261158in}}%
\pgfpathlineto{\pgfqpoint{3.147957in}{1.264275in}}%
\pgfpathlineto{\pgfqpoint{3.150072in}{1.262383in}}%
\pgfpathlineto{\pgfqpoint{3.152186in}{1.263339in}}%
\pgfpathlineto{\pgfqpoint{3.156416in}{1.273712in}}%
\pgfpathlineto{\pgfqpoint{3.158530in}{1.275494in}}%
\pgfpathlineto{\pgfqpoint{3.160645in}{1.274323in}}%
\pgfpathlineto{\pgfqpoint{3.162760in}{1.276194in}}%
\pgfpathlineto{\pgfqpoint{3.164874in}{1.275856in}}%
\pgfpathlineto{\pgfqpoint{3.166989in}{1.277397in}}%
\pgfpathlineto{\pgfqpoint{3.169104in}{1.273398in}}%
\pgfpathlineto{\pgfqpoint{3.171219in}{1.276722in}}%
\pgfpathlineto{\pgfqpoint{3.175448in}{1.273782in}}%
\pgfpathlineto{\pgfqpoint{3.177563in}{1.276461in}}%
\pgfpathlineto{\pgfqpoint{3.179677in}{1.281798in}}%
\pgfpathlineto{\pgfqpoint{3.181792in}{1.281535in}}%
\pgfpathlineto{\pgfqpoint{3.186021in}{1.273221in}}%
\pgfpathlineto{\pgfqpoint{3.188136in}{1.260869in}}%
\pgfpathlineto{\pgfqpoint{3.190251in}{1.263905in}}%
\pgfpathlineto{\pgfqpoint{3.192366in}{1.263126in}}%
\pgfpathlineto{\pgfqpoint{3.194480in}{1.266767in}}%
\pgfpathlineto{\pgfqpoint{3.198710in}{1.269068in}}%
\pgfpathlineto{\pgfqpoint{3.200824in}{1.264373in}}%
\pgfpathlineto{\pgfqpoint{3.202939in}{1.264782in}}%
\pgfpathlineto{\pgfqpoint{3.205054in}{1.270714in}}%
\pgfpathlineto{\pgfqpoint{3.209283in}{1.264423in}}%
\pgfpathlineto{\pgfqpoint{3.211398in}{1.267928in}}%
\pgfpathlineto{\pgfqpoint{3.213512in}{1.266536in}}%
\pgfpathlineto{\pgfqpoint{3.215627in}{1.263662in}}%
\pgfpathlineto{\pgfqpoint{3.217742in}{1.270434in}}%
\pgfpathlineto{\pgfqpoint{3.219857in}{1.256003in}}%
\pgfpathlineto{\pgfqpoint{3.221971in}{1.260154in}}%
\pgfpathlineto{\pgfqpoint{3.224086in}{1.260344in}}%
\pgfpathlineto{\pgfqpoint{3.228315in}{1.272864in}}%
\pgfpathlineto{\pgfqpoint{3.230430in}{1.272876in}}%
\pgfpathlineto{\pgfqpoint{3.232545in}{1.271313in}}%
\pgfpathlineto{\pgfqpoint{3.234659in}{1.279906in}}%
\pgfpathlineto{\pgfqpoint{3.236774in}{1.282189in}}%
\pgfpathlineto{\pgfqpoint{3.238889in}{1.288060in}}%
\pgfpathlineto{\pgfqpoint{3.241003in}{1.285902in}}%
\pgfpathlineto{\pgfqpoint{3.243118in}{1.286168in}}%
\pgfpathlineto{\pgfqpoint{3.245233in}{1.290880in}}%
\pgfpathlineto{\pgfqpoint{3.247348in}{1.291089in}}%
\pgfpathlineto{\pgfqpoint{3.249462in}{1.295597in}}%
\pgfpathlineto{\pgfqpoint{3.251577in}{1.296071in}}%
\pgfpathlineto{\pgfqpoint{3.253692in}{1.294292in}}%
\pgfpathlineto{\pgfqpoint{3.255806in}{1.290529in}}%
\pgfpathlineto{\pgfqpoint{3.257921in}{1.292024in}}%
\pgfpathlineto{\pgfqpoint{3.260036in}{1.289388in}}%
\pgfpathlineto{\pgfqpoint{3.262150in}{1.282871in}}%
\pgfpathlineto{\pgfqpoint{3.264265in}{1.281722in}}%
\pgfpathlineto{\pgfqpoint{3.268494in}{1.263935in}}%
\pgfpathlineto{\pgfqpoint{3.270609in}{1.265094in}}%
\pgfpathlineto{\pgfqpoint{3.272724in}{1.261954in}}%
\pgfpathlineto{\pgfqpoint{3.274839in}{1.262595in}}%
\pgfpathlineto{\pgfqpoint{3.279068in}{1.254424in}}%
\pgfpathlineto{\pgfqpoint{3.281183in}{1.259346in}}%
\pgfpathlineto{\pgfqpoint{3.283297in}{1.258488in}}%
\pgfpathlineto{\pgfqpoint{3.287527in}{1.263261in}}%
\pgfpathlineto{\pgfqpoint{3.289641in}{1.257740in}}%
\pgfpathlineto{\pgfqpoint{3.291756in}{1.255786in}}%
\pgfpathlineto{\pgfqpoint{3.293871in}{1.249183in}}%
\pgfpathlineto{\pgfqpoint{3.295985in}{1.250157in}}%
\pgfpathlineto{\pgfqpoint{3.298100in}{1.248537in}}%
\pgfpathlineto{\pgfqpoint{3.300215in}{1.243700in}}%
\pgfpathlineto{\pgfqpoint{3.302330in}{1.241793in}}%
\pgfpathlineto{\pgfqpoint{3.310788in}{1.250767in}}%
\pgfpathlineto{\pgfqpoint{3.312903in}{1.249915in}}%
\pgfpathlineto{\pgfqpoint{3.315018in}{1.242202in}}%
\pgfpathlineto{\pgfqpoint{3.317132in}{1.243224in}}%
\pgfpathlineto{\pgfqpoint{3.319247in}{1.240120in}}%
\pgfpathlineto{\pgfqpoint{3.321362in}{1.241979in}}%
\pgfpathlineto{\pgfqpoint{3.323477in}{1.237221in}}%
\pgfpathlineto{\pgfqpoint{3.325591in}{1.241333in}}%
\pgfpathlineto{\pgfqpoint{3.327706in}{1.236673in}}%
\pgfpathlineto{\pgfqpoint{3.329821in}{1.235865in}}%
\pgfpathlineto{\pgfqpoint{3.334050in}{1.230677in}}%
\pgfpathlineto{\pgfqpoint{3.336165in}{1.236920in}}%
\pgfpathlineto{\pgfqpoint{3.342509in}{1.221955in}}%
\pgfpathlineto{\pgfqpoint{3.346738in}{1.212878in}}%
\pgfpathlineto{\pgfqpoint{3.348853in}{1.209258in}}%
\pgfpathlineto{\pgfqpoint{3.350968in}{1.210698in}}%
\pgfpathlineto{\pgfqpoint{3.353082in}{1.207902in}}%
\pgfpathlineto{\pgfqpoint{3.355197in}{1.217221in}}%
\pgfpathlineto{\pgfqpoint{3.357312in}{1.221115in}}%
\pgfpathlineto{\pgfqpoint{3.359426in}{1.218549in}}%
\pgfpathlineto{\pgfqpoint{3.365770in}{1.239039in}}%
\pgfpathlineto{\pgfqpoint{3.367885in}{1.240362in}}%
\pgfpathlineto{\pgfqpoint{3.370000in}{1.247191in}}%
\pgfpathlineto{\pgfqpoint{3.372114in}{1.246357in}}%
\pgfpathlineto{\pgfqpoint{3.374229in}{1.244038in}}%
\pgfpathlineto{\pgfqpoint{3.378459in}{1.235414in}}%
\pgfpathlineto{\pgfqpoint{3.380573in}{1.237549in}}%
\pgfpathlineto{\pgfqpoint{3.382688in}{1.245837in}}%
\pgfpathlineto{\pgfqpoint{3.384803in}{1.246784in}}%
\pgfpathlineto{\pgfqpoint{3.386917in}{1.241379in}}%
\pgfpathlineto{\pgfqpoint{3.389032in}{1.241055in}}%
\pgfpathlineto{\pgfqpoint{3.393261in}{1.236267in}}%
\pgfpathlineto{\pgfqpoint{3.395376in}{1.236411in}}%
\pgfpathlineto{\pgfqpoint{3.397491in}{1.238772in}}%
\pgfpathlineto{\pgfqpoint{3.399605in}{1.247333in}}%
\pgfpathlineto{\pgfqpoint{3.401720in}{1.245944in}}%
\pgfpathlineto{\pgfqpoint{3.403835in}{1.246027in}}%
\pgfpathlineto{\pgfqpoint{3.405950in}{1.244126in}}%
\pgfpathlineto{\pgfqpoint{3.408064in}{1.238372in}}%
\pgfpathlineto{\pgfqpoint{3.410179in}{1.236883in}}%
\pgfpathlineto{\pgfqpoint{3.414408in}{1.245796in}}%
\pgfpathlineto{\pgfqpoint{3.416523in}{1.245560in}}%
\pgfpathlineto{\pgfqpoint{3.418638in}{1.239189in}}%
\pgfpathlineto{\pgfqpoint{3.420752in}{1.239276in}}%
\pgfpathlineto{\pgfqpoint{3.422867in}{1.242551in}}%
\pgfpathlineto{\pgfqpoint{3.424982in}{1.243309in}}%
\pgfpathlineto{\pgfqpoint{3.427097in}{1.255623in}}%
\pgfpathlineto{\pgfqpoint{3.433441in}{1.253956in}}%
\pgfpathlineto{\pgfqpoint{3.435555in}{1.254120in}}%
\pgfpathlineto{\pgfqpoint{3.437670in}{1.258483in}}%
\pgfpathlineto{\pgfqpoint{3.439785in}{1.258569in}}%
\pgfpathlineto{\pgfqpoint{3.441899in}{1.260375in}}%
\pgfpathlineto{\pgfqpoint{3.444014in}{1.266980in}}%
\pgfpathlineto{\pgfqpoint{3.450358in}{1.272868in}}%
\pgfpathlineto{\pgfqpoint{3.452473in}{1.266945in}}%
\pgfpathlineto{\pgfqpoint{3.454588in}{1.267500in}}%
\pgfpathlineto{\pgfqpoint{3.456702in}{1.263253in}}%
\pgfpathlineto{\pgfqpoint{3.458817in}{1.254797in}}%
\pgfpathlineto{\pgfqpoint{3.463046in}{1.256030in}}%
\pgfpathlineto{\pgfqpoint{3.465161in}{1.251300in}}%
\pgfpathlineto{\pgfqpoint{3.467276in}{1.251262in}}%
\pgfpathlineto{\pgfqpoint{3.471505in}{1.233212in}}%
\pgfpathlineto{\pgfqpoint{3.473620in}{1.230625in}}%
\pgfpathlineto{\pgfqpoint{3.475734in}{1.233501in}}%
\pgfpathlineto{\pgfqpoint{3.479964in}{1.227382in}}%
\pgfpathlineto{\pgfqpoint{3.482079in}{1.227502in}}%
\pgfpathlineto{\pgfqpoint{3.484193in}{1.225833in}}%
\pgfpathlineto{\pgfqpoint{3.486308in}{1.233542in}}%
\pgfpathlineto{\pgfqpoint{3.488423in}{1.235578in}}%
\pgfpathlineto{\pgfqpoint{3.490537in}{1.233445in}}%
\pgfpathlineto{\pgfqpoint{3.492652in}{1.234805in}}%
\pgfpathlineto{\pgfqpoint{3.496881in}{1.231260in}}%
\pgfpathlineto{\pgfqpoint{3.498996in}{1.229379in}}%
\pgfpathlineto{\pgfqpoint{3.501111in}{1.230004in}}%
\pgfpathlineto{\pgfqpoint{3.503225in}{1.228859in}}%
\pgfpathlineto{\pgfqpoint{3.505340in}{1.232859in}}%
\pgfpathlineto{\pgfqpoint{3.511684in}{1.252094in}}%
\pgfpathlineto{\pgfqpoint{3.513799in}{1.254452in}}%
\pgfpathlineto{\pgfqpoint{3.515914in}{1.253579in}}%
\pgfpathlineto{\pgfqpoint{3.518028in}{1.254490in}}%
\pgfpathlineto{\pgfqpoint{3.520143in}{1.259164in}}%
\pgfpathlineto{\pgfqpoint{3.522258in}{1.253427in}}%
\pgfpathlineto{\pgfqpoint{3.526487in}{1.252451in}}%
\pgfpathlineto{\pgfqpoint{3.528602in}{1.249272in}}%
\pgfpathlineto{\pgfqpoint{3.530716in}{1.248552in}}%
\pgfpathlineto{\pgfqpoint{3.532831in}{1.252573in}}%
\pgfpathlineto{\pgfqpoint{3.537061in}{1.255360in}}%
\pgfpathlineto{\pgfqpoint{3.541290in}{1.248823in}}%
\pgfpathlineto{\pgfqpoint{3.543405in}{1.253888in}}%
\pgfpathlineto{\pgfqpoint{3.549749in}{1.250136in}}%
\pgfpathlineto{\pgfqpoint{3.551863in}{1.246118in}}%
\pgfpathlineto{\pgfqpoint{3.556093in}{1.247668in}}%
\pgfpathlineto{\pgfqpoint{3.560322in}{1.252542in}}%
\pgfpathlineto{\pgfqpoint{3.562437in}{1.245273in}}%
\pgfpathlineto{\pgfqpoint{3.564552in}{1.244518in}}%
\pgfpathlineto{\pgfqpoint{3.566666in}{1.242201in}}%
\pgfpathlineto{\pgfqpoint{3.568781in}{1.244086in}}%
\pgfpathlineto{\pgfqpoint{3.573010in}{1.233618in}}%
\pgfpathlineto{\pgfqpoint{3.575125in}{1.231505in}}%
\pgfpathlineto{\pgfqpoint{3.579354in}{1.236281in}}%
\pgfpathlineto{\pgfqpoint{3.581469in}{1.240080in}}%
\pgfpathlineto{\pgfqpoint{3.583584in}{1.236498in}}%
\pgfpathlineto{\pgfqpoint{3.585699in}{1.236293in}}%
\pgfpathlineto{\pgfqpoint{3.587813in}{1.237384in}}%
\pgfpathlineto{\pgfqpoint{3.589928in}{1.240667in}}%
\pgfpathlineto{\pgfqpoint{3.592043in}{1.235067in}}%
\pgfpathlineto{\pgfqpoint{3.600501in}{1.231035in}}%
\pgfpathlineto{\pgfqpoint{3.602616in}{1.231884in}}%
\pgfpathlineto{\pgfqpoint{3.604731in}{1.235522in}}%
\pgfpathlineto{\pgfqpoint{3.606845in}{1.236026in}}%
\pgfpathlineto{\pgfqpoint{3.611075in}{1.239190in}}%
\pgfpathlineto{\pgfqpoint{3.613190in}{1.237971in}}%
\pgfpathlineto{\pgfqpoint{3.615304in}{1.230747in}}%
\pgfpathlineto{\pgfqpoint{3.619534in}{1.236835in}}%
\pgfpathlineto{\pgfqpoint{3.625878in}{1.230281in}}%
\pgfpathlineto{\pgfqpoint{3.630107in}{1.232003in}}%
\pgfpathlineto{\pgfqpoint{3.632222in}{1.230901in}}%
\pgfpathlineto{\pgfqpoint{3.634336in}{1.234311in}}%
\pgfpathlineto{\pgfqpoint{3.636451in}{1.234359in}}%
\pgfpathlineto{\pgfqpoint{3.642795in}{1.225360in}}%
\pgfpathlineto{\pgfqpoint{3.644910in}{1.229292in}}%
\pgfpathlineto{\pgfqpoint{3.647025in}{1.223526in}}%
\pgfpathlineto{\pgfqpoint{3.649139in}{1.222614in}}%
\pgfpathlineto{\pgfqpoint{3.651254in}{1.224092in}}%
\pgfpathlineto{\pgfqpoint{3.655483in}{1.230726in}}%
\pgfpathlineto{\pgfqpoint{3.659713in}{1.224212in}}%
\pgfpathlineto{\pgfqpoint{3.661828in}{1.221828in}}%
\pgfpathlineto{\pgfqpoint{3.663942in}{1.223968in}}%
\pgfpathlineto{\pgfqpoint{3.666057in}{1.221020in}}%
\pgfpathlineto{\pgfqpoint{3.668172in}{1.223326in}}%
\pgfpathlineto{\pgfqpoint{3.670286in}{1.228075in}}%
\pgfpathlineto{\pgfqpoint{3.674516in}{1.244112in}}%
\pgfpathlineto{\pgfqpoint{3.676630in}{1.246259in}}%
\pgfpathlineto{\pgfqpoint{3.678745in}{1.245884in}}%
\pgfpathlineto{\pgfqpoint{3.682974in}{1.254097in}}%
\pgfpathlineto{\pgfqpoint{3.687204in}{1.262664in}}%
\pgfpathlineto{\pgfqpoint{3.689319in}{1.256663in}}%
\pgfpathlineto{\pgfqpoint{3.693548in}{1.251779in}}%
\pgfpathlineto{\pgfqpoint{3.695663in}{1.256960in}}%
\pgfpathlineto{\pgfqpoint{3.697777in}{1.256531in}}%
\pgfpathlineto{\pgfqpoint{3.702007in}{1.241145in}}%
\pgfpathlineto{\pgfqpoint{3.704121in}{1.246956in}}%
\pgfpathlineto{\pgfqpoint{3.706236in}{1.239864in}}%
\pgfpathlineto{\pgfqpoint{3.714695in}{1.229527in}}%
\pgfpathlineto{\pgfqpoint{3.716810in}{1.221344in}}%
\pgfpathlineto{\pgfqpoint{3.718924in}{1.226175in}}%
\pgfpathlineto{\pgfqpoint{3.721039in}{1.218319in}}%
\pgfpathlineto{\pgfqpoint{3.723154in}{1.220883in}}%
\pgfpathlineto{\pgfqpoint{3.727383in}{1.221638in}}%
\pgfpathlineto{\pgfqpoint{3.733727in}{1.231969in}}%
\pgfpathlineto{\pgfqpoint{3.737956in}{1.246480in}}%
\pgfpathlineto{\pgfqpoint{3.740071in}{1.243404in}}%
\pgfpathlineto{\pgfqpoint{3.746415in}{1.244805in}}%
\pgfpathlineto{\pgfqpoint{3.748530in}{1.241302in}}%
\pgfpathlineto{\pgfqpoint{3.750645in}{1.243963in}}%
\pgfpathlineto{\pgfqpoint{3.752759in}{1.242901in}}%
\pgfpathlineto{\pgfqpoint{3.754874in}{1.236364in}}%
\pgfpathlineto{\pgfqpoint{3.759103in}{1.236912in}}%
\pgfpathlineto{\pgfqpoint{3.763333in}{1.228260in}}%
\pgfpathlineto{\pgfqpoint{3.765447in}{1.230209in}}%
\pgfpathlineto{\pgfqpoint{3.767562in}{1.238698in}}%
\pgfpathlineto{\pgfqpoint{3.769677in}{1.239649in}}%
\pgfpathlineto{\pgfqpoint{3.776021in}{1.228036in}}%
\pgfpathlineto{\pgfqpoint{3.778136in}{1.234525in}}%
\pgfpathlineto{\pgfqpoint{3.780250in}{1.237377in}}%
\pgfpathlineto{\pgfqpoint{3.784480in}{1.228350in}}%
\pgfpathlineto{\pgfqpoint{3.786594in}{1.223633in}}%
\pgfpathlineto{\pgfqpoint{3.790824in}{1.240171in}}%
\pgfpathlineto{\pgfqpoint{3.792939in}{1.241245in}}%
\pgfpathlineto{\pgfqpoint{3.797168in}{1.240537in}}%
\pgfpathlineto{\pgfqpoint{3.803512in}{1.244382in}}%
\pgfpathlineto{\pgfqpoint{3.807741in}{1.240377in}}%
\pgfpathlineto{\pgfqpoint{3.809856in}{1.236139in}}%
\pgfpathlineto{\pgfqpoint{3.811971in}{1.235908in}}%
\pgfpathlineto{\pgfqpoint{3.814085in}{1.232928in}}%
\pgfpathlineto{\pgfqpoint{3.826774in}{1.235946in}}%
\pgfpathlineto{\pgfqpoint{3.828888in}{1.238397in}}%
\pgfpathlineto{\pgfqpoint{3.833118in}{1.235372in}}%
\pgfpathlineto{\pgfqpoint{3.835232in}{1.236564in}}%
\pgfpathlineto{\pgfqpoint{3.837347in}{1.244303in}}%
\pgfpathlineto{\pgfqpoint{3.839462in}{1.241529in}}%
\pgfpathlineto{\pgfqpoint{3.841576in}{1.236569in}}%
\pgfpathlineto{\pgfqpoint{3.843691in}{1.228463in}}%
\pgfpathlineto{\pgfqpoint{3.845806in}{1.227496in}}%
\pgfpathlineto{\pgfqpoint{3.850035in}{1.239390in}}%
\pgfpathlineto{\pgfqpoint{3.852150in}{1.239158in}}%
\pgfpathlineto{\pgfqpoint{3.854265in}{1.235273in}}%
\pgfpathlineto{\pgfqpoint{3.856379in}{1.239851in}}%
\pgfpathlineto{\pgfqpoint{3.858494in}{1.239030in}}%
\pgfpathlineto{\pgfqpoint{3.862723in}{1.243324in}}%
\pgfpathlineto{\pgfqpoint{3.869067in}{1.236264in}}%
\pgfpathlineto{\pgfqpoint{3.873297in}{1.248909in}}%
\pgfpathlineto{\pgfqpoint{3.875412in}{1.245929in}}%
\pgfpathlineto{\pgfqpoint{3.877526in}{1.251033in}}%
\pgfpathlineto{\pgfqpoint{3.879641in}{1.250380in}}%
\pgfpathlineto{\pgfqpoint{3.881756in}{1.254250in}}%
\pgfpathlineto{\pgfqpoint{3.883870in}{1.255460in}}%
\pgfpathlineto{\pgfqpoint{3.885985in}{1.261533in}}%
\pgfpathlineto{\pgfqpoint{3.888100in}{1.259863in}}%
\pgfpathlineto{\pgfqpoint{3.890214in}{1.263480in}}%
\pgfpathlineto{\pgfqpoint{3.892329in}{1.261411in}}%
\pgfpathlineto{\pgfqpoint{3.894444in}{1.264823in}}%
\pgfpathlineto{\pgfqpoint{3.896559in}{1.275114in}}%
\pgfpathlineto{\pgfqpoint{3.905017in}{1.279791in}}%
\pgfpathlineto{\pgfqpoint{3.907132in}{1.278828in}}%
\pgfpathlineto{\pgfqpoint{3.911361in}{1.284469in}}%
\pgfpathlineto{\pgfqpoint{3.915591in}{1.296259in}}%
\pgfpathlineto{\pgfqpoint{3.917705in}{1.294012in}}%
\pgfpathlineto{\pgfqpoint{3.919820in}{1.287802in}}%
\pgfpathlineto{\pgfqpoint{3.921935in}{1.297816in}}%
\pgfpathlineto{\pgfqpoint{3.924050in}{1.299581in}}%
\pgfpathlineto{\pgfqpoint{3.926164in}{1.305321in}}%
\pgfpathlineto{\pgfqpoint{3.930394in}{1.302759in}}%
\pgfpathlineto{\pgfqpoint{3.932508in}{1.303869in}}%
\pgfpathlineto{\pgfqpoint{3.936738in}{1.307577in}}%
\pgfpathlineto{\pgfqpoint{3.938852in}{1.308430in}}%
\pgfpathlineto{\pgfqpoint{3.940967in}{1.305176in}}%
\pgfpathlineto{\pgfqpoint{3.943082in}{1.309054in}}%
\pgfpathlineto{\pgfqpoint{3.945196in}{1.301017in}}%
\pgfpathlineto{\pgfqpoint{3.947311in}{1.303931in}}%
\pgfpathlineto{\pgfqpoint{3.951541in}{1.292696in}}%
\pgfpathlineto{\pgfqpoint{3.953655in}{1.294922in}}%
\pgfpathlineto{\pgfqpoint{3.955770in}{1.299582in}}%
\pgfpathlineto{\pgfqpoint{3.957885in}{1.290878in}}%
\pgfpathlineto{\pgfqpoint{3.959999in}{1.292173in}}%
\pgfpathlineto{\pgfqpoint{3.962114in}{1.295141in}}%
\pgfpathlineto{\pgfqpoint{3.964229in}{1.294172in}}%
\pgfpathlineto{\pgfqpoint{3.970573in}{1.280865in}}%
\pgfpathlineto{\pgfqpoint{3.972687in}{1.278752in}}%
\pgfpathlineto{\pgfqpoint{3.974802in}{1.280591in}}%
\pgfpathlineto{\pgfqpoint{3.976917in}{1.280709in}}%
\pgfpathlineto{\pgfqpoint{3.979032in}{1.275994in}}%
\pgfpathlineto{\pgfqpoint{3.981146in}{1.277495in}}%
\pgfpathlineto{\pgfqpoint{3.983261in}{1.277270in}}%
\pgfpathlineto{\pgfqpoint{3.985376in}{1.273805in}}%
\pgfpathlineto{\pgfqpoint{3.987490in}{1.275276in}}%
\pgfpathlineto{\pgfqpoint{3.989605in}{1.280502in}}%
\pgfpathlineto{\pgfqpoint{3.991720in}{1.275938in}}%
\pgfpathlineto{\pgfqpoint{3.993834in}{1.275965in}}%
\pgfpathlineto{\pgfqpoint{3.995949in}{1.278648in}}%
\pgfpathlineto{\pgfqpoint{3.998064in}{1.276146in}}%
\pgfpathlineto{\pgfqpoint{4.000178in}{1.269184in}}%
\pgfpathlineto{\pgfqpoint{4.002293in}{1.269599in}}%
\pgfpathlineto{\pgfqpoint{4.004408in}{1.263701in}}%
\pgfpathlineto{\pgfqpoint{4.006523in}{1.266199in}}%
\pgfpathlineto{\pgfqpoint{4.008637in}{1.273207in}}%
\pgfpathlineto{\pgfqpoint{4.010752in}{1.275642in}}%
\pgfpathlineto{\pgfqpoint{4.012867in}{1.272831in}}%
\pgfpathlineto{\pgfqpoint{4.014981in}{1.274967in}}%
\pgfpathlineto{\pgfqpoint{4.017096in}{1.273324in}}%
\pgfpathlineto{\pgfqpoint{4.019211in}{1.278759in}}%
\pgfpathlineto{\pgfqpoint{4.021325in}{1.276462in}}%
\pgfpathlineto{\pgfqpoint{4.023440in}{1.278650in}}%
\pgfpathlineto{\pgfqpoint{4.025555in}{1.275676in}}%
\pgfpathlineto{\pgfqpoint{4.027670in}{1.281659in}}%
\pgfpathlineto{\pgfqpoint{4.031899in}{1.283666in}}%
\pgfpathlineto{\pgfqpoint{4.036128in}{1.276908in}}%
\pgfpathlineto{\pgfqpoint{4.038243in}{1.271291in}}%
\pgfpathlineto{\pgfqpoint{4.044587in}{1.283528in}}%
\pgfpathlineto{\pgfqpoint{4.046702in}{1.275296in}}%
\pgfpathlineto{\pgfqpoint{4.048816in}{1.277135in}}%
\pgfpathlineto{\pgfqpoint{4.053046in}{1.269957in}}%
\pgfpathlineto{\pgfqpoint{4.055161in}{1.268503in}}%
\pgfpathlineto{\pgfqpoint{4.057275in}{1.265589in}}%
\pgfpathlineto{\pgfqpoint{4.061505in}{1.268017in}}%
\pgfpathlineto{\pgfqpoint{4.063619in}{1.268037in}}%
\pgfpathlineto{\pgfqpoint{4.067849in}{1.257483in}}%
\pgfpathlineto{\pgfqpoint{4.072078in}{1.249020in}}%
\pgfpathlineto{\pgfqpoint{4.074193in}{1.251019in}}%
\pgfpathlineto{\pgfqpoint{4.076307in}{1.245741in}}%
\pgfpathlineto{\pgfqpoint{4.078422in}{1.247609in}}%
\pgfpathlineto{\pgfqpoint{4.080537in}{1.253155in}}%
\pgfpathlineto{\pgfqpoint{4.082652in}{1.253682in}}%
\pgfpathlineto{\pgfqpoint{4.084766in}{1.248936in}}%
\pgfpathlineto{\pgfqpoint{4.086881in}{1.254939in}}%
\pgfpathlineto{\pgfqpoint{4.088996in}{1.248365in}}%
\pgfpathlineto{\pgfqpoint{4.091110in}{1.248073in}}%
\pgfpathlineto{\pgfqpoint{4.093225in}{1.241880in}}%
\pgfpathlineto{\pgfqpoint{4.095340in}{1.243284in}}%
\pgfpathlineto{\pgfqpoint{4.097454in}{1.241319in}}%
\pgfpathlineto{\pgfqpoint{4.099569in}{1.243943in}}%
\pgfpathlineto{\pgfqpoint{4.105913in}{1.238468in}}%
\pgfpathlineto{\pgfqpoint{4.110143in}{1.240054in}}%
\pgfpathlineto{\pgfqpoint{4.114372in}{1.236489in}}%
\pgfpathlineto{\pgfqpoint{4.116487in}{1.237248in}}%
\pgfpathlineto{\pgfqpoint{4.118601in}{1.243678in}}%
\pgfpathlineto{\pgfqpoint{4.120716in}{1.242834in}}%
\pgfpathlineto{\pgfqpoint{4.122831in}{1.244483in}}%
\pgfpathlineto{\pgfqpoint{4.124945in}{1.241951in}}%
\pgfpathlineto{\pgfqpoint{4.127060in}{1.245396in}}%
\pgfpathlineto{\pgfqpoint{4.131290in}{1.237362in}}%
\pgfpathlineto{\pgfqpoint{4.133404in}{1.239882in}}%
\pgfpathlineto{\pgfqpoint{4.139748in}{1.238253in}}%
\pgfpathlineto{\pgfqpoint{4.143978in}{1.243492in}}%
\pgfpathlineto{\pgfqpoint{4.146092in}{1.242198in}}%
\pgfpathlineto{\pgfqpoint{4.150322in}{1.248664in}}%
\pgfpathlineto{\pgfqpoint{4.152436in}{1.243190in}}%
\pgfpathlineto{\pgfqpoint{4.154551in}{1.241914in}}%
\pgfpathlineto{\pgfqpoint{4.156666in}{1.244888in}}%
\pgfpathlineto{\pgfqpoint{4.158781in}{1.244659in}}%
\pgfpathlineto{\pgfqpoint{4.160895in}{1.239912in}}%
\pgfpathlineto{\pgfqpoint{4.163010in}{1.238182in}}%
\pgfpathlineto{\pgfqpoint{4.167239in}{1.237131in}}%
\pgfpathlineto{\pgfqpoint{4.171469in}{1.249440in}}%
\pgfpathlineto{\pgfqpoint{4.173583in}{1.243952in}}%
\pgfpathlineto{\pgfqpoint{4.175698in}{1.243695in}}%
\pgfpathlineto{\pgfqpoint{4.177813in}{1.246910in}}%
\pgfpathlineto{\pgfqpoint{4.179927in}{1.241944in}}%
\pgfpathlineto{\pgfqpoint{4.182042in}{1.243201in}}%
\pgfpathlineto{\pgfqpoint{4.184157in}{1.242786in}}%
\pgfpathlineto{\pgfqpoint{4.186272in}{1.250833in}}%
\pgfpathlineto{\pgfqpoint{4.188386in}{1.244871in}}%
\pgfpathlineto{\pgfqpoint{4.190501in}{1.252399in}}%
\pgfpathlineto{\pgfqpoint{4.194730in}{1.240970in}}%
\pgfpathlineto{\pgfqpoint{4.198960in}{1.245329in}}%
\pgfpathlineto{\pgfqpoint{4.205304in}{1.236254in}}%
\pgfpathlineto{\pgfqpoint{4.207418in}{1.238267in}}%
\pgfpathlineto{\pgfqpoint{4.209533in}{1.241828in}}%
\pgfpathlineto{\pgfqpoint{4.211648in}{1.236728in}}%
\pgfpathlineto{\pgfqpoint{4.215877in}{1.232510in}}%
\pgfpathlineto{\pgfqpoint{4.220107in}{1.237898in}}%
\pgfpathlineto{\pgfqpoint{4.222221in}{1.234158in}}%
\pgfpathlineto{\pgfqpoint{4.224336in}{1.241756in}}%
\pgfpathlineto{\pgfqpoint{4.226451in}{1.245064in}}%
\pgfpathlineto{\pgfqpoint{4.228565in}{1.241612in}}%
\pgfpathlineto{\pgfqpoint{4.230680in}{1.232153in}}%
\pgfpathlineto{\pgfqpoint{4.232795in}{1.229147in}}%
\pgfpathlineto{\pgfqpoint{4.234910in}{1.231883in}}%
\pgfpathlineto{\pgfqpoint{4.237024in}{1.220221in}}%
\pgfpathlineto{\pgfqpoint{4.241254in}{1.232583in}}%
\pgfpathlineto{\pgfqpoint{4.247598in}{1.224517in}}%
\pgfpathlineto{\pgfqpoint{4.251827in}{1.230649in}}%
\pgfpathlineto{\pgfqpoint{4.253942in}{1.238682in}}%
\pgfpathlineto{\pgfqpoint{4.256056in}{1.240599in}}%
\pgfpathlineto{\pgfqpoint{4.258171in}{1.249141in}}%
\pgfpathlineto{\pgfqpoint{4.260286in}{1.242921in}}%
\pgfpathlineto{\pgfqpoint{4.262401in}{1.243630in}}%
\pgfpathlineto{\pgfqpoint{4.264515in}{1.247905in}}%
\pgfpathlineto{\pgfqpoint{4.266630in}{1.248929in}}%
\pgfpathlineto{\pgfqpoint{4.268745in}{1.245402in}}%
\pgfpathlineto{\pgfqpoint{4.272974in}{1.244996in}}%
\pgfpathlineto{\pgfqpoint{4.275089in}{1.237869in}}%
\pgfpathlineto{\pgfqpoint{4.277203in}{1.245389in}}%
\pgfpathlineto{\pgfqpoint{4.279318in}{1.245777in}}%
\pgfpathlineto{\pgfqpoint{4.281433in}{1.247685in}}%
\pgfpathlineto{\pgfqpoint{4.283547in}{1.244744in}}%
\pgfpathlineto{\pgfqpoint{4.285662in}{1.239721in}}%
\pgfpathlineto{\pgfqpoint{4.287777in}{1.242753in}}%
\pgfpathlineto{\pgfqpoint{4.289892in}{1.242026in}}%
\pgfpathlineto{\pgfqpoint{4.292006in}{1.244067in}}%
\pgfpathlineto{\pgfqpoint{4.294121in}{1.237942in}}%
\pgfpathlineto{\pgfqpoint{4.298350in}{1.239895in}}%
\pgfpathlineto{\pgfqpoint{4.300465in}{1.237685in}}%
\pgfpathlineto{\pgfqpoint{4.302580in}{1.233553in}}%
\pgfpathlineto{\pgfqpoint{4.304694in}{1.234331in}}%
\pgfpathlineto{\pgfqpoint{4.306809in}{1.227472in}}%
\pgfpathlineto{\pgfqpoint{4.311038in}{1.225798in}}%
\pgfpathlineto{\pgfqpoint{4.313153in}{1.226368in}}%
\pgfpathlineto{\pgfqpoint{4.315268in}{1.230905in}}%
\pgfpathlineto{\pgfqpoint{4.317383in}{1.224367in}}%
\pgfpathlineto{\pgfqpoint{4.319497in}{1.226978in}}%
\pgfpathlineto{\pgfqpoint{4.321612in}{1.223974in}}%
\pgfpathlineto{\pgfqpoint{4.323727in}{1.217324in}}%
\pgfpathlineto{\pgfqpoint{4.325841in}{1.215062in}}%
\pgfpathlineto{\pgfqpoint{4.332185in}{1.218812in}}%
\pgfpathlineto{\pgfqpoint{4.334300in}{1.226480in}}%
\pgfpathlineto{\pgfqpoint{4.336415in}{1.226230in}}%
\pgfpathlineto{\pgfqpoint{4.338529in}{1.220248in}}%
\pgfpathlineto{\pgfqpoint{4.340644in}{1.225755in}}%
\pgfpathlineto{\pgfqpoint{4.344874in}{1.221442in}}%
\pgfpathlineto{\pgfqpoint{4.346988in}{1.223418in}}%
\pgfpathlineto{\pgfqpoint{4.349103in}{1.218876in}}%
\pgfpathlineto{\pgfqpoint{4.355447in}{1.213795in}}%
\pgfpathlineto{\pgfqpoint{4.357562in}{1.220227in}}%
\pgfpathlineto{\pgfqpoint{4.359676in}{1.216977in}}%
\pgfpathlineto{\pgfqpoint{4.363906in}{1.223486in}}%
\pgfpathlineto{\pgfqpoint{4.366021in}{1.215560in}}%
\pgfpathlineto{\pgfqpoint{4.368135in}{1.222564in}}%
\pgfpathlineto{\pgfqpoint{4.370250in}{1.215815in}}%
\pgfpathlineto{\pgfqpoint{4.372365in}{1.223353in}}%
\pgfpathlineto{\pgfqpoint{4.374479in}{1.218313in}}%
\pgfpathlineto{\pgfqpoint{4.378709in}{1.224715in}}%
\pgfpathlineto{\pgfqpoint{4.380823in}{1.235424in}}%
\pgfpathlineto{\pgfqpoint{4.382938in}{1.230331in}}%
\pgfpathlineto{\pgfqpoint{4.385053in}{1.236357in}}%
\pgfpathlineto{\pgfqpoint{4.387167in}{1.233474in}}%
\pgfpathlineto{\pgfqpoint{4.389282in}{1.234583in}}%
\pgfpathlineto{\pgfqpoint{4.393512in}{1.232915in}}%
\pgfpathlineto{\pgfqpoint{4.395626in}{1.226448in}}%
\pgfpathlineto{\pgfqpoint{4.397741in}{1.228638in}}%
\pgfpathlineto{\pgfqpoint{4.399856in}{1.232990in}}%
\pgfpathlineto{\pgfqpoint{4.401970in}{1.242497in}}%
\pgfpathlineto{\pgfqpoint{4.404085in}{1.238842in}}%
\pgfpathlineto{\pgfqpoint{4.406200in}{1.237653in}}%
\pgfpathlineto{\pgfqpoint{4.410429in}{1.249302in}}%
\pgfpathlineto{\pgfqpoint{4.412544in}{1.247545in}}%
\pgfpathlineto{\pgfqpoint{4.414658in}{1.248305in}}%
\pgfpathlineto{\pgfqpoint{4.416773in}{1.250422in}}%
\pgfpathlineto{\pgfqpoint{4.418888in}{1.250196in}}%
\pgfpathlineto{\pgfqpoint{4.421003in}{1.247183in}}%
\pgfpathlineto{\pgfqpoint{4.423117in}{1.252796in}}%
\pgfpathlineto{\pgfqpoint{4.425232in}{1.253005in}}%
\pgfpathlineto{\pgfqpoint{4.427347in}{1.251095in}}%
\pgfpathlineto{\pgfqpoint{4.429461in}{1.254729in}}%
\pgfpathlineto{\pgfqpoint{4.433691in}{1.246284in}}%
\pgfpathlineto{\pgfqpoint{4.435805in}{1.243128in}}%
\pgfpathlineto{\pgfqpoint{4.437920in}{1.244765in}}%
\pgfpathlineto{\pgfqpoint{4.440035in}{1.250515in}}%
\pgfpathlineto{\pgfqpoint{4.442149in}{1.250710in}}%
\pgfpathlineto{\pgfqpoint{4.444264in}{1.253364in}}%
\pgfpathlineto{\pgfqpoint{4.446379in}{1.253278in}}%
\pgfpathlineto{\pgfqpoint{4.450608in}{1.265479in}}%
\pgfpathlineto{\pgfqpoint{4.452723in}{1.265082in}}%
\pgfpathlineto{\pgfqpoint{4.454838in}{1.266927in}}%
\pgfpathlineto{\pgfqpoint{4.456952in}{1.261136in}}%
\pgfpathlineto{\pgfqpoint{4.459067in}{1.269428in}}%
\pgfpathlineto{\pgfqpoint{4.461182in}{1.273010in}}%
\pgfpathlineto{\pgfqpoint{4.465411in}{1.266706in}}%
\pgfpathlineto{\pgfqpoint{4.467526in}{1.258309in}}%
\pgfpathlineto{\pgfqpoint{4.469641in}{1.254837in}}%
\pgfpathlineto{\pgfqpoint{4.471755in}{1.256517in}}%
\pgfpathlineto{\pgfqpoint{4.473870in}{1.254551in}}%
\pgfpathlineto{\pgfqpoint{4.475985in}{1.258059in}}%
\pgfpathlineto{\pgfqpoint{4.478099in}{1.257001in}}%
\pgfpathlineto{\pgfqpoint{4.480214in}{1.258338in}}%
\pgfpathlineto{\pgfqpoint{4.482329in}{1.252836in}}%
\pgfpathlineto{\pgfqpoint{4.484443in}{1.252077in}}%
\pgfpathlineto{\pgfqpoint{4.490787in}{1.260604in}}%
\pgfpathlineto{\pgfqpoint{4.492902in}{1.256139in}}%
\pgfpathlineto{\pgfqpoint{4.495017in}{1.254961in}}%
\pgfpathlineto{\pgfqpoint{4.497132in}{1.251824in}}%
\pgfpathlineto{\pgfqpoint{4.499246in}{1.259670in}}%
\pgfpathlineto{\pgfqpoint{4.501361in}{1.256379in}}%
\pgfpathlineto{\pgfqpoint{4.503476in}{1.263127in}}%
\pgfpathlineto{\pgfqpoint{4.505590in}{1.262856in}}%
\pgfpathlineto{\pgfqpoint{4.507705in}{1.260320in}}%
\pgfpathlineto{\pgfqpoint{4.509820in}{1.261321in}}%
\pgfpathlineto{\pgfqpoint{4.511934in}{1.265346in}}%
\pgfpathlineto{\pgfqpoint{4.516164in}{1.265708in}}%
\pgfpathlineto{\pgfqpoint{4.518278in}{1.259763in}}%
\pgfpathlineto{\pgfqpoint{4.520393in}{1.262878in}}%
\pgfpathlineto{\pgfqpoint{4.522508in}{1.270588in}}%
\pgfpathlineto{\pgfqpoint{4.526737in}{1.264458in}}%
\pgfpathlineto{\pgfqpoint{4.528852in}{1.267638in}}%
\pgfpathlineto{\pgfqpoint{4.535196in}{1.257045in}}%
\pgfpathlineto{\pgfqpoint{4.539425in}{1.266984in}}%
\pgfpathlineto{\pgfqpoint{4.541540in}{1.265994in}}%
\pgfpathlineto{\pgfqpoint{4.543655in}{1.266526in}}%
\pgfpathlineto{\pgfqpoint{4.545769in}{1.261566in}}%
\pgfpathlineto{\pgfqpoint{4.549999in}{1.269834in}}%
\pgfpathlineto{\pgfqpoint{4.552114in}{1.264447in}}%
\pgfpathlineto{\pgfqpoint{4.554228in}{1.272691in}}%
\pgfpathlineto{\pgfqpoint{4.556343in}{1.273164in}}%
\pgfpathlineto{\pgfqpoint{4.560572in}{1.271288in}}%
\pgfpathlineto{\pgfqpoint{4.562687in}{1.268895in}}%
\pgfpathlineto{\pgfqpoint{4.566916in}{1.279777in}}%
\pgfpathlineto{\pgfqpoint{4.569031in}{1.278822in}}%
\pgfpathlineto{\pgfqpoint{4.571146in}{1.283801in}}%
\pgfpathlineto{\pgfqpoint{4.573260in}{1.280268in}}%
\pgfpathlineto{\pgfqpoint{4.575375in}{1.279623in}}%
\pgfpathlineto{\pgfqpoint{4.577490in}{1.277493in}}%
\pgfpathlineto{\pgfqpoint{4.579605in}{1.273095in}}%
\pgfpathlineto{\pgfqpoint{4.581719in}{1.282643in}}%
\pgfpathlineto{\pgfqpoint{4.585949in}{1.280182in}}%
\pgfpathlineto{\pgfqpoint{4.588063in}{1.275547in}}%
\pgfpathlineto{\pgfqpoint{4.592293in}{1.274683in}}%
\pgfpathlineto{\pgfqpoint{4.594407in}{1.270149in}}%
\pgfpathlineto{\pgfqpoint{4.596522in}{1.268720in}}%
\pgfpathlineto{\pgfqpoint{4.598637in}{1.259303in}}%
\pgfpathlineto{\pgfqpoint{4.600752in}{1.260866in}}%
\pgfpathlineto{\pgfqpoint{4.602866in}{1.266624in}}%
\pgfpathlineto{\pgfqpoint{4.604981in}{1.265165in}}%
\pgfpathlineto{\pgfqpoint{4.607096in}{1.257147in}}%
\pgfpathlineto{\pgfqpoint{4.609210in}{1.254783in}}%
\pgfpathlineto{\pgfqpoint{4.611325in}{1.248148in}}%
\pgfpathlineto{\pgfqpoint{4.615554in}{1.244675in}}%
\pgfpathlineto{\pgfqpoint{4.617669in}{1.246419in}}%
\pgfpathlineto{\pgfqpoint{4.619784in}{1.246259in}}%
\pgfpathlineto{\pgfqpoint{4.621898in}{1.248018in}}%
\pgfpathlineto{\pgfqpoint{4.624013in}{1.245422in}}%
\pgfpathlineto{\pgfqpoint{4.630357in}{1.255287in}}%
\pgfpathlineto{\pgfqpoint{4.632472in}{1.252645in}}%
\pgfpathlineto{\pgfqpoint{4.634587in}{1.263170in}}%
\pgfpathlineto{\pgfqpoint{4.636701in}{1.260780in}}%
\pgfpathlineto{\pgfqpoint{4.638816in}{1.271198in}}%
\pgfpathlineto{\pgfqpoint{4.645160in}{1.272477in}}%
\pgfpathlineto{\pgfqpoint{4.651504in}{1.281577in}}%
\pgfpathlineto{\pgfqpoint{4.655734in}{1.273735in}}%
\pgfpathlineto{\pgfqpoint{4.659963in}{1.286517in}}%
\pgfpathlineto{\pgfqpoint{4.662078in}{1.286128in}}%
\pgfpathlineto{\pgfqpoint{4.664192in}{1.290633in}}%
\pgfpathlineto{\pgfqpoint{4.666307in}{1.291466in}}%
\pgfpathlineto{\pgfqpoint{4.670536in}{1.274714in}}%
\pgfpathlineto{\pgfqpoint{4.672651in}{1.281771in}}%
\pgfpathlineto{\pgfqpoint{4.674766in}{1.280934in}}%
\pgfpathlineto{\pgfqpoint{4.676880in}{1.276253in}}%
\pgfpathlineto{\pgfqpoint{4.678995in}{1.278539in}}%
\pgfpathlineto{\pgfqpoint{4.683225in}{1.279477in}}%
\pgfpathlineto{\pgfqpoint{4.687454in}{1.283348in}}%
\pgfpathlineto{\pgfqpoint{4.691683in}{1.274778in}}%
\pgfpathlineto{\pgfqpoint{4.693798in}{1.273067in}}%
\pgfpathlineto{\pgfqpoint{4.695913in}{1.275974in}}%
\pgfpathlineto{\pgfqpoint{4.698027in}{1.275838in}}%
\pgfpathlineto{\pgfqpoint{4.700142in}{1.281286in}}%
\pgfpathlineto{\pgfqpoint{4.702257in}{1.278155in}}%
\pgfpathlineto{\pgfqpoint{4.704372in}{1.277497in}}%
\pgfpathlineto{\pgfqpoint{4.706486in}{1.280030in}}%
\pgfpathlineto{\pgfqpoint{4.708601in}{1.280099in}}%
\pgfpathlineto{\pgfqpoint{4.712830in}{1.286879in}}%
\pgfpathlineto{\pgfqpoint{4.714945in}{1.286058in}}%
\pgfpathlineto{\pgfqpoint{4.717060in}{1.288237in}}%
\pgfpathlineto{\pgfqpoint{4.719174in}{1.294188in}}%
\pgfpathlineto{\pgfqpoint{4.721289in}{1.293336in}}%
\pgfpathlineto{\pgfqpoint{4.723404in}{1.295642in}}%
\pgfpathlineto{\pgfqpoint{4.725518in}{1.295392in}}%
\pgfpathlineto{\pgfqpoint{4.727633in}{1.293357in}}%
\pgfpathlineto{\pgfqpoint{4.729748in}{1.288979in}}%
\pgfpathlineto{\pgfqpoint{4.731863in}{1.290952in}}%
\pgfpathlineto{\pgfqpoint{4.733977in}{1.283169in}}%
\pgfpathlineto{\pgfqpoint{4.736092in}{1.286383in}}%
\pgfpathlineto{\pgfqpoint{4.740321in}{1.278915in}}%
\pgfpathlineto{\pgfqpoint{4.742436in}{1.276357in}}%
\pgfpathlineto{\pgfqpoint{4.746665in}{1.283364in}}%
\pgfpathlineto{\pgfqpoint{4.750895in}{1.278331in}}%
\pgfpathlineto{\pgfqpoint{4.753009in}{1.284605in}}%
\pgfpathlineto{\pgfqpoint{4.755124in}{1.285215in}}%
\pgfpathlineto{\pgfqpoint{4.757239in}{1.279620in}}%
\pgfpathlineto{\pgfqpoint{4.759354in}{1.283140in}}%
\pgfpathlineto{\pgfqpoint{4.761468in}{1.291159in}}%
\pgfpathlineto{\pgfqpoint{4.765698in}{1.294320in}}%
\pgfpathlineto{\pgfqpoint{4.769927in}{1.308553in}}%
\pgfpathlineto{\pgfqpoint{4.772042in}{1.308804in}}%
\pgfpathlineto{\pgfqpoint{4.774156in}{1.306521in}}%
\pgfpathlineto{\pgfqpoint{4.776271in}{1.301321in}}%
\pgfpathlineto{\pgfqpoint{4.778386in}{1.292613in}}%
\pgfpathlineto{\pgfqpoint{4.780500in}{1.291774in}}%
\pgfpathlineto{\pgfqpoint{4.782615in}{1.293858in}}%
\pgfpathlineto{\pgfqpoint{4.784730in}{1.292608in}}%
\pgfpathlineto{\pgfqpoint{4.786845in}{1.285482in}}%
\pgfpathlineto{\pgfqpoint{4.788959in}{1.285685in}}%
\pgfpathlineto{\pgfqpoint{4.791074in}{1.284659in}}%
\pgfpathlineto{\pgfqpoint{4.793189in}{1.286601in}}%
\pgfpathlineto{\pgfqpoint{4.799533in}{1.276330in}}%
\pgfpathlineto{\pgfqpoint{4.801647in}{1.276641in}}%
\pgfpathlineto{\pgfqpoint{4.803762in}{1.282811in}}%
\pgfpathlineto{\pgfqpoint{4.807991in}{1.288974in}}%
\pgfpathlineto{\pgfqpoint{4.810106in}{1.284186in}}%
\pgfpathlineto{\pgfqpoint{4.812221in}{1.282262in}}%
\pgfpathlineto{\pgfqpoint{4.814336in}{1.275999in}}%
\pgfpathlineto{\pgfqpoint{4.816450in}{1.284770in}}%
\pgfpathlineto{\pgfqpoint{4.818565in}{1.283134in}}%
\pgfpathlineto{\pgfqpoint{4.824909in}{1.297060in}}%
\pgfpathlineto{\pgfqpoint{4.827024in}{1.292732in}}%
\pgfpathlineto{\pgfqpoint{4.829138in}{1.292510in}}%
\pgfpathlineto{\pgfqpoint{4.831253in}{1.300465in}}%
\pgfpathlineto{\pgfqpoint{4.835483in}{1.302173in}}%
\pgfpathlineto{\pgfqpoint{4.839712in}{1.289610in}}%
\pgfpathlineto{\pgfqpoint{4.841827in}{1.290225in}}%
\pgfpathlineto{\pgfqpoint{4.843941in}{1.299112in}}%
\pgfpathlineto{\pgfqpoint{4.846056in}{1.291652in}}%
\pgfpathlineto{\pgfqpoint{4.848171in}{1.293980in}}%
\pgfpathlineto{\pgfqpoint{4.850285in}{1.292460in}}%
\pgfpathlineto{\pgfqpoint{4.852400in}{1.296169in}}%
\pgfpathlineto{\pgfqpoint{4.856629in}{1.290666in}}%
\pgfpathlineto{\pgfqpoint{4.862974in}{1.293102in}}%
\pgfpathlineto{\pgfqpoint{4.865088in}{1.283232in}}%
\pgfpathlineto{\pgfqpoint{4.867203in}{1.284322in}}%
\pgfpathlineto{\pgfqpoint{4.871432in}{1.277085in}}%
\pgfpathlineto{\pgfqpoint{4.875662in}{1.280079in}}%
\pgfpathlineto{\pgfqpoint{4.877776in}{1.278314in}}%
\pgfpathlineto{\pgfqpoint{4.886235in}{1.291658in}}%
\pgfpathlineto{\pgfqpoint{4.890465in}{1.292832in}}%
\pgfpathlineto{\pgfqpoint{4.892579in}{1.297011in}}%
\pgfpathlineto{\pgfqpoint{4.896809in}{1.309439in}}%
\pgfpathlineto{\pgfqpoint{4.898923in}{1.304821in}}%
\pgfpathlineto{\pgfqpoint{4.901038in}{1.306601in}}%
\pgfpathlineto{\pgfqpoint{4.903153in}{1.312483in}}%
\pgfpathlineto{\pgfqpoint{4.905267in}{1.308419in}}%
\pgfpathlineto{\pgfqpoint{4.907382in}{1.313177in}}%
\pgfpathlineto{\pgfqpoint{4.909497in}{1.310934in}}%
\pgfpathlineto{\pgfqpoint{4.913726in}{1.318922in}}%
\pgfpathlineto{\pgfqpoint{4.915841in}{1.319939in}}%
\pgfpathlineto{\pgfqpoint{4.917956in}{1.317001in}}%
\pgfpathlineto{\pgfqpoint{4.920070in}{1.317375in}}%
\pgfpathlineto{\pgfqpoint{4.922185in}{1.321489in}}%
\pgfpathlineto{\pgfqpoint{4.924300in}{1.319235in}}%
\pgfpathlineto{\pgfqpoint{4.926414in}{1.319303in}}%
\pgfpathlineto{\pgfqpoint{4.928529in}{1.313153in}}%
\pgfpathlineto{\pgfqpoint{4.930644in}{1.315732in}}%
\pgfpathlineto{\pgfqpoint{4.932758in}{1.313407in}}%
\pgfpathlineto{\pgfqpoint{4.934873in}{1.315054in}}%
\pgfpathlineto{\pgfqpoint{4.936988in}{1.312767in}}%
\pgfpathlineto{\pgfqpoint{4.939103in}{1.312494in}}%
\pgfpathlineto{\pgfqpoint{4.941217in}{1.305718in}}%
\pgfpathlineto{\pgfqpoint{4.945447in}{1.305824in}}%
\pgfpathlineto{\pgfqpoint{4.951791in}{1.311960in}}%
\pgfpathlineto{\pgfqpoint{4.953905in}{1.308128in}}%
\pgfpathlineto{\pgfqpoint{4.956020in}{1.312257in}}%
\pgfpathlineto{\pgfqpoint{4.958135in}{1.311289in}}%
\pgfpathlineto{\pgfqpoint{4.960249in}{1.319497in}}%
\pgfpathlineto{\pgfqpoint{4.962364in}{1.313761in}}%
\pgfpathlineto{\pgfqpoint{4.964479in}{1.316952in}}%
\pgfpathlineto{\pgfqpoint{4.966594in}{1.323056in}}%
\pgfpathlineto{\pgfqpoint{4.968708in}{1.322802in}}%
\pgfpathlineto{\pgfqpoint{4.970823in}{1.325697in}}%
\pgfpathlineto{\pgfqpoint{4.972938in}{1.323763in}}%
\pgfpathlineto{\pgfqpoint{4.979282in}{1.331690in}}%
\pgfpathlineto{\pgfqpoint{4.981396in}{1.327491in}}%
\pgfpathlineto{\pgfqpoint{4.983511in}{1.329218in}}%
\pgfpathlineto{\pgfqpoint{4.987740in}{1.340966in}}%
\pgfpathlineto{\pgfqpoint{4.989855in}{1.340721in}}%
\pgfpathlineto{\pgfqpoint{4.991970in}{1.337271in}}%
\pgfpathlineto{\pgfqpoint{4.994085in}{1.339105in}}%
\pgfpathlineto{\pgfqpoint{4.996199in}{1.332374in}}%
\pgfpathlineto{\pgfqpoint{4.998314in}{1.337673in}}%
\pgfpathlineto{\pgfqpoint{5.002543in}{1.329329in}}%
\pgfpathlineto{\pgfqpoint{5.004658in}{1.321270in}}%
\pgfpathlineto{\pgfqpoint{5.006773in}{1.322359in}}%
\pgfpathlineto{\pgfqpoint{5.008887in}{1.314073in}}%
\pgfpathlineto{\pgfqpoint{5.011002in}{1.311451in}}%
\pgfpathlineto{\pgfqpoint{5.013117in}{1.306748in}}%
\pgfpathlineto{\pgfqpoint{5.015231in}{1.309637in}}%
\pgfpathlineto{\pgfqpoint{5.017346in}{1.318705in}}%
\pgfpathlineto{\pgfqpoint{5.019461in}{1.316107in}}%
\pgfpathlineto{\pgfqpoint{5.021576in}{1.318451in}}%
\pgfpathlineto{\pgfqpoint{5.023690in}{1.322624in}}%
\pgfpathlineto{\pgfqpoint{5.025805in}{1.330397in}}%
\pgfpathlineto{\pgfqpoint{5.027920in}{1.333849in}}%
\pgfpathlineto{\pgfqpoint{5.030034in}{1.331636in}}%
\pgfpathlineto{\pgfqpoint{5.036378in}{1.330548in}}%
\pgfpathlineto{\pgfqpoint{5.042722in}{1.318907in}}%
\pgfpathlineto{\pgfqpoint{5.044837in}{1.319482in}}%
\pgfpathlineto{\pgfqpoint{5.046952in}{1.318694in}}%
\pgfpathlineto{\pgfqpoint{5.051181in}{1.326642in}}%
\pgfpathlineto{\pgfqpoint{5.053296in}{1.327285in}}%
\pgfpathlineto{\pgfqpoint{5.057525in}{1.334114in}}%
\pgfpathlineto{\pgfqpoint{5.059640in}{1.333679in}}%
\pgfpathlineto{\pgfqpoint{5.061755in}{1.342709in}}%
\pgfpathlineto{\pgfqpoint{5.063869in}{1.339123in}}%
\pgfpathlineto{\pgfqpoint{5.065984in}{1.345721in}}%
\pgfpathlineto{\pgfqpoint{5.068099in}{1.338719in}}%
\pgfpathlineto{\pgfqpoint{5.070214in}{1.339855in}}%
\pgfpathlineto{\pgfqpoint{5.072328in}{1.344174in}}%
\pgfpathlineto{\pgfqpoint{5.074443in}{1.344417in}}%
\pgfpathlineto{\pgfqpoint{5.076558in}{1.339778in}}%
\pgfpathlineto{\pgfqpoint{5.078672in}{1.342172in}}%
\pgfpathlineto{\pgfqpoint{5.080787in}{1.347121in}}%
\pgfpathlineto{\pgfqpoint{5.082902in}{1.342543in}}%
\pgfpathlineto{\pgfqpoint{5.087131in}{1.348138in}}%
\pgfpathlineto{\pgfqpoint{5.089246in}{1.347635in}}%
\pgfpathlineto{\pgfqpoint{5.091360in}{1.348781in}}%
\pgfpathlineto{\pgfqpoint{5.093475in}{1.337164in}}%
\pgfpathlineto{\pgfqpoint{5.095590in}{1.333690in}}%
\pgfpathlineto{\pgfqpoint{5.097705in}{1.334500in}}%
\pgfpathlineto{\pgfqpoint{5.101934in}{1.332999in}}%
\pgfpathlineto{\pgfqpoint{5.106163in}{1.323298in}}%
\pgfpathlineto{\pgfqpoint{5.108278in}{1.327450in}}%
\pgfpathlineto{\pgfqpoint{5.110393in}{1.325739in}}%
\pgfpathlineto{\pgfqpoint{5.112507in}{1.327713in}}%
\pgfpathlineto{\pgfqpoint{5.120966in}{1.316846in}}%
\pgfpathlineto{\pgfqpoint{5.123081in}{1.327121in}}%
\pgfpathlineto{\pgfqpoint{5.125196in}{1.324133in}}%
\pgfpathlineto{\pgfqpoint{5.127310in}{1.328859in}}%
\pgfpathlineto{\pgfqpoint{5.129425in}{1.328976in}}%
\pgfpathlineto{\pgfqpoint{5.133654in}{1.334165in}}%
\pgfpathlineto{\pgfqpoint{5.135769in}{1.337715in}}%
\pgfpathlineto{\pgfqpoint{5.139998in}{1.335039in}}%
\pgfpathlineto{\pgfqpoint{5.142113in}{1.337017in}}%
\pgfpathlineto{\pgfqpoint{5.144228in}{1.342024in}}%
\pgfpathlineto{\pgfqpoint{5.146342in}{1.340550in}}%
\pgfpathlineto{\pgfqpoint{5.150572in}{1.344606in}}%
\pgfpathlineto{\pgfqpoint{5.161145in}{1.356755in}}%
\pgfpathlineto{\pgfqpoint{5.163260in}{1.354216in}}%
\pgfpathlineto{\pgfqpoint{5.167489in}{1.342325in}}%
\pgfpathlineto{\pgfqpoint{5.169604in}{1.346635in}}%
\pgfpathlineto{\pgfqpoint{5.171719in}{1.345038in}}%
\pgfpathlineto{\pgfqpoint{5.178063in}{1.345914in}}%
\pgfpathlineto{\pgfqpoint{5.180178in}{1.350908in}}%
\pgfpathlineto{\pgfqpoint{5.184407in}{1.343359in}}%
\pgfpathlineto{\pgfqpoint{5.186522in}{1.339652in}}%
\pgfpathlineto{\pgfqpoint{5.188636in}{1.343482in}}%
\pgfpathlineto{\pgfqpoint{5.188636in}{1.343482in}}%
\pgfusepath{stroke}%
\end{pgfscope}%
\begin{pgfscope}%
\pgfpathrectangle{\pgfqpoint{0.750000in}{0.275000in}}{\pgfqpoint{4.650000in}{1.925000in}}%
\pgfusepath{clip}%
\pgfsetroundcap%
\pgfsetroundjoin%
\pgfsetlinewidth{1.003750pt}%
\definecolor{currentstroke}{rgb}{0.596078,0.305882,0.639216}%
\pgfsetstrokecolor{currentstroke}%
\pgfsetdash{}{0pt}%
\pgfpathmoveto{\pgfqpoint{0.961364in}{1.271153in}}%
\pgfpathlineto{\pgfqpoint{0.963478in}{1.280602in}}%
\pgfpathlineto{\pgfqpoint{0.965593in}{1.276823in}}%
\pgfpathlineto{\pgfqpoint{0.967708in}{1.279005in}}%
\pgfpathlineto{\pgfqpoint{0.969822in}{1.283762in}}%
\pgfpathlineto{\pgfqpoint{0.971937in}{1.282643in}}%
\pgfpathlineto{\pgfqpoint{0.974052in}{1.290317in}}%
\pgfpathlineto{\pgfqpoint{0.976166in}{1.292596in}}%
\pgfpathlineto{\pgfqpoint{0.978281in}{1.288869in}}%
\pgfpathlineto{\pgfqpoint{0.980396in}{1.289275in}}%
\pgfpathlineto{\pgfqpoint{0.982511in}{1.284903in}}%
\pgfpathlineto{\pgfqpoint{0.984625in}{1.287137in}}%
\pgfpathlineto{\pgfqpoint{0.986740in}{1.283610in}}%
\pgfpathlineto{\pgfqpoint{0.988855in}{1.284000in}}%
\pgfpathlineto{\pgfqpoint{0.990969in}{1.286458in}}%
\pgfpathlineto{\pgfqpoint{0.993084in}{1.291925in}}%
\pgfpathlineto{\pgfqpoint{0.997313in}{1.288027in}}%
\pgfpathlineto{\pgfqpoint{0.999428in}{1.276384in}}%
\pgfpathlineto{\pgfqpoint{1.001543in}{1.282139in}}%
\pgfpathlineto{\pgfqpoint{1.005772in}{1.281232in}}%
\pgfpathlineto{\pgfqpoint{1.012116in}{1.273784in}}%
\pgfpathlineto{\pgfqpoint{1.014231in}{1.281518in}}%
\pgfpathlineto{\pgfqpoint{1.016346in}{1.279555in}}%
\pgfpathlineto{\pgfqpoint{1.020575in}{1.293206in}}%
\pgfpathlineto{\pgfqpoint{1.022690in}{1.292731in}}%
\pgfpathlineto{\pgfqpoint{1.024804in}{1.290690in}}%
\pgfpathlineto{\pgfqpoint{1.029034in}{1.283161in}}%
\pgfpathlineto{\pgfqpoint{1.035378in}{1.290062in}}%
\pgfpathlineto{\pgfqpoint{1.037493in}{1.288235in}}%
\pgfpathlineto{\pgfqpoint{1.039607in}{1.281891in}}%
\pgfpathlineto{\pgfqpoint{1.041722in}{1.271047in}}%
\pgfpathlineto{\pgfqpoint{1.043837in}{1.270655in}}%
\pgfpathlineto{\pgfqpoint{1.045951in}{1.273203in}}%
\pgfpathlineto{\pgfqpoint{1.048066in}{1.270536in}}%
\pgfpathlineto{\pgfqpoint{1.052295in}{1.284320in}}%
\pgfpathlineto{\pgfqpoint{1.056525in}{1.273031in}}%
\pgfpathlineto{\pgfqpoint{1.058640in}{1.275612in}}%
\pgfpathlineto{\pgfqpoint{1.060754in}{1.275675in}}%
\pgfpathlineto{\pgfqpoint{1.064984in}{1.279055in}}%
\pgfpathlineto{\pgfqpoint{1.067098in}{1.278433in}}%
\pgfpathlineto{\pgfqpoint{1.071328in}{1.288514in}}%
\pgfpathlineto{\pgfqpoint{1.073442in}{1.283988in}}%
\pgfpathlineto{\pgfqpoint{1.075557in}{1.292880in}}%
\pgfpathlineto{\pgfqpoint{1.077672in}{1.289092in}}%
\pgfpathlineto{\pgfqpoint{1.079786in}{1.287703in}}%
\pgfpathlineto{\pgfqpoint{1.081901in}{1.288042in}}%
\pgfpathlineto{\pgfqpoint{1.084016in}{1.280188in}}%
\pgfpathlineto{\pgfqpoint{1.088245in}{1.283919in}}%
\pgfpathlineto{\pgfqpoint{1.090360in}{1.287914in}}%
\pgfpathlineto{\pgfqpoint{1.092475in}{1.286823in}}%
\pgfpathlineto{\pgfqpoint{1.094589in}{1.287003in}}%
\pgfpathlineto{\pgfqpoint{1.098819in}{1.291015in}}%
\pgfpathlineto{\pgfqpoint{1.100933in}{1.289476in}}%
\pgfpathlineto{\pgfqpoint{1.103048in}{1.291493in}}%
\pgfpathlineto{\pgfqpoint{1.105163in}{1.289580in}}%
\pgfpathlineto{\pgfqpoint{1.109392in}{1.295374in}}%
\pgfpathlineto{\pgfqpoint{1.111507in}{1.302601in}}%
\pgfpathlineto{\pgfqpoint{1.113622in}{1.294597in}}%
\pgfpathlineto{\pgfqpoint{1.117851in}{1.292560in}}%
\pgfpathlineto{\pgfqpoint{1.119966in}{1.293414in}}%
\pgfpathlineto{\pgfqpoint{1.122080in}{1.286718in}}%
\pgfpathlineto{\pgfqpoint{1.124195in}{1.284509in}}%
\pgfpathlineto{\pgfqpoint{1.126310in}{1.293068in}}%
\pgfpathlineto{\pgfqpoint{1.128424in}{1.292494in}}%
\pgfpathlineto{\pgfqpoint{1.132654in}{1.285854in}}%
\pgfpathlineto{\pgfqpoint{1.134769in}{1.288134in}}%
\pgfpathlineto{\pgfqpoint{1.136883in}{1.285755in}}%
\pgfpathlineto{\pgfqpoint{1.138998in}{1.285290in}}%
\pgfpathlineto{\pgfqpoint{1.141113in}{1.286935in}}%
\pgfpathlineto{\pgfqpoint{1.147457in}{1.286994in}}%
\pgfpathlineto{\pgfqpoint{1.149571in}{1.285329in}}%
\pgfpathlineto{\pgfqpoint{1.153801in}{1.287642in}}%
\pgfpathlineto{\pgfqpoint{1.155915in}{1.285628in}}%
\pgfpathlineto{\pgfqpoint{1.160145in}{1.275478in}}%
\pgfpathlineto{\pgfqpoint{1.164374in}{1.286746in}}%
\pgfpathlineto{\pgfqpoint{1.170718in}{1.270962in}}%
\pgfpathlineto{\pgfqpoint{1.174948in}{1.271450in}}%
\pgfpathlineto{\pgfqpoint{1.177062in}{1.274843in}}%
\pgfpathlineto{\pgfqpoint{1.185521in}{1.242689in}}%
\pgfpathlineto{\pgfqpoint{1.189751in}{1.246817in}}%
\pgfpathlineto{\pgfqpoint{1.191865in}{1.244941in}}%
\pgfpathlineto{\pgfqpoint{1.193980in}{1.241166in}}%
\pgfpathlineto{\pgfqpoint{1.196095in}{1.244043in}}%
\pgfpathlineto{\pgfqpoint{1.200324in}{1.253130in}}%
\pgfpathlineto{\pgfqpoint{1.202439in}{1.265460in}}%
\pgfpathlineto{\pgfqpoint{1.204553in}{1.259435in}}%
\pgfpathlineto{\pgfqpoint{1.206668in}{1.261684in}}%
\pgfpathlineto{\pgfqpoint{1.208783in}{1.258491in}}%
\pgfpathlineto{\pgfqpoint{1.210897in}{1.264767in}}%
\pgfpathlineto{\pgfqpoint{1.215127in}{1.270114in}}%
\pgfpathlineto{\pgfqpoint{1.217242in}{1.265559in}}%
\pgfpathlineto{\pgfqpoint{1.223586in}{1.272005in}}%
\pgfpathlineto{\pgfqpoint{1.227815in}{1.267657in}}%
\pgfpathlineto{\pgfqpoint{1.229930in}{1.266679in}}%
\pgfpathlineto{\pgfqpoint{1.234159in}{1.270110in}}%
\pgfpathlineto{\pgfqpoint{1.236274in}{1.281562in}}%
\pgfpathlineto{\pgfqpoint{1.238389in}{1.280847in}}%
\pgfpathlineto{\pgfqpoint{1.242618in}{1.295065in}}%
\pgfpathlineto{\pgfqpoint{1.244733in}{1.289353in}}%
\pgfpathlineto{\pgfqpoint{1.246847in}{1.288905in}}%
\pgfpathlineto{\pgfqpoint{1.248962in}{1.284494in}}%
\pgfpathlineto{\pgfqpoint{1.251077in}{1.282991in}}%
\pgfpathlineto{\pgfqpoint{1.255306in}{1.287195in}}%
\pgfpathlineto{\pgfqpoint{1.257421in}{1.286084in}}%
\pgfpathlineto{\pgfqpoint{1.259535in}{1.287668in}}%
\pgfpathlineto{\pgfqpoint{1.261650in}{1.286145in}}%
\pgfpathlineto{\pgfqpoint{1.263765in}{1.287974in}}%
\pgfpathlineto{\pgfqpoint{1.265880in}{1.296384in}}%
\pgfpathlineto{\pgfqpoint{1.267994in}{1.292128in}}%
\pgfpathlineto{\pgfqpoint{1.270109in}{1.298656in}}%
\pgfpathlineto{\pgfqpoint{1.274338in}{1.299339in}}%
\pgfpathlineto{\pgfqpoint{1.276453in}{1.305048in}}%
\pgfpathlineto{\pgfqpoint{1.278568in}{1.301690in}}%
\pgfpathlineto{\pgfqpoint{1.282797in}{1.309281in}}%
\pgfpathlineto{\pgfqpoint{1.284912in}{1.304464in}}%
\pgfpathlineto{\pgfqpoint{1.289141in}{1.305161in}}%
\pgfpathlineto{\pgfqpoint{1.291256in}{1.315970in}}%
\pgfpathlineto{\pgfqpoint{1.297600in}{1.322800in}}%
\pgfpathlineto{\pgfqpoint{1.299715in}{1.316361in}}%
\pgfpathlineto{\pgfqpoint{1.301829in}{1.322944in}}%
\pgfpathlineto{\pgfqpoint{1.303944in}{1.323391in}}%
\pgfpathlineto{\pgfqpoint{1.306059in}{1.330179in}}%
\pgfpathlineto{\pgfqpoint{1.308173in}{1.325717in}}%
\pgfpathlineto{\pgfqpoint{1.310288in}{1.317881in}}%
\pgfpathlineto{\pgfqpoint{1.314517in}{1.318054in}}%
\pgfpathlineto{\pgfqpoint{1.316632in}{1.322219in}}%
\pgfpathlineto{\pgfqpoint{1.318747in}{1.318981in}}%
\pgfpathlineto{\pgfqpoint{1.320862in}{1.318851in}}%
\pgfpathlineto{\pgfqpoint{1.322976in}{1.321888in}}%
\pgfpathlineto{\pgfqpoint{1.325091in}{1.322566in}}%
\pgfpathlineto{\pgfqpoint{1.329320in}{1.322182in}}%
\pgfpathlineto{\pgfqpoint{1.331435in}{1.328559in}}%
\pgfpathlineto{\pgfqpoint{1.333550in}{1.324468in}}%
\pgfpathlineto{\pgfqpoint{1.335664in}{1.323920in}}%
\pgfpathlineto{\pgfqpoint{1.337779in}{1.326435in}}%
\pgfpathlineto{\pgfqpoint{1.339894in}{1.333610in}}%
\pgfpathlineto{\pgfqpoint{1.342009in}{1.330943in}}%
\pgfpathlineto{\pgfqpoint{1.344123in}{1.325164in}}%
\pgfpathlineto{\pgfqpoint{1.346238in}{1.329441in}}%
\pgfpathlineto{\pgfqpoint{1.348353in}{1.329797in}}%
\pgfpathlineto{\pgfqpoint{1.350467in}{1.328288in}}%
\pgfpathlineto{\pgfqpoint{1.352582in}{1.330325in}}%
\pgfpathlineto{\pgfqpoint{1.354697in}{1.322962in}}%
\pgfpathlineto{\pgfqpoint{1.356811in}{1.323984in}}%
\pgfpathlineto{\pgfqpoint{1.358926in}{1.320897in}}%
\pgfpathlineto{\pgfqpoint{1.361041in}{1.314762in}}%
\pgfpathlineto{\pgfqpoint{1.363155in}{1.323849in}}%
\pgfpathlineto{\pgfqpoint{1.365270in}{1.325987in}}%
\pgfpathlineto{\pgfqpoint{1.367385in}{1.322435in}}%
\pgfpathlineto{\pgfqpoint{1.369500in}{1.322093in}}%
\pgfpathlineto{\pgfqpoint{1.371614in}{1.323638in}}%
\pgfpathlineto{\pgfqpoint{1.373729in}{1.328880in}}%
\pgfpathlineto{\pgfqpoint{1.375844in}{1.327067in}}%
\pgfpathlineto{\pgfqpoint{1.377958in}{1.323369in}}%
\pgfpathlineto{\pgfqpoint{1.380073in}{1.321950in}}%
\pgfpathlineto{\pgfqpoint{1.382188in}{1.318960in}}%
\pgfpathlineto{\pgfqpoint{1.386417in}{1.319477in}}%
\pgfpathlineto{\pgfqpoint{1.390646in}{1.327672in}}%
\pgfpathlineto{\pgfqpoint{1.396991in}{1.331193in}}%
\pgfpathlineto{\pgfqpoint{1.399105in}{1.338607in}}%
\pgfpathlineto{\pgfqpoint{1.401220in}{1.333854in}}%
\pgfpathlineto{\pgfqpoint{1.403335in}{1.333682in}}%
\pgfpathlineto{\pgfqpoint{1.405449in}{1.337602in}}%
\pgfpathlineto{\pgfqpoint{1.407564in}{1.347185in}}%
\pgfpathlineto{\pgfqpoint{1.416023in}{1.337198in}}%
\pgfpathlineto{\pgfqpoint{1.418137in}{1.335494in}}%
\pgfpathlineto{\pgfqpoint{1.420252in}{1.335961in}}%
\pgfpathlineto{\pgfqpoint{1.422367in}{1.333892in}}%
\pgfpathlineto{\pgfqpoint{1.426596in}{1.343390in}}%
\pgfpathlineto{\pgfqpoint{1.430826in}{1.340052in}}%
\pgfpathlineto{\pgfqpoint{1.432940in}{1.337308in}}%
\pgfpathlineto{\pgfqpoint{1.435055in}{1.336876in}}%
\pgfpathlineto{\pgfqpoint{1.437170in}{1.341017in}}%
\pgfpathlineto{\pgfqpoint{1.439284in}{1.338683in}}%
\pgfpathlineto{\pgfqpoint{1.441399in}{1.328757in}}%
\pgfpathlineto{\pgfqpoint{1.443514in}{1.333108in}}%
\pgfpathlineto{\pgfqpoint{1.445628in}{1.332200in}}%
\pgfpathlineto{\pgfqpoint{1.447743in}{1.326988in}}%
\pgfpathlineto{\pgfqpoint{1.451973in}{1.331066in}}%
\pgfpathlineto{\pgfqpoint{1.454087in}{1.325055in}}%
\pgfpathlineto{\pgfqpoint{1.456202in}{1.324808in}}%
\pgfpathlineto{\pgfqpoint{1.458317in}{1.322605in}}%
\pgfpathlineto{\pgfqpoint{1.460431in}{1.315552in}}%
\pgfpathlineto{\pgfqpoint{1.462546in}{1.316041in}}%
\pgfpathlineto{\pgfqpoint{1.466775in}{1.327941in}}%
\pgfpathlineto{\pgfqpoint{1.471005in}{1.321906in}}%
\pgfpathlineto{\pgfqpoint{1.473120in}{1.324081in}}%
\pgfpathlineto{\pgfqpoint{1.475234in}{1.315139in}}%
\pgfpathlineto{\pgfqpoint{1.481578in}{1.305779in}}%
\pgfpathlineto{\pgfqpoint{1.483693in}{1.304789in}}%
\pgfpathlineto{\pgfqpoint{1.487922in}{1.311993in}}%
\pgfpathlineto{\pgfqpoint{1.494266in}{1.297521in}}%
\pgfpathlineto{\pgfqpoint{1.496381in}{1.300742in}}%
\pgfpathlineto{\pgfqpoint{1.498496in}{1.297817in}}%
\pgfpathlineto{\pgfqpoint{1.504840in}{1.310556in}}%
\pgfpathlineto{\pgfqpoint{1.506955in}{1.307550in}}%
\pgfpathlineto{\pgfqpoint{1.509069in}{1.316370in}}%
\pgfpathlineto{\pgfqpoint{1.513299in}{1.298600in}}%
\pgfpathlineto{\pgfqpoint{1.515413in}{1.303520in}}%
\pgfpathlineto{\pgfqpoint{1.517528in}{1.304113in}}%
\pgfpathlineto{\pgfqpoint{1.519643in}{1.302602in}}%
\pgfpathlineto{\pgfqpoint{1.525987in}{1.306607in}}%
\pgfpathlineto{\pgfqpoint{1.528102in}{1.320862in}}%
\pgfpathlineto{\pgfqpoint{1.534446in}{1.320498in}}%
\pgfpathlineto{\pgfqpoint{1.538675in}{1.315199in}}%
\pgfpathlineto{\pgfqpoint{1.540790in}{1.315075in}}%
\pgfpathlineto{\pgfqpoint{1.542904in}{1.313671in}}%
\pgfpathlineto{\pgfqpoint{1.547134in}{1.318569in}}%
\pgfpathlineto{\pgfqpoint{1.549248in}{1.319122in}}%
\pgfpathlineto{\pgfqpoint{1.551363in}{1.322377in}}%
\pgfpathlineto{\pgfqpoint{1.553478in}{1.314823in}}%
\pgfpathlineto{\pgfqpoint{1.555593in}{1.313284in}}%
\pgfpathlineto{\pgfqpoint{1.557707in}{1.317041in}}%
\pgfpathlineto{\pgfqpoint{1.559822in}{1.308960in}}%
\pgfpathlineto{\pgfqpoint{1.561937in}{1.308782in}}%
\pgfpathlineto{\pgfqpoint{1.564051in}{1.311195in}}%
\pgfpathlineto{\pgfqpoint{1.566166in}{1.310255in}}%
\pgfpathlineto{\pgfqpoint{1.568281in}{1.318183in}}%
\pgfpathlineto{\pgfqpoint{1.572510in}{1.312587in}}%
\pgfpathlineto{\pgfqpoint{1.574625in}{1.312854in}}%
\pgfpathlineto{\pgfqpoint{1.576740in}{1.310008in}}%
\pgfpathlineto{\pgfqpoint{1.578854in}{1.315205in}}%
\pgfpathlineto{\pgfqpoint{1.580969in}{1.315612in}}%
\pgfpathlineto{\pgfqpoint{1.583084in}{1.314769in}}%
\pgfpathlineto{\pgfqpoint{1.585198in}{1.317516in}}%
\pgfpathlineto{\pgfqpoint{1.587313in}{1.313412in}}%
\pgfpathlineto{\pgfqpoint{1.593657in}{1.325032in}}%
\pgfpathlineto{\pgfqpoint{1.595772in}{1.326962in}}%
\pgfpathlineto{\pgfqpoint{1.597886in}{1.318707in}}%
\pgfpathlineto{\pgfqpoint{1.602116in}{1.312359in}}%
\pgfpathlineto{\pgfqpoint{1.604231in}{1.314579in}}%
\pgfpathlineto{\pgfqpoint{1.606345in}{1.313420in}}%
\pgfpathlineto{\pgfqpoint{1.608460in}{1.313690in}}%
\pgfpathlineto{\pgfqpoint{1.612689in}{1.318720in}}%
\pgfpathlineto{\pgfqpoint{1.614804in}{1.317062in}}%
\pgfpathlineto{\pgfqpoint{1.616919in}{1.312672in}}%
\pgfpathlineto{\pgfqpoint{1.619033in}{1.320429in}}%
\pgfpathlineto{\pgfqpoint{1.621148in}{1.321783in}}%
\pgfpathlineto{\pgfqpoint{1.627492in}{1.331260in}}%
\pgfpathlineto{\pgfqpoint{1.629607in}{1.330759in}}%
\pgfpathlineto{\pgfqpoint{1.631722in}{1.323659in}}%
\pgfpathlineto{\pgfqpoint{1.635951in}{1.330500in}}%
\pgfpathlineto{\pgfqpoint{1.640180in}{1.324921in}}%
\pgfpathlineto{\pgfqpoint{1.644410in}{1.331218in}}%
\pgfpathlineto{\pgfqpoint{1.646524in}{1.331285in}}%
\pgfpathlineto{\pgfqpoint{1.648639in}{1.329680in}}%
\pgfpathlineto{\pgfqpoint{1.650754in}{1.326059in}}%
\pgfpathlineto{\pgfqpoint{1.652868in}{1.330486in}}%
\pgfpathlineto{\pgfqpoint{1.654983in}{1.326897in}}%
\pgfpathlineto{\pgfqpoint{1.657098in}{1.331924in}}%
\pgfpathlineto{\pgfqpoint{1.659213in}{1.326692in}}%
\pgfpathlineto{\pgfqpoint{1.663442in}{1.332039in}}%
\pgfpathlineto{\pgfqpoint{1.665557in}{1.323556in}}%
\pgfpathlineto{\pgfqpoint{1.667671in}{1.325816in}}%
\pgfpathlineto{\pgfqpoint{1.669786in}{1.320562in}}%
\pgfpathlineto{\pgfqpoint{1.676130in}{1.312685in}}%
\pgfpathlineto{\pgfqpoint{1.678245in}{1.314123in}}%
\pgfpathlineto{\pgfqpoint{1.680359in}{1.311945in}}%
\pgfpathlineto{\pgfqpoint{1.686704in}{1.323576in}}%
\pgfpathlineto{\pgfqpoint{1.688818in}{1.317894in}}%
\pgfpathlineto{\pgfqpoint{1.690933in}{1.318450in}}%
\pgfpathlineto{\pgfqpoint{1.693048in}{1.312559in}}%
\pgfpathlineto{\pgfqpoint{1.697277in}{1.310083in}}%
\pgfpathlineto{\pgfqpoint{1.701506in}{1.316380in}}%
\pgfpathlineto{\pgfqpoint{1.705736in}{1.319680in}}%
\pgfpathlineto{\pgfqpoint{1.709965in}{1.333037in}}%
\pgfpathlineto{\pgfqpoint{1.712080in}{1.327372in}}%
\pgfpathlineto{\pgfqpoint{1.714195in}{1.328234in}}%
\pgfpathlineto{\pgfqpoint{1.718424in}{1.325839in}}%
\pgfpathlineto{\pgfqpoint{1.720539in}{1.318375in}}%
\pgfpathlineto{\pgfqpoint{1.722653in}{1.324107in}}%
\pgfpathlineto{\pgfqpoint{1.726883in}{1.319753in}}%
\pgfpathlineto{\pgfqpoint{1.728997in}{1.313884in}}%
\pgfpathlineto{\pgfqpoint{1.731112in}{1.315728in}}%
\pgfpathlineto{\pgfqpoint{1.737456in}{1.300231in}}%
\pgfpathlineto{\pgfqpoint{1.741686in}{1.306852in}}%
\pgfpathlineto{\pgfqpoint{1.743800in}{1.305844in}}%
\pgfpathlineto{\pgfqpoint{1.750144in}{1.324018in}}%
\pgfpathlineto{\pgfqpoint{1.754374in}{1.320283in}}%
\pgfpathlineto{\pgfqpoint{1.756488in}{1.321012in}}%
\pgfpathlineto{\pgfqpoint{1.758603in}{1.326018in}}%
\pgfpathlineto{\pgfqpoint{1.760718in}{1.321424in}}%
\pgfpathlineto{\pgfqpoint{1.762833in}{1.323320in}}%
\pgfpathlineto{\pgfqpoint{1.764947in}{1.327695in}}%
\pgfpathlineto{\pgfqpoint{1.767062in}{1.326923in}}%
\pgfpathlineto{\pgfqpoint{1.769177in}{1.323292in}}%
\pgfpathlineto{\pgfqpoint{1.773406in}{1.324270in}}%
\pgfpathlineto{\pgfqpoint{1.775521in}{1.323237in}}%
\pgfpathlineto{\pgfqpoint{1.777635in}{1.328430in}}%
\pgfpathlineto{\pgfqpoint{1.779750in}{1.321898in}}%
\pgfpathlineto{\pgfqpoint{1.781865in}{1.322635in}}%
\pgfpathlineto{\pgfqpoint{1.786094in}{1.314551in}}%
\pgfpathlineto{\pgfqpoint{1.788209in}{1.316528in}}%
\pgfpathlineto{\pgfqpoint{1.790324in}{1.322058in}}%
\pgfpathlineto{\pgfqpoint{1.794553in}{1.317462in}}%
\pgfpathlineto{\pgfqpoint{1.796668in}{1.313609in}}%
\pgfpathlineto{\pgfqpoint{1.798782in}{1.312295in}}%
\pgfpathlineto{\pgfqpoint{1.800897in}{1.315963in}}%
\pgfpathlineto{\pgfqpoint{1.803012in}{1.316133in}}%
\pgfpathlineto{\pgfqpoint{1.805126in}{1.320102in}}%
\pgfpathlineto{\pgfqpoint{1.807241in}{1.320239in}}%
\pgfpathlineto{\pgfqpoint{1.809356in}{1.324141in}}%
\pgfpathlineto{\pgfqpoint{1.811471in}{1.322778in}}%
\pgfpathlineto{\pgfqpoint{1.815700in}{1.323134in}}%
\pgfpathlineto{\pgfqpoint{1.817815in}{1.324355in}}%
\pgfpathlineto{\pgfqpoint{1.819929in}{1.322850in}}%
\pgfpathlineto{\pgfqpoint{1.822044in}{1.327595in}}%
\pgfpathlineto{\pgfqpoint{1.824159in}{1.327215in}}%
\pgfpathlineto{\pgfqpoint{1.826273in}{1.331064in}}%
\pgfpathlineto{\pgfqpoint{1.828388in}{1.321736in}}%
\pgfpathlineto{\pgfqpoint{1.830503in}{1.320511in}}%
\pgfpathlineto{\pgfqpoint{1.832617in}{1.323005in}}%
\pgfpathlineto{\pgfqpoint{1.836847in}{1.321623in}}%
\pgfpathlineto{\pgfqpoint{1.838962in}{1.325721in}}%
\pgfpathlineto{\pgfqpoint{1.841076in}{1.320588in}}%
\pgfpathlineto{\pgfqpoint{1.843191in}{1.321837in}}%
\pgfpathlineto{\pgfqpoint{1.845306in}{1.330268in}}%
\pgfpathlineto{\pgfqpoint{1.849535in}{1.326847in}}%
\pgfpathlineto{\pgfqpoint{1.853764in}{1.325781in}}%
\pgfpathlineto{\pgfqpoint{1.855879in}{1.326518in}}%
\pgfpathlineto{\pgfqpoint{1.857994in}{1.323048in}}%
\pgfpathlineto{\pgfqpoint{1.860108in}{1.311576in}}%
\pgfpathlineto{\pgfqpoint{1.862223in}{1.307856in}}%
\pgfpathlineto{\pgfqpoint{1.866453in}{1.308497in}}%
\pgfpathlineto{\pgfqpoint{1.868567in}{1.313992in}}%
\pgfpathlineto{\pgfqpoint{1.877026in}{1.307241in}}%
\pgfpathlineto{\pgfqpoint{1.879141in}{1.309610in}}%
\pgfpathlineto{\pgfqpoint{1.881255in}{1.316797in}}%
\pgfpathlineto{\pgfqpoint{1.883370in}{1.316462in}}%
\pgfpathlineto{\pgfqpoint{1.885485in}{1.318588in}}%
\pgfpathlineto{\pgfqpoint{1.887599in}{1.316971in}}%
\pgfpathlineto{\pgfqpoint{1.889714in}{1.312623in}}%
\pgfpathlineto{\pgfqpoint{1.891829in}{1.312714in}}%
\pgfpathlineto{\pgfqpoint{1.893944in}{1.318119in}}%
\pgfpathlineto{\pgfqpoint{1.896058in}{1.320108in}}%
\pgfpathlineto{\pgfqpoint{1.898173in}{1.319817in}}%
\pgfpathlineto{\pgfqpoint{1.900288in}{1.317062in}}%
\pgfpathlineto{\pgfqpoint{1.902402in}{1.316235in}}%
\pgfpathlineto{\pgfqpoint{1.904517in}{1.322110in}}%
\pgfpathlineto{\pgfqpoint{1.908746in}{1.326146in}}%
\pgfpathlineto{\pgfqpoint{1.910861in}{1.321995in}}%
\pgfpathlineto{\pgfqpoint{1.912976in}{1.322461in}}%
\pgfpathlineto{\pgfqpoint{1.915090in}{1.329352in}}%
\pgfpathlineto{\pgfqpoint{1.919320in}{1.332359in}}%
\pgfpathlineto{\pgfqpoint{1.921435in}{1.336054in}}%
\pgfpathlineto{\pgfqpoint{1.923549in}{1.337327in}}%
\pgfpathlineto{\pgfqpoint{1.927779in}{1.320972in}}%
\pgfpathlineto{\pgfqpoint{1.929893in}{1.330121in}}%
\pgfpathlineto{\pgfqpoint{1.932008in}{1.327886in}}%
\pgfpathlineto{\pgfqpoint{1.936237in}{1.333302in}}%
\pgfpathlineto{\pgfqpoint{1.938352in}{1.325257in}}%
\pgfpathlineto{\pgfqpoint{1.940467in}{1.325618in}}%
\pgfpathlineto{\pgfqpoint{1.944696in}{1.314058in}}%
\pgfpathlineto{\pgfqpoint{1.946811in}{1.312361in}}%
\pgfpathlineto{\pgfqpoint{1.951040in}{1.324100in}}%
\pgfpathlineto{\pgfqpoint{1.953155in}{1.326248in}}%
\pgfpathlineto{\pgfqpoint{1.955270in}{1.334695in}}%
\pgfpathlineto{\pgfqpoint{1.957384in}{1.330600in}}%
\pgfpathlineto{\pgfqpoint{1.959499in}{1.329054in}}%
\pgfpathlineto{\pgfqpoint{1.965843in}{1.334553in}}%
\pgfpathlineto{\pgfqpoint{1.967958in}{1.332198in}}%
\pgfpathlineto{\pgfqpoint{1.970073in}{1.339390in}}%
\pgfpathlineto{\pgfqpoint{1.972187in}{1.334111in}}%
\pgfpathlineto{\pgfqpoint{1.976417in}{1.343529in}}%
\pgfpathlineto{\pgfqpoint{1.980646in}{1.351078in}}%
\pgfpathlineto{\pgfqpoint{1.982761in}{1.346948in}}%
\pgfpathlineto{\pgfqpoint{1.986990in}{1.353238in}}%
\pgfpathlineto{\pgfqpoint{1.989105in}{1.351530in}}%
\pgfpathlineto{\pgfqpoint{1.991219in}{1.354817in}}%
\pgfpathlineto{\pgfqpoint{1.999678in}{1.341022in}}%
\pgfpathlineto{\pgfqpoint{2.001793in}{1.340486in}}%
\pgfpathlineto{\pgfqpoint{2.003908in}{1.346981in}}%
\pgfpathlineto{\pgfqpoint{2.006022in}{1.347919in}}%
\pgfpathlineto{\pgfqpoint{2.008137in}{1.346617in}}%
\pgfpathlineto{\pgfqpoint{2.010252in}{1.341517in}}%
\pgfpathlineto{\pgfqpoint{2.012366in}{1.344328in}}%
\pgfpathlineto{\pgfqpoint{2.014481in}{1.344714in}}%
\pgfpathlineto{\pgfqpoint{2.016596in}{1.343420in}}%
\pgfpathlineto{\pgfqpoint{2.020825in}{1.333094in}}%
\pgfpathlineto{\pgfqpoint{2.022940in}{1.337791in}}%
\pgfpathlineto{\pgfqpoint{2.025055in}{1.329406in}}%
\pgfpathlineto{\pgfqpoint{2.027169in}{1.326727in}}%
\pgfpathlineto{\pgfqpoint{2.029284in}{1.325944in}}%
\pgfpathlineto{\pgfqpoint{2.031399in}{1.323829in}}%
\pgfpathlineto{\pgfqpoint{2.033513in}{1.328590in}}%
\pgfpathlineto{\pgfqpoint{2.037743in}{1.326518in}}%
\pgfpathlineto{\pgfqpoint{2.039857in}{1.327538in}}%
\pgfpathlineto{\pgfqpoint{2.041972in}{1.319585in}}%
\pgfpathlineto{\pgfqpoint{2.046202in}{1.332294in}}%
\pgfpathlineto{\pgfqpoint{2.050431in}{1.326241in}}%
\pgfpathlineto{\pgfqpoint{2.052546in}{1.331200in}}%
\pgfpathlineto{\pgfqpoint{2.058890in}{1.317911in}}%
\pgfpathlineto{\pgfqpoint{2.063119in}{1.318108in}}%
\pgfpathlineto{\pgfqpoint{2.067348in}{1.312627in}}%
\pgfpathlineto{\pgfqpoint{2.069463in}{1.315575in}}%
\pgfpathlineto{\pgfqpoint{2.071578in}{1.315599in}}%
\pgfpathlineto{\pgfqpoint{2.073693in}{1.307566in}}%
\pgfpathlineto{\pgfqpoint{2.075807in}{1.312608in}}%
\pgfpathlineto{\pgfqpoint{2.088495in}{1.293999in}}%
\pgfpathlineto{\pgfqpoint{2.090610in}{1.287957in}}%
\pgfpathlineto{\pgfqpoint{2.096954in}{1.296548in}}%
\pgfpathlineto{\pgfqpoint{2.103298in}{1.283704in}}%
\pgfpathlineto{\pgfqpoint{2.107528in}{1.282709in}}%
\pgfpathlineto{\pgfqpoint{2.113872in}{1.291467in}}%
\pgfpathlineto{\pgfqpoint{2.115986in}{1.293701in}}%
\pgfpathlineto{\pgfqpoint{2.118101in}{1.304126in}}%
\pgfpathlineto{\pgfqpoint{2.120216in}{1.303338in}}%
\pgfpathlineto{\pgfqpoint{2.122330in}{1.305889in}}%
\pgfpathlineto{\pgfqpoint{2.124445in}{1.310621in}}%
\pgfpathlineto{\pgfqpoint{2.126560in}{1.311701in}}%
\pgfpathlineto{\pgfqpoint{2.128675in}{1.307469in}}%
\pgfpathlineto{\pgfqpoint{2.132904in}{1.304821in}}%
\pgfpathlineto{\pgfqpoint{2.139248in}{1.307356in}}%
\pgfpathlineto{\pgfqpoint{2.143477in}{1.293958in}}%
\pgfpathlineto{\pgfqpoint{2.145592in}{1.296390in}}%
\pgfpathlineto{\pgfqpoint{2.149822in}{1.293175in}}%
\pgfpathlineto{\pgfqpoint{2.151936in}{1.296129in}}%
\pgfpathlineto{\pgfqpoint{2.158280in}{1.311147in}}%
\pgfpathlineto{\pgfqpoint{2.160395in}{1.307767in}}%
\pgfpathlineto{\pgfqpoint{2.162510in}{1.309094in}}%
\pgfpathlineto{\pgfqpoint{2.164624in}{1.317066in}}%
\pgfpathlineto{\pgfqpoint{2.166739in}{1.317820in}}%
\pgfpathlineto{\pgfqpoint{2.168854in}{1.316668in}}%
\pgfpathlineto{\pgfqpoint{2.170968in}{1.320246in}}%
\pgfpathlineto{\pgfqpoint{2.175198in}{1.313154in}}%
\pgfpathlineto{\pgfqpoint{2.177313in}{1.313992in}}%
\pgfpathlineto{\pgfqpoint{2.179427in}{1.317209in}}%
\pgfpathlineto{\pgfqpoint{2.181542in}{1.313312in}}%
\pgfpathlineto{\pgfqpoint{2.183657in}{1.311779in}}%
\pgfpathlineto{\pgfqpoint{2.185771in}{1.308269in}}%
\pgfpathlineto{\pgfqpoint{2.187886in}{1.307225in}}%
\pgfpathlineto{\pgfqpoint{2.190001in}{1.310727in}}%
\pgfpathlineto{\pgfqpoint{2.192115in}{1.306477in}}%
\pgfpathlineto{\pgfqpoint{2.194230in}{1.311140in}}%
\pgfpathlineto{\pgfqpoint{2.196345in}{1.311821in}}%
\pgfpathlineto{\pgfqpoint{2.198459in}{1.317638in}}%
\pgfpathlineto{\pgfqpoint{2.200574in}{1.318350in}}%
\pgfpathlineto{\pgfqpoint{2.202689in}{1.320979in}}%
\pgfpathlineto{\pgfqpoint{2.204804in}{1.317845in}}%
\pgfpathlineto{\pgfqpoint{2.209033in}{1.305805in}}%
\pgfpathlineto{\pgfqpoint{2.211148in}{1.314618in}}%
\pgfpathlineto{\pgfqpoint{2.213262in}{1.313570in}}%
\pgfpathlineto{\pgfqpoint{2.217492in}{1.322625in}}%
\pgfpathlineto{\pgfqpoint{2.221721in}{1.318542in}}%
\pgfpathlineto{\pgfqpoint{2.223836in}{1.319469in}}%
\pgfpathlineto{\pgfqpoint{2.225950in}{1.317880in}}%
\pgfpathlineto{\pgfqpoint{2.228065in}{1.310825in}}%
\pgfpathlineto{\pgfqpoint{2.230180in}{1.311377in}}%
\pgfpathlineto{\pgfqpoint{2.236524in}{1.317485in}}%
\pgfpathlineto{\pgfqpoint{2.240753in}{1.304707in}}%
\pgfpathlineto{\pgfqpoint{2.242868in}{1.303181in}}%
\pgfpathlineto{\pgfqpoint{2.244983in}{1.306419in}}%
\pgfpathlineto{\pgfqpoint{2.249212in}{1.307098in}}%
\pgfpathlineto{\pgfqpoint{2.251327in}{1.307739in}}%
\pgfpathlineto{\pgfqpoint{2.253441in}{1.309698in}}%
\pgfpathlineto{\pgfqpoint{2.255556in}{1.307319in}}%
\pgfpathlineto{\pgfqpoint{2.257671in}{1.296771in}}%
\pgfpathlineto{\pgfqpoint{2.259786in}{1.298762in}}%
\pgfpathlineto{\pgfqpoint{2.261900in}{1.297523in}}%
\pgfpathlineto{\pgfqpoint{2.264015in}{1.310018in}}%
\pgfpathlineto{\pgfqpoint{2.266130in}{1.308116in}}%
\pgfpathlineto{\pgfqpoint{2.272474in}{1.320833in}}%
\pgfpathlineto{\pgfqpoint{2.274588in}{1.325040in}}%
\pgfpathlineto{\pgfqpoint{2.276703in}{1.325132in}}%
\pgfpathlineto{\pgfqpoint{2.278818in}{1.331540in}}%
\pgfpathlineto{\pgfqpoint{2.283047in}{1.318637in}}%
\pgfpathlineto{\pgfqpoint{2.285162in}{1.319835in}}%
\pgfpathlineto{\pgfqpoint{2.287277in}{1.316829in}}%
\pgfpathlineto{\pgfqpoint{2.289391in}{1.318720in}}%
\pgfpathlineto{\pgfqpoint{2.291506in}{1.312481in}}%
\pgfpathlineto{\pgfqpoint{2.293621in}{1.317554in}}%
\pgfpathlineto{\pgfqpoint{2.297850in}{1.310161in}}%
\pgfpathlineto{\pgfqpoint{2.299965in}{1.315031in}}%
\pgfpathlineto{\pgfqpoint{2.302079in}{1.313798in}}%
\pgfpathlineto{\pgfqpoint{2.304194in}{1.318590in}}%
\pgfpathlineto{\pgfqpoint{2.306309in}{1.315039in}}%
\pgfpathlineto{\pgfqpoint{2.308424in}{1.315854in}}%
\pgfpathlineto{\pgfqpoint{2.310538in}{1.319694in}}%
\pgfpathlineto{\pgfqpoint{2.312653in}{1.308724in}}%
\pgfpathlineto{\pgfqpoint{2.314768in}{1.315496in}}%
\pgfpathlineto{\pgfqpoint{2.316882in}{1.313515in}}%
\pgfpathlineto{\pgfqpoint{2.318997in}{1.318852in}}%
\pgfpathlineto{\pgfqpoint{2.321112in}{1.316985in}}%
\pgfpathlineto{\pgfqpoint{2.323226in}{1.319923in}}%
\pgfpathlineto{\pgfqpoint{2.327456in}{1.321720in}}%
\pgfpathlineto{\pgfqpoint{2.329570in}{1.319023in}}%
\pgfpathlineto{\pgfqpoint{2.331685in}{1.319174in}}%
\pgfpathlineto{\pgfqpoint{2.335915in}{1.298598in}}%
\pgfpathlineto{\pgfqpoint{2.338029in}{1.301551in}}%
\pgfpathlineto{\pgfqpoint{2.340144in}{1.301110in}}%
\pgfpathlineto{\pgfqpoint{2.342259in}{1.304363in}}%
\pgfpathlineto{\pgfqpoint{2.344373in}{1.301505in}}%
\pgfpathlineto{\pgfqpoint{2.346488in}{1.302387in}}%
\pgfpathlineto{\pgfqpoint{2.348603in}{1.297929in}}%
\pgfpathlineto{\pgfqpoint{2.354947in}{1.303475in}}%
\pgfpathlineto{\pgfqpoint{2.359176in}{1.306630in}}%
\pgfpathlineto{\pgfqpoint{2.361291in}{1.307558in}}%
\pgfpathlineto{\pgfqpoint{2.363406in}{1.303427in}}%
\pgfpathlineto{\pgfqpoint{2.365520in}{1.305105in}}%
\pgfpathlineto{\pgfqpoint{2.373979in}{1.302190in}}%
\pgfpathlineto{\pgfqpoint{2.376094in}{1.304245in}}%
\pgfpathlineto{\pgfqpoint{2.382438in}{1.315172in}}%
\pgfpathlineto{\pgfqpoint{2.386667in}{1.330183in}}%
\pgfpathlineto{\pgfqpoint{2.390897in}{1.331197in}}%
\pgfpathlineto{\pgfqpoint{2.393011in}{1.326929in}}%
\pgfpathlineto{\pgfqpoint{2.395126in}{1.327956in}}%
\pgfpathlineto{\pgfqpoint{2.399355in}{1.334783in}}%
\pgfpathlineto{\pgfqpoint{2.401470in}{1.327836in}}%
\pgfpathlineto{\pgfqpoint{2.403585in}{1.316611in}}%
\pgfpathlineto{\pgfqpoint{2.405699in}{1.313899in}}%
\pgfpathlineto{\pgfqpoint{2.407814in}{1.313894in}}%
\pgfpathlineto{\pgfqpoint{2.409929in}{1.307785in}}%
\pgfpathlineto{\pgfqpoint{2.412044in}{1.319726in}}%
\pgfpathlineto{\pgfqpoint{2.414158in}{1.317818in}}%
\pgfpathlineto{\pgfqpoint{2.418388in}{1.310565in}}%
\pgfpathlineto{\pgfqpoint{2.420502in}{1.314934in}}%
\pgfpathlineto{\pgfqpoint{2.422617in}{1.310707in}}%
\pgfpathlineto{\pgfqpoint{2.424732in}{1.311564in}}%
\pgfpathlineto{\pgfqpoint{2.426846in}{1.307445in}}%
\pgfpathlineto{\pgfqpoint{2.428961in}{1.313160in}}%
\pgfpathlineto{\pgfqpoint{2.431076in}{1.313824in}}%
\pgfpathlineto{\pgfqpoint{2.433190in}{1.317084in}}%
\pgfpathlineto{\pgfqpoint{2.435305in}{1.324327in}}%
\pgfpathlineto{\pgfqpoint{2.439535in}{1.324250in}}%
\pgfpathlineto{\pgfqpoint{2.441649in}{1.319563in}}%
\pgfpathlineto{\pgfqpoint{2.443764in}{1.322083in}}%
\pgfpathlineto{\pgfqpoint{2.445879in}{1.317634in}}%
\pgfpathlineto{\pgfqpoint{2.447993in}{1.320137in}}%
\pgfpathlineto{\pgfqpoint{2.450108in}{1.319956in}}%
\pgfpathlineto{\pgfqpoint{2.452223in}{1.321734in}}%
\pgfpathlineto{\pgfqpoint{2.454337in}{1.326885in}}%
\pgfpathlineto{\pgfqpoint{2.456452in}{1.335049in}}%
\pgfpathlineto{\pgfqpoint{2.458567in}{1.338387in}}%
\pgfpathlineto{\pgfqpoint{2.460681in}{1.338383in}}%
\pgfpathlineto{\pgfqpoint{2.462796in}{1.342076in}}%
\pgfpathlineto{\pgfqpoint{2.467026in}{1.342107in}}%
\pgfpathlineto{\pgfqpoint{2.471255in}{1.326942in}}%
\pgfpathlineto{\pgfqpoint{2.473370in}{1.329336in}}%
\pgfpathlineto{\pgfqpoint{2.475484in}{1.326059in}}%
\pgfpathlineto{\pgfqpoint{2.477599in}{1.319227in}}%
\pgfpathlineto{\pgfqpoint{2.479714in}{1.316425in}}%
\pgfpathlineto{\pgfqpoint{2.481828in}{1.321300in}}%
\pgfpathlineto{\pgfqpoint{2.486058in}{1.321315in}}%
\pgfpathlineto{\pgfqpoint{2.492402in}{1.324269in}}%
\pgfpathlineto{\pgfqpoint{2.494517in}{1.320230in}}%
\pgfpathlineto{\pgfqpoint{2.498746in}{1.334669in}}%
\pgfpathlineto{\pgfqpoint{2.500861in}{1.335761in}}%
\pgfpathlineto{\pgfqpoint{2.502975in}{1.327739in}}%
\pgfpathlineto{\pgfqpoint{2.507205in}{1.344212in}}%
\pgfpathlineto{\pgfqpoint{2.511434in}{1.346663in}}%
\pgfpathlineto{\pgfqpoint{2.513549in}{1.342027in}}%
\pgfpathlineto{\pgfqpoint{2.515664in}{1.343062in}}%
\pgfpathlineto{\pgfqpoint{2.517778in}{1.348574in}}%
\pgfpathlineto{\pgfqpoint{2.522008in}{1.346468in}}%
\pgfpathlineto{\pgfqpoint{2.524122in}{1.345630in}}%
\pgfpathlineto{\pgfqpoint{2.526237in}{1.343115in}}%
\pgfpathlineto{\pgfqpoint{2.528352in}{1.343222in}}%
\pgfpathlineto{\pgfqpoint{2.536810in}{1.331952in}}%
\pgfpathlineto{\pgfqpoint{2.541040in}{1.321834in}}%
\pgfpathlineto{\pgfqpoint{2.543155in}{1.324397in}}%
\pgfpathlineto{\pgfqpoint{2.545269in}{1.331070in}}%
\pgfpathlineto{\pgfqpoint{2.549499in}{1.335393in}}%
\pgfpathlineto{\pgfqpoint{2.551613in}{1.331489in}}%
\pgfpathlineto{\pgfqpoint{2.557957in}{1.335012in}}%
\pgfpathlineto{\pgfqpoint{2.562187in}{1.328298in}}%
\pgfpathlineto{\pgfqpoint{2.564301in}{1.326534in}}%
\pgfpathlineto{\pgfqpoint{2.566416in}{1.332445in}}%
\pgfpathlineto{\pgfqpoint{2.572760in}{1.337115in}}%
\pgfpathlineto{\pgfqpoint{2.574875in}{1.339554in}}%
\pgfpathlineto{\pgfqpoint{2.576990in}{1.338203in}}%
\pgfpathlineto{\pgfqpoint{2.581219in}{1.329762in}}%
\pgfpathlineto{\pgfqpoint{2.585448in}{1.337064in}}%
\pgfpathlineto{\pgfqpoint{2.587563in}{1.333039in}}%
\pgfpathlineto{\pgfqpoint{2.589678in}{1.331684in}}%
\pgfpathlineto{\pgfqpoint{2.591792in}{1.331783in}}%
\pgfpathlineto{\pgfqpoint{2.593907in}{1.324982in}}%
\pgfpathlineto{\pgfqpoint{2.596022in}{1.324954in}}%
\pgfpathlineto{\pgfqpoint{2.598137in}{1.323619in}}%
\pgfpathlineto{\pgfqpoint{2.602366in}{1.331279in}}%
\pgfpathlineto{\pgfqpoint{2.604481in}{1.330136in}}%
\pgfpathlineto{\pgfqpoint{2.606595in}{1.323972in}}%
\pgfpathlineto{\pgfqpoint{2.608710in}{1.326124in}}%
\pgfpathlineto{\pgfqpoint{2.612939in}{1.328235in}}%
\pgfpathlineto{\pgfqpoint{2.615054in}{1.319800in}}%
\pgfpathlineto{\pgfqpoint{2.617169in}{1.320575in}}%
\pgfpathlineto{\pgfqpoint{2.619284in}{1.326110in}}%
\pgfpathlineto{\pgfqpoint{2.621398in}{1.326747in}}%
\pgfpathlineto{\pgfqpoint{2.623513in}{1.328907in}}%
\pgfpathlineto{\pgfqpoint{2.627742in}{1.328707in}}%
\pgfpathlineto{\pgfqpoint{2.629857in}{1.332421in}}%
\pgfpathlineto{\pgfqpoint{2.631972in}{1.330321in}}%
\pgfpathlineto{\pgfqpoint{2.638316in}{1.335213in}}%
\pgfpathlineto{\pgfqpoint{2.642545in}{1.322641in}}%
\pgfpathlineto{\pgfqpoint{2.644660in}{1.327970in}}%
\pgfpathlineto{\pgfqpoint{2.646775in}{1.324470in}}%
\pgfpathlineto{\pgfqpoint{2.651004in}{1.325035in}}%
\pgfpathlineto{\pgfqpoint{2.653119in}{1.325792in}}%
\pgfpathlineto{\pgfqpoint{2.655233in}{1.323811in}}%
\pgfpathlineto{\pgfqpoint{2.657348in}{1.323741in}}%
\pgfpathlineto{\pgfqpoint{2.659463in}{1.330577in}}%
\pgfpathlineto{\pgfqpoint{2.661577in}{1.329521in}}%
\pgfpathlineto{\pgfqpoint{2.663692in}{1.325589in}}%
\pgfpathlineto{\pgfqpoint{2.665807in}{1.318937in}}%
\pgfpathlineto{\pgfqpoint{2.667921in}{1.322557in}}%
\pgfpathlineto{\pgfqpoint{2.670036in}{1.319800in}}%
\pgfpathlineto{\pgfqpoint{2.672151in}{1.314838in}}%
\pgfpathlineto{\pgfqpoint{2.674266in}{1.314018in}}%
\pgfpathlineto{\pgfqpoint{2.678495in}{1.319303in}}%
\pgfpathlineto{\pgfqpoint{2.680610in}{1.319775in}}%
\pgfpathlineto{\pgfqpoint{2.684839in}{1.328852in}}%
\pgfpathlineto{\pgfqpoint{2.686954in}{1.328605in}}%
\pgfpathlineto{\pgfqpoint{2.689068in}{1.334164in}}%
\pgfpathlineto{\pgfqpoint{2.693298in}{1.328143in}}%
\pgfpathlineto{\pgfqpoint{2.695412in}{1.328247in}}%
\pgfpathlineto{\pgfqpoint{2.697527in}{1.320708in}}%
\pgfpathlineto{\pgfqpoint{2.699642in}{1.322279in}}%
\pgfpathlineto{\pgfqpoint{2.703871in}{1.336718in}}%
\pgfpathlineto{\pgfqpoint{2.705986in}{1.338415in}}%
\pgfpathlineto{\pgfqpoint{2.708101in}{1.333396in}}%
\pgfpathlineto{\pgfqpoint{2.710215in}{1.337172in}}%
\pgfpathlineto{\pgfqpoint{2.712330in}{1.330252in}}%
\pgfpathlineto{\pgfqpoint{2.716559in}{1.325130in}}%
\pgfpathlineto{\pgfqpoint{2.718674in}{1.321825in}}%
\pgfpathlineto{\pgfqpoint{2.720789in}{1.328080in}}%
\pgfpathlineto{\pgfqpoint{2.722903in}{1.323969in}}%
\pgfpathlineto{\pgfqpoint{2.725018in}{1.325799in}}%
\pgfpathlineto{\pgfqpoint{2.729248in}{1.322036in}}%
\pgfpathlineto{\pgfqpoint{2.731362in}{1.322822in}}%
\pgfpathlineto{\pgfqpoint{2.733477in}{1.328361in}}%
\pgfpathlineto{\pgfqpoint{2.737706in}{1.345053in}}%
\pgfpathlineto{\pgfqpoint{2.739821in}{1.337117in}}%
\pgfpathlineto{\pgfqpoint{2.741936in}{1.334179in}}%
\pgfpathlineto{\pgfqpoint{2.744050in}{1.327383in}}%
\pgfpathlineto{\pgfqpoint{2.746165in}{1.326526in}}%
\pgfpathlineto{\pgfqpoint{2.748280in}{1.321110in}}%
\pgfpathlineto{\pgfqpoint{2.750395in}{1.321306in}}%
\pgfpathlineto{\pgfqpoint{2.756739in}{1.327774in}}%
\pgfpathlineto{\pgfqpoint{2.758853in}{1.323708in}}%
\pgfpathlineto{\pgfqpoint{2.763083in}{1.323954in}}%
\pgfpathlineto{\pgfqpoint{2.765197in}{1.320546in}}%
\pgfpathlineto{\pgfqpoint{2.767312in}{1.319978in}}%
\pgfpathlineto{\pgfqpoint{2.771541in}{1.311322in}}%
\pgfpathlineto{\pgfqpoint{2.773656in}{1.313534in}}%
\pgfpathlineto{\pgfqpoint{2.775771in}{1.313818in}}%
\pgfpathlineto{\pgfqpoint{2.777886in}{1.312972in}}%
\pgfpathlineto{\pgfqpoint{2.780000in}{1.318203in}}%
\pgfpathlineto{\pgfqpoint{2.782115in}{1.316871in}}%
\pgfpathlineto{\pgfqpoint{2.784230in}{1.321049in}}%
\pgfpathlineto{\pgfqpoint{2.786344in}{1.318235in}}%
\pgfpathlineto{\pgfqpoint{2.792688in}{1.329194in}}%
\pgfpathlineto{\pgfqpoint{2.796918in}{1.323423in}}%
\pgfpathlineto{\pgfqpoint{2.799032in}{1.325770in}}%
\pgfpathlineto{\pgfqpoint{2.801147in}{1.331260in}}%
\pgfpathlineto{\pgfqpoint{2.803262in}{1.330495in}}%
\pgfpathlineto{\pgfqpoint{2.807491in}{1.337890in}}%
\pgfpathlineto{\pgfqpoint{2.809606in}{1.347842in}}%
\pgfpathlineto{\pgfqpoint{2.811721in}{1.348836in}}%
\pgfpathlineto{\pgfqpoint{2.813835in}{1.351320in}}%
\pgfpathlineto{\pgfqpoint{2.815950in}{1.351995in}}%
\pgfpathlineto{\pgfqpoint{2.818065in}{1.347338in}}%
\pgfpathlineto{\pgfqpoint{2.820179in}{1.348786in}}%
\pgfpathlineto{\pgfqpoint{2.824409in}{1.357273in}}%
\pgfpathlineto{\pgfqpoint{2.826523in}{1.346878in}}%
\pgfpathlineto{\pgfqpoint{2.830753in}{1.356723in}}%
\pgfpathlineto{\pgfqpoint{2.832868in}{1.354669in}}%
\pgfpathlineto{\pgfqpoint{2.834982in}{1.356609in}}%
\pgfpathlineto{\pgfqpoint{2.837097in}{1.365992in}}%
\pgfpathlineto{\pgfqpoint{2.841326in}{1.368003in}}%
\pgfpathlineto{\pgfqpoint{2.843441in}{1.372399in}}%
\pgfpathlineto{\pgfqpoint{2.845556in}{1.370072in}}%
\pgfpathlineto{\pgfqpoint{2.847670in}{1.370866in}}%
\pgfpathlineto{\pgfqpoint{2.849785in}{1.377048in}}%
\pgfpathlineto{\pgfqpoint{2.851900in}{1.377433in}}%
\pgfpathlineto{\pgfqpoint{2.854015in}{1.382720in}}%
\pgfpathlineto{\pgfqpoint{2.860359in}{1.380735in}}%
\pgfpathlineto{\pgfqpoint{2.864588in}{1.377413in}}%
\pgfpathlineto{\pgfqpoint{2.866703in}{1.378677in}}%
\pgfpathlineto{\pgfqpoint{2.870932in}{1.371940in}}%
\pgfpathlineto{\pgfqpoint{2.877276in}{1.361567in}}%
\pgfpathlineto{\pgfqpoint{2.879391in}{1.359939in}}%
\pgfpathlineto{\pgfqpoint{2.883620in}{1.362972in}}%
\pgfpathlineto{\pgfqpoint{2.885735in}{1.361590in}}%
\pgfpathlineto{\pgfqpoint{2.887850in}{1.356566in}}%
\pgfpathlineto{\pgfqpoint{2.892079in}{1.361221in}}%
\pgfpathlineto{\pgfqpoint{2.894194in}{1.360675in}}%
\pgfpathlineto{\pgfqpoint{2.896308in}{1.357601in}}%
\pgfpathlineto{\pgfqpoint{2.902652in}{1.368864in}}%
\pgfpathlineto{\pgfqpoint{2.904767in}{1.367518in}}%
\pgfpathlineto{\pgfqpoint{2.906882in}{1.372962in}}%
\pgfpathlineto{\pgfqpoint{2.911111in}{1.365039in}}%
\pgfpathlineto{\pgfqpoint{2.913226in}{1.360526in}}%
\pgfpathlineto{\pgfqpoint{2.919570in}{1.356678in}}%
\pgfpathlineto{\pgfqpoint{2.932258in}{1.384596in}}%
\pgfpathlineto{\pgfqpoint{2.938602in}{1.375389in}}%
\pgfpathlineto{\pgfqpoint{2.940717in}{1.383232in}}%
\pgfpathlineto{\pgfqpoint{2.942832in}{1.376483in}}%
\pgfpathlineto{\pgfqpoint{2.949176in}{1.395321in}}%
\pgfpathlineto{\pgfqpoint{2.951290in}{1.396876in}}%
\pgfpathlineto{\pgfqpoint{2.953405in}{1.395437in}}%
\pgfpathlineto{\pgfqpoint{2.955520in}{1.398208in}}%
\pgfpathlineto{\pgfqpoint{2.957634in}{1.396320in}}%
\pgfpathlineto{\pgfqpoint{2.959749in}{1.392310in}}%
\pgfpathlineto{\pgfqpoint{2.961864in}{1.394806in}}%
\pgfpathlineto{\pgfqpoint{2.963979in}{1.395056in}}%
\pgfpathlineto{\pgfqpoint{2.966093in}{1.393576in}}%
\pgfpathlineto{\pgfqpoint{2.968208in}{1.403014in}}%
\pgfpathlineto{\pgfqpoint{2.970323in}{1.402466in}}%
\pgfpathlineto{\pgfqpoint{2.974552in}{1.395541in}}%
\pgfpathlineto{\pgfqpoint{2.983011in}{1.417069in}}%
\pgfpathlineto{\pgfqpoint{2.985126in}{1.413213in}}%
\pgfpathlineto{\pgfqpoint{2.987240in}{1.414768in}}%
\pgfpathlineto{\pgfqpoint{2.989355in}{1.411751in}}%
\pgfpathlineto{\pgfqpoint{2.991470in}{1.416433in}}%
\pgfpathlineto{\pgfqpoint{2.995699in}{1.411297in}}%
\pgfpathlineto{\pgfqpoint{2.997814in}{1.412097in}}%
\pgfpathlineto{\pgfqpoint{2.999928in}{1.416029in}}%
\pgfpathlineto{\pgfqpoint{3.008387in}{1.416047in}}%
\pgfpathlineto{\pgfqpoint{3.010502in}{1.422298in}}%
\pgfpathlineto{\pgfqpoint{3.012617in}{1.418971in}}%
\pgfpathlineto{\pgfqpoint{3.016846in}{1.422897in}}%
\pgfpathlineto{\pgfqpoint{3.018961in}{1.413752in}}%
\pgfpathlineto{\pgfqpoint{3.021075in}{1.413766in}}%
\pgfpathlineto{\pgfqpoint{3.023190in}{1.410833in}}%
\pgfpathlineto{\pgfqpoint{3.025305in}{1.416916in}}%
\pgfpathlineto{\pgfqpoint{3.027419in}{1.416652in}}%
\pgfpathlineto{\pgfqpoint{3.031649in}{1.420196in}}%
\pgfpathlineto{\pgfqpoint{3.033763in}{1.427886in}}%
\pgfpathlineto{\pgfqpoint{3.035878in}{1.425093in}}%
\pgfpathlineto{\pgfqpoint{3.037993in}{1.425880in}}%
\pgfpathlineto{\pgfqpoint{3.042222in}{1.430123in}}%
\pgfpathlineto{\pgfqpoint{3.046452in}{1.430191in}}%
\pgfpathlineto{\pgfqpoint{3.048566in}{1.435823in}}%
\pgfpathlineto{\pgfqpoint{3.050681in}{1.433423in}}%
\pgfpathlineto{\pgfqpoint{3.052796in}{1.438714in}}%
\pgfpathlineto{\pgfqpoint{3.054910in}{1.431401in}}%
\pgfpathlineto{\pgfqpoint{3.057025in}{1.428738in}}%
\pgfpathlineto{\pgfqpoint{3.059140in}{1.430689in}}%
\pgfpathlineto{\pgfqpoint{3.061254in}{1.429200in}}%
\pgfpathlineto{\pgfqpoint{3.063369in}{1.437936in}}%
\pgfpathlineto{\pgfqpoint{3.069713in}{1.444404in}}%
\pgfpathlineto{\pgfqpoint{3.071828in}{1.438717in}}%
\pgfpathlineto{\pgfqpoint{3.073943in}{1.436511in}}%
\pgfpathlineto{\pgfqpoint{3.076057in}{1.442754in}}%
\pgfpathlineto{\pgfqpoint{3.078172in}{1.437277in}}%
\pgfpathlineto{\pgfqpoint{3.084516in}{1.443032in}}%
\pgfpathlineto{\pgfqpoint{3.086631in}{1.437659in}}%
\pgfpathlineto{\pgfqpoint{3.088746in}{1.438887in}}%
\pgfpathlineto{\pgfqpoint{3.092975in}{1.434765in}}%
\pgfpathlineto{\pgfqpoint{3.097204in}{1.425027in}}%
\pgfpathlineto{\pgfqpoint{3.101434in}{1.433049in}}%
\pgfpathlineto{\pgfqpoint{3.103548in}{1.434129in}}%
\pgfpathlineto{\pgfqpoint{3.105663in}{1.432619in}}%
\pgfpathlineto{\pgfqpoint{3.107778in}{1.433476in}}%
\pgfpathlineto{\pgfqpoint{3.109892in}{1.430703in}}%
\pgfpathlineto{\pgfqpoint{3.112007in}{1.433225in}}%
\pgfpathlineto{\pgfqpoint{3.114122in}{1.429031in}}%
\pgfpathlineto{\pgfqpoint{3.116237in}{1.428074in}}%
\pgfpathlineto{\pgfqpoint{3.118351in}{1.428848in}}%
\pgfpathlineto{\pgfqpoint{3.120466in}{1.432894in}}%
\pgfpathlineto{\pgfqpoint{3.122581in}{1.432845in}}%
\pgfpathlineto{\pgfqpoint{3.124695in}{1.430980in}}%
\pgfpathlineto{\pgfqpoint{3.126810in}{1.436616in}}%
\pgfpathlineto{\pgfqpoint{3.128925in}{1.437168in}}%
\pgfpathlineto{\pgfqpoint{3.131039in}{1.440687in}}%
\pgfpathlineto{\pgfqpoint{3.133154in}{1.437179in}}%
\pgfpathlineto{\pgfqpoint{3.135269in}{1.436527in}}%
\pgfpathlineto{\pgfqpoint{3.137383in}{1.443290in}}%
\pgfpathlineto{\pgfqpoint{3.141613in}{1.440034in}}%
\pgfpathlineto{\pgfqpoint{3.143728in}{1.435250in}}%
\pgfpathlineto{\pgfqpoint{3.145842in}{1.436673in}}%
\pgfpathlineto{\pgfqpoint{3.147957in}{1.432890in}}%
\pgfpathlineto{\pgfqpoint{3.150072in}{1.440760in}}%
\pgfpathlineto{\pgfqpoint{3.152186in}{1.443075in}}%
\pgfpathlineto{\pgfqpoint{3.154301in}{1.441614in}}%
\pgfpathlineto{\pgfqpoint{3.158530in}{1.444963in}}%
\pgfpathlineto{\pgfqpoint{3.160645in}{1.444865in}}%
\pgfpathlineto{\pgfqpoint{3.162760in}{1.443588in}}%
\pgfpathlineto{\pgfqpoint{3.164874in}{1.440601in}}%
\pgfpathlineto{\pgfqpoint{3.166989in}{1.441468in}}%
\pgfpathlineto{\pgfqpoint{3.169104in}{1.435421in}}%
\pgfpathlineto{\pgfqpoint{3.171219in}{1.434321in}}%
\pgfpathlineto{\pgfqpoint{3.173333in}{1.435321in}}%
\pgfpathlineto{\pgfqpoint{3.175448in}{1.442176in}}%
\pgfpathlineto{\pgfqpoint{3.177563in}{1.442315in}}%
\pgfpathlineto{\pgfqpoint{3.179677in}{1.447188in}}%
\pgfpathlineto{\pgfqpoint{3.181792in}{1.445729in}}%
\pgfpathlineto{\pgfqpoint{3.183907in}{1.446600in}}%
\pgfpathlineto{\pgfqpoint{3.186021in}{1.438258in}}%
\pgfpathlineto{\pgfqpoint{3.190251in}{1.435275in}}%
\pgfpathlineto{\pgfqpoint{3.194480in}{1.437216in}}%
\pgfpathlineto{\pgfqpoint{3.196595in}{1.436817in}}%
\pgfpathlineto{\pgfqpoint{3.198710in}{1.430062in}}%
\pgfpathlineto{\pgfqpoint{3.200824in}{1.434475in}}%
\pgfpathlineto{\pgfqpoint{3.202939in}{1.432467in}}%
\pgfpathlineto{\pgfqpoint{3.205054in}{1.435527in}}%
\pgfpathlineto{\pgfqpoint{3.209283in}{1.424008in}}%
\pgfpathlineto{\pgfqpoint{3.213512in}{1.434372in}}%
\pgfpathlineto{\pgfqpoint{3.215627in}{1.433819in}}%
\pgfpathlineto{\pgfqpoint{3.217742in}{1.440627in}}%
\pgfpathlineto{\pgfqpoint{3.219857in}{1.441868in}}%
\pgfpathlineto{\pgfqpoint{3.221971in}{1.432850in}}%
\pgfpathlineto{\pgfqpoint{3.224086in}{1.432996in}}%
\pgfpathlineto{\pgfqpoint{3.228315in}{1.427775in}}%
\pgfpathlineto{\pgfqpoint{3.230430in}{1.431562in}}%
\pgfpathlineto{\pgfqpoint{3.234659in}{1.430463in}}%
\pgfpathlineto{\pgfqpoint{3.236774in}{1.424713in}}%
\pgfpathlineto{\pgfqpoint{3.238889in}{1.424513in}}%
\pgfpathlineto{\pgfqpoint{3.241003in}{1.429970in}}%
\pgfpathlineto{\pgfqpoint{3.243118in}{1.426401in}}%
\pgfpathlineto{\pgfqpoint{3.245233in}{1.427663in}}%
\pgfpathlineto{\pgfqpoint{3.247348in}{1.425041in}}%
\pgfpathlineto{\pgfqpoint{3.249462in}{1.418326in}}%
\pgfpathlineto{\pgfqpoint{3.251577in}{1.419194in}}%
\pgfpathlineto{\pgfqpoint{3.253692in}{1.427628in}}%
\pgfpathlineto{\pgfqpoint{3.255806in}{1.424843in}}%
\pgfpathlineto{\pgfqpoint{3.257921in}{1.409206in}}%
\pgfpathlineto{\pgfqpoint{3.260036in}{1.409382in}}%
\pgfpathlineto{\pgfqpoint{3.272724in}{1.384180in}}%
\pgfpathlineto{\pgfqpoint{3.274839in}{1.395243in}}%
\pgfpathlineto{\pgfqpoint{3.279068in}{1.395977in}}%
\pgfpathlineto{\pgfqpoint{3.281183in}{1.391322in}}%
\pgfpathlineto{\pgfqpoint{3.283297in}{1.399005in}}%
\pgfpathlineto{\pgfqpoint{3.285412in}{1.400230in}}%
\pgfpathlineto{\pgfqpoint{3.287527in}{1.405276in}}%
\pgfpathlineto{\pgfqpoint{3.289641in}{1.398166in}}%
\pgfpathlineto{\pgfqpoint{3.291756in}{1.401067in}}%
\pgfpathlineto{\pgfqpoint{3.293871in}{1.401312in}}%
\pgfpathlineto{\pgfqpoint{3.295985in}{1.399807in}}%
\pgfpathlineto{\pgfqpoint{3.302330in}{1.386182in}}%
\pgfpathlineto{\pgfqpoint{3.306559in}{1.397172in}}%
\pgfpathlineto{\pgfqpoint{3.308674in}{1.397025in}}%
\pgfpathlineto{\pgfqpoint{3.310788in}{1.392514in}}%
\pgfpathlineto{\pgfqpoint{3.312903in}{1.392302in}}%
\pgfpathlineto{\pgfqpoint{3.315018in}{1.390915in}}%
\pgfpathlineto{\pgfqpoint{3.317132in}{1.394637in}}%
\pgfpathlineto{\pgfqpoint{3.319247in}{1.391229in}}%
\pgfpathlineto{\pgfqpoint{3.323477in}{1.375942in}}%
\pgfpathlineto{\pgfqpoint{3.325591in}{1.378232in}}%
\pgfpathlineto{\pgfqpoint{3.327706in}{1.376642in}}%
\pgfpathlineto{\pgfqpoint{3.329821in}{1.381044in}}%
\pgfpathlineto{\pgfqpoint{3.331935in}{1.380837in}}%
\pgfpathlineto{\pgfqpoint{3.334050in}{1.383666in}}%
\pgfpathlineto{\pgfqpoint{3.338279in}{1.380864in}}%
\pgfpathlineto{\pgfqpoint{3.342509in}{1.396562in}}%
\pgfpathlineto{\pgfqpoint{3.344623in}{1.392577in}}%
\pgfpathlineto{\pgfqpoint{3.346738in}{1.391290in}}%
\pgfpathlineto{\pgfqpoint{3.350968in}{1.386460in}}%
\pgfpathlineto{\pgfqpoint{3.357312in}{1.377789in}}%
\pgfpathlineto{\pgfqpoint{3.359426in}{1.375089in}}%
\pgfpathlineto{\pgfqpoint{3.361541in}{1.376153in}}%
\pgfpathlineto{\pgfqpoint{3.363656in}{1.379589in}}%
\pgfpathlineto{\pgfqpoint{3.367885in}{1.377856in}}%
\pgfpathlineto{\pgfqpoint{3.370000in}{1.381190in}}%
\pgfpathlineto{\pgfqpoint{3.372114in}{1.377907in}}%
\pgfpathlineto{\pgfqpoint{3.374229in}{1.371640in}}%
\pgfpathlineto{\pgfqpoint{3.378459in}{1.379448in}}%
\pgfpathlineto{\pgfqpoint{3.380573in}{1.377450in}}%
\pgfpathlineto{\pgfqpoint{3.382688in}{1.382732in}}%
\pgfpathlineto{\pgfqpoint{3.384803in}{1.380431in}}%
\pgfpathlineto{\pgfqpoint{3.389032in}{1.381437in}}%
\pgfpathlineto{\pgfqpoint{3.395376in}{1.362234in}}%
\pgfpathlineto{\pgfqpoint{3.397491in}{1.363520in}}%
\pgfpathlineto{\pgfqpoint{3.401720in}{1.358714in}}%
\pgfpathlineto{\pgfqpoint{3.405950in}{1.359131in}}%
\pgfpathlineto{\pgfqpoint{3.408064in}{1.355316in}}%
\pgfpathlineto{\pgfqpoint{3.410179in}{1.359018in}}%
\pgfpathlineto{\pgfqpoint{3.412294in}{1.357572in}}%
\pgfpathlineto{\pgfqpoint{3.414408in}{1.357974in}}%
\pgfpathlineto{\pgfqpoint{3.418638in}{1.354709in}}%
\pgfpathlineto{\pgfqpoint{3.422867in}{1.343608in}}%
\pgfpathlineto{\pgfqpoint{3.427097in}{1.353360in}}%
\pgfpathlineto{\pgfqpoint{3.429211in}{1.352721in}}%
\pgfpathlineto{\pgfqpoint{3.431326in}{1.359728in}}%
\pgfpathlineto{\pgfqpoint{3.433441in}{1.359108in}}%
\pgfpathlineto{\pgfqpoint{3.435555in}{1.360399in}}%
\pgfpathlineto{\pgfqpoint{3.437670in}{1.354038in}}%
\pgfpathlineto{\pgfqpoint{3.439785in}{1.343463in}}%
\pgfpathlineto{\pgfqpoint{3.441899in}{1.339543in}}%
\pgfpathlineto{\pgfqpoint{3.444014in}{1.339317in}}%
\pgfpathlineto{\pgfqpoint{3.446129in}{1.345412in}}%
\pgfpathlineto{\pgfqpoint{3.448243in}{1.342098in}}%
\pgfpathlineto{\pgfqpoint{3.450358in}{1.343902in}}%
\pgfpathlineto{\pgfqpoint{3.452473in}{1.343196in}}%
\pgfpathlineto{\pgfqpoint{3.456702in}{1.333620in}}%
\pgfpathlineto{\pgfqpoint{3.460932in}{1.342307in}}%
\pgfpathlineto{\pgfqpoint{3.463046in}{1.342469in}}%
\pgfpathlineto{\pgfqpoint{3.465161in}{1.335740in}}%
\pgfpathlineto{\pgfqpoint{3.469390in}{1.351166in}}%
\pgfpathlineto{\pgfqpoint{3.471505in}{1.342990in}}%
\pgfpathlineto{\pgfqpoint{3.473620in}{1.346382in}}%
\pgfpathlineto{\pgfqpoint{3.475734in}{1.340131in}}%
\pgfpathlineto{\pgfqpoint{3.482079in}{1.349083in}}%
\pgfpathlineto{\pgfqpoint{3.484193in}{1.347087in}}%
\pgfpathlineto{\pgfqpoint{3.486308in}{1.341659in}}%
\pgfpathlineto{\pgfqpoint{3.488423in}{1.348691in}}%
\pgfpathlineto{\pgfqpoint{3.490537in}{1.348651in}}%
\pgfpathlineto{\pgfqpoint{3.492652in}{1.342297in}}%
\pgfpathlineto{\pgfqpoint{3.496881in}{1.345636in}}%
\pgfpathlineto{\pgfqpoint{3.498996in}{1.349083in}}%
\pgfpathlineto{\pgfqpoint{3.501111in}{1.355633in}}%
\pgfpathlineto{\pgfqpoint{3.509570in}{1.362254in}}%
\pgfpathlineto{\pgfqpoint{3.513799in}{1.353743in}}%
\pgfpathlineto{\pgfqpoint{3.515914in}{1.355955in}}%
\pgfpathlineto{\pgfqpoint{3.518028in}{1.356069in}}%
\pgfpathlineto{\pgfqpoint{3.520143in}{1.359773in}}%
\pgfpathlineto{\pgfqpoint{3.522258in}{1.352944in}}%
\pgfpathlineto{\pgfqpoint{3.526487in}{1.354246in}}%
\pgfpathlineto{\pgfqpoint{3.528602in}{1.346112in}}%
\pgfpathlineto{\pgfqpoint{3.532831in}{1.341562in}}%
\pgfpathlineto{\pgfqpoint{3.534946in}{1.339889in}}%
\pgfpathlineto{\pgfqpoint{3.537061in}{1.334249in}}%
\pgfpathlineto{\pgfqpoint{3.539175in}{1.337536in}}%
\pgfpathlineto{\pgfqpoint{3.541290in}{1.335958in}}%
\pgfpathlineto{\pgfqpoint{3.543405in}{1.340661in}}%
\pgfpathlineto{\pgfqpoint{3.545519in}{1.339480in}}%
\pgfpathlineto{\pgfqpoint{3.549749in}{1.341152in}}%
\pgfpathlineto{\pgfqpoint{3.551863in}{1.343364in}}%
\pgfpathlineto{\pgfqpoint{3.553978in}{1.351107in}}%
\pgfpathlineto{\pgfqpoint{3.556093in}{1.341534in}}%
\pgfpathlineto{\pgfqpoint{3.558208in}{1.343904in}}%
\pgfpathlineto{\pgfqpoint{3.560322in}{1.349938in}}%
\pgfpathlineto{\pgfqpoint{3.562437in}{1.349769in}}%
\pgfpathlineto{\pgfqpoint{3.564552in}{1.341626in}}%
\pgfpathlineto{\pgfqpoint{3.566666in}{1.343527in}}%
\pgfpathlineto{\pgfqpoint{3.570896in}{1.338161in}}%
\pgfpathlineto{\pgfqpoint{3.573010in}{1.341793in}}%
\pgfpathlineto{\pgfqpoint{3.575125in}{1.340769in}}%
\pgfpathlineto{\pgfqpoint{3.579354in}{1.336294in}}%
\pgfpathlineto{\pgfqpoint{3.581469in}{1.343551in}}%
\pgfpathlineto{\pgfqpoint{3.585699in}{1.343044in}}%
\pgfpathlineto{\pgfqpoint{3.587813in}{1.346883in}}%
\pgfpathlineto{\pgfqpoint{3.589928in}{1.338210in}}%
\pgfpathlineto{\pgfqpoint{3.592043in}{1.342939in}}%
\pgfpathlineto{\pgfqpoint{3.594157in}{1.340657in}}%
\pgfpathlineto{\pgfqpoint{3.596272in}{1.335380in}}%
\pgfpathlineto{\pgfqpoint{3.598387in}{1.337538in}}%
\pgfpathlineto{\pgfqpoint{3.600501in}{1.329808in}}%
\pgfpathlineto{\pgfqpoint{3.602616in}{1.328014in}}%
\pgfpathlineto{\pgfqpoint{3.608960in}{1.344969in}}%
\pgfpathlineto{\pgfqpoint{3.613190in}{1.352465in}}%
\pgfpathlineto{\pgfqpoint{3.617419in}{1.356727in}}%
\pgfpathlineto{\pgfqpoint{3.619534in}{1.353763in}}%
\pgfpathlineto{\pgfqpoint{3.621648in}{1.354632in}}%
\pgfpathlineto{\pgfqpoint{3.625878in}{1.363573in}}%
\pgfpathlineto{\pgfqpoint{3.627992in}{1.355433in}}%
\pgfpathlineto{\pgfqpoint{3.630107in}{1.352703in}}%
\pgfpathlineto{\pgfqpoint{3.632222in}{1.352985in}}%
\pgfpathlineto{\pgfqpoint{3.636451in}{1.341987in}}%
\pgfpathlineto{\pgfqpoint{3.638566in}{1.344299in}}%
\pgfpathlineto{\pgfqpoint{3.640681in}{1.344562in}}%
\pgfpathlineto{\pgfqpoint{3.642795in}{1.346406in}}%
\pgfpathlineto{\pgfqpoint{3.644910in}{1.361401in}}%
\pgfpathlineto{\pgfqpoint{3.651254in}{1.357418in}}%
\pgfpathlineto{\pgfqpoint{3.653369in}{1.351872in}}%
\pgfpathlineto{\pgfqpoint{3.655483in}{1.350781in}}%
\pgfpathlineto{\pgfqpoint{3.657598in}{1.354222in}}%
\pgfpathlineto{\pgfqpoint{3.659713in}{1.351664in}}%
\pgfpathlineto{\pgfqpoint{3.661828in}{1.362233in}}%
\pgfpathlineto{\pgfqpoint{3.663942in}{1.359980in}}%
\pgfpathlineto{\pgfqpoint{3.666057in}{1.365985in}}%
\pgfpathlineto{\pgfqpoint{3.670286in}{1.363509in}}%
\pgfpathlineto{\pgfqpoint{3.678745in}{1.378392in}}%
\pgfpathlineto{\pgfqpoint{3.680860in}{1.378473in}}%
\pgfpathlineto{\pgfqpoint{3.682974in}{1.381864in}}%
\pgfpathlineto{\pgfqpoint{3.685089in}{1.375777in}}%
\pgfpathlineto{\pgfqpoint{3.687204in}{1.377468in}}%
\pgfpathlineto{\pgfqpoint{3.691433in}{1.384057in}}%
\pgfpathlineto{\pgfqpoint{3.693548in}{1.386779in}}%
\pgfpathlineto{\pgfqpoint{3.695663in}{1.387012in}}%
\pgfpathlineto{\pgfqpoint{3.697777in}{1.388698in}}%
\pgfpathlineto{\pgfqpoint{3.702007in}{1.385473in}}%
\pgfpathlineto{\pgfqpoint{3.704121in}{1.385713in}}%
\pgfpathlineto{\pgfqpoint{3.708351in}{1.384105in}}%
\pgfpathlineto{\pgfqpoint{3.712580in}{1.389202in}}%
\pgfpathlineto{\pgfqpoint{3.714695in}{1.394033in}}%
\pgfpathlineto{\pgfqpoint{3.716810in}{1.391837in}}%
\pgfpathlineto{\pgfqpoint{3.718924in}{1.391801in}}%
\pgfpathlineto{\pgfqpoint{3.721039in}{1.387367in}}%
\pgfpathlineto{\pgfqpoint{3.723154in}{1.388882in}}%
\pgfpathlineto{\pgfqpoint{3.725268in}{1.403222in}}%
\pgfpathlineto{\pgfqpoint{3.727383in}{1.408502in}}%
\pgfpathlineto{\pgfqpoint{3.731612in}{1.410204in}}%
\pgfpathlineto{\pgfqpoint{3.735842in}{1.411587in}}%
\pgfpathlineto{\pgfqpoint{3.737956in}{1.409885in}}%
\pgfpathlineto{\pgfqpoint{3.740071in}{1.413680in}}%
\pgfpathlineto{\pgfqpoint{3.744301in}{1.407188in}}%
\pgfpathlineto{\pgfqpoint{3.746415in}{1.419700in}}%
\pgfpathlineto{\pgfqpoint{3.750645in}{1.416433in}}%
\pgfpathlineto{\pgfqpoint{3.752759in}{1.417609in}}%
\pgfpathlineto{\pgfqpoint{3.756989in}{1.417486in}}%
\pgfpathlineto{\pgfqpoint{3.759103in}{1.419970in}}%
\pgfpathlineto{\pgfqpoint{3.761218in}{1.426769in}}%
\pgfpathlineto{\pgfqpoint{3.763333in}{1.420040in}}%
\pgfpathlineto{\pgfqpoint{3.765447in}{1.422921in}}%
\pgfpathlineto{\pgfqpoint{3.767562in}{1.422669in}}%
\pgfpathlineto{\pgfqpoint{3.769677in}{1.417008in}}%
\pgfpathlineto{\pgfqpoint{3.771792in}{1.423364in}}%
\pgfpathlineto{\pgfqpoint{3.773906in}{1.416552in}}%
\pgfpathlineto{\pgfqpoint{3.778136in}{1.420977in}}%
\pgfpathlineto{\pgfqpoint{3.780250in}{1.419391in}}%
\pgfpathlineto{\pgfqpoint{3.782365in}{1.423571in}}%
\pgfpathlineto{\pgfqpoint{3.784480in}{1.424746in}}%
\pgfpathlineto{\pgfqpoint{3.788709in}{1.413661in}}%
\pgfpathlineto{\pgfqpoint{3.792939in}{1.412572in}}%
\pgfpathlineto{\pgfqpoint{3.795053in}{1.423869in}}%
\pgfpathlineto{\pgfqpoint{3.797168in}{1.415415in}}%
\pgfpathlineto{\pgfqpoint{3.799283in}{1.415090in}}%
\pgfpathlineto{\pgfqpoint{3.803512in}{1.423627in}}%
\pgfpathlineto{\pgfqpoint{3.805627in}{1.421308in}}%
\pgfpathlineto{\pgfqpoint{3.807741in}{1.424176in}}%
\pgfpathlineto{\pgfqpoint{3.809856in}{1.421356in}}%
\pgfpathlineto{\pgfqpoint{3.811971in}{1.425679in}}%
\pgfpathlineto{\pgfqpoint{3.814085in}{1.423018in}}%
\pgfpathlineto{\pgfqpoint{3.818315in}{1.439611in}}%
\pgfpathlineto{\pgfqpoint{3.820430in}{1.444912in}}%
\pgfpathlineto{\pgfqpoint{3.822544in}{1.442796in}}%
\pgfpathlineto{\pgfqpoint{3.828888in}{1.452263in}}%
\pgfpathlineto{\pgfqpoint{3.831003in}{1.451712in}}%
\pgfpathlineto{\pgfqpoint{3.833118in}{1.453582in}}%
\pgfpathlineto{\pgfqpoint{3.841576in}{1.470400in}}%
\pgfpathlineto{\pgfqpoint{3.843691in}{1.465191in}}%
\pgfpathlineto{\pgfqpoint{3.847921in}{1.472731in}}%
\pgfpathlineto{\pgfqpoint{3.850035in}{1.476381in}}%
\pgfpathlineto{\pgfqpoint{3.854265in}{1.466895in}}%
\pgfpathlineto{\pgfqpoint{3.856379in}{1.468616in}}%
\pgfpathlineto{\pgfqpoint{3.858494in}{1.472582in}}%
\pgfpathlineto{\pgfqpoint{3.860609in}{1.471475in}}%
\pgfpathlineto{\pgfqpoint{3.864838in}{1.480937in}}%
\pgfpathlineto{\pgfqpoint{3.866953in}{1.479396in}}%
\pgfpathlineto{\pgfqpoint{3.869067in}{1.476039in}}%
\pgfpathlineto{\pgfqpoint{3.873297in}{1.478310in}}%
\pgfpathlineto{\pgfqpoint{3.877526in}{1.474583in}}%
\pgfpathlineto{\pgfqpoint{3.879641in}{1.466285in}}%
\pgfpathlineto{\pgfqpoint{3.881756in}{1.467048in}}%
\pgfpathlineto{\pgfqpoint{3.883870in}{1.462495in}}%
\pgfpathlineto{\pgfqpoint{3.885985in}{1.462400in}}%
\pgfpathlineto{\pgfqpoint{3.888100in}{1.463694in}}%
\pgfpathlineto{\pgfqpoint{3.892329in}{1.453912in}}%
\pgfpathlineto{\pgfqpoint{3.894444in}{1.455347in}}%
\pgfpathlineto{\pgfqpoint{3.898673in}{1.452380in}}%
\pgfpathlineto{\pgfqpoint{3.900788in}{1.452702in}}%
\pgfpathlineto{\pgfqpoint{3.902903in}{1.451179in}}%
\pgfpathlineto{\pgfqpoint{3.907132in}{1.451347in}}%
\pgfpathlineto{\pgfqpoint{3.913476in}{1.441161in}}%
\pgfpathlineto{\pgfqpoint{3.917705in}{1.443111in}}%
\pgfpathlineto{\pgfqpoint{3.919820in}{1.442385in}}%
\pgfpathlineto{\pgfqpoint{3.921935in}{1.439882in}}%
\pgfpathlineto{\pgfqpoint{3.924050in}{1.443850in}}%
\pgfpathlineto{\pgfqpoint{3.926164in}{1.439366in}}%
\pgfpathlineto{\pgfqpoint{3.928279in}{1.431329in}}%
\pgfpathlineto{\pgfqpoint{3.930394in}{1.427702in}}%
\pgfpathlineto{\pgfqpoint{3.938852in}{1.426260in}}%
\pgfpathlineto{\pgfqpoint{3.945196in}{1.412074in}}%
\pgfpathlineto{\pgfqpoint{3.949426in}{1.418411in}}%
\pgfpathlineto{\pgfqpoint{3.951541in}{1.410863in}}%
\pgfpathlineto{\pgfqpoint{3.953655in}{1.414043in}}%
\pgfpathlineto{\pgfqpoint{3.955770in}{1.408546in}}%
\pgfpathlineto{\pgfqpoint{3.957885in}{1.408481in}}%
\pgfpathlineto{\pgfqpoint{3.959999in}{1.405080in}}%
\pgfpathlineto{\pgfqpoint{3.962114in}{1.405005in}}%
\pgfpathlineto{\pgfqpoint{3.964229in}{1.406930in}}%
\pgfpathlineto{\pgfqpoint{3.968458in}{1.412256in}}%
\pgfpathlineto{\pgfqpoint{3.970573in}{1.411020in}}%
\pgfpathlineto{\pgfqpoint{3.972687in}{1.415482in}}%
\pgfpathlineto{\pgfqpoint{3.974802in}{1.416724in}}%
\pgfpathlineto{\pgfqpoint{3.976917in}{1.415863in}}%
\pgfpathlineto{\pgfqpoint{3.979032in}{1.424034in}}%
\pgfpathlineto{\pgfqpoint{3.983261in}{1.425863in}}%
\pgfpathlineto{\pgfqpoint{3.985376in}{1.425200in}}%
\pgfpathlineto{\pgfqpoint{3.987490in}{1.428872in}}%
\pgfpathlineto{\pgfqpoint{3.989605in}{1.422729in}}%
\pgfpathlineto{\pgfqpoint{3.991720in}{1.422287in}}%
\pgfpathlineto{\pgfqpoint{3.995949in}{1.431601in}}%
\pgfpathlineto{\pgfqpoint{3.998064in}{1.425678in}}%
\pgfpathlineto{\pgfqpoint{4.002293in}{1.422994in}}%
\pgfpathlineto{\pgfqpoint{4.004408in}{1.424321in}}%
\pgfpathlineto{\pgfqpoint{4.008637in}{1.412474in}}%
\pgfpathlineto{\pgfqpoint{4.010752in}{1.414435in}}%
\pgfpathlineto{\pgfqpoint{4.012867in}{1.414709in}}%
\pgfpathlineto{\pgfqpoint{4.014981in}{1.413327in}}%
\pgfpathlineto{\pgfqpoint{4.017096in}{1.414994in}}%
\pgfpathlineto{\pgfqpoint{4.019211in}{1.413305in}}%
\pgfpathlineto{\pgfqpoint{4.023440in}{1.420814in}}%
\pgfpathlineto{\pgfqpoint{4.025555in}{1.415850in}}%
\pgfpathlineto{\pgfqpoint{4.029784in}{1.413049in}}%
\pgfpathlineto{\pgfqpoint{4.031899in}{1.410139in}}%
\pgfpathlineto{\pgfqpoint{4.034014in}{1.411229in}}%
\pgfpathlineto{\pgfqpoint{4.036128in}{1.406734in}}%
\pgfpathlineto{\pgfqpoint{4.038243in}{1.407142in}}%
\pgfpathlineto{\pgfqpoint{4.040358in}{1.403080in}}%
\pgfpathlineto{\pgfqpoint{4.044587in}{1.407772in}}%
\pgfpathlineto{\pgfqpoint{4.046702in}{1.408163in}}%
\pgfpathlineto{\pgfqpoint{4.053046in}{1.397779in}}%
\pgfpathlineto{\pgfqpoint{4.055161in}{1.396441in}}%
\pgfpathlineto{\pgfqpoint{4.057275in}{1.398122in}}%
\pgfpathlineto{\pgfqpoint{4.059390in}{1.395395in}}%
\pgfpathlineto{\pgfqpoint{4.061505in}{1.398034in}}%
\pgfpathlineto{\pgfqpoint{4.063619in}{1.395911in}}%
\pgfpathlineto{\pgfqpoint{4.065734in}{1.397525in}}%
\pgfpathlineto{\pgfqpoint{4.067849in}{1.401195in}}%
\pgfpathlineto{\pgfqpoint{4.069963in}{1.407227in}}%
\pgfpathlineto{\pgfqpoint{4.072078in}{1.408804in}}%
\pgfpathlineto{\pgfqpoint{4.074193in}{1.419167in}}%
\pgfpathlineto{\pgfqpoint{4.076307in}{1.421709in}}%
\pgfpathlineto{\pgfqpoint{4.080537in}{1.423380in}}%
\pgfpathlineto{\pgfqpoint{4.082652in}{1.421767in}}%
\pgfpathlineto{\pgfqpoint{4.084766in}{1.422631in}}%
\pgfpathlineto{\pgfqpoint{4.086881in}{1.416915in}}%
\pgfpathlineto{\pgfqpoint{4.088996in}{1.422021in}}%
\pgfpathlineto{\pgfqpoint{4.093225in}{1.419392in}}%
\pgfpathlineto{\pgfqpoint{4.095340in}{1.415700in}}%
\pgfpathlineto{\pgfqpoint{4.097454in}{1.421340in}}%
\pgfpathlineto{\pgfqpoint{4.099569in}{1.423228in}}%
\pgfpathlineto{\pgfqpoint{4.103798in}{1.414532in}}%
\pgfpathlineto{\pgfqpoint{4.105913in}{1.424816in}}%
\pgfpathlineto{\pgfqpoint{4.108028in}{1.421783in}}%
\pgfpathlineto{\pgfqpoint{4.110143in}{1.422020in}}%
\pgfpathlineto{\pgfqpoint{4.112257in}{1.419536in}}%
\pgfpathlineto{\pgfqpoint{4.116487in}{1.421548in}}%
\pgfpathlineto{\pgfqpoint{4.118601in}{1.425101in}}%
\pgfpathlineto{\pgfqpoint{4.120716in}{1.424794in}}%
\pgfpathlineto{\pgfqpoint{4.122831in}{1.420110in}}%
\pgfpathlineto{\pgfqpoint{4.127060in}{1.440221in}}%
\pgfpathlineto{\pgfqpoint{4.129175in}{1.444656in}}%
\pgfpathlineto{\pgfqpoint{4.131290in}{1.444552in}}%
\pgfpathlineto{\pgfqpoint{4.133404in}{1.438247in}}%
\pgfpathlineto{\pgfqpoint{4.135519in}{1.443486in}}%
\pgfpathlineto{\pgfqpoint{4.137634in}{1.434887in}}%
\pgfpathlineto{\pgfqpoint{4.139748in}{1.438600in}}%
\pgfpathlineto{\pgfqpoint{4.143978in}{1.435112in}}%
\pgfpathlineto{\pgfqpoint{4.146092in}{1.426801in}}%
\pgfpathlineto{\pgfqpoint{4.148207in}{1.432172in}}%
\pgfpathlineto{\pgfqpoint{4.150322in}{1.424717in}}%
\pgfpathlineto{\pgfqpoint{4.152436in}{1.424025in}}%
\pgfpathlineto{\pgfqpoint{4.154551in}{1.429693in}}%
\pgfpathlineto{\pgfqpoint{4.158781in}{1.432258in}}%
\pgfpathlineto{\pgfqpoint{4.160895in}{1.436998in}}%
\pgfpathlineto{\pgfqpoint{4.163010in}{1.438322in}}%
\pgfpathlineto{\pgfqpoint{4.165125in}{1.441071in}}%
\pgfpathlineto{\pgfqpoint{4.167239in}{1.441891in}}%
\pgfpathlineto{\pgfqpoint{4.169354in}{1.448339in}}%
\pgfpathlineto{\pgfqpoint{4.171469in}{1.447287in}}%
\pgfpathlineto{\pgfqpoint{4.173583in}{1.449789in}}%
\pgfpathlineto{\pgfqpoint{4.177813in}{1.459519in}}%
\pgfpathlineto{\pgfqpoint{4.179927in}{1.462619in}}%
\pgfpathlineto{\pgfqpoint{4.184157in}{1.462347in}}%
\pgfpathlineto{\pgfqpoint{4.186272in}{1.458732in}}%
\pgfpathlineto{\pgfqpoint{4.188386in}{1.468611in}}%
\pgfpathlineto{\pgfqpoint{4.190501in}{1.461137in}}%
\pgfpathlineto{\pgfqpoint{4.192616in}{1.465216in}}%
\pgfpathlineto{\pgfqpoint{4.194730in}{1.456735in}}%
\pgfpathlineto{\pgfqpoint{4.196845in}{1.463274in}}%
\pgfpathlineto{\pgfqpoint{4.198960in}{1.462123in}}%
\pgfpathlineto{\pgfqpoint{4.201074in}{1.463322in}}%
\pgfpathlineto{\pgfqpoint{4.205304in}{1.461936in}}%
\pgfpathlineto{\pgfqpoint{4.207418in}{1.457834in}}%
\pgfpathlineto{\pgfqpoint{4.209533in}{1.457907in}}%
\pgfpathlineto{\pgfqpoint{4.211648in}{1.452107in}}%
\pgfpathlineto{\pgfqpoint{4.213763in}{1.453725in}}%
\pgfpathlineto{\pgfqpoint{4.215877in}{1.453063in}}%
\pgfpathlineto{\pgfqpoint{4.217992in}{1.455669in}}%
\pgfpathlineto{\pgfqpoint{4.222221in}{1.465287in}}%
\pgfpathlineto{\pgfqpoint{4.224336in}{1.463153in}}%
\pgfpathlineto{\pgfqpoint{4.226451in}{1.457304in}}%
\pgfpathlineto{\pgfqpoint{4.228565in}{1.455431in}}%
\pgfpathlineto{\pgfqpoint{4.230680in}{1.459753in}}%
\pgfpathlineto{\pgfqpoint{4.232795in}{1.459878in}}%
\pgfpathlineto{\pgfqpoint{4.234910in}{1.463132in}}%
\pgfpathlineto{\pgfqpoint{4.237024in}{1.451168in}}%
\pgfpathlineto{\pgfqpoint{4.239139in}{1.450767in}}%
\pgfpathlineto{\pgfqpoint{4.243368in}{1.448569in}}%
\pgfpathlineto{\pgfqpoint{4.245483in}{1.453044in}}%
\pgfpathlineto{\pgfqpoint{4.247598in}{1.453249in}}%
\pgfpathlineto{\pgfqpoint{4.249712in}{1.459911in}}%
\pgfpathlineto{\pgfqpoint{4.251827in}{1.455454in}}%
\pgfpathlineto{\pgfqpoint{4.256056in}{1.456836in}}%
\pgfpathlineto{\pgfqpoint{4.258171in}{1.450930in}}%
\pgfpathlineto{\pgfqpoint{4.260286in}{1.452037in}}%
\pgfpathlineto{\pgfqpoint{4.262401in}{1.455945in}}%
\pgfpathlineto{\pgfqpoint{4.264515in}{1.452959in}}%
\pgfpathlineto{\pgfqpoint{4.268745in}{1.461224in}}%
\pgfpathlineto{\pgfqpoint{4.270859in}{1.462220in}}%
\pgfpathlineto{\pgfqpoint{4.272974in}{1.458855in}}%
\pgfpathlineto{\pgfqpoint{4.275089in}{1.458733in}}%
\pgfpathlineto{\pgfqpoint{4.277203in}{1.455799in}}%
\pgfpathlineto{\pgfqpoint{4.279318in}{1.463776in}}%
\pgfpathlineto{\pgfqpoint{4.281433in}{1.463920in}}%
\pgfpathlineto{\pgfqpoint{4.283547in}{1.461180in}}%
\pgfpathlineto{\pgfqpoint{4.285662in}{1.463746in}}%
\pgfpathlineto{\pgfqpoint{4.287777in}{1.455182in}}%
\pgfpathlineto{\pgfqpoint{4.289892in}{1.453430in}}%
\pgfpathlineto{\pgfqpoint{4.292006in}{1.450185in}}%
\pgfpathlineto{\pgfqpoint{4.294121in}{1.456413in}}%
\pgfpathlineto{\pgfqpoint{4.296236in}{1.458354in}}%
\pgfpathlineto{\pgfqpoint{4.298350in}{1.452178in}}%
\pgfpathlineto{\pgfqpoint{4.300465in}{1.451245in}}%
\pgfpathlineto{\pgfqpoint{4.302580in}{1.446434in}}%
\pgfpathlineto{\pgfqpoint{4.304694in}{1.446333in}}%
\pgfpathlineto{\pgfqpoint{4.306809in}{1.442212in}}%
\pgfpathlineto{\pgfqpoint{4.308924in}{1.443700in}}%
\pgfpathlineto{\pgfqpoint{4.311038in}{1.443695in}}%
\pgfpathlineto{\pgfqpoint{4.315268in}{1.439286in}}%
\pgfpathlineto{\pgfqpoint{4.317383in}{1.431077in}}%
\pgfpathlineto{\pgfqpoint{4.319497in}{1.427770in}}%
\pgfpathlineto{\pgfqpoint{4.323727in}{1.441230in}}%
\pgfpathlineto{\pgfqpoint{4.325841in}{1.443436in}}%
\pgfpathlineto{\pgfqpoint{4.327956in}{1.448180in}}%
\pgfpathlineto{\pgfqpoint{4.330071in}{1.449631in}}%
\pgfpathlineto{\pgfqpoint{4.332185in}{1.445406in}}%
\pgfpathlineto{\pgfqpoint{4.336415in}{1.448262in}}%
\pgfpathlineto{\pgfqpoint{4.338529in}{1.446787in}}%
\pgfpathlineto{\pgfqpoint{4.340644in}{1.442993in}}%
\pgfpathlineto{\pgfqpoint{4.342759in}{1.441875in}}%
\pgfpathlineto{\pgfqpoint{4.344874in}{1.446561in}}%
\pgfpathlineto{\pgfqpoint{4.353332in}{1.437038in}}%
\pgfpathlineto{\pgfqpoint{4.355447in}{1.437510in}}%
\pgfpathlineto{\pgfqpoint{4.359676in}{1.441643in}}%
\pgfpathlineto{\pgfqpoint{4.363906in}{1.436293in}}%
\pgfpathlineto{\pgfqpoint{4.366021in}{1.443929in}}%
\pgfpathlineto{\pgfqpoint{4.368135in}{1.447051in}}%
\pgfpathlineto{\pgfqpoint{4.370250in}{1.443200in}}%
\pgfpathlineto{\pgfqpoint{4.372365in}{1.445495in}}%
\pgfpathlineto{\pgfqpoint{4.374479in}{1.442255in}}%
\pgfpathlineto{\pgfqpoint{4.376594in}{1.442524in}}%
\pgfpathlineto{\pgfqpoint{4.378709in}{1.447274in}}%
\pgfpathlineto{\pgfqpoint{4.380823in}{1.439098in}}%
\pgfpathlineto{\pgfqpoint{4.382938in}{1.438078in}}%
\pgfpathlineto{\pgfqpoint{4.385053in}{1.440017in}}%
\pgfpathlineto{\pgfqpoint{4.387167in}{1.439093in}}%
\pgfpathlineto{\pgfqpoint{4.389282in}{1.423918in}}%
\pgfpathlineto{\pgfqpoint{4.393512in}{1.420448in}}%
\pgfpathlineto{\pgfqpoint{4.395626in}{1.424995in}}%
\pgfpathlineto{\pgfqpoint{4.397741in}{1.423494in}}%
\pgfpathlineto{\pgfqpoint{4.399856in}{1.418689in}}%
\pgfpathlineto{\pgfqpoint{4.401970in}{1.418756in}}%
\pgfpathlineto{\pgfqpoint{4.404085in}{1.425588in}}%
\pgfpathlineto{\pgfqpoint{4.414658in}{1.430243in}}%
\pgfpathlineto{\pgfqpoint{4.421003in}{1.422145in}}%
\pgfpathlineto{\pgfqpoint{4.423117in}{1.414441in}}%
\pgfpathlineto{\pgfqpoint{4.425232in}{1.412068in}}%
\pgfpathlineto{\pgfqpoint{4.427347in}{1.411530in}}%
\pgfpathlineto{\pgfqpoint{4.429461in}{1.417898in}}%
\pgfpathlineto{\pgfqpoint{4.435805in}{1.407123in}}%
\pgfpathlineto{\pgfqpoint{4.437920in}{1.407479in}}%
\pgfpathlineto{\pgfqpoint{4.440035in}{1.409259in}}%
\pgfpathlineto{\pgfqpoint{4.442149in}{1.415081in}}%
\pgfpathlineto{\pgfqpoint{4.444264in}{1.413338in}}%
\pgfpathlineto{\pgfqpoint{4.446379in}{1.410139in}}%
\pgfpathlineto{\pgfqpoint{4.450608in}{1.409566in}}%
\pgfpathlineto{\pgfqpoint{4.454838in}{1.403798in}}%
\pgfpathlineto{\pgfqpoint{4.456952in}{1.400866in}}%
\pgfpathlineto{\pgfqpoint{4.459067in}{1.401027in}}%
\pgfpathlineto{\pgfqpoint{4.463296in}{1.398755in}}%
\pgfpathlineto{\pgfqpoint{4.465411in}{1.399742in}}%
\pgfpathlineto{\pgfqpoint{4.467526in}{1.403015in}}%
\pgfpathlineto{\pgfqpoint{4.469641in}{1.401078in}}%
\pgfpathlineto{\pgfqpoint{4.471755in}{1.396317in}}%
\pgfpathlineto{\pgfqpoint{4.473870in}{1.394738in}}%
\pgfpathlineto{\pgfqpoint{4.480214in}{1.399576in}}%
\pgfpathlineto{\pgfqpoint{4.482329in}{1.397802in}}%
\pgfpathlineto{\pgfqpoint{4.484443in}{1.402171in}}%
\pgfpathlineto{\pgfqpoint{4.486558in}{1.395226in}}%
\pgfpathlineto{\pgfqpoint{4.488673in}{1.395670in}}%
\pgfpathlineto{\pgfqpoint{4.490787in}{1.390595in}}%
\pgfpathlineto{\pgfqpoint{4.495017in}{1.391213in}}%
\pgfpathlineto{\pgfqpoint{4.497132in}{1.386835in}}%
\pgfpathlineto{\pgfqpoint{4.501361in}{1.402891in}}%
\pgfpathlineto{\pgfqpoint{4.503476in}{1.403664in}}%
\pgfpathlineto{\pgfqpoint{4.505590in}{1.401369in}}%
\pgfpathlineto{\pgfqpoint{4.507705in}{1.409875in}}%
\pgfpathlineto{\pgfqpoint{4.509820in}{1.412621in}}%
\pgfpathlineto{\pgfqpoint{4.511934in}{1.408783in}}%
\pgfpathlineto{\pgfqpoint{4.516164in}{1.419269in}}%
\pgfpathlineto{\pgfqpoint{4.518278in}{1.419311in}}%
\pgfpathlineto{\pgfqpoint{4.520393in}{1.415116in}}%
\pgfpathlineto{\pgfqpoint{4.524623in}{1.415382in}}%
\pgfpathlineto{\pgfqpoint{4.526737in}{1.405872in}}%
\pgfpathlineto{\pgfqpoint{4.530967in}{1.411811in}}%
\pgfpathlineto{\pgfqpoint{4.533081in}{1.407949in}}%
\pgfpathlineto{\pgfqpoint{4.537311in}{1.417246in}}%
\pgfpathlineto{\pgfqpoint{4.539425in}{1.413550in}}%
\pgfpathlineto{\pgfqpoint{4.541540in}{1.416965in}}%
\pgfpathlineto{\pgfqpoint{4.543655in}{1.416109in}}%
\pgfpathlineto{\pgfqpoint{4.545769in}{1.417003in}}%
\pgfpathlineto{\pgfqpoint{4.547884in}{1.416219in}}%
\pgfpathlineto{\pgfqpoint{4.549999in}{1.409115in}}%
\pgfpathlineto{\pgfqpoint{4.552114in}{1.409620in}}%
\pgfpathlineto{\pgfqpoint{4.554228in}{1.406405in}}%
\pgfpathlineto{\pgfqpoint{4.556343in}{1.407894in}}%
\pgfpathlineto{\pgfqpoint{4.558458in}{1.406389in}}%
\pgfpathlineto{\pgfqpoint{4.560572in}{1.398448in}}%
\pgfpathlineto{\pgfqpoint{4.562687in}{1.404992in}}%
\pgfpathlineto{\pgfqpoint{4.566916in}{1.400003in}}%
\pgfpathlineto{\pgfqpoint{4.569031in}{1.403104in}}%
\pgfpathlineto{\pgfqpoint{4.571146in}{1.400598in}}%
\pgfpathlineto{\pgfqpoint{4.573260in}{1.407532in}}%
\pgfpathlineto{\pgfqpoint{4.577490in}{1.412206in}}%
\pgfpathlineto{\pgfqpoint{4.579605in}{1.403438in}}%
\pgfpathlineto{\pgfqpoint{4.581719in}{1.405453in}}%
\pgfpathlineto{\pgfqpoint{4.585949in}{1.394606in}}%
\pgfpathlineto{\pgfqpoint{4.590178in}{1.402525in}}%
\pgfpathlineto{\pgfqpoint{4.592293in}{1.405166in}}%
\pgfpathlineto{\pgfqpoint{4.594407in}{1.401766in}}%
\pgfpathlineto{\pgfqpoint{4.598637in}{1.402362in}}%
\pgfpathlineto{\pgfqpoint{4.600752in}{1.412245in}}%
\pgfpathlineto{\pgfqpoint{4.602866in}{1.414037in}}%
\pgfpathlineto{\pgfqpoint{4.609210in}{1.427192in}}%
\pgfpathlineto{\pgfqpoint{4.611325in}{1.433342in}}%
\pgfpathlineto{\pgfqpoint{4.613440in}{1.431798in}}%
\pgfpathlineto{\pgfqpoint{4.615554in}{1.436174in}}%
\pgfpathlineto{\pgfqpoint{4.617669in}{1.431909in}}%
\pgfpathlineto{\pgfqpoint{4.619784in}{1.432706in}}%
\pgfpathlineto{\pgfqpoint{4.626128in}{1.449313in}}%
\pgfpathlineto{\pgfqpoint{4.632472in}{1.444375in}}%
\pgfpathlineto{\pgfqpoint{4.636701in}{1.447218in}}%
\pgfpathlineto{\pgfqpoint{4.638816in}{1.445352in}}%
\pgfpathlineto{\pgfqpoint{4.640931in}{1.441141in}}%
\pgfpathlineto{\pgfqpoint{4.645160in}{1.447307in}}%
\pgfpathlineto{\pgfqpoint{4.649389in}{1.442806in}}%
\pgfpathlineto{\pgfqpoint{4.651504in}{1.433525in}}%
\pgfpathlineto{\pgfqpoint{4.655734in}{1.444795in}}%
\pgfpathlineto{\pgfqpoint{4.657848in}{1.446999in}}%
\pgfpathlineto{\pgfqpoint{4.659963in}{1.443571in}}%
\pgfpathlineto{\pgfqpoint{4.662078in}{1.448754in}}%
\pgfpathlineto{\pgfqpoint{4.664192in}{1.446979in}}%
\pgfpathlineto{\pgfqpoint{4.666307in}{1.447063in}}%
\pgfpathlineto{\pgfqpoint{4.668422in}{1.445537in}}%
\pgfpathlineto{\pgfqpoint{4.670536in}{1.445677in}}%
\pgfpathlineto{\pgfqpoint{4.672651in}{1.444038in}}%
\pgfpathlineto{\pgfqpoint{4.674766in}{1.444660in}}%
\pgfpathlineto{\pgfqpoint{4.687454in}{1.421565in}}%
\pgfpathlineto{\pgfqpoint{4.691683in}{1.425999in}}%
\pgfpathlineto{\pgfqpoint{4.693798in}{1.428389in}}%
\pgfpathlineto{\pgfqpoint{4.695913in}{1.432593in}}%
\pgfpathlineto{\pgfqpoint{4.700142in}{1.425760in}}%
\pgfpathlineto{\pgfqpoint{4.704372in}{1.427496in}}%
\pgfpathlineto{\pgfqpoint{4.706486in}{1.422190in}}%
\pgfpathlineto{\pgfqpoint{4.710716in}{1.421528in}}%
\pgfpathlineto{\pgfqpoint{4.712830in}{1.428165in}}%
\pgfpathlineto{\pgfqpoint{4.714945in}{1.422423in}}%
\pgfpathlineto{\pgfqpoint{4.717060in}{1.423295in}}%
\pgfpathlineto{\pgfqpoint{4.723404in}{1.435119in}}%
\pgfpathlineto{\pgfqpoint{4.729748in}{1.423372in}}%
\pgfpathlineto{\pgfqpoint{4.731863in}{1.425901in}}%
\pgfpathlineto{\pgfqpoint{4.740321in}{1.425210in}}%
\pgfpathlineto{\pgfqpoint{4.744551in}{1.413109in}}%
\pgfpathlineto{\pgfqpoint{4.746665in}{1.414970in}}%
\pgfpathlineto{\pgfqpoint{4.748780in}{1.414876in}}%
\pgfpathlineto{\pgfqpoint{4.750895in}{1.411895in}}%
\pgfpathlineto{\pgfqpoint{4.755124in}{1.414760in}}%
\pgfpathlineto{\pgfqpoint{4.757239in}{1.409221in}}%
\pgfpathlineto{\pgfqpoint{4.759354in}{1.411005in}}%
\pgfpathlineto{\pgfqpoint{4.761468in}{1.414979in}}%
\pgfpathlineto{\pgfqpoint{4.765698in}{1.412090in}}%
\pgfpathlineto{\pgfqpoint{4.769927in}{1.404572in}}%
\pgfpathlineto{\pgfqpoint{4.772042in}{1.404562in}}%
\pgfpathlineto{\pgfqpoint{4.774156in}{1.407238in}}%
\pgfpathlineto{\pgfqpoint{4.776271in}{1.413094in}}%
\pgfpathlineto{\pgfqpoint{4.778386in}{1.410398in}}%
\pgfpathlineto{\pgfqpoint{4.780500in}{1.402654in}}%
\pgfpathlineto{\pgfqpoint{4.782615in}{1.402993in}}%
\pgfpathlineto{\pgfqpoint{4.784730in}{1.402102in}}%
\pgfpathlineto{\pgfqpoint{4.791074in}{1.389870in}}%
\pgfpathlineto{\pgfqpoint{4.793189in}{1.391678in}}%
\pgfpathlineto{\pgfqpoint{4.795303in}{1.396344in}}%
\pgfpathlineto{\pgfqpoint{4.797418in}{1.396204in}}%
\pgfpathlineto{\pgfqpoint{4.799533in}{1.391030in}}%
\pgfpathlineto{\pgfqpoint{4.803762in}{1.406026in}}%
\pgfpathlineto{\pgfqpoint{4.805877in}{1.407602in}}%
\pgfpathlineto{\pgfqpoint{4.810106in}{1.413788in}}%
\pgfpathlineto{\pgfqpoint{4.812221in}{1.411145in}}%
\pgfpathlineto{\pgfqpoint{4.814336in}{1.416834in}}%
\pgfpathlineto{\pgfqpoint{4.816450in}{1.418086in}}%
\pgfpathlineto{\pgfqpoint{4.818565in}{1.420846in}}%
\pgfpathlineto{\pgfqpoint{4.820680in}{1.420310in}}%
\pgfpathlineto{\pgfqpoint{4.822794in}{1.417682in}}%
\pgfpathlineto{\pgfqpoint{4.824909in}{1.409552in}}%
\pgfpathlineto{\pgfqpoint{4.833368in}{1.415784in}}%
\pgfpathlineto{\pgfqpoint{4.835483in}{1.425706in}}%
\pgfpathlineto{\pgfqpoint{4.837597in}{1.427329in}}%
\pgfpathlineto{\pgfqpoint{4.839712in}{1.427247in}}%
\pgfpathlineto{\pgfqpoint{4.841827in}{1.429247in}}%
\pgfpathlineto{\pgfqpoint{4.846056in}{1.414827in}}%
\pgfpathlineto{\pgfqpoint{4.848171in}{1.415213in}}%
\pgfpathlineto{\pgfqpoint{4.852400in}{1.420812in}}%
\pgfpathlineto{\pgfqpoint{4.856629in}{1.418293in}}%
\pgfpathlineto{\pgfqpoint{4.858744in}{1.420587in}}%
\pgfpathlineto{\pgfqpoint{4.860859in}{1.419973in}}%
\pgfpathlineto{\pgfqpoint{4.862974in}{1.424426in}}%
\pgfpathlineto{\pgfqpoint{4.865088in}{1.418352in}}%
\pgfpathlineto{\pgfqpoint{4.869318in}{1.421231in}}%
\pgfpathlineto{\pgfqpoint{4.871432in}{1.415772in}}%
\pgfpathlineto{\pgfqpoint{4.873547in}{1.406753in}}%
\pgfpathlineto{\pgfqpoint{4.875662in}{1.405002in}}%
\pgfpathlineto{\pgfqpoint{4.877776in}{1.412050in}}%
\pgfpathlineto{\pgfqpoint{4.879891in}{1.411886in}}%
\pgfpathlineto{\pgfqpoint{4.882006in}{1.404864in}}%
\pgfpathlineto{\pgfqpoint{4.884120in}{1.408919in}}%
\pgfpathlineto{\pgfqpoint{4.890465in}{1.403527in}}%
\pgfpathlineto{\pgfqpoint{4.894694in}{1.406206in}}%
\pgfpathlineto{\pgfqpoint{4.896809in}{1.405684in}}%
\pgfpathlineto{\pgfqpoint{4.901038in}{1.413783in}}%
\pgfpathlineto{\pgfqpoint{4.909497in}{1.412600in}}%
\pgfpathlineto{\pgfqpoint{4.911611in}{1.419491in}}%
\pgfpathlineto{\pgfqpoint{4.913726in}{1.418280in}}%
\pgfpathlineto{\pgfqpoint{4.922185in}{1.399569in}}%
\pgfpathlineto{\pgfqpoint{4.924300in}{1.406730in}}%
\pgfpathlineto{\pgfqpoint{4.928529in}{1.411802in}}%
\pgfpathlineto{\pgfqpoint{4.930644in}{1.414837in}}%
\pgfpathlineto{\pgfqpoint{4.932758in}{1.408974in}}%
\pgfpathlineto{\pgfqpoint{4.936988in}{1.415135in}}%
\pgfpathlineto{\pgfqpoint{4.939103in}{1.421424in}}%
\pgfpathlineto{\pgfqpoint{4.941217in}{1.413126in}}%
\pgfpathlineto{\pgfqpoint{4.943332in}{1.414621in}}%
\pgfpathlineto{\pgfqpoint{4.945447in}{1.418616in}}%
\pgfpathlineto{\pgfqpoint{4.947561in}{1.415291in}}%
\pgfpathlineto{\pgfqpoint{4.951791in}{1.415650in}}%
\pgfpathlineto{\pgfqpoint{4.953905in}{1.417686in}}%
\pgfpathlineto{\pgfqpoint{4.958135in}{1.413747in}}%
\pgfpathlineto{\pgfqpoint{4.962364in}{1.424128in}}%
\pgfpathlineto{\pgfqpoint{4.966594in}{1.413589in}}%
\pgfpathlineto{\pgfqpoint{4.972938in}{1.404319in}}%
\pgfpathlineto{\pgfqpoint{4.975052in}{1.413738in}}%
\pgfpathlineto{\pgfqpoint{4.979282in}{1.413412in}}%
\pgfpathlineto{\pgfqpoint{4.981396in}{1.401522in}}%
\pgfpathlineto{\pgfqpoint{4.983511in}{1.408335in}}%
\pgfpathlineto{\pgfqpoint{4.985626in}{1.409803in}}%
\pgfpathlineto{\pgfqpoint{4.987740in}{1.404674in}}%
\pgfpathlineto{\pgfqpoint{4.991970in}{1.404204in}}%
\pgfpathlineto{\pgfqpoint{4.994085in}{1.407951in}}%
\pgfpathlineto{\pgfqpoint{4.996199in}{1.417709in}}%
\pgfpathlineto{\pgfqpoint{5.000429in}{1.424217in}}%
\pgfpathlineto{\pgfqpoint{5.004658in}{1.407443in}}%
\pgfpathlineto{\pgfqpoint{5.006773in}{1.412331in}}%
\pgfpathlineto{\pgfqpoint{5.008887in}{1.420516in}}%
\pgfpathlineto{\pgfqpoint{5.011002in}{1.420293in}}%
\pgfpathlineto{\pgfqpoint{5.013117in}{1.424612in}}%
\pgfpathlineto{\pgfqpoint{5.015231in}{1.421342in}}%
\pgfpathlineto{\pgfqpoint{5.017346in}{1.424781in}}%
\pgfpathlineto{\pgfqpoint{5.019461in}{1.425946in}}%
\pgfpathlineto{\pgfqpoint{5.021576in}{1.422922in}}%
\pgfpathlineto{\pgfqpoint{5.023690in}{1.422854in}}%
\pgfpathlineto{\pgfqpoint{5.027920in}{1.432539in}}%
\pgfpathlineto{\pgfqpoint{5.030034in}{1.429202in}}%
\pgfpathlineto{\pgfqpoint{5.032149in}{1.433046in}}%
\pgfpathlineto{\pgfqpoint{5.034264in}{1.434316in}}%
\pgfpathlineto{\pgfqpoint{5.036378in}{1.434254in}}%
\pgfpathlineto{\pgfqpoint{5.040608in}{1.441201in}}%
\pgfpathlineto{\pgfqpoint{5.042722in}{1.441181in}}%
\pgfpathlineto{\pgfqpoint{5.044837in}{1.435789in}}%
\pgfpathlineto{\pgfqpoint{5.049067in}{1.438741in}}%
\pgfpathlineto{\pgfqpoint{5.051181in}{1.442560in}}%
\pgfpathlineto{\pgfqpoint{5.053296in}{1.450957in}}%
\pgfpathlineto{\pgfqpoint{5.055411in}{1.450063in}}%
\pgfpathlineto{\pgfqpoint{5.057525in}{1.447622in}}%
\pgfpathlineto{\pgfqpoint{5.059640in}{1.449359in}}%
\pgfpathlineto{\pgfqpoint{5.061755in}{1.447790in}}%
\pgfpathlineto{\pgfqpoint{5.063869in}{1.448201in}}%
\pgfpathlineto{\pgfqpoint{5.068099in}{1.451349in}}%
\pgfpathlineto{\pgfqpoint{5.070214in}{1.448684in}}%
\pgfpathlineto{\pgfqpoint{5.072328in}{1.451846in}}%
\pgfpathlineto{\pgfqpoint{5.074443in}{1.450944in}}%
\pgfpathlineto{\pgfqpoint{5.076558in}{1.447958in}}%
\pgfpathlineto{\pgfqpoint{5.078672in}{1.455731in}}%
\pgfpathlineto{\pgfqpoint{5.080787in}{1.455834in}}%
\pgfpathlineto{\pgfqpoint{5.082902in}{1.461035in}}%
\pgfpathlineto{\pgfqpoint{5.085016in}{1.461511in}}%
\pgfpathlineto{\pgfqpoint{5.087131in}{1.467044in}}%
\pgfpathlineto{\pgfqpoint{5.091360in}{1.470364in}}%
\pgfpathlineto{\pgfqpoint{5.097705in}{1.482091in}}%
\pgfpathlineto{\pgfqpoint{5.101934in}{1.481804in}}%
\pgfpathlineto{\pgfqpoint{5.104049in}{1.477949in}}%
\pgfpathlineto{\pgfqpoint{5.108278in}{1.477280in}}%
\pgfpathlineto{\pgfqpoint{5.110393in}{1.479224in}}%
\pgfpathlineto{\pgfqpoint{5.112507in}{1.478633in}}%
\pgfpathlineto{\pgfqpoint{5.114622in}{1.471654in}}%
\pgfpathlineto{\pgfqpoint{5.116737in}{1.476482in}}%
\pgfpathlineto{\pgfqpoint{5.118851in}{1.475660in}}%
\pgfpathlineto{\pgfqpoint{5.120966in}{1.472827in}}%
\pgfpathlineto{\pgfqpoint{5.123081in}{1.466661in}}%
\pgfpathlineto{\pgfqpoint{5.125196in}{1.467295in}}%
\pgfpathlineto{\pgfqpoint{5.129425in}{1.473737in}}%
\pgfpathlineto{\pgfqpoint{5.131540in}{1.476314in}}%
\pgfpathlineto{\pgfqpoint{5.133654in}{1.476684in}}%
\pgfpathlineto{\pgfqpoint{5.135769in}{1.474193in}}%
\pgfpathlineto{\pgfqpoint{5.137884in}{1.479254in}}%
\pgfpathlineto{\pgfqpoint{5.142113in}{1.479515in}}%
\pgfpathlineto{\pgfqpoint{5.144228in}{1.477619in}}%
\pgfpathlineto{\pgfqpoint{5.146342in}{1.478028in}}%
\pgfpathlineto{\pgfqpoint{5.148457in}{1.483344in}}%
\pgfpathlineto{\pgfqpoint{5.150572in}{1.483977in}}%
\pgfpathlineto{\pgfqpoint{5.152687in}{1.482409in}}%
\pgfpathlineto{\pgfqpoint{5.154801in}{1.477683in}}%
\pgfpathlineto{\pgfqpoint{5.156916in}{1.476345in}}%
\pgfpathlineto{\pgfqpoint{5.159031in}{1.472754in}}%
\pgfpathlineto{\pgfqpoint{5.163260in}{1.456310in}}%
\pgfpathlineto{\pgfqpoint{5.165375in}{1.460105in}}%
\pgfpathlineto{\pgfqpoint{5.167489in}{1.457657in}}%
\pgfpathlineto{\pgfqpoint{5.169604in}{1.458755in}}%
\pgfpathlineto{\pgfqpoint{5.171719in}{1.458298in}}%
\pgfpathlineto{\pgfqpoint{5.173834in}{1.465725in}}%
\pgfpathlineto{\pgfqpoint{5.175948in}{1.462275in}}%
\pgfpathlineto{\pgfqpoint{5.178063in}{1.467908in}}%
\pgfpathlineto{\pgfqpoint{5.182292in}{1.462526in}}%
\pgfpathlineto{\pgfqpoint{5.184407in}{1.463145in}}%
\pgfpathlineto{\pgfqpoint{5.186522in}{1.459058in}}%
\pgfpathlineto{\pgfqpoint{5.188636in}{1.463735in}}%
\pgfpathlineto{\pgfqpoint{5.188636in}{1.463735in}}%
\pgfusepath{stroke}%
\end{pgfscope}%
\begin{pgfscope}%
\pgfpathrectangle{\pgfqpoint{0.750000in}{0.275000in}}{\pgfqpoint{4.650000in}{1.925000in}}%
\pgfusepath{clip}%
\pgfsetroundcap%
\pgfsetroundjoin%
\pgfsetlinewidth{1.003750pt}%
\definecolor{currentstroke}{rgb}{0.600000,0.600000,0.600000}%
\pgfsetstrokecolor{currentstroke}%
\pgfsetdash{}{0pt}%
\pgfpathmoveto{\pgfqpoint{0.961364in}{1.302013in}}%
\pgfpathlineto{\pgfqpoint{0.963478in}{1.291789in}}%
\pgfpathlineto{\pgfqpoint{0.965593in}{1.292980in}}%
\pgfpathlineto{\pgfqpoint{0.971937in}{1.305525in}}%
\pgfpathlineto{\pgfqpoint{0.974052in}{1.304644in}}%
\pgfpathlineto{\pgfqpoint{0.976166in}{1.300214in}}%
\pgfpathlineto{\pgfqpoint{0.978281in}{1.305968in}}%
\pgfpathlineto{\pgfqpoint{0.980396in}{1.307741in}}%
\pgfpathlineto{\pgfqpoint{0.982511in}{1.307390in}}%
\pgfpathlineto{\pgfqpoint{0.986740in}{1.310474in}}%
\pgfpathlineto{\pgfqpoint{0.990969in}{1.304881in}}%
\pgfpathlineto{\pgfqpoint{0.993084in}{1.302481in}}%
\pgfpathlineto{\pgfqpoint{0.999428in}{1.300182in}}%
\pgfpathlineto{\pgfqpoint{1.001543in}{1.306088in}}%
\pgfpathlineto{\pgfqpoint{1.007887in}{1.308250in}}%
\pgfpathlineto{\pgfqpoint{1.010002in}{1.306157in}}%
\pgfpathlineto{\pgfqpoint{1.012116in}{1.299110in}}%
\pgfpathlineto{\pgfqpoint{1.016346in}{1.299625in}}%
\pgfpathlineto{\pgfqpoint{1.018460in}{1.299004in}}%
\pgfpathlineto{\pgfqpoint{1.020575in}{1.303630in}}%
\pgfpathlineto{\pgfqpoint{1.022690in}{1.297699in}}%
\pgfpathlineto{\pgfqpoint{1.024804in}{1.305476in}}%
\pgfpathlineto{\pgfqpoint{1.026919in}{1.306155in}}%
\pgfpathlineto{\pgfqpoint{1.031149in}{1.313013in}}%
\pgfpathlineto{\pgfqpoint{1.033263in}{1.311899in}}%
\pgfpathlineto{\pgfqpoint{1.035378in}{1.315227in}}%
\pgfpathlineto{\pgfqpoint{1.037493in}{1.320724in}}%
\pgfpathlineto{\pgfqpoint{1.039607in}{1.320377in}}%
\pgfpathlineto{\pgfqpoint{1.041722in}{1.315022in}}%
\pgfpathlineto{\pgfqpoint{1.043837in}{1.317902in}}%
\pgfpathlineto{\pgfqpoint{1.045951in}{1.315326in}}%
\pgfpathlineto{\pgfqpoint{1.050181in}{1.317055in}}%
\pgfpathlineto{\pgfqpoint{1.052295in}{1.324636in}}%
\pgfpathlineto{\pgfqpoint{1.054410in}{1.314554in}}%
\pgfpathlineto{\pgfqpoint{1.056525in}{1.309948in}}%
\pgfpathlineto{\pgfqpoint{1.058640in}{1.311550in}}%
\pgfpathlineto{\pgfqpoint{1.067098in}{1.336942in}}%
\pgfpathlineto{\pgfqpoint{1.071328in}{1.322684in}}%
\pgfpathlineto{\pgfqpoint{1.073442in}{1.327330in}}%
\pgfpathlineto{\pgfqpoint{1.075557in}{1.327455in}}%
\pgfpathlineto{\pgfqpoint{1.077672in}{1.328853in}}%
\pgfpathlineto{\pgfqpoint{1.079786in}{1.323238in}}%
\pgfpathlineto{\pgfqpoint{1.081901in}{1.321265in}}%
\pgfpathlineto{\pgfqpoint{1.084016in}{1.317554in}}%
\pgfpathlineto{\pgfqpoint{1.086131in}{1.321379in}}%
\pgfpathlineto{\pgfqpoint{1.090360in}{1.315445in}}%
\pgfpathlineto{\pgfqpoint{1.092475in}{1.324794in}}%
\pgfpathlineto{\pgfqpoint{1.094589in}{1.325429in}}%
\pgfpathlineto{\pgfqpoint{1.096704in}{1.324389in}}%
\pgfpathlineto{\pgfqpoint{1.100933in}{1.316206in}}%
\pgfpathlineto{\pgfqpoint{1.103048in}{1.321450in}}%
\pgfpathlineto{\pgfqpoint{1.105163in}{1.318583in}}%
\pgfpathlineto{\pgfqpoint{1.107278in}{1.318077in}}%
\pgfpathlineto{\pgfqpoint{1.109392in}{1.320215in}}%
\pgfpathlineto{\pgfqpoint{1.111507in}{1.325006in}}%
\pgfpathlineto{\pgfqpoint{1.113622in}{1.325514in}}%
\pgfpathlineto{\pgfqpoint{1.115736in}{1.328512in}}%
\pgfpathlineto{\pgfqpoint{1.117851in}{1.333529in}}%
\pgfpathlineto{\pgfqpoint{1.119966in}{1.335650in}}%
\pgfpathlineto{\pgfqpoint{1.124195in}{1.323618in}}%
\pgfpathlineto{\pgfqpoint{1.126310in}{1.319706in}}%
\pgfpathlineto{\pgfqpoint{1.128424in}{1.328207in}}%
\pgfpathlineto{\pgfqpoint{1.130539in}{1.330839in}}%
\pgfpathlineto{\pgfqpoint{1.132654in}{1.326253in}}%
\pgfpathlineto{\pgfqpoint{1.134769in}{1.327990in}}%
\pgfpathlineto{\pgfqpoint{1.136883in}{1.327984in}}%
\pgfpathlineto{\pgfqpoint{1.138998in}{1.333297in}}%
\pgfpathlineto{\pgfqpoint{1.141113in}{1.331879in}}%
\pgfpathlineto{\pgfqpoint{1.143227in}{1.336910in}}%
\pgfpathlineto{\pgfqpoint{1.145342in}{1.333659in}}%
\pgfpathlineto{\pgfqpoint{1.147457in}{1.334820in}}%
\pgfpathlineto{\pgfqpoint{1.149571in}{1.334519in}}%
\pgfpathlineto{\pgfqpoint{1.151686in}{1.338274in}}%
\pgfpathlineto{\pgfqpoint{1.155915in}{1.329036in}}%
\pgfpathlineto{\pgfqpoint{1.158030in}{1.328154in}}%
\pgfpathlineto{\pgfqpoint{1.160145in}{1.338621in}}%
\pgfpathlineto{\pgfqpoint{1.162260in}{1.339517in}}%
\pgfpathlineto{\pgfqpoint{1.164374in}{1.344675in}}%
\pgfpathlineto{\pgfqpoint{1.170718in}{1.333093in}}%
\pgfpathlineto{\pgfqpoint{1.172833in}{1.337809in}}%
\pgfpathlineto{\pgfqpoint{1.174948in}{1.339561in}}%
\pgfpathlineto{\pgfqpoint{1.177062in}{1.342959in}}%
\pgfpathlineto{\pgfqpoint{1.179177in}{1.336556in}}%
\pgfpathlineto{\pgfqpoint{1.181292in}{1.335730in}}%
\pgfpathlineto{\pgfqpoint{1.183406in}{1.331004in}}%
\pgfpathlineto{\pgfqpoint{1.189751in}{1.324990in}}%
\pgfpathlineto{\pgfqpoint{1.196095in}{1.332680in}}%
\pgfpathlineto{\pgfqpoint{1.198209in}{1.335000in}}%
\pgfpathlineto{\pgfqpoint{1.202439in}{1.333457in}}%
\pgfpathlineto{\pgfqpoint{1.204553in}{1.338245in}}%
\pgfpathlineto{\pgfqpoint{1.208783in}{1.329169in}}%
\pgfpathlineto{\pgfqpoint{1.213012in}{1.344883in}}%
\pgfpathlineto{\pgfqpoint{1.215127in}{1.344703in}}%
\pgfpathlineto{\pgfqpoint{1.217242in}{1.341687in}}%
\pgfpathlineto{\pgfqpoint{1.219356in}{1.344029in}}%
\pgfpathlineto{\pgfqpoint{1.221471in}{1.341346in}}%
\pgfpathlineto{\pgfqpoint{1.223586in}{1.343504in}}%
\pgfpathlineto{\pgfqpoint{1.227815in}{1.352224in}}%
\pgfpathlineto{\pgfqpoint{1.229930in}{1.352205in}}%
\pgfpathlineto{\pgfqpoint{1.232044in}{1.342046in}}%
\pgfpathlineto{\pgfqpoint{1.234159in}{1.346064in}}%
\pgfpathlineto{\pgfqpoint{1.236274in}{1.353184in}}%
\pgfpathlineto{\pgfqpoint{1.238389in}{1.356012in}}%
\pgfpathlineto{\pgfqpoint{1.240503in}{1.354784in}}%
\pgfpathlineto{\pgfqpoint{1.242618in}{1.347010in}}%
\pgfpathlineto{\pgfqpoint{1.244733in}{1.351210in}}%
\pgfpathlineto{\pgfqpoint{1.246847in}{1.349877in}}%
\pgfpathlineto{\pgfqpoint{1.251077in}{1.351017in}}%
\pgfpathlineto{\pgfqpoint{1.253191in}{1.345931in}}%
\pgfpathlineto{\pgfqpoint{1.255306in}{1.347922in}}%
\pgfpathlineto{\pgfqpoint{1.259535in}{1.362318in}}%
\pgfpathlineto{\pgfqpoint{1.261650in}{1.360908in}}%
\pgfpathlineto{\pgfqpoint{1.265880in}{1.368918in}}%
\pgfpathlineto{\pgfqpoint{1.267994in}{1.358449in}}%
\pgfpathlineto{\pgfqpoint{1.270109in}{1.358275in}}%
\pgfpathlineto{\pgfqpoint{1.276453in}{1.342621in}}%
\pgfpathlineto{\pgfqpoint{1.278568in}{1.346713in}}%
\pgfpathlineto{\pgfqpoint{1.280682in}{1.343353in}}%
\pgfpathlineto{\pgfqpoint{1.282797in}{1.342675in}}%
\pgfpathlineto{\pgfqpoint{1.293371in}{1.357634in}}%
\pgfpathlineto{\pgfqpoint{1.295485in}{1.362270in}}%
\pgfpathlineto{\pgfqpoint{1.297600in}{1.362003in}}%
\pgfpathlineto{\pgfqpoint{1.299715in}{1.356383in}}%
\pgfpathlineto{\pgfqpoint{1.303944in}{1.353864in}}%
\pgfpathlineto{\pgfqpoint{1.306059in}{1.360555in}}%
\pgfpathlineto{\pgfqpoint{1.308173in}{1.357460in}}%
\pgfpathlineto{\pgfqpoint{1.312403in}{1.375954in}}%
\pgfpathlineto{\pgfqpoint{1.316632in}{1.375602in}}%
\pgfpathlineto{\pgfqpoint{1.318747in}{1.372026in}}%
\pgfpathlineto{\pgfqpoint{1.322976in}{1.370075in}}%
\pgfpathlineto{\pgfqpoint{1.325091in}{1.375318in}}%
\pgfpathlineto{\pgfqpoint{1.327206in}{1.368916in}}%
\pgfpathlineto{\pgfqpoint{1.333550in}{1.375500in}}%
\pgfpathlineto{\pgfqpoint{1.335664in}{1.373760in}}%
\pgfpathlineto{\pgfqpoint{1.337779in}{1.375284in}}%
\pgfpathlineto{\pgfqpoint{1.339894in}{1.371264in}}%
\pgfpathlineto{\pgfqpoint{1.350467in}{1.383019in}}%
\pgfpathlineto{\pgfqpoint{1.352582in}{1.384211in}}%
\pgfpathlineto{\pgfqpoint{1.354697in}{1.389781in}}%
\pgfpathlineto{\pgfqpoint{1.356811in}{1.385924in}}%
\pgfpathlineto{\pgfqpoint{1.363155in}{1.400308in}}%
\pgfpathlineto{\pgfqpoint{1.367385in}{1.410891in}}%
\pgfpathlineto{\pgfqpoint{1.369500in}{1.411516in}}%
\pgfpathlineto{\pgfqpoint{1.371614in}{1.414727in}}%
\pgfpathlineto{\pgfqpoint{1.373729in}{1.414087in}}%
\pgfpathlineto{\pgfqpoint{1.377958in}{1.409530in}}%
\pgfpathlineto{\pgfqpoint{1.380073in}{1.404973in}}%
\pgfpathlineto{\pgfqpoint{1.384302in}{1.414784in}}%
\pgfpathlineto{\pgfqpoint{1.390646in}{1.401984in}}%
\pgfpathlineto{\pgfqpoint{1.392761in}{1.408106in}}%
\pgfpathlineto{\pgfqpoint{1.394876in}{1.399651in}}%
\pgfpathlineto{\pgfqpoint{1.396991in}{1.399278in}}%
\pgfpathlineto{\pgfqpoint{1.399105in}{1.390542in}}%
\pgfpathlineto{\pgfqpoint{1.401220in}{1.402197in}}%
\pgfpathlineto{\pgfqpoint{1.403335in}{1.403230in}}%
\pgfpathlineto{\pgfqpoint{1.405449in}{1.402182in}}%
\pgfpathlineto{\pgfqpoint{1.407564in}{1.395465in}}%
\pgfpathlineto{\pgfqpoint{1.409679in}{1.397850in}}%
\pgfpathlineto{\pgfqpoint{1.416023in}{1.384968in}}%
\pgfpathlineto{\pgfqpoint{1.418137in}{1.387768in}}%
\pgfpathlineto{\pgfqpoint{1.420252in}{1.388492in}}%
\pgfpathlineto{\pgfqpoint{1.422367in}{1.392564in}}%
\pgfpathlineto{\pgfqpoint{1.424482in}{1.392487in}}%
\pgfpathlineto{\pgfqpoint{1.426596in}{1.386033in}}%
\pgfpathlineto{\pgfqpoint{1.428711in}{1.386093in}}%
\pgfpathlineto{\pgfqpoint{1.430826in}{1.384131in}}%
\pgfpathlineto{\pgfqpoint{1.432940in}{1.387849in}}%
\pgfpathlineto{\pgfqpoint{1.435055in}{1.387992in}}%
\pgfpathlineto{\pgfqpoint{1.437170in}{1.377775in}}%
\pgfpathlineto{\pgfqpoint{1.441399in}{1.381348in}}%
\pgfpathlineto{\pgfqpoint{1.445628in}{1.375170in}}%
\pgfpathlineto{\pgfqpoint{1.447743in}{1.376425in}}%
\pgfpathlineto{\pgfqpoint{1.449858in}{1.375810in}}%
\pgfpathlineto{\pgfqpoint{1.451973in}{1.379693in}}%
\pgfpathlineto{\pgfqpoint{1.458317in}{1.377303in}}%
\pgfpathlineto{\pgfqpoint{1.460431in}{1.379246in}}%
\pgfpathlineto{\pgfqpoint{1.462546in}{1.378231in}}%
\pgfpathlineto{\pgfqpoint{1.464661in}{1.378698in}}%
\pgfpathlineto{\pgfqpoint{1.466775in}{1.387135in}}%
\pgfpathlineto{\pgfqpoint{1.468890in}{1.387479in}}%
\pgfpathlineto{\pgfqpoint{1.471005in}{1.384074in}}%
\pgfpathlineto{\pgfqpoint{1.473120in}{1.389556in}}%
\pgfpathlineto{\pgfqpoint{1.479464in}{1.393909in}}%
\pgfpathlineto{\pgfqpoint{1.481578in}{1.393410in}}%
\pgfpathlineto{\pgfqpoint{1.483693in}{1.390927in}}%
\pgfpathlineto{\pgfqpoint{1.487922in}{1.397833in}}%
\pgfpathlineto{\pgfqpoint{1.490037in}{1.400007in}}%
\pgfpathlineto{\pgfqpoint{1.492152in}{1.406833in}}%
\pgfpathlineto{\pgfqpoint{1.494266in}{1.395312in}}%
\pgfpathlineto{\pgfqpoint{1.496381in}{1.393366in}}%
\pgfpathlineto{\pgfqpoint{1.498496in}{1.397567in}}%
\pgfpathlineto{\pgfqpoint{1.500611in}{1.394102in}}%
\pgfpathlineto{\pgfqpoint{1.502725in}{1.397158in}}%
\pgfpathlineto{\pgfqpoint{1.504840in}{1.391450in}}%
\pgfpathlineto{\pgfqpoint{1.506955in}{1.389388in}}%
\pgfpathlineto{\pgfqpoint{1.511184in}{1.392950in}}%
\pgfpathlineto{\pgfqpoint{1.513299in}{1.388270in}}%
\pgfpathlineto{\pgfqpoint{1.517528in}{1.398566in}}%
\pgfpathlineto{\pgfqpoint{1.519643in}{1.392627in}}%
\pgfpathlineto{\pgfqpoint{1.525987in}{1.407589in}}%
\pgfpathlineto{\pgfqpoint{1.528102in}{1.408932in}}%
\pgfpathlineto{\pgfqpoint{1.530216in}{1.405758in}}%
\pgfpathlineto{\pgfqpoint{1.532331in}{1.407035in}}%
\pgfpathlineto{\pgfqpoint{1.534446in}{1.404837in}}%
\pgfpathlineto{\pgfqpoint{1.538675in}{1.411359in}}%
\pgfpathlineto{\pgfqpoint{1.540790in}{1.416297in}}%
\pgfpathlineto{\pgfqpoint{1.542904in}{1.416734in}}%
\pgfpathlineto{\pgfqpoint{1.547134in}{1.428295in}}%
\pgfpathlineto{\pgfqpoint{1.549248in}{1.423400in}}%
\pgfpathlineto{\pgfqpoint{1.551363in}{1.423242in}}%
\pgfpathlineto{\pgfqpoint{1.555593in}{1.413055in}}%
\pgfpathlineto{\pgfqpoint{1.557707in}{1.405974in}}%
\pgfpathlineto{\pgfqpoint{1.559822in}{1.407826in}}%
\pgfpathlineto{\pgfqpoint{1.564051in}{1.399144in}}%
\pgfpathlineto{\pgfqpoint{1.570395in}{1.411413in}}%
\pgfpathlineto{\pgfqpoint{1.576740in}{1.393875in}}%
\pgfpathlineto{\pgfqpoint{1.580969in}{1.396551in}}%
\pgfpathlineto{\pgfqpoint{1.583084in}{1.402480in}}%
\pgfpathlineto{\pgfqpoint{1.585198in}{1.391617in}}%
\pgfpathlineto{\pgfqpoint{1.587313in}{1.388560in}}%
\pgfpathlineto{\pgfqpoint{1.591542in}{1.387103in}}%
\pgfpathlineto{\pgfqpoint{1.593657in}{1.383780in}}%
\pgfpathlineto{\pgfqpoint{1.595772in}{1.386309in}}%
\pgfpathlineto{\pgfqpoint{1.597886in}{1.393283in}}%
\pgfpathlineto{\pgfqpoint{1.600001in}{1.394323in}}%
\pgfpathlineto{\pgfqpoint{1.606345in}{1.378239in}}%
\pgfpathlineto{\pgfqpoint{1.608460in}{1.373871in}}%
\pgfpathlineto{\pgfqpoint{1.610575in}{1.372109in}}%
\pgfpathlineto{\pgfqpoint{1.612689in}{1.375874in}}%
\pgfpathlineto{\pgfqpoint{1.614804in}{1.368159in}}%
\pgfpathlineto{\pgfqpoint{1.616919in}{1.367412in}}%
\pgfpathlineto{\pgfqpoint{1.619033in}{1.371318in}}%
\pgfpathlineto{\pgfqpoint{1.621148in}{1.378055in}}%
\pgfpathlineto{\pgfqpoint{1.623263in}{1.377446in}}%
\pgfpathlineto{\pgfqpoint{1.625377in}{1.367862in}}%
\pgfpathlineto{\pgfqpoint{1.629607in}{1.373725in}}%
\pgfpathlineto{\pgfqpoint{1.631722in}{1.374749in}}%
\pgfpathlineto{\pgfqpoint{1.633836in}{1.374294in}}%
\pgfpathlineto{\pgfqpoint{1.635951in}{1.383625in}}%
\pgfpathlineto{\pgfqpoint{1.638066in}{1.384988in}}%
\pgfpathlineto{\pgfqpoint{1.640180in}{1.393162in}}%
\pgfpathlineto{\pgfqpoint{1.644410in}{1.385540in}}%
\pgfpathlineto{\pgfqpoint{1.648639in}{1.391587in}}%
\pgfpathlineto{\pgfqpoint{1.652868in}{1.383961in}}%
\pgfpathlineto{\pgfqpoint{1.654983in}{1.383874in}}%
\pgfpathlineto{\pgfqpoint{1.661327in}{1.376744in}}%
\pgfpathlineto{\pgfqpoint{1.667671in}{1.372916in}}%
\pgfpathlineto{\pgfqpoint{1.671901in}{1.377911in}}%
\pgfpathlineto{\pgfqpoint{1.674015in}{1.383119in}}%
\pgfpathlineto{\pgfqpoint{1.676130in}{1.382783in}}%
\pgfpathlineto{\pgfqpoint{1.678245in}{1.384215in}}%
\pgfpathlineto{\pgfqpoint{1.680359in}{1.383998in}}%
\pgfpathlineto{\pgfqpoint{1.682474in}{1.392706in}}%
\pgfpathlineto{\pgfqpoint{1.686704in}{1.385129in}}%
\pgfpathlineto{\pgfqpoint{1.690933in}{1.379783in}}%
\pgfpathlineto{\pgfqpoint{1.693048in}{1.374428in}}%
\pgfpathlineto{\pgfqpoint{1.695162in}{1.380643in}}%
\pgfpathlineto{\pgfqpoint{1.697277in}{1.378220in}}%
\pgfpathlineto{\pgfqpoint{1.699392in}{1.377832in}}%
\pgfpathlineto{\pgfqpoint{1.703621in}{1.385836in}}%
\pgfpathlineto{\pgfqpoint{1.705736in}{1.386089in}}%
\pgfpathlineto{\pgfqpoint{1.709965in}{1.390239in}}%
\pgfpathlineto{\pgfqpoint{1.714195in}{1.386358in}}%
\pgfpathlineto{\pgfqpoint{1.716309in}{1.383994in}}%
\pgfpathlineto{\pgfqpoint{1.718424in}{1.388676in}}%
\pgfpathlineto{\pgfqpoint{1.720539in}{1.387116in}}%
\pgfpathlineto{\pgfqpoint{1.722653in}{1.376767in}}%
\pgfpathlineto{\pgfqpoint{1.724768in}{1.380410in}}%
\pgfpathlineto{\pgfqpoint{1.726883in}{1.381163in}}%
\pgfpathlineto{\pgfqpoint{1.728997in}{1.379008in}}%
\pgfpathlineto{\pgfqpoint{1.733227in}{1.378447in}}%
\pgfpathlineto{\pgfqpoint{1.735342in}{1.383163in}}%
\pgfpathlineto{\pgfqpoint{1.737456in}{1.385138in}}%
\pgfpathlineto{\pgfqpoint{1.739571in}{1.383075in}}%
\pgfpathlineto{\pgfqpoint{1.741686in}{1.388117in}}%
\pgfpathlineto{\pgfqpoint{1.743800in}{1.388900in}}%
\pgfpathlineto{\pgfqpoint{1.750144in}{1.401587in}}%
\pgfpathlineto{\pgfqpoint{1.754374in}{1.411265in}}%
\pgfpathlineto{\pgfqpoint{1.756488in}{1.414266in}}%
\pgfpathlineto{\pgfqpoint{1.758603in}{1.410985in}}%
\pgfpathlineto{\pgfqpoint{1.760718in}{1.423123in}}%
\pgfpathlineto{\pgfqpoint{1.764947in}{1.432411in}}%
\pgfpathlineto{\pgfqpoint{1.767062in}{1.432683in}}%
\pgfpathlineto{\pgfqpoint{1.769177in}{1.428724in}}%
\pgfpathlineto{\pgfqpoint{1.771291in}{1.431314in}}%
\pgfpathlineto{\pgfqpoint{1.773406in}{1.429969in}}%
\pgfpathlineto{\pgfqpoint{1.779750in}{1.445520in}}%
\pgfpathlineto{\pgfqpoint{1.781865in}{1.440853in}}%
\pgfpathlineto{\pgfqpoint{1.783979in}{1.443239in}}%
\pgfpathlineto{\pgfqpoint{1.786094in}{1.436421in}}%
\pgfpathlineto{\pgfqpoint{1.788209in}{1.436483in}}%
\pgfpathlineto{\pgfqpoint{1.790324in}{1.428565in}}%
\pgfpathlineto{\pgfqpoint{1.792438in}{1.435259in}}%
\pgfpathlineto{\pgfqpoint{1.794553in}{1.436827in}}%
\pgfpathlineto{\pgfqpoint{1.796668in}{1.434626in}}%
\pgfpathlineto{\pgfqpoint{1.798782in}{1.420544in}}%
\pgfpathlineto{\pgfqpoint{1.805126in}{1.410123in}}%
\pgfpathlineto{\pgfqpoint{1.807241in}{1.400688in}}%
\pgfpathlineto{\pgfqpoint{1.811471in}{1.397859in}}%
\pgfpathlineto{\pgfqpoint{1.813585in}{1.400524in}}%
\pgfpathlineto{\pgfqpoint{1.817815in}{1.397838in}}%
\pgfpathlineto{\pgfqpoint{1.819929in}{1.404652in}}%
\pgfpathlineto{\pgfqpoint{1.822044in}{1.406878in}}%
\pgfpathlineto{\pgfqpoint{1.826273in}{1.405659in}}%
\pgfpathlineto{\pgfqpoint{1.828388in}{1.403839in}}%
\pgfpathlineto{\pgfqpoint{1.832617in}{1.406297in}}%
\pgfpathlineto{\pgfqpoint{1.834732in}{1.411662in}}%
\pgfpathlineto{\pgfqpoint{1.836847in}{1.421413in}}%
\pgfpathlineto{\pgfqpoint{1.838962in}{1.421525in}}%
\pgfpathlineto{\pgfqpoint{1.841076in}{1.423264in}}%
\pgfpathlineto{\pgfqpoint{1.843191in}{1.419949in}}%
\pgfpathlineto{\pgfqpoint{1.845306in}{1.420276in}}%
\pgfpathlineto{\pgfqpoint{1.847420in}{1.423649in}}%
\pgfpathlineto{\pgfqpoint{1.849535in}{1.434629in}}%
\pgfpathlineto{\pgfqpoint{1.851650in}{1.431638in}}%
\pgfpathlineto{\pgfqpoint{1.853764in}{1.430872in}}%
\pgfpathlineto{\pgfqpoint{1.855879in}{1.425843in}}%
\pgfpathlineto{\pgfqpoint{1.857994in}{1.424094in}}%
\pgfpathlineto{\pgfqpoint{1.860108in}{1.429288in}}%
\pgfpathlineto{\pgfqpoint{1.866453in}{1.424373in}}%
\pgfpathlineto{\pgfqpoint{1.870682in}{1.426809in}}%
\pgfpathlineto{\pgfqpoint{1.874911in}{1.419052in}}%
\pgfpathlineto{\pgfqpoint{1.877026in}{1.419600in}}%
\pgfpathlineto{\pgfqpoint{1.879141in}{1.411929in}}%
\pgfpathlineto{\pgfqpoint{1.883370in}{1.406659in}}%
\pgfpathlineto{\pgfqpoint{1.885485in}{1.412123in}}%
\pgfpathlineto{\pgfqpoint{1.887599in}{1.412116in}}%
\pgfpathlineto{\pgfqpoint{1.889714in}{1.403533in}}%
\pgfpathlineto{\pgfqpoint{1.891829in}{1.406602in}}%
\pgfpathlineto{\pgfqpoint{1.893944in}{1.418423in}}%
\pgfpathlineto{\pgfqpoint{1.896058in}{1.418126in}}%
\pgfpathlineto{\pgfqpoint{1.898173in}{1.414661in}}%
\pgfpathlineto{\pgfqpoint{1.900288in}{1.414669in}}%
\pgfpathlineto{\pgfqpoint{1.902402in}{1.417064in}}%
\pgfpathlineto{\pgfqpoint{1.906632in}{1.430326in}}%
\pgfpathlineto{\pgfqpoint{1.908746in}{1.428924in}}%
\pgfpathlineto{\pgfqpoint{1.912976in}{1.435294in}}%
\pgfpathlineto{\pgfqpoint{1.919320in}{1.444220in}}%
\pgfpathlineto{\pgfqpoint{1.921435in}{1.440962in}}%
\pgfpathlineto{\pgfqpoint{1.923549in}{1.440691in}}%
\pgfpathlineto{\pgfqpoint{1.925664in}{1.442736in}}%
\pgfpathlineto{\pgfqpoint{1.934123in}{1.426653in}}%
\pgfpathlineto{\pgfqpoint{1.936237in}{1.426336in}}%
\pgfpathlineto{\pgfqpoint{1.938352in}{1.430883in}}%
\pgfpathlineto{\pgfqpoint{1.940467in}{1.429327in}}%
\pgfpathlineto{\pgfqpoint{1.942582in}{1.430250in}}%
\pgfpathlineto{\pgfqpoint{1.946811in}{1.427741in}}%
\pgfpathlineto{\pgfqpoint{1.948926in}{1.427476in}}%
\pgfpathlineto{\pgfqpoint{1.951040in}{1.422490in}}%
\pgfpathlineto{\pgfqpoint{1.953155in}{1.426660in}}%
\pgfpathlineto{\pgfqpoint{1.959499in}{1.430562in}}%
\pgfpathlineto{\pgfqpoint{1.961614in}{1.436401in}}%
\pgfpathlineto{\pgfqpoint{1.963728in}{1.434162in}}%
\pgfpathlineto{\pgfqpoint{1.967958in}{1.439363in}}%
\pgfpathlineto{\pgfqpoint{1.970073in}{1.439164in}}%
\pgfpathlineto{\pgfqpoint{1.972187in}{1.440176in}}%
\pgfpathlineto{\pgfqpoint{1.974302in}{1.442598in}}%
\pgfpathlineto{\pgfqpoint{1.976417in}{1.439578in}}%
\pgfpathlineto{\pgfqpoint{1.978531in}{1.441135in}}%
\pgfpathlineto{\pgfqpoint{1.980646in}{1.438836in}}%
\pgfpathlineto{\pgfqpoint{1.982761in}{1.441018in}}%
\pgfpathlineto{\pgfqpoint{1.984875in}{1.439412in}}%
\pgfpathlineto{\pgfqpoint{1.986990in}{1.436092in}}%
\pgfpathlineto{\pgfqpoint{1.989105in}{1.439368in}}%
\pgfpathlineto{\pgfqpoint{1.991219in}{1.436287in}}%
\pgfpathlineto{\pgfqpoint{1.995449in}{1.444540in}}%
\pgfpathlineto{\pgfqpoint{1.997564in}{1.439616in}}%
\pgfpathlineto{\pgfqpoint{1.999678in}{1.445362in}}%
\pgfpathlineto{\pgfqpoint{2.003908in}{1.443586in}}%
\pgfpathlineto{\pgfqpoint{2.006022in}{1.445269in}}%
\pgfpathlineto{\pgfqpoint{2.008137in}{1.440437in}}%
\pgfpathlineto{\pgfqpoint{2.010252in}{1.447736in}}%
\pgfpathlineto{\pgfqpoint{2.012366in}{1.460928in}}%
\pgfpathlineto{\pgfqpoint{2.014481in}{1.463417in}}%
\pgfpathlineto{\pgfqpoint{2.016596in}{1.462744in}}%
\pgfpathlineto{\pgfqpoint{2.018710in}{1.469180in}}%
\pgfpathlineto{\pgfqpoint{2.020825in}{1.470722in}}%
\pgfpathlineto{\pgfqpoint{2.022940in}{1.469758in}}%
\pgfpathlineto{\pgfqpoint{2.025055in}{1.470245in}}%
\pgfpathlineto{\pgfqpoint{2.027169in}{1.469520in}}%
\pgfpathlineto{\pgfqpoint{2.029284in}{1.462292in}}%
\pgfpathlineto{\pgfqpoint{2.031399in}{1.468941in}}%
\pgfpathlineto{\pgfqpoint{2.035628in}{1.462186in}}%
\pgfpathlineto{\pgfqpoint{2.037743in}{1.463051in}}%
\pgfpathlineto{\pgfqpoint{2.039857in}{1.461209in}}%
\pgfpathlineto{\pgfqpoint{2.041972in}{1.463667in}}%
\pgfpathlineto{\pgfqpoint{2.044087in}{1.463305in}}%
\pgfpathlineto{\pgfqpoint{2.046202in}{1.465012in}}%
\pgfpathlineto{\pgfqpoint{2.050431in}{1.461305in}}%
\pgfpathlineto{\pgfqpoint{2.052546in}{1.460432in}}%
\pgfpathlineto{\pgfqpoint{2.056775in}{1.451662in}}%
\pgfpathlineto{\pgfqpoint{2.058890in}{1.450239in}}%
\pgfpathlineto{\pgfqpoint{2.061004in}{1.454544in}}%
\pgfpathlineto{\pgfqpoint{2.063119in}{1.446681in}}%
\pgfpathlineto{\pgfqpoint{2.065234in}{1.450358in}}%
\pgfpathlineto{\pgfqpoint{2.067348in}{1.445954in}}%
\pgfpathlineto{\pgfqpoint{2.069463in}{1.448857in}}%
\pgfpathlineto{\pgfqpoint{2.071578in}{1.441199in}}%
\pgfpathlineto{\pgfqpoint{2.073693in}{1.441108in}}%
\pgfpathlineto{\pgfqpoint{2.075807in}{1.432591in}}%
\pgfpathlineto{\pgfqpoint{2.077922in}{1.431986in}}%
\pgfpathlineto{\pgfqpoint{2.080037in}{1.434849in}}%
\pgfpathlineto{\pgfqpoint{2.082151in}{1.432406in}}%
\pgfpathlineto{\pgfqpoint{2.084266in}{1.432075in}}%
\pgfpathlineto{\pgfqpoint{2.086381in}{1.423766in}}%
\pgfpathlineto{\pgfqpoint{2.088495in}{1.428263in}}%
\pgfpathlineto{\pgfqpoint{2.090610in}{1.428268in}}%
\pgfpathlineto{\pgfqpoint{2.092725in}{1.425174in}}%
\pgfpathlineto{\pgfqpoint{2.099069in}{1.433542in}}%
\pgfpathlineto{\pgfqpoint{2.101184in}{1.436677in}}%
\pgfpathlineto{\pgfqpoint{2.103298in}{1.433532in}}%
\pgfpathlineto{\pgfqpoint{2.105413in}{1.427743in}}%
\pgfpathlineto{\pgfqpoint{2.107528in}{1.433719in}}%
\pgfpathlineto{\pgfqpoint{2.111757in}{1.432173in}}%
\pgfpathlineto{\pgfqpoint{2.113872in}{1.430041in}}%
\pgfpathlineto{\pgfqpoint{2.115986in}{1.430987in}}%
\pgfpathlineto{\pgfqpoint{2.120216in}{1.443646in}}%
\pgfpathlineto{\pgfqpoint{2.122330in}{1.437312in}}%
\pgfpathlineto{\pgfqpoint{2.124445in}{1.435455in}}%
\pgfpathlineto{\pgfqpoint{2.126560in}{1.428690in}}%
\pgfpathlineto{\pgfqpoint{2.128675in}{1.427635in}}%
\pgfpathlineto{\pgfqpoint{2.130789in}{1.430942in}}%
\pgfpathlineto{\pgfqpoint{2.135019in}{1.425557in}}%
\pgfpathlineto{\pgfqpoint{2.137133in}{1.430349in}}%
\pgfpathlineto{\pgfqpoint{2.141363in}{1.419393in}}%
\pgfpathlineto{\pgfqpoint{2.143477in}{1.419601in}}%
\pgfpathlineto{\pgfqpoint{2.145592in}{1.421257in}}%
\pgfpathlineto{\pgfqpoint{2.149822in}{1.428454in}}%
\pgfpathlineto{\pgfqpoint{2.151936in}{1.430186in}}%
\pgfpathlineto{\pgfqpoint{2.154051in}{1.428727in}}%
\pgfpathlineto{\pgfqpoint{2.156166in}{1.422412in}}%
\pgfpathlineto{\pgfqpoint{2.162510in}{1.417963in}}%
\pgfpathlineto{\pgfqpoint{2.166739in}{1.411498in}}%
\pgfpathlineto{\pgfqpoint{2.168854in}{1.410771in}}%
\pgfpathlineto{\pgfqpoint{2.170968in}{1.408057in}}%
\pgfpathlineto{\pgfqpoint{2.173083in}{1.410907in}}%
\pgfpathlineto{\pgfqpoint{2.175198in}{1.405667in}}%
\pgfpathlineto{\pgfqpoint{2.177313in}{1.407890in}}%
\pgfpathlineto{\pgfqpoint{2.179427in}{1.412451in}}%
\pgfpathlineto{\pgfqpoint{2.183657in}{1.426768in}}%
\pgfpathlineto{\pgfqpoint{2.190001in}{1.435850in}}%
\pgfpathlineto{\pgfqpoint{2.192115in}{1.430650in}}%
\pgfpathlineto{\pgfqpoint{2.198459in}{1.438489in}}%
\pgfpathlineto{\pgfqpoint{2.200574in}{1.443087in}}%
\pgfpathlineto{\pgfqpoint{2.202689in}{1.440123in}}%
\pgfpathlineto{\pgfqpoint{2.204804in}{1.446571in}}%
\pgfpathlineto{\pgfqpoint{2.209033in}{1.441703in}}%
\pgfpathlineto{\pgfqpoint{2.213262in}{1.452777in}}%
\pgfpathlineto{\pgfqpoint{2.215377in}{1.452394in}}%
\pgfpathlineto{\pgfqpoint{2.217492in}{1.443236in}}%
\pgfpathlineto{\pgfqpoint{2.219606in}{1.444512in}}%
\pgfpathlineto{\pgfqpoint{2.221721in}{1.438960in}}%
\pgfpathlineto{\pgfqpoint{2.223836in}{1.439148in}}%
\pgfpathlineto{\pgfqpoint{2.225950in}{1.444370in}}%
\pgfpathlineto{\pgfqpoint{2.228065in}{1.442865in}}%
\pgfpathlineto{\pgfqpoint{2.230180in}{1.439854in}}%
\pgfpathlineto{\pgfqpoint{2.232295in}{1.438980in}}%
\pgfpathlineto{\pgfqpoint{2.234409in}{1.428482in}}%
\pgfpathlineto{\pgfqpoint{2.236524in}{1.429644in}}%
\pgfpathlineto{\pgfqpoint{2.238639in}{1.436613in}}%
\pgfpathlineto{\pgfqpoint{2.242868in}{1.436344in}}%
\pgfpathlineto{\pgfqpoint{2.244983in}{1.432413in}}%
\pgfpathlineto{\pgfqpoint{2.247097in}{1.431601in}}%
\pgfpathlineto{\pgfqpoint{2.249212in}{1.427417in}}%
\pgfpathlineto{\pgfqpoint{2.251327in}{1.428173in}}%
\pgfpathlineto{\pgfqpoint{2.253441in}{1.430355in}}%
\pgfpathlineto{\pgfqpoint{2.257671in}{1.430595in}}%
\pgfpathlineto{\pgfqpoint{2.259786in}{1.421367in}}%
\pgfpathlineto{\pgfqpoint{2.264015in}{1.417818in}}%
\pgfpathlineto{\pgfqpoint{2.266130in}{1.422619in}}%
\pgfpathlineto{\pgfqpoint{2.268244in}{1.422542in}}%
\pgfpathlineto{\pgfqpoint{2.272474in}{1.414868in}}%
\pgfpathlineto{\pgfqpoint{2.274588in}{1.417075in}}%
\pgfpathlineto{\pgfqpoint{2.276703in}{1.415663in}}%
\pgfpathlineto{\pgfqpoint{2.278818in}{1.416641in}}%
\pgfpathlineto{\pgfqpoint{2.280933in}{1.415477in}}%
\pgfpathlineto{\pgfqpoint{2.283047in}{1.408906in}}%
\pgfpathlineto{\pgfqpoint{2.285162in}{1.406686in}}%
\pgfpathlineto{\pgfqpoint{2.287277in}{1.407919in}}%
\pgfpathlineto{\pgfqpoint{2.291506in}{1.406708in}}%
\pgfpathlineto{\pgfqpoint{2.293621in}{1.409574in}}%
\pgfpathlineto{\pgfqpoint{2.295735in}{1.404999in}}%
\pgfpathlineto{\pgfqpoint{2.297850in}{1.403158in}}%
\pgfpathlineto{\pgfqpoint{2.299965in}{1.398733in}}%
\pgfpathlineto{\pgfqpoint{2.306309in}{1.410825in}}%
\pgfpathlineto{\pgfqpoint{2.308424in}{1.408632in}}%
\pgfpathlineto{\pgfqpoint{2.310538in}{1.403101in}}%
\pgfpathlineto{\pgfqpoint{2.312653in}{1.400954in}}%
\pgfpathlineto{\pgfqpoint{2.314768in}{1.404380in}}%
\pgfpathlineto{\pgfqpoint{2.321112in}{1.392193in}}%
\pgfpathlineto{\pgfqpoint{2.323226in}{1.388970in}}%
\pgfpathlineto{\pgfqpoint{2.325341in}{1.393860in}}%
\pgfpathlineto{\pgfqpoint{2.329570in}{1.395889in}}%
\pgfpathlineto{\pgfqpoint{2.331685in}{1.390674in}}%
\pgfpathlineto{\pgfqpoint{2.333800in}{1.395401in}}%
\pgfpathlineto{\pgfqpoint{2.335915in}{1.395839in}}%
\pgfpathlineto{\pgfqpoint{2.338029in}{1.404005in}}%
\pgfpathlineto{\pgfqpoint{2.340144in}{1.399379in}}%
\pgfpathlineto{\pgfqpoint{2.342259in}{1.401698in}}%
\pgfpathlineto{\pgfqpoint{2.344373in}{1.399524in}}%
\pgfpathlineto{\pgfqpoint{2.348603in}{1.412229in}}%
\pgfpathlineto{\pgfqpoint{2.350717in}{1.414490in}}%
\pgfpathlineto{\pgfqpoint{2.352832in}{1.406932in}}%
\pgfpathlineto{\pgfqpoint{2.354947in}{1.408977in}}%
\pgfpathlineto{\pgfqpoint{2.357061in}{1.408327in}}%
\pgfpathlineto{\pgfqpoint{2.361291in}{1.410935in}}%
\pgfpathlineto{\pgfqpoint{2.363406in}{1.423512in}}%
\pgfpathlineto{\pgfqpoint{2.367635in}{1.425191in}}%
\pgfpathlineto{\pgfqpoint{2.369750in}{1.422393in}}%
\pgfpathlineto{\pgfqpoint{2.371864in}{1.432993in}}%
\pgfpathlineto{\pgfqpoint{2.373979in}{1.433921in}}%
\pgfpathlineto{\pgfqpoint{2.378208in}{1.442664in}}%
\pgfpathlineto{\pgfqpoint{2.380323in}{1.443606in}}%
\pgfpathlineto{\pgfqpoint{2.382438in}{1.439245in}}%
\pgfpathlineto{\pgfqpoint{2.384553in}{1.441109in}}%
\pgfpathlineto{\pgfqpoint{2.386667in}{1.447276in}}%
\pgfpathlineto{\pgfqpoint{2.388782in}{1.448952in}}%
\pgfpathlineto{\pgfqpoint{2.390897in}{1.447972in}}%
\pgfpathlineto{\pgfqpoint{2.393011in}{1.452384in}}%
\pgfpathlineto{\pgfqpoint{2.395126in}{1.448064in}}%
\pgfpathlineto{\pgfqpoint{2.397241in}{1.450732in}}%
\pgfpathlineto{\pgfqpoint{2.399355in}{1.449501in}}%
\pgfpathlineto{\pgfqpoint{2.403585in}{1.437259in}}%
\pgfpathlineto{\pgfqpoint{2.405699in}{1.445033in}}%
\pgfpathlineto{\pgfqpoint{2.407814in}{1.445181in}}%
\pgfpathlineto{\pgfqpoint{2.412044in}{1.449858in}}%
\pgfpathlineto{\pgfqpoint{2.414158in}{1.439857in}}%
\pgfpathlineto{\pgfqpoint{2.416273in}{1.446518in}}%
\pgfpathlineto{\pgfqpoint{2.418388in}{1.443669in}}%
\pgfpathlineto{\pgfqpoint{2.420502in}{1.443325in}}%
\pgfpathlineto{\pgfqpoint{2.422617in}{1.439266in}}%
\pgfpathlineto{\pgfqpoint{2.424732in}{1.430932in}}%
\pgfpathlineto{\pgfqpoint{2.426846in}{1.429982in}}%
\pgfpathlineto{\pgfqpoint{2.428961in}{1.433559in}}%
\pgfpathlineto{\pgfqpoint{2.431076in}{1.433762in}}%
\pgfpathlineto{\pgfqpoint{2.433190in}{1.440402in}}%
\pgfpathlineto{\pgfqpoint{2.435305in}{1.437394in}}%
\pgfpathlineto{\pgfqpoint{2.437420in}{1.444118in}}%
\pgfpathlineto{\pgfqpoint{2.439535in}{1.437765in}}%
\pgfpathlineto{\pgfqpoint{2.441649in}{1.435110in}}%
\pgfpathlineto{\pgfqpoint{2.443764in}{1.440416in}}%
\pgfpathlineto{\pgfqpoint{2.445879in}{1.442026in}}%
\pgfpathlineto{\pgfqpoint{2.447993in}{1.440933in}}%
\pgfpathlineto{\pgfqpoint{2.450108in}{1.443488in}}%
\pgfpathlineto{\pgfqpoint{2.452223in}{1.448915in}}%
\pgfpathlineto{\pgfqpoint{2.456452in}{1.441540in}}%
\pgfpathlineto{\pgfqpoint{2.458567in}{1.434394in}}%
\pgfpathlineto{\pgfqpoint{2.460681in}{1.433357in}}%
\pgfpathlineto{\pgfqpoint{2.462796in}{1.426166in}}%
\pgfpathlineto{\pgfqpoint{2.464911in}{1.435179in}}%
\pgfpathlineto{\pgfqpoint{2.467026in}{1.434995in}}%
\pgfpathlineto{\pgfqpoint{2.469140in}{1.441627in}}%
\pgfpathlineto{\pgfqpoint{2.471255in}{1.437595in}}%
\pgfpathlineto{\pgfqpoint{2.473370in}{1.441255in}}%
\pgfpathlineto{\pgfqpoint{2.475484in}{1.448236in}}%
\pgfpathlineto{\pgfqpoint{2.477599in}{1.447153in}}%
\pgfpathlineto{\pgfqpoint{2.479714in}{1.443731in}}%
\pgfpathlineto{\pgfqpoint{2.481828in}{1.444081in}}%
\pgfpathlineto{\pgfqpoint{2.486058in}{1.448754in}}%
\pgfpathlineto{\pgfqpoint{2.492402in}{1.465701in}}%
\pgfpathlineto{\pgfqpoint{2.494517in}{1.468408in}}%
\pgfpathlineto{\pgfqpoint{2.498746in}{1.465017in}}%
\pgfpathlineto{\pgfqpoint{2.500861in}{1.468937in}}%
\pgfpathlineto{\pgfqpoint{2.502975in}{1.470424in}}%
\pgfpathlineto{\pgfqpoint{2.505090in}{1.465901in}}%
\pgfpathlineto{\pgfqpoint{2.507205in}{1.468141in}}%
\pgfpathlineto{\pgfqpoint{2.513549in}{1.462720in}}%
\pgfpathlineto{\pgfqpoint{2.515664in}{1.465629in}}%
\pgfpathlineto{\pgfqpoint{2.517778in}{1.464332in}}%
\pgfpathlineto{\pgfqpoint{2.524122in}{1.450432in}}%
\pgfpathlineto{\pgfqpoint{2.526237in}{1.449389in}}%
\pgfpathlineto{\pgfqpoint{2.528352in}{1.446874in}}%
\pgfpathlineto{\pgfqpoint{2.532581in}{1.445147in}}%
\pgfpathlineto{\pgfqpoint{2.534696in}{1.447204in}}%
\pgfpathlineto{\pgfqpoint{2.538925in}{1.463506in}}%
\pgfpathlineto{\pgfqpoint{2.541040in}{1.465753in}}%
\pgfpathlineto{\pgfqpoint{2.543155in}{1.464619in}}%
\pgfpathlineto{\pgfqpoint{2.547384in}{1.467211in}}%
\pgfpathlineto{\pgfqpoint{2.549499in}{1.458983in}}%
\pgfpathlineto{\pgfqpoint{2.553728in}{1.457987in}}%
\pgfpathlineto{\pgfqpoint{2.555843in}{1.446650in}}%
\pgfpathlineto{\pgfqpoint{2.557957in}{1.443585in}}%
\pgfpathlineto{\pgfqpoint{2.560072in}{1.445016in}}%
\pgfpathlineto{\pgfqpoint{2.562187in}{1.454252in}}%
\pgfpathlineto{\pgfqpoint{2.564301in}{1.452540in}}%
\pgfpathlineto{\pgfqpoint{2.566416in}{1.458450in}}%
\pgfpathlineto{\pgfqpoint{2.570646in}{1.457818in}}%
\pgfpathlineto{\pgfqpoint{2.572760in}{1.466264in}}%
\pgfpathlineto{\pgfqpoint{2.574875in}{1.466474in}}%
\pgfpathlineto{\pgfqpoint{2.576990in}{1.467931in}}%
\pgfpathlineto{\pgfqpoint{2.579104in}{1.467626in}}%
\pgfpathlineto{\pgfqpoint{2.581219in}{1.468962in}}%
\pgfpathlineto{\pgfqpoint{2.583334in}{1.476218in}}%
\pgfpathlineto{\pgfqpoint{2.585448in}{1.473915in}}%
\pgfpathlineto{\pgfqpoint{2.587563in}{1.475628in}}%
\pgfpathlineto{\pgfqpoint{2.589678in}{1.472835in}}%
\pgfpathlineto{\pgfqpoint{2.591792in}{1.465430in}}%
\pgfpathlineto{\pgfqpoint{2.598137in}{1.478695in}}%
\pgfpathlineto{\pgfqpoint{2.600251in}{1.475986in}}%
\pgfpathlineto{\pgfqpoint{2.602366in}{1.470849in}}%
\pgfpathlineto{\pgfqpoint{2.604481in}{1.462742in}}%
\pgfpathlineto{\pgfqpoint{2.608710in}{1.464133in}}%
\pgfpathlineto{\pgfqpoint{2.615054in}{1.476114in}}%
\pgfpathlineto{\pgfqpoint{2.617169in}{1.473750in}}%
\pgfpathlineto{\pgfqpoint{2.621398in}{1.474265in}}%
\pgfpathlineto{\pgfqpoint{2.623513in}{1.482187in}}%
\pgfpathlineto{\pgfqpoint{2.625628in}{1.484142in}}%
\pgfpathlineto{\pgfqpoint{2.627742in}{1.483942in}}%
\pgfpathlineto{\pgfqpoint{2.629857in}{1.485732in}}%
\pgfpathlineto{\pgfqpoint{2.631972in}{1.486037in}}%
\pgfpathlineto{\pgfqpoint{2.634086in}{1.489742in}}%
\pgfpathlineto{\pgfqpoint{2.636201in}{1.489622in}}%
\pgfpathlineto{\pgfqpoint{2.638316in}{1.486076in}}%
\pgfpathlineto{\pgfqpoint{2.640430in}{1.487838in}}%
\pgfpathlineto{\pgfqpoint{2.642545in}{1.480255in}}%
\pgfpathlineto{\pgfqpoint{2.644660in}{1.478392in}}%
\pgfpathlineto{\pgfqpoint{2.646775in}{1.479613in}}%
\pgfpathlineto{\pgfqpoint{2.648889in}{1.476883in}}%
\pgfpathlineto{\pgfqpoint{2.651004in}{1.471545in}}%
\pgfpathlineto{\pgfqpoint{2.653119in}{1.472447in}}%
\pgfpathlineto{\pgfqpoint{2.657348in}{1.464140in}}%
\pgfpathlineto{\pgfqpoint{2.659463in}{1.464524in}}%
\pgfpathlineto{\pgfqpoint{2.661577in}{1.458077in}}%
\pgfpathlineto{\pgfqpoint{2.665807in}{1.461535in}}%
\pgfpathlineto{\pgfqpoint{2.667921in}{1.455657in}}%
\pgfpathlineto{\pgfqpoint{2.674266in}{1.470853in}}%
\pgfpathlineto{\pgfqpoint{2.676380in}{1.470212in}}%
\pgfpathlineto{\pgfqpoint{2.678495in}{1.473551in}}%
\pgfpathlineto{\pgfqpoint{2.680610in}{1.467753in}}%
\pgfpathlineto{\pgfqpoint{2.684839in}{1.477212in}}%
\pgfpathlineto{\pgfqpoint{2.689068in}{1.479852in}}%
\pgfpathlineto{\pgfqpoint{2.691183in}{1.479933in}}%
\pgfpathlineto{\pgfqpoint{2.693298in}{1.482931in}}%
\pgfpathlineto{\pgfqpoint{2.695412in}{1.481783in}}%
\pgfpathlineto{\pgfqpoint{2.699642in}{1.492626in}}%
\pgfpathlineto{\pgfqpoint{2.703871in}{1.481420in}}%
\pgfpathlineto{\pgfqpoint{2.710215in}{1.488633in}}%
\pgfpathlineto{\pgfqpoint{2.712330in}{1.484911in}}%
\pgfpathlineto{\pgfqpoint{2.714445in}{1.486135in}}%
\pgfpathlineto{\pgfqpoint{2.716559in}{1.489674in}}%
\pgfpathlineto{\pgfqpoint{2.718674in}{1.485685in}}%
\pgfpathlineto{\pgfqpoint{2.720789in}{1.487606in}}%
\pgfpathlineto{\pgfqpoint{2.725018in}{1.486776in}}%
\pgfpathlineto{\pgfqpoint{2.727133in}{1.487538in}}%
\pgfpathlineto{\pgfqpoint{2.729248in}{1.486418in}}%
\pgfpathlineto{\pgfqpoint{2.731362in}{1.486935in}}%
\pgfpathlineto{\pgfqpoint{2.735592in}{1.493368in}}%
\pgfpathlineto{\pgfqpoint{2.737706in}{1.489710in}}%
\pgfpathlineto{\pgfqpoint{2.739821in}{1.493447in}}%
\pgfpathlineto{\pgfqpoint{2.744050in}{1.491717in}}%
\pgfpathlineto{\pgfqpoint{2.748280in}{1.492319in}}%
\pgfpathlineto{\pgfqpoint{2.750395in}{1.482683in}}%
\pgfpathlineto{\pgfqpoint{2.752509in}{1.487097in}}%
\pgfpathlineto{\pgfqpoint{2.754624in}{1.486883in}}%
\pgfpathlineto{\pgfqpoint{2.758853in}{1.493527in}}%
\pgfpathlineto{\pgfqpoint{2.760968in}{1.490952in}}%
\pgfpathlineto{\pgfqpoint{2.763083in}{1.493998in}}%
\pgfpathlineto{\pgfqpoint{2.765197in}{1.499498in}}%
\pgfpathlineto{\pgfqpoint{2.767312in}{1.498880in}}%
\pgfpathlineto{\pgfqpoint{2.769427in}{1.499642in}}%
\pgfpathlineto{\pgfqpoint{2.771541in}{1.502740in}}%
\pgfpathlineto{\pgfqpoint{2.775771in}{1.501091in}}%
\pgfpathlineto{\pgfqpoint{2.777886in}{1.510544in}}%
\pgfpathlineto{\pgfqpoint{2.780000in}{1.508784in}}%
\pgfpathlineto{\pgfqpoint{2.782115in}{1.512490in}}%
\pgfpathlineto{\pgfqpoint{2.784230in}{1.508930in}}%
\pgfpathlineto{\pgfqpoint{2.786344in}{1.509646in}}%
\pgfpathlineto{\pgfqpoint{2.790574in}{1.524060in}}%
\pgfpathlineto{\pgfqpoint{2.794803in}{1.517649in}}%
\pgfpathlineto{\pgfqpoint{2.796918in}{1.516064in}}%
\pgfpathlineto{\pgfqpoint{2.799032in}{1.511743in}}%
\pgfpathlineto{\pgfqpoint{2.801147in}{1.510160in}}%
\pgfpathlineto{\pgfqpoint{2.803262in}{1.514709in}}%
\pgfpathlineto{\pgfqpoint{2.805377in}{1.515509in}}%
\pgfpathlineto{\pgfqpoint{2.807491in}{1.514451in}}%
\pgfpathlineto{\pgfqpoint{2.809606in}{1.511227in}}%
\pgfpathlineto{\pgfqpoint{2.813835in}{1.522945in}}%
\pgfpathlineto{\pgfqpoint{2.818065in}{1.524056in}}%
\pgfpathlineto{\pgfqpoint{2.822294in}{1.525765in}}%
\pgfpathlineto{\pgfqpoint{2.824409in}{1.530279in}}%
\pgfpathlineto{\pgfqpoint{2.826523in}{1.526202in}}%
\pgfpathlineto{\pgfqpoint{2.828638in}{1.525703in}}%
\pgfpathlineto{\pgfqpoint{2.830753in}{1.527265in}}%
\pgfpathlineto{\pgfqpoint{2.832868in}{1.533729in}}%
\pgfpathlineto{\pgfqpoint{2.837097in}{1.536777in}}%
\pgfpathlineto{\pgfqpoint{2.841326in}{1.528202in}}%
\pgfpathlineto{\pgfqpoint{2.843441in}{1.529991in}}%
\pgfpathlineto{\pgfqpoint{2.845556in}{1.530048in}}%
\pgfpathlineto{\pgfqpoint{2.849785in}{1.539156in}}%
\pgfpathlineto{\pgfqpoint{2.856129in}{1.524634in}}%
\pgfpathlineto{\pgfqpoint{2.858244in}{1.522620in}}%
\pgfpathlineto{\pgfqpoint{2.860359in}{1.517718in}}%
\pgfpathlineto{\pgfqpoint{2.862473in}{1.520707in}}%
\pgfpathlineto{\pgfqpoint{2.864588in}{1.519068in}}%
\pgfpathlineto{\pgfqpoint{2.866703in}{1.515179in}}%
\pgfpathlineto{\pgfqpoint{2.868817in}{1.517940in}}%
\pgfpathlineto{\pgfqpoint{2.870932in}{1.513965in}}%
\pgfpathlineto{\pgfqpoint{2.873047in}{1.514335in}}%
\pgfpathlineto{\pgfqpoint{2.875161in}{1.512404in}}%
\pgfpathlineto{\pgfqpoint{2.877276in}{1.516699in}}%
\pgfpathlineto{\pgfqpoint{2.881506in}{1.510876in}}%
\pgfpathlineto{\pgfqpoint{2.883620in}{1.512581in}}%
\pgfpathlineto{\pgfqpoint{2.885735in}{1.512490in}}%
\pgfpathlineto{\pgfqpoint{2.887850in}{1.515076in}}%
\pgfpathlineto{\pgfqpoint{2.889964in}{1.521949in}}%
\pgfpathlineto{\pgfqpoint{2.892079in}{1.520831in}}%
\pgfpathlineto{\pgfqpoint{2.894194in}{1.521817in}}%
\pgfpathlineto{\pgfqpoint{2.896308in}{1.520405in}}%
\pgfpathlineto{\pgfqpoint{2.898423in}{1.524932in}}%
\pgfpathlineto{\pgfqpoint{2.900538in}{1.521141in}}%
\pgfpathlineto{\pgfqpoint{2.904767in}{1.522708in}}%
\pgfpathlineto{\pgfqpoint{2.908997in}{1.512364in}}%
\pgfpathlineto{\pgfqpoint{2.911111in}{1.520583in}}%
\pgfpathlineto{\pgfqpoint{2.913226in}{1.519584in}}%
\pgfpathlineto{\pgfqpoint{2.915341in}{1.515549in}}%
\pgfpathlineto{\pgfqpoint{2.917455in}{1.517782in}}%
\pgfpathlineto{\pgfqpoint{2.919570in}{1.516178in}}%
\pgfpathlineto{\pgfqpoint{2.921685in}{1.520006in}}%
\pgfpathlineto{\pgfqpoint{2.923799in}{1.517368in}}%
\pgfpathlineto{\pgfqpoint{2.928029in}{1.519854in}}%
\pgfpathlineto{\pgfqpoint{2.930143in}{1.522849in}}%
\pgfpathlineto{\pgfqpoint{2.934373in}{1.523386in}}%
\pgfpathlineto{\pgfqpoint{2.936488in}{1.525590in}}%
\pgfpathlineto{\pgfqpoint{2.940717in}{1.513411in}}%
\pgfpathlineto{\pgfqpoint{2.942832in}{1.512007in}}%
\pgfpathlineto{\pgfqpoint{2.944946in}{1.512524in}}%
\pgfpathlineto{\pgfqpoint{2.947061in}{1.515793in}}%
\pgfpathlineto{\pgfqpoint{2.949176in}{1.523427in}}%
\pgfpathlineto{\pgfqpoint{2.951290in}{1.523783in}}%
\pgfpathlineto{\pgfqpoint{2.953405in}{1.522810in}}%
\pgfpathlineto{\pgfqpoint{2.957634in}{1.515293in}}%
\pgfpathlineto{\pgfqpoint{2.961864in}{1.512373in}}%
\pgfpathlineto{\pgfqpoint{2.966093in}{1.498317in}}%
\pgfpathlineto{\pgfqpoint{2.968208in}{1.504372in}}%
\pgfpathlineto{\pgfqpoint{2.972437in}{1.497583in}}%
\pgfpathlineto{\pgfqpoint{2.974552in}{1.502837in}}%
\pgfpathlineto{\pgfqpoint{2.976667in}{1.500274in}}%
\pgfpathlineto{\pgfqpoint{2.978781in}{1.495937in}}%
\pgfpathlineto{\pgfqpoint{2.980896in}{1.503605in}}%
\pgfpathlineto{\pgfqpoint{2.983011in}{1.505432in}}%
\pgfpathlineto{\pgfqpoint{2.985126in}{1.499501in}}%
\pgfpathlineto{\pgfqpoint{2.991470in}{1.511516in}}%
\pgfpathlineto{\pgfqpoint{2.995699in}{1.509899in}}%
\pgfpathlineto{\pgfqpoint{2.997814in}{1.513476in}}%
\pgfpathlineto{\pgfqpoint{2.999928in}{1.509664in}}%
\pgfpathlineto{\pgfqpoint{3.002043in}{1.509333in}}%
\pgfpathlineto{\pgfqpoint{3.004158in}{1.512468in}}%
\pgfpathlineto{\pgfqpoint{3.006272in}{1.518089in}}%
\pgfpathlineto{\pgfqpoint{3.008387in}{1.514337in}}%
\pgfpathlineto{\pgfqpoint{3.010502in}{1.515419in}}%
\pgfpathlineto{\pgfqpoint{3.012617in}{1.508604in}}%
\pgfpathlineto{\pgfqpoint{3.016846in}{1.511662in}}%
\pgfpathlineto{\pgfqpoint{3.021075in}{1.510188in}}%
\pgfpathlineto{\pgfqpoint{3.023190in}{1.514546in}}%
\pgfpathlineto{\pgfqpoint{3.025305in}{1.512160in}}%
\pgfpathlineto{\pgfqpoint{3.027419in}{1.512061in}}%
\pgfpathlineto{\pgfqpoint{3.029534in}{1.513442in}}%
\pgfpathlineto{\pgfqpoint{3.031649in}{1.512175in}}%
\pgfpathlineto{\pgfqpoint{3.033763in}{1.506542in}}%
\pgfpathlineto{\pgfqpoint{3.035878in}{1.511954in}}%
\pgfpathlineto{\pgfqpoint{3.037993in}{1.510030in}}%
\pgfpathlineto{\pgfqpoint{3.040108in}{1.515907in}}%
\pgfpathlineto{\pgfqpoint{3.042222in}{1.517334in}}%
\pgfpathlineto{\pgfqpoint{3.044337in}{1.520339in}}%
\pgfpathlineto{\pgfqpoint{3.048566in}{1.516800in}}%
\pgfpathlineto{\pgfqpoint{3.052796in}{1.519043in}}%
\pgfpathlineto{\pgfqpoint{3.054910in}{1.523780in}}%
\pgfpathlineto{\pgfqpoint{3.059140in}{1.525715in}}%
\pgfpathlineto{\pgfqpoint{3.061254in}{1.523853in}}%
\pgfpathlineto{\pgfqpoint{3.067599in}{1.529282in}}%
\pgfpathlineto{\pgfqpoint{3.069713in}{1.543283in}}%
\pgfpathlineto{\pgfqpoint{3.071828in}{1.538319in}}%
\pgfpathlineto{\pgfqpoint{3.073943in}{1.540778in}}%
\pgfpathlineto{\pgfqpoint{3.076057in}{1.539315in}}%
\pgfpathlineto{\pgfqpoint{3.078172in}{1.536081in}}%
\pgfpathlineto{\pgfqpoint{3.082401in}{1.528283in}}%
\pgfpathlineto{\pgfqpoint{3.084516in}{1.532382in}}%
\pgfpathlineto{\pgfqpoint{3.086631in}{1.531395in}}%
\pgfpathlineto{\pgfqpoint{3.088746in}{1.532394in}}%
\pgfpathlineto{\pgfqpoint{3.090860in}{1.530420in}}%
\pgfpathlineto{\pgfqpoint{3.092975in}{1.531109in}}%
\pgfpathlineto{\pgfqpoint{3.097204in}{1.524700in}}%
\pgfpathlineto{\pgfqpoint{3.099319in}{1.521348in}}%
\pgfpathlineto{\pgfqpoint{3.101434in}{1.532452in}}%
\pgfpathlineto{\pgfqpoint{3.103548in}{1.532083in}}%
\pgfpathlineto{\pgfqpoint{3.105663in}{1.541428in}}%
\pgfpathlineto{\pgfqpoint{3.109892in}{1.537806in}}%
\pgfpathlineto{\pgfqpoint{3.112007in}{1.532750in}}%
\pgfpathlineto{\pgfqpoint{3.114122in}{1.535442in}}%
\pgfpathlineto{\pgfqpoint{3.116237in}{1.528082in}}%
\pgfpathlineto{\pgfqpoint{3.118351in}{1.527077in}}%
\pgfpathlineto{\pgfqpoint{3.120466in}{1.524505in}}%
\pgfpathlineto{\pgfqpoint{3.122581in}{1.519264in}}%
\pgfpathlineto{\pgfqpoint{3.124695in}{1.523148in}}%
\pgfpathlineto{\pgfqpoint{3.126810in}{1.523131in}}%
\pgfpathlineto{\pgfqpoint{3.128925in}{1.519592in}}%
\pgfpathlineto{\pgfqpoint{3.131039in}{1.526110in}}%
\pgfpathlineto{\pgfqpoint{3.133154in}{1.526995in}}%
\pgfpathlineto{\pgfqpoint{3.137383in}{1.522361in}}%
\pgfpathlineto{\pgfqpoint{3.141613in}{1.527638in}}%
\pgfpathlineto{\pgfqpoint{3.145842in}{1.536205in}}%
\pgfpathlineto{\pgfqpoint{3.152186in}{1.532043in}}%
\pgfpathlineto{\pgfqpoint{3.154301in}{1.532307in}}%
\pgfpathlineto{\pgfqpoint{3.156416in}{1.524592in}}%
\pgfpathlineto{\pgfqpoint{3.158530in}{1.530112in}}%
\pgfpathlineto{\pgfqpoint{3.160645in}{1.532435in}}%
\pgfpathlineto{\pgfqpoint{3.162760in}{1.531189in}}%
\pgfpathlineto{\pgfqpoint{3.164874in}{1.536524in}}%
\pgfpathlineto{\pgfqpoint{3.169104in}{1.534278in}}%
\pgfpathlineto{\pgfqpoint{3.173333in}{1.549317in}}%
\pgfpathlineto{\pgfqpoint{3.177563in}{1.550341in}}%
\pgfpathlineto{\pgfqpoint{3.181792in}{1.542782in}}%
\pgfpathlineto{\pgfqpoint{3.183907in}{1.544212in}}%
\pgfpathlineto{\pgfqpoint{3.186021in}{1.537043in}}%
\pgfpathlineto{\pgfqpoint{3.188136in}{1.549179in}}%
\pgfpathlineto{\pgfqpoint{3.192366in}{1.548826in}}%
\pgfpathlineto{\pgfqpoint{3.194480in}{1.539350in}}%
\pgfpathlineto{\pgfqpoint{3.196595in}{1.539960in}}%
\pgfpathlineto{\pgfqpoint{3.198710in}{1.528519in}}%
\pgfpathlineto{\pgfqpoint{3.200824in}{1.529405in}}%
\pgfpathlineto{\pgfqpoint{3.209283in}{1.557184in}}%
\pgfpathlineto{\pgfqpoint{3.211398in}{1.556010in}}%
\pgfpathlineto{\pgfqpoint{3.213512in}{1.560953in}}%
\pgfpathlineto{\pgfqpoint{3.215627in}{1.559863in}}%
\pgfpathlineto{\pgfqpoint{3.217742in}{1.555968in}}%
\pgfpathlineto{\pgfqpoint{3.219857in}{1.556081in}}%
\pgfpathlineto{\pgfqpoint{3.221971in}{1.551596in}}%
\pgfpathlineto{\pgfqpoint{3.224086in}{1.551679in}}%
\pgfpathlineto{\pgfqpoint{3.228315in}{1.544301in}}%
\pgfpathlineto{\pgfqpoint{3.230430in}{1.545340in}}%
\pgfpathlineto{\pgfqpoint{3.232545in}{1.547677in}}%
\pgfpathlineto{\pgfqpoint{3.234659in}{1.547444in}}%
\pgfpathlineto{\pgfqpoint{3.238889in}{1.542240in}}%
\pgfpathlineto{\pgfqpoint{3.241003in}{1.542899in}}%
\pgfpathlineto{\pgfqpoint{3.243118in}{1.547645in}}%
\pgfpathlineto{\pgfqpoint{3.245233in}{1.546944in}}%
\pgfpathlineto{\pgfqpoint{3.247348in}{1.544714in}}%
\pgfpathlineto{\pgfqpoint{3.257921in}{1.557900in}}%
\pgfpathlineto{\pgfqpoint{3.260036in}{1.558760in}}%
\pgfpathlineto{\pgfqpoint{3.262150in}{1.563376in}}%
\pgfpathlineto{\pgfqpoint{3.264265in}{1.559871in}}%
\pgfpathlineto{\pgfqpoint{3.266380in}{1.565978in}}%
\pgfpathlineto{\pgfqpoint{3.272724in}{1.567242in}}%
\pgfpathlineto{\pgfqpoint{3.276953in}{1.573061in}}%
\pgfpathlineto{\pgfqpoint{3.279068in}{1.573821in}}%
\pgfpathlineto{\pgfqpoint{3.281183in}{1.571043in}}%
\pgfpathlineto{\pgfqpoint{3.283297in}{1.574717in}}%
\pgfpathlineto{\pgfqpoint{3.285412in}{1.575422in}}%
\pgfpathlineto{\pgfqpoint{3.287527in}{1.565765in}}%
\pgfpathlineto{\pgfqpoint{3.289641in}{1.566141in}}%
\pgfpathlineto{\pgfqpoint{3.291756in}{1.559128in}}%
\pgfpathlineto{\pgfqpoint{3.298100in}{1.572033in}}%
\pgfpathlineto{\pgfqpoint{3.300215in}{1.573877in}}%
\pgfpathlineto{\pgfqpoint{3.302330in}{1.567116in}}%
\pgfpathlineto{\pgfqpoint{3.306559in}{1.571702in}}%
\pgfpathlineto{\pgfqpoint{3.308674in}{1.568022in}}%
\pgfpathlineto{\pgfqpoint{3.312903in}{1.577052in}}%
\pgfpathlineto{\pgfqpoint{3.315018in}{1.574159in}}%
\pgfpathlineto{\pgfqpoint{3.319247in}{1.586850in}}%
\pgfpathlineto{\pgfqpoint{3.323477in}{1.586478in}}%
\pgfpathlineto{\pgfqpoint{3.325591in}{1.593691in}}%
\pgfpathlineto{\pgfqpoint{3.327706in}{1.592273in}}%
\pgfpathlineto{\pgfqpoint{3.329821in}{1.595755in}}%
\pgfpathlineto{\pgfqpoint{3.331935in}{1.594077in}}%
\pgfpathlineto{\pgfqpoint{3.334050in}{1.605711in}}%
\pgfpathlineto{\pgfqpoint{3.336165in}{1.605230in}}%
\pgfpathlineto{\pgfqpoint{3.338279in}{1.603459in}}%
\pgfpathlineto{\pgfqpoint{3.340394in}{1.598386in}}%
\pgfpathlineto{\pgfqpoint{3.344623in}{1.582612in}}%
\pgfpathlineto{\pgfqpoint{3.346738in}{1.587407in}}%
\pgfpathlineto{\pgfqpoint{3.348853in}{1.589123in}}%
\pgfpathlineto{\pgfqpoint{3.355197in}{1.611044in}}%
\pgfpathlineto{\pgfqpoint{3.357312in}{1.613411in}}%
\pgfpathlineto{\pgfqpoint{3.359426in}{1.611626in}}%
\pgfpathlineto{\pgfqpoint{3.361541in}{1.606075in}}%
\pgfpathlineto{\pgfqpoint{3.363656in}{1.609462in}}%
\pgfpathlineto{\pgfqpoint{3.365770in}{1.610241in}}%
\pgfpathlineto{\pgfqpoint{3.370000in}{1.599089in}}%
\pgfpathlineto{\pgfqpoint{3.372114in}{1.600438in}}%
\pgfpathlineto{\pgfqpoint{3.374229in}{1.595295in}}%
\pgfpathlineto{\pgfqpoint{3.376344in}{1.600469in}}%
\pgfpathlineto{\pgfqpoint{3.380573in}{1.591246in}}%
\pgfpathlineto{\pgfqpoint{3.382688in}{1.593745in}}%
\pgfpathlineto{\pgfqpoint{3.384803in}{1.591492in}}%
\pgfpathlineto{\pgfqpoint{3.386917in}{1.592242in}}%
\pgfpathlineto{\pgfqpoint{3.389032in}{1.590427in}}%
\pgfpathlineto{\pgfqpoint{3.393261in}{1.572453in}}%
\pgfpathlineto{\pgfqpoint{3.395376in}{1.572050in}}%
\pgfpathlineto{\pgfqpoint{3.397491in}{1.567744in}}%
\pgfpathlineto{\pgfqpoint{3.399605in}{1.560352in}}%
\pgfpathlineto{\pgfqpoint{3.401720in}{1.562421in}}%
\pgfpathlineto{\pgfqpoint{3.405950in}{1.570235in}}%
\pgfpathlineto{\pgfqpoint{3.408064in}{1.568800in}}%
\pgfpathlineto{\pgfqpoint{3.410179in}{1.570154in}}%
\pgfpathlineto{\pgfqpoint{3.412294in}{1.568577in}}%
\pgfpathlineto{\pgfqpoint{3.416523in}{1.558589in}}%
\pgfpathlineto{\pgfqpoint{3.418638in}{1.566305in}}%
\pgfpathlineto{\pgfqpoint{3.420752in}{1.559062in}}%
\pgfpathlineto{\pgfqpoint{3.422867in}{1.564796in}}%
\pgfpathlineto{\pgfqpoint{3.427097in}{1.569864in}}%
\pgfpathlineto{\pgfqpoint{3.431326in}{1.564995in}}%
\pgfpathlineto{\pgfqpoint{3.433441in}{1.564394in}}%
\pgfpathlineto{\pgfqpoint{3.437670in}{1.568884in}}%
\pgfpathlineto{\pgfqpoint{3.439785in}{1.569170in}}%
\pgfpathlineto{\pgfqpoint{3.444014in}{1.563244in}}%
\pgfpathlineto{\pgfqpoint{3.446129in}{1.557406in}}%
\pgfpathlineto{\pgfqpoint{3.448243in}{1.557638in}}%
\pgfpathlineto{\pgfqpoint{3.450358in}{1.553557in}}%
\pgfpathlineto{\pgfqpoint{3.452473in}{1.554904in}}%
\pgfpathlineto{\pgfqpoint{3.454588in}{1.554182in}}%
\pgfpathlineto{\pgfqpoint{3.456702in}{1.562017in}}%
\pgfpathlineto{\pgfqpoint{3.458817in}{1.560652in}}%
\pgfpathlineto{\pgfqpoint{3.460932in}{1.556338in}}%
\pgfpathlineto{\pgfqpoint{3.463046in}{1.560942in}}%
\pgfpathlineto{\pgfqpoint{3.467276in}{1.551563in}}%
\pgfpathlineto{\pgfqpoint{3.471505in}{1.546607in}}%
\pgfpathlineto{\pgfqpoint{3.473620in}{1.549731in}}%
\pgfpathlineto{\pgfqpoint{3.475734in}{1.548747in}}%
\pgfpathlineto{\pgfqpoint{3.479964in}{1.556905in}}%
\pgfpathlineto{\pgfqpoint{3.482079in}{1.555572in}}%
\pgfpathlineto{\pgfqpoint{3.486308in}{1.547118in}}%
\pgfpathlineto{\pgfqpoint{3.488423in}{1.549590in}}%
\pgfpathlineto{\pgfqpoint{3.490537in}{1.554000in}}%
\pgfpathlineto{\pgfqpoint{3.492652in}{1.555828in}}%
\pgfpathlineto{\pgfqpoint{3.494767in}{1.554309in}}%
\pgfpathlineto{\pgfqpoint{3.496881in}{1.559779in}}%
\pgfpathlineto{\pgfqpoint{3.503225in}{1.565154in}}%
\pgfpathlineto{\pgfqpoint{3.505340in}{1.563500in}}%
\pgfpathlineto{\pgfqpoint{3.507455in}{1.570715in}}%
\pgfpathlineto{\pgfqpoint{3.509570in}{1.570818in}}%
\pgfpathlineto{\pgfqpoint{3.511684in}{1.567794in}}%
\pgfpathlineto{\pgfqpoint{3.515914in}{1.570910in}}%
\pgfpathlineto{\pgfqpoint{3.518028in}{1.568572in}}%
\pgfpathlineto{\pgfqpoint{3.520143in}{1.571073in}}%
\pgfpathlineto{\pgfqpoint{3.522258in}{1.569451in}}%
\pgfpathlineto{\pgfqpoint{3.528602in}{1.558702in}}%
\pgfpathlineto{\pgfqpoint{3.532831in}{1.559288in}}%
\pgfpathlineto{\pgfqpoint{3.537061in}{1.570974in}}%
\pgfpathlineto{\pgfqpoint{3.539175in}{1.566705in}}%
\pgfpathlineto{\pgfqpoint{3.543405in}{1.564847in}}%
\pgfpathlineto{\pgfqpoint{3.545519in}{1.566283in}}%
\pgfpathlineto{\pgfqpoint{3.549749in}{1.565287in}}%
\pgfpathlineto{\pgfqpoint{3.553978in}{1.576662in}}%
\pgfpathlineto{\pgfqpoint{3.560322in}{1.568777in}}%
\pgfpathlineto{\pgfqpoint{3.564552in}{1.566677in}}%
\pgfpathlineto{\pgfqpoint{3.568781in}{1.568193in}}%
\pgfpathlineto{\pgfqpoint{3.570896in}{1.573096in}}%
\pgfpathlineto{\pgfqpoint{3.573010in}{1.571068in}}%
\pgfpathlineto{\pgfqpoint{3.575125in}{1.577606in}}%
\pgfpathlineto{\pgfqpoint{3.577240in}{1.573686in}}%
\pgfpathlineto{\pgfqpoint{3.579354in}{1.575487in}}%
\pgfpathlineto{\pgfqpoint{3.581469in}{1.572779in}}%
\pgfpathlineto{\pgfqpoint{3.585699in}{1.565388in}}%
\pgfpathlineto{\pgfqpoint{3.587813in}{1.565942in}}%
\pgfpathlineto{\pgfqpoint{3.592043in}{1.556816in}}%
\pgfpathlineto{\pgfqpoint{3.594157in}{1.560005in}}%
\pgfpathlineto{\pgfqpoint{3.598387in}{1.561687in}}%
\pgfpathlineto{\pgfqpoint{3.600501in}{1.565434in}}%
\pgfpathlineto{\pgfqpoint{3.602616in}{1.563831in}}%
\pgfpathlineto{\pgfqpoint{3.606845in}{1.563580in}}%
\pgfpathlineto{\pgfqpoint{3.608960in}{1.559181in}}%
\pgfpathlineto{\pgfqpoint{3.611075in}{1.560170in}}%
\pgfpathlineto{\pgfqpoint{3.613190in}{1.568908in}}%
\pgfpathlineto{\pgfqpoint{3.615304in}{1.563641in}}%
\pgfpathlineto{\pgfqpoint{3.617419in}{1.571032in}}%
\pgfpathlineto{\pgfqpoint{3.619534in}{1.571618in}}%
\pgfpathlineto{\pgfqpoint{3.623763in}{1.576940in}}%
\pgfpathlineto{\pgfqpoint{3.625878in}{1.576224in}}%
\pgfpathlineto{\pgfqpoint{3.627992in}{1.571661in}}%
\pgfpathlineto{\pgfqpoint{3.632222in}{1.574135in}}%
\pgfpathlineto{\pgfqpoint{3.636451in}{1.570985in}}%
\pgfpathlineto{\pgfqpoint{3.638566in}{1.571188in}}%
\pgfpathlineto{\pgfqpoint{3.642795in}{1.566490in}}%
\pgfpathlineto{\pgfqpoint{3.647025in}{1.557261in}}%
\pgfpathlineto{\pgfqpoint{3.649139in}{1.561805in}}%
\pgfpathlineto{\pgfqpoint{3.651254in}{1.561117in}}%
\pgfpathlineto{\pgfqpoint{3.653369in}{1.558242in}}%
\pgfpathlineto{\pgfqpoint{3.655483in}{1.560970in}}%
\pgfpathlineto{\pgfqpoint{3.659713in}{1.561513in}}%
\pgfpathlineto{\pgfqpoint{3.661828in}{1.563216in}}%
\pgfpathlineto{\pgfqpoint{3.663942in}{1.568202in}}%
\pgfpathlineto{\pgfqpoint{3.666057in}{1.562056in}}%
\pgfpathlineto{\pgfqpoint{3.670286in}{1.564935in}}%
\pgfpathlineto{\pgfqpoint{3.672401in}{1.575211in}}%
\pgfpathlineto{\pgfqpoint{3.676630in}{1.577150in}}%
\pgfpathlineto{\pgfqpoint{3.678745in}{1.578605in}}%
\pgfpathlineto{\pgfqpoint{3.680860in}{1.581583in}}%
\pgfpathlineto{\pgfqpoint{3.682974in}{1.586612in}}%
\pgfpathlineto{\pgfqpoint{3.687204in}{1.587515in}}%
\pgfpathlineto{\pgfqpoint{3.689319in}{1.583056in}}%
\pgfpathlineto{\pgfqpoint{3.693548in}{1.584365in}}%
\pgfpathlineto{\pgfqpoint{3.695663in}{1.578477in}}%
\pgfpathlineto{\pgfqpoint{3.697777in}{1.580370in}}%
\pgfpathlineto{\pgfqpoint{3.699892in}{1.579919in}}%
\pgfpathlineto{\pgfqpoint{3.702007in}{1.585309in}}%
\pgfpathlineto{\pgfqpoint{3.704121in}{1.579058in}}%
\pgfpathlineto{\pgfqpoint{3.706236in}{1.581741in}}%
\pgfpathlineto{\pgfqpoint{3.708351in}{1.578667in}}%
\pgfpathlineto{\pgfqpoint{3.714695in}{1.583526in}}%
\pgfpathlineto{\pgfqpoint{3.718924in}{1.600849in}}%
\pgfpathlineto{\pgfqpoint{3.723154in}{1.602086in}}%
\pgfpathlineto{\pgfqpoint{3.725268in}{1.608587in}}%
\pgfpathlineto{\pgfqpoint{3.727383in}{1.605928in}}%
\pgfpathlineto{\pgfqpoint{3.731612in}{1.609842in}}%
\pgfpathlineto{\pgfqpoint{3.733727in}{1.604563in}}%
\pgfpathlineto{\pgfqpoint{3.735842in}{1.608475in}}%
\pgfpathlineto{\pgfqpoint{3.737956in}{1.607055in}}%
\pgfpathlineto{\pgfqpoint{3.740071in}{1.603719in}}%
\pgfpathlineto{\pgfqpoint{3.742186in}{1.605230in}}%
\pgfpathlineto{\pgfqpoint{3.744301in}{1.602135in}}%
\pgfpathlineto{\pgfqpoint{3.746415in}{1.596694in}}%
\pgfpathlineto{\pgfqpoint{3.750645in}{1.594058in}}%
\pgfpathlineto{\pgfqpoint{3.752759in}{1.598446in}}%
\pgfpathlineto{\pgfqpoint{3.754874in}{1.599534in}}%
\pgfpathlineto{\pgfqpoint{3.756989in}{1.602107in}}%
\pgfpathlineto{\pgfqpoint{3.759103in}{1.599423in}}%
\pgfpathlineto{\pgfqpoint{3.761218in}{1.608336in}}%
\pgfpathlineto{\pgfqpoint{3.763333in}{1.610589in}}%
\pgfpathlineto{\pgfqpoint{3.765447in}{1.605123in}}%
\pgfpathlineto{\pgfqpoint{3.767562in}{1.606975in}}%
\pgfpathlineto{\pgfqpoint{3.769677in}{1.606894in}}%
\pgfpathlineto{\pgfqpoint{3.771792in}{1.603243in}}%
\pgfpathlineto{\pgfqpoint{3.773906in}{1.604810in}}%
\pgfpathlineto{\pgfqpoint{3.778136in}{1.610846in}}%
\pgfpathlineto{\pgfqpoint{3.784480in}{1.617489in}}%
\pgfpathlineto{\pgfqpoint{3.786594in}{1.624481in}}%
\pgfpathlineto{\pgfqpoint{3.788709in}{1.625199in}}%
\pgfpathlineto{\pgfqpoint{3.790824in}{1.621495in}}%
\pgfpathlineto{\pgfqpoint{3.792939in}{1.630691in}}%
\pgfpathlineto{\pgfqpoint{3.795053in}{1.632658in}}%
\pgfpathlineto{\pgfqpoint{3.797168in}{1.630915in}}%
\pgfpathlineto{\pgfqpoint{3.799283in}{1.640441in}}%
\pgfpathlineto{\pgfqpoint{3.801397in}{1.642174in}}%
\pgfpathlineto{\pgfqpoint{3.807741in}{1.654317in}}%
\pgfpathlineto{\pgfqpoint{3.809856in}{1.652395in}}%
\pgfpathlineto{\pgfqpoint{3.811971in}{1.642085in}}%
\pgfpathlineto{\pgfqpoint{3.816200in}{1.648232in}}%
\pgfpathlineto{\pgfqpoint{3.822544in}{1.629746in}}%
\pgfpathlineto{\pgfqpoint{3.824659in}{1.628040in}}%
\pgfpathlineto{\pgfqpoint{3.828888in}{1.631559in}}%
\pgfpathlineto{\pgfqpoint{3.831003in}{1.636427in}}%
\pgfpathlineto{\pgfqpoint{3.833118in}{1.631455in}}%
\pgfpathlineto{\pgfqpoint{3.835232in}{1.629645in}}%
\pgfpathlineto{\pgfqpoint{3.837347in}{1.632878in}}%
\pgfpathlineto{\pgfqpoint{3.839462in}{1.633093in}}%
\pgfpathlineto{\pgfqpoint{3.841576in}{1.630569in}}%
\pgfpathlineto{\pgfqpoint{3.843691in}{1.638110in}}%
\pgfpathlineto{\pgfqpoint{3.845806in}{1.637332in}}%
\pgfpathlineto{\pgfqpoint{3.847921in}{1.638396in}}%
\pgfpathlineto{\pgfqpoint{3.850035in}{1.634179in}}%
\pgfpathlineto{\pgfqpoint{3.852150in}{1.641130in}}%
\pgfpathlineto{\pgfqpoint{3.856379in}{1.634003in}}%
\pgfpathlineto{\pgfqpoint{3.858494in}{1.640815in}}%
\pgfpathlineto{\pgfqpoint{3.860609in}{1.634378in}}%
\pgfpathlineto{\pgfqpoint{3.864838in}{1.636911in}}%
\pgfpathlineto{\pgfqpoint{3.866953in}{1.629363in}}%
\pgfpathlineto{\pgfqpoint{3.869067in}{1.626604in}}%
\pgfpathlineto{\pgfqpoint{3.871182in}{1.618768in}}%
\pgfpathlineto{\pgfqpoint{3.875412in}{1.623223in}}%
\pgfpathlineto{\pgfqpoint{3.879641in}{1.614277in}}%
\pgfpathlineto{\pgfqpoint{3.881756in}{1.615287in}}%
\pgfpathlineto{\pgfqpoint{3.883870in}{1.614138in}}%
\pgfpathlineto{\pgfqpoint{3.885985in}{1.611012in}}%
\pgfpathlineto{\pgfqpoint{3.888100in}{1.611292in}}%
\pgfpathlineto{\pgfqpoint{3.890214in}{1.609017in}}%
\pgfpathlineto{\pgfqpoint{3.894444in}{1.601726in}}%
\pgfpathlineto{\pgfqpoint{3.896559in}{1.598464in}}%
\pgfpathlineto{\pgfqpoint{3.900788in}{1.606421in}}%
\pgfpathlineto{\pgfqpoint{3.902903in}{1.607767in}}%
\pgfpathlineto{\pgfqpoint{3.905017in}{1.598720in}}%
\pgfpathlineto{\pgfqpoint{3.907132in}{1.598216in}}%
\pgfpathlineto{\pgfqpoint{3.909247in}{1.600954in}}%
\pgfpathlineto{\pgfqpoint{3.915591in}{1.602951in}}%
\pgfpathlineto{\pgfqpoint{3.917705in}{1.598595in}}%
\pgfpathlineto{\pgfqpoint{3.921935in}{1.596471in}}%
\pgfpathlineto{\pgfqpoint{3.924050in}{1.593928in}}%
\pgfpathlineto{\pgfqpoint{3.928279in}{1.598066in}}%
\pgfpathlineto{\pgfqpoint{3.930394in}{1.602922in}}%
\pgfpathlineto{\pgfqpoint{3.932508in}{1.599294in}}%
\pgfpathlineto{\pgfqpoint{3.934623in}{1.598421in}}%
\pgfpathlineto{\pgfqpoint{3.936738in}{1.600048in}}%
\pgfpathlineto{\pgfqpoint{3.943082in}{1.614276in}}%
\pgfpathlineto{\pgfqpoint{3.945196in}{1.616457in}}%
\pgfpathlineto{\pgfqpoint{3.947311in}{1.616052in}}%
\pgfpathlineto{\pgfqpoint{3.951541in}{1.620536in}}%
\pgfpathlineto{\pgfqpoint{3.953655in}{1.625190in}}%
\pgfpathlineto{\pgfqpoint{3.955770in}{1.626906in}}%
\pgfpathlineto{\pgfqpoint{3.959999in}{1.625078in}}%
\pgfpathlineto{\pgfqpoint{3.966343in}{1.614514in}}%
\pgfpathlineto{\pgfqpoint{3.968458in}{1.614083in}}%
\pgfpathlineto{\pgfqpoint{3.970573in}{1.610194in}}%
\pgfpathlineto{\pgfqpoint{3.974802in}{1.609337in}}%
\pgfpathlineto{\pgfqpoint{3.979032in}{1.596001in}}%
\pgfpathlineto{\pgfqpoint{3.981146in}{1.599474in}}%
\pgfpathlineto{\pgfqpoint{3.983261in}{1.598551in}}%
\pgfpathlineto{\pgfqpoint{3.985376in}{1.604216in}}%
\pgfpathlineto{\pgfqpoint{3.987490in}{1.601985in}}%
\pgfpathlineto{\pgfqpoint{3.989605in}{1.602040in}}%
\pgfpathlineto{\pgfqpoint{3.991720in}{1.606233in}}%
\pgfpathlineto{\pgfqpoint{3.993834in}{1.614439in}}%
\pgfpathlineto{\pgfqpoint{3.995949in}{1.612506in}}%
\pgfpathlineto{\pgfqpoint{4.002293in}{1.614059in}}%
\pgfpathlineto{\pgfqpoint{4.004408in}{1.610376in}}%
\pgfpathlineto{\pgfqpoint{4.006523in}{1.610644in}}%
\pgfpathlineto{\pgfqpoint{4.008637in}{1.605758in}}%
\pgfpathlineto{\pgfqpoint{4.010752in}{1.606512in}}%
\pgfpathlineto{\pgfqpoint{4.012867in}{1.609782in}}%
\pgfpathlineto{\pgfqpoint{4.014981in}{1.615525in}}%
\pgfpathlineto{\pgfqpoint{4.017096in}{1.611041in}}%
\pgfpathlineto{\pgfqpoint{4.021325in}{1.612937in}}%
\pgfpathlineto{\pgfqpoint{4.023440in}{1.604859in}}%
\pgfpathlineto{\pgfqpoint{4.027670in}{1.605578in}}%
\pgfpathlineto{\pgfqpoint{4.031899in}{1.599418in}}%
\pgfpathlineto{\pgfqpoint{4.034014in}{1.598829in}}%
\pgfpathlineto{\pgfqpoint{4.036128in}{1.603270in}}%
\pgfpathlineto{\pgfqpoint{4.040358in}{1.605345in}}%
\pgfpathlineto{\pgfqpoint{4.042472in}{1.613088in}}%
\pgfpathlineto{\pgfqpoint{4.044587in}{1.609597in}}%
\pgfpathlineto{\pgfqpoint{4.046702in}{1.620097in}}%
\pgfpathlineto{\pgfqpoint{4.048816in}{1.623672in}}%
\pgfpathlineto{\pgfqpoint{4.053046in}{1.622800in}}%
\pgfpathlineto{\pgfqpoint{4.055161in}{1.618887in}}%
\pgfpathlineto{\pgfqpoint{4.057275in}{1.630460in}}%
\pgfpathlineto{\pgfqpoint{4.061505in}{1.637857in}}%
\pgfpathlineto{\pgfqpoint{4.067849in}{1.646073in}}%
\pgfpathlineto{\pgfqpoint{4.069963in}{1.636654in}}%
\pgfpathlineto{\pgfqpoint{4.072078in}{1.637873in}}%
\pgfpathlineto{\pgfqpoint{4.074193in}{1.634387in}}%
\pgfpathlineto{\pgfqpoint{4.076307in}{1.635299in}}%
\pgfpathlineto{\pgfqpoint{4.082652in}{1.625983in}}%
\pgfpathlineto{\pgfqpoint{4.093225in}{1.618834in}}%
\pgfpathlineto{\pgfqpoint{4.095340in}{1.618919in}}%
\pgfpathlineto{\pgfqpoint{4.097454in}{1.612487in}}%
\pgfpathlineto{\pgfqpoint{4.108028in}{1.603418in}}%
\pgfpathlineto{\pgfqpoint{4.110143in}{1.604578in}}%
\pgfpathlineto{\pgfqpoint{4.112257in}{1.597432in}}%
\pgfpathlineto{\pgfqpoint{4.116487in}{1.601064in}}%
\pgfpathlineto{\pgfqpoint{4.118601in}{1.595554in}}%
\pgfpathlineto{\pgfqpoint{4.120716in}{1.598332in}}%
\pgfpathlineto{\pgfqpoint{4.124945in}{1.599796in}}%
\pgfpathlineto{\pgfqpoint{4.129175in}{1.585488in}}%
\pgfpathlineto{\pgfqpoint{4.133404in}{1.587612in}}%
\pgfpathlineto{\pgfqpoint{4.135519in}{1.593747in}}%
\pgfpathlineto{\pgfqpoint{4.137634in}{1.590985in}}%
\pgfpathlineto{\pgfqpoint{4.139748in}{1.592839in}}%
\pgfpathlineto{\pgfqpoint{4.141863in}{1.590853in}}%
\pgfpathlineto{\pgfqpoint{4.143978in}{1.590987in}}%
\pgfpathlineto{\pgfqpoint{4.150322in}{1.577537in}}%
\pgfpathlineto{\pgfqpoint{4.152436in}{1.578610in}}%
\pgfpathlineto{\pgfqpoint{4.156666in}{1.587493in}}%
\pgfpathlineto{\pgfqpoint{4.158781in}{1.583473in}}%
\pgfpathlineto{\pgfqpoint{4.167239in}{1.590848in}}%
\pgfpathlineto{\pgfqpoint{4.169354in}{1.590405in}}%
\pgfpathlineto{\pgfqpoint{4.173583in}{1.594695in}}%
\pgfpathlineto{\pgfqpoint{4.175698in}{1.594755in}}%
\pgfpathlineto{\pgfqpoint{4.177813in}{1.592138in}}%
\pgfpathlineto{\pgfqpoint{4.179927in}{1.586101in}}%
\pgfpathlineto{\pgfqpoint{4.182042in}{1.587084in}}%
\pgfpathlineto{\pgfqpoint{4.184157in}{1.584850in}}%
\pgfpathlineto{\pgfqpoint{4.186272in}{1.584756in}}%
\pgfpathlineto{\pgfqpoint{4.188386in}{1.587381in}}%
\pgfpathlineto{\pgfqpoint{4.192616in}{1.582239in}}%
\pgfpathlineto{\pgfqpoint{4.194730in}{1.581966in}}%
\pgfpathlineto{\pgfqpoint{4.196845in}{1.570508in}}%
\pgfpathlineto{\pgfqpoint{4.198960in}{1.565467in}}%
\pgfpathlineto{\pgfqpoint{4.201074in}{1.570355in}}%
\pgfpathlineto{\pgfqpoint{4.205304in}{1.567566in}}%
\pgfpathlineto{\pgfqpoint{4.209533in}{1.575274in}}%
\pgfpathlineto{\pgfqpoint{4.211648in}{1.574085in}}%
\pgfpathlineto{\pgfqpoint{4.215877in}{1.583653in}}%
\pgfpathlineto{\pgfqpoint{4.220107in}{1.574311in}}%
\pgfpathlineto{\pgfqpoint{4.222221in}{1.574740in}}%
\pgfpathlineto{\pgfqpoint{4.226451in}{1.592071in}}%
\pgfpathlineto{\pgfqpoint{4.230680in}{1.594396in}}%
\pgfpathlineto{\pgfqpoint{4.232795in}{1.592446in}}%
\pgfpathlineto{\pgfqpoint{4.234910in}{1.578300in}}%
\pgfpathlineto{\pgfqpoint{4.239139in}{1.576354in}}%
\pgfpathlineto{\pgfqpoint{4.241254in}{1.578788in}}%
\pgfpathlineto{\pgfqpoint{4.243368in}{1.578199in}}%
\pgfpathlineto{\pgfqpoint{4.245483in}{1.588249in}}%
\pgfpathlineto{\pgfqpoint{4.249712in}{1.585551in}}%
\pgfpathlineto{\pgfqpoint{4.256056in}{1.573597in}}%
\pgfpathlineto{\pgfqpoint{4.258171in}{1.572117in}}%
\pgfpathlineto{\pgfqpoint{4.260286in}{1.574785in}}%
\pgfpathlineto{\pgfqpoint{4.262401in}{1.572176in}}%
\pgfpathlineto{\pgfqpoint{4.264515in}{1.572259in}}%
\pgfpathlineto{\pgfqpoint{4.266630in}{1.570941in}}%
\pgfpathlineto{\pgfqpoint{4.268745in}{1.566981in}}%
\pgfpathlineto{\pgfqpoint{4.270859in}{1.567476in}}%
\pgfpathlineto{\pgfqpoint{4.272974in}{1.566080in}}%
\pgfpathlineto{\pgfqpoint{4.277203in}{1.560001in}}%
\pgfpathlineto{\pgfqpoint{4.279318in}{1.559962in}}%
\pgfpathlineto{\pgfqpoint{4.281433in}{1.556883in}}%
\pgfpathlineto{\pgfqpoint{4.283547in}{1.562320in}}%
\pgfpathlineto{\pgfqpoint{4.285662in}{1.556123in}}%
\pgfpathlineto{\pgfqpoint{4.287777in}{1.559395in}}%
\pgfpathlineto{\pgfqpoint{4.292006in}{1.547955in}}%
\pgfpathlineto{\pgfqpoint{4.296236in}{1.549860in}}%
\pgfpathlineto{\pgfqpoint{4.298350in}{1.548712in}}%
\pgfpathlineto{\pgfqpoint{4.300465in}{1.549518in}}%
\pgfpathlineto{\pgfqpoint{4.302580in}{1.544200in}}%
\pgfpathlineto{\pgfqpoint{4.304694in}{1.544034in}}%
\pgfpathlineto{\pgfqpoint{4.306809in}{1.547335in}}%
\pgfpathlineto{\pgfqpoint{4.308924in}{1.546737in}}%
\pgfpathlineto{\pgfqpoint{4.313153in}{1.551852in}}%
\pgfpathlineto{\pgfqpoint{4.321612in}{1.534089in}}%
\pgfpathlineto{\pgfqpoint{4.323727in}{1.543369in}}%
\pgfpathlineto{\pgfqpoint{4.325841in}{1.537424in}}%
\pgfpathlineto{\pgfqpoint{4.330071in}{1.540286in}}%
\pgfpathlineto{\pgfqpoint{4.332185in}{1.540067in}}%
\pgfpathlineto{\pgfqpoint{4.336415in}{1.534027in}}%
\pgfpathlineto{\pgfqpoint{4.338529in}{1.530444in}}%
\pgfpathlineto{\pgfqpoint{4.340644in}{1.533435in}}%
\pgfpathlineto{\pgfqpoint{4.342759in}{1.531330in}}%
\pgfpathlineto{\pgfqpoint{4.344874in}{1.533295in}}%
\pgfpathlineto{\pgfqpoint{4.346988in}{1.531915in}}%
\pgfpathlineto{\pgfqpoint{4.349103in}{1.532401in}}%
\pgfpathlineto{\pgfqpoint{4.353332in}{1.538736in}}%
\pgfpathlineto{\pgfqpoint{4.357562in}{1.546126in}}%
\pgfpathlineto{\pgfqpoint{4.359676in}{1.543082in}}%
\pgfpathlineto{\pgfqpoint{4.361791in}{1.548100in}}%
\pgfpathlineto{\pgfqpoint{4.370250in}{1.552974in}}%
\pgfpathlineto{\pgfqpoint{4.372365in}{1.548825in}}%
\pgfpathlineto{\pgfqpoint{4.374479in}{1.550298in}}%
\pgfpathlineto{\pgfqpoint{4.376594in}{1.549633in}}%
\pgfpathlineto{\pgfqpoint{4.378709in}{1.544571in}}%
\pgfpathlineto{\pgfqpoint{4.380823in}{1.536002in}}%
\pgfpathlineto{\pgfqpoint{4.382938in}{1.537458in}}%
\pgfpathlineto{\pgfqpoint{4.385053in}{1.537049in}}%
\pgfpathlineto{\pgfqpoint{4.387167in}{1.538509in}}%
\pgfpathlineto{\pgfqpoint{4.389282in}{1.542453in}}%
\pgfpathlineto{\pgfqpoint{4.391397in}{1.539440in}}%
\pgfpathlineto{\pgfqpoint{4.393512in}{1.538871in}}%
\pgfpathlineto{\pgfqpoint{4.395626in}{1.536467in}}%
\pgfpathlineto{\pgfqpoint{4.397741in}{1.538660in}}%
\pgfpathlineto{\pgfqpoint{4.399856in}{1.547369in}}%
\pgfpathlineto{\pgfqpoint{4.401970in}{1.549787in}}%
\pgfpathlineto{\pgfqpoint{4.404085in}{1.550227in}}%
\pgfpathlineto{\pgfqpoint{4.408314in}{1.542686in}}%
\pgfpathlineto{\pgfqpoint{4.410429in}{1.541546in}}%
\pgfpathlineto{\pgfqpoint{4.412544in}{1.544224in}}%
\pgfpathlineto{\pgfqpoint{4.414658in}{1.539114in}}%
\pgfpathlineto{\pgfqpoint{4.418888in}{1.542713in}}%
\pgfpathlineto{\pgfqpoint{4.423117in}{1.539093in}}%
\pgfpathlineto{\pgfqpoint{4.425232in}{1.541852in}}%
\pgfpathlineto{\pgfqpoint{4.427347in}{1.540593in}}%
\pgfpathlineto{\pgfqpoint{4.429461in}{1.541544in}}%
\pgfpathlineto{\pgfqpoint{4.431576in}{1.550458in}}%
\pgfpathlineto{\pgfqpoint{4.433691in}{1.550088in}}%
\pgfpathlineto{\pgfqpoint{4.435805in}{1.558518in}}%
\pgfpathlineto{\pgfqpoint{4.442149in}{1.549302in}}%
\pgfpathlineto{\pgfqpoint{4.444264in}{1.554575in}}%
\pgfpathlineto{\pgfqpoint{4.448494in}{1.556075in}}%
\pgfpathlineto{\pgfqpoint{4.452723in}{1.544386in}}%
\pgfpathlineto{\pgfqpoint{4.454838in}{1.548498in}}%
\pgfpathlineto{\pgfqpoint{4.456952in}{1.556793in}}%
\pgfpathlineto{\pgfqpoint{4.459067in}{1.556008in}}%
\pgfpathlineto{\pgfqpoint{4.461182in}{1.561508in}}%
\pgfpathlineto{\pgfqpoint{4.463296in}{1.560022in}}%
\pgfpathlineto{\pgfqpoint{4.465411in}{1.561866in}}%
\pgfpathlineto{\pgfqpoint{4.469641in}{1.561953in}}%
\pgfpathlineto{\pgfqpoint{4.473870in}{1.555671in}}%
\pgfpathlineto{\pgfqpoint{4.480214in}{1.569186in}}%
\pgfpathlineto{\pgfqpoint{4.482329in}{1.566984in}}%
\pgfpathlineto{\pgfqpoint{4.484443in}{1.575502in}}%
\pgfpathlineto{\pgfqpoint{4.486558in}{1.567698in}}%
\pgfpathlineto{\pgfqpoint{4.488673in}{1.575941in}}%
\pgfpathlineto{\pgfqpoint{4.490787in}{1.573394in}}%
\pgfpathlineto{\pgfqpoint{4.492902in}{1.573412in}}%
\pgfpathlineto{\pgfqpoint{4.495017in}{1.568583in}}%
\pgfpathlineto{\pgfqpoint{4.499246in}{1.573387in}}%
\pgfpathlineto{\pgfqpoint{4.501361in}{1.570752in}}%
\pgfpathlineto{\pgfqpoint{4.507705in}{1.582944in}}%
\pgfpathlineto{\pgfqpoint{4.509820in}{1.581749in}}%
\pgfpathlineto{\pgfqpoint{4.514049in}{1.588608in}}%
\pgfpathlineto{\pgfqpoint{4.516164in}{1.588037in}}%
\pgfpathlineto{\pgfqpoint{4.518278in}{1.589536in}}%
\pgfpathlineto{\pgfqpoint{4.520393in}{1.589077in}}%
\pgfpathlineto{\pgfqpoint{4.522508in}{1.592921in}}%
\pgfpathlineto{\pgfqpoint{4.524623in}{1.593184in}}%
\pgfpathlineto{\pgfqpoint{4.526737in}{1.583944in}}%
\pgfpathlineto{\pgfqpoint{4.528852in}{1.582995in}}%
\pgfpathlineto{\pgfqpoint{4.530967in}{1.578155in}}%
\pgfpathlineto{\pgfqpoint{4.533081in}{1.580306in}}%
\pgfpathlineto{\pgfqpoint{4.539425in}{1.594547in}}%
\pgfpathlineto{\pgfqpoint{4.541540in}{1.591765in}}%
\pgfpathlineto{\pgfqpoint{4.543655in}{1.585148in}}%
\pgfpathlineto{\pgfqpoint{4.545769in}{1.590611in}}%
\pgfpathlineto{\pgfqpoint{4.547884in}{1.589615in}}%
\pgfpathlineto{\pgfqpoint{4.549999in}{1.593245in}}%
\pgfpathlineto{\pgfqpoint{4.552114in}{1.590145in}}%
\pgfpathlineto{\pgfqpoint{4.554228in}{1.590005in}}%
\pgfpathlineto{\pgfqpoint{4.556343in}{1.585907in}}%
\pgfpathlineto{\pgfqpoint{4.560572in}{1.571553in}}%
\pgfpathlineto{\pgfqpoint{4.564802in}{1.577074in}}%
\pgfpathlineto{\pgfqpoint{4.569031in}{1.580111in}}%
\pgfpathlineto{\pgfqpoint{4.571146in}{1.577186in}}%
\pgfpathlineto{\pgfqpoint{4.573260in}{1.580183in}}%
\pgfpathlineto{\pgfqpoint{4.575375in}{1.578487in}}%
\pgfpathlineto{\pgfqpoint{4.577490in}{1.579122in}}%
\pgfpathlineto{\pgfqpoint{4.579605in}{1.575728in}}%
\pgfpathlineto{\pgfqpoint{4.581719in}{1.567909in}}%
\pgfpathlineto{\pgfqpoint{4.583834in}{1.568888in}}%
\pgfpathlineto{\pgfqpoint{4.585949in}{1.571195in}}%
\pgfpathlineto{\pgfqpoint{4.588063in}{1.570467in}}%
\pgfpathlineto{\pgfqpoint{4.590178in}{1.562203in}}%
\pgfpathlineto{\pgfqpoint{4.594407in}{1.567812in}}%
\pgfpathlineto{\pgfqpoint{4.596522in}{1.559716in}}%
\pgfpathlineto{\pgfqpoint{4.598637in}{1.557241in}}%
\pgfpathlineto{\pgfqpoint{4.600752in}{1.557621in}}%
\pgfpathlineto{\pgfqpoint{4.602866in}{1.560192in}}%
\pgfpathlineto{\pgfqpoint{4.604981in}{1.558697in}}%
\pgfpathlineto{\pgfqpoint{4.611325in}{1.564566in}}%
\pgfpathlineto{\pgfqpoint{4.613440in}{1.560281in}}%
\pgfpathlineto{\pgfqpoint{4.617669in}{1.568033in}}%
\pgfpathlineto{\pgfqpoint{4.619784in}{1.568273in}}%
\pgfpathlineto{\pgfqpoint{4.621898in}{1.556550in}}%
\pgfpathlineto{\pgfqpoint{4.624013in}{1.559773in}}%
\pgfpathlineto{\pgfqpoint{4.632472in}{1.581630in}}%
\pgfpathlineto{\pgfqpoint{4.634587in}{1.584914in}}%
\pgfpathlineto{\pgfqpoint{4.636701in}{1.579433in}}%
\pgfpathlineto{\pgfqpoint{4.638816in}{1.587702in}}%
\pgfpathlineto{\pgfqpoint{4.640931in}{1.582726in}}%
\pgfpathlineto{\pgfqpoint{4.643045in}{1.593296in}}%
\pgfpathlineto{\pgfqpoint{4.645160in}{1.590642in}}%
\pgfpathlineto{\pgfqpoint{4.647275in}{1.591811in}}%
\pgfpathlineto{\pgfqpoint{4.649389in}{1.590864in}}%
\pgfpathlineto{\pgfqpoint{4.651504in}{1.587511in}}%
\pgfpathlineto{\pgfqpoint{4.653619in}{1.592775in}}%
\pgfpathlineto{\pgfqpoint{4.657848in}{1.588306in}}%
\pgfpathlineto{\pgfqpoint{4.664192in}{1.594243in}}%
\pgfpathlineto{\pgfqpoint{4.666307in}{1.589135in}}%
\pgfpathlineto{\pgfqpoint{4.668422in}{1.587560in}}%
\pgfpathlineto{\pgfqpoint{4.670536in}{1.590637in}}%
\pgfpathlineto{\pgfqpoint{4.672651in}{1.590115in}}%
\pgfpathlineto{\pgfqpoint{4.674766in}{1.587192in}}%
\pgfpathlineto{\pgfqpoint{4.676880in}{1.598944in}}%
\pgfpathlineto{\pgfqpoint{4.678995in}{1.603331in}}%
\pgfpathlineto{\pgfqpoint{4.683225in}{1.603212in}}%
\pgfpathlineto{\pgfqpoint{4.685339in}{1.597026in}}%
\pgfpathlineto{\pgfqpoint{4.687454in}{1.594630in}}%
\pgfpathlineto{\pgfqpoint{4.689569in}{1.594673in}}%
\pgfpathlineto{\pgfqpoint{4.691683in}{1.592995in}}%
\pgfpathlineto{\pgfqpoint{4.693798in}{1.595384in}}%
\pgfpathlineto{\pgfqpoint{4.695913in}{1.591846in}}%
\pgfpathlineto{\pgfqpoint{4.698027in}{1.590893in}}%
\pgfpathlineto{\pgfqpoint{4.700142in}{1.591873in}}%
\pgfpathlineto{\pgfqpoint{4.702257in}{1.599873in}}%
\pgfpathlineto{\pgfqpoint{4.704372in}{1.600051in}}%
\pgfpathlineto{\pgfqpoint{4.710716in}{1.597149in}}%
\pgfpathlineto{\pgfqpoint{4.717060in}{1.602337in}}%
\pgfpathlineto{\pgfqpoint{4.719174in}{1.598585in}}%
\pgfpathlineto{\pgfqpoint{4.721289in}{1.603580in}}%
\pgfpathlineto{\pgfqpoint{4.723404in}{1.599235in}}%
\pgfpathlineto{\pgfqpoint{4.725518in}{1.600787in}}%
\pgfpathlineto{\pgfqpoint{4.727633in}{1.598745in}}%
\pgfpathlineto{\pgfqpoint{4.729748in}{1.594299in}}%
\pgfpathlineto{\pgfqpoint{4.731863in}{1.600102in}}%
\pgfpathlineto{\pgfqpoint{4.733977in}{1.597003in}}%
\pgfpathlineto{\pgfqpoint{4.736092in}{1.596123in}}%
\pgfpathlineto{\pgfqpoint{4.738207in}{1.590249in}}%
\pgfpathlineto{\pgfqpoint{4.740321in}{1.588810in}}%
\pgfpathlineto{\pgfqpoint{4.742436in}{1.592443in}}%
\pgfpathlineto{\pgfqpoint{4.744551in}{1.590547in}}%
\pgfpathlineto{\pgfqpoint{4.746665in}{1.584018in}}%
\pgfpathlineto{\pgfqpoint{4.748780in}{1.589107in}}%
\pgfpathlineto{\pgfqpoint{4.750895in}{1.586449in}}%
\pgfpathlineto{\pgfqpoint{4.755124in}{1.586538in}}%
\pgfpathlineto{\pgfqpoint{4.757239in}{1.592498in}}%
\pgfpathlineto{\pgfqpoint{4.759354in}{1.590456in}}%
\pgfpathlineto{\pgfqpoint{4.761468in}{1.592506in}}%
\pgfpathlineto{\pgfqpoint{4.765698in}{1.602753in}}%
\pgfpathlineto{\pgfqpoint{4.767812in}{1.599345in}}%
\pgfpathlineto{\pgfqpoint{4.769927in}{1.602771in}}%
\pgfpathlineto{\pgfqpoint{4.772042in}{1.603320in}}%
\pgfpathlineto{\pgfqpoint{4.774156in}{1.599599in}}%
\pgfpathlineto{\pgfqpoint{4.776271in}{1.599187in}}%
\pgfpathlineto{\pgfqpoint{4.780500in}{1.605205in}}%
\pgfpathlineto{\pgfqpoint{4.784730in}{1.592295in}}%
\pgfpathlineto{\pgfqpoint{4.791074in}{1.591349in}}%
\pgfpathlineto{\pgfqpoint{4.793189in}{1.592453in}}%
\pgfpathlineto{\pgfqpoint{4.795303in}{1.595869in}}%
\pgfpathlineto{\pgfqpoint{4.797418in}{1.587559in}}%
\pgfpathlineto{\pgfqpoint{4.799533in}{1.590117in}}%
\pgfpathlineto{\pgfqpoint{4.801647in}{1.584549in}}%
\pgfpathlineto{\pgfqpoint{4.803762in}{1.582501in}}%
\pgfpathlineto{\pgfqpoint{4.805877in}{1.587622in}}%
\pgfpathlineto{\pgfqpoint{4.807991in}{1.587773in}}%
\pgfpathlineto{\pgfqpoint{4.810106in}{1.592326in}}%
\pgfpathlineto{\pgfqpoint{4.812221in}{1.588089in}}%
\pgfpathlineto{\pgfqpoint{4.814336in}{1.591816in}}%
\pgfpathlineto{\pgfqpoint{4.816450in}{1.588463in}}%
\pgfpathlineto{\pgfqpoint{4.818565in}{1.591788in}}%
\pgfpathlineto{\pgfqpoint{4.820680in}{1.600401in}}%
\pgfpathlineto{\pgfqpoint{4.822794in}{1.604309in}}%
\pgfpathlineto{\pgfqpoint{4.829138in}{1.601235in}}%
\pgfpathlineto{\pgfqpoint{4.831253in}{1.598112in}}%
\pgfpathlineto{\pgfqpoint{4.833368in}{1.607525in}}%
\pgfpathlineto{\pgfqpoint{4.835483in}{1.609029in}}%
\pgfpathlineto{\pgfqpoint{4.841827in}{1.604116in}}%
\pgfpathlineto{\pgfqpoint{4.843941in}{1.608493in}}%
\pgfpathlineto{\pgfqpoint{4.846056in}{1.604375in}}%
\pgfpathlineto{\pgfqpoint{4.848171in}{1.612339in}}%
\pgfpathlineto{\pgfqpoint{4.854515in}{1.615130in}}%
\pgfpathlineto{\pgfqpoint{4.856629in}{1.615543in}}%
\pgfpathlineto{\pgfqpoint{4.858744in}{1.618288in}}%
\pgfpathlineto{\pgfqpoint{4.865088in}{1.612435in}}%
\pgfpathlineto{\pgfqpoint{4.867203in}{1.613051in}}%
\pgfpathlineto{\pgfqpoint{4.869318in}{1.611704in}}%
\pgfpathlineto{\pgfqpoint{4.871432in}{1.605534in}}%
\pgfpathlineto{\pgfqpoint{4.873547in}{1.608075in}}%
\pgfpathlineto{\pgfqpoint{4.875662in}{1.607146in}}%
\pgfpathlineto{\pgfqpoint{4.877776in}{1.604299in}}%
\pgfpathlineto{\pgfqpoint{4.879891in}{1.606631in}}%
\pgfpathlineto{\pgfqpoint{4.882006in}{1.611855in}}%
\pgfpathlineto{\pgfqpoint{4.888350in}{1.613632in}}%
\pgfpathlineto{\pgfqpoint{4.890465in}{1.618168in}}%
\pgfpathlineto{\pgfqpoint{4.892579in}{1.613196in}}%
\pgfpathlineto{\pgfqpoint{4.894694in}{1.612125in}}%
\pgfpathlineto{\pgfqpoint{4.901038in}{1.602251in}}%
\pgfpathlineto{\pgfqpoint{4.903153in}{1.608318in}}%
\pgfpathlineto{\pgfqpoint{4.907382in}{1.611774in}}%
\pgfpathlineto{\pgfqpoint{4.911611in}{1.599583in}}%
\pgfpathlineto{\pgfqpoint{4.913726in}{1.596994in}}%
\pgfpathlineto{\pgfqpoint{4.915841in}{1.598537in}}%
\pgfpathlineto{\pgfqpoint{4.917956in}{1.605675in}}%
\pgfpathlineto{\pgfqpoint{4.920070in}{1.602481in}}%
\pgfpathlineto{\pgfqpoint{4.922185in}{1.602523in}}%
\pgfpathlineto{\pgfqpoint{4.924300in}{1.605066in}}%
\pgfpathlineto{\pgfqpoint{4.926414in}{1.604040in}}%
\pgfpathlineto{\pgfqpoint{4.930644in}{1.614264in}}%
\pgfpathlineto{\pgfqpoint{4.934873in}{1.610876in}}%
\pgfpathlineto{\pgfqpoint{4.936988in}{1.621925in}}%
\pgfpathlineto{\pgfqpoint{4.939103in}{1.623474in}}%
\pgfpathlineto{\pgfqpoint{4.941217in}{1.630098in}}%
\pgfpathlineto{\pgfqpoint{4.943332in}{1.629015in}}%
\pgfpathlineto{\pgfqpoint{4.947561in}{1.624859in}}%
\pgfpathlineto{\pgfqpoint{4.949676in}{1.634966in}}%
\pgfpathlineto{\pgfqpoint{4.951791in}{1.631765in}}%
\pgfpathlineto{\pgfqpoint{4.953905in}{1.636343in}}%
\pgfpathlineto{\pgfqpoint{4.956020in}{1.636372in}}%
\pgfpathlineto{\pgfqpoint{4.958135in}{1.630577in}}%
\pgfpathlineto{\pgfqpoint{4.962364in}{1.632041in}}%
\pgfpathlineto{\pgfqpoint{4.964479in}{1.632230in}}%
\pgfpathlineto{\pgfqpoint{4.968708in}{1.641454in}}%
\pgfpathlineto{\pgfqpoint{4.970823in}{1.643466in}}%
\pgfpathlineto{\pgfqpoint{4.972938in}{1.647885in}}%
\pgfpathlineto{\pgfqpoint{4.977167in}{1.639855in}}%
\pgfpathlineto{\pgfqpoint{4.981396in}{1.632211in}}%
\pgfpathlineto{\pgfqpoint{4.985626in}{1.621906in}}%
\pgfpathlineto{\pgfqpoint{4.987740in}{1.618677in}}%
\pgfpathlineto{\pgfqpoint{4.989855in}{1.618006in}}%
\pgfpathlineto{\pgfqpoint{4.991970in}{1.625155in}}%
\pgfpathlineto{\pgfqpoint{4.994085in}{1.620629in}}%
\pgfpathlineto{\pgfqpoint{4.996199in}{1.625351in}}%
\pgfpathlineto{\pgfqpoint{4.998314in}{1.626688in}}%
\pgfpathlineto{\pgfqpoint{5.002543in}{1.639606in}}%
\pgfpathlineto{\pgfqpoint{5.004658in}{1.639265in}}%
\pgfpathlineto{\pgfqpoint{5.006773in}{1.634752in}}%
\pgfpathlineto{\pgfqpoint{5.008887in}{1.636251in}}%
\pgfpathlineto{\pgfqpoint{5.011002in}{1.641070in}}%
\pgfpathlineto{\pgfqpoint{5.013117in}{1.641612in}}%
\pgfpathlineto{\pgfqpoint{5.015231in}{1.643493in}}%
\pgfpathlineto{\pgfqpoint{5.017346in}{1.641883in}}%
\pgfpathlineto{\pgfqpoint{5.019461in}{1.650425in}}%
\pgfpathlineto{\pgfqpoint{5.021576in}{1.653734in}}%
\pgfpathlineto{\pgfqpoint{5.025805in}{1.639674in}}%
\pgfpathlineto{\pgfqpoint{5.027920in}{1.640053in}}%
\pgfpathlineto{\pgfqpoint{5.030034in}{1.635113in}}%
\pgfpathlineto{\pgfqpoint{5.032149in}{1.636200in}}%
\pgfpathlineto{\pgfqpoint{5.034264in}{1.635317in}}%
\pgfpathlineto{\pgfqpoint{5.036378in}{1.635924in}}%
\pgfpathlineto{\pgfqpoint{5.040608in}{1.642662in}}%
\pgfpathlineto{\pgfqpoint{5.044837in}{1.639391in}}%
\pgfpathlineto{\pgfqpoint{5.046952in}{1.640837in}}%
\pgfpathlineto{\pgfqpoint{5.051181in}{1.630969in}}%
\pgfpathlineto{\pgfqpoint{5.053296in}{1.631935in}}%
\pgfpathlineto{\pgfqpoint{5.055411in}{1.628738in}}%
\pgfpathlineto{\pgfqpoint{5.057525in}{1.631309in}}%
\pgfpathlineto{\pgfqpoint{5.059640in}{1.641269in}}%
\pgfpathlineto{\pgfqpoint{5.063869in}{1.644839in}}%
\pgfpathlineto{\pgfqpoint{5.068099in}{1.647177in}}%
\pgfpathlineto{\pgfqpoint{5.070214in}{1.651385in}}%
\pgfpathlineto{\pgfqpoint{5.072328in}{1.640747in}}%
\pgfpathlineto{\pgfqpoint{5.078672in}{1.648944in}}%
\pgfpathlineto{\pgfqpoint{5.080787in}{1.641955in}}%
\pgfpathlineto{\pgfqpoint{5.082902in}{1.643918in}}%
\pgfpathlineto{\pgfqpoint{5.087131in}{1.632888in}}%
\pgfpathlineto{\pgfqpoint{5.089246in}{1.631647in}}%
\pgfpathlineto{\pgfqpoint{5.091360in}{1.627384in}}%
\pgfpathlineto{\pgfqpoint{5.095590in}{1.636686in}}%
\pgfpathlineto{\pgfqpoint{5.097705in}{1.634744in}}%
\pgfpathlineto{\pgfqpoint{5.099819in}{1.639560in}}%
\pgfpathlineto{\pgfqpoint{5.101934in}{1.640565in}}%
\pgfpathlineto{\pgfqpoint{5.104049in}{1.642987in}}%
\pgfpathlineto{\pgfqpoint{5.106163in}{1.639266in}}%
\pgfpathlineto{\pgfqpoint{5.108278in}{1.646384in}}%
\pgfpathlineto{\pgfqpoint{5.110393in}{1.648921in}}%
\pgfpathlineto{\pgfqpoint{5.112507in}{1.647683in}}%
\pgfpathlineto{\pgfqpoint{5.114622in}{1.653674in}}%
\pgfpathlineto{\pgfqpoint{5.116737in}{1.655150in}}%
\pgfpathlineto{\pgfqpoint{5.123081in}{1.667592in}}%
\pgfpathlineto{\pgfqpoint{5.125196in}{1.667467in}}%
\pgfpathlineto{\pgfqpoint{5.129425in}{1.671802in}}%
\pgfpathlineto{\pgfqpoint{5.131540in}{1.677066in}}%
\pgfpathlineto{\pgfqpoint{5.133654in}{1.679197in}}%
\pgfpathlineto{\pgfqpoint{5.135769in}{1.670320in}}%
\pgfpathlineto{\pgfqpoint{5.137884in}{1.669374in}}%
\pgfpathlineto{\pgfqpoint{5.139998in}{1.666810in}}%
\pgfpathlineto{\pgfqpoint{5.144228in}{1.670202in}}%
\pgfpathlineto{\pgfqpoint{5.146342in}{1.668185in}}%
\pgfpathlineto{\pgfqpoint{5.152687in}{1.653453in}}%
\pgfpathlineto{\pgfqpoint{5.154801in}{1.653520in}}%
\pgfpathlineto{\pgfqpoint{5.156916in}{1.648575in}}%
\pgfpathlineto{\pgfqpoint{5.163260in}{1.660730in}}%
\pgfpathlineto{\pgfqpoint{5.165375in}{1.658624in}}%
\pgfpathlineto{\pgfqpoint{5.167489in}{1.658914in}}%
\pgfpathlineto{\pgfqpoint{5.169604in}{1.656828in}}%
\pgfpathlineto{\pgfqpoint{5.171719in}{1.658768in}}%
\pgfpathlineto{\pgfqpoint{5.173834in}{1.658785in}}%
\pgfpathlineto{\pgfqpoint{5.175948in}{1.657049in}}%
\pgfpathlineto{\pgfqpoint{5.178063in}{1.660939in}}%
\pgfpathlineto{\pgfqpoint{5.180178in}{1.656567in}}%
\pgfpathlineto{\pgfqpoint{5.182292in}{1.658945in}}%
\pgfpathlineto{\pgfqpoint{5.184407in}{1.666209in}}%
\pgfpathlineto{\pgfqpoint{5.188636in}{1.660296in}}%
\pgfpathlineto{\pgfqpoint{5.188636in}{1.660296in}}%
\pgfusepath{stroke}%
\end{pgfscope}%
\begin{pgfscope}%
\pgfpathrectangle{\pgfqpoint{0.750000in}{0.275000in}}{\pgfqpoint{4.650000in}{1.925000in}}%
\pgfusepath{clip}%
\pgfsetroundcap%
\pgfsetroundjoin%
\pgfsetlinewidth{1.003750pt}%
\definecolor{currentstroke}{rgb}{0.894118,0.101961,0.109804}%
\pgfsetstrokecolor{currentstroke}%
\pgfsetdash{}{0pt}%
\pgfpathmoveto{\pgfqpoint{0.961364in}{1.332873in}}%
\pgfpathlineto{\pgfqpoint{0.963478in}{1.336899in}}%
\pgfpathlineto{\pgfqpoint{0.965593in}{1.336748in}}%
\pgfpathlineto{\pgfqpoint{0.971937in}{1.352004in}}%
\pgfpathlineto{\pgfqpoint{0.974052in}{1.349076in}}%
\pgfpathlineto{\pgfqpoint{0.976166in}{1.342616in}}%
\pgfpathlineto{\pgfqpoint{0.980396in}{1.350681in}}%
\pgfpathlineto{\pgfqpoint{0.982511in}{1.349747in}}%
\pgfpathlineto{\pgfqpoint{0.984625in}{1.352453in}}%
\pgfpathlineto{\pgfqpoint{0.988855in}{1.363133in}}%
\pgfpathlineto{\pgfqpoint{0.990969in}{1.364578in}}%
\pgfpathlineto{\pgfqpoint{0.993084in}{1.367534in}}%
\pgfpathlineto{\pgfqpoint{0.995199in}{1.365368in}}%
\pgfpathlineto{\pgfqpoint{0.997313in}{1.366419in}}%
\pgfpathlineto{\pgfqpoint{0.999428in}{1.363033in}}%
\pgfpathlineto{\pgfqpoint{1.001543in}{1.363633in}}%
\pgfpathlineto{\pgfqpoint{1.003658in}{1.355278in}}%
\pgfpathlineto{\pgfqpoint{1.005772in}{1.354329in}}%
\pgfpathlineto{\pgfqpoint{1.007887in}{1.349284in}}%
\pgfpathlineto{\pgfqpoint{1.010002in}{1.350871in}}%
\pgfpathlineto{\pgfqpoint{1.012116in}{1.349979in}}%
\pgfpathlineto{\pgfqpoint{1.014231in}{1.346747in}}%
\pgfpathlineto{\pgfqpoint{1.016346in}{1.347274in}}%
\pgfpathlineto{\pgfqpoint{1.018460in}{1.343661in}}%
\pgfpathlineto{\pgfqpoint{1.020575in}{1.348748in}}%
\pgfpathlineto{\pgfqpoint{1.024804in}{1.340351in}}%
\pgfpathlineto{\pgfqpoint{1.026919in}{1.343694in}}%
\pgfpathlineto{\pgfqpoint{1.029034in}{1.356190in}}%
\pgfpathlineto{\pgfqpoint{1.031149in}{1.356232in}}%
\pgfpathlineto{\pgfqpoint{1.033263in}{1.353547in}}%
\pgfpathlineto{\pgfqpoint{1.035378in}{1.356236in}}%
\pgfpathlineto{\pgfqpoint{1.037493in}{1.354528in}}%
\pgfpathlineto{\pgfqpoint{1.041722in}{1.353775in}}%
\pgfpathlineto{\pgfqpoint{1.048066in}{1.337362in}}%
\pgfpathlineto{\pgfqpoint{1.050181in}{1.331497in}}%
\pgfpathlineto{\pgfqpoint{1.052295in}{1.329043in}}%
\pgfpathlineto{\pgfqpoint{1.060754in}{1.354450in}}%
\pgfpathlineto{\pgfqpoint{1.062869in}{1.352513in}}%
\pgfpathlineto{\pgfqpoint{1.064984in}{1.347578in}}%
\pgfpathlineto{\pgfqpoint{1.067098in}{1.347872in}}%
\pgfpathlineto{\pgfqpoint{1.071328in}{1.357705in}}%
\pgfpathlineto{\pgfqpoint{1.073442in}{1.357533in}}%
\pgfpathlineto{\pgfqpoint{1.075557in}{1.358619in}}%
\pgfpathlineto{\pgfqpoint{1.077672in}{1.354007in}}%
\pgfpathlineto{\pgfqpoint{1.081901in}{1.354420in}}%
\pgfpathlineto{\pgfqpoint{1.084016in}{1.362826in}}%
\pgfpathlineto{\pgfqpoint{1.088245in}{1.367968in}}%
\pgfpathlineto{\pgfqpoint{1.090360in}{1.364738in}}%
\pgfpathlineto{\pgfqpoint{1.092475in}{1.365469in}}%
\pgfpathlineto{\pgfqpoint{1.096704in}{1.363722in}}%
\pgfpathlineto{\pgfqpoint{1.098819in}{1.366235in}}%
\pgfpathlineto{\pgfqpoint{1.100933in}{1.366414in}}%
\pgfpathlineto{\pgfqpoint{1.103048in}{1.370553in}}%
\pgfpathlineto{\pgfqpoint{1.105163in}{1.368553in}}%
\pgfpathlineto{\pgfqpoint{1.107278in}{1.370597in}}%
\pgfpathlineto{\pgfqpoint{1.111507in}{1.371072in}}%
\pgfpathlineto{\pgfqpoint{1.113622in}{1.377506in}}%
\pgfpathlineto{\pgfqpoint{1.119966in}{1.386867in}}%
\pgfpathlineto{\pgfqpoint{1.122080in}{1.380931in}}%
\pgfpathlineto{\pgfqpoint{1.124195in}{1.382701in}}%
\pgfpathlineto{\pgfqpoint{1.126310in}{1.386090in}}%
\pgfpathlineto{\pgfqpoint{1.130539in}{1.378579in}}%
\pgfpathlineto{\pgfqpoint{1.134769in}{1.382138in}}%
\pgfpathlineto{\pgfqpoint{1.136883in}{1.380343in}}%
\pgfpathlineto{\pgfqpoint{1.138998in}{1.380558in}}%
\pgfpathlineto{\pgfqpoint{1.141113in}{1.379421in}}%
\pgfpathlineto{\pgfqpoint{1.145342in}{1.369493in}}%
\pgfpathlineto{\pgfqpoint{1.149571in}{1.375079in}}%
\pgfpathlineto{\pgfqpoint{1.151686in}{1.373546in}}%
\pgfpathlineto{\pgfqpoint{1.153801in}{1.373424in}}%
\pgfpathlineto{\pgfqpoint{1.160145in}{1.380353in}}%
\pgfpathlineto{\pgfqpoint{1.162260in}{1.376136in}}%
\pgfpathlineto{\pgfqpoint{1.164374in}{1.379630in}}%
\pgfpathlineto{\pgfqpoint{1.166489in}{1.377179in}}%
\pgfpathlineto{\pgfqpoint{1.168604in}{1.381435in}}%
\pgfpathlineto{\pgfqpoint{1.170718in}{1.373613in}}%
\pgfpathlineto{\pgfqpoint{1.177062in}{1.381856in}}%
\pgfpathlineto{\pgfqpoint{1.179177in}{1.384977in}}%
\pgfpathlineto{\pgfqpoint{1.181292in}{1.385467in}}%
\pgfpathlineto{\pgfqpoint{1.183406in}{1.393076in}}%
\pgfpathlineto{\pgfqpoint{1.185521in}{1.390014in}}%
\pgfpathlineto{\pgfqpoint{1.187636in}{1.383860in}}%
\pgfpathlineto{\pgfqpoint{1.189751in}{1.383021in}}%
\pgfpathlineto{\pgfqpoint{1.191865in}{1.388181in}}%
\pgfpathlineto{\pgfqpoint{1.193980in}{1.389021in}}%
\pgfpathlineto{\pgfqpoint{1.196095in}{1.386504in}}%
\pgfpathlineto{\pgfqpoint{1.198209in}{1.385821in}}%
\pgfpathlineto{\pgfqpoint{1.200324in}{1.389464in}}%
\pgfpathlineto{\pgfqpoint{1.202439in}{1.389471in}}%
\pgfpathlineto{\pgfqpoint{1.206668in}{1.398693in}}%
\pgfpathlineto{\pgfqpoint{1.208783in}{1.402587in}}%
\pgfpathlineto{\pgfqpoint{1.210897in}{1.411924in}}%
\pgfpathlineto{\pgfqpoint{1.213012in}{1.408419in}}%
\pgfpathlineto{\pgfqpoint{1.215127in}{1.402653in}}%
\pgfpathlineto{\pgfqpoint{1.217242in}{1.407326in}}%
\pgfpathlineto{\pgfqpoint{1.219356in}{1.408957in}}%
\pgfpathlineto{\pgfqpoint{1.221471in}{1.405344in}}%
\pgfpathlineto{\pgfqpoint{1.223586in}{1.406570in}}%
\pgfpathlineto{\pgfqpoint{1.227815in}{1.417108in}}%
\pgfpathlineto{\pgfqpoint{1.229930in}{1.415347in}}%
\pgfpathlineto{\pgfqpoint{1.232044in}{1.418554in}}%
\pgfpathlineto{\pgfqpoint{1.234159in}{1.419155in}}%
\pgfpathlineto{\pgfqpoint{1.236274in}{1.421699in}}%
\pgfpathlineto{\pgfqpoint{1.238389in}{1.421533in}}%
\pgfpathlineto{\pgfqpoint{1.240503in}{1.426119in}}%
\pgfpathlineto{\pgfqpoint{1.244733in}{1.420502in}}%
\pgfpathlineto{\pgfqpoint{1.248962in}{1.424118in}}%
\pgfpathlineto{\pgfqpoint{1.251077in}{1.418147in}}%
\pgfpathlineto{\pgfqpoint{1.253191in}{1.425269in}}%
\pgfpathlineto{\pgfqpoint{1.255306in}{1.424165in}}%
\pgfpathlineto{\pgfqpoint{1.257421in}{1.418028in}}%
\pgfpathlineto{\pgfqpoint{1.259535in}{1.423117in}}%
\pgfpathlineto{\pgfqpoint{1.263765in}{1.426161in}}%
\pgfpathlineto{\pgfqpoint{1.265880in}{1.434918in}}%
\pgfpathlineto{\pgfqpoint{1.267994in}{1.438104in}}%
\pgfpathlineto{\pgfqpoint{1.272224in}{1.438496in}}%
\pgfpathlineto{\pgfqpoint{1.276453in}{1.446793in}}%
\pgfpathlineto{\pgfqpoint{1.280682in}{1.435101in}}%
\pgfpathlineto{\pgfqpoint{1.282797in}{1.435954in}}%
\pgfpathlineto{\pgfqpoint{1.284912in}{1.438453in}}%
\pgfpathlineto{\pgfqpoint{1.287026in}{1.429197in}}%
\pgfpathlineto{\pgfqpoint{1.291256in}{1.433889in}}%
\pgfpathlineto{\pgfqpoint{1.293371in}{1.428952in}}%
\pgfpathlineto{\pgfqpoint{1.295485in}{1.430630in}}%
\pgfpathlineto{\pgfqpoint{1.297600in}{1.436230in}}%
\pgfpathlineto{\pgfqpoint{1.299715in}{1.438404in}}%
\pgfpathlineto{\pgfqpoint{1.301829in}{1.443912in}}%
\pgfpathlineto{\pgfqpoint{1.303944in}{1.453104in}}%
\pgfpathlineto{\pgfqpoint{1.306059in}{1.454035in}}%
\pgfpathlineto{\pgfqpoint{1.310288in}{1.448941in}}%
\pgfpathlineto{\pgfqpoint{1.314517in}{1.450163in}}%
\pgfpathlineto{\pgfqpoint{1.318747in}{1.461312in}}%
\pgfpathlineto{\pgfqpoint{1.320862in}{1.462735in}}%
\pgfpathlineto{\pgfqpoint{1.322976in}{1.459599in}}%
\pgfpathlineto{\pgfqpoint{1.325091in}{1.460840in}}%
\pgfpathlineto{\pgfqpoint{1.329320in}{1.457127in}}%
\pgfpathlineto{\pgfqpoint{1.331435in}{1.458001in}}%
\pgfpathlineto{\pgfqpoint{1.333550in}{1.453782in}}%
\pgfpathlineto{\pgfqpoint{1.335664in}{1.454998in}}%
\pgfpathlineto{\pgfqpoint{1.339894in}{1.447829in}}%
\pgfpathlineto{\pgfqpoint{1.346238in}{1.461752in}}%
\pgfpathlineto{\pgfqpoint{1.350467in}{1.461535in}}%
\pgfpathlineto{\pgfqpoint{1.352582in}{1.470079in}}%
\pgfpathlineto{\pgfqpoint{1.356811in}{1.464153in}}%
\pgfpathlineto{\pgfqpoint{1.358926in}{1.465847in}}%
\pgfpathlineto{\pgfqpoint{1.361041in}{1.459703in}}%
\pgfpathlineto{\pgfqpoint{1.363155in}{1.457145in}}%
\pgfpathlineto{\pgfqpoint{1.365270in}{1.449487in}}%
\pgfpathlineto{\pgfqpoint{1.367385in}{1.450178in}}%
\pgfpathlineto{\pgfqpoint{1.369500in}{1.448924in}}%
\pgfpathlineto{\pgfqpoint{1.373729in}{1.463713in}}%
\pgfpathlineto{\pgfqpoint{1.375844in}{1.460956in}}%
\pgfpathlineto{\pgfqpoint{1.377958in}{1.465413in}}%
\pgfpathlineto{\pgfqpoint{1.380073in}{1.458545in}}%
\pgfpathlineto{\pgfqpoint{1.382188in}{1.464859in}}%
\pgfpathlineto{\pgfqpoint{1.386417in}{1.457248in}}%
\pgfpathlineto{\pgfqpoint{1.388532in}{1.459913in}}%
\pgfpathlineto{\pgfqpoint{1.390646in}{1.468641in}}%
\pgfpathlineto{\pgfqpoint{1.392761in}{1.472501in}}%
\pgfpathlineto{\pgfqpoint{1.394876in}{1.471282in}}%
\pgfpathlineto{\pgfqpoint{1.396991in}{1.476784in}}%
\pgfpathlineto{\pgfqpoint{1.399105in}{1.476742in}}%
\pgfpathlineto{\pgfqpoint{1.403335in}{1.467001in}}%
\pgfpathlineto{\pgfqpoint{1.405449in}{1.465247in}}%
\pgfpathlineto{\pgfqpoint{1.407564in}{1.461983in}}%
\pgfpathlineto{\pgfqpoint{1.409679in}{1.466158in}}%
\pgfpathlineto{\pgfqpoint{1.416023in}{1.471356in}}%
\pgfpathlineto{\pgfqpoint{1.418137in}{1.473969in}}%
\pgfpathlineto{\pgfqpoint{1.422367in}{1.473299in}}%
\pgfpathlineto{\pgfqpoint{1.424482in}{1.467026in}}%
\pgfpathlineto{\pgfqpoint{1.426596in}{1.469075in}}%
\pgfpathlineto{\pgfqpoint{1.430826in}{1.466379in}}%
\pgfpathlineto{\pgfqpoint{1.432940in}{1.467874in}}%
\pgfpathlineto{\pgfqpoint{1.437170in}{1.464638in}}%
\pgfpathlineto{\pgfqpoint{1.439284in}{1.466522in}}%
\pgfpathlineto{\pgfqpoint{1.441399in}{1.458322in}}%
\pgfpathlineto{\pgfqpoint{1.443514in}{1.456315in}}%
\pgfpathlineto{\pgfqpoint{1.447743in}{1.468140in}}%
\pgfpathlineto{\pgfqpoint{1.449858in}{1.468490in}}%
\pgfpathlineto{\pgfqpoint{1.451973in}{1.466351in}}%
\pgfpathlineto{\pgfqpoint{1.458317in}{1.479425in}}%
\pgfpathlineto{\pgfqpoint{1.460431in}{1.489057in}}%
\pgfpathlineto{\pgfqpoint{1.462546in}{1.484175in}}%
\pgfpathlineto{\pgfqpoint{1.475234in}{1.491405in}}%
\pgfpathlineto{\pgfqpoint{1.477349in}{1.477795in}}%
\pgfpathlineto{\pgfqpoint{1.479464in}{1.475848in}}%
\pgfpathlineto{\pgfqpoint{1.481578in}{1.479674in}}%
\pgfpathlineto{\pgfqpoint{1.485808in}{1.473236in}}%
\pgfpathlineto{\pgfqpoint{1.487922in}{1.473653in}}%
\pgfpathlineto{\pgfqpoint{1.490037in}{1.476552in}}%
\pgfpathlineto{\pgfqpoint{1.492152in}{1.476465in}}%
\pgfpathlineto{\pgfqpoint{1.494266in}{1.470659in}}%
\pgfpathlineto{\pgfqpoint{1.496381in}{1.468402in}}%
\pgfpathlineto{\pgfqpoint{1.500611in}{1.468294in}}%
\pgfpathlineto{\pgfqpoint{1.502725in}{1.471485in}}%
\pgfpathlineto{\pgfqpoint{1.506955in}{1.472319in}}%
\pgfpathlineto{\pgfqpoint{1.509069in}{1.461758in}}%
\pgfpathlineto{\pgfqpoint{1.511184in}{1.456981in}}%
\pgfpathlineto{\pgfqpoint{1.513299in}{1.461865in}}%
\pgfpathlineto{\pgfqpoint{1.517528in}{1.460363in}}%
\pgfpathlineto{\pgfqpoint{1.521757in}{1.465264in}}%
\pgfpathlineto{\pgfqpoint{1.523872in}{1.459640in}}%
\pgfpathlineto{\pgfqpoint{1.525987in}{1.457336in}}%
\pgfpathlineto{\pgfqpoint{1.528102in}{1.466927in}}%
\pgfpathlineto{\pgfqpoint{1.532331in}{1.466140in}}%
\pgfpathlineto{\pgfqpoint{1.534446in}{1.470444in}}%
\pgfpathlineto{\pgfqpoint{1.536560in}{1.467455in}}%
\pgfpathlineto{\pgfqpoint{1.538675in}{1.468187in}}%
\pgfpathlineto{\pgfqpoint{1.540790in}{1.470637in}}%
\pgfpathlineto{\pgfqpoint{1.542904in}{1.470948in}}%
\pgfpathlineto{\pgfqpoint{1.545019in}{1.466951in}}%
\pgfpathlineto{\pgfqpoint{1.547134in}{1.469392in}}%
\pgfpathlineto{\pgfqpoint{1.549248in}{1.479222in}}%
\pgfpathlineto{\pgfqpoint{1.551363in}{1.475609in}}%
\pgfpathlineto{\pgfqpoint{1.555593in}{1.462352in}}%
\pgfpathlineto{\pgfqpoint{1.561937in}{1.476347in}}%
\pgfpathlineto{\pgfqpoint{1.566166in}{1.469710in}}%
\pgfpathlineto{\pgfqpoint{1.568281in}{1.468176in}}%
\pgfpathlineto{\pgfqpoint{1.570395in}{1.460549in}}%
\pgfpathlineto{\pgfqpoint{1.572510in}{1.465702in}}%
\pgfpathlineto{\pgfqpoint{1.576740in}{1.468490in}}%
\pgfpathlineto{\pgfqpoint{1.578854in}{1.474851in}}%
\pgfpathlineto{\pgfqpoint{1.580969in}{1.485950in}}%
\pgfpathlineto{\pgfqpoint{1.585198in}{1.489171in}}%
\pgfpathlineto{\pgfqpoint{1.587313in}{1.488805in}}%
\pgfpathlineto{\pgfqpoint{1.589428in}{1.490288in}}%
\pgfpathlineto{\pgfqpoint{1.597886in}{1.481725in}}%
\pgfpathlineto{\pgfqpoint{1.602116in}{1.484090in}}%
\pgfpathlineto{\pgfqpoint{1.606345in}{1.495480in}}%
\pgfpathlineto{\pgfqpoint{1.608460in}{1.495012in}}%
\pgfpathlineto{\pgfqpoint{1.610575in}{1.499312in}}%
\pgfpathlineto{\pgfqpoint{1.612689in}{1.496659in}}%
\pgfpathlineto{\pgfqpoint{1.614804in}{1.491863in}}%
\pgfpathlineto{\pgfqpoint{1.616919in}{1.499349in}}%
\pgfpathlineto{\pgfqpoint{1.621148in}{1.500838in}}%
\pgfpathlineto{\pgfqpoint{1.623263in}{1.504311in}}%
\pgfpathlineto{\pgfqpoint{1.625377in}{1.494588in}}%
\pgfpathlineto{\pgfqpoint{1.627492in}{1.498115in}}%
\pgfpathlineto{\pgfqpoint{1.629607in}{1.497388in}}%
\pgfpathlineto{\pgfqpoint{1.631722in}{1.492125in}}%
\pgfpathlineto{\pgfqpoint{1.635951in}{1.491223in}}%
\pgfpathlineto{\pgfqpoint{1.638066in}{1.493221in}}%
\pgfpathlineto{\pgfqpoint{1.640180in}{1.499440in}}%
\pgfpathlineto{\pgfqpoint{1.642295in}{1.497385in}}%
\pgfpathlineto{\pgfqpoint{1.644410in}{1.490488in}}%
\pgfpathlineto{\pgfqpoint{1.646524in}{1.491938in}}%
\pgfpathlineto{\pgfqpoint{1.650754in}{1.489240in}}%
\pgfpathlineto{\pgfqpoint{1.652868in}{1.492382in}}%
\pgfpathlineto{\pgfqpoint{1.654983in}{1.490975in}}%
\pgfpathlineto{\pgfqpoint{1.661327in}{1.494700in}}%
\pgfpathlineto{\pgfqpoint{1.663442in}{1.494177in}}%
\pgfpathlineto{\pgfqpoint{1.665557in}{1.485415in}}%
\pgfpathlineto{\pgfqpoint{1.667671in}{1.489039in}}%
\pgfpathlineto{\pgfqpoint{1.669786in}{1.497647in}}%
\pgfpathlineto{\pgfqpoint{1.671901in}{1.493205in}}%
\pgfpathlineto{\pgfqpoint{1.674015in}{1.491907in}}%
\pgfpathlineto{\pgfqpoint{1.676130in}{1.495794in}}%
\pgfpathlineto{\pgfqpoint{1.678245in}{1.493644in}}%
\pgfpathlineto{\pgfqpoint{1.680359in}{1.502028in}}%
\pgfpathlineto{\pgfqpoint{1.682474in}{1.503697in}}%
\pgfpathlineto{\pgfqpoint{1.684589in}{1.503184in}}%
\pgfpathlineto{\pgfqpoint{1.686704in}{1.503889in}}%
\pgfpathlineto{\pgfqpoint{1.690933in}{1.496498in}}%
\pgfpathlineto{\pgfqpoint{1.693048in}{1.498423in}}%
\pgfpathlineto{\pgfqpoint{1.695162in}{1.496550in}}%
\pgfpathlineto{\pgfqpoint{1.699392in}{1.488808in}}%
\pgfpathlineto{\pgfqpoint{1.701506in}{1.489461in}}%
\pgfpathlineto{\pgfqpoint{1.703621in}{1.492497in}}%
\pgfpathlineto{\pgfqpoint{1.705736in}{1.485244in}}%
\pgfpathlineto{\pgfqpoint{1.707851in}{1.482115in}}%
\pgfpathlineto{\pgfqpoint{1.712080in}{1.493425in}}%
\pgfpathlineto{\pgfqpoint{1.714195in}{1.492988in}}%
\pgfpathlineto{\pgfqpoint{1.718424in}{1.481368in}}%
\pgfpathlineto{\pgfqpoint{1.720539in}{1.485123in}}%
\pgfpathlineto{\pgfqpoint{1.726883in}{1.469052in}}%
\pgfpathlineto{\pgfqpoint{1.728997in}{1.473906in}}%
\pgfpathlineto{\pgfqpoint{1.731112in}{1.482200in}}%
\pgfpathlineto{\pgfqpoint{1.733227in}{1.485521in}}%
\pgfpathlineto{\pgfqpoint{1.735342in}{1.483059in}}%
\pgfpathlineto{\pgfqpoint{1.737456in}{1.484510in}}%
\pgfpathlineto{\pgfqpoint{1.739571in}{1.483529in}}%
\pgfpathlineto{\pgfqpoint{1.741686in}{1.480019in}}%
\pgfpathlineto{\pgfqpoint{1.743800in}{1.484521in}}%
\pgfpathlineto{\pgfqpoint{1.745915in}{1.483690in}}%
\pgfpathlineto{\pgfqpoint{1.748030in}{1.477971in}}%
\pgfpathlineto{\pgfqpoint{1.750144in}{1.478624in}}%
\pgfpathlineto{\pgfqpoint{1.752259in}{1.482816in}}%
\pgfpathlineto{\pgfqpoint{1.754374in}{1.481351in}}%
\pgfpathlineto{\pgfqpoint{1.756488in}{1.474772in}}%
\pgfpathlineto{\pgfqpoint{1.760718in}{1.475240in}}%
\pgfpathlineto{\pgfqpoint{1.762833in}{1.467406in}}%
\pgfpathlineto{\pgfqpoint{1.771291in}{1.476219in}}%
\pgfpathlineto{\pgfqpoint{1.773406in}{1.476311in}}%
\pgfpathlineto{\pgfqpoint{1.775521in}{1.478939in}}%
\pgfpathlineto{\pgfqpoint{1.777635in}{1.478060in}}%
\pgfpathlineto{\pgfqpoint{1.781865in}{1.483796in}}%
\pgfpathlineto{\pgfqpoint{1.783979in}{1.488130in}}%
\pgfpathlineto{\pgfqpoint{1.786094in}{1.486350in}}%
\pgfpathlineto{\pgfqpoint{1.790324in}{1.491598in}}%
\pgfpathlineto{\pgfqpoint{1.794553in}{1.500293in}}%
\pgfpathlineto{\pgfqpoint{1.796668in}{1.493531in}}%
\pgfpathlineto{\pgfqpoint{1.798782in}{1.497616in}}%
\pgfpathlineto{\pgfqpoint{1.800897in}{1.497724in}}%
\pgfpathlineto{\pgfqpoint{1.805126in}{1.508098in}}%
\pgfpathlineto{\pgfqpoint{1.807241in}{1.509137in}}%
\pgfpathlineto{\pgfqpoint{1.809356in}{1.517383in}}%
\pgfpathlineto{\pgfqpoint{1.811471in}{1.513213in}}%
\pgfpathlineto{\pgfqpoint{1.813585in}{1.519001in}}%
\pgfpathlineto{\pgfqpoint{1.815700in}{1.515260in}}%
\pgfpathlineto{\pgfqpoint{1.817815in}{1.520788in}}%
\pgfpathlineto{\pgfqpoint{1.819929in}{1.509645in}}%
\pgfpathlineto{\pgfqpoint{1.822044in}{1.514688in}}%
\pgfpathlineto{\pgfqpoint{1.824159in}{1.516472in}}%
\pgfpathlineto{\pgfqpoint{1.826273in}{1.520282in}}%
\pgfpathlineto{\pgfqpoint{1.828388in}{1.516291in}}%
\pgfpathlineto{\pgfqpoint{1.830503in}{1.514809in}}%
\pgfpathlineto{\pgfqpoint{1.832617in}{1.511902in}}%
\pgfpathlineto{\pgfqpoint{1.834732in}{1.517597in}}%
\pgfpathlineto{\pgfqpoint{1.838962in}{1.511888in}}%
\pgfpathlineto{\pgfqpoint{1.843191in}{1.519802in}}%
\pgfpathlineto{\pgfqpoint{1.845306in}{1.515227in}}%
\pgfpathlineto{\pgfqpoint{1.847420in}{1.514689in}}%
\pgfpathlineto{\pgfqpoint{1.851650in}{1.519372in}}%
\pgfpathlineto{\pgfqpoint{1.853764in}{1.519871in}}%
\pgfpathlineto{\pgfqpoint{1.857994in}{1.514906in}}%
\pgfpathlineto{\pgfqpoint{1.860108in}{1.506357in}}%
\pgfpathlineto{\pgfqpoint{1.862223in}{1.508715in}}%
\pgfpathlineto{\pgfqpoint{1.866453in}{1.522849in}}%
\pgfpathlineto{\pgfqpoint{1.868567in}{1.523240in}}%
\pgfpathlineto{\pgfqpoint{1.870682in}{1.531794in}}%
\pgfpathlineto{\pgfqpoint{1.877026in}{1.520063in}}%
\pgfpathlineto{\pgfqpoint{1.879141in}{1.519802in}}%
\pgfpathlineto{\pgfqpoint{1.881255in}{1.529041in}}%
\pgfpathlineto{\pgfqpoint{1.883370in}{1.529319in}}%
\pgfpathlineto{\pgfqpoint{1.885485in}{1.528359in}}%
\pgfpathlineto{\pgfqpoint{1.887599in}{1.528933in}}%
\pgfpathlineto{\pgfqpoint{1.889714in}{1.526494in}}%
\pgfpathlineto{\pgfqpoint{1.891829in}{1.526020in}}%
\pgfpathlineto{\pgfqpoint{1.893944in}{1.529603in}}%
\pgfpathlineto{\pgfqpoint{1.896058in}{1.528830in}}%
\pgfpathlineto{\pgfqpoint{1.900288in}{1.532569in}}%
\pgfpathlineto{\pgfqpoint{1.904517in}{1.531477in}}%
\pgfpathlineto{\pgfqpoint{1.906632in}{1.528692in}}%
\pgfpathlineto{\pgfqpoint{1.908746in}{1.531488in}}%
\pgfpathlineto{\pgfqpoint{1.910861in}{1.530629in}}%
\pgfpathlineto{\pgfqpoint{1.915090in}{1.540989in}}%
\pgfpathlineto{\pgfqpoint{1.921435in}{1.532266in}}%
\pgfpathlineto{\pgfqpoint{1.925664in}{1.535524in}}%
\pgfpathlineto{\pgfqpoint{1.927779in}{1.535832in}}%
\pgfpathlineto{\pgfqpoint{1.929893in}{1.538720in}}%
\pgfpathlineto{\pgfqpoint{1.932008in}{1.536076in}}%
\pgfpathlineto{\pgfqpoint{1.934123in}{1.544701in}}%
\pgfpathlineto{\pgfqpoint{1.936237in}{1.540286in}}%
\pgfpathlineto{\pgfqpoint{1.938352in}{1.544742in}}%
\pgfpathlineto{\pgfqpoint{1.944696in}{1.528968in}}%
\pgfpathlineto{\pgfqpoint{1.946811in}{1.539176in}}%
\pgfpathlineto{\pgfqpoint{1.948926in}{1.535856in}}%
\pgfpathlineto{\pgfqpoint{1.951040in}{1.542713in}}%
\pgfpathlineto{\pgfqpoint{1.955270in}{1.536733in}}%
\pgfpathlineto{\pgfqpoint{1.957384in}{1.531535in}}%
\pgfpathlineto{\pgfqpoint{1.959499in}{1.532561in}}%
\pgfpathlineto{\pgfqpoint{1.961614in}{1.527505in}}%
\pgfpathlineto{\pgfqpoint{1.965843in}{1.527661in}}%
\pgfpathlineto{\pgfqpoint{1.970073in}{1.525113in}}%
\pgfpathlineto{\pgfqpoint{1.980646in}{1.502995in}}%
\pgfpathlineto{\pgfqpoint{1.982761in}{1.506821in}}%
\pgfpathlineto{\pgfqpoint{1.986990in}{1.507513in}}%
\pgfpathlineto{\pgfqpoint{1.989105in}{1.512742in}}%
\pgfpathlineto{\pgfqpoint{1.993334in}{1.510287in}}%
\pgfpathlineto{\pgfqpoint{1.995449in}{1.508489in}}%
\pgfpathlineto{\pgfqpoint{1.999678in}{1.522055in}}%
\pgfpathlineto{\pgfqpoint{2.001793in}{1.517595in}}%
\pgfpathlineto{\pgfqpoint{2.003908in}{1.515811in}}%
\pgfpathlineto{\pgfqpoint{2.006022in}{1.518031in}}%
\pgfpathlineto{\pgfqpoint{2.010252in}{1.515501in}}%
\pgfpathlineto{\pgfqpoint{2.012366in}{1.511134in}}%
\pgfpathlineto{\pgfqpoint{2.014481in}{1.518108in}}%
\pgfpathlineto{\pgfqpoint{2.016596in}{1.512717in}}%
\pgfpathlineto{\pgfqpoint{2.020825in}{1.521349in}}%
\pgfpathlineto{\pgfqpoint{2.022940in}{1.519692in}}%
\pgfpathlineto{\pgfqpoint{2.025055in}{1.528359in}}%
\pgfpathlineto{\pgfqpoint{2.027169in}{1.532115in}}%
\pgfpathlineto{\pgfqpoint{2.029284in}{1.526590in}}%
\pgfpathlineto{\pgfqpoint{2.031399in}{1.528181in}}%
\pgfpathlineto{\pgfqpoint{2.033513in}{1.524125in}}%
\pgfpathlineto{\pgfqpoint{2.035628in}{1.529133in}}%
\pgfpathlineto{\pgfqpoint{2.037743in}{1.525744in}}%
\pgfpathlineto{\pgfqpoint{2.039857in}{1.529256in}}%
\pgfpathlineto{\pgfqpoint{2.046202in}{1.518710in}}%
\pgfpathlineto{\pgfqpoint{2.050431in}{1.506736in}}%
\pgfpathlineto{\pgfqpoint{2.052546in}{1.508492in}}%
\pgfpathlineto{\pgfqpoint{2.054660in}{1.504057in}}%
\pgfpathlineto{\pgfqpoint{2.056775in}{1.510649in}}%
\pgfpathlineto{\pgfqpoint{2.058890in}{1.501486in}}%
\pgfpathlineto{\pgfqpoint{2.061004in}{1.501089in}}%
\pgfpathlineto{\pgfqpoint{2.065234in}{1.504435in}}%
\pgfpathlineto{\pgfqpoint{2.067348in}{1.497226in}}%
\pgfpathlineto{\pgfqpoint{2.069463in}{1.496583in}}%
\pgfpathlineto{\pgfqpoint{2.073693in}{1.505581in}}%
\pgfpathlineto{\pgfqpoint{2.075807in}{1.504338in}}%
\pgfpathlineto{\pgfqpoint{2.077922in}{1.505544in}}%
\pgfpathlineto{\pgfqpoint{2.080037in}{1.509732in}}%
\pgfpathlineto{\pgfqpoint{2.082151in}{1.507856in}}%
\pgfpathlineto{\pgfqpoint{2.086381in}{1.513201in}}%
\pgfpathlineto{\pgfqpoint{2.090610in}{1.508008in}}%
\pgfpathlineto{\pgfqpoint{2.094839in}{1.512901in}}%
\pgfpathlineto{\pgfqpoint{2.096954in}{1.504884in}}%
\pgfpathlineto{\pgfqpoint{2.099069in}{1.507241in}}%
\pgfpathlineto{\pgfqpoint{2.103298in}{1.500639in}}%
\pgfpathlineto{\pgfqpoint{2.105413in}{1.508182in}}%
\pgfpathlineto{\pgfqpoint{2.107528in}{1.511468in}}%
\pgfpathlineto{\pgfqpoint{2.109642in}{1.506529in}}%
\pgfpathlineto{\pgfqpoint{2.111757in}{1.513874in}}%
\pgfpathlineto{\pgfqpoint{2.113872in}{1.508688in}}%
\pgfpathlineto{\pgfqpoint{2.115986in}{1.511865in}}%
\pgfpathlineto{\pgfqpoint{2.124445in}{1.532438in}}%
\pgfpathlineto{\pgfqpoint{2.128675in}{1.535348in}}%
\pgfpathlineto{\pgfqpoint{2.130789in}{1.544422in}}%
\pgfpathlineto{\pgfqpoint{2.135019in}{1.540328in}}%
\pgfpathlineto{\pgfqpoint{2.137133in}{1.539157in}}%
\pgfpathlineto{\pgfqpoint{2.139248in}{1.535840in}}%
\pgfpathlineto{\pgfqpoint{2.141363in}{1.539283in}}%
\pgfpathlineto{\pgfqpoint{2.143477in}{1.536286in}}%
\pgfpathlineto{\pgfqpoint{2.145592in}{1.545650in}}%
\pgfpathlineto{\pgfqpoint{2.147707in}{1.540008in}}%
\pgfpathlineto{\pgfqpoint{2.149822in}{1.543509in}}%
\pgfpathlineto{\pgfqpoint{2.151936in}{1.540308in}}%
\pgfpathlineto{\pgfqpoint{2.154051in}{1.543199in}}%
\pgfpathlineto{\pgfqpoint{2.156166in}{1.552336in}}%
\pgfpathlineto{\pgfqpoint{2.158280in}{1.555093in}}%
\pgfpathlineto{\pgfqpoint{2.160395in}{1.546848in}}%
\pgfpathlineto{\pgfqpoint{2.166739in}{1.540941in}}%
\pgfpathlineto{\pgfqpoint{2.170968in}{1.541432in}}%
\pgfpathlineto{\pgfqpoint{2.173083in}{1.534717in}}%
\pgfpathlineto{\pgfqpoint{2.179427in}{1.531312in}}%
\pgfpathlineto{\pgfqpoint{2.181542in}{1.533751in}}%
\pgfpathlineto{\pgfqpoint{2.183657in}{1.534034in}}%
\pgfpathlineto{\pgfqpoint{2.185771in}{1.531996in}}%
\pgfpathlineto{\pgfqpoint{2.187886in}{1.537118in}}%
\pgfpathlineto{\pgfqpoint{2.190001in}{1.539056in}}%
\pgfpathlineto{\pgfqpoint{2.192115in}{1.536498in}}%
\pgfpathlineto{\pgfqpoint{2.196345in}{1.537320in}}%
\pgfpathlineto{\pgfqpoint{2.198459in}{1.533374in}}%
\pgfpathlineto{\pgfqpoint{2.200574in}{1.535470in}}%
\pgfpathlineto{\pgfqpoint{2.202689in}{1.531653in}}%
\pgfpathlineto{\pgfqpoint{2.204804in}{1.525420in}}%
\pgfpathlineto{\pgfqpoint{2.206918in}{1.524079in}}%
\pgfpathlineto{\pgfqpoint{2.213262in}{1.533449in}}%
\pgfpathlineto{\pgfqpoint{2.219606in}{1.533094in}}%
\pgfpathlineto{\pgfqpoint{2.221721in}{1.531219in}}%
\pgfpathlineto{\pgfqpoint{2.223836in}{1.524968in}}%
\pgfpathlineto{\pgfqpoint{2.225950in}{1.529845in}}%
\pgfpathlineto{\pgfqpoint{2.228065in}{1.525227in}}%
\pgfpathlineto{\pgfqpoint{2.230180in}{1.529518in}}%
\pgfpathlineto{\pgfqpoint{2.232295in}{1.525498in}}%
\pgfpathlineto{\pgfqpoint{2.236524in}{1.533688in}}%
\pgfpathlineto{\pgfqpoint{2.238639in}{1.534206in}}%
\pgfpathlineto{\pgfqpoint{2.240753in}{1.530512in}}%
\pgfpathlineto{\pgfqpoint{2.242868in}{1.529425in}}%
\pgfpathlineto{\pgfqpoint{2.244983in}{1.533919in}}%
\pgfpathlineto{\pgfqpoint{2.247097in}{1.535583in}}%
\pgfpathlineto{\pgfqpoint{2.249212in}{1.535060in}}%
\pgfpathlineto{\pgfqpoint{2.251327in}{1.538580in}}%
\pgfpathlineto{\pgfqpoint{2.253441in}{1.536925in}}%
\pgfpathlineto{\pgfqpoint{2.255556in}{1.538466in}}%
\pgfpathlineto{\pgfqpoint{2.259786in}{1.535052in}}%
\pgfpathlineto{\pgfqpoint{2.261900in}{1.536230in}}%
\pgfpathlineto{\pgfqpoint{2.266130in}{1.533715in}}%
\pgfpathlineto{\pgfqpoint{2.268244in}{1.535133in}}%
\pgfpathlineto{\pgfqpoint{2.270359in}{1.531338in}}%
\pgfpathlineto{\pgfqpoint{2.272474in}{1.537212in}}%
\pgfpathlineto{\pgfqpoint{2.274588in}{1.533561in}}%
\pgfpathlineto{\pgfqpoint{2.276703in}{1.538633in}}%
\pgfpathlineto{\pgfqpoint{2.278818in}{1.538441in}}%
\pgfpathlineto{\pgfqpoint{2.280933in}{1.536515in}}%
\pgfpathlineto{\pgfqpoint{2.287277in}{1.519060in}}%
\pgfpathlineto{\pgfqpoint{2.289391in}{1.525584in}}%
\pgfpathlineto{\pgfqpoint{2.291506in}{1.527731in}}%
\pgfpathlineto{\pgfqpoint{2.293621in}{1.526820in}}%
\pgfpathlineto{\pgfqpoint{2.295735in}{1.528408in}}%
\pgfpathlineto{\pgfqpoint{2.297850in}{1.532693in}}%
\pgfpathlineto{\pgfqpoint{2.299965in}{1.533765in}}%
\pgfpathlineto{\pgfqpoint{2.302079in}{1.533187in}}%
\pgfpathlineto{\pgfqpoint{2.306309in}{1.526802in}}%
\pgfpathlineto{\pgfqpoint{2.308424in}{1.529162in}}%
\pgfpathlineto{\pgfqpoint{2.310538in}{1.518604in}}%
\pgfpathlineto{\pgfqpoint{2.312653in}{1.519038in}}%
\pgfpathlineto{\pgfqpoint{2.318997in}{1.506125in}}%
\pgfpathlineto{\pgfqpoint{2.321112in}{1.504430in}}%
\pgfpathlineto{\pgfqpoint{2.323226in}{1.505322in}}%
\pgfpathlineto{\pgfqpoint{2.325341in}{1.503307in}}%
\pgfpathlineto{\pgfqpoint{2.327456in}{1.505491in}}%
\pgfpathlineto{\pgfqpoint{2.329570in}{1.503626in}}%
\pgfpathlineto{\pgfqpoint{2.331685in}{1.494404in}}%
\pgfpathlineto{\pgfqpoint{2.333800in}{1.493664in}}%
\pgfpathlineto{\pgfqpoint{2.335915in}{1.491540in}}%
\pgfpathlineto{\pgfqpoint{2.338029in}{1.492252in}}%
\pgfpathlineto{\pgfqpoint{2.342259in}{1.484047in}}%
\pgfpathlineto{\pgfqpoint{2.346488in}{1.489358in}}%
\pgfpathlineto{\pgfqpoint{2.348603in}{1.497172in}}%
\pgfpathlineto{\pgfqpoint{2.350717in}{1.497587in}}%
\pgfpathlineto{\pgfqpoint{2.352832in}{1.499608in}}%
\pgfpathlineto{\pgfqpoint{2.357061in}{1.511916in}}%
\pgfpathlineto{\pgfqpoint{2.359176in}{1.506553in}}%
\pgfpathlineto{\pgfqpoint{2.365520in}{1.521699in}}%
\pgfpathlineto{\pgfqpoint{2.367635in}{1.512313in}}%
\pgfpathlineto{\pgfqpoint{2.369750in}{1.514392in}}%
\pgfpathlineto{\pgfqpoint{2.371864in}{1.520641in}}%
\pgfpathlineto{\pgfqpoint{2.373979in}{1.522158in}}%
\pgfpathlineto{\pgfqpoint{2.378208in}{1.531087in}}%
\pgfpathlineto{\pgfqpoint{2.380323in}{1.528635in}}%
\pgfpathlineto{\pgfqpoint{2.382438in}{1.528906in}}%
\pgfpathlineto{\pgfqpoint{2.384553in}{1.527357in}}%
\pgfpathlineto{\pgfqpoint{2.386667in}{1.532887in}}%
\pgfpathlineto{\pgfqpoint{2.388782in}{1.531710in}}%
\pgfpathlineto{\pgfqpoint{2.390897in}{1.535065in}}%
\pgfpathlineto{\pgfqpoint{2.393011in}{1.535539in}}%
\pgfpathlineto{\pgfqpoint{2.395126in}{1.537642in}}%
\pgfpathlineto{\pgfqpoint{2.399355in}{1.534917in}}%
\pgfpathlineto{\pgfqpoint{2.407814in}{1.514799in}}%
\pgfpathlineto{\pgfqpoint{2.409929in}{1.522163in}}%
\pgfpathlineto{\pgfqpoint{2.416273in}{1.512083in}}%
\pgfpathlineto{\pgfqpoint{2.420502in}{1.521326in}}%
\pgfpathlineto{\pgfqpoint{2.424732in}{1.518102in}}%
\pgfpathlineto{\pgfqpoint{2.426846in}{1.524369in}}%
\pgfpathlineto{\pgfqpoint{2.428961in}{1.526625in}}%
\pgfpathlineto{\pgfqpoint{2.431076in}{1.525345in}}%
\pgfpathlineto{\pgfqpoint{2.435305in}{1.530471in}}%
\pgfpathlineto{\pgfqpoint{2.437420in}{1.530892in}}%
\pgfpathlineto{\pgfqpoint{2.439535in}{1.533022in}}%
\pgfpathlineto{\pgfqpoint{2.441649in}{1.530742in}}%
\pgfpathlineto{\pgfqpoint{2.445879in}{1.521821in}}%
\pgfpathlineto{\pgfqpoint{2.447993in}{1.510751in}}%
\pgfpathlineto{\pgfqpoint{2.450108in}{1.509133in}}%
\pgfpathlineto{\pgfqpoint{2.452223in}{1.512834in}}%
\pgfpathlineto{\pgfqpoint{2.454337in}{1.519160in}}%
\pgfpathlineto{\pgfqpoint{2.456452in}{1.508295in}}%
\pgfpathlineto{\pgfqpoint{2.460681in}{1.522243in}}%
\pgfpathlineto{\pgfqpoint{2.462796in}{1.518337in}}%
\pgfpathlineto{\pgfqpoint{2.467026in}{1.529081in}}%
\pgfpathlineto{\pgfqpoint{2.471255in}{1.524764in}}%
\pgfpathlineto{\pgfqpoint{2.473370in}{1.520808in}}%
\pgfpathlineto{\pgfqpoint{2.475484in}{1.525602in}}%
\pgfpathlineto{\pgfqpoint{2.477599in}{1.523323in}}%
\pgfpathlineto{\pgfqpoint{2.483943in}{1.529958in}}%
\pgfpathlineto{\pgfqpoint{2.488172in}{1.521067in}}%
\pgfpathlineto{\pgfqpoint{2.492402in}{1.527682in}}%
\pgfpathlineto{\pgfqpoint{2.494517in}{1.529866in}}%
\pgfpathlineto{\pgfqpoint{2.496631in}{1.533750in}}%
\pgfpathlineto{\pgfqpoint{2.498746in}{1.531412in}}%
\pgfpathlineto{\pgfqpoint{2.502975in}{1.534672in}}%
\pgfpathlineto{\pgfqpoint{2.509319in}{1.528917in}}%
\pgfpathlineto{\pgfqpoint{2.511434in}{1.528303in}}%
\pgfpathlineto{\pgfqpoint{2.513549in}{1.524435in}}%
\pgfpathlineto{\pgfqpoint{2.517778in}{1.524843in}}%
\pgfpathlineto{\pgfqpoint{2.519893in}{1.525817in}}%
\pgfpathlineto{\pgfqpoint{2.522008in}{1.522645in}}%
\pgfpathlineto{\pgfqpoint{2.524122in}{1.526324in}}%
\pgfpathlineto{\pgfqpoint{2.526237in}{1.527260in}}%
\pgfpathlineto{\pgfqpoint{2.528352in}{1.525048in}}%
\pgfpathlineto{\pgfqpoint{2.530466in}{1.521065in}}%
\pgfpathlineto{\pgfqpoint{2.532581in}{1.522865in}}%
\pgfpathlineto{\pgfqpoint{2.534696in}{1.521748in}}%
\pgfpathlineto{\pgfqpoint{2.536810in}{1.525789in}}%
\pgfpathlineto{\pgfqpoint{2.538925in}{1.524376in}}%
\pgfpathlineto{\pgfqpoint{2.543155in}{1.518625in}}%
\pgfpathlineto{\pgfqpoint{2.547384in}{1.534576in}}%
\pgfpathlineto{\pgfqpoint{2.549499in}{1.536368in}}%
\pgfpathlineto{\pgfqpoint{2.551613in}{1.534774in}}%
\pgfpathlineto{\pgfqpoint{2.560072in}{1.543997in}}%
\pgfpathlineto{\pgfqpoint{2.562187in}{1.537985in}}%
\pgfpathlineto{\pgfqpoint{2.564301in}{1.535795in}}%
\pgfpathlineto{\pgfqpoint{2.566416in}{1.536004in}}%
\pgfpathlineto{\pgfqpoint{2.568531in}{1.539333in}}%
\pgfpathlineto{\pgfqpoint{2.570646in}{1.535525in}}%
\pgfpathlineto{\pgfqpoint{2.572760in}{1.537671in}}%
\pgfpathlineto{\pgfqpoint{2.574875in}{1.543718in}}%
\pgfpathlineto{\pgfqpoint{2.576990in}{1.544961in}}%
\pgfpathlineto{\pgfqpoint{2.579104in}{1.542919in}}%
\pgfpathlineto{\pgfqpoint{2.581219in}{1.550187in}}%
\pgfpathlineto{\pgfqpoint{2.585448in}{1.536337in}}%
\pgfpathlineto{\pgfqpoint{2.587563in}{1.534485in}}%
\pgfpathlineto{\pgfqpoint{2.589678in}{1.540237in}}%
\pgfpathlineto{\pgfqpoint{2.591792in}{1.534797in}}%
\pgfpathlineto{\pgfqpoint{2.593907in}{1.536505in}}%
\pgfpathlineto{\pgfqpoint{2.596022in}{1.534349in}}%
\pgfpathlineto{\pgfqpoint{2.600251in}{1.536433in}}%
\pgfpathlineto{\pgfqpoint{2.602366in}{1.543259in}}%
\pgfpathlineto{\pgfqpoint{2.604481in}{1.541247in}}%
\pgfpathlineto{\pgfqpoint{2.606595in}{1.541915in}}%
\pgfpathlineto{\pgfqpoint{2.608710in}{1.536309in}}%
\pgfpathlineto{\pgfqpoint{2.610825in}{1.534275in}}%
\pgfpathlineto{\pgfqpoint{2.612939in}{1.524401in}}%
\pgfpathlineto{\pgfqpoint{2.615054in}{1.528892in}}%
\pgfpathlineto{\pgfqpoint{2.617169in}{1.527728in}}%
\pgfpathlineto{\pgfqpoint{2.619284in}{1.525221in}}%
\pgfpathlineto{\pgfqpoint{2.621398in}{1.526630in}}%
\pgfpathlineto{\pgfqpoint{2.625628in}{1.525661in}}%
\pgfpathlineto{\pgfqpoint{2.629857in}{1.515777in}}%
\pgfpathlineto{\pgfqpoint{2.631972in}{1.517817in}}%
\pgfpathlineto{\pgfqpoint{2.634086in}{1.522303in}}%
\pgfpathlineto{\pgfqpoint{2.636201in}{1.523043in}}%
\pgfpathlineto{\pgfqpoint{2.640430in}{1.531830in}}%
\pgfpathlineto{\pgfqpoint{2.642545in}{1.531327in}}%
\pgfpathlineto{\pgfqpoint{2.644660in}{1.533140in}}%
\pgfpathlineto{\pgfqpoint{2.646775in}{1.530909in}}%
\pgfpathlineto{\pgfqpoint{2.648889in}{1.531758in}}%
\pgfpathlineto{\pgfqpoint{2.653119in}{1.540170in}}%
\pgfpathlineto{\pgfqpoint{2.655233in}{1.539894in}}%
\pgfpathlineto{\pgfqpoint{2.657348in}{1.534228in}}%
\pgfpathlineto{\pgfqpoint{2.663692in}{1.551606in}}%
\pgfpathlineto{\pgfqpoint{2.665807in}{1.550923in}}%
\pgfpathlineto{\pgfqpoint{2.672151in}{1.553668in}}%
\pgfpathlineto{\pgfqpoint{2.674266in}{1.549506in}}%
\pgfpathlineto{\pgfqpoint{2.676380in}{1.548161in}}%
\pgfpathlineto{\pgfqpoint{2.680610in}{1.538726in}}%
\pgfpathlineto{\pgfqpoint{2.686954in}{1.536167in}}%
\pgfpathlineto{\pgfqpoint{2.689068in}{1.533576in}}%
\pgfpathlineto{\pgfqpoint{2.691183in}{1.524085in}}%
\pgfpathlineto{\pgfqpoint{2.693298in}{1.535129in}}%
\pgfpathlineto{\pgfqpoint{2.695412in}{1.535935in}}%
\pgfpathlineto{\pgfqpoint{2.697527in}{1.538330in}}%
\pgfpathlineto{\pgfqpoint{2.699642in}{1.548999in}}%
\pgfpathlineto{\pgfqpoint{2.701757in}{1.543704in}}%
\pgfpathlineto{\pgfqpoint{2.703871in}{1.544924in}}%
\pgfpathlineto{\pgfqpoint{2.705986in}{1.555111in}}%
\pgfpathlineto{\pgfqpoint{2.708101in}{1.550906in}}%
\pgfpathlineto{\pgfqpoint{2.710215in}{1.557715in}}%
\pgfpathlineto{\pgfqpoint{2.712330in}{1.548929in}}%
\pgfpathlineto{\pgfqpoint{2.714445in}{1.556458in}}%
\pgfpathlineto{\pgfqpoint{2.720789in}{1.552558in}}%
\pgfpathlineto{\pgfqpoint{2.725018in}{1.550352in}}%
\pgfpathlineto{\pgfqpoint{2.727133in}{1.547353in}}%
\pgfpathlineto{\pgfqpoint{2.729248in}{1.546985in}}%
\pgfpathlineto{\pgfqpoint{2.731362in}{1.548675in}}%
\pgfpathlineto{\pgfqpoint{2.737706in}{1.546524in}}%
\pgfpathlineto{\pgfqpoint{2.746165in}{1.562925in}}%
\pgfpathlineto{\pgfqpoint{2.748280in}{1.563450in}}%
\pgfpathlineto{\pgfqpoint{2.750395in}{1.559196in}}%
\pgfpathlineto{\pgfqpoint{2.752509in}{1.564704in}}%
\pgfpathlineto{\pgfqpoint{2.754624in}{1.557757in}}%
\pgfpathlineto{\pgfqpoint{2.756739in}{1.555396in}}%
\pgfpathlineto{\pgfqpoint{2.760968in}{1.565689in}}%
\pgfpathlineto{\pgfqpoint{2.763083in}{1.560738in}}%
\pgfpathlineto{\pgfqpoint{2.765197in}{1.560638in}}%
\pgfpathlineto{\pgfqpoint{2.767312in}{1.564491in}}%
\pgfpathlineto{\pgfqpoint{2.769427in}{1.564788in}}%
\pgfpathlineto{\pgfqpoint{2.771541in}{1.566715in}}%
\pgfpathlineto{\pgfqpoint{2.773656in}{1.564079in}}%
\pgfpathlineto{\pgfqpoint{2.777886in}{1.564171in}}%
\pgfpathlineto{\pgfqpoint{2.780000in}{1.576587in}}%
\pgfpathlineto{\pgfqpoint{2.782115in}{1.574731in}}%
\pgfpathlineto{\pgfqpoint{2.784230in}{1.570950in}}%
\pgfpathlineto{\pgfqpoint{2.786344in}{1.573583in}}%
\pgfpathlineto{\pgfqpoint{2.788459in}{1.573528in}}%
\pgfpathlineto{\pgfqpoint{2.794803in}{1.564222in}}%
\pgfpathlineto{\pgfqpoint{2.796918in}{1.568751in}}%
\pgfpathlineto{\pgfqpoint{2.799032in}{1.567294in}}%
\pgfpathlineto{\pgfqpoint{2.801147in}{1.570346in}}%
\pgfpathlineto{\pgfqpoint{2.803262in}{1.567905in}}%
\pgfpathlineto{\pgfqpoint{2.807491in}{1.573056in}}%
\pgfpathlineto{\pgfqpoint{2.809606in}{1.569275in}}%
\pgfpathlineto{\pgfqpoint{2.811721in}{1.570320in}}%
\pgfpathlineto{\pgfqpoint{2.815950in}{1.568382in}}%
\pgfpathlineto{\pgfqpoint{2.818065in}{1.569485in}}%
\pgfpathlineto{\pgfqpoint{2.820179in}{1.574456in}}%
\pgfpathlineto{\pgfqpoint{2.822294in}{1.576284in}}%
\pgfpathlineto{\pgfqpoint{2.824409in}{1.573990in}}%
\pgfpathlineto{\pgfqpoint{2.826523in}{1.569145in}}%
\pgfpathlineto{\pgfqpoint{2.828638in}{1.572668in}}%
\pgfpathlineto{\pgfqpoint{2.830753in}{1.573155in}}%
\pgfpathlineto{\pgfqpoint{2.832868in}{1.570666in}}%
\pgfpathlineto{\pgfqpoint{2.834982in}{1.571424in}}%
\pgfpathlineto{\pgfqpoint{2.839212in}{1.577278in}}%
\pgfpathlineto{\pgfqpoint{2.841326in}{1.573296in}}%
\pgfpathlineto{\pgfqpoint{2.845556in}{1.578355in}}%
\pgfpathlineto{\pgfqpoint{2.847670in}{1.573906in}}%
\pgfpathlineto{\pgfqpoint{2.849785in}{1.577145in}}%
\pgfpathlineto{\pgfqpoint{2.851900in}{1.586137in}}%
\pgfpathlineto{\pgfqpoint{2.854015in}{1.580604in}}%
\pgfpathlineto{\pgfqpoint{2.856129in}{1.584583in}}%
\pgfpathlineto{\pgfqpoint{2.858244in}{1.584479in}}%
\pgfpathlineto{\pgfqpoint{2.860359in}{1.587265in}}%
\pgfpathlineto{\pgfqpoint{2.862473in}{1.582451in}}%
\pgfpathlineto{\pgfqpoint{2.864588in}{1.585162in}}%
\pgfpathlineto{\pgfqpoint{2.866703in}{1.585202in}}%
\pgfpathlineto{\pgfqpoint{2.873047in}{1.592096in}}%
\pgfpathlineto{\pgfqpoint{2.875161in}{1.586439in}}%
\pgfpathlineto{\pgfqpoint{2.879391in}{1.596320in}}%
\pgfpathlineto{\pgfqpoint{2.881506in}{1.596179in}}%
\pgfpathlineto{\pgfqpoint{2.885735in}{1.583587in}}%
\pgfpathlineto{\pgfqpoint{2.889964in}{1.582341in}}%
\pgfpathlineto{\pgfqpoint{2.892079in}{1.589102in}}%
\pgfpathlineto{\pgfqpoint{2.894194in}{1.590189in}}%
\pgfpathlineto{\pgfqpoint{2.896308in}{1.592692in}}%
\pgfpathlineto{\pgfqpoint{2.898423in}{1.588167in}}%
\pgfpathlineto{\pgfqpoint{2.902652in}{1.587743in}}%
\pgfpathlineto{\pgfqpoint{2.908997in}{1.598638in}}%
\pgfpathlineto{\pgfqpoint{2.911111in}{1.596658in}}%
\pgfpathlineto{\pgfqpoint{2.915341in}{1.599420in}}%
\pgfpathlineto{\pgfqpoint{2.917455in}{1.599931in}}%
\pgfpathlineto{\pgfqpoint{2.921685in}{1.582528in}}%
\pgfpathlineto{\pgfqpoint{2.923799in}{1.585980in}}%
\pgfpathlineto{\pgfqpoint{2.928029in}{1.585538in}}%
\pgfpathlineto{\pgfqpoint{2.932258in}{1.591472in}}%
\pgfpathlineto{\pgfqpoint{2.936488in}{1.590407in}}%
\pgfpathlineto{\pgfqpoint{2.940717in}{1.586909in}}%
\pgfpathlineto{\pgfqpoint{2.942832in}{1.590847in}}%
\pgfpathlineto{\pgfqpoint{2.944946in}{1.589804in}}%
\pgfpathlineto{\pgfqpoint{2.947061in}{1.576740in}}%
\pgfpathlineto{\pgfqpoint{2.953405in}{1.577680in}}%
\pgfpathlineto{\pgfqpoint{2.955520in}{1.568467in}}%
\pgfpathlineto{\pgfqpoint{2.957634in}{1.573745in}}%
\pgfpathlineto{\pgfqpoint{2.959749in}{1.570533in}}%
\pgfpathlineto{\pgfqpoint{2.961864in}{1.572711in}}%
\pgfpathlineto{\pgfqpoint{2.963979in}{1.572103in}}%
\pgfpathlineto{\pgfqpoint{2.966093in}{1.569809in}}%
\pgfpathlineto{\pgfqpoint{2.970323in}{1.573954in}}%
\pgfpathlineto{\pgfqpoint{2.972437in}{1.576790in}}%
\pgfpathlineto{\pgfqpoint{2.974552in}{1.577062in}}%
\pgfpathlineto{\pgfqpoint{2.976667in}{1.587616in}}%
\pgfpathlineto{\pgfqpoint{2.978781in}{1.587397in}}%
\pgfpathlineto{\pgfqpoint{2.980896in}{1.584757in}}%
\pgfpathlineto{\pgfqpoint{2.983011in}{1.586588in}}%
\pgfpathlineto{\pgfqpoint{2.985126in}{1.581988in}}%
\pgfpathlineto{\pgfqpoint{2.987240in}{1.584655in}}%
\pgfpathlineto{\pgfqpoint{2.993584in}{1.580944in}}%
\pgfpathlineto{\pgfqpoint{2.997814in}{1.583764in}}%
\pgfpathlineto{\pgfqpoint{2.999928in}{1.589165in}}%
\pgfpathlineto{\pgfqpoint{3.004158in}{1.581843in}}%
\pgfpathlineto{\pgfqpoint{3.006272in}{1.572010in}}%
\pgfpathlineto{\pgfqpoint{3.008387in}{1.578882in}}%
\pgfpathlineto{\pgfqpoint{3.010502in}{1.578210in}}%
\pgfpathlineto{\pgfqpoint{3.012617in}{1.582642in}}%
\pgfpathlineto{\pgfqpoint{3.014731in}{1.584308in}}%
\pgfpathlineto{\pgfqpoint{3.016846in}{1.578445in}}%
\pgfpathlineto{\pgfqpoint{3.021075in}{1.576446in}}%
\pgfpathlineto{\pgfqpoint{3.023190in}{1.573234in}}%
\pgfpathlineto{\pgfqpoint{3.025305in}{1.572551in}}%
\pgfpathlineto{\pgfqpoint{3.027419in}{1.573254in}}%
\pgfpathlineto{\pgfqpoint{3.029534in}{1.575209in}}%
\pgfpathlineto{\pgfqpoint{3.033763in}{1.584615in}}%
\pgfpathlineto{\pgfqpoint{3.035878in}{1.581466in}}%
\pgfpathlineto{\pgfqpoint{3.037993in}{1.589648in}}%
\pgfpathlineto{\pgfqpoint{3.046452in}{1.582669in}}%
\pgfpathlineto{\pgfqpoint{3.048566in}{1.581425in}}%
\pgfpathlineto{\pgfqpoint{3.050681in}{1.585885in}}%
\pgfpathlineto{\pgfqpoint{3.052796in}{1.583762in}}%
\pgfpathlineto{\pgfqpoint{3.054910in}{1.586568in}}%
\pgfpathlineto{\pgfqpoint{3.057025in}{1.587361in}}%
\pgfpathlineto{\pgfqpoint{3.059140in}{1.583428in}}%
\pgfpathlineto{\pgfqpoint{3.063369in}{1.567575in}}%
\pgfpathlineto{\pgfqpoint{3.065484in}{1.569652in}}%
\pgfpathlineto{\pgfqpoint{3.067599in}{1.565476in}}%
\pgfpathlineto{\pgfqpoint{3.069713in}{1.565293in}}%
\pgfpathlineto{\pgfqpoint{3.071828in}{1.571025in}}%
\pgfpathlineto{\pgfqpoint{3.073943in}{1.569949in}}%
\pgfpathlineto{\pgfqpoint{3.076057in}{1.571930in}}%
\pgfpathlineto{\pgfqpoint{3.078172in}{1.576017in}}%
\pgfpathlineto{\pgfqpoint{3.080287in}{1.570280in}}%
\pgfpathlineto{\pgfqpoint{3.082401in}{1.576046in}}%
\pgfpathlineto{\pgfqpoint{3.084516in}{1.577829in}}%
\pgfpathlineto{\pgfqpoint{3.088746in}{1.585602in}}%
\pgfpathlineto{\pgfqpoint{3.101434in}{1.593066in}}%
\pgfpathlineto{\pgfqpoint{3.105663in}{1.601459in}}%
\pgfpathlineto{\pgfqpoint{3.107778in}{1.602455in}}%
\pgfpathlineto{\pgfqpoint{3.114122in}{1.589030in}}%
\pgfpathlineto{\pgfqpoint{3.120466in}{1.589743in}}%
\pgfpathlineto{\pgfqpoint{3.122581in}{1.593060in}}%
\pgfpathlineto{\pgfqpoint{3.124695in}{1.591263in}}%
\pgfpathlineto{\pgfqpoint{3.128925in}{1.591137in}}%
\pgfpathlineto{\pgfqpoint{3.133154in}{1.599551in}}%
\pgfpathlineto{\pgfqpoint{3.135269in}{1.593433in}}%
\pgfpathlineto{\pgfqpoint{3.137383in}{1.593244in}}%
\pgfpathlineto{\pgfqpoint{3.145842in}{1.604434in}}%
\pgfpathlineto{\pgfqpoint{3.147957in}{1.601642in}}%
\pgfpathlineto{\pgfqpoint{3.150072in}{1.604099in}}%
\pgfpathlineto{\pgfqpoint{3.152186in}{1.602320in}}%
\pgfpathlineto{\pgfqpoint{3.156416in}{1.607841in}}%
\pgfpathlineto{\pgfqpoint{3.158530in}{1.607082in}}%
\pgfpathlineto{\pgfqpoint{3.164874in}{1.617831in}}%
\pgfpathlineto{\pgfqpoint{3.166989in}{1.611858in}}%
\pgfpathlineto{\pgfqpoint{3.169104in}{1.611026in}}%
\pgfpathlineto{\pgfqpoint{3.171219in}{1.607709in}}%
\pgfpathlineto{\pgfqpoint{3.173333in}{1.609266in}}%
\pgfpathlineto{\pgfqpoint{3.175448in}{1.606916in}}%
\pgfpathlineto{\pgfqpoint{3.183907in}{1.585833in}}%
\pgfpathlineto{\pgfqpoint{3.188136in}{1.582210in}}%
\pgfpathlineto{\pgfqpoint{3.190251in}{1.585588in}}%
\pgfpathlineto{\pgfqpoint{3.192366in}{1.579492in}}%
\pgfpathlineto{\pgfqpoint{3.194480in}{1.578072in}}%
\pgfpathlineto{\pgfqpoint{3.196595in}{1.580189in}}%
\pgfpathlineto{\pgfqpoint{3.198710in}{1.579848in}}%
\pgfpathlineto{\pgfqpoint{3.200824in}{1.577910in}}%
\pgfpathlineto{\pgfqpoint{3.202939in}{1.578681in}}%
\pgfpathlineto{\pgfqpoint{3.205054in}{1.583134in}}%
\pgfpathlineto{\pgfqpoint{3.207168in}{1.582454in}}%
\pgfpathlineto{\pgfqpoint{3.209283in}{1.592359in}}%
\pgfpathlineto{\pgfqpoint{3.211398in}{1.589996in}}%
\pgfpathlineto{\pgfqpoint{3.213512in}{1.600275in}}%
\pgfpathlineto{\pgfqpoint{3.215627in}{1.604173in}}%
\pgfpathlineto{\pgfqpoint{3.217742in}{1.601107in}}%
\pgfpathlineto{\pgfqpoint{3.219857in}{1.603045in}}%
\pgfpathlineto{\pgfqpoint{3.221971in}{1.619030in}}%
\pgfpathlineto{\pgfqpoint{3.224086in}{1.625997in}}%
\pgfpathlineto{\pgfqpoint{3.226201in}{1.625842in}}%
\pgfpathlineto{\pgfqpoint{3.230430in}{1.619336in}}%
\pgfpathlineto{\pgfqpoint{3.232545in}{1.618581in}}%
\pgfpathlineto{\pgfqpoint{3.234659in}{1.614375in}}%
\pgfpathlineto{\pgfqpoint{3.236774in}{1.614474in}}%
\pgfpathlineto{\pgfqpoint{3.241003in}{1.603379in}}%
\pgfpathlineto{\pgfqpoint{3.243118in}{1.602510in}}%
\pgfpathlineto{\pgfqpoint{3.245233in}{1.605131in}}%
\pgfpathlineto{\pgfqpoint{3.247348in}{1.605475in}}%
\pgfpathlineto{\pgfqpoint{3.249462in}{1.610553in}}%
\pgfpathlineto{\pgfqpoint{3.251577in}{1.610739in}}%
\pgfpathlineto{\pgfqpoint{3.255806in}{1.621124in}}%
\pgfpathlineto{\pgfqpoint{3.257921in}{1.617344in}}%
\pgfpathlineto{\pgfqpoint{3.260036in}{1.618007in}}%
\pgfpathlineto{\pgfqpoint{3.262150in}{1.620378in}}%
\pgfpathlineto{\pgfqpoint{3.266380in}{1.630768in}}%
\pgfpathlineto{\pgfqpoint{3.268494in}{1.632914in}}%
\pgfpathlineto{\pgfqpoint{3.270609in}{1.624583in}}%
\pgfpathlineto{\pgfqpoint{3.272724in}{1.631112in}}%
\pgfpathlineto{\pgfqpoint{3.274839in}{1.631169in}}%
\pgfpathlineto{\pgfqpoint{3.279068in}{1.621996in}}%
\pgfpathlineto{\pgfqpoint{3.281183in}{1.621923in}}%
\pgfpathlineto{\pgfqpoint{3.283297in}{1.617894in}}%
\pgfpathlineto{\pgfqpoint{3.285412in}{1.624256in}}%
\pgfpathlineto{\pgfqpoint{3.287527in}{1.624681in}}%
\pgfpathlineto{\pgfqpoint{3.291756in}{1.618288in}}%
\pgfpathlineto{\pgfqpoint{3.293871in}{1.622316in}}%
\pgfpathlineto{\pgfqpoint{3.295985in}{1.620705in}}%
\pgfpathlineto{\pgfqpoint{3.300215in}{1.630981in}}%
\pgfpathlineto{\pgfqpoint{3.302330in}{1.630300in}}%
\pgfpathlineto{\pgfqpoint{3.304444in}{1.624853in}}%
\pgfpathlineto{\pgfqpoint{3.308674in}{1.635549in}}%
\pgfpathlineto{\pgfqpoint{3.310788in}{1.634909in}}%
\pgfpathlineto{\pgfqpoint{3.315018in}{1.642791in}}%
\pgfpathlineto{\pgfqpoint{3.317132in}{1.639987in}}%
\pgfpathlineto{\pgfqpoint{3.321362in}{1.645643in}}%
\pgfpathlineto{\pgfqpoint{3.323477in}{1.643355in}}%
\pgfpathlineto{\pgfqpoint{3.325591in}{1.646397in}}%
\pgfpathlineto{\pgfqpoint{3.327706in}{1.646942in}}%
\pgfpathlineto{\pgfqpoint{3.329821in}{1.649075in}}%
\pgfpathlineto{\pgfqpoint{3.331935in}{1.647078in}}%
\pgfpathlineto{\pgfqpoint{3.334050in}{1.648776in}}%
\pgfpathlineto{\pgfqpoint{3.338279in}{1.640730in}}%
\pgfpathlineto{\pgfqpoint{3.340394in}{1.643819in}}%
\pgfpathlineto{\pgfqpoint{3.344623in}{1.653039in}}%
\pgfpathlineto{\pgfqpoint{3.348853in}{1.646963in}}%
\pgfpathlineto{\pgfqpoint{3.350968in}{1.646924in}}%
\pgfpathlineto{\pgfqpoint{3.353082in}{1.639092in}}%
\pgfpathlineto{\pgfqpoint{3.355197in}{1.637504in}}%
\pgfpathlineto{\pgfqpoint{3.359426in}{1.646259in}}%
\pgfpathlineto{\pgfqpoint{3.361541in}{1.645325in}}%
\pgfpathlineto{\pgfqpoint{3.365770in}{1.639868in}}%
\pgfpathlineto{\pgfqpoint{3.370000in}{1.643235in}}%
\pgfpathlineto{\pgfqpoint{3.372114in}{1.650141in}}%
\pgfpathlineto{\pgfqpoint{3.374229in}{1.645671in}}%
\pgfpathlineto{\pgfqpoint{3.376344in}{1.644274in}}%
\pgfpathlineto{\pgfqpoint{3.378459in}{1.648243in}}%
\pgfpathlineto{\pgfqpoint{3.380573in}{1.647873in}}%
\pgfpathlineto{\pgfqpoint{3.382688in}{1.653421in}}%
\pgfpathlineto{\pgfqpoint{3.384803in}{1.652261in}}%
\pgfpathlineto{\pgfqpoint{3.389032in}{1.645244in}}%
\pgfpathlineto{\pgfqpoint{3.391147in}{1.639832in}}%
\pgfpathlineto{\pgfqpoint{3.393261in}{1.641616in}}%
\pgfpathlineto{\pgfqpoint{3.397491in}{1.652074in}}%
\pgfpathlineto{\pgfqpoint{3.401720in}{1.652557in}}%
\pgfpathlineto{\pgfqpoint{3.403835in}{1.649134in}}%
\pgfpathlineto{\pgfqpoint{3.405950in}{1.655664in}}%
\pgfpathlineto{\pgfqpoint{3.408064in}{1.654439in}}%
\pgfpathlineto{\pgfqpoint{3.410179in}{1.651055in}}%
\pgfpathlineto{\pgfqpoint{3.412294in}{1.658266in}}%
\pgfpathlineto{\pgfqpoint{3.416523in}{1.661342in}}%
\pgfpathlineto{\pgfqpoint{3.418638in}{1.666932in}}%
\pgfpathlineto{\pgfqpoint{3.422867in}{1.685030in}}%
\pgfpathlineto{\pgfqpoint{3.424982in}{1.686895in}}%
\pgfpathlineto{\pgfqpoint{3.427097in}{1.695961in}}%
\pgfpathlineto{\pgfqpoint{3.431326in}{1.691338in}}%
\pgfpathlineto{\pgfqpoint{3.433441in}{1.693957in}}%
\pgfpathlineto{\pgfqpoint{3.435555in}{1.693983in}}%
\pgfpathlineto{\pgfqpoint{3.437670in}{1.697804in}}%
\pgfpathlineto{\pgfqpoint{3.439785in}{1.695320in}}%
\pgfpathlineto{\pgfqpoint{3.441899in}{1.698489in}}%
\pgfpathlineto{\pgfqpoint{3.448243in}{1.693261in}}%
\pgfpathlineto{\pgfqpoint{3.450358in}{1.694963in}}%
\pgfpathlineto{\pgfqpoint{3.456702in}{1.688289in}}%
\pgfpathlineto{\pgfqpoint{3.460932in}{1.686701in}}%
\pgfpathlineto{\pgfqpoint{3.465161in}{1.693732in}}%
\pgfpathlineto{\pgfqpoint{3.469390in}{1.681415in}}%
\pgfpathlineto{\pgfqpoint{3.471505in}{1.687580in}}%
\pgfpathlineto{\pgfqpoint{3.473620in}{1.682272in}}%
\pgfpathlineto{\pgfqpoint{3.482079in}{1.701103in}}%
\pgfpathlineto{\pgfqpoint{3.484193in}{1.699760in}}%
\pgfpathlineto{\pgfqpoint{3.488423in}{1.700787in}}%
\pgfpathlineto{\pgfqpoint{3.494767in}{1.713185in}}%
\pgfpathlineto{\pgfqpoint{3.496881in}{1.721780in}}%
\pgfpathlineto{\pgfqpoint{3.501111in}{1.719798in}}%
\pgfpathlineto{\pgfqpoint{3.503225in}{1.716912in}}%
\pgfpathlineto{\pgfqpoint{3.505340in}{1.717142in}}%
\pgfpathlineto{\pgfqpoint{3.507455in}{1.712731in}}%
\pgfpathlineto{\pgfqpoint{3.509570in}{1.713514in}}%
\pgfpathlineto{\pgfqpoint{3.515914in}{1.708667in}}%
\pgfpathlineto{\pgfqpoint{3.518028in}{1.711309in}}%
\pgfpathlineto{\pgfqpoint{3.520143in}{1.709940in}}%
\pgfpathlineto{\pgfqpoint{3.522258in}{1.703910in}}%
\pgfpathlineto{\pgfqpoint{3.524372in}{1.704223in}}%
\pgfpathlineto{\pgfqpoint{3.526487in}{1.707357in}}%
\pgfpathlineto{\pgfqpoint{3.530716in}{1.706221in}}%
\pgfpathlineto{\pgfqpoint{3.532831in}{1.709273in}}%
\pgfpathlineto{\pgfqpoint{3.534946in}{1.708928in}}%
\pgfpathlineto{\pgfqpoint{3.537061in}{1.705412in}}%
\pgfpathlineto{\pgfqpoint{3.539175in}{1.704388in}}%
\pgfpathlineto{\pgfqpoint{3.545519in}{1.714159in}}%
\pgfpathlineto{\pgfqpoint{3.547634in}{1.715406in}}%
\pgfpathlineto{\pgfqpoint{3.549749in}{1.715293in}}%
\pgfpathlineto{\pgfqpoint{3.553978in}{1.720346in}}%
\pgfpathlineto{\pgfqpoint{3.556093in}{1.718708in}}%
\pgfpathlineto{\pgfqpoint{3.558208in}{1.718737in}}%
\pgfpathlineto{\pgfqpoint{3.562437in}{1.708877in}}%
\pgfpathlineto{\pgfqpoint{3.566666in}{1.702743in}}%
\pgfpathlineto{\pgfqpoint{3.568781in}{1.706045in}}%
\pgfpathlineto{\pgfqpoint{3.570896in}{1.700944in}}%
\pgfpathlineto{\pgfqpoint{3.573010in}{1.705289in}}%
\pgfpathlineto{\pgfqpoint{3.577240in}{1.718835in}}%
\pgfpathlineto{\pgfqpoint{3.579354in}{1.714301in}}%
\pgfpathlineto{\pgfqpoint{3.581469in}{1.719483in}}%
\pgfpathlineto{\pgfqpoint{3.583584in}{1.716725in}}%
\pgfpathlineto{\pgfqpoint{3.585699in}{1.717357in}}%
\pgfpathlineto{\pgfqpoint{3.587813in}{1.715917in}}%
\pgfpathlineto{\pgfqpoint{3.592043in}{1.723033in}}%
\pgfpathlineto{\pgfqpoint{3.594157in}{1.721399in}}%
\pgfpathlineto{\pgfqpoint{3.598387in}{1.720833in}}%
\pgfpathlineto{\pgfqpoint{3.602616in}{1.723037in}}%
\pgfpathlineto{\pgfqpoint{3.604731in}{1.727062in}}%
\pgfpathlineto{\pgfqpoint{3.606845in}{1.727226in}}%
\pgfpathlineto{\pgfqpoint{3.608960in}{1.731313in}}%
\pgfpathlineto{\pgfqpoint{3.611075in}{1.725573in}}%
\pgfpathlineto{\pgfqpoint{3.613190in}{1.725783in}}%
\pgfpathlineto{\pgfqpoint{3.615304in}{1.727909in}}%
\pgfpathlineto{\pgfqpoint{3.619534in}{1.727104in}}%
\pgfpathlineto{\pgfqpoint{3.621648in}{1.729158in}}%
\pgfpathlineto{\pgfqpoint{3.623763in}{1.726752in}}%
\pgfpathlineto{\pgfqpoint{3.625878in}{1.729400in}}%
\pgfpathlineto{\pgfqpoint{3.627992in}{1.724183in}}%
\pgfpathlineto{\pgfqpoint{3.630107in}{1.729620in}}%
\pgfpathlineto{\pgfqpoint{3.632222in}{1.728153in}}%
\pgfpathlineto{\pgfqpoint{3.634336in}{1.724539in}}%
\pgfpathlineto{\pgfqpoint{3.636451in}{1.724916in}}%
\pgfpathlineto{\pgfqpoint{3.638566in}{1.726917in}}%
\pgfpathlineto{\pgfqpoint{3.640681in}{1.726459in}}%
\pgfpathlineto{\pgfqpoint{3.642795in}{1.727364in}}%
\pgfpathlineto{\pgfqpoint{3.644910in}{1.726153in}}%
\pgfpathlineto{\pgfqpoint{3.647025in}{1.723042in}}%
\pgfpathlineto{\pgfqpoint{3.649139in}{1.725535in}}%
\pgfpathlineto{\pgfqpoint{3.651254in}{1.721207in}}%
\pgfpathlineto{\pgfqpoint{3.653369in}{1.723165in}}%
\pgfpathlineto{\pgfqpoint{3.655483in}{1.716567in}}%
\pgfpathlineto{\pgfqpoint{3.657598in}{1.718949in}}%
\pgfpathlineto{\pgfqpoint{3.661828in}{1.714481in}}%
\pgfpathlineto{\pgfqpoint{3.666057in}{1.703955in}}%
\pgfpathlineto{\pgfqpoint{3.668172in}{1.699855in}}%
\pgfpathlineto{\pgfqpoint{3.672401in}{1.698514in}}%
\pgfpathlineto{\pgfqpoint{3.676630in}{1.699110in}}%
\pgfpathlineto{\pgfqpoint{3.680860in}{1.702050in}}%
\pgfpathlineto{\pgfqpoint{3.682974in}{1.698247in}}%
\pgfpathlineto{\pgfqpoint{3.685089in}{1.706804in}}%
\pgfpathlineto{\pgfqpoint{3.689319in}{1.705500in}}%
\pgfpathlineto{\pgfqpoint{3.691433in}{1.709625in}}%
\pgfpathlineto{\pgfqpoint{3.693548in}{1.707032in}}%
\pgfpathlineto{\pgfqpoint{3.695663in}{1.715567in}}%
\pgfpathlineto{\pgfqpoint{3.697777in}{1.717125in}}%
\pgfpathlineto{\pgfqpoint{3.702007in}{1.712454in}}%
\pgfpathlineto{\pgfqpoint{3.706236in}{1.716097in}}%
\pgfpathlineto{\pgfqpoint{3.710465in}{1.723093in}}%
\pgfpathlineto{\pgfqpoint{3.714695in}{1.719606in}}%
\pgfpathlineto{\pgfqpoint{3.716810in}{1.726897in}}%
\pgfpathlineto{\pgfqpoint{3.718924in}{1.726291in}}%
\pgfpathlineto{\pgfqpoint{3.721039in}{1.730700in}}%
\pgfpathlineto{\pgfqpoint{3.723154in}{1.727261in}}%
\pgfpathlineto{\pgfqpoint{3.725268in}{1.727710in}}%
\pgfpathlineto{\pgfqpoint{3.729498in}{1.739261in}}%
\pgfpathlineto{\pgfqpoint{3.733727in}{1.733734in}}%
\pgfpathlineto{\pgfqpoint{3.735842in}{1.737443in}}%
\pgfpathlineto{\pgfqpoint{3.737956in}{1.737454in}}%
\pgfpathlineto{\pgfqpoint{3.742186in}{1.735519in}}%
\pgfpathlineto{\pgfqpoint{3.744301in}{1.736636in}}%
\pgfpathlineto{\pgfqpoint{3.746415in}{1.735628in}}%
\pgfpathlineto{\pgfqpoint{3.748530in}{1.741574in}}%
\pgfpathlineto{\pgfqpoint{3.750645in}{1.741667in}}%
\pgfpathlineto{\pgfqpoint{3.754874in}{1.751292in}}%
\pgfpathlineto{\pgfqpoint{3.756989in}{1.752577in}}%
\pgfpathlineto{\pgfqpoint{3.759103in}{1.747347in}}%
\pgfpathlineto{\pgfqpoint{3.761218in}{1.746818in}}%
\pgfpathlineto{\pgfqpoint{3.765447in}{1.740865in}}%
\pgfpathlineto{\pgfqpoint{3.767562in}{1.738367in}}%
\pgfpathlineto{\pgfqpoint{3.769677in}{1.748422in}}%
\pgfpathlineto{\pgfqpoint{3.776021in}{1.745620in}}%
\pgfpathlineto{\pgfqpoint{3.778136in}{1.753008in}}%
\pgfpathlineto{\pgfqpoint{3.782365in}{1.743952in}}%
\pgfpathlineto{\pgfqpoint{3.784480in}{1.740123in}}%
\pgfpathlineto{\pgfqpoint{3.786594in}{1.739276in}}%
\pgfpathlineto{\pgfqpoint{3.792939in}{1.750687in}}%
\pgfpathlineto{\pgfqpoint{3.795053in}{1.751047in}}%
\pgfpathlineto{\pgfqpoint{3.797168in}{1.758087in}}%
\pgfpathlineto{\pgfqpoint{3.799283in}{1.757269in}}%
\pgfpathlineto{\pgfqpoint{3.801397in}{1.751773in}}%
\pgfpathlineto{\pgfqpoint{3.803512in}{1.752674in}}%
\pgfpathlineto{\pgfqpoint{3.805627in}{1.747233in}}%
\pgfpathlineto{\pgfqpoint{3.807741in}{1.747428in}}%
\pgfpathlineto{\pgfqpoint{3.809856in}{1.743332in}}%
\pgfpathlineto{\pgfqpoint{3.811971in}{1.734330in}}%
\pgfpathlineto{\pgfqpoint{3.816200in}{1.743616in}}%
\pgfpathlineto{\pgfqpoint{3.818315in}{1.744503in}}%
\pgfpathlineto{\pgfqpoint{3.820430in}{1.747730in}}%
\pgfpathlineto{\pgfqpoint{3.822544in}{1.748305in}}%
\pgfpathlineto{\pgfqpoint{3.826774in}{1.761305in}}%
\pgfpathlineto{\pgfqpoint{3.828888in}{1.764712in}}%
\pgfpathlineto{\pgfqpoint{3.831003in}{1.764377in}}%
\pgfpathlineto{\pgfqpoint{3.833118in}{1.760630in}}%
\pgfpathlineto{\pgfqpoint{3.835232in}{1.764138in}}%
\pgfpathlineto{\pgfqpoint{3.839462in}{1.775969in}}%
\pgfpathlineto{\pgfqpoint{3.841576in}{1.779358in}}%
\pgfpathlineto{\pgfqpoint{3.845806in}{1.768741in}}%
\pgfpathlineto{\pgfqpoint{3.847921in}{1.768849in}}%
\pgfpathlineto{\pgfqpoint{3.852150in}{1.787142in}}%
\pgfpathlineto{\pgfqpoint{3.854265in}{1.786922in}}%
\pgfpathlineto{\pgfqpoint{3.858494in}{1.791260in}}%
\pgfpathlineto{\pgfqpoint{3.860609in}{1.803207in}}%
\pgfpathlineto{\pgfqpoint{3.862723in}{1.802747in}}%
\pgfpathlineto{\pgfqpoint{3.869067in}{1.809175in}}%
\pgfpathlineto{\pgfqpoint{3.871182in}{1.809577in}}%
\pgfpathlineto{\pgfqpoint{3.873297in}{1.813344in}}%
\pgfpathlineto{\pgfqpoint{3.875412in}{1.809466in}}%
\pgfpathlineto{\pgfqpoint{3.877526in}{1.810036in}}%
\pgfpathlineto{\pgfqpoint{3.879641in}{1.807355in}}%
\pgfpathlineto{\pgfqpoint{3.881756in}{1.808558in}}%
\pgfpathlineto{\pgfqpoint{3.883870in}{1.804111in}}%
\pgfpathlineto{\pgfqpoint{3.885985in}{1.802883in}}%
\pgfpathlineto{\pgfqpoint{3.888100in}{1.805758in}}%
\pgfpathlineto{\pgfqpoint{3.890214in}{1.802955in}}%
\pgfpathlineto{\pgfqpoint{3.892329in}{1.797110in}}%
\pgfpathlineto{\pgfqpoint{3.894444in}{1.797910in}}%
\pgfpathlineto{\pgfqpoint{3.900788in}{1.786888in}}%
\pgfpathlineto{\pgfqpoint{3.902903in}{1.784733in}}%
\pgfpathlineto{\pgfqpoint{3.907132in}{1.775265in}}%
\pgfpathlineto{\pgfqpoint{3.911361in}{1.784470in}}%
\pgfpathlineto{\pgfqpoint{3.913476in}{1.783255in}}%
\pgfpathlineto{\pgfqpoint{3.919820in}{1.786472in}}%
\pgfpathlineto{\pgfqpoint{3.921935in}{1.787422in}}%
\pgfpathlineto{\pgfqpoint{3.924050in}{1.791257in}}%
\pgfpathlineto{\pgfqpoint{3.928279in}{1.803634in}}%
\pgfpathlineto{\pgfqpoint{3.930394in}{1.804591in}}%
\pgfpathlineto{\pgfqpoint{3.932508in}{1.798506in}}%
\pgfpathlineto{\pgfqpoint{3.938852in}{1.802940in}}%
\pgfpathlineto{\pgfqpoint{3.945196in}{1.807361in}}%
\pgfpathlineto{\pgfqpoint{3.949426in}{1.797946in}}%
\pgfpathlineto{\pgfqpoint{3.953655in}{1.805108in}}%
\pgfpathlineto{\pgfqpoint{3.955770in}{1.802835in}}%
\pgfpathlineto{\pgfqpoint{3.957885in}{1.803073in}}%
\pgfpathlineto{\pgfqpoint{3.959999in}{1.808282in}}%
\pgfpathlineto{\pgfqpoint{3.962114in}{1.802946in}}%
\pgfpathlineto{\pgfqpoint{3.966343in}{1.801430in}}%
\pgfpathlineto{\pgfqpoint{3.968458in}{1.808247in}}%
\pgfpathlineto{\pgfqpoint{3.970573in}{1.802999in}}%
\pgfpathlineto{\pgfqpoint{3.974802in}{1.807605in}}%
\pgfpathlineto{\pgfqpoint{3.976917in}{1.803543in}}%
\pgfpathlineto{\pgfqpoint{3.979032in}{1.803446in}}%
\pgfpathlineto{\pgfqpoint{3.983261in}{1.811797in}}%
\pgfpathlineto{\pgfqpoint{3.985376in}{1.822668in}}%
\pgfpathlineto{\pgfqpoint{3.987490in}{1.818281in}}%
\pgfpathlineto{\pgfqpoint{3.989605in}{1.817839in}}%
\pgfpathlineto{\pgfqpoint{3.991720in}{1.821968in}}%
\pgfpathlineto{\pgfqpoint{3.998064in}{1.822395in}}%
\pgfpathlineto{\pgfqpoint{4.000178in}{1.818578in}}%
\pgfpathlineto{\pgfqpoint{4.008637in}{1.795741in}}%
\pgfpathlineto{\pgfqpoint{4.010752in}{1.797426in}}%
\pgfpathlineto{\pgfqpoint{4.012867in}{1.795642in}}%
\pgfpathlineto{\pgfqpoint{4.014981in}{1.795547in}}%
\pgfpathlineto{\pgfqpoint{4.017096in}{1.801425in}}%
\pgfpathlineto{\pgfqpoint{4.019211in}{1.799180in}}%
\pgfpathlineto{\pgfqpoint{4.023440in}{1.804483in}}%
\pgfpathlineto{\pgfqpoint{4.025555in}{1.800608in}}%
\pgfpathlineto{\pgfqpoint{4.027670in}{1.801625in}}%
\pgfpathlineto{\pgfqpoint{4.034014in}{1.799785in}}%
\pgfpathlineto{\pgfqpoint{4.036128in}{1.800232in}}%
\pgfpathlineto{\pgfqpoint{4.040358in}{1.797455in}}%
\pgfpathlineto{\pgfqpoint{4.042472in}{1.805367in}}%
\pgfpathlineto{\pgfqpoint{4.044587in}{1.801981in}}%
\pgfpathlineto{\pgfqpoint{4.046702in}{1.805381in}}%
\pgfpathlineto{\pgfqpoint{4.048816in}{1.802802in}}%
\pgfpathlineto{\pgfqpoint{4.050931in}{1.806923in}}%
\pgfpathlineto{\pgfqpoint{4.053046in}{1.807159in}}%
\pgfpathlineto{\pgfqpoint{4.055161in}{1.809420in}}%
\pgfpathlineto{\pgfqpoint{4.057275in}{1.801027in}}%
\pgfpathlineto{\pgfqpoint{4.059390in}{1.799715in}}%
\pgfpathlineto{\pgfqpoint{4.063619in}{1.801864in}}%
\pgfpathlineto{\pgfqpoint{4.065734in}{1.799946in}}%
\pgfpathlineto{\pgfqpoint{4.069963in}{1.808356in}}%
\pgfpathlineto{\pgfqpoint{4.072078in}{1.815552in}}%
\pgfpathlineto{\pgfqpoint{4.074193in}{1.818661in}}%
\pgfpathlineto{\pgfqpoint{4.076307in}{1.813148in}}%
\pgfpathlineto{\pgfqpoint{4.078422in}{1.812879in}}%
\pgfpathlineto{\pgfqpoint{4.080537in}{1.823595in}}%
\pgfpathlineto{\pgfqpoint{4.084766in}{1.829724in}}%
\pgfpathlineto{\pgfqpoint{4.086881in}{1.828985in}}%
\pgfpathlineto{\pgfqpoint{4.088996in}{1.829660in}}%
\pgfpathlineto{\pgfqpoint{4.091110in}{1.831743in}}%
\pgfpathlineto{\pgfqpoint{4.093225in}{1.831470in}}%
\pgfpathlineto{\pgfqpoint{4.095340in}{1.835288in}}%
\pgfpathlineto{\pgfqpoint{4.099569in}{1.829171in}}%
\pgfpathlineto{\pgfqpoint{4.103798in}{1.834408in}}%
\pgfpathlineto{\pgfqpoint{4.108028in}{1.842645in}}%
\pgfpathlineto{\pgfqpoint{4.114372in}{1.838902in}}%
\pgfpathlineto{\pgfqpoint{4.116487in}{1.833861in}}%
\pgfpathlineto{\pgfqpoint{4.120716in}{1.841217in}}%
\pgfpathlineto{\pgfqpoint{4.124945in}{1.836248in}}%
\pgfpathlineto{\pgfqpoint{4.127060in}{1.833705in}}%
\pgfpathlineto{\pgfqpoint{4.129175in}{1.842704in}}%
\pgfpathlineto{\pgfqpoint{4.131290in}{1.839752in}}%
\pgfpathlineto{\pgfqpoint{4.133404in}{1.840675in}}%
\pgfpathlineto{\pgfqpoint{4.135519in}{1.836386in}}%
\pgfpathlineto{\pgfqpoint{4.137634in}{1.835454in}}%
\pgfpathlineto{\pgfqpoint{4.141863in}{1.836842in}}%
\pgfpathlineto{\pgfqpoint{4.143978in}{1.832123in}}%
\pgfpathlineto{\pgfqpoint{4.148207in}{1.820407in}}%
\pgfpathlineto{\pgfqpoint{4.150322in}{1.822234in}}%
\pgfpathlineto{\pgfqpoint{4.152436in}{1.830404in}}%
\pgfpathlineto{\pgfqpoint{4.156666in}{1.835521in}}%
\pgfpathlineto{\pgfqpoint{4.160895in}{1.829427in}}%
\pgfpathlineto{\pgfqpoint{4.163010in}{1.833479in}}%
\pgfpathlineto{\pgfqpoint{4.165125in}{1.834088in}}%
\pgfpathlineto{\pgfqpoint{4.167239in}{1.838414in}}%
\pgfpathlineto{\pgfqpoint{4.169354in}{1.839375in}}%
\pgfpathlineto{\pgfqpoint{4.171469in}{1.842840in}}%
\pgfpathlineto{\pgfqpoint{4.182042in}{1.834551in}}%
\pgfpathlineto{\pgfqpoint{4.184157in}{1.831542in}}%
\pgfpathlineto{\pgfqpoint{4.188386in}{1.831162in}}%
\pgfpathlineto{\pgfqpoint{4.192616in}{1.836041in}}%
\pgfpathlineto{\pgfqpoint{4.196845in}{1.834160in}}%
\pgfpathlineto{\pgfqpoint{4.198960in}{1.840229in}}%
\pgfpathlineto{\pgfqpoint{4.201074in}{1.840911in}}%
\pgfpathlineto{\pgfqpoint{4.203189in}{1.838945in}}%
\pgfpathlineto{\pgfqpoint{4.207418in}{1.845528in}}%
\pgfpathlineto{\pgfqpoint{4.209533in}{1.845987in}}%
\pgfpathlineto{\pgfqpoint{4.211648in}{1.851415in}}%
\pgfpathlineto{\pgfqpoint{4.213763in}{1.850439in}}%
\pgfpathlineto{\pgfqpoint{4.217992in}{1.860342in}}%
\pgfpathlineto{\pgfqpoint{4.222221in}{1.853419in}}%
\pgfpathlineto{\pgfqpoint{4.226451in}{1.856981in}}%
\pgfpathlineto{\pgfqpoint{4.228565in}{1.851705in}}%
\pgfpathlineto{\pgfqpoint{4.232795in}{1.852494in}}%
\pgfpathlineto{\pgfqpoint{4.234910in}{1.848386in}}%
\pgfpathlineto{\pgfqpoint{4.237024in}{1.853652in}}%
\pgfpathlineto{\pgfqpoint{4.241254in}{1.856530in}}%
\pgfpathlineto{\pgfqpoint{4.243368in}{1.855949in}}%
\pgfpathlineto{\pgfqpoint{4.245483in}{1.852879in}}%
\pgfpathlineto{\pgfqpoint{4.247598in}{1.862399in}}%
\pgfpathlineto{\pgfqpoint{4.249712in}{1.863052in}}%
\pgfpathlineto{\pgfqpoint{4.253942in}{1.857218in}}%
\pgfpathlineto{\pgfqpoint{4.256056in}{1.850435in}}%
\pgfpathlineto{\pgfqpoint{4.258171in}{1.849476in}}%
\pgfpathlineto{\pgfqpoint{4.260286in}{1.855389in}}%
\pgfpathlineto{\pgfqpoint{4.262401in}{1.856226in}}%
\pgfpathlineto{\pgfqpoint{4.264515in}{1.859530in}}%
\pgfpathlineto{\pgfqpoint{4.266630in}{1.858671in}}%
\pgfpathlineto{\pgfqpoint{4.268745in}{1.852425in}}%
\pgfpathlineto{\pgfqpoint{4.270859in}{1.853403in}}%
\pgfpathlineto{\pgfqpoint{4.275089in}{1.852622in}}%
\pgfpathlineto{\pgfqpoint{4.277203in}{1.855082in}}%
\pgfpathlineto{\pgfqpoint{4.279318in}{1.859832in}}%
\pgfpathlineto{\pgfqpoint{4.281433in}{1.850437in}}%
\pgfpathlineto{\pgfqpoint{4.283547in}{1.846943in}}%
\pgfpathlineto{\pgfqpoint{4.285662in}{1.848015in}}%
\pgfpathlineto{\pgfqpoint{4.287777in}{1.852885in}}%
\pgfpathlineto{\pgfqpoint{4.289892in}{1.849224in}}%
\pgfpathlineto{\pgfqpoint{4.292006in}{1.851366in}}%
\pgfpathlineto{\pgfqpoint{4.300465in}{1.840992in}}%
\pgfpathlineto{\pgfqpoint{4.302580in}{1.844070in}}%
\pgfpathlineto{\pgfqpoint{4.304694in}{1.842161in}}%
\pgfpathlineto{\pgfqpoint{4.306809in}{1.847907in}}%
\pgfpathlineto{\pgfqpoint{4.308924in}{1.843170in}}%
\pgfpathlineto{\pgfqpoint{4.311038in}{1.848066in}}%
\pgfpathlineto{\pgfqpoint{4.315268in}{1.841189in}}%
\pgfpathlineto{\pgfqpoint{4.317383in}{1.837947in}}%
\pgfpathlineto{\pgfqpoint{4.319497in}{1.840383in}}%
\pgfpathlineto{\pgfqpoint{4.321612in}{1.845508in}}%
\pgfpathlineto{\pgfqpoint{4.323727in}{1.837822in}}%
\pgfpathlineto{\pgfqpoint{4.325841in}{1.842489in}}%
\pgfpathlineto{\pgfqpoint{4.330071in}{1.834877in}}%
\pgfpathlineto{\pgfqpoint{4.334300in}{1.842765in}}%
\pgfpathlineto{\pgfqpoint{4.336415in}{1.839420in}}%
\pgfpathlineto{\pgfqpoint{4.338529in}{1.840484in}}%
\pgfpathlineto{\pgfqpoint{4.340644in}{1.835261in}}%
\pgfpathlineto{\pgfqpoint{4.342759in}{1.840783in}}%
\pgfpathlineto{\pgfqpoint{4.344874in}{1.841420in}}%
\pgfpathlineto{\pgfqpoint{4.346988in}{1.843907in}}%
\pgfpathlineto{\pgfqpoint{4.349103in}{1.844395in}}%
\pgfpathlineto{\pgfqpoint{4.353332in}{1.838557in}}%
\pgfpathlineto{\pgfqpoint{4.355447in}{1.837882in}}%
\pgfpathlineto{\pgfqpoint{4.357562in}{1.834766in}}%
\pgfpathlineto{\pgfqpoint{4.361791in}{1.840839in}}%
\pgfpathlineto{\pgfqpoint{4.363906in}{1.838383in}}%
\pgfpathlineto{\pgfqpoint{4.366021in}{1.837914in}}%
\pgfpathlineto{\pgfqpoint{4.370250in}{1.823136in}}%
\pgfpathlineto{\pgfqpoint{4.376594in}{1.834813in}}%
\pgfpathlineto{\pgfqpoint{4.378709in}{1.830958in}}%
\pgfpathlineto{\pgfqpoint{4.380823in}{1.830579in}}%
\pgfpathlineto{\pgfqpoint{4.382938in}{1.838855in}}%
\pgfpathlineto{\pgfqpoint{4.385053in}{1.837090in}}%
\pgfpathlineto{\pgfqpoint{4.387167in}{1.843586in}}%
\pgfpathlineto{\pgfqpoint{4.389282in}{1.836665in}}%
\pgfpathlineto{\pgfqpoint{4.391397in}{1.838844in}}%
\pgfpathlineto{\pgfqpoint{4.399856in}{1.821442in}}%
\pgfpathlineto{\pgfqpoint{4.401970in}{1.822763in}}%
\pgfpathlineto{\pgfqpoint{4.404085in}{1.819312in}}%
\pgfpathlineto{\pgfqpoint{4.406200in}{1.824589in}}%
\pgfpathlineto{\pgfqpoint{4.408314in}{1.823440in}}%
\pgfpathlineto{\pgfqpoint{4.410429in}{1.824968in}}%
\pgfpathlineto{\pgfqpoint{4.412544in}{1.819606in}}%
\pgfpathlineto{\pgfqpoint{4.414658in}{1.817544in}}%
\pgfpathlineto{\pgfqpoint{4.416773in}{1.818935in}}%
\pgfpathlineto{\pgfqpoint{4.418888in}{1.812069in}}%
\pgfpathlineto{\pgfqpoint{4.425232in}{1.816888in}}%
\pgfpathlineto{\pgfqpoint{4.427347in}{1.811695in}}%
\pgfpathlineto{\pgfqpoint{4.429461in}{1.812371in}}%
\pgfpathlineto{\pgfqpoint{4.435805in}{1.821265in}}%
\pgfpathlineto{\pgfqpoint{4.442149in}{1.831172in}}%
\pgfpathlineto{\pgfqpoint{4.444264in}{1.835836in}}%
\pgfpathlineto{\pgfqpoint{4.446379in}{1.836080in}}%
\pgfpathlineto{\pgfqpoint{4.450608in}{1.838246in}}%
\pgfpathlineto{\pgfqpoint{4.452723in}{1.837879in}}%
\pgfpathlineto{\pgfqpoint{4.454838in}{1.842083in}}%
\pgfpathlineto{\pgfqpoint{4.456952in}{1.834576in}}%
\pgfpathlineto{\pgfqpoint{4.461182in}{1.837939in}}%
\pgfpathlineto{\pgfqpoint{4.463296in}{1.833151in}}%
\pgfpathlineto{\pgfqpoint{4.465411in}{1.831411in}}%
\pgfpathlineto{\pgfqpoint{4.467526in}{1.837935in}}%
\pgfpathlineto{\pgfqpoint{4.469641in}{1.837356in}}%
\pgfpathlineto{\pgfqpoint{4.471755in}{1.839121in}}%
\pgfpathlineto{\pgfqpoint{4.473870in}{1.836006in}}%
\pgfpathlineto{\pgfqpoint{4.475985in}{1.843805in}}%
\pgfpathlineto{\pgfqpoint{4.478099in}{1.844442in}}%
\pgfpathlineto{\pgfqpoint{4.480214in}{1.836828in}}%
\pgfpathlineto{\pgfqpoint{4.482329in}{1.839002in}}%
\pgfpathlineto{\pgfqpoint{4.484443in}{1.844742in}}%
\pgfpathlineto{\pgfqpoint{4.488673in}{1.840525in}}%
\pgfpathlineto{\pgfqpoint{4.492902in}{1.848528in}}%
\pgfpathlineto{\pgfqpoint{4.495017in}{1.859581in}}%
\pgfpathlineto{\pgfqpoint{4.497132in}{1.851771in}}%
\pgfpathlineto{\pgfqpoint{4.499246in}{1.857608in}}%
\pgfpathlineto{\pgfqpoint{4.501361in}{1.858061in}}%
\pgfpathlineto{\pgfqpoint{4.503476in}{1.856005in}}%
\pgfpathlineto{\pgfqpoint{4.505590in}{1.863223in}}%
\pgfpathlineto{\pgfqpoint{4.507705in}{1.864694in}}%
\pgfpathlineto{\pgfqpoint{4.509820in}{1.874232in}}%
\pgfpathlineto{\pgfqpoint{4.511934in}{1.866693in}}%
\pgfpathlineto{\pgfqpoint{4.516164in}{1.877626in}}%
\pgfpathlineto{\pgfqpoint{4.518278in}{1.878024in}}%
\pgfpathlineto{\pgfqpoint{4.520393in}{1.872730in}}%
\pgfpathlineto{\pgfqpoint{4.522508in}{1.877497in}}%
\pgfpathlineto{\pgfqpoint{4.526737in}{1.880571in}}%
\pgfpathlineto{\pgfqpoint{4.528852in}{1.884949in}}%
\pgfpathlineto{\pgfqpoint{4.530967in}{1.880872in}}%
\pgfpathlineto{\pgfqpoint{4.533081in}{1.873844in}}%
\pgfpathlineto{\pgfqpoint{4.535196in}{1.876093in}}%
\pgfpathlineto{\pgfqpoint{4.537311in}{1.882674in}}%
\pgfpathlineto{\pgfqpoint{4.539425in}{1.880327in}}%
\pgfpathlineto{\pgfqpoint{4.543655in}{1.878325in}}%
\pgfpathlineto{\pgfqpoint{4.545769in}{1.880566in}}%
\pgfpathlineto{\pgfqpoint{4.547884in}{1.875164in}}%
\pgfpathlineto{\pgfqpoint{4.549999in}{1.877426in}}%
\pgfpathlineto{\pgfqpoint{4.552114in}{1.875585in}}%
\pgfpathlineto{\pgfqpoint{4.556343in}{1.867629in}}%
\pgfpathlineto{\pgfqpoint{4.558458in}{1.868731in}}%
\pgfpathlineto{\pgfqpoint{4.562687in}{1.865738in}}%
\pgfpathlineto{\pgfqpoint{4.564802in}{1.865115in}}%
\pgfpathlineto{\pgfqpoint{4.566916in}{1.862393in}}%
\pgfpathlineto{\pgfqpoint{4.569031in}{1.863436in}}%
\pgfpathlineto{\pgfqpoint{4.571146in}{1.862595in}}%
\pgfpathlineto{\pgfqpoint{4.573260in}{1.864305in}}%
\pgfpathlineto{\pgfqpoint{4.575375in}{1.859076in}}%
\pgfpathlineto{\pgfqpoint{4.577490in}{1.861382in}}%
\pgfpathlineto{\pgfqpoint{4.579605in}{1.859782in}}%
\pgfpathlineto{\pgfqpoint{4.581719in}{1.862219in}}%
\pgfpathlineto{\pgfqpoint{4.583834in}{1.861452in}}%
\pgfpathlineto{\pgfqpoint{4.585949in}{1.866984in}}%
\pgfpathlineto{\pgfqpoint{4.588063in}{1.867982in}}%
\pgfpathlineto{\pgfqpoint{4.590178in}{1.870895in}}%
\pgfpathlineto{\pgfqpoint{4.592293in}{1.866505in}}%
\pgfpathlineto{\pgfqpoint{4.594407in}{1.866633in}}%
\pgfpathlineto{\pgfqpoint{4.598637in}{1.859410in}}%
\pgfpathlineto{\pgfqpoint{4.600752in}{1.862579in}}%
\pgfpathlineto{\pgfqpoint{4.602866in}{1.861625in}}%
\pgfpathlineto{\pgfqpoint{4.604981in}{1.872645in}}%
\pgfpathlineto{\pgfqpoint{4.609210in}{1.880022in}}%
\pgfpathlineto{\pgfqpoint{4.611325in}{1.880453in}}%
\pgfpathlineto{\pgfqpoint{4.615554in}{1.898090in}}%
\pgfpathlineto{\pgfqpoint{4.617669in}{1.899112in}}%
\pgfpathlineto{\pgfqpoint{4.619784in}{1.904040in}}%
\pgfpathlineto{\pgfqpoint{4.621898in}{1.905418in}}%
\pgfpathlineto{\pgfqpoint{4.624013in}{1.903722in}}%
\pgfpathlineto{\pgfqpoint{4.628243in}{1.896829in}}%
\pgfpathlineto{\pgfqpoint{4.630357in}{1.900070in}}%
\pgfpathlineto{\pgfqpoint{4.634587in}{1.898274in}}%
\pgfpathlineto{\pgfqpoint{4.636701in}{1.891673in}}%
\pgfpathlineto{\pgfqpoint{4.638816in}{1.895733in}}%
\pgfpathlineto{\pgfqpoint{4.640931in}{1.890603in}}%
\pgfpathlineto{\pgfqpoint{4.645160in}{1.901230in}}%
\pgfpathlineto{\pgfqpoint{4.647275in}{1.902475in}}%
\pgfpathlineto{\pgfqpoint{4.649389in}{1.899683in}}%
\pgfpathlineto{\pgfqpoint{4.651504in}{1.888768in}}%
\pgfpathlineto{\pgfqpoint{4.653619in}{1.891067in}}%
\pgfpathlineto{\pgfqpoint{4.664192in}{1.863335in}}%
\pgfpathlineto{\pgfqpoint{4.666307in}{1.867882in}}%
\pgfpathlineto{\pgfqpoint{4.668422in}{1.867331in}}%
\pgfpathlineto{\pgfqpoint{4.672651in}{1.874037in}}%
\pgfpathlineto{\pgfqpoint{4.674766in}{1.873736in}}%
\pgfpathlineto{\pgfqpoint{4.678995in}{1.877085in}}%
\pgfpathlineto{\pgfqpoint{4.685339in}{1.886025in}}%
\pgfpathlineto{\pgfqpoint{4.687454in}{1.892106in}}%
\pgfpathlineto{\pgfqpoint{4.691683in}{1.883403in}}%
\pgfpathlineto{\pgfqpoint{4.693798in}{1.884700in}}%
\pgfpathlineto{\pgfqpoint{4.695913in}{1.881417in}}%
\pgfpathlineto{\pgfqpoint{4.698027in}{1.891150in}}%
\pgfpathlineto{\pgfqpoint{4.700142in}{1.894301in}}%
\pgfpathlineto{\pgfqpoint{4.702257in}{1.894721in}}%
\pgfpathlineto{\pgfqpoint{4.704372in}{1.899581in}}%
\pgfpathlineto{\pgfqpoint{4.708601in}{1.889896in}}%
\pgfpathlineto{\pgfqpoint{4.712830in}{1.895713in}}%
\pgfpathlineto{\pgfqpoint{4.714945in}{1.901736in}}%
\pgfpathlineto{\pgfqpoint{4.721289in}{1.908942in}}%
\pgfpathlineto{\pgfqpoint{4.723404in}{1.908483in}}%
\pgfpathlineto{\pgfqpoint{4.725518in}{1.915057in}}%
\pgfpathlineto{\pgfqpoint{4.727633in}{1.914139in}}%
\pgfpathlineto{\pgfqpoint{4.731863in}{1.901064in}}%
\pgfpathlineto{\pgfqpoint{4.733977in}{1.893961in}}%
\pgfpathlineto{\pgfqpoint{4.736092in}{1.895638in}}%
\pgfpathlineto{\pgfqpoint{4.738207in}{1.894512in}}%
\pgfpathlineto{\pgfqpoint{4.744551in}{1.902347in}}%
\pgfpathlineto{\pgfqpoint{4.746665in}{1.901898in}}%
\pgfpathlineto{\pgfqpoint{4.748780in}{1.905689in}}%
\pgfpathlineto{\pgfqpoint{4.750895in}{1.905186in}}%
\pgfpathlineto{\pgfqpoint{4.753009in}{1.906743in}}%
\pgfpathlineto{\pgfqpoint{4.757239in}{1.899518in}}%
\pgfpathlineto{\pgfqpoint{4.761468in}{1.903094in}}%
\pgfpathlineto{\pgfqpoint{4.763583in}{1.906554in}}%
\pgfpathlineto{\pgfqpoint{4.765698in}{1.906137in}}%
\pgfpathlineto{\pgfqpoint{4.767812in}{1.915856in}}%
\pgfpathlineto{\pgfqpoint{4.772042in}{1.914344in}}%
\pgfpathlineto{\pgfqpoint{4.774156in}{1.913758in}}%
\pgfpathlineto{\pgfqpoint{4.776271in}{1.923572in}}%
\pgfpathlineto{\pgfqpoint{4.778386in}{1.926595in}}%
\pgfpathlineto{\pgfqpoint{4.780500in}{1.921896in}}%
\pgfpathlineto{\pgfqpoint{4.782615in}{1.913562in}}%
\pgfpathlineto{\pgfqpoint{4.784730in}{1.922708in}}%
\pgfpathlineto{\pgfqpoint{4.786845in}{1.918910in}}%
\pgfpathlineto{\pgfqpoint{4.788959in}{1.926228in}}%
\pgfpathlineto{\pgfqpoint{4.793189in}{1.934346in}}%
\pgfpathlineto{\pgfqpoint{4.795303in}{1.933519in}}%
\pgfpathlineto{\pgfqpoint{4.797418in}{1.926674in}}%
\pgfpathlineto{\pgfqpoint{4.799533in}{1.929719in}}%
\pgfpathlineto{\pgfqpoint{4.801647in}{1.923088in}}%
\pgfpathlineto{\pgfqpoint{4.805877in}{1.931554in}}%
\pgfpathlineto{\pgfqpoint{4.810106in}{1.924254in}}%
\pgfpathlineto{\pgfqpoint{4.812221in}{1.916755in}}%
\pgfpathlineto{\pgfqpoint{4.814336in}{1.920155in}}%
\pgfpathlineto{\pgfqpoint{4.820680in}{1.906645in}}%
\pgfpathlineto{\pgfqpoint{4.822794in}{1.905351in}}%
\pgfpathlineto{\pgfqpoint{4.824909in}{1.908780in}}%
\pgfpathlineto{\pgfqpoint{4.827024in}{1.905288in}}%
\pgfpathlineto{\pgfqpoint{4.829138in}{1.896052in}}%
\pgfpathlineto{\pgfqpoint{4.831253in}{1.900958in}}%
\pgfpathlineto{\pgfqpoint{4.833368in}{1.901813in}}%
\pgfpathlineto{\pgfqpoint{4.837597in}{1.896310in}}%
\pgfpathlineto{\pgfqpoint{4.839712in}{1.896513in}}%
\pgfpathlineto{\pgfqpoint{4.841827in}{1.899862in}}%
\pgfpathlineto{\pgfqpoint{4.843941in}{1.896193in}}%
\pgfpathlineto{\pgfqpoint{4.846056in}{1.896649in}}%
\pgfpathlineto{\pgfqpoint{4.850285in}{1.892000in}}%
\pgfpathlineto{\pgfqpoint{4.852400in}{1.891324in}}%
\pgfpathlineto{\pgfqpoint{4.854515in}{1.894920in}}%
\pgfpathlineto{\pgfqpoint{4.856629in}{1.895281in}}%
\pgfpathlineto{\pgfqpoint{4.862974in}{1.901173in}}%
\pgfpathlineto{\pgfqpoint{4.865088in}{1.900031in}}%
\pgfpathlineto{\pgfqpoint{4.869318in}{1.888242in}}%
\pgfpathlineto{\pgfqpoint{4.871432in}{1.896257in}}%
\pgfpathlineto{\pgfqpoint{4.877776in}{1.900451in}}%
\pgfpathlineto{\pgfqpoint{4.882006in}{1.886497in}}%
\pgfpathlineto{\pgfqpoint{4.886235in}{1.869083in}}%
\pgfpathlineto{\pgfqpoint{4.888350in}{1.866107in}}%
\pgfpathlineto{\pgfqpoint{4.890465in}{1.870347in}}%
\pgfpathlineto{\pgfqpoint{4.894694in}{1.869963in}}%
\pgfpathlineto{\pgfqpoint{4.898923in}{1.866787in}}%
\pgfpathlineto{\pgfqpoint{4.901038in}{1.870756in}}%
\pgfpathlineto{\pgfqpoint{4.903153in}{1.866681in}}%
\pgfpathlineto{\pgfqpoint{4.909497in}{1.882833in}}%
\pgfpathlineto{\pgfqpoint{4.911611in}{1.883872in}}%
\pgfpathlineto{\pgfqpoint{4.915841in}{1.882323in}}%
\pgfpathlineto{\pgfqpoint{4.922185in}{1.856680in}}%
\pgfpathlineto{\pgfqpoint{4.930644in}{1.872725in}}%
\pgfpathlineto{\pgfqpoint{4.934873in}{1.869638in}}%
\pgfpathlineto{\pgfqpoint{4.936988in}{1.880196in}}%
\pgfpathlineto{\pgfqpoint{4.939103in}{1.867034in}}%
\pgfpathlineto{\pgfqpoint{4.941217in}{1.867083in}}%
\pgfpathlineto{\pgfqpoint{4.943332in}{1.871156in}}%
\pgfpathlineto{\pgfqpoint{4.945447in}{1.871038in}}%
\pgfpathlineto{\pgfqpoint{4.947561in}{1.875898in}}%
\pgfpathlineto{\pgfqpoint{4.949676in}{1.874383in}}%
\pgfpathlineto{\pgfqpoint{4.951791in}{1.871407in}}%
\pgfpathlineto{\pgfqpoint{4.953905in}{1.870883in}}%
\pgfpathlineto{\pgfqpoint{4.956020in}{1.873191in}}%
\pgfpathlineto{\pgfqpoint{4.958135in}{1.867793in}}%
\pgfpathlineto{\pgfqpoint{4.960249in}{1.871420in}}%
\pgfpathlineto{\pgfqpoint{4.962364in}{1.869191in}}%
\pgfpathlineto{\pgfqpoint{4.964479in}{1.859529in}}%
\pgfpathlineto{\pgfqpoint{4.966594in}{1.858915in}}%
\pgfpathlineto{\pgfqpoint{4.972938in}{1.844123in}}%
\pgfpathlineto{\pgfqpoint{4.975052in}{1.844644in}}%
\pgfpathlineto{\pgfqpoint{4.977167in}{1.847666in}}%
\pgfpathlineto{\pgfqpoint{4.981396in}{1.840500in}}%
\pgfpathlineto{\pgfqpoint{4.983511in}{1.849699in}}%
\pgfpathlineto{\pgfqpoint{4.985626in}{1.852047in}}%
\pgfpathlineto{\pgfqpoint{4.989855in}{1.844473in}}%
\pgfpathlineto{\pgfqpoint{4.994085in}{1.831111in}}%
\pgfpathlineto{\pgfqpoint{4.996199in}{1.837707in}}%
\pgfpathlineto{\pgfqpoint{4.998314in}{1.835388in}}%
\pgfpathlineto{\pgfqpoint{5.000429in}{1.840095in}}%
\pgfpathlineto{\pgfqpoint{5.002543in}{1.836709in}}%
\pgfpathlineto{\pgfqpoint{5.006773in}{1.825681in}}%
\pgfpathlineto{\pgfqpoint{5.011002in}{1.820577in}}%
\pgfpathlineto{\pgfqpoint{5.013117in}{1.824713in}}%
\pgfpathlineto{\pgfqpoint{5.015231in}{1.833380in}}%
\pgfpathlineto{\pgfqpoint{5.019461in}{1.839042in}}%
\pgfpathlineto{\pgfqpoint{5.021576in}{1.844462in}}%
\pgfpathlineto{\pgfqpoint{5.025805in}{1.836659in}}%
\pgfpathlineto{\pgfqpoint{5.027920in}{1.839129in}}%
\pgfpathlineto{\pgfqpoint{5.030034in}{1.834746in}}%
\pgfpathlineto{\pgfqpoint{5.032149in}{1.833075in}}%
\pgfpathlineto{\pgfqpoint{5.034264in}{1.828208in}}%
\pgfpathlineto{\pgfqpoint{5.036378in}{1.829197in}}%
\pgfpathlineto{\pgfqpoint{5.040608in}{1.827201in}}%
\pgfpathlineto{\pgfqpoint{5.042722in}{1.828007in}}%
\pgfpathlineto{\pgfqpoint{5.049067in}{1.823775in}}%
\pgfpathlineto{\pgfqpoint{5.051181in}{1.825358in}}%
\pgfpathlineto{\pgfqpoint{5.055411in}{1.833780in}}%
\pgfpathlineto{\pgfqpoint{5.059640in}{1.829809in}}%
\pgfpathlineto{\pgfqpoint{5.061755in}{1.832193in}}%
\pgfpathlineto{\pgfqpoint{5.063869in}{1.839825in}}%
\pgfpathlineto{\pgfqpoint{5.065984in}{1.828438in}}%
\pgfpathlineto{\pgfqpoint{5.068099in}{1.832340in}}%
\pgfpathlineto{\pgfqpoint{5.070214in}{1.827535in}}%
\pgfpathlineto{\pgfqpoint{5.072328in}{1.830845in}}%
\pgfpathlineto{\pgfqpoint{5.076558in}{1.826896in}}%
\pgfpathlineto{\pgfqpoint{5.078672in}{1.827370in}}%
\pgfpathlineto{\pgfqpoint{5.080787in}{1.831854in}}%
\pgfpathlineto{\pgfqpoint{5.082902in}{1.827959in}}%
\pgfpathlineto{\pgfqpoint{5.087131in}{1.827353in}}%
\pgfpathlineto{\pgfqpoint{5.089246in}{1.827987in}}%
\pgfpathlineto{\pgfqpoint{5.091360in}{1.838138in}}%
\pgfpathlineto{\pgfqpoint{5.093475in}{1.836345in}}%
\pgfpathlineto{\pgfqpoint{5.095590in}{1.841738in}}%
\pgfpathlineto{\pgfqpoint{5.097705in}{1.842777in}}%
\pgfpathlineto{\pgfqpoint{5.099819in}{1.839415in}}%
\pgfpathlineto{\pgfqpoint{5.101934in}{1.842547in}}%
\pgfpathlineto{\pgfqpoint{5.104049in}{1.843392in}}%
\pgfpathlineto{\pgfqpoint{5.106163in}{1.840899in}}%
\pgfpathlineto{\pgfqpoint{5.110393in}{1.854069in}}%
\pgfpathlineto{\pgfqpoint{5.112507in}{1.850744in}}%
\pgfpathlineto{\pgfqpoint{5.114622in}{1.850794in}}%
\pgfpathlineto{\pgfqpoint{5.116737in}{1.848540in}}%
\pgfpathlineto{\pgfqpoint{5.120966in}{1.857088in}}%
\pgfpathlineto{\pgfqpoint{5.123081in}{1.865114in}}%
\pgfpathlineto{\pgfqpoint{5.125196in}{1.861623in}}%
\pgfpathlineto{\pgfqpoint{5.127310in}{1.851442in}}%
\pgfpathlineto{\pgfqpoint{5.129425in}{1.855845in}}%
\pgfpathlineto{\pgfqpoint{5.133654in}{1.850881in}}%
\pgfpathlineto{\pgfqpoint{5.135769in}{1.857060in}}%
\pgfpathlineto{\pgfqpoint{5.137884in}{1.856984in}}%
\pgfpathlineto{\pgfqpoint{5.148457in}{1.872082in}}%
\pgfpathlineto{\pgfqpoint{5.152687in}{1.874324in}}%
\pgfpathlineto{\pgfqpoint{5.156916in}{1.881607in}}%
\pgfpathlineto{\pgfqpoint{5.159031in}{1.879020in}}%
\pgfpathlineto{\pgfqpoint{5.161145in}{1.879578in}}%
\pgfpathlineto{\pgfqpoint{5.163260in}{1.882314in}}%
\pgfpathlineto{\pgfqpoint{5.165375in}{1.888985in}}%
\pgfpathlineto{\pgfqpoint{5.167489in}{1.890112in}}%
\pgfpathlineto{\pgfqpoint{5.169604in}{1.888162in}}%
\pgfpathlineto{\pgfqpoint{5.178063in}{1.873040in}}%
\pgfpathlineto{\pgfqpoint{5.180178in}{1.864202in}}%
\pgfpathlineto{\pgfqpoint{5.182292in}{1.861987in}}%
\pgfpathlineto{\pgfqpoint{5.184407in}{1.855137in}}%
\pgfpathlineto{\pgfqpoint{5.186522in}{1.853382in}}%
\pgfpathlineto{\pgfqpoint{5.188636in}{1.849029in}}%
\pgfpathlineto{\pgfqpoint{5.188636in}{1.849029in}}%
\pgfusepath{stroke}%
\end{pgfscope}%
\begin{pgfscope}%
\pgfpathrectangle{\pgfqpoint{0.750000in}{0.275000in}}{\pgfqpoint{4.650000in}{1.925000in}}%
\pgfusepath{clip}%
\pgfsetroundcap%
\pgfsetroundjoin%
\pgfsetlinewidth{1.003750pt}%
\definecolor{currentstroke}{rgb}{0.870588,0.870588,0.000000}%
\pgfsetstrokecolor{currentstroke}%
\pgfsetdash{}{0pt}%
\pgfpathmoveto{\pgfqpoint{0.961364in}{1.363733in}}%
\pgfpathlineto{\pgfqpoint{0.963478in}{1.372409in}}%
\pgfpathlineto{\pgfqpoint{0.965593in}{1.371964in}}%
\pgfpathlineto{\pgfqpoint{0.967708in}{1.373891in}}%
\pgfpathlineto{\pgfqpoint{0.974052in}{1.394522in}}%
\pgfpathlineto{\pgfqpoint{0.978281in}{1.396105in}}%
\pgfpathlineto{\pgfqpoint{0.982511in}{1.407604in}}%
\pgfpathlineto{\pgfqpoint{0.984625in}{1.409018in}}%
\pgfpathlineto{\pgfqpoint{0.986740in}{1.407906in}}%
\pgfpathlineto{\pgfqpoint{0.988855in}{1.410227in}}%
\pgfpathlineto{\pgfqpoint{0.997313in}{1.412910in}}%
\pgfpathlineto{\pgfqpoint{1.001543in}{1.401957in}}%
\pgfpathlineto{\pgfqpoint{1.003658in}{1.406967in}}%
\pgfpathlineto{\pgfqpoint{1.005772in}{1.404929in}}%
\pgfpathlineto{\pgfqpoint{1.010002in}{1.395244in}}%
\pgfpathlineto{\pgfqpoint{1.012116in}{1.396978in}}%
\pgfpathlineto{\pgfqpoint{1.016346in}{1.394769in}}%
\pgfpathlineto{\pgfqpoint{1.018460in}{1.398526in}}%
\pgfpathlineto{\pgfqpoint{1.020575in}{1.397751in}}%
\pgfpathlineto{\pgfqpoint{1.022690in}{1.390568in}}%
\pgfpathlineto{\pgfqpoint{1.029034in}{1.397206in}}%
\pgfpathlineto{\pgfqpoint{1.033263in}{1.390572in}}%
\pgfpathlineto{\pgfqpoint{1.037493in}{1.389792in}}%
\pgfpathlineto{\pgfqpoint{1.041722in}{1.383169in}}%
\pgfpathlineto{\pgfqpoint{1.043837in}{1.392133in}}%
\pgfpathlineto{\pgfqpoint{1.045951in}{1.395260in}}%
\pgfpathlineto{\pgfqpoint{1.050181in}{1.387419in}}%
\pgfpathlineto{\pgfqpoint{1.052295in}{1.389310in}}%
\pgfpathlineto{\pgfqpoint{1.054410in}{1.394417in}}%
\pgfpathlineto{\pgfqpoint{1.056525in}{1.390866in}}%
\pgfpathlineto{\pgfqpoint{1.058640in}{1.389623in}}%
\pgfpathlineto{\pgfqpoint{1.060754in}{1.391863in}}%
\pgfpathlineto{\pgfqpoint{1.062869in}{1.389304in}}%
\pgfpathlineto{\pgfqpoint{1.064984in}{1.382437in}}%
\pgfpathlineto{\pgfqpoint{1.071328in}{1.388516in}}%
\pgfpathlineto{\pgfqpoint{1.073442in}{1.389247in}}%
\pgfpathlineto{\pgfqpoint{1.075557in}{1.385525in}}%
\pgfpathlineto{\pgfqpoint{1.081901in}{1.409756in}}%
\pgfpathlineto{\pgfqpoint{1.084016in}{1.404215in}}%
\pgfpathlineto{\pgfqpoint{1.086131in}{1.409033in}}%
\pgfpathlineto{\pgfqpoint{1.090360in}{1.412415in}}%
\pgfpathlineto{\pgfqpoint{1.096704in}{1.422413in}}%
\pgfpathlineto{\pgfqpoint{1.098819in}{1.423802in}}%
\pgfpathlineto{\pgfqpoint{1.103048in}{1.423171in}}%
\pgfpathlineto{\pgfqpoint{1.105163in}{1.419877in}}%
\pgfpathlineto{\pgfqpoint{1.107278in}{1.420970in}}%
\pgfpathlineto{\pgfqpoint{1.113622in}{1.415851in}}%
\pgfpathlineto{\pgfqpoint{1.115736in}{1.416206in}}%
\pgfpathlineto{\pgfqpoint{1.117851in}{1.424693in}}%
\pgfpathlineto{\pgfqpoint{1.119966in}{1.423144in}}%
\pgfpathlineto{\pgfqpoint{1.122080in}{1.423724in}}%
\pgfpathlineto{\pgfqpoint{1.128424in}{1.438364in}}%
\pgfpathlineto{\pgfqpoint{1.130539in}{1.439478in}}%
\pgfpathlineto{\pgfqpoint{1.132654in}{1.442388in}}%
\pgfpathlineto{\pgfqpoint{1.136883in}{1.438693in}}%
\pgfpathlineto{\pgfqpoint{1.138998in}{1.441346in}}%
\pgfpathlineto{\pgfqpoint{1.141113in}{1.436707in}}%
\pgfpathlineto{\pgfqpoint{1.145342in}{1.446922in}}%
\pgfpathlineto{\pgfqpoint{1.147457in}{1.458106in}}%
\pgfpathlineto{\pgfqpoint{1.153801in}{1.463885in}}%
\pgfpathlineto{\pgfqpoint{1.155915in}{1.463343in}}%
\pgfpathlineto{\pgfqpoint{1.160145in}{1.471181in}}%
\pgfpathlineto{\pgfqpoint{1.162260in}{1.473163in}}%
\pgfpathlineto{\pgfqpoint{1.164374in}{1.471608in}}%
\pgfpathlineto{\pgfqpoint{1.168604in}{1.472187in}}%
\pgfpathlineto{\pgfqpoint{1.172833in}{1.470692in}}%
\pgfpathlineto{\pgfqpoint{1.174948in}{1.484113in}}%
\pgfpathlineto{\pgfqpoint{1.177062in}{1.487535in}}%
\pgfpathlineto{\pgfqpoint{1.179177in}{1.484693in}}%
\pgfpathlineto{\pgfqpoint{1.183406in}{1.489074in}}%
\pgfpathlineto{\pgfqpoint{1.185521in}{1.487002in}}%
\pgfpathlineto{\pgfqpoint{1.191865in}{1.490189in}}%
\pgfpathlineto{\pgfqpoint{1.193980in}{1.487594in}}%
\pgfpathlineto{\pgfqpoint{1.196095in}{1.487414in}}%
\pgfpathlineto{\pgfqpoint{1.200324in}{1.482560in}}%
\pgfpathlineto{\pgfqpoint{1.202439in}{1.474595in}}%
\pgfpathlineto{\pgfqpoint{1.204553in}{1.480214in}}%
\pgfpathlineto{\pgfqpoint{1.206668in}{1.492161in}}%
\pgfpathlineto{\pgfqpoint{1.208783in}{1.494924in}}%
\pgfpathlineto{\pgfqpoint{1.210897in}{1.485074in}}%
\pgfpathlineto{\pgfqpoint{1.213012in}{1.484549in}}%
\pgfpathlineto{\pgfqpoint{1.215127in}{1.481432in}}%
\pgfpathlineto{\pgfqpoint{1.217242in}{1.481036in}}%
\pgfpathlineto{\pgfqpoint{1.221471in}{1.487417in}}%
\pgfpathlineto{\pgfqpoint{1.223586in}{1.477249in}}%
\pgfpathlineto{\pgfqpoint{1.225700in}{1.484203in}}%
\pgfpathlineto{\pgfqpoint{1.227815in}{1.481187in}}%
\pgfpathlineto{\pgfqpoint{1.229930in}{1.475145in}}%
\pgfpathlineto{\pgfqpoint{1.234159in}{1.479892in}}%
\pgfpathlineto{\pgfqpoint{1.236274in}{1.480526in}}%
\pgfpathlineto{\pgfqpoint{1.238389in}{1.484451in}}%
\pgfpathlineto{\pgfqpoint{1.240503in}{1.476949in}}%
\pgfpathlineto{\pgfqpoint{1.242618in}{1.475279in}}%
\pgfpathlineto{\pgfqpoint{1.244733in}{1.480971in}}%
\pgfpathlineto{\pgfqpoint{1.246847in}{1.480950in}}%
\pgfpathlineto{\pgfqpoint{1.248962in}{1.487787in}}%
\pgfpathlineto{\pgfqpoint{1.253191in}{1.477821in}}%
\pgfpathlineto{\pgfqpoint{1.255306in}{1.477112in}}%
\pgfpathlineto{\pgfqpoint{1.257421in}{1.481237in}}%
\pgfpathlineto{\pgfqpoint{1.261650in}{1.471373in}}%
\pgfpathlineto{\pgfqpoint{1.263765in}{1.472981in}}%
\pgfpathlineto{\pgfqpoint{1.267994in}{1.483521in}}%
\pgfpathlineto{\pgfqpoint{1.270109in}{1.485733in}}%
\pgfpathlineto{\pgfqpoint{1.272224in}{1.485234in}}%
\pgfpathlineto{\pgfqpoint{1.274338in}{1.486225in}}%
\pgfpathlineto{\pgfqpoint{1.276453in}{1.493404in}}%
\pgfpathlineto{\pgfqpoint{1.278568in}{1.495227in}}%
\pgfpathlineto{\pgfqpoint{1.282797in}{1.495236in}}%
\pgfpathlineto{\pgfqpoint{1.284912in}{1.499641in}}%
\pgfpathlineto{\pgfqpoint{1.287026in}{1.495689in}}%
\pgfpathlineto{\pgfqpoint{1.289141in}{1.494415in}}%
\pgfpathlineto{\pgfqpoint{1.291256in}{1.495708in}}%
\pgfpathlineto{\pgfqpoint{1.293371in}{1.495524in}}%
\pgfpathlineto{\pgfqpoint{1.297600in}{1.505454in}}%
\pgfpathlineto{\pgfqpoint{1.301829in}{1.510327in}}%
\pgfpathlineto{\pgfqpoint{1.303944in}{1.514135in}}%
\pgfpathlineto{\pgfqpoint{1.306059in}{1.513855in}}%
\pgfpathlineto{\pgfqpoint{1.308173in}{1.520372in}}%
\pgfpathlineto{\pgfqpoint{1.314517in}{1.524807in}}%
\pgfpathlineto{\pgfqpoint{1.316632in}{1.535279in}}%
\pgfpathlineto{\pgfqpoint{1.318747in}{1.539562in}}%
\pgfpathlineto{\pgfqpoint{1.320862in}{1.538904in}}%
\pgfpathlineto{\pgfqpoint{1.322976in}{1.535265in}}%
\pgfpathlineto{\pgfqpoint{1.325091in}{1.529290in}}%
\pgfpathlineto{\pgfqpoint{1.331435in}{1.530981in}}%
\pgfpathlineto{\pgfqpoint{1.333550in}{1.525135in}}%
\pgfpathlineto{\pgfqpoint{1.335664in}{1.533778in}}%
\pgfpathlineto{\pgfqpoint{1.339894in}{1.535406in}}%
\pgfpathlineto{\pgfqpoint{1.342009in}{1.534767in}}%
\pgfpathlineto{\pgfqpoint{1.348353in}{1.541566in}}%
\pgfpathlineto{\pgfqpoint{1.350467in}{1.538406in}}%
\pgfpathlineto{\pgfqpoint{1.352582in}{1.540964in}}%
\pgfpathlineto{\pgfqpoint{1.354697in}{1.546473in}}%
\pgfpathlineto{\pgfqpoint{1.356811in}{1.547347in}}%
\pgfpathlineto{\pgfqpoint{1.358926in}{1.554219in}}%
\pgfpathlineto{\pgfqpoint{1.361041in}{1.553900in}}%
\pgfpathlineto{\pgfqpoint{1.365270in}{1.543869in}}%
\pgfpathlineto{\pgfqpoint{1.367385in}{1.547218in}}%
\pgfpathlineto{\pgfqpoint{1.369500in}{1.546458in}}%
\pgfpathlineto{\pgfqpoint{1.371614in}{1.547392in}}%
\pgfpathlineto{\pgfqpoint{1.375844in}{1.534336in}}%
\pgfpathlineto{\pgfqpoint{1.380073in}{1.536479in}}%
\pgfpathlineto{\pgfqpoint{1.382188in}{1.530362in}}%
\pgfpathlineto{\pgfqpoint{1.384302in}{1.533971in}}%
\pgfpathlineto{\pgfqpoint{1.388532in}{1.533893in}}%
\pgfpathlineto{\pgfqpoint{1.390646in}{1.540001in}}%
\pgfpathlineto{\pgfqpoint{1.392761in}{1.536511in}}%
\pgfpathlineto{\pgfqpoint{1.394876in}{1.540910in}}%
\pgfpathlineto{\pgfqpoint{1.396991in}{1.538139in}}%
\pgfpathlineto{\pgfqpoint{1.399105in}{1.542628in}}%
\pgfpathlineto{\pgfqpoint{1.403335in}{1.537551in}}%
\pgfpathlineto{\pgfqpoint{1.405449in}{1.540619in}}%
\pgfpathlineto{\pgfqpoint{1.409679in}{1.541523in}}%
\pgfpathlineto{\pgfqpoint{1.411793in}{1.545759in}}%
\pgfpathlineto{\pgfqpoint{1.413908in}{1.539155in}}%
\pgfpathlineto{\pgfqpoint{1.416023in}{1.541608in}}%
\pgfpathlineto{\pgfqpoint{1.418137in}{1.545772in}}%
\pgfpathlineto{\pgfqpoint{1.422367in}{1.539490in}}%
\pgfpathlineto{\pgfqpoint{1.424482in}{1.535062in}}%
\pgfpathlineto{\pgfqpoint{1.426596in}{1.536762in}}%
\pgfpathlineto{\pgfqpoint{1.428711in}{1.531472in}}%
\pgfpathlineto{\pgfqpoint{1.432940in}{1.534277in}}%
\pgfpathlineto{\pgfqpoint{1.435055in}{1.526724in}}%
\pgfpathlineto{\pgfqpoint{1.437170in}{1.526473in}}%
\pgfpathlineto{\pgfqpoint{1.439284in}{1.528171in}}%
\pgfpathlineto{\pgfqpoint{1.441399in}{1.527396in}}%
\pgfpathlineto{\pgfqpoint{1.445628in}{1.533444in}}%
\pgfpathlineto{\pgfqpoint{1.447743in}{1.533441in}}%
\pgfpathlineto{\pgfqpoint{1.449858in}{1.542854in}}%
\pgfpathlineto{\pgfqpoint{1.451973in}{1.545128in}}%
\pgfpathlineto{\pgfqpoint{1.454087in}{1.545151in}}%
\pgfpathlineto{\pgfqpoint{1.456202in}{1.541860in}}%
\pgfpathlineto{\pgfqpoint{1.458317in}{1.535166in}}%
\pgfpathlineto{\pgfqpoint{1.460431in}{1.533472in}}%
\pgfpathlineto{\pgfqpoint{1.462546in}{1.533358in}}%
\pgfpathlineto{\pgfqpoint{1.466775in}{1.525694in}}%
\pgfpathlineto{\pgfqpoint{1.471005in}{1.530189in}}%
\pgfpathlineto{\pgfqpoint{1.473120in}{1.531929in}}%
\pgfpathlineto{\pgfqpoint{1.477349in}{1.521180in}}%
\pgfpathlineto{\pgfqpoint{1.479464in}{1.521242in}}%
\pgfpathlineto{\pgfqpoint{1.481578in}{1.515456in}}%
\pgfpathlineto{\pgfqpoint{1.483693in}{1.516521in}}%
\pgfpathlineto{\pgfqpoint{1.485808in}{1.515288in}}%
\pgfpathlineto{\pgfqpoint{1.490037in}{1.505108in}}%
\pgfpathlineto{\pgfqpoint{1.492152in}{1.506383in}}%
\pgfpathlineto{\pgfqpoint{1.494266in}{1.512889in}}%
\pgfpathlineto{\pgfqpoint{1.496381in}{1.511078in}}%
\pgfpathlineto{\pgfqpoint{1.498496in}{1.516680in}}%
\pgfpathlineto{\pgfqpoint{1.500611in}{1.515190in}}%
\pgfpathlineto{\pgfqpoint{1.502725in}{1.511903in}}%
\pgfpathlineto{\pgfqpoint{1.504840in}{1.511516in}}%
\pgfpathlineto{\pgfqpoint{1.506955in}{1.507702in}}%
\pgfpathlineto{\pgfqpoint{1.511184in}{1.493319in}}%
\pgfpathlineto{\pgfqpoint{1.513299in}{1.502199in}}%
\pgfpathlineto{\pgfqpoint{1.515413in}{1.500440in}}%
\pgfpathlineto{\pgfqpoint{1.517528in}{1.497004in}}%
\pgfpathlineto{\pgfqpoint{1.519643in}{1.500878in}}%
\pgfpathlineto{\pgfqpoint{1.521757in}{1.499143in}}%
\pgfpathlineto{\pgfqpoint{1.523872in}{1.495908in}}%
\pgfpathlineto{\pgfqpoint{1.525987in}{1.500890in}}%
\pgfpathlineto{\pgfqpoint{1.532331in}{1.505923in}}%
\pgfpathlineto{\pgfqpoint{1.534446in}{1.505053in}}%
\pgfpathlineto{\pgfqpoint{1.536560in}{1.509386in}}%
\pgfpathlineto{\pgfqpoint{1.540790in}{1.502144in}}%
\pgfpathlineto{\pgfqpoint{1.545019in}{1.508003in}}%
\pgfpathlineto{\pgfqpoint{1.553478in}{1.532478in}}%
\pgfpathlineto{\pgfqpoint{1.555593in}{1.532687in}}%
\pgfpathlineto{\pgfqpoint{1.557707in}{1.535924in}}%
\pgfpathlineto{\pgfqpoint{1.559822in}{1.542246in}}%
\pgfpathlineto{\pgfqpoint{1.561937in}{1.542136in}}%
\pgfpathlineto{\pgfqpoint{1.566166in}{1.550066in}}%
\pgfpathlineto{\pgfqpoint{1.568281in}{1.549992in}}%
\pgfpathlineto{\pgfqpoint{1.570395in}{1.557180in}}%
\pgfpathlineto{\pgfqpoint{1.574625in}{1.545780in}}%
\pgfpathlineto{\pgfqpoint{1.576740in}{1.541025in}}%
\pgfpathlineto{\pgfqpoint{1.578854in}{1.540600in}}%
\pgfpathlineto{\pgfqpoint{1.580969in}{1.549164in}}%
\pgfpathlineto{\pgfqpoint{1.585198in}{1.546987in}}%
\pgfpathlineto{\pgfqpoint{1.587313in}{1.543060in}}%
\pgfpathlineto{\pgfqpoint{1.589428in}{1.543369in}}%
\pgfpathlineto{\pgfqpoint{1.591542in}{1.540834in}}%
\pgfpathlineto{\pgfqpoint{1.593657in}{1.541721in}}%
\pgfpathlineto{\pgfqpoint{1.595772in}{1.540871in}}%
\pgfpathlineto{\pgfqpoint{1.597886in}{1.537378in}}%
\pgfpathlineto{\pgfqpoint{1.600001in}{1.536085in}}%
\pgfpathlineto{\pgfqpoint{1.602116in}{1.531366in}}%
\pgfpathlineto{\pgfqpoint{1.604231in}{1.531099in}}%
\pgfpathlineto{\pgfqpoint{1.606345in}{1.533564in}}%
\pgfpathlineto{\pgfqpoint{1.608460in}{1.538496in}}%
\pgfpathlineto{\pgfqpoint{1.610575in}{1.532656in}}%
\pgfpathlineto{\pgfqpoint{1.612689in}{1.537913in}}%
\pgfpathlineto{\pgfqpoint{1.614804in}{1.534769in}}%
\pgfpathlineto{\pgfqpoint{1.616919in}{1.534385in}}%
\pgfpathlineto{\pgfqpoint{1.619033in}{1.532630in}}%
\pgfpathlineto{\pgfqpoint{1.621148in}{1.535637in}}%
\pgfpathlineto{\pgfqpoint{1.623263in}{1.533529in}}%
\pgfpathlineto{\pgfqpoint{1.625377in}{1.540124in}}%
\pgfpathlineto{\pgfqpoint{1.627492in}{1.541220in}}%
\pgfpathlineto{\pgfqpoint{1.629607in}{1.539784in}}%
\pgfpathlineto{\pgfqpoint{1.631722in}{1.540247in}}%
\pgfpathlineto{\pgfqpoint{1.633836in}{1.536263in}}%
\pgfpathlineto{\pgfqpoint{1.635951in}{1.538894in}}%
\pgfpathlineto{\pgfqpoint{1.638066in}{1.534951in}}%
\pgfpathlineto{\pgfqpoint{1.640180in}{1.538349in}}%
\pgfpathlineto{\pgfqpoint{1.642295in}{1.538082in}}%
\pgfpathlineto{\pgfqpoint{1.646524in}{1.534851in}}%
\pgfpathlineto{\pgfqpoint{1.650754in}{1.539810in}}%
\pgfpathlineto{\pgfqpoint{1.654983in}{1.550480in}}%
\pgfpathlineto{\pgfqpoint{1.657098in}{1.550304in}}%
\pgfpathlineto{\pgfqpoint{1.659213in}{1.556113in}}%
\pgfpathlineto{\pgfqpoint{1.661327in}{1.557783in}}%
\pgfpathlineto{\pgfqpoint{1.663442in}{1.561941in}}%
\pgfpathlineto{\pgfqpoint{1.665557in}{1.557775in}}%
\pgfpathlineto{\pgfqpoint{1.667671in}{1.559253in}}%
\pgfpathlineto{\pgfqpoint{1.669786in}{1.558329in}}%
\pgfpathlineto{\pgfqpoint{1.671901in}{1.559031in}}%
\pgfpathlineto{\pgfqpoint{1.674015in}{1.568130in}}%
\pgfpathlineto{\pgfqpoint{1.676130in}{1.567792in}}%
\pgfpathlineto{\pgfqpoint{1.678245in}{1.570345in}}%
\pgfpathlineto{\pgfqpoint{1.680359in}{1.563130in}}%
\pgfpathlineto{\pgfqpoint{1.682474in}{1.562476in}}%
\pgfpathlineto{\pgfqpoint{1.684589in}{1.559645in}}%
\pgfpathlineto{\pgfqpoint{1.688818in}{1.545700in}}%
\pgfpathlineto{\pgfqpoint{1.690933in}{1.543868in}}%
\pgfpathlineto{\pgfqpoint{1.693048in}{1.546343in}}%
\pgfpathlineto{\pgfqpoint{1.695162in}{1.554749in}}%
\pgfpathlineto{\pgfqpoint{1.697277in}{1.557682in}}%
\pgfpathlineto{\pgfqpoint{1.699392in}{1.554040in}}%
\pgfpathlineto{\pgfqpoint{1.701506in}{1.555335in}}%
\pgfpathlineto{\pgfqpoint{1.703621in}{1.554085in}}%
\pgfpathlineto{\pgfqpoint{1.707851in}{1.558383in}}%
\pgfpathlineto{\pgfqpoint{1.709965in}{1.558357in}}%
\pgfpathlineto{\pgfqpoint{1.712080in}{1.552996in}}%
\pgfpathlineto{\pgfqpoint{1.714195in}{1.551150in}}%
\pgfpathlineto{\pgfqpoint{1.716309in}{1.553284in}}%
\pgfpathlineto{\pgfqpoint{1.718424in}{1.548752in}}%
\pgfpathlineto{\pgfqpoint{1.720539in}{1.555238in}}%
\pgfpathlineto{\pgfqpoint{1.722653in}{1.550294in}}%
\pgfpathlineto{\pgfqpoint{1.724768in}{1.556192in}}%
\pgfpathlineto{\pgfqpoint{1.728997in}{1.553428in}}%
\pgfpathlineto{\pgfqpoint{1.731112in}{1.553312in}}%
\pgfpathlineto{\pgfqpoint{1.733227in}{1.556358in}}%
\pgfpathlineto{\pgfqpoint{1.735342in}{1.551915in}}%
\pgfpathlineto{\pgfqpoint{1.741686in}{1.555572in}}%
\pgfpathlineto{\pgfqpoint{1.743800in}{1.553702in}}%
\pgfpathlineto{\pgfqpoint{1.745915in}{1.555281in}}%
\pgfpathlineto{\pgfqpoint{1.752259in}{1.568927in}}%
\pgfpathlineto{\pgfqpoint{1.754374in}{1.565239in}}%
\pgfpathlineto{\pgfqpoint{1.756488in}{1.565547in}}%
\pgfpathlineto{\pgfqpoint{1.758603in}{1.564037in}}%
\pgfpathlineto{\pgfqpoint{1.762833in}{1.565725in}}%
\pgfpathlineto{\pgfqpoint{1.764947in}{1.553993in}}%
\pgfpathlineto{\pgfqpoint{1.767062in}{1.550874in}}%
\pgfpathlineto{\pgfqpoint{1.771291in}{1.558634in}}%
\pgfpathlineto{\pgfqpoint{1.773406in}{1.562703in}}%
\pgfpathlineto{\pgfqpoint{1.775521in}{1.570417in}}%
\pgfpathlineto{\pgfqpoint{1.777635in}{1.572321in}}%
\pgfpathlineto{\pgfqpoint{1.779750in}{1.575906in}}%
\pgfpathlineto{\pgfqpoint{1.781865in}{1.574919in}}%
\pgfpathlineto{\pgfqpoint{1.783979in}{1.579503in}}%
\pgfpathlineto{\pgfqpoint{1.786094in}{1.578610in}}%
\pgfpathlineto{\pgfqpoint{1.790324in}{1.578747in}}%
\pgfpathlineto{\pgfqpoint{1.792438in}{1.571582in}}%
\pgfpathlineto{\pgfqpoint{1.794553in}{1.577487in}}%
\pgfpathlineto{\pgfqpoint{1.796668in}{1.577400in}}%
\pgfpathlineto{\pgfqpoint{1.798782in}{1.581502in}}%
\pgfpathlineto{\pgfqpoint{1.800897in}{1.581033in}}%
\pgfpathlineto{\pgfqpoint{1.805126in}{1.582770in}}%
\pgfpathlineto{\pgfqpoint{1.807241in}{1.582191in}}%
\pgfpathlineto{\pgfqpoint{1.809356in}{1.584899in}}%
\pgfpathlineto{\pgfqpoint{1.811471in}{1.590825in}}%
\pgfpathlineto{\pgfqpoint{1.813585in}{1.592246in}}%
\pgfpathlineto{\pgfqpoint{1.815700in}{1.597057in}}%
\pgfpathlineto{\pgfqpoint{1.817815in}{1.591909in}}%
\pgfpathlineto{\pgfqpoint{1.830503in}{1.617190in}}%
\pgfpathlineto{\pgfqpoint{1.834732in}{1.617151in}}%
\pgfpathlineto{\pgfqpoint{1.836847in}{1.626733in}}%
\pgfpathlineto{\pgfqpoint{1.838962in}{1.628270in}}%
\pgfpathlineto{\pgfqpoint{1.841076in}{1.619791in}}%
\pgfpathlineto{\pgfqpoint{1.845306in}{1.628966in}}%
\pgfpathlineto{\pgfqpoint{1.847420in}{1.629218in}}%
\pgfpathlineto{\pgfqpoint{1.849535in}{1.631270in}}%
\pgfpathlineto{\pgfqpoint{1.851650in}{1.630810in}}%
\pgfpathlineto{\pgfqpoint{1.853764in}{1.632804in}}%
\pgfpathlineto{\pgfqpoint{1.855879in}{1.631715in}}%
\pgfpathlineto{\pgfqpoint{1.857994in}{1.628709in}}%
\pgfpathlineto{\pgfqpoint{1.862223in}{1.631940in}}%
\pgfpathlineto{\pgfqpoint{1.864338in}{1.632258in}}%
\pgfpathlineto{\pgfqpoint{1.866453in}{1.626701in}}%
\pgfpathlineto{\pgfqpoint{1.868567in}{1.626181in}}%
\pgfpathlineto{\pgfqpoint{1.872797in}{1.629225in}}%
\pgfpathlineto{\pgfqpoint{1.879141in}{1.635056in}}%
\pgfpathlineto{\pgfqpoint{1.881255in}{1.642810in}}%
\pgfpathlineto{\pgfqpoint{1.883370in}{1.639275in}}%
\pgfpathlineto{\pgfqpoint{1.887599in}{1.644744in}}%
\pgfpathlineto{\pgfqpoint{1.896058in}{1.630627in}}%
\pgfpathlineto{\pgfqpoint{1.898173in}{1.633585in}}%
\pgfpathlineto{\pgfqpoint{1.900288in}{1.633413in}}%
\pgfpathlineto{\pgfqpoint{1.904517in}{1.636582in}}%
\pgfpathlineto{\pgfqpoint{1.910861in}{1.635667in}}%
\pgfpathlineto{\pgfqpoint{1.912976in}{1.645236in}}%
\pgfpathlineto{\pgfqpoint{1.915090in}{1.643227in}}%
\pgfpathlineto{\pgfqpoint{1.919320in}{1.648301in}}%
\pgfpathlineto{\pgfqpoint{1.921435in}{1.651660in}}%
\pgfpathlineto{\pgfqpoint{1.925664in}{1.661072in}}%
\pgfpathlineto{\pgfqpoint{1.927779in}{1.654424in}}%
\pgfpathlineto{\pgfqpoint{1.932008in}{1.664025in}}%
\pgfpathlineto{\pgfqpoint{1.934123in}{1.662015in}}%
\pgfpathlineto{\pgfqpoint{1.936237in}{1.663366in}}%
\pgfpathlineto{\pgfqpoint{1.938352in}{1.663286in}}%
\pgfpathlineto{\pgfqpoint{1.940467in}{1.658243in}}%
\pgfpathlineto{\pgfqpoint{1.942582in}{1.661057in}}%
\pgfpathlineto{\pgfqpoint{1.946811in}{1.675639in}}%
\pgfpathlineto{\pgfqpoint{1.948926in}{1.667411in}}%
\pgfpathlineto{\pgfqpoint{1.951040in}{1.672058in}}%
\pgfpathlineto{\pgfqpoint{1.953155in}{1.673593in}}%
\pgfpathlineto{\pgfqpoint{1.955270in}{1.671759in}}%
\pgfpathlineto{\pgfqpoint{1.957384in}{1.667305in}}%
\pgfpathlineto{\pgfqpoint{1.959499in}{1.671568in}}%
\pgfpathlineto{\pgfqpoint{1.963728in}{1.660589in}}%
\pgfpathlineto{\pgfqpoint{1.965843in}{1.651706in}}%
\pgfpathlineto{\pgfqpoint{1.967958in}{1.653041in}}%
\pgfpathlineto{\pgfqpoint{1.970073in}{1.651211in}}%
\pgfpathlineto{\pgfqpoint{1.972187in}{1.652581in}}%
\pgfpathlineto{\pgfqpoint{1.974302in}{1.649075in}}%
\pgfpathlineto{\pgfqpoint{1.976417in}{1.649184in}}%
\pgfpathlineto{\pgfqpoint{1.978531in}{1.641616in}}%
\pgfpathlineto{\pgfqpoint{1.980646in}{1.645834in}}%
\pgfpathlineto{\pgfqpoint{1.982761in}{1.637070in}}%
\pgfpathlineto{\pgfqpoint{1.984875in}{1.644517in}}%
\pgfpathlineto{\pgfqpoint{1.986990in}{1.643297in}}%
\pgfpathlineto{\pgfqpoint{1.989105in}{1.647542in}}%
\pgfpathlineto{\pgfqpoint{1.991219in}{1.637576in}}%
\pgfpathlineto{\pgfqpoint{1.993334in}{1.642912in}}%
\pgfpathlineto{\pgfqpoint{1.999678in}{1.634272in}}%
\pgfpathlineto{\pgfqpoint{2.001793in}{1.632831in}}%
\pgfpathlineto{\pgfqpoint{2.008137in}{1.655115in}}%
\pgfpathlineto{\pgfqpoint{2.010252in}{1.657448in}}%
\pgfpathlineto{\pgfqpoint{2.012366in}{1.661677in}}%
\pgfpathlineto{\pgfqpoint{2.018710in}{1.680611in}}%
\pgfpathlineto{\pgfqpoint{2.020825in}{1.678488in}}%
\pgfpathlineto{\pgfqpoint{2.022940in}{1.682167in}}%
\pgfpathlineto{\pgfqpoint{2.025055in}{1.677864in}}%
\pgfpathlineto{\pgfqpoint{2.027169in}{1.681659in}}%
\pgfpathlineto{\pgfqpoint{2.029284in}{1.680163in}}%
\pgfpathlineto{\pgfqpoint{2.031399in}{1.685703in}}%
\pgfpathlineto{\pgfqpoint{2.035628in}{1.684925in}}%
\pgfpathlineto{\pgfqpoint{2.039857in}{1.672526in}}%
\pgfpathlineto{\pgfqpoint{2.041972in}{1.673446in}}%
\pgfpathlineto{\pgfqpoint{2.044087in}{1.676833in}}%
\pgfpathlineto{\pgfqpoint{2.046202in}{1.674589in}}%
\pgfpathlineto{\pgfqpoint{2.048316in}{1.674652in}}%
\pgfpathlineto{\pgfqpoint{2.061004in}{1.685052in}}%
\pgfpathlineto{\pgfqpoint{2.063119in}{1.682748in}}%
\pgfpathlineto{\pgfqpoint{2.065234in}{1.688471in}}%
\pgfpathlineto{\pgfqpoint{2.067348in}{1.689730in}}%
\pgfpathlineto{\pgfqpoint{2.069463in}{1.679295in}}%
\pgfpathlineto{\pgfqpoint{2.071578in}{1.682914in}}%
\pgfpathlineto{\pgfqpoint{2.073693in}{1.683593in}}%
\pgfpathlineto{\pgfqpoint{2.075807in}{1.687184in}}%
\pgfpathlineto{\pgfqpoint{2.077922in}{1.679465in}}%
\pgfpathlineto{\pgfqpoint{2.080037in}{1.677334in}}%
\pgfpathlineto{\pgfqpoint{2.082151in}{1.679185in}}%
\pgfpathlineto{\pgfqpoint{2.086381in}{1.688965in}}%
\pgfpathlineto{\pgfqpoint{2.088495in}{1.686865in}}%
\pgfpathlineto{\pgfqpoint{2.090610in}{1.689517in}}%
\pgfpathlineto{\pgfqpoint{2.092725in}{1.696312in}}%
\pgfpathlineto{\pgfqpoint{2.094839in}{1.695287in}}%
\pgfpathlineto{\pgfqpoint{2.096954in}{1.700423in}}%
\pgfpathlineto{\pgfqpoint{2.099069in}{1.700667in}}%
\pgfpathlineto{\pgfqpoint{2.103298in}{1.710118in}}%
\pgfpathlineto{\pgfqpoint{2.105413in}{1.713130in}}%
\pgfpathlineto{\pgfqpoint{2.107528in}{1.712385in}}%
\pgfpathlineto{\pgfqpoint{2.109642in}{1.714020in}}%
\pgfpathlineto{\pgfqpoint{2.113872in}{1.707520in}}%
\pgfpathlineto{\pgfqpoint{2.115986in}{1.707041in}}%
\pgfpathlineto{\pgfqpoint{2.120216in}{1.719624in}}%
\pgfpathlineto{\pgfqpoint{2.122330in}{1.716374in}}%
\pgfpathlineto{\pgfqpoint{2.124445in}{1.709441in}}%
\pgfpathlineto{\pgfqpoint{2.126560in}{1.712091in}}%
\pgfpathlineto{\pgfqpoint{2.128675in}{1.709462in}}%
\pgfpathlineto{\pgfqpoint{2.130789in}{1.702806in}}%
\pgfpathlineto{\pgfqpoint{2.137133in}{1.712888in}}%
\pgfpathlineto{\pgfqpoint{2.139248in}{1.707426in}}%
\pgfpathlineto{\pgfqpoint{2.141363in}{1.705426in}}%
\pgfpathlineto{\pgfqpoint{2.147707in}{1.720434in}}%
\pgfpathlineto{\pgfqpoint{2.149822in}{1.722090in}}%
\pgfpathlineto{\pgfqpoint{2.154051in}{1.711334in}}%
\pgfpathlineto{\pgfqpoint{2.160395in}{1.706502in}}%
\pgfpathlineto{\pgfqpoint{2.164624in}{1.708732in}}%
\pgfpathlineto{\pgfqpoint{2.166739in}{1.707112in}}%
\pgfpathlineto{\pgfqpoint{2.168854in}{1.707911in}}%
\pgfpathlineto{\pgfqpoint{2.170968in}{1.707080in}}%
\pgfpathlineto{\pgfqpoint{2.173083in}{1.700094in}}%
\pgfpathlineto{\pgfqpoint{2.177313in}{1.714838in}}%
\pgfpathlineto{\pgfqpoint{2.179427in}{1.713800in}}%
\pgfpathlineto{\pgfqpoint{2.181542in}{1.710793in}}%
\pgfpathlineto{\pgfqpoint{2.183657in}{1.712065in}}%
\pgfpathlineto{\pgfqpoint{2.185771in}{1.709507in}}%
\pgfpathlineto{\pgfqpoint{2.187886in}{1.711366in}}%
\pgfpathlineto{\pgfqpoint{2.190001in}{1.704612in}}%
\pgfpathlineto{\pgfqpoint{2.192115in}{1.702364in}}%
\pgfpathlineto{\pgfqpoint{2.194230in}{1.704000in}}%
\pgfpathlineto{\pgfqpoint{2.196345in}{1.707721in}}%
\pgfpathlineto{\pgfqpoint{2.198459in}{1.706521in}}%
\pgfpathlineto{\pgfqpoint{2.202689in}{1.690116in}}%
\pgfpathlineto{\pgfqpoint{2.209033in}{1.696214in}}%
\pgfpathlineto{\pgfqpoint{2.211148in}{1.707000in}}%
\pgfpathlineto{\pgfqpoint{2.213262in}{1.709336in}}%
\pgfpathlineto{\pgfqpoint{2.215377in}{1.716060in}}%
\pgfpathlineto{\pgfqpoint{2.217492in}{1.717094in}}%
\pgfpathlineto{\pgfqpoint{2.221721in}{1.709877in}}%
\pgfpathlineto{\pgfqpoint{2.223836in}{1.713272in}}%
\pgfpathlineto{\pgfqpoint{2.225950in}{1.710088in}}%
\pgfpathlineto{\pgfqpoint{2.228065in}{1.711717in}}%
\pgfpathlineto{\pgfqpoint{2.230180in}{1.705210in}}%
\pgfpathlineto{\pgfqpoint{2.232295in}{1.715386in}}%
\pgfpathlineto{\pgfqpoint{2.234409in}{1.715474in}}%
\pgfpathlineto{\pgfqpoint{2.238639in}{1.711413in}}%
\pgfpathlineto{\pgfqpoint{2.240753in}{1.705187in}}%
\pgfpathlineto{\pgfqpoint{2.242868in}{1.708890in}}%
\pgfpathlineto{\pgfqpoint{2.244983in}{1.709956in}}%
\pgfpathlineto{\pgfqpoint{2.253441in}{1.706526in}}%
\pgfpathlineto{\pgfqpoint{2.255556in}{1.708685in}}%
\pgfpathlineto{\pgfqpoint{2.257671in}{1.703989in}}%
\pgfpathlineto{\pgfqpoint{2.259786in}{1.705621in}}%
\pgfpathlineto{\pgfqpoint{2.261900in}{1.705381in}}%
\pgfpathlineto{\pgfqpoint{2.264015in}{1.711296in}}%
\pgfpathlineto{\pgfqpoint{2.266130in}{1.710812in}}%
\pgfpathlineto{\pgfqpoint{2.268244in}{1.707327in}}%
\pgfpathlineto{\pgfqpoint{2.270359in}{1.706685in}}%
\pgfpathlineto{\pgfqpoint{2.272474in}{1.707414in}}%
\pgfpathlineto{\pgfqpoint{2.274588in}{1.709585in}}%
\pgfpathlineto{\pgfqpoint{2.276703in}{1.709497in}}%
\pgfpathlineto{\pgfqpoint{2.278818in}{1.705336in}}%
\pgfpathlineto{\pgfqpoint{2.285162in}{1.704698in}}%
\pgfpathlineto{\pgfqpoint{2.291506in}{1.696686in}}%
\pgfpathlineto{\pgfqpoint{2.295735in}{1.702285in}}%
\pgfpathlineto{\pgfqpoint{2.297850in}{1.700733in}}%
\pgfpathlineto{\pgfqpoint{2.299965in}{1.704921in}}%
\pgfpathlineto{\pgfqpoint{2.302079in}{1.706391in}}%
\pgfpathlineto{\pgfqpoint{2.306309in}{1.720107in}}%
\pgfpathlineto{\pgfqpoint{2.310538in}{1.736858in}}%
\pgfpathlineto{\pgfqpoint{2.314768in}{1.725475in}}%
\pgfpathlineto{\pgfqpoint{2.316882in}{1.730137in}}%
\pgfpathlineto{\pgfqpoint{2.318997in}{1.727251in}}%
\pgfpathlineto{\pgfqpoint{2.321112in}{1.728109in}}%
\pgfpathlineto{\pgfqpoint{2.323226in}{1.727364in}}%
\pgfpathlineto{\pgfqpoint{2.327456in}{1.716461in}}%
\pgfpathlineto{\pgfqpoint{2.329570in}{1.717392in}}%
\pgfpathlineto{\pgfqpoint{2.331685in}{1.720068in}}%
\pgfpathlineto{\pgfqpoint{2.333800in}{1.718305in}}%
\pgfpathlineto{\pgfqpoint{2.338029in}{1.718784in}}%
\pgfpathlineto{\pgfqpoint{2.344373in}{1.708559in}}%
\pgfpathlineto{\pgfqpoint{2.346488in}{1.708627in}}%
\pgfpathlineto{\pgfqpoint{2.348603in}{1.706343in}}%
\pgfpathlineto{\pgfqpoint{2.350717in}{1.707684in}}%
\pgfpathlineto{\pgfqpoint{2.352832in}{1.704119in}}%
\pgfpathlineto{\pgfqpoint{2.357061in}{1.702618in}}%
\pgfpathlineto{\pgfqpoint{2.359176in}{1.703653in}}%
\pgfpathlineto{\pgfqpoint{2.363406in}{1.709991in}}%
\pgfpathlineto{\pgfqpoint{2.365520in}{1.706806in}}%
\pgfpathlineto{\pgfqpoint{2.369750in}{1.710902in}}%
\pgfpathlineto{\pgfqpoint{2.371864in}{1.705124in}}%
\pgfpathlineto{\pgfqpoint{2.373979in}{1.710027in}}%
\pgfpathlineto{\pgfqpoint{2.376094in}{1.709128in}}%
\pgfpathlineto{\pgfqpoint{2.378208in}{1.710557in}}%
\pgfpathlineto{\pgfqpoint{2.380323in}{1.707311in}}%
\pgfpathlineto{\pgfqpoint{2.382438in}{1.715224in}}%
\pgfpathlineto{\pgfqpoint{2.386667in}{1.700962in}}%
\pgfpathlineto{\pgfqpoint{2.388782in}{1.699081in}}%
\pgfpathlineto{\pgfqpoint{2.393011in}{1.700324in}}%
\pgfpathlineto{\pgfqpoint{2.395126in}{1.704761in}}%
\pgfpathlineto{\pgfqpoint{2.397241in}{1.702352in}}%
\pgfpathlineto{\pgfqpoint{2.401470in}{1.706551in}}%
\pgfpathlineto{\pgfqpoint{2.405699in}{1.702442in}}%
\pgfpathlineto{\pgfqpoint{2.407814in}{1.701265in}}%
\pgfpathlineto{\pgfqpoint{2.409929in}{1.710927in}}%
\pgfpathlineto{\pgfqpoint{2.412044in}{1.710132in}}%
\pgfpathlineto{\pgfqpoint{2.414158in}{1.713550in}}%
\pgfpathlineto{\pgfqpoint{2.418388in}{1.714706in}}%
\pgfpathlineto{\pgfqpoint{2.420502in}{1.714272in}}%
\pgfpathlineto{\pgfqpoint{2.424732in}{1.724917in}}%
\pgfpathlineto{\pgfqpoint{2.426846in}{1.720741in}}%
\pgfpathlineto{\pgfqpoint{2.433190in}{1.726266in}}%
\pgfpathlineto{\pgfqpoint{2.435305in}{1.722347in}}%
\pgfpathlineto{\pgfqpoint{2.437420in}{1.731301in}}%
\pgfpathlineto{\pgfqpoint{2.439535in}{1.727829in}}%
\pgfpathlineto{\pgfqpoint{2.441649in}{1.733633in}}%
\pgfpathlineto{\pgfqpoint{2.443764in}{1.729781in}}%
\pgfpathlineto{\pgfqpoint{2.445879in}{1.734597in}}%
\pgfpathlineto{\pgfqpoint{2.447993in}{1.736534in}}%
\pgfpathlineto{\pgfqpoint{2.450108in}{1.742103in}}%
\pgfpathlineto{\pgfqpoint{2.452223in}{1.741558in}}%
\pgfpathlineto{\pgfqpoint{2.454337in}{1.751645in}}%
\pgfpathlineto{\pgfqpoint{2.456452in}{1.752674in}}%
\pgfpathlineto{\pgfqpoint{2.458567in}{1.751903in}}%
\pgfpathlineto{\pgfqpoint{2.460681in}{1.747785in}}%
\pgfpathlineto{\pgfqpoint{2.462796in}{1.758501in}}%
\pgfpathlineto{\pgfqpoint{2.464911in}{1.758113in}}%
\pgfpathlineto{\pgfqpoint{2.469140in}{1.750170in}}%
\pgfpathlineto{\pgfqpoint{2.473370in}{1.762153in}}%
\pgfpathlineto{\pgfqpoint{2.475484in}{1.754413in}}%
\pgfpathlineto{\pgfqpoint{2.479714in}{1.761516in}}%
\pgfpathlineto{\pgfqpoint{2.481828in}{1.763927in}}%
\pgfpathlineto{\pgfqpoint{2.483943in}{1.764016in}}%
\pgfpathlineto{\pgfqpoint{2.486058in}{1.758639in}}%
\pgfpathlineto{\pgfqpoint{2.488172in}{1.757145in}}%
\pgfpathlineto{\pgfqpoint{2.490287in}{1.760558in}}%
\pgfpathlineto{\pgfqpoint{2.492402in}{1.760430in}}%
\pgfpathlineto{\pgfqpoint{2.494517in}{1.761609in}}%
\pgfpathlineto{\pgfqpoint{2.496631in}{1.748270in}}%
\pgfpathlineto{\pgfqpoint{2.498746in}{1.749206in}}%
\pgfpathlineto{\pgfqpoint{2.500861in}{1.746709in}}%
\pgfpathlineto{\pgfqpoint{2.502975in}{1.753019in}}%
\pgfpathlineto{\pgfqpoint{2.505090in}{1.755554in}}%
\pgfpathlineto{\pgfqpoint{2.509319in}{1.749475in}}%
\pgfpathlineto{\pgfqpoint{2.511434in}{1.751836in}}%
\pgfpathlineto{\pgfqpoint{2.513549in}{1.751143in}}%
\pgfpathlineto{\pgfqpoint{2.515664in}{1.754358in}}%
\pgfpathlineto{\pgfqpoint{2.517778in}{1.749653in}}%
\pgfpathlineto{\pgfqpoint{2.519893in}{1.755040in}}%
\pgfpathlineto{\pgfqpoint{2.524122in}{1.758037in}}%
\pgfpathlineto{\pgfqpoint{2.528352in}{1.765487in}}%
\pgfpathlineto{\pgfqpoint{2.530466in}{1.766884in}}%
\pgfpathlineto{\pgfqpoint{2.532581in}{1.770450in}}%
\pgfpathlineto{\pgfqpoint{2.534696in}{1.767543in}}%
\pgfpathlineto{\pgfqpoint{2.536810in}{1.768975in}}%
\pgfpathlineto{\pgfqpoint{2.541040in}{1.758467in}}%
\pgfpathlineto{\pgfqpoint{2.545269in}{1.755114in}}%
\pgfpathlineto{\pgfqpoint{2.547384in}{1.756814in}}%
\pgfpathlineto{\pgfqpoint{2.549499in}{1.755959in}}%
\pgfpathlineto{\pgfqpoint{2.551613in}{1.763190in}}%
\pgfpathlineto{\pgfqpoint{2.553728in}{1.764363in}}%
\pgfpathlineto{\pgfqpoint{2.555843in}{1.763782in}}%
\pgfpathlineto{\pgfqpoint{2.557957in}{1.764930in}}%
\pgfpathlineto{\pgfqpoint{2.560072in}{1.764775in}}%
\pgfpathlineto{\pgfqpoint{2.562187in}{1.776299in}}%
\pgfpathlineto{\pgfqpoint{2.564301in}{1.774001in}}%
\pgfpathlineto{\pgfqpoint{2.566416in}{1.780090in}}%
\pgfpathlineto{\pgfqpoint{2.568531in}{1.777743in}}%
\pgfpathlineto{\pgfqpoint{2.570646in}{1.778645in}}%
\pgfpathlineto{\pgfqpoint{2.572760in}{1.776480in}}%
\pgfpathlineto{\pgfqpoint{2.576990in}{1.776206in}}%
\pgfpathlineto{\pgfqpoint{2.579104in}{1.770440in}}%
\pgfpathlineto{\pgfqpoint{2.581219in}{1.770527in}}%
\pgfpathlineto{\pgfqpoint{2.585448in}{1.773951in}}%
\pgfpathlineto{\pgfqpoint{2.587563in}{1.774552in}}%
\pgfpathlineto{\pgfqpoint{2.591792in}{1.769007in}}%
\pgfpathlineto{\pgfqpoint{2.593907in}{1.775200in}}%
\pgfpathlineto{\pgfqpoint{2.596022in}{1.770442in}}%
\pgfpathlineto{\pgfqpoint{2.598137in}{1.775549in}}%
\pgfpathlineto{\pgfqpoint{2.600251in}{1.768773in}}%
\pgfpathlineto{\pgfqpoint{2.602366in}{1.769420in}}%
\pgfpathlineto{\pgfqpoint{2.604481in}{1.766254in}}%
\pgfpathlineto{\pgfqpoint{2.606595in}{1.765818in}}%
\pgfpathlineto{\pgfqpoint{2.612939in}{1.753915in}}%
\pgfpathlineto{\pgfqpoint{2.615054in}{1.750402in}}%
\pgfpathlineto{\pgfqpoint{2.619284in}{1.757077in}}%
\pgfpathlineto{\pgfqpoint{2.623513in}{1.758145in}}%
\pgfpathlineto{\pgfqpoint{2.625628in}{1.762456in}}%
\pgfpathlineto{\pgfqpoint{2.627742in}{1.750988in}}%
\pgfpathlineto{\pgfqpoint{2.636201in}{1.740013in}}%
\pgfpathlineto{\pgfqpoint{2.638316in}{1.731605in}}%
\pgfpathlineto{\pgfqpoint{2.642545in}{1.730794in}}%
\pgfpathlineto{\pgfqpoint{2.644660in}{1.727545in}}%
\pgfpathlineto{\pgfqpoint{2.646775in}{1.727768in}}%
\pgfpathlineto{\pgfqpoint{2.648889in}{1.731808in}}%
\pgfpathlineto{\pgfqpoint{2.651004in}{1.730427in}}%
\pgfpathlineto{\pgfqpoint{2.653119in}{1.739005in}}%
\pgfpathlineto{\pgfqpoint{2.657348in}{1.732310in}}%
\pgfpathlineto{\pgfqpoint{2.659463in}{1.740795in}}%
\pgfpathlineto{\pgfqpoint{2.661577in}{1.740936in}}%
\pgfpathlineto{\pgfqpoint{2.663692in}{1.735892in}}%
\pgfpathlineto{\pgfqpoint{2.665807in}{1.737209in}}%
\pgfpathlineto{\pgfqpoint{2.667921in}{1.742567in}}%
\pgfpathlineto{\pgfqpoint{2.670036in}{1.743716in}}%
\pgfpathlineto{\pgfqpoint{2.672151in}{1.741997in}}%
\pgfpathlineto{\pgfqpoint{2.676380in}{1.735542in}}%
\pgfpathlineto{\pgfqpoint{2.678495in}{1.746476in}}%
\pgfpathlineto{\pgfqpoint{2.680610in}{1.748240in}}%
\pgfpathlineto{\pgfqpoint{2.682724in}{1.745798in}}%
\pgfpathlineto{\pgfqpoint{2.684839in}{1.735882in}}%
\pgfpathlineto{\pgfqpoint{2.686954in}{1.739316in}}%
\pgfpathlineto{\pgfqpoint{2.689068in}{1.738662in}}%
\pgfpathlineto{\pgfqpoint{2.691183in}{1.744779in}}%
\pgfpathlineto{\pgfqpoint{2.693298in}{1.742897in}}%
\pgfpathlineto{\pgfqpoint{2.697527in}{1.752289in}}%
\pgfpathlineto{\pgfqpoint{2.699642in}{1.751528in}}%
\pgfpathlineto{\pgfqpoint{2.701757in}{1.752254in}}%
\pgfpathlineto{\pgfqpoint{2.705986in}{1.748597in}}%
\pgfpathlineto{\pgfqpoint{2.710215in}{1.759927in}}%
\pgfpathlineto{\pgfqpoint{2.712330in}{1.756535in}}%
\pgfpathlineto{\pgfqpoint{2.714445in}{1.755725in}}%
\pgfpathlineto{\pgfqpoint{2.716559in}{1.758892in}}%
\pgfpathlineto{\pgfqpoint{2.718674in}{1.755107in}}%
\pgfpathlineto{\pgfqpoint{2.720789in}{1.760742in}}%
\pgfpathlineto{\pgfqpoint{2.725018in}{1.749602in}}%
\pgfpathlineto{\pgfqpoint{2.727133in}{1.748335in}}%
\pgfpathlineto{\pgfqpoint{2.731362in}{1.751140in}}%
\pgfpathlineto{\pgfqpoint{2.733477in}{1.749335in}}%
\pgfpathlineto{\pgfqpoint{2.735592in}{1.739364in}}%
\pgfpathlineto{\pgfqpoint{2.737706in}{1.739662in}}%
\pgfpathlineto{\pgfqpoint{2.739821in}{1.732517in}}%
\pgfpathlineto{\pgfqpoint{2.746165in}{1.734734in}}%
\pgfpathlineto{\pgfqpoint{2.752509in}{1.713878in}}%
\pgfpathlineto{\pgfqpoint{2.756739in}{1.717020in}}%
\pgfpathlineto{\pgfqpoint{2.758853in}{1.723619in}}%
\pgfpathlineto{\pgfqpoint{2.760968in}{1.726303in}}%
\pgfpathlineto{\pgfqpoint{2.763083in}{1.731941in}}%
\pgfpathlineto{\pgfqpoint{2.765197in}{1.733032in}}%
\pgfpathlineto{\pgfqpoint{2.769427in}{1.726066in}}%
\pgfpathlineto{\pgfqpoint{2.771541in}{1.726173in}}%
\pgfpathlineto{\pgfqpoint{2.773656in}{1.720078in}}%
\pgfpathlineto{\pgfqpoint{2.775771in}{1.730112in}}%
\pgfpathlineto{\pgfqpoint{2.777886in}{1.731082in}}%
\pgfpathlineto{\pgfqpoint{2.780000in}{1.740171in}}%
\pgfpathlineto{\pgfqpoint{2.782115in}{1.734469in}}%
\pgfpathlineto{\pgfqpoint{2.784230in}{1.732540in}}%
\pgfpathlineto{\pgfqpoint{2.786344in}{1.735163in}}%
\pgfpathlineto{\pgfqpoint{2.788459in}{1.730967in}}%
\pgfpathlineto{\pgfqpoint{2.790574in}{1.729432in}}%
\pgfpathlineto{\pgfqpoint{2.792688in}{1.730474in}}%
\pgfpathlineto{\pgfqpoint{2.794803in}{1.723803in}}%
\pgfpathlineto{\pgfqpoint{2.796918in}{1.728173in}}%
\pgfpathlineto{\pgfqpoint{2.805377in}{1.728479in}}%
\pgfpathlineto{\pgfqpoint{2.807491in}{1.731088in}}%
\pgfpathlineto{\pgfqpoint{2.809606in}{1.726765in}}%
\pgfpathlineto{\pgfqpoint{2.811721in}{1.736360in}}%
\pgfpathlineto{\pgfqpoint{2.813835in}{1.737107in}}%
\pgfpathlineto{\pgfqpoint{2.820179in}{1.743831in}}%
\pgfpathlineto{\pgfqpoint{2.822294in}{1.738321in}}%
\pgfpathlineto{\pgfqpoint{2.824409in}{1.741879in}}%
\pgfpathlineto{\pgfqpoint{2.826523in}{1.742762in}}%
\pgfpathlineto{\pgfqpoint{2.828638in}{1.741613in}}%
\pgfpathlineto{\pgfqpoint{2.830753in}{1.745056in}}%
\pgfpathlineto{\pgfqpoint{2.832868in}{1.741101in}}%
\pgfpathlineto{\pgfqpoint{2.834982in}{1.741903in}}%
\pgfpathlineto{\pgfqpoint{2.839212in}{1.736413in}}%
\pgfpathlineto{\pgfqpoint{2.841326in}{1.730381in}}%
\pgfpathlineto{\pgfqpoint{2.843441in}{1.731427in}}%
\pgfpathlineto{\pgfqpoint{2.847670in}{1.722559in}}%
\pgfpathlineto{\pgfqpoint{2.849785in}{1.726882in}}%
\pgfpathlineto{\pgfqpoint{2.854015in}{1.721924in}}%
\pgfpathlineto{\pgfqpoint{2.856129in}{1.722391in}}%
\pgfpathlineto{\pgfqpoint{2.860359in}{1.730133in}}%
\pgfpathlineto{\pgfqpoint{2.862473in}{1.726320in}}%
\pgfpathlineto{\pgfqpoint{2.866703in}{1.728183in}}%
\pgfpathlineto{\pgfqpoint{2.868817in}{1.731279in}}%
\pgfpathlineto{\pgfqpoint{2.870932in}{1.737036in}}%
\pgfpathlineto{\pgfqpoint{2.873047in}{1.731655in}}%
\pgfpathlineto{\pgfqpoint{2.875161in}{1.729898in}}%
\pgfpathlineto{\pgfqpoint{2.877276in}{1.729921in}}%
\pgfpathlineto{\pgfqpoint{2.883620in}{1.710677in}}%
\pgfpathlineto{\pgfqpoint{2.885735in}{1.707523in}}%
\pgfpathlineto{\pgfqpoint{2.887850in}{1.701281in}}%
\pgfpathlineto{\pgfqpoint{2.889964in}{1.708081in}}%
\pgfpathlineto{\pgfqpoint{2.894194in}{1.711893in}}%
\pgfpathlineto{\pgfqpoint{2.898423in}{1.722952in}}%
\pgfpathlineto{\pgfqpoint{2.900538in}{1.716826in}}%
\pgfpathlineto{\pgfqpoint{2.902652in}{1.721071in}}%
\pgfpathlineto{\pgfqpoint{2.904767in}{1.722249in}}%
\pgfpathlineto{\pgfqpoint{2.906882in}{1.729213in}}%
\pgfpathlineto{\pgfqpoint{2.908997in}{1.730293in}}%
\pgfpathlineto{\pgfqpoint{2.911111in}{1.733688in}}%
\pgfpathlineto{\pgfqpoint{2.913226in}{1.741464in}}%
\pgfpathlineto{\pgfqpoint{2.915341in}{1.743580in}}%
\pgfpathlineto{\pgfqpoint{2.917455in}{1.743133in}}%
\pgfpathlineto{\pgfqpoint{2.919570in}{1.744394in}}%
\pgfpathlineto{\pgfqpoint{2.921685in}{1.742044in}}%
\pgfpathlineto{\pgfqpoint{2.923799in}{1.741774in}}%
\pgfpathlineto{\pgfqpoint{2.925914in}{1.738301in}}%
\pgfpathlineto{\pgfqpoint{2.930143in}{1.746056in}}%
\pgfpathlineto{\pgfqpoint{2.936488in}{1.746600in}}%
\pgfpathlineto{\pgfqpoint{2.938602in}{1.744004in}}%
\pgfpathlineto{\pgfqpoint{2.942832in}{1.749678in}}%
\pgfpathlineto{\pgfqpoint{2.947061in}{1.752126in}}%
\pgfpathlineto{\pgfqpoint{2.949176in}{1.756368in}}%
\pgfpathlineto{\pgfqpoint{2.951290in}{1.755526in}}%
\pgfpathlineto{\pgfqpoint{2.953405in}{1.749282in}}%
\pgfpathlineto{\pgfqpoint{2.955520in}{1.749717in}}%
\pgfpathlineto{\pgfqpoint{2.957634in}{1.740844in}}%
\pgfpathlineto{\pgfqpoint{2.961864in}{1.746843in}}%
\pgfpathlineto{\pgfqpoint{2.963979in}{1.752241in}}%
\pgfpathlineto{\pgfqpoint{2.966093in}{1.749621in}}%
\pgfpathlineto{\pgfqpoint{2.968208in}{1.753722in}}%
\pgfpathlineto{\pgfqpoint{2.974552in}{1.748010in}}%
\pgfpathlineto{\pgfqpoint{2.976667in}{1.742347in}}%
\pgfpathlineto{\pgfqpoint{2.978781in}{1.744743in}}%
\pgfpathlineto{\pgfqpoint{2.983011in}{1.737930in}}%
\pgfpathlineto{\pgfqpoint{2.989355in}{1.734894in}}%
\pgfpathlineto{\pgfqpoint{2.993584in}{1.743370in}}%
\pgfpathlineto{\pgfqpoint{2.995699in}{1.745880in}}%
\pgfpathlineto{\pgfqpoint{2.997814in}{1.752339in}}%
\pgfpathlineto{\pgfqpoint{2.999928in}{1.755092in}}%
\pgfpathlineto{\pgfqpoint{3.002043in}{1.755513in}}%
\pgfpathlineto{\pgfqpoint{3.006272in}{1.762229in}}%
\pgfpathlineto{\pgfqpoint{3.008387in}{1.756362in}}%
\pgfpathlineto{\pgfqpoint{3.010502in}{1.758962in}}%
\pgfpathlineto{\pgfqpoint{3.014731in}{1.754062in}}%
\pgfpathlineto{\pgfqpoint{3.016846in}{1.756820in}}%
\pgfpathlineto{\pgfqpoint{3.018961in}{1.757214in}}%
\pgfpathlineto{\pgfqpoint{3.021075in}{1.754638in}}%
\pgfpathlineto{\pgfqpoint{3.023190in}{1.753888in}}%
\pgfpathlineto{\pgfqpoint{3.025305in}{1.749162in}}%
\pgfpathlineto{\pgfqpoint{3.027419in}{1.748414in}}%
\pgfpathlineto{\pgfqpoint{3.031649in}{1.758524in}}%
\pgfpathlineto{\pgfqpoint{3.033763in}{1.762177in}}%
\pgfpathlineto{\pgfqpoint{3.035878in}{1.762873in}}%
\pgfpathlineto{\pgfqpoint{3.044337in}{1.770727in}}%
\pgfpathlineto{\pgfqpoint{3.046452in}{1.771113in}}%
\pgfpathlineto{\pgfqpoint{3.048566in}{1.773323in}}%
\pgfpathlineto{\pgfqpoint{3.050681in}{1.767043in}}%
\pgfpathlineto{\pgfqpoint{3.052796in}{1.764450in}}%
\pgfpathlineto{\pgfqpoint{3.054910in}{1.765836in}}%
\pgfpathlineto{\pgfqpoint{3.059140in}{1.759059in}}%
\pgfpathlineto{\pgfqpoint{3.061254in}{1.763221in}}%
\pgfpathlineto{\pgfqpoint{3.065484in}{1.756898in}}%
\pgfpathlineto{\pgfqpoint{3.069713in}{1.755030in}}%
\pgfpathlineto{\pgfqpoint{3.073943in}{1.760549in}}%
\pgfpathlineto{\pgfqpoint{3.078172in}{1.759896in}}%
\pgfpathlineto{\pgfqpoint{3.080287in}{1.754606in}}%
\pgfpathlineto{\pgfqpoint{3.082401in}{1.760828in}}%
\pgfpathlineto{\pgfqpoint{3.084516in}{1.760747in}}%
\pgfpathlineto{\pgfqpoint{3.088746in}{1.757840in}}%
\pgfpathlineto{\pgfqpoint{3.090860in}{1.752785in}}%
\pgfpathlineto{\pgfqpoint{3.092975in}{1.751840in}}%
\pgfpathlineto{\pgfqpoint{3.095090in}{1.749437in}}%
\pgfpathlineto{\pgfqpoint{3.097204in}{1.754038in}}%
\pgfpathlineto{\pgfqpoint{3.101434in}{1.750614in}}%
\pgfpathlineto{\pgfqpoint{3.103548in}{1.745906in}}%
\pgfpathlineto{\pgfqpoint{3.105663in}{1.738057in}}%
\pgfpathlineto{\pgfqpoint{3.107778in}{1.741859in}}%
\pgfpathlineto{\pgfqpoint{3.109892in}{1.742840in}}%
\pgfpathlineto{\pgfqpoint{3.112007in}{1.748144in}}%
\pgfpathlineto{\pgfqpoint{3.114122in}{1.750372in}}%
\pgfpathlineto{\pgfqpoint{3.116237in}{1.750756in}}%
\pgfpathlineto{\pgfqpoint{3.120466in}{1.760378in}}%
\pgfpathlineto{\pgfqpoint{3.122581in}{1.764989in}}%
\pgfpathlineto{\pgfqpoint{3.124695in}{1.755946in}}%
\pgfpathlineto{\pgfqpoint{3.128925in}{1.769298in}}%
\pgfpathlineto{\pgfqpoint{3.131039in}{1.770489in}}%
\pgfpathlineto{\pgfqpoint{3.133154in}{1.770129in}}%
\pgfpathlineto{\pgfqpoint{3.135269in}{1.763573in}}%
\pgfpathlineto{\pgfqpoint{3.139498in}{1.765697in}}%
\pgfpathlineto{\pgfqpoint{3.141613in}{1.763753in}}%
\pgfpathlineto{\pgfqpoint{3.145842in}{1.765471in}}%
\pgfpathlineto{\pgfqpoint{3.147957in}{1.763421in}}%
\pgfpathlineto{\pgfqpoint{3.150072in}{1.767847in}}%
\pgfpathlineto{\pgfqpoint{3.154301in}{1.759329in}}%
\pgfpathlineto{\pgfqpoint{3.156416in}{1.760219in}}%
\pgfpathlineto{\pgfqpoint{3.158530in}{1.759304in}}%
\pgfpathlineto{\pgfqpoint{3.164874in}{1.769596in}}%
\pgfpathlineto{\pgfqpoint{3.171219in}{1.759102in}}%
\pgfpathlineto{\pgfqpoint{3.173333in}{1.759160in}}%
\pgfpathlineto{\pgfqpoint{3.177563in}{1.747684in}}%
\pgfpathlineto{\pgfqpoint{3.181792in}{1.750773in}}%
\pgfpathlineto{\pgfqpoint{3.183907in}{1.744647in}}%
\pgfpathlineto{\pgfqpoint{3.188136in}{1.743911in}}%
\pgfpathlineto{\pgfqpoint{3.198710in}{1.751686in}}%
\pgfpathlineto{\pgfqpoint{3.205054in}{1.751584in}}%
\pgfpathlineto{\pgfqpoint{3.207168in}{1.752779in}}%
\pgfpathlineto{\pgfqpoint{3.209283in}{1.752021in}}%
\pgfpathlineto{\pgfqpoint{3.213512in}{1.761746in}}%
\pgfpathlineto{\pgfqpoint{3.217742in}{1.766171in}}%
\pgfpathlineto{\pgfqpoint{3.219857in}{1.760165in}}%
\pgfpathlineto{\pgfqpoint{3.221971in}{1.764590in}}%
\pgfpathlineto{\pgfqpoint{3.224086in}{1.763143in}}%
\pgfpathlineto{\pgfqpoint{3.226201in}{1.765676in}}%
\pgfpathlineto{\pgfqpoint{3.230430in}{1.773312in}}%
\pgfpathlineto{\pgfqpoint{3.232545in}{1.771019in}}%
\pgfpathlineto{\pgfqpoint{3.234659in}{1.775954in}}%
\pgfpathlineto{\pgfqpoint{3.236774in}{1.777926in}}%
\pgfpathlineto{\pgfqpoint{3.238889in}{1.781529in}}%
\pgfpathlineto{\pgfqpoint{3.241003in}{1.779423in}}%
\pgfpathlineto{\pgfqpoint{3.243118in}{1.780025in}}%
\pgfpathlineto{\pgfqpoint{3.245233in}{1.782636in}}%
\pgfpathlineto{\pgfqpoint{3.247348in}{1.788826in}}%
\pgfpathlineto{\pgfqpoint{3.251577in}{1.789866in}}%
\pgfpathlineto{\pgfqpoint{3.253692in}{1.790365in}}%
\pgfpathlineto{\pgfqpoint{3.255806in}{1.788074in}}%
\pgfpathlineto{\pgfqpoint{3.257921in}{1.783497in}}%
\pgfpathlineto{\pgfqpoint{3.260036in}{1.790260in}}%
\pgfpathlineto{\pgfqpoint{3.262150in}{1.792170in}}%
\pgfpathlineto{\pgfqpoint{3.264265in}{1.787278in}}%
\pgfpathlineto{\pgfqpoint{3.266380in}{1.791525in}}%
\pgfpathlineto{\pgfqpoint{3.268494in}{1.791570in}}%
\pgfpathlineto{\pgfqpoint{3.270609in}{1.789868in}}%
\pgfpathlineto{\pgfqpoint{3.272724in}{1.780611in}}%
\pgfpathlineto{\pgfqpoint{3.274839in}{1.777832in}}%
\pgfpathlineto{\pgfqpoint{3.276953in}{1.782345in}}%
\pgfpathlineto{\pgfqpoint{3.279068in}{1.790029in}}%
\pgfpathlineto{\pgfqpoint{3.283297in}{1.780654in}}%
\pgfpathlineto{\pgfqpoint{3.285412in}{1.785440in}}%
\pgfpathlineto{\pgfqpoint{3.287527in}{1.784480in}}%
\pgfpathlineto{\pgfqpoint{3.291756in}{1.771192in}}%
\pgfpathlineto{\pgfqpoint{3.293871in}{1.766259in}}%
\pgfpathlineto{\pgfqpoint{3.295985in}{1.769713in}}%
\pgfpathlineto{\pgfqpoint{3.298100in}{1.767390in}}%
\pgfpathlineto{\pgfqpoint{3.300215in}{1.760412in}}%
\pgfpathlineto{\pgfqpoint{3.304444in}{1.760692in}}%
\pgfpathlineto{\pgfqpoint{3.306559in}{1.763583in}}%
\pgfpathlineto{\pgfqpoint{3.308674in}{1.755727in}}%
\pgfpathlineto{\pgfqpoint{3.310788in}{1.764811in}}%
\pgfpathlineto{\pgfqpoint{3.312903in}{1.761120in}}%
\pgfpathlineto{\pgfqpoint{3.315018in}{1.760645in}}%
\pgfpathlineto{\pgfqpoint{3.317132in}{1.762345in}}%
\pgfpathlineto{\pgfqpoint{3.321362in}{1.762879in}}%
\pgfpathlineto{\pgfqpoint{3.325591in}{1.768092in}}%
\pgfpathlineto{\pgfqpoint{3.327706in}{1.767122in}}%
\pgfpathlineto{\pgfqpoint{3.329821in}{1.763213in}}%
\pgfpathlineto{\pgfqpoint{3.331935in}{1.762341in}}%
\pgfpathlineto{\pgfqpoint{3.334050in}{1.757131in}}%
\pgfpathlineto{\pgfqpoint{3.336165in}{1.761925in}}%
\pgfpathlineto{\pgfqpoint{3.338279in}{1.762851in}}%
\pgfpathlineto{\pgfqpoint{3.340394in}{1.759537in}}%
\pgfpathlineto{\pgfqpoint{3.344623in}{1.757621in}}%
\pgfpathlineto{\pgfqpoint{3.346738in}{1.753251in}}%
\pgfpathlineto{\pgfqpoint{3.348853in}{1.754131in}}%
\pgfpathlineto{\pgfqpoint{3.350968in}{1.748133in}}%
\pgfpathlineto{\pgfqpoint{3.353082in}{1.748418in}}%
\pgfpathlineto{\pgfqpoint{3.355197in}{1.750224in}}%
\pgfpathlineto{\pgfqpoint{3.359426in}{1.742649in}}%
\pgfpathlineto{\pgfqpoint{3.363656in}{1.749136in}}%
\pgfpathlineto{\pgfqpoint{3.365770in}{1.757254in}}%
\pgfpathlineto{\pgfqpoint{3.367885in}{1.754447in}}%
\pgfpathlineto{\pgfqpoint{3.370000in}{1.758525in}}%
\pgfpathlineto{\pgfqpoint{3.372114in}{1.765631in}}%
\pgfpathlineto{\pgfqpoint{3.374229in}{1.765930in}}%
\pgfpathlineto{\pgfqpoint{3.376344in}{1.767678in}}%
\pgfpathlineto{\pgfqpoint{3.378459in}{1.761775in}}%
\pgfpathlineto{\pgfqpoint{3.380573in}{1.764997in}}%
\pgfpathlineto{\pgfqpoint{3.382688in}{1.764502in}}%
\pgfpathlineto{\pgfqpoint{3.384803in}{1.769652in}}%
\pgfpathlineto{\pgfqpoint{3.386917in}{1.768218in}}%
\pgfpathlineto{\pgfqpoint{3.389032in}{1.763207in}}%
\pgfpathlineto{\pgfqpoint{3.391147in}{1.763642in}}%
\pgfpathlineto{\pgfqpoint{3.393261in}{1.765840in}}%
\pgfpathlineto{\pgfqpoint{3.395376in}{1.760163in}}%
\pgfpathlineto{\pgfqpoint{3.397491in}{1.758153in}}%
\pgfpathlineto{\pgfqpoint{3.399605in}{1.765473in}}%
\pgfpathlineto{\pgfqpoint{3.401720in}{1.763724in}}%
\pgfpathlineto{\pgfqpoint{3.405950in}{1.766931in}}%
\pgfpathlineto{\pgfqpoint{3.408064in}{1.764647in}}%
\pgfpathlineto{\pgfqpoint{3.410179in}{1.764736in}}%
\pgfpathlineto{\pgfqpoint{3.414408in}{1.761056in}}%
\pgfpathlineto{\pgfqpoint{3.416523in}{1.774566in}}%
\pgfpathlineto{\pgfqpoint{3.418638in}{1.769994in}}%
\pgfpathlineto{\pgfqpoint{3.420752in}{1.769451in}}%
\pgfpathlineto{\pgfqpoint{3.422867in}{1.772499in}}%
\pgfpathlineto{\pgfqpoint{3.424982in}{1.772740in}}%
\pgfpathlineto{\pgfqpoint{3.431326in}{1.789658in}}%
\pgfpathlineto{\pgfqpoint{3.433441in}{1.790605in}}%
\pgfpathlineto{\pgfqpoint{3.435555in}{1.789049in}}%
\pgfpathlineto{\pgfqpoint{3.439785in}{1.792345in}}%
\pgfpathlineto{\pgfqpoint{3.444014in}{1.786842in}}%
\pgfpathlineto{\pgfqpoint{3.446129in}{1.788594in}}%
\pgfpathlineto{\pgfqpoint{3.448243in}{1.785641in}}%
\pgfpathlineto{\pgfqpoint{3.450358in}{1.791961in}}%
\pgfpathlineto{\pgfqpoint{3.452473in}{1.791902in}}%
\pgfpathlineto{\pgfqpoint{3.454588in}{1.789176in}}%
\pgfpathlineto{\pgfqpoint{3.456702in}{1.792020in}}%
\pgfpathlineto{\pgfqpoint{3.458817in}{1.787346in}}%
\pgfpathlineto{\pgfqpoint{3.460932in}{1.794370in}}%
\pgfpathlineto{\pgfqpoint{3.463046in}{1.787302in}}%
\pgfpathlineto{\pgfqpoint{3.467276in}{1.792529in}}%
\pgfpathlineto{\pgfqpoint{3.469390in}{1.794519in}}%
\pgfpathlineto{\pgfqpoint{3.479964in}{1.788081in}}%
\pgfpathlineto{\pgfqpoint{3.482079in}{1.795075in}}%
\pgfpathlineto{\pgfqpoint{3.484193in}{1.792471in}}%
\pgfpathlineto{\pgfqpoint{3.488423in}{1.805152in}}%
\pgfpathlineto{\pgfqpoint{3.492652in}{1.810681in}}%
\pgfpathlineto{\pgfqpoint{3.494767in}{1.809929in}}%
\pgfpathlineto{\pgfqpoint{3.498996in}{1.814649in}}%
\pgfpathlineto{\pgfqpoint{3.501111in}{1.811576in}}%
\pgfpathlineto{\pgfqpoint{3.503225in}{1.804545in}}%
\pgfpathlineto{\pgfqpoint{3.505340in}{1.801548in}}%
\pgfpathlineto{\pgfqpoint{3.507455in}{1.803026in}}%
\pgfpathlineto{\pgfqpoint{3.509570in}{1.809026in}}%
\pgfpathlineto{\pgfqpoint{3.511684in}{1.820730in}}%
\pgfpathlineto{\pgfqpoint{3.515914in}{1.816768in}}%
\pgfpathlineto{\pgfqpoint{3.518028in}{1.818984in}}%
\pgfpathlineto{\pgfqpoint{3.520143in}{1.814741in}}%
\pgfpathlineto{\pgfqpoint{3.522258in}{1.813932in}}%
\pgfpathlineto{\pgfqpoint{3.524372in}{1.811654in}}%
\pgfpathlineto{\pgfqpoint{3.526487in}{1.813212in}}%
\pgfpathlineto{\pgfqpoint{3.528602in}{1.810358in}}%
\pgfpathlineto{\pgfqpoint{3.530716in}{1.811444in}}%
\pgfpathlineto{\pgfqpoint{3.532831in}{1.810083in}}%
\pgfpathlineto{\pgfqpoint{3.539175in}{1.815069in}}%
\pgfpathlineto{\pgfqpoint{3.543405in}{1.810804in}}%
\pgfpathlineto{\pgfqpoint{3.549749in}{1.827900in}}%
\pgfpathlineto{\pgfqpoint{3.553978in}{1.828552in}}%
\pgfpathlineto{\pgfqpoint{3.556093in}{1.829775in}}%
\pgfpathlineto{\pgfqpoint{3.560322in}{1.825197in}}%
\pgfpathlineto{\pgfqpoint{3.568781in}{1.817370in}}%
\pgfpathlineto{\pgfqpoint{3.570896in}{1.814350in}}%
\pgfpathlineto{\pgfqpoint{3.573010in}{1.817429in}}%
\pgfpathlineto{\pgfqpoint{3.575125in}{1.815581in}}%
\pgfpathlineto{\pgfqpoint{3.577240in}{1.818460in}}%
\pgfpathlineto{\pgfqpoint{3.581469in}{1.807531in}}%
\pgfpathlineto{\pgfqpoint{3.583584in}{1.813967in}}%
\pgfpathlineto{\pgfqpoint{3.585699in}{1.812119in}}%
\pgfpathlineto{\pgfqpoint{3.587813in}{1.818562in}}%
\pgfpathlineto{\pgfqpoint{3.589928in}{1.814616in}}%
\pgfpathlineto{\pgfqpoint{3.592043in}{1.814421in}}%
\pgfpathlineto{\pgfqpoint{3.596272in}{1.821914in}}%
\pgfpathlineto{\pgfqpoint{3.598387in}{1.817788in}}%
\pgfpathlineto{\pgfqpoint{3.600501in}{1.817660in}}%
\pgfpathlineto{\pgfqpoint{3.604731in}{1.821359in}}%
\pgfpathlineto{\pgfqpoint{3.608960in}{1.832022in}}%
\pgfpathlineto{\pgfqpoint{3.613190in}{1.837410in}}%
\pgfpathlineto{\pgfqpoint{3.615304in}{1.837014in}}%
\pgfpathlineto{\pgfqpoint{3.617419in}{1.838542in}}%
\pgfpathlineto{\pgfqpoint{3.619534in}{1.838624in}}%
\pgfpathlineto{\pgfqpoint{3.621648in}{1.844060in}}%
\pgfpathlineto{\pgfqpoint{3.623763in}{1.845119in}}%
\pgfpathlineto{\pgfqpoint{3.625878in}{1.841332in}}%
\pgfpathlineto{\pgfqpoint{3.630107in}{1.846078in}}%
\pgfpathlineto{\pgfqpoint{3.632222in}{1.850623in}}%
\pgfpathlineto{\pgfqpoint{3.636451in}{1.845242in}}%
\pgfpathlineto{\pgfqpoint{3.638566in}{1.858879in}}%
\pgfpathlineto{\pgfqpoint{3.640681in}{1.858812in}}%
\pgfpathlineto{\pgfqpoint{3.642795in}{1.856149in}}%
\pgfpathlineto{\pgfqpoint{3.649139in}{1.855509in}}%
\pgfpathlineto{\pgfqpoint{3.651254in}{1.855880in}}%
\pgfpathlineto{\pgfqpoint{3.653369in}{1.858349in}}%
\pgfpathlineto{\pgfqpoint{3.657598in}{1.855796in}}%
\pgfpathlineto{\pgfqpoint{3.663942in}{1.842251in}}%
\pgfpathlineto{\pgfqpoint{3.666057in}{1.837328in}}%
\pgfpathlineto{\pgfqpoint{3.668172in}{1.838905in}}%
\pgfpathlineto{\pgfqpoint{3.670286in}{1.839000in}}%
\pgfpathlineto{\pgfqpoint{3.674516in}{1.832318in}}%
\pgfpathlineto{\pgfqpoint{3.676630in}{1.834662in}}%
\pgfpathlineto{\pgfqpoint{3.678745in}{1.833421in}}%
\pgfpathlineto{\pgfqpoint{3.680860in}{1.836137in}}%
\pgfpathlineto{\pgfqpoint{3.682974in}{1.836399in}}%
\pgfpathlineto{\pgfqpoint{3.687204in}{1.840021in}}%
\pgfpathlineto{\pgfqpoint{3.689319in}{1.832116in}}%
\pgfpathlineto{\pgfqpoint{3.693548in}{1.835576in}}%
\pgfpathlineto{\pgfqpoint{3.695663in}{1.846288in}}%
\pgfpathlineto{\pgfqpoint{3.697777in}{1.849034in}}%
\pgfpathlineto{\pgfqpoint{3.699892in}{1.849621in}}%
\pgfpathlineto{\pgfqpoint{3.704121in}{1.844362in}}%
\pgfpathlineto{\pgfqpoint{3.706236in}{1.845824in}}%
\pgfpathlineto{\pgfqpoint{3.708351in}{1.841637in}}%
\pgfpathlineto{\pgfqpoint{3.712580in}{1.843988in}}%
\pgfpathlineto{\pgfqpoint{3.714695in}{1.840067in}}%
\pgfpathlineto{\pgfqpoint{3.725268in}{1.856874in}}%
\pgfpathlineto{\pgfqpoint{3.727383in}{1.855832in}}%
\pgfpathlineto{\pgfqpoint{3.729498in}{1.858256in}}%
\pgfpathlineto{\pgfqpoint{3.731612in}{1.855816in}}%
\pgfpathlineto{\pgfqpoint{3.733727in}{1.849208in}}%
\pgfpathlineto{\pgfqpoint{3.735842in}{1.853049in}}%
\pgfpathlineto{\pgfqpoint{3.737956in}{1.853923in}}%
\pgfpathlineto{\pgfqpoint{3.746415in}{1.870072in}}%
\pgfpathlineto{\pgfqpoint{3.748530in}{1.882238in}}%
\pgfpathlineto{\pgfqpoint{3.750645in}{1.881783in}}%
\pgfpathlineto{\pgfqpoint{3.752759in}{1.883139in}}%
\pgfpathlineto{\pgfqpoint{3.756989in}{1.889798in}}%
\pgfpathlineto{\pgfqpoint{3.759103in}{1.890118in}}%
\pgfpathlineto{\pgfqpoint{3.761218in}{1.895044in}}%
\pgfpathlineto{\pgfqpoint{3.763333in}{1.893828in}}%
\pgfpathlineto{\pgfqpoint{3.769677in}{1.903759in}}%
\pgfpathlineto{\pgfqpoint{3.771792in}{1.904994in}}%
\pgfpathlineto{\pgfqpoint{3.776021in}{1.919504in}}%
\pgfpathlineto{\pgfqpoint{3.778136in}{1.911573in}}%
\pgfpathlineto{\pgfqpoint{3.782365in}{1.921018in}}%
\pgfpathlineto{\pgfqpoint{3.784480in}{1.909751in}}%
\pgfpathlineto{\pgfqpoint{3.786594in}{1.911088in}}%
\pgfpathlineto{\pgfqpoint{3.788709in}{1.916049in}}%
\pgfpathlineto{\pgfqpoint{3.792939in}{1.906742in}}%
\pgfpathlineto{\pgfqpoint{3.795053in}{1.907177in}}%
\pgfpathlineto{\pgfqpoint{3.797168in}{1.913792in}}%
\pgfpathlineto{\pgfqpoint{3.799283in}{1.916035in}}%
\pgfpathlineto{\pgfqpoint{3.801397in}{1.922352in}}%
\pgfpathlineto{\pgfqpoint{3.803512in}{1.917697in}}%
\pgfpathlineto{\pgfqpoint{3.805627in}{1.920151in}}%
\pgfpathlineto{\pgfqpoint{3.807741in}{1.915870in}}%
\pgfpathlineto{\pgfqpoint{3.809856in}{1.918461in}}%
\pgfpathlineto{\pgfqpoint{3.814085in}{1.916425in}}%
\pgfpathlineto{\pgfqpoint{3.818315in}{1.925690in}}%
\pgfpathlineto{\pgfqpoint{3.820430in}{1.921539in}}%
\pgfpathlineto{\pgfqpoint{3.822544in}{1.924449in}}%
\pgfpathlineto{\pgfqpoint{3.826774in}{1.922164in}}%
\pgfpathlineto{\pgfqpoint{3.828888in}{1.917352in}}%
\pgfpathlineto{\pgfqpoint{3.831003in}{1.909548in}}%
\pgfpathlineto{\pgfqpoint{3.835232in}{1.917733in}}%
\pgfpathlineto{\pgfqpoint{3.837347in}{1.916039in}}%
\pgfpathlineto{\pgfqpoint{3.839462in}{1.911672in}}%
\pgfpathlineto{\pgfqpoint{3.841576in}{1.920284in}}%
\pgfpathlineto{\pgfqpoint{3.843691in}{1.923869in}}%
\pgfpathlineto{\pgfqpoint{3.845806in}{1.925200in}}%
\pgfpathlineto{\pgfqpoint{3.847921in}{1.934771in}}%
\pgfpathlineto{\pgfqpoint{3.850035in}{1.935920in}}%
\pgfpathlineto{\pgfqpoint{3.852150in}{1.939342in}}%
\pgfpathlineto{\pgfqpoint{3.854265in}{1.935955in}}%
\pgfpathlineto{\pgfqpoint{3.856379in}{1.944288in}}%
\pgfpathlineto{\pgfqpoint{3.858494in}{1.938392in}}%
\pgfpathlineto{\pgfqpoint{3.860609in}{1.928184in}}%
\pgfpathlineto{\pgfqpoint{3.866953in}{1.925764in}}%
\pgfpathlineto{\pgfqpoint{3.869067in}{1.927085in}}%
\pgfpathlineto{\pgfqpoint{3.871182in}{1.924822in}}%
\pgfpathlineto{\pgfqpoint{3.873297in}{1.930484in}}%
\pgfpathlineto{\pgfqpoint{3.877526in}{1.928196in}}%
\pgfpathlineto{\pgfqpoint{3.881756in}{1.930600in}}%
\pgfpathlineto{\pgfqpoint{3.885985in}{1.918457in}}%
\pgfpathlineto{\pgfqpoint{3.888100in}{1.919522in}}%
\pgfpathlineto{\pgfqpoint{3.890214in}{1.917064in}}%
\pgfpathlineto{\pgfqpoint{3.892329in}{1.920258in}}%
\pgfpathlineto{\pgfqpoint{3.894444in}{1.920676in}}%
\pgfpathlineto{\pgfqpoint{3.900788in}{1.914113in}}%
\pgfpathlineto{\pgfqpoint{3.902903in}{1.914018in}}%
\pgfpathlineto{\pgfqpoint{3.905017in}{1.912093in}}%
\pgfpathlineto{\pgfqpoint{3.911361in}{1.928697in}}%
\pgfpathlineto{\pgfqpoint{3.913476in}{1.921643in}}%
\pgfpathlineto{\pgfqpoint{3.915591in}{1.925722in}}%
\pgfpathlineto{\pgfqpoint{3.917705in}{1.927132in}}%
\pgfpathlineto{\pgfqpoint{3.921935in}{1.924979in}}%
\pgfpathlineto{\pgfqpoint{3.924050in}{1.929351in}}%
\pgfpathlineto{\pgfqpoint{3.926164in}{1.919540in}}%
\pgfpathlineto{\pgfqpoint{3.930394in}{1.934272in}}%
\pgfpathlineto{\pgfqpoint{3.932508in}{1.934120in}}%
\pgfpathlineto{\pgfqpoint{3.936738in}{1.929285in}}%
\pgfpathlineto{\pgfqpoint{3.938852in}{1.926899in}}%
\pgfpathlineto{\pgfqpoint{3.943082in}{1.935806in}}%
\pgfpathlineto{\pgfqpoint{3.947311in}{1.933529in}}%
\pgfpathlineto{\pgfqpoint{3.949426in}{1.935297in}}%
\pgfpathlineto{\pgfqpoint{3.951541in}{1.931604in}}%
\pgfpathlineto{\pgfqpoint{3.953655in}{1.933668in}}%
\pgfpathlineto{\pgfqpoint{3.955770in}{1.933427in}}%
\pgfpathlineto{\pgfqpoint{3.964229in}{1.917336in}}%
\pgfpathlineto{\pgfqpoint{3.972687in}{1.931433in}}%
\pgfpathlineto{\pgfqpoint{3.974802in}{1.932041in}}%
\pgfpathlineto{\pgfqpoint{3.979032in}{1.943604in}}%
\pgfpathlineto{\pgfqpoint{3.985376in}{1.931140in}}%
\pgfpathlineto{\pgfqpoint{3.987490in}{1.929461in}}%
\pgfpathlineto{\pgfqpoint{3.989605in}{1.936690in}}%
\pgfpathlineto{\pgfqpoint{3.991720in}{1.939086in}}%
\pgfpathlineto{\pgfqpoint{3.995949in}{1.948332in}}%
\pgfpathlineto{\pgfqpoint{4.002293in}{1.944743in}}%
\pgfpathlineto{\pgfqpoint{4.004408in}{1.946730in}}%
\pgfpathlineto{\pgfqpoint{4.006523in}{1.943877in}}%
\pgfpathlineto{\pgfqpoint{4.008637in}{1.949847in}}%
\pgfpathlineto{\pgfqpoint{4.010752in}{1.947627in}}%
\pgfpathlineto{\pgfqpoint{4.012867in}{1.950677in}}%
\pgfpathlineto{\pgfqpoint{4.017096in}{1.946000in}}%
\pgfpathlineto{\pgfqpoint{4.019211in}{1.939079in}}%
\pgfpathlineto{\pgfqpoint{4.021325in}{1.939004in}}%
\pgfpathlineto{\pgfqpoint{4.023440in}{1.942351in}}%
\pgfpathlineto{\pgfqpoint{4.025555in}{1.942717in}}%
\pgfpathlineto{\pgfqpoint{4.027670in}{1.941759in}}%
\pgfpathlineto{\pgfqpoint{4.029784in}{1.938360in}}%
\pgfpathlineto{\pgfqpoint{4.031899in}{1.940107in}}%
\pgfpathlineto{\pgfqpoint{4.034014in}{1.936409in}}%
\pgfpathlineto{\pgfqpoint{4.036128in}{1.939134in}}%
\pgfpathlineto{\pgfqpoint{4.038243in}{1.938182in}}%
\pgfpathlineto{\pgfqpoint{4.040358in}{1.943772in}}%
\pgfpathlineto{\pgfqpoint{4.044587in}{1.941688in}}%
\pgfpathlineto{\pgfqpoint{4.048816in}{1.950803in}}%
\pgfpathlineto{\pgfqpoint{4.050931in}{1.957408in}}%
\pgfpathlineto{\pgfqpoint{4.055161in}{1.949267in}}%
\pgfpathlineto{\pgfqpoint{4.057275in}{1.953619in}}%
\pgfpathlineto{\pgfqpoint{4.059390in}{1.952200in}}%
\pgfpathlineto{\pgfqpoint{4.061505in}{1.946749in}}%
\pgfpathlineto{\pgfqpoint{4.063619in}{1.953388in}}%
\pgfpathlineto{\pgfqpoint{4.067849in}{1.951565in}}%
\pgfpathlineto{\pgfqpoint{4.069963in}{1.951649in}}%
\pgfpathlineto{\pgfqpoint{4.074193in}{1.937800in}}%
\pgfpathlineto{\pgfqpoint{4.076307in}{1.937678in}}%
\pgfpathlineto{\pgfqpoint{4.078422in}{1.930609in}}%
\pgfpathlineto{\pgfqpoint{4.080537in}{1.935550in}}%
\pgfpathlineto{\pgfqpoint{4.082652in}{1.936343in}}%
\pgfpathlineto{\pgfqpoint{4.084766in}{1.935717in}}%
\pgfpathlineto{\pgfqpoint{4.086881in}{1.931481in}}%
\pgfpathlineto{\pgfqpoint{4.088996in}{1.930478in}}%
\pgfpathlineto{\pgfqpoint{4.091110in}{1.928028in}}%
\pgfpathlineto{\pgfqpoint{4.095340in}{1.927140in}}%
\pgfpathlineto{\pgfqpoint{4.097454in}{1.929883in}}%
\pgfpathlineto{\pgfqpoint{4.099569in}{1.918657in}}%
\pgfpathlineto{\pgfqpoint{4.103798in}{1.924920in}}%
\pgfpathlineto{\pgfqpoint{4.105913in}{1.922725in}}%
\pgfpathlineto{\pgfqpoint{4.108028in}{1.925604in}}%
\pgfpathlineto{\pgfqpoint{4.110143in}{1.925020in}}%
\pgfpathlineto{\pgfqpoint{4.114372in}{1.919211in}}%
\pgfpathlineto{\pgfqpoint{4.116487in}{1.920537in}}%
\pgfpathlineto{\pgfqpoint{4.122831in}{1.931132in}}%
\pgfpathlineto{\pgfqpoint{4.124945in}{1.931383in}}%
\pgfpathlineto{\pgfqpoint{4.131290in}{1.940045in}}%
\pgfpathlineto{\pgfqpoint{4.133404in}{1.936537in}}%
\pgfpathlineto{\pgfqpoint{4.135519in}{1.941314in}}%
\pgfpathlineto{\pgfqpoint{4.137634in}{1.938407in}}%
\pgfpathlineto{\pgfqpoint{4.139748in}{1.940733in}}%
\pgfpathlineto{\pgfqpoint{4.141863in}{1.936035in}}%
\pgfpathlineto{\pgfqpoint{4.143978in}{1.940096in}}%
\pgfpathlineto{\pgfqpoint{4.148207in}{1.932932in}}%
\pgfpathlineto{\pgfqpoint{4.150322in}{1.937085in}}%
\pgfpathlineto{\pgfqpoint{4.152436in}{1.938746in}}%
\pgfpathlineto{\pgfqpoint{4.154551in}{1.946121in}}%
\pgfpathlineto{\pgfqpoint{4.156666in}{1.949037in}}%
\pgfpathlineto{\pgfqpoint{4.158781in}{1.944601in}}%
\pgfpathlineto{\pgfqpoint{4.163010in}{1.946916in}}%
\pgfpathlineto{\pgfqpoint{4.165125in}{1.944687in}}%
\pgfpathlineto{\pgfqpoint{4.167239in}{1.946389in}}%
\pgfpathlineto{\pgfqpoint{4.169354in}{1.944105in}}%
\pgfpathlineto{\pgfqpoint{4.171469in}{1.939654in}}%
\pgfpathlineto{\pgfqpoint{4.179927in}{1.942481in}}%
\pgfpathlineto{\pgfqpoint{4.182042in}{1.942859in}}%
\pgfpathlineto{\pgfqpoint{4.184157in}{1.950721in}}%
\pgfpathlineto{\pgfqpoint{4.186272in}{1.949303in}}%
\pgfpathlineto{\pgfqpoint{4.190501in}{1.952654in}}%
\pgfpathlineto{\pgfqpoint{4.192616in}{1.948360in}}%
\pgfpathlineto{\pgfqpoint{4.194730in}{1.947804in}}%
\pgfpathlineto{\pgfqpoint{4.196845in}{1.943811in}}%
\pgfpathlineto{\pgfqpoint{4.198960in}{1.945532in}}%
\pgfpathlineto{\pgfqpoint{4.203189in}{1.953865in}}%
\pgfpathlineto{\pgfqpoint{4.207418in}{1.947869in}}%
\pgfpathlineto{\pgfqpoint{4.209533in}{1.943139in}}%
\pgfpathlineto{\pgfqpoint{4.211648in}{1.946020in}}%
\pgfpathlineto{\pgfqpoint{4.213763in}{1.944142in}}%
\pgfpathlineto{\pgfqpoint{4.215877in}{1.944413in}}%
\pgfpathlineto{\pgfqpoint{4.217992in}{1.939577in}}%
\pgfpathlineto{\pgfqpoint{4.220107in}{1.938590in}}%
\pgfpathlineto{\pgfqpoint{4.222221in}{1.934841in}}%
\pgfpathlineto{\pgfqpoint{4.224336in}{1.941089in}}%
\pgfpathlineto{\pgfqpoint{4.226451in}{1.939931in}}%
\pgfpathlineto{\pgfqpoint{4.228565in}{1.935758in}}%
\pgfpathlineto{\pgfqpoint{4.230680in}{1.943228in}}%
\pgfpathlineto{\pgfqpoint{4.232795in}{1.938116in}}%
\pgfpathlineto{\pgfqpoint{4.234910in}{1.940026in}}%
\pgfpathlineto{\pgfqpoint{4.237024in}{1.933357in}}%
\pgfpathlineto{\pgfqpoint{4.239139in}{1.936920in}}%
\pgfpathlineto{\pgfqpoint{4.241254in}{1.937324in}}%
\pgfpathlineto{\pgfqpoint{4.247598in}{1.950541in}}%
\pgfpathlineto{\pgfqpoint{4.249712in}{1.949793in}}%
\pgfpathlineto{\pgfqpoint{4.251827in}{1.947625in}}%
\pgfpathlineto{\pgfqpoint{4.253942in}{1.950341in}}%
\pgfpathlineto{\pgfqpoint{4.256056in}{1.950835in}}%
\pgfpathlineto{\pgfqpoint{4.258171in}{1.953223in}}%
\pgfpathlineto{\pgfqpoint{4.260286in}{1.949464in}}%
\pgfpathlineto{\pgfqpoint{4.262401in}{1.951669in}}%
\pgfpathlineto{\pgfqpoint{4.266630in}{1.940158in}}%
\pgfpathlineto{\pgfqpoint{4.268745in}{1.942077in}}%
\pgfpathlineto{\pgfqpoint{4.270859in}{1.940178in}}%
\pgfpathlineto{\pgfqpoint{4.272974in}{1.936175in}}%
\pgfpathlineto{\pgfqpoint{4.275089in}{1.941363in}}%
\pgfpathlineto{\pgfqpoint{4.277203in}{1.941575in}}%
\pgfpathlineto{\pgfqpoint{4.279318in}{1.932788in}}%
\pgfpathlineto{\pgfqpoint{4.283547in}{1.939493in}}%
\pgfpathlineto{\pgfqpoint{4.287777in}{1.935629in}}%
\pgfpathlineto{\pgfqpoint{4.289892in}{1.941985in}}%
\pgfpathlineto{\pgfqpoint{4.292006in}{1.952754in}}%
\pgfpathlineto{\pgfqpoint{4.294121in}{1.952375in}}%
\pgfpathlineto{\pgfqpoint{4.296236in}{1.950319in}}%
\pgfpathlineto{\pgfqpoint{4.300465in}{1.957836in}}%
\pgfpathlineto{\pgfqpoint{4.302580in}{1.955722in}}%
\pgfpathlineto{\pgfqpoint{4.306809in}{1.944013in}}%
\pgfpathlineto{\pgfqpoint{4.308924in}{1.945855in}}%
\pgfpathlineto{\pgfqpoint{4.313153in}{1.946959in}}%
\pgfpathlineto{\pgfqpoint{4.315268in}{1.949029in}}%
\pgfpathlineto{\pgfqpoint{4.317383in}{1.958285in}}%
\pgfpathlineto{\pgfqpoint{4.321612in}{1.951180in}}%
\pgfpathlineto{\pgfqpoint{4.323727in}{1.949264in}}%
\pgfpathlineto{\pgfqpoint{4.325841in}{1.950280in}}%
\pgfpathlineto{\pgfqpoint{4.327956in}{1.948132in}}%
\pgfpathlineto{\pgfqpoint{4.330071in}{1.948798in}}%
\pgfpathlineto{\pgfqpoint{4.332185in}{1.945252in}}%
\pgfpathlineto{\pgfqpoint{4.334300in}{1.936763in}}%
\pgfpathlineto{\pgfqpoint{4.336415in}{1.936791in}}%
\pgfpathlineto{\pgfqpoint{4.340644in}{1.946897in}}%
\pgfpathlineto{\pgfqpoint{4.344874in}{1.946959in}}%
\pgfpathlineto{\pgfqpoint{4.346988in}{1.949516in}}%
\pgfpathlineto{\pgfqpoint{4.349103in}{1.956121in}}%
\pgfpathlineto{\pgfqpoint{4.351218in}{1.957943in}}%
\pgfpathlineto{\pgfqpoint{4.353332in}{1.961564in}}%
\pgfpathlineto{\pgfqpoint{4.357562in}{1.963738in}}%
\pgfpathlineto{\pgfqpoint{4.359676in}{1.969508in}}%
\pgfpathlineto{\pgfqpoint{4.361791in}{1.965710in}}%
\pgfpathlineto{\pgfqpoint{4.366021in}{1.976921in}}%
\pgfpathlineto{\pgfqpoint{4.368135in}{1.978111in}}%
\pgfpathlineto{\pgfqpoint{4.370250in}{1.982755in}}%
\pgfpathlineto{\pgfqpoint{4.372365in}{1.984361in}}%
\pgfpathlineto{\pgfqpoint{4.374479in}{1.988133in}}%
\pgfpathlineto{\pgfqpoint{4.376594in}{1.981570in}}%
\pgfpathlineto{\pgfqpoint{4.378709in}{1.980699in}}%
\pgfpathlineto{\pgfqpoint{4.380823in}{1.978471in}}%
\pgfpathlineto{\pgfqpoint{4.382938in}{1.986135in}}%
\pgfpathlineto{\pgfqpoint{4.385053in}{1.983748in}}%
\pgfpathlineto{\pgfqpoint{4.387167in}{1.987262in}}%
\pgfpathlineto{\pgfqpoint{4.389282in}{1.988009in}}%
\pgfpathlineto{\pgfqpoint{4.391397in}{1.986699in}}%
\pgfpathlineto{\pgfqpoint{4.393512in}{1.986740in}}%
\pgfpathlineto{\pgfqpoint{4.395626in}{1.988205in}}%
\pgfpathlineto{\pgfqpoint{4.397741in}{1.981857in}}%
\pgfpathlineto{\pgfqpoint{4.399856in}{1.982438in}}%
\pgfpathlineto{\pgfqpoint{4.404085in}{1.990777in}}%
\pgfpathlineto{\pgfqpoint{4.406200in}{1.989717in}}%
\pgfpathlineto{\pgfqpoint{4.408314in}{1.990961in}}%
\pgfpathlineto{\pgfqpoint{4.410429in}{1.989095in}}%
\pgfpathlineto{\pgfqpoint{4.412544in}{1.989228in}}%
\pgfpathlineto{\pgfqpoint{4.414658in}{1.991219in}}%
\pgfpathlineto{\pgfqpoint{4.418888in}{1.986096in}}%
\pgfpathlineto{\pgfqpoint{4.423117in}{1.981971in}}%
\pgfpathlineto{\pgfqpoint{4.427347in}{1.974493in}}%
\pgfpathlineto{\pgfqpoint{4.429461in}{1.975188in}}%
\pgfpathlineto{\pgfqpoint{4.431576in}{1.973425in}}%
\pgfpathlineto{\pgfqpoint{4.435805in}{1.963466in}}%
\pgfpathlineto{\pgfqpoint{4.442149in}{1.956312in}}%
\pgfpathlineto{\pgfqpoint{4.444264in}{1.960721in}}%
\pgfpathlineto{\pgfqpoint{4.448494in}{1.942747in}}%
\pgfpathlineto{\pgfqpoint{4.450608in}{1.946803in}}%
\pgfpathlineto{\pgfqpoint{4.454838in}{1.937901in}}%
\pgfpathlineto{\pgfqpoint{4.456952in}{1.943346in}}%
\pgfpathlineto{\pgfqpoint{4.459067in}{1.945712in}}%
\pgfpathlineto{\pgfqpoint{4.461182in}{1.941300in}}%
\pgfpathlineto{\pgfqpoint{4.463296in}{1.942226in}}%
\pgfpathlineto{\pgfqpoint{4.465411in}{1.938209in}}%
\pgfpathlineto{\pgfqpoint{4.469641in}{1.938233in}}%
\pgfpathlineto{\pgfqpoint{4.471755in}{1.930396in}}%
\pgfpathlineto{\pgfqpoint{4.475985in}{1.929875in}}%
\pgfpathlineto{\pgfqpoint{4.478099in}{1.937263in}}%
\pgfpathlineto{\pgfqpoint{4.480214in}{1.934772in}}%
\pgfpathlineto{\pgfqpoint{4.482329in}{1.928757in}}%
\pgfpathlineto{\pgfqpoint{4.484443in}{1.933690in}}%
\pgfpathlineto{\pgfqpoint{4.488673in}{1.938063in}}%
\pgfpathlineto{\pgfqpoint{4.490787in}{1.945496in}}%
\pgfpathlineto{\pgfqpoint{4.492902in}{1.940245in}}%
\pgfpathlineto{\pgfqpoint{4.495017in}{1.938240in}}%
\pgfpathlineto{\pgfqpoint{4.497132in}{1.949790in}}%
\pgfpathlineto{\pgfqpoint{4.499246in}{1.946082in}}%
\pgfpathlineto{\pgfqpoint{4.501361in}{1.939795in}}%
\pgfpathlineto{\pgfqpoint{4.503476in}{1.938051in}}%
\pgfpathlineto{\pgfqpoint{4.505590in}{1.939704in}}%
\pgfpathlineto{\pgfqpoint{4.509820in}{1.950783in}}%
\pgfpathlineto{\pgfqpoint{4.511934in}{1.950284in}}%
\pgfpathlineto{\pgfqpoint{4.516164in}{1.941923in}}%
\pgfpathlineto{\pgfqpoint{4.518278in}{1.946879in}}%
\pgfpathlineto{\pgfqpoint{4.520393in}{1.948245in}}%
\pgfpathlineto{\pgfqpoint{4.526737in}{1.964856in}}%
\pgfpathlineto{\pgfqpoint{4.533081in}{1.953396in}}%
\pgfpathlineto{\pgfqpoint{4.537311in}{1.958789in}}%
\pgfpathlineto{\pgfqpoint{4.539425in}{1.960172in}}%
\pgfpathlineto{\pgfqpoint{4.541540in}{1.964679in}}%
\pgfpathlineto{\pgfqpoint{4.543655in}{1.960153in}}%
\pgfpathlineto{\pgfqpoint{4.547884in}{1.956507in}}%
\pgfpathlineto{\pgfqpoint{4.552114in}{1.940388in}}%
\pgfpathlineto{\pgfqpoint{4.556343in}{1.946354in}}%
\pgfpathlineto{\pgfqpoint{4.560572in}{1.940977in}}%
\pgfpathlineto{\pgfqpoint{4.562687in}{1.935587in}}%
\pgfpathlineto{\pgfqpoint{4.564802in}{1.934930in}}%
\pgfpathlineto{\pgfqpoint{4.566916in}{1.935891in}}%
\pgfpathlineto{\pgfqpoint{4.569031in}{1.941780in}}%
\pgfpathlineto{\pgfqpoint{4.571146in}{1.943150in}}%
\pgfpathlineto{\pgfqpoint{4.573260in}{1.939309in}}%
\pgfpathlineto{\pgfqpoint{4.577490in}{1.941289in}}%
\pgfpathlineto{\pgfqpoint{4.579605in}{1.945385in}}%
\pgfpathlineto{\pgfqpoint{4.583834in}{1.945808in}}%
\pgfpathlineto{\pgfqpoint{4.585949in}{1.947425in}}%
\pgfpathlineto{\pgfqpoint{4.588063in}{1.950849in}}%
\pgfpathlineto{\pgfqpoint{4.590178in}{1.943498in}}%
\pgfpathlineto{\pgfqpoint{4.592293in}{1.943983in}}%
\pgfpathlineto{\pgfqpoint{4.594407in}{1.952639in}}%
\pgfpathlineto{\pgfqpoint{4.598637in}{1.958276in}}%
\pgfpathlineto{\pgfqpoint{4.602866in}{1.955204in}}%
\pgfpathlineto{\pgfqpoint{4.604981in}{1.952344in}}%
\pgfpathlineto{\pgfqpoint{4.607096in}{1.954615in}}%
\pgfpathlineto{\pgfqpoint{4.609210in}{1.950879in}}%
\pgfpathlineto{\pgfqpoint{4.613440in}{1.948622in}}%
\pgfpathlineto{\pgfqpoint{4.615554in}{1.948621in}}%
\pgfpathlineto{\pgfqpoint{4.617669in}{1.944165in}}%
\pgfpathlineto{\pgfqpoint{4.621898in}{1.943752in}}%
\pgfpathlineto{\pgfqpoint{4.624013in}{1.940422in}}%
\pgfpathlineto{\pgfqpoint{4.628243in}{1.940186in}}%
\pgfpathlineto{\pgfqpoint{4.630357in}{1.942431in}}%
\pgfpathlineto{\pgfqpoint{4.634587in}{1.963445in}}%
\pgfpathlineto{\pgfqpoint{4.636701in}{1.958804in}}%
\pgfpathlineto{\pgfqpoint{4.638816in}{1.961216in}}%
\pgfpathlineto{\pgfqpoint{4.645160in}{1.962343in}}%
\pgfpathlineto{\pgfqpoint{4.647275in}{1.960042in}}%
\pgfpathlineto{\pgfqpoint{4.651504in}{1.962236in}}%
\pgfpathlineto{\pgfqpoint{4.653619in}{1.958650in}}%
\pgfpathlineto{\pgfqpoint{4.655734in}{1.963957in}}%
\pgfpathlineto{\pgfqpoint{4.657848in}{1.961527in}}%
\pgfpathlineto{\pgfqpoint{4.659963in}{1.954865in}}%
\pgfpathlineto{\pgfqpoint{4.662078in}{1.953976in}}%
\pgfpathlineto{\pgfqpoint{4.664192in}{1.950796in}}%
\pgfpathlineto{\pgfqpoint{4.666307in}{1.940953in}}%
\pgfpathlineto{\pgfqpoint{4.672651in}{1.950001in}}%
\pgfpathlineto{\pgfqpoint{4.674766in}{1.947312in}}%
\pgfpathlineto{\pgfqpoint{4.676880in}{1.951949in}}%
\pgfpathlineto{\pgfqpoint{4.681110in}{1.949192in}}%
\pgfpathlineto{\pgfqpoint{4.683225in}{1.947081in}}%
\pgfpathlineto{\pgfqpoint{4.687454in}{1.955418in}}%
\pgfpathlineto{\pgfqpoint{4.689569in}{1.957182in}}%
\pgfpathlineto{\pgfqpoint{4.693798in}{1.951946in}}%
\pgfpathlineto{\pgfqpoint{4.698027in}{1.946390in}}%
\pgfpathlineto{\pgfqpoint{4.700142in}{1.950625in}}%
\pgfpathlineto{\pgfqpoint{4.702257in}{1.951138in}}%
\pgfpathlineto{\pgfqpoint{4.704372in}{1.950438in}}%
\pgfpathlineto{\pgfqpoint{4.706486in}{1.955613in}}%
\pgfpathlineto{\pgfqpoint{4.708601in}{1.951491in}}%
\pgfpathlineto{\pgfqpoint{4.710716in}{1.956609in}}%
\pgfpathlineto{\pgfqpoint{4.712830in}{1.957610in}}%
\pgfpathlineto{\pgfqpoint{4.719174in}{1.954192in}}%
\pgfpathlineto{\pgfqpoint{4.721289in}{1.960630in}}%
\pgfpathlineto{\pgfqpoint{4.723404in}{1.958938in}}%
\pgfpathlineto{\pgfqpoint{4.725518in}{1.959823in}}%
\pgfpathlineto{\pgfqpoint{4.727633in}{1.962056in}}%
\pgfpathlineto{\pgfqpoint{4.729748in}{1.957816in}}%
\pgfpathlineto{\pgfqpoint{4.731863in}{1.957665in}}%
\pgfpathlineto{\pgfqpoint{4.733977in}{1.955441in}}%
\pgfpathlineto{\pgfqpoint{4.736092in}{1.957930in}}%
\pgfpathlineto{\pgfqpoint{4.738207in}{1.955757in}}%
\pgfpathlineto{\pgfqpoint{4.740321in}{1.958963in}}%
\pgfpathlineto{\pgfqpoint{4.744551in}{1.973415in}}%
\pgfpathlineto{\pgfqpoint{4.746665in}{1.978253in}}%
\pgfpathlineto{\pgfqpoint{4.748780in}{1.979128in}}%
\pgfpathlineto{\pgfqpoint{4.757239in}{1.997438in}}%
\pgfpathlineto{\pgfqpoint{4.759354in}{1.997479in}}%
\pgfpathlineto{\pgfqpoint{4.761468in}{1.996265in}}%
\pgfpathlineto{\pgfqpoint{4.765698in}{1.986986in}}%
\pgfpathlineto{\pgfqpoint{4.767812in}{1.989653in}}%
\pgfpathlineto{\pgfqpoint{4.772042in}{1.991214in}}%
\pgfpathlineto{\pgfqpoint{4.776271in}{1.997196in}}%
\pgfpathlineto{\pgfqpoint{4.782615in}{1.998674in}}%
\pgfpathlineto{\pgfqpoint{4.788959in}{1.996218in}}%
\pgfpathlineto{\pgfqpoint{4.791074in}{1.997491in}}%
\pgfpathlineto{\pgfqpoint{4.793189in}{1.990171in}}%
\pgfpathlineto{\pgfqpoint{4.795303in}{1.993270in}}%
\pgfpathlineto{\pgfqpoint{4.797418in}{1.993948in}}%
\pgfpathlineto{\pgfqpoint{4.801647in}{1.981220in}}%
\pgfpathlineto{\pgfqpoint{4.803762in}{1.981475in}}%
\pgfpathlineto{\pgfqpoint{4.805877in}{1.983446in}}%
\pgfpathlineto{\pgfqpoint{4.807991in}{1.975601in}}%
\pgfpathlineto{\pgfqpoint{4.810106in}{1.972960in}}%
\pgfpathlineto{\pgfqpoint{4.812221in}{1.975739in}}%
\pgfpathlineto{\pgfqpoint{4.816450in}{1.972628in}}%
\pgfpathlineto{\pgfqpoint{4.822794in}{1.955187in}}%
\pgfpathlineto{\pgfqpoint{4.827024in}{1.956356in}}%
\pgfpathlineto{\pgfqpoint{4.829138in}{1.953545in}}%
\pgfpathlineto{\pgfqpoint{4.831253in}{1.959519in}}%
\pgfpathlineto{\pgfqpoint{4.833368in}{1.959885in}}%
\pgfpathlineto{\pgfqpoint{4.837597in}{1.950750in}}%
\pgfpathlineto{\pgfqpoint{4.839712in}{1.954516in}}%
\pgfpathlineto{\pgfqpoint{4.841827in}{1.964706in}}%
\pgfpathlineto{\pgfqpoint{4.843941in}{1.953720in}}%
\pgfpathlineto{\pgfqpoint{4.846056in}{1.951749in}}%
\pgfpathlineto{\pgfqpoint{4.848171in}{1.944715in}}%
\pgfpathlineto{\pgfqpoint{4.852400in}{1.947096in}}%
\pgfpathlineto{\pgfqpoint{4.854515in}{1.949476in}}%
\pgfpathlineto{\pgfqpoint{4.856629in}{1.944797in}}%
\pgfpathlineto{\pgfqpoint{4.858744in}{1.943993in}}%
\pgfpathlineto{\pgfqpoint{4.860859in}{1.945515in}}%
\pgfpathlineto{\pgfqpoint{4.862974in}{1.941837in}}%
\pgfpathlineto{\pgfqpoint{4.865088in}{1.945516in}}%
\pgfpathlineto{\pgfqpoint{4.869318in}{1.944257in}}%
\pgfpathlineto{\pgfqpoint{4.873547in}{1.949844in}}%
\pgfpathlineto{\pgfqpoint{4.875662in}{1.944609in}}%
\pgfpathlineto{\pgfqpoint{4.879891in}{1.943281in}}%
\pgfpathlineto{\pgfqpoint{4.884120in}{1.936391in}}%
\pgfpathlineto{\pgfqpoint{4.886235in}{1.934570in}}%
\pgfpathlineto{\pgfqpoint{4.888350in}{1.939907in}}%
\pgfpathlineto{\pgfqpoint{4.892579in}{1.938414in}}%
\pgfpathlineto{\pgfqpoint{4.894694in}{1.940428in}}%
\pgfpathlineto{\pgfqpoint{4.896809in}{1.937257in}}%
\pgfpathlineto{\pgfqpoint{4.898923in}{1.937629in}}%
\pgfpathlineto{\pgfqpoint{4.901038in}{1.939793in}}%
\pgfpathlineto{\pgfqpoint{4.903153in}{1.949527in}}%
\pgfpathlineto{\pgfqpoint{4.905267in}{1.952915in}}%
\pgfpathlineto{\pgfqpoint{4.907382in}{1.961641in}}%
\pgfpathlineto{\pgfqpoint{4.909497in}{1.963517in}}%
\pgfpathlineto{\pgfqpoint{4.911611in}{1.974760in}}%
\pgfpathlineto{\pgfqpoint{4.913726in}{1.978904in}}%
\pgfpathlineto{\pgfqpoint{4.915841in}{1.975488in}}%
\pgfpathlineto{\pgfqpoint{4.920070in}{1.987949in}}%
\pgfpathlineto{\pgfqpoint{4.922185in}{1.989817in}}%
\pgfpathlineto{\pgfqpoint{4.926414in}{1.983951in}}%
\pgfpathlineto{\pgfqpoint{4.928529in}{1.988532in}}%
\pgfpathlineto{\pgfqpoint{4.930644in}{1.986526in}}%
\pgfpathlineto{\pgfqpoint{4.934873in}{1.995310in}}%
\pgfpathlineto{\pgfqpoint{4.939103in}{1.998343in}}%
\pgfpathlineto{\pgfqpoint{4.941217in}{1.997140in}}%
\pgfpathlineto{\pgfqpoint{4.943332in}{1.997940in}}%
\pgfpathlineto{\pgfqpoint{4.945447in}{1.996984in}}%
\pgfpathlineto{\pgfqpoint{4.947561in}{1.997280in}}%
\pgfpathlineto{\pgfqpoint{4.949676in}{2.006687in}}%
\pgfpathlineto{\pgfqpoint{4.951791in}{2.007603in}}%
\pgfpathlineto{\pgfqpoint{4.953905in}{2.001834in}}%
\pgfpathlineto{\pgfqpoint{4.956020in}{2.000268in}}%
\pgfpathlineto{\pgfqpoint{4.960249in}{2.009142in}}%
\pgfpathlineto{\pgfqpoint{4.962364in}{2.009451in}}%
\pgfpathlineto{\pgfqpoint{4.964479in}{2.011930in}}%
\pgfpathlineto{\pgfqpoint{4.966594in}{2.007783in}}%
\pgfpathlineto{\pgfqpoint{4.968708in}{2.007311in}}%
\pgfpathlineto{\pgfqpoint{4.972938in}{2.011511in}}%
\pgfpathlineto{\pgfqpoint{4.975052in}{2.007231in}}%
\pgfpathlineto{\pgfqpoint{4.977167in}{2.010469in}}%
\pgfpathlineto{\pgfqpoint{4.979282in}{2.009530in}}%
\pgfpathlineto{\pgfqpoint{4.981396in}{2.011385in}}%
\pgfpathlineto{\pgfqpoint{4.983511in}{2.010725in}}%
\pgfpathlineto{\pgfqpoint{4.985626in}{2.012560in}}%
\pgfpathlineto{\pgfqpoint{4.987740in}{2.007274in}}%
\pgfpathlineto{\pgfqpoint{4.989855in}{2.008639in}}%
\pgfpathlineto{\pgfqpoint{4.994085in}{2.002758in}}%
\pgfpathlineto{\pgfqpoint{4.996199in}{2.001777in}}%
\pgfpathlineto{\pgfqpoint{5.002543in}{2.007877in}}%
\pgfpathlineto{\pgfqpoint{5.004658in}{2.004785in}}%
\pgfpathlineto{\pgfqpoint{5.006773in}{2.007753in}}%
\pgfpathlineto{\pgfqpoint{5.008887in}{2.003896in}}%
\pgfpathlineto{\pgfqpoint{5.013117in}{2.006559in}}%
\pgfpathlineto{\pgfqpoint{5.015231in}{2.002080in}}%
\pgfpathlineto{\pgfqpoint{5.017346in}{2.003738in}}%
\pgfpathlineto{\pgfqpoint{5.021576in}{1.996907in}}%
\pgfpathlineto{\pgfqpoint{5.023690in}{1.998515in}}%
\pgfpathlineto{\pgfqpoint{5.025805in}{1.991800in}}%
\pgfpathlineto{\pgfqpoint{5.027920in}{1.995099in}}%
\pgfpathlineto{\pgfqpoint{5.030034in}{1.990009in}}%
\pgfpathlineto{\pgfqpoint{5.032149in}{1.992134in}}%
\pgfpathlineto{\pgfqpoint{5.034264in}{1.997616in}}%
\pgfpathlineto{\pgfqpoint{5.036378in}{1.999320in}}%
\pgfpathlineto{\pgfqpoint{5.040608in}{2.006012in}}%
\pgfpathlineto{\pgfqpoint{5.042722in}{2.003415in}}%
\pgfpathlineto{\pgfqpoint{5.049067in}{2.017042in}}%
\pgfpathlineto{\pgfqpoint{5.051181in}{2.015336in}}%
\pgfpathlineto{\pgfqpoint{5.053296in}{2.017740in}}%
\pgfpathlineto{\pgfqpoint{5.055411in}{2.018064in}}%
\pgfpathlineto{\pgfqpoint{5.061755in}{2.033066in}}%
\pgfpathlineto{\pgfqpoint{5.065984in}{2.031751in}}%
\pgfpathlineto{\pgfqpoint{5.068099in}{2.033922in}}%
\pgfpathlineto{\pgfqpoint{5.070214in}{2.037847in}}%
\pgfpathlineto{\pgfqpoint{5.072328in}{2.034266in}}%
\pgfpathlineto{\pgfqpoint{5.074443in}{2.036727in}}%
\pgfpathlineto{\pgfqpoint{5.078672in}{2.037881in}}%
\pgfpathlineto{\pgfqpoint{5.080787in}{2.036507in}}%
\pgfpathlineto{\pgfqpoint{5.085016in}{2.040174in}}%
\pgfpathlineto{\pgfqpoint{5.091360in}{2.037828in}}%
\pgfpathlineto{\pgfqpoint{5.093475in}{2.033489in}}%
\pgfpathlineto{\pgfqpoint{5.095590in}{2.035820in}}%
\pgfpathlineto{\pgfqpoint{5.097705in}{2.030062in}}%
\pgfpathlineto{\pgfqpoint{5.108278in}{2.058702in}}%
\pgfpathlineto{\pgfqpoint{5.110393in}{2.056287in}}%
\pgfpathlineto{\pgfqpoint{5.112507in}{2.056919in}}%
\pgfpathlineto{\pgfqpoint{5.114622in}{2.056096in}}%
\pgfpathlineto{\pgfqpoint{5.116737in}{2.062295in}}%
\pgfpathlineto{\pgfqpoint{5.120966in}{2.057956in}}%
\pgfpathlineto{\pgfqpoint{5.123081in}{2.062610in}}%
\pgfpathlineto{\pgfqpoint{5.127310in}{2.061009in}}%
\pgfpathlineto{\pgfqpoint{5.129425in}{2.063944in}}%
\pgfpathlineto{\pgfqpoint{5.131540in}{2.060908in}}%
\pgfpathlineto{\pgfqpoint{5.133654in}{2.060198in}}%
\pgfpathlineto{\pgfqpoint{5.137884in}{2.071025in}}%
\pgfpathlineto{\pgfqpoint{5.142113in}{2.073040in}}%
\pgfpathlineto{\pgfqpoint{5.146342in}{2.078359in}}%
\pgfpathlineto{\pgfqpoint{5.148457in}{2.075034in}}%
\pgfpathlineto{\pgfqpoint{5.150572in}{2.079656in}}%
\pgfpathlineto{\pgfqpoint{5.152687in}{2.080112in}}%
\pgfpathlineto{\pgfqpoint{5.154801in}{2.071474in}}%
\pgfpathlineto{\pgfqpoint{5.159031in}{2.078921in}}%
\pgfpathlineto{\pgfqpoint{5.161145in}{2.088342in}}%
\pgfpathlineto{\pgfqpoint{5.163260in}{2.091072in}}%
\pgfpathlineto{\pgfqpoint{5.165375in}{2.090090in}}%
\pgfpathlineto{\pgfqpoint{5.167489in}{2.087005in}}%
\pgfpathlineto{\pgfqpoint{5.171719in}{2.091970in}}%
\pgfpathlineto{\pgfqpoint{5.173834in}{2.089309in}}%
\pgfpathlineto{\pgfqpoint{5.175948in}{2.097959in}}%
\pgfpathlineto{\pgfqpoint{5.178063in}{2.100077in}}%
\pgfpathlineto{\pgfqpoint{5.180178in}{2.100197in}}%
\pgfpathlineto{\pgfqpoint{5.188636in}{2.112500in}}%
\pgfpathlineto{\pgfqpoint{5.188636in}{2.112500in}}%
\pgfusepath{stroke}%
\end{pgfscope}%
\begin{pgfscope}%
\pgfsetrectcap%
\pgfsetmiterjoin%
\pgfsetlinewidth{0.000000pt}%
\definecolor{currentstroke}{rgb}{1.000000,1.000000,1.000000}%
\pgfsetstrokecolor{currentstroke}%
\pgfsetdash{}{0pt}%
\pgfpathmoveto{\pgfqpoint{0.750000in}{0.275000in}}%
\pgfpathlineto{\pgfqpoint{0.750000in}{2.200000in}}%
\pgfusepath{}%
\end{pgfscope}%
\begin{pgfscope}%
\pgfsetrectcap%
\pgfsetmiterjoin%
\pgfsetlinewidth{0.000000pt}%
\definecolor{currentstroke}{rgb}{1.000000,1.000000,1.000000}%
\pgfsetstrokecolor{currentstroke}%
\pgfsetdash{}{0pt}%
\pgfpathmoveto{\pgfqpoint{5.400000in}{0.275000in}}%
\pgfpathlineto{\pgfqpoint{5.400000in}{2.200000in}}%
\pgfusepath{}%
\end{pgfscope}%
\begin{pgfscope}%
\pgfsetrectcap%
\pgfsetmiterjoin%
\pgfsetlinewidth{0.000000pt}%
\definecolor{currentstroke}{rgb}{1.000000,1.000000,1.000000}%
\pgfsetstrokecolor{currentstroke}%
\pgfsetdash{}{0pt}%
\pgfpathmoveto{\pgfqpoint{0.750000in}{0.275000in}}%
\pgfpathlineto{\pgfqpoint{5.400000in}{0.275000in}}%
\pgfusepath{}%
\end{pgfscope}%
\begin{pgfscope}%
\pgfsetrectcap%
\pgfsetmiterjoin%
\pgfsetlinewidth{0.000000pt}%
\definecolor{currentstroke}{rgb}{1.000000,1.000000,1.000000}%
\pgfsetstrokecolor{currentstroke}%
\pgfsetdash{}{0pt}%
\pgfpathmoveto{\pgfqpoint{0.750000in}{2.200000in}}%
\pgfpathlineto{\pgfqpoint{5.400000in}{2.200000in}}%
\pgfusepath{}%
\end{pgfscope}%
\end{pgfpicture}%
\makeatother%
\endgroup%

    \caption{Superposition of simulation of a Dyson Brownian motion and its finite-variation part.\label{fig:dyson_comparison}}
\end{figure}

In Figure \ref{fig:four_dyson} there are four different path simulations for a Dyson Brownian motion with dimension equal to four, five, seven and nine. The initial condition in the third last cases is with $n$ equally spaced points around zero. In the first case, the initial condition is slightly different because the two points in the middle have a separation of size one, while the surrounding points are separated by 0.1 of them. This appears to have little effect on this system, but in Chapter \ref{ch:determinist} we will see that in the deterministic case, it has an effect.

Although the distances between paths in every simulation of Figure \ref{fig:four_dyson} are stochastic, we can notice that the spacing between consecutive points is more or less uniform, which is what we would expect as the particles reject each other with a similar strength. Also, it is interesting to notice that, although the number of paths increases in every chart of the figure, the total path spacing until $t=10$ is more or less uniform in all of them. 

\begin{figure}[h!] 
    %% Creator: Matplotlib, PGF backend
%%
%% To include the figure in your LaTeX document, write
%%   \input{<filename>.pgf}
%%
%% Make sure the required packages are loaded in your preamble
%%   \usepackage{pgf}
%%
%% Also ensure that all the required font packages are loaded; for instance,
%% the lmodern package is sometimes necessary when using math font.
%%   \usepackage{lmodern}
%%
%% Figures using additional raster images can only be included by \input if
%% they are in the same directory as the main LaTeX file. For loading figures
%% from other directories you can use the `import` package
%%   \usepackage{import}
%%
%% and then include the figures with
%%   \import{<path to file>}{<filename>.pgf}
%%
%% Matplotlib used the following preamble
%%   
%%   \makeatletter\@ifpackageloaded{underscore}{}{\usepackage[strings]{underscore}}\makeatother
%%
\begingroup%
\makeatletter%
\begin{pgfpicture}%
\pgfpathrectangle{\pgfpointorigin}{\pgfqpoint{6.000000in}{6.000000in}}%
\pgfusepath{use as bounding box, clip}%
\begin{pgfscope}%
\pgfsetbuttcap%
\pgfsetmiterjoin%
\definecolor{currentfill}{rgb}{1.000000,1.000000,1.000000}%
\pgfsetfillcolor{currentfill}%
\pgfsetlinewidth{0.000000pt}%
\definecolor{currentstroke}{rgb}{1.000000,1.000000,1.000000}%
\pgfsetstrokecolor{currentstroke}%
\pgfsetdash{}{0pt}%
\pgfpathmoveto{\pgfqpoint{0.000000in}{0.000000in}}%
\pgfpathlineto{\pgfqpoint{6.000000in}{0.000000in}}%
\pgfpathlineto{\pgfqpoint{6.000000in}{6.000000in}}%
\pgfpathlineto{\pgfqpoint{0.000000in}{6.000000in}}%
\pgfpathlineto{\pgfqpoint{0.000000in}{0.000000in}}%
\pgfpathclose%
\pgfusepath{fill}%
\end{pgfscope}%
\begin{pgfscope}%
\pgfsetbuttcap%
\pgfsetmiterjoin%
\definecolor{currentfill}{rgb}{0.917647,0.917647,0.949020}%
\pgfsetfillcolor{currentfill}%
\pgfsetlinewidth{0.000000pt}%
\definecolor{currentstroke}{rgb}{0.000000,0.000000,0.000000}%
\pgfsetstrokecolor{currentstroke}%
\pgfsetstrokeopacity{0.000000}%
\pgfsetdash{}{0pt}%
\pgfpathmoveto{\pgfqpoint{0.750000in}{3.180000in}}%
\pgfpathlineto{\pgfqpoint{2.863636in}{3.180000in}}%
\pgfpathlineto{\pgfqpoint{2.863636in}{5.280000in}}%
\pgfpathlineto{\pgfqpoint{0.750000in}{5.280000in}}%
\pgfpathlineto{\pgfqpoint{0.750000in}{3.180000in}}%
\pgfpathclose%
\pgfusepath{fill}%
\end{pgfscope}%
\begin{pgfscope}%
\pgfpathrectangle{\pgfqpoint{0.750000in}{3.180000in}}{\pgfqpoint{2.113636in}{2.100000in}}%
\pgfusepath{clip}%
\pgfsetroundcap%
\pgfsetroundjoin%
\pgfsetlinewidth{1.003750pt}%
\definecolor{currentstroke}{rgb}{1.000000,1.000000,1.000000}%
\pgfsetstrokecolor{currentstroke}%
\pgfsetdash{}{0pt}%
\pgfpathmoveto{\pgfqpoint{0.846074in}{3.180000in}}%
\pgfpathlineto{\pgfqpoint{0.846074in}{5.280000in}}%
\pgfusepath{stroke}%
\end{pgfscope}%
\begin{pgfscope}%
\definecolor{textcolor}{rgb}{0.150000,0.150000,0.150000}%
\pgfsetstrokecolor{textcolor}%
\pgfsetfillcolor{textcolor}%
\pgftext[x=0.846074in,y=3.082778in,,top]{\color{textcolor}\rmfamily\fontsize{10.000000}{12.000000}\selectfont \(\displaystyle {0.0}\)}%
\end{pgfscope}%
\begin{pgfscope}%
\pgfpathrectangle{\pgfqpoint{0.750000in}{3.180000in}}{\pgfqpoint{2.113636in}{2.100000in}}%
\pgfusepath{clip}%
\pgfsetroundcap%
\pgfsetroundjoin%
\pgfsetlinewidth{1.003750pt}%
\definecolor{currentstroke}{rgb}{1.000000,1.000000,1.000000}%
\pgfsetstrokecolor{currentstroke}%
\pgfsetdash{}{0pt}%
\pgfpathmoveto{\pgfqpoint{1.326446in}{3.180000in}}%
\pgfpathlineto{\pgfqpoint{1.326446in}{5.280000in}}%
\pgfusepath{stroke}%
\end{pgfscope}%
\begin{pgfscope}%
\definecolor{textcolor}{rgb}{0.150000,0.150000,0.150000}%
\pgfsetstrokecolor{textcolor}%
\pgfsetfillcolor{textcolor}%
\pgftext[x=1.326446in,y=3.082778in,,top]{\color{textcolor}\rmfamily\fontsize{10.000000}{12.000000}\selectfont \(\displaystyle {2.5}\)}%
\end{pgfscope}%
\begin{pgfscope}%
\pgfpathrectangle{\pgfqpoint{0.750000in}{3.180000in}}{\pgfqpoint{2.113636in}{2.100000in}}%
\pgfusepath{clip}%
\pgfsetroundcap%
\pgfsetroundjoin%
\pgfsetlinewidth{1.003750pt}%
\definecolor{currentstroke}{rgb}{1.000000,1.000000,1.000000}%
\pgfsetstrokecolor{currentstroke}%
\pgfsetdash{}{0pt}%
\pgfpathmoveto{\pgfqpoint{1.806818in}{3.180000in}}%
\pgfpathlineto{\pgfqpoint{1.806818in}{5.280000in}}%
\pgfusepath{stroke}%
\end{pgfscope}%
\begin{pgfscope}%
\definecolor{textcolor}{rgb}{0.150000,0.150000,0.150000}%
\pgfsetstrokecolor{textcolor}%
\pgfsetfillcolor{textcolor}%
\pgftext[x=1.806818in,y=3.082778in,,top]{\color{textcolor}\rmfamily\fontsize{10.000000}{12.000000}\selectfont \(\displaystyle {5.0}\)}%
\end{pgfscope}%
\begin{pgfscope}%
\pgfpathrectangle{\pgfqpoint{0.750000in}{3.180000in}}{\pgfqpoint{2.113636in}{2.100000in}}%
\pgfusepath{clip}%
\pgfsetroundcap%
\pgfsetroundjoin%
\pgfsetlinewidth{1.003750pt}%
\definecolor{currentstroke}{rgb}{1.000000,1.000000,1.000000}%
\pgfsetstrokecolor{currentstroke}%
\pgfsetdash{}{0pt}%
\pgfpathmoveto{\pgfqpoint{2.287190in}{3.180000in}}%
\pgfpathlineto{\pgfqpoint{2.287190in}{5.280000in}}%
\pgfusepath{stroke}%
\end{pgfscope}%
\begin{pgfscope}%
\definecolor{textcolor}{rgb}{0.150000,0.150000,0.150000}%
\pgfsetstrokecolor{textcolor}%
\pgfsetfillcolor{textcolor}%
\pgftext[x=2.287190in,y=3.082778in,,top]{\color{textcolor}\rmfamily\fontsize{10.000000}{12.000000}\selectfont \(\displaystyle {7.5}\)}%
\end{pgfscope}%
\begin{pgfscope}%
\pgfpathrectangle{\pgfqpoint{0.750000in}{3.180000in}}{\pgfqpoint{2.113636in}{2.100000in}}%
\pgfusepath{clip}%
\pgfsetroundcap%
\pgfsetroundjoin%
\pgfsetlinewidth{1.003750pt}%
\definecolor{currentstroke}{rgb}{1.000000,1.000000,1.000000}%
\pgfsetstrokecolor{currentstroke}%
\pgfsetdash{}{0pt}%
\pgfpathmoveto{\pgfqpoint{2.767562in}{3.180000in}}%
\pgfpathlineto{\pgfqpoint{2.767562in}{5.280000in}}%
\pgfusepath{stroke}%
\end{pgfscope}%
\begin{pgfscope}%
\definecolor{textcolor}{rgb}{0.150000,0.150000,0.150000}%
\pgfsetstrokecolor{textcolor}%
\pgfsetfillcolor{textcolor}%
\pgftext[x=2.767562in,y=3.082778in,,top]{\color{textcolor}\rmfamily\fontsize{10.000000}{12.000000}\selectfont \(\displaystyle {10.0}\)}%
\end{pgfscope}%
\begin{pgfscope}%
\definecolor{textcolor}{rgb}{0.150000,0.150000,0.150000}%
\pgfsetstrokecolor{textcolor}%
\pgfsetfillcolor{textcolor}%
\pgftext[x=1.806818in,y=2.903766in,,top]{\color{textcolor}\rmfamily\fontsize{11.000000}{13.200000}\selectfont time (\(\displaystyle t\))}%
\end{pgfscope}%
\begin{pgfscope}%
\pgfpathrectangle{\pgfqpoint{0.750000in}{3.180000in}}{\pgfqpoint{2.113636in}{2.100000in}}%
\pgfusepath{clip}%
\pgfsetroundcap%
\pgfsetroundjoin%
\pgfsetlinewidth{1.003750pt}%
\definecolor{currentstroke}{rgb}{1.000000,1.000000,1.000000}%
\pgfsetstrokecolor{currentstroke}%
\pgfsetdash{}{0pt}%
\pgfpathmoveto{\pgfqpoint{0.750000in}{3.273945in}}%
\pgfpathlineto{\pgfqpoint{2.863636in}{3.273945in}}%
\pgfusepath{stroke}%
\end{pgfscope}%
\begin{pgfscope}%
\definecolor{textcolor}{rgb}{0.150000,0.150000,0.150000}%
\pgfsetstrokecolor{textcolor}%
\pgfsetfillcolor{textcolor}%
\pgftext[x=0.405863in, y=3.225720in, left, base]{\color{textcolor}\rmfamily\fontsize{10.000000}{12.000000}\selectfont \(\displaystyle {\ensuremath{-}10}\)}%
\end{pgfscope}%
\begin{pgfscope}%
\pgfpathrectangle{\pgfqpoint{0.750000in}{3.180000in}}{\pgfqpoint{2.113636in}{2.100000in}}%
\pgfusepath{clip}%
\pgfsetroundcap%
\pgfsetroundjoin%
\pgfsetlinewidth{1.003750pt}%
\definecolor{currentstroke}{rgb}{1.000000,1.000000,1.000000}%
\pgfsetstrokecolor{currentstroke}%
\pgfsetdash{}{0pt}%
\pgfpathmoveto{\pgfqpoint{0.750000in}{3.710994in}}%
\pgfpathlineto{\pgfqpoint{2.863636in}{3.710994in}}%
\pgfusepath{stroke}%
\end{pgfscope}%
\begin{pgfscope}%
\definecolor{textcolor}{rgb}{0.150000,0.150000,0.150000}%
\pgfsetstrokecolor{textcolor}%
\pgfsetfillcolor{textcolor}%
\pgftext[x=0.475308in, y=3.662769in, left, base]{\color{textcolor}\rmfamily\fontsize{10.000000}{12.000000}\selectfont \(\displaystyle {\ensuremath{-}5}\)}%
\end{pgfscope}%
\begin{pgfscope}%
\pgfpathrectangle{\pgfqpoint{0.750000in}{3.180000in}}{\pgfqpoint{2.113636in}{2.100000in}}%
\pgfusepath{clip}%
\pgfsetroundcap%
\pgfsetroundjoin%
\pgfsetlinewidth{1.003750pt}%
\definecolor{currentstroke}{rgb}{1.000000,1.000000,1.000000}%
\pgfsetstrokecolor{currentstroke}%
\pgfsetdash{}{0pt}%
\pgfpathmoveto{\pgfqpoint{0.750000in}{4.148043in}}%
\pgfpathlineto{\pgfqpoint{2.863636in}{4.148043in}}%
\pgfusepath{stroke}%
\end{pgfscope}%
\begin{pgfscope}%
\definecolor{textcolor}{rgb}{0.150000,0.150000,0.150000}%
\pgfsetstrokecolor{textcolor}%
\pgfsetfillcolor{textcolor}%
\pgftext[x=0.583333in, y=4.099818in, left, base]{\color{textcolor}\rmfamily\fontsize{10.000000}{12.000000}\selectfont \(\displaystyle {0}\)}%
\end{pgfscope}%
\begin{pgfscope}%
\pgfpathrectangle{\pgfqpoint{0.750000in}{3.180000in}}{\pgfqpoint{2.113636in}{2.100000in}}%
\pgfusepath{clip}%
\pgfsetroundcap%
\pgfsetroundjoin%
\pgfsetlinewidth{1.003750pt}%
\definecolor{currentstroke}{rgb}{1.000000,1.000000,1.000000}%
\pgfsetstrokecolor{currentstroke}%
\pgfsetdash{}{0pt}%
\pgfpathmoveto{\pgfqpoint{0.750000in}{4.585092in}}%
\pgfpathlineto{\pgfqpoint{2.863636in}{4.585092in}}%
\pgfusepath{stroke}%
\end{pgfscope}%
\begin{pgfscope}%
\definecolor{textcolor}{rgb}{0.150000,0.150000,0.150000}%
\pgfsetstrokecolor{textcolor}%
\pgfsetfillcolor{textcolor}%
\pgftext[x=0.583333in, y=4.536867in, left, base]{\color{textcolor}\rmfamily\fontsize{10.000000}{12.000000}\selectfont \(\displaystyle {5}\)}%
\end{pgfscope}%
\begin{pgfscope}%
\pgfpathrectangle{\pgfqpoint{0.750000in}{3.180000in}}{\pgfqpoint{2.113636in}{2.100000in}}%
\pgfusepath{clip}%
\pgfsetroundcap%
\pgfsetroundjoin%
\pgfsetlinewidth{1.003750pt}%
\definecolor{currentstroke}{rgb}{1.000000,1.000000,1.000000}%
\pgfsetstrokecolor{currentstroke}%
\pgfsetdash{}{0pt}%
\pgfpathmoveto{\pgfqpoint{0.750000in}{5.022141in}}%
\pgfpathlineto{\pgfqpoint{2.863636in}{5.022141in}}%
\pgfusepath{stroke}%
\end{pgfscope}%
\begin{pgfscope}%
\definecolor{textcolor}{rgb}{0.150000,0.150000,0.150000}%
\pgfsetstrokecolor{textcolor}%
\pgfsetfillcolor{textcolor}%
\pgftext[x=0.513888in, y=4.973916in, left, base]{\color{textcolor}\rmfamily\fontsize{10.000000}{12.000000}\selectfont \(\displaystyle {10}\)}%
\end{pgfscope}%
\begin{pgfscope}%
\definecolor{textcolor}{rgb}{0.150000,0.150000,0.150000}%
\pgfsetstrokecolor{textcolor}%
\pgfsetfillcolor{textcolor}%
\pgftext[x=0.350308in,y=4.230000in,,bottom,rotate=90.000000]{\color{textcolor}\rmfamily\fontsize{11.000000}{13.200000}\selectfont Position}%
\end{pgfscope}%
\begin{pgfscope}%
\pgfpathrectangle{\pgfqpoint{0.750000in}{3.180000in}}{\pgfqpoint{2.113636in}{2.100000in}}%
\pgfusepath{clip}%
\pgfsetroundcap%
\pgfsetroundjoin%
\pgfsetlinewidth{0.602250pt}%
\definecolor{currentstroke}{rgb}{0.215686,0.494118,0.721569}%
\pgfsetstrokecolor{currentstroke}%
\pgfsetdash{}{0pt}%
\pgfpathmoveto{\pgfqpoint{0.846074in}{4.051892in}}%
\pgfpathlineto{\pgfqpoint{0.849921in}{4.032016in}}%
\pgfpathlineto{\pgfqpoint{0.851845in}{4.030584in}}%
\pgfpathlineto{\pgfqpoint{0.857615in}{4.001980in}}%
\pgfpathlineto{\pgfqpoint{0.859538in}{4.005400in}}%
\pgfpathlineto{\pgfqpoint{0.865308in}{3.978908in}}%
\pgfpathlineto{\pgfqpoint{0.867232in}{3.981374in}}%
\pgfpathlineto{\pgfqpoint{0.869155in}{3.974027in}}%
\pgfpathlineto{\pgfqpoint{0.871079in}{3.972313in}}%
\pgfpathlineto{\pgfqpoint{0.873002in}{3.952731in}}%
\pgfpathlineto{\pgfqpoint{0.876849in}{3.968392in}}%
\pgfpathlineto{\pgfqpoint{0.878772in}{3.986792in}}%
\pgfpathlineto{\pgfqpoint{0.880696in}{3.991102in}}%
\pgfpathlineto{\pgfqpoint{0.884543in}{3.964128in}}%
\pgfpathlineto{\pgfqpoint{0.886466in}{3.963124in}}%
\pgfpathlineto{\pgfqpoint{0.888389in}{3.965439in}}%
\pgfpathlineto{\pgfqpoint{0.890313in}{3.965680in}}%
\pgfpathlineto{\pgfqpoint{0.894160in}{3.948988in}}%
\pgfpathlineto{\pgfqpoint{0.896083in}{3.950326in}}%
\pgfpathlineto{\pgfqpoint{0.898006in}{3.938421in}}%
\pgfpathlineto{\pgfqpoint{0.899930in}{3.936611in}}%
\pgfpathlineto{\pgfqpoint{0.901853in}{3.952470in}}%
\pgfpathlineto{\pgfqpoint{0.905700in}{3.934294in}}%
\pgfpathlineto{\pgfqpoint{0.907624in}{3.924922in}}%
\pgfpathlineto{\pgfqpoint{0.911470in}{3.921484in}}%
\pgfpathlineto{\pgfqpoint{0.915317in}{3.944211in}}%
\pgfpathlineto{\pgfqpoint{0.917241in}{3.952774in}}%
\pgfpathlineto{\pgfqpoint{0.919164in}{3.927958in}}%
\pgfpathlineto{\pgfqpoint{0.921087in}{3.923857in}}%
\pgfpathlineto{\pgfqpoint{0.923011in}{3.930999in}}%
\pgfpathlineto{\pgfqpoint{0.926858in}{3.925361in}}%
\pgfpathlineto{\pgfqpoint{0.928781in}{3.924476in}}%
\pgfpathlineto{\pgfqpoint{0.930704in}{3.933497in}}%
\pgfpathlineto{\pgfqpoint{0.934551in}{3.894083in}}%
\pgfpathlineto{\pgfqpoint{0.936475in}{3.908489in}}%
\pgfpathlineto{\pgfqpoint{0.938398in}{3.913389in}}%
\pgfpathlineto{\pgfqpoint{0.940322in}{3.910843in}}%
\pgfpathlineto{\pgfqpoint{0.944168in}{3.920019in}}%
\pgfpathlineto{\pgfqpoint{0.946092in}{3.921582in}}%
\pgfpathlineto{\pgfqpoint{0.948015in}{3.916408in}}%
\pgfpathlineto{\pgfqpoint{0.953785in}{3.952143in}}%
\pgfpathlineto{\pgfqpoint{0.955709in}{3.951978in}}%
\pgfpathlineto{\pgfqpoint{0.959556in}{3.980576in}}%
\pgfpathlineto{\pgfqpoint{0.965326in}{3.956724in}}%
\pgfpathlineto{\pgfqpoint{0.967249in}{3.969111in}}%
\pgfpathlineto{\pgfqpoint{0.969173in}{3.969652in}}%
\pgfpathlineto{\pgfqpoint{0.971096in}{3.980285in}}%
\pgfpathlineto{\pgfqpoint{0.973020in}{3.972196in}}%
\pgfpathlineto{\pgfqpoint{0.974943in}{3.977714in}}%
\pgfpathlineto{\pgfqpoint{0.976866in}{3.974039in}}%
\pgfpathlineto{\pgfqpoint{0.978790in}{3.967019in}}%
\pgfpathlineto{\pgfqpoint{0.984560in}{3.964714in}}%
\pgfpathlineto{\pgfqpoint{0.986483in}{3.969657in}}%
\pgfpathlineto{\pgfqpoint{0.988407in}{3.978897in}}%
\pgfpathlineto{\pgfqpoint{0.990330in}{3.976718in}}%
\pgfpathlineto{\pgfqpoint{0.992254in}{3.983844in}}%
\pgfpathlineto{\pgfqpoint{0.994177in}{3.983878in}}%
\pgfpathlineto{\pgfqpoint{0.998024in}{3.979310in}}%
\pgfpathlineto{\pgfqpoint{0.999947in}{3.982714in}}%
\pgfpathlineto{\pgfqpoint{1.001871in}{3.991784in}}%
\pgfpathlineto{\pgfqpoint{1.003794in}{3.987606in}}%
\pgfpathlineto{\pgfqpoint{1.005717in}{3.989117in}}%
\pgfpathlineto{\pgfqpoint{1.009564in}{3.977724in}}%
\pgfpathlineto{\pgfqpoint{1.011488in}{3.990806in}}%
\pgfpathlineto{\pgfqpoint{1.015335in}{3.986391in}}%
\pgfpathlineto{\pgfqpoint{1.017258in}{3.989229in}}%
\pgfpathlineto{\pgfqpoint{1.019181in}{3.983790in}}%
\pgfpathlineto{\pgfqpoint{1.021105in}{3.971574in}}%
\pgfpathlineto{\pgfqpoint{1.023028in}{3.975953in}}%
\pgfpathlineto{\pgfqpoint{1.024952in}{3.987171in}}%
\pgfpathlineto{\pgfqpoint{1.026875in}{3.975802in}}%
\pgfpathlineto{\pgfqpoint{1.028798in}{3.976875in}}%
\pgfpathlineto{\pgfqpoint{1.032645in}{3.997983in}}%
\pgfpathlineto{\pgfqpoint{1.034569in}{3.998595in}}%
\pgfpathlineto{\pgfqpoint{1.036492in}{4.000587in}}%
\pgfpathlineto{\pgfqpoint{1.038415in}{4.004562in}}%
\pgfpathlineto{\pgfqpoint{1.040339in}{4.013859in}}%
\pgfpathlineto{\pgfqpoint{1.042262in}{3.999713in}}%
\pgfpathlineto{\pgfqpoint{1.044186in}{4.008986in}}%
\pgfpathlineto{\pgfqpoint{1.046109in}{4.006914in}}%
\pgfpathlineto{\pgfqpoint{1.048033in}{3.991506in}}%
\pgfpathlineto{\pgfqpoint{1.049956in}{3.988872in}}%
\pgfpathlineto{\pgfqpoint{1.051879in}{3.996577in}}%
\pgfpathlineto{\pgfqpoint{1.055726in}{3.974040in}}%
\pgfpathlineto{\pgfqpoint{1.057650in}{3.974349in}}%
\pgfpathlineto{\pgfqpoint{1.059573in}{3.978156in}}%
\pgfpathlineto{\pgfqpoint{1.061496in}{3.973349in}}%
\pgfpathlineto{\pgfqpoint{1.065343in}{3.955903in}}%
\pgfpathlineto{\pgfqpoint{1.067267in}{3.952870in}}%
\pgfpathlineto{\pgfqpoint{1.069190in}{3.961988in}}%
\pgfpathlineto{\pgfqpoint{1.071113in}{3.948958in}}%
\pgfpathlineto{\pgfqpoint{1.073037in}{3.953230in}}%
\pgfpathlineto{\pgfqpoint{1.074960in}{3.945646in}}%
\pgfpathlineto{\pgfqpoint{1.076884in}{3.948315in}}%
\pgfpathlineto{\pgfqpoint{1.078807in}{3.957993in}}%
\pgfpathlineto{\pgfqpoint{1.080731in}{3.957493in}}%
\pgfpathlineto{\pgfqpoint{1.084577in}{3.944545in}}%
\pgfpathlineto{\pgfqpoint{1.086501in}{3.940574in}}%
\pgfpathlineto{\pgfqpoint{1.088424in}{3.924747in}}%
\pgfpathlineto{\pgfqpoint{1.090348in}{3.920602in}}%
\pgfpathlineto{\pgfqpoint{1.092271in}{3.921930in}}%
\pgfpathlineto{\pgfqpoint{1.094194in}{3.913765in}}%
\pgfpathlineto{\pgfqpoint{1.096118in}{3.933991in}}%
\pgfpathlineto{\pgfqpoint{1.098041in}{3.934252in}}%
\pgfpathlineto{\pgfqpoint{1.101888in}{3.920728in}}%
\pgfpathlineto{\pgfqpoint{1.103811in}{3.924592in}}%
\pgfpathlineto{\pgfqpoint{1.105735in}{3.940966in}}%
\pgfpathlineto{\pgfqpoint{1.107658in}{3.945838in}}%
\pgfpathlineto{\pgfqpoint{1.109582in}{3.956158in}}%
\pgfpathlineto{\pgfqpoint{1.111505in}{3.958436in}}%
\pgfpathlineto{\pgfqpoint{1.113429in}{3.963697in}}%
\pgfpathlineto{\pgfqpoint{1.115352in}{3.973159in}}%
\pgfpathlineto{\pgfqpoint{1.117275in}{3.965284in}}%
\pgfpathlineto{\pgfqpoint{1.119199in}{3.973618in}}%
\pgfpathlineto{\pgfqpoint{1.123046in}{3.940851in}}%
\pgfpathlineto{\pgfqpoint{1.124969in}{3.934696in}}%
\pgfpathlineto{\pgfqpoint{1.126892in}{3.936119in}}%
\pgfpathlineto{\pgfqpoint{1.134586in}{3.898128in}}%
\pgfpathlineto{\pgfqpoint{1.138433in}{3.920898in}}%
\pgfpathlineto{\pgfqpoint{1.140356in}{3.911946in}}%
\pgfpathlineto{\pgfqpoint{1.142280in}{3.923853in}}%
\pgfpathlineto{\pgfqpoint{1.144203in}{3.920191in}}%
\pgfpathlineto{\pgfqpoint{1.146126in}{3.937131in}}%
\pgfpathlineto{\pgfqpoint{1.148050in}{3.943757in}}%
\pgfpathlineto{\pgfqpoint{1.149973in}{3.942310in}}%
\pgfpathlineto{\pgfqpoint{1.151897in}{3.943213in}}%
\pgfpathlineto{\pgfqpoint{1.153820in}{3.937119in}}%
\pgfpathlineto{\pgfqpoint{1.157667in}{3.933668in}}%
\pgfpathlineto{\pgfqpoint{1.161514in}{3.917012in}}%
\pgfpathlineto{\pgfqpoint{1.163437in}{3.916634in}}%
\pgfpathlineto{\pgfqpoint{1.165361in}{3.945306in}}%
\pgfpathlineto{\pgfqpoint{1.167284in}{3.950274in}}%
\pgfpathlineto{\pgfqpoint{1.169207in}{3.940689in}}%
\pgfpathlineto{\pgfqpoint{1.171131in}{3.946255in}}%
\pgfpathlineto{\pgfqpoint{1.173054in}{3.943032in}}%
\pgfpathlineto{\pgfqpoint{1.174978in}{3.947106in}}%
\pgfpathlineto{\pgfqpoint{1.176901in}{3.958303in}}%
\pgfpathlineto{\pgfqpoint{1.178824in}{3.952021in}}%
\pgfpathlineto{\pgfqpoint{1.180748in}{3.949755in}}%
\pgfpathlineto{\pgfqpoint{1.182671in}{3.941938in}}%
\pgfpathlineto{\pgfqpoint{1.184595in}{3.914309in}}%
\pgfpathlineto{\pgfqpoint{1.188442in}{3.895428in}}%
\pgfpathlineto{\pgfqpoint{1.190365in}{3.881718in}}%
\pgfpathlineto{\pgfqpoint{1.194212in}{3.869755in}}%
\pgfpathlineto{\pgfqpoint{1.196135in}{3.885990in}}%
\pgfpathlineto{\pgfqpoint{1.198059in}{3.885412in}}%
\pgfpathlineto{\pgfqpoint{1.201905in}{3.871921in}}%
\pgfpathlineto{\pgfqpoint{1.203829in}{3.889018in}}%
\pgfpathlineto{\pgfqpoint{1.205752in}{3.894099in}}%
\pgfpathlineto{\pgfqpoint{1.209599in}{3.879864in}}%
\pgfpathlineto{\pgfqpoint{1.211522in}{3.881374in}}%
\pgfpathlineto{\pgfqpoint{1.213446in}{3.875021in}}%
\pgfpathlineto{\pgfqpoint{1.215369in}{3.880546in}}%
\pgfpathlineto{\pgfqpoint{1.219216in}{3.861731in}}%
\pgfpathlineto{\pgfqpoint{1.221140in}{3.875895in}}%
\pgfpathlineto{\pgfqpoint{1.223063in}{3.876136in}}%
\pgfpathlineto{\pgfqpoint{1.224986in}{3.873896in}}%
\pgfpathlineto{\pgfqpoint{1.226910in}{3.869086in}}%
\pgfpathlineto{\pgfqpoint{1.228833in}{3.878029in}}%
\pgfpathlineto{\pgfqpoint{1.234603in}{3.868531in}}%
\pgfpathlineto{\pgfqpoint{1.238450in}{3.895737in}}%
\pgfpathlineto{\pgfqpoint{1.240374in}{3.884222in}}%
\pgfpathlineto{\pgfqpoint{1.244220in}{3.914714in}}%
\pgfpathlineto{\pgfqpoint{1.246144in}{3.898401in}}%
\pgfpathlineto{\pgfqpoint{1.248067in}{3.909036in}}%
\pgfpathlineto{\pgfqpoint{1.249991in}{3.913334in}}%
\pgfpathlineto{\pgfqpoint{1.251914in}{3.914442in}}%
\pgfpathlineto{\pgfqpoint{1.253838in}{3.912594in}}%
\pgfpathlineto{\pgfqpoint{1.255761in}{3.906572in}}%
\pgfpathlineto{\pgfqpoint{1.257684in}{3.934257in}}%
\pgfpathlineto{\pgfqpoint{1.263455in}{3.924497in}}%
\pgfpathlineto{\pgfqpoint{1.265378in}{3.914442in}}%
\pgfpathlineto{\pgfqpoint{1.267301in}{3.895547in}}%
\pgfpathlineto{\pgfqpoint{1.269225in}{3.889094in}}%
\pgfpathlineto{\pgfqpoint{1.271148in}{3.903938in}}%
\pgfpathlineto{\pgfqpoint{1.273072in}{3.890507in}}%
\pgfpathlineto{\pgfqpoint{1.278842in}{3.929100in}}%
\pgfpathlineto{\pgfqpoint{1.280765in}{3.928131in}}%
\pgfpathlineto{\pgfqpoint{1.282689in}{3.919584in}}%
\pgfpathlineto{\pgfqpoint{1.284612in}{3.932658in}}%
\pgfpathlineto{\pgfqpoint{1.288459in}{3.927825in}}%
\pgfpathlineto{\pgfqpoint{1.290382in}{3.935608in}}%
\pgfpathlineto{\pgfqpoint{1.292306in}{3.949148in}}%
\pgfpathlineto{\pgfqpoint{1.294229in}{3.950652in}}%
\pgfpathlineto{\pgfqpoint{1.296153in}{3.959043in}}%
\pgfpathlineto{\pgfqpoint{1.298076in}{3.976370in}}%
\pgfpathlineto{\pgfqpoint{1.299999in}{3.969510in}}%
\pgfpathlineto{\pgfqpoint{1.301923in}{3.972346in}}%
\pgfpathlineto{\pgfqpoint{1.303846in}{3.973033in}}%
\pgfpathlineto{\pgfqpoint{1.305770in}{3.963015in}}%
\pgfpathlineto{\pgfqpoint{1.311540in}{4.000036in}}%
\pgfpathlineto{\pgfqpoint{1.313463in}{3.994934in}}%
\pgfpathlineto{\pgfqpoint{1.315387in}{3.999981in}}%
\pgfpathlineto{\pgfqpoint{1.321157in}{3.978399in}}%
\pgfpathlineto{\pgfqpoint{1.323080in}{3.976262in}}%
\pgfpathlineto{\pgfqpoint{1.325004in}{3.978342in}}%
\pgfpathlineto{\pgfqpoint{1.326927in}{3.967117in}}%
\pgfpathlineto{\pgfqpoint{1.332697in}{3.999982in}}%
\pgfpathlineto{\pgfqpoint{1.334621in}{4.004335in}}%
\pgfpathlineto{\pgfqpoint{1.336544in}{4.012462in}}%
\pgfpathlineto{\pgfqpoint{1.338468in}{4.010329in}}%
\pgfpathlineto{\pgfqpoint{1.340391in}{4.010212in}}%
\pgfpathlineto{\pgfqpoint{1.342314in}{4.015548in}}%
\pgfpathlineto{\pgfqpoint{1.344238in}{4.009781in}}%
\pgfpathlineto{\pgfqpoint{1.346161in}{4.019054in}}%
\pgfpathlineto{\pgfqpoint{1.350008in}{4.003275in}}%
\pgfpathlineto{\pgfqpoint{1.353855in}{3.973854in}}%
\pgfpathlineto{\pgfqpoint{1.355778in}{3.975273in}}%
\pgfpathlineto{\pgfqpoint{1.357702in}{3.978335in}}%
\pgfpathlineto{\pgfqpoint{1.359625in}{3.991221in}}%
\pgfpathlineto{\pgfqpoint{1.361549in}{3.988986in}}%
\pgfpathlineto{\pgfqpoint{1.363472in}{3.980994in}}%
\pgfpathlineto{\pgfqpoint{1.365395in}{3.983558in}}%
\pgfpathlineto{\pgfqpoint{1.367319in}{3.997876in}}%
\pgfpathlineto{\pgfqpoint{1.369242in}{4.002351in}}%
\pgfpathlineto{\pgfqpoint{1.371166in}{3.997578in}}%
\pgfpathlineto{\pgfqpoint{1.373089in}{4.008690in}}%
\pgfpathlineto{\pgfqpoint{1.375012in}{4.008737in}}%
\pgfpathlineto{\pgfqpoint{1.376936in}{4.012246in}}%
\pgfpathlineto{\pgfqpoint{1.378859in}{4.002422in}}%
\pgfpathlineto{\pgfqpoint{1.380783in}{4.007239in}}%
\pgfpathlineto{\pgfqpoint{1.382706in}{4.000429in}}%
\pgfpathlineto{\pgfqpoint{1.386553in}{4.021682in}}%
\pgfpathlineto{\pgfqpoint{1.388476in}{4.026990in}}%
\pgfpathlineto{\pgfqpoint{1.390400in}{4.021528in}}%
\pgfpathlineto{\pgfqpoint{1.392323in}{4.005786in}}%
\pgfpathlineto{\pgfqpoint{1.394247in}{4.004705in}}%
\pgfpathlineto{\pgfqpoint{1.396170in}{3.993991in}}%
\pgfpathlineto{\pgfqpoint{1.400017in}{3.990319in}}%
\pgfpathlineto{\pgfqpoint{1.401940in}{3.981952in}}%
\pgfpathlineto{\pgfqpoint{1.405787in}{3.952445in}}%
\pgfpathlineto{\pgfqpoint{1.407710in}{3.960091in}}%
\pgfpathlineto{\pgfqpoint{1.411557in}{3.939267in}}%
\pgfpathlineto{\pgfqpoint{1.413481in}{3.940647in}}%
\pgfpathlineto{\pgfqpoint{1.415404in}{3.938925in}}%
\pgfpathlineto{\pgfqpoint{1.417327in}{3.942726in}}%
\pgfpathlineto{\pgfqpoint{1.419251in}{3.940815in}}%
\pgfpathlineto{\pgfqpoint{1.421174in}{3.935572in}}%
\pgfpathlineto{\pgfqpoint{1.423098in}{3.940581in}}%
\pgfpathlineto{\pgfqpoint{1.426945in}{3.912758in}}%
\pgfpathlineto{\pgfqpoint{1.432715in}{3.904682in}}%
\pgfpathlineto{\pgfqpoint{1.434638in}{3.915637in}}%
\pgfpathlineto{\pgfqpoint{1.436562in}{3.917560in}}%
\pgfpathlineto{\pgfqpoint{1.438485in}{3.923447in}}%
\pgfpathlineto{\pgfqpoint{1.440408in}{3.925366in}}%
\pgfpathlineto{\pgfqpoint{1.442332in}{3.906785in}}%
\pgfpathlineto{\pgfqpoint{1.444255in}{3.900432in}}%
\pgfpathlineto{\pgfqpoint{1.450025in}{3.923112in}}%
\pgfpathlineto{\pgfqpoint{1.451949in}{3.910163in}}%
\pgfpathlineto{\pgfqpoint{1.453872in}{3.921607in}}%
\pgfpathlineto{\pgfqpoint{1.455796in}{3.898245in}}%
\pgfpathlineto{\pgfqpoint{1.457719in}{3.892545in}}%
\pgfpathlineto{\pgfqpoint{1.459642in}{3.898180in}}%
\pgfpathlineto{\pgfqpoint{1.465413in}{3.878134in}}%
\pgfpathlineto{\pgfqpoint{1.467336in}{3.881983in}}%
\pgfpathlineto{\pgfqpoint{1.469260in}{3.880108in}}%
\pgfpathlineto{\pgfqpoint{1.471183in}{3.896149in}}%
\pgfpathlineto{\pgfqpoint{1.473106in}{3.882133in}}%
\pgfpathlineto{\pgfqpoint{1.475030in}{3.897522in}}%
\pgfpathlineto{\pgfqpoint{1.476953in}{3.890378in}}%
\pgfpathlineto{\pgfqpoint{1.478877in}{3.894624in}}%
\pgfpathlineto{\pgfqpoint{1.480800in}{3.894709in}}%
\pgfpathlineto{\pgfqpoint{1.482723in}{3.876759in}}%
\pgfpathlineto{\pgfqpoint{1.486570in}{3.872823in}}%
\pgfpathlineto{\pgfqpoint{1.488494in}{3.879849in}}%
\pgfpathlineto{\pgfqpoint{1.490417in}{3.898451in}}%
\pgfpathlineto{\pgfqpoint{1.492340in}{3.884898in}}%
\pgfpathlineto{\pgfqpoint{1.494264in}{3.883310in}}%
\pgfpathlineto{\pgfqpoint{1.496187in}{3.887794in}}%
\pgfpathlineto{\pgfqpoint{1.498111in}{3.885082in}}%
\pgfpathlineto{\pgfqpoint{1.500034in}{3.874727in}}%
\pgfpathlineto{\pgfqpoint{1.501958in}{3.877517in}}%
\pgfpathlineto{\pgfqpoint{1.505804in}{3.874486in}}%
\pgfpathlineto{\pgfqpoint{1.509651in}{3.891875in}}%
\pgfpathlineto{\pgfqpoint{1.511575in}{3.894536in}}%
\pgfpathlineto{\pgfqpoint{1.513498in}{3.890482in}}%
\pgfpathlineto{\pgfqpoint{1.515421in}{3.897218in}}%
\pgfpathlineto{\pgfqpoint{1.517345in}{3.892033in}}%
\pgfpathlineto{\pgfqpoint{1.519268in}{3.879394in}}%
\pgfpathlineto{\pgfqpoint{1.521192in}{3.888183in}}%
\pgfpathlineto{\pgfqpoint{1.523115in}{3.876021in}}%
\pgfpathlineto{\pgfqpoint{1.525038in}{3.878476in}}%
\pgfpathlineto{\pgfqpoint{1.526962in}{3.883797in}}%
\pgfpathlineto{\pgfqpoint{1.528885in}{3.879024in}}%
\pgfpathlineto{\pgfqpoint{1.532732in}{3.891262in}}%
\pgfpathlineto{\pgfqpoint{1.534656in}{3.891771in}}%
\pgfpathlineto{\pgfqpoint{1.536579in}{3.882543in}}%
\pgfpathlineto{\pgfqpoint{1.538502in}{3.882932in}}%
\pgfpathlineto{\pgfqpoint{1.540426in}{3.891685in}}%
\pgfpathlineto{\pgfqpoint{1.542349in}{3.885359in}}%
\pgfpathlineto{\pgfqpoint{1.544273in}{3.897423in}}%
\pgfpathlineto{\pgfqpoint{1.546196in}{3.896509in}}%
\pgfpathlineto{\pgfqpoint{1.548119in}{3.902753in}}%
\pgfpathlineto{\pgfqpoint{1.551966in}{3.942192in}}%
\pgfpathlineto{\pgfqpoint{1.555813in}{3.918164in}}%
\pgfpathlineto{\pgfqpoint{1.557736in}{3.913769in}}%
\pgfpathlineto{\pgfqpoint{1.559660in}{3.916711in}}%
\pgfpathlineto{\pgfqpoint{1.561583in}{3.913052in}}%
\pgfpathlineto{\pgfqpoint{1.563507in}{3.920388in}}%
\pgfpathlineto{\pgfqpoint{1.565430in}{3.920726in}}%
\pgfpathlineto{\pgfqpoint{1.571200in}{3.887977in}}%
\pgfpathlineto{\pgfqpoint{1.573124in}{3.891378in}}%
\pgfpathlineto{\pgfqpoint{1.575047in}{3.882894in}}%
\pgfpathlineto{\pgfqpoint{1.578894in}{3.902724in}}%
\pgfpathlineto{\pgfqpoint{1.580817in}{3.906431in}}%
\pgfpathlineto{\pgfqpoint{1.584664in}{3.882839in}}%
\pgfpathlineto{\pgfqpoint{1.586588in}{3.879024in}}%
\pgfpathlineto{\pgfqpoint{1.588511in}{3.878480in}}%
\pgfpathlineto{\pgfqpoint{1.590434in}{3.870299in}}%
\pgfpathlineto{\pgfqpoint{1.592358in}{3.868684in}}%
\pgfpathlineto{\pgfqpoint{1.594281in}{3.877479in}}%
\pgfpathlineto{\pgfqpoint{1.596205in}{3.891814in}}%
\pgfpathlineto{\pgfqpoint{1.598128in}{3.883080in}}%
\pgfpathlineto{\pgfqpoint{1.600051in}{3.894575in}}%
\pgfpathlineto{\pgfqpoint{1.601975in}{3.888726in}}%
\pgfpathlineto{\pgfqpoint{1.603898in}{3.889370in}}%
\pgfpathlineto{\pgfqpoint{1.607745in}{3.867188in}}%
\pgfpathlineto{\pgfqpoint{1.609669in}{3.874320in}}%
\pgfpathlineto{\pgfqpoint{1.611592in}{3.860085in}}%
\pgfpathlineto{\pgfqpoint{1.613515in}{3.870629in}}%
\pgfpathlineto{\pgfqpoint{1.615439in}{3.870690in}}%
\pgfpathlineto{\pgfqpoint{1.617362in}{3.863582in}}%
\pgfpathlineto{\pgfqpoint{1.619286in}{3.867004in}}%
\pgfpathlineto{\pgfqpoint{1.621209in}{3.861270in}}%
\pgfpathlineto{\pgfqpoint{1.623132in}{3.860902in}}%
\pgfpathlineto{\pgfqpoint{1.625056in}{3.841482in}}%
\pgfpathlineto{\pgfqpoint{1.626979in}{3.842904in}}%
\pgfpathlineto{\pgfqpoint{1.628903in}{3.848115in}}%
\pgfpathlineto{\pgfqpoint{1.630826in}{3.844564in}}%
\pgfpathlineto{\pgfqpoint{1.632749in}{3.836973in}}%
\pgfpathlineto{\pgfqpoint{1.634673in}{3.835172in}}%
\pgfpathlineto{\pgfqpoint{1.636596in}{3.826605in}}%
\pgfpathlineto{\pgfqpoint{1.638520in}{3.835648in}}%
\pgfpathlineto{\pgfqpoint{1.640443in}{3.834680in}}%
\pgfpathlineto{\pgfqpoint{1.642367in}{3.841442in}}%
\pgfpathlineto{\pgfqpoint{1.644290in}{3.835696in}}%
\pgfpathlineto{\pgfqpoint{1.646213in}{3.821057in}}%
\pgfpathlineto{\pgfqpoint{1.648137in}{3.825845in}}%
\pgfpathlineto{\pgfqpoint{1.650060in}{3.826721in}}%
\pgfpathlineto{\pgfqpoint{1.651984in}{3.822058in}}%
\pgfpathlineto{\pgfqpoint{1.655830in}{3.826319in}}%
\pgfpathlineto{\pgfqpoint{1.657754in}{3.820581in}}%
\pgfpathlineto{\pgfqpoint{1.661601in}{3.823901in}}%
\pgfpathlineto{\pgfqpoint{1.663524in}{3.828976in}}%
\pgfpathlineto{\pgfqpoint{1.665447in}{3.838452in}}%
\pgfpathlineto{\pgfqpoint{1.669294in}{3.843695in}}%
\pgfpathlineto{\pgfqpoint{1.671218in}{3.835711in}}%
\pgfpathlineto{\pgfqpoint{1.673141in}{3.833850in}}%
\pgfpathlineto{\pgfqpoint{1.675065in}{3.849987in}}%
\pgfpathlineto{\pgfqpoint{1.676988in}{3.855183in}}%
\pgfpathlineto{\pgfqpoint{1.678911in}{3.854036in}}%
\pgfpathlineto{\pgfqpoint{1.680835in}{3.847766in}}%
\pgfpathlineto{\pgfqpoint{1.682758in}{3.847336in}}%
\pgfpathlineto{\pgfqpoint{1.684682in}{3.838932in}}%
\pgfpathlineto{\pgfqpoint{1.686605in}{3.851699in}}%
\pgfpathlineto{\pgfqpoint{1.688528in}{3.854259in}}%
\pgfpathlineto{\pgfqpoint{1.692375in}{3.881278in}}%
\pgfpathlineto{\pgfqpoint{1.694299in}{3.864426in}}%
\pgfpathlineto{\pgfqpoint{1.696222in}{3.873593in}}%
\pgfpathlineto{\pgfqpoint{1.700069in}{3.844855in}}%
\pgfpathlineto{\pgfqpoint{1.703916in}{3.857137in}}%
\pgfpathlineto{\pgfqpoint{1.705839in}{3.842253in}}%
\pgfpathlineto{\pgfqpoint{1.709686in}{3.844652in}}%
\pgfpathlineto{\pgfqpoint{1.711609in}{3.824379in}}%
\pgfpathlineto{\pgfqpoint{1.715456in}{3.825117in}}%
\pgfpathlineto{\pgfqpoint{1.717380in}{3.814876in}}%
\pgfpathlineto{\pgfqpoint{1.719303in}{3.815775in}}%
\pgfpathlineto{\pgfqpoint{1.721226in}{3.812146in}}%
\pgfpathlineto{\pgfqpoint{1.723150in}{3.825988in}}%
\pgfpathlineto{\pgfqpoint{1.726997in}{3.802674in}}%
\pgfpathlineto{\pgfqpoint{1.732767in}{3.819315in}}%
\pgfpathlineto{\pgfqpoint{1.736614in}{3.802028in}}%
\pgfpathlineto{\pgfqpoint{1.738537in}{3.796430in}}%
\pgfpathlineto{\pgfqpoint{1.740461in}{3.804668in}}%
\pgfpathlineto{\pgfqpoint{1.742384in}{3.800707in}}%
\pgfpathlineto{\pgfqpoint{1.744307in}{3.801662in}}%
\pgfpathlineto{\pgfqpoint{1.746231in}{3.819444in}}%
\pgfpathlineto{\pgfqpoint{1.748154in}{3.816221in}}%
\pgfpathlineto{\pgfqpoint{1.750078in}{3.822612in}}%
\pgfpathlineto{\pgfqpoint{1.752001in}{3.822797in}}%
\pgfpathlineto{\pgfqpoint{1.753924in}{3.832341in}}%
\pgfpathlineto{\pgfqpoint{1.755848in}{3.826503in}}%
\pgfpathlineto{\pgfqpoint{1.757771in}{3.826415in}}%
\pgfpathlineto{\pgfqpoint{1.761618in}{3.810337in}}%
\pgfpathlineto{\pgfqpoint{1.763541in}{3.814947in}}%
\pgfpathlineto{\pgfqpoint{1.767388in}{3.840758in}}%
\pgfpathlineto{\pgfqpoint{1.769312in}{3.839948in}}%
\pgfpathlineto{\pgfqpoint{1.773158in}{3.851438in}}%
\pgfpathlineto{\pgfqpoint{1.775082in}{3.838253in}}%
\pgfpathlineto{\pgfqpoint{1.777005in}{3.846624in}}%
\pgfpathlineto{\pgfqpoint{1.778929in}{3.831745in}}%
\pgfpathlineto{\pgfqpoint{1.780852in}{3.825552in}}%
\pgfpathlineto{\pgfqpoint{1.784699in}{3.806361in}}%
\pgfpathlineto{\pgfqpoint{1.786622in}{3.799213in}}%
\pgfpathlineto{\pgfqpoint{1.788546in}{3.785501in}}%
\pgfpathlineto{\pgfqpoint{1.790469in}{3.793427in}}%
\pgfpathlineto{\pgfqpoint{1.792393in}{3.795704in}}%
\pgfpathlineto{\pgfqpoint{1.796239in}{3.790334in}}%
\pgfpathlineto{\pgfqpoint{1.798163in}{3.796640in}}%
\pgfpathlineto{\pgfqpoint{1.800086in}{3.796254in}}%
\pgfpathlineto{\pgfqpoint{1.802010in}{3.773701in}}%
\pgfpathlineto{\pgfqpoint{1.803933in}{3.768514in}}%
\pgfpathlineto{\pgfqpoint{1.805856in}{3.752481in}}%
\pgfpathlineto{\pgfqpoint{1.807780in}{3.745906in}}%
\pgfpathlineto{\pgfqpoint{1.813550in}{3.773913in}}%
\pgfpathlineto{\pgfqpoint{1.815474in}{3.773561in}}%
\pgfpathlineto{\pgfqpoint{1.817397in}{3.774345in}}%
\pgfpathlineto{\pgfqpoint{1.819320in}{3.767519in}}%
\pgfpathlineto{\pgfqpoint{1.821244in}{3.764840in}}%
\pgfpathlineto{\pgfqpoint{1.825091in}{3.775258in}}%
\pgfpathlineto{\pgfqpoint{1.827014in}{3.772918in}}%
\pgfpathlineto{\pgfqpoint{1.830861in}{3.794594in}}%
\pgfpathlineto{\pgfqpoint{1.832784in}{3.787569in}}%
\pgfpathlineto{\pgfqpoint{1.834708in}{3.789226in}}%
\pgfpathlineto{\pgfqpoint{1.838554in}{3.807786in}}%
\pgfpathlineto{\pgfqpoint{1.840478in}{3.805014in}}%
\pgfpathlineto{\pgfqpoint{1.842401in}{3.806801in}}%
\pgfpathlineto{\pgfqpoint{1.844325in}{3.794402in}}%
\pgfpathlineto{\pgfqpoint{1.846248in}{3.772355in}}%
\pgfpathlineto{\pgfqpoint{1.850095in}{3.785150in}}%
\pgfpathlineto{\pgfqpoint{1.852018in}{3.782191in}}%
\pgfpathlineto{\pgfqpoint{1.853942in}{3.792908in}}%
\pgfpathlineto{\pgfqpoint{1.855865in}{3.785919in}}%
\pgfpathlineto{\pgfqpoint{1.857789in}{3.792460in}}%
\pgfpathlineto{\pgfqpoint{1.859712in}{3.771554in}}%
\pgfpathlineto{\pgfqpoint{1.861635in}{3.780141in}}%
\pgfpathlineto{\pgfqpoint{1.863559in}{3.776953in}}%
\pgfpathlineto{\pgfqpoint{1.865482in}{3.769237in}}%
\pgfpathlineto{\pgfqpoint{1.867406in}{3.776921in}}%
\pgfpathlineto{\pgfqpoint{1.869329in}{3.773740in}}%
\pgfpathlineto{\pgfqpoint{1.873176in}{3.737452in}}%
\pgfpathlineto{\pgfqpoint{1.875099in}{3.743835in}}%
\pgfpathlineto{\pgfqpoint{1.877023in}{3.760723in}}%
\pgfpathlineto{\pgfqpoint{1.878946in}{3.757974in}}%
\pgfpathlineto{\pgfqpoint{1.880870in}{3.736191in}}%
\pgfpathlineto{\pgfqpoint{1.882793in}{3.730596in}}%
\pgfpathlineto{\pgfqpoint{1.884716in}{3.741959in}}%
\pgfpathlineto{\pgfqpoint{1.886640in}{3.736288in}}%
\pgfpathlineto{\pgfqpoint{1.888563in}{3.739563in}}%
\pgfpathlineto{\pgfqpoint{1.890487in}{3.726155in}}%
\pgfpathlineto{\pgfqpoint{1.892410in}{3.729413in}}%
\pgfpathlineto{\pgfqpoint{1.894333in}{3.735783in}}%
\pgfpathlineto{\pgfqpoint{1.896257in}{3.731456in}}%
\pgfpathlineto{\pgfqpoint{1.898180in}{3.708593in}}%
\pgfpathlineto{\pgfqpoint{1.900104in}{3.699496in}}%
\pgfpathlineto{\pgfqpoint{1.902027in}{3.698498in}}%
\pgfpathlineto{\pgfqpoint{1.905874in}{3.673830in}}%
\pgfpathlineto{\pgfqpoint{1.907797in}{3.676360in}}%
\pgfpathlineto{\pgfqpoint{1.911644in}{3.672332in}}%
\pgfpathlineto{\pgfqpoint{1.913567in}{3.675085in}}%
\pgfpathlineto{\pgfqpoint{1.915491in}{3.672365in}}%
\pgfpathlineto{\pgfqpoint{1.917414in}{3.681012in}}%
\pgfpathlineto{\pgfqpoint{1.919338in}{3.681877in}}%
\pgfpathlineto{\pgfqpoint{1.925108in}{3.640129in}}%
\pgfpathlineto{\pgfqpoint{1.927031in}{3.643568in}}%
\pgfpathlineto{\pgfqpoint{1.928955in}{3.626031in}}%
\pgfpathlineto{\pgfqpoint{1.932802in}{3.629113in}}%
\pgfpathlineto{\pgfqpoint{1.936648in}{3.649632in}}%
\pgfpathlineto{\pgfqpoint{1.938572in}{3.643079in}}%
\pgfpathlineto{\pgfqpoint{1.940495in}{3.642961in}}%
\pgfpathlineto{\pgfqpoint{1.942419in}{3.638646in}}%
\pgfpathlineto{\pgfqpoint{1.944342in}{3.637063in}}%
\pgfpathlineto{\pgfqpoint{1.946265in}{3.631881in}}%
\pgfpathlineto{\pgfqpoint{1.948189in}{3.639625in}}%
\pgfpathlineto{\pgfqpoint{1.950112in}{3.631549in}}%
\pgfpathlineto{\pgfqpoint{1.952036in}{3.639972in}}%
\pgfpathlineto{\pgfqpoint{1.955883in}{3.621770in}}%
\pgfpathlineto{\pgfqpoint{1.957806in}{3.618229in}}%
\pgfpathlineto{\pgfqpoint{1.959729in}{3.619590in}}%
\pgfpathlineto{\pgfqpoint{1.961653in}{3.632327in}}%
\pgfpathlineto{\pgfqpoint{1.963576in}{3.632686in}}%
\pgfpathlineto{\pgfqpoint{1.965500in}{3.643374in}}%
\pgfpathlineto{\pgfqpoint{1.967423in}{3.639024in}}%
\pgfpathlineto{\pgfqpoint{1.978963in}{3.588611in}}%
\pgfpathlineto{\pgfqpoint{1.980887in}{3.591493in}}%
\pgfpathlineto{\pgfqpoint{1.982810in}{3.576796in}}%
\pgfpathlineto{\pgfqpoint{1.984734in}{3.574864in}}%
\pgfpathlineto{\pgfqpoint{1.986657in}{3.579606in}}%
\pgfpathlineto{\pgfqpoint{1.988581in}{3.576236in}}%
\pgfpathlineto{\pgfqpoint{1.990504in}{3.562581in}}%
\pgfpathlineto{\pgfqpoint{1.994351in}{3.569612in}}%
\pgfpathlineto{\pgfqpoint{1.996274in}{3.558973in}}%
\pgfpathlineto{\pgfqpoint{1.998198in}{3.563863in}}%
\pgfpathlineto{\pgfqpoint{2.002044in}{3.581228in}}%
\pgfpathlineto{\pgfqpoint{2.003968in}{3.573263in}}%
\pgfpathlineto{\pgfqpoint{2.005891in}{3.574258in}}%
\pgfpathlineto{\pgfqpoint{2.011661in}{3.585809in}}%
\pgfpathlineto{\pgfqpoint{2.013585in}{3.573451in}}%
\pgfpathlineto{\pgfqpoint{2.015508in}{3.571842in}}%
\pgfpathlineto{\pgfqpoint{2.017432in}{3.584698in}}%
\pgfpathlineto{\pgfqpoint{2.019355in}{3.575534in}}%
\pgfpathlineto{\pgfqpoint{2.021279in}{3.574981in}}%
\pgfpathlineto{\pgfqpoint{2.023202in}{3.582062in}}%
\pgfpathlineto{\pgfqpoint{2.025125in}{3.561358in}}%
\pgfpathlineto{\pgfqpoint{2.027049in}{3.553908in}}%
\pgfpathlineto{\pgfqpoint{2.028972in}{3.539713in}}%
\pgfpathlineto{\pgfqpoint{2.034742in}{3.572819in}}%
\pgfpathlineto{\pgfqpoint{2.036666in}{3.573144in}}%
\pgfpathlineto{\pgfqpoint{2.040513in}{3.578333in}}%
\pgfpathlineto{\pgfqpoint{2.042436in}{3.576902in}}%
\pgfpathlineto{\pgfqpoint{2.044359in}{3.571751in}}%
\pgfpathlineto{\pgfqpoint{2.046283in}{3.579778in}}%
\pgfpathlineto{\pgfqpoint{2.050130in}{3.562341in}}%
\pgfpathlineto{\pgfqpoint{2.052053in}{3.565321in}}%
\pgfpathlineto{\pgfqpoint{2.053976in}{3.558480in}}%
\pgfpathlineto{\pgfqpoint{2.055900in}{3.559325in}}%
\pgfpathlineto{\pgfqpoint{2.059747in}{3.545052in}}%
\pgfpathlineto{\pgfqpoint{2.063594in}{3.548469in}}%
\pgfpathlineto{\pgfqpoint{2.065517in}{3.552500in}}%
\pgfpathlineto{\pgfqpoint{2.067440in}{3.552746in}}%
\pgfpathlineto{\pgfqpoint{2.071287in}{3.547312in}}%
\pgfpathlineto{\pgfqpoint{2.073211in}{3.561521in}}%
\pgfpathlineto{\pgfqpoint{2.075134in}{3.556797in}}%
\pgfpathlineto{\pgfqpoint{2.077057in}{3.555204in}}%
\pgfpathlineto{\pgfqpoint{2.078981in}{3.545984in}}%
\pgfpathlineto{\pgfqpoint{2.086674in}{3.567240in}}%
\pgfpathlineto{\pgfqpoint{2.090521in}{3.545892in}}%
\pgfpathlineto{\pgfqpoint{2.092445in}{3.551393in}}%
\pgfpathlineto{\pgfqpoint{2.096292in}{3.543525in}}%
\pgfpathlineto{\pgfqpoint{2.098215in}{3.553673in}}%
\pgfpathlineto{\pgfqpoint{2.100138in}{3.544878in}}%
\pgfpathlineto{\pgfqpoint{2.102062in}{3.543770in}}%
\pgfpathlineto{\pgfqpoint{2.103985in}{3.557415in}}%
\pgfpathlineto{\pgfqpoint{2.105909in}{3.557501in}}%
\pgfpathlineto{\pgfqpoint{2.107832in}{3.568876in}}%
\pgfpathlineto{\pgfqpoint{2.109755in}{3.564155in}}%
\pgfpathlineto{\pgfqpoint{2.111679in}{3.569655in}}%
\pgfpathlineto{\pgfqpoint{2.113602in}{3.582423in}}%
\pgfpathlineto{\pgfqpoint{2.115526in}{3.574958in}}%
\pgfpathlineto{\pgfqpoint{2.119372in}{3.595769in}}%
\pgfpathlineto{\pgfqpoint{2.121296in}{3.601644in}}%
\pgfpathlineto{\pgfqpoint{2.125143in}{3.605826in}}%
\pgfpathlineto{\pgfqpoint{2.127066in}{3.623013in}}%
\pgfpathlineto{\pgfqpoint{2.128990in}{3.618749in}}%
\pgfpathlineto{\pgfqpoint{2.130913in}{3.621432in}}%
\pgfpathlineto{\pgfqpoint{2.132836in}{3.632242in}}%
\pgfpathlineto{\pgfqpoint{2.134760in}{3.608114in}}%
\pgfpathlineto{\pgfqpoint{2.136683in}{3.605929in}}%
\pgfpathlineto{\pgfqpoint{2.138607in}{3.595958in}}%
\pgfpathlineto{\pgfqpoint{2.142453in}{3.602779in}}%
\pgfpathlineto{\pgfqpoint{2.146300in}{3.591028in}}%
\pgfpathlineto{\pgfqpoint{2.148224in}{3.594186in}}%
\pgfpathlineto{\pgfqpoint{2.150147in}{3.578753in}}%
\pgfpathlineto{\pgfqpoint{2.152070in}{3.577032in}}%
\pgfpathlineto{\pgfqpoint{2.153994in}{3.587851in}}%
\pgfpathlineto{\pgfqpoint{2.155917in}{3.589443in}}%
\pgfpathlineto{\pgfqpoint{2.157841in}{3.597033in}}%
\pgfpathlineto{\pgfqpoint{2.159764in}{3.593580in}}%
\pgfpathlineto{\pgfqpoint{2.163611in}{3.592903in}}%
\pgfpathlineto{\pgfqpoint{2.165534in}{3.594053in}}%
\pgfpathlineto{\pgfqpoint{2.167458in}{3.588519in}}%
\pgfpathlineto{\pgfqpoint{2.171305in}{3.618081in}}%
\pgfpathlineto{\pgfqpoint{2.173228in}{3.614092in}}%
\pgfpathlineto{\pgfqpoint{2.175151in}{3.615605in}}%
\pgfpathlineto{\pgfqpoint{2.177075in}{3.608133in}}%
\pgfpathlineto{\pgfqpoint{2.178998in}{3.594198in}}%
\pgfpathlineto{\pgfqpoint{2.180922in}{3.601835in}}%
\pgfpathlineto{\pgfqpoint{2.182845in}{3.603253in}}%
\pgfpathlineto{\pgfqpoint{2.184768in}{3.601765in}}%
\pgfpathlineto{\pgfqpoint{2.188615in}{3.618504in}}%
\pgfpathlineto{\pgfqpoint{2.192462in}{3.600874in}}%
\pgfpathlineto{\pgfqpoint{2.196309in}{3.592071in}}%
\pgfpathlineto{\pgfqpoint{2.198232in}{3.595306in}}%
\pgfpathlineto{\pgfqpoint{2.200156in}{3.600940in}}%
\pgfpathlineto{\pgfqpoint{2.205926in}{3.576328in}}%
\pgfpathlineto{\pgfqpoint{2.209773in}{3.567119in}}%
\pgfpathlineto{\pgfqpoint{2.211696in}{3.567181in}}%
\pgfpathlineto{\pgfqpoint{2.213620in}{3.560443in}}%
\pgfpathlineto{\pgfqpoint{2.215543in}{3.547860in}}%
\pgfpathlineto{\pgfqpoint{2.219390in}{3.544503in}}%
\pgfpathlineto{\pgfqpoint{2.221313in}{3.551441in}}%
\pgfpathlineto{\pgfqpoint{2.223237in}{3.551556in}}%
\pgfpathlineto{\pgfqpoint{2.229007in}{3.544019in}}%
\pgfpathlineto{\pgfqpoint{2.232854in}{3.527230in}}%
\pgfpathlineto{\pgfqpoint{2.234777in}{3.526612in}}%
\pgfpathlineto{\pgfqpoint{2.236701in}{3.548349in}}%
\pgfpathlineto{\pgfqpoint{2.238624in}{3.542648in}}%
\pgfpathlineto{\pgfqpoint{2.240547in}{3.544109in}}%
\pgfpathlineto{\pgfqpoint{2.242471in}{3.532678in}}%
\pgfpathlineto{\pgfqpoint{2.244394in}{3.541612in}}%
\pgfpathlineto{\pgfqpoint{2.246318in}{3.534588in}}%
\pgfpathlineto{\pgfqpoint{2.248241in}{3.518690in}}%
\pgfpathlineto{\pgfqpoint{2.250164in}{3.514994in}}%
\pgfpathlineto{\pgfqpoint{2.252088in}{3.516425in}}%
\pgfpathlineto{\pgfqpoint{2.254011in}{3.511632in}}%
\pgfpathlineto{\pgfqpoint{2.255935in}{3.512019in}}%
\pgfpathlineto{\pgfqpoint{2.257858in}{3.501279in}}%
\pgfpathlineto{\pgfqpoint{2.259781in}{3.479773in}}%
\pgfpathlineto{\pgfqpoint{2.261705in}{3.498891in}}%
\pgfpathlineto{\pgfqpoint{2.263628in}{3.500860in}}%
\pgfpathlineto{\pgfqpoint{2.265552in}{3.499523in}}%
\pgfpathlineto{\pgfqpoint{2.267475in}{3.501070in}}%
\pgfpathlineto{\pgfqpoint{2.269399in}{3.505986in}}%
\pgfpathlineto{\pgfqpoint{2.273245in}{3.482631in}}%
\pgfpathlineto{\pgfqpoint{2.275169in}{3.489535in}}%
\pgfpathlineto{\pgfqpoint{2.277092in}{3.488314in}}%
\pgfpathlineto{\pgfqpoint{2.279016in}{3.492659in}}%
\pgfpathlineto{\pgfqpoint{2.282862in}{3.486655in}}%
\pgfpathlineto{\pgfqpoint{2.284786in}{3.488191in}}%
\pgfpathlineto{\pgfqpoint{2.290556in}{3.466034in}}%
\pgfpathlineto{\pgfqpoint{2.292479in}{3.488159in}}%
\pgfpathlineto{\pgfqpoint{2.296326in}{3.505591in}}%
\pgfpathlineto{\pgfqpoint{2.298250in}{3.490092in}}%
\pgfpathlineto{\pgfqpoint{2.300173in}{3.486469in}}%
\pgfpathlineto{\pgfqpoint{2.302097in}{3.489291in}}%
\pgfpathlineto{\pgfqpoint{2.304020in}{3.477942in}}%
\pgfpathlineto{\pgfqpoint{2.305943in}{3.479874in}}%
\pgfpathlineto{\pgfqpoint{2.309790in}{3.467796in}}%
\pgfpathlineto{\pgfqpoint{2.311714in}{3.476861in}}%
\pgfpathlineto{\pgfqpoint{2.313637in}{3.479974in}}%
\pgfpathlineto{\pgfqpoint{2.319407in}{3.511897in}}%
\pgfpathlineto{\pgfqpoint{2.321331in}{3.501423in}}%
\pgfpathlineto{\pgfqpoint{2.323254in}{3.511030in}}%
\pgfpathlineto{\pgfqpoint{2.325177in}{3.514164in}}%
\pgfpathlineto{\pgfqpoint{2.327101in}{3.506575in}}%
\pgfpathlineto{\pgfqpoint{2.329024in}{3.507374in}}%
\pgfpathlineto{\pgfqpoint{2.330948in}{3.492895in}}%
\pgfpathlineto{\pgfqpoint{2.332871in}{3.487928in}}%
\pgfpathlineto{\pgfqpoint{2.334795in}{3.478547in}}%
\pgfpathlineto{\pgfqpoint{2.336718in}{3.478913in}}%
\pgfpathlineto{\pgfqpoint{2.338641in}{3.474809in}}%
\pgfpathlineto{\pgfqpoint{2.342488in}{3.437347in}}%
\pgfpathlineto{\pgfqpoint{2.346335in}{3.442830in}}%
\pgfpathlineto{\pgfqpoint{2.348258in}{3.437329in}}%
\pgfpathlineto{\pgfqpoint{2.350182in}{3.425322in}}%
\pgfpathlineto{\pgfqpoint{2.352105in}{3.429524in}}%
\pgfpathlineto{\pgfqpoint{2.355952in}{3.413751in}}%
\pgfpathlineto{\pgfqpoint{2.357875in}{3.413505in}}%
\pgfpathlineto{\pgfqpoint{2.359799in}{3.408102in}}%
\pgfpathlineto{\pgfqpoint{2.361722in}{3.410393in}}%
\pgfpathlineto{\pgfqpoint{2.363646in}{3.429380in}}%
\pgfpathlineto{\pgfqpoint{2.365569in}{3.432389in}}%
\pgfpathlineto{\pgfqpoint{2.367492in}{3.432246in}}%
\pgfpathlineto{\pgfqpoint{2.371339in}{3.424457in}}%
\pgfpathlineto{\pgfqpoint{2.373263in}{3.430056in}}%
\pgfpathlineto{\pgfqpoint{2.375186in}{3.427411in}}%
\pgfpathlineto{\pgfqpoint{2.379033in}{3.412617in}}%
\pgfpathlineto{\pgfqpoint{2.380956in}{3.403163in}}%
\pgfpathlineto{\pgfqpoint{2.382880in}{3.407449in}}%
\pgfpathlineto{\pgfqpoint{2.386727in}{3.438418in}}%
\pgfpathlineto{\pgfqpoint{2.390573in}{3.455998in}}%
\pgfpathlineto{\pgfqpoint{2.392497in}{3.431886in}}%
\pgfpathlineto{\pgfqpoint{2.394420in}{3.434912in}}%
\pgfpathlineto{\pgfqpoint{2.396344in}{3.435564in}}%
\pgfpathlineto{\pgfqpoint{2.398267in}{3.450697in}}%
\pgfpathlineto{\pgfqpoint{2.402114in}{3.426164in}}%
\pgfpathlineto{\pgfqpoint{2.404037in}{3.425573in}}%
\pgfpathlineto{\pgfqpoint{2.405961in}{3.426145in}}%
\pgfpathlineto{\pgfqpoint{2.407884in}{3.423841in}}%
\pgfpathlineto{\pgfqpoint{2.409808in}{3.428099in}}%
\pgfpathlineto{\pgfqpoint{2.411731in}{3.421436in}}%
\pgfpathlineto{\pgfqpoint{2.413654in}{3.410472in}}%
\pgfpathlineto{\pgfqpoint{2.415578in}{3.412152in}}%
\pgfpathlineto{\pgfqpoint{2.417501in}{3.420111in}}%
\pgfpathlineto{\pgfqpoint{2.421348in}{3.411995in}}%
\pgfpathlineto{\pgfqpoint{2.423271in}{3.415548in}}%
\pgfpathlineto{\pgfqpoint{2.427118in}{3.407950in}}%
\pgfpathlineto{\pgfqpoint{2.429042in}{3.394143in}}%
\pgfpathlineto{\pgfqpoint{2.430965in}{3.396598in}}%
\pgfpathlineto{\pgfqpoint{2.432888in}{3.394061in}}%
\pgfpathlineto{\pgfqpoint{2.434812in}{3.400828in}}%
\pgfpathlineto{\pgfqpoint{2.436735in}{3.396918in}}%
\pgfpathlineto{\pgfqpoint{2.438659in}{3.399718in}}%
\pgfpathlineto{\pgfqpoint{2.440582in}{3.415297in}}%
\pgfpathlineto{\pgfqpoint{2.442506in}{3.417658in}}%
\pgfpathlineto{\pgfqpoint{2.444429in}{3.431125in}}%
\pgfpathlineto{\pgfqpoint{2.446352in}{3.424384in}}%
\pgfpathlineto{\pgfqpoint{2.450199in}{3.452258in}}%
\pgfpathlineto{\pgfqpoint{2.452123in}{3.452475in}}%
\pgfpathlineto{\pgfqpoint{2.454046in}{3.438405in}}%
\pgfpathlineto{\pgfqpoint{2.455969in}{3.446047in}}%
\pgfpathlineto{\pgfqpoint{2.457893in}{3.444428in}}%
\pgfpathlineto{\pgfqpoint{2.459816in}{3.438222in}}%
\pgfpathlineto{\pgfqpoint{2.461740in}{3.439263in}}%
\pgfpathlineto{\pgfqpoint{2.463663in}{3.434702in}}%
\pgfpathlineto{\pgfqpoint{2.465586in}{3.425393in}}%
\pgfpathlineto{\pgfqpoint{2.467510in}{3.438547in}}%
\pgfpathlineto{\pgfqpoint{2.469433in}{3.441531in}}%
\pgfpathlineto{\pgfqpoint{2.471357in}{3.441807in}}%
\pgfpathlineto{\pgfqpoint{2.479050in}{3.393383in}}%
\pgfpathlineto{\pgfqpoint{2.484821in}{3.403444in}}%
\pgfpathlineto{\pgfqpoint{2.486744in}{3.414818in}}%
\pgfpathlineto{\pgfqpoint{2.488667in}{3.414016in}}%
\pgfpathlineto{\pgfqpoint{2.490591in}{3.396168in}}%
\pgfpathlineto{\pgfqpoint{2.492514in}{3.397333in}}%
\pgfpathlineto{\pgfqpoint{2.494438in}{3.389375in}}%
\pgfpathlineto{\pgfqpoint{2.496361in}{3.396021in}}%
\pgfpathlineto{\pgfqpoint{2.498284in}{3.397777in}}%
\pgfpathlineto{\pgfqpoint{2.502131in}{3.416368in}}%
\pgfpathlineto{\pgfqpoint{2.507901in}{3.390964in}}%
\pgfpathlineto{\pgfqpoint{2.509825in}{3.403151in}}%
\pgfpathlineto{\pgfqpoint{2.513672in}{3.392408in}}%
\pgfpathlineto{\pgfqpoint{2.515595in}{3.408495in}}%
\pgfpathlineto{\pgfqpoint{2.517519in}{3.407916in}}%
\pgfpathlineto{\pgfqpoint{2.523289in}{3.449324in}}%
\pgfpathlineto{\pgfqpoint{2.525212in}{3.451678in}}%
\pgfpathlineto{\pgfqpoint{2.527136in}{3.451200in}}%
\pgfpathlineto{\pgfqpoint{2.530982in}{3.437393in}}%
\pgfpathlineto{\pgfqpoint{2.532906in}{3.437442in}}%
\pgfpathlineto{\pgfqpoint{2.536753in}{3.422995in}}%
\pgfpathlineto{\pgfqpoint{2.538676in}{3.420536in}}%
\pgfpathlineto{\pgfqpoint{2.540599in}{3.421388in}}%
\pgfpathlineto{\pgfqpoint{2.542523in}{3.416898in}}%
\pgfpathlineto{\pgfqpoint{2.544446in}{3.427034in}}%
\pgfpathlineto{\pgfqpoint{2.546370in}{3.410119in}}%
\pgfpathlineto{\pgfqpoint{2.548293in}{3.414372in}}%
\pgfpathlineto{\pgfqpoint{2.550217in}{3.389384in}}%
\pgfpathlineto{\pgfqpoint{2.552140in}{3.394093in}}%
\pgfpathlineto{\pgfqpoint{2.554063in}{3.404539in}}%
\pgfpathlineto{\pgfqpoint{2.555987in}{3.402783in}}%
\pgfpathlineto{\pgfqpoint{2.559834in}{3.411340in}}%
\pgfpathlineto{\pgfqpoint{2.563680in}{3.426729in}}%
\pgfpathlineto{\pgfqpoint{2.565604in}{3.423148in}}%
\pgfpathlineto{\pgfqpoint{2.567527in}{3.429940in}}%
\pgfpathlineto{\pgfqpoint{2.569451in}{3.422250in}}%
\pgfpathlineto{\pgfqpoint{2.573297in}{3.425542in}}%
\pgfpathlineto{\pgfqpoint{2.575221in}{3.435714in}}%
\pgfpathlineto{\pgfqpoint{2.577144in}{3.429866in}}%
\pgfpathlineto{\pgfqpoint{2.584838in}{3.467799in}}%
\pgfpathlineto{\pgfqpoint{2.586761in}{3.469963in}}%
\pgfpathlineto{\pgfqpoint{2.588685in}{3.466147in}}%
\pgfpathlineto{\pgfqpoint{2.590608in}{3.470105in}}%
\pgfpathlineto{\pgfqpoint{2.592532in}{3.457261in}}%
\pgfpathlineto{\pgfqpoint{2.596378in}{3.458387in}}%
\pgfpathlineto{\pgfqpoint{2.598302in}{3.463140in}}%
\pgfpathlineto{\pgfqpoint{2.600225in}{3.460253in}}%
\pgfpathlineto{\pgfqpoint{2.602149in}{3.446934in}}%
\pgfpathlineto{\pgfqpoint{2.604072in}{3.442938in}}%
\pgfpathlineto{\pgfqpoint{2.605995in}{3.429025in}}%
\pgfpathlineto{\pgfqpoint{2.607919in}{3.425621in}}%
\pgfpathlineto{\pgfqpoint{2.609842in}{3.435152in}}%
\pgfpathlineto{\pgfqpoint{2.611766in}{3.424560in}}%
\pgfpathlineto{\pgfqpoint{2.615613in}{3.419091in}}%
\pgfpathlineto{\pgfqpoint{2.617536in}{3.413639in}}%
\pgfpathlineto{\pgfqpoint{2.619459in}{3.420882in}}%
\pgfpathlineto{\pgfqpoint{2.623306in}{3.413593in}}%
\pgfpathlineto{\pgfqpoint{2.627153in}{3.392315in}}%
\pgfpathlineto{\pgfqpoint{2.629076in}{3.408425in}}%
\pgfpathlineto{\pgfqpoint{2.631000in}{3.408611in}}%
\pgfpathlineto{\pgfqpoint{2.634847in}{3.398826in}}%
\pgfpathlineto{\pgfqpoint{2.636770in}{3.399689in}}%
\pgfpathlineto{\pgfqpoint{2.638693in}{3.390434in}}%
\pgfpathlineto{\pgfqpoint{2.640617in}{3.389911in}}%
\pgfpathlineto{\pgfqpoint{2.644464in}{3.373224in}}%
\pgfpathlineto{\pgfqpoint{2.648311in}{3.389141in}}%
\pgfpathlineto{\pgfqpoint{2.650234in}{3.393651in}}%
\pgfpathlineto{\pgfqpoint{2.656004in}{3.388020in}}%
\pgfpathlineto{\pgfqpoint{2.661774in}{3.399896in}}%
\pgfpathlineto{\pgfqpoint{2.667545in}{3.365036in}}%
\pgfpathlineto{\pgfqpoint{2.669468in}{3.375780in}}%
\pgfpathlineto{\pgfqpoint{2.671391in}{3.372800in}}%
\pgfpathlineto{\pgfqpoint{2.675238in}{3.339889in}}%
\pgfpathlineto{\pgfqpoint{2.677162in}{3.339760in}}%
\pgfpathlineto{\pgfqpoint{2.679085in}{3.323772in}}%
\pgfpathlineto{\pgfqpoint{2.681008in}{3.325425in}}%
\pgfpathlineto{\pgfqpoint{2.682932in}{3.320882in}}%
\pgfpathlineto{\pgfqpoint{2.684855in}{3.334958in}}%
\pgfpathlineto{\pgfqpoint{2.686779in}{3.337068in}}%
\pgfpathlineto{\pgfqpoint{2.690626in}{3.321587in}}%
\pgfpathlineto{\pgfqpoint{2.696396in}{3.313167in}}%
\pgfpathlineto{\pgfqpoint{2.698319in}{3.311425in}}%
\pgfpathlineto{\pgfqpoint{2.700243in}{3.295338in}}%
\pgfpathlineto{\pgfqpoint{2.702166in}{3.299535in}}%
\pgfpathlineto{\pgfqpoint{2.704089in}{3.294584in}}%
\pgfpathlineto{\pgfqpoint{2.706013in}{3.296851in}}%
\pgfpathlineto{\pgfqpoint{2.707936in}{3.306683in}}%
\pgfpathlineto{\pgfqpoint{2.711783in}{3.286587in}}%
\pgfpathlineto{\pgfqpoint{2.713706in}{3.293296in}}%
\pgfpathlineto{\pgfqpoint{2.715630in}{3.281081in}}%
\pgfpathlineto{\pgfqpoint{2.717553in}{3.298129in}}%
\pgfpathlineto{\pgfqpoint{2.719477in}{3.294486in}}%
\pgfpathlineto{\pgfqpoint{2.721400in}{3.275455in}}%
\pgfpathlineto{\pgfqpoint{2.723324in}{3.287463in}}%
\pgfpathlineto{\pgfqpoint{2.725247in}{3.290933in}}%
\pgfpathlineto{\pgfqpoint{2.727170in}{3.302795in}}%
\pgfpathlineto{\pgfqpoint{2.729094in}{3.298723in}}%
\pgfpathlineto{\pgfqpoint{2.732941in}{3.304224in}}%
\pgfpathlineto{\pgfqpoint{2.734864in}{3.301225in}}%
\pgfpathlineto{\pgfqpoint{2.736787in}{3.302879in}}%
\pgfpathlineto{\pgfqpoint{2.738711in}{3.315119in}}%
\pgfpathlineto{\pgfqpoint{2.740634in}{3.311278in}}%
\pgfpathlineto{\pgfqpoint{2.742558in}{3.317785in}}%
\pgfpathlineto{\pgfqpoint{2.744481in}{3.317837in}}%
\pgfpathlineto{\pgfqpoint{2.746404in}{3.313092in}}%
\pgfpathlineto{\pgfqpoint{2.748328in}{3.314497in}}%
\pgfpathlineto{\pgfqpoint{2.750251in}{3.322776in}}%
\pgfpathlineto{\pgfqpoint{2.752175in}{3.324838in}}%
\pgfpathlineto{\pgfqpoint{2.756022in}{3.305471in}}%
\pgfpathlineto{\pgfqpoint{2.757945in}{3.313923in}}%
\pgfpathlineto{\pgfqpoint{2.761792in}{3.342550in}}%
\pgfpathlineto{\pgfqpoint{2.763715in}{3.336485in}}%
\pgfpathlineto{\pgfqpoint{2.765639in}{3.315129in}}%
\pgfpathlineto{\pgfqpoint{2.767562in}{3.324094in}}%
\pgfpathlineto{\pgfqpoint{2.767562in}{3.324094in}}%
\pgfusepath{stroke}%
\end{pgfscope}%
\begin{pgfscope}%
\pgfpathrectangle{\pgfqpoint{0.750000in}{3.180000in}}{\pgfqpoint{2.113636in}{2.100000in}}%
\pgfusepath{clip}%
\pgfsetroundcap%
\pgfsetroundjoin%
\pgfsetlinewidth{0.602250pt}%
\definecolor{currentstroke}{rgb}{1.000000,0.498039,0.000000}%
\pgfsetstrokecolor{currentstroke}%
\pgfsetdash{}{0pt}%
\pgfpathmoveto{\pgfqpoint{0.846074in}{4.060633in}}%
\pgfpathlineto{\pgfqpoint{0.847998in}{4.068528in}}%
\pgfpathlineto{\pgfqpoint{0.849921in}{4.065856in}}%
\pgfpathlineto{\pgfqpoint{0.851845in}{4.057994in}}%
\pgfpathlineto{\pgfqpoint{0.853768in}{4.038251in}}%
\pgfpathlineto{\pgfqpoint{0.859538in}{4.070337in}}%
\pgfpathlineto{\pgfqpoint{0.861462in}{4.068072in}}%
\pgfpathlineto{\pgfqpoint{0.863385in}{4.079882in}}%
\pgfpathlineto{\pgfqpoint{0.865308in}{4.060926in}}%
\pgfpathlineto{\pgfqpoint{0.867232in}{4.063809in}}%
\pgfpathlineto{\pgfqpoint{0.869155in}{4.079497in}}%
\pgfpathlineto{\pgfqpoint{0.871079in}{4.074256in}}%
\pgfpathlineto{\pgfqpoint{0.874926in}{4.055974in}}%
\pgfpathlineto{\pgfqpoint{0.876849in}{4.059981in}}%
\pgfpathlineto{\pgfqpoint{0.878772in}{4.076380in}}%
\pgfpathlineto{\pgfqpoint{0.880696in}{4.074718in}}%
\pgfpathlineto{\pgfqpoint{0.882619in}{4.077218in}}%
\pgfpathlineto{\pgfqpoint{0.890313in}{4.098579in}}%
\pgfpathlineto{\pgfqpoint{0.892236in}{4.096899in}}%
\pgfpathlineto{\pgfqpoint{0.894160in}{4.118738in}}%
\pgfpathlineto{\pgfqpoint{0.896083in}{4.122035in}}%
\pgfpathlineto{\pgfqpoint{0.899930in}{4.091594in}}%
\pgfpathlineto{\pgfqpoint{0.901853in}{4.090403in}}%
\pgfpathlineto{\pgfqpoint{0.905700in}{4.068393in}}%
\pgfpathlineto{\pgfqpoint{0.907624in}{4.064602in}}%
\pgfpathlineto{\pgfqpoint{0.909547in}{4.056537in}}%
\pgfpathlineto{\pgfqpoint{0.911470in}{4.042014in}}%
\pgfpathlineto{\pgfqpoint{0.913394in}{4.047772in}}%
\pgfpathlineto{\pgfqpoint{0.915317in}{4.047238in}}%
\pgfpathlineto{\pgfqpoint{0.917241in}{4.059226in}}%
\pgfpathlineto{\pgfqpoint{0.921087in}{4.061106in}}%
\pgfpathlineto{\pgfqpoint{0.923011in}{4.051216in}}%
\pgfpathlineto{\pgfqpoint{0.924934in}{4.055093in}}%
\pgfpathlineto{\pgfqpoint{0.926858in}{4.040961in}}%
\pgfpathlineto{\pgfqpoint{0.930704in}{4.045508in}}%
\pgfpathlineto{\pgfqpoint{0.932628in}{4.045630in}}%
\pgfpathlineto{\pgfqpoint{0.934551in}{4.043782in}}%
\pgfpathlineto{\pgfqpoint{0.938398in}{4.072401in}}%
\pgfpathlineto{\pgfqpoint{0.940322in}{4.071755in}}%
\pgfpathlineto{\pgfqpoint{0.942245in}{4.074267in}}%
\pgfpathlineto{\pgfqpoint{0.944168in}{4.084352in}}%
\pgfpathlineto{\pgfqpoint{0.948015in}{4.059045in}}%
\pgfpathlineto{\pgfqpoint{0.949939in}{4.067068in}}%
\pgfpathlineto{\pgfqpoint{0.951862in}{4.058865in}}%
\pgfpathlineto{\pgfqpoint{0.953785in}{4.058655in}}%
\pgfpathlineto{\pgfqpoint{0.955709in}{4.083221in}}%
\pgfpathlineto{\pgfqpoint{0.957632in}{4.084232in}}%
\pgfpathlineto{\pgfqpoint{0.959556in}{4.078939in}}%
\pgfpathlineto{\pgfqpoint{0.961479in}{4.081527in}}%
\pgfpathlineto{\pgfqpoint{0.963402in}{4.074512in}}%
\pgfpathlineto{\pgfqpoint{0.967249in}{4.085810in}}%
\pgfpathlineto{\pgfqpoint{0.971096in}{4.078191in}}%
\pgfpathlineto{\pgfqpoint{0.973020in}{4.079922in}}%
\pgfpathlineto{\pgfqpoint{0.974943in}{4.069824in}}%
\pgfpathlineto{\pgfqpoint{0.976866in}{4.075745in}}%
\pgfpathlineto{\pgfqpoint{0.980713in}{4.093693in}}%
\pgfpathlineto{\pgfqpoint{0.982637in}{4.087276in}}%
\pgfpathlineto{\pgfqpoint{0.984560in}{4.085382in}}%
\pgfpathlineto{\pgfqpoint{0.990330in}{4.063220in}}%
\pgfpathlineto{\pgfqpoint{0.992254in}{4.071076in}}%
\pgfpathlineto{\pgfqpoint{0.994177in}{4.071823in}}%
\pgfpathlineto{\pgfqpoint{0.996100in}{4.078727in}}%
\pgfpathlineto{\pgfqpoint{0.998024in}{4.079949in}}%
\pgfpathlineto{\pgfqpoint{0.999947in}{4.074475in}}%
\pgfpathlineto{\pgfqpoint{1.001871in}{4.072463in}}%
\pgfpathlineto{\pgfqpoint{1.003794in}{4.066656in}}%
\pgfpathlineto{\pgfqpoint{1.005717in}{4.055442in}}%
\pgfpathlineto{\pgfqpoint{1.007641in}{4.067933in}}%
\pgfpathlineto{\pgfqpoint{1.009564in}{4.069826in}}%
\pgfpathlineto{\pgfqpoint{1.011488in}{4.054282in}}%
\pgfpathlineto{\pgfqpoint{1.013411in}{4.058058in}}%
\pgfpathlineto{\pgfqpoint{1.015335in}{4.052681in}}%
\pgfpathlineto{\pgfqpoint{1.017258in}{4.065139in}}%
\pgfpathlineto{\pgfqpoint{1.019181in}{4.042019in}}%
\pgfpathlineto{\pgfqpoint{1.023028in}{4.037498in}}%
\pgfpathlineto{\pgfqpoint{1.024952in}{4.031168in}}%
\pgfpathlineto{\pgfqpoint{1.026875in}{4.020542in}}%
\pgfpathlineto{\pgfqpoint{1.034569in}{4.046952in}}%
\pgfpathlineto{\pgfqpoint{1.036492in}{4.056503in}}%
\pgfpathlineto{\pgfqpoint{1.038415in}{4.080631in}}%
\pgfpathlineto{\pgfqpoint{1.040339in}{4.083701in}}%
\pgfpathlineto{\pgfqpoint{1.044186in}{4.082796in}}%
\pgfpathlineto{\pgfqpoint{1.048033in}{4.092725in}}%
\pgfpathlineto{\pgfqpoint{1.049956in}{4.082630in}}%
\pgfpathlineto{\pgfqpoint{1.053803in}{4.094890in}}%
\pgfpathlineto{\pgfqpoint{1.055726in}{4.084370in}}%
\pgfpathlineto{\pgfqpoint{1.057650in}{4.093948in}}%
\pgfpathlineto{\pgfqpoint{1.059573in}{4.083935in}}%
\pgfpathlineto{\pgfqpoint{1.061496in}{4.087907in}}%
\pgfpathlineto{\pgfqpoint{1.063420in}{4.077933in}}%
\pgfpathlineto{\pgfqpoint{1.065343in}{4.082860in}}%
\pgfpathlineto{\pgfqpoint{1.067267in}{4.082134in}}%
\pgfpathlineto{\pgfqpoint{1.069190in}{4.094730in}}%
\pgfpathlineto{\pgfqpoint{1.071113in}{4.084040in}}%
\pgfpathlineto{\pgfqpoint{1.073037in}{4.086357in}}%
\pgfpathlineto{\pgfqpoint{1.074960in}{4.082960in}}%
\pgfpathlineto{\pgfqpoint{1.076884in}{4.072673in}}%
\pgfpathlineto{\pgfqpoint{1.078807in}{4.073663in}}%
\pgfpathlineto{\pgfqpoint{1.080731in}{4.072903in}}%
\pgfpathlineto{\pgfqpoint{1.082654in}{4.064872in}}%
\pgfpathlineto{\pgfqpoint{1.084577in}{4.063506in}}%
\pgfpathlineto{\pgfqpoint{1.086501in}{4.071068in}}%
\pgfpathlineto{\pgfqpoint{1.090348in}{4.053894in}}%
\pgfpathlineto{\pgfqpoint{1.092271in}{4.051685in}}%
\pgfpathlineto{\pgfqpoint{1.094194in}{4.045529in}}%
\pgfpathlineto{\pgfqpoint{1.096118in}{4.050536in}}%
\pgfpathlineto{\pgfqpoint{1.098041in}{4.048708in}}%
\pgfpathlineto{\pgfqpoint{1.101888in}{4.061962in}}%
\pgfpathlineto{\pgfqpoint{1.105735in}{4.053404in}}%
\pgfpathlineto{\pgfqpoint{1.109582in}{4.037105in}}%
\pgfpathlineto{\pgfqpoint{1.111505in}{4.028458in}}%
\pgfpathlineto{\pgfqpoint{1.113429in}{4.037987in}}%
\pgfpathlineto{\pgfqpoint{1.115352in}{4.032206in}}%
\pgfpathlineto{\pgfqpoint{1.117275in}{4.048729in}}%
\pgfpathlineto{\pgfqpoint{1.121122in}{4.038712in}}%
\pgfpathlineto{\pgfqpoint{1.123046in}{4.043530in}}%
\pgfpathlineto{\pgfqpoint{1.124969in}{4.035667in}}%
\pgfpathlineto{\pgfqpoint{1.126892in}{4.040832in}}%
\pgfpathlineto{\pgfqpoint{1.128816in}{4.038571in}}%
\pgfpathlineto{\pgfqpoint{1.130739in}{4.032065in}}%
\pgfpathlineto{\pgfqpoint{1.132663in}{4.016628in}}%
\pgfpathlineto{\pgfqpoint{1.134586in}{4.032792in}}%
\pgfpathlineto{\pgfqpoint{1.136509in}{4.032604in}}%
\pgfpathlineto{\pgfqpoint{1.138433in}{4.027813in}}%
\pgfpathlineto{\pgfqpoint{1.142280in}{4.055553in}}%
\pgfpathlineto{\pgfqpoint{1.146126in}{4.065610in}}%
\pgfpathlineto{\pgfqpoint{1.148050in}{4.058207in}}%
\pgfpathlineto{\pgfqpoint{1.149973in}{4.063473in}}%
\pgfpathlineto{\pgfqpoint{1.151897in}{4.078539in}}%
\pgfpathlineto{\pgfqpoint{1.153820in}{4.082964in}}%
\pgfpathlineto{\pgfqpoint{1.155744in}{4.080376in}}%
\pgfpathlineto{\pgfqpoint{1.157667in}{4.082440in}}%
\pgfpathlineto{\pgfqpoint{1.159590in}{4.078152in}}%
\pgfpathlineto{\pgfqpoint{1.161514in}{4.080600in}}%
\pgfpathlineto{\pgfqpoint{1.163437in}{4.065776in}}%
\pgfpathlineto{\pgfqpoint{1.165361in}{4.065445in}}%
\pgfpathlineto{\pgfqpoint{1.169207in}{4.059616in}}%
\pgfpathlineto{\pgfqpoint{1.173054in}{4.072540in}}%
\pgfpathlineto{\pgfqpoint{1.174978in}{4.059536in}}%
\pgfpathlineto{\pgfqpoint{1.178824in}{4.054591in}}%
\pgfpathlineto{\pgfqpoint{1.180748in}{4.049266in}}%
\pgfpathlineto{\pgfqpoint{1.184595in}{4.059894in}}%
\pgfpathlineto{\pgfqpoint{1.186518in}{4.057808in}}%
\pgfpathlineto{\pgfqpoint{1.190365in}{4.034966in}}%
\pgfpathlineto{\pgfqpoint{1.192288in}{4.042871in}}%
\pgfpathlineto{\pgfqpoint{1.194212in}{4.026051in}}%
\pgfpathlineto{\pgfqpoint{1.199982in}{4.012527in}}%
\pgfpathlineto{\pgfqpoint{1.201905in}{4.021220in}}%
\pgfpathlineto{\pgfqpoint{1.203829in}{4.004287in}}%
\pgfpathlineto{\pgfqpoint{1.205752in}{4.019032in}}%
\pgfpathlineto{\pgfqpoint{1.207676in}{4.024133in}}%
\pgfpathlineto{\pgfqpoint{1.209599in}{4.012745in}}%
\pgfpathlineto{\pgfqpoint{1.211522in}{4.026836in}}%
\pgfpathlineto{\pgfqpoint{1.213446in}{4.004585in}}%
\pgfpathlineto{\pgfqpoint{1.215369in}{4.007517in}}%
\pgfpathlineto{\pgfqpoint{1.221140in}{4.021600in}}%
\pgfpathlineto{\pgfqpoint{1.223063in}{4.022046in}}%
\pgfpathlineto{\pgfqpoint{1.224986in}{4.013851in}}%
\pgfpathlineto{\pgfqpoint{1.226910in}{4.015357in}}%
\pgfpathlineto{\pgfqpoint{1.228833in}{4.014749in}}%
\pgfpathlineto{\pgfqpoint{1.232680in}{3.998400in}}%
\pgfpathlineto{\pgfqpoint{1.234603in}{4.002484in}}%
\pgfpathlineto{\pgfqpoint{1.238450in}{3.994956in}}%
\pgfpathlineto{\pgfqpoint{1.240374in}{4.005233in}}%
\pgfpathlineto{\pgfqpoint{1.244220in}{4.000140in}}%
\pgfpathlineto{\pgfqpoint{1.246144in}{3.990505in}}%
\pgfpathlineto{\pgfqpoint{1.248067in}{3.990593in}}%
\pgfpathlineto{\pgfqpoint{1.249991in}{3.982330in}}%
\pgfpathlineto{\pgfqpoint{1.251914in}{3.969006in}}%
\pgfpathlineto{\pgfqpoint{1.253838in}{3.973073in}}%
\pgfpathlineto{\pgfqpoint{1.257684in}{3.999982in}}%
\pgfpathlineto{\pgfqpoint{1.259608in}{4.005860in}}%
\pgfpathlineto{\pgfqpoint{1.261531in}{3.990486in}}%
\pgfpathlineto{\pgfqpoint{1.263455in}{3.994917in}}%
\pgfpathlineto{\pgfqpoint{1.265378in}{3.996450in}}%
\pgfpathlineto{\pgfqpoint{1.269225in}{4.004030in}}%
\pgfpathlineto{\pgfqpoint{1.273072in}{4.007715in}}%
\pgfpathlineto{\pgfqpoint{1.274995in}{4.024218in}}%
\pgfpathlineto{\pgfqpoint{1.276918in}{4.022172in}}%
\pgfpathlineto{\pgfqpoint{1.278842in}{4.005712in}}%
\pgfpathlineto{\pgfqpoint{1.282689in}{4.012446in}}%
\pgfpathlineto{\pgfqpoint{1.286536in}{4.027783in}}%
\pgfpathlineto{\pgfqpoint{1.288459in}{4.025466in}}%
\pgfpathlineto{\pgfqpoint{1.290382in}{4.025265in}}%
\pgfpathlineto{\pgfqpoint{1.294229in}{4.011211in}}%
\pgfpathlineto{\pgfqpoint{1.296153in}{4.027521in}}%
\pgfpathlineto{\pgfqpoint{1.298076in}{4.033806in}}%
\pgfpathlineto{\pgfqpoint{1.299999in}{4.023620in}}%
\pgfpathlineto{\pgfqpoint{1.303846in}{4.038746in}}%
\pgfpathlineto{\pgfqpoint{1.307693in}{4.020773in}}%
\pgfpathlineto{\pgfqpoint{1.309616in}{4.032806in}}%
\pgfpathlineto{\pgfqpoint{1.311540in}{4.031208in}}%
\pgfpathlineto{\pgfqpoint{1.317310in}{4.050260in}}%
\pgfpathlineto{\pgfqpoint{1.319233in}{4.045183in}}%
\pgfpathlineto{\pgfqpoint{1.321157in}{4.053638in}}%
\pgfpathlineto{\pgfqpoint{1.323080in}{4.076788in}}%
\pgfpathlineto{\pgfqpoint{1.325004in}{4.080286in}}%
\pgfpathlineto{\pgfqpoint{1.326927in}{4.077828in}}%
\pgfpathlineto{\pgfqpoint{1.328851in}{4.062408in}}%
\pgfpathlineto{\pgfqpoint{1.330774in}{4.073106in}}%
\pgfpathlineto{\pgfqpoint{1.332697in}{4.068639in}}%
\pgfpathlineto{\pgfqpoint{1.334621in}{4.074437in}}%
\pgfpathlineto{\pgfqpoint{1.338468in}{4.099001in}}%
\pgfpathlineto{\pgfqpoint{1.342314in}{4.089946in}}%
\pgfpathlineto{\pgfqpoint{1.344238in}{4.095587in}}%
\pgfpathlineto{\pgfqpoint{1.346161in}{4.085422in}}%
\pgfpathlineto{\pgfqpoint{1.350008in}{4.099962in}}%
\pgfpathlineto{\pgfqpoint{1.351931in}{4.103400in}}%
\pgfpathlineto{\pgfqpoint{1.355778in}{4.102773in}}%
\pgfpathlineto{\pgfqpoint{1.357702in}{4.104371in}}%
\pgfpathlineto{\pgfqpoint{1.359625in}{4.101072in}}%
\pgfpathlineto{\pgfqpoint{1.361549in}{4.092625in}}%
\pgfpathlineto{\pgfqpoint{1.365395in}{4.108359in}}%
\pgfpathlineto{\pgfqpoint{1.367319in}{4.103796in}}%
\pgfpathlineto{\pgfqpoint{1.369242in}{4.108935in}}%
\pgfpathlineto{\pgfqpoint{1.373089in}{4.098393in}}%
\pgfpathlineto{\pgfqpoint{1.375012in}{4.073785in}}%
\pgfpathlineto{\pgfqpoint{1.376936in}{4.079959in}}%
\pgfpathlineto{\pgfqpoint{1.378859in}{4.079198in}}%
\pgfpathlineto{\pgfqpoint{1.380783in}{4.069969in}}%
\pgfpathlineto{\pgfqpoint{1.382706in}{4.069145in}}%
\pgfpathlineto{\pgfqpoint{1.384629in}{4.080303in}}%
\pgfpathlineto{\pgfqpoint{1.388476in}{4.085546in}}%
\pgfpathlineto{\pgfqpoint{1.390400in}{4.096381in}}%
\pgfpathlineto{\pgfqpoint{1.392323in}{4.098043in}}%
\pgfpathlineto{\pgfqpoint{1.394247in}{4.095067in}}%
\pgfpathlineto{\pgfqpoint{1.398093in}{4.082040in}}%
\pgfpathlineto{\pgfqpoint{1.400017in}{4.079246in}}%
\pgfpathlineto{\pgfqpoint{1.403864in}{4.066830in}}%
\pgfpathlineto{\pgfqpoint{1.405787in}{4.090136in}}%
\pgfpathlineto{\pgfqpoint{1.407710in}{4.076512in}}%
\pgfpathlineto{\pgfqpoint{1.409634in}{4.080277in}}%
\pgfpathlineto{\pgfqpoint{1.411557in}{4.081422in}}%
\pgfpathlineto{\pgfqpoint{1.413481in}{4.093615in}}%
\pgfpathlineto{\pgfqpoint{1.415404in}{4.097895in}}%
\pgfpathlineto{\pgfqpoint{1.417327in}{4.088548in}}%
\pgfpathlineto{\pgfqpoint{1.419251in}{4.098084in}}%
\pgfpathlineto{\pgfqpoint{1.421174in}{4.099857in}}%
\pgfpathlineto{\pgfqpoint{1.423098in}{4.087142in}}%
\pgfpathlineto{\pgfqpoint{1.426945in}{4.081719in}}%
\pgfpathlineto{\pgfqpoint{1.436562in}{4.117980in}}%
\pgfpathlineto{\pgfqpoint{1.440408in}{4.112915in}}%
\pgfpathlineto{\pgfqpoint{1.442332in}{4.105716in}}%
\pgfpathlineto{\pgfqpoint{1.444255in}{4.106701in}}%
\pgfpathlineto{\pgfqpoint{1.446179in}{4.112017in}}%
\pgfpathlineto{\pgfqpoint{1.448102in}{4.124475in}}%
\pgfpathlineto{\pgfqpoint{1.450025in}{4.117993in}}%
\pgfpathlineto{\pgfqpoint{1.451949in}{4.101603in}}%
\pgfpathlineto{\pgfqpoint{1.453872in}{4.101631in}}%
\pgfpathlineto{\pgfqpoint{1.461566in}{4.109603in}}%
\pgfpathlineto{\pgfqpoint{1.465413in}{4.097814in}}%
\pgfpathlineto{\pgfqpoint{1.467336in}{4.097121in}}%
\pgfpathlineto{\pgfqpoint{1.469260in}{4.117422in}}%
\pgfpathlineto{\pgfqpoint{1.471183in}{4.118982in}}%
\pgfpathlineto{\pgfqpoint{1.473106in}{4.116232in}}%
\pgfpathlineto{\pgfqpoint{1.476953in}{4.139269in}}%
\pgfpathlineto{\pgfqpoint{1.478877in}{4.140027in}}%
\pgfpathlineto{\pgfqpoint{1.480800in}{4.127321in}}%
\pgfpathlineto{\pgfqpoint{1.482723in}{4.125529in}}%
\pgfpathlineto{\pgfqpoint{1.484647in}{4.111574in}}%
\pgfpathlineto{\pgfqpoint{1.486570in}{4.111775in}}%
\pgfpathlineto{\pgfqpoint{1.488494in}{4.131796in}}%
\pgfpathlineto{\pgfqpoint{1.492340in}{4.120489in}}%
\pgfpathlineto{\pgfqpoint{1.494264in}{4.123442in}}%
\pgfpathlineto{\pgfqpoint{1.496187in}{4.123111in}}%
\pgfpathlineto{\pgfqpoint{1.498111in}{4.129408in}}%
\pgfpathlineto{\pgfqpoint{1.500034in}{4.143560in}}%
\pgfpathlineto{\pgfqpoint{1.505804in}{4.158963in}}%
\pgfpathlineto{\pgfqpoint{1.507728in}{4.142646in}}%
\pgfpathlineto{\pgfqpoint{1.509651in}{4.145807in}}%
\pgfpathlineto{\pgfqpoint{1.513498in}{4.126175in}}%
\pgfpathlineto{\pgfqpoint{1.515421in}{4.133356in}}%
\pgfpathlineto{\pgfqpoint{1.519268in}{4.122535in}}%
\pgfpathlineto{\pgfqpoint{1.521192in}{4.134354in}}%
\pgfpathlineto{\pgfqpoint{1.523115in}{4.131546in}}%
\pgfpathlineto{\pgfqpoint{1.525038in}{4.131939in}}%
\pgfpathlineto{\pgfqpoint{1.528885in}{4.138274in}}%
\pgfpathlineto{\pgfqpoint{1.532732in}{4.118090in}}%
\pgfpathlineto{\pgfqpoint{1.536579in}{4.143059in}}%
\pgfpathlineto{\pgfqpoint{1.538502in}{4.131051in}}%
\pgfpathlineto{\pgfqpoint{1.542349in}{4.156182in}}%
\pgfpathlineto{\pgfqpoint{1.544273in}{4.147332in}}%
\pgfpathlineto{\pgfqpoint{1.546196in}{4.144669in}}%
\pgfpathlineto{\pgfqpoint{1.548119in}{4.150901in}}%
\pgfpathlineto{\pgfqpoint{1.550043in}{4.161022in}}%
\pgfpathlineto{\pgfqpoint{1.551966in}{4.157016in}}%
\pgfpathlineto{\pgfqpoint{1.555813in}{4.173581in}}%
\pgfpathlineto{\pgfqpoint{1.557736in}{4.178385in}}%
\pgfpathlineto{\pgfqpoint{1.561583in}{4.139508in}}%
\pgfpathlineto{\pgfqpoint{1.565430in}{4.167373in}}%
\pgfpathlineto{\pgfqpoint{1.567354in}{4.171064in}}%
\pgfpathlineto{\pgfqpoint{1.571200in}{4.159105in}}%
\pgfpathlineto{\pgfqpoint{1.573124in}{4.171886in}}%
\pgfpathlineto{\pgfqpoint{1.576971in}{4.149411in}}%
\pgfpathlineto{\pgfqpoint{1.580817in}{4.151242in}}%
\pgfpathlineto{\pgfqpoint{1.582741in}{4.150498in}}%
\pgfpathlineto{\pgfqpoint{1.584664in}{4.151071in}}%
\pgfpathlineto{\pgfqpoint{1.586588in}{4.146627in}}%
\pgfpathlineto{\pgfqpoint{1.588511in}{4.147723in}}%
\pgfpathlineto{\pgfqpoint{1.590434in}{4.153561in}}%
\pgfpathlineto{\pgfqpoint{1.592358in}{4.153128in}}%
\pgfpathlineto{\pgfqpoint{1.594281in}{4.161658in}}%
\pgfpathlineto{\pgfqpoint{1.600051in}{4.129017in}}%
\pgfpathlineto{\pgfqpoint{1.601975in}{4.118362in}}%
\pgfpathlineto{\pgfqpoint{1.603898in}{4.114203in}}%
\pgfpathlineto{\pgfqpoint{1.609669in}{4.136947in}}%
\pgfpathlineto{\pgfqpoint{1.611592in}{4.147946in}}%
\pgfpathlineto{\pgfqpoint{1.613515in}{4.149457in}}%
\pgfpathlineto{\pgfqpoint{1.615439in}{4.148430in}}%
\pgfpathlineto{\pgfqpoint{1.617362in}{4.161501in}}%
\pgfpathlineto{\pgfqpoint{1.619286in}{4.166682in}}%
\pgfpathlineto{\pgfqpoint{1.621209in}{4.166014in}}%
\pgfpathlineto{\pgfqpoint{1.623132in}{4.185632in}}%
\pgfpathlineto{\pgfqpoint{1.625056in}{4.188050in}}%
\pgfpathlineto{\pgfqpoint{1.630826in}{4.202814in}}%
\pgfpathlineto{\pgfqpoint{1.632749in}{4.200119in}}%
\pgfpathlineto{\pgfqpoint{1.634673in}{4.207572in}}%
\pgfpathlineto{\pgfqpoint{1.638520in}{4.214358in}}%
\pgfpathlineto{\pgfqpoint{1.640443in}{4.211476in}}%
\pgfpathlineto{\pgfqpoint{1.642367in}{4.223918in}}%
\pgfpathlineto{\pgfqpoint{1.644290in}{4.217895in}}%
\pgfpathlineto{\pgfqpoint{1.646213in}{4.218160in}}%
\pgfpathlineto{\pgfqpoint{1.648137in}{4.219851in}}%
\pgfpathlineto{\pgfqpoint{1.650060in}{4.229951in}}%
\pgfpathlineto{\pgfqpoint{1.653907in}{4.212683in}}%
\pgfpathlineto{\pgfqpoint{1.655830in}{4.226347in}}%
\pgfpathlineto{\pgfqpoint{1.663524in}{4.203814in}}%
\pgfpathlineto{\pgfqpoint{1.665447in}{4.204248in}}%
\pgfpathlineto{\pgfqpoint{1.669294in}{4.196257in}}%
\pgfpathlineto{\pgfqpoint{1.671218in}{4.210264in}}%
\pgfpathlineto{\pgfqpoint{1.676988in}{4.200059in}}%
\pgfpathlineto{\pgfqpoint{1.678911in}{4.187347in}}%
\pgfpathlineto{\pgfqpoint{1.682758in}{4.180673in}}%
\pgfpathlineto{\pgfqpoint{1.686605in}{4.165215in}}%
\pgfpathlineto{\pgfqpoint{1.690452in}{4.182098in}}%
\pgfpathlineto{\pgfqpoint{1.692375in}{4.184051in}}%
\pgfpathlineto{\pgfqpoint{1.698145in}{4.224683in}}%
\pgfpathlineto{\pgfqpoint{1.700069in}{4.225882in}}%
\pgfpathlineto{\pgfqpoint{1.701992in}{4.217743in}}%
\pgfpathlineto{\pgfqpoint{1.703916in}{4.214737in}}%
\pgfpathlineto{\pgfqpoint{1.709686in}{4.174628in}}%
\pgfpathlineto{\pgfqpoint{1.713533in}{4.195286in}}%
\pgfpathlineto{\pgfqpoint{1.719303in}{4.200474in}}%
\pgfpathlineto{\pgfqpoint{1.721226in}{4.207027in}}%
\pgfpathlineto{\pgfqpoint{1.723150in}{4.199749in}}%
\pgfpathlineto{\pgfqpoint{1.728920in}{4.209146in}}%
\pgfpathlineto{\pgfqpoint{1.730843in}{4.203098in}}%
\pgfpathlineto{\pgfqpoint{1.732767in}{4.206958in}}%
\pgfpathlineto{\pgfqpoint{1.734690in}{4.207995in}}%
\pgfpathlineto{\pgfqpoint{1.736614in}{4.215314in}}%
\pgfpathlineto{\pgfqpoint{1.740461in}{4.204828in}}%
\pgfpathlineto{\pgfqpoint{1.744307in}{4.229938in}}%
\pgfpathlineto{\pgfqpoint{1.748154in}{4.199789in}}%
\pgfpathlineto{\pgfqpoint{1.750078in}{4.206013in}}%
\pgfpathlineto{\pgfqpoint{1.752001in}{4.203298in}}%
\pgfpathlineto{\pgfqpoint{1.755848in}{4.210043in}}%
\pgfpathlineto{\pgfqpoint{1.757771in}{4.200787in}}%
\pgfpathlineto{\pgfqpoint{1.761618in}{4.209938in}}%
\pgfpathlineto{\pgfqpoint{1.763541in}{4.211770in}}%
\pgfpathlineto{\pgfqpoint{1.767388in}{4.224240in}}%
\pgfpathlineto{\pgfqpoint{1.769312in}{4.237148in}}%
\pgfpathlineto{\pgfqpoint{1.771235in}{4.234012in}}%
\pgfpathlineto{\pgfqpoint{1.773158in}{4.236444in}}%
\pgfpathlineto{\pgfqpoint{1.775082in}{4.246855in}}%
\pgfpathlineto{\pgfqpoint{1.778929in}{4.244472in}}%
\pgfpathlineto{\pgfqpoint{1.780852in}{4.235364in}}%
\pgfpathlineto{\pgfqpoint{1.782776in}{4.241456in}}%
\pgfpathlineto{\pgfqpoint{1.784699in}{4.251935in}}%
\pgfpathlineto{\pgfqpoint{1.786622in}{4.249929in}}%
\pgfpathlineto{\pgfqpoint{1.788546in}{4.258930in}}%
\pgfpathlineto{\pgfqpoint{1.790469in}{4.280500in}}%
\pgfpathlineto{\pgfqpoint{1.792393in}{4.277751in}}%
\pgfpathlineto{\pgfqpoint{1.794316in}{4.282559in}}%
\pgfpathlineto{\pgfqpoint{1.796239in}{4.278922in}}%
\pgfpathlineto{\pgfqpoint{1.798163in}{4.269145in}}%
\pgfpathlineto{\pgfqpoint{1.800086in}{4.274593in}}%
\pgfpathlineto{\pgfqpoint{1.802010in}{4.287271in}}%
\pgfpathlineto{\pgfqpoint{1.803933in}{4.288246in}}%
\pgfpathlineto{\pgfqpoint{1.805856in}{4.279506in}}%
\pgfpathlineto{\pgfqpoint{1.807780in}{4.287294in}}%
\pgfpathlineto{\pgfqpoint{1.809703in}{4.268068in}}%
\pgfpathlineto{\pgfqpoint{1.811627in}{4.268637in}}%
\pgfpathlineto{\pgfqpoint{1.813550in}{4.291519in}}%
\pgfpathlineto{\pgfqpoint{1.815474in}{4.274825in}}%
\pgfpathlineto{\pgfqpoint{1.817397in}{4.276611in}}%
\pgfpathlineto{\pgfqpoint{1.819320in}{4.289054in}}%
\pgfpathlineto{\pgfqpoint{1.823167in}{4.266655in}}%
\pgfpathlineto{\pgfqpoint{1.825091in}{4.270713in}}%
\pgfpathlineto{\pgfqpoint{1.827014in}{4.285273in}}%
\pgfpathlineto{\pgfqpoint{1.828937in}{4.278863in}}%
\pgfpathlineto{\pgfqpoint{1.830861in}{4.283146in}}%
\pgfpathlineto{\pgfqpoint{1.832784in}{4.279684in}}%
\pgfpathlineto{\pgfqpoint{1.834708in}{4.273103in}}%
\pgfpathlineto{\pgfqpoint{1.836631in}{4.271481in}}%
\pgfpathlineto{\pgfqpoint{1.838554in}{4.277651in}}%
\pgfpathlineto{\pgfqpoint{1.840478in}{4.274418in}}%
\pgfpathlineto{\pgfqpoint{1.842401in}{4.247539in}}%
\pgfpathlineto{\pgfqpoint{1.844325in}{4.240938in}}%
\pgfpathlineto{\pgfqpoint{1.846248in}{4.246470in}}%
\pgfpathlineto{\pgfqpoint{1.848172in}{4.247022in}}%
\pgfpathlineto{\pgfqpoint{1.850095in}{4.239597in}}%
\pgfpathlineto{\pgfqpoint{1.852018in}{4.250516in}}%
\pgfpathlineto{\pgfqpoint{1.855865in}{4.240553in}}%
\pgfpathlineto{\pgfqpoint{1.857789in}{4.242471in}}%
\pgfpathlineto{\pgfqpoint{1.859712in}{4.235293in}}%
\pgfpathlineto{\pgfqpoint{1.861635in}{4.245051in}}%
\pgfpathlineto{\pgfqpoint{1.863559in}{4.239904in}}%
\pgfpathlineto{\pgfqpoint{1.867406in}{4.264984in}}%
\pgfpathlineto{\pgfqpoint{1.869329in}{4.258722in}}%
\pgfpathlineto{\pgfqpoint{1.875099in}{4.265386in}}%
\pgfpathlineto{\pgfqpoint{1.877023in}{4.268973in}}%
\pgfpathlineto{\pgfqpoint{1.878946in}{4.277281in}}%
\pgfpathlineto{\pgfqpoint{1.880870in}{4.274492in}}%
\pgfpathlineto{\pgfqpoint{1.882793in}{4.280745in}}%
\pgfpathlineto{\pgfqpoint{1.886640in}{4.257237in}}%
\pgfpathlineto{\pgfqpoint{1.888563in}{4.258072in}}%
\pgfpathlineto{\pgfqpoint{1.890487in}{4.257295in}}%
\pgfpathlineto{\pgfqpoint{1.892410in}{4.250349in}}%
\pgfpathlineto{\pgfqpoint{1.894333in}{4.251086in}}%
\pgfpathlineto{\pgfqpoint{1.896257in}{4.247428in}}%
\pgfpathlineto{\pgfqpoint{1.898180in}{4.251448in}}%
\pgfpathlineto{\pgfqpoint{1.900104in}{4.262262in}}%
\pgfpathlineto{\pgfqpoint{1.902027in}{4.260421in}}%
\pgfpathlineto{\pgfqpoint{1.905874in}{4.281173in}}%
\pgfpathlineto{\pgfqpoint{1.907797in}{4.287756in}}%
\pgfpathlineto{\pgfqpoint{1.909721in}{4.283304in}}%
\pgfpathlineto{\pgfqpoint{1.911644in}{4.283154in}}%
\pgfpathlineto{\pgfqpoint{1.913567in}{4.271642in}}%
\pgfpathlineto{\pgfqpoint{1.917414in}{4.283311in}}%
\pgfpathlineto{\pgfqpoint{1.919338in}{4.285576in}}%
\pgfpathlineto{\pgfqpoint{1.923185in}{4.270078in}}%
\pgfpathlineto{\pgfqpoint{1.925108in}{4.277259in}}%
\pgfpathlineto{\pgfqpoint{1.927031in}{4.274781in}}%
\pgfpathlineto{\pgfqpoint{1.928955in}{4.276626in}}%
\pgfpathlineto{\pgfqpoint{1.934725in}{4.301864in}}%
\pgfpathlineto{\pgfqpoint{1.936648in}{4.305786in}}%
\pgfpathlineto{\pgfqpoint{1.938572in}{4.289232in}}%
\pgfpathlineto{\pgfqpoint{1.940495in}{4.286358in}}%
\pgfpathlineto{\pgfqpoint{1.944342in}{4.293755in}}%
\pgfpathlineto{\pgfqpoint{1.946265in}{4.289484in}}%
\pgfpathlineto{\pgfqpoint{1.950112in}{4.285361in}}%
\pgfpathlineto{\pgfqpoint{1.953959in}{4.264357in}}%
\pgfpathlineto{\pgfqpoint{1.957806in}{4.270527in}}%
\pgfpathlineto{\pgfqpoint{1.959729in}{4.266777in}}%
\pgfpathlineto{\pgfqpoint{1.963576in}{4.276073in}}%
\pgfpathlineto{\pgfqpoint{1.965500in}{4.273435in}}%
\pgfpathlineto{\pgfqpoint{1.969346in}{4.283092in}}%
\pgfpathlineto{\pgfqpoint{1.973193in}{4.273165in}}%
\pgfpathlineto{\pgfqpoint{1.975117in}{4.283483in}}%
\pgfpathlineto{\pgfqpoint{1.977040in}{4.274867in}}%
\pgfpathlineto{\pgfqpoint{1.980887in}{4.296469in}}%
\pgfpathlineto{\pgfqpoint{1.982810in}{4.298715in}}%
\pgfpathlineto{\pgfqpoint{1.984734in}{4.295779in}}%
\pgfpathlineto{\pgfqpoint{1.990504in}{4.318006in}}%
\pgfpathlineto{\pgfqpoint{1.992427in}{4.320023in}}%
\pgfpathlineto{\pgfqpoint{1.994351in}{4.324784in}}%
\pgfpathlineto{\pgfqpoint{1.998198in}{4.325170in}}%
\pgfpathlineto{\pgfqpoint{2.000121in}{4.313566in}}%
\pgfpathlineto{\pgfqpoint{2.002044in}{4.316433in}}%
\pgfpathlineto{\pgfqpoint{2.003968in}{4.324257in}}%
\pgfpathlineto{\pgfqpoint{2.005891in}{4.314718in}}%
\pgfpathlineto{\pgfqpoint{2.007815in}{4.315188in}}%
\pgfpathlineto{\pgfqpoint{2.009738in}{4.311254in}}%
\pgfpathlineto{\pgfqpoint{2.011661in}{4.299300in}}%
\pgfpathlineto{\pgfqpoint{2.013585in}{4.305255in}}%
\pgfpathlineto{\pgfqpoint{2.019355in}{4.280095in}}%
\pgfpathlineto{\pgfqpoint{2.021279in}{4.278204in}}%
\pgfpathlineto{\pgfqpoint{2.023202in}{4.267224in}}%
\pgfpathlineto{\pgfqpoint{2.027049in}{4.284477in}}%
\pgfpathlineto{\pgfqpoint{2.030896in}{4.262589in}}%
\pgfpathlineto{\pgfqpoint{2.032819in}{4.269873in}}%
\pgfpathlineto{\pgfqpoint{2.034742in}{4.272129in}}%
\pgfpathlineto{\pgfqpoint{2.036666in}{4.268365in}}%
\pgfpathlineto{\pgfqpoint{2.038589in}{4.277699in}}%
\pgfpathlineto{\pgfqpoint{2.040513in}{4.280335in}}%
\pgfpathlineto{\pgfqpoint{2.042436in}{4.289903in}}%
\pgfpathlineto{\pgfqpoint{2.044359in}{4.291940in}}%
\pgfpathlineto{\pgfqpoint{2.048206in}{4.305573in}}%
\pgfpathlineto{\pgfqpoint{2.050130in}{4.324805in}}%
\pgfpathlineto{\pgfqpoint{2.052053in}{4.318383in}}%
\pgfpathlineto{\pgfqpoint{2.053976in}{4.329385in}}%
\pgfpathlineto{\pgfqpoint{2.055900in}{4.325151in}}%
\pgfpathlineto{\pgfqpoint{2.057823in}{4.325189in}}%
\pgfpathlineto{\pgfqpoint{2.059747in}{4.319205in}}%
\pgfpathlineto{\pgfqpoint{2.061670in}{4.319516in}}%
\pgfpathlineto{\pgfqpoint{2.069364in}{4.361358in}}%
\pgfpathlineto{\pgfqpoint{2.071287in}{4.353412in}}%
\pgfpathlineto{\pgfqpoint{2.073211in}{4.363630in}}%
\pgfpathlineto{\pgfqpoint{2.075134in}{4.357323in}}%
\pgfpathlineto{\pgfqpoint{2.078981in}{4.365701in}}%
\pgfpathlineto{\pgfqpoint{2.080904in}{4.356181in}}%
\pgfpathlineto{\pgfqpoint{2.084751in}{4.372569in}}%
\pgfpathlineto{\pgfqpoint{2.088598in}{4.363662in}}%
\pgfpathlineto{\pgfqpoint{2.090521in}{4.353469in}}%
\pgfpathlineto{\pgfqpoint{2.092445in}{4.357674in}}%
\pgfpathlineto{\pgfqpoint{2.094368in}{4.367839in}}%
\pgfpathlineto{\pgfqpoint{2.098215in}{4.360325in}}%
\pgfpathlineto{\pgfqpoint{2.102062in}{4.339722in}}%
\pgfpathlineto{\pgfqpoint{2.105909in}{4.350863in}}%
\pgfpathlineto{\pgfqpoint{2.107832in}{4.358126in}}%
\pgfpathlineto{\pgfqpoint{2.111679in}{4.349711in}}%
\pgfpathlineto{\pgfqpoint{2.113602in}{4.354588in}}%
\pgfpathlineto{\pgfqpoint{2.115526in}{4.341591in}}%
\pgfpathlineto{\pgfqpoint{2.117449in}{4.350937in}}%
\pgfpathlineto{\pgfqpoint{2.119372in}{4.340475in}}%
\pgfpathlineto{\pgfqpoint{2.121296in}{4.343633in}}%
\pgfpathlineto{\pgfqpoint{2.123219in}{4.335993in}}%
\pgfpathlineto{\pgfqpoint{2.125143in}{4.338908in}}%
\pgfpathlineto{\pgfqpoint{2.127066in}{4.325669in}}%
\pgfpathlineto{\pgfqpoint{2.128990in}{4.333648in}}%
\pgfpathlineto{\pgfqpoint{2.134760in}{4.301523in}}%
\pgfpathlineto{\pgfqpoint{2.138607in}{4.312391in}}%
\pgfpathlineto{\pgfqpoint{2.142453in}{4.333276in}}%
\pgfpathlineto{\pgfqpoint{2.144377in}{4.331826in}}%
\pgfpathlineto{\pgfqpoint{2.146300in}{4.339304in}}%
\pgfpathlineto{\pgfqpoint{2.148224in}{4.328181in}}%
\pgfpathlineto{\pgfqpoint{2.150147in}{4.328133in}}%
\pgfpathlineto{\pgfqpoint{2.153994in}{4.335722in}}%
\pgfpathlineto{\pgfqpoint{2.155917in}{4.340512in}}%
\pgfpathlineto{\pgfqpoint{2.157841in}{4.339250in}}%
\pgfpathlineto{\pgfqpoint{2.159764in}{4.356678in}}%
\pgfpathlineto{\pgfqpoint{2.161688in}{4.363015in}}%
\pgfpathlineto{\pgfqpoint{2.163611in}{4.362129in}}%
\pgfpathlineto{\pgfqpoint{2.165534in}{4.351308in}}%
\pgfpathlineto{\pgfqpoint{2.167458in}{4.351553in}}%
\pgfpathlineto{\pgfqpoint{2.169381in}{4.364078in}}%
\pgfpathlineto{\pgfqpoint{2.171305in}{4.359789in}}%
\pgfpathlineto{\pgfqpoint{2.173228in}{4.378591in}}%
\pgfpathlineto{\pgfqpoint{2.175151in}{4.378006in}}%
\pgfpathlineto{\pgfqpoint{2.178998in}{4.391919in}}%
\pgfpathlineto{\pgfqpoint{2.180922in}{4.398895in}}%
\pgfpathlineto{\pgfqpoint{2.182845in}{4.391622in}}%
\pgfpathlineto{\pgfqpoint{2.184768in}{4.373506in}}%
\pgfpathlineto{\pgfqpoint{2.186692in}{4.379040in}}%
\pgfpathlineto{\pgfqpoint{2.188615in}{4.367365in}}%
\pgfpathlineto{\pgfqpoint{2.190539in}{4.372435in}}%
\pgfpathlineto{\pgfqpoint{2.192462in}{4.363794in}}%
\pgfpathlineto{\pgfqpoint{2.194386in}{4.364598in}}%
\pgfpathlineto{\pgfqpoint{2.196309in}{4.377324in}}%
\pgfpathlineto{\pgfqpoint{2.198232in}{4.374931in}}%
\pgfpathlineto{\pgfqpoint{2.200156in}{4.364367in}}%
\pgfpathlineto{\pgfqpoint{2.202079in}{4.369289in}}%
\pgfpathlineto{\pgfqpoint{2.204003in}{4.349298in}}%
\pgfpathlineto{\pgfqpoint{2.205926in}{4.348911in}}%
\pgfpathlineto{\pgfqpoint{2.207849in}{4.339863in}}%
\pgfpathlineto{\pgfqpoint{2.215543in}{4.346695in}}%
\pgfpathlineto{\pgfqpoint{2.217466in}{4.346588in}}%
\pgfpathlineto{\pgfqpoint{2.219390in}{4.343726in}}%
\pgfpathlineto{\pgfqpoint{2.221313in}{4.345740in}}%
\pgfpathlineto{\pgfqpoint{2.223237in}{4.353213in}}%
\pgfpathlineto{\pgfqpoint{2.225160in}{4.351063in}}%
\pgfpathlineto{\pgfqpoint{2.229007in}{4.337290in}}%
\pgfpathlineto{\pgfqpoint{2.230930in}{4.340199in}}%
\pgfpathlineto{\pgfqpoint{2.232854in}{4.336863in}}%
\pgfpathlineto{\pgfqpoint{2.234777in}{4.351705in}}%
\pgfpathlineto{\pgfqpoint{2.236701in}{4.354635in}}%
\pgfpathlineto{\pgfqpoint{2.238624in}{4.345457in}}%
\pgfpathlineto{\pgfqpoint{2.242471in}{4.344086in}}%
\pgfpathlineto{\pgfqpoint{2.244394in}{4.366001in}}%
\pgfpathlineto{\pgfqpoint{2.248241in}{4.353152in}}%
\pgfpathlineto{\pgfqpoint{2.250164in}{4.351606in}}%
\pgfpathlineto{\pgfqpoint{2.252088in}{4.332503in}}%
\pgfpathlineto{\pgfqpoint{2.255935in}{4.338545in}}%
\pgfpathlineto{\pgfqpoint{2.259781in}{4.312223in}}%
\pgfpathlineto{\pgfqpoint{2.261705in}{4.314353in}}%
\pgfpathlineto{\pgfqpoint{2.265552in}{4.306701in}}%
\pgfpathlineto{\pgfqpoint{2.269399in}{4.328623in}}%
\pgfpathlineto{\pgfqpoint{2.273245in}{4.344121in}}%
\pgfpathlineto{\pgfqpoint{2.277092in}{4.322691in}}%
\pgfpathlineto{\pgfqpoint{2.279016in}{4.328149in}}%
\pgfpathlineto{\pgfqpoint{2.280939in}{4.318914in}}%
\pgfpathlineto{\pgfqpoint{2.284786in}{4.310292in}}%
\pgfpathlineto{\pgfqpoint{2.286709in}{4.317266in}}%
\pgfpathlineto{\pgfqpoint{2.290556in}{4.321328in}}%
\pgfpathlineto{\pgfqpoint{2.292479in}{4.323356in}}%
\pgfpathlineto{\pgfqpoint{2.296326in}{4.339484in}}%
\pgfpathlineto{\pgfqpoint{2.298250in}{4.325912in}}%
\pgfpathlineto{\pgfqpoint{2.300173in}{4.321806in}}%
\pgfpathlineto{\pgfqpoint{2.305943in}{4.300479in}}%
\pgfpathlineto{\pgfqpoint{2.307867in}{4.304647in}}%
\pgfpathlineto{\pgfqpoint{2.311714in}{4.288527in}}%
\pgfpathlineto{\pgfqpoint{2.313637in}{4.276013in}}%
\pgfpathlineto{\pgfqpoint{2.315560in}{4.271894in}}%
\pgfpathlineto{\pgfqpoint{2.317484in}{4.275565in}}%
\pgfpathlineto{\pgfqpoint{2.319407in}{4.282851in}}%
\pgfpathlineto{\pgfqpoint{2.321331in}{4.284782in}}%
\pgfpathlineto{\pgfqpoint{2.323254in}{4.274701in}}%
\pgfpathlineto{\pgfqpoint{2.325177in}{4.281224in}}%
\pgfpathlineto{\pgfqpoint{2.327101in}{4.283619in}}%
\pgfpathlineto{\pgfqpoint{2.329024in}{4.289571in}}%
\pgfpathlineto{\pgfqpoint{2.330948in}{4.290417in}}%
\pgfpathlineto{\pgfqpoint{2.332871in}{4.295651in}}%
\pgfpathlineto{\pgfqpoint{2.334795in}{4.296454in}}%
\pgfpathlineto{\pgfqpoint{2.336718in}{4.306867in}}%
\pgfpathlineto{\pgfqpoint{2.338641in}{4.305541in}}%
\pgfpathlineto{\pgfqpoint{2.340565in}{4.307125in}}%
\pgfpathlineto{\pgfqpoint{2.346335in}{4.330022in}}%
\pgfpathlineto{\pgfqpoint{2.348258in}{4.329321in}}%
\pgfpathlineto{\pgfqpoint{2.350182in}{4.305221in}}%
\pgfpathlineto{\pgfqpoint{2.352105in}{4.299576in}}%
\pgfpathlineto{\pgfqpoint{2.354029in}{4.307574in}}%
\pgfpathlineto{\pgfqpoint{2.355952in}{4.309257in}}%
\pgfpathlineto{\pgfqpoint{2.357875in}{4.313111in}}%
\pgfpathlineto{\pgfqpoint{2.361722in}{4.301953in}}%
\pgfpathlineto{\pgfqpoint{2.363646in}{4.293333in}}%
\pgfpathlineto{\pgfqpoint{2.365569in}{4.299562in}}%
\pgfpathlineto{\pgfqpoint{2.367492in}{4.292417in}}%
\pgfpathlineto{\pgfqpoint{2.369416in}{4.296079in}}%
\pgfpathlineto{\pgfqpoint{2.371339in}{4.272367in}}%
\pgfpathlineto{\pgfqpoint{2.373263in}{4.267698in}}%
\pgfpathlineto{\pgfqpoint{2.377110in}{4.270774in}}%
\pgfpathlineto{\pgfqpoint{2.379033in}{4.254679in}}%
\pgfpathlineto{\pgfqpoint{2.380956in}{4.253495in}}%
\pgfpathlineto{\pgfqpoint{2.384803in}{4.233833in}}%
\pgfpathlineto{\pgfqpoint{2.386727in}{4.239057in}}%
\pgfpathlineto{\pgfqpoint{2.388650in}{4.248155in}}%
\pgfpathlineto{\pgfqpoint{2.390573in}{4.245952in}}%
\pgfpathlineto{\pgfqpoint{2.392497in}{4.248609in}}%
\pgfpathlineto{\pgfqpoint{2.394420in}{4.243392in}}%
\pgfpathlineto{\pgfqpoint{2.396344in}{4.251828in}}%
\pgfpathlineto{\pgfqpoint{2.398267in}{4.255096in}}%
\pgfpathlineto{\pgfqpoint{2.400190in}{4.252847in}}%
\pgfpathlineto{\pgfqpoint{2.402114in}{4.252844in}}%
\pgfpathlineto{\pgfqpoint{2.404037in}{4.248738in}}%
\pgfpathlineto{\pgfqpoint{2.405961in}{4.240223in}}%
\pgfpathlineto{\pgfqpoint{2.407884in}{4.242839in}}%
\pgfpathlineto{\pgfqpoint{2.409808in}{4.223578in}}%
\pgfpathlineto{\pgfqpoint{2.411731in}{4.220504in}}%
\pgfpathlineto{\pgfqpoint{2.413654in}{4.223599in}}%
\pgfpathlineto{\pgfqpoint{2.415578in}{4.213916in}}%
\pgfpathlineto{\pgfqpoint{2.417501in}{4.228472in}}%
\pgfpathlineto{\pgfqpoint{2.419425in}{4.229395in}}%
\pgfpathlineto{\pgfqpoint{2.421348in}{4.232446in}}%
\pgfpathlineto{\pgfqpoint{2.423271in}{4.240129in}}%
\pgfpathlineto{\pgfqpoint{2.425195in}{4.238542in}}%
\pgfpathlineto{\pgfqpoint{2.429042in}{4.220580in}}%
\pgfpathlineto{\pgfqpoint{2.430965in}{4.226809in}}%
\pgfpathlineto{\pgfqpoint{2.432888in}{4.202057in}}%
\pgfpathlineto{\pgfqpoint{2.434812in}{4.206235in}}%
\pgfpathlineto{\pgfqpoint{2.438659in}{4.186881in}}%
\pgfpathlineto{\pgfqpoint{2.440582in}{4.178216in}}%
\pgfpathlineto{\pgfqpoint{2.442506in}{4.186683in}}%
\pgfpathlineto{\pgfqpoint{2.446352in}{4.157697in}}%
\pgfpathlineto{\pgfqpoint{2.450199in}{4.178646in}}%
\pgfpathlineto{\pgfqpoint{2.452123in}{4.174851in}}%
\pgfpathlineto{\pgfqpoint{2.455969in}{4.195909in}}%
\pgfpathlineto{\pgfqpoint{2.461740in}{4.165396in}}%
\pgfpathlineto{\pgfqpoint{2.463663in}{4.173057in}}%
\pgfpathlineto{\pgfqpoint{2.465586in}{4.172601in}}%
\pgfpathlineto{\pgfqpoint{2.467510in}{4.175274in}}%
\pgfpathlineto{\pgfqpoint{2.469433in}{4.174353in}}%
\pgfpathlineto{\pgfqpoint{2.471357in}{4.182340in}}%
\pgfpathlineto{\pgfqpoint{2.475204in}{4.167783in}}%
\pgfpathlineto{\pgfqpoint{2.477127in}{4.171566in}}%
\pgfpathlineto{\pgfqpoint{2.479050in}{4.171059in}}%
\pgfpathlineto{\pgfqpoint{2.480974in}{4.167547in}}%
\pgfpathlineto{\pgfqpoint{2.482897in}{4.157471in}}%
\pgfpathlineto{\pgfqpoint{2.486744in}{4.184556in}}%
\pgfpathlineto{\pgfqpoint{2.490591in}{4.184437in}}%
\pgfpathlineto{\pgfqpoint{2.492514in}{4.196249in}}%
\pgfpathlineto{\pgfqpoint{2.494438in}{4.187999in}}%
\pgfpathlineto{\pgfqpoint{2.496361in}{4.189240in}}%
\pgfpathlineto{\pgfqpoint{2.498284in}{4.185290in}}%
\pgfpathlineto{\pgfqpoint{2.500208in}{4.193708in}}%
\pgfpathlineto{\pgfqpoint{2.505978in}{4.184009in}}%
\pgfpathlineto{\pgfqpoint{2.507901in}{4.197166in}}%
\pgfpathlineto{\pgfqpoint{2.515595in}{4.199085in}}%
\pgfpathlineto{\pgfqpoint{2.517519in}{4.200193in}}%
\pgfpathlineto{\pgfqpoint{2.519442in}{4.194878in}}%
\pgfpathlineto{\pgfqpoint{2.523289in}{4.205763in}}%
\pgfpathlineto{\pgfqpoint{2.529059in}{4.166078in}}%
\pgfpathlineto{\pgfqpoint{2.530982in}{4.172827in}}%
\pgfpathlineto{\pgfqpoint{2.532906in}{4.155627in}}%
\pgfpathlineto{\pgfqpoint{2.538676in}{4.178033in}}%
\pgfpathlineto{\pgfqpoint{2.540599in}{4.177099in}}%
\pgfpathlineto{\pgfqpoint{2.542523in}{4.188054in}}%
\pgfpathlineto{\pgfqpoint{2.546370in}{4.176332in}}%
\pgfpathlineto{\pgfqpoint{2.548293in}{4.185890in}}%
\pgfpathlineto{\pgfqpoint{2.550217in}{4.182815in}}%
\pgfpathlineto{\pgfqpoint{2.552140in}{4.185871in}}%
\pgfpathlineto{\pgfqpoint{2.555987in}{4.200823in}}%
\pgfpathlineto{\pgfqpoint{2.557910in}{4.195401in}}%
\pgfpathlineto{\pgfqpoint{2.559834in}{4.211713in}}%
\pgfpathlineto{\pgfqpoint{2.561757in}{4.207920in}}%
\pgfpathlineto{\pgfqpoint{2.565604in}{4.247183in}}%
\pgfpathlineto{\pgfqpoint{2.567527in}{4.247134in}}%
\pgfpathlineto{\pgfqpoint{2.569451in}{4.237954in}}%
\pgfpathlineto{\pgfqpoint{2.571374in}{4.250377in}}%
\pgfpathlineto{\pgfqpoint{2.573297in}{4.254634in}}%
\pgfpathlineto{\pgfqpoint{2.582915in}{4.211712in}}%
\pgfpathlineto{\pgfqpoint{2.588685in}{4.264740in}}%
\pgfpathlineto{\pgfqpoint{2.590608in}{4.263863in}}%
\pgfpathlineto{\pgfqpoint{2.592532in}{4.252112in}}%
\pgfpathlineto{\pgfqpoint{2.594455in}{4.251599in}}%
\pgfpathlineto{\pgfqpoint{2.596378in}{4.258372in}}%
\pgfpathlineto{\pgfqpoint{2.598302in}{4.236858in}}%
\pgfpathlineto{\pgfqpoint{2.600225in}{4.248994in}}%
\pgfpathlineto{\pgfqpoint{2.602149in}{4.252258in}}%
\pgfpathlineto{\pgfqpoint{2.605995in}{4.221892in}}%
\pgfpathlineto{\pgfqpoint{2.607919in}{4.224028in}}%
\pgfpathlineto{\pgfqpoint{2.609842in}{4.242558in}}%
\pgfpathlineto{\pgfqpoint{2.611766in}{4.247494in}}%
\pgfpathlineto{\pgfqpoint{2.613689in}{4.237731in}}%
\pgfpathlineto{\pgfqpoint{2.617536in}{4.236971in}}%
\pgfpathlineto{\pgfqpoint{2.619459in}{4.240042in}}%
\pgfpathlineto{\pgfqpoint{2.621383in}{4.236268in}}%
\pgfpathlineto{\pgfqpoint{2.625230in}{4.202609in}}%
\pgfpathlineto{\pgfqpoint{2.629076in}{4.212674in}}%
\pgfpathlineto{\pgfqpoint{2.632923in}{4.204645in}}%
\pgfpathlineto{\pgfqpoint{2.634847in}{4.208395in}}%
\pgfpathlineto{\pgfqpoint{2.636770in}{4.208472in}}%
\pgfpathlineto{\pgfqpoint{2.640617in}{4.197168in}}%
\pgfpathlineto{\pgfqpoint{2.642540in}{4.203459in}}%
\pgfpathlineto{\pgfqpoint{2.644464in}{4.198744in}}%
\pgfpathlineto{\pgfqpoint{2.646387in}{4.189801in}}%
\pgfpathlineto{\pgfqpoint{2.648311in}{4.201307in}}%
\pgfpathlineto{\pgfqpoint{2.650234in}{4.195571in}}%
\pgfpathlineto{\pgfqpoint{2.652157in}{4.199286in}}%
\pgfpathlineto{\pgfqpoint{2.654081in}{4.199691in}}%
\pgfpathlineto{\pgfqpoint{2.656004in}{4.204163in}}%
\pgfpathlineto{\pgfqpoint{2.657928in}{4.201859in}}%
\pgfpathlineto{\pgfqpoint{2.659851in}{4.195978in}}%
\pgfpathlineto{\pgfqpoint{2.661774in}{4.205908in}}%
\pgfpathlineto{\pgfqpoint{2.663698in}{4.205087in}}%
\pgfpathlineto{\pgfqpoint{2.665621in}{4.209542in}}%
\pgfpathlineto{\pgfqpoint{2.669468in}{4.206430in}}%
\pgfpathlineto{\pgfqpoint{2.671391in}{4.208774in}}%
\pgfpathlineto{\pgfqpoint{2.673315in}{4.194331in}}%
\pgfpathlineto{\pgfqpoint{2.675238in}{4.205695in}}%
\pgfpathlineto{\pgfqpoint{2.677162in}{4.209698in}}%
\pgfpathlineto{\pgfqpoint{2.679085in}{4.219567in}}%
\pgfpathlineto{\pgfqpoint{2.681008in}{4.246703in}}%
\pgfpathlineto{\pgfqpoint{2.682932in}{4.245609in}}%
\pgfpathlineto{\pgfqpoint{2.684855in}{4.252624in}}%
\pgfpathlineto{\pgfqpoint{2.686779in}{4.243905in}}%
\pgfpathlineto{\pgfqpoint{2.688702in}{4.249211in}}%
\pgfpathlineto{\pgfqpoint{2.690626in}{4.230509in}}%
\pgfpathlineto{\pgfqpoint{2.692549in}{4.239353in}}%
\pgfpathlineto{\pgfqpoint{2.696396in}{4.223080in}}%
\pgfpathlineto{\pgfqpoint{2.698319in}{4.211961in}}%
\pgfpathlineto{\pgfqpoint{2.702166in}{4.209850in}}%
\pgfpathlineto{\pgfqpoint{2.704089in}{4.217687in}}%
\pgfpathlineto{\pgfqpoint{2.706013in}{4.214799in}}%
\pgfpathlineto{\pgfqpoint{2.707936in}{4.197198in}}%
\pgfpathlineto{\pgfqpoint{2.709860in}{4.205032in}}%
\pgfpathlineto{\pgfqpoint{2.711783in}{4.204117in}}%
\pgfpathlineto{\pgfqpoint{2.713706in}{4.215898in}}%
\pgfpathlineto{\pgfqpoint{2.715630in}{4.200160in}}%
\pgfpathlineto{\pgfqpoint{2.717553in}{4.203413in}}%
\pgfpathlineto{\pgfqpoint{2.719477in}{4.202334in}}%
\pgfpathlineto{\pgfqpoint{2.721400in}{4.203868in}}%
\pgfpathlineto{\pgfqpoint{2.723324in}{4.189512in}}%
\pgfpathlineto{\pgfqpoint{2.725247in}{4.197881in}}%
\pgfpathlineto{\pgfqpoint{2.727170in}{4.181468in}}%
\pgfpathlineto{\pgfqpoint{2.729094in}{4.180537in}}%
\pgfpathlineto{\pgfqpoint{2.731017in}{4.164380in}}%
\pgfpathlineto{\pgfqpoint{2.736787in}{4.150092in}}%
\pgfpathlineto{\pgfqpoint{2.740634in}{4.163312in}}%
\pgfpathlineto{\pgfqpoint{2.744481in}{4.148215in}}%
\pgfpathlineto{\pgfqpoint{2.746404in}{4.140074in}}%
\pgfpathlineto{\pgfqpoint{2.748328in}{4.150521in}}%
\pgfpathlineto{\pgfqpoint{2.750251in}{4.147516in}}%
\pgfpathlineto{\pgfqpoint{2.752175in}{4.159217in}}%
\pgfpathlineto{\pgfqpoint{2.757945in}{4.171165in}}%
\pgfpathlineto{\pgfqpoint{2.759868in}{4.177176in}}%
\pgfpathlineto{\pgfqpoint{2.763715in}{4.204284in}}%
\pgfpathlineto{\pgfqpoint{2.765639in}{4.201571in}}%
\pgfpathlineto{\pgfqpoint{2.767562in}{4.216954in}}%
\pgfpathlineto{\pgfqpoint{2.767562in}{4.216954in}}%
\pgfusepath{stroke}%
\end{pgfscope}%
\begin{pgfscope}%
\pgfpathrectangle{\pgfqpoint{0.750000in}{3.180000in}}{\pgfqpoint{2.113636in}{2.100000in}}%
\pgfusepath{clip}%
\pgfsetroundcap%
\pgfsetroundjoin%
\pgfsetlinewidth{0.602250pt}%
\definecolor{currentstroke}{rgb}{0.301961,0.686275,0.290196}%
\pgfsetstrokecolor{currentstroke}%
\pgfsetdash{}{0pt}%
\pgfpathmoveto{\pgfqpoint{0.846074in}{4.322863in}}%
\pgfpathlineto{\pgfqpoint{0.847998in}{4.310921in}}%
\pgfpathlineto{\pgfqpoint{0.851845in}{4.313777in}}%
\pgfpathlineto{\pgfqpoint{0.855691in}{4.327151in}}%
\pgfpathlineto{\pgfqpoint{0.857615in}{4.324975in}}%
\pgfpathlineto{\pgfqpoint{0.859538in}{4.324982in}}%
\pgfpathlineto{\pgfqpoint{0.861462in}{4.318568in}}%
\pgfpathlineto{\pgfqpoint{0.863385in}{4.318894in}}%
\pgfpathlineto{\pgfqpoint{0.865308in}{4.315796in}}%
\pgfpathlineto{\pgfqpoint{0.869155in}{4.301141in}}%
\pgfpathlineto{\pgfqpoint{0.871079in}{4.305376in}}%
\pgfpathlineto{\pgfqpoint{0.873002in}{4.304133in}}%
\pgfpathlineto{\pgfqpoint{0.874926in}{4.299751in}}%
\pgfpathlineto{\pgfqpoint{0.876849in}{4.315905in}}%
\pgfpathlineto{\pgfqpoint{0.878772in}{4.301021in}}%
\pgfpathlineto{\pgfqpoint{0.880696in}{4.315184in}}%
\pgfpathlineto{\pgfqpoint{0.882619in}{4.312488in}}%
\pgfpathlineto{\pgfqpoint{0.884543in}{4.324222in}}%
\pgfpathlineto{\pgfqpoint{0.886466in}{4.318530in}}%
\pgfpathlineto{\pgfqpoint{0.888389in}{4.295703in}}%
\pgfpathlineto{\pgfqpoint{0.890313in}{4.305044in}}%
\pgfpathlineto{\pgfqpoint{0.896083in}{4.276078in}}%
\pgfpathlineto{\pgfqpoint{0.898006in}{4.282773in}}%
\pgfpathlineto{\pgfqpoint{0.899930in}{4.274657in}}%
\pgfpathlineto{\pgfqpoint{0.901853in}{4.259570in}}%
\pgfpathlineto{\pgfqpoint{0.903777in}{4.261582in}}%
\pgfpathlineto{\pgfqpoint{0.907624in}{4.252615in}}%
\pgfpathlineto{\pgfqpoint{0.909547in}{4.255339in}}%
\pgfpathlineto{\pgfqpoint{0.913394in}{4.237666in}}%
\pgfpathlineto{\pgfqpoint{0.917241in}{4.258380in}}%
\pgfpathlineto{\pgfqpoint{0.919164in}{4.254834in}}%
\pgfpathlineto{\pgfqpoint{0.923011in}{4.277706in}}%
\pgfpathlineto{\pgfqpoint{0.924934in}{4.271642in}}%
\pgfpathlineto{\pgfqpoint{0.926858in}{4.275193in}}%
\pgfpathlineto{\pgfqpoint{0.930704in}{4.291910in}}%
\pgfpathlineto{\pgfqpoint{0.932628in}{4.276456in}}%
\pgfpathlineto{\pgfqpoint{0.934551in}{4.286994in}}%
\pgfpathlineto{\pgfqpoint{0.936475in}{4.276850in}}%
\pgfpathlineto{\pgfqpoint{0.938398in}{4.281311in}}%
\pgfpathlineto{\pgfqpoint{0.940322in}{4.281284in}}%
\pgfpathlineto{\pgfqpoint{0.942245in}{4.283820in}}%
\pgfpathlineto{\pgfqpoint{0.944168in}{4.276546in}}%
\pgfpathlineto{\pgfqpoint{0.946092in}{4.299953in}}%
\pgfpathlineto{\pgfqpoint{0.948015in}{4.291632in}}%
\pgfpathlineto{\pgfqpoint{0.949939in}{4.291015in}}%
\pgfpathlineto{\pgfqpoint{0.951862in}{4.277698in}}%
\pgfpathlineto{\pgfqpoint{0.953785in}{4.278009in}}%
\pgfpathlineto{\pgfqpoint{0.957632in}{4.260306in}}%
\pgfpathlineto{\pgfqpoint{0.959556in}{4.263805in}}%
\pgfpathlineto{\pgfqpoint{0.961479in}{4.274489in}}%
\pgfpathlineto{\pgfqpoint{0.963402in}{4.274932in}}%
\pgfpathlineto{\pgfqpoint{0.965326in}{4.290407in}}%
\pgfpathlineto{\pgfqpoint{0.967249in}{4.288267in}}%
\pgfpathlineto{\pgfqpoint{0.969173in}{4.278178in}}%
\pgfpathlineto{\pgfqpoint{0.973020in}{4.310796in}}%
\pgfpathlineto{\pgfqpoint{0.976866in}{4.299168in}}%
\pgfpathlineto{\pgfqpoint{0.978790in}{4.301153in}}%
\pgfpathlineto{\pgfqpoint{0.984560in}{4.276345in}}%
\pgfpathlineto{\pgfqpoint{0.986483in}{4.274215in}}%
\pgfpathlineto{\pgfqpoint{0.988407in}{4.264582in}}%
\pgfpathlineto{\pgfqpoint{0.990330in}{4.265797in}}%
\pgfpathlineto{\pgfqpoint{0.992254in}{4.273177in}}%
\pgfpathlineto{\pgfqpoint{0.994177in}{4.273525in}}%
\pgfpathlineto{\pgfqpoint{0.998024in}{4.286419in}}%
\pgfpathlineto{\pgfqpoint{0.999947in}{4.279944in}}%
\pgfpathlineto{\pgfqpoint{1.001871in}{4.292748in}}%
\pgfpathlineto{\pgfqpoint{1.005717in}{4.274910in}}%
\pgfpathlineto{\pgfqpoint{1.007641in}{4.284924in}}%
\pgfpathlineto{\pgfqpoint{1.009564in}{4.274178in}}%
\pgfpathlineto{\pgfqpoint{1.011488in}{4.279979in}}%
\pgfpathlineto{\pgfqpoint{1.013411in}{4.274696in}}%
\pgfpathlineto{\pgfqpoint{1.015335in}{4.259246in}}%
\pgfpathlineto{\pgfqpoint{1.017258in}{4.271417in}}%
\pgfpathlineto{\pgfqpoint{1.019181in}{4.253720in}}%
\pgfpathlineto{\pgfqpoint{1.021105in}{4.264842in}}%
\pgfpathlineto{\pgfqpoint{1.023028in}{4.244759in}}%
\pgfpathlineto{\pgfqpoint{1.024952in}{4.268949in}}%
\pgfpathlineto{\pgfqpoint{1.026875in}{4.269845in}}%
\pgfpathlineto{\pgfqpoint{1.028798in}{4.273018in}}%
\pgfpathlineto{\pgfqpoint{1.032645in}{4.263365in}}%
\pgfpathlineto{\pgfqpoint{1.034569in}{4.267488in}}%
\pgfpathlineto{\pgfqpoint{1.036492in}{4.278136in}}%
\pgfpathlineto{\pgfqpoint{1.038415in}{4.276790in}}%
\pgfpathlineto{\pgfqpoint{1.040339in}{4.273619in}}%
\pgfpathlineto{\pgfqpoint{1.042262in}{4.260696in}}%
\pgfpathlineto{\pgfqpoint{1.044186in}{4.276880in}}%
\pgfpathlineto{\pgfqpoint{1.048033in}{4.250238in}}%
\pgfpathlineto{\pgfqpoint{1.049956in}{4.248135in}}%
\pgfpathlineto{\pgfqpoint{1.051879in}{4.253635in}}%
\pgfpathlineto{\pgfqpoint{1.053803in}{4.253695in}}%
\pgfpathlineto{\pgfqpoint{1.055726in}{4.243352in}}%
\pgfpathlineto{\pgfqpoint{1.057650in}{4.245009in}}%
\pgfpathlineto{\pgfqpoint{1.059573in}{4.241098in}}%
\pgfpathlineto{\pgfqpoint{1.061496in}{4.227931in}}%
\pgfpathlineto{\pgfqpoint{1.065343in}{4.242642in}}%
\pgfpathlineto{\pgfqpoint{1.067267in}{4.236142in}}%
\pgfpathlineto{\pgfqpoint{1.069190in}{4.214277in}}%
\pgfpathlineto{\pgfqpoint{1.071113in}{4.212914in}}%
\pgfpathlineto{\pgfqpoint{1.073037in}{4.213016in}}%
\pgfpathlineto{\pgfqpoint{1.074960in}{4.211438in}}%
\pgfpathlineto{\pgfqpoint{1.076884in}{4.223875in}}%
\pgfpathlineto{\pgfqpoint{1.078807in}{4.225835in}}%
\pgfpathlineto{\pgfqpoint{1.080731in}{4.245335in}}%
\pgfpathlineto{\pgfqpoint{1.082654in}{4.244642in}}%
\pgfpathlineto{\pgfqpoint{1.084577in}{4.239116in}}%
\pgfpathlineto{\pgfqpoint{1.086501in}{4.249456in}}%
\pgfpathlineto{\pgfqpoint{1.088424in}{4.249362in}}%
\pgfpathlineto{\pgfqpoint{1.090348in}{4.247057in}}%
\pgfpathlineto{\pgfqpoint{1.092271in}{4.231049in}}%
\pgfpathlineto{\pgfqpoint{1.096118in}{4.260563in}}%
\pgfpathlineto{\pgfqpoint{1.098041in}{4.269352in}}%
\pgfpathlineto{\pgfqpoint{1.099965in}{4.266172in}}%
\pgfpathlineto{\pgfqpoint{1.103811in}{4.249262in}}%
\pgfpathlineto{\pgfqpoint{1.105735in}{4.259545in}}%
\pgfpathlineto{\pgfqpoint{1.107658in}{4.263406in}}%
\pgfpathlineto{\pgfqpoint{1.109582in}{4.271126in}}%
\pgfpathlineto{\pgfqpoint{1.111505in}{4.257082in}}%
\pgfpathlineto{\pgfqpoint{1.115352in}{4.263814in}}%
\pgfpathlineto{\pgfqpoint{1.117275in}{4.257739in}}%
\pgfpathlineto{\pgfqpoint{1.119199in}{4.256625in}}%
\pgfpathlineto{\pgfqpoint{1.123046in}{4.269700in}}%
\pgfpathlineto{\pgfqpoint{1.124969in}{4.276260in}}%
\pgfpathlineto{\pgfqpoint{1.128816in}{4.265768in}}%
\pgfpathlineto{\pgfqpoint{1.130739in}{4.265039in}}%
\pgfpathlineto{\pgfqpoint{1.132663in}{4.254487in}}%
\pgfpathlineto{\pgfqpoint{1.136509in}{4.250816in}}%
\pgfpathlineto{\pgfqpoint{1.140356in}{4.274060in}}%
\pgfpathlineto{\pgfqpoint{1.142280in}{4.286283in}}%
\pgfpathlineto{\pgfqpoint{1.146126in}{4.290846in}}%
\pgfpathlineto{\pgfqpoint{1.148050in}{4.297584in}}%
\pgfpathlineto{\pgfqpoint{1.149973in}{4.290266in}}%
\pgfpathlineto{\pgfqpoint{1.151897in}{4.294814in}}%
\pgfpathlineto{\pgfqpoint{1.153820in}{4.295686in}}%
\pgfpathlineto{\pgfqpoint{1.155744in}{4.310789in}}%
\pgfpathlineto{\pgfqpoint{1.157667in}{4.305261in}}%
\pgfpathlineto{\pgfqpoint{1.159590in}{4.304292in}}%
\pgfpathlineto{\pgfqpoint{1.161514in}{4.295713in}}%
\pgfpathlineto{\pgfqpoint{1.163437in}{4.301533in}}%
\pgfpathlineto{\pgfqpoint{1.165361in}{4.299905in}}%
\pgfpathlineto{\pgfqpoint{1.167284in}{4.300850in}}%
\pgfpathlineto{\pgfqpoint{1.171131in}{4.289857in}}%
\pgfpathlineto{\pgfqpoint{1.173054in}{4.290677in}}%
\pgfpathlineto{\pgfqpoint{1.174978in}{4.281003in}}%
\pgfpathlineto{\pgfqpoint{1.178824in}{4.296628in}}%
\pgfpathlineto{\pgfqpoint{1.180748in}{4.291528in}}%
\pgfpathlineto{\pgfqpoint{1.182671in}{4.300537in}}%
\pgfpathlineto{\pgfqpoint{1.184595in}{4.280407in}}%
\pgfpathlineto{\pgfqpoint{1.186518in}{4.284859in}}%
\pgfpathlineto{\pgfqpoint{1.188442in}{4.292323in}}%
\pgfpathlineto{\pgfqpoint{1.190365in}{4.308350in}}%
\pgfpathlineto{\pgfqpoint{1.194212in}{4.285178in}}%
\pgfpathlineto{\pgfqpoint{1.198059in}{4.277216in}}%
\pgfpathlineto{\pgfqpoint{1.201905in}{4.288593in}}%
\pgfpathlineto{\pgfqpoint{1.207676in}{4.282260in}}%
\pgfpathlineto{\pgfqpoint{1.209599in}{4.282400in}}%
\pgfpathlineto{\pgfqpoint{1.211522in}{4.260492in}}%
\pgfpathlineto{\pgfqpoint{1.213446in}{4.253338in}}%
\pgfpathlineto{\pgfqpoint{1.215369in}{4.263948in}}%
\pgfpathlineto{\pgfqpoint{1.217293in}{4.265005in}}%
\pgfpathlineto{\pgfqpoint{1.221140in}{4.250237in}}%
\pgfpathlineto{\pgfqpoint{1.223063in}{4.247325in}}%
\pgfpathlineto{\pgfqpoint{1.226910in}{4.235874in}}%
\pgfpathlineto{\pgfqpoint{1.230757in}{4.208899in}}%
\pgfpathlineto{\pgfqpoint{1.232680in}{4.213154in}}%
\pgfpathlineto{\pgfqpoint{1.234603in}{4.206964in}}%
\pgfpathlineto{\pgfqpoint{1.236527in}{4.219282in}}%
\pgfpathlineto{\pgfqpoint{1.238450in}{4.212143in}}%
\pgfpathlineto{\pgfqpoint{1.240374in}{4.209298in}}%
\pgfpathlineto{\pgfqpoint{1.242297in}{4.199898in}}%
\pgfpathlineto{\pgfqpoint{1.244220in}{4.211936in}}%
\pgfpathlineto{\pgfqpoint{1.248067in}{4.200826in}}%
\pgfpathlineto{\pgfqpoint{1.249991in}{4.187669in}}%
\pgfpathlineto{\pgfqpoint{1.251914in}{4.199149in}}%
\pgfpathlineto{\pgfqpoint{1.253838in}{4.188602in}}%
\pgfpathlineto{\pgfqpoint{1.255761in}{4.189551in}}%
\pgfpathlineto{\pgfqpoint{1.259608in}{4.182262in}}%
\pgfpathlineto{\pgfqpoint{1.261531in}{4.184699in}}%
\pgfpathlineto{\pgfqpoint{1.263455in}{4.178602in}}%
\pgfpathlineto{\pgfqpoint{1.265378in}{4.178984in}}%
\pgfpathlineto{\pgfqpoint{1.269225in}{4.209893in}}%
\pgfpathlineto{\pgfqpoint{1.271148in}{4.196354in}}%
\pgfpathlineto{\pgfqpoint{1.276918in}{4.209338in}}%
\pgfpathlineto{\pgfqpoint{1.278842in}{4.206301in}}%
\pgfpathlineto{\pgfqpoint{1.280765in}{4.193866in}}%
\pgfpathlineto{\pgfqpoint{1.282689in}{4.202397in}}%
\pgfpathlineto{\pgfqpoint{1.286536in}{4.202269in}}%
\pgfpathlineto{\pgfqpoint{1.288459in}{4.197230in}}%
\pgfpathlineto{\pgfqpoint{1.290382in}{4.195915in}}%
\pgfpathlineto{\pgfqpoint{1.292306in}{4.188608in}}%
\pgfpathlineto{\pgfqpoint{1.294229in}{4.196976in}}%
\pgfpathlineto{\pgfqpoint{1.296153in}{4.194955in}}%
\pgfpathlineto{\pgfqpoint{1.298076in}{4.201962in}}%
\pgfpathlineto{\pgfqpoint{1.301923in}{4.190406in}}%
\pgfpathlineto{\pgfqpoint{1.303846in}{4.185013in}}%
\pgfpathlineto{\pgfqpoint{1.305770in}{4.193868in}}%
\pgfpathlineto{\pgfqpoint{1.307693in}{4.190959in}}%
\pgfpathlineto{\pgfqpoint{1.309616in}{4.200640in}}%
\pgfpathlineto{\pgfqpoint{1.311540in}{4.195723in}}%
\pgfpathlineto{\pgfqpoint{1.315387in}{4.205643in}}%
\pgfpathlineto{\pgfqpoint{1.317310in}{4.202623in}}%
\pgfpathlineto{\pgfqpoint{1.321157in}{4.210596in}}%
\pgfpathlineto{\pgfqpoint{1.323080in}{4.215515in}}%
\pgfpathlineto{\pgfqpoint{1.325004in}{4.211489in}}%
\pgfpathlineto{\pgfqpoint{1.326927in}{4.223514in}}%
\pgfpathlineto{\pgfqpoint{1.330774in}{4.202576in}}%
\pgfpathlineto{\pgfqpoint{1.332697in}{4.194921in}}%
\pgfpathlineto{\pgfqpoint{1.334621in}{4.212179in}}%
\pgfpathlineto{\pgfqpoint{1.336544in}{4.199560in}}%
\pgfpathlineto{\pgfqpoint{1.338468in}{4.195977in}}%
\pgfpathlineto{\pgfqpoint{1.340391in}{4.211745in}}%
\pgfpathlineto{\pgfqpoint{1.342314in}{4.203718in}}%
\pgfpathlineto{\pgfqpoint{1.344238in}{4.213989in}}%
\pgfpathlineto{\pgfqpoint{1.346161in}{4.209088in}}%
\pgfpathlineto{\pgfqpoint{1.348085in}{4.209547in}}%
\pgfpathlineto{\pgfqpoint{1.350008in}{4.208344in}}%
\pgfpathlineto{\pgfqpoint{1.351931in}{4.199522in}}%
\pgfpathlineto{\pgfqpoint{1.355778in}{4.217809in}}%
\pgfpathlineto{\pgfqpoint{1.357702in}{4.217005in}}%
\pgfpathlineto{\pgfqpoint{1.359625in}{4.211120in}}%
\pgfpathlineto{\pgfqpoint{1.361549in}{4.226713in}}%
\pgfpathlineto{\pgfqpoint{1.363472in}{4.222311in}}%
\pgfpathlineto{\pgfqpoint{1.365395in}{4.236522in}}%
\pgfpathlineto{\pgfqpoint{1.367319in}{4.238264in}}%
\pgfpathlineto{\pgfqpoint{1.369242in}{4.233420in}}%
\pgfpathlineto{\pgfqpoint{1.371166in}{4.232398in}}%
\pgfpathlineto{\pgfqpoint{1.373089in}{4.255630in}}%
\pgfpathlineto{\pgfqpoint{1.375012in}{4.253402in}}%
\pgfpathlineto{\pgfqpoint{1.376936in}{4.248655in}}%
\pgfpathlineto{\pgfqpoint{1.380783in}{4.258728in}}%
\pgfpathlineto{\pgfqpoint{1.382706in}{4.260052in}}%
\pgfpathlineto{\pgfqpoint{1.384629in}{4.254207in}}%
\pgfpathlineto{\pgfqpoint{1.392323in}{4.297345in}}%
\pgfpathlineto{\pgfqpoint{1.396170in}{4.307047in}}%
\pgfpathlineto{\pgfqpoint{1.401940in}{4.326644in}}%
\pgfpathlineto{\pgfqpoint{1.403864in}{4.327780in}}%
\pgfpathlineto{\pgfqpoint{1.405787in}{4.327317in}}%
\pgfpathlineto{\pgfqpoint{1.407710in}{4.334207in}}%
\pgfpathlineto{\pgfqpoint{1.409634in}{4.349450in}}%
\pgfpathlineto{\pgfqpoint{1.411557in}{4.337757in}}%
\pgfpathlineto{\pgfqpoint{1.415404in}{4.354552in}}%
\pgfpathlineto{\pgfqpoint{1.417327in}{4.352429in}}%
\pgfpathlineto{\pgfqpoint{1.419251in}{4.336472in}}%
\pgfpathlineto{\pgfqpoint{1.421174in}{4.332849in}}%
\pgfpathlineto{\pgfqpoint{1.423098in}{4.337004in}}%
\pgfpathlineto{\pgfqpoint{1.425021in}{4.333212in}}%
\pgfpathlineto{\pgfqpoint{1.428868in}{4.341119in}}%
\pgfpathlineto{\pgfqpoint{1.430791in}{4.327770in}}%
\pgfpathlineto{\pgfqpoint{1.432715in}{4.340921in}}%
\pgfpathlineto{\pgfqpoint{1.436562in}{4.340628in}}%
\pgfpathlineto{\pgfqpoint{1.438485in}{4.357644in}}%
\pgfpathlineto{\pgfqpoint{1.440408in}{4.349401in}}%
\pgfpathlineto{\pgfqpoint{1.442332in}{4.358668in}}%
\pgfpathlineto{\pgfqpoint{1.450025in}{4.333682in}}%
\pgfpathlineto{\pgfqpoint{1.451949in}{4.339364in}}%
\pgfpathlineto{\pgfqpoint{1.455796in}{4.320237in}}%
\pgfpathlineto{\pgfqpoint{1.457719in}{4.324580in}}%
\pgfpathlineto{\pgfqpoint{1.461566in}{4.350415in}}%
\pgfpathlineto{\pgfqpoint{1.463489in}{4.360889in}}%
\pgfpathlineto{\pgfqpoint{1.465413in}{4.362942in}}%
\pgfpathlineto{\pgfqpoint{1.467336in}{4.358945in}}%
\pgfpathlineto{\pgfqpoint{1.469260in}{4.363839in}}%
\pgfpathlineto{\pgfqpoint{1.471183in}{4.359292in}}%
\pgfpathlineto{\pgfqpoint{1.473106in}{4.345961in}}%
\pgfpathlineto{\pgfqpoint{1.475030in}{4.361032in}}%
\pgfpathlineto{\pgfqpoint{1.476953in}{4.362416in}}%
\pgfpathlineto{\pgfqpoint{1.480800in}{4.352333in}}%
\pgfpathlineto{\pgfqpoint{1.482723in}{4.345329in}}%
\pgfpathlineto{\pgfqpoint{1.486570in}{4.364034in}}%
\pgfpathlineto{\pgfqpoint{1.488494in}{4.363620in}}%
\pgfpathlineto{\pgfqpoint{1.490417in}{4.377706in}}%
\pgfpathlineto{\pgfqpoint{1.492340in}{4.372929in}}%
\pgfpathlineto{\pgfqpoint{1.494264in}{4.384116in}}%
\pgfpathlineto{\pgfqpoint{1.496187in}{4.372849in}}%
\pgfpathlineto{\pgfqpoint{1.498111in}{4.379650in}}%
\pgfpathlineto{\pgfqpoint{1.500034in}{4.373183in}}%
\pgfpathlineto{\pgfqpoint{1.501958in}{4.388209in}}%
\pgfpathlineto{\pgfqpoint{1.503881in}{4.390151in}}%
\pgfpathlineto{\pgfqpoint{1.505804in}{4.385620in}}%
\pgfpathlineto{\pgfqpoint{1.507728in}{4.402211in}}%
\pgfpathlineto{\pgfqpoint{1.509651in}{4.395408in}}%
\pgfpathlineto{\pgfqpoint{1.511575in}{4.397182in}}%
\pgfpathlineto{\pgfqpoint{1.513498in}{4.394035in}}%
\pgfpathlineto{\pgfqpoint{1.515421in}{4.399552in}}%
\pgfpathlineto{\pgfqpoint{1.519268in}{4.420968in}}%
\pgfpathlineto{\pgfqpoint{1.523115in}{4.414123in}}%
\pgfpathlineto{\pgfqpoint{1.525038in}{4.402100in}}%
\pgfpathlineto{\pgfqpoint{1.526962in}{4.402742in}}%
\pgfpathlineto{\pgfqpoint{1.528885in}{4.407484in}}%
\pgfpathlineto{\pgfqpoint{1.532732in}{4.390686in}}%
\pgfpathlineto{\pgfqpoint{1.534656in}{4.393053in}}%
\pgfpathlineto{\pgfqpoint{1.538502in}{4.406822in}}%
\pgfpathlineto{\pgfqpoint{1.542349in}{4.387619in}}%
\pgfpathlineto{\pgfqpoint{1.546196in}{4.392548in}}%
\pgfpathlineto{\pgfqpoint{1.548119in}{4.402247in}}%
\pgfpathlineto{\pgfqpoint{1.550043in}{4.395257in}}%
\pgfpathlineto{\pgfqpoint{1.551966in}{4.393179in}}%
\pgfpathlineto{\pgfqpoint{1.555813in}{4.384677in}}%
\pgfpathlineto{\pgfqpoint{1.557736in}{4.370583in}}%
\pgfpathlineto{\pgfqpoint{1.561583in}{4.374976in}}%
\pgfpathlineto{\pgfqpoint{1.563507in}{4.364533in}}%
\pgfpathlineto{\pgfqpoint{1.565430in}{4.361003in}}%
\pgfpathlineto{\pgfqpoint{1.569277in}{4.371900in}}%
\pgfpathlineto{\pgfqpoint{1.571200in}{4.377658in}}%
\pgfpathlineto{\pgfqpoint{1.573124in}{4.379066in}}%
\pgfpathlineto{\pgfqpoint{1.575047in}{4.363435in}}%
\pgfpathlineto{\pgfqpoint{1.576971in}{4.363201in}}%
\pgfpathlineto{\pgfqpoint{1.578894in}{4.365987in}}%
\pgfpathlineto{\pgfqpoint{1.582741in}{4.376836in}}%
\pgfpathlineto{\pgfqpoint{1.584664in}{4.364079in}}%
\pgfpathlineto{\pgfqpoint{1.588511in}{4.379359in}}%
\pgfpathlineto{\pgfqpoint{1.590434in}{4.374806in}}%
\pgfpathlineto{\pgfqpoint{1.598128in}{4.418641in}}%
\pgfpathlineto{\pgfqpoint{1.601975in}{4.401355in}}%
\pgfpathlineto{\pgfqpoint{1.603898in}{4.411311in}}%
\pgfpathlineto{\pgfqpoint{1.605822in}{4.406976in}}%
\pgfpathlineto{\pgfqpoint{1.611592in}{4.414556in}}%
\pgfpathlineto{\pgfqpoint{1.613515in}{4.395117in}}%
\pgfpathlineto{\pgfqpoint{1.615439in}{4.393032in}}%
\pgfpathlineto{\pgfqpoint{1.617362in}{4.394299in}}%
\pgfpathlineto{\pgfqpoint{1.619286in}{4.401293in}}%
\pgfpathlineto{\pgfqpoint{1.621209in}{4.400893in}}%
\pgfpathlineto{\pgfqpoint{1.625056in}{4.403075in}}%
\pgfpathlineto{\pgfqpoint{1.626979in}{4.394450in}}%
\pgfpathlineto{\pgfqpoint{1.630826in}{4.420021in}}%
\pgfpathlineto{\pgfqpoint{1.634673in}{4.414105in}}%
\pgfpathlineto{\pgfqpoint{1.636596in}{4.404598in}}%
\pgfpathlineto{\pgfqpoint{1.638520in}{4.405622in}}%
\pgfpathlineto{\pgfqpoint{1.640443in}{4.403926in}}%
\pgfpathlineto{\pgfqpoint{1.642367in}{4.411768in}}%
\pgfpathlineto{\pgfqpoint{1.646213in}{4.398636in}}%
\pgfpathlineto{\pgfqpoint{1.650060in}{4.407426in}}%
\pgfpathlineto{\pgfqpoint{1.653907in}{4.425599in}}%
\pgfpathlineto{\pgfqpoint{1.655830in}{4.427402in}}%
\pgfpathlineto{\pgfqpoint{1.659677in}{4.445205in}}%
\pgfpathlineto{\pgfqpoint{1.661601in}{4.444537in}}%
\pgfpathlineto{\pgfqpoint{1.669294in}{4.469610in}}%
\pgfpathlineto{\pgfqpoint{1.671218in}{4.467341in}}%
\pgfpathlineto{\pgfqpoint{1.673141in}{4.475299in}}%
\pgfpathlineto{\pgfqpoint{1.676988in}{4.459950in}}%
\pgfpathlineto{\pgfqpoint{1.680835in}{4.473843in}}%
\pgfpathlineto{\pgfqpoint{1.682758in}{4.467840in}}%
\pgfpathlineto{\pgfqpoint{1.684682in}{4.470921in}}%
\pgfpathlineto{\pgfqpoint{1.690452in}{4.463783in}}%
\pgfpathlineto{\pgfqpoint{1.692375in}{4.454064in}}%
\pgfpathlineto{\pgfqpoint{1.694299in}{4.454316in}}%
\pgfpathlineto{\pgfqpoint{1.696222in}{4.458916in}}%
\pgfpathlineto{\pgfqpoint{1.698145in}{4.469745in}}%
\pgfpathlineto{\pgfqpoint{1.700069in}{4.471569in}}%
\pgfpathlineto{\pgfqpoint{1.703916in}{4.491938in}}%
\pgfpathlineto{\pgfqpoint{1.705839in}{4.483581in}}%
\pgfpathlineto{\pgfqpoint{1.707763in}{4.504785in}}%
\pgfpathlineto{\pgfqpoint{1.709686in}{4.503581in}}%
\pgfpathlineto{\pgfqpoint{1.711609in}{4.500747in}}%
\pgfpathlineto{\pgfqpoint{1.715456in}{4.485851in}}%
\pgfpathlineto{\pgfqpoint{1.717380in}{4.483479in}}%
\pgfpathlineto{\pgfqpoint{1.719303in}{4.484762in}}%
\pgfpathlineto{\pgfqpoint{1.723150in}{4.505931in}}%
\pgfpathlineto{\pgfqpoint{1.726997in}{4.495167in}}%
\pgfpathlineto{\pgfqpoint{1.728920in}{4.503536in}}%
\pgfpathlineto{\pgfqpoint{1.730843in}{4.502539in}}%
\pgfpathlineto{\pgfqpoint{1.732767in}{4.497504in}}%
\pgfpathlineto{\pgfqpoint{1.734690in}{4.479796in}}%
\pgfpathlineto{\pgfqpoint{1.736614in}{4.476602in}}%
\pgfpathlineto{\pgfqpoint{1.738537in}{4.460728in}}%
\pgfpathlineto{\pgfqpoint{1.740461in}{4.472885in}}%
\pgfpathlineto{\pgfqpoint{1.742384in}{4.469672in}}%
\pgfpathlineto{\pgfqpoint{1.744307in}{4.459122in}}%
\pgfpathlineto{\pgfqpoint{1.746231in}{4.466248in}}%
\pgfpathlineto{\pgfqpoint{1.748154in}{4.466815in}}%
\pgfpathlineto{\pgfqpoint{1.750078in}{4.461293in}}%
\pgfpathlineto{\pgfqpoint{1.752001in}{4.446923in}}%
\pgfpathlineto{\pgfqpoint{1.753924in}{4.460831in}}%
\pgfpathlineto{\pgfqpoint{1.755848in}{4.449653in}}%
\pgfpathlineto{\pgfqpoint{1.757771in}{4.448660in}}%
\pgfpathlineto{\pgfqpoint{1.761618in}{4.455959in}}%
\pgfpathlineto{\pgfqpoint{1.769312in}{4.496415in}}%
\pgfpathlineto{\pgfqpoint{1.771235in}{4.497386in}}%
\pgfpathlineto{\pgfqpoint{1.773158in}{4.497035in}}%
\pgfpathlineto{\pgfqpoint{1.775082in}{4.498670in}}%
\pgfpathlineto{\pgfqpoint{1.777005in}{4.497971in}}%
\pgfpathlineto{\pgfqpoint{1.778929in}{4.524686in}}%
\pgfpathlineto{\pgfqpoint{1.786622in}{4.549076in}}%
\pgfpathlineto{\pgfqpoint{1.788546in}{4.538040in}}%
\pgfpathlineto{\pgfqpoint{1.790469in}{4.546804in}}%
\pgfpathlineto{\pgfqpoint{1.792393in}{4.541562in}}%
\pgfpathlineto{\pgfqpoint{1.794316in}{4.552599in}}%
\pgfpathlineto{\pgfqpoint{1.796239in}{4.554191in}}%
\pgfpathlineto{\pgfqpoint{1.798163in}{4.552768in}}%
\pgfpathlineto{\pgfqpoint{1.800086in}{4.548710in}}%
\pgfpathlineto{\pgfqpoint{1.803933in}{4.530388in}}%
\pgfpathlineto{\pgfqpoint{1.805856in}{4.523376in}}%
\pgfpathlineto{\pgfqpoint{1.811627in}{4.553736in}}%
\pgfpathlineto{\pgfqpoint{1.813550in}{4.533585in}}%
\pgfpathlineto{\pgfqpoint{1.815474in}{4.536552in}}%
\pgfpathlineto{\pgfqpoint{1.817397in}{4.535444in}}%
\pgfpathlineto{\pgfqpoint{1.819320in}{4.560845in}}%
\pgfpathlineto{\pgfqpoint{1.821244in}{4.569279in}}%
\pgfpathlineto{\pgfqpoint{1.823167in}{4.570670in}}%
\pgfpathlineto{\pgfqpoint{1.827014in}{4.560886in}}%
\pgfpathlineto{\pgfqpoint{1.828937in}{4.559473in}}%
\pgfpathlineto{\pgfqpoint{1.830861in}{4.562286in}}%
\pgfpathlineto{\pgfqpoint{1.832784in}{4.559041in}}%
\pgfpathlineto{\pgfqpoint{1.834708in}{4.561033in}}%
\pgfpathlineto{\pgfqpoint{1.838554in}{4.553286in}}%
\pgfpathlineto{\pgfqpoint{1.840478in}{4.562282in}}%
\pgfpathlineto{\pgfqpoint{1.844325in}{4.563376in}}%
\pgfpathlineto{\pgfqpoint{1.846248in}{4.552305in}}%
\pgfpathlineto{\pgfqpoint{1.848172in}{4.552953in}}%
\pgfpathlineto{\pgfqpoint{1.852018in}{4.562856in}}%
\pgfpathlineto{\pgfqpoint{1.853942in}{4.546355in}}%
\pgfpathlineto{\pgfqpoint{1.855865in}{4.555588in}}%
\pgfpathlineto{\pgfqpoint{1.857789in}{4.557768in}}%
\pgfpathlineto{\pgfqpoint{1.859712in}{4.551722in}}%
\pgfpathlineto{\pgfqpoint{1.863559in}{4.566968in}}%
\pgfpathlineto{\pgfqpoint{1.865482in}{4.588290in}}%
\pgfpathlineto{\pgfqpoint{1.867406in}{4.584474in}}%
\pgfpathlineto{\pgfqpoint{1.869329in}{4.594375in}}%
\pgfpathlineto{\pgfqpoint{1.871252in}{4.589610in}}%
\pgfpathlineto{\pgfqpoint{1.873176in}{4.567453in}}%
\pgfpathlineto{\pgfqpoint{1.875099in}{4.588899in}}%
\pgfpathlineto{\pgfqpoint{1.877023in}{4.577578in}}%
\pgfpathlineto{\pgfqpoint{1.878946in}{4.583426in}}%
\pgfpathlineto{\pgfqpoint{1.882793in}{4.583266in}}%
\pgfpathlineto{\pgfqpoint{1.884716in}{4.593232in}}%
\pgfpathlineto{\pgfqpoint{1.890487in}{4.565343in}}%
\pgfpathlineto{\pgfqpoint{1.892410in}{4.566564in}}%
\pgfpathlineto{\pgfqpoint{1.894333in}{4.541126in}}%
\pgfpathlineto{\pgfqpoint{1.898180in}{4.547792in}}%
\pgfpathlineto{\pgfqpoint{1.900104in}{4.562745in}}%
\pgfpathlineto{\pgfqpoint{1.902027in}{4.563628in}}%
\pgfpathlineto{\pgfqpoint{1.903950in}{4.562269in}}%
\pgfpathlineto{\pgfqpoint{1.905874in}{4.558246in}}%
\pgfpathlineto{\pgfqpoint{1.909721in}{4.559056in}}%
\pgfpathlineto{\pgfqpoint{1.911644in}{4.570884in}}%
\pgfpathlineto{\pgfqpoint{1.913567in}{4.570547in}}%
\pgfpathlineto{\pgfqpoint{1.915491in}{4.582077in}}%
\pgfpathlineto{\pgfqpoint{1.917414in}{4.574131in}}%
\pgfpathlineto{\pgfqpoint{1.921261in}{4.576782in}}%
\pgfpathlineto{\pgfqpoint{1.923185in}{4.572282in}}%
\pgfpathlineto{\pgfqpoint{1.925108in}{4.572772in}}%
\pgfpathlineto{\pgfqpoint{1.927031in}{4.587297in}}%
\pgfpathlineto{\pgfqpoint{1.928955in}{4.582527in}}%
\pgfpathlineto{\pgfqpoint{1.930878in}{4.588243in}}%
\pgfpathlineto{\pgfqpoint{1.936648in}{4.561678in}}%
\pgfpathlineto{\pgfqpoint{1.940495in}{4.566741in}}%
\pgfpathlineto{\pgfqpoint{1.946265in}{4.597960in}}%
\pgfpathlineto{\pgfqpoint{1.948189in}{4.583764in}}%
\pgfpathlineto{\pgfqpoint{1.953959in}{4.566001in}}%
\pgfpathlineto{\pgfqpoint{1.955883in}{4.577079in}}%
\pgfpathlineto{\pgfqpoint{1.959729in}{4.582771in}}%
\pgfpathlineto{\pgfqpoint{1.961653in}{4.578979in}}%
\pgfpathlineto{\pgfqpoint{1.969346in}{4.630874in}}%
\pgfpathlineto{\pgfqpoint{1.975117in}{4.627104in}}%
\pgfpathlineto{\pgfqpoint{1.977040in}{4.627661in}}%
\pgfpathlineto{\pgfqpoint{1.978963in}{4.621125in}}%
\pgfpathlineto{\pgfqpoint{1.988581in}{4.662431in}}%
\pgfpathlineto{\pgfqpoint{1.990504in}{4.658170in}}%
\pgfpathlineto{\pgfqpoint{1.992427in}{4.657477in}}%
\pgfpathlineto{\pgfqpoint{1.994351in}{4.655011in}}%
\pgfpathlineto{\pgfqpoint{1.996274in}{4.656275in}}%
\pgfpathlineto{\pgfqpoint{1.998198in}{4.652463in}}%
\pgfpathlineto{\pgfqpoint{2.000121in}{4.665763in}}%
\pgfpathlineto{\pgfqpoint{2.002044in}{4.655428in}}%
\pgfpathlineto{\pgfqpoint{2.003968in}{4.657790in}}%
\pgfpathlineto{\pgfqpoint{2.005891in}{4.654377in}}%
\pgfpathlineto{\pgfqpoint{2.007815in}{4.656017in}}%
\pgfpathlineto{\pgfqpoint{2.009738in}{4.661890in}}%
\pgfpathlineto{\pgfqpoint{2.011661in}{4.659993in}}%
\pgfpathlineto{\pgfqpoint{2.013585in}{4.665908in}}%
\pgfpathlineto{\pgfqpoint{2.015508in}{4.667767in}}%
\pgfpathlineto{\pgfqpoint{2.019355in}{4.658025in}}%
\pgfpathlineto{\pgfqpoint{2.021279in}{4.656691in}}%
\pgfpathlineto{\pgfqpoint{2.023202in}{4.639301in}}%
\pgfpathlineto{\pgfqpoint{2.027049in}{4.657684in}}%
\pgfpathlineto{\pgfqpoint{2.028972in}{4.639336in}}%
\pgfpathlineto{\pgfqpoint{2.030896in}{4.645175in}}%
\pgfpathlineto{\pgfqpoint{2.032819in}{4.635329in}}%
\pgfpathlineto{\pgfqpoint{2.034742in}{4.636800in}}%
\pgfpathlineto{\pgfqpoint{2.038589in}{4.653063in}}%
\pgfpathlineto{\pgfqpoint{2.042436in}{4.659736in}}%
\pgfpathlineto{\pgfqpoint{2.044359in}{4.658499in}}%
\pgfpathlineto{\pgfqpoint{2.048206in}{4.630105in}}%
\pgfpathlineto{\pgfqpoint{2.050130in}{4.631767in}}%
\pgfpathlineto{\pgfqpoint{2.052053in}{4.628589in}}%
\pgfpathlineto{\pgfqpoint{2.053976in}{4.638262in}}%
\pgfpathlineto{\pgfqpoint{2.059747in}{4.620165in}}%
\pgfpathlineto{\pgfqpoint{2.061670in}{4.606133in}}%
\pgfpathlineto{\pgfqpoint{2.063594in}{4.609291in}}%
\pgfpathlineto{\pgfqpoint{2.065517in}{4.608141in}}%
\pgfpathlineto{\pgfqpoint{2.067440in}{4.595625in}}%
\pgfpathlineto{\pgfqpoint{2.069364in}{4.593873in}}%
\pgfpathlineto{\pgfqpoint{2.071287in}{4.570356in}}%
\pgfpathlineto{\pgfqpoint{2.073211in}{4.565501in}}%
\pgfpathlineto{\pgfqpoint{2.075134in}{4.565633in}}%
\pgfpathlineto{\pgfqpoint{2.077057in}{4.580037in}}%
\pgfpathlineto{\pgfqpoint{2.078981in}{4.578846in}}%
\pgfpathlineto{\pgfqpoint{2.082828in}{4.615001in}}%
\pgfpathlineto{\pgfqpoint{2.084751in}{4.620110in}}%
\pgfpathlineto{\pgfqpoint{2.086674in}{4.631370in}}%
\pgfpathlineto{\pgfqpoint{2.088598in}{4.629413in}}%
\pgfpathlineto{\pgfqpoint{2.090521in}{4.639897in}}%
\pgfpathlineto{\pgfqpoint{2.092445in}{4.632428in}}%
\pgfpathlineto{\pgfqpoint{2.096292in}{4.641980in}}%
\pgfpathlineto{\pgfqpoint{2.098215in}{4.632189in}}%
\pgfpathlineto{\pgfqpoint{2.100138in}{4.641838in}}%
\pgfpathlineto{\pgfqpoint{2.102062in}{4.657227in}}%
\pgfpathlineto{\pgfqpoint{2.103985in}{4.647386in}}%
\pgfpathlineto{\pgfqpoint{2.105909in}{4.651522in}}%
\pgfpathlineto{\pgfqpoint{2.109755in}{4.635866in}}%
\pgfpathlineto{\pgfqpoint{2.111679in}{4.627262in}}%
\pgfpathlineto{\pgfqpoint{2.113602in}{4.632823in}}%
\pgfpathlineto{\pgfqpoint{2.115526in}{4.618896in}}%
\pgfpathlineto{\pgfqpoint{2.121296in}{4.640820in}}%
\pgfpathlineto{\pgfqpoint{2.123219in}{4.632735in}}%
\pgfpathlineto{\pgfqpoint{2.125143in}{4.630868in}}%
\pgfpathlineto{\pgfqpoint{2.127066in}{4.626212in}}%
\pgfpathlineto{\pgfqpoint{2.128990in}{4.625001in}}%
\pgfpathlineto{\pgfqpoint{2.132836in}{4.619603in}}%
\pgfpathlineto{\pgfqpoint{2.136683in}{4.617400in}}%
\pgfpathlineto{\pgfqpoint{2.138607in}{4.616699in}}%
\pgfpathlineto{\pgfqpoint{2.140530in}{4.620374in}}%
\pgfpathlineto{\pgfqpoint{2.142453in}{4.610244in}}%
\pgfpathlineto{\pgfqpoint{2.146300in}{4.624214in}}%
\pgfpathlineto{\pgfqpoint{2.148224in}{4.618850in}}%
\pgfpathlineto{\pgfqpoint{2.150147in}{4.623282in}}%
\pgfpathlineto{\pgfqpoint{2.152070in}{4.622034in}}%
\pgfpathlineto{\pgfqpoint{2.157841in}{4.598808in}}%
\pgfpathlineto{\pgfqpoint{2.159764in}{4.596632in}}%
\pgfpathlineto{\pgfqpoint{2.161688in}{4.574450in}}%
\pgfpathlineto{\pgfqpoint{2.163611in}{4.576176in}}%
\pgfpathlineto{\pgfqpoint{2.165534in}{4.567710in}}%
\pgfpathlineto{\pgfqpoint{2.167458in}{4.568300in}}%
\pgfpathlineto{\pgfqpoint{2.169381in}{4.560420in}}%
\pgfpathlineto{\pgfqpoint{2.171305in}{4.558128in}}%
\pgfpathlineto{\pgfqpoint{2.173228in}{4.561051in}}%
\pgfpathlineto{\pgfqpoint{2.175151in}{4.585983in}}%
\pgfpathlineto{\pgfqpoint{2.177075in}{4.587921in}}%
\pgfpathlineto{\pgfqpoint{2.180922in}{4.605903in}}%
\pgfpathlineto{\pgfqpoint{2.186692in}{4.585115in}}%
\pgfpathlineto{\pgfqpoint{2.188615in}{4.598125in}}%
\pgfpathlineto{\pgfqpoint{2.190539in}{4.598300in}}%
\pgfpathlineto{\pgfqpoint{2.192462in}{4.595991in}}%
\pgfpathlineto{\pgfqpoint{2.194386in}{4.598083in}}%
\pgfpathlineto{\pgfqpoint{2.196309in}{4.594884in}}%
\pgfpathlineto{\pgfqpoint{2.198232in}{4.585838in}}%
\pgfpathlineto{\pgfqpoint{2.200156in}{4.585093in}}%
\pgfpathlineto{\pgfqpoint{2.202079in}{4.596967in}}%
\pgfpathlineto{\pgfqpoint{2.204003in}{4.589767in}}%
\pgfpathlineto{\pgfqpoint{2.207849in}{4.595803in}}%
\pgfpathlineto{\pgfqpoint{2.209773in}{4.583007in}}%
\pgfpathlineto{\pgfqpoint{2.211696in}{4.580813in}}%
\pgfpathlineto{\pgfqpoint{2.213620in}{4.597633in}}%
\pgfpathlineto{\pgfqpoint{2.215543in}{4.581645in}}%
\pgfpathlineto{\pgfqpoint{2.217466in}{4.592057in}}%
\pgfpathlineto{\pgfqpoint{2.219390in}{4.590282in}}%
\pgfpathlineto{\pgfqpoint{2.223237in}{4.574157in}}%
\pgfpathlineto{\pgfqpoint{2.225160in}{4.581616in}}%
\pgfpathlineto{\pgfqpoint{2.227083in}{4.578721in}}%
\pgfpathlineto{\pgfqpoint{2.229007in}{4.585497in}}%
\pgfpathlineto{\pgfqpoint{2.230930in}{4.581785in}}%
\pgfpathlineto{\pgfqpoint{2.232854in}{4.581031in}}%
\pgfpathlineto{\pgfqpoint{2.234777in}{4.571908in}}%
\pgfpathlineto{\pgfqpoint{2.236701in}{4.577135in}}%
\pgfpathlineto{\pgfqpoint{2.240547in}{4.603202in}}%
\pgfpathlineto{\pgfqpoint{2.242471in}{4.591733in}}%
\pgfpathlineto{\pgfqpoint{2.244394in}{4.596896in}}%
\pgfpathlineto{\pgfqpoint{2.246318in}{4.592024in}}%
\pgfpathlineto{\pgfqpoint{2.248241in}{4.591922in}}%
\pgfpathlineto{\pgfqpoint{2.252088in}{4.562027in}}%
\pgfpathlineto{\pgfqpoint{2.255935in}{4.578651in}}%
\pgfpathlineto{\pgfqpoint{2.257858in}{4.602859in}}%
\pgfpathlineto{\pgfqpoint{2.259781in}{4.602514in}}%
\pgfpathlineto{\pgfqpoint{2.263628in}{4.556999in}}%
\pgfpathlineto{\pgfqpoint{2.265552in}{4.560254in}}%
\pgfpathlineto{\pgfqpoint{2.267475in}{4.566710in}}%
\pgfpathlineto{\pgfqpoint{2.271322in}{4.589815in}}%
\pgfpathlineto{\pgfqpoint{2.273245in}{4.581455in}}%
\pgfpathlineto{\pgfqpoint{2.275169in}{4.587170in}}%
\pgfpathlineto{\pgfqpoint{2.277092in}{4.600870in}}%
\pgfpathlineto{\pgfqpoint{2.279016in}{4.594262in}}%
\pgfpathlineto{\pgfqpoint{2.280939in}{4.579836in}}%
\pgfpathlineto{\pgfqpoint{2.282862in}{4.580999in}}%
\pgfpathlineto{\pgfqpoint{2.284786in}{4.587566in}}%
\pgfpathlineto{\pgfqpoint{2.286709in}{4.589416in}}%
\pgfpathlineto{\pgfqpoint{2.288633in}{4.594908in}}%
\pgfpathlineto{\pgfqpoint{2.290556in}{4.605926in}}%
\pgfpathlineto{\pgfqpoint{2.292479in}{4.610216in}}%
\pgfpathlineto{\pgfqpoint{2.294403in}{4.622006in}}%
\pgfpathlineto{\pgfqpoint{2.296326in}{4.613180in}}%
\pgfpathlineto{\pgfqpoint{2.298250in}{4.610404in}}%
\pgfpathlineto{\pgfqpoint{2.300173in}{4.623876in}}%
\pgfpathlineto{\pgfqpoint{2.302097in}{4.611552in}}%
\pgfpathlineto{\pgfqpoint{2.304020in}{4.610479in}}%
\pgfpathlineto{\pgfqpoint{2.309790in}{4.596147in}}%
\pgfpathlineto{\pgfqpoint{2.313637in}{4.617136in}}%
\pgfpathlineto{\pgfqpoint{2.317484in}{4.638338in}}%
\pgfpathlineto{\pgfqpoint{2.319407in}{4.626800in}}%
\pgfpathlineto{\pgfqpoint{2.323254in}{4.640656in}}%
\pgfpathlineto{\pgfqpoint{2.325177in}{4.634901in}}%
\pgfpathlineto{\pgfqpoint{2.329024in}{4.674505in}}%
\pgfpathlineto{\pgfqpoint{2.330948in}{4.679126in}}%
\pgfpathlineto{\pgfqpoint{2.332871in}{4.657386in}}%
\pgfpathlineto{\pgfqpoint{2.334795in}{4.657370in}}%
\pgfpathlineto{\pgfqpoint{2.336718in}{4.642321in}}%
\pgfpathlineto{\pgfqpoint{2.340565in}{4.660138in}}%
\pgfpathlineto{\pgfqpoint{2.342488in}{4.654792in}}%
\pgfpathlineto{\pgfqpoint{2.344412in}{4.666014in}}%
\pgfpathlineto{\pgfqpoint{2.346335in}{4.670590in}}%
\pgfpathlineto{\pgfqpoint{2.348258in}{4.678926in}}%
\pgfpathlineto{\pgfqpoint{2.352105in}{4.667281in}}%
\pgfpathlineto{\pgfqpoint{2.354029in}{4.686219in}}%
\pgfpathlineto{\pgfqpoint{2.355952in}{4.684380in}}%
\pgfpathlineto{\pgfqpoint{2.357875in}{4.686551in}}%
\pgfpathlineto{\pgfqpoint{2.359799in}{4.675763in}}%
\pgfpathlineto{\pgfqpoint{2.361722in}{4.678882in}}%
\pgfpathlineto{\pgfqpoint{2.363646in}{4.671170in}}%
\pgfpathlineto{\pgfqpoint{2.365569in}{4.672799in}}%
\pgfpathlineto{\pgfqpoint{2.367492in}{4.670565in}}%
\pgfpathlineto{\pgfqpoint{2.369416in}{4.687993in}}%
\pgfpathlineto{\pgfqpoint{2.371339in}{4.684361in}}%
\pgfpathlineto{\pgfqpoint{2.373263in}{4.689726in}}%
\pgfpathlineto{\pgfqpoint{2.377110in}{4.678979in}}%
\pgfpathlineto{\pgfqpoint{2.379033in}{4.670846in}}%
\pgfpathlineto{\pgfqpoint{2.380956in}{4.657335in}}%
\pgfpathlineto{\pgfqpoint{2.382880in}{4.656824in}}%
\pgfpathlineto{\pgfqpoint{2.384803in}{4.646456in}}%
\pgfpathlineto{\pgfqpoint{2.386727in}{4.646835in}}%
\pgfpathlineto{\pgfqpoint{2.388650in}{4.652399in}}%
\pgfpathlineto{\pgfqpoint{2.390573in}{4.646870in}}%
\pgfpathlineto{\pgfqpoint{2.392497in}{4.628010in}}%
\pgfpathlineto{\pgfqpoint{2.394420in}{4.642564in}}%
\pgfpathlineto{\pgfqpoint{2.400190in}{4.641309in}}%
\pgfpathlineto{\pgfqpoint{2.402114in}{4.631356in}}%
\pgfpathlineto{\pgfqpoint{2.404037in}{4.634534in}}%
\pgfpathlineto{\pgfqpoint{2.405961in}{4.626459in}}%
\pgfpathlineto{\pgfqpoint{2.409808in}{4.659395in}}%
\pgfpathlineto{\pgfqpoint{2.413654in}{4.668245in}}%
\pgfpathlineto{\pgfqpoint{2.415578in}{4.669953in}}%
\pgfpathlineto{\pgfqpoint{2.417501in}{4.673970in}}%
\pgfpathlineto{\pgfqpoint{2.419425in}{4.685505in}}%
\pgfpathlineto{\pgfqpoint{2.421348in}{4.667265in}}%
\pgfpathlineto{\pgfqpoint{2.423271in}{4.663294in}}%
\pgfpathlineto{\pgfqpoint{2.425195in}{4.671814in}}%
\pgfpathlineto{\pgfqpoint{2.427118in}{4.672098in}}%
\pgfpathlineto{\pgfqpoint{2.429042in}{4.663738in}}%
\pgfpathlineto{\pgfqpoint{2.430965in}{4.666161in}}%
\pgfpathlineto{\pgfqpoint{2.436735in}{4.699264in}}%
\pgfpathlineto{\pgfqpoint{2.438659in}{4.702057in}}%
\pgfpathlineto{\pgfqpoint{2.442506in}{4.686113in}}%
\pgfpathlineto{\pgfqpoint{2.444429in}{4.691874in}}%
\pgfpathlineto{\pgfqpoint{2.446352in}{4.688718in}}%
\pgfpathlineto{\pgfqpoint{2.448276in}{4.671966in}}%
\pgfpathlineto{\pgfqpoint{2.450199in}{4.676889in}}%
\pgfpathlineto{\pgfqpoint{2.454046in}{4.695765in}}%
\pgfpathlineto{\pgfqpoint{2.455969in}{4.706476in}}%
\pgfpathlineto{\pgfqpoint{2.457893in}{4.701983in}}%
\pgfpathlineto{\pgfqpoint{2.459816in}{4.713824in}}%
\pgfpathlineto{\pgfqpoint{2.461740in}{4.713728in}}%
\pgfpathlineto{\pgfqpoint{2.463663in}{4.717146in}}%
\pgfpathlineto{\pgfqpoint{2.465586in}{4.713371in}}%
\pgfpathlineto{\pgfqpoint{2.467510in}{4.715591in}}%
\pgfpathlineto{\pgfqpoint{2.469433in}{4.713425in}}%
\pgfpathlineto{\pgfqpoint{2.471357in}{4.729299in}}%
\pgfpathlineto{\pgfqpoint{2.473280in}{4.724857in}}%
\pgfpathlineto{\pgfqpoint{2.477127in}{4.743095in}}%
\pgfpathlineto{\pgfqpoint{2.479050in}{4.737269in}}%
\pgfpathlineto{\pgfqpoint{2.482897in}{4.733775in}}%
\pgfpathlineto{\pgfqpoint{2.484821in}{4.747548in}}%
\pgfpathlineto{\pgfqpoint{2.488667in}{4.721500in}}%
\pgfpathlineto{\pgfqpoint{2.492514in}{4.691090in}}%
\pgfpathlineto{\pgfqpoint{2.494438in}{4.682971in}}%
\pgfpathlineto{\pgfqpoint{2.496361in}{4.695007in}}%
\pgfpathlineto{\pgfqpoint{2.498284in}{4.691559in}}%
\pgfpathlineto{\pgfqpoint{2.502131in}{4.717819in}}%
\pgfpathlineto{\pgfqpoint{2.504055in}{4.698701in}}%
\pgfpathlineto{\pgfqpoint{2.505978in}{4.699973in}}%
\pgfpathlineto{\pgfqpoint{2.507901in}{4.689057in}}%
\pgfpathlineto{\pgfqpoint{2.509825in}{4.690585in}}%
\pgfpathlineto{\pgfqpoint{2.511748in}{4.686758in}}%
\pgfpathlineto{\pgfqpoint{2.513672in}{4.686677in}}%
\pgfpathlineto{\pgfqpoint{2.515595in}{4.689875in}}%
\pgfpathlineto{\pgfqpoint{2.519442in}{4.709235in}}%
\pgfpathlineto{\pgfqpoint{2.527136in}{4.693287in}}%
\pgfpathlineto{\pgfqpoint{2.530982in}{4.707541in}}%
\pgfpathlineto{\pgfqpoint{2.532906in}{4.709910in}}%
\pgfpathlineto{\pgfqpoint{2.534829in}{4.701938in}}%
\pgfpathlineto{\pgfqpoint{2.536753in}{4.700161in}}%
\pgfpathlineto{\pgfqpoint{2.538676in}{4.700360in}}%
\pgfpathlineto{\pgfqpoint{2.540599in}{4.698852in}}%
\pgfpathlineto{\pgfqpoint{2.542523in}{4.694829in}}%
\pgfpathlineto{\pgfqpoint{2.544446in}{4.704702in}}%
\pgfpathlineto{\pgfqpoint{2.546370in}{4.697463in}}%
\pgfpathlineto{\pgfqpoint{2.548293in}{4.713722in}}%
\pgfpathlineto{\pgfqpoint{2.550217in}{4.718936in}}%
\pgfpathlineto{\pgfqpoint{2.552140in}{4.730278in}}%
\pgfpathlineto{\pgfqpoint{2.554063in}{4.727486in}}%
\pgfpathlineto{\pgfqpoint{2.557910in}{4.740832in}}%
\pgfpathlineto{\pgfqpoint{2.559834in}{4.741742in}}%
\pgfpathlineto{\pgfqpoint{2.561757in}{4.748743in}}%
\pgfpathlineto{\pgfqpoint{2.563680in}{4.739566in}}%
\pgfpathlineto{\pgfqpoint{2.565604in}{4.714483in}}%
\pgfpathlineto{\pgfqpoint{2.567527in}{4.723511in}}%
\pgfpathlineto{\pgfqpoint{2.571374in}{4.709954in}}%
\pgfpathlineto{\pgfqpoint{2.573297in}{4.693042in}}%
\pgfpathlineto{\pgfqpoint{2.575221in}{4.696366in}}%
\pgfpathlineto{\pgfqpoint{2.577144in}{4.692936in}}%
\pgfpathlineto{\pgfqpoint{2.579068in}{4.701717in}}%
\pgfpathlineto{\pgfqpoint{2.580991in}{4.688994in}}%
\pgfpathlineto{\pgfqpoint{2.582915in}{4.690031in}}%
\pgfpathlineto{\pgfqpoint{2.584838in}{4.693843in}}%
\pgfpathlineto{\pgfqpoint{2.586761in}{4.688036in}}%
\pgfpathlineto{\pgfqpoint{2.588685in}{4.701346in}}%
\pgfpathlineto{\pgfqpoint{2.590608in}{4.693213in}}%
\pgfpathlineto{\pgfqpoint{2.592532in}{4.692602in}}%
\pgfpathlineto{\pgfqpoint{2.594455in}{4.693331in}}%
\pgfpathlineto{\pgfqpoint{2.600225in}{4.674671in}}%
\pgfpathlineto{\pgfqpoint{2.602149in}{4.682848in}}%
\pgfpathlineto{\pgfqpoint{2.604072in}{4.680082in}}%
\pgfpathlineto{\pgfqpoint{2.605995in}{4.668963in}}%
\pgfpathlineto{\pgfqpoint{2.607919in}{4.667820in}}%
\pgfpathlineto{\pgfqpoint{2.609842in}{4.654864in}}%
\pgfpathlineto{\pgfqpoint{2.611766in}{4.655613in}}%
\pgfpathlineto{\pgfqpoint{2.613689in}{4.653776in}}%
\pgfpathlineto{\pgfqpoint{2.617536in}{4.656061in}}%
\pgfpathlineto{\pgfqpoint{2.627153in}{4.688450in}}%
\pgfpathlineto{\pgfqpoint{2.629076in}{4.686012in}}%
\pgfpathlineto{\pgfqpoint{2.631000in}{4.672174in}}%
\pgfpathlineto{\pgfqpoint{2.632923in}{4.683353in}}%
\pgfpathlineto{\pgfqpoint{2.634847in}{4.701691in}}%
\pgfpathlineto{\pgfqpoint{2.636770in}{4.697256in}}%
\pgfpathlineto{\pgfqpoint{2.638693in}{4.688335in}}%
\pgfpathlineto{\pgfqpoint{2.640617in}{4.697338in}}%
\pgfpathlineto{\pgfqpoint{2.642540in}{4.690244in}}%
\pgfpathlineto{\pgfqpoint{2.646387in}{4.669336in}}%
\pgfpathlineto{\pgfqpoint{2.648311in}{4.669648in}}%
\pgfpathlineto{\pgfqpoint{2.650234in}{4.654345in}}%
\pgfpathlineto{\pgfqpoint{2.652157in}{4.650207in}}%
\pgfpathlineto{\pgfqpoint{2.654081in}{4.662680in}}%
\pgfpathlineto{\pgfqpoint{2.659851in}{4.661303in}}%
\pgfpathlineto{\pgfqpoint{2.661774in}{4.646231in}}%
\pgfpathlineto{\pgfqpoint{2.663698in}{4.642621in}}%
\pgfpathlineto{\pgfqpoint{2.665621in}{4.641948in}}%
\pgfpathlineto{\pgfqpoint{2.667545in}{4.628948in}}%
\pgfpathlineto{\pgfqpoint{2.669468in}{4.635533in}}%
\pgfpathlineto{\pgfqpoint{2.671391in}{4.657864in}}%
\pgfpathlineto{\pgfqpoint{2.673315in}{4.656074in}}%
\pgfpathlineto{\pgfqpoint{2.675238in}{4.662691in}}%
\pgfpathlineto{\pgfqpoint{2.679085in}{4.649987in}}%
\pgfpathlineto{\pgfqpoint{2.682932in}{4.674783in}}%
\pgfpathlineto{\pgfqpoint{2.684855in}{4.687767in}}%
\pgfpathlineto{\pgfqpoint{2.688702in}{4.667046in}}%
\pgfpathlineto{\pgfqpoint{2.690626in}{4.662124in}}%
\pgfpathlineto{\pgfqpoint{2.696396in}{4.672186in}}%
\pgfpathlineto{\pgfqpoint{2.698319in}{4.673236in}}%
\pgfpathlineto{\pgfqpoint{2.700243in}{4.687607in}}%
\pgfpathlineto{\pgfqpoint{2.702166in}{4.687450in}}%
\pgfpathlineto{\pgfqpoint{2.704089in}{4.683413in}}%
\pgfpathlineto{\pgfqpoint{2.706013in}{4.710893in}}%
\pgfpathlineto{\pgfqpoint{2.707936in}{4.705977in}}%
\pgfpathlineto{\pgfqpoint{2.709860in}{4.705908in}}%
\pgfpathlineto{\pgfqpoint{2.711783in}{4.713306in}}%
\pgfpathlineto{\pgfqpoint{2.713706in}{4.708529in}}%
\pgfpathlineto{\pgfqpoint{2.715630in}{4.719544in}}%
\pgfpathlineto{\pgfqpoint{2.717553in}{4.721028in}}%
\pgfpathlineto{\pgfqpoint{2.721400in}{4.697393in}}%
\pgfpathlineto{\pgfqpoint{2.723324in}{4.681639in}}%
\pgfpathlineto{\pgfqpoint{2.725247in}{4.677024in}}%
\pgfpathlineto{\pgfqpoint{2.727170in}{4.689928in}}%
\pgfpathlineto{\pgfqpoint{2.729094in}{4.691273in}}%
\pgfpathlineto{\pgfqpoint{2.732941in}{4.672001in}}%
\pgfpathlineto{\pgfqpoint{2.734864in}{4.672505in}}%
\pgfpathlineto{\pgfqpoint{2.736787in}{4.674904in}}%
\pgfpathlineto{\pgfqpoint{2.738711in}{4.684384in}}%
\pgfpathlineto{\pgfqpoint{2.740634in}{4.684620in}}%
\pgfpathlineto{\pgfqpoint{2.744481in}{4.688261in}}%
\pgfpathlineto{\pgfqpoint{2.746404in}{4.692543in}}%
\pgfpathlineto{\pgfqpoint{2.748328in}{4.700224in}}%
\pgfpathlineto{\pgfqpoint{2.752175in}{4.692208in}}%
\pgfpathlineto{\pgfqpoint{2.754098in}{4.687581in}}%
\pgfpathlineto{\pgfqpoint{2.756022in}{4.697834in}}%
\pgfpathlineto{\pgfqpoint{2.757945in}{4.666196in}}%
\pgfpathlineto{\pgfqpoint{2.759868in}{4.657229in}}%
\pgfpathlineto{\pgfqpoint{2.763715in}{4.678539in}}%
\pgfpathlineto{\pgfqpoint{2.767562in}{4.694041in}}%
\pgfpathlineto{\pgfqpoint{2.767562in}{4.694041in}}%
\pgfusepath{stroke}%
\end{pgfscope}%
\begin{pgfscope}%
\pgfpathrectangle{\pgfqpoint{0.750000in}{3.180000in}}{\pgfqpoint{2.113636in}{2.100000in}}%
\pgfusepath{clip}%
\pgfsetroundcap%
\pgfsetroundjoin%
\pgfsetlinewidth{0.602250pt}%
\definecolor{currentstroke}{rgb}{0.968627,0.505882,0.749020}%
\pgfsetstrokecolor{currentstroke}%
\pgfsetdash{}{0pt}%
\pgfpathmoveto{\pgfqpoint{0.846074in}{4.331604in}}%
\pgfpathlineto{\pgfqpoint{0.849921in}{4.355857in}}%
\pgfpathlineto{\pgfqpoint{0.851845in}{4.366327in}}%
\pgfpathlineto{\pgfqpoint{0.853768in}{4.363560in}}%
\pgfpathlineto{\pgfqpoint{0.855691in}{4.364241in}}%
\pgfpathlineto{\pgfqpoint{0.857615in}{4.358181in}}%
\pgfpathlineto{\pgfqpoint{0.859538in}{4.360423in}}%
\pgfpathlineto{\pgfqpoint{0.861462in}{4.351196in}}%
\pgfpathlineto{\pgfqpoint{0.865308in}{4.366472in}}%
\pgfpathlineto{\pgfqpoint{0.867232in}{4.364796in}}%
\pgfpathlineto{\pgfqpoint{0.869155in}{4.383528in}}%
\pgfpathlineto{\pgfqpoint{0.871079in}{4.381292in}}%
\pgfpathlineto{\pgfqpoint{0.873002in}{4.391416in}}%
\pgfpathlineto{\pgfqpoint{0.874926in}{4.371148in}}%
\pgfpathlineto{\pgfqpoint{0.876849in}{4.369099in}}%
\pgfpathlineto{\pgfqpoint{0.878772in}{4.351232in}}%
\pgfpathlineto{\pgfqpoint{0.880696in}{4.362511in}}%
\pgfpathlineto{\pgfqpoint{0.884543in}{4.365889in}}%
\pgfpathlineto{\pgfqpoint{0.886466in}{4.366212in}}%
\pgfpathlineto{\pgfqpoint{0.888389in}{4.372186in}}%
\pgfpathlineto{\pgfqpoint{0.890313in}{4.365136in}}%
\pgfpathlineto{\pgfqpoint{0.892236in}{4.363267in}}%
\pgfpathlineto{\pgfqpoint{0.894160in}{4.350019in}}%
\pgfpathlineto{\pgfqpoint{0.896083in}{4.349302in}}%
\pgfpathlineto{\pgfqpoint{0.898006in}{4.347317in}}%
\pgfpathlineto{\pgfqpoint{0.899930in}{4.354241in}}%
\pgfpathlineto{\pgfqpoint{0.903777in}{4.379434in}}%
\pgfpathlineto{\pgfqpoint{0.905700in}{4.375838in}}%
\pgfpathlineto{\pgfqpoint{0.907624in}{4.357225in}}%
\pgfpathlineto{\pgfqpoint{0.909547in}{4.366873in}}%
\pgfpathlineto{\pgfqpoint{0.911470in}{4.358074in}}%
\pgfpathlineto{\pgfqpoint{0.913394in}{4.356621in}}%
\pgfpathlineto{\pgfqpoint{0.915317in}{4.353127in}}%
\pgfpathlineto{\pgfqpoint{0.919164in}{4.331105in}}%
\pgfpathlineto{\pgfqpoint{0.921087in}{4.338556in}}%
\pgfpathlineto{\pgfqpoint{0.923011in}{4.340165in}}%
\pgfpathlineto{\pgfqpoint{0.928781in}{4.360808in}}%
\pgfpathlineto{\pgfqpoint{0.930704in}{4.359396in}}%
\pgfpathlineto{\pgfqpoint{0.932628in}{4.362023in}}%
\pgfpathlineto{\pgfqpoint{0.936475in}{4.397542in}}%
\pgfpathlineto{\pgfqpoint{0.938398in}{4.395287in}}%
\pgfpathlineto{\pgfqpoint{0.942245in}{4.410732in}}%
\pgfpathlineto{\pgfqpoint{0.944168in}{4.421735in}}%
\pgfpathlineto{\pgfqpoint{0.946092in}{4.413438in}}%
\pgfpathlineto{\pgfqpoint{0.948015in}{4.421928in}}%
\pgfpathlineto{\pgfqpoint{0.949939in}{4.407659in}}%
\pgfpathlineto{\pgfqpoint{0.951862in}{4.401756in}}%
\pgfpathlineto{\pgfqpoint{0.953785in}{4.387071in}}%
\pgfpathlineto{\pgfqpoint{0.955709in}{4.386822in}}%
\pgfpathlineto{\pgfqpoint{0.957632in}{4.400873in}}%
\pgfpathlineto{\pgfqpoint{0.959556in}{4.391184in}}%
\pgfpathlineto{\pgfqpoint{0.961479in}{4.389708in}}%
\pgfpathlineto{\pgfqpoint{0.963402in}{4.402627in}}%
\pgfpathlineto{\pgfqpoint{0.965326in}{4.400946in}}%
\pgfpathlineto{\pgfqpoint{0.967249in}{4.414784in}}%
\pgfpathlineto{\pgfqpoint{0.969173in}{4.402993in}}%
\pgfpathlineto{\pgfqpoint{0.971096in}{4.403819in}}%
\pgfpathlineto{\pgfqpoint{0.973020in}{4.410650in}}%
\pgfpathlineto{\pgfqpoint{0.978790in}{4.385915in}}%
\pgfpathlineto{\pgfqpoint{0.980713in}{4.383936in}}%
\pgfpathlineto{\pgfqpoint{0.982637in}{4.384734in}}%
\pgfpathlineto{\pgfqpoint{0.986483in}{4.400864in}}%
\pgfpathlineto{\pgfqpoint{0.988407in}{4.408254in}}%
\pgfpathlineto{\pgfqpoint{0.990330in}{4.420768in}}%
\pgfpathlineto{\pgfqpoint{0.992254in}{4.422985in}}%
\pgfpathlineto{\pgfqpoint{0.994177in}{4.406727in}}%
\pgfpathlineto{\pgfqpoint{0.996100in}{4.420378in}}%
\pgfpathlineto{\pgfqpoint{0.999947in}{4.405820in}}%
\pgfpathlineto{\pgfqpoint{1.001871in}{4.421465in}}%
\pgfpathlineto{\pgfqpoint{1.003794in}{4.413784in}}%
\pgfpathlineto{\pgfqpoint{1.005717in}{4.414218in}}%
\pgfpathlineto{\pgfqpoint{1.007641in}{4.420552in}}%
\pgfpathlineto{\pgfqpoint{1.009564in}{4.431025in}}%
\pgfpathlineto{\pgfqpoint{1.011488in}{4.460804in}}%
\pgfpathlineto{\pgfqpoint{1.013411in}{4.460855in}}%
\pgfpathlineto{\pgfqpoint{1.017258in}{4.471621in}}%
\pgfpathlineto{\pgfqpoint{1.019181in}{4.454821in}}%
\pgfpathlineto{\pgfqpoint{1.021105in}{4.467969in}}%
\pgfpathlineto{\pgfqpoint{1.023028in}{4.462666in}}%
\pgfpathlineto{\pgfqpoint{1.024952in}{4.445296in}}%
\pgfpathlineto{\pgfqpoint{1.026875in}{4.460233in}}%
\pgfpathlineto{\pgfqpoint{1.028798in}{4.459495in}}%
\pgfpathlineto{\pgfqpoint{1.030722in}{4.462020in}}%
\pgfpathlineto{\pgfqpoint{1.032645in}{4.451031in}}%
\pgfpathlineto{\pgfqpoint{1.038415in}{4.499184in}}%
\pgfpathlineto{\pgfqpoint{1.040339in}{4.512348in}}%
\pgfpathlineto{\pgfqpoint{1.042262in}{4.507743in}}%
\pgfpathlineto{\pgfqpoint{1.044186in}{4.507437in}}%
\pgfpathlineto{\pgfqpoint{1.046109in}{4.515555in}}%
\pgfpathlineto{\pgfqpoint{1.048033in}{4.529327in}}%
\pgfpathlineto{\pgfqpoint{1.049956in}{4.525114in}}%
\pgfpathlineto{\pgfqpoint{1.051879in}{4.517445in}}%
\pgfpathlineto{\pgfqpoint{1.053803in}{4.518938in}}%
\pgfpathlineto{\pgfqpoint{1.055726in}{4.524338in}}%
\pgfpathlineto{\pgfqpoint{1.057650in}{4.521534in}}%
\pgfpathlineto{\pgfqpoint{1.063420in}{4.545745in}}%
\pgfpathlineto{\pgfqpoint{1.065343in}{4.535443in}}%
\pgfpathlineto{\pgfqpoint{1.069190in}{4.530485in}}%
\pgfpathlineto{\pgfqpoint{1.071113in}{4.533831in}}%
\pgfpathlineto{\pgfqpoint{1.073037in}{4.524682in}}%
\pgfpathlineto{\pgfqpoint{1.074960in}{4.532508in}}%
\pgfpathlineto{\pgfqpoint{1.078807in}{4.508128in}}%
\pgfpathlineto{\pgfqpoint{1.082654in}{4.496595in}}%
\pgfpathlineto{\pgfqpoint{1.084577in}{4.510202in}}%
\pgfpathlineto{\pgfqpoint{1.088424in}{4.500878in}}%
\pgfpathlineto{\pgfqpoint{1.090348in}{4.507542in}}%
\pgfpathlineto{\pgfqpoint{1.094194in}{4.499254in}}%
\pgfpathlineto{\pgfqpoint{1.096118in}{4.500358in}}%
\pgfpathlineto{\pgfqpoint{1.098041in}{4.485573in}}%
\pgfpathlineto{\pgfqpoint{1.099965in}{4.492277in}}%
\pgfpathlineto{\pgfqpoint{1.101888in}{4.485543in}}%
\pgfpathlineto{\pgfqpoint{1.103811in}{4.487543in}}%
\pgfpathlineto{\pgfqpoint{1.105735in}{4.495424in}}%
\pgfpathlineto{\pgfqpoint{1.107658in}{4.485742in}}%
\pgfpathlineto{\pgfqpoint{1.111505in}{4.484127in}}%
\pgfpathlineto{\pgfqpoint{1.115352in}{4.468493in}}%
\pgfpathlineto{\pgfqpoint{1.117275in}{4.470693in}}%
\pgfpathlineto{\pgfqpoint{1.119199in}{4.489199in}}%
\pgfpathlineto{\pgfqpoint{1.121122in}{4.491178in}}%
\pgfpathlineto{\pgfqpoint{1.123046in}{4.513297in}}%
\pgfpathlineto{\pgfqpoint{1.124969in}{4.496285in}}%
\pgfpathlineto{\pgfqpoint{1.126892in}{4.489184in}}%
\pgfpathlineto{\pgfqpoint{1.130739in}{4.501932in}}%
\pgfpathlineto{\pgfqpoint{1.132663in}{4.494746in}}%
\pgfpathlineto{\pgfqpoint{1.136509in}{4.510095in}}%
\pgfpathlineto{\pgfqpoint{1.138433in}{4.498755in}}%
\pgfpathlineto{\pgfqpoint{1.140356in}{4.518648in}}%
\pgfpathlineto{\pgfqpoint{1.142280in}{4.525645in}}%
\pgfpathlineto{\pgfqpoint{1.144203in}{4.512127in}}%
\pgfpathlineto{\pgfqpoint{1.146126in}{4.521137in}}%
\pgfpathlineto{\pgfqpoint{1.151897in}{4.510873in}}%
\pgfpathlineto{\pgfqpoint{1.153820in}{4.494137in}}%
\pgfpathlineto{\pgfqpoint{1.159590in}{4.503483in}}%
\pgfpathlineto{\pgfqpoint{1.161514in}{4.497788in}}%
\pgfpathlineto{\pgfqpoint{1.163437in}{4.502416in}}%
\pgfpathlineto{\pgfqpoint{1.165361in}{4.489771in}}%
\pgfpathlineto{\pgfqpoint{1.169207in}{4.504068in}}%
\pgfpathlineto{\pgfqpoint{1.171131in}{4.490549in}}%
\pgfpathlineto{\pgfqpoint{1.173054in}{4.505387in}}%
\pgfpathlineto{\pgfqpoint{1.174978in}{4.506978in}}%
\pgfpathlineto{\pgfqpoint{1.176901in}{4.498608in}}%
\pgfpathlineto{\pgfqpoint{1.180748in}{4.527712in}}%
\pgfpathlineto{\pgfqpoint{1.182671in}{4.527847in}}%
\pgfpathlineto{\pgfqpoint{1.184595in}{4.526864in}}%
\pgfpathlineto{\pgfqpoint{1.186518in}{4.524080in}}%
\pgfpathlineto{\pgfqpoint{1.192288in}{4.559367in}}%
\pgfpathlineto{\pgfqpoint{1.196135in}{4.540677in}}%
\pgfpathlineto{\pgfqpoint{1.198059in}{4.541907in}}%
\pgfpathlineto{\pgfqpoint{1.199982in}{4.540645in}}%
\pgfpathlineto{\pgfqpoint{1.201905in}{4.528483in}}%
\pgfpathlineto{\pgfqpoint{1.205752in}{4.552381in}}%
\pgfpathlineto{\pgfqpoint{1.207676in}{4.543924in}}%
\pgfpathlineto{\pgfqpoint{1.213446in}{4.554993in}}%
\pgfpathlineto{\pgfqpoint{1.215369in}{4.566620in}}%
\pgfpathlineto{\pgfqpoint{1.217293in}{4.570748in}}%
\pgfpathlineto{\pgfqpoint{1.219216in}{4.566002in}}%
\pgfpathlineto{\pgfqpoint{1.226910in}{4.612843in}}%
\pgfpathlineto{\pgfqpoint{1.228833in}{4.621569in}}%
\pgfpathlineto{\pgfqpoint{1.230757in}{4.636267in}}%
\pgfpathlineto{\pgfqpoint{1.232680in}{4.629205in}}%
\pgfpathlineto{\pgfqpoint{1.234603in}{4.627734in}}%
\pgfpathlineto{\pgfqpoint{1.236527in}{4.645207in}}%
\pgfpathlineto{\pgfqpoint{1.238450in}{4.649281in}}%
\pgfpathlineto{\pgfqpoint{1.242297in}{4.634475in}}%
\pgfpathlineto{\pgfqpoint{1.244220in}{4.638381in}}%
\pgfpathlineto{\pgfqpoint{1.246144in}{4.634340in}}%
\pgfpathlineto{\pgfqpoint{1.248067in}{4.639983in}}%
\pgfpathlineto{\pgfqpoint{1.249991in}{4.650431in}}%
\pgfpathlineto{\pgfqpoint{1.253838in}{4.680419in}}%
\pgfpathlineto{\pgfqpoint{1.255761in}{4.689598in}}%
\pgfpathlineto{\pgfqpoint{1.257684in}{4.689317in}}%
\pgfpathlineto{\pgfqpoint{1.263455in}{4.708672in}}%
\pgfpathlineto{\pgfqpoint{1.265378in}{4.704431in}}%
\pgfpathlineto{\pgfqpoint{1.267301in}{4.707769in}}%
\pgfpathlineto{\pgfqpoint{1.269225in}{4.704759in}}%
\pgfpathlineto{\pgfqpoint{1.271148in}{4.689262in}}%
\pgfpathlineto{\pgfqpoint{1.273072in}{4.692230in}}%
\pgfpathlineto{\pgfqpoint{1.274995in}{4.668116in}}%
\pgfpathlineto{\pgfqpoint{1.276918in}{4.670159in}}%
\pgfpathlineto{\pgfqpoint{1.280765in}{4.690285in}}%
\pgfpathlineto{\pgfqpoint{1.282689in}{4.693635in}}%
\pgfpathlineto{\pgfqpoint{1.284612in}{4.703233in}}%
\pgfpathlineto{\pgfqpoint{1.286536in}{4.694548in}}%
\pgfpathlineto{\pgfqpoint{1.288459in}{4.697945in}}%
\pgfpathlineto{\pgfqpoint{1.290382in}{4.705798in}}%
\pgfpathlineto{\pgfqpoint{1.292306in}{4.704874in}}%
\pgfpathlineto{\pgfqpoint{1.296153in}{4.689196in}}%
\pgfpathlineto{\pgfqpoint{1.299999in}{4.716971in}}%
\pgfpathlineto{\pgfqpoint{1.305770in}{4.701326in}}%
\pgfpathlineto{\pgfqpoint{1.307693in}{4.714628in}}%
\pgfpathlineto{\pgfqpoint{1.309616in}{4.699503in}}%
\pgfpathlineto{\pgfqpoint{1.311540in}{4.703945in}}%
\pgfpathlineto{\pgfqpoint{1.313463in}{4.693853in}}%
\pgfpathlineto{\pgfqpoint{1.315387in}{4.701730in}}%
\pgfpathlineto{\pgfqpoint{1.317310in}{4.693439in}}%
\pgfpathlineto{\pgfqpoint{1.321157in}{4.703489in}}%
\pgfpathlineto{\pgfqpoint{1.323080in}{4.703420in}}%
\pgfpathlineto{\pgfqpoint{1.325004in}{4.693766in}}%
\pgfpathlineto{\pgfqpoint{1.326927in}{4.699425in}}%
\pgfpathlineto{\pgfqpoint{1.328851in}{4.691623in}}%
\pgfpathlineto{\pgfqpoint{1.332697in}{4.693707in}}%
\pgfpathlineto{\pgfqpoint{1.334621in}{4.693364in}}%
\pgfpathlineto{\pgfqpoint{1.336544in}{4.687777in}}%
\pgfpathlineto{\pgfqpoint{1.338468in}{4.687698in}}%
\pgfpathlineto{\pgfqpoint{1.340391in}{4.691512in}}%
\pgfpathlineto{\pgfqpoint{1.342314in}{4.706189in}}%
\pgfpathlineto{\pgfqpoint{1.344238in}{4.711267in}}%
\pgfpathlineto{\pgfqpoint{1.346161in}{4.699007in}}%
\pgfpathlineto{\pgfqpoint{1.348085in}{4.699920in}}%
\pgfpathlineto{\pgfqpoint{1.350008in}{4.695834in}}%
\pgfpathlineto{\pgfqpoint{1.351931in}{4.682689in}}%
\pgfpathlineto{\pgfqpoint{1.355778in}{4.696249in}}%
\pgfpathlineto{\pgfqpoint{1.357702in}{4.697387in}}%
\pgfpathlineto{\pgfqpoint{1.359625in}{4.699980in}}%
\pgfpathlineto{\pgfqpoint{1.361549in}{4.698450in}}%
\pgfpathlineto{\pgfqpoint{1.363472in}{4.689158in}}%
\pgfpathlineto{\pgfqpoint{1.365395in}{4.705839in}}%
\pgfpathlineto{\pgfqpoint{1.367319in}{4.696236in}}%
\pgfpathlineto{\pgfqpoint{1.373089in}{4.725282in}}%
\pgfpathlineto{\pgfqpoint{1.375012in}{4.719383in}}%
\pgfpathlineto{\pgfqpoint{1.376936in}{4.739032in}}%
\pgfpathlineto{\pgfqpoint{1.378859in}{4.722598in}}%
\pgfpathlineto{\pgfqpoint{1.380783in}{4.724191in}}%
\pgfpathlineto{\pgfqpoint{1.382706in}{4.732698in}}%
\pgfpathlineto{\pgfqpoint{1.386553in}{4.703512in}}%
\pgfpathlineto{\pgfqpoint{1.388476in}{4.705253in}}%
\pgfpathlineto{\pgfqpoint{1.390400in}{4.694382in}}%
\pgfpathlineto{\pgfqpoint{1.392323in}{4.694689in}}%
\pgfpathlineto{\pgfqpoint{1.394247in}{4.690380in}}%
\pgfpathlineto{\pgfqpoint{1.396170in}{4.696431in}}%
\pgfpathlineto{\pgfqpoint{1.398093in}{4.711900in}}%
\pgfpathlineto{\pgfqpoint{1.400017in}{4.709240in}}%
\pgfpathlineto{\pgfqpoint{1.401940in}{4.719263in}}%
\pgfpathlineto{\pgfqpoint{1.405787in}{4.715748in}}%
\pgfpathlineto{\pgfqpoint{1.407710in}{4.709802in}}%
\pgfpathlineto{\pgfqpoint{1.409634in}{4.712121in}}%
\pgfpathlineto{\pgfqpoint{1.411557in}{4.709202in}}%
\pgfpathlineto{\pgfqpoint{1.413481in}{4.711918in}}%
\pgfpathlineto{\pgfqpoint{1.415404in}{4.702440in}}%
\pgfpathlineto{\pgfqpoint{1.417327in}{4.707213in}}%
\pgfpathlineto{\pgfqpoint{1.425021in}{4.668717in}}%
\pgfpathlineto{\pgfqpoint{1.428868in}{4.680755in}}%
\pgfpathlineto{\pgfqpoint{1.430791in}{4.663016in}}%
\pgfpathlineto{\pgfqpoint{1.432715in}{4.673839in}}%
\pgfpathlineto{\pgfqpoint{1.434638in}{4.663376in}}%
\pgfpathlineto{\pgfqpoint{1.436562in}{4.662611in}}%
\pgfpathlineto{\pgfqpoint{1.438485in}{4.648694in}}%
\pgfpathlineto{\pgfqpoint{1.440408in}{4.654810in}}%
\pgfpathlineto{\pgfqpoint{1.442332in}{4.646452in}}%
\pgfpathlineto{\pgfqpoint{1.448102in}{4.671844in}}%
\pgfpathlineto{\pgfqpoint{1.450025in}{4.669885in}}%
\pgfpathlineto{\pgfqpoint{1.451949in}{4.671799in}}%
\pgfpathlineto{\pgfqpoint{1.453872in}{4.666689in}}%
\pgfpathlineto{\pgfqpoint{1.455796in}{4.671688in}}%
\pgfpathlineto{\pgfqpoint{1.457719in}{4.671649in}}%
\pgfpathlineto{\pgfqpoint{1.461566in}{4.706273in}}%
\pgfpathlineto{\pgfqpoint{1.463489in}{4.706119in}}%
\pgfpathlineto{\pgfqpoint{1.465413in}{4.695491in}}%
\pgfpathlineto{\pgfqpoint{1.471183in}{4.710524in}}%
\pgfpathlineto{\pgfqpoint{1.473106in}{4.703772in}}%
\pgfpathlineto{\pgfqpoint{1.475030in}{4.702736in}}%
\pgfpathlineto{\pgfqpoint{1.476953in}{4.727049in}}%
\pgfpathlineto{\pgfqpoint{1.480800in}{4.721653in}}%
\pgfpathlineto{\pgfqpoint{1.482723in}{4.724204in}}%
\pgfpathlineto{\pgfqpoint{1.484647in}{4.714753in}}%
\pgfpathlineto{\pgfqpoint{1.486570in}{4.710881in}}%
\pgfpathlineto{\pgfqpoint{1.488494in}{4.715168in}}%
\pgfpathlineto{\pgfqpoint{1.492340in}{4.714209in}}%
\pgfpathlineto{\pgfqpoint{1.494264in}{4.707249in}}%
\pgfpathlineto{\pgfqpoint{1.498111in}{4.727180in}}%
\pgfpathlineto{\pgfqpoint{1.500034in}{4.718410in}}%
\pgfpathlineto{\pgfqpoint{1.501958in}{4.732789in}}%
\pgfpathlineto{\pgfqpoint{1.503881in}{4.736034in}}%
\pgfpathlineto{\pgfqpoint{1.505804in}{4.734084in}}%
\pgfpathlineto{\pgfqpoint{1.507728in}{4.744367in}}%
\pgfpathlineto{\pgfqpoint{1.511575in}{4.739410in}}%
\pgfpathlineto{\pgfqpoint{1.515421in}{4.753285in}}%
\pgfpathlineto{\pgfqpoint{1.517345in}{4.745245in}}%
\pgfpathlineto{\pgfqpoint{1.519268in}{4.748140in}}%
\pgfpathlineto{\pgfqpoint{1.523115in}{4.734974in}}%
\pgfpathlineto{\pgfqpoint{1.526962in}{4.730577in}}%
\pgfpathlineto{\pgfqpoint{1.532732in}{4.776090in}}%
\pgfpathlineto{\pgfqpoint{1.534656in}{4.770536in}}%
\pgfpathlineto{\pgfqpoint{1.536579in}{4.784304in}}%
\pgfpathlineto{\pgfqpoint{1.538502in}{4.783468in}}%
\pgfpathlineto{\pgfqpoint{1.542349in}{4.803071in}}%
\pgfpathlineto{\pgfqpoint{1.546196in}{4.788916in}}%
\pgfpathlineto{\pgfqpoint{1.548119in}{4.790794in}}%
\pgfpathlineto{\pgfqpoint{1.551966in}{4.777513in}}%
\pgfpathlineto{\pgfqpoint{1.553890in}{4.767239in}}%
\pgfpathlineto{\pgfqpoint{1.555813in}{4.788562in}}%
\pgfpathlineto{\pgfqpoint{1.557736in}{4.784453in}}%
\pgfpathlineto{\pgfqpoint{1.559660in}{4.783587in}}%
\pgfpathlineto{\pgfqpoint{1.561583in}{4.780542in}}%
\pgfpathlineto{\pgfqpoint{1.565430in}{4.787314in}}%
\pgfpathlineto{\pgfqpoint{1.567354in}{4.786610in}}%
\pgfpathlineto{\pgfqpoint{1.571200in}{4.772526in}}%
\pgfpathlineto{\pgfqpoint{1.575047in}{4.771077in}}%
\pgfpathlineto{\pgfqpoint{1.576971in}{4.783401in}}%
\pgfpathlineto{\pgfqpoint{1.578894in}{4.784003in}}%
\pgfpathlineto{\pgfqpoint{1.582741in}{4.803591in}}%
\pgfpathlineto{\pgfqpoint{1.584664in}{4.795243in}}%
\pgfpathlineto{\pgfqpoint{1.586588in}{4.756859in}}%
\pgfpathlineto{\pgfqpoint{1.588511in}{4.764031in}}%
\pgfpathlineto{\pgfqpoint{1.590434in}{4.762913in}}%
\pgfpathlineto{\pgfqpoint{1.592358in}{4.781625in}}%
\pgfpathlineto{\pgfqpoint{1.594281in}{4.786404in}}%
\pgfpathlineto{\pgfqpoint{1.596205in}{4.785788in}}%
\pgfpathlineto{\pgfqpoint{1.598128in}{4.791222in}}%
\pgfpathlineto{\pgfqpoint{1.600051in}{4.786494in}}%
\pgfpathlineto{\pgfqpoint{1.605822in}{4.820584in}}%
\pgfpathlineto{\pgfqpoint{1.609669in}{4.793122in}}%
\pgfpathlineto{\pgfqpoint{1.611592in}{4.806492in}}%
\pgfpathlineto{\pgfqpoint{1.613515in}{4.793750in}}%
\pgfpathlineto{\pgfqpoint{1.615439in}{4.794174in}}%
\pgfpathlineto{\pgfqpoint{1.617362in}{4.786279in}}%
\pgfpathlineto{\pgfqpoint{1.619286in}{4.796736in}}%
\pgfpathlineto{\pgfqpoint{1.621209in}{4.786179in}}%
\pgfpathlineto{\pgfqpoint{1.623132in}{4.796904in}}%
\pgfpathlineto{\pgfqpoint{1.625056in}{4.791138in}}%
\pgfpathlineto{\pgfqpoint{1.628903in}{4.802217in}}%
\pgfpathlineto{\pgfqpoint{1.630826in}{4.791740in}}%
\pgfpathlineto{\pgfqpoint{1.632749in}{4.797105in}}%
\pgfpathlineto{\pgfqpoint{1.634673in}{4.788727in}}%
\pgfpathlineto{\pgfqpoint{1.636596in}{4.791610in}}%
\pgfpathlineto{\pgfqpoint{1.638520in}{4.789969in}}%
\pgfpathlineto{\pgfqpoint{1.640443in}{4.791912in}}%
\pgfpathlineto{\pgfqpoint{1.644290in}{4.786022in}}%
\pgfpathlineto{\pgfqpoint{1.646213in}{4.793906in}}%
\pgfpathlineto{\pgfqpoint{1.648137in}{4.794214in}}%
\pgfpathlineto{\pgfqpoint{1.650060in}{4.796093in}}%
\pgfpathlineto{\pgfqpoint{1.655830in}{4.764060in}}%
\pgfpathlineto{\pgfqpoint{1.657754in}{4.764800in}}%
\pgfpathlineto{\pgfqpoint{1.659677in}{4.776868in}}%
\pgfpathlineto{\pgfqpoint{1.661601in}{4.768853in}}%
\pgfpathlineto{\pgfqpoint{1.665447in}{4.785296in}}%
\pgfpathlineto{\pgfqpoint{1.667371in}{4.775874in}}%
\pgfpathlineto{\pgfqpoint{1.671218in}{4.791678in}}%
\pgfpathlineto{\pgfqpoint{1.673141in}{4.791554in}}%
\pgfpathlineto{\pgfqpoint{1.675065in}{4.800431in}}%
\pgfpathlineto{\pgfqpoint{1.676988in}{4.800938in}}%
\pgfpathlineto{\pgfqpoint{1.678911in}{4.778736in}}%
\pgfpathlineto{\pgfqpoint{1.680835in}{4.781062in}}%
\pgfpathlineto{\pgfqpoint{1.682758in}{4.777771in}}%
\pgfpathlineto{\pgfqpoint{1.684682in}{4.771000in}}%
\pgfpathlineto{\pgfqpoint{1.688528in}{4.781063in}}%
\pgfpathlineto{\pgfqpoint{1.692375in}{4.775083in}}%
\pgfpathlineto{\pgfqpoint{1.694299in}{4.771427in}}%
\pgfpathlineto{\pgfqpoint{1.701992in}{4.795226in}}%
\pgfpathlineto{\pgfqpoint{1.703916in}{4.796157in}}%
\pgfpathlineto{\pgfqpoint{1.711609in}{4.828916in}}%
\pgfpathlineto{\pgfqpoint{1.713533in}{4.832112in}}%
\pgfpathlineto{\pgfqpoint{1.715456in}{4.837757in}}%
\pgfpathlineto{\pgfqpoint{1.717380in}{4.835110in}}%
\pgfpathlineto{\pgfqpoint{1.719303in}{4.843412in}}%
\pgfpathlineto{\pgfqpoint{1.721226in}{4.836798in}}%
\pgfpathlineto{\pgfqpoint{1.723150in}{4.839493in}}%
\pgfpathlineto{\pgfqpoint{1.726997in}{4.865177in}}%
\pgfpathlineto{\pgfqpoint{1.728920in}{4.871391in}}%
\pgfpathlineto{\pgfqpoint{1.730843in}{4.868317in}}%
\pgfpathlineto{\pgfqpoint{1.734690in}{4.869100in}}%
\pgfpathlineto{\pgfqpoint{1.736614in}{4.874997in}}%
\pgfpathlineto{\pgfqpoint{1.740461in}{4.893975in}}%
\pgfpathlineto{\pgfqpoint{1.742384in}{4.874638in}}%
\pgfpathlineto{\pgfqpoint{1.744307in}{4.891116in}}%
\pgfpathlineto{\pgfqpoint{1.746231in}{4.890718in}}%
\pgfpathlineto{\pgfqpoint{1.748154in}{4.885984in}}%
\pgfpathlineto{\pgfqpoint{1.750078in}{4.865939in}}%
\pgfpathlineto{\pgfqpoint{1.752001in}{4.859472in}}%
\pgfpathlineto{\pgfqpoint{1.753924in}{4.862797in}}%
\pgfpathlineto{\pgfqpoint{1.755848in}{4.854710in}}%
\pgfpathlineto{\pgfqpoint{1.757771in}{4.852046in}}%
\pgfpathlineto{\pgfqpoint{1.759695in}{4.858560in}}%
\pgfpathlineto{\pgfqpoint{1.761618in}{4.858836in}}%
\pgfpathlineto{\pgfqpoint{1.763541in}{4.846992in}}%
\pgfpathlineto{\pgfqpoint{1.765465in}{4.857497in}}%
\pgfpathlineto{\pgfqpoint{1.767388in}{4.852673in}}%
\pgfpathlineto{\pgfqpoint{1.769312in}{4.854801in}}%
\pgfpathlineto{\pgfqpoint{1.773158in}{4.849372in}}%
\pgfpathlineto{\pgfqpoint{1.775082in}{4.832374in}}%
\pgfpathlineto{\pgfqpoint{1.778929in}{4.823974in}}%
\pgfpathlineto{\pgfqpoint{1.780852in}{4.826968in}}%
\pgfpathlineto{\pgfqpoint{1.782776in}{4.816870in}}%
\pgfpathlineto{\pgfqpoint{1.784699in}{4.825929in}}%
\pgfpathlineto{\pgfqpoint{1.786622in}{4.817788in}}%
\pgfpathlineto{\pgfqpoint{1.788546in}{4.820246in}}%
\pgfpathlineto{\pgfqpoint{1.790469in}{4.825944in}}%
\pgfpathlineto{\pgfqpoint{1.794316in}{4.848908in}}%
\pgfpathlineto{\pgfqpoint{1.796239in}{4.851612in}}%
\pgfpathlineto{\pgfqpoint{1.800086in}{4.868730in}}%
\pgfpathlineto{\pgfqpoint{1.802010in}{4.865540in}}%
\pgfpathlineto{\pgfqpoint{1.803933in}{4.856727in}}%
\pgfpathlineto{\pgfqpoint{1.805856in}{4.858063in}}%
\pgfpathlineto{\pgfqpoint{1.809703in}{4.863553in}}%
\pgfpathlineto{\pgfqpoint{1.811627in}{4.860010in}}%
\pgfpathlineto{\pgfqpoint{1.815474in}{4.869904in}}%
\pgfpathlineto{\pgfqpoint{1.817397in}{4.871977in}}%
\pgfpathlineto{\pgfqpoint{1.819320in}{4.880948in}}%
\pgfpathlineto{\pgfqpoint{1.821244in}{4.870751in}}%
\pgfpathlineto{\pgfqpoint{1.823167in}{4.850088in}}%
\pgfpathlineto{\pgfqpoint{1.825091in}{4.851739in}}%
\pgfpathlineto{\pgfqpoint{1.827014in}{4.840169in}}%
\pgfpathlineto{\pgfqpoint{1.828937in}{4.845608in}}%
\pgfpathlineto{\pgfqpoint{1.830861in}{4.829772in}}%
\pgfpathlineto{\pgfqpoint{1.832784in}{4.825724in}}%
\pgfpathlineto{\pgfqpoint{1.834708in}{4.825158in}}%
\pgfpathlineto{\pgfqpoint{1.840478in}{4.851076in}}%
\pgfpathlineto{\pgfqpoint{1.842401in}{4.850372in}}%
\pgfpathlineto{\pgfqpoint{1.844325in}{4.844649in}}%
\pgfpathlineto{\pgfqpoint{1.846248in}{4.843744in}}%
\pgfpathlineto{\pgfqpoint{1.848172in}{4.837617in}}%
\pgfpathlineto{\pgfqpoint{1.850095in}{4.827430in}}%
\pgfpathlineto{\pgfqpoint{1.853942in}{4.834617in}}%
\pgfpathlineto{\pgfqpoint{1.855865in}{4.825067in}}%
\pgfpathlineto{\pgfqpoint{1.857789in}{4.808582in}}%
\pgfpathlineto{\pgfqpoint{1.859712in}{4.812971in}}%
\pgfpathlineto{\pgfqpoint{1.863559in}{4.843153in}}%
\pgfpathlineto{\pgfqpoint{1.865482in}{4.842818in}}%
\pgfpathlineto{\pgfqpoint{1.867406in}{4.856353in}}%
\pgfpathlineto{\pgfqpoint{1.869329in}{4.859269in}}%
\pgfpathlineto{\pgfqpoint{1.871252in}{4.880324in}}%
\pgfpathlineto{\pgfqpoint{1.873176in}{4.881797in}}%
\pgfpathlineto{\pgfqpoint{1.875099in}{4.875801in}}%
\pgfpathlineto{\pgfqpoint{1.877023in}{4.890637in}}%
\pgfpathlineto{\pgfqpoint{1.878946in}{4.886054in}}%
\pgfpathlineto{\pgfqpoint{1.880870in}{4.905763in}}%
\pgfpathlineto{\pgfqpoint{1.882793in}{4.901448in}}%
\pgfpathlineto{\pgfqpoint{1.884716in}{4.905753in}}%
\pgfpathlineto{\pgfqpoint{1.886640in}{4.914950in}}%
\pgfpathlineto{\pgfqpoint{1.890487in}{4.905502in}}%
\pgfpathlineto{\pgfqpoint{1.892410in}{4.891320in}}%
\pgfpathlineto{\pgfqpoint{1.894333in}{4.886961in}}%
\pgfpathlineto{\pgfqpoint{1.898180in}{4.862487in}}%
\pgfpathlineto{\pgfqpoint{1.900104in}{4.866773in}}%
\pgfpathlineto{\pgfqpoint{1.902027in}{4.855367in}}%
\pgfpathlineto{\pgfqpoint{1.905874in}{4.868086in}}%
\pgfpathlineto{\pgfqpoint{1.907797in}{4.860691in}}%
\pgfpathlineto{\pgfqpoint{1.909721in}{4.873368in}}%
\pgfpathlineto{\pgfqpoint{1.911644in}{4.860077in}}%
\pgfpathlineto{\pgfqpoint{1.913567in}{4.854636in}}%
\pgfpathlineto{\pgfqpoint{1.915491in}{4.833226in}}%
\pgfpathlineto{\pgfqpoint{1.917414in}{4.833116in}}%
\pgfpathlineto{\pgfqpoint{1.919338in}{4.827241in}}%
\pgfpathlineto{\pgfqpoint{1.921261in}{4.834786in}}%
\pgfpathlineto{\pgfqpoint{1.923185in}{4.833149in}}%
\pgfpathlineto{\pgfqpoint{1.925108in}{4.837434in}}%
\pgfpathlineto{\pgfqpoint{1.927031in}{4.831224in}}%
\pgfpathlineto{\pgfqpoint{1.928955in}{4.840472in}}%
\pgfpathlineto{\pgfqpoint{1.932802in}{4.839849in}}%
\pgfpathlineto{\pgfqpoint{1.934725in}{4.844990in}}%
\pgfpathlineto{\pgfqpoint{1.936648in}{4.843185in}}%
\pgfpathlineto{\pgfqpoint{1.938572in}{4.849042in}}%
\pgfpathlineto{\pgfqpoint{1.942419in}{4.845684in}}%
\pgfpathlineto{\pgfqpoint{1.944342in}{4.848464in}}%
\pgfpathlineto{\pgfqpoint{1.948189in}{4.872521in}}%
\pgfpathlineto{\pgfqpoint{1.950112in}{4.867710in}}%
\pgfpathlineto{\pgfqpoint{1.952036in}{4.877365in}}%
\pgfpathlineto{\pgfqpoint{1.953959in}{4.861170in}}%
\pgfpathlineto{\pgfqpoint{1.955883in}{4.859383in}}%
\pgfpathlineto{\pgfqpoint{1.957806in}{4.864589in}}%
\pgfpathlineto{\pgfqpoint{1.959729in}{4.865218in}}%
\pgfpathlineto{\pgfqpoint{1.961653in}{4.869269in}}%
\pgfpathlineto{\pgfqpoint{1.963576in}{4.862734in}}%
\pgfpathlineto{\pgfqpoint{1.965500in}{4.862417in}}%
\pgfpathlineto{\pgfqpoint{1.967423in}{4.857101in}}%
\pgfpathlineto{\pgfqpoint{1.971270in}{4.821745in}}%
\pgfpathlineto{\pgfqpoint{1.973193in}{4.817119in}}%
\pgfpathlineto{\pgfqpoint{1.975117in}{4.816849in}}%
\pgfpathlineto{\pgfqpoint{1.977040in}{4.813996in}}%
\pgfpathlineto{\pgfqpoint{1.978963in}{4.820944in}}%
\pgfpathlineto{\pgfqpoint{1.982810in}{4.805880in}}%
\pgfpathlineto{\pgfqpoint{1.984734in}{4.781635in}}%
\pgfpathlineto{\pgfqpoint{1.988581in}{4.770540in}}%
\pgfpathlineto{\pgfqpoint{1.990504in}{4.787247in}}%
\pgfpathlineto{\pgfqpoint{1.992427in}{4.771715in}}%
\pgfpathlineto{\pgfqpoint{1.994351in}{4.770619in}}%
\pgfpathlineto{\pgfqpoint{1.998198in}{4.782299in}}%
\pgfpathlineto{\pgfqpoint{2.000121in}{4.799274in}}%
\pgfpathlineto{\pgfqpoint{2.002044in}{4.797265in}}%
\pgfpathlineto{\pgfqpoint{2.007815in}{4.801031in}}%
\pgfpathlineto{\pgfqpoint{2.009738in}{4.785977in}}%
\pgfpathlineto{\pgfqpoint{2.011661in}{4.789444in}}%
\pgfpathlineto{\pgfqpoint{2.013585in}{4.790286in}}%
\pgfpathlineto{\pgfqpoint{2.015508in}{4.784678in}}%
\pgfpathlineto{\pgfqpoint{2.017432in}{4.788753in}}%
\pgfpathlineto{\pgfqpoint{2.021279in}{4.805509in}}%
\pgfpathlineto{\pgfqpoint{2.025125in}{4.814550in}}%
\pgfpathlineto{\pgfqpoint{2.027049in}{4.803585in}}%
\pgfpathlineto{\pgfqpoint{2.028972in}{4.803308in}}%
\pgfpathlineto{\pgfqpoint{2.030896in}{4.798903in}}%
\pgfpathlineto{\pgfqpoint{2.032819in}{4.801021in}}%
\pgfpathlineto{\pgfqpoint{2.034742in}{4.799509in}}%
\pgfpathlineto{\pgfqpoint{2.036666in}{4.787194in}}%
\pgfpathlineto{\pgfqpoint{2.038589in}{4.802243in}}%
\pgfpathlineto{\pgfqpoint{2.040513in}{4.798485in}}%
\pgfpathlineto{\pgfqpoint{2.042436in}{4.799668in}}%
\pgfpathlineto{\pgfqpoint{2.044359in}{4.804834in}}%
\pgfpathlineto{\pgfqpoint{2.048206in}{4.780401in}}%
\pgfpathlineto{\pgfqpoint{2.050130in}{4.781832in}}%
\pgfpathlineto{\pgfqpoint{2.052053in}{4.774126in}}%
\pgfpathlineto{\pgfqpoint{2.053976in}{4.774152in}}%
\pgfpathlineto{\pgfqpoint{2.055900in}{4.768781in}}%
\pgfpathlineto{\pgfqpoint{2.057823in}{4.781526in}}%
\pgfpathlineto{\pgfqpoint{2.059747in}{4.786080in}}%
\pgfpathlineto{\pgfqpoint{2.061670in}{4.800314in}}%
\pgfpathlineto{\pgfqpoint{2.063594in}{4.804907in}}%
\pgfpathlineto{\pgfqpoint{2.065517in}{4.806156in}}%
\pgfpathlineto{\pgfqpoint{2.067440in}{4.823384in}}%
\pgfpathlineto{\pgfqpoint{2.071287in}{4.804138in}}%
\pgfpathlineto{\pgfqpoint{2.073211in}{4.805740in}}%
\pgfpathlineto{\pgfqpoint{2.075134in}{4.810497in}}%
\pgfpathlineto{\pgfqpoint{2.077057in}{4.803967in}}%
\pgfpathlineto{\pgfqpoint{2.078981in}{4.801446in}}%
\pgfpathlineto{\pgfqpoint{2.080904in}{4.810938in}}%
\pgfpathlineto{\pgfqpoint{2.082828in}{4.793336in}}%
\pgfpathlineto{\pgfqpoint{2.084751in}{4.795623in}}%
\pgfpathlineto{\pgfqpoint{2.086674in}{4.796152in}}%
\pgfpathlineto{\pgfqpoint{2.088598in}{4.788458in}}%
\pgfpathlineto{\pgfqpoint{2.090521in}{4.786790in}}%
\pgfpathlineto{\pgfqpoint{2.092445in}{4.796416in}}%
\pgfpathlineto{\pgfqpoint{2.094368in}{4.793885in}}%
\pgfpathlineto{\pgfqpoint{2.100138in}{4.807548in}}%
\pgfpathlineto{\pgfqpoint{2.102062in}{4.828969in}}%
\pgfpathlineto{\pgfqpoint{2.103985in}{4.826083in}}%
\pgfpathlineto{\pgfqpoint{2.105909in}{4.837461in}}%
\pgfpathlineto{\pgfqpoint{2.107832in}{4.830093in}}%
\pgfpathlineto{\pgfqpoint{2.109755in}{4.829727in}}%
\pgfpathlineto{\pgfqpoint{2.111679in}{4.843007in}}%
\pgfpathlineto{\pgfqpoint{2.113602in}{4.834209in}}%
\pgfpathlineto{\pgfqpoint{2.115526in}{4.856549in}}%
\pgfpathlineto{\pgfqpoint{2.117449in}{4.857798in}}%
\pgfpathlineto{\pgfqpoint{2.119372in}{4.867746in}}%
\pgfpathlineto{\pgfqpoint{2.123219in}{4.845858in}}%
\pgfpathlineto{\pgfqpoint{2.125143in}{4.851889in}}%
\pgfpathlineto{\pgfqpoint{2.127066in}{4.853872in}}%
\pgfpathlineto{\pgfqpoint{2.128990in}{4.859056in}}%
\pgfpathlineto{\pgfqpoint{2.132836in}{4.891432in}}%
\pgfpathlineto{\pgfqpoint{2.134760in}{4.883471in}}%
\pgfpathlineto{\pgfqpoint{2.136683in}{4.888332in}}%
\pgfpathlineto{\pgfqpoint{2.138607in}{4.888553in}}%
\pgfpathlineto{\pgfqpoint{2.144377in}{4.915367in}}%
\pgfpathlineto{\pgfqpoint{2.146300in}{4.915543in}}%
\pgfpathlineto{\pgfqpoint{2.150147in}{4.948636in}}%
\pgfpathlineto{\pgfqpoint{2.152070in}{4.941795in}}%
\pgfpathlineto{\pgfqpoint{2.153994in}{4.941364in}}%
\pgfpathlineto{\pgfqpoint{2.157841in}{4.917596in}}%
\pgfpathlineto{\pgfqpoint{2.159764in}{4.923200in}}%
\pgfpathlineto{\pgfqpoint{2.161688in}{4.918401in}}%
\pgfpathlineto{\pgfqpoint{2.163611in}{4.909626in}}%
\pgfpathlineto{\pgfqpoint{2.165534in}{4.908938in}}%
\pgfpathlineto{\pgfqpoint{2.167458in}{4.920265in}}%
\pgfpathlineto{\pgfqpoint{2.169381in}{4.913357in}}%
\pgfpathlineto{\pgfqpoint{2.171305in}{4.913127in}}%
\pgfpathlineto{\pgfqpoint{2.173228in}{4.919452in}}%
\pgfpathlineto{\pgfqpoint{2.175151in}{4.921304in}}%
\pgfpathlineto{\pgfqpoint{2.177075in}{4.930384in}}%
\pgfpathlineto{\pgfqpoint{2.182845in}{4.934439in}}%
\pgfpathlineto{\pgfqpoint{2.184768in}{4.927639in}}%
\pgfpathlineto{\pgfqpoint{2.188615in}{4.906239in}}%
\pgfpathlineto{\pgfqpoint{2.190539in}{4.906796in}}%
\pgfpathlineto{\pgfqpoint{2.192462in}{4.893326in}}%
\pgfpathlineto{\pgfqpoint{2.194386in}{4.895290in}}%
\pgfpathlineto{\pgfqpoint{2.196309in}{4.889595in}}%
\pgfpathlineto{\pgfqpoint{2.198232in}{4.897723in}}%
\pgfpathlineto{\pgfqpoint{2.200156in}{4.898316in}}%
\pgfpathlineto{\pgfqpoint{2.202079in}{4.897588in}}%
\pgfpathlineto{\pgfqpoint{2.204003in}{4.907242in}}%
\pgfpathlineto{\pgfqpoint{2.207849in}{4.912166in}}%
\pgfpathlineto{\pgfqpoint{2.209773in}{4.902044in}}%
\pgfpathlineto{\pgfqpoint{2.211696in}{4.904295in}}%
\pgfpathlineto{\pgfqpoint{2.213620in}{4.915871in}}%
\pgfpathlineto{\pgfqpoint{2.215543in}{4.934312in}}%
\pgfpathlineto{\pgfqpoint{2.217466in}{4.936195in}}%
\pgfpathlineto{\pgfqpoint{2.219390in}{4.933339in}}%
\pgfpathlineto{\pgfqpoint{2.221313in}{4.920531in}}%
\pgfpathlineto{\pgfqpoint{2.223237in}{4.923739in}}%
\pgfpathlineto{\pgfqpoint{2.225160in}{4.940378in}}%
\pgfpathlineto{\pgfqpoint{2.227083in}{4.937502in}}%
\pgfpathlineto{\pgfqpoint{2.230930in}{4.919588in}}%
\pgfpathlineto{\pgfqpoint{2.232854in}{4.916871in}}%
\pgfpathlineto{\pgfqpoint{2.236701in}{4.948230in}}%
\pgfpathlineto{\pgfqpoint{2.238624in}{4.934480in}}%
\pgfpathlineto{\pgfqpoint{2.240547in}{4.936295in}}%
\pgfpathlineto{\pgfqpoint{2.242471in}{4.934632in}}%
\pgfpathlineto{\pgfqpoint{2.244394in}{4.940088in}}%
\pgfpathlineto{\pgfqpoint{2.246318in}{4.959108in}}%
\pgfpathlineto{\pgfqpoint{2.248241in}{4.948958in}}%
\pgfpathlineto{\pgfqpoint{2.252088in}{4.954003in}}%
\pgfpathlineto{\pgfqpoint{2.255935in}{4.972523in}}%
\pgfpathlineto{\pgfqpoint{2.259781in}{4.958056in}}%
\pgfpathlineto{\pgfqpoint{2.261705in}{4.941345in}}%
\pgfpathlineto{\pgfqpoint{2.263628in}{4.939860in}}%
\pgfpathlineto{\pgfqpoint{2.265552in}{4.926261in}}%
\pgfpathlineto{\pgfqpoint{2.267475in}{4.931805in}}%
\pgfpathlineto{\pgfqpoint{2.269399in}{4.931035in}}%
\pgfpathlineto{\pgfqpoint{2.271322in}{4.935957in}}%
\pgfpathlineto{\pgfqpoint{2.273245in}{4.921982in}}%
\pgfpathlineto{\pgfqpoint{2.277092in}{4.926077in}}%
\pgfpathlineto{\pgfqpoint{2.279016in}{4.941342in}}%
\pgfpathlineto{\pgfqpoint{2.280939in}{4.937483in}}%
\pgfpathlineto{\pgfqpoint{2.282862in}{4.941518in}}%
\pgfpathlineto{\pgfqpoint{2.284786in}{4.955891in}}%
\pgfpathlineto{\pgfqpoint{2.286709in}{4.958839in}}%
\pgfpathlineto{\pgfqpoint{2.288633in}{4.949667in}}%
\pgfpathlineto{\pgfqpoint{2.290556in}{4.957074in}}%
\pgfpathlineto{\pgfqpoint{2.292479in}{4.944873in}}%
\pgfpathlineto{\pgfqpoint{2.294403in}{4.944063in}}%
\pgfpathlineto{\pgfqpoint{2.298250in}{4.940260in}}%
\pgfpathlineto{\pgfqpoint{2.300173in}{4.947456in}}%
\pgfpathlineto{\pgfqpoint{2.302097in}{4.960184in}}%
\pgfpathlineto{\pgfqpoint{2.304020in}{4.963995in}}%
\pgfpathlineto{\pgfqpoint{2.305943in}{4.957783in}}%
\pgfpathlineto{\pgfqpoint{2.307867in}{4.956832in}}%
\pgfpathlineto{\pgfqpoint{2.309790in}{4.959178in}}%
\pgfpathlineto{\pgfqpoint{2.313637in}{4.949613in}}%
\pgfpathlineto{\pgfqpoint{2.315560in}{4.952662in}}%
\pgfpathlineto{\pgfqpoint{2.317484in}{4.963682in}}%
\pgfpathlineto{\pgfqpoint{2.319407in}{4.944765in}}%
\pgfpathlineto{\pgfqpoint{2.321331in}{4.944367in}}%
\pgfpathlineto{\pgfqpoint{2.323254in}{4.966797in}}%
\pgfpathlineto{\pgfqpoint{2.329024in}{4.991431in}}%
\pgfpathlineto{\pgfqpoint{2.330948in}{5.008998in}}%
\pgfpathlineto{\pgfqpoint{2.332871in}{5.004247in}}%
\pgfpathlineto{\pgfqpoint{2.334795in}{5.016497in}}%
\pgfpathlineto{\pgfqpoint{2.336718in}{5.013291in}}%
\pgfpathlineto{\pgfqpoint{2.340565in}{5.042167in}}%
\pgfpathlineto{\pgfqpoint{2.342488in}{5.040668in}}%
\pgfpathlineto{\pgfqpoint{2.344412in}{5.041244in}}%
\pgfpathlineto{\pgfqpoint{2.348258in}{5.040588in}}%
\pgfpathlineto{\pgfqpoint{2.350182in}{5.033204in}}%
\pgfpathlineto{\pgfqpoint{2.354029in}{5.028754in}}%
\pgfpathlineto{\pgfqpoint{2.355952in}{5.018850in}}%
\pgfpathlineto{\pgfqpoint{2.357875in}{5.018106in}}%
\pgfpathlineto{\pgfqpoint{2.361722in}{5.007120in}}%
\pgfpathlineto{\pgfqpoint{2.365569in}{4.994340in}}%
\pgfpathlineto{\pgfqpoint{2.367492in}{4.995834in}}%
\pgfpathlineto{\pgfqpoint{2.369416in}{5.002562in}}%
\pgfpathlineto{\pgfqpoint{2.371339in}{5.000986in}}%
\pgfpathlineto{\pgfqpoint{2.373263in}{5.011853in}}%
\pgfpathlineto{\pgfqpoint{2.375186in}{4.998135in}}%
\pgfpathlineto{\pgfqpoint{2.377110in}{5.006756in}}%
\pgfpathlineto{\pgfqpoint{2.379033in}{5.010239in}}%
\pgfpathlineto{\pgfqpoint{2.384803in}{4.987296in}}%
\pgfpathlineto{\pgfqpoint{2.386727in}{4.987587in}}%
\pgfpathlineto{\pgfqpoint{2.388650in}{4.981094in}}%
\pgfpathlineto{\pgfqpoint{2.390573in}{4.999633in}}%
\pgfpathlineto{\pgfqpoint{2.392497in}{5.000444in}}%
\pgfpathlineto{\pgfqpoint{2.394420in}{4.998505in}}%
\pgfpathlineto{\pgfqpoint{2.396344in}{5.011636in}}%
\pgfpathlineto{\pgfqpoint{2.402114in}{5.029212in}}%
\pgfpathlineto{\pgfqpoint{2.404037in}{5.029593in}}%
\pgfpathlineto{\pgfqpoint{2.405961in}{5.032415in}}%
\pgfpathlineto{\pgfqpoint{2.407884in}{5.033112in}}%
\pgfpathlineto{\pgfqpoint{2.411731in}{5.025147in}}%
\pgfpathlineto{\pgfqpoint{2.413654in}{5.025167in}}%
\pgfpathlineto{\pgfqpoint{2.415578in}{5.032543in}}%
\pgfpathlineto{\pgfqpoint{2.417501in}{5.035086in}}%
\pgfpathlineto{\pgfqpoint{2.419425in}{5.030453in}}%
\pgfpathlineto{\pgfqpoint{2.427118in}{5.061034in}}%
\pgfpathlineto{\pgfqpoint{2.429042in}{5.057345in}}%
\pgfpathlineto{\pgfqpoint{2.430965in}{5.065626in}}%
\pgfpathlineto{\pgfqpoint{2.432888in}{5.057803in}}%
\pgfpathlineto{\pgfqpoint{2.434812in}{5.060848in}}%
\pgfpathlineto{\pgfqpoint{2.436735in}{5.059636in}}%
\pgfpathlineto{\pgfqpoint{2.444429in}{5.088389in}}%
\pgfpathlineto{\pgfqpoint{2.446352in}{5.071695in}}%
\pgfpathlineto{\pgfqpoint{2.450199in}{5.061694in}}%
\pgfpathlineto{\pgfqpoint{2.452123in}{5.051445in}}%
\pgfpathlineto{\pgfqpoint{2.455969in}{5.070463in}}%
\pgfpathlineto{\pgfqpoint{2.457893in}{5.067691in}}%
\pgfpathlineto{\pgfqpoint{2.459816in}{5.088194in}}%
\pgfpathlineto{\pgfqpoint{2.461740in}{5.086945in}}%
\pgfpathlineto{\pgfqpoint{2.463663in}{5.091261in}}%
\pgfpathlineto{\pgfqpoint{2.465586in}{5.085308in}}%
\pgfpathlineto{\pgfqpoint{2.467510in}{5.099546in}}%
\pgfpathlineto{\pgfqpoint{2.471357in}{5.081918in}}%
\pgfpathlineto{\pgfqpoint{2.473280in}{5.077678in}}%
\pgfpathlineto{\pgfqpoint{2.475204in}{5.080470in}}%
\pgfpathlineto{\pgfqpoint{2.477127in}{5.068731in}}%
\pgfpathlineto{\pgfqpoint{2.479050in}{5.070238in}}%
\pgfpathlineto{\pgfqpoint{2.480974in}{5.068963in}}%
\pgfpathlineto{\pgfqpoint{2.482897in}{5.073783in}}%
\pgfpathlineto{\pgfqpoint{2.484821in}{5.086150in}}%
\pgfpathlineto{\pgfqpoint{2.486744in}{5.085935in}}%
\pgfpathlineto{\pgfqpoint{2.488667in}{5.075643in}}%
\pgfpathlineto{\pgfqpoint{2.492514in}{5.099493in}}%
\pgfpathlineto{\pgfqpoint{2.494438in}{5.098510in}}%
\pgfpathlineto{\pgfqpoint{2.496361in}{5.099204in}}%
\pgfpathlineto{\pgfqpoint{2.504055in}{5.085634in}}%
\pgfpathlineto{\pgfqpoint{2.505978in}{5.095970in}}%
\pgfpathlineto{\pgfqpoint{2.507901in}{5.083404in}}%
\pgfpathlineto{\pgfqpoint{2.509825in}{5.089694in}}%
\pgfpathlineto{\pgfqpoint{2.511748in}{5.078075in}}%
\pgfpathlineto{\pgfqpoint{2.513672in}{5.073560in}}%
\pgfpathlineto{\pgfqpoint{2.517519in}{5.051357in}}%
\pgfpathlineto{\pgfqpoint{2.519442in}{5.049175in}}%
\pgfpathlineto{\pgfqpoint{2.521365in}{5.043527in}}%
\pgfpathlineto{\pgfqpoint{2.523289in}{5.049764in}}%
\pgfpathlineto{\pgfqpoint{2.525212in}{5.045405in}}%
\pgfpathlineto{\pgfqpoint{2.527136in}{5.046178in}}%
\pgfpathlineto{\pgfqpoint{2.529059in}{5.041203in}}%
\pgfpathlineto{\pgfqpoint{2.530982in}{5.040200in}}%
\pgfpathlineto{\pgfqpoint{2.532906in}{5.044383in}}%
\pgfpathlineto{\pgfqpoint{2.536753in}{5.065356in}}%
\pgfpathlineto{\pgfqpoint{2.538676in}{5.056831in}}%
\pgfpathlineto{\pgfqpoint{2.540599in}{5.075809in}}%
\pgfpathlineto{\pgfqpoint{2.542523in}{5.077905in}}%
\pgfpathlineto{\pgfqpoint{2.544446in}{5.076810in}}%
\pgfpathlineto{\pgfqpoint{2.546370in}{5.060453in}}%
\pgfpathlineto{\pgfqpoint{2.548293in}{5.069573in}}%
\pgfpathlineto{\pgfqpoint{2.550217in}{5.066947in}}%
\pgfpathlineto{\pgfqpoint{2.552140in}{5.073178in}}%
\pgfpathlineto{\pgfqpoint{2.554063in}{5.064830in}}%
\pgfpathlineto{\pgfqpoint{2.557910in}{5.077232in}}%
\pgfpathlineto{\pgfqpoint{2.559834in}{5.074974in}}%
\pgfpathlineto{\pgfqpoint{2.561757in}{5.058770in}}%
\pgfpathlineto{\pgfqpoint{2.563680in}{5.057023in}}%
\pgfpathlineto{\pgfqpoint{2.565604in}{5.073294in}}%
\pgfpathlineto{\pgfqpoint{2.567527in}{5.075953in}}%
\pgfpathlineto{\pgfqpoint{2.569451in}{5.069546in}}%
\pgfpathlineto{\pgfqpoint{2.571374in}{5.071606in}}%
\pgfpathlineto{\pgfqpoint{2.573297in}{5.077787in}}%
\pgfpathlineto{\pgfqpoint{2.575221in}{5.078702in}}%
\pgfpathlineto{\pgfqpoint{2.580991in}{5.120791in}}%
\pgfpathlineto{\pgfqpoint{2.582915in}{5.119562in}}%
\pgfpathlineto{\pgfqpoint{2.584838in}{5.108480in}}%
\pgfpathlineto{\pgfqpoint{2.586761in}{5.107232in}}%
\pgfpathlineto{\pgfqpoint{2.590608in}{5.127928in}}%
\pgfpathlineto{\pgfqpoint{2.592532in}{5.140471in}}%
\pgfpathlineto{\pgfqpoint{2.594455in}{5.140795in}}%
\pgfpathlineto{\pgfqpoint{2.596378in}{5.129940in}}%
\pgfpathlineto{\pgfqpoint{2.598302in}{5.150193in}}%
\pgfpathlineto{\pgfqpoint{2.600225in}{5.142829in}}%
\pgfpathlineto{\pgfqpoint{2.602149in}{5.141951in}}%
\pgfpathlineto{\pgfqpoint{2.605995in}{5.150750in}}%
\pgfpathlineto{\pgfqpoint{2.607919in}{5.162542in}}%
\pgfpathlineto{\pgfqpoint{2.609842in}{5.164439in}}%
\pgfpathlineto{\pgfqpoint{2.613689in}{5.184464in}}%
\pgfpathlineto{\pgfqpoint{2.617536in}{5.184274in}}%
\pgfpathlineto{\pgfqpoint{2.619459in}{5.165061in}}%
\pgfpathlineto{\pgfqpoint{2.621383in}{5.172529in}}%
\pgfpathlineto{\pgfqpoint{2.623306in}{5.166418in}}%
\pgfpathlineto{\pgfqpoint{2.625230in}{5.154917in}}%
\pgfpathlineto{\pgfqpoint{2.627153in}{5.153809in}}%
\pgfpathlineto{\pgfqpoint{2.629076in}{5.149734in}}%
\pgfpathlineto{\pgfqpoint{2.632923in}{5.169401in}}%
\pgfpathlineto{\pgfqpoint{2.634847in}{5.170467in}}%
\pgfpathlineto{\pgfqpoint{2.636770in}{5.164306in}}%
\pgfpathlineto{\pgfqpoint{2.638693in}{5.172094in}}%
\pgfpathlineto{\pgfqpoint{2.642540in}{5.159686in}}%
\pgfpathlineto{\pgfqpoint{2.644464in}{5.160018in}}%
\pgfpathlineto{\pgfqpoint{2.646387in}{5.166028in}}%
\pgfpathlineto{\pgfqpoint{2.648311in}{5.162227in}}%
\pgfpathlineto{\pgfqpoint{2.652157in}{5.147912in}}%
\pgfpathlineto{\pgfqpoint{2.654081in}{5.156316in}}%
\pgfpathlineto{\pgfqpoint{2.656004in}{5.155726in}}%
\pgfpathlineto{\pgfqpoint{2.657928in}{5.150193in}}%
\pgfpathlineto{\pgfqpoint{2.661774in}{5.163638in}}%
\pgfpathlineto{\pgfqpoint{2.663698in}{5.163403in}}%
\pgfpathlineto{\pgfqpoint{2.665621in}{5.153220in}}%
\pgfpathlineto{\pgfqpoint{2.667545in}{5.160055in}}%
\pgfpathlineto{\pgfqpoint{2.671391in}{5.162430in}}%
\pgfpathlineto{\pgfqpoint{2.675238in}{5.152384in}}%
\pgfpathlineto{\pgfqpoint{2.677162in}{5.149054in}}%
\pgfpathlineto{\pgfqpoint{2.679085in}{5.138821in}}%
\pgfpathlineto{\pgfqpoint{2.681008in}{5.145538in}}%
\pgfpathlineto{\pgfqpoint{2.684855in}{5.133707in}}%
\pgfpathlineto{\pgfqpoint{2.686779in}{5.158158in}}%
\pgfpathlineto{\pgfqpoint{2.688702in}{5.149097in}}%
\pgfpathlineto{\pgfqpoint{2.690626in}{5.151598in}}%
\pgfpathlineto{\pgfqpoint{2.692549in}{5.163861in}}%
\pgfpathlineto{\pgfqpoint{2.694472in}{5.138109in}}%
\pgfpathlineto{\pgfqpoint{2.700243in}{5.128903in}}%
\pgfpathlineto{\pgfqpoint{2.702166in}{5.136033in}}%
\pgfpathlineto{\pgfqpoint{2.706013in}{5.117862in}}%
\pgfpathlineto{\pgfqpoint{2.709860in}{5.121047in}}%
\pgfpathlineto{\pgfqpoint{2.713706in}{5.117901in}}%
\pgfpathlineto{\pgfqpoint{2.717553in}{5.145605in}}%
\pgfpathlineto{\pgfqpoint{2.721400in}{5.123346in}}%
\pgfpathlineto{\pgfqpoint{2.725247in}{5.147268in}}%
\pgfpathlineto{\pgfqpoint{2.727170in}{5.125909in}}%
\pgfpathlineto{\pgfqpoint{2.729094in}{5.123332in}}%
\pgfpathlineto{\pgfqpoint{2.731017in}{5.122878in}}%
\pgfpathlineto{\pgfqpoint{2.732941in}{5.095863in}}%
\pgfpathlineto{\pgfqpoint{2.734864in}{5.103682in}}%
\pgfpathlineto{\pgfqpoint{2.736787in}{5.098371in}}%
\pgfpathlineto{\pgfqpoint{2.740634in}{5.119704in}}%
\pgfpathlineto{\pgfqpoint{2.744481in}{5.095725in}}%
\pgfpathlineto{\pgfqpoint{2.746404in}{5.082747in}}%
\pgfpathlineto{\pgfqpoint{2.748328in}{5.082536in}}%
\pgfpathlineto{\pgfqpoint{2.750251in}{5.069349in}}%
\pgfpathlineto{\pgfqpoint{2.756022in}{5.057375in}}%
\pgfpathlineto{\pgfqpoint{2.757945in}{5.055543in}}%
\pgfpathlineto{\pgfqpoint{2.759868in}{5.057287in}}%
\pgfpathlineto{\pgfqpoint{2.763715in}{5.022931in}}%
\pgfpathlineto{\pgfqpoint{2.765639in}{5.021154in}}%
\pgfpathlineto{\pgfqpoint{2.767562in}{5.012918in}}%
\pgfpathlineto{\pgfqpoint{2.767562in}{5.012918in}}%
\pgfusepath{stroke}%
\end{pgfscope}%
\begin{pgfscope}%
\pgfsetrectcap%
\pgfsetmiterjoin%
\pgfsetlinewidth{0.000000pt}%
\definecolor{currentstroke}{rgb}{1.000000,1.000000,1.000000}%
\pgfsetstrokecolor{currentstroke}%
\pgfsetdash{}{0pt}%
\pgfpathmoveto{\pgfqpoint{0.750000in}{3.180000in}}%
\pgfpathlineto{\pgfqpoint{0.750000in}{5.280000in}}%
\pgfusepath{}%
\end{pgfscope}%
\begin{pgfscope}%
\pgfsetrectcap%
\pgfsetmiterjoin%
\pgfsetlinewidth{0.000000pt}%
\definecolor{currentstroke}{rgb}{1.000000,1.000000,1.000000}%
\pgfsetstrokecolor{currentstroke}%
\pgfsetdash{}{0pt}%
\pgfpathmoveto{\pgfqpoint{2.863636in}{3.180000in}}%
\pgfpathlineto{\pgfqpoint{2.863636in}{5.280000in}}%
\pgfusepath{}%
\end{pgfscope}%
\begin{pgfscope}%
\pgfsetrectcap%
\pgfsetmiterjoin%
\pgfsetlinewidth{0.000000pt}%
\definecolor{currentstroke}{rgb}{1.000000,1.000000,1.000000}%
\pgfsetstrokecolor{currentstroke}%
\pgfsetdash{}{0pt}%
\pgfpathmoveto{\pgfqpoint{0.750000in}{3.180000in}}%
\pgfpathlineto{\pgfqpoint{2.863636in}{3.180000in}}%
\pgfusepath{}%
\end{pgfscope}%
\begin{pgfscope}%
\pgfsetrectcap%
\pgfsetmiterjoin%
\pgfsetlinewidth{0.000000pt}%
\definecolor{currentstroke}{rgb}{1.000000,1.000000,1.000000}%
\pgfsetstrokecolor{currentstroke}%
\pgfsetdash{}{0pt}%
\pgfpathmoveto{\pgfqpoint{0.750000in}{5.280000in}}%
\pgfpathlineto{\pgfqpoint{2.863636in}{5.280000in}}%
\pgfusepath{}%
\end{pgfscope}%
\begin{pgfscope}%
\pgfsetbuttcap%
\pgfsetmiterjoin%
\definecolor{currentfill}{rgb}{0.917647,0.917647,0.949020}%
\pgfsetfillcolor{currentfill}%
\pgfsetlinewidth{0.000000pt}%
\definecolor{currentstroke}{rgb}{0.000000,0.000000,0.000000}%
\pgfsetstrokecolor{currentstroke}%
\pgfsetstrokeopacity{0.000000}%
\pgfsetdash{}{0pt}%
\pgfpathmoveto{\pgfqpoint{3.286364in}{3.180000in}}%
\pgfpathlineto{\pgfqpoint{5.400000in}{3.180000in}}%
\pgfpathlineto{\pgfqpoint{5.400000in}{5.280000in}}%
\pgfpathlineto{\pgfqpoint{3.286364in}{5.280000in}}%
\pgfpathlineto{\pgfqpoint{3.286364in}{3.180000in}}%
\pgfpathclose%
\pgfusepath{fill}%
\end{pgfscope}%
\begin{pgfscope}%
\pgfpathrectangle{\pgfqpoint{3.286364in}{3.180000in}}{\pgfqpoint{2.113636in}{2.100000in}}%
\pgfusepath{clip}%
\pgfsetroundcap%
\pgfsetroundjoin%
\pgfsetlinewidth{1.003750pt}%
\definecolor{currentstroke}{rgb}{1.000000,1.000000,1.000000}%
\pgfsetstrokecolor{currentstroke}%
\pgfsetdash{}{0pt}%
\pgfpathmoveto{\pgfqpoint{3.382438in}{3.180000in}}%
\pgfpathlineto{\pgfqpoint{3.382438in}{5.280000in}}%
\pgfusepath{stroke}%
\end{pgfscope}%
\begin{pgfscope}%
\definecolor{textcolor}{rgb}{0.150000,0.150000,0.150000}%
\pgfsetstrokecolor{textcolor}%
\pgfsetfillcolor{textcolor}%
\pgftext[x=3.382438in,y=3.082778in,,top]{\color{textcolor}\rmfamily\fontsize{10.000000}{12.000000}\selectfont \(\displaystyle {0.0}\)}%
\end{pgfscope}%
\begin{pgfscope}%
\pgfpathrectangle{\pgfqpoint{3.286364in}{3.180000in}}{\pgfqpoint{2.113636in}{2.100000in}}%
\pgfusepath{clip}%
\pgfsetroundcap%
\pgfsetroundjoin%
\pgfsetlinewidth{1.003750pt}%
\definecolor{currentstroke}{rgb}{1.000000,1.000000,1.000000}%
\pgfsetstrokecolor{currentstroke}%
\pgfsetdash{}{0pt}%
\pgfpathmoveto{\pgfqpoint{3.862810in}{3.180000in}}%
\pgfpathlineto{\pgfqpoint{3.862810in}{5.280000in}}%
\pgfusepath{stroke}%
\end{pgfscope}%
\begin{pgfscope}%
\definecolor{textcolor}{rgb}{0.150000,0.150000,0.150000}%
\pgfsetstrokecolor{textcolor}%
\pgfsetfillcolor{textcolor}%
\pgftext[x=3.862810in,y=3.082778in,,top]{\color{textcolor}\rmfamily\fontsize{10.000000}{12.000000}\selectfont \(\displaystyle {2.5}\)}%
\end{pgfscope}%
\begin{pgfscope}%
\pgfpathrectangle{\pgfqpoint{3.286364in}{3.180000in}}{\pgfqpoint{2.113636in}{2.100000in}}%
\pgfusepath{clip}%
\pgfsetroundcap%
\pgfsetroundjoin%
\pgfsetlinewidth{1.003750pt}%
\definecolor{currentstroke}{rgb}{1.000000,1.000000,1.000000}%
\pgfsetstrokecolor{currentstroke}%
\pgfsetdash{}{0pt}%
\pgfpathmoveto{\pgfqpoint{4.343182in}{3.180000in}}%
\pgfpathlineto{\pgfqpoint{4.343182in}{5.280000in}}%
\pgfusepath{stroke}%
\end{pgfscope}%
\begin{pgfscope}%
\definecolor{textcolor}{rgb}{0.150000,0.150000,0.150000}%
\pgfsetstrokecolor{textcolor}%
\pgfsetfillcolor{textcolor}%
\pgftext[x=4.343182in,y=3.082778in,,top]{\color{textcolor}\rmfamily\fontsize{10.000000}{12.000000}\selectfont \(\displaystyle {5.0}\)}%
\end{pgfscope}%
\begin{pgfscope}%
\pgfpathrectangle{\pgfqpoint{3.286364in}{3.180000in}}{\pgfqpoint{2.113636in}{2.100000in}}%
\pgfusepath{clip}%
\pgfsetroundcap%
\pgfsetroundjoin%
\pgfsetlinewidth{1.003750pt}%
\definecolor{currentstroke}{rgb}{1.000000,1.000000,1.000000}%
\pgfsetstrokecolor{currentstroke}%
\pgfsetdash{}{0pt}%
\pgfpathmoveto{\pgfqpoint{4.823554in}{3.180000in}}%
\pgfpathlineto{\pgfqpoint{4.823554in}{5.280000in}}%
\pgfusepath{stroke}%
\end{pgfscope}%
\begin{pgfscope}%
\definecolor{textcolor}{rgb}{0.150000,0.150000,0.150000}%
\pgfsetstrokecolor{textcolor}%
\pgfsetfillcolor{textcolor}%
\pgftext[x=4.823554in,y=3.082778in,,top]{\color{textcolor}\rmfamily\fontsize{10.000000}{12.000000}\selectfont \(\displaystyle {7.5}\)}%
\end{pgfscope}%
\begin{pgfscope}%
\pgfpathrectangle{\pgfqpoint{3.286364in}{3.180000in}}{\pgfqpoint{2.113636in}{2.100000in}}%
\pgfusepath{clip}%
\pgfsetroundcap%
\pgfsetroundjoin%
\pgfsetlinewidth{1.003750pt}%
\definecolor{currentstroke}{rgb}{1.000000,1.000000,1.000000}%
\pgfsetstrokecolor{currentstroke}%
\pgfsetdash{}{0pt}%
\pgfpathmoveto{\pgfqpoint{5.303926in}{3.180000in}}%
\pgfpathlineto{\pgfqpoint{5.303926in}{5.280000in}}%
\pgfusepath{stroke}%
\end{pgfscope}%
\begin{pgfscope}%
\definecolor{textcolor}{rgb}{0.150000,0.150000,0.150000}%
\pgfsetstrokecolor{textcolor}%
\pgfsetfillcolor{textcolor}%
\pgftext[x=5.303926in,y=3.082778in,,top]{\color{textcolor}\rmfamily\fontsize{10.000000}{12.000000}\selectfont \(\displaystyle {10.0}\)}%
\end{pgfscope}%
\begin{pgfscope}%
\definecolor{textcolor}{rgb}{0.150000,0.150000,0.150000}%
\pgfsetstrokecolor{textcolor}%
\pgfsetfillcolor{textcolor}%
\pgftext[x=4.343182in,y=2.903766in,,top]{\color{textcolor}\rmfamily\fontsize{11.000000}{13.200000}\selectfont time (\(\displaystyle t\))}%
\end{pgfscope}%
\begin{pgfscope}%
\pgfpathrectangle{\pgfqpoint{3.286364in}{3.180000in}}{\pgfqpoint{2.113636in}{2.100000in}}%
\pgfusepath{clip}%
\pgfsetroundcap%
\pgfsetroundjoin%
\pgfsetlinewidth{1.003750pt}%
\definecolor{currentstroke}{rgb}{1.000000,1.000000,1.000000}%
\pgfsetstrokecolor{currentstroke}%
\pgfsetdash{}{0pt}%
\pgfpathmoveto{\pgfqpoint{3.286364in}{3.212094in}}%
\pgfpathlineto{\pgfqpoint{5.400000in}{3.212094in}}%
\pgfusepath{stroke}%
\end{pgfscope}%
\begin{pgfscope}%
\definecolor{textcolor}{rgb}{0.150000,0.150000,0.150000}%
\pgfsetstrokecolor{textcolor}%
\pgfsetfillcolor{textcolor}%
\pgftext[x=2.942227in, y=3.163869in, left, base]{\color{textcolor}\rmfamily\fontsize{10.000000}{12.000000}\selectfont \(\displaystyle {\ensuremath{-}10}\)}%
\end{pgfscope}%
\begin{pgfscope}%
\pgfpathrectangle{\pgfqpoint{3.286364in}{3.180000in}}{\pgfqpoint{2.113636in}{2.100000in}}%
\pgfusepath{clip}%
\pgfsetroundcap%
\pgfsetroundjoin%
\pgfsetlinewidth{1.003750pt}%
\definecolor{currentstroke}{rgb}{1.000000,1.000000,1.000000}%
\pgfsetstrokecolor{currentstroke}%
\pgfsetdash{}{0pt}%
\pgfpathmoveto{\pgfqpoint{3.286364in}{3.681445in}}%
\pgfpathlineto{\pgfqpoint{5.400000in}{3.681445in}}%
\pgfusepath{stroke}%
\end{pgfscope}%
\begin{pgfscope}%
\definecolor{textcolor}{rgb}{0.150000,0.150000,0.150000}%
\pgfsetstrokecolor{textcolor}%
\pgfsetfillcolor{textcolor}%
\pgftext[x=3.011672in, y=3.633220in, left, base]{\color{textcolor}\rmfamily\fontsize{10.000000}{12.000000}\selectfont \(\displaystyle {\ensuremath{-}5}\)}%
\end{pgfscope}%
\begin{pgfscope}%
\pgfpathrectangle{\pgfqpoint{3.286364in}{3.180000in}}{\pgfqpoint{2.113636in}{2.100000in}}%
\pgfusepath{clip}%
\pgfsetroundcap%
\pgfsetroundjoin%
\pgfsetlinewidth{1.003750pt}%
\definecolor{currentstroke}{rgb}{1.000000,1.000000,1.000000}%
\pgfsetstrokecolor{currentstroke}%
\pgfsetdash{}{0pt}%
\pgfpathmoveto{\pgfqpoint{3.286364in}{4.150796in}}%
\pgfpathlineto{\pgfqpoint{5.400000in}{4.150796in}}%
\pgfusepath{stroke}%
\end{pgfscope}%
\begin{pgfscope}%
\definecolor{textcolor}{rgb}{0.150000,0.150000,0.150000}%
\pgfsetstrokecolor{textcolor}%
\pgfsetfillcolor{textcolor}%
\pgftext[x=3.119697in, y=4.102571in, left, base]{\color{textcolor}\rmfamily\fontsize{10.000000}{12.000000}\selectfont \(\displaystyle {0}\)}%
\end{pgfscope}%
\begin{pgfscope}%
\pgfpathrectangle{\pgfqpoint{3.286364in}{3.180000in}}{\pgfqpoint{2.113636in}{2.100000in}}%
\pgfusepath{clip}%
\pgfsetroundcap%
\pgfsetroundjoin%
\pgfsetlinewidth{1.003750pt}%
\definecolor{currentstroke}{rgb}{1.000000,1.000000,1.000000}%
\pgfsetstrokecolor{currentstroke}%
\pgfsetdash{}{0pt}%
\pgfpathmoveto{\pgfqpoint{3.286364in}{4.620147in}}%
\pgfpathlineto{\pgfqpoint{5.400000in}{4.620147in}}%
\pgfusepath{stroke}%
\end{pgfscope}%
\begin{pgfscope}%
\definecolor{textcolor}{rgb}{0.150000,0.150000,0.150000}%
\pgfsetstrokecolor{textcolor}%
\pgfsetfillcolor{textcolor}%
\pgftext[x=3.119697in, y=4.571922in, left, base]{\color{textcolor}\rmfamily\fontsize{10.000000}{12.000000}\selectfont \(\displaystyle {5}\)}%
\end{pgfscope}%
\begin{pgfscope}%
\pgfpathrectangle{\pgfqpoint{3.286364in}{3.180000in}}{\pgfqpoint{2.113636in}{2.100000in}}%
\pgfusepath{clip}%
\pgfsetroundcap%
\pgfsetroundjoin%
\pgfsetlinewidth{1.003750pt}%
\definecolor{currentstroke}{rgb}{1.000000,1.000000,1.000000}%
\pgfsetstrokecolor{currentstroke}%
\pgfsetdash{}{0pt}%
\pgfpathmoveto{\pgfqpoint{3.286364in}{5.089498in}}%
\pgfpathlineto{\pgfqpoint{5.400000in}{5.089498in}}%
\pgfusepath{stroke}%
\end{pgfscope}%
\begin{pgfscope}%
\definecolor{textcolor}{rgb}{0.150000,0.150000,0.150000}%
\pgfsetstrokecolor{textcolor}%
\pgfsetfillcolor{textcolor}%
\pgftext[x=3.050252in, y=5.041273in, left, base]{\color{textcolor}\rmfamily\fontsize{10.000000}{12.000000}\selectfont \(\displaystyle {10}\)}%
\end{pgfscope}%
\begin{pgfscope}%
\definecolor{textcolor}{rgb}{0.150000,0.150000,0.150000}%
\pgfsetstrokecolor{textcolor}%
\pgfsetfillcolor{textcolor}%
\pgftext[x=2.886672in,y=4.230000in,,bottom,rotate=90.000000]{\color{textcolor}\rmfamily\fontsize{11.000000}{13.200000}\selectfont Position}%
\end{pgfscope}%
\begin{pgfscope}%
\pgfpathrectangle{\pgfqpoint{3.286364in}{3.180000in}}{\pgfqpoint{2.113636in}{2.100000in}}%
\pgfusepath{clip}%
\pgfsetroundcap%
\pgfsetroundjoin%
\pgfsetlinewidth{0.602250pt}%
\definecolor{currentstroke}{rgb}{0.215686,0.494118,0.721569}%
\pgfsetstrokecolor{currentstroke}%
\pgfsetdash{}{0pt}%
\pgfpathmoveto{\pgfqpoint{3.382438in}{4.056926in}}%
\pgfpathlineto{\pgfqpoint{3.386285in}{4.042433in}}%
\pgfpathlineto{\pgfqpoint{3.388208in}{4.047452in}}%
\pgfpathlineto{\pgfqpoint{3.392055in}{4.019357in}}%
\pgfpathlineto{\pgfqpoint{3.393978in}{4.026099in}}%
\pgfpathlineto{\pgfqpoint{3.395902in}{4.023449in}}%
\pgfpathlineto{\pgfqpoint{3.401672in}{3.998255in}}%
\pgfpathlineto{\pgfqpoint{3.403596in}{4.003676in}}%
\pgfpathlineto{\pgfqpoint{3.417059in}{3.906653in}}%
\pgfpathlineto{\pgfqpoint{3.418983in}{3.917956in}}%
\pgfpathlineto{\pgfqpoint{3.422830in}{3.909617in}}%
\pgfpathlineto{\pgfqpoint{3.426676in}{3.923546in}}%
\pgfpathlineto{\pgfqpoint{3.428600in}{3.918505in}}%
\pgfpathlineto{\pgfqpoint{3.430523in}{3.921109in}}%
\pgfpathlineto{\pgfqpoint{3.432447in}{3.920774in}}%
\pgfpathlineto{\pgfqpoint{3.434370in}{3.939333in}}%
\pgfpathlineto{\pgfqpoint{3.436294in}{3.917440in}}%
\pgfpathlineto{\pgfqpoint{3.438217in}{3.925512in}}%
\pgfpathlineto{\pgfqpoint{3.440140in}{3.927686in}}%
\pgfpathlineto{\pgfqpoint{3.442064in}{3.927921in}}%
\pgfpathlineto{\pgfqpoint{3.443987in}{3.915462in}}%
\pgfpathlineto{\pgfqpoint{3.445911in}{3.914990in}}%
\pgfpathlineto{\pgfqpoint{3.449757in}{3.939733in}}%
\pgfpathlineto{\pgfqpoint{3.453604in}{3.931014in}}%
\pgfpathlineto{\pgfqpoint{3.455528in}{3.941296in}}%
\pgfpathlineto{\pgfqpoint{3.457451in}{3.940207in}}%
\pgfpathlineto{\pgfqpoint{3.459374in}{3.960032in}}%
\pgfpathlineto{\pgfqpoint{3.461298in}{3.945664in}}%
\pgfpathlineto{\pgfqpoint{3.463221in}{3.949244in}}%
\pgfpathlineto{\pgfqpoint{3.465145in}{3.956216in}}%
\pgfpathlineto{\pgfqpoint{3.467068in}{3.971633in}}%
\pgfpathlineto{\pgfqpoint{3.468992in}{3.968709in}}%
\pgfpathlineto{\pgfqpoint{3.470915in}{3.968066in}}%
\pgfpathlineto{\pgfqpoint{3.474762in}{3.988494in}}%
\pgfpathlineto{\pgfqpoint{3.476685in}{4.005950in}}%
\pgfpathlineto{\pgfqpoint{3.478609in}{4.010257in}}%
\pgfpathlineto{\pgfqpoint{3.480532in}{3.983789in}}%
\pgfpathlineto{\pgfqpoint{3.484379in}{4.000127in}}%
\pgfpathlineto{\pgfqpoint{3.486302in}{4.001366in}}%
\pgfpathlineto{\pgfqpoint{3.492072in}{3.981378in}}%
\pgfpathlineto{\pgfqpoint{3.493996in}{3.996147in}}%
\pgfpathlineto{\pgfqpoint{3.495919in}{4.002199in}}%
\pgfpathlineto{\pgfqpoint{3.501689in}{3.981106in}}%
\pgfpathlineto{\pgfqpoint{3.503613in}{3.978477in}}%
\pgfpathlineto{\pgfqpoint{3.505536in}{3.978643in}}%
\pgfpathlineto{\pgfqpoint{3.507460in}{3.984910in}}%
\pgfpathlineto{\pgfqpoint{3.509383in}{3.998978in}}%
\pgfpathlineto{\pgfqpoint{3.513230in}{4.003533in}}%
\pgfpathlineto{\pgfqpoint{3.515153in}{4.006727in}}%
\pgfpathlineto{\pgfqpoint{3.517077in}{3.997444in}}%
\pgfpathlineto{\pgfqpoint{3.519000in}{4.013888in}}%
\pgfpathlineto{\pgfqpoint{3.524770in}{3.986454in}}%
\pgfpathlineto{\pgfqpoint{3.526694in}{3.996314in}}%
\pgfpathlineto{\pgfqpoint{3.528617in}{3.992995in}}%
\pgfpathlineto{\pgfqpoint{3.530541in}{3.995067in}}%
\pgfpathlineto{\pgfqpoint{3.532464in}{3.991976in}}%
\pgfpathlineto{\pgfqpoint{3.534387in}{3.975620in}}%
\pgfpathlineto{\pgfqpoint{3.538234in}{3.965672in}}%
\pgfpathlineto{\pgfqpoint{3.540158in}{3.964762in}}%
\pgfpathlineto{\pgfqpoint{3.542081in}{3.961347in}}%
\pgfpathlineto{\pgfqpoint{3.544005in}{3.950188in}}%
\pgfpathlineto{\pgfqpoint{3.545928in}{3.949111in}}%
\pgfpathlineto{\pgfqpoint{3.549775in}{3.936613in}}%
\pgfpathlineto{\pgfqpoint{3.551698in}{3.937265in}}%
\pgfpathlineto{\pgfqpoint{3.553622in}{3.927976in}}%
\pgfpathlineto{\pgfqpoint{3.555545in}{3.932871in}}%
\pgfpathlineto{\pgfqpoint{3.559392in}{3.923924in}}%
\pgfpathlineto{\pgfqpoint{3.561315in}{3.928845in}}%
\pgfpathlineto{\pgfqpoint{3.563239in}{3.919798in}}%
\pgfpathlineto{\pgfqpoint{3.565162in}{3.918628in}}%
\pgfpathlineto{\pgfqpoint{3.567085in}{3.905756in}}%
\pgfpathlineto{\pgfqpoint{3.569009in}{3.911613in}}%
\pgfpathlineto{\pgfqpoint{3.570932in}{3.903228in}}%
\pgfpathlineto{\pgfqpoint{3.572856in}{3.907991in}}%
\pgfpathlineto{\pgfqpoint{3.576703in}{3.888305in}}%
\pgfpathlineto{\pgfqpoint{3.578626in}{3.880904in}}%
\pgfpathlineto{\pgfqpoint{3.580549in}{3.886077in}}%
\pgfpathlineto{\pgfqpoint{3.584396in}{3.907168in}}%
\pgfpathlineto{\pgfqpoint{3.588243in}{3.905860in}}%
\pgfpathlineto{\pgfqpoint{3.590166in}{3.904044in}}%
\pgfpathlineto{\pgfqpoint{3.592090in}{3.895431in}}%
\pgfpathlineto{\pgfqpoint{3.594013in}{3.893232in}}%
\pgfpathlineto{\pgfqpoint{3.595937in}{3.886157in}}%
\pgfpathlineto{\pgfqpoint{3.597860in}{3.885618in}}%
\pgfpathlineto{\pgfqpoint{3.601707in}{3.898261in}}%
\pgfpathlineto{\pgfqpoint{3.603630in}{3.898772in}}%
\pgfpathlineto{\pgfqpoint{3.607477in}{3.892184in}}%
\pgfpathlineto{\pgfqpoint{3.609401in}{3.891817in}}%
\pgfpathlineto{\pgfqpoint{3.611324in}{3.892643in}}%
\pgfpathlineto{\pgfqpoint{3.613247in}{3.890887in}}%
\pgfpathlineto{\pgfqpoint{3.615171in}{3.875618in}}%
\pgfpathlineto{\pgfqpoint{3.620941in}{3.902748in}}%
\pgfpathlineto{\pgfqpoint{3.624788in}{3.900705in}}%
\pgfpathlineto{\pgfqpoint{3.626711in}{3.909136in}}%
\pgfpathlineto{\pgfqpoint{3.628635in}{3.902607in}}%
\pgfpathlineto{\pgfqpoint{3.630558in}{3.912857in}}%
\pgfpathlineto{\pgfqpoint{3.632481in}{3.916809in}}%
\pgfpathlineto{\pgfqpoint{3.638252in}{3.902924in}}%
\pgfpathlineto{\pgfqpoint{3.640175in}{3.892157in}}%
\pgfpathlineto{\pgfqpoint{3.642099in}{3.893944in}}%
\pgfpathlineto{\pgfqpoint{3.644022in}{3.910108in}}%
\pgfpathlineto{\pgfqpoint{3.647869in}{3.894573in}}%
\pgfpathlineto{\pgfqpoint{3.649792in}{3.896266in}}%
\pgfpathlineto{\pgfqpoint{3.655562in}{3.883164in}}%
\pgfpathlineto{\pgfqpoint{3.667103in}{3.943215in}}%
\pgfpathlineto{\pgfqpoint{3.669026in}{3.937897in}}%
\pgfpathlineto{\pgfqpoint{3.670950in}{3.927739in}}%
\pgfpathlineto{\pgfqpoint{3.672873in}{3.927351in}}%
\pgfpathlineto{\pgfqpoint{3.674796in}{3.919164in}}%
\pgfpathlineto{\pgfqpoint{3.676720in}{3.901313in}}%
\pgfpathlineto{\pgfqpoint{3.678643in}{3.910862in}}%
\pgfpathlineto{\pgfqpoint{3.680567in}{3.913247in}}%
\pgfpathlineto{\pgfqpoint{3.682490in}{3.899686in}}%
\pgfpathlineto{\pgfqpoint{3.684414in}{3.894729in}}%
\pgfpathlineto{\pgfqpoint{3.686337in}{3.899049in}}%
\pgfpathlineto{\pgfqpoint{3.688260in}{3.894416in}}%
\pgfpathlineto{\pgfqpoint{3.692107in}{3.877972in}}%
\pgfpathlineto{\pgfqpoint{3.694031in}{3.884747in}}%
\pgfpathlineto{\pgfqpoint{3.695954in}{3.897749in}}%
\pgfpathlineto{\pgfqpoint{3.697877in}{3.881297in}}%
\pgfpathlineto{\pgfqpoint{3.699801in}{3.885198in}}%
\pgfpathlineto{\pgfqpoint{3.703648in}{3.903307in}}%
\pgfpathlineto{\pgfqpoint{3.707494in}{3.886657in}}%
\pgfpathlineto{\pgfqpoint{3.709418in}{3.886282in}}%
\pgfpathlineto{\pgfqpoint{3.711341in}{3.878957in}}%
\pgfpathlineto{\pgfqpoint{3.713265in}{3.851751in}}%
\pgfpathlineto{\pgfqpoint{3.717112in}{3.846878in}}%
\pgfpathlineto{\pgfqpoint{3.719035in}{3.850215in}}%
\pgfpathlineto{\pgfqpoint{3.720958in}{3.848758in}}%
\pgfpathlineto{\pgfqpoint{3.724805in}{3.831518in}}%
\pgfpathlineto{\pgfqpoint{3.732499in}{3.863341in}}%
\pgfpathlineto{\pgfqpoint{3.734422in}{3.862542in}}%
\pgfpathlineto{\pgfqpoint{3.736346in}{3.865962in}}%
\pgfpathlineto{\pgfqpoint{3.738269in}{3.862952in}}%
\pgfpathlineto{\pgfqpoint{3.740192in}{3.876780in}}%
\pgfpathlineto{\pgfqpoint{3.745963in}{3.845009in}}%
\pgfpathlineto{\pgfqpoint{3.747886in}{3.851334in}}%
\pgfpathlineto{\pgfqpoint{3.749810in}{3.840423in}}%
\pgfpathlineto{\pgfqpoint{3.751733in}{3.837739in}}%
\pgfpathlineto{\pgfqpoint{3.753656in}{3.822140in}}%
\pgfpathlineto{\pgfqpoint{3.757503in}{3.811310in}}%
\pgfpathlineto{\pgfqpoint{3.759427in}{3.812634in}}%
\pgfpathlineto{\pgfqpoint{3.761350in}{3.806143in}}%
\pgfpathlineto{\pgfqpoint{3.763273in}{3.813036in}}%
\pgfpathlineto{\pgfqpoint{3.767120in}{3.836081in}}%
\pgfpathlineto{\pgfqpoint{3.769044in}{3.835571in}}%
\pgfpathlineto{\pgfqpoint{3.770967in}{3.839961in}}%
\pgfpathlineto{\pgfqpoint{3.776737in}{3.875382in}}%
\pgfpathlineto{\pgfqpoint{3.780584in}{3.848050in}}%
\pgfpathlineto{\pgfqpoint{3.782508in}{3.849261in}}%
\pgfpathlineto{\pgfqpoint{3.784431in}{3.833700in}}%
\pgfpathlineto{\pgfqpoint{3.786354in}{3.827797in}}%
\pgfpathlineto{\pgfqpoint{3.790201in}{3.835332in}}%
\pgfpathlineto{\pgfqpoint{3.792125in}{3.840565in}}%
\pgfpathlineto{\pgfqpoint{3.794048in}{3.840085in}}%
\pgfpathlineto{\pgfqpoint{3.795971in}{3.837578in}}%
\pgfpathlineto{\pgfqpoint{3.797895in}{3.823033in}}%
\pgfpathlineto{\pgfqpoint{3.799818in}{3.820512in}}%
\pgfpathlineto{\pgfqpoint{3.801742in}{3.831732in}}%
\pgfpathlineto{\pgfqpoint{3.803665in}{3.833167in}}%
\pgfpathlineto{\pgfqpoint{3.809435in}{3.859470in}}%
\pgfpathlineto{\pgfqpoint{3.811359in}{3.857284in}}%
\pgfpathlineto{\pgfqpoint{3.815205in}{3.872428in}}%
\pgfpathlineto{\pgfqpoint{3.817129in}{3.856199in}}%
\pgfpathlineto{\pgfqpoint{3.819052in}{3.865902in}}%
\pgfpathlineto{\pgfqpoint{3.820976in}{3.885748in}}%
\pgfpathlineto{\pgfqpoint{3.822899in}{3.868611in}}%
\pgfpathlineto{\pgfqpoint{3.824823in}{3.880692in}}%
\pgfpathlineto{\pgfqpoint{3.828669in}{3.851943in}}%
\pgfpathlineto{\pgfqpoint{3.832516in}{3.837484in}}%
\pgfpathlineto{\pgfqpoint{3.834440in}{3.836981in}}%
\pgfpathlineto{\pgfqpoint{3.838286in}{3.847901in}}%
\pgfpathlineto{\pgfqpoint{3.842133in}{3.851211in}}%
\pgfpathlineto{\pgfqpoint{3.844057in}{3.848872in}}%
\pgfpathlineto{\pgfqpoint{3.845980in}{3.863951in}}%
\pgfpathlineto{\pgfqpoint{3.847903in}{3.859369in}}%
\pgfpathlineto{\pgfqpoint{3.849827in}{3.859470in}}%
\pgfpathlineto{\pgfqpoint{3.851750in}{3.862138in}}%
\pgfpathlineto{\pgfqpoint{3.853674in}{3.861350in}}%
\pgfpathlineto{\pgfqpoint{3.855597in}{3.868948in}}%
\pgfpathlineto{\pgfqpoint{3.857521in}{3.860138in}}%
\pgfpathlineto{\pgfqpoint{3.861367in}{3.893597in}}%
\pgfpathlineto{\pgfqpoint{3.863291in}{3.896443in}}%
\pgfpathlineto{\pgfqpoint{3.867138in}{3.884115in}}%
\pgfpathlineto{\pgfqpoint{3.869061in}{3.888256in}}%
\pgfpathlineto{\pgfqpoint{3.870984in}{3.876363in}}%
\pgfpathlineto{\pgfqpoint{3.872908in}{3.875200in}}%
\pgfpathlineto{\pgfqpoint{3.874831in}{3.872501in}}%
\pgfpathlineto{\pgfqpoint{3.876755in}{3.881013in}}%
\pgfpathlineto{\pgfqpoint{3.878678in}{3.870680in}}%
\pgfpathlineto{\pgfqpoint{3.880601in}{3.882206in}}%
\pgfpathlineto{\pgfqpoint{3.882525in}{3.872935in}}%
\pgfpathlineto{\pgfqpoint{3.884448in}{3.882046in}}%
\pgfpathlineto{\pgfqpoint{3.886372in}{3.882641in}}%
\pgfpathlineto{\pgfqpoint{3.888295in}{3.887228in}}%
\pgfpathlineto{\pgfqpoint{3.890219in}{3.879820in}}%
\pgfpathlineto{\pgfqpoint{3.895989in}{3.907549in}}%
\pgfpathlineto{\pgfqpoint{3.897912in}{3.908786in}}%
\pgfpathlineto{\pgfqpoint{3.903682in}{3.935332in}}%
\pgfpathlineto{\pgfqpoint{3.905606in}{3.900459in}}%
\pgfpathlineto{\pgfqpoint{3.907529in}{3.909377in}}%
\pgfpathlineto{\pgfqpoint{3.909453in}{3.902301in}}%
\pgfpathlineto{\pgfqpoint{3.911376in}{3.926617in}}%
\pgfpathlineto{\pgfqpoint{3.913299in}{3.919436in}}%
\pgfpathlineto{\pgfqpoint{3.915223in}{3.920260in}}%
\pgfpathlineto{\pgfqpoint{3.919070in}{3.933744in}}%
\pgfpathlineto{\pgfqpoint{3.922917in}{3.928719in}}%
\pgfpathlineto{\pgfqpoint{3.924840in}{3.926270in}}%
\pgfpathlineto{\pgfqpoint{3.926763in}{3.915410in}}%
\pgfpathlineto{\pgfqpoint{3.930610in}{3.914317in}}%
\pgfpathlineto{\pgfqpoint{3.932534in}{3.890223in}}%
\pgfpathlineto{\pgfqpoint{3.934457in}{3.897448in}}%
\pgfpathlineto{\pgfqpoint{3.936380in}{3.895680in}}%
\pgfpathlineto{\pgfqpoint{3.944074in}{3.921296in}}%
\pgfpathlineto{\pgfqpoint{3.945997in}{3.920737in}}%
\pgfpathlineto{\pgfqpoint{3.947921in}{3.914990in}}%
\pgfpathlineto{\pgfqpoint{3.949844in}{3.915585in}}%
\pgfpathlineto{\pgfqpoint{3.951768in}{3.904404in}}%
\pgfpathlineto{\pgfqpoint{3.953691in}{3.904369in}}%
\pgfpathlineto{\pgfqpoint{3.959461in}{3.889123in}}%
\pgfpathlineto{\pgfqpoint{3.961385in}{3.883925in}}%
\pgfpathlineto{\pgfqpoint{3.963308in}{3.888000in}}%
\pgfpathlineto{\pgfqpoint{3.965232in}{3.873713in}}%
\pgfpathlineto{\pgfqpoint{3.967155in}{3.874814in}}%
\pgfpathlineto{\pgfqpoint{3.969078in}{3.871708in}}%
\pgfpathlineto{\pgfqpoint{3.972925in}{3.841640in}}%
\pgfpathlineto{\pgfqpoint{3.974849in}{3.844156in}}%
\pgfpathlineto{\pgfqpoint{3.978695in}{3.839320in}}%
\pgfpathlineto{\pgfqpoint{3.982542in}{3.856839in}}%
\pgfpathlineto{\pgfqpoint{3.984466in}{3.873388in}}%
\pgfpathlineto{\pgfqpoint{3.986389in}{3.876892in}}%
\pgfpathlineto{\pgfqpoint{3.988312in}{3.890466in}}%
\pgfpathlineto{\pgfqpoint{3.990236in}{3.893102in}}%
\pgfpathlineto{\pgfqpoint{3.992159in}{3.875298in}}%
\pgfpathlineto{\pgfqpoint{3.994083in}{3.880832in}}%
\pgfpathlineto{\pgfqpoint{3.996006in}{3.881065in}}%
\pgfpathlineto{\pgfqpoint{3.999853in}{3.861606in}}%
\pgfpathlineto{\pgfqpoint{4.001776in}{3.864809in}}%
\pgfpathlineto{\pgfqpoint{4.003700in}{3.862533in}}%
\pgfpathlineto{\pgfqpoint{4.005623in}{3.880766in}}%
\pgfpathlineto{\pgfqpoint{4.009470in}{3.823169in}}%
\pgfpathlineto{\pgfqpoint{4.011393in}{3.826039in}}%
\pgfpathlineto{\pgfqpoint{4.017164in}{3.779004in}}%
\pgfpathlineto{\pgfqpoint{4.021010in}{3.783425in}}%
\pgfpathlineto{\pgfqpoint{4.024857in}{3.772629in}}%
\pgfpathlineto{\pgfqpoint{4.026781in}{3.771619in}}%
\pgfpathlineto{\pgfqpoint{4.028704in}{3.777635in}}%
\pgfpathlineto{\pgfqpoint{4.030628in}{3.777679in}}%
\pgfpathlineto{\pgfqpoint{4.034474in}{3.753171in}}%
\pgfpathlineto{\pgfqpoint{4.036398in}{3.758757in}}%
\pgfpathlineto{\pgfqpoint{4.038321in}{3.750764in}}%
\pgfpathlineto{\pgfqpoint{4.040245in}{3.750010in}}%
\pgfpathlineto{\pgfqpoint{4.044091in}{3.728727in}}%
\pgfpathlineto{\pgfqpoint{4.046015in}{3.718764in}}%
\pgfpathlineto{\pgfqpoint{4.047938in}{3.722194in}}%
\pgfpathlineto{\pgfqpoint{4.049862in}{3.718806in}}%
\pgfpathlineto{\pgfqpoint{4.051785in}{3.710208in}}%
\pgfpathlineto{\pgfqpoint{4.055632in}{3.715412in}}%
\pgfpathlineto{\pgfqpoint{4.059479in}{3.693668in}}%
\pgfpathlineto{\pgfqpoint{4.061402in}{3.694045in}}%
\pgfpathlineto{\pgfqpoint{4.065249in}{3.680613in}}%
\pgfpathlineto{\pgfqpoint{4.067172in}{3.696178in}}%
\pgfpathlineto{\pgfqpoint{4.069096in}{3.690370in}}%
\pgfpathlineto{\pgfqpoint{4.071019in}{3.695122in}}%
\pgfpathlineto{\pgfqpoint{4.074866in}{3.698642in}}%
\pgfpathlineto{\pgfqpoint{4.076789in}{3.692881in}}%
\pgfpathlineto{\pgfqpoint{4.078713in}{3.696488in}}%
\pgfpathlineto{\pgfqpoint{4.080636in}{3.684482in}}%
\pgfpathlineto{\pgfqpoint{4.082560in}{3.686699in}}%
\pgfpathlineto{\pgfqpoint{4.084483in}{3.679341in}}%
\pgfpathlineto{\pgfqpoint{4.086406in}{3.679534in}}%
\pgfpathlineto{\pgfqpoint{4.088330in}{3.673257in}}%
\pgfpathlineto{\pgfqpoint{4.090253in}{3.687803in}}%
\pgfpathlineto{\pgfqpoint{4.092177in}{3.685573in}}%
\pgfpathlineto{\pgfqpoint{4.094100in}{3.695109in}}%
\pgfpathlineto{\pgfqpoint{4.096024in}{3.690433in}}%
\pgfpathlineto{\pgfqpoint{4.097947in}{3.689523in}}%
\pgfpathlineto{\pgfqpoint{4.099870in}{3.679159in}}%
\pgfpathlineto{\pgfqpoint{4.105641in}{3.716851in}}%
\pgfpathlineto{\pgfqpoint{4.107564in}{3.709204in}}%
\pgfpathlineto{\pgfqpoint{4.109487in}{3.716755in}}%
\pgfpathlineto{\pgfqpoint{4.111411in}{3.708015in}}%
\pgfpathlineto{\pgfqpoint{4.113334in}{3.717167in}}%
\pgfpathlineto{\pgfqpoint{4.115258in}{3.716301in}}%
\pgfpathlineto{\pgfqpoint{4.117181in}{3.710353in}}%
\pgfpathlineto{\pgfqpoint{4.121028in}{3.709098in}}%
\pgfpathlineto{\pgfqpoint{4.122951in}{3.714459in}}%
\pgfpathlineto{\pgfqpoint{4.124875in}{3.715830in}}%
\pgfpathlineto{\pgfqpoint{4.126798in}{3.705553in}}%
\pgfpathlineto{\pgfqpoint{4.128721in}{3.720805in}}%
\pgfpathlineto{\pgfqpoint{4.130645in}{3.723948in}}%
\pgfpathlineto{\pgfqpoint{4.134492in}{3.738979in}}%
\pgfpathlineto{\pgfqpoint{4.138339in}{3.729013in}}%
\pgfpathlineto{\pgfqpoint{4.140262in}{3.739452in}}%
\pgfpathlineto{\pgfqpoint{4.142185in}{3.725097in}}%
\pgfpathlineto{\pgfqpoint{4.144109in}{3.723161in}}%
\pgfpathlineto{\pgfqpoint{4.146032in}{3.735687in}}%
\pgfpathlineto{\pgfqpoint{4.151802in}{3.714938in}}%
\pgfpathlineto{\pgfqpoint{4.153726in}{3.713707in}}%
\pgfpathlineto{\pgfqpoint{4.155649in}{3.706356in}}%
\pgfpathlineto{\pgfqpoint{4.161419in}{3.670236in}}%
\pgfpathlineto{\pgfqpoint{4.165266in}{3.670909in}}%
\pgfpathlineto{\pgfqpoint{4.167190in}{3.666755in}}%
\pgfpathlineto{\pgfqpoint{4.169113in}{3.667974in}}%
\pgfpathlineto{\pgfqpoint{4.171037in}{3.638550in}}%
\pgfpathlineto{\pgfqpoint{4.174883in}{3.626595in}}%
\pgfpathlineto{\pgfqpoint{4.176807in}{3.626696in}}%
\pgfpathlineto{\pgfqpoint{4.178730in}{3.621928in}}%
\pgfpathlineto{\pgfqpoint{4.180654in}{3.629076in}}%
\pgfpathlineto{\pgfqpoint{4.186424in}{3.618354in}}%
\pgfpathlineto{\pgfqpoint{4.188347in}{3.630474in}}%
\pgfpathlineto{\pgfqpoint{4.190271in}{3.633406in}}%
\pgfpathlineto{\pgfqpoint{4.194117in}{3.631204in}}%
\pgfpathlineto{\pgfqpoint{4.197964in}{3.659655in}}%
\pgfpathlineto{\pgfqpoint{4.199888in}{3.657426in}}%
\pgfpathlineto{\pgfqpoint{4.201811in}{3.665235in}}%
\pgfpathlineto{\pgfqpoint{4.203735in}{3.667890in}}%
\pgfpathlineto{\pgfqpoint{4.205658in}{3.676478in}}%
\pgfpathlineto{\pgfqpoint{4.209505in}{3.708904in}}%
\pgfpathlineto{\pgfqpoint{4.211428in}{3.719274in}}%
\pgfpathlineto{\pgfqpoint{4.213352in}{3.717636in}}%
\pgfpathlineto{\pgfqpoint{4.215275in}{3.713706in}}%
\pgfpathlineto{\pgfqpoint{4.217198in}{3.715757in}}%
\pgfpathlineto{\pgfqpoint{4.219122in}{3.710809in}}%
\pgfpathlineto{\pgfqpoint{4.221045in}{3.700224in}}%
\pgfpathlineto{\pgfqpoint{4.224892in}{3.701358in}}%
\pgfpathlineto{\pgfqpoint{4.226815in}{3.698949in}}%
\pgfpathlineto{\pgfqpoint{4.228739in}{3.683855in}}%
\pgfpathlineto{\pgfqpoint{4.230662in}{3.690663in}}%
\pgfpathlineto{\pgfqpoint{4.232586in}{3.691891in}}%
\pgfpathlineto{\pgfqpoint{4.234509in}{3.687769in}}%
\pgfpathlineto{\pgfqpoint{4.236433in}{3.693144in}}%
\pgfpathlineto{\pgfqpoint{4.238356in}{3.690467in}}%
\pgfpathlineto{\pgfqpoint{4.240279in}{3.685464in}}%
\pgfpathlineto{\pgfqpoint{4.242203in}{3.690735in}}%
\pgfpathlineto{\pgfqpoint{4.244126in}{3.682897in}}%
\pgfpathlineto{\pgfqpoint{4.247973in}{3.689204in}}%
\pgfpathlineto{\pgfqpoint{4.249896in}{3.708340in}}%
\pgfpathlineto{\pgfqpoint{4.251820in}{3.711804in}}%
\pgfpathlineto{\pgfqpoint{4.255667in}{3.713294in}}%
\pgfpathlineto{\pgfqpoint{4.257590in}{3.716639in}}%
\pgfpathlineto{\pgfqpoint{4.261437in}{3.694809in}}%
\pgfpathlineto{\pgfqpoint{4.263360in}{3.697673in}}%
\pgfpathlineto{\pgfqpoint{4.265284in}{3.687511in}}%
\pgfpathlineto{\pgfqpoint{4.267207in}{3.690331in}}%
\pgfpathlineto{\pgfqpoint{4.269130in}{3.700605in}}%
\pgfpathlineto{\pgfqpoint{4.271054in}{3.682322in}}%
\pgfpathlineto{\pgfqpoint{4.272977in}{3.678628in}}%
\pgfpathlineto{\pgfqpoint{4.276824in}{3.658950in}}%
\pgfpathlineto{\pgfqpoint{4.278748in}{3.666282in}}%
\pgfpathlineto{\pgfqpoint{4.282594in}{3.637556in}}%
\pgfpathlineto{\pgfqpoint{4.284518in}{3.615761in}}%
\pgfpathlineto{\pgfqpoint{4.286441in}{3.624413in}}%
\pgfpathlineto{\pgfqpoint{4.290288in}{3.605209in}}%
\pgfpathlineto{\pgfqpoint{4.292211in}{3.620677in}}%
\pgfpathlineto{\pgfqpoint{4.294135in}{3.611436in}}%
\pgfpathlineto{\pgfqpoint{4.296058in}{3.612950in}}%
\pgfpathlineto{\pgfqpoint{4.297982in}{3.607489in}}%
\pgfpathlineto{\pgfqpoint{4.301828in}{3.584415in}}%
\pgfpathlineto{\pgfqpoint{4.303752in}{3.584038in}}%
\pgfpathlineto{\pgfqpoint{4.305675in}{3.606478in}}%
\pgfpathlineto{\pgfqpoint{4.309522in}{3.621660in}}%
\pgfpathlineto{\pgfqpoint{4.311446in}{3.621840in}}%
\pgfpathlineto{\pgfqpoint{4.313369in}{3.645317in}}%
\pgfpathlineto{\pgfqpoint{4.317216in}{3.633777in}}%
\pgfpathlineto{\pgfqpoint{4.321063in}{3.602315in}}%
\pgfpathlineto{\pgfqpoint{4.322986in}{3.605437in}}%
\pgfpathlineto{\pgfqpoint{4.326833in}{3.635535in}}%
\pgfpathlineto{\pgfqpoint{4.330680in}{3.615019in}}%
\pgfpathlineto{\pgfqpoint{4.332603in}{3.615543in}}%
\pgfpathlineto{\pgfqpoint{4.334526in}{3.618966in}}%
\pgfpathlineto{\pgfqpoint{4.338373in}{3.602376in}}%
\pgfpathlineto{\pgfqpoint{4.342220in}{3.564742in}}%
\pgfpathlineto{\pgfqpoint{4.344144in}{3.552476in}}%
\pgfpathlineto{\pgfqpoint{4.347990in}{3.560448in}}%
\pgfpathlineto{\pgfqpoint{4.351837in}{3.572639in}}%
\pgfpathlineto{\pgfqpoint{4.353761in}{3.571547in}}%
\pgfpathlineto{\pgfqpoint{4.355684in}{3.572267in}}%
\pgfpathlineto{\pgfqpoint{4.357607in}{3.570676in}}%
\pgfpathlineto{\pgfqpoint{4.359531in}{3.567548in}}%
\pgfpathlineto{\pgfqpoint{4.361454in}{3.569406in}}%
\pgfpathlineto{\pgfqpoint{4.363378in}{3.566152in}}%
\pgfpathlineto{\pgfqpoint{4.367224in}{3.553691in}}%
\pgfpathlineto{\pgfqpoint{4.369148in}{3.555898in}}%
\pgfpathlineto{\pgfqpoint{4.371071in}{3.549794in}}%
\pgfpathlineto{\pgfqpoint{4.372995in}{3.570628in}}%
\pgfpathlineto{\pgfqpoint{4.376842in}{3.549843in}}%
\pgfpathlineto{\pgfqpoint{4.378765in}{3.549437in}}%
\pgfpathlineto{\pgfqpoint{4.380688in}{3.527493in}}%
\pgfpathlineto{\pgfqpoint{4.382612in}{3.529862in}}%
\pgfpathlineto{\pgfqpoint{4.384535in}{3.518062in}}%
\pgfpathlineto{\pgfqpoint{4.386459in}{3.523311in}}%
\pgfpathlineto{\pgfqpoint{4.390305in}{3.517004in}}%
\pgfpathlineto{\pgfqpoint{4.394152in}{3.545803in}}%
\pgfpathlineto{\pgfqpoint{4.396076in}{3.539152in}}%
\pgfpathlineto{\pgfqpoint{4.397999in}{3.537263in}}%
\pgfpathlineto{\pgfqpoint{4.399922in}{3.538985in}}%
\pgfpathlineto{\pgfqpoint{4.401846in}{3.545958in}}%
\pgfpathlineto{\pgfqpoint{4.405693in}{3.531110in}}%
\pgfpathlineto{\pgfqpoint{4.407616in}{3.521112in}}%
\pgfpathlineto{\pgfqpoint{4.409539in}{3.522356in}}%
\pgfpathlineto{\pgfqpoint{4.411463in}{3.506188in}}%
\pgfpathlineto{\pgfqpoint{4.413386in}{3.512349in}}%
\pgfpathlineto{\pgfqpoint{4.415310in}{3.506713in}}%
\pgfpathlineto{\pgfqpoint{4.417233in}{3.506930in}}%
\pgfpathlineto{\pgfqpoint{4.419157in}{3.495994in}}%
\pgfpathlineto{\pgfqpoint{4.421080in}{3.491579in}}%
\pgfpathlineto{\pgfqpoint{4.426850in}{3.453320in}}%
\pgfpathlineto{\pgfqpoint{4.428774in}{3.454732in}}%
\pgfpathlineto{\pgfqpoint{4.430697in}{3.458363in}}%
\pgfpathlineto{\pgfqpoint{4.432620in}{3.459547in}}%
\pgfpathlineto{\pgfqpoint{4.434544in}{3.451573in}}%
\pgfpathlineto{\pgfqpoint{4.436467in}{3.459230in}}%
\pgfpathlineto{\pgfqpoint{4.438391in}{3.451819in}}%
\pgfpathlineto{\pgfqpoint{4.440314in}{3.457761in}}%
\pgfpathlineto{\pgfqpoint{4.444161in}{3.456291in}}%
\pgfpathlineto{\pgfqpoint{4.446084in}{3.447871in}}%
\pgfpathlineto{\pgfqpoint{4.448008in}{3.447579in}}%
\pgfpathlineto{\pgfqpoint{4.451855in}{3.430444in}}%
\pgfpathlineto{\pgfqpoint{4.455701in}{3.455325in}}%
\pgfpathlineto{\pgfqpoint{4.459548in}{3.443158in}}%
\pgfpathlineto{\pgfqpoint{4.463395in}{3.434535in}}%
\pgfpathlineto{\pgfqpoint{4.465318in}{3.443645in}}%
\pgfpathlineto{\pgfqpoint{4.467242in}{3.434943in}}%
\pgfpathlineto{\pgfqpoint{4.469165in}{3.442117in}}%
\pgfpathlineto{\pgfqpoint{4.471089in}{3.437674in}}%
\pgfpathlineto{\pgfqpoint{4.474935in}{3.422348in}}%
\pgfpathlineto{\pgfqpoint{4.478782in}{3.409745in}}%
\pgfpathlineto{\pgfqpoint{4.480706in}{3.416243in}}%
\pgfpathlineto{\pgfqpoint{4.482629in}{3.416969in}}%
\pgfpathlineto{\pgfqpoint{4.484553in}{3.420489in}}%
\pgfpathlineto{\pgfqpoint{4.486476in}{3.413827in}}%
\pgfpathlineto{\pgfqpoint{4.488399in}{3.415223in}}%
\pgfpathlineto{\pgfqpoint{4.490323in}{3.420657in}}%
\pgfpathlineto{\pgfqpoint{4.492246in}{3.407372in}}%
\pgfpathlineto{\pgfqpoint{4.494170in}{3.411550in}}%
\pgfpathlineto{\pgfqpoint{4.499940in}{3.377186in}}%
\pgfpathlineto{\pgfqpoint{4.501863in}{3.375895in}}%
\pgfpathlineto{\pgfqpoint{4.503787in}{3.372619in}}%
\pgfpathlineto{\pgfqpoint{4.505710in}{3.395995in}}%
\pgfpathlineto{\pgfqpoint{4.509557in}{3.387642in}}%
\pgfpathlineto{\pgfqpoint{4.511480in}{3.402247in}}%
\pgfpathlineto{\pgfqpoint{4.513404in}{3.399165in}}%
\pgfpathlineto{\pgfqpoint{4.515327in}{3.401083in}}%
\pgfpathlineto{\pgfqpoint{4.517251in}{3.399018in}}%
\pgfpathlineto{\pgfqpoint{4.519174in}{3.389476in}}%
\pgfpathlineto{\pgfqpoint{4.521097in}{3.388217in}}%
\pgfpathlineto{\pgfqpoint{4.523021in}{3.384300in}}%
\pgfpathlineto{\pgfqpoint{4.524944in}{3.383453in}}%
\pgfpathlineto{\pgfqpoint{4.528791in}{3.358332in}}%
\pgfpathlineto{\pgfqpoint{4.530714in}{3.359846in}}%
\pgfpathlineto{\pgfqpoint{4.532638in}{3.371370in}}%
\pgfpathlineto{\pgfqpoint{4.538408in}{3.380994in}}%
\pgfpathlineto{\pgfqpoint{4.544178in}{3.414000in}}%
\pgfpathlineto{\pgfqpoint{4.546102in}{3.417203in}}%
\pgfpathlineto{\pgfqpoint{4.548025in}{3.429210in}}%
\pgfpathlineto{\pgfqpoint{4.549949in}{3.420715in}}%
\pgfpathlineto{\pgfqpoint{4.551872in}{3.421601in}}%
\pgfpathlineto{\pgfqpoint{4.553795in}{3.418542in}}%
\pgfpathlineto{\pgfqpoint{4.555719in}{3.407279in}}%
\pgfpathlineto{\pgfqpoint{4.557642in}{3.409852in}}%
\pgfpathlineto{\pgfqpoint{4.559566in}{3.389887in}}%
\pgfpathlineto{\pgfqpoint{4.561489in}{3.404058in}}%
\pgfpathlineto{\pgfqpoint{4.563412in}{3.399331in}}%
\pgfpathlineto{\pgfqpoint{4.565336in}{3.401977in}}%
\pgfpathlineto{\pgfqpoint{4.569183in}{3.371922in}}%
\pgfpathlineto{\pgfqpoint{4.571106in}{3.379459in}}%
\pgfpathlineto{\pgfqpoint{4.573029in}{3.377451in}}%
\pgfpathlineto{\pgfqpoint{4.576876in}{3.361551in}}%
\pgfpathlineto{\pgfqpoint{4.582646in}{3.339087in}}%
\pgfpathlineto{\pgfqpoint{4.584570in}{3.342176in}}%
\pgfpathlineto{\pgfqpoint{4.586493in}{3.357297in}}%
\pgfpathlineto{\pgfqpoint{4.588417in}{3.358564in}}%
\pgfpathlineto{\pgfqpoint{4.590340in}{3.366123in}}%
\pgfpathlineto{\pgfqpoint{4.592264in}{3.363579in}}%
\pgfpathlineto{\pgfqpoint{4.594187in}{3.349225in}}%
\pgfpathlineto{\pgfqpoint{4.596110in}{3.355817in}}%
\pgfpathlineto{\pgfqpoint{4.599957in}{3.336333in}}%
\pgfpathlineto{\pgfqpoint{4.603804in}{3.359860in}}%
\pgfpathlineto{\pgfqpoint{4.607651in}{3.371742in}}%
\pgfpathlineto{\pgfqpoint{4.609574in}{3.368271in}}%
\pgfpathlineto{\pgfqpoint{4.611498in}{3.369486in}}%
\pgfpathlineto{\pgfqpoint{4.615344in}{3.351137in}}%
\pgfpathlineto{\pgfqpoint{4.617268in}{3.344905in}}%
\pgfpathlineto{\pgfqpoint{4.621115in}{3.349983in}}%
\pgfpathlineto{\pgfqpoint{4.623038in}{3.337930in}}%
\pgfpathlineto{\pgfqpoint{4.626885in}{3.341806in}}%
\pgfpathlineto{\pgfqpoint{4.628808in}{3.340749in}}%
\pgfpathlineto{\pgfqpoint{4.632655in}{3.325947in}}%
\pgfpathlineto{\pgfqpoint{4.634579in}{3.327766in}}%
\pgfpathlineto{\pgfqpoint{4.636502in}{3.308011in}}%
\pgfpathlineto{\pgfqpoint{4.640349in}{3.310701in}}%
\pgfpathlineto{\pgfqpoint{4.642272in}{3.308120in}}%
\pgfpathlineto{\pgfqpoint{4.644196in}{3.292099in}}%
\pgfpathlineto{\pgfqpoint{4.646119in}{3.297198in}}%
\pgfpathlineto{\pgfqpoint{4.648042in}{3.297349in}}%
\pgfpathlineto{\pgfqpoint{4.649966in}{3.291863in}}%
\pgfpathlineto{\pgfqpoint{4.651889in}{3.290284in}}%
\pgfpathlineto{\pgfqpoint{4.653813in}{3.286687in}}%
\pgfpathlineto{\pgfqpoint{4.655736in}{3.307344in}}%
\pgfpathlineto{\pgfqpoint{4.657660in}{3.304502in}}%
\pgfpathlineto{\pgfqpoint{4.661506in}{3.315295in}}%
\pgfpathlineto{\pgfqpoint{4.663430in}{3.315578in}}%
\pgfpathlineto{\pgfqpoint{4.669200in}{3.311009in}}%
\pgfpathlineto{\pgfqpoint{4.671123in}{3.319134in}}%
\pgfpathlineto{\pgfqpoint{4.673047in}{3.316877in}}%
\pgfpathlineto{\pgfqpoint{4.674970in}{3.320859in}}%
\pgfpathlineto{\pgfqpoint{4.676894in}{3.316723in}}%
\pgfpathlineto{\pgfqpoint{4.678817in}{3.304606in}}%
\pgfpathlineto{\pgfqpoint{4.680740in}{3.321409in}}%
\pgfpathlineto{\pgfqpoint{4.682664in}{3.322328in}}%
\pgfpathlineto{\pgfqpoint{4.688434in}{3.354430in}}%
\pgfpathlineto{\pgfqpoint{4.690358in}{3.355376in}}%
\pgfpathlineto{\pgfqpoint{4.692281in}{3.372102in}}%
\pgfpathlineto{\pgfqpoint{4.694204in}{3.370168in}}%
\pgfpathlineto{\pgfqpoint{4.696128in}{3.371471in}}%
\pgfpathlineto{\pgfqpoint{4.698051in}{3.364986in}}%
\pgfpathlineto{\pgfqpoint{4.703821in}{3.376525in}}%
\pgfpathlineto{\pgfqpoint{4.705745in}{3.389670in}}%
\pgfpathlineto{\pgfqpoint{4.707668in}{3.386298in}}%
\pgfpathlineto{\pgfqpoint{4.709592in}{3.403311in}}%
\pgfpathlineto{\pgfqpoint{4.715362in}{3.376090in}}%
\pgfpathlineto{\pgfqpoint{4.717285in}{3.372308in}}%
\pgfpathlineto{\pgfqpoint{4.719209in}{3.372012in}}%
\pgfpathlineto{\pgfqpoint{4.721132in}{3.374472in}}%
\pgfpathlineto{\pgfqpoint{4.726902in}{3.365880in}}%
\pgfpathlineto{\pgfqpoint{4.728826in}{3.374772in}}%
\pgfpathlineto{\pgfqpoint{4.730749in}{3.367293in}}%
\pgfpathlineto{\pgfqpoint{4.732673in}{3.368166in}}%
\pgfpathlineto{\pgfqpoint{4.736519in}{3.347748in}}%
\pgfpathlineto{\pgfqpoint{4.738443in}{3.357763in}}%
\pgfpathlineto{\pgfqpoint{4.740366in}{3.346646in}}%
\pgfpathlineto{\pgfqpoint{4.742290in}{3.347179in}}%
\pgfpathlineto{\pgfqpoint{4.744213in}{3.354474in}}%
\pgfpathlineto{\pgfqpoint{4.746136in}{3.354859in}}%
\pgfpathlineto{\pgfqpoint{4.748060in}{3.338827in}}%
\pgfpathlineto{\pgfqpoint{4.749983in}{3.350270in}}%
\pgfpathlineto{\pgfqpoint{4.751907in}{3.340921in}}%
\pgfpathlineto{\pgfqpoint{4.755753in}{3.351452in}}%
\pgfpathlineto{\pgfqpoint{4.757677in}{3.348516in}}%
\pgfpathlineto{\pgfqpoint{4.759600in}{3.351117in}}%
\pgfpathlineto{\pgfqpoint{4.761524in}{3.355948in}}%
\pgfpathlineto{\pgfqpoint{4.765371in}{3.378040in}}%
\pgfpathlineto{\pgfqpoint{4.767294in}{3.371917in}}%
\pgfpathlineto{\pgfqpoint{4.769217in}{3.369964in}}%
\pgfpathlineto{\pgfqpoint{4.776911in}{3.417854in}}%
\pgfpathlineto{\pgfqpoint{4.778834in}{3.414136in}}%
\pgfpathlineto{\pgfqpoint{4.780758in}{3.397985in}}%
\pgfpathlineto{\pgfqpoint{4.782681in}{3.403961in}}%
\pgfpathlineto{\pgfqpoint{4.784605in}{3.413851in}}%
\pgfpathlineto{\pgfqpoint{4.788451in}{3.390205in}}%
\pgfpathlineto{\pgfqpoint{4.790375in}{3.388536in}}%
\pgfpathlineto{\pgfqpoint{4.792298in}{3.397827in}}%
\pgfpathlineto{\pgfqpoint{4.794222in}{3.397598in}}%
\pgfpathlineto{\pgfqpoint{4.796145in}{3.405622in}}%
\pgfpathlineto{\pgfqpoint{4.798069in}{3.399366in}}%
\pgfpathlineto{\pgfqpoint{4.801915in}{3.414406in}}%
\pgfpathlineto{\pgfqpoint{4.803839in}{3.408854in}}%
\pgfpathlineto{\pgfqpoint{4.805762in}{3.423094in}}%
\pgfpathlineto{\pgfqpoint{4.809609in}{3.421877in}}%
\pgfpathlineto{\pgfqpoint{4.811532in}{3.427144in}}%
\pgfpathlineto{\pgfqpoint{4.813456in}{3.422267in}}%
\pgfpathlineto{\pgfqpoint{4.815379in}{3.421706in}}%
\pgfpathlineto{\pgfqpoint{4.823073in}{3.399563in}}%
\pgfpathlineto{\pgfqpoint{4.824996in}{3.400532in}}%
\pgfpathlineto{\pgfqpoint{4.826920in}{3.396802in}}%
\pgfpathlineto{\pgfqpoint{4.828843in}{3.409819in}}%
\pgfpathlineto{\pgfqpoint{4.830767in}{3.407561in}}%
\pgfpathlineto{\pgfqpoint{4.834613in}{3.434826in}}%
\pgfpathlineto{\pgfqpoint{4.836537in}{3.421997in}}%
\pgfpathlineto{\pgfqpoint{4.838460in}{3.418139in}}%
\pgfpathlineto{\pgfqpoint{4.840384in}{3.418845in}}%
\pgfpathlineto{\pgfqpoint{4.842307in}{3.421293in}}%
\pgfpathlineto{\pgfqpoint{4.844230in}{3.420130in}}%
\pgfpathlineto{\pgfqpoint{4.846154in}{3.414507in}}%
\pgfpathlineto{\pgfqpoint{4.848077in}{3.416523in}}%
\pgfpathlineto{\pgfqpoint{4.850001in}{3.420986in}}%
\pgfpathlineto{\pgfqpoint{4.853847in}{3.445601in}}%
\pgfpathlineto{\pgfqpoint{4.855771in}{3.447633in}}%
\pgfpathlineto{\pgfqpoint{4.857694in}{3.458152in}}%
\pgfpathlineto{\pgfqpoint{4.859618in}{3.459858in}}%
\pgfpathlineto{\pgfqpoint{4.861541in}{3.477910in}}%
\pgfpathlineto{\pgfqpoint{4.863464in}{3.470712in}}%
\pgfpathlineto{\pgfqpoint{4.865388in}{3.454827in}}%
\pgfpathlineto{\pgfqpoint{4.867311in}{3.461156in}}%
\pgfpathlineto{\pgfqpoint{4.869235in}{3.455749in}}%
\pgfpathlineto{\pgfqpoint{4.871158in}{3.463622in}}%
\pgfpathlineto{\pgfqpoint{4.873082in}{3.480483in}}%
\pgfpathlineto{\pgfqpoint{4.875005in}{3.479813in}}%
\pgfpathlineto{\pgfqpoint{4.878852in}{3.496894in}}%
\pgfpathlineto{\pgfqpoint{4.880775in}{3.501856in}}%
\pgfpathlineto{\pgfqpoint{4.882699in}{3.510009in}}%
\pgfpathlineto{\pgfqpoint{4.884622in}{3.510216in}}%
\pgfpathlineto{\pgfqpoint{4.890392in}{3.483953in}}%
\pgfpathlineto{\pgfqpoint{4.892316in}{3.463888in}}%
\pgfpathlineto{\pgfqpoint{4.894239in}{3.473971in}}%
\pgfpathlineto{\pgfqpoint{4.896162in}{3.476203in}}%
\pgfpathlineto{\pgfqpoint{4.898086in}{3.471858in}}%
\pgfpathlineto{\pgfqpoint{4.900009in}{3.474207in}}%
\pgfpathlineto{\pgfqpoint{4.901933in}{3.471341in}}%
\pgfpathlineto{\pgfqpoint{4.903856in}{3.464377in}}%
\pgfpathlineto{\pgfqpoint{4.905780in}{3.470109in}}%
\pgfpathlineto{\pgfqpoint{4.907703in}{3.467492in}}%
\pgfpathlineto{\pgfqpoint{4.909626in}{3.467756in}}%
\pgfpathlineto{\pgfqpoint{4.911550in}{3.449616in}}%
\pgfpathlineto{\pgfqpoint{4.915397in}{3.454910in}}%
\pgfpathlineto{\pgfqpoint{4.917320in}{3.467338in}}%
\pgfpathlineto{\pgfqpoint{4.921167in}{3.456969in}}%
\pgfpathlineto{\pgfqpoint{4.923090in}{3.439688in}}%
\pgfpathlineto{\pgfqpoint{4.925014in}{3.447587in}}%
\pgfpathlineto{\pgfqpoint{4.926937in}{3.449208in}}%
\pgfpathlineto{\pgfqpoint{4.932707in}{3.424216in}}%
\pgfpathlineto{\pgfqpoint{4.934631in}{3.439405in}}%
\pgfpathlineto{\pgfqpoint{4.940401in}{3.422338in}}%
\pgfpathlineto{\pgfqpoint{4.942324in}{3.423654in}}%
\pgfpathlineto{\pgfqpoint{4.944248in}{3.417142in}}%
\pgfpathlineto{\pgfqpoint{4.946171in}{3.415292in}}%
\pgfpathlineto{\pgfqpoint{4.948095in}{3.406333in}}%
\pgfpathlineto{\pgfqpoint{4.950018in}{3.404449in}}%
\pgfpathlineto{\pgfqpoint{4.953865in}{3.388933in}}%
\pgfpathlineto{\pgfqpoint{4.955788in}{3.372876in}}%
\pgfpathlineto{\pgfqpoint{4.957712in}{3.374815in}}%
\pgfpathlineto{\pgfqpoint{4.961558in}{3.353157in}}%
\pgfpathlineto{\pgfqpoint{4.963482in}{3.347981in}}%
\pgfpathlineto{\pgfqpoint{4.967329in}{3.335103in}}%
\pgfpathlineto{\pgfqpoint{4.973099in}{3.348895in}}%
\pgfpathlineto{\pgfqpoint{4.976946in}{3.372332in}}%
\pgfpathlineto{\pgfqpoint{4.978869in}{3.383064in}}%
\pgfpathlineto{\pgfqpoint{4.980793in}{3.385481in}}%
\pgfpathlineto{\pgfqpoint{4.982716in}{3.385969in}}%
\pgfpathlineto{\pgfqpoint{4.984639in}{3.388589in}}%
\pgfpathlineto{\pgfqpoint{4.986563in}{3.383517in}}%
\pgfpathlineto{\pgfqpoint{4.990410in}{3.380136in}}%
\pgfpathlineto{\pgfqpoint{4.992333in}{3.376422in}}%
\pgfpathlineto{\pgfqpoint{4.994256in}{3.386873in}}%
\pgfpathlineto{\pgfqpoint{4.996180in}{3.378945in}}%
\pgfpathlineto{\pgfqpoint{4.998103in}{3.355681in}}%
\pgfpathlineto{\pgfqpoint{5.000027in}{3.356122in}}%
\pgfpathlineto{\pgfqpoint{5.003874in}{3.359808in}}%
\pgfpathlineto{\pgfqpoint{5.007720in}{3.335896in}}%
\pgfpathlineto{\pgfqpoint{5.009644in}{3.337372in}}%
\pgfpathlineto{\pgfqpoint{5.011567in}{3.332881in}}%
\pgfpathlineto{\pgfqpoint{5.017337in}{3.351057in}}%
\pgfpathlineto{\pgfqpoint{5.021184in}{3.372709in}}%
\pgfpathlineto{\pgfqpoint{5.025031in}{3.346332in}}%
\pgfpathlineto{\pgfqpoint{5.026954in}{3.351964in}}%
\pgfpathlineto{\pgfqpoint{5.028878in}{3.350875in}}%
\pgfpathlineto{\pgfqpoint{5.032725in}{3.328235in}}%
\pgfpathlineto{\pgfqpoint{5.034648in}{3.334896in}}%
\pgfpathlineto{\pgfqpoint{5.038495in}{3.316853in}}%
\pgfpathlineto{\pgfqpoint{5.040418in}{3.319361in}}%
\pgfpathlineto{\pgfqpoint{5.042342in}{3.319881in}}%
\pgfpathlineto{\pgfqpoint{5.044265in}{3.305956in}}%
\pgfpathlineto{\pgfqpoint{5.046189in}{3.311665in}}%
\pgfpathlineto{\pgfqpoint{5.048112in}{3.300275in}}%
\pgfpathlineto{\pgfqpoint{5.050035in}{3.301876in}}%
\pgfpathlineto{\pgfqpoint{5.051959in}{3.296722in}}%
\pgfpathlineto{\pgfqpoint{5.053882in}{3.305008in}}%
\pgfpathlineto{\pgfqpoint{5.055806in}{3.304368in}}%
\pgfpathlineto{\pgfqpoint{5.057729in}{3.294916in}}%
\pgfpathlineto{\pgfqpoint{5.059652in}{3.305646in}}%
\pgfpathlineto{\pgfqpoint{5.061576in}{3.284708in}}%
\pgfpathlineto{\pgfqpoint{5.065423in}{3.319034in}}%
\pgfpathlineto{\pgfqpoint{5.067346in}{3.319088in}}%
\pgfpathlineto{\pgfqpoint{5.069269in}{3.316372in}}%
\pgfpathlineto{\pgfqpoint{5.071193in}{3.317427in}}%
\pgfpathlineto{\pgfqpoint{5.073116in}{3.323196in}}%
\pgfpathlineto{\pgfqpoint{5.076963in}{3.308166in}}%
\pgfpathlineto{\pgfqpoint{5.078887in}{3.309309in}}%
\pgfpathlineto{\pgfqpoint{5.080810in}{3.294973in}}%
\pgfpathlineto{\pgfqpoint{5.082733in}{3.292435in}}%
\pgfpathlineto{\pgfqpoint{5.084657in}{3.286508in}}%
\pgfpathlineto{\pgfqpoint{5.086580in}{3.296526in}}%
\pgfpathlineto{\pgfqpoint{5.088504in}{3.294574in}}%
\pgfpathlineto{\pgfqpoint{5.092350in}{3.298933in}}%
\pgfpathlineto{\pgfqpoint{5.094274in}{3.298254in}}%
\pgfpathlineto{\pgfqpoint{5.096197in}{3.281599in}}%
\pgfpathlineto{\pgfqpoint{5.098121in}{3.275455in}}%
\pgfpathlineto{\pgfqpoint{5.100044in}{3.278048in}}%
\pgfpathlineto{\pgfqpoint{5.103891in}{3.298913in}}%
\pgfpathlineto{\pgfqpoint{5.105814in}{3.299021in}}%
\pgfpathlineto{\pgfqpoint{5.107738in}{3.301771in}}%
\pgfpathlineto{\pgfqpoint{5.109661in}{3.318916in}}%
\pgfpathlineto{\pgfqpoint{5.111585in}{3.314281in}}%
\pgfpathlineto{\pgfqpoint{5.113508in}{3.314416in}}%
\pgfpathlineto{\pgfqpoint{5.115431in}{3.311616in}}%
\pgfpathlineto{\pgfqpoint{5.117355in}{3.315086in}}%
\pgfpathlineto{\pgfqpoint{5.119278in}{3.309211in}}%
\pgfpathlineto{\pgfqpoint{5.121202in}{3.310201in}}%
\pgfpathlineto{\pgfqpoint{5.126972in}{3.291910in}}%
\pgfpathlineto{\pgfqpoint{5.130819in}{3.319064in}}%
\pgfpathlineto{\pgfqpoint{5.132742in}{3.326193in}}%
\pgfpathlineto{\pgfqpoint{5.134665in}{3.322468in}}%
\pgfpathlineto{\pgfqpoint{5.138512in}{3.301061in}}%
\pgfpathlineto{\pgfqpoint{5.140436in}{3.301602in}}%
\pgfpathlineto{\pgfqpoint{5.142359in}{3.291732in}}%
\pgfpathlineto{\pgfqpoint{5.144283in}{3.295317in}}%
\pgfpathlineto{\pgfqpoint{5.146206in}{3.313017in}}%
\pgfpathlineto{\pgfqpoint{5.148129in}{3.315730in}}%
\pgfpathlineto{\pgfqpoint{5.150053in}{3.321705in}}%
\pgfpathlineto{\pgfqpoint{5.151976in}{3.319916in}}%
\pgfpathlineto{\pgfqpoint{5.153900in}{3.300805in}}%
\pgfpathlineto{\pgfqpoint{5.155823in}{3.300150in}}%
\pgfpathlineto{\pgfqpoint{5.157746in}{3.305418in}}%
\pgfpathlineto{\pgfqpoint{5.159670in}{3.304466in}}%
\pgfpathlineto{\pgfqpoint{5.161593in}{3.314783in}}%
\pgfpathlineto{\pgfqpoint{5.165440in}{3.306784in}}%
\pgfpathlineto{\pgfqpoint{5.167363in}{3.303211in}}%
\pgfpathlineto{\pgfqpoint{5.169287in}{3.284261in}}%
\pgfpathlineto{\pgfqpoint{5.173134in}{3.295261in}}%
\pgfpathlineto{\pgfqpoint{5.175057in}{3.294293in}}%
\pgfpathlineto{\pgfqpoint{5.176980in}{3.306575in}}%
\pgfpathlineto{\pgfqpoint{5.178904in}{3.296366in}}%
\pgfpathlineto{\pgfqpoint{5.180827in}{3.299628in}}%
\pgfpathlineto{\pgfqpoint{5.182751in}{3.295056in}}%
\pgfpathlineto{\pgfqpoint{5.184674in}{3.325226in}}%
\pgfpathlineto{\pgfqpoint{5.186598in}{3.325959in}}%
\pgfpathlineto{\pgfqpoint{5.188521in}{3.312891in}}%
\pgfpathlineto{\pgfqpoint{5.190444in}{3.313031in}}%
\pgfpathlineto{\pgfqpoint{5.192368in}{3.309908in}}%
\pgfpathlineto{\pgfqpoint{5.194291in}{3.309197in}}%
\pgfpathlineto{\pgfqpoint{5.196215in}{3.312429in}}%
\pgfpathlineto{\pgfqpoint{5.198138in}{3.310557in}}%
\pgfpathlineto{\pgfqpoint{5.200061in}{3.312770in}}%
\pgfpathlineto{\pgfqpoint{5.203908in}{3.321801in}}%
\pgfpathlineto{\pgfqpoint{5.207755in}{3.316401in}}%
\pgfpathlineto{\pgfqpoint{5.211602in}{3.310557in}}%
\pgfpathlineto{\pgfqpoint{5.213525in}{3.330043in}}%
\pgfpathlineto{\pgfqpoint{5.215449in}{3.327989in}}%
\pgfpathlineto{\pgfqpoint{5.217372in}{3.348424in}}%
\pgfpathlineto{\pgfqpoint{5.221219in}{3.330944in}}%
\pgfpathlineto{\pgfqpoint{5.223142in}{3.332975in}}%
\pgfpathlineto{\pgfqpoint{5.225066in}{3.343955in}}%
\pgfpathlineto{\pgfqpoint{5.226989in}{3.336770in}}%
\pgfpathlineto{\pgfqpoint{5.228913in}{3.347749in}}%
\pgfpathlineto{\pgfqpoint{5.230836in}{3.350102in}}%
\pgfpathlineto{\pgfqpoint{5.232759in}{3.358491in}}%
\pgfpathlineto{\pgfqpoint{5.238530in}{3.331265in}}%
\pgfpathlineto{\pgfqpoint{5.240453in}{3.332523in}}%
\pgfpathlineto{\pgfqpoint{5.242376in}{3.340610in}}%
\pgfpathlineto{\pgfqpoint{5.244300in}{3.337769in}}%
\pgfpathlineto{\pgfqpoint{5.246223in}{3.332864in}}%
\pgfpathlineto{\pgfqpoint{5.248147in}{3.335242in}}%
\pgfpathlineto{\pgfqpoint{5.251994in}{3.365588in}}%
\pgfpathlineto{\pgfqpoint{5.253917in}{3.361521in}}%
\pgfpathlineto{\pgfqpoint{5.257764in}{3.363600in}}%
\pgfpathlineto{\pgfqpoint{5.261611in}{3.349035in}}%
\pgfpathlineto{\pgfqpoint{5.263534in}{3.356692in}}%
\pgfpathlineto{\pgfqpoint{5.265457in}{3.357424in}}%
\pgfpathlineto{\pgfqpoint{5.267381in}{3.351624in}}%
\pgfpathlineto{\pgfqpoint{5.269304in}{3.351618in}}%
\pgfpathlineto{\pgfqpoint{5.271228in}{3.359982in}}%
\pgfpathlineto{\pgfqpoint{5.273151in}{3.352449in}}%
\pgfpathlineto{\pgfqpoint{5.275074in}{3.364184in}}%
\pgfpathlineto{\pgfqpoint{5.276998in}{3.358691in}}%
\pgfpathlineto{\pgfqpoint{5.278921in}{3.363936in}}%
\pgfpathlineto{\pgfqpoint{5.280845in}{3.364097in}}%
\pgfpathlineto{\pgfqpoint{5.282768in}{3.354142in}}%
\pgfpathlineto{\pgfqpoint{5.286615in}{3.367212in}}%
\pgfpathlineto{\pgfqpoint{5.288538in}{3.368029in}}%
\pgfpathlineto{\pgfqpoint{5.290462in}{3.377010in}}%
\pgfpathlineto{\pgfqpoint{5.292385in}{3.363470in}}%
\pgfpathlineto{\pgfqpoint{5.294309in}{3.362186in}}%
\pgfpathlineto{\pgfqpoint{5.296232in}{3.363475in}}%
\pgfpathlineto{\pgfqpoint{5.298155in}{3.367521in}}%
\pgfpathlineto{\pgfqpoint{5.300079in}{3.374910in}}%
\pgfpathlineto{\pgfqpoint{5.302002in}{3.376398in}}%
\pgfpathlineto{\pgfqpoint{5.303926in}{3.375104in}}%
\pgfpathlineto{\pgfqpoint{5.303926in}{3.375104in}}%
\pgfusepath{stroke}%
\end{pgfscope}%
\begin{pgfscope}%
\pgfpathrectangle{\pgfqpoint{3.286364in}{3.180000in}}{\pgfqpoint{2.113636in}{2.100000in}}%
\pgfusepath{clip}%
\pgfsetroundcap%
\pgfsetroundjoin%
\pgfsetlinewidth{0.602250pt}%
\definecolor{currentstroke}{rgb}{1.000000,0.498039,0.000000}%
\pgfsetstrokecolor{currentstroke}%
\pgfsetdash{}{0pt}%
\pgfpathmoveto{\pgfqpoint{3.382438in}{4.103861in}}%
\pgfpathlineto{\pgfqpoint{3.392055in}{4.079396in}}%
\pgfpathlineto{\pgfqpoint{3.393978in}{4.089484in}}%
\pgfpathlineto{\pgfqpoint{3.397825in}{4.065065in}}%
\pgfpathlineto{\pgfqpoint{3.399749in}{4.081670in}}%
\pgfpathlineto{\pgfqpoint{3.401672in}{4.070830in}}%
\pgfpathlineto{\pgfqpoint{3.403596in}{4.070669in}}%
\pgfpathlineto{\pgfqpoint{3.405519in}{4.064878in}}%
\pgfpathlineto{\pgfqpoint{3.407442in}{4.069868in}}%
\pgfpathlineto{\pgfqpoint{3.409366in}{4.059482in}}%
\pgfpathlineto{\pgfqpoint{3.411289in}{4.058834in}}%
\pgfpathlineto{\pgfqpoint{3.413213in}{4.053151in}}%
\pgfpathlineto{\pgfqpoint{3.415136in}{4.054997in}}%
\pgfpathlineto{\pgfqpoint{3.420906in}{3.988320in}}%
\pgfpathlineto{\pgfqpoint{3.422830in}{3.990388in}}%
\pgfpathlineto{\pgfqpoint{3.424753in}{4.002605in}}%
\pgfpathlineto{\pgfqpoint{3.426676in}{4.005658in}}%
\pgfpathlineto{\pgfqpoint{3.430523in}{4.047738in}}%
\pgfpathlineto{\pgfqpoint{3.432447in}{4.045340in}}%
\pgfpathlineto{\pgfqpoint{3.434370in}{4.040991in}}%
\pgfpathlineto{\pgfqpoint{3.436294in}{4.044812in}}%
\pgfpathlineto{\pgfqpoint{3.440140in}{4.056695in}}%
\pgfpathlineto{\pgfqpoint{3.442064in}{4.055172in}}%
\pgfpathlineto{\pgfqpoint{3.443987in}{4.059095in}}%
\pgfpathlineto{\pgfqpoint{3.445911in}{4.055028in}}%
\pgfpathlineto{\pgfqpoint{3.447834in}{4.042464in}}%
\pgfpathlineto{\pgfqpoint{3.449757in}{4.044980in}}%
\pgfpathlineto{\pgfqpoint{3.451681in}{4.068358in}}%
\pgfpathlineto{\pgfqpoint{3.453604in}{4.067828in}}%
\pgfpathlineto{\pgfqpoint{3.457451in}{4.063128in}}%
\pgfpathlineto{\pgfqpoint{3.459374in}{4.049267in}}%
\pgfpathlineto{\pgfqpoint{3.461298in}{4.054348in}}%
\pgfpathlineto{\pgfqpoint{3.463221in}{4.049414in}}%
\pgfpathlineto{\pgfqpoint{3.465145in}{4.061083in}}%
\pgfpathlineto{\pgfqpoint{3.467068in}{4.057804in}}%
\pgfpathlineto{\pgfqpoint{3.468992in}{4.067959in}}%
\pgfpathlineto{\pgfqpoint{3.470915in}{4.056912in}}%
\pgfpathlineto{\pgfqpoint{3.472838in}{4.075188in}}%
\pgfpathlineto{\pgfqpoint{3.474762in}{4.070262in}}%
\pgfpathlineto{\pgfqpoint{3.476685in}{4.084158in}}%
\pgfpathlineto{\pgfqpoint{3.480532in}{4.090568in}}%
\pgfpathlineto{\pgfqpoint{3.482455in}{4.084709in}}%
\pgfpathlineto{\pgfqpoint{3.484379in}{4.075005in}}%
\pgfpathlineto{\pgfqpoint{3.486302in}{4.080580in}}%
\pgfpathlineto{\pgfqpoint{3.488226in}{4.070559in}}%
\pgfpathlineto{\pgfqpoint{3.492072in}{4.085501in}}%
\pgfpathlineto{\pgfqpoint{3.493996in}{4.078879in}}%
\pgfpathlineto{\pgfqpoint{3.495919in}{4.088519in}}%
\pgfpathlineto{\pgfqpoint{3.497843in}{4.081310in}}%
\pgfpathlineto{\pgfqpoint{3.499766in}{4.088177in}}%
\pgfpathlineto{\pgfqpoint{3.501689in}{4.082805in}}%
\pgfpathlineto{\pgfqpoint{3.503613in}{4.088768in}}%
\pgfpathlineto{\pgfqpoint{3.505536in}{4.089311in}}%
\pgfpathlineto{\pgfqpoint{3.507460in}{4.095130in}}%
\pgfpathlineto{\pgfqpoint{3.509383in}{4.094464in}}%
\pgfpathlineto{\pgfqpoint{3.513230in}{4.077612in}}%
\pgfpathlineto{\pgfqpoint{3.515153in}{4.081291in}}%
\pgfpathlineto{\pgfqpoint{3.519000in}{4.056860in}}%
\pgfpathlineto{\pgfqpoint{3.520924in}{4.056607in}}%
\pgfpathlineto{\pgfqpoint{3.522847in}{4.048821in}}%
\pgfpathlineto{\pgfqpoint{3.524770in}{4.055254in}}%
\pgfpathlineto{\pgfqpoint{3.526694in}{4.056871in}}%
\pgfpathlineto{\pgfqpoint{3.532464in}{4.071986in}}%
\pgfpathlineto{\pgfqpoint{3.534387in}{4.072304in}}%
\pgfpathlineto{\pgfqpoint{3.536311in}{4.086656in}}%
\pgfpathlineto{\pgfqpoint{3.538234in}{4.086139in}}%
\pgfpathlineto{\pgfqpoint{3.540158in}{4.089699in}}%
\pgfpathlineto{\pgfqpoint{3.542081in}{4.084029in}}%
\pgfpathlineto{\pgfqpoint{3.544005in}{4.089470in}}%
\pgfpathlineto{\pgfqpoint{3.547851in}{4.086140in}}%
\pgfpathlineto{\pgfqpoint{3.549775in}{4.065971in}}%
\pgfpathlineto{\pgfqpoint{3.551698in}{4.069746in}}%
\pgfpathlineto{\pgfqpoint{3.555545in}{4.046318in}}%
\pgfpathlineto{\pgfqpoint{3.557468in}{4.063624in}}%
\pgfpathlineto{\pgfqpoint{3.559392in}{4.053575in}}%
\pgfpathlineto{\pgfqpoint{3.561315in}{4.056090in}}%
\pgfpathlineto{\pgfqpoint{3.563239in}{4.056227in}}%
\pgfpathlineto{\pgfqpoint{3.565162in}{4.047298in}}%
\pgfpathlineto{\pgfqpoint{3.569009in}{4.053509in}}%
\pgfpathlineto{\pgfqpoint{3.570932in}{4.073952in}}%
\pgfpathlineto{\pgfqpoint{3.572856in}{4.068842in}}%
\pgfpathlineto{\pgfqpoint{3.574779in}{4.074192in}}%
\pgfpathlineto{\pgfqpoint{3.576703in}{4.072528in}}%
\pgfpathlineto{\pgfqpoint{3.578626in}{4.076143in}}%
\pgfpathlineto{\pgfqpoint{3.582473in}{4.059114in}}%
\pgfpathlineto{\pgfqpoint{3.584396in}{4.061002in}}%
\pgfpathlineto{\pgfqpoint{3.586320in}{4.060522in}}%
\pgfpathlineto{\pgfqpoint{3.588243in}{4.061455in}}%
\pgfpathlineto{\pgfqpoint{3.590166in}{4.055663in}}%
\pgfpathlineto{\pgfqpoint{3.592090in}{4.067014in}}%
\pgfpathlineto{\pgfqpoint{3.594013in}{4.066263in}}%
\pgfpathlineto{\pgfqpoint{3.597860in}{4.050363in}}%
\pgfpathlineto{\pgfqpoint{3.601707in}{4.077873in}}%
\pgfpathlineto{\pgfqpoint{3.603630in}{4.072339in}}%
\pgfpathlineto{\pgfqpoint{3.605554in}{4.070881in}}%
\pgfpathlineto{\pgfqpoint{3.609401in}{4.084160in}}%
\pgfpathlineto{\pgfqpoint{3.611324in}{4.076893in}}%
\pgfpathlineto{\pgfqpoint{3.613247in}{4.081299in}}%
\pgfpathlineto{\pgfqpoint{3.615171in}{4.072510in}}%
\pgfpathlineto{\pgfqpoint{3.617094in}{4.072659in}}%
\pgfpathlineto{\pgfqpoint{3.620941in}{4.088627in}}%
\pgfpathlineto{\pgfqpoint{3.622864in}{4.092536in}}%
\pgfpathlineto{\pgfqpoint{3.624788in}{4.101100in}}%
\pgfpathlineto{\pgfqpoint{3.626711in}{4.097494in}}%
\pgfpathlineto{\pgfqpoint{3.628635in}{4.089232in}}%
\pgfpathlineto{\pgfqpoint{3.632481in}{4.106962in}}%
\pgfpathlineto{\pgfqpoint{3.634405in}{4.103557in}}%
\pgfpathlineto{\pgfqpoint{3.636328in}{4.110016in}}%
\pgfpathlineto{\pgfqpoint{3.640175in}{4.129742in}}%
\pgfpathlineto{\pgfqpoint{3.642099in}{4.132936in}}%
\pgfpathlineto{\pgfqpoint{3.644022in}{4.130973in}}%
\pgfpathlineto{\pgfqpoint{3.645945in}{4.127016in}}%
\pgfpathlineto{\pgfqpoint{3.647869in}{4.107549in}}%
\pgfpathlineto{\pgfqpoint{3.651716in}{4.103358in}}%
\pgfpathlineto{\pgfqpoint{3.655562in}{4.086457in}}%
\pgfpathlineto{\pgfqpoint{3.657486in}{4.086205in}}%
\pgfpathlineto{\pgfqpoint{3.659409in}{4.105312in}}%
\pgfpathlineto{\pgfqpoint{3.661333in}{4.106566in}}%
\pgfpathlineto{\pgfqpoint{3.665179in}{4.100333in}}%
\pgfpathlineto{\pgfqpoint{3.667103in}{4.100842in}}%
\pgfpathlineto{\pgfqpoint{3.670950in}{4.084783in}}%
\pgfpathlineto{\pgfqpoint{3.672873in}{4.085788in}}%
\pgfpathlineto{\pgfqpoint{3.674796in}{4.079357in}}%
\pgfpathlineto{\pgfqpoint{3.676720in}{4.078234in}}%
\pgfpathlineto{\pgfqpoint{3.678643in}{4.073828in}}%
\pgfpathlineto{\pgfqpoint{3.682490in}{4.085313in}}%
\pgfpathlineto{\pgfqpoint{3.684414in}{4.086988in}}%
\pgfpathlineto{\pgfqpoint{3.686337in}{4.075239in}}%
\pgfpathlineto{\pgfqpoint{3.688260in}{4.080788in}}%
\pgfpathlineto{\pgfqpoint{3.690184in}{4.099661in}}%
\pgfpathlineto{\pgfqpoint{3.692107in}{4.094205in}}%
\pgfpathlineto{\pgfqpoint{3.695954in}{4.095157in}}%
\pgfpathlineto{\pgfqpoint{3.697877in}{4.093275in}}%
\pgfpathlineto{\pgfqpoint{3.701724in}{4.122172in}}%
\pgfpathlineto{\pgfqpoint{3.703648in}{4.098189in}}%
\pgfpathlineto{\pgfqpoint{3.705571in}{4.098708in}}%
\pgfpathlineto{\pgfqpoint{3.709418in}{4.109114in}}%
\pgfpathlineto{\pgfqpoint{3.711341in}{4.106605in}}%
\pgfpathlineto{\pgfqpoint{3.713265in}{4.109757in}}%
\pgfpathlineto{\pgfqpoint{3.717112in}{4.091820in}}%
\pgfpathlineto{\pgfqpoint{3.719035in}{4.092349in}}%
\pgfpathlineto{\pgfqpoint{3.720958in}{4.089928in}}%
\pgfpathlineto{\pgfqpoint{3.724805in}{4.097711in}}%
\pgfpathlineto{\pgfqpoint{3.726729in}{4.106213in}}%
\pgfpathlineto{\pgfqpoint{3.728652in}{4.091492in}}%
\pgfpathlineto{\pgfqpoint{3.732499in}{4.094363in}}%
\pgfpathlineto{\pgfqpoint{3.734422in}{4.106714in}}%
\pgfpathlineto{\pgfqpoint{3.736346in}{4.109065in}}%
\pgfpathlineto{\pgfqpoint{3.738269in}{4.097549in}}%
\pgfpathlineto{\pgfqpoint{3.740192in}{4.099780in}}%
\pgfpathlineto{\pgfqpoint{3.742116in}{4.108253in}}%
\pgfpathlineto{\pgfqpoint{3.744039in}{4.101510in}}%
\pgfpathlineto{\pgfqpoint{3.745963in}{4.104684in}}%
\pgfpathlineto{\pgfqpoint{3.749810in}{4.088365in}}%
\pgfpathlineto{\pgfqpoint{3.753656in}{4.063120in}}%
\pgfpathlineto{\pgfqpoint{3.755580in}{4.052292in}}%
\pgfpathlineto{\pgfqpoint{3.759427in}{4.061836in}}%
\pgfpathlineto{\pgfqpoint{3.761350in}{4.059187in}}%
\pgfpathlineto{\pgfqpoint{3.763273in}{4.070793in}}%
\pgfpathlineto{\pgfqpoint{3.765197in}{4.052792in}}%
\pgfpathlineto{\pgfqpoint{3.767120in}{4.053280in}}%
\pgfpathlineto{\pgfqpoint{3.769044in}{4.051047in}}%
\pgfpathlineto{\pgfqpoint{3.772890in}{4.029229in}}%
\pgfpathlineto{\pgfqpoint{3.774814in}{4.021068in}}%
\pgfpathlineto{\pgfqpoint{3.776737in}{4.018914in}}%
\pgfpathlineto{\pgfqpoint{3.778661in}{4.007649in}}%
\pgfpathlineto{\pgfqpoint{3.780584in}{4.019216in}}%
\pgfpathlineto{\pgfqpoint{3.784431in}{4.027874in}}%
\pgfpathlineto{\pgfqpoint{3.786354in}{4.036282in}}%
\pgfpathlineto{\pgfqpoint{3.788278in}{4.029579in}}%
\pgfpathlineto{\pgfqpoint{3.792125in}{4.052475in}}%
\pgfpathlineto{\pgfqpoint{3.794048in}{4.052106in}}%
\pgfpathlineto{\pgfqpoint{3.797895in}{4.043599in}}%
\pgfpathlineto{\pgfqpoint{3.799818in}{4.058163in}}%
\pgfpathlineto{\pgfqpoint{3.803665in}{4.056438in}}%
\pgfpathlineto{\pgfqpoint{3.805588in}{4.050206in}}%
\pgfpathlineto{\pgfqpoint{3.809435in}{4.057098in}}%
\pgfpathlineto{\pgfqpoint{3.811359in}{4.051738in}}%
\pgfpathlineto{\pgfqpoint{3.813282in}{4.035259in}}%
\pgfpathlineto{\pgfqpoint{3.815205in}{4.037453in}}%
\pgfpathlineto{\pgfqpoint{3.817129in}{4.041662in}}%
\pgfpathlineto{\pgfqpoint{3.820976in}{4.018760in}}%
\pgfpathlineto{\pgfqpoint{3.822899in}{4.029961in}}%
\pgfpathlineto{\pgfqpoint{3.824823in}{4.025391in}}%
\pgfpathlineto{\pgfqpoint{3.826746in}{4.029954in}}%
\pgfpathlineto{\pgfqpoint{3.828669in}{4.037564in}}%
\pgfpathlineto{\pgfqpoint{3.832516in}{4.018558in}}%
\pgfpathlineto{\pgfqpoint{3.834440in}{4.030068in}}%
\pgfpathlineto{\pgfqpoint{3.842133in}{4.014131in}}%
\pgfpathlineto{\pgfqpoint{3.844057in}{4.015118in}}%
\pgfpathlineto{\pgfqpoint{3.845980in}{4.011696in}}%
\pgfpathlineto{\pgfqpoint{3.847903in}{4.014051in}}%
\pgfpathlineto{\pgfqpoint{3.851750in}{4.014167in}}%
\pgfpathlineto{\pgfqpoint{3.853674in}{4.025346in}}%
\pgfpathlineto{\pgfqpoint{3.855597in}{4.018156in}}%
\pgfpathlineto{\pgfqpoint{3.857521in}{4.027040in}}%
\pgfpathlineto{\pgfqpoint{3.859444in}{4.029548in}}%
\pgfpathlineto{\pgfqpoint{3.861367in}{4.045834in}}%
\pgfpathlineto{\pgfqpoint{3.863291in}{4.042460in}}%
\pgfpathlineto{\pgfqpoint{3.867138in}{4.026963in}}%
\pgfpathlineto{\pgfqpoint{3.869061in}{4.030583in}}%
\pgfpathlineto{\pgfqpoint{3.870984in}{4.029454in}}%
\pgfpathlineto{\pgfqpoint{3.872908in}{4.031534in}}%
\pgfpathlineto{\pgfqpoint{3.874831in}{4.039995in}}%
\pgfpathlineto{\pgfqpoint{3.878678in}{4.024625in}}%
\pgfpathlineto{\pgfqpoint{3.886372in}{3.994510in}}%
\pgfpathlineto{\pgfqpoint{3.888295in}{4.010442in}}%
\pgfpathlineto{\pgfqpoint{3.890219in}{4.011708in}}%
\pgfpathlineto{\pgfqpoint{3.892142in}{4.020579in}}%
\pgfpathlineto{\pgfqpoint{3.894065in}{4.021686in}}%
\pgfpathlineto{\pgfqpoint{3.897912in}{4.003664in}}%
\pgfpathlineto{\pgfqpoint{3.899836in}{4.006335in}}%
\pgfpathlineto{\pgfqpoint{3.901759in}{3.999699in}}%
\pgfpathlineto{\pgfqpoint{3.903682in}{4.024377in}}%
\pgfpathlineto{\pgfqpoint{3.905606in}{4.034120in}}%
\pgfpathlineto{\pgfqpoint{3.907529in}{4.035565in}}%
\pgfpathlineto{\pgfqpoint{3.909453in}{4.053511in}}%
\pgfpathlineto{\pgfqpoint{3.913299in}{4.033785in}}%
\pgfpathlineto{\pgfqpoint{3.915223in}{4.026713in}}%
\pgfpathlineto{\pgfqpoint{3.917146in}{4.038797in}}%
\pgfpathlineto{\pgfqpoint{3.919070in}{4.042353in}}%
\pgfpathlineto{\pgfqpoint{3.920993in}{4.056585in}}%
\pgfpathlineto{\pgfqpoint{3.924840in}{4.062382in}}%
\pgfpathlineto{\pgfqpoint{3.928687in}{4.078616in}}%
\pgfpathlineto{\pgfqpoint{3.932534in}{4.070839in}}%
\pgfpathlineto{\pgfqpoint{3.934457in}{4.073030in}}%
\pgfpathlineto{\pgfqpoint{3.936380in}{4.072659in}}%
\pgfpathlineto{\pgfqpoint{3.940227in}{4.070056in}}%
\pgfpathlineto{\pgfqpoint{3.942151in}{4.051484in}}%
\pgfpathlineto{\pgfqpoint{3.947921in}{4.042049in}}%
\pgfpathlineto{\pgfqpoint{3.949844in}{4.044424in}}%
\pgfpathlineto{\pgfqpoint{3.951768in}{4.038100in}}%
\pgfpathlineto{\pgfqpoint{3.953691in}{4.055648in}}%
\pgfpathlineto{\pgfqpoint{3.955614in}{4.052258in}}%
\pgfpathlineto{\pgfqpoint{3.957538in}{4.060361in}}%
\pgfpathlineto{\pgfqpoint{3.961385in}{4.042560in}}%
\pgfpathlineto{\pgfqpoint{3.965232in}{4.049889in}}%
\pgfpathlineto{\pgfqpoint{3.971002in}{4.083461in}}%
\pgfpathlineto{\pgfqpoint{3.974849in}{4.086072in}}%
\pgfpathlineto{\pgfqpoint{3.976772in}{4.094331in}}%
\pgfpathlineto{\pgfqpoint{3.978695in}{4.091192in}}%
\pgfpathlineto{\pgfqpoint{3.980619in}{4.103073in}}%
\pgfpathlineto{\pgfqpoint{3.984466in}{4.088824in}}%
\pgfpathlineto{\pgfqpoint{3.986389in}{4.085556in}}%
\pgfpathlineto{\pgfqpoint{3.988312in}{4.087150in}}%
\pgfpathlineto{\pgfqpoint{3.992159in}{4.070030in}}%
\pgfpathlineto{\pgfqpoint{3.994083in}{4.082322in}}%
\pgfpathlineto{\pgfqpoint{3.996006in}{4.086736in}}%
\pgfpathlineto{\pgfqpoint{3.997930in}{4.080338in}}%
\pgfpathlineto{\pgfqpoint{3.999853in}{4.080891in}}%
\pgfpathlineto{\pgfqpoint{4.001776in}{4.077435in}}%
\pgfpathlineto{\pgfqpoint{4.003700in}{4.067471in}}%
\pgfpathlineto{\pgfqpoint{4.007547in}{4.088114in}}%
\pgfpathlineto{\pgfqpoint{4.009470in}{4.086847in}}%
\pgfpathlineto{\pgfqpoint{4.011393in}{4.092198in}}%
\pgfpathlineto{\pgfqpoint{4.013317in}{4.102517in}}%
\pgfpathlineto{\pgfqpoint{4.015240in}{4.093109in}}%
\pgfpathlineto{\pgfqpoint{4.017164in}{4.075505in}}%
\pgfpathlineto{\pgfqpoint{4.019087in}{4.079777in}}%
\pgfpathlineto{\pgfqpoint{4.022934in}{4.095274in}}%
\pgfpathlineto{\pgfqpoint{4.024857in}{4.097246in}}%
\pgfpathlineto{\pgfqpoint{4.028704in}{4.109544in}}%
\pgfpathlineto{\pgfqpoint{4.030628in}{4.107623in}}%
\pgfpathlineto{\pgfqpoint{4.034474in}{4.099285in}}%
\pgfpathlineto{\pgfqpoint{4.036398in}{4.097154in}}%
\pgfpathlineto{\pgfqpoint{4.038321in}{4.088784in}}%
\pgfpathlineto{\pgfqpoint{4.042168in}{4.106247in}}%
\pgfpathlineto{\pgfqpoint{4.046015in}{4.108190in}}%
\pgfpathlineto{\pgfqpoint{4.047938in}{4.118339in}}%
\pgfpathlineto{\pgfqpoint{4.049862in}{4.110559in}}%
\pgfpathlineto{\pgfqpoint{4.051785in}{4.116225in}}%
\pgfpathlineto{\pgfqpoint{4.053708in}{4.111911in}}%
\pgfpathlineto{\pgfqpoint{4.055632in}{4.118431in}}%
\pgfpathlineto{\pgfqpoint{4.057555in}{4.110371in}}%
\pgfpathlineto{\pgfqpoint{4.059479in}{4.108429in}}%
\pgfpathlineto{\pgfqpoint{4.061402in}{4.127824in}}%
\pgfpathlineto{\pgfqpoint{4.063326in}{4.125712in}}%
\pgfpathlineto{\pgfqpoint{4.065249in}{4.137701in}}%
\pgfpathlineto{\pgfqpoint{4.072943in}{4.093291in}}%
\pgfpathlineto{\pgfqpoint{4.074866in}{4.088827in}}%
\pgfpathlineto{\pgfqpoint{4.076789in}{4.098483in}}%
\pgfpathlineto{\pgfqpoint{4.078713in}{4.099009in}}%
\pgfpathlineto{\pgfqpoint{4.080636in}{4.103972in}}%
\pgfpathlineto{\pgfqpoint{4.082560in}{4.102229in}}%
\pgfpathlineto{\pgfqpoint{4.084483in}{4.118338in}}%
\pgfpathlineto{\pgfqpoint{4.086406in}{4.122154in}}%
\pgfpathlineto{\pgfqpoint{4.088330in}{4.144846in}}%
\pgfpathlineto{\pgfqpoint{4.090253in}{4.144687in}}%
\pgfpathlineto{\pgfqpoint{4.092177in}{4.141864in}}%
\pgfpathlineto{\pgfqpoint{4.094100in}{4.134454in}}%
\pgfpathlineto{\pgfqpoint{4.096024in}{4.143109in}}%
\pgfpathlineto{\pgfqpoint{4.097947in}{4.135952in}}%
\pgfpathlineto{\pgfqpoint{4.099870in}{4.133594in}}%
\pgfpathlineto{\pgfqpoint{4.105641in}{4.150133in}}%
\pgfpathlineto{\pgfqpoint{4.107564in}{4.148613in}}%
\pgfpathlineto{\pgfqpoint{4.109487in}{4.133796in}}%
\pgfpathlineto{\pgfqpoint{4.111411in}{4.133448in}}%
\pgfpathlineto{\pgfqpoint{4.115258in}{4.125076in}}%
\pgfpathlineto{\pgfqpoint{4.119104in}{4.103554in}}%
\pgfpathlineto{\pgfqpoint{4.121028in}{4.111048in}}%
\pgfpathlineto{\pgfqpoint{4.122951in}{4.108987in}}%
\pgfpathlineto{\pgfqpoint{4.124875in}{4.096014in}}%
\pgfpathlineto{\pgfqpoint{4.126798in}{4.102163in}}%
\pgfpathlineto{\pgfqpoint{4.130645in}{4.126792in}}%
\pgfpathlineto{\pgfqpoint{4.132568in}{4.134393in}}%
\pgfpathlineto{\pgfqpoint{4.134492in}{4.120800in}}%
\pgfpathlineto{\pgfqpoint{4.136415in}{4.117919in}}%
\pgfpathlineto{\pgfqpoint{4.140262in}{4.102512in}}%
\pgfpathlineto{\pgfqpoint{4.144109in}{4.118218in}}%
\pgfpathlineto{\pgfqpoint{4.146032in}{4.104323in}}%
\pgfpathlineto{\pgfqpoint{4.151802in}{4.127129in}}%
\pgfpathlineto{\pgfqpoint{4.153726in}{4.128172in}}%
\pgfpathlineto{\pgfqpoint{4.155649in}{4.126930in}}%
\pgfpathlineto{\pgfqpoint{4.157573in}{4.122690in}}%
\pgfpathlineto{\pgfqpoint{4.159496in}{4.114510in}}%
\pgfpathlineto{\pgfqpoint{4.161419in}{4.127594in}}%
\pgfpathlineto{\pgfqpoint{4.163343in}{4.120453in}}%
\pgfpathlineto{\pgfqpoint{4.165266in}{4.126627in}}%
\pgfpathlineto{\pgfqpoint{4.169113in}{4.141657in}}%
\pgfpathlineto{\pgfqpoint{4.172960in}{4.155973in}}%
\pgfpathlineto{\pgfqpoint{4.174883in}{4.157684in}}%
\pgfpathlineto{\pgfqpoint{4.176807in}{4.136007in}}%
\pgfpathlineto{\pgfqpoint{4.178730in}{4.135237in}}%
\pgfpathlineto{\pgfqpoint{4.184500in}{4.098663in}}%
\pgfpathlineto{\pgfqpoint{4.186424in}{4.112933in}}%
\pgfpathlineto{\pgfqpoint{4.188347in}{4.118076in}}%
\pgfpathlineto{\pgfqpoint{4.192194in}{4.091562in}}%
\pgfpathlineto{\pgfqpoint{4.197964in}{4.112739in}}%
\pgfpathlineto{\pgfqpoint{4.199888in}{4.115210in}}%
\pgfpathlineto{\pgfqpoint{4.201811in}{4.126932in}}%
\pgfpathlineto{\pgfqpoint{4.203735in}{4.117575in}}%
\pgfpathlineto{\pgfqpoint{4.205658in}{4.114360in}}%
\pgfpathlineto{\pgfqpoint{4.207581in}{4.128918in}}%
\pgfpathlineto{\pgfqpoint{4.209505in}{4.123862in}}%
\pgfpathlineto{\pgfqpoint{4.211428in}{4.125909in}}%
\pgfpathlineto{\pgfqpoint{4.213352in}{4.132622in}}%
\pgfpathlineto{\pgfqpoint{4.215275in}{4.132673in}}%
\pgfpathlineto{\pgfqpoint{4.217198in}{4.124913in}}%
\pgfpathlineto{\pgfqpoint{4.219122in}{4.131499in}}%
\pgfpathlineto{\pgfqpoint{4.221045in}{4.110575in}}%
\pgfpathlineto{\pgfqpoint{4.222969in}{4.106314in}}%
\pgfpathlineto{\pgfqpoint{4.224892in}{4.088643in}}%
\pgfpathlineto{\pgfqpoint{4.226815in}{4.086739in}}%
\pgfpathlineto{\pgfqpoint{4.230662in}{4.110407in}}%
\pgfpathlineto{\pgfqpoint{4.234509in}{4.089007in}}%
\pgfpathlineto{\pgfqpoint{4.236433in}{4.089707in}}%
\pgfpathlineto{\pgfqpoint{4.238356in}{4.105231in}}%
\pgfpathlineto{\pgfqpoint{4.240279in}{4.100753in}}%
\pgfpathlineto{\pgfqpoint{4.244126in}{4.120845in}}%
\pgfpathlineto{\pgfqpoint{4.246050in}{4.117421in}}%
\pgfpathlineto{\pgfqpoint{4.247973in}{4.124038in}}%
\pgfpathlineto{\pgfqpoint{4.251820in}{4.102976in}}%
\pgfpathlineto{\pgfqpoint{4.253743in}{4.106403in}}%
\pgfpathlineto{\pgfqpoint{4.255667in}{4.114474in}}%
\pgfpathlineto{\pgfqpoint{4.257590in}{4.103045in}}%
\pgfpathlineto{\pgfqpoint{4.259513in}{4.103460in}}%
\pgfpathlineto{\pgfqpoint{4.261437in}{4.087992in}}%
\pgfpathlineto{\pgfqpoint{4.263360in}{4.086516in}}%
\pgfpathlineto{\pgfqpoint{4.265284in}{4.095497in}}%
\pgfpathlineto{\pgfqpoint{4.267207in}{4.096680in}}%
\pgfpathlineto{\pgfqpoint{4.269130in}{4.075377in}}%
\pgfpathlineto{\pgfqpoint{4.271054in}{4.089155in}}%
\pgfpathlineto{\pgfqpoint{4.272977in}{4.111064in}}%
\pgfpathlineto{\pgfqpoint{4.274901in}{4.105775in}}%
\pgfpathlineto{\pgfqpoint{4.276824in}{4.128016in}}%
\pgfpathlineto{\pgfqpoint{4.278748in}{4.118029in}}%
\pgfpathlineto{\pgfqpoint{4.280671in}{4.138887in}}%
\pgfpathlineto{\pgfqpoint{4.282594in}{4.139550in}}%
\pgfpathlineto{\pgfqpoint{4.286441in}{4.156908in}}%
\pgfpathlineto{\pgfqpoint{4.288365in}{4.156341in}}%
\pgfpathlineto{\pgfqpoint{4.290288in}{4.157633in}}%
\pgfpathlineto{\pgfqpoint{4.292211in}{4.169743in}}%
\pgfpathlineto{\pgfqpoint{4.294135in}{4.168882in}}%
\pgfpathlineto{\pgfqpoint{4.296058in}{4.177892in}}%
\pgfpathlineto{\pgfqpoint{4.297982in}{4.167437in}}%
\pgfpathlineto{\pgfqpoint{4.299905in}{4.171363in}}%
\pgfpathlineto{\pgfqpoint{4.301828in}{4.163620in}}%
\pgfpathlineto{\pgfqpoint{4.305675in}{4.171731in}}%
\pgfpathlineto{\pgfqpoint{4.309522in}{4.136861in}}%
\pgfpathlineto{\pgfqpoint{4.311446in}{4.140910in}}%
\pgfpathlineto{\pgfqpoint{4.313369in}{4.119975in}}%
\pgfpathlineto{\pgfqpoint{4.315292in}{4.114018in}}%
\pgfpathlineto{\pgfqpoint{4.317216in}{4.114353in}}%
\pgfpathlineto{\pgfqpoint{4.319139in}{4.118236in}}%
\pgfpathlineto{\pgfqpoint{4.321063in}{4.117926in}}%
\pgfpathlineto{\pgfqpoint{4.322986in}{4.113665in}}%
\pgfpathlineto{\pgfqpoint{4.324909in}{4.112938in}}%
\pgfpathlineto{\pgfqpoint{4.328756in}{4.102273in}}%
\pgfpathlineto{\pgfqpoint{4.332603in}{4.115892in}}%
\pgfpathlineto{\pgfqpoint{4.336450in}{4.128782in}}%
\pgfpathlineto{\pgfqpoint{4.338373in}{4.116085in}}%
\pgfpathlineto{\pgfqpoint{4.342220in}{4.144058in}}%
\pgfpathlineto{\pgfqpoint{4.344144in}{4.125797in}}%
\pgfpathlineto{\pgfqpoint{4.346067in}{4.143348in}}%
\pgfpathlineto{\pgfqpoint{4.347990in}{4.122754in}}%
\pgfpathlineto{\pgfqpoint{4.349914in}{4.127822in}}%
\pgfpathlineto{\pgfqpoint{4.351837in}{4.128044in}}%
\pgfpathlineto{\pgfqpoint{4.353761in}{4.123880in}}%
\pgfpathlineto{\pgfqpoint{4.357607in}{4.148772in}}%
\pgfpathlineto{\pgfqpoint{4.361454in}{4.139720in}}%
\pgfpathlineto{\pgfqpoint{4.363378in}{4.133390in}}%
\pgfpathlineto{\pgfqpoint{4.367224in}{4.153012in}}%
\pgfpathlineto{\pgfqpoint{4.369148in}{4.144069in}}%
\pgfpathlineto{\pgfqpoint{4.371071in}{4.143194in}}%
\pgfpathlineto{\pgfqpoint{4.372995in}{4.140147in}}%
\pgfpathlineto{\pgfqpoint{4.374918in}{4.142612in}}%
\pgfpathlineto{\pgfqpoint{4.376842in}{4.162947in}}%
\pgfpathlineto{\pgfqpoint{4.378765in}{4.163735in}}%
\pgfpathlineto{\pgfqpoint{4.380688in}{4.155494in}}%
\pgfpathlineto{\pgfqpoint{4.384535in}{4.161876in}}%
\pgfpathlineto{\pgfqpoint{4.386459in}{4.156621in}}%
\pgfpathlineto{\pgfqpoint{4.388382in}{4.168918in}}%
\pgfpathlineto{\pgfqpoint{4.392229in}{4.152171in}}%
\pgfpathlineto{\pgfqpoint{4.394152in}{4.118807in}}%
\pgfpathlineto{\pgfqpoint{4.396076in}{4.127346in}}%
\pgfpathlineto{\pgfqpoint{4.397999in}{4.124966in}}%
\pgfpathlineto{\pgfqpoint{4.399922in}{4.148062in}}%
\pgfpathlineto{\pgfqpoint{4.401846in}{4.130600in}}%
\pgfpathlineto{\pgfqpoint{4.403769in}{4.129285in}}%
\pgfpathlineto{\pgfqpoint{4.405693in}{4.121420in}}%
\pgfpathlineto{\pgfqpoint{4.407616in}{4.134578in}}%
\pgfpathlineto{\pgfqpoint{4.411463in}{4.129983in}}%
\pgfpathlineto{\pgfqpoint{4.413386in}{4.135523in}}%
\pgfpathlineto{\pgfqpoint{4.415310in}{4.134971in}}%
\pgfpathlineto{\pgfqpoint{4.417233in}{4.118049in}}%
\pgfpathlineto{\pgfqpoint{4.421080in}{4.106634in}}%
\pgfpathlineto{\pgfqpoint{4.423003in}{4.113207in}}%
\pgfpathlineto{\pgfqpoint{4.424927in}{4.115296in}}%
\pgfpathlineto{\pgfqpoint{4.426850in}{4.120317in}}%
\pgfpathlineto{\pgfqpoint{4.428774in}{4.117397in}}%
\pgfpathlineto{\pgfqpoint{4.430697in}{4.120040in}}%
\pgfpathlineto{\pgfqpoint{4.432620in}{4.145953in}}%
\pgfpathlineto{\pgfqpoint{4.436467in}{4.140349in}}%
\pgfpathlineto{\pgfqpoint{4.438391in}{4.153834in}}%
\pgfpathlineto{\pgfqpoint{4.444161in}{4.131548in}}%
\pgfpathlineto{\pgfqpoint{4.448008in}{4.112106in}}%
\pgfpathlineto{\pgfqpoint{4.449931in}{4.113482in}}%
\pgfpathlineto{\pgfqpoint{4.451855in}{4.104549in}}%
\pgfpathlineto{\pgfqpoint{4.453778in}{4.101476in}}%
\pgfpathlineto{\pgfqpoint{4.455701in}{4.094864in}}%
\pgfpathlineto{\pgfqpoint{4.457625in}{4.092250in}}%
\pgfpathlineto{\pgfqpoint{4.459548in}{4.092244in}}%
\pgfpathlineto{\pgfqpoint{4.463395in}{4.068318in}}%
\pgfpathlineto{\pgfqpoint{4.467242in}{4.102206in}}%
\pgfpathlineto{\pgfqpoint{4.469165in}{4.104276in}}%
\pgfpathlineto{\pgfqpoint{4.471089in}{4.101568in}}%
\pgfpathlineto{\pgfqpoint{4.473012in}{4.089131in}}%
\pgfpathlineto{\pgfqpoint{4.474935in}{4.100244in}}%
\pgfpathlineto{\pgfqpoint{4.476859in}{4.084727in}}%
\pgfpathlineto{\pgfqpoint{4.478782in}{4.080239in}}%
\pgfpathlineto{\pgfqpoint{4.480706in}{4.091363in}}%
\pgfpathlineto{\pgfqpoint{4.482629in}{4.086144in}}%
\pgfpathlineto{\pgfqpoint{4.486476in}{4.093101in}}%
\pgfpathlineto{\pgfqpoint{4.488399in}{4.096920in}}%
\pgfpathlineto{\pgfqpoint{4.490323in}{4.085273in}}%
\pgfpathlineto{\pgfqpoint{4.492246in}{4.085963in}}%
\pgfpathlineto{\pgfqpoint{4.494170in}{4.084854in}}%
\pgfpathlineto{\pgfqpoint{4.496093in}{4.079789in}}%
\pgfpathlineto{\pgfqpoint{4.499940in}{4.110107in}}%
\pgfpathlineto{\pgfqpoint{4.501863in}{4.086222in}}%
\pgfpathlineto{\pgfqpoint{4.505710in}{4.095108in}}%
\pgfpathlineto{\pgfqpoint{4.507633in}{4.103817in}}%
\pgfpathlineto{\pgfqpoint{4.509557in}{4.099827in}}%
\pgfpathlineto{\pgfqpoint{4.511480in}{4.101186in}}%
\pgfpathlineto{\pgfqpoint{4.513404in}{4.090424in}}%
\pgfpathlineto{\pgfqpoint{4.515327in}{4.108265in}}%
\pgfpathlineto{\pgfqpoint{4.517251in}{4.103890in}}%
\pgfpathlineto{\pgfqpoint{4.519174in}{4.095965in}}%
\pgfpathlineto{\pgfqpoint{4.521097in}{4.103114in}}%
\pgfpathlineto{\pgfqpoint{4.524944in}{4.062998in}}%
\pgfpathlineto{\pgfqpoint{4.526868in}{4.065829in}}%
\pgfpathlineto{\pgfqpoint{4.528791in}{4.063292in}}%
\pgfpathlineto{\pgfqpoint{4.530714in}{4.069987in}}%
\pgfpathlineto{\pgfqpoint{4.534561in}{4.063181in}}%
\pgfpathlineto{\pgfqpoint{4.536485in}{4.065764in}}%
\pgfpathlineto{\pgfqpoint{4.538408in}{4.075141in}}%
\pgfpathlineto{\pgfqpoint{4.542255in}{4.054795in}}%
\pgfpathlineto{\pgfqpoint{4.544178in}{4.046313in}}%
\pgfpathlineto{\pgfqpoint{4.546102in}{4.043145in}}%
\pgfpathlineto{\pgfqpoint{4.549949in}{4.031830in}}%
\pgfpathlineto{\pgfqpoint{4.551872in}{4.030314in}}%
\pgfpathlineto{\pgfqpoint{4.553795in}{4.027235in}}%
\pgfpathlineto{\pgfqpoint{4.555719in}{4.027050in}}%
\pgfpathlineto{\pgfqpoint{4.561489in}{4.008975in}}%
\pgfpathlineto{\pgfqpoint{4.563412in}{3.996059in}}%
\pgfpathlineto{\pgfqpoint{4.565336in}{4.007751in}}%
\pgfpathlineto{\pgfqpoint{4.567259in}{4.004309in}}%
\pgfpathlineto{\pgfqpoint{4.569183in}{3.987263in}}%
\pgfpathlineto{\pgfqpoint{4.571106in}{3.989151in}}%
\pgfpathlineto{\pgfqpoint{4.573029in}{3.998781in}}%
\pgfpathlineto{\pgfqpoint{4.574953in}{3.984226in}}%
\pgfpathlineto{\pgfqpoint{4.576876in}{3.990672in}}%
\pgfpathlineto{\pgfqpoint{4.578800in}{4.014821in}}%
\pgfpathlineto{\pgfqpoint{4.580723in}{4.016698in}}%
\pgfpathlineto{\pgfqpoint{4.586493in}{4.054690in}}%
\pgfpathlineto{\pgfqpoint{4.588417in}{4.050107in}}%
\pgfpathlineto{\pgfqpoint{4.590340in}{4.042379in}}%
\pgfpathlineto{\pgfqpoint{4.592264in}{4.047205in}}%
\pgfpathlineto{\pgfqpoint{4.594187in}{4.045023in}}%
\pgfpathlineto{\pgfqpoint{4.596110in}{4.048091in}}%
\pgfpathlineto{\pgfqpoint{4.599957in}{4.067351in}}%
\pgfpathlineto{\pgfqpoint{4.603804in}{4.064006in}}%
\pgfpathlineto{\pgfqpoint{4.605727in}{4.056221in}}%
\pgfpathlineto{\pgfqpoint{4.607651in}{4.073447in}}%
\pgfpathlineto{\pgfqpoint{4.611498in}{4.062801in}}%
\pgfpathlineto{\pgfqpoint{4.615344in}{4.043670in}}%
\pgfpathlineto{\pgfqpoint{4.617268in}{4.044399in}}%
\pgfpathlineto{\pgfqpoint{4.619191in}{4.032117in}}%
\pgfpathlineto{\pgfqpoint{4.621115in}{4.034281in}}%
\pgfpathlineto{\pgfqpoint{4.623038in}{4.043981in}}%
\pgfpathlineto{\pgfqpoint{4.624962in}{4.047663in}}%
\pgfpathlineto{\pgfqpoint{4.628808in}{4.062746in}}%
\pgfpathlineto{\pgfqpoint{4.630732in}{4.042245in}}%
\pgfpathlineto{\pgfqpoint{4.632655in}{4.052745in}}%
\pgfpathlineto{\pgfqpoint{4.634579in}{4.043179in}}%
\pgfpathlineto{\pgfqpoint{4.636502in}{4.052012in}}%
\pgfpathlineto{\pgfqpoint{4.640349in}{4.030504in}}%
\pgfpathlineto{\pgfqpoint{4.642272in}{4.033028in}}%
\pgfpathlineto{\pgfqpoint{4.644196in}{4.031325in}}%
\pgfpathlineto{\pgfqpoint{4.646119in}{4.032735in}}%
\pgfpathlineto{\pgfqpoint{4.648042in}{4.030095in}}%
\pgfpathlineto{\pgfqpoint{4.649966in}{4.030588in}}%
\pgfpathlineto{\pgfqpoint{4.651889in}{4.048636in}}%
\pgfpathlineto{\pgfqpoint{4.653813in}{4.042805in}}%
\pgfpathlineto{\pgfqpoint{4.655736in}{4.041850in}}%
\pgfpathlineto{\pgfqpoint{4.657660in}{4.049788in}}%
\pgfpathlineto{\pgfqpoint{4.659583in}{4.041403in}}%
\pgfpathlineto{\pgfqpoint{4.661506in}{4.041093in}}%
\pgfpathlineto{\pgfqpoint{4.663430in}{4.025642in}}%
\pgfpathlineto{\pgfqpoint{4.667277in}{4.013206in}}%
\pgfpathlineto{\pgfqpoint{4.669200in}{4.023646in}}%
\pgfpathlineto{\pgfqpoint{4.671123in}{4.006579in}}%
\pgfpathlineto{\pgfqpoint{4.673047in}{4.001361in}}%
\pgfpathlineto{\pgfqpoint{4.674970in}{4.004721in}}%
\pgfpathlineto{\pgfqpoint{4.676894in}{4.002050in}}%
\pgfpathlineto{\pgfqpoint{4.678817in}{3.993632in}}%
\pgfpathlineto{\pgfqpoint{4.680740in}{3.995384in}}%
\pgfpathlineto{\pgfqpoint{4.688434in}{3.952512in}}%
\pgfpathlineto{\pgfqpoint{4.690358in}{3.959712in}}%
\pgfpathlineto{\pgfqpoint{4.696128in}{3.968446in}}%
\pgfpathlineto{\pgfqpoint{4.698051in}{3.958659in}}%
\pgfpathlineto{\pgfqpoint{4.701898in}{3.967767in}}%
\pgfpathlineto{\pgfqpoint{4.703821in}{3.968349in}}%
\pgfpathlineto{\pgfqpoint{4.707668in}{3.939756in}}%
\pgfpathlineto{\pgfqpoint{4.709592in}{3.942464in}}%
\pgfpathlineto{\pgfqpoint{4.713438in}{3.965129in}}%
\pgfpathlineto{\pgfqpoint{4.715362in}{3.984368in}}%
\pgfpathlineto{\pgfqpoint{4.721132in}{4.003790in}}%
\pgfpathlineto{\pgfqpoint{4.723055in}{4.007030in}}%
\pgfpathlineto{\pgfqpoint{4.730749in}{3.979636in}}%
\pgfpathlineto{\pgfqpoint{4.732673in}{3.971288in}}%
\pgfpathlineto{\pgfqpoint{4.736519in}{4.008092in}}%
\pgfpathlineto{\pgfqpoint{4.738443in}{4.008988in}}%
\pgfpathlineto{\pgfqpoint{4.744213in}{4.019874in}}%
\pgfpathlineto{\pgfqpoint{4.746136in}{4.020826in}}%
\pgfpathlineto{\pgfqpoint{4.748060in}{4.023371in}}%
\pgfpathlineto{\pgfqpoint{4.749983in}{4.014000in}}%
\pgfpathlineto{\pgfqpoint{4.755753in}{4.007983in}}%
\pgfpathlineto{\pgfqpoint{4.757677in}{4.010681in}}%
\pgfpathlineto{\pgfqpoint{4.759600in}{4.009459in}}%
\pgfpathlineto{\pgfqpoint{4.761524in}{4.015670in}}%
\pgfpathlineto{\pgfqpoint{4.763447in}{4.008875in}}%
\pgfpathlineto{\pgfqpoint{4.765371in}{4.015594in}}%
\pgfpathlineto{\pgfqpoint{4.767294in}{4.017388in}}%
\pgfpathlineto{\pgfqpoint{4.769217in}{4.021266in}}%
\pgfpathlineto{\pgfqpoint{4.771141in}{4.016319in}}%
\pgfpathlineto{\pgfqpoint{4.773064in}{4.030990in}}%
\pgfpathlineto{\pgfqpoint{4.774988in}{4.012524in}}%
\pgfpathlineto{\pgfqpoint{4.776911in}{4.005948in}}%
\pgfpathlineto{\pgfqpoint{4.778834in}{3.989253in}}%
\pgfpathlineto{\pgfqpoint{4.780758in}{4.000148in}}%
\pgfpathlineto{\pgfqpoint{4.782681in}{4.001466in}}%
\pgfpathlineto{\pgfqpoint{4.784605in}{4.001178in}}%
\pgfpathlineto{\pgfqpoint{4.786528in}{3.997674in}}%
\pgfpathlineto{\pgfqpoint{4.788451in}{4.003037in}}%
\pgfpathlineto{\pgfqpoint{4.790375in}{4.003303in}}%
\pgfpathlineto{\pgfqpoint{4.792298in}{4.001575in}}%
\pgfpathlineto{\pgfqpoint{4.794222in}{4.001699in}}%
\pgfpathlineto{\pgfqpoint{4.796145in}{4.005479in}}%
\pgfpathlineto{\pgfqpoint{4.801915in}{4.031676in}}%
\pgfpathlineto{\pgfqpoint{4.803839in}{4.028057in}}%
\pgfpathlineto{\pgfqpoint{4.807686in}{4.041850in}}%
\pgfpathlineto{\pgfqpoint{4.809609in}{4.061160in}}%
\pgfpathlineto{\pgfqpoint{4.811532in}{4.057217in}}%
\pgfpathlineto{\pgfqpoint{4.813456in}{4.056725in}}%
\pgfpathlineto{\pgfqpoint{4.815379in}{4.047032in}}%
\pgfpathlineto{\pgfqpoint{4.817303in}{4.048668in}}%
\pgfpathlineto{\pgfqpoint{4.819226in}{4.045076in}}%
\pgfpathlineto{\pgfqpoint{4.821149in}{4.046172in}}%
\pgfpathlineto{\pgfqpoint{4.824996in}{4.052611in}}%
\pgfpathlineto{\pgfqpoint{4.826920in}{4.047142in}}%
\pgfpathlineto{\pgfqpoint{4.828843in}{4.035096in}}%
\pgfpathlineto{\pgfqpoint{4.830767in}{4.058624in}}%
\pgfpathlineto{\pgfqpoint{4.834613in}{4.035018in}}%
\pgfpathlineto{\pgfqpoint{4.836537in}{4.030544in}}%
\pgfpathlineto{\pgfqpoint{4.838460in}{4.033456in}}%
\pgfpathlineto{\pgfqpoint{4.840384in}{4.025716in}}%
\pgfpathlineto{\pgfqpoint{4.842307in}{4.034741in}}%
\pgfpathlineto{\pgfqpoint{4.844230in}{4.031033in}}%
\pgfpathlineto{\pgfqpoint{4.846154in}{4.032464in}}%
\pgfpathlineto{\pgfqpoint{4.851924in}{4.058527in}}%
\pgfpathlineto{\pgfqpoint{4.853847in}{4.072008in}}%
\pgfpathlineto{\pgfqpoint{4.855771in}{4.066788in}}%
\pgfpathlineto{\pgfqpoint{4.857694in}{4.065359in}}%
\pgfpathlineto{\pgfqpoint{4.859618in}{4.059816in}}%
\pgfpathlineto{\pgfqpoint{4.861541in}{4.081608in}}%
\pgfpathlineto{\pgfqpoint{4.867311in}{4.060600in}}%
\pgfpathlineto{\pgfqpoint{4.869235in}{4.066599in}}%
\pgfpathlineto{\pgfqpoint{4.871158in}{4.066755in}}%
\pgfpathlineto{\pgfqpoint{4.873082in}{4.058947in}}%
\pgfpathlineto{\pgfqpoint{4.875005in}{4.082253in}}%
\pgfpathlineto{\pgfqpoint{4.876928in}{4.081448in}}%
\pgfpathlineto{\pgfqpoint{4.878852in}{4.063435in}}%
\pgfpathlineto{\pgfqpoint{4.880775in}{4.076924in}}%
\pgfpathlineto{\pgfqpoint{4.888469in}{4.090980in}}%
\pgfpathlineto{\pgfqpoint{4.890392in}{4.087173in}}%
\pgfpathlineto{\pgfqpoint{4.894239in}{4.094512in}}%
\pgfpathlineto{\pgfqpoint{4.896162in}{4.089630in}}%
\pgfpathlineto{\pgfqpoint{4.898086in}{4.097768in}}%
\pgfpathlineto{\pgfqpoint{4.900009in}{4.095717in}}%
\pgfpathlineto{\pgfqpoint{4.903856in}{4.104749in}}%
\pgfpathlineto{\pgfqpoint{4.905780in}{4.099470in}}%
\pgfpathlineto{\pgfqpoint{4.907703in}{4.124381in}}%
\pgfpathlineto{\pgfqpoint{4.909626in}{4.115379in}}%
\pgfpathlineto{\pgfqpoint{4.913473in}{4.133155in}}%
\pgfpathlineto{\pgfqpoint{4.915397in}{4.137798in}}%
\pgfpathlineto{\pgfqpoint{4.917320in}{4.153968in}}%
\pgfpathlineto{\pgfqpoint{4.919243in}{4.180329in}}%
\pgfpathlineto{\pgfqpoint{4.921167in}{4.189654in}}%
\pgfpathlineto{\pgfqpoint{4.923090in}{4.183493in}}%
\pgfpathlineto{\pgfqpoint{4.926937in}{4.207919in}}%
\pgfpathlineto{\pgfqpoint{4.928860in}{4.204953in}}%
\pgfpathlineto{\pgfqpoint{4.930784in}{4.198501in}}%
\pgfpathlineto{\pgfqpoint{4.932707in}{4.184701in}}%
\pgfpathlineto{\pgfqpoint{4.934631in}{4.190985in}}%
\pgfpathlineto{\pgfqpoint{4.936554in}{4.179394in}}%
\pgfpathlineto{\pgfqpoint{4.938478in}{4.177500in}}%
\pgfpathlineto{\pgfqpoint{4.942324in}{4.209812in}}%
\pgfpathlineto{\pgfqpoint{4.944248in}{4.212350in}}%
\pgfpathlineto{\pgfqpoint{4.946171in}{4.205419in}}%
\pgfpathlineto{\pgfqpoint{4.948095in}{4.211745in}}%
\pgfpathlineto{\pgfqpoint{4.951941in}{4.193907in}}%
\pgfpathlineto{\pgfqpoint{4.955788in}{4.191435in}}%
\pgfpathlineto{\pgfqpoint{4.957712in}{4.220944in}}%
\pgfpathlineto{\pgfqpoint{4.959635in}{4.227494in}}%
\pgfpathlineto{\pgfqpoint{4.961558in}{4.229841in}}%
\pgfpathlineto{\pgfqpoint{4.971176in}{4.209925in}}%
\pgfpathlineto{\pgfqpoint{4.973099in}{4.197286in}}%
\pgfpathlineto{\pgfqpoint{4.975022in}{4.212190in}}%
\pgfpathlineto{\pgfqpoint{4.976946in}{4.216831in}}%
\pgfpathlineto{\pgfqpoint{4.978869in}{4.203811in}}%
\pgfpathlineto{\pgfqpoint{4.980793in}{4.212667in}}%
\pgfpathlineto{\pgfqpoint{4.982716in}{4.212028in}}%
\pgfpathlineto{\pgfqpoint{4.986563in}{4.227620in}}%
\pgfpathlineto{\pgfqpoint{4.988486in}{4.216842in}}%
\pgfpathlineto{\pgfqpoint{4.992333in}{4.233007in}}%
\pgfpathlineto{\pgfqpoint{4.994256in}{4.225419in}}%
\pgfpathlineto{\pgfqpoint{4.996180in}{4.225510in}}%
\pgfpathlineto{\pgfqpoint{4.998103in}{4.223846in}}%
\pgfpathlineto{\pgfqpoint{5.001950in}{4.200983in}}%
\pgfpathlineto{\pgfqpoint{5.003874in}{4.205155in}}%
\pgfpathlineto{\pgfqpoint{5.005797in}{4.199411in}}%
\pgfpathlineto{\pgfqpoint{5.007720in}{4.199972in}}%
\pgfpathlineto{\pgfqpoint{5.009644in}{4.190343in}}%
\pgfpathlineto{\pgfqpoint{5.011567in}{4.206514in}}%
\pgfpathlineto{\pgfqpoint{5.015414in}{4.186981in}}%
\pgfpathlineto{\pgfqpoint{5.017337in}{4.188414in}}%
\pgfpathlineto{\pgfqpoint{5.019261in}{4.196533in}}%
\pgfpathlineto{\pgfqpoint{5.021184in}{4.215560in}}%
\pgfpathlineto{\pgfqpoint{5.026954in}{4.226887in}}%
\pgfpathlineto{\pgfqpoint{5.028878in}{4.235463in}}%
\pgfpathlineto{\pgfqpoint{5.030801in}{4.236826in}}%
\pgfpathlineto{\pgfqpoint{5.032725in}{4.234366in}}%
\pgfpathlineto{\pgfqpoint{5.036571in}{4.207329in}}%
\pgfpathlineto{\pgfqpoint{5.038495in}{4.208937in}}%
\pgfpathlineto{\pgfqpoint{5.040418in}{4.207257in}}%
\pgfpathlineto{\pgfqpoint{5.042342in}{4.213904in}}%
\pgfpathlineto{\pgfqpoint{5.044265in}{4.203033in}}%
\pgfpathlineto{\pgfqpoint{5.048112in}{4.226142in}}%
\pgfpathlineto{\pgfqpoint{5.050035in}{4.228613in}}%
\pgfpathlineto{\pgfqpoint{5.053882in}{4.211305in}}%
\pgfpathlineto{\pgfqpoint{5.055806in}{4.215368in}}%
\pgfpathlineto{\pgfqpoint{5.057729in}{4.211398in}}%
\pgfpathlineto{\pgfqpoint{5.059652in}{4.187730in}}%
\pgfpathlineto{\pgfqpoint{5.061576in}{4.206379in}}%
\pgfpathlineto{\pgfqpoint{5.063499in}{4.200905in}}%
\pgfpathlineto{\pgfqpoint{5.067346in}{4.169537in}}%
\pgfpathlineto{\pgfqpoint{5.071193in}{4.192593in}}%
\pgfpathlineto{\pgfqpoint{5.073116in}{4.184822in}}%
\pgfpathlineto{\pgfqpoint{5.076963in}{4.206650in}}%
\pgfpathlineto{\pgfqpoint{5.078887in}{4.200187in}}%
\pgfpathlineto{\pgfqpoint{5.080810in}{4.203299in}}%
\pgfpathlineto{\pgfqpoint{5.084657in}{4.214874in}}%
\pgfpathlineto{\pgfqpoint{5.086580in}{4.212314in}}%
\pgfpathlineto{\pgfqpoint{5.088504in}{4.206266in}}%
\pgfpathlineto{\pgfqpoint{5.090427in}{4.204317in}}%
\pgfpathlineto{\pgfqpoint{5.092350in}{4.205746in}}%
\pgfpathlineto{\pgfqpoint{5.094274in}{4.217104in}}%
\pgfpathlineto{\pgfqpoint{5.096197in}{4.211540in}}%
\pgfpathlineto{\pgfqpoint{5.098121in}{4.216674in}}%
\pgfpathlineto{\pgfqpoint{5.103891in}{4.190257in}}%
\pgfpathlineto{\pgfqpoint{5.105814in}{4.193689in}}%
\pgfpathlineto{\pgfqpoint{5.109661in}{4.176840in}}%
\pgfpathlineto{\pgfqpoint{5.111585in}{4.180338in}}%
\pgfpathlineto{\pgfqpoint{5.113508in}{4.190438in}}%
\pgfpathlineto{\pgfqpoint{5.115431in}{4.176662in}}%
\pgfpathlineto{\pgfqpoint{5.119278in}{4.170557in}}%
\pgfpathlineto{\pgfqpoint{5.123125in}{4.156022in}}%
\pgfpathlineto{\pgfqpoint{5.125048in}{4.155184in}}%
\pgfpathlineto{\pgfqpoint{5.126972in}{4.155834in}}%
\pgfpathlineto{\pgfqpoint{5.128895in}{4.152093in}}%
\pgfpathlineto{\pgfqpoint{5.130819in}{4.134803in}}%
\pgfpathlineto{\pgfqpoint{5.132742in}{4.128267in}}%
\pgfpathlineto{\pgfqpoint{5.134665in}{4.129420in}}%
\pgfpathlineto{\pgfqpoint{5.136589in}{4.128630in}}%
\pgfpathlineto{\pgfqpoint{5.138512in}{4.112927in}}%
\pgfpathlineto{\pgfqpoint{5.140436in}{4.109252in}}%
\pgfpathlineto{\pgfqpoint{5.144283in}{4.118009in}}%
\pgfpathlineto{\pgfqpoint{5.146206in}{4.130719in}}%
\pgfpathlineto{\pgfqpoint{5.148129in}{4.130747in}}%
\pgfpathlineto{\pgfqpoint{5.150053in}{4.136682in}}%
\pgfpathlineto{\pgfqpoint{5.151976in}{4.151938in}}%
\pgfpathlineto{\pgfqpoint{5.153900in}{4.151402in}}%
\pgfpathlineto{\pgfqpoint{5.155823in}{4.152348in}}%
\pgfpathlineto{\pgfqpoint{5.157746in}{4.162550in}}%
\pgfpathlineto{\pgfqpoint{5.159670in}{4.156432in}}%
\pgfpathlineto{\pgfqpoint{5.161593in}{4.161088in}}%
\pgfpathlineto{\pgfqpoint{5.163517in}{4.175219in}}%
\pgfpathlineto{\pgfqpoint{5.165440in}{4.167957in}}%
\pgfpathlineto{\pgfqpoint{5.167363in}{4.174084in}}%
\pgfpathlineto{\pgfqpoint{5.169287in}{4.175784in}}%
\pgfpathlineto{\pgfqpoint{5.171210in}{4.180165in}}%
\pgfpathlineto{\pgfqpoint{5.173134in}{4.167137in}}%
\pgfpathlineto{\pgfqpoint{5.175057in}{4.167200in}}%
\pgfpathlineto{\pgfqpoint{5.178904in}{4.182774in}}%
\pgfpathlineto{\pgfqpoint{5.182751in}{4.150315in}}%
\pgfpathlineto{\pgfqpoint{5.184674in}{4.141161in}}%
\pgfpathlineto{\pgfqpoint{5.186598in}{4.150021in}}%
\pgfpathlineto{\pgfqpoint{5.188521in}{4.152329in}}%
\pgfpathlineto{\pgfqpoint{5.190444in}{4.146036in}}%
\pgfpathlineto{\pgfqpoint{5.192368in}{4.145728in}}%
\pgfpathlineto{\pgfqpoint{5.194291in}{4.147078in}}%
\pgfpathlineto{\pgfqpoint{5.198138in}{4.164763in}}%
\pgfpathlineto{\pgfqpoint{5.201985in}{4.135626in}}%
\pgfpathlineto{\pgfqpoint{5.203908in}{4.130710in}}%
\pgfpathlineto{\pgfqpoint{5.205832in}{4.134478in}}%
\pgfpathlineto{\pgfqpoint{5.207755in}{4.128979in}}%
\pgfpathlineto{\pgfqpoint{5.211602in}{4.145738in}}%
\pgfpathlineto{\pgfqpoint{5.213525in}{4.145127in}}%
\pgfpathlineto{\pgfqpoint{5.217372in}{4.134509in}}%
\pgfpathlineto{\pgfqpoint{5.219296in}{4.130948in}}%
\pgfpathlineto{\pgfqpoint{5.221219in}{4.137603in}}%
\pgfpathlineto{\pgfqpoint{5.223142in}{4.119171in}}%
\pgfpathlineto{\pgfqpoint{5.225066in}{4.125948in}}%
\pgfpathlineto{\pgfqpoint{5.226989in}{4.112892in}}%
\pgfpathlineto{\pgfqpoint{5.228913in}{4.118390in}}%
\pgfpathlineto{\pgfqpoint{5.230836in}{4.114987in}}%
\pgfpathlineto{\pgfqpoint{5.232759in}{4.124606in}}%
\pgfpathlineto{\pgfqpoint{5.234683in}{4.122146in}}%
\pgfpathlineto{\pgfqpoint{5.236606in}{4.131460in}}%
\pgfpathlineto{\pgfqpoint{5.238530in}{4.112630in}}%
\pgfpathlineto{\pgfqpoint{5.240453in}{4.112856in}}%
\pgfpathlineto{\pgfqpoint{5.244300in}{4.098330in}}%
\pgfpathlineto{\pgfqpoint{5.248147in}{4.071140in}}%
\pgfpathlineto{\pgfqpoint{5.250070in}{4.074204in}}%
\pgfpathlineto{\pgfqpoint{5.251994in}{4.074305in}}%
\pgfpathlineto{\pgfqpoint{5.253917in}{4.068804in}}%
\pgfpathlineto{\pgfqpoint{5.255840in}{4.076125in}}%
\pgfpathlineto{\pgfqpoint{5.257764in}{4.069876in}}%
\pgfpathlineto{\pgfqpoint{5.261611in}{4.030385in}}%
\pgfpathlineto{\pgfqpoint{5.263534in}{4.037430in}}%
\pgfpathlineto{\pgfqpoint{5.265457in}{4.039930in}}%
\pgfpathlineto{\pgfqpoint{5.269304in}{4.055748in}}%
\pgfpathlineto{\pgfqpoint{5.271228in}{4.057854in}}%
\pgfpathlineto{\pgfqpoint{5.275074in}{4.088027in}}%
\pgfpathlineto{\pgfqpoint{5.276998in}{4.080977in}}%
\pgfpathlineto{\pgfqpoint{5.278921in}{4.057753in}}%
\pgfpathlineto{\pgfqpoint{5.280845in}{4.066943in}}%
\pgfpathlineto{\pgfqpoint{5.284692in}{4.043131in}}%
\pgfpathlineto{\pgfqpoint{5.286615in}{4.041645in}}%
\pgfpathlineto{\pgfqpoint{5.288538in}{4.035194in}}%
\pgfpathlineto{\pgfqpoint{5.290462in}{4.036968in}}%
\pgfpathlineto{\pgfqpoint{5.292385in}{4.035072in}}%
\pgfpathlineto{\pgfqpoint{5.294309in}{4.026167in}}%
\pgfpathlineto{\pgfqpoint{5.296232in}{4.034095in}}%
\pgfpathlineto{\pgfqpoint{5.298155in}{4.033271in}}%
\pgfpathlineto{\pgfqpoint{5.303926in}{4.058619in}}%
\pgfpathlineto{\pgfqpoint{5.303926in}{4.058619in}}%
\pgfusepath{stroke}%
\end{pgfscope}%
\begin{pgfscope}%
\pgfpathrectangle{\pgfqpoint{3.286364in}{3.180000in}}{\pgfqpoint{2.113636in}{2.100000in}}%
\pgfusepath{clip}%
\pgfsetroundcap%
\pgfsetroundjoin%
\pgfsetlinewidth{0.602250pt}%
\definecolor{currentstroke}{rgb}{0.301961,0.686275,0.290196}%
\pgfsetstrokecolor{currentstroke}%
\pgfsetdash{}{0pt}%
\pgfpathmoveto{\pgfqpoint{3.382438in}{4.150796in}}%
\pgfpathlineto{\pgfqpoint{3.388208in}{4.141598in}}%
\pgfpathlineto{\pgfqpoint{3.390132in}{4.142457in}}%
\pgfpathlineto{\pgfqpoint{3.392055in}{4.128350in}}%
\pgfpathlineto{\pgfqpoint{3.393978in}{4.139302in}}%
\pgfpathlineto{\pgfqpoint{3.395902in}{4.135793in}}%
\pgfpathlineto{\pgfqpoint{3.397825in}{4.135975in}}%
\pgfpathlineto{\pgfqpoint{3.399749in}{4.144971in}}%
\pgfpathlineto{\pgfqpoint{3.401672in}{4.145098in}}%
\pgfpathlineto{\pgfqpoint{3.403596in}{4.141389in}}%
\pgfpathlineto{\pgfqpoint{3.411289in}{4.102820in}}%
\pgfpathlineto{\pgfqpoint{3.413213in}{4.107459in}}%
\pgfpathlineto{\pgfqpoint{3.415136in}{4.081881in}}%
\pgfpathlineto{\pgfqpoint{3.418983in}{4.105554in}}%
\pgfpathlineto{\pgfqpoint{3.422830in}{4.110833in}}%
\pgfpathlineto{\pgfqpoint{3.424753in}{4.111334in}}%
\pgfpathlineto{\pgfqpoint{3.428600in}{4.136861in}}%
\pgfpathlineto{\pgfqpoint{3.430523in}{4.129651in}}%
\pgfpathlineto{\pgfqpoint{3.434370in}{4.147542in}}%
\pgfpathlineto{\pgfqpoint{3.438217in}{4.140881in}}%
\pgfpathlineto{\pgfqpoint{3.442064in}{4.154380in}}%
\pgfpathlineto{\pgfqpoint{3.443987in}{4.149941in}}%
\pgfpathlineto{\pgfqpoint{3.445911in}{4.162069in}}%
\pgfpathlineto{\pgfqpoint{3.447834in}{4.152157in}}%
\pgfpathlineto{\pgfqpoint{3.449757in}{4.151970in}}%
\pgfpathlineto{\pgfqpoint{3.451681in}{4.149367in}}%
\pgfpathlineto{\pgfqpoint{3.453604in}{4.156441in}}%
\pgfpathlineto{\pgfqpoint{3.455528in}{4.143799in}}%
\pgfpathlineto{\pgfqpoint{3.457451in}{4.142818in}}%
\pgfpathlineto{\pgfqpoint{3.463221in}{4.098741in}}%
\pgfpathlineto{\pgfqpoint{3.467068in}{4.089711in}}%
\pgfpathlineto{\pgfqpoint{3.470915in}{4.132198in}}%
\pgfpathlineto{\pgfqpoint{3.472838in}{4.133526in}}%
\pgfpathlineto{\pgfqpoint{3.474762in}{4.144114in}}%
\pgfpathlineto{\pgfqpoint{3.476685in}{4.143590in}}%
\pgfpathlineto{\pgfqpoint{3.478609in}{4.144275in}}%
\pgfpathlineto{\pgfqpoint{3.480532in}{4.136827in}}%
\pgfpathlineto{\pgfqpoint{3.482455in}{4.141658in}}%
\pgfpathlineto{\pgfqpoint{3.488226in}{4.185672in}}%
\pgfpathlineto{\pgfqpoint{3.490149in}{4.176755in}}%
\pgfpathlineto{\pgfqpoint{3.492072in}{4.177548in}}%
\pgfpathlineto{\pgfqpoint{3.493996in}{4.181884in}}%
\pgfpathlineto{\pgfqpoint{3.495919in}{4.182754in}}%
\pgfpathlineto{\pgfqpoint{3.497843in}{4.173095in}}%
\pgfpathlineto{\pgfqpoint{3.499766in}{4.171510in}}%
\pgfpathlineto{\pgfqpoint{3.505536in}{4.193504in}}%
\pgfpathlineto{\pgfqpoint{3.509383in}{4.178511in}}%
\pgfpathlineto{\pgfqpoint{3.511307in}{4.186168in}}%
\pgfpathlineto{\pgfqpoint{3.513230in}{4.168494in}}%
\pgfpathlineto{\pgfqpoint{3.515153in}{4.170523in}}%
\pgfpathlineto{\pgfqpoint{3.517077in}{4.162942in}}%
\pgfpathlineto{\pgfqpoint{3.519000in}{4.168795in}}%
\pgfpathlineto{\pgfqpoint{3.520924in}{4.183864in}}%
\pgfpathlineto{\pgfqpoint{3.522847in}{4.188954in}}%
\pgfpathlineto{\pgfqpoint{3.524770in}{4.188290in}}%
\pgfpathlineto{\pgfqpoint{3.526694in}{4.183645in}}%
\pgfpathlineto{\pgfqpoint{3.530541in}{4.167690in}}%
\pgfpathlineto{\pgfqpoint{3.534387in}{4.196672in}}%
\pgfpathlineto{\pgfqpoint{3.536311in}{4.192392in}}%
\pgfpathlineto{\pgfqpoint{3.540158in}{4.175876in}}%
\pgfpathlineto{\pgfqpoint{3.542081in}{4.191842in}}%
\pgfpathlineto{\pgfqpoint{3.544005in}{4.189710in}}%
\pgfpathlineto{\pgfqpoint{3.545928in}{4.199815in}}%
\pgfpathlineto{\pgfqpoint{3.547851in}{4.189403in}}%
\pgfpathlineto{\pgfqpoint{3.549775in}{4.199431in}}%
\pgfpathlineto{\pgfqpoint{3.551698in}{4.199518in}}%
\pgfpathlineto{\pgfqpoint{3.553622in}{4.192686in}}%
\pgfpathlineto{\pgfqpoint{3.555545in}{4.193429in}}%
\pgfpathlineto{\pgfqpoint{3.557468in}{4.201022in}}%
\pgfpathlineto{\pgfqpoint{3.561315in}{4.221366in}}%
\pgfpathlineto{\pgfqpoint{3.563239in}{4.212119in}}%
\pgfpathlineto{\pgfqpoint{3.565162in}{4.223074in}}%
\pgfpathlineto{\pgfqpoint{3.567085in}{4.222568in}}%
\pgfpathlineto{\pgfqpoint{3.569009in}{4.225047in}}%
\pgfpathlineto{\pgfqpoint{3.570932in}{4.215978in}}%
\pgfpathlineto{\pgfqpoint{3.574779in}{4.217586in}}%
\pgfpathlineto{\pgfqpoint{3.576703in}{4.220484in}}%
\pgfpathlineto{\pgfqpoint{3.578626in}{4.209092in}}%
\pgfpathlineto{\pgfqpoint{3.582473in}{4.225355in}}%
\pgfpathlineto{\pgfqpoint{3.584396in}{4.216415in}}%
\pgfpathlineto{\pgfqpoint{3.586320in}{4.221326in}}%
\pgfpathlineto{\pgfqpoint{3.588243in}{4.213653in}}%
\pgfpathlineto{\pgfqpoint{3.590166in}{4.211205in}}%
\pgfpathlineto{\pgfqpoint{3.592090in}{4.222313in}}%
\pgfpathlineto{\pgfqpoint{3.594013in}{4.220868in}}%
\pgfpathlineto{\pgfqpoint{3.595937in}{4.243293in}}%
\pgfpathlineto{\pgfqpoint{3.597860in}{4.247423in}}%
\pgfpathlineto{\pgfqpoint{3.601707in}{4.250605in}}%
\pgfpathlineto{\pgfqpoint{3.603630in}{4.227285in}}%
\pgfpathlineto{\pgfqpoint{3.605554in}{4.238894in}}%
\pgfpathlineto{\pgfqpoint{3.613247in}{4.191408in}}%
\pgfpathlineto{\pgfqpoint{3.615171in}{4.193479in}}%
\pgfpathlineto{\pgfqpoint{3.617094in}{4.212073in}}%
\pgfpathlineto{\pgfqpoint{3.619018in}{4.207888in}}%
\pgfpathlineto{\pgfqpoint{3.620941in}{4.196516in}}%
\pgfpathlineto{\pgfqpoint{3.622864in}{4.203738in}}%
\pgfpathlineto{\pgfqpoint{3.624788in}{4.192553in}}%
\pgfpathlineto{\pgfqpoint{3.626711in}{4.194448in}}%
\pgfpathlineto{\pgfqpoint{3.628635in}{4.191038in}}%
\pgfpathlineto{\pgfqpoint{3.630558in}{4.192409in}}%
\pgfpathlineto{\pgfqpoint{3.632481in}{4.201112in}}%
\pgfpathlineto{\pgfqpoint{3.634405in}{4.202889in}}%
\pgfpathlineto{\pgfqpoint{3.636328in}{4.210614in}}%
\pgfpathlineto{\pgfqpoint{3.638252in}{4.187519in}}%
\pgfpathlineto{\pgfqpoint{3.640175in}{4.185030in}}%
\pgfpathlineto{\pgfqpoint{3.642099in}{4.184583in}}%
\pgfpathlineto{\pgfqpoint{3.645945in}{4.170548in}}%
\pgfpathlineto{\pgfqpoint{3.649792in}{4.175495in}}%
\pgfpathlineto{\pgfqpoint{3.651716in}{4.186441in}}%
\pgfpathlineto{\pgfqpoint{3.653639in}{4.190134in}}%
\pgfpathlineto{\pgfqpoint{3.655562in}{4.187275in}}%
\pgfpathlineto{\pgfqpoint{3.659409in}{4.165367in}}%
\pgfpathlineto{\pgfqpoint{3.661333in}{4.173995in}}%
\pgfpathlineto{\pgfqpoint{3.663256in}{4.174290in}}%
\pgfpathlineto{\pgfqpoint{3.669026in}{4.161811in}}%
\pgfpathlineto{\pgfqpoint{3.672873in}{4.150144in}}%
\pgfpathlineto{\pgfqpoint{3.676720in}{4.167631in}}%
\pgfpathlineto{\pgfqpoint{3.684414in}{4.150124in}}%
\pgfpathlineto{\pgfqpoint{3.686337in}{4.153540in}}%
\pgfpathlineto{\pgfqpoint{3.688260in}{4.163361in}}%
\pgfpathlineto{\pgfqpoint{3.690184in}{4.163969in}}%
\pgfpathlineto{\pgfqpoint{3.692107in}{4.169826in}}%
\pgfpathlineto{\pgfqpoint{3.694031in}{4.162695in}}%
\pgfpathlineto{\pgfqpoint{3.695954in}{4.166993in}}%
\pgfpathlineto{\pgfqpoint{3.697877in}{4.188824in}}%
\pgfpathlineto{\pgfqpoint{3.699801in}{4.172659in}}%
\pgfpathlineto{\pgfqpoint{3.701724in}{4.174434in}}%
\pgfpathlineto{\pgfqpoint{3.705571in}{4.189829in}}%
\pgfpathlineto{\pgfqpoint{3.707494in}{4.212356in}}%
\pgfpathlineto{\pgfqpoint{3.711341in}{4.199584in}}%
\pgfpathlineto{\pgfqpoint{3.713265in}{4.211760in}}%
\pgfpathlineto{\pgfqpoint{3.715188in}{4.212694in}}%
\pgfpathlineto{\pgfqpoint{3.719035in}{4.195157in}}%
\pgfpathlineto{\pgfqpoint{3.720958in}{4.203870in}}%
\pgfpathlineto{\pgfqpoint{3.722882in}{4.197782in}}%
\pgfpathlineto{\pgfqpoint{3.724805in}{4.196512in}}%
\pgfpathlineto{\pgfqpoint{3.726729in}{4.183185in}}%
\pgfpathlineto{\pgfqpoint{3.728652in}{4.181636in}}%
\pgfpathlineto{\pgfqpoint{3.730575in}{4.185093in}}%
\pgfpathlineto{\pgfqpoint{3.732499in}{4.180099in}}%
\pgfpathlineto{\pgfqpoint{3.734422in}{4.178699in}}%
\pgfpathlineto{\pgfqpoint{3.736346in}{4.179203in}}%
\pgfpathlineto{\pgfqpoint{3.738269in}{4.188363in}}%
\pgfpathlineto{\pgfqpoint{3.740192in}{4.190931in}}%
\pgfpathlineto{\pgfqpoint{3.742116in}{4.202474in}}%
\pgfpathlineto{\pgfqpoint{3.744039in}{4.202575in}}%
\pgfpathlineto{\pgfqpoint{3.745963in}{4.198774in}}%
\pgfpathlineto{\pgfqpoint{3.749810in}{4.217709in}}%
\pgfpathlineto{\pgfqpoint{3.753656in}{4.220563in}}%
\pgfpathlineto{\pgfqpoint{3.755580in}{4.199430in}}%
\pgfpathlineto{\pgfqpoint{3.757503in}{4.204527in}}%
\pgfpathlineto{\pgfqpoint{3.759427in}{4.192371in}}%
\pgfpathlineto{\pgfqpoint{3.761350in}{4.205685in}}%
\pgfpathlineto{\pgfqpoint{3.763273in}{4.200094in}}%
\pgfpathlineto{\pgfqpoint{3.765197in}{4.216782in}}%
\pgfpathlineto{\pgfqpoint{3.767120in}{4.220206in}}%
\pgfpathlineto{\pgfqpoint{3.769044in}{4.216413in}}%
\pgfpathlineto{\pgfqpoint{3.770967in}{4.221734in}}%
\pgfpathlineto{\pgfqpoint{3.772890in}{4.230804in}}%
\pgfpathlineto{\pgfqpoint{3.774814in}{4.227147in}}%
\pgfpathlineto{\pgfqpoint{3.776737in}{4.234825in}}%
\pgfpathlineto{\pgfqpoint{3.778661in}{4.227910in}}%
\pgfpathlineto{\pgfqpoint{3.780584in}{4.234477in}}%
\pgfpathlineto{\pgfqpoint{3.782508in}{4.222066in}}%
\pgfpathlineto{\pgfqpoint{3.784431in}{4.227381in}}%
\pgfpathlineto{\pgfqpoint{3.790201in}{4.255684in}}%
\pgfpathlineto{\pgfqpoint{3.792125in}{4.257465in}}%
\pgfpathlineto{\pgfqpoint{3.795971in}{4.273764in}}%
\pgfpathlineto{\pgfqpoint{3.797895in}{4.281484in}}%
\pgfpathlineto{\pgfqpoint{3.799818in}{4.275113in}}%
\pgfpathlineto{\pgfqpoint{3.805588in}{4.313884in}}%
\pgfpathlineto{\pgfqpoint{3.807512in}{4.308858in}}%
\pgfpathlineto{\pgfqpoint{3.809435in}{4.314555in}}%
\pgfpathlineto{\pgfqpoint{3.815205in}{4.301565in}}%
\pgfpathlineto{\pgfqpoint{3.819052in}{4.280688in}}%
\pgfpathlineto{\pgfqpoint{3.820976in}{4.272239in}}%
\pgfpathlineto{\pgfqpoint{3.822899in}{4.288805in}}%
\pgfpathlineto{\pgfqpoint{3.826746in}{4.264218in}}%
\pgfpathlineto{\pgfqpoint{3.830593in}{4.253870in}}%
\pgfpathlineto{\pgfqpoint{3.834440in}{4.238613in}}%
\pgfpathlineto{\pgfqpoint{3.836363in}{4.222340in}}%
\pgfpathlineto{\pgfqpoint{3.838286in}{4.220402in}}%
\pgfpathlineto{\pgfqpoint{3.842133in}{4.202635in}}%
\pgfpathlineto{\pgfqpoint{3.844057in}{4.213125in}}%
\pgfpathlineto{\pgfqpoint{3.845980in}{4.217080in}}%
\pgfpathlineto{\pgfqpoint{3.849827in}{4.230412in}}%
\pgfpathlineto{\pgfqpoint{3.851750in}{4.230643in}}%
\pgfpathlineto{\pgfqpoint{3.853674in}{4.225982in}}%
\pgfpathlineto{\pgfqpoint{3.857521in}{4.241417in}}%
\pgfpathlineto{\pgfqpoint{3.861367in}{4.231785in}}%
\pgfpathlineto{\pgfqpoint{3.867138in}{4.183968in}}%
\pgfpathlineto{\pgfqpoint{3.870984in}{4.181480in}}%
\pgfpathlineto{\pgfqpoint{3.874831in}{4.198792in}}%
\pgfpathlineto{\pgfqpoint{3.880601in}{4.180532in}}%
\pgfpathlineto{\pgfqpoint{3.884448in}{4.193477in}}%
\pgfpathlineto{\pgfqpoint{3.886372in}{4.178746in}}%
\pgfpathlineto{\pgfqpoint{3.888295in}{4.180108in}}%
\pgfpathlineto{\pgfqpoint{3.890219in}{4.173770in}}%
\pgfpathlineto{\pgfqpoint{3.892142in}{4.174788in}}%
\pgfpathlineto{\pgfqpoint{3.895989in}{4.201633in}}%
\pgfpathlineto{\pgfqpoint{3.897912in}{4.222656in}}%
\pgfpathlineto{\pgfqpoint{3.899836in}{4.226132in}}%
\pgfpathlineto{\pgfqpoint{3.901759in}{4.234577in}}%
\pgfpathlineto{\pgfqpoint{3.903682in}{4.224607in}}%
\pgfpathlineto{\pgfqpoint{3.905606in}{4.220692in}}%
\pgfpathlineto{\pgfqpoint{3.907529in}{4.225745in}}%
\pgfpathlineto{\pgfqpoint{3.909453in}{4.226496in}}%
\pgfpathlineto{\pgfqpoint{3.911376in}{4.258325in}}%
\pgfpathlineto{\pgfqpoint{3.913299in}{4.263787in}}%
\pgfpathlineto{\pgfqpoint{3.915223in}{4.263870in}}%
\pgfpathlineto{\pgfqpoint{3.917146in}{4.284974in}}%
\pgfpathlineto{\pgfqpoint{3.920993in}{4.289568in}}%
\pgfpathlineto{\pgfqpoint{3.922917in}{4.293117in}}%
\pgfpathlineto{\pgfqpoint{3.924840in}{4.281437in}}%
\pgfpathlineto{\pgfqpoint{3.926763in}{4.279515in}}%
\pgfpathlineto{\pgfqpoint{3.928687in}{4.250396in}}%
\pgfpathlineto{\pgfqpoint{3.930610in}{4.260056in}}%
\pgfpathlineto{\pgfqpoint{3.932534in}{4.245701in}}%
\pgfpathlineto{\pgfqpoint{3.934457in}{4.254044in}}%
\pgfpathlineto{\pgfqpoint{3.936380in}{4.253762in}}%
\pgfpathlineto{\pgfqpoint{3.938304in}{4.248118in}}%
\pgfpathlineto{\pgfqpoint{3.940227in}{4.236456in}}%
\pgfpathlineto{\pgfqpoint{3.942151in}{4.216913in}}%
\pgfpathlineto{\pgfqpoint{3.944074in}{4.230199in}}%
\pgfpathlineto{\pgfqpoint{3.945997in}{4.234301in}}%
\pgfpathlineto{\pgfqpoint{3.947921in}{4.256452in}}%
\pgfpathlineto{\pgfqpoint{3.949844in}{4.261583in}}%
\pgfpathlineto{\pgfqpoint{3.953691in}{4.243407in}}%
\pgfpathlineto{\pgfqpoint{3.955614in}{4.267329in}}%
\pgfpathlineto{\pgfqpoint{3.959461in}{4.273482in}}%
\pgfpathlineto{\pgfqpoint{3.963308in}{4.279971in}}%
\pgfpathlineto{\pgfqpoint{3.965232in}{4.288064in}}%
\pgfpathlineto{\pgfqpoint{3.971002in}{4.266534in}}%
\pgfpathlineto{\pgfqpoint{3.972925in}{4.273620in}}%
\pgfpathlineto{\pgfqpoint{3.974849in}{4.257616in}}%
\pgfpathlineto{\pgfqpoint{3.980619in}{4.299138in}}%
\pgfpathlineto{\pgfqpoint{3.984466in}{4.317762in}}%
\pgfpathlineto{\pgfqpoint{3.988312in}{4.334682in}}%
\pgfpathlineto{\pgfqpoint{3.990236in}{4.329009in}}%
\pgfpathlineto{\pgfqpoint{3.992159in}{4.331809in}}%
\pgfpathlineto{\pgfqpoint{3.994083in}{4.329586in}}%
\pgfpathlineto{\pgfqpoint{3.996006in}{4.335999in}}%
\pgfpathlineto{\pgfqpoint{4.003700in}{4.301239in}}%
\pgfpathlineto{\pgfqpoint{4.005623in}{4.303930in}}%
\pgfpathlineto{\pgfqpoint{4.007547in}{4.311540in}}%
\pgfpathlineto{\pgfqpoint{4.011393in}{4.296372in}}%
\pgfpathlineto{\pgfqpoint{4.015240in}{4.289809in}}%
\pgfpathlineto{\pgfqpoint{4.017164in}{4.308287in}}%
\pgfpathlineto{\pgfqpoint{4.019087in}{4.304381in}}%
\pgfpathlineto{\pgfqpoint{4.022934in}{4.339679in}}%
\pgfpathlineto{\pgfqpoint{4.026781in}{4.334503in}}%
\pgfpathlineto{\pgfqpoint{4.028704in}{4.339121in}}%
\pgfpathlineto{\pgfqpoint{4.032551in}{4.322577in}}%
\pgfpathlineto{\pgfqpoint{4.034474in}{4.337909in}}%
\pgfpathlineto{\pgfqpoint{4.036398in}{4.343407in}}%
\pgfpathlineto{\pgfqpoint{4.038321in}{4.343290in}}%
\pgfpathlineto{\pgfqpoint{4.040245in}{4.336104in}}%
\pgfpathlineto{\pgfqpoint{4.044091in}{4.346274in}}%
\pgfpathlineto{\pgfqpoint{4.046015in}{4.351756in}}%
\pgfpathlineto{\pgfqpoint{4.047938in}{4.348433in}}%
\pgfpathlineto{\pgfqpoint{4.049862in}{4.364238in}}%
\pgfpathlineto{\pgfqpoint{4.051785in}{4.369474in}}%
\pgfpathlineto{\pgfqpoint{4.053708in}{4.355574in}}%
\pgfpathlineto{\pgfqpoint{4.055632in}{4.371197in}}%
\pgfpathlineto{\pgfqpoint{4.057555in}{4.374586in}}%
\pgfpathlineto{\pgfqpoint{4.059479in}{4.374624in}}%
\pgfpathlineto{\pgfqpoint{4.061402in}{4.381628in}}%
\pgfpathlineto{\pgfqpoint{4.065249in}{4.363170in}}%
\pgfpathlineto{\pgfqpoint{4.069096in}{4.389436in}}%
\pgfpathlineto{\pgfqpoint{4.071019in}{4.384134in}}%
\pgfpathlineto{\pgfqpoint{4.072943in}{4.384656in}}%
\pgfpathlineto{\pgfqpoint{4.074866in}{4.378738in}}%
\pgfpathlineto{\pgfqpoint{4.076789in}{4.387761in}}%
\pgfpathlineto{\pgfqpoint{4.080636in}{4.364476in}}%
\pgfpathlineto{\pgfqpoint{4.082560in}{4.375825in}}%
\pgfpathlineto{\pgfqpoint{4.084483in}{4.373651in}}%
\pgfpathlineto{\pgfqpoint{4.086406in}{4.362450in}}%
\pgfpathlineto{\pgfqpoint{4.088330in}{4.365945in}}%
\pgfpathlineto{\pgfqpoint{4.090253in}{4.341026in}}%
\pgfpathlineto{\pgfqpoint{4.092177in}{4.349189in}}%
\pgfpathlineto{\pgfqpoint{4.094100in}{4.352498in}}%
\pgfpathlineto{\pgfqpoint{4.096024in}{4.342675in}}%
\pgfpathlineto{\pgfqpoint{4.101794in}{4.330443in}}%
\pgfpathlineto{\pgfqpoint{4.107564in}{4.336111in}}%
\pgfpathlineto{\pgfqpoint{4.109487in}{4.328960in}}%
\pgfpathlineto{\pgfqpoint{4.113334in}{4.283139in}}%
\pgfpathlineto{\pgfqpoint{4.115258in}{4.281440in}}%
\pgfpathlineto{\pgfqpoint{4.117181in}{4.268561in}}%
\pgfpathlineto{\pgfqpoint{4.119104in}{4.268474in}}%
\pgfpathlineto{\pgfqpoint{4.121028in}{4.273168in}}%
\pgfpathlineto{\pgfqpoint{4.128721in}{4.259436in}}%
\pgfpathlineto{\pgfqpoint{4.132568in}{4.245825in}}%
\pgfpathlineto{\pgfqpoint{4.134492in}{4.256166in}}%
\pgfpathlineto{\pgfqpoint{4.136415in}{4.253937in}}%
\pgfpathlineto{\pgfqpoint{4.138339in}{4.268092in}}%
\pgfpathlineto{\pgfqpoint{4.140262in}{4.294207in}}%
\pgfpathlineto{\pgfqpoint{4.142185in}{4.292761in}}%
\pgfpathlineto{\pgfqpoint{4.144109in}{4.300880in}}%
\pgfpathlineto{\pgfqpoint{4.147956in}{4.262700in}}%
\pgfpathlineto{\pgfqpoint{4.149879in}{4.259866in}}%
\pgfpathlineto{\pgfqpoint{4.151802in}{4.252378in}}%
\pgfpathlineto{\pgfqpoint{4.153726in}{4.266692in}}%
\pgfpathlineto{\pgfqpoint{4.155649in}{4.271983in}}%
\pgfpathlineto{\pgfqpoint{4.157573in}{4.265307in}}%
\pgfpathlineto{\pgfqpoint{4.163343in}{4.264766in}}%
\pgfpathlineto{\pgfqpoint{4.165266in}{4.283386in}}%
\pgfpathlineto{\pgfqpoint{4.167190in}{4.283171in}}%
\pgfpathlineto{\pgfqpoint{4.171037in}{4.267140in}}%
\pgfpathlineto{\pgfqpoint{4.174883in}{4.283797in}}%
\pgfpathlineto{\pgfqpoint{4.176807in}{4.297904in}}%
\pgfpathlineto{\pgfqpoint{4.180654in}{4.275181in}}%
\pgfpathlineto{\pgfqpoint{4.182577in}{4.279509in}}%
\pgfpathlineto{\pgfqpoint{4.184500in}{4.270211in}}%
\pgfpathlineto{\pgfqpoint{4.186424in}{4.276469in}}%
\pgfpathlineto{\pgfqpoint{4.190271in}{4.297890in}}%
\pgfpathlineto{\pgfqpoint{4.192194in}{4.297011in}}%
\pgfpathlineto{\pgfqpoint{4.194117in}{4.293424in}}%
\pgfpathlineto{\pgfqpoint{4.196041in}{4.303032in}}%
\pgfpathlineto{\pgfqpoint{4.197964in}{4.295857in}}%
\pgfpathlineto{\pgfqpoint{4.199888in}{4.294636in}}%
\pgfpathlineto{\pgfqpoint{4.203735in}{4.301659in}}%
\pgfpathlineto{\pgfqpoint{4.205658in}{4.306069in}}%
\pgfpathlineto{\pgfqpoint{4.207581in}{4.289406in}}%
\pgfpathlineto{\pgfqpoint{4.209505in}{4.285635in}}%
\pgfpathlineto{\pgfqpoint{4.211428in}{4.270171in}}%
\pgfpathlineto{\pgfqpoint{4.213352in}{4.276484in}}%
\pgfpathlineto{\pgfqpoint{4.217198in}{4.300818in}}%
\pgfpathlineto{\pgfqpoint{4.219122in}{4.286645in}}%
\pgfpathlineto{\pgfqpoint{4.221045in}{4.302644in}}%
\pgfpathlineto{\pgfqpoint{4.222969in}{4.301952in}}%
\pgfpathlineto{\pgfqpoint{4.224892in}{4.291097in}}%
\pgfpathlineto{\pgfqpoint{4.226815in}{4.292734in}}%
\pgfpathlineto{\pgfqpoint{4.228739in}{4.298805in}}%
\pgfpathlineto{\pgfqpoint{4.230662in}{4.300432in}}%
\pgfpathlineto{\pgfqpoint{4.232586in}{4.299594in}}%
\pgfpathlineto{\pgfqpoint{4.234509in}{4.306140in}}%
\pgfpathlineto{\pgfqpoint{4.236433in}{4.324438in}}%
\pgfpathlineto{\pgfqpoint{4.238356in}{4.308891in}}%
\pgfpathlineto{\pgfqpoint{4.240279in}{4.305975in}}%
\pgfpathlineto{\pgfqpoint{4.242203in}{4.299708in}}%
\pgfpathlineto{\pgfqpoint{4.244126in}{4.297310in}}%
\pgfpathlineto{\pgfqpoint{4.246050in}{4.288165in}}%
\pgfpathlineto{\pgfqpoint{4.247973in}{4.293022in}}%
\pgfpathlineto{\pgfqpoint{4.249896in}{4.319519in}}%
\pgfpathlineto{\pgfqpoint{4.253743in}{4.294669in}}%
\pgfpathlineto{\pgfqpoint{4.255667in}{4.273685in}}%
\pgfpathlineto{\pgfqpoint{4.257590in}{4.281443in}}%
\pgfpathlineto{\pgfqpoint{4.259513in}{4.301427in}}%
\pgfpathlineto{\pgfqpoint{4.261437in}{4.299377in}}%
\pgfpathlineto{\pgfqpoint{4.263360in}{4.301681in}}%
\pgfpathlineto{\pgfqpoint{4.265284in}{4.298772in}}%
\pgfpathlineto{\pgfqpoint{4.267207in}{4.286676in}}%
\pgfpathlineto{\pgfqpoint{4.271054in}{4.293766in}}%
\pgfpathlineto{\pgfqpoint{4.272977in}{4.298218in}}%
\pgfpathlineto{\pgfqpoint{4.274901in}{4.292129in}}%
\pgfpathlineto{\pgfqpoint{4.276824in}{4.306901in}}%
\pgfpathlineto{\pgfqpoint{4.280671in}{4.286562in}}%
\pgfpathlineto{\pgfqpoint{4.282594in}{4.282929in}}%
\pgfpathlineto{\pgfqpoint{4.284518in}{4.286174in}}%
\pgfpathlineto{\pgfqpoint{4.288365in}{4.313346in}}%
\pgfpathlineto{\pgfqpoint{4.290288in}{4.303366in}}%
\pgfpathlineto{\pgfqpoint{4.292211in}{4.315880in}}%
\pgfpathlineto{\pgfqpoint{4.296058in}{4.298328in}}%
\pgfpathlineto{\pgfqpoint{4.297982in}{4.287213in}}%
\pgfpathlineto{\pgfqpoint{4.301828in}{4.301963in}}%
\pgfpathlineto{\pgfqpoint{4.307599in}{4.293264in}}%
\pgfpathlineto{\pgfqpoint{4.309522in}{4.278132in}}%
\pgfpathlineto{\pgfqpoint{4.311446in}{4.278192in}}%
\pgfpathlineto{\pgfqpoint{4.313369in}{4.275129in}}%
\pgfpathlineto{\pgfqpoint{4.315292in}{4.266620in}}%
\pgfpathlineto{\pgfqpoint{4.319139in}{4.281731in}}%
\pgfpathlineto{\pgfqpoint{4.321063in}{4.276261in}}%
\pgfpathlineto{\pgfqpoint{4.328756in}{4.313439in}}%
\pgfpathlineto{\pgfqpoint{4.330680in}{4.315990in}}%
\pgfpathlineto{\pgfqpoint{4.332603in}{4.332005in}}%
\pgfpathlineto{\pgfqpoint{4.334526in}{4.331041in}}%
\pgfpathlineto{\pgfqpoint{4.336450in}{4.334471in}}%
\pgfpathlineto{\pgfqpoint{4.338373in}{4.320078in}}%
\pgfpathlineto{\pgfqpoint{4.340297in}{4.322898in}}%
\pgfpathlineto{\pgfqpoint{4.344144in}{4.319310in}}%
\pgfpathlineto{\pgfqpoint{4.346067in}{4.327607in}}%
\pgfpathlineto{\pgfqpoint{4.349914in}{4.326437in}}%
\pgfpathlineto{\pgfqpoint{4.351837in}{4.329510in}}%
\pgfpathlineto{\pgfqpoint{4.353761in}{4.337329in}}%
\pgfpathlineto{\pgfqpoint{4.355684in}{4.338188in}}%
\pgfpathlineto{\pgfqpoint{4.361454in}{4.382581in}}%
\pgfpathlineto{\pgfqpoint{4.363378in}{4.374911in}}%
\pgfpathlineto{\pgfqpoint{4.365301in}{4.375273in}}%
\pgfpathlineto{\pgfqpoint{4.367224in}{4.372148in}}%
\pgfpathlineto{\pgfqpoint{4.369148in}{4.386952in}}%
\pgfpathlineto{\pgfqpoint{4.371071in}{4.381254in}}%
\pgfpathlineto{\pgfqpoint{4.372995in}{4.381568in}}%
\pgfpathlineto{\pgfqpoint{4.374918in}{4.374484in}}%
\pgfpathlineto{\pgfqpoint{4.376842in}{4.372873in}}%
\pgfpathlineto{\pgfqpoint{4.378765in}{4.356842in}}%
\pgfpathlineto{\pgfqpoint{4.380688in}{4.354508in}}%
\pgfpathlineto{\pgfqpoint{4.384535in}{4.344244in}}%
\pgfpathlineto{\pgfqpoint{4.386459in}{4.358884in}}%
\pgfpathlineto{\pgfqpoint{4.388382in}{4.355547in}}%
\pgfpathlineto{\pgfqpoint{4.390305in}{4.356960in}}%
\pgfpathlineto{\pgfqpoint{4.392229in}{4.355041in}}%
\pgfpathlineto{\pgfqpoint{4.394152in}{4.358837in}}%
\pgfpathlineto{\pgfqpoint{4.397999in}{4.342533in}}%
\pgfpathlineto{\pgfqpoint{4.399922in}{4.352814in}}%
\pgfpathlineto{\pgfqpoint{4.401846in}{4.354515in}}%
\pgfpathlineto{\pgfqpoint{4.403769in}{4.368986in}}%
\pgfpathlineto{\pgfqpoint{4.405693in}{4.356887in}}%
\pgfpathlineto{\pgfqpoint{4.407616in}{4.358106in}}%
\pgfpathlineto{\pgfqpoint{4.409539in}{4.338206in}}%
\pgfpathlineto{\pgfqpoint{4.413386in}{4.330644in}}%
\pgfpathlineto{\pgfqpoint{4.415310in}{4.323492in}}%
\pgfpathlineto{\pgfqpoint{4.417233in}{4.324636in}}%
\pgfpathlineto{\pgfqpoint{4.419157in}{4.323336in}}%
\pgfpathlineto{\pgfqpoint{4.421080in}{4.317819in}}%
\pgfpathlineto{\pgfqpoint{4.424927in}{4.328635in}}%
\pgfpathlineto{\pgfqpoint{4.426850in}{4.320894in}}%
\pgfpathlineto{\pgfqpoint{4.428774in}{4.335602in}}%
\pgfpathlineto{\pgfqpoint{4.430697in}{4.327544in}}%
\pgfpathlineto{\pgfqpoint{4.432620in}{4.341248in}}%
\pgfpathlineto{\pgfqpoint{4.434544in}{4.331986in}}%
\pgfpathlineto{\pgfqpoint{4.436467in}{4.331407in}}%
\pgfpathlineto{\pgfqpoint{4.438391in}{4.317633in}}%
\pgfpathlineto{\pgfqpoint{4.440314in}{4.329511in}}%
\pgfpathlineto{\pgfqpoint{4.442237in}{4.321801in}}%
\pgfpathlineto{\pgfqpoint{4.444161in}{4.324290in}}%
\pgfpathlineto{\pgfqpoint{4.446084in}{4.329343in}}%
\pgfpathlineto{\pgfqpoint{4.453778in}{4.284114in}}%
\pgfpathlineto{\pgfqpoint{4.455701in}{4.299031in}}%
\pgfpathlineto{\pgfqpoint{4.459548in}{4.297584in}}%
\pgfpathlineto{\pgfqpoint{4.461472in}{4.301582in}}%
\pgfpathlineto{\pgfqpoint{4.465318in}{4.335192in}}%
\pgfpathlineto{\pgfqpoint{4.467242in}{4.336918in}}%
\pgfpathlineto{\pgfqpoint{4.469165in}{4.345758in}}%
\pgfpathlineto{\pgfqpoint{4.471089in}{4.344467in}}%
\pgfpathlineto{\pgfqpoint{4.473012in}{4.338297in}}%
\pgfpathlineto{\pgfqpoint{4.474935in}{4.342579in}}%
\pgfpathlineto{\pgfqpoint{4.478782in}{4.320717in}}%
\pgfpathlineto{\pgfqpoint{4.480706in}{4.324483in}}%
\pgfpathlineto{\pgfqpoint{4.484553in}{4.342689in}}%
\pgfpathlineto{\pgfqpoint{4.486476in}{4.340892in}}%
\pgfpathlineto{\pgfqpoint{4.488399in}{4.347579in}}%
\pgfpathlineto{\pgfqpoint{4.492246in}{4.352807in}}%
\pgfpathlineto{\pgfqpoint{4.494170in}{4.335742in}}%
\pgfpathlineto{\pgfqpoint{4.499940in}{4.349952in}}%
\pgfpathlineto{\pgfqpoint{4.501863in}{4.347003in}}%
\pgfpathlineto{\pgfqpoint{4.503787in}{4.355487in}}%
\pgfpathlineto{\pgfqpoint{4.505710in}{4.348046in}}%
\pgfpathlineto{\pgfqpoint{4.507633in}{4.357427in}}%
\pgfpathlineto{\pgfqpoint{4.509557in}{4.352203in}}%
\pgfpathlineto{\pgfqpoint{4.513404in}{4.366879in}}%
\pgfpathlineto{\pgfqpoint{4.515327in}{4.360728in}}%
\pgfpathlineto{\pgfqpoint{4.517251in}{4.373294in}}%
\pgfpathlineto{\pgfqpoint{4.519174in}{4.372141in}}%
\pgfpathlineto{\pgfqpoint{4.521097in}{4.391084in}}%
\pgfpathlineto{\pgfqpoint{4.523021in}{4.387896in}}%
\pgfpathlineto{\pgfqpoint{4.524944in}{4.387114in}}%
\pgfpathlineto{\pgfqpoint{4.526868in}{4.383890in}}%
\pgfpathlineto{\pgfqpoint{4.528791in}{4.392660in}}%
\pgfpathlineto{\pgfqpoint{4.530714in}{4.420881in}}%
\pgfpathlineto{\pgfqpoint{4.532638in}{4.416692in}}%
\pgfpathlineto{\pgfqpoint{4.536485in}{4.429550in}}%
\pgfpathlineto{\pgfqpoint{4.538408in}{4.443269in}}%
\pgfpathlineto{\pgfqpoint{4.540331in}{4.440126in}}%
\pgfpathlineto{\pgfqpoint{4.542255in}{4.442658in}}%
\pgfpathlineto{\pgfqpoint{4.544178in}{4.459328in}}%
\pgfpathlineto{\pgfqpoint{4.546102in}{4.460961in}}%
\pgfpathlineto{\pgfqpoint{4.548025in}{4.452426in}}%
\pgfpathlineto{\pgfqpoint{4.551872in}{4.450699in}}%
\pgfpathlineto{\pgfqpoint{4.553795in}{4.436000in}}%
\pgfpathlineto{\pgfqpoint{4.555719in}{4.440092in}}%
\pgfpathlineto{\pgfqpoint{4.557642in}{4.438922in}}%
\pgfpathlineto{\pgfqpoint{4.563412in}{4.452892in}}%
\pgfpathlineto{\pgfqpoint{4.567259in}{4.435175in}}%
\pgfpathlineto{\pgfqpoint{4.569183in}{4.453387in}}%
\pgfpathlineto{\pgfqpoint{4.571106in}{4.451204in}}%
\pgfpathlineto{\pgfqpoint{4.573029in}{4.454835in}}%
\pgfpathlineto{\pgfqpoint{4.574953in}{4.448485in}}%
\pgfpathlineto{\pgfqpoint{4.576876in}{4.447822in}}%
\pgfpathlineto{\pgfqpoint{4.578800in}{4.467506in}}%
\pgfpathlineto{\pgfqpoint{4.580723in}{4.462959in}}%
\pgfpathlineto{\pgfqpoint{4.584570in}{4.482278in}}%
\pgfpathlineto{\pgfqpoint{4.586493in}{4.473233in}}%
\pgfpathlineto{\pgfqpoint{4.588417in}{4.473746in}}%
\pgfpathlineto{\pgfqpoint{4.590340in}{4.471850in}}%
\pgfpathlineto{\pgfqpoint{4.592264in}{4.463094in}}%
\pgfpathlineto{\pgfqpoint{4.594187in}{4.482609in}}%
\pgfpathlineto{\pgfqpoint{4.596110in}{4.467745in}}%
\pgfpathlineto{\pgfqpoint{4.598034in}{4.467357in}}%
\pgfpathlineto{\pgfqpoint{4.601881in}{4.452183in}}%
\pgfpathlineto{\pgfqpoint{4.605727in}{4.433081in}}%
\pgfpathlineto{\pgfqpoint{4.609574in}{4.433769in}}%
\pgfpathlineto{\pgfqpoint{4.611498in}{4.430232in}}%
\pgfpathlineto{\pgfqpoint{4.613421in}{4.417903in}}%
\pgfpathlineto{\pgfqpoint{4.615344in}{4.422241in}}%
\pgfpathlineto{\pgfqpoint{4.617268in}{4.418991in}}%
\pgfpathlineto{\pgfqpoint{4.619191in}{4.419307in}}%
\pgfpathlineto{\pgfqpoint{4.621115in}{4.426626in}}%
\pgfpathlineto{\pgfqpoint{4.623038in}{4.420968in}}%
\pgfpathlineto{\pgfqpoint{4.624962in}{4.421508in}}%
\pgfpathlineto{\pgfqpoint{4.626885in}{4.408400in}}%
\pgfpathlineto{\pgfqpoint{4.628808in}{4.403520in}}%
\pgfpathlineto{\pgfqpoint{4.630732in}{4.390353in}}%
\pgfpathlineto{\pgfqpoint{4.632655in}{4.395319in}}%
\pgfpathlineto{\pgfqpoint{4.634579in}{4.393717in}}%
\pgfpathlineto{\pgfqpoint{4.636502in}{4.399780in}}%
\pgfpathlineto{\pgfqpoint{4.638425in}{4.400946in}}%
\pgfpathlineto{\pgfqpoint{4.640349in}{4.388938in}}%
\pgfpathlineto{\pgfqpoint{4.642272in}{4.386624in}}%
\pgfpathlineto{\pgfqpoint{4.644196in}{4.380542in}}%
\pgfpathlineto{\pgfqpoint{4.646119in}{4.370520in}}%
\pgfpathlineto{\pgfqpoint{4.649966in}{4.391027in}}%
\pgfpathlineto{\pgfqpoint{4.651889in}{4.387342in}}%
\pgfpathlineto{\pgfqpoint{4.653813in}{4.392670in}}%
\pgfpathlineto{\pgfqpoint{4.657660in}{4.373390in}}%
\pgfpathlineto{\pgfqpoint{4.661506in}{4.373087in}}%
\pgfpathlineto{\pgfqpoint{4.663430in}{4.365540in}}%
\pgfpathlineto{\pgfqpoint{4.665353in}{4.363119in}}%
\pgfpathlineto{\pgfqpoint{4.667277in}{4.362541in}}%
\pgfpathlineto{\pgfqpoint{4.669200in}{4.366043in}}%
\pgfpathlineto{\pgfqpoint{4.673047in}{4.340583in}}%
\pgfpathlineto{\pgfqpoint{4.674970in}{4.346103in}}%
\pgfpathlineto{\pgfqpoint{4.676894in}{4.331174in}}%
\pgfpathlineto{\pgfqpoint{4.680740in}{4.351265in}}%
\pgfpathlineto{\pgfqpoint{4.682664in}{4.332775in}}%
\pgfpathlineto{\pgfqpoint{4.684587in}{4.332330in}}%
\pgfpathlineto{\pgfqpoint{4.686511in}{4.348499in}}%
\pgfpathlineto{\pgfqpoint{4.690358in}{4.332163in}}%
\pgfpathlineto{\pgfqpoint{4.692281in}{4.351067in}}%
\pgfpathlineto{\pgfqpoint{4.694204in}{4.354782in}}%
\pgfpathlineto{\pgfqpoint{4.696128in}{4.318589in}}%
\pgfpathlineto{\pgfqpoint{4.698051in}{4.303475in}}%
\pgfpathlineto{\pgfqpoint{4.701898in}{4.308667in}}%
\pgfpathlineto{\pgfqpoint{4.703821in}{4.319337in}}%
\pgfpathlineto{\pgfqpoint{4.707668in}{4.305256in}}%
\pgfpathlineto{\pgfqpoint{4.709592in}{4.309629in}}%
\pgfpathlineto{\pgfqpoint{4.711515in}{4.308340in}}%
\pgfpathlineto{\pgfqpoint{4.713438in}{4.298759in}}%
\pgfpathlineto{\pgfqpoint{4.715362in}{4.308738in}}%
\pgfpathlineto{\pgfqpoint{4.717285in}{4.300154in}}%
\pgfpathlineto{\pgfqpoint{4.719209in}{4.286172in}}%
\pgfpathlineto{\pgfqpoint{4.721132in}{4.258847in}}%
\pgfpathlineto{\pgfqpoint{4.723055in}{4.254057in}}%
\pgfpathlineto{\pgfqpoint{4.724979in}{4.259800in}}%
\pgfpathlineto{\pgfqpoint{4.726902in}{4.258968in}}%
\pgfpathlineto{\pgfqpoint{4.728826in}{4.253151in}}%
\pgfpathlineto{\pgfqpoint{4.730749in}{4.236200in}}%
\pgfpathlineto{\pgfqpoint{4.732673in}{4.252556in}}%
\pgfpathlineto{\pgfqpoint{4.734596in}{4.243489in}}%
\pgfpathlineto{\pgfqpoint{4.736519in}{4.267700in}}%
\pgfpathlineto{\pgfqpoint{4.740366in}{4.237561in}}%
\pgfpathlineto{\pgfqpoint{4.746136in}{4.268369in}}%
\pgfpathlineto{\pgfqpoint{4.749983in}{4.280900in}}%
\pgfpathlineto{\pgfqpoint{4.751907in}{4.275521in}}%
\pgfpathlineto{\pgfqpoint{4.753830in}{4.260486in}}%
\pgfpathlineto{\pgfqpoint{4.757677in}{4.256491in}}%
\pgfpathlineto{\pgfqpoint{4.761524in}{4.245229in}}%
\pgfpathlineto{\pgfqpoint{4.763447in}{4.243424in}}%
\pgfpathlineto{\pgfqpoint{4.765371in}{4.238565in}}%
\pgfpathlineto{\pgfqpoint{4.767294in}{4.236853in}}%
\pgfpathlineto{\pgfqpoint{4.769217in}{4.231568in}}%
\pgfpathlineto{\pgfqpoint{4.774988in}{4.245521in}}%
\pgfpathlineto{\pgfqpoint{4.776911in}{4.245145in}}%
\pgfpathlineto{\pgfqpoint{4.778834in}{4.241425in}}%
\pgfpathlineto{\pgfqpoint{4.780758in}{4.245432in}}%
\pgfpathlineto{\pgfqpoint{4.782681in}{4.245933in}}%
\pgfpathlineto{\pgfqpoint{4.786528in}{4.263436in}}%
\pgfpathlineto{\pgfqpoint{4.790375in}{4.304730in}}%
\pgfpathlineto{\pgfqpoint{4.794222in}{4.325235in}}%
\pgfpathlineto{\pgfqpoint{4.796145in}{4.327366in}}%
\pgfpathlineto{\pgfqpoint{4.799992in}{4.324455in}}%
\pgfpathlineto{\pgfqpoint{4.801915in}{4.324497in}}%
\pgfpathlineto{\pgfqpoint{4.805762in}{4.282897in}}%
\pgfpathlineto{\pgfqpoint{4.807686in}{4.296938in}}%
\pgfpathlineto{\pgfqpoint{4.809609in}{4.297530in}}%
\pgfpathlineto{\pgfqpoint{4.811532in}{4.309263in}}%
\pgfpathlineto{\pgfqpoint{4.813456in}{4.310396in}}%
\pgfpathlineto{\pgfqpoint{4.815379in}{4.306232in}}%
\pgfpathlineto{\pgfqpoint{4.819226in}{4.292663in}}%
\pgfpathlineto{\pgfqpoint{4.821149in}{4.308789in}}%
\pgfpathlineto{\pgfqpoint{4.823073in}{4.312015in}}%
\pgfpathlineto{\pgfqpoint{4.824996in}{4.307591in}}%
\pgfpathlineto{\pgfqpoint{4.826920in}{4.299740in}}%
\pgfpathlineto{\pgfqpoint{4.828843in}{4.315936in}}%
\pgfpathlineto{\pgfqpoint{4.830767in}{4.305658in}}%
\pgfpathlineto{\pgfqpoint{4.832690in}{4.310295in}}%
\pgfpathlineto{\pgfqpoint{4.834613in}{4.347007in}}%
\pgfpathlineto{\pgfqpoint{4.836537in}{4.355581in}}%
\pgfpathlineto{\pgfqpoint{4.838460in}{4.355674in}}%
\pgfpathlineto{\pgfqpoint{4.840384in}{4.345938in}}%
\pgfpathlineto{\pgfqpoint{4.842307in}{4.350904in}}%
\pgfpathlineto{\pgfqpoint{4.844230in}{4.346248in}}%
\pgfpathlineto{\pgfqpoint{4.846154in}{4.350128in}}%
\pgfpathlineto{\pgfqpoint{4.848077in}{4.341907in}}%
\pgfpathlineto{\pgfqpoint{4.850001in}{4.357525in}}%
\pgfpathlineto{\pgfqpoint{4.851924in}{4.361934in}}%
\pgfpathlineto{\pgfqpoint{4.853847in}{4.356413in}}%
\pgfpathlineto{\pgfqpoint{4.855771in}{4.340361in}}%
\pgfpathlineto{\pgfqpoint{4.857694in}{4.353639in}}%
\pgfpathlineto{\pgfqpoint{4.859618in}{4.352449in}}%
\pgfpathlineto{\pgfqpoint{4.863464in}{4.347985in}}%
\pgfpathlineto{\pgfqpoint{4.865388in}{4.336596in}}%
\pgfpathlineto{\pgfqpoint{4.867311in}{4.334796in}}%
\pgfpathlineto{\pgfqpoint{4.869235in}{4.337549in}}%
\pgfpathlineto{\pgfqpoint{4.871158in}{4.336834in}}%
\pgfpathlineto{\pgfqpoint{4.873082in}{4.347148in}}%
\pgfpathlineto{\pgfqpoint{4.875005in}{4.346148in}}%
\pgfpathlineto{\pgfqpoint{4.876928in}{4.335624in}}%
\pgfpathlineto{\pgfqpoint{4.878852in}{4.315843in}}%
\pgfpathlineto{\pgfqpoint{4.880775in}{4.325972in}}%
\pgfpathlineto{\pgfqpoint{4.882699in}{4.328980in}}%
\pgfpathlineto{\pgfqpoint{4.884622in}{4.336663in}}%
\pgfpathlineto{\pgfqpoint{4.886545in}{4.336804in}}%
\pgfpathlineto{\pgfqpoint{4.890392in}{4.317610in}}%
\pgfpathlineto{\pgfqpoint{4.892316in}{4.313995in}}%
\pgfpathlineto{\pgfqpoint{4.894239in}{4.313824in}}%
\pgfpathlineto{\pgfqpoint{4.896162in}{4.308623in}}%
\pgfpathlineto{\pgfqpoint{4.898086in}{4.309561in}}%
\pgfpathlineto{\pgfqpoint{4.901933in}{4.339780in}}%
\pgfpathlineto{\pgfqpoint{4.903856in}{4.337982in}}%
\pgfpathlineto{\pgfqpoint{4.907703in}{4.324176in}}%
\pgfpathlineto{\pgfqpoint{4.909626in}{4.336453in}}%
\pgfpathlineto{\pgfqpoint{4.911550in}{4.336608in}}%
\pgfpathlineto{\pgfqpoint{4.915397in}{4.351679in}}%
\pgfpathlineto{\pgfqpoint{4.925014in}{4.311019in}}%
\pgfpathlineto{\pgfqpoint{4.926937in}{4.314696in}}%
\pgfpathlineto{\pgfqpoint{4.928860in}{4.294889in}}%
\pgfpathlineto{\pgfqpoint{4.930784in}{4.288053in}}%
\pgfpathlineto{\pgfqpoint{4.932707in}{4.286208in}}%
\pgfpathlineto{\pgfqpoint{4.934631in}{4.276128in}}%
\pgfpathlineto{\pgfqpoint{4.936554in}{4.276595in}}%
\pgfpathlineto{\pgfqpoint{4.938478in}{4.257986in}}%
\pgfpathlineto{\pgfqpoint{4.940401in}{4.260005in}}%
\pgfpathlineto{\pgfqpoint{4.942324in}{4.272795in}}%
\pgfpathlineto{\pgfqpoint{4.944248in}{4.268101in}}%
\pgfpathlineto{\pgfqpoint{4.950018in}{4.297628in}}%
\pgfpathlineto{\pgfqpoint{4.951941in}{4.314479in}}%
\pgfpathlineto{\pgfqpoint{4.953865in}{4.316731in}}%
\pgfpathlineto{\pgfqpoint{4.957712in}{4.331945in}}%
\pgfpathlineto{\pgfqpoint{4.961558in}{4.329325in}}%
\pgfpathlineto{\pgfqpoint{4.963482in}{4.323091in}}%
\pgfpathlineto{\pgfqpoint{4.967329in}{4.303256in}}%
\pgfpathlineto{\pgfqpoint{4.969252in}{4.312649in}}%
\pgfpathlineto{\pgfqpoint{4.971176in}{4.307771in}}%
\pgfpathlineto{\pgfqpoint{4.973099in}{4.321868in}}%
\pgfpathlineto{\pgfqpoint{4.976946in}{4.300510in}}%
\pgfpathlineto{\pgfqpoint{4.978869in}{4.309054in}}%
\pgfpathlineto{\pgfqpoint{4.982716in}{4.292552in}}%
\pgfpathlineto{\pgfqpoint{4.984639in}{4.299105in}}%
\pgfpathlineto{\pgfqpoint{4.988486in}{4.321361in}}%
\pgfpathlineto{\pgfqpoint{4.990410in}{4.326170in}}%
\pgfpathlineto{\pgfqpoint{4.992333in}{4.327561in}}%
\pgfpathlineto{\pgfqpoint{4.996180in}{4.349578in}}%
\pgfpathlineto{\pgfqpoint{4.998103in}{4.347070in}}%
\pgfpathlineto{\pgfqpoint{5.000027in}{4.349305in}}%
\pgfpathlineto{\pgfqpoint{5.001950in}{4.367490in}}%
\pgfpathlineto{\pgfqpoint{5.003874in}{4.367564in}}%
\pgfpathlineto{\pgfqpoint{5.005797in}{4.350663in}}%
\pgfpathlineto{\pgfqpoint{5.009644in}{4.354271in}}%
\pgfpathlineto{\pgfqpoint{5.011567in}{4.361544in}}%
\pgfpathlineto{\pgfqpoint{5.013491in}{4.380045in}}%
\pgfpathlineto{\pgfqpoint{5.015414in}{4.379887in}}%
\pgfpathlineto{\pgfqpoint{5.017337in}{4.381153in}}%
\pgfpathlineto{\pgfqpoint{5.019261in}{4.381047in}}%
\pgfpathlineto{\pgfqpoint{5.021184in}{4.376399in}}%
\pgfpathlineto{\pgfqpoint{5.023108in}{4.379095in}}%
\pgfpathlineto{\pgfqpoint{5.025031in}{4.379390in}}%
\pgfpathlineto{\pgfqpoint{5.030801in}{4.388661in}}%
\pgfpathlineto{\pgfqpoint{5.032725in}{4.375746in}}%
\pgfpathlineto{\pgfqpoint{5.034648in}{4.381876in}}%
\pgfpathlineto{\pgfqpoint{5.036571in}{4.362948in}}%
\pgfpathlineto{\pgfqpoint{5.038495in}{4.367581in}}%
\pgfpathlineto{\pgfqpoint{5.040418in}{4.356320in}}%
\pgfpathlineto{\pgfqpoint{5.042342in}{4.354210in}}%
\pgfpathlineto{\pgfqpoint{5.044265in}{4.336389in}}%
\pgfpathlineto{\pgfqpoint{5.046189in}{4.338773in}}%
\pgfpathlineto{\pgfqpoint{5.048112in}{4.347085in}}%
\pgfpathlineto{\pgfqpoint{5.050035in}{4.346949in}}%
\pgfpathlineto{\pgfqpoint{5.051959in}{4.358272in}}%
\pgfpathlineto{\pgfqpoint{5.053882in}{4.342478in}}%
\pgfpathlineto{\pgfqpoint{5.055806in}{4.347706in}}%
\pgfpathlineto{\pgfqpoint{5.057729in}{4.335844in}}%
\pgfpathlineto{\pgfqpoint{5.059652in}{4.333116in}}%
\pgfpathlineto{\pgfqpoint{5.061576in}{4.328450in}}%
\pgfpathlineto{\pgfqpoint{5.063499in}{4.328497in}}%
\pgfpathlineto{\pgfqpoint{5.065423in}{4.350791in}}%
\pgfpathlineto{\pgfqpoint{5.067346in}{4.350497in}}%
\pgfpathlineto{\pgfqpoint{5.069269in}{4.341323in}}%
\pgfpathlineto{\pgfqpoint{5.075040in}{4.330945in}}%
\pgfpathlineto{\pgfqpoint{5.076963in}{4.339734in}}%
\pgfpathlineto{\pgfqpoint{5.078887in}{4.338795in}}%
\pgfpathlineto{\pgfqpoint{5.080810in}{4.344209in}}%
\pgfpathlineto{\pgfqpoint{5.082733in}{4.357719in}}%
\pgfpathlineto{\pgfqpoint{5.084657in}{4.382263in}}%
\pgfpathlineto{\pgfqpoint{5.086580in}{4.373922in}}%
\pgfpathlineto{\pgfqpoint{5.090427in}{4.351261in}}%
\pgfpathlineto{\pgfqpoint{5.092350in}{4.347574in}}%
\pgfpathlineto{\pgfqpoint{5.094274in}{4.319130in}}%
\pgfpathlineto{\pgfqpoint{5.096197in}{4.328889in}}%
\pgfpathlineto{\pgfqpoint{5.100044in}{4.304881in}}%
\pgfpathlineto{\pgfqpoint{5.101967in}{4.303936in}}%
\pgfpathlineto{\pgfqpoint{5.103891in}{4.309703in}}%
\pgfpathlineto{\pgfqpoint{5.107738in}{4.291402in}}%
\pgfpathlineto{\pgfqpoint{5.111585in}{4.308133in}}%
\pgfpathlineto{\pgfqpoint{5.115431in}{4.298100in}}%
\pgfpathlineto{\pgfqpoint{5.117355in}{4.317618in}}%
\pgfpathlineto{\pgfqpoint{5.119278in}{4.318062in}}%
\pgfpathlineto{\pgfqpoint{5.121202in}{4.334096in}}%
\pgfpathlineto{\pgfqpoint{5.123125in}{4.326358in}}%
\pgfpathlineto{\pgfqpoint{5.125048in}{4.336132in}}%
\pgfpathlineto{\pgfqpoint{5.126972in}{4.338260in}}%
\pgfpathlineto{\pgfqpoint{5.130819in}{4.367087in}}%
\pgfpathlineto{\pgfqpoint{5.132742in}{4.368482in}}%
\pgfpathlineto{\pgfqpoint{5.134665in}{4.366123in}}%
\pgfpathlineto{\pgfqpoint{5.136589in}{4.346692in}}%
\pgfpathlineto{\pgfqpoint{5.138512in}{4.358034in}}%
\pgfpathlineto{\pgfqpoint{5.140436in}{4.357188in}}%
\pgfpathlineto{\pgfqpoint{5.142359in}{4.354022in}}%
\pgfpathlineto{\pgfqpoint{5.144283in}{4.355770in}}%
\pgfpathlineto{\pgfqpoint{5.146206in}{4.351228in}}%
\pgfpathlineto{\pgfqpoint{5.148129in}{4.352134in}}%
\pgfpathlineto{\pgfqpoint{5.150053in}{4.355042in}}%
\pgfpathlineto{\pgfqpoint{5.151976in}{4.344122in}}%
\pgfpathlineto{\pgfqpoint{5.153900in}{4.345585in}}%
\pgfpathlineto{\pgfqpoint{5.155823in}{4.358238in}}%
\pgfpathlineto{\pgfqpoint{5.157746in}{4.346576in}}%
\pgfpathlineto{\pgfqpoint{5.161593in}{4.340396in}}%
\pgfpathlineto{\pgfqpoint{5.163517in}{4.341102in}}%
\pgfpathlineto{\pgfqpoint{5.165440in}{4.339212in}}%
\pgfpathlineto{\pgfqpoint{5.167363in}{4.345912in}}%
\pgfpathlineto{\pgfqpoint{5.171210in}{4.344032in}}%
\pgfpathlineto{\pgfqpoint{5.173134in}{4.339514in}}%
\pgfpathlineto{\pgfqpoint{5.176980in}{4.359060in}}%
\pgfpathlineto{\pgfqpoint{5.178904in}{4.354610in}}%
\pgfpathlineto{\pgfqpoint{5.180827in}{4.341886in}}%
\pgfpathlineto{\pgfqpoint{5.182751in}{4.341432in}}%
\pgfpathlineto{\pgfqpoint{5.184674in}{4.329477in}}%
\pgfpathlineto{\pgfqpoint{5.186598in}{4.335864in}}%
\pgfpathlineto{\pgfqpoint{5.188521in}{4.329623in}}%
\pgfpathlineto{\pgfqpoint{5.190444in}{4.329916in}}%
\pgfpathlineto{\pgfqpoint{5.192368in}{4.334156in}}%
\pgfpathlineto{\pgfqpoint{5.194291in}{4.332146in}}%
\pgfpathlineto{\pgfqpoint{5.196215in}{4.325285in}}%
\pgfpathlineto{\pgfqpoint{5.198138in}{4.332466in}}%
\pgfpathlineto{\pgfqpoint{5.200061in}{4.323628in}}%
\pgfpathlineto{\pgfqpoint{5.201985in}{4.332878in}}%
\pgfpathlineto{\pgfqpoint{5.203908in}{4.329967in}}%
\pgfpathlineto{\pgfqpoint{5.205832in}{4.331487in}}%
\pgfpathlineto{\pgfqpoint{5.207755in}{4.336081in}}%
\pgfpathlineto{\pgfqpoint{5.209678in}{4.332019in}}%
\pgfpathlineto{\pgfqpoint{5.211602in}{4.358570in}}%
\pgfpathlineto{\pgfqpoint{5.213525in}{4.342480in}}%
\pgfpathlineto{\pgfqpoint{5.215449in}{4.351013in}}%
\pgfpathlineto{\pgfqpoint{5.217372in}{4.334767in}}%
\pgfpathlineto{\pgfqpoint{5.219296in}{4.335237in}}%
\pgfpathlineto{\pgfqpoint{5.221219in}{4.351975in}}%
\pgfpathlineto{\pgfqpoint{5.223142in}{4.341183in}}%
\pgfpathlineto{\pgfqpoint{5.225066in}{4.353245in}}%
\pgfpathlineto{\pgfqpoint{5.226989in}{4.353339in}}%
\pgfpathlineto{\pgfqpoint{5.228913in}{4.362664in}}%
\pgfpathlineto{\pgfqpoint{5.230836in}{4.361112in}}%
\pgfpathlineto{\pgfqpoint{5.232759in}{4.362233in}}%
\pgfpathlineto{\pgfqpoint{5.240453in}{4.413826in}}%
\pgfpathlineto{\pgfqpoint{5.242376in}{4.413817in}}%
\pgfpathlineto{\pgfqpoint{5.244300in}{4.391916in}}%
\pgfpathlineto{\pgfqpoint{5.246223in}{4.396278in}}%
\pgfpathlineto{\pgfqpoint{5.248147in}{4.418059in}}%
\pgfpathlineto{\pgfqpoint{5.250070in}{4.425378in}}%
\pgfpathlineto{\pgfqpoint{5.251994in}{4.409567in}}%
\pgfpathlineto{\pgfqpoint{5.253917in}{4.417870in}}%
\pgfpathlineto{\pgfqpoint{5.255840in}{4.409057in}}%
\pgfpathlineto{\pgfqpoint{5.257764in}{4.413505in}}%
\pgfpathlineto{\pgfqpoint{5.259687in}{4.412837in}}%
\pgfpathlineto{\pgfqpoint{5.261611in}{4.413751in}}%
\pgfpathlineto{\pgfqpoint{5.263534in}{4.401362in}}%
\pgfpathlineto{\pgfqpoint{5.265457in}{4.399240in}}%
\pgfpathlineto{\pgfqpoint{5.267381in}{4.386901in}}%
\pgfpathlineto{\pgfqpoint{5.269304in}{4.392742in}}%
\pgfpathlineto{\pgfqpoint{5.271228in}{4.374455in}}%
\pgfpathlineto{\pgfqpoint{5.275074in}{4.367597in}}%
\pgfpathlineto{\pgfqpoint{5.276998in}{4.384900in}}%
\pgfpathlineto{\pgfqpoint{5.278921in}{4.385633in}}%
\pgfpathlineto{\pgfqpoint{5.280845in}{4.387947in}}%
\pgfpathlineto{\pgfqpoint{5.282768in}{4.392909in}}%
\pgfpathlineto{\pgfqpoint{5.284692in}{4.381844in}}%
\pgfpathlineto{\pgfqpoint{5.286615in}{4.385295in}}%
\pgfpathlineto{\pgfqpoint{5.290462in}{4.360854in}}%
\pgfpathlineto{\pgfqpoint{5.292385in}{4.374818in}}%
\pgfpathlineto{\pgfqpoint{5.294309in}{4.367847in}}%
\pgfpathlineto{\pgfqpoint{5.296232in}{4.356320in}}%
\pgfpathlineto{\pgfqpoint{5.300079in}{4.373569in}}%
\pgfpathlineto{\pgfqpoint{5.302002in}{4.372949in}}%
\pgfpathlineto{\pgfqpoint{5.303926in}{4.376124in}}%
\pgfpathlineto{\pgfqpoint{5.303926in}{4.376124in}}%
\pgfusepath{stroke}%
\end{pgfscope}%
\begin{pgfscope}%
\pgfpathrectangle{\pgfqpoint{3.286364in}{3.180000in}}{\pgfqpoint{2.113636in}{2.100000in}}%
\pgfusepath{clip}%
\pgfsetroundcap%
\pgfsetroundjoin%
\pgfsetlinewidth{0.602250pt}%
\definecolor{currentstroke}{rgb}{0.968627,0.505882,0.749020}%
\pgfsetstrokecolor{currentstroke}%
\pgfsetdash{}{0pt}%
\pgfpathmoveto{\pgfqpoint{3.382438in}{4.197731in}}%
\pgfpathlineto{\pgfqpoint{3.384361in}{4.205081in}}%
\pgfpathlineto{\pgfqpoint{3.386285in}{4.206295in}}%
\pgfpathlineto{\pgfqpoint{3.388208in}{4.193667in}}%
\pgfpathlineto{\pgfqpoint{3.390132in}{4.191116in}}%
\pgfpathlineto{\pgfqpoint{3.392055in}{4.211421in}}%
\pgfpathlineto{\pgfqpoint{3.393978in}{4.204509in}}%
\pgfpathlineto{\pgfqpoint{3.397825in}{4.223071in}}%
\pgfpathlineto{\pgfqpoint{3.399749in}{4.216026in}}%
\pgfpathlineto{\pgfqpoint{3.403596in}{4.212922in}}%
\pgfpathlineto{\pgfqpoint{3.405519in}{4.220572in}}%
\pgfpathlineto{\pgfqpoint{3.407442in}{4.215307in}}%
\pgfpathlineto{\pgfqpoint{3.409366in}{4.231387in}}%
\pgfpathlineto{\pgfqpoint{3.411289in}{4.222758in}}%
\pgfpathlineto{\pgfqpoint{3.413213in}{4.237869in}}%
\pgfpathlineto{\pgfqpoint{3.418983in}{4.235324in}}%
\pgfpathlineto{\pgfqpoint{3.420906in}{4.227984in}}%
\pgfpathlineto{\pgfqpoint{3.424753in}{4.233527in}}%
\pgfpathlineto{\pgfqpoint{3.428600in}{4.209585in}}%
\pgfpathlineto{\pgfqpoint{3.430523in}{4.226892in}}%
\pgfpathlineto{\pgfqpoint{3.432447in}{4.232716in}}%
\pgfpathlineto{\pgfqpoint{3.434370in}{4.242604in}}%
\pgfpathlineto{\pgfqpoint{3.436294in}{4.241968in}}%
\pgfpathlineto{\pgfqpoint{3.438217in}{4.243539in}}%
\pgfpathlineto{\pgfqpoint{3.442064in}{4.236838in}}%
\pgfpathlineto{\pgfqpoint{3.443987in}{4.234965in}}%
\pgfpathlineto{\pgfqpoint{3.445911in}{4.228536in}}%
\pgfpathlineto{\pgfqpoint{3.449757in}{4.199186in}}%
\pgfpathlineto{\pgfqpoint{3.451681in}{4.195961in}}%
\pgfpathlineto{\pgfqpoint{3.453604in}{4.188137in}}%
\pgfpathlineto{\pgfqpoint{3.455528in}{4.192520in}}%
\pgfpathlineto{\pgfqpoint{3.457451in}{4.193899in}}%
\pgfpathlineto{\pgfqpoint{3.459374in}{4.179318in}}%
\pgfpathlineto{\pgfqpoint{3.461298in}{4.177793in}}%
\pgfpathlineto{\pgfqpoint{3.463221in}{4.168722in}}%
\pgfpathlineto{\pgfqpoint{3.465145in}{4.168537in}}%
\pgfpathlineto{\pgfqpoint{3.468992in}{4.189901in}}%
\pgfpathlineto{\pgfqpoint{3.470915in}{4.174207in}}%
\pgfpathlineto{\pgfqpoint{3.472838in}{4.174623in}}%
\pgfpathlineto{\pgfqpoint{3.474762in}{4.170366in}}%
\pgfpathlineto{\pgfqpoint{3.476685in}{4.173480in}}%
\pgfpathlineto{\pgfqpoint{3.478609in}{4.170819in}}%
\pgfpathlineto{\pgfqpoint{3.480532in}{4.200336in}}%
\pgfpathlineto{\pgfqpoint{3.482455in}{4.206458in}}%
\pgfpathlineto{\pgfqpoint{3.484379in}{4.222832in}}%
\pgfpathlineto{\pgfqpoint{3.488226in}{4.231293in}}%
\pgfpathlineto{\pgfqpoint{3.490149in}{4.247528in}}%
\pgfpathlineto{\pgfqpoint{3.492072in}{4.240063in}}%
\pgfpathlineto{\pgfqpoint{3.493996in}{4.244478in}}%
\pgfpathlineto{\pgfqpoint{3.497843in}{4.219645in}}%
\pgfpathlineto{\pgfqpoint{3.501689in}{4.239093in}}%
\pgfpathlineto{\pgfqpoint{3.503613in}{4.241943in}}%
\pgfpathlineto{\pgfqpoint{3.505536in}{4.256120in}}%
\pgfpathlineto{\pgfqpoint{3.507460in}{4.260422in}}%
\pgfpathlineto{\pgfqpoint{3.509383in}{4.270885in}}%
\pgfpathlineto{\pgfqpoint{3.511307in}{4.266466in}}%
\pgfpathlineto{\pgfqpoint{3.515153in}{4.298373in}}%
\pgfpathlineto{\pgfqpoint{3.519000in}{4.302260in}}%
\pgfpathlineto{\pgfqpoint{3.520924in}{4.301048in}}%
\pgfpathlineto{\pgfqpoint{3.522847in}{4.297735in}}%
\pgfpathlineto{\pgfqpoint{3.530541in}{4.335715in}}%
\pgfpathlineto{\pgfqpoint{3.532464in}{4.339177in}}%
\pgfpathlineto{\pgfqpoint{3.534387in}{4.345804in}}%
\pgfpathlineto{\pgfqpoint{3.536311in}{4.347647in}}%
\pgfpathlineto{\pgfqpoint{3.538234in}{4.358240in}}%
\pgfpathlineto{\pgfqpoint{3.542081in}{4.365380in}}%
\pgfpathlineto{\pgfqpoint{3.544005in}{4.380423in}}%
\pgfpathlineto{\pgfqpoint{3.545928in}{4.380657in}}%
\pgfpathlineto{\pgfqpoint{3.551698in}{4.401925in}}%
\pgfpathlineto{\pgfqpoint{3.553622in}{4.399619in}}%
\pgfpathlineto{\pgfqpoint{3.555545in}{4.417470in}}%
\pgfpathlineto{\pgfqpoint{3.557468in}{4.420701in}}%
\pgfpathlineto{\pgfqpoint{3.559392in}{4.416859in}}%
\pgfpathlineto{\pgfqpoint{3.561315in}{4.417262in}}%
\pgfpathlineto{\pgfqpoint{3.563239in}{4.408296in}}%
\pgfpathlineto{\pgfqpoint{3.565162in}{4.415865in}}%
\pgfpathlineto{\pgfqpoint{3.567085in}{4.412051in}}%
\pgfpathlineto{\pgfqpoint{3.569009in}{4.415109in}}%
\pgfpathlineto{\pgfqpoint{3.570932in}{4.413894in}}%
\pgfpathlineto{\pgfqpoint{3.574779in}{4.420910in}}%
\pgfpathlineto{\pgfqpoint{3.576703in}{4.418754in}}%
\pgfpathlineto{\pgfqpoint{3.578626in}{4.420969in}}%
\pgfpathlineto{\pgfqpoint{3.580549in}{4.413958in}}%
\pgfpathlineto{\pgfqpoint{3.582473in}{4.428370in}}%
\pgfpathlineto{\pgfqpoint{3.584396in}{4.424873in}}%
\pgfpathlineto{\pgfqpoint{3.586320in}{4.427196in}}%
\pgfpathlineto{\pgfqpoint{3.588243in}{4.433631in}}%
\pgfpathlineto{\pgfqpoint{3.594013in}{4.490921in}}%
\pgfpathlineto{\pgfqpoint{3.595937in}{4.498736in}}%
\pgfpathlineto{\pgfqpoint{3.597860in}{4.501389in}}%
\pgfpathlineto{\pgfqpoint{3.601707in}{4.478967in}}%
\pgfpathlineto{\pgfqpoint{3.603630in}{4.479970in}}%
\pgfpathlineto{\pgfqpoint{3.605554in}{4.493942in}}%
\pgfpathlineto{\pgfqpoint{3.607477in}{4.489462in}}%
\pgfpathlineto{\pgfqpoint{3.609401in}{4.479747in}}%
\pgfpathlineto{\pgfqpoint{3.615171in}{4.496902in}}%
\pgfpathlineto{\pgfqpoint{3.617094in}{4.506296in}}%
\pgfpathlineto{\pgfqpoint{3.619018in}{4.500032in}}%
\pgfpathlineto{\pgfqpoint{3.620941in}{4.503639in}}%
\pgfpathlineto{\pgfqpoint{3.622864in}{4.515302in}}%
\pgfpathlineto{\pgfqpoint{3.624788in}{4.506323in}}%
\pgfpathlineto{\pgfqpoint{3.626711in}{4.507691in}}%
\pgfpathlineto{\pgfqpoint{3.628635in}{4.495742in}}%
\pgfpathlineto{\pgfqpoint{3.630558in}{4.496340in}}%
\pgfpathlineto{\pgfqpoint{3.636328in}{4.484244in}}%
\pgfpathlineto{\pgfqpoint{3.638252in}{4.498966in}}%
\pgfpathlineto{\pgfqpoint{3.640175in}{4.477832in}}%
\pgfpathlineto{\pgfqpoint{3.642099in}{4.484568in}}%
\pgfpathlineto{\pgfqpoint{3.644022in}{4.496625in}}%
\pgfpathlineto{\pgfqpoint{3.645945in}{4.488744in}}%
\pgfpathlineto{\pgfqpoint{3.647869in}{4.487198in}}%
\pgfpathlineto{\pgfqpoint{3.651716in}{4.474678in}}%
\pgfpathlineto{\pgfqpoint{3.653639in}{4.475033in}}%
\pgfpathlineto{\pgfqpoint{3.655562in}{4.462800in}}%
\pgfpathlineto{\pgfqpoint{3.657486in}{4.464071in}}%
\pgfpathlineto{\pgfqpoint{3.659409in}{4.467132in}}%
\pgfpathlineto{\pgfqpoint{3.667103in}{4.427663in}}%
\pgfpathlineto{\pgfqpoint{3.669026in}{4.423574in}}%
\pgfpathlineto{\pgfqpoint{3.670950in}{4.425219in}}%
\pgfpathlineto{\pgfqpoint{3.672873in}{4.429035in}}%
\pgfpathlineto{\pgfqpoint{3.674796in}{4.416513in}}%
\pgfpathlineto{\pgfqpoint{3.678643in}{4.452156in}}%
\pgfpathlineto{\pgfqpoint{3.680567in}{4.437542in}}%
\pgfpathlineto{\pgfqpoint{3.684414in}{4.464172in}}%
\pgfpathlineto{\pgfqpoint{3.686337in}{4.479165in}}%
\pgfpathlineto{\pgfqpoint{3.688260in}{4.479817in}}%
\pgfpathlineto{\pgfqpoint{3.690184in}{4.482898in}}%
\pgfpathlineto{\pgfqpoint{3.692107in}{4.489583in}}%
\pgfpathlineto{\pgfqpoint{3.694031in}{4.486315in}}%
\pgfpathlineto{\pgfqpoint{3.695954in}{4.485245in}}%
\pgfpathlineto{\pgfqpoint{3.697877in}{4.477241in}}%
\pgfpathlineto{\pgfqpoint{3.699801in}{4.478831in}}%
\pgfpathlineto{\pgfqpoint{3.701724in}{4.477999in}}%
\pgfpathlineto{\pgfqpoint{3.705571in}{4.459951in}}%
\pgfpathlineto{\pgfqpoint{3.707494in}{4.465912in}}%
\pgfpathlineto{\pgfqpoint{3.711341in}{4.460715in}}%
\pgfpathlineto{\pgfqpoint{3.713265in}{4.478609in}}%
\pgfpathlineto{\pgfqpoint{3.715188in}{4.477249in}}%
\pgfpathlineto{\pgfqpoint{3.720958in}{4.497009in}}%
\pgfpathlineto{\pgfqpoint{3.722882in}{4.491220in}}%
\pgfpathlineto{\pgfqpoint{3.726729in}{4.468555in}}%
\pgfpathlineto{\pgfqpoint{3.728652in}{4.456359in}}%
\pgfpathlineto{\pgfqpoint{3.730575in}{4.456037in}}%
\pgfpathlineto{\pgfqpoint{3.734422in}{4.449772in}}%
\pgfpathlineto{\pgfqpoint{3.736346in}{4.457691in}}%
\pgfpathlineto{\pgfqpoint{3.738269in}{4.457497in}}%
\pgfpathlineto{\pgfqpoint{3.740192in}{4.454876in}}%
\pgfpathlineto{\pgfqpoint{3.742116in}{4.450152in}}%
\pgfpathlineto{\pgfqpoint{3.744039in}{4.448582in}}%
\pgfpathlineto{\pgfqpoint{3.749810in}{4.486104in}}%
\pgfpathlineto{\pgfqpoint{3.751733in}{4.490195in}}%
\pgfpathlineto{\pgfqpoint{3.753656in}{4.479419in}}%
\pgfpathlineto{\pgfqpoint{3.759427in}{4.479120in}}%
\pgfpathlineto{\pgfqpoint{3.761350in}{4.485535in}}%
\pgfpathlineto{\pgfqpoint{3.769044in}{4.525890in}}%
\pgfpathlineto{\pgfqpoint{3.770967in}{4.525555in}}%
\pgfpathlineto{\pgfqpoint{3.772890in}{4.529926in}}%
\pgfpathlineto{\pgfqpoint{3.774814in}{4.531035in}}%
\pgfpathlineto{\pgfqpoint{3.776737in}{4.534057in}}%
\pgfpathlineto{\pgfqpoint{3.778661in}{4.521888in}}%
\pgfpathlineto{\pgfqpoint{3.780584in}{4.529191in}}%
\pgfpathlineto{\pgfqpoint{3.782508in}{4.524183in}}%
\pgfpathlineto{\pgfqpoint{3.784431in}{4.523068in}}%
\pgfpathlineto{\pgfqpoint{3.786354in}{4.539164in}}%
\pgfpathlineto{\pgfqpoint{3.790201in}{4.546560in}}%
\pgfpathlineto{\pgfqpoint{3.792125in}{4.543489in}}%
\pgfpathlineto{\pgfqpoint{3.794048in}{4.535915in}}%
\pgfpathlineto{\pgfqpoint{3.795971in}{4.551588in}}%
\pgfpathlineto{\pgfqpoint{3.797895in}{4.549867in}}%
\pgfpathlineto{\pgfqpoint{3.799818in}{4.538875in}}%
\pgfpathlineto{\pgfqpoint{3.801742in}{4.543521in}}%
\pgfpathlineto{\pgfqpoint{3.803665in}{4.541272in}}%
\pgfpathlineto{\pgfqpoint{3.805588in}{4.532992in}}%
\pgfpathlineto{\pgfqpoint{3.809435in}{4.552199in}}%
\pgfpathlineto{\pgfqpoint{3.813282in}{4.547718in}}%
\pgfpathlineto{\pgfqpoint{3.815205in}{4.550654in}}%
\pgfpathlineto{\pgfqpoint{3.819052in}{4.536092in}}%
\pgfpathlineto{\pgfqpoint{3.820976in}{4.538268in}}%
\pgfpathlineto{\pgfqpoint{3.822899in}{4.549376in}}%
\pgfpathlineto{\pgfqpoint{3.824823in}{4.536413in}}%
\pgfpathlineto{\pgfqpoint{3.826746in}{4.537579in}}%
\pgfpathlineto{\pgfqpoint{3.828669in}{4.540565in}}%
\pgfpathlineto{\pgfqpoint{3.830593in}{4.523086in}}%
\pgfpathlineto{\pgfqpoint{3.832516in}{4.519170in}}%
\pgfpathlineto{\pgfqpoint{3.834440in}{4.510827in}}%
\pgfpathlineto{\pgfqpoint{3.836363in}{4.512036in}}%
\pgfpathlineto{\pgfqpoint{3.840210in}{4.483393in}}%
\pgfpathlineto{\pgfqpoint{3.845980in}{4.547518in}}%
\pgfpathlineto{\pgfqpoint{3.847903in}{4.545554in}}%
\pgfpathlineto{\pgfqpoint{3.849827in}{4.534535in}}%
\pgfpathlineto{\pgfqpoint{3.851750in}{4.536922in}}%
\pgfpathlineto{\pgfqpoint{3.853674in}{4.535157in}}%
\pgfpathlineto{\pgfqpoint{3.855597in}{4.528136in}}%
\pgfpathlineto{\pgfqpoint{3.859444in}{4.556484in}}%
\pgfpathlineto{\pgfqpoint{3.861367in}{4.552850in}}%
\pgfpathlineto{\pgfqpoint{3.867138in}{4.556486in}}%
\pgfpathlineto{\pgfqpoint{3.869061in}{4.554246in}}%
\pgfpathlineto{\pgfqpoint{3.870984in}{4.540088in}}%
\pgfpathlineto{\pgfqpoint{3.874831in}{4.555713in}}%
\pgfpathlineto{\pgfqpoint{3.876755in}{4.548765in}}%
\pgfpathlineto{\pgfqpoint{3.878678in}{4.549075in}}%
\pgfpathlineto{\pgfqpoint{3.880601in}{4.554464in}}%
\pgfpathlineto{\pgfqpoint{3.884448in}{4.543306in}}%
\pgfpathlineto{\pgfqpoint{3.886372in}{4.548040in}}%
\pgfpathlineto{\pgfqpoint{3.888295in}{4.549704in}}%
\pgfpathlineto{\pgfqpoint{3.892142in}{4.548134in}}%
\pgfpathlineto{\pgfqpoint{3.894065in}{4.543205in}}%
\pgfpathlineto{\pgfqpoint{3.897912in}{4.565248in}}%
\pgfpathlineto{\pgfqpoint{3.899836in}{4.565571in}}%
\pgfpathlineto{\pgfqpoint{3.901759in}{4.555932in}}%
\pgfpathlineto{\pgfqpoint{3.903682in}{4.562784in}}%
\pgfpathlineto{\pgfqpoint{3.905606in}{4.551174in}}%
\pgfpathlineto{\pgfqpoint{3.907529in}{4.552366in}}%
\pgfpathlineto{\pgfqpoint{3.909453in}{4.557011in}}%
\pgfpathlineto{\pgfqpoint{3.911376in}{4.576417in}}%
\pgfpathlineto{\pgfqpoint{3.913299in}{4.576521in}}%
\pgfpathlineto{\pgfqpoint{3.915223in}{4.584630in}}%
\pgfpathlineto{\pgfqpoint{3.917146in}{4.564945in}}%
\pgfpathlineto{\pgfqpoint{3.919070in}{4.568887in}}%
\pgfpathlineto{\pgfqpoint{3.920993in}{4.582318in}}%
\pgfpathlineto{\pgfqpoint{3.924840in}{4.570241in}}%
\pgfpathlineto{\pgfqpoint{3.926763in}{4.590129in}}%
\pgfpathlineto{\pgfqpoint{3.930610in}{4.601685in}}%
\pgfpathlineto{\pgfqpoint{3.932534in}{4.600286in}}%
\pgfpathlineto{\pgfqpoint{3.934457in}{4.609823in}}%
\pgfpathlineto{\pgfqpoint{3.936380in}{4.606434in}}%
\pgfpathlineto{\pgfqpoint{3.938304in}{4.612861in}}%
\pgfpathlineto{\pgfqpoint{3.940227in}{4.603268in}}%
\pgfpathlineto{\pgfqpoint{3.942151in}{4.606948in}}%
\pgfpathlineto{\pgfqpoint{3.947921in}{4.598655in}}%
\pgfpathlineto{\pgfqpoint{3.949844in}{4.590396in}}%
\pgfpathlineto{\pgfqpoint{3.951768in}{4.592256in}}%
\pgfpathlineto{\pgfqpoint{3.953691in}{4.591708in}}%
\pgfpathlineto{\pgfqpoint{3.955614in}{4.596951in}}%
\pgfpathlineto{\pgfqpoint{3.957538in}{4.581077in}}%
\pgfpathlineto{\pgfqpoint{3.959461in}{4.579505in}}%
\pgfpathlineto{\pgfqpoint{3.961385in}{4.588954in}}%
\pgfpathlineto{\pgfqpoint{3.963308in}{4.585021in}}%
\pgfpathlineto{\pgfqpoint{3.969078in}{4.608347in}}%
\pgfpathlineto{\pgfqpoint{3.971002in}{4.610968in}}%
\pgfpathlineto{\pgfqpoint{3.972925in}{4.602694in}}%
\pgfpathlineto{\pgfqpoint{3.974849in}{4.617960in}}%
\pgfpathlineto{\pgfqpoint{3.976772in}{4.613832in}}%
\pgfpathlineto{\pgfqpoint{3.980619in}{4.575845in}}%
\pgfpathlineto{\pgfqpoint{3.982542in}{4.579245in}}%
\pgfpathlineto{\pgfqpoint{3.984466in}{4.557376in}}%
\pgfpathlineto{\pgfqpoint{3.992159in}{4.597952in}}%
\pgfpathlineto{\pgfqpoint{3.994083in}{4.600520in}}%
\pgfpathlineto{\pgfqpoint{3.996006in}{4.611432in}}%
\pgfpathlineto{\pgfqpoint{3.999853in}{4.585533in}}%
\pgfpathlineto{\pgfqpoint{4.001776in}{4.588314in}}%
\pgfpathlineto{\pgfqpoint{4.005623in}{4.610900in}}%
\pgfpathlineto{\pgfqpoint{4.007547in}{4.623443in}}%
\pgfpathlineto{\pgfqpoint{4.017164in}{4.591153in}}%
\pgfpathlineto{\pgfqpoint{4.019087in}{4.598333in}}%
\pgfpathlineto{\pgfqpoint{4.021010in}{4.592792in}}%
\pgfpathlineto{\pgfqpoint{4.022934in}{4.579184in}}%
\pgfpathlineto{\pgfqpoint{4.024857in}{4.583058in}}%
\pgfpathlineto{\pgfqpoint{4.030628in}{4.611734in}}%
\pgfpathlineto{\pgfqpoint{4.032551in}{4.606782in}}%
\pgfpathlineto{\pgfqpoint{4.034474in}{4.614393in}}%
\pgfpathlineto{\pgfqpoint{4.036398in}{4.613992in}}%
\pgfpathlineto{\pgfqpoint{4.040245in}{4.639270in}}%
\pgfpathlineto{\pgfqpoint{4.042168in}{4.638253in}}%
\pgfpathlineto{\pgfqpoint{4.044091in}{4.651255in}}%
\pgfpathlineto{\pgfqpoint{4.046015in}{4.644782in}}%
\pgfpathlineto{\pgfqpoint{4.047938in}{4.647836in}}%
\pgfpathlineto{\pgfqpoint{4.049862in}{4.660951in}}%
\pgfpathlineto{\pgfqpoint{4.051785in}{4.659147in}}%
\pgfpathlineto{\pgfqpoint{4.053708in}{4.637002in}}%
\pgfpathlineto{\pgfqpoint{4.057555in}{4.652817in}}%
\pgfpathlineto{\pgfqpoint{4.059479in}{4.648631in}}%
\pgfpathlineto{\pgfqpoint{4.061402in}{4.633250in}}%
\pgfpathlineto{\pgfqpoint{4.067172in}{4.625153in}}%
\pgfpathlineto{\pgfqpoint{4.069096in}{4.608772in}}%
\pgfpathlineto{\pgfqpoint{4.071019in}{4.625006in}}%
\pgfpathlineto{\pgfqpoint{4.078713in}{4.605572in}}%
\pgfpathlineto{\pgfqpoint{4.080636in}{4.605681in}}%
\pgfpathlineto{\pgfqpoint{4.082560in}{4.613805in}}%
\pgfpathlineto{\pgfqpoint{4.088330in}{4.595488in}}%
\pgfpathlineto{\pgfqpoint{4.092177in}{4.591373in}}%
\pgfpathlineto{\pgfqpoint{4.094100in}{4.573782in}}%
\pgfpathlineto{\pgfqpoint{4.096024in}{4.569716in}}%
\pgfpathlineto{\pgfqpoint{4.097947in}{4.574374in}}%
\pgfpathlineto{\pgfqpoint{4.099870in}{4.574489in}}%
\pgfpathlineto{\pgfqpoint{4.101794in}{4.557823in}}%
\pgfpathlineto{\pgfqpoint{4.103717in}{4.572037in}}%
\pgfpathlineto{\pgfqpoint{4.105641in}{4.572661in}}%
\pgfpathlineto{\pgfqpoint{4.107564in}{4.580770in}}%
\pgfpathlineto{\pgfqpoint{4.109487in}{4.582127in}}%
\pgfpathlineto{\pgfqpoint{4.111411in}{4.574719in}}%
\pgfpathlineto{\pgfqpoint{4.113334in}{4.579904in}}%
\pgfpathlineto{\pgfqpoint{4.115258in}{4.579407in}}%
\pgfpathlineto{\pgfqpoint{4.117181in}{4.590331in}}%
\pgfpathlineto{\pgfqpoint{4.121028in}{4.575103in}}%
\pgfpathlineto{\pgfqpoint{4.122951in}{4.561920in}}%
\pgfpathlineto{\pgfqpoint{4.124875in}{4.571892in}}%
\pgfpathlineto{\pgfqpoint{4.130645in}{4.547439in}}%
\pgfpathlineto{\pgfqpoint{4.134492in}{4.529115in}}%
\pgfpathlineto{\pgfqpoint{4.136415in}{4.532142in}}%
\pgfpathlineto{\pgfqpoint{4.138339in}{4.540658in}}%
\pgfpathlineto{\pgfqpoint{4.140262in}{4.543479in}}%
\pgfpathlineto{\pgfqpoint{4.144109in}{4.534303in}}%
\pgfpathlineto{\pgfqpoint{4.147956in}{4.530387in}}%
\pgfpathlineto{\pgfqpoint{4.151802in}{4.526230in}}%
\pgfpathlineto{\pgfqpoint{4.153726in}{4.536275in}}%
\pgfpathlineto{\pgfqpoint{4.155649in}{4.533249in}}%
\pgfpathlineto{\pgfqpoint{4.157573in}{4.521943in}}%
\pgfpathlineto{\pgfqpoint{4.159496in}{4.528030in}}%
\pgfpathlineto{\pgfqpoint{4.161419in}{4.521485in}}%
\pgfpathlineto{\pgfqpoint{4.163343in}{4.522508in}}%
\pgfpathlineto{\pgfqpoint{4.165266in}{4.525520in}}%
\pgfpathlineto{\pgfqpoint{4.167190in}{4.532750in}}%
\pgfpathlineto{\pgfqpoint{4.169113in}{4.534250in}}%
\pgfpathlineto{\pgfqpoint{4.171037in}{4.529724in}}%
\pgfpathlineto{\pgfqpoint{4.172960in}{4.542038in}}%
\pgfpathlineto{\pgfqpoint{4.174883in}{4.542019in}}%
\pgfpathlineto{\pgfqpoint{4.180654in}{4.555993in}}%
\pgfpathlineto{\pgfqpoint{4.182577in}{4.572571in}}%
\pgfpathlineto{\pgfqpoint{4.188347in}{4.535443in}}%
\pgfpathlineto{\pgfqpoint{4.192194in}{4.528782in}}%
\pgfpathlineto{\pgfqpoint{4.194117in}{4.530523in}}%
\pgfpathlineto{\pgfqpoint{4.197964in}{4.517710in}}%
\pgfpathlineto{\pgfqpoint{4.199888in}{4.516952in}}%
\pgfpathlineto{\pgfqpoint{4.201811in}{4.509246in}}%
\pgfpathlineto{\pgfqpoint{4.203735in}{4.512905in}}%
\pgfpathlineto{\pgfqpoint{4.205658in}{4.493599in}}%
\pgfpathlineto{\pgfqpoint{4.207581in}{4.490563in}}%
\pgfpathlineto{\pgfqpoint{4.209505in}{4.495199in}}%
\pgfpathlineto{\pgfqpoint{4.211428in}{4.486694in}}%
\pgfpathlineto{\pgfqpoint{4.213352in}{4.486582in}}%
\pgfpathlineto{\pgfqpoint{4.215275in}{4.489734in}}%
\pgfpathlineto{\pgfqpoint{4.217198in}{4.499082in}}%
\pgfpathlineto{\pgfqpoint{4.219122in}{4.501268in}}%
\pgfpathlineto{\pgfqpoint{4.221045in}{4.490092in}}%
\pgfpathlineto{\pgfqpoint{4.224892in}{4.494534in}}%
\pgfpathlineto{\pgfqpoint{4.226815in}{4.481396in}}%
\pgfpathlineto{\pgfqpoint{4.228739in}{4.491534in}}%
\pgfpathlineto{\pgfqpoint{4.232586in}{4.481294in}}%
\pgfpathlineto{\pgfqpoint{4.234509in}{4.490982in}}%
\pgfpathlineto{\pgfqpoint{4.236433in}{4.491427in}}%
\pgfpathlineto{\pgfqpoint{4.238356in}{4.480523in}}%
\pgfpathlineto{\pgfqpoint{4.242203in}{4.474017in}}%
\pgfpathlineto{\pgfqpoint{4.244126in}{4.468691in}}%
\pgfpathlineto{\pgfqpoint{4.246050in}{4.454827in}}%
\pgfpathlineto{\pgfqpoint{4.247973in}{4.449270in}}%
\pgfpathlineto{\pgfqpoint{4.249896in}{4.469379in}}%
\pgfpathlineto{\pgfqpoint{4.251820in}{4.476175in}}%
\pgfpathlineto{\pgfqpoint{4.255667in}{4.446387in}}%
\pgfpathlineto{\pgfqpoint{4.257590in}{4.452661in}}%
\pgfpathlineto{\pgfqpoint{4.259513in}{4.440478in}}%
\pgfpathlineto{\pgfqpoint{4.261437in}{4.444652in}}%
\pgfpathlineto{\pgfqpoint{4.263360in}{4.444053in}}%
\pgfpathlineto{\pgfqpoint{4.265284in}{4.441023in}}%
\pgfpathlineto{\pgfqpoint{4.271054in}{4.468803in}}%
\pgfpathlineto{\pgfqpoint{4.272977in}{4.467177in}}%
\pgfpathlineto{\pgfqpoint{4.276824in}{4.485444in}}%
\pgfpathlineto{\pgfqpoint{4.278748in}{4.479893in}}%
\pgfpathlineto{\pgfqpoint{4.280671in}{4.481808in}}%
\pgfpathlineto{\pgfqpoint{4.282594in}{4.465962in}}%
\pgfpathlineto{\pgfqpoint{4.286441in}{4.469254in}}%
\pgfpathlineto{\pgfqpoint{4.288365in}{4.464284in}}%
\pgfpathlineto{\pgfqpoint{4.290288in}{4.471780in}}%
\pgfpathlineto{\pgfqpoint{4.294135in}{4.518369in}}%
\pgfpathlineto{\pgfqpoint{4.296058in}{4.516410in}}%
\pgfpathlineto{\pgfqpoint{4.299905in}{4.527825in}}%
\pgfpathlineto{\pgfqpoint{4.301828in}{4.522344in}}%
\pgfpathlineto{\pgfqpoint{4.303752in}{4.511175in}}%
\pgfpathlineto{\pgfqpoint{4.305675in}{4.516222in}}%
\pgfpathlineto{\pgfqpoint{4.307599in}{4.501557in}}%
\pgfpathlineto{\pgfqpoint{4.309522in}{4.511475in}}%
\pgfpathlineto{\pgfqpoint{4.311446in}{4.499375in}}%
\pgfpathlineto{\pgfqpoint{4.313369in}{4.498741in}}%
\pgfpathlineto{\pgfqpoint{4.315292in}{4.496116in}}%
\pgfpathlineto{\pgfqpoint{4.317216in}{4.495914in}}%
\pgfpathlineto{\pgfqpoint{4.319139in}{4.486412in}}%
\pgfpathlineto{\pgfqpoint{4.321063in}{4.492749in}}%
\pgfpathlineto{\pgfqpoint{4.322986in}{4.479846in}}%
\pgfpathlineto{\pgfqpoint{4.326833in}{4.499185in}}%
\pgfpathlineto{\pgfqpoint{4.328756in}{4.487150in}}%
\pgfpathlineto{\pgfqpoint{4.330680in}{4.485408in}}%
\pgfpathlineto{\pgfqpoint{4.332603in}{4.480514in}}%
\pgfpathlineto{\pgfqpoint{4.334526in}{4.481652in}}%
\pgfpathlineto{\pgfqpoint{4.336450in}{4.464184in}}%
\pgfpathlineto{\pgfqpoint{4.338373in}{4.468176in}}%
\pgfpathlineto{\pgfqpoint{4.340297in}{4.460015in}}%
\pgfpathlineto{\pgfqpoint{4.342220in}{4.477259in}}%
\pgfpathlineto{\pgfqpoint{4.344144in}{4.469099in}}%
\pgfpathlineto{\pgfqpoint{4.346067in}{4.473117in}}%
\pgfpathlineto{\pgfqpoint{4.347990in}{4.467635in}}%
\pgfpathlineto{\pgfqpoint{4.349914in}{4.474577in}}%
\pgfpathlineto{\pgfqpoint{4.351837in}{4.474994in}}%
\pgfpathlineto{\pgfqpoint{4.353761in}{4.489706in}}%
\pgfpathlineto{\pgfqpoint{4.355684in}{4.494821in}}%
\pgfpathlineto{\pgfqpoint{4.357607in}{4.505518in}}%
\pgfpathlineto{\pgfqpoint{4.359531in}{4.505414in}}%
\pgfpathlineto{\pgfqpoint{4.361454in}{4.497159in}}%
\pgfpathlineto{\pgfqpoint{4.363378in}{4.513148in}}%
\pgfpathlineto{\pgfqpoint{4.365301in}{4.505513in}}%
\pgfpathlineto{\pgfqpoint{4.369148in}{4.529163in}}%
\pgfpathlineto{\pgfqpoint{4.371071in}{4.524862in}}%
\pgfpathlineto{\pgfqpoint{4.372995in}{4.541314in}}%
\pgfpathlineto{\pgfqpoint{4.374918in}{4.538135in}}%
\pgfpathlineto{\pgfqpoint{4.376842in}{4.546766in}}%
\pgfpathlineto{\pgfqpoint{4.382612in}{4.594624in}}%
\pgfpathlineto{\pgfqpoint{4.394152in}{4.623507in}}%
\pgfpathlineto{\pgfqpoint{4.396076in}{4.621263in}}%
\pgfpathlineto{\pgfqpoint{4.397999in}{4.629855in}}%
\pgfpathlineto{\pgfqpoint{4.399922in}{4.614402in}}%
\pgfpathlineto{\pgfqpoint{4.401846in}{4.611218in}}%
\pgfpathlineto{\pgfqpoint{4.403769in}{4.601392in}}%
\pgfpathlineto{\pgfqpoint{4.405693in}{4.597882in}}%
\pgfpathlineto{\pgfqpoint{4.409539in}{4.613743in}}%
\pgfpathlineto{\pgfqpoint{4.411463in}{4.620015in}}%
\pgfpathlineto{\pgfqpoint{4.413386in}{4.619553in}}%
\pgfpathlineto{\pgfqpoint{4.415310in}{4.596914in}}%
\pgfpathlineto{\pgfqpoint{4.417233in}{4.601309in}}%
\pgfpathlineto{\pgfqpoint{4.419157in}{4.600878in}}%
\pgfpathlineto{\pgfqpoint{4.421080in}{4.621780in}}%
\pgfpathlineto{\pgfqpoint{4.424927in}{4.617802in}}%
\pgfpathlineto{\pgfqpoint{4.426850in}{4.614264in}}%
\pgfpathlineto{\pgfqpoint{4.428774in}{4.620140in}}%
\pgfpathlineto{\pgfqpoint{4.430697in}{4.605800in}}%
\pgfpathlineto{\pgfqpoint{4.432620in}{4.611616in}}%
\pgfpathlineto{\pgfqpoint{4.434544in}{4.611820in}}%
\pgfpathlineto{\pgfqpoint{4.436467in}{4.630025in}}%
\pgfpathlineto{\pgfqpoint{4.438391in}{4.628033in}}%
\pgfpathlineto{\pgfqpoint{4.440314in}{4.613952in}}%
\pgfpathlineto{\pgfqpoint{4.444161in}{4.637686in}}%
\pgfpathlineto{\pgfqpoint{4.446084in}{4.641744in}}%
\pgfpathlineto{\pgfqpoint{4.448008in}{4.659998in}}%
\pgfpathlineto{\pgfqpoint{4.449931in}{4.651630in}}%
\pgfpathlineto{\pgfqpoint{4.451855in}{4.650221in}}%
\pgfpathlineto{\pgfqpoint{4.455701in}{4.663610in}}%
\pgfpathlineto{\pgfqpoint{4.457625in}{4.661485in}}%
\pgfpathlineto{\pgfqpoint{4.459548in}{4.668115in}}%
\pgfpathlineto{\pgfqpoint{4.463395in}{4.657417in}}%
\pgfpathlineto{\pgfqpoint{4.467242in}{4.660625in}}%
\pgfpathlineto{\pgfqpoint{4.469165in}{4.680854in}}%
\pgfpathlineto{\pgfqpoint{4.471089in}{4.683178in}}%
\pgfpathlineto{\pgfqpoint{4.473012in}{4.700981in}}%
\pgfpathlineto{\pgfqpoint{4.474935in}{4.683772in}}%
\pgfpathlineto{\pgfqpoint{4.480706in}{4.672090in}}%
\pgfpathlineto{\pgfqpoint{4.484553in}{4.694876in}}%
\pgfpathlineto{\pgfqpoint{4.486476in}{4.699261in}}%
\pgfpathlineto{\pgfqpoint{4.488399in}{4.684487in}}%
\pgfpathlineto{\pgfqpoint{4.492246in}{4.678461in}}%
\pgfpathlineto{\pgfqpoint{4.494170in}{4.671777in}}%
\pgfpathlineto{\pgfqpoint{4.496093in}{4.653844in}}%
\pgfpathlineto{\pgfqpoint{4.498016in}{4.652810in}}%
\pgfpathlineto{\pgfqpoint{4.499940in}{4.662274in}}%
\pgfpathlineto{\pgfqpoint{4.501863in}{4.662962in}}%
\pgfpathlineto{\pgfqpoint{4.503787in}{4.660348in}}%
\pgfpathlineto{\pgfqpoint{4.511480in}{4.698152in}}%
\pgfpathlineto{\pgfqpoint{4.513404in}{4.709919in}}%
\pgfpathlineto{\pgfqpoint{4.515327in}{4.702886in}}%
\pgfpathlineto{\pgfqpoint{4.517251in}{4.716358in}}%
\pgfpathlineto{\pgfqpoint{4.519174in}{4.715685in}}%
\pgfpathlineto{\pgfqpoint{4.523021in}{4.731954in}}%
\pgfpathlineto{\pgfqpoint{4.524944in}{4.718121in}}%
\pgfpathlineto{\pgfqpoint{4.526868in}{4.718003in}}%
\pgfpathlineto{\pgfqpoint{4.530714in}{4.708865in}}%
\pgfpathlineto{\pgfqpoint{4.536485in}{4.733595in}}%
\pgfpathlineto{\pgfqpoint{4.538408in}{4.735345in}}%
\pgfpathlineto{\pgfqpoint{4.540331in}{4.743034in}}%
\pgfpathlineto{\pgfqpoint{4.542255in}{4.732705in}}%
\pgfpathlineto{\pgfqpoint{4.544178in}{4.730761in}}%
\pgfpathlineto{\pgfqpoint{4.546102in}{4.732990in}}%
\pgfpathlineto{\pgfqpoint{4.548025in}{4.743950in}}%
\pgfpathlineto{\pgfqpoint{4.551872in}{4.724832in}}%
\pgfpathlineto{\pgfqpoint{4.553795in}{4.715920in}}%
\pgfpathlineto{\pgfqpoint{4.555719in}{4.726478in}}%
\pgfpathlineto{\pgfqpoint{4.559566in}{4.691685in}}%
\pgfpathlineto{\pgfqpoint{4.563412in}{4.707011in}}%
\pgfpathlineto{\pgfqpoint{4.565336in}{4.709772in}}%
\pgfpathlineto{\pgfqpoint{4.567259in}{4.718026in}}%
\pgfpathlineto{\pgfqpoint{4.569183in}{4.709319in}}%
\pgfpathlineto{\pgfqpoint{4.571106in}{4.710076in}}%
\pgfpathlineto{\pgfqpoint{4.578800in}{4.751812in}}%
\pgfpathlineto{\pgfqpoint{4.580723in}{4.749748in}}%
\pgfpathlineto{\pgfqpoint{4.588417in}{4.715525in}}%
\pgfpathlineto{\pgfqpoint{4.590340in}{4.699072in}}%
\pgfpathlineto{\pgfqpoint{4.594187in}{4.697209in}}%
\pgfpathlineto{\pgfqpoint{4.596110in}{4.674467in}}%
\pgfpathlineto{\pgfqpoint{4.599957in}{4.659313in}}%
\pgfpathlineto{\pgfqpoint{4.601881in}{4.677972in}}%
\pgfpathlineto{\pgfqpoint{4.603804in}{4.679611in}}%
\pgfpathlineto{\pgfqpoint{4.605727in}{4.666717in}}%
\pgfpathlineto{\pgfqpoint{4.607651in}{4.662555in}}%
\pgfpathlineto{\pgfqpoint{4.609574in}{4.691890in}}%
\pgfpathlineto{\pgfqpoint{4.611498in}{4.699884in}}%
\pgfpathlineto{\pgfqpoint{4.613421in}{4.700451in}}%
\pgfpathlineto{\pgfqpoint{4.619191in}{4.719369in}}%
\pgfpathlineto{\pgfqpoint{4.623038in}{4.761406in}}%
\pgfpathlineto{\pgfqpoint{4.624962in}{4.741262in}}%
\pgfpathlineto{\pgfqpoint{4.626885in}{4.754877in}}%
\pgfpathlineto{\pgfqpoint{4.628808in}{4.745241in}}%
\pgfpathlineto{\pgfqpoint{4.630732in}{4.745574in}}%
\pgfpathlineto{\pgfqpoint{4.632655in}{4.739286in}}%
\pgfpathlineto{\pgfqpoint{4.636502in}{4.754888in}}%
\pgfpathlineto{\pgfqpoint{4.638425in}{4.749983in}}%
\pgfpathlineto{\pgfqpoint{4.640349in}{4.758488in}}%
\pgfpathlineto{\pgfqpoint{4.642272in}{4.752374in}}%
\pgfpathlineto{\pgfqpoint{4.646119in}{4.756403in}}%
\pgfpathlineto{\pgfqpoint{4.648042in}{4.754666in}}%
\pgfpathlineto{\pgfqpoint{4.649966in}{4.749963in}}%
\pgfpathlineto{\pgfqpoint{4.651889in}{4.748636in}}%
\pgfpathlineto{\pgfqpoint{4.653813in}{4.748892in}}%
\pgfpathlineto{\pgfqpoint{4.655736in}{4.733160in}}%
\pgfpathlineto{\pgfqpoint{4.657660in}{4.737872in}}%
\pgfpathlineto{\pgfqpoint{4.659583in}{4.727478in}}%
\pgfpathlineto{\pgfqpoint{4.661506in}{4.730359in}}%
\pgfpathlineto{\pgfqpoint{4.663430in}{4.726142in}}%
\pgfpathlineto{\pgfqpoint{4.665353in}{4.726346in}}%
\pgfpathlineto{\pgfqpoint{4.669200in}{4.748358in}}%
\pgfpathlineto{\pgfqpoint{4.671123in}{4.737250in}}%
\pgfpathlineto{\pgfqpoint{4.674970in}{4.746616in}}%
\pgfpathlineto{\pgfqpoint{4.676894in}{4.764182in}}%
\pgfpathlineto{\pgfqpoint{4.678817in}{4.764009in}}%
\pgfpathlineto{\pgfqpoint{4.680740in}{4.762382in}}%
\pgfpathlineto{\pgfqpoint{4.682664in}{4.763813in}}%
\pgfpathlineto{\pgfqpoint{4.684587in}{4.762033in}}%
\pgfpathlineto{\pgfqpoint{4.686511in}{4.743779in}}%
\pgfpathlineto{\pgfqpoint{4.688434in}{4.742113in}}%
\pgfpathlineto{\pgfqpoint{4.690358in}{4.736259in}}%
\pgfpathlineto{\pgfqpoint{4.692281in}{4.719226in}}%
\pgfpathlineto{\pgfqpoint{4.696128in}{4.743094in}}%
\pgfpathlineto{\pgfqpoint{4.698051in}{4.733261in}}%
\pgfpathlineto{\pgfqpoint{4.701898in}{4.736348in}}%
\pgfpathlineto{\pgfqpoint{4.703821in}{4.741999in}}%
\pgfpathlineto{\pgfqpoint{4.705745in}{4.723015in}}%
\pgfpathlineto{\pgfqpoint{4.707668in}{4.725991in}}%
\pgfpathlineto{\pgfqpoint{4.709592in}{4.732761in}}%
\pgfpathlineto{\pgfqpoint{4.711515in}{4.726789in}}%
\pgfpathlineto{\pgfqpoint{4.715362in}{4.725504in}}%
\pgfpathlineto{\pgfqpoint{4.717285in}{4.728377in}}%
\pgfpathlineto{\pgfqpoint{4.721132in}{4.754750in}}%
\pgfpathlineto{\pgfqpoint{4.724979in}{4.722345in}}%
\pgfpathlineto{\pgfqpoint{4.726902in}{4.731820in}}%
\pgfpathlineto{\pgfqpoint{4.728826in}{4.725038in}}%
\pgfpathlineto{\pgfqpoint{4.730749in}{4.728554in}}%
\pgfpathlineto{\pgfqpoint{4.732673in}{4.711795in}}%
\pgfpathlineto{\pgfqpoint{4.736519in}{4.726535in}}%
\pgfpathlineto{\pgfqpoint{4.742290in}{4.755546in}}%
\pgfpathlineto{\pgfqpoint{4.744213in}{4.743524in}}%
\pgfpathlineto{\pgfqpoint{4.748060in}{4.756977in}}%
\pgfpathlineto{\pgfqpoint{4.749983in}{4.769353in}}%
\pgfpathlineto{\pgfqpoint{4.751907in}{4.770212in}}%
\pgfpathlineto{\pgfqpoint{4.753830in}{4.760129in}}%
\pgfpathlineto{\pgfqpoint{4.755753in}{4.743632in}}%
\pgfpathlineto{\pgfqpoint{4.757677in}{4.751602in}}%
\pgfpathlineto{\pgfqpoint{4.759600in}{4.741754in}}%
\pgfpathlineto{\pgfqpoint{4.763447in}{4.756492in}}%
\pgfpathlineto{\pgfqpoint{4.767294in}{4.740597in}}%
\pgfpathlineto{\pgfqpoint{4.771141in}{4.753368in}}%
\pgfpathlineto{\pgfqpoint{4.774988in}{4.761135in}}%
\pgfpathlineto{\pgfqpoint{4.776911in}{4.751874in}}%
\pgfpathlineto{\pgfqpoint{4.778834in}{4.736371in}}%
\pgfpathlineto{\pgfqpoint{4.782681in}{4.740305in}}%
\pgfpathlineto{\pgfqpoint{4.784605in}{4.743547in}}%
\pgfpathlineto{\pgfqpoint{4.786528in}{4.753930in}}%
\pgfpathlineto{\pgfqpoint{4.788451in}{4.745686in}}%
\pgfpathlineto{\pgfqpoint{4.792298in}{4.748457in}}%
\pgfpathlineto{\pgfqpoint{4.794222in}{4.735232in}}%
\pgfpathlineto{\pgfqpoint{4.796145in}{4.743688in}}%
\pgfpathlineto{\pgfqpoint{4.801915in}{4.779624in}}%
\pgfpathlineto{\pgfqpoint{4.805762in}{4.752510in}}%
\pgfpathlineto{\pgfqpoint{4.807686in}{4.747805in}}%
\pgfpathlineto{\pgfqpoint{4.811532in}{4.763530in}}%
\pgfpathlineto{\pgfqpoint{4.813456in}{4.746780in}}%
\pgfpathlineto{\pgfqpoint{4.815379in}{4.743751in}}%
\pgfpathlineto{\pgfqpoint{4.817303in}{4.745385in}}%
\pgfpathlineto{\pgfqpoint{4.821149in}{4.738224in}}%
\pgfpathlineto{\pgfqpoint{4.823073in}{4.732273in}}%
\pgfpathlineto{\pgfqpoint{4.824996in}{4.711447in}}%
\pgfpathlineto{\pgfqpoint{4.826920in}{4.719933in}}%
\pgfpathlineto{\pgfqpoint{4.828843in}{4.720738in}}%
\pgfpathlineto{\pgfqpoint{4.830767in}{4.705048in}}%
\pgfpathlineto{\pgfqpoint{4.832690in}{4.699124in}}%
\pgfpathlineto{\pgfqpoint{4.834613in}{4.700188in}}%
\pgfpathlineto{\pgfqpoint{4.836537in}{4.705403in}}%
\pgfpathlineto{\pgfqpoint{4.840384in}{4.703631in}}%
\pgfpathlineto{\pgfqpoint{4.842307in}{4.695234in}}%
\pgfpathlineto{\pgfqpoint{4.844230in}{4.698307in}}%
\pgfpathlineto{\pgfqpoint{4.846154in}{4.707221in}}%
\pgfpathlineto{\pgfqpoint{4.848077in}{4.707192in}}%
\pgfpathlineto{\pgfqpoint{4.850001in}{4.700651in}}%
\pgfpathlineto{\pgfqpoint{4.851924in}{4.709568in}}%
\pgfpathlineto{\pgfqpoint{4.853847in}{4.695085in}}%
\pgfpathlineto{\pgfqpoint{4.855771in}{4.690140in}}%
\pgfpathlineto{\pgfqpoint{4.857694in}{4.688375in}}%
\pgfpathlineto{\pgfqpoint{4.859618in}{4.680772in}}%
\pgfpathlineto{\pgfqpoint{4.861541in}{4.689886in}}%
\pgfpathlineto{\pgfqpoint{4.863464in}{4.692306in}}%
\pgfpathlineto{\pgfqpoint{4.869235in}{4.663786in}}%
\pgfpathlineto{\pgfqpoint{4.871158in}{4.672527in}}%
\pgfpathlineto{\pgfqpoint{4.873082in}{4.675080in}}%
\pgfpathlineto{\pgfqpoint{4.875005in}{4.673463in}}%
\pgfpathlineto{\pgfqpoint{4.876928in}{4.662377in}}%
\pgfpathlineto{\pgfqpoint{4.878852in}{4.667401in}}%
\pgfpathlineto{\pgfqpoint{4.882699in}{4.669480in}}%
\pgfpathlineto{\pgfqpoint{4.884622in}{4.670077in}}%
\pgfpathlineto{\pgfqpoint{4.886545in}{4.662924in}}%
\pgfpathlineto{\pgfqpoint{4.888469in}{4.660900in}}%
\pgfpathlineto{\pgfqpoint{4.890392in}{4.665074in}}%
\pgfpathlineto{\pgfqpoint{4.894239in}{4.664351in}}%
\pgfpathlineto{\pgfqpoint{4.898086in}{4.668477in}}%
\pgfpathlineto{\pgfqpoint{4.901933in}{4.642784in}}%
\pgfpathlineto{\pgfqpoint{4.903856in}{4.652047in}}%
\pgfpathlineto{\pgfqpoint{4.905780in}{4.650793in}}%
\pgfpathlineto{\pgfqpoint{4.907703in}{4.663001in}}%
\pgfpathlineto{\pgfqpoint{4.909626in}{4.639637in}}%
\pgfpathlineto{\pgfqpoint{4.913473in}{4.626462in}}%
\pgfpathlineto{\pgfqpoint{4.917320in}{4.612663in}}%
\pgfpathlineto{\pgfqpoint{4.921167in}{4.628330in}}%
\pgfpathlineto{\pgfqpoint{4.923090in}{4.623567in}}%
\pgfpathlineto{\pgfqpoint{4.926937in}{4.602690in}}%
\pgfpathlineto{\pgfqpoint{4.930784in}{4.616761in}}%
\pgfpathlineto{\pgfqpoint{4.934631in}{4.600219in}}%
\pgfpathlineto{\pgfqpoint{4.936554in}{4.606477in}}%
\pgfpathlineto{\pgfqpoint{4.940401in}{4.629703in}}%
\pgfpathlineto{\pgfqpoint{4.944248in}{4.630583in}}%
\pgfpathlineto{\pgfqpoint{4.948095in}{4.660520in}}%
\pgfpathlineto{\pgfqpoint{4.951941in}{4.636452in}}%
\pgfpathlineto{\pgfqpoint{4.959635in}{4.671039in}}%
\pgfpathlineto{\pgfqpoint{4.961558in}{4.684886in}}%
\pgfpathlineto{\pgfqpoint{4.963482in}{4.711387in}}%
\pgfpathlineto{\pgfqpoint{4.965405in}{4.699499in}}%
\pgfpathlineto{\pgfqpoint{4.967329in}{4.722836in}}%
\pgfpathlineto{\pgfqpoint{4.971176in}{4.721151in}}%
\pgfpathlineto{\pgfqpoint{4.973099in}{4.717438in}}%
\pgfpathlineto{\pgfqpoint{4.975022in}{4.699984in}}%
\pgfpathlineto{\pgfqpoint{4.976946in}{4.700588in}}%
\pgfpathlineto{\pgfqpoint{4.982716in}{4.669118in}}%
\pgfpathlineto{\pgfqpoint{4.984639in}{4.659586in}}%
\pgfpathlineto{\pgfqpoint{4.986563in}{4.660194in}}%
\pgfpathlineto{\pgfqpoint{4.992333in}{4.677960in}}%
\pgfpathlineto{\pgfqpoint{4.994256in}{4.661719in}}%
\pgfpathlineto{\pgfqpoint{4.996180in}{4.661542in}}%
\pgfpathlineto{\pgfqpoint{4.998103in}{4.667765in}}%
\pgfpathlineto{\pgfqpoint{5.000027in}{4.659859in}}%
\pgfpathlineto{\pgfqpoint{5.003874in}{4.684353in}}%
\pgfpathlineto{\pgfqpoint{5.005797in}{4.692709in}}%
\pgfpathlineto{\pgfqpoint{5.007720in}{4.690277in}}%
\pgfpathlineto{\pgfqpoint{5.011567in}{4.679600in}}%
\pgfpathlineto{\pgfqpoint{5.013491in}{4.693000in}}%
\pgfpathlineto{\pgfqpoint{5.015414in}{4.693518in}}%
\pgfpathlineto{\pgfqpoint{5.017337in}{4.683922in}}%
\pgfpathlineto{\pgfqpoint{5.019261in}{4.687626in}}%
\pgfpathlineto{\pgfqpoint{5.025031in}{4.643488in}}%
\pgfpathlineto{\pgfqpoint{5.026954in}{4.629960in}}%
\pgfpathlineto{\pgfqpoint{5.028878in}{4.626803in}}%
\pgfpathlineto{\pgfqpoint{5.032725in}{4.618601in}}%
\pgfpathlineto{\pgfqpoint{5.034648in}{4.623307in}}%
\pgfpathlineto{\pgfqpoint{5.036571in}{4.621471in}}%
\pgfpathlineto{\pgfqpoint{5.038495in}{4.623386in}}%
\pgfpathlineto{\pgfqpoint{5.040418in}{4.614844in}}%
\pgfpathlineto{\pgfqpoint{5.046189in}{4.642856in}}%
\pgfpathlineto{\pgfqpoint{5.048112in}{4.628067in}}%
\pgfpathlineto{\pgfqpoint{5.050035in}{4.633315in}}%
\pgfpathlineto{\pgfqpoint{5.051959in}{4.619635in}}%
\pgfpathlineto{\pgfqpoint{5.053882in}{4.629627in}}%
\pgfpathlineto{\pgfqpoint{5.057729in}{4.605991in}}%
\pgfpathlineto{\pgfqpoint{5.059652in}{4.622097in}}%
\pgfpathlineto{\pgfqpoint{5.061576in}{4.627538in}}%
\pgfpathlineto{\pgfqpoint{5.063499in}{4.627607in}}%
\pgfpathlineto{\pgfqpoint{5.065423in}{4.609107in}}%
\pgfpathlineto{\pgfqpoint{5.067346in}{4.614581in}}%
\pgfpathlineto{\pgfqpoint{5.069269in}{4.594991in}}%
\pgfpathlineto{\pgfqpoint{5.071193in}{4.592448in}}%
\pgfpathlineto{\pgfqpoint{5.073116in}{4.604839in}}%
\pgfpathlineto{\pgfqpoint{5.075040in}{4.602669in}}%
\pgfpathlineto{\pgfqpoint{5.080810in}{4.572606in}}%
\pgfpathlineto{\pgfqpoint{5.082733in}{4.570988in}}%
\pgfpathlineto{\pgfqpoint{5.086580in}{4.590060in}}%
\pgfpathlineto{\pgfqpoint{5.088504in}{4.621291in}}%
\pgfpathlineto{\pgfqpoint{5.092350in}{4.602582in}}%
\pgfpathlineto{\pgfqpoint{5.094274in}{4.607098in}}%
\pgfpathlineto{\pgfqpoint{5.096197in}{4.606386in}}%
\pgfpathlineto{\pgfqpoint{5.098121in}{4.625656in}}%
\pgfpathlineto{\pgfqpoint{5.100044in}{4.627507in}}%
\pgfpathlineto{\pgfqpoint{5.101967in}{4.606294in}}%
\pgfpathlineto{\pgfqpoint{5.103891in}{4.608053in}}%
\pgfpathlineto{\pgfqpoint{5.105814in}{4.606877in}}%
\pgfpathlineto{\pgfqpoint{5.107738in}{4.607898in}}%
\pgfpathlineto{\pgfqpoint{5.111585in}{4.587114in}}%
\pgfpathlineto{\pgfqpoint{5.113508in}{4.581685in}}%
\pgfpathlineto{\pgfqpoint{5.115431in}{4.584418in}}%
\pgfpathlineto{\pgfqpoint{5.119278in}{4.602054in}}%
\pgfpathlineto{\pgfqpoint{5.121202in}{4.621364in}}%
\pgfpathlineto{\pgfqpoint{5.123125in}{4.628771in}}%
\pgfpathlineto{\pgfqpoint{5.126972in}{4.590468in}}%
\pgfpathlineto{\pgfqpoint{5.128895in}{4.591232in}}%
\pgfpathlineto{\pgfqpoint{5.130819in}{4.587983in}}%
\pgfpathlineto{\pgfqpoint{5.132742in}{4.582367in}}%
\pgfpathlineto{\pgfqpoint{5.136589in}{4.592922in}}%
\pgfpathlineto{\pgfqpoint{5.138512in}{4.592012in}}%
\pgfpathlineto{\pgfqpoint{5.142359in}{4.619781in}}%
\pgfpathlineto{\pgfqpoint{5.144283in}{4.606590in}}%
\pgfpathlineto{\pgfqpoint{5.146206in}{4.630763in}}%
\pgfpathlineto{\pgfqpoint{5.153900in}{4.665992in}}%
\pgfpathlineto{\pgfqpoint{5.155823in}{4.670357in}}%
\pgfpathlineto{\pgfqpoint{5.157746in}{4.661217in}}%
\pgfpathlineto{\pgfqpoint{5.159670in}{4.658442in}}%
\pgfpathlineto{\pgfqpoint{5.161593in}{4.660860in}}%
\pgfpathlineto{\pgfqpoint{5.163517in}{4.657515in}}%
\pgfpathlineto{\pgfqpoint{5.165440in}{4.666929in}}%
\pgfpathlineto{\pgfqpoint{5.167363in}{4.663242in}}%
\pgfpathlineto{\pgfqpoint{5.169287in}{4.663778in}}%
\pgfpathlineto{\pgfqpoint{5.173134in}{4.656511in}}%
\pgfpathlineto{\pgfqpoint{5.175057in}{4.657894in}}%
\pgfpathlineto{\pgfqpoint{5.178904in}{4.635271in}}%
\pgfpathlineto{\pgfqpoint{5.180827in}{4.639163in}}%
\pgfpathlineto{\pgfqpoint{5.182751in}{4.637978in}}%
\pgfpathlineto{\pgfqpoint{5.186598in}{4.646202in}}%
\pgfpathlineto{\pgfqpoint{5.190444in}{4.653932in}}%
\pgfpathlineto{\pgfqpoint{5.192368in}{4.638453in}}%
\pgfpathlineto{\pgfqpoint{5.194291in}{4.633220in}}%
\pgfpathlineto{\pgfqpoint{5.196215in}{4.638484in}}%
\pgfpathlineto{\pgfqpoint{5.198138in}{4.625802in}}%
\pgfpathlineto{\pgfqpoint{5.200061in}{4.622225in}}%
\pgfpathlineto{\pgfqpoint{5.201985in}{4.623793in}}%
\pgfpathlineto{\pgfqpoint{5.207755in}{4.655411in}}%
\pgfpathlineto{\pgfqpoint{5.209678in}{4.653514in}}%
\pgfpathlineto{\pgfqpoint{5.211602in}{4.662045in}}%
\pgfpathlineto{\pgfqpoint{5.213525in}{4.639177in}}%
\pgfpathlineto{\pgfqpoint{5.217372in}{4.670567in}}%
\pgfpathlineto{\pgfqpoint{5.219296in}{4.667779in}}%
\pgfpathlineto{\pgfqpoint{5.221219in}{4.674590in}}%
\pgfpathlineto{\pgfqpoint{5.223142in}{4.663005in}}%
\pgfpathlineto{\pgfqpoint{5.225066in}{4.665462in}}%
\pgfpathlineto{\pgfqpoint{5.226989in}{4.672259in}}%
\pgfpathlineto{\pgfqpoint{5.230836in}{4.657884in}}%
\pgfpathlineto{\pgfqpoint{5.232759in}{4.662400in}}%
\pgfpathlineto{\pgfqpoint{5.234683in}{4.661395in}}%
\pgfpathlineto{\pgfqpoint{5.236606in}{4.678617in}}%
\pgfpathlineto{\pgfqpoint{5.238530in}{4.661770in}}%
\pgfpathlineto{\pgfqpoint{5.242376in}{4.653109in}}%
\pgfpathlineto{\pgfqpoint{5.246223in}{4.677384in}}%
\pgfpathlineto{\pgfqpoint{5.250070in}{4.693700in}}%
\pgfpathlineto{\pgfqpoint{5.251994in}{4.687387in}}%
\pgfpathlineto{\pgfqpoint{5.253917in}{4.687550in}}%
\pgfpathlineto{\pgfqpoint{5.255840in}{4.683397in}}%
\pgfpathlineto{\pgfqpoint{5.257764in}{4.689798in}}%
\pgfpathlineto{\pgfqpoint{5.259687in}{4.690399in}}%
\pgfpathlineto{\pgfqpoint{5.261611in}{4.698165in}}%
\pgfpathlineto{\pgfqpoint{5.263534in}{4.695326in}}%
\pgfpathlineto{\pgfqpoint{5.265457in}{4.685872in}}%
\pgfpathlineto{\pgfqpoint{5.267381in}{4.686386in}}%
\pgfpathlineto{\pgfqpoint{5.271228in}{4.681985in}}%
\pgfpathlineto{\pgfqpoint{5.273151in}{4.650097in}}%
\pgfpathlineto{\pgfqpoint{5.275074in}{4.639903in}}%
\pgfpathlineto{\pgfqpoint{5.276998in}{4.637773in}}%
\pgfpathlineto{\pgfqpoint{5.278921in}{4.633638in}}%
\pgfpathlineto{\pgfqpoint{5.280845in}{4.632321in}}%
\pgfpathlineto{\pgfqpoint{5.284692in}{4.611330in}}%
\pgfpathlineto{\pgfqpoint{5.286615in}{4.617603in}}%
\pgfpathlineto{\pgfqpoint{5.288538in}{4.618322in}}%
\pgfpathlineto{\pgfqpoint{5.290462in}{4.626702in}}%
\pgfpathlineto{\pgfqpoint{5.294309in}{4.613809in}}%
\pgfpathlineto{\pgfqpoint{5.296232in}{4.617417in}}%
\pgfpathlineto{\pgfqpoint{5.298155in}{4.629425in}}%
\pgfpathlineto{\pgfqpoint{5.300079in}{4.618245in}}%
\pgfpathlineto{\pgfqpoint{5.302002in}{4.634966in}}%
\pgfpathlineto{\pgfqpoint{5.303926in}{4.639852in}}%
\pgfpathlineto{\pgfqpoint{5.303926in}{4.639852in}}%
\pgfusepath{stroke}%
\end{pgfscope}%
\begin{pgfscope}%
\pgfpathrectangle{\pgfqpoint{3.286364in}{3.180000in}}{\pgfqpoint{2.113636in}{2.100000in}}%
\pgfusepath{clip}%
\pgfsetroundcap%
\pgfsetroundjoin%
\pgfsetlinewidth{0.602250pt}%
\definecolor{currentstroke}{rgb}{0.650980,0.337255,0.156863}%
\pgfsetstrokecolor{currentstroke}%
\pgfsetdash{}{0pt}%
\pgfpathmoveto{\pgfqpoint{3.382438in}{4.244666in}}%
\pgfpathlineto{\pgfqpoint{3.384361in}{4.234966in}}%
\pgfpathlineto{\pgfqpoint{3.386285in}{4.233689in}}%
\pgfpathlineto{\pgfqpoint{3.388208in}{4.216590in}}%
\pgfpathlineto{\pgfqpoint{3.390132in}{4.233475in}}%
\pgfpathlineto{\pgfqpoint{3.392055in}{4.237049in}}%
\pgfpathlineto{\pgfqpoint{3.397825in}{4.262596in}}%
\pgfpathlineto{\pgfqpoint{3.399749in}{4.258320in}}%
\pgfpathlineto{\pgfqpoint{3.401672in}{4.268358in}}%
\pgfpathlineto{\pgfqpoint{3.403596in}{4.267467in}}%
\pgfpathlineto{\pgfqpoint{3.407442in}{4.259085in}}%
\pgfpathlineto{\pgfqpoint{3.409366in}{4.262680in}}%
\pgfpathlineto{\pgfqpoint{3.411289in}{4.261266in}}%
\pgfpathlineto{\pgfqpoint{3.415136in}{4.277034in}}%
\pgfpathlineto{\pgfqpoint{3.417059in}{4.275210in}}%
\pgfpathlineto{\pgfqpoint{3.420906in}{4.307943in}}%
\pgfpathlineto{\pgfqpoint{3.422830in}{4.306736in}}%
\pgfpathlineto{\pgfqpoint{3.424753in}{4.340018in}}%
\pgfpathlineto{\pgfqpoint{3.426676in}{4.340997in}}%
\pgfpathlineto{\pgfqpoint{3.428600in}{4.352879in}}%
\pgfpathlineto{\pgfqpoint{3.432447in}{4.343139in}}%
\pgfpathlineto{\pgfqpoint{3.434370in}{4.333488in}}%
\pgfpathlineto{\pgfqpoint{3.436294in}{4.340847in}}%
\pgfpathlineto{\pgfqpoint{3.438217in}{4.323090in}}%
\pgfpathlineto{\pgfqpoint{3.440140in}{4.320015in}}%
\pgfpathlineto{\pgfqpoint{3.442064in}{4.330749in}}%
\pgfpathlineto{\pgfqpoint{3.443987in}{4.331150in}}%
\pgfpathlineto{\pgfqpoint{3.445911in}{4.333092in}}%
\pgfpathlineto{\pgfqpoint{3.447834in}{4.331768in}}%
\pgfpathlineto{\pgfqpoint{3.451681in}{4.355102in}}%
\pgfpathlineto{\pgfqpoint{3.453604in}{4.352076in}}%
\pgfpathlineto{\pgfqpoint{3.455528in}{4.351678in}}%
\pgfpathlineto{\pgfqpoint{3.457451in}{4.357282in}}%
\pgfpathlineto{\pgfqpoint{3.461298in}{4.343973in}}%
\pgfpathlineto{\pgfqpoint{3.467068in}{4.371033in}}%
\pgfpathlineto{\pgfqpoint{3.468992in}{4.364016in}}%
\pgfpathlineto{\pgfqpoint{3.470915in}{4.369647in}}%
\pgfpathlineto{\pgfqpoint{3.474762in}{4.404354in}}%
\pgfpathlineto{\pgfqpoint{3.478609in}{4.429287in}}%
\pgfpathlineto{\pgfqpoint{3.480532in}{4.427422in}}%
\pgfpathlineto{\pgfqpoint{3.482455in}{4.423422in}}%
\pgfpathlineto{\pgfqpoint{3.484379in}{4.429825in}}%
\pgfpathlineto{\pgfqpoint{3.488226in}{4.422855in}}%
\pgfpathlineto{\pgfqpoint{3.490149in}{4.426331in}}%
\pgfpathlineto{\pgfqpoint{3.492072in}{4.420732in}}%
\pgfpathlineto{\pgfqpoint{3.493996in}{4.400685in}}%
\pgfpathlineto{\pgfqpoint{3.495919in}{4.395529in}}%
\pgfpathlineto{\pgfqpoint{3.497843in}{4.401466in}}%
\pgfpathlineto{\pgfqpoint{3.499766in}{4.394444in}}%
\pgfpathlineto{\pgfqpoint{3.503613in}{4.423396in}}%
\pgfpathlineto{\pgfqpoint{3.505536in}{4.421944in}}%
\pgfpathlineto{\pgfqpoint{3.507460in}{4.437393in}}%
\pgfpathlineto{\pgfqpoint{3.509383in}{4.438220in}}%
\pgfpathlineto{\pgfqpoint{3.511307in}{4.431162in}}%
\pgfpathlineto{\pgfqpoint{3.513230in}{4.434133in}}%
\pgfpathlineto{\pgfqpoint{3.517077in}{4.427963in}}%
\pgfpathlineto{\pgfqpoint{3.519000in}{4.431493in}}%
\pgfpathlineto{\pgfqpoint{3.520924in}{4.438614in}}%
\pgfpathlineto{\pgfqpoint{3.522847in}{4.434317in}}%
\pgfpathlineto{\pgfqpoint{3.524770in}{4.443288in}}%
\pgfpathlineto{\pgfqpoint{3.526694in}{4.439649in}}%
\pgfpathlineto{\pgfqpoint{3.528617in}{4.453858in}}%
\pgfpathlineto{\pgfqpoint{3.530541in}{4.458201in}}%
\pgfpathlineto{\pgfqpoint{3.532464in}{4.472236in}}%
\pgfpathlineto{\pgfqpoint{3.534387in}{4.466931in}}%
\pgfpathlineto{\pgfqpoint{3.538234in}{4.482139in}}%
\pgfpathlineto{\pgfqpoint{3.540158in}{4.476128in}}%
\pgfpathlineto{\pgfqpoint{3.542081in}{4.482178in}}%
\pgfpathlineto{\pgfqpoint{3.544005in}{4.459367in}}%
\pgfpathlineto{\pgfqpoint{3.545928in}{4.459707in}}%
\pgfpathlineto{\pgfqpoint{3.547851in}{4.468334in}}%
\pgfpathlineto{\pgfqpoint{3.551698in}{4.470568in}}%
\pgfpathlineto{\pgfqpoint{3.553622in}{4.482225in}}%
\pgfpathlineto{\pgfqpoint{3.555545in}{4.469333in}}%
\pgfpathlineto{\pgfqpoint{3.557468in}{4.470774in}}%
\pgfpathlineto{\pgfqpoint{3.559392in}{4.470684in}}%
\pgfpathlineto{\pgfqpoint{3.561315in}{4.466071in}}%
\pgfpathlineto{\pgfqpoint{3.563239in}{4.483798in}}%
\pgfpathlineto{\pgfqpoint{3.567085in}{4.492973in}}%
\pgfpathlineto{\pgfqpoint{3.569009in}{4.490612in}}%
\pgfpathlineto{\pgfqpoint{3.570932in}{4.494321in}}%
\pgfpathlineto{\pgfqpoint{3.574779in}{4.470877in}}%
\pgfpathlineto{\pgfqpoint{3.576703in}{4.479831in}}%
\pgfpathlineto{\pgfqpoint{3.578626in}{4.478132in}}%
\pgfpathlineto{\pgfqpoint{3.580549in}{4.483182in}}%
\pgfpathlineto{\pgfqpoint{3.584396in}{4.511553in}}%
\pgfpathlineto{\pgfqpoint{3.586320in}{4.514569in}}%
\pgfpathlineto{\pgfqpoint{3.592090in}{4.533928in}}%
\pgfpathlineto{\pgfqpoint{3.595937in}{4.563052in}}%
\pgfpathlineto{\pgfqpoint{3.597860in}{4.564414in}}%
\pgfpathlineto{\pgfqpoint{3.599783in}{4.548588in}}%
\pgfpathlineto{\pgfqpoint{3.601707in}{4.559188in}}%
\pgfpathlineto{\pgfqpoint{3.605554in}{4.547676in}}%
\pgfpathlineto{\pgfqpoint{3.607477in}{4.537825in}}%
\pgfpathlineto{\pgfqpoint{3.609401in}{4.549340in}}%
\pgfpathlineto{\pgfqpoint{3.611324in}{4.549528in}}%
\pgfpathlineto{\pgfqpoint{3.613247in}{4.548379in}}%
\pgfpathlineto{\pgfqpoint{3.615171in}{4.560176in}}%
\pgfpathlineto{\pgfqpoint{3.617094in}{4.555704in}}%
\pgfpathlineto{\pgfqpoint{3.620941in}{4.574062in}}%
\pgfpathlineto{\pgfqpoint{3.622864in}{4.562063in}}%
\pgfpathlineto{\pgfqpoint{3.624788in}{4.566283in}}%
\pgfpathlineto{\pgfqpoint{3.626711in}{4.584566in}}%
\pgfpathlineto{\pgfqpoint{3.628635in}{4.591952in}}%
\pgfpathlineto{\pgfqpoint{3.638252in}{4.599875in}}%
\pgfpathlineto{\pgfqpoint{3.640175in}{4.601501in}}%
\pgfpathlineto{\pgfqpoint{3.642099in}{4.606533in}}%
\pgfpathlineto{\pgfqpoint{3.644022in}{4.601155in}}%
\pgfpathlineto{\pgfqpoint{3.647869in}{4.614442in}}%
\pgfpathlineto{\pgfqpoint{3.649792in}{4.611017in}}%
\pgfpathlineto{\pgfqpoint{3.651716in}{4.615605in}}%
\pgfpathlineto{\pgfqpoint{3.653639in}{4.593507in}}%
\pgfpathlineto{\pgfqpoint{3.659409in}{4.568494in}}%
\pgfpathlineto{\pgfqpoint{3.661333in}{4.578376in}}%
\pgfpathlineto{\pgfqpoint{3.665179in}{4.570625in}}%
\pgfpathlineto{\pgfqpoint{3.667103in}{4.559498in}}%
\pgfpathlineto{\pgfqpoint{3.669026in}{4.561552in}}%
\pgfpathlineto{\pgfqpoint{3.670950in}{4.550849in}}%
\pgfpathlineto{\pgfqpoint{3.672873in}{4.551820in}}%
\pgfpathlineto{\pgfqpoint{3.678643in}{4.559073in}}%
\pgfpathlineto{\pgfqpoint{3.682490in}{4.564986in}}%
\pgfpathlineto{\pgfqpoint{3.686337in}{4.550143in}}%
\pgfpathlineto{\pgfqpoint{3.688260in}{4.549122in}}%
\pgfpathlineto{\pgfqpoint{3.690184in}{4.558962in}}%
\pgfpathlineto{\pgfqpoint{3.692107in}{4.554199in}}%
\pgfpathlineto{\pgfqpoint{3.694031in}{4.552537in}}%
\pgfpathlineto{\pgfqpoint{3.697877in}{4.556917in}}%
\pgfpathlineto{\pgfqpoint{3.701724in}{4.569364in}}%
\pgfpathlineto{\pgfqpoint{3.703648in}{4.562847in}}%
\pgfpathlineto{\pgfqpoint{3.705571in}{4.543658in}}%
\pgfpathlineto{\pgfqpoint{3.707494in}{4.544332in}}%
\pgfpathlineto{\pgfqpoint{3.709418in}{4.556145in}}%
\pgfpathlineto{\pgfqpoint{3.711341in}{4.554791in}}%
\pgfpathlineto{\pgfqpoint{3.713265in}{4.554818in}}%
\pgfpathlineto{\pgfqpoint{3.715188in}{4.552372in}}%
\pgfpathlineto{\pgfqpoint{3.717112in}{4.553935in}}%
\pgfpathlineto{\pgfqpoint{3.719035in}{4.572224in}}%
\pgfpathlineto{\pgfqpoint{3.720958in}{4.579838in}}%
\pgfpathlineto{\pgfqpoint{3.722882in}{4.576435in}}%
\pgfpathlineto{\pgfqpoint{3.724805in}{4.582488in}}%
\pgfpathlineto{\pgfqpoint{3.726729in}{4.572134in}}%
\pgfpathlineto{\pgfqpoint{3.728652in}{4.571846in}}%
\pgfpathlineto{\pgfqpoint{3.730575in}{4.567539in}}%
\pgfpathlineto{\pgfqpoint{3.734422in}{4.581064in}}%
\pgfpathlineto{\pgfqpoint{3.736346in}{4.583151in}}%
\pgfpathlineto{\pgfqpoint{3.738269in}{4.581144in}}%
\pgfpathlineto{\pgfqpoint{3.740192in}{4.593530in}}%
\pgfpathlineto{\pgfqpoint{3.742116in}{4.590402in}}%
\pgfpathlineto{\pgfqpoint{3.745963in}{4.590961in}}%
\pgfpathlineto{\pgfqpoint{3.747886in}{4.585317in}}%
\pgfpathlineto{\pgfqpoint{3.749810in}{4.584675in}}%
\pgfpathlineto{\pgfqpoint{3.751733in}{4.593302in}}%
\pgfpathlineto{\pgfqpoint{3.753656in}{4.596631in}}%
\pgfpathlineto{\pgfqpoint{3.755580in}{4.590485in}}%
\pgfpathlineto{\pgfqpoint{3.757503in}{4.603773in}}%
\pgfpathlineto{\pgfqpoint{3.759427in}{4.607981in}}%
\pgfpathlineto{\pgfqpoint{3.761350in}{4.625356in}}%
\pgfpathlineto{\pgfqpoint{3.763273in}{4.630884in}}%
\pgfpathlineto{\pgfqpoint{3.765197in}{4.629589in}}%
\pgfpathlineto{\pgfqpoint{3.767120in}{4.633475in}}%
\pgfpathlineto{\pgfqpoint{3.769044in}{4.618989in}}%
\pgfpathlineto{\pgfqpoint{3.770967in}{4.617132in}}%
\pgfpathlineto{\pgfqpoint{3.772890in}{4.626155in}}%
\pgfpathlineto{\pgfqpoint{3.774814in}{4.627195in}}%
\pgfpathlineto{\pgfqpoint{3.776737in}{4.630393in}}%
\pgfpathlineto{\pgfqpoint{3.778661in}{4.617650in}}%
\pgfpathlineto{\pgfqpoint{3.780584in}{4.621781in}}%
\pgfpathlineto{\pgfqpoint{3.782508in}{4.620006in}}%
\pgfpathlineto{\pgfqpoint{3.784431in}{4.620818in}}%
\pgfpathlineto{\pgfqpoint{3.786354in}{4.617414in}}%
\pgfpathlineto{\pgfqpoint{3.788278in}{4.624185in}}%
\pgfpathlineto{\pgfqpoint{3.790201in}{4.626193in}}%
\pgfpathlineto{\pgfqpoint{3.792125in}{4.633328in}}%
\pgfpathlineto{\pgfqpoint{3.794048in}{4.624582in}}%
\pgfpathlineto{\pgfqpoint{3.795971in}{4.645973in}}%
\pgfpathlineto{\pgfqpoint{3.799818in}{4.643340in}}%
\pgfpathlineto{\pgfqpoint{3.801742in}{4.643708in}}%
\pgfpathlineto{\pgfqpoint{3.803665in}{4.649583in}}%
\pgfpathlineto{\pgfqpoint{3.805588in}{4.671531in}}%
\pgfpathlineto{\pgfqpoint{3.807512in}{4.675418in}}%
\pgfpathlineto{\pgfqpoint{3.809435in}{4.676411in}}%
\pgfpathlineto{\pgfqpoint{3.811359in}{4.669371in}}%
\pgfpathlineto{\pgfqpoint{3.813282in}{4.670725in}}%
\pgfpathlineto{\pgfqpoint{3.815205in}{4.693506in}}%
\pgfpathlineto{\pgfqpoint{3.817129in}{4.674788in}}%
\pgfpathlineto{\pgfqpoint{3.819052in}{4.668201in}}%
\pgfpathlineto{\pgfqpoint{3.820976in}{4.647280in}}%
\pgfpathlineto{\pgfqpoint{3.822899in}{4.652286in}}%
\pgfpathlineto{\pgfqpoint{3.824823in}{4.648671in}}%
\pgfpathlineto{\pgfqpoint{3.826746in}{4.650975in}}%
\pgfpathlineto{\pgfqpoint{3.830593in}{4.671728in}}%
\pgfpathlineto{\pgfqpoint{3.834440in}{4.665398in}}%
\pgfpathlineto{\pgfqpoint{3.836363in}{4.658552in}}%
\pgfpathlineto{\pgfqpoint{3.838286in}{4.665699in}}%
\pgfpathlineto{\pgfqpoint{3.840210in}{4.658493in}}%
\pgfpathlineto{\pgfqpoint{3.842133in}{4.657802in}}%
\pgfpathlineto{\pgfqpoint{3.844057in}{4.666656in}}%
\pgfpathlineto{\pgfqpoint{3.847903in}{4.640252in}}%
\pgfpathlineto{\pgfqpoint{3.849827in}{4.655218in}}%
\pgfpathlineto{\pgfqpoint{3.851750in}{4.658554in}}%
\pgfpathlineto{\pgfqpoint{3.853674in}{4.666667in}}%
\pgfpathlineto{\pgfqpoint{3.855597in}{4.662265in}}%
\pgfpathlineto{\pgfqpoint{3.857521in}{4.664360in}}%
\pgfpathlineto{\pgfqpoint{3.861367in}{4.682654in}}%
\pgfpathlineto{\pgfqpoint{3.863291in}{4.682324in}}%
\pgfpathlineto{\pgfqpoint{3.865214in}{4.685248in}}%
\pgfpathlineto{\pgfqpoint{3.867138in}{4.680872in}}%
\pgfpathlineto{\pgfqpoint{3.869061in}{4.680541in}}%
\pgfpathlineto{\pgfqpoint{3.870984in}{4.676678in}}%
\pgfpathlineto{\pgfqpoint{3.872908in}{4.685686in}}%
\pgfpathlineto{\pgfqpoint{3.874831in}{4.700920in}}%
\pgfpathlineto{\pgfqpoint{3.876755in}{4.702827in}}%
\pgfpathlineto{\pgfqpoint{3.880601in}{4.724390in}}%
\pgfpathlineto{\pgfqpoint{3.882525in}{4.718281in}}%
\pgfpathlineto{\pgfqpoint{3.884448in}{4.719533in}}%
\pgfpathlineto{\pgfqpoint{3.888295in}{4.707949in}}%
\pgfpathlineto{\pgfqpoint{3.890219in}{4.690434in}}%
\pgfpathlineto{\pgfqpoint{3.894065in}{4.695264in}}%
\pgfpathlineto{\pgfqpoint{3.895989in}{4.684328in}}%
\pgfpathlineto{\pgfqpoint{3.897912in}{4.688479in}}%
\pgfpathlineto{\pgfqpoint{3.899836in}{4.664931in}}%
\pgfpathlineto{\pgfqpoint{3.901759in}{4.677606in}}%
\pgfpathlineto{\pgfqpoint{3.903682in}{4.676638in}}%
\pgfpathlineto{\pgfqpoint{3.905606in}{4.673712in}}%
\pgfpathlineto{\pgfqpoint{3.907529in}{4.686162in}}%
\pgfpathlineto{\pgfqpoint{3.909453in}{4.690248in}}%
\pgfpathlineto{\pgfqpoint{3.911376in}{4.684469in}}%
\pgfpathlineto{\pgfqpoint{3.913299in}{4.690761in}}%
\pgfpathlineto{\pgfqpoint{3.917146in}{4.691539in}}%
\pgfpathlineto{\pgfqpoint{3.920993in}{4.676104in}}%
\pgfpathlineto{\pgfqpoint{3.924840in}{4.697782in}}%
\pgfpathlineto{\pgfqpoint{3.926763in}{4.702252in}}%
\pgfpathlineto{\pgfqpoint{3.930610in}{4.728874in}}%
\pgfpathlineto{\pgfqpoint{3.932534in}{4.732052in}}%
\pgfpathlineto{\pgfqpoint{3.936380in}{4.726198in}}%
\pgfpathlineto{\pgfqpoint{3.938304in}{4.742445in}}%
\pgfpathlineto{\pgfqpoint{3.940227in}{4.743298in}}%
\pgfpathlineto{\pgfqpoint{3.944074in}{4.750586in}}%
\pgfpathlineto{\pgfqpoint{3.945997in}{4.746045in}}%
\pgfpathlineto{\pgfqpoint{3.947921in}{4.764282in}}%
\pgfpathlineto{\pgfqpoint{3.951768in}{4.774789in}}%
\pgfpathlineto{\pgfqpoint{3.953691in}{4.775125in}}%
\pgfpathlineto{\pgfqpoint{3.965232in}{4.826830in}}%
\pgfpathlineto{\pgfqpoint{3.967155in}{4.821361in}}%
\pgfpathlineto{\pgfqpoint{3.969078in}{4.807009in}}%
\pgfpathlineto{\pgfqpoint{3.971002in}{4.816712in}}%
\pgfpathlineto{\pgfqpoint{3.974849in}{4.804249in}}%
\pgfpathlineto{\pgfqpoint{3.978695in}{4.832250in}}%
\pgfpathlineto{\pgfqpoint{3.980619in}{4.835806in}}%
\pgfpathlineto{\pgfqpoint{3.982542in}{4.824358in}}%
\pgfpathlineto{\pgfqpoint{3.984466in}{4.841490in}}%
\pgfpathlineto{\pgfqpoint{3.986389in}{4.846491in}}%
\pgfpathlineto{\pgfqpoint{3.988312in}{4.858263in}}%
\pgfpathlineto{\pgfqpoint{3.990236in}{4.848342in}}%
\pgfpathlineto{\pgfqpoint{3.992159in}{4.860308in}}%
\pgfpathlineto{\pgfqpoint{3.994083in}{4.861534in}}%
\pgfpathlineto{\pgfqpoint{3.996006in}{4.868251in}}%
\pgfpathlineto{\pgfqpoint{3.997930in}{4.887963in}}%
\pgfpathlineto{\pgfqpoint{3.999853in}{4.886749in}}%
\pgfpathlineto{\pgfqpoint{4.001776in}{4.876462in}}%
\pgfpathlineto{\pgfqpoint{4.003700in}{4.881134in}}%
\pgfpathlineto{\pgfqpoint{4.005623in}{4.889522in}}%
\pgfpathlineto{\pgfqpoint{4.009470in}{4.882065in}}%
\pgfpathlineto{\pgfqpoint{4.013317in}{4.908939in}}%
\pgfpathlineto{\pgfqpoint{4.015240in}{4.908384in}}%
\pgfpathlineto{\pgfqpoint{4.017164in}{4.918583in}}%
\pgfpathlineto{\pgfqpoint{4.019087in}{4.919169in}}%
\pgfpathlineto{\pgfqpoint{4.021010in}{4.925896in}}%
\pgfpathlineto{\pgfqpoint{4.022934in}{4.925340in}}%
\pgfpathlineto{\pgfqpoint{4.024857in}{4.915345in}}%
\pgfpathlineto{\pgfqpoint{4.026781in}{4.913695in}}%
\pgfpathlineto{\pgfqpoint{4.028704in}{4.921789in}}%
\pgfpathlineto{\pgfqpoint{4.030628in}{4.921829in}}%
\pgfpathlineto{\pgfqpoint{4.034474in}{4.918195in}}%
\pgfpathlineto{\pgfqpoint{4.036398in}{4.923192in}}%
\pgfpathlineto{\pgfqpoint{4.038321in}{4.936468in}}%
\pgfpathlineto{\pgfqpoint{4.040245in}{4.931049in}}%
\pgfpathlineto{\pgfqpoint{4.042168in}{4.934420in}}%
\pgfpathlineto{\pgfqpoint{4.044091in}{4.933061in}}%
\pgfpathlineto{\pgfqpoint{4.046015in}{4.927713in}}%
\pgfpathlineto{\pgfqpoint{4.047938in}{4.929268in}}%
\pgfpathlineto{\pgfqpoint{4.049862in}{4.929083in}}%
\pgfpathlineto{\pgfqpoint{4.051785in}{4.931834in}}%
\pgfpathlineto{\pgfqpoint{4.053708in}{4.925071in}}%
\pgfpathlineto{\pgfqpoint{4.055632in}{4.927080in}}%
\pgfpathlineto{\pgfqpoint{4.057555in}{4.936267in}}%
\pgfpathlineto{\pgfqpoint{4.059479in}{4.939960in}}%
\pgfpathlineto{\pgfqpoint{4.061402in}{4.946458in}}%
\pgfpathlineto{\pgfqpoint{4.063326in}{4.948371in}}%
\pgfpathlineto{\pgfqpoint{4.065249in}{4.936117in}}%
\pgfpathlineto{\pgfqpoint{4.067172in}{4.942567in}}%
\pgfpathlineto{\pgfqpoint{4.069096in}{4.915689in}}%
\pgfpathlineto{\pgfqpoint{4.071019in}{4.923038in}}%
\pgfpathlineto{\pgfqpoint{4.074866in}{4.902446in}}%
\pgfpathlineto{\pgfqpoint{4.076789in}{4.916946in}}%
\pgfpathlineto{\pgfqpoint{4.078713in}{4.917301in}}%
\pgfpathlineto{\pgfqpoint{4.080636in}{4.919858in}}%
\pgfpathlineto{\pgfqpoint{4.082560in}{4.913028in}}%
\pgfpathlineto{\pgfqpoint{4.084483in}{4.893300in}}%
\pgfpathlineto{\pgfqpoint{4.086406in}{4.901328in}}%
\pgfpathlineto{\pgfqpoint{4.088330in}{4.893096in}}%
\pgfpathlineto{\pgfqpoint{4.090253in}{4.900735in}}%
\pgfpathlineto{\pgfqpoint{4.092177in}{4.902057in}}%
\pgfpathlineto{\pgfqpoint{4.094100in}{4.904911in}}%
\pgfpathlineto{\pgfqpoint{4.096024in}{4.901882in}}%
\pgfpathlineto{\pgfqpoint{4.097947in}{4.914517in}}%
\pgfpathlineto{\pgfqpoint{4.099870in}{4.913719in}}%
\pgfpathlineto{\pgfqpoint{4.103717in}{4.887138in}}%
\pgfpathlineto{\pgfqpoint{4.105641in}{4.904962in}}%
\pgfpathlineto{\pgfqpoint{4.107564in}{4.884134in}}%
\pgfpathlineto{\pgfqpoint{4.109487in}{4.886259in}}%
\pgfpathlineto{\pgfqpoint{4.111411in}{4.905243in}}%
\pgfpathlineto{\pgfqpoint{4.113334in}{4.904841in}}%
\pgfpathlineto{\pgfqpoint{4.115258in}{4.900289in}}%
\pgfpathlineto{\pgfqpoint{4.117181in}{4.912023in}}%
\pgfpathlineto{\pgfqpoint{4.119104in}{4.915130in}}%
\pgfpathlineto{\pgfqpoint{4.121028in}{4.908262in}}%
\pgfpathlineto{\pgfqpoint{4.122951in}{4.889812in}}%
\pgfpathlineto{\pgfqpoint{4.124875in}{4.901846in}}%
\pgfpathlineto{\pgfqpoint{4.126798in}{4.894254in}}%
\pgfpathlineto{\pgfqpoint{4.130645in}{4.902233in}}%
\pgfpathlineto{\pgfqpoint{4.134492in}{4.920392in}}%
\pgfpathlineto{\pgfqpoint{4.136415in}{4.916582in}}%
\pgfpathlineto{\pgfqpoint{4.138339in}{4.921663in}}%
\pgfpathlineto{\pgfqpoint{4.140262in}{4.920840in}}%
\pgfpathlineto{\pgfqpoint{4.142185in}{4.916703in}}%
\pgfpathlineto{\pgfqpoint{4.144109in}{4.927539in}}%
\pgfpathlineto{\pgfqpoint{4.146032in}{4.921009in}}%
\pgfpathlineto{\pgfqpoint{4.147956in}{4.928511in}}%
\pgfpathlineto{\pgfqpoint{4.149879in}{4.920913in}}%
\pgfpathlineto{\pgfqpoint{4.151802in}{4.921616in}}%
\pgfpathlineto{\pgfqpoint{4.153726in}{4.928338in}}%
\pgfpathlineto{\pgfqpoint{4.155649in}{4.922879in}}%
\pgfpathlineto{\pgfqpoint{4.157573in}{4.904679in}}%
\pgfpathlineto{\pgfqpoint{4.161419in}{4.907486in}}%
\pgfpathlineto{\pgfqpoint{4.163343in}{4.895864in}}%
\pgfpathlineto{\pgfqpoint{4.165266in}{4.895857in}}%
\pgfpathlineto{\pgfqpoint{4.167190in}{4.883672in}}%
\pgfpathlineto{\pgfqpoint{4.169113in}{4.894848in}}%
\pgfpathlineto{\pgfqpoint{4.171037in}{4.895187in}}%
\pgfpathlineto{\pgfqpoint{4.174883in}{4.908735in}}%
\pgfpathlineto{\pgfqpoint{4.178730in}{4.947677in}}%
\pgfpathlineto{\pgfqpoint{4.182577in}{4.934860in}}%
\pgfpathlineto{\pgfqpoint{4.184500in}{4.933647in}}%
\pgfpathlineto{\pgfqpoint{4.186424in}{4.935277in}}%
\pgfpathlineto{\pgfqpoint{4.188347in}{4.952737in}}%
\pgfpathlineto{\pgfqpoint{4.190271in}{4.945048in}}%
\pgfpathlineto{\pgfqpoint{4.192194in}{4.928815in}}%
\pgfpathlineto{\pgfqpoint{4.194117in}{4.926308in}}%
\pgfpathlineto{\pgfqpoint{4.196041in}{4.938388in}}%
\pgfpathlineto{\pgfqpoint{4.197964in}{4.922336in}}%
\pgfpathlineto{\pgfqpoint{4.199888in}{4.921825in}}%
\pgfpathlineto{\pgfqpoint{4.203735in}{4.898141in}}%
\pgfpathlineto{\pgfqpoint{4.205658in}{4.904147in}}%
\pgfpathlineto{\pgfqpoint{4.207581in}{4.916664in}}%
\pgfpathlineto{\pgfqpoint{4.209505in}{4.908382in}}%
\pgfpathlineto{\pgfqpoint{4.211428in}{4.913274in}}%
\pgfpathlineto{\pgfqpoint{4.213352in}{4.905833in}}%
\pgfpathlineto{\pgfqpoint{4.221045in}{4.927926in}}%
\pgfpathlineto{\pgfqpoint{4.222969in}{4.917611in}}%
\pgfpathlineto{\pgfqpoint{4.228739in}{4.947861in}}%
\pgfpathlineto{\pgfqpoint{4.230662in}{4.945542in}}%
\pgfpathlineto{\pgfqpoint{4.232586in}{4.948348in}}%
\pgfpathlineto{\pgfqpoint{4.234509in}{4.958344in}}%
\pgfpathlineto{\pgfqpoint{4.236433in}{4.955629in}}%
\pgfpathlineto{\pgfqpoint{4.238356in}{4.966150in}}%
\pgfpathlineto{\pgfqpoint{4.240279in}{4.965061in}}%
\pgfpathlineto{\pgfqpoint{4.242203in}{4.968222in}}%
\pgfpathlineto{\pgfqpoint{4.244126in}{4.955427in}}%
\pgfpathlineto{\pgfqpoint{4.246050in}{4.959416in}}%
\pgfpathlineto{\pgfqpoint{4.247973in}{4.948158in}}%
\pgfpathlineto{\pgfqpoint{4.249896in}{4.952126in}}%
\pgfpathlineto{\pgfqpoint{4.253743in}{4.976914in}}%
\pgfpathlineto{\pgfqpoint{4.255667in}{4.990837in}}%
\pgfpathlineto{\pgfqpoint{4.259513in}{4.975625in}}%
\pgfpathlineto{\pgfqpoint{4.261437in}{4.978685in}}%
\pgfpathlineto{\pgfqpoint{4.263360in}{4.974845in}}%
\pgfpathlineto{\pgfqpoint{4.265284in}{4.962987in}}%
\pgfpathlineto{\pgfqpoint{4.267207in}{4.964388in}}%
\pgfpathlineto{\pgfqpoint{4.269130in}{4.957472in}}%
\pgfpathlineto{\pgfqpoint{4.271054in}{4.982696in}}%
\pgfpathlineto{\pgfqpoint{4.272977in}{4.992876in}}%
\pgfpathlineto{\pgfqpoint{4.274901in}{4.979157in}}%
\pgfpathlineto{\pgfqpoint{4.276824in}{4.987647in}}%
\pgfpathlineto{\pgfqpoint{4.278748in}{4.970281in}}%
\pgfpathlineto{\pgfqpoint{4.280671in}{4.968395in}}%
\pgfpathlineto{\pgfqpoint{4.282594in}{4.972381in}}%
\pgfpathlineto{\pgfqpoint{4.284518in}{4.973013in}}%
\pgfpathlineto{\pgfqpoint{4.286441in}{4.970663in}}%
\pgfpathlineto{\pgfqpoint{4.288365in}{4.978688in}}%
\pgfpathlineto{\pgfqpoint{4.290288in}{4.976954in}}%
\pgfpathlineto{\pgfqpoint{4.292211in}{4.980494in}}%
\pgfpathlineto{\pgfqpoint{4.294135in}{4.974205in}}%
\pgfpathlineto{\pgfqpoint{4.297982in}{4.931360in}}%
\pgfpathlineto{\pgfqpoint{4.299905in}{4.930966in}}%
\pgfpathlineto{\pgfqpoint{4.301828in}{4.938746in}}%
\pgfpathlineto{\pgfqpoint{4.303752in}{4.923035in}}%
\pgfpathlineto{\pgfqpoint{4.305675in}{4.939050in}}%
\pgfpathlineto{\pgfqpoint{4.307599in}{4.944682in}}%
\pgfpathlineto{\pgfqpoint{4.309522in}{4.934601in}}%
\pgfpathlineto{\pgfqpoint{4.311446in}{4.961970in}}%
\pgfpathlineto{\pgfqpoint{4.313369in}{4.954343in}}%
\pgfpathlineto{\pgfqpoint{4.315292in}{4.960472in}}%
\pgfpathlineto{\pgfqpoint{4.317216in}{4.947500in}}%
\pgfpathlineto{\pgfqpoint{4.319139in}{4.944249in}}%
\pgfpathlineto{\pgfqpoint{4.321063in}{4.945011in}}%
\pgfpathlineto{\pgfqpoint{4.322986in}{4.954819in}}%
\pgfpathlineto{\pgfqpoint{4.324909in}{4.951860in}}%
\pgfpathlineto{\pgfqpoint{4.328756in}{4.937445in}}%
\pgfpathlineto{\pgfqpoint{4.330680in}{4.941213in}}%
\pgfpathlineto{\pgfqpoint{4.336450in}{4.959863in}}%
\pgfpathlineto{\pgfqpoint{4.338373in}{4.960844in}}%
\pgfpathlineto{\pgfqpoint{4.340297in}{4.954581in}}%
\pgfpathlineto{\pgfqpoint{4.342220in}{4.960407in}}%
\pgfpathlineto{\pgfqpoint{4.344144in}{4.940819in}}%
\pgfpathlineto{\pgfqpoint{4.347990in}{4.949716in}}%
\pgfpathlineto{\pgfqpoint{4.349914in}{4.964442in}}%
\pgfpathlineto{\pgfqpoint{4.351837in}{4.957169in}}%
\pgfpathlineto{\pgfqpoint{4.353761in}{4.954243in}}%
\pgfpathlineto{\pgfqpoint{4.355684in}{4.954928in}}%
\pgfpathlineto{\pgfqpoint{4.357607in}{4.936540in}}%
\pgfpathlineto{\pgfqpoint{4.359531in}{4.933190in}}%
\pgfpathlineto{\pgfqpoint{4.363378in}{4.949901in}}%
\pgfpathlineto{\pgfqpoint{4.365301in}{4.968967in}}%
\pgfpathlineto{\pgfqpoint{4.367224in}{4.971563in}}%
\pgfpathlineto{\pgfqpoint{4.369148in}{4.979199in}}%
\pgfpathlineto{\pgfqpoint{4.371071in}{4.972696in}}%
\pgfpathlineto{\pgfqpoint{4.372995in}{4.957425in}}%
\pgfpathlineto{\pgfqpoint{4.374918in}{4.932523in}}%
\pgfpathlineto{\pgfqpoint{4.376842in}{4.931090in}}%
\pgfpathlineto{\pgfqpoint{4.378765in}{4.924188in}}%
\pgfpathlineto{\pgfqpoint{4.380688in}{4.925589in}}%
\pgfpathlineto{\pgfqpoint{4.382612in}{4.958713in}}%
\pgfpathlineto{\pgfqpoint{4.384535in}{4.937126in}}%
\pgfpathlineto{\pgfqpoint{4.386459in}{4.935057in}}%
\pgfpathlineto{\pgfqpoint{4.388382in}{4.935953in}}%
\pgfpathlineto{\pgfqpoint{4.392229in}{4.952795in}}%
\pgfpathlineto{\pgfqpoint{4.394152in}{4.955445in}}%
\pgfpathlineto{\pgfqpoint{4.399922in}{4.917770in}}%
\pgfpathlineto{\pgfqpoint{4.403769in}{4.921465in}}%
\pgfpathlineto{\pgfqpoint{4.405693in}{4.921646in}}%
\pgfpathlineto{\pgfqpoint{4.407616in}{4.906325in}}%
\pgfpathlineto{\pgfqpoint{4.409539in}{4.919881in}}%
\pgfpathlineto{\pgfqpoint{4.413386in}{4.908747in}}%
\pgfpathlineto{\pgfqpoint{4.415310in}{4.909393in}}%
\pgfpathlineto{\pgfqpoint{4.417233in}{4.898546in}}%
\pgfpathlineto{\pgfqpoint{4.419157in}{4.901056in}}%
\pgfpathlineto{\pgfqpoint{4.426850in}{4.944272in}}%
\pgfpathlineto{\pgfqpoint{4.428774in}{4.948124in}}%
\pgfpathlineto{\pgfqpoint{4.430697in}{4.967283in}}%
\pgfpathlineto{\pgfqpoint{4.432620in}{4.964360in}}%
\pgfpathlineto{\pgfqpoint{4.434544in}{4.953941in}}%
\pgfpathlineto{\pgfqpoint{4.436467in}{4.954088in}}%
\pgfpathlineto{\pgfqpoint{4.438391in}{4.952508in}}%
\pgfpathlineto{\pgfqpoint{4.440314in}{4.967737in}}%
\pgfpathlineto{\pgfqpoint{4.446084in}{4.927351in}}%
\pgfpathlineto{\pgfqpoint{4.448008in}{4.930282in}}%
\pgfpathlineto{\pgfqpoint{4.449931in}{4.914030in}}%
\pgfpathlineto{\pgfqpoint{4.451855in}{4.911033in}}%
\pgfpathlineto{\pgfqpoint{4.453778in}{4.902577in}}%
\pgfpathlineto{\pgfqpoint{4.455701in}{4.918250in}}%
\pgfpathlineto{\pgfqpoint{4.459548in}{4.911478in}}%
\pgfpathlineto{\pgfqpoint{4.461472in}{4.929823in}}%
\pgfpathlineto{\pgfqpoint{4.463395in}{4.933346in}}%
\pgfpathlineto{\pgfqpoint{4.465318in}{4.929396in}}%
\pgfpathlineto{\pgfqpoint{4.467242in}{4.947671in}}%
\pgfpathlineto{\pgfqpoint{4.469165in}{4.946830in}}%
\pgfpathlineto{\pgfqpoint{4.474935in}{4.966718in}}%
\pgfpathlineto{\pgfqpoint{4.478782in}{4.957727in}}%
\pgfpathlineto{\pgfqpoint{4.480706in}{4.958596in}}%
\pgfpathlineto{\pgfqpoint{4.482629in}{4.954239in}}%
\pgfpathlineto{\pgfqpoint{4.484553in}{4.969091in}}%
\pgfpathlineto{\pgfqpoint{4.486476in}{4.943811in}}%
\pgfpathlineto{\pgfqpoint{4.488399in}{4.954559in}}%
\pgfpathlineto{\pgfqpoint{4.490323in}{4.958174in}}%
\pgfpathlineto{\pgfqpoint{4.492246in}{4.951313in}}%
\pgfpathlineto{\pgfqpoint{4.494170in}{4.963932in}}%
\pgfpathlineto{\pgfqpoint{4.496093in}{4.956293in}}%
\pgfpathlineto{\pgfqpoint{4.498016in}{4.967671in}}%
\pgfpathlineto{\pgfqpoint{4.499940in}{4.989203in}}%
\pgfpathlineto{\pgfqpoint{4.501863in}{4.969029in}}%
\pgfpathlineto{\pgfqpoint{4.503787in}{4.973227in}}%
\pgfpathlineto{\pgfqpoint{4.505710in}{4.966577in}}%
\pgfpathlineto{\pgfqpoint{4.507633in}{4.977308in}}%
\pgfpathlineto{\pgfqpoint{4.509557in}{4.974264in}}%
\pgfpathlineto{\pgfqpoint{4.513404in}{4.954809in}}%
\pgfpathlineto{\pgfqpoint{4.515327in}{4.965247in}}%
\pgfpathlineto{\pgfqpoint{4.517251in}{4.964303in}}%
\pgfpathlineto{\pgfqpoint{4.519174in}{4.966817in}}%
\pgfpathlineto{\pgfqpoint{4.521097in}{4.983504in}}%
\pgfpathlineto{\pgfqpoint{4.523021in}{4.983410in}}%
\pgfpathlineto{\pgfqpoint{4.524944in}{4.981004in}}%
\pgfpathlineto{\pgfqpoint{4.526868in}{4.987525in}}%
\pgfpathlineto{\pgfqpoint{4.528791in}{4.980536in}}%
\pgfpathlineto{\pgfqpoint{4.530714in}{4.985040in}}%
\pgfpathlineto{\pgfqpoint{4.536485in}{4.956612in}}%
\pgfpathlineto{\pgfqpoint{4.540331in}{4.976390in}}%
\pgfpathlineto{\pgfqpoint{4.542255in}{4.972193in}}%
\pgfpathlineto{\pgfqpoint{4.544178in}{4.976733in}}%
\pgfpathlineto{\pgfqpoint{4.546102in}{4.969845in}}%
\pgfpathlineto{\pgfqpoint{4.548025in}{4.972907in}}%
\pgfpathlineto{\pgfqpoint{4.549949in}{4.971355in}}%
\pgfpathlineto{\pgfqpoint{4.551872in}{4.972453in}}%
\pgfpathlineto{\pgfqpoint{4.553795in}{4.965264in}}%
\pgfpathlineto{\pgfqpoint{4.557642in}{4.976106in}}%
\pgfpathlineto{\pgfqpoint{4.559566in}{4.976262in}}%
\pgfpathlineto{\pgfqpoint{4.563412in}{4.952704in}}%
\pgfpathlineto{\pgfqpoint{4.567259in}{4.970844in}}%
\pgfpathlineto{\pgfqpoint{4.569183in}{4.965195in}}%
\pgfpathlineto{\pgfqpoint{4.571106in}{4.966918in}}%
\pgfpathlineto{\pgfqpoint{4.573029in}{4.966968in}}%
\pgfpathlineto{\pgfqpoint{4.574953in}{4.951888in}}%
\pgfpathlineto{\pgfqpoint{4.576876in}{4.967747in}}%
\pgfpathlineto{\pgfqpoint{4.578800in}{4.968687in}}%
\pgfpathlineto{\pgfqpoint{4.580723in}{4.979448in}}%
\pgfpathlineto{\pgfqpoint{4.582646in}{4.981991in}}%
\pgfpathlineto{\pgfqpoint{4.584570in}{4.999971in}}%
\pgfpathlineto{\pgfqpoint{4.588417in}{4.981516in}}%
\pgfpathlineto{\pgfqpoint{4.590340in}{4.984972in}}%
\pgfpathlineto{\pgfqpoint{4.592264in}{4.992448in}}%
\pgfpathlineto{\pgfqpoint{4.594187in}{4.986472in}}%
\pgfpathlineto{\pgfqpoint{4.596110in}{4.975788in}}%
\pgfpathlineto{\pgfqpoint{4.599957in}{4.974134in}}%
\pgfpathlineto{\pgfqpoint{4.601881in}{4.955953in}}%
\pgfpathlineto{\pgfqpoint{4.603804in}{4.949522in}}%
\pgfpathlineto{\pgfqpoint{4.609574in}{4.975057in}}%
\pgfpathlineto{\pgfqpoint{4.613421in}{4.957209in}}%
\pgfpathlineto{\pgfqpoint{4.615344in}{4.971214in}}%
\pgfpathlineto{\pgfqpoint{4.617268in}{4.975897in}}%
\pgfpathlineto{\pgfqpoint{4.624962in}{5.012548in}}%
\pgfpathlineto{\pgfqpoint{4.626885in}{5.012663in}}%
\pgfpathlineto{\pgfqpoint{4.628808in}{5.004424in}}%
\pgfpathlineto{\pgfqpoint{4.630732in}{4.981357in}}%
\pgfpathlineto{\pgfqpoint{4.632655in}{4.976232in}}%
\pgfpathlineto{\pgfqpoint{4.640349in}{5.027357in}}%
\pgfpathlineto{\pgfqpoint{4.642272in}{5.025923in}}%
\pgfpathlineto{\pgfqpoint{4.644196in}{5.030340in}}%
\pgfpathlineto{\pgfqpoint{4.646119in}{5.025433in}}%
\pgfpathlineto{\pgfqpoint{4.648042in}{5.033598in}}%
\pgfpathlineto{\pgfqpoint{4.649966in}{5.035875in}}%
\pgfpathlineto{\pgfqpoint{4.651889in}{5.022045in}}%
\pgfpathlineto{\pgfqpoint{4.653813in}{5.023363in}}%
\pgfpathlineto{\pgfqpoint{4.655736in}{5.032464in}}%
\pgfpathlineto{\pgfqpoint{4.657660in}{5.030908in}}%
\pgfpathlineto{\pgfqpoint{4.661506in}{5.042959in}}%
\pgfpathlineto{\pgfqpoint{4.663430in}{5.036845in}}%
\pgfpathlineto{\pgfqpoint{4.665353in}{5.025486in}}%
\pgfpathlineto{\pgfqpoint{4.667277in}{5.026666in}}%
\pgfpathlineto{\pgfqpoint{4.671123in}{5.045466in}}%
\pgfpathlineto{\pgfqpoint{4.673047in}{5.037577in}}%
\pgfpathlineto{\pgfqpoint{4.674970in}{5.039896in}}%
\pgfpathlineto{\pgfqpoint{4.676894in}{5.031352in}}%
\pgfpathlineto{\pgfqpoint{4.678817in}{5.032554in}}%
\pgfpathlineto{\pgfqpoint{4.680740in}{5.038774in}}%
\pgfpathlineto{\pgfqpoint{4.682664in}{5.051388in}}%
\pgfpathlineto{\pgfqpoint{4.684587in}{5.046405in}}%
\pgfpathlineto{\pgfqpoint{4.686511in}{5.056169in}}%
\pgfpathlineto{\pgfqpoint{4.690358in}{5.042438in}}%
\pgfpathlineto{\pgfqpoint{4.694204in}{5.075161in}}%
\pgfpathlineto{\pgfqpoint{4.696128in}{5.081381in}}%
\pgfpathlineto{\pgfqpoint{4.698051in}{5.081210in}}%
\pgfpathlineto{\pgfqpoint{4.699975in}{5.077481in}}%
\pgfpathlineto{\pgfqpoint{4.701898in}{5.081342in}}%
\pgfpathlineto{\pgfqpoint{4.705745in}{5.095975in}}%
\pgfpathlineto{\pgfqpoint{4.709592in}{5.112106in}}%
\pgfpathlineto{\pgfqpoint{4.711515in}{5.108489in}}%
\pgfpathlineto{\pgfqpoint{4.713438in}{5.111983in}}%
\pgfpathlineto{\pgfqpoint{4.715362in}{5.096152in}}%
\pgfpathlineto{\pgfqpoint{4.717285in}{5.102019in}}%
\pgfpathlineto{\pgfqpoint{4.719209in}{5.102286in}}%
\pgfpathlineto{\pgfqpoint{4.721132in}{5.108619in}}%
\pgfpathlineto{\pgfqpoint{4.723055in}{5.093395in}}%
\pgfpathlineto{\pgfqpoint{4.724979in}{5.091276in}}%
\pgfpathlineto{\pgfqpoint{4.726902in}{5.078864in}}%
\pgfpathlineto{\pgfqpoint{4.728826in}{5.074579in}}%
\pgfpathlineto{\pgfqpoint{4.730749in}{5.066845in}}%
\pgfpathlineto{\pgfqpoint{4.734596in}{5.067029in}}%
\pgfpathlineto{\pgfqpoint{4.738443in}{5.078199in}}%
\pgfpathlineto{\pgfqpoint{4.740366in}{5.066825in}}%
\pgfpathlineto{\pgfqpoint{4.744213in}{5.091369in}}%
\pgfpathlineto{\pgfqpoint{4.746136in}{5.091454in}}%
\pgfpathlineto{\pgfqpoint{4.748060in}{5.096273in}}%
\pgfpathlineto{\pgfqpoint{4.749983in}{5.104493in}}%
\pgfpathlineto{\pgfqpoint{4.753830in}{5.128452in}}%
\pgfpathlineto{\pgfqpoint{4.757677in}{5.146624in}}%
\pgfpathlineto{\pgfqpoint{4.759600in}{5.162923in}}%
\pgfpathlineto{\pgfqpoint{4.761524in}{5.160305in}}%
\pgfpathlineto{\pgfqpoint{4.763447in}{5.168352in}}%
\pgfpathlineto{\pgfqpoint{4.765371in}{5.157463in}}%
\pgfpathlineto{\pgfqpoint{4.767294in}{5.157064in}}%
\pgfpathlineto{\pgfqpoint{4.769217in}{5.153072in}}%
\pgfpathlineto{\pgfqpoint{4.771141in}{5.171252in}}%
\pgfpathlineto{\pgfqpoint{4.773064in}{5.153059in}}%
\pgfpathlineto{\pgfqpoint{4.774988in}{5.155084in}}%
\pgfpathlineto{\pgfqpoint{4.776911in}{5.153494in}}%
\pgfpathlineto{\pgfqpoint{4.778834in}{5.140535in}}%
\pgfpathlineto{\pgfqpoint{4.784605in}{5.171140in}}%
\pgfpathlineto{\pgfqpoint{4.788451in}{5.156325in}}%
\pgfpathlineto{\pgfqpoint{4.790375in}{5.158147in}}%
\pgfpathlineto{\pgfqpoint{4.792298in}{5.154017in}}%
\pgfpathlineto{\pgfqpoint{4.796145in}{5.140000in}}%
\pgfpathlineto{\pgfqpoint{4.798069in}{5.135298in}}%
\pgfpathlineto{\pgfqpoint{4.799992in}{5.118377in}}%
\pgfpathlineto{\pgfqpoint{4.801915in}{5.119350in}}%
\pgfpathlineto{\pgfqpoint{4.803839in}{5.126025in}}%
\pgfpathlineto{\pgfqpoint{4.805762in}{5.140485in}}%
\pgfpathlineto{\pgfqpoint{4.809609in}{5.130344in}}%
\pgfpathlineto{\pgfqpoint{4.811532in}{5.137132in}}%
\pgfpathlineto{\pgfqpoint{4.815379in}{5.160456in}}%
\pgfpathlineto{\pgfqpoint{4.817303in}{5.153632in}}%
\pgfpathlineto{\pgfqpoint{4.819226in}{5.151731in}}%
\pgfpathlineto{\pgfqpoint{4.821149in}{5.143144in}}%
\pgfpathlineto{\pgfqpoint{4.823073in}{5.159690in}}%
\pgfpathlineto{\pgfqpoint{4.824996in}{5.162285in}}%
\pgfpathlineto{\pgfqpoint{4.826920in}{5.143643in}}%
\pgfpathlineto{\pgfqpoint{4.830767in}{5.177128in}}%
\pgfpathlineto{\pgfqpoint{4.832690in}{5.170496in}}%
\pgfpathlineto{\pgfqpoint{4.836537in}{5.146487in}}%
\pgfpathlineto{\pgfqpoint{4.840384in}{5.157759in}}%
\pgfpathlineto{\pgfqpoint{4.848077in}{5.158489in}}%
\pgfpathlineto{\pgfqpoint{4.850001in}{5.165364in}}%
\pgfpathlineto{\pgfqpoint{4.851924in}{5.153510in}}%
\pgfpathlineto{\pgfqpoint{4.857694in}{5.166259in}}%
\pgfpathlineto{\pgfqpoint{4.859618in}{5.162969in}}%
\pgfpathlineto{\pgfqpoint{4.861541in}{5.184545in}}%
\pgfpathlineto{\pgfqpoint{4.863464in}{5.176345in}}%
\pgfpathlineto{\pgfqpoint{4.865388in}{5.174075in}}%
\pgfpathlineto{\pgfqpoint{4.867311in}{5.168356in}}%
\pgfpathlineto{\pgfqpoint{4.869235in}{5.147647in}}%
\pgfpathlineto{\pgfqpoint{4.871158in}{5.150303in}}%
\pgfpathlineto{\pgfqpoint{4.873082in}{5.134330in}}%
\pgfpathlineto{\pgfqpoint{4.876928in}{5.153739in}}%
\pgfpathlineto{\pgfqpoint{4.880775in}{5.136905in}}%
\pgfpathlineto{\pgfqpoint{4.882699in}{5.140551in}}%
\pgfpathlineto{\pgfqpoint{4.884622in}{5.128035in}}%
\pgfpathlineto{\pgfqpoint{4.886545in}{5.131310in}}%
\pgfpathlineto{\pgfqpoint{4.888469in}{5.138449in}}%
\pgfpathlineto{\pgfqpoint{4.890392in}{5.112788in}}%
\pgfpathlineto{\pgfqpoint{4.892316in}{5.130630in}}%
\pgfpathlineto{\pgfqpoint{4.894239in}{5.120941in}}%
\pgfpathlineto{\pgfqpoint{4.896162in}{5.117957in}}%
\pgfpathlineto{\pgfqpoint{4.900009in}{5.108982in}}%
\pgfpathlineto{\pgfqpoint{4.901933in}{5.091016in}}%
\pgfpathlineto{\pgfqpoint{4.903856in}{5.098232in}}%
\pgfpathlineto{\pgfqpoint{4.909626in}{5.079391in}}%
\pgfpathlineto{\pgfqpoint{4.911550in}{5.074687in}}%
\pgfpathlineto{\pgfqpoint{4.913473in}{5.073201in}}%
\pgfpathlineto{\pgfqpoint{4.915397in}{5.062128in}}%
\pgfpathlineto{\pgfqpoint{4.917320in}{5.071596in}}%
\pgfpathlineto{\pgfqpoint{4.919243in}{5.071269in}}%
\pgfpathlineto{\pgfqpoint{4.921167in}{5.061802in}}%
\pgfpathlineto{\pgfqpoint{4.923090in}{5.066060in}}%
\pgfpathlineto{\pgfqpoint{4.925014in}{5.061839in}}%
\pgfpathlineto{\pgfqpoint{4.928860in}{5.047023in}}%
\pgfpathlineto{\pgfqpoint{4.930784in}{5.053602in}}%
\pgfpathlineto{\pgfqpoint{4.932707in}{5.035509in}}%
\pgfpathlineto{\pgfqpoint{4.934631in}{5.042007in}}%
\pgfpathlineto{\pgfqpoint{4.936554in}{5.042791in}}%
\pgfpathlineto{\pgfqpoint{4.938478in}{5.032785in}}%
\pgfpathlineto{\pgfqpoint{4.940401in}{5.030110in}}%
\pgfpathlineto{\pgfqpoint{4.942324in}{5.040828in}}%
\pgfpathlineto{\pgfqpoint{4.944248in}{5.023179in}}%
\pgfpathlineto{\pgfqpoint{4.948095in}{5.010400in}}%
\pgfpathlineto{\pgfqpoint{4.950018in}{5.011737in}}%
\pgfpathlineto{\pgfqpoint{4.951941in}{5.009284in}}%
\pgfpathlineto{\pgfqpoint{4.953865in}{5.018981in}}%
\pgfpathlineto{\pgfqpoint{4.955788in}{5.010445in}}%
\pgfpathlineto{\pgfqpoint{4.957712in}{4.996370in}}%
\pgfpathlineto{\pgfqpoint{4.959635in}{4.996506in}}%
\pgfpathlineto{\pgfqpoint{4.961558in}{4.994878in}}%
\pgfpathlineto{\pgfqpoint{4.963482in}{5.001070in}}%
\pgfpathlineto{\pgfqpoint{4.965405in}{4.994635in}}%
\pgfpathlineto{\pgfqpoint{4.967329in}{4.992815in}}%
\pgfpathlineto{\pgfqpoint{4.969252in}{4.989250in}}%
\pgfpathlineto{\pgfqpoint{4.971176in}{5.000473in}}%
\pgfpathlineto{\pgfqpoint{4.973099in}{5.002594in}}%
\pgfpathlineto{\pgfqpoint{4.975022in}{5.008510in}}%
\pgfpathlineto{\pgfqpoint{4.976946in}{5.022453in}}%
\pgfpathlineto{\pgfqpoint{4.978869in}{5.025137in}}%
\pgfpathlineto{\pgfqpoint{4.980793in}{5.050785in}}%
\pgfpathlineto{\pgfqpoint{4.982716in}{5.053229in}}%
\pgfpathlineto{\pgfqpoint{4.984639in}{5.053127in}}%
\pgfpathlineto{\pgfqpoint{4.986563in}{5.067029in}}%
\pgfpathlineto{\pgfqpoint{4.988486in}{5.057237in}}%
\pgfpathlineto{\pgfqpoint{4.990410in}{5.068034in}}%
\pgfpathlineto{\pgfqpoint{4.992333in}{5.065477in}}%
\pgfpathlineto{\pgfqpoint{4.994256in}{5.069984in}}%
\pgfpathlineto{\pgfqpoint{4.998103in}{5.058335in}}%
\pgfpathlineto{\pgfqpoint{5.000027in}{5.046131in}}%
\pgfpathlineto{\pgfqpoint{5.001950in}{5.057046in}}%
\pgfpathlineto{\pgfqpoint{5.005797in}{5.035548in}}%
\pgfpathlineto{\pgfqpoint{5.007720in}{5.043324in}}%
\pgfpathlineto{\pgfqpoint{5.009644in}{5.045510in}}%
\pgfpathlineto{\pgfqpoint{5.011567in}{5.031993in}}%
\pgfpathlineto{\pgfqpoint{5.017337in}{5.058428in}}%
\pgfpathlineto{\pgfqpoint{5.019261in}{5.058531in}}%
\pgfpathlineto{\pgfqpoint{5.021184in}{5.064661in}}%
\pgfpathlineto{\pgfqpoint{5.023108in}{5.056893in}}%
\pgfpathlineto{\pgfqpoint{5.025031in}{5.055654in}}%
\pgfpathlineto{\pgfqpoint{5.026954in}{5.051021in}}%
\pgfpathlineto{\pgfqpoint{5.028878in}{5.057034in}}%
\pgfpathlineto{\pgfqpoint{5.030801in}{5.040796in}}%
\pgfpathlineto{\pgfqpoint{5.032725in}{5.053978in}}%
\pgfpathlineto{\pgfqpoint{5.036571in}{5.050139in}}%
\pgfpathlineto{\pgfqpoint{5.040418in}{5.031241in}}%
\pgfpathlineto{\pgfqpoint{5.042342in}{5.024705in}}%
\pgfpathlineto{\pgfqpoint{5.044265in}{5.033524in}}%
\pgfpathlineto{\pgfqpoint{5.046189in}{5.021029in}}%
\pgfpathlineto{\pgfqpoint{5.048112in}{5.020617in}}%
\pgfpathlineto{\pgfqpoint{5.050035in}{5.011311in}}%
\pgfpathlineto{\pgfqpoint{5.051959in}{5.024911in}}%
\pgfpathlineto{\pgfqpoint{5.053882in}{5.010331in}}%
\pgfpathlineto{\pgfqpoint{5.055806in}{5.012663in}}%
\pgfpathlineto{\pgfqpoint{5.057729in}{5.010939in}}%
\pgfpathlineto{\pgfqpoint{5.061576in}{5.022327in}}%
\pgfpathlineto{\pgfqpoint{5.063499in}{5.021651in}}%
\pgfpathlineto{\pgfqpoint{5.065423in}{5.025529in}}%
\pgfpathlineto{\pgfqpoint{5.067346in}{5.038300in}}%
\pgfpathlineto{\pgfqpoint{5.069269in}{5.030531in}}%
\pgfpathlineto{\pgfqpoint{5.071193in}{5.033980in}}%
\pgfpathlineto{\pgfqpoint{5.075040in}{5.019024in}}%
\pgfpathlineto{\pgfqpoint{5.076963in}{5.020082in}}%
\pgfpathlineto{\pgfqpoint{5.080810in}{5.024685in}}%
\pgfpathlineto{\pgfqpoint{5.082733in}{5.033873in}}%
\pgfpathlineto{\pgfqpoint{5.084657in}{5.030802in}}%
\pgfpathlineto{\pgfqpoint{5.088504in}{5.058162in}}%
\pgfpathlineto{\pgfqpoint{5.092350in}{5.044630in}}%
\pgfpathlineto{\pgfqpoint{5.098121in}{5.062443in}}%
\pgfpathlineto{\pgfqpoint{5.100044in}{5.045238in}}%
\pgfpathlineto{\pgfqpoint{5.101967in}{5.048600in}}%
\pgfpathlineto{\pgfqpoint{5.103891in}{5.034604in}}%
\pgfpathlineto{\pgfqpoint{5.105814in}{5.038323in}}%
\pgfpathlineto{\pgfqpoint{5.107738in}{5.039008in}}%
\pgfpathlineto{\pgfqpoint{5.109661in}{5.042582in}}%
\pgfpathlineto{\pgfqpoint{5.111585in}{5.035082in}}%
\pgfpathlineto{\pgfqpoint{5.113508in}{5.053431in}}%
\pgfpathlineto{\pgfqpoint{5.115431in}{5.045486in}}%
\pgfpathlineto{\pgfqpoint{5.117355in}{5.047814in}}%
\pgfpathlineto{\pgfqpoint{5.119278in}{5.055671in}}%
\pgfpathlineto{\pgfqpoint{5.121202in}{5.044784in}}%
\pgfpathlineto{\pgfqpoint{5.123125in}{5.048117in}}%
\pgfpathlineto{\pgfqpoint{5.125048in}{5.023274in}}%
\pgfpathlineto{\pgfqpoint{5.126972in}{5.014575in}}%
\pgfpathlineto{\pgfqpoint{5.128895in}{5.000530in}}%
\pgfpathlineto{\pgfqpoint{5.130819in}{5.015235in}}%
\pgfpathlineto{\pgfqpoint{5.132742in}{5.000896in}}%
\pgfpathlineto{\pgfqpoint{5.136589in}{4.993587in}}%
\pgfpathlineto{\pgfqpoint{5.138512in}{5.011219in}}%
\pgfpathlineto{\pgfqpoint{5.140436in}{5.002497in}}%
\pgfpathlineto{\pgfqpoint{5.142359in}{5.019245in}}%
\pgfpathlineto{\pgfqpoint{5.144283in}{5.025530in}}%
\pgfpathlineto{\pgfqpoint{5.146206in}{5.042615in}}%
\pgfpathlineto{\pgfqpoint{5.148129in}{5.036241in}}%
\pgfpathlineto{\pgfqpoint{5.150053in}{5.023368in}}%
\pgfpathlineto{\pgfqpoint{5.151976in}{5.027603in}}%
\pgfpathlineto{\pgfqpoint{5.155823in}{5.006671in}}%
\pgfpathlineto{\pgfqpoint{5.157746in}{5.001656in}}%
\pgfpathlineto{\pgfqpoint{5.159670in}{5.003548in}}%
\pgfpathlineto{\pgfqpoint{5.161593in}{4.994497in}}%
\pgfpathlineto{\pgfqpoint{5.163517in}{4.997742in}}%
\pgfpathlineto{\pgfqpoint{5.167363in}{4.977301in}}%
\pgfpathlineto{\pgfqpoint{5.169287in}{4.976397in}}%
\pgfpathlineto{\pgfqpoint{5.171210in}{4.969590in}}%
\pgfpathlineto{\pgfqpoint{5.175057in}{4.980610in}}%
\pgfpathlineto{\pgfqpoint{5.176980in}{4.979953in}}%
\pgfpathlineto{\pgfqpoint{5.180827in}{4.969464in}}%
\pgfpathlineto{\pgfqpoint{5.182751in}{4.943421in}}%
\pgfpathlineto{\pgfqpoint{5.184674in}{4.936185in}}%
\pgfpathlineto{\pgfqpoint{5.186598in}{4.948284in}}%
\pgfpathlineto{\pgfqpoint{5.188521in}{4.942116in}}%
\pgfpathlineto{\pgfqpoint{5.192368in}{4.948955in}}%
\pgfpathlineto{\pgfqpoint{5.194291in}{4.942293in}}%
\pgfpathlineto{\pgfqpoint{5.196215in}{4.955828in}}%
\pgfpathlineto{\pgfqpoint{5.200061in}{4.945552in}}%
\pgfpathlineto{\pgfqpoint{5.201985in}{4.959683in}}%
\pgfpathlineto{\pgfqpoint{5.203908in}{4.960589in}}%
\pgfpathlineto{\pgfqpoint{5.205832in}{4.976156in}}%
\pgfpathlineto{\pgfqpoint{5.207755in}{4.979627in}}%
\pgfpathlineto{\pgfqpoint{5.209678in}{4.968571in}}%
\pgfpathlineto{\pgfqpoint{5.215449in}{4.959933in}}%
\pgfpathlineto{\pgfqpoint{5.217372in}{4.961327in}}%
\pgfpathlineto{\pgfqpoint{5.219296in}{4.959541in}}%
\pgfpathlineto{\pgfqpoint{5.223142in}{4.971067in}}%
\pgfpathlineto{\pgfqpoint{5.225066in}{4.982820in}}%
\pgfpathlineto{\pgfqpoint{5.226989in}{4.975298in}}%
\pgfpathlineto{\pgfqpoint{5.228913in}{4.975138in}}%
\pgfpathlineto{\pgfqpoint{5.230836in}{4.972062in}}%
\pgfpathlineto{\pgfqpoint{5.232759in}{4.962780in}}%
\pgfpathlineto{\pgfqpoint{5.234683in}{4.965911in}}%
\pgfpathlineto{\pgfqpoint{5.238530in}{4.973899in}}%
\pgfpathlineto{\pgfqpoint{5.240453in}{4.969117in}}%
\pgfpathlineto{\pgfqpoint{5.244300in}{4.974855in}}%
\pgfpathlineto{\pgfqpoint{5.246223in}{4.972503in}}%
\pgfpathlineto{\pgfqpoint{5.248147in}{4.951283in}}%
\pgfpathlineto{\pgfqpoint{5.250070in}{4.961296in}}%
\pgfpathlineto{\pgfqpoint{5.251994in}{4.963791in}}%
\pgfpathlineto{\pgfqpoint{5.253917in}{4.962435in}}%
\pgfpathlineto{\pgfqpoint{5.255840in}{4.978745in}}%
\pgfpathlineto{\pgfqpoint{5.259687in}{4.965864in}}%
\pgfpathlineto{\pgfqpoint{5.261611in}{4.963059in}}%
\pgfpathlineto{\pgfqpoint{5.263534in}{4.963857in}}%
\pgfpathlineto{\pgfqpoint{5.265457in}{4.947410in}}%
\pgfpathlineto{\pgfqpoint{5.267381in}{4.946424in}}%
\pgfpathlineto{\pgfqpoint{5.269304in}{4.955970in}}%
\pgfpathlineto{\pgfqpoint{5.276998in}{4.964808in}}%
\pgfpathlineto{\pgfqpoint{5.278921in}{4.957820in}}%
\pgfpathlineto{\pgfqpoint{5.280845in}{4.962749in}}%
\pgfpathlineto{\pgfqpoint{5.286615in}{4.954675in}}%
\pgfpathlineto{\pgfqpoint{5.288538in}{4.953142in}}%
\pgfpathlineto{\pgfqpoint{5.290462in}{4.946365in}}%
\pgfpathlineto{\pgfqpoint{5.292385in}{4.952944in}}%
\pgfpathlineto{\pgfqpoint{5.294309in}{4.952849in}}%
\pgfpathlineto{\pgfqpoint{5.296232in}{4.943835in}}%
\pgfpathlineto{\pgfqpoint{5.298155in}{4.944009in}}%
\pgfpathlineto{\pgfqpoint{5.300079in}{4.928525in}}%
\pgfpathlineto{\pgfqpoint{5.302002in}{4.935300in}}%
\pgfpathlineto{\pgfqpoint{5.303926in}{4.929259in}}%
\pgfpathlineto{\pgfqpoint{5.303926in}{4.929259in}}%
\pgfusepath{stroke}%
\end{pgfscope}%
\begin{pgfscope}%
\pgfsetrectcap%
\pgfsetmiterjoin%
\pgfsetlinewidth{0.000000pt}%
\definecolor{currentstroke}{rgb}{1.000000,1.000000,1.000000}%
\pgfsetstrokecolor{currentstroke}%
\pgfsetdash{}{0pt}%
\pgfpathmoveto{\pgfqpoint{3.286364in}{3.180000in}}%
\pgfpathlineto{\pgfqpoint{3.286364in}{5.280000in}}%
\pgfusepath{}%
\end{pgfscope}%
\begin{pgfscope}%
\pgfsetrectcap%
\pgfsetmiterjoin%
\pgfsetlinewidth{0.000000pt}%
\definecolor{currentstroke}{rgb}{1.000000,1.000000,1.000000}%
\pgfsetstrokecolor{currentstroke}%
\pgfsetdash{}{0pt}%
\pgfpathmoveto{\pgfqpoint{5.400000in}{3.180000in}}%
\pgfpathlineto{\pgfqpoint{5.400000in}{5.280000in}}%
\pgfusepath{}%
\end{pgfscope}%
\begin{pgfscope}%
\pgfsetrectcap%
\pgfsetmiterjoin%
\pgfsetlinewidth{0.000000pt}%
\definecolor{currentstroke}{rgb}{1.000000,1.000000,1.000000}%
\pgfsetstrokecolor{currentstroke}%
\pgfsetdash{}{0pt}%
\pgfpathmoveto{\pgfqpoint{3.286364in}{3.180000in}}%
\pgfpathlineto{\pgfqpoint{5.400000in}{3.180000in}}%
\pgfusepath{}%
\end{pgfscope}%
\begin{pgfscope}%
\pgfsetrectcap%
\pgfsetmiterjoin%
\pgfsetlinewidth{0.000000pt}%
\definecolor{currentstroke}{rgb}{1.000000,1.000000,1.000000}%
\pgfsetstrokecolor{currentstroke}%
\pgfsetdash{}{0pt}%
\pgfpathmoveto{\pgfqpoint{3.286364in}{5.280000in}}%
\pgfpathlineto{\pgfqpoint{5.400000in}{5.280000in}}%
\pgfusepath{}%
\end{pgfscope}%
\begin{pgfscope}%
\pgfsetbuttcap%
\pgfsetmiterjoin%
\definecolor{currentfill}{rgb}{0.917647,0.917647,0.949020}%
\pgfsetfillcolor{currentfill}%
\pgfsetlinewidth{0.000000pt}%
\definecolor{currentstroke}{rgb}{0.000000,0.000000,0.000000}%
\pgfsetstrokecolor{currentstroke}%
\pgfsetstrokeopacity{0.000000}%
\pgfsetdash{}{0pt}%
\pgfpathmoveto{\pgfqpoint{0.750000in}{0.660000in}}%
\pgfpathlineto{\pgfqpoint{2.863636in}{0.660000in}}%
\pgfpathlineto{\pgfqpoint{2.863636in}{2.760000in}}%
\pgfpathlineto{\pgfqpoint{0.750000in}{2.760000in}}%
\pgfpathlineto{\pgfqpoint{0.750000in}{0.660000in}}%
\pgfpathclose%
\pgfusepath{fill}%
\end{pgfscope}%
\begin{pgfscope}%
\pgfpathrectangle{\pgfqpoint{0.750000in}{0.660000in}}{\pgfqpoint{2.113636in}{2.100000in}}%
\pgfusepath{clip}%
\pgfsetroundcap%
\pgfsetroundjoin%
\pgfsetlinewidth{1.003750pt}%
\definecolor{currentstroke}{rgb}{1.000000,1.000000,1.000000}%
\pgfsetstrokecolor{currentstroke}%
\pgfsetdash{}{0pt}%
\pgfpathmoveto{\pgfqpoint{0.846074in}{0.660000in}}%
\pgfpathlineto{\pgfqpoint{0.846074in}{2.760000in}}%
\pgfusepath{stroke}%
\end{pgfscope}%
\begin{pgfscope}%
\definecolor{textcolor}{rgb}{0.150000,0.150000,0.150000}%
\pgfsetstrokecolor{textcolor}%
\pgfsetfillcolor{textcolor}%
\pgftext[x=0.846074in,y=0.562778in,,top]{\color{textcolor}\rmfamily\fontsize{10.000000}{12.000000}\selectfont \(\displaystyle {0.0}\)}%
\end{pgfscope}%
\begin{pgfscope}%
\pgfpathrectangle{\pgfqpoint{0.750000in}{0.660000in}}{\pgfqpoint{2.113636in}{2.100000in}}%
\pgfusepath{clip}%
\pgfsetroundcap%
\pgfsetroundjoin%
\pgfsetlinewidth{1.003750pt}%
\definecolor{currentstroke}{rgb}{1.000000,1.000000,1.000000}%
\pgfsetstrokecolor{currentstroke}%
\pgfsetdash{}{0pt}%
\pgfpathmoveto{\pgfqpoint{1.326446in}{0.660000in}}%
\pgfpathlineto{\pgfqpoint{1.326446in}{2.760000in}}%
\pgfusepath{stroke}%
\end{pgfscope}%
\begin{pgfscope}%
\definecolor{textcolor}{rgb}{0.150000,0.150000,0.150000}%
\pgfsetstrokecolor{textcolor}%
\pgfsetfillcolor{textcolor}%
\pgftext[x=1.326446in,y=0.562778in,,top]{\color{textcolor}\rmfamily\fontsize{10.000000}{12.000000}\selectfont \(\displaystyle {2.5}\)}%
\end{pgfscope}%
\begin{pgfscope}%
\pgfpathrectangle{\pgfqpoint{0.750000in}{0.660000in}}{\pgfqpoint{2.113636in}{2.100000in}}%
\pgfusepath{clip}%
\pgfsetroundcap%
\pgfsetroundjoin%
\pgfsetlinewidth{1.003750pt}%
\definecolor{currentstroke}{rgb}{1.000000,1.000000,1.000000}%
\pgfsetstrokecolor{currentstroke}%
\pgfsetdash{}{0pt}%
\pgfpathmoveto{\pgfqpoint{1.806818in}{0.660000in}}%
\pgfpathlineto{\pgfqpoint{1.806818in}{2.760000in}}%
\pgfusepath{stroke}%
\end{pgfscope}%
\begin{pgfscope}%
\definecolor{textcolor}{rgb}{0.150000,0.150000,0.150000}%
\pgfsetstrokecolor{textcolor}%
\pgfsetfillcolor{textcolor}%
\pgftext[x=1.806818in,y=0.562778in,,top]{\color{textcolor}\rmfamily\fontsize{10.000000}{12.000000}\selectfont \(\displaystyle {5.0}\)}%
\end{pgfscope}%
\begin{pgfscope}%
\pgfpathrectangle{\pgfqpoint{0.750000in}{0.660000in}}{\pgfqpoint{2.113636in}{2.100000in}}%
\pgfusepath{clip}%
\pgfsetroundcap%
\pgfsetroundjoin%
\pgfsetlinewidth{1.003750pt}%
\definecolor{currentstroke}{rgb}{1.000000,1.000000,1.000000}%
\pgfsetstrokecolor{currentstroke}%
\pgfsetdash{}{0pt}%
\pgfpathmoveto{\pgfqpoint{2.287190in}{0.660000in}}%
\pgfpathlineto{\pgfqpoint{2.287190in}{2.760000in}}%
\pgfusepath{stroke}%
\end{pgfscope}%
\begin{pgfscope}%
\definecolor{textcolor}{rgb}{0.150000,0.150000,0.150000}%
\pgfsetstrokecolor{textcolor}%
\pgfsetfillcolor{textcolor}%
\pgftext[x=2.287190in,y=0.562778in,,top]{\color{textcolor}\rmfamily\fontsize{10.000000}{12.000000}\selectfont \(\displaystyle {7.5}\)}%
\end{pgfscope}%
\begin{pgfscope}%
\pgfpathrectangle{\pgfqpoint{0.750000in}{0.660000in}}{\pgfqpoint{2.113636in}{2.100000in}}%
\pgfusepath{clip}%
\pgfsetroundcap%
\pgfsetroundjoin%
\pgfsetlinewidth{1.003750pt}%
\definecolor{currentstroke}{rgb}{1.000000,1.000000,1.000000}%
\pgfsetstrokecolor{currentstroke}%
\pgfsetdash{}{0pt}%
\pgfpathmoveto{\pgfqpoint{2.767562in}{0.660000in}}%
\pgfpathlineto{\pgfqpoint{2.767562in}{2.760000in}}%
\pgfusepath{stroke}%
\end{pgfscope}%
\begin{pgfscope}%
\definecolor{textcolor}{rgb}{0.150000,0.150000,0.150000}%
\pgfsetstrokecolor{textcolor}%
\pgfsetfillcolor{textcolor}%
\pgftext[x=2.767562in,y=0.562778in,,top]{\color{textcolor}\rmfamily\fontsize{10.000000}{12.000000}\selectfont \(\displaystyle {10.0}\)}%
\end{pgfscope}%
\begin{pgfscope}%
\definecolor{textcolor}{rgb}{0.150000,0.150000,0.150000}%
\pgfsetstrokecolor{textcolor}%
\pgfsetfillcolor{textcolor}%
\pgftext[x=1.806818in,y=0.383766in,,top]{\color{textcolor}\rmfamily\fontsize{11.000000}{13.200000}\selectfont time (\(\displaystyle t\))}%
\end{pgfscope}%
\begin{pgfscope}%
\pgfpathrectangle{\pgfqpoint{0.750000in}{0.660000in}}{\pgfqpoint{2.113636in}{2.100000in}}%
\pgfusepath{clip}%
\pgfsetroundcap%
\pgfsetroundjoin%
\pgfsetlinewidth{1.003750pt}%
\definecolor{currentstroke}{rgb}{1.000000,1.000000,1.000000}%
\pgfsetstrokecolor{currentstroke}%
\pgfsetdash{}{0pt}%
\pgfpathmoveto{\pgfqpoint{0.750000in}{1.017037in}}%
\pgfpathlineto{\pgfqpoint{2.863636in}{1.017037in}}%
\pgfusepath{stroke}%
\end{pgfscope}%
\begin{pgfscope}%
\definecolor{textcolor}{rgb}{0.150000,0.150000,0.150000}%
\pgfsetstrokecolor{textcolor}%
\pgfsetfillcolor{textcolor}%
\pgftext[x=0.405863in, y=0.968812in, left, base]{\color{textcolor}\rmfamily\fontsize{10.000000}{12.000000}\selectfont \(\displaystyle {\ensuremath{-}10}\)}%
\end{pgfscope}%
\begin{pgfscope}%
\pgfpathrectangle{\pgfqpoint{0.750000in}{0.660000in}}{\pgfqpoint{2.113636in}{2.100000in}}%
\pgfusepath{clip}%
\pgfsetroundcap%
\pgfsetroundjoin%
\pgfsetlinewidth{1.003750pt}%
\definecolor{currentstroke}{rgb}{1.000000,1.000000,1.000000}%
\pgfsetstrokecolor{currentstroke}%
\pgfsetdash{}{0pt}%
\pgfpathmoveto{\pgfqpoint{0.750000in}{1.382741in}}%
\pgfpathlineto{\pgfqpoint{2.863636in}{1.382741in}}%
\pgfusepath{stroke}%
\end{pgfscope}%
\begin{pgfscope}%
\definecolor{textcolor}{rgb}{0.150000,0.150000,0.150000}%
\pgfsetstrokecolor{textcolor}%
\pgfsetfillcolor{textcolor}%
\pgftext[x=0.475308in, y=1.334516in, left, base]{\color{textcolor}\rmfamily\fontsize{10.000000}{12.000000}\selectfont \(\displaystyle {\ensuremath{-}5}\)}%
\end{pgfscope}%
\begin{pgfscope}%
\pgfpathrectangle{\pgfqpoint{0.750000in}{0.660000in}}{\pgfqpoint{2.113636in}{2.100000in}}%
\pgfusepath{clip}%
\pgfsetroundcap%
\pgfsetroundjoin%
\pgfsetlinewidth{1.003750pt}%
\definecolor{currentstroke}{rgb}{1.000000,1.000000,1.000000}%
\pgfsetstrokecolor{currentstroke}%
\pgfsetdash{}{0pt}%
\pgfpathmoveto{\pgfqpoint{0.750000in}{1.748444in}}%
\pgfpathlineto{\pgfqpoint{2.863636in}{1.748444in}}%
\pgfusepath{stroke}%
\end{pgfscope}%
\begin{pgfscope}%
\definecolor{textcolor}{rgb}{0.150000,0.150000,0.150000}%
\pgfsetstrokecolor{textcolor}%
\pgfsetfillcolor{textcolor}%
\pgftext[x=0.583333in, y=1.700219in, left, base]{\color{textcolor}\rmfamily\fontsize{10.000000}{12.000000}\selectfont \(\displaystyle {0}\)}%
\end{pgfscope}%
\begin{pgfscope}%
\pgfpathrectangle{\pgfqpoint{0.750000in}{0.660000in}}{\pgfqpoint{2.113636in}{2.100000in}}%
\pgfusepath{clip}%
\pgfsetroundcap%
\pgfsetroundjoin%
\pgfsetlinewidth{1.003750pt}%
\definecolor{currentstroke}{rgb}{1.000000,1.000000,1.000000}%
\pgfsetstrokecolor{currentstroke}%
\pgfsetdash{}{0pt}%
\pgfpathmoveto{\pgfqpoint{0.750000in}{2.114148in}}%
\pgfpathlineto{\pgfqpoint{2.863636in}{2.114148in}}%
\pgfusepath{stroke}%
\end{pgfscope}%
\begin{pgfscope}%
\definecolor{textcolor}{rgb}{0.150000,0.150000,0.150000}%
\pgfsetstrokecolor{textcolor}%
\pgfsetfillcolor{textcolor}%
\pgftext[x=0.583333in, y=2.065923in, left, base]{\color{textcolor}\rmfamily\fontsize{10.000000}{12.000000}\selectfont \(\displaystyle {5}\)}%
\end{pgfscope}%
\begin{pgfscope}%
\pgfpathrectangle{\pgfqpoint{0.750000in}{0.660000in}}{\pgfqpoint{2.113636in}{2.100000in}}%
\pgfusepath{clip}%
\pgfsetroundcap%
\pgfsetroundjoin%
\pgfsetlinewidth{1.003750pt}%
\definecolor{currentstroke}{rgb}{1.000000,1.000000,1.000000}%
\pgfsetstrokecolor{currentstroke}%
\pgfsetdash{}{0pt}%
\pgfpathmoveto{\pgfqpoint{0.750000in}{2.479851in}}%
\pgfpathlineto{\pgfqpoint{2.863636in}{2.479851in}}%
\pgfusepath{stroke}%
\end{pgfscope}%
\begin{pgfscope}%
\definecolor{textcolor}{rgb}{0.150000,0.150000,0.150000}%
\pgfsetstrokecolor{textcolor}%
\pgfsetfillcolor{textcolor}%
\pgftext[x=0.513888in, y=2.431626in, left, base]{\color{textcolor}\rmfamily\fontsize{10.000000}{12.000000}\selectfont \(\displaystyle {10}\)}%
\end{pgfscope}%
\begin{pgfscope}%
\definecolor{textcolor}{rgb}{0.150000,0.150000,0.150000}%
\pgfsetstrokecolor{textcolor}%
\pgfsetfillcolor{textcolor}%
\pgftext[x=0.350308in,y=1.710000in,,bottom,rotate=90.000000]{\color{textcolor}\rmfamily\fontsize{11.000000}{13.200000}\selectfont Position}%
\end{pgfscope}%
\begin{pgfscope}%
\pgfpathrectangle{\pgfqpoint{0.750000in}{0.660000in}}{\pgfqpoint{2.113636in}{2.100000in}}%
\pgfusepath{clip}%
\pgfsetroundcap%
\pgfsetroundjoin%
\pgfsetlinewidth{0.602250pt}%
\definecolor{currentstroke}{rgb}{0.215686,0.494118,0.721569}%
\pgfsetstrokecolor{currentstroke}%
\pgfsetdash{}{0pt}%
\pgfpathmoveto{\pgfqpoint{0.846074in}{1.565593in}}%
\pgfpathlineto{\pgfqpoint{0.847998in}{1.569071in}}%
\pgfpathlineto{\pgfqpoint{0.849921in}{1.581050in}}%
\pgfpathlineto{\pgfqpoint{0.853768in}{1.586793in}}%
\pgfpathlineto{\pgfqpoint{0.855691in}{1.577058in}}%
\pgfpathlineto{\pgfqpoint{0.857615in}{1.574916in}}%
\pgfpathlineto{\pgfqpoint{0.859538in}{1.561164in}}%
\pgfpathlineto{\pgfqpoint{0.861462in}{1.562601in}}%
\pgfpathlineto{\pgfqpoint{0.863385in}{1.571610in}}%
\pgfpathlineto{\pgfqpoint{0.865308in}{1.572976in}}%
\pgfpathlineto{\pgfqpoint{0.867232in}{1.561960in}}%
\pgfpathlineto{\pgfqpoint{0.869155in}{1.563380in}}%
\pgfpathlineto{\pgfqpoint{0.873002in}{1.575302in}}%
\pgfpathlineto{\pgfqpoint{0.874926in}{1.572538in}}%
\pgfpathlineto{\pgfqpoint{0.876849in}{1.574700in}}%
\pgfpathlineto{\pgfqpoint{0.882619in}{1.544005in}}%
\pgfpathlineto{\pgfqpoint{0.884543in}{1.540792in}}%
\pgfpathlineto{\pgfqpoint{0.886466in}{1.542855in}}%
\pgfpathlineto{\pgfqpoint{0.888389in}{1.542681in}}%
\pgfpathlineto{\pgfqpoint{0.890313in}{1.534333in}}%
\pgfpathlineto{\pgfqpoint{0.892236in}{1.531290in}}%
\pgfpathlineto{\pgfqpoint{0.894160in}{1.518572in}}%
\pgfpathlineto{\pgfqpoint{0.896083in}{1.520421in}}%
\pgfpathlineto{\pgfqpoint{0.898006in}{1.519911in}}%
\pgfpathlineto{\pgfqpoint{0.899930in}{1.514233in}}%
\pgfpathlineto{\pgfqpoint{0.901853in}{1.504612in}}%
\pgfpathlineto{\pgfqpoint{0.903777in}{1.504831in}}%
\pgfpathlineto{\pgfqpoint{0.905700in}{1.501428in}}%
\pgfpathlineto{\pgfqpoint{0.907624in}{1.515954in}}%
\pgfpathlineto{\pgfqpoint{0.909547in}{1.507070in}}%
\pgfpathlineto{\pgfqpoint{0.913394in}{1.506883in}}%
\pgfpathlineto{\pgfqpoint{0.915317in}{1.503333in}}%
\pgfpathlineto{\pgfqpoint{0.921087in}{1.537037in}}%
\pgfpathlineto{\pgfqpoint{0.923011in}{1.521674in}}%
\pgfpathlineto{\pgfqpoint{0.924934in}{1.516996in}}%
\pgfpathlineto{\pgfqpoint{0.928781in}{1.502312in}}%
\pgfpathlineto{\pgfqpoint{0.930704in}{1.484153in}}%
\pgfpathlineto{\pgfqpoint{0.934551in}{1.472385in}}%
\pgfpathlineto{\pgfqpoint{0.936475in}{1.476600in}}%
\pgfpathlineto{\pgfqpoint{0.938398in}{1.484316in}}%
\pgfpathlineto{\pgfqpoint{0.944168in}{1.491139in}}%
\pgfpathlineto{\pgfqpoint{0.946092in}{1.501567in}}%
\pgfpathlineto{\pgfqpoint{0.948015in}{1.499821in}}%
\pgfpathlineto{\pgfqpoint{0.951862in}{1.484608in}}%
\pgfpathlineto{\pgfqpoint{0.953785in}{1.489059in}}%
\pgfpathlineto{\pgfqpoint{0.955709in}{1.484314in}}%
\pgfpathlineto{\pgfqpoint{0.957632in}{1.493668in}}%
\pgfpathlineto{\pgfqpoint{0.959556in}{1.480128in}}%
\pgfpathlineto{\pgfqpoint{0.961479in}{1.475689in}}%
\pgfpathlineto{\pgfqpoint{0.963402in}{1.479289in}}%
\pgfpathlineto{\pgfqpoint{0.965326in}{1.476078in}}%
\pgfpathlineto{\pgfqpoint{0.969173in}{1.489347in}}%
\pgfpathlineto{\pgfqpoint{0.971096in}{1.471568in}}%
\pgfpathlineto{\pgfqpoint{0.973020in}{1.477205in}}%
\pgfpathlineto{\pgfqpoint{0.974943in}{1.469637in}}%
\pgfpathlineto{\pgfqpoint{0.976866in}{1.474655in}}%
\pgfpathlineto{\pgfqpoint{0.982637in}{1.451933in}}%
\pgfpathlineto{\pgfqpoint{0.984560in}{1.452865in}}%
\pgfpathlineto{\pgfqpoint{0.986483in}{1.441744in}}%
\pgfpathlineto{\pgfqpoint{0.992254in}{1.457107in}}%
\pgfpathlineto{\pgfqpoint{0.994177in}{1.446586in}}%
\pgfpathlineto{\pgfqpoint{0.996100in}{1.446237in}}%
\pgfpathlineto{\pgfqpoint{0.998024in}{1.448622in}}%
\pgfpathlineto{\pgfqpoint{0.999947in}{1.441795in}}%
\pgfpathlineto{\pgfqpoint{1.001871in}{1.440289in}}%
\pgfpathlineto{\pgfqpoint{1.005717in}{1.445193in}}%
\pgfpathlineto{\pgfqpoint{1.009564in}{1.422944in}}%
\pgfpathlineto{\pgfqpoint{1.013411in}{1.411665in}}%
\pgfpathlineto{\pgfqpoint{1.015335in}{1.403316in}}%
\pgfpathlineto{\pgfqpoint{1.017258in}{1.402420in}}%
\pgfpathlineto{\pgfqpoint{1.019181in}{1.396083in}}%
\pgfpathlineto{\pgfqpoint{1.023028in}{1.404210in}}%
\pgfpathlineto{\pgfqpoint{1.024952in}{1.402708in}}%
\pgfpathlineto{\pgfqpoint{1.026875in}{1.408885in}}%
\pgfpathlineto{\pgfqpoint{1.028798in}{1.405590in}}%
\pgfpathlineto{\pgfqpoint{1.030722in}{1.417993in}}%
\pgfpathlineto{\pgfqpoint{1.034569in}{1.407557in}}%
\pgfpathlineto{\pgfqpoint{1.036492in}{1.409336in}}%
\pgfpathlineto{\pgfqpoint{1.038415in}{1.413036in}}%
\pgfpathlineto{\pgfqpoint{1.044186in}{1.439057in}}%
\pgfpathlineto{\pgfqpoint{1.046109in}{1.437151in}}%
\pgfpathlineto{\pgfqpoint{1.048033in}{1.431214in}}%
\pgfpathlineto{\pgfqpoint{1.049956in}{1.429959in}}%
\pgfpathlineto{\pgfqpoint{1.051879in}{1.424955in}}%
\pgfpathlineto{\pgfqpoint{1.053803in}{1.430394in}}%
\pgfpathlineto{\pgfqpoint{1.055726in}{1.420708in}}%
\pgfpathlineto{\pgfqpoint{1.057650in}{1.420853in}}%
\pgfpathlineto{\pgfqpoint{1.061496in}{1.425072in}}%
\pgfpathlineto{\pgfqpoint{1.063420in}{1.409358in}}%
\pgfpathlineto{\pgfqpoint{1.065343in}{1.405347in}}%
\pgfpathlineto{\pgfqpoint{1.067267in}{1.419000in}}%
\pgfpathlineto{\pgfqpoint{1.069190in}{1.415989in}}%
\pgfpathlineto{\pgfqpoint{1.071113in}{1.415779in}}%
\pgfpathlineto{\pgfqpoint{1.073037in}{1.414450in}}%
\pgfpathlineto{\pgfqpoint{1.074960in}{1.399207in}}%
\pgfpathlineto{\pgfqpoint{1.076884in}{1.400114in}}%
\pgfpathlineto{\pgfqpoint{1.078807in}{1.399381in}}%
\pgfpathlineto{\pgfqpoint{1.080731in}{1.387634in}}%
\pgfpathlineto{\pgfqpoint{1.082654in}{1.405031in}}%
\pgfpathlineto{\pgfqpoint{1.084577in}{1.404978in}}%
\pgfpathlineto{\pgfqpoint{1.086501in}{1.400212in}}%
\pgfpathlineto{\pgfqpoint{1.088424in}{1.401322in}}%
\pgfpathlineto{\pgfqpoint{1.090348in}{1.385234in}}%
\pgfpathlineto{\pgfqpoint{1.096118in}{1.392297in}}%
\pgfpathlineto{\pgfqpoint{1.098041in}{1.381386in}}%
\pgfpathlineto{\pgfqpoint{1.101888in}{1.378982in}}%
\pgfpathlineto{\pgfqpoint{1.105735in}{1.388761in}}%
\pgfpathlineto{\pgfqpoint{1.109582in}{1.366627in}}%
\pgfpathlineto{\pgfqpoint{1.111505in}{1.382291in}}%
\pgfpathlineto{\pgfqpoint{1.117275in}{1.400137in}}%
\pgfpathlineto{\pgfqpoint{1.119199in}{1.400296in}}%
\pgfpathlineto{\pgfqpoint{1.121122in}{1.403172in}}%
\pgfpathlineto{\pgfqpoint{1.123046in}{1.399288in}}%
\pgfpathlineto{\pgfqpoint{1.124969in}{1.390452in}}%
\pgfpathlineto{\pgfqpoint{1.132663in}{1.419460in}}%
\pgfpathlineto{\pgfqpoint{1.134586in}{1.419312in}}%
\pgfpathlineto{\pgfqpoint{1.136509in}{1.412993in}}%
\pgfpathlineto{\pgfqpoint{1.138433in}{1.412092in}}%
\pgfpathlineto{\pgfqpoint{1.140356in}{1.405416in}}%
\pgfpathlineto{\pgfqpoint{1.144203in}{1.402548in}}%
\pgfpathlineto{\pgfqpoint{1.146126in}{1.390985in}}%
\pgfpathlineto{\pgfqpoint{1.148050in}{1.387760in}}%
\pgfpathlineto{\pgfqpoint{1.149973in}{1.390542in}}%
\pgfpathlineto{\pgfqpoint{1.151897in}{1.389093in}}%
\pgfpathlineto{\pgfqpoint{1.155744in}{1.374274in}}%
\pgfpathlineto{\pgfqpoint{1.157667in}{1.371972in}}%
\pgfpathlineto{\pgfqpoint{1.159590in}{1.377936in}}%
\pgfpathlineto{\pgfqpoint{1.161514in}{1.379332in}}%
\pgfpathlineto{\pgfqpoint{1.163437in}{1.373532in}}%
\pgfpathlineto{\pgfqpoint{1.165361in}{1.372997in}}%
\pgfpathlineto{\pgfqpoint{1.167284in}{1.364520in}}%
\pgfpathlineto{\pgfqpoint{1.169207in}{1.375077in}}%
\pgfpathlineto{\pgfqpoint{1.173054in}{1.369387in}}%
\pgfpathlineto{\pgfqpoint{1.176901in}{1.349082in}}%
\pgfpathlineto{\pgfqpoint{1.178824in}{1.358789in}}%
\pgfpathlineto{\pgfqpoint{1.180748in}{1.361226in}}%
\pgfpathlineto{\pgfqpoint{1.182671in}{1.366126in}}%
\pgfpathlineto{\pgfqpoint{1.186518in}{1.365178in}}%
\pgfpathlineto{\pgfqpoint{1.188442in}{1.363226in}}%
\pgfpathlineto{\pgfqpoint{1.190365in}{1.357048in}}%
\pgfpathlineto{\pgfqpoint{1.192288in}{1.364258in}}%
\pgfpathlineto{\pgfqpoint{1.194212in}{1.354479in}}%
\pgfpathlineto{\pgfqpoint{1.196135in}{1.351729in}}%
\pgfpathlineto{\pgfqpoint{1.198059in}{1.356618in}}%
\pgfpathlineto{\pgfqpoint{1.199982in}{1.352384in}}%
\pgfpathlineto{\pgfqpoint{1.201905in}{1.351874in}}%
\pgfpathlineto{\pgfqpoint{1.203829in}{1.355182in}}%
\pgfpathlineto{\pgfqpoint{1.205752in}{1.361199in}}%
\pgfpathlineto{\pgfqpoint{1.209599in}{1.361679in}}%
\pgfpathlineto{\pgfqpoint{1.211522in}{1.361183in}}%
\pgfpathlineto{\pgfqpoint{1.215369in}{1.345274in}}%
\pgfpathlineto{\pgfqpoint{1.217293in}{1.336812in}}%
\pgfpathlineto{\pgfqpoint{1.219216in}{1.333431in}}%
\pgfpathlineto{\pgfqpoint{1.221140in}{1.338211in}}%
\pgfpathlineto{\pgfqpoint{1.223063in}{1.337057in}}%
\pgfpathlineto{\pgfqpoint{1.226910in}{1.331913in}}%
\pgfpathlineto{\pgfqpoint{1.228833in}{1.325135in}}%
\pgfpathlineto{\pgfqpoint{1.230757in}{1.325003in}}%
\pgfpathlineto{\pgfqpoint{1.232680in}{1.319182in}}%
\pgfpathlineto{\pgfqpoint{1.234603in}{1.308102in}}%
\pgfpathlineto{\pgfqpoint{1.236527in}{1.314363in}}%
\pgfpathlineto{\pgfqpoint{1.238450in}{1.315016in}}%
\pgfpathlineto{\pgfqpoint{1.240374in}{1.318214in}}%
\pgfpathlineto{\pgfqpoint{1.242297in}{1.326997in}}%
\pgfpathlineto{\pgfqpoint{1.246144in}{1.317649in}}%
\pgfpathlineto{\pgfqpoint{1.248067in}{1.322924in}}%
\pgfpathlineto{\pgfqpoint{1.251914in}{1.299535in}}%
\pgfpathlineto{\pgfqpoint{1.253838in}{1.298592in}}%
\pgfpathlineto{\pgfqpoint{1.255761in}{1.289041in}}%
\pgfpathlineto{\pgfqpoint{1.257684in}{1.293721in}}%
\pgfpathlineto{\pgfqpoint{1.259608in}{1.289164in}}%
\pgfpathlineto{\pgfqpoint{1.261531in}{1.301428in}}%
\pgfpathlineto{\pgfqpoint{1.263455in}{1.302667in}}%
\pgfpathlineto{\pgfqpoint{1.265378in}{1.300193in}}%
\pgfpathlineto{\pgfqpoint{1.267301in}{1.304276in}}%
\pgfpathlineto{\pgfqpoint{1.269225in}{1.302465in}}%
\pgfpathlineto{\pgfqpoint{1.271148in}{1.305441in}}%
\pgfpathlineto{\pgfqpoint{1.273072in}{1.305610in}}%
\pgfpathlineto{\pgfqpoint{1.274995in}{1.311858in}}%
\pgfpathlineto{\pgfqpoint{1.276918in}{1.306973in}}%
\pgfpathlineto{\pgfqpoint{1.278842in}{1.321772in}}%
\pgfpathlineto{\pgfqpoint{1.280765in}{1.306945in}}%
\pgfpathlineto{\pgfqpoint{1.284612in}{1.313604in}}%
\pgfpathlineto{\pgfqpoint{1.288459in}{1.321817in}}%
\pgfpathlineto{\pgfqpoint{1.290382in}{1.324250in}}%
\pgfpathlineto{\pgfqpoint{1.292306in}{1.315527in}}%
\pgfpathlineto{\pgfqpoint{1.294229in}{1.313459in}}%
\pgfpathlineto{\pgfqpoint{1.296153in}{1.315452in}}%
\pgfpathlineto{\pgfqpoint{1.298076in}{1.312393in}}%
\pgfpathlineto{\pgfqpoint{1.299999in}{1.315815in}}%
\pgfpathlineto{\pgfqpoint{1.303846in}{1.328633in}}%
\pgfpathlineto{\pgfqpoint{1.307693in}{1.336109in}}%
\pgfpathlineto{\pgfqpoint{1.309616in}{1.339596in}}%
\pgfpathlineto{\pgfqpoint{1.311540in}{1.340021in}}%
\pgfpathlineto{\pgfqpoint{1.315387in}{1.320002in}}%
\pgfpathlineto{\pgfqpoint{1.317310in}{1.320855in}}%
\pgfpathlineto{\pgfqpoint{1.319233in}{1.320285in}}%
\pgfpathlineto{\pgfqpoint{1.321157in}{1.315727in}}%
\pgfpathlineto{\pgfqpoint{1.323080in}{1.304227in}}%
\pgfpathlineto{\pgfqpoint{1.325004in}{1.299982in}}%
\pgfpathlineto{\pgfqpoint{1.326927in}{1.307635in}}%
\pgfpathlineto{\pgfqpoint{1.330774in}{1.299501in}}%
\pgfpathlineto{\pgfqpoint{1.334621in}{1.322499in}}%
\pgfpathlineto{\pgfqpoint{1.336544in}{1.320621in}}%
\pgfpathlineto{\pgfqpoint{1.338468in}{1.326054in}}%
\pgfpathlineto{\pgfqpoint{1.340391in}{1.335417in}}%
\pgfpathlineto{\pgfqpoint{1.342314in}{1.338203in}}%
\pgfpathlineto{\pgfqpoint{1.346161in}{1.329334in}}%
\pgfpathlineto{\pgfqpoint{1.348085in}{1.335361in}}%
\pgfpathlineto{\pgfqpoint{1.351931in}{1.320281in}}%
\pgfpathlineto{\pgfqpoint{1.353855in}{1.328650in}}%
\pgfpathlineto{\pgfqpoint{1.355778in}{1.327786in}}%
\pgfpathlineto{\pgfqpoint{1.357702in}{1.329236in}}%
\pgfpathlineto{\pgfqpoint{1.359625in}{1.332493in}}%
\pgfpathlineto{\pgfqpoint{1.361549in}{1.319422in}}%
\pgfpathlineto{\pgfqpoint{1.363472in}{1.327996in}}%
\pgfpathlineto{\pgfqpoint{1.365395in}{1.320801in}}%
\pgfpathlineto{\pgfqpoint{1.367319in}{1.308749in}}%
\pgfpathlineto{\pgfqpoint{1.369242in}{1.306252in}}%
\pgfpathlineto{\pgfqpoint{1.371166in}{1.294048in}}%
\pgfpathlineto{\pgfqpoint{1.376936in}{1.319090in}}%
\pgfpathlineto{\pgfqpoint{1.378859in}{1.317102in}}%
\pgfpathlineto{\pgfqpoint{1.384629in}{1.320810in}}%
\pgfpathlineto{\pgfqpoint{1.386553in}{1.317153in}}%
\pgfpathlineto{\pgfqpoint{1.388476in}{1.318409in}}%
\pgfpathlineto{\pgfqpoint{1.390400in}{1.313321in}}%
\pgfpathlineto{\pgfqpoint{1.392323in}{1.315329in}}%
\pgfpathlineto{\pgfqpoint{1.394247in}{1.321018in}}%
\pgfpathlineto{\pgfqpoint{1.396170in}{1.322884in}}%
\pgfpathlineto{\pgfqpoint{1.398093in}{1.313657in}}%
\pgfpathlineto{\pgfqpoint{1.401940in}{1.322862in}}%
\pgfpathlineto{\pgfqpoint{1.403864in}{1.325754in}}%
\pgfpathlineto{\pgfqpoint{1.405787in}{1.318371in}}%
\pgfpathlineto{\pgfqpoint{1.407710in}{1.326627in}}%
\pgfpathlineto{\pgfqpoint{1.409634in}{1.320210in}}%
\pgfpathlineto{\pgfqpoint{1.413481in}{1.321994in}}%
\pgfpathlineto{\pgfqpoint{1.415404in}{1.317381in}}%
\pgfpathlineto{\pgfqpoint{1.417327in}{1.318811in}}%
\pgfpathlineto{\pgfqpoint{1.419251in}{1.314267in}}%
\pgfpathlineto{\pgfqpoint{1.421174in}{1.316173in}}%
\pgfpathlineto{\pgfqpoint{1.423098in}{1.312063in}}%
\pgfpathlineto{\pgfqpoint{1.425021in}{1.317116in}}%
\pgfpathlineto{\pgfqpoint{1.426945in}{1.312502in}}%
\pgfpathlineto{\pgfqpoint{1.428868in}{1.312083in}}%
\pgfpathlineto{\pgfqpoint{1.430791in}{1.292900in}}%
\pgfpathlineto{\pgfqpoint{1.432715in}{1.285027in}}%
\pgfpathlineto{\pgfqpoint{1.434638in}{1.287780in}}%
\pgfpathlineto{\pgfqpoint{1.436562in}{1.277356in}}%
\pgfpathlineto{\pgfqpoint{1.438485in}{1.282253in}}%
\pgfpathlineto{\pgfqpoint{1.440408in}{1.280701in}}%
\pgfpathlineto{\pgfqpoint{1.442332in}{1.276058in}}%
\pgfpathlineto{\pgfqpoint{1.444255in}{1.283480in}}%
\pgfpathlineto{\pgfqpoint{1.448102in}{1.250772in}}%
\pgfpathlineto{\pgfqpoint{1.453872in}{1.282638in}}%
\pgfpathlineto{\pgfqpoint{1.455796in}{1.283903in}}%
\pgfpathlineto{\pgfqpoint{1.457719in}{1.289483in}}%
\pgfpathlineto{\pgfqpoint{1.459642in}{1.291653in}}%
\pgfpathlineto{\pgfqpoint{1.461566in}{1.296163in}}%
\pgfpathlineto{\pgfqpoint{1.463489in}{1.292962in}}%
\pgfpathlineto{\pgfqpoint{1.465413in}{1.293675in}}%
\pgfpathlineto{\pgfqpoint{1.469260in}{1.297991in}}%
\pgfpathlineto{\pgfqpoint{1.471183in}{1.293584in}}%
\pgfpathlineto{\pgfqpoint{1.473106in}{1.292019in}}%
\pgfpathlineto{\pgfqpoint{1.475030in}{1.281920in}}%
\pgfpathlineto{\pgfqpoint{1.476953in}{1.286830in}}%
\pgfpathlineto{\pgfqpoint{1.478877in}{1.296478in}}%
\pgfpathlineto{\pgfqpoint{1.482723in}{1.283672in}}%
\pgfpathlineto{\pgfqpoint{1.486570in}{1.270186in}}%
\pgfpathlineto{\pgfqpoint{1.488494in}{1.269001in}}%
\pgfpathlineto{\pgfqpoint{1.492340in}{1.252452in}}%
\pgfpathlineto{\pgfqpoint{1.494264in}{1.261114in}}%
\pgfpathlineto{\pgfqpoint{1.496187in}{1.263592in}}%
\pgfpathlineto{\pgfqpoint{1.498111in}{1.259973in}}%
\pgfpathlineto{\pgfqpoint{1.501958in}{1.263276in}}%
\pgfpathlineto{\pgfqpoint{1.503881in}{1.271236in}}%
\pgfpathlineto{\pgfqpoint{1.505804in}{1.270861in}}%
\pgfpathlineto{\pgfqpoint{1.507728in}{1.274165in}}%
\pgfpathlineto{\pgfqpoint{1.509651in}{1.285796in}}%
\pgfpathlineto{\pgfqpoint{1.511575in}{1.287667in}}%
\pgfpathlineto{\pgfqpoint{1.513498in}{1.279800in}}%
\pgfpathlineto{\pgfqpoint{1.515421in}{1.258329in}}%
\pgfpathlineto{\pgfqpoint{1.517345in}{1.265026in}}%
\pgfpathlineto{\pgfqpoint{1.519268in}{1.280715in}}%
\pgfpathlineto{\pgfqpoint{1.523115in}{1.266116in}}%
\pgfpathlineto{\pgfqpoint{1.525038in}{1.266919in}}%
\pgfpathlineto{\pgfqpoint{1.526962in}{1.262368in}}%
\pgfpathlineto{\pgfqpoint{1.528885in}{1.282113in}}%
\pgfpathlineto{\pgfqpoint{1.530809in}{1.278838in}}%
\pgfpathlineto{\pgfqpoint{1.532732in}{1.286899in}}%
\pgfpathlineto{\pgfqpoint{1.534656in}{1.278878in}}%
\pgfpathlineto{\pgfqpoint{1.536579in}{1.288508in}}%
\pgfpathlineto{\pgfqpoint{1.540426in}{1.255097in}}%
\pgfpathlineto{\pgfqpoint{1.542349in}{1.270053in}}%
\pgfpathlineto{\pgfqpoint{1.544273in}{1.273170in}}%
\pgfpathlineto{\pgfqpoint{1.548119in}{1.261365in}}%
\pgfpathlineto{\pgfqpoint{1.550043in}{1.256784in}}%
\pgfpathlineto{\pgfqpoint{1.551966in}{1.256544in}}%
\pgfpathlineto{\pgfqpoint{1.555813in}{1.259960in}}%
\pgfpathlineto{\pgfqpoint{1.557736in}{1.252488in}}%
\pgfpathlineto{\pgfqpoint{1.559660in}{1.257016in}}%
\pgfpathlineto{\pgfqpoint{1.561583in}{1.258071in}}%
\pgfpathlineto{\pgfqpoint{1.563507in}{1.254269in}}%
\pgfpathlineto{\pgfqpoint{1.565430in}{1.244516in}}%
\pgfpathlineto{\pgfqpoint{1.569277in}{1.243993in}}%
\pgfpathlineto{\pgfqpoint{1.573124in}{1.226223in}}%
\pgfpathlineto{\pgfqpoint{1.575047in}{1.226624in}}%
\pgfpathlineto{\pgfqpoint{1.576971in}{1.230090in}}%
\pgfpathlineto{\pgfqpoint{1.578894in}{1.228557in}}%
\pgfpathlineto{\pgfqpoint{1.580817in}{1.230866in}}%
\pgfpathlineto{\pgfqpoint{1.582741in}{1.236552in}}%
\pgfpathlineto{\pgfqpoint{1.584664in}{1.237440in}}%
\pgfpathlineto{\pgfqpoint{1.586588in}{1.234824in}}%
\pgfpathlineto{\pgfqpoint{1.588511in}{1.212649in}}%
\pgfpathlineto{\pgfqpoint{1.590434in}{1.204730in}}%
\pgfpathlineto{\pgfqpoint{1.592358in}{1.201893in}}%
\pgfpathlineto{\pgfqpoint{1.594281in}{1.212712in}}%
\pgfpathlineto{\pgfqpoint{1.596205in}{1.213773in}}%
\pgfpathlineto{\pgfqpoint{1.598128in}{1.203777in}}%
\pgfpathlineto{\pgfqpoint{1.600051in}{1.200028in}}%
\pgfpathlineto{\pgfqpoint{1.601975in}{1.200581in}}%
\pgfpathlineto{\pgfqpoint{1.603898in}{1.196556in}}%
\pgfpathlineto{\pgfqpoint{1.605822in}{1.203184in}}%
\pgfpathlineto{\pgfqpoint{1.607745in}{1.197477in}}%
\pgfpathlineto{\pgfqpoint{1.609669in}{1.204791in}}%
\pgfpathlineto{\pgfqpoint{1.611592in}{1.206093in}}%
\pgfpathlineto{\pgfqpoint{1.613515in}{1.200550in}}%
\pgfpathlineto{\pgfqpoint{1.615439in}{1.189200in}}%
\pgfpathlineto{\pgfqpoint{1.617362in}{1.188075in}}%
\pgfpathlineto{\pgfqpoint{1.619286in}{1.188548in}}%
\pgfpathlineto{\pgfqpoint{1.621209in}{1.174674in}}%
\pgfpathlineto{\pgfqpoint{1.623132in}{1.180790in}}%
\pgfpathlineto{\pgfqpoint{1.625056in}{1.171582in}}%
\pgfpathlineto{\pgfqpoint{1.626979in}{1.178339in}}%
\pgfpathlineto{\pgfqpoint{1.628903in}{1.180600in}}%
\pgfpathlineto{\pgfqpoint{1.632749in}{1.167624in}}%
\pgfpathlineto{\pgfqpoint{1.634673in}{1.168982in}}%
\pgfpathlineto{\pgfqpoint{1.636596in}{1.147019in}}%
\pgfpathlineto{\pgfqpoint{1.638520in}{1.147538in}}%
\pgfpathlineto{\pgfqpoint{1.640443in}{1.134163in}}%
\pgfpathlineto{\pgfqpoint{1.642367in}{1.135234in}}%
\pgfpathlineto{\pgfqpoint{1.644290in}{1.133004in}}%
\pgfpathlineto{\pgfqpoint{1.646213in}{1.125858in}}%
\pgfpathlineto{\pgfqpoint{1.648137in}{1.145073in}}%
\pgfpathlineto{\pgfqpoint{1.650060in}{1.142291in}}%
\pgfpathlineto{\pgfqpoint{1.651984in}{1.148739in}}%
\pgfpathlineto{\pgfqpoint{1.653907in}{1.149651in}}%
\pgfpathlineto{\pgfqpoint{1.655830in}{1.146668in}}%
\pgfpathlineto{\pgfqpoint{1.657754in}{1.146247in}}%
\pgfpathlineto{\pgfqpoint{1.659677in}{1.140061in}}%
\pgfpathlineto{\pgfqpoint{1.661601in}{1.127620in}}%
\pgfpathlineto{\pgfqpoint{1.663524in}{1.131259in}}%
\pgfpathlineto{\pgfqpoint{1.665447in}{1.118044in}}%
\pgfpathlineto{\pgfqpoint{1.667371in}{1.119924in}}%
\pgfpathlineto{\pgfqpoint{1.669294in}{1.123689in}}%
\pgfpathlineto{\pgfqpoint{1.671218in}{1.137009in}}%
\pgfpathlineto{\pgfqpoint{1.675065in}{1.134605in}}%
\pgfpathlineto{\pgfqpoint{1.676988in}{1.141413in}}%
\pgfpathlineto{\pgfqpoint{1.680835in}{1.142600in}}%
\pgfpathlineto{\pgfqpoint{1.682758in}{1.145219in}}%
\pgfpathlineto{\pgfqpoint{1.684682in}{1.152836in}}%
\pgfpathlineto{\pgfqpoint{1.686605in}{1.145426in}}%
\pgfpathlineto{\pgfqpoint{1.688528in}{1.150844in}}%
\pgfpathlineto{\pgfqpoint{1.694299in}{1.132327in}}%
\pgfpathlineto{\pgfqpoint{1.698145in}{1.139582in}}%
\pgfpathlineto{\pgfqpoint{1.700069in}{1.147146in}}%
\pgfpathlineto{\pgfqpoint{1.701992in}{1.146367in}}%
\pgfpathlineto{\pgfqpoint{1.703916in}{1.151205in}}%
\pgfpathlineto{\pgfqpoint{1.705839in}{1.149304in}}%
\pgfpathlineto{\pgfqpoint{1.707763in}{1.151219in}}%
\pgfpathlineto{\pgfqpoint{1.709686in}{1.141967in}}%
\pgfpathlineto{\pgfqpoint{1.715456in}{1.167096in}}%
\pgfpathlineto{\pgfqpoint{1.717380in}{1.150608in}}%
\pgfpathlineto{\pgfqpoint{1.721226in}{1.168148in}}%
\pgfpathlineto{\pgfqpoint{1.725073in}{1.152030in}}%
\pgfpathlineto{\pgfqpoint{1.726997in}{1.156990in}}%
\pgfpathlineto{\pgfqpoint{1.730843in}{1.155496in}}%
\pgfpathlineto{\pgfqpoint{1.740461in}{1.106792in}}%
\pgfpathlineto{\pgfqpoint{1.744307in}{1.121837in}}%
\pgfpathlineto{\pgfqpoint{1.748154in}{1.125811in}}%
\pgfpathlineto{\pgfqpoint{1.750078in}{1.135705in}}%
\pgfpathlineto{\pgfqpoint{1.752001in}{1.127031in}}%
\pgfpathlineto{\pgfqpoint{1.753924in}{1.134469in}}%
\pgfpathlineto{\pgfqpoint{1.755848in}{1.136929in}}%
\pgfpathlineto{\pgfqpoint{1.757771in}{1.127314in}}%
\pgfpathlineto{\pgfqpoint{1.759695in}{1.124332in}}%
\pgfpathlineto{\pgfqpoint{1.763541in}{1.136275in}}%
\pgfpathlineto{\pgfqpoint{1.765465in}{1.134207in}}%
\pgfpathlineto{\pgfqpoint{1.767388in}{1.135929in}}%
\pgfpathlineto{\pgfqpoint{1.769312in}{1.142055in}}%
\pgfpathlineto{\pgfqpoint{1.771235in}{1.143240in}}%
\pgfpathlineto{\pgfqpoint{1.773158in}{1.141771in}}%
\pgfpathlineto{\pgfqpoint{1.775082in}{1.149230in}}%
\pgfpathlineto{\pgfqpoint{1.777005in}{1.149357in}}%
\pgfpathlineto{\pgfqpoint{1.778929in}{1.136076in}}%
\pgfpathlineto{\pgfqpoint{1.780852in}{1.134647in}}%
\pgfpathlineto{\pgfqpoint{1.788546in}{1.112121in}}%
\pgfpathlineto{\pgfqpoint{1.790469in}{1.124632in}}%
\pgfpathlineto{\pgfqpoint{1.792393in}{1.125745in}}%
\pgfpathlineto{\pgfqpoint{1.794316in}{1.117153in}}%
\pgfpathlineto{\pgfqpoint{1.796239in}{1.116579in}}%
\pgfpathlineto{\pgfqpoint{1.803933in}{1.086844in}}%
\pgfpathlineto{\pgfqpoint{1.805856in}{1.088774in}}%
\pgfpathlineto{\pgfqpoint{1.807780in}{1.087081in}}%
\pgfpathlineto{\pgfqpoint{1.813550in}{1.105292in}}%
\pgfpathlineto{\pgfqpoint{1.815474in}{1.101004in}}%
\pgfpathlineto{\pgfqpoint{1.817397in}{1.115310in}}%
\pgfpathlineto{\pgfqpoint{1.821244in}{1.116751in}}%
\pgfpathlineto{\pgfqpoint{1.823167in}{1.117338in}}%
\pgfpathlineto{\pgfqpoint{1.828937in}{1.094494in}}%
\pgfpathlineto{\pgfqpoint{1.832784in}{1.080722in}}%
\pgfpathlineto{\pgfqpoint{1.834708in}{1.082498in}}%
\pgfpathlineto{\pgfqpoint{1.836631in}{1.099552in}}%
\pgfpathlineto{\pgfqpoint{1.838554in}{1.097198in}}%
\pgfpathlineto{\pgfqpoint{1.840478in}{1.090661in}}%
\pgfpathlineto{\pgfqpoint{1.842401in}{1.091706in}}%
\pgfpathlineto{\pgfqpoint{1.844325in}{1.094089in}}%
\pgfpathlineto{\pgfqpoint{1.846248in}{1.105317in}}%
\pgfpathlineto{\pgfqpoint{1.848172in}{1.100491in}}%
\pgfpathlineto{\pgfqpoint{1.850095in}{1.112591in}}%
\pgfpathlineto{\pgfqpoint{1.852018in}{1.108802in}}%
\pgfpathlineto{\pgfqpoint{1.853942in}{1.117877in}}%
\pgfpathlineto{\pgfqpoint{1.855865in}{1.118941in}}%
\pgfpathlineto{\pgfqpoint{1.857789in}{1.117094in}}%
\pgfpathlineto{\pgfqpoint{1.859712in}{1.120347in}}%
\pgfpathlineto{\pgfqpoint{1.863559in}{1.108196in}}%
\pgfpathlineto{\pgfqpoint{1.865482in}{1.109370in}}%
\pgfpathlineto{\pgfqpoint{1.867406in}{1.104418in}}%
\pgfpathlineto{\pgfqpoint{1.869329in}{1.087193in}}%
\pgfpathlineto{\pgfqpoint{1.873176in}{1.101549in}}%
\pgfpathlineto{\pgfqpoint{1.875099in}{1.100488in}}%
\pgfpathlineto{\pgfqpoint{1.877023in}{1.095337in}}%
\pgfpathlineto{\pgfqpoint{1.878946in}{1.098832in}}%
\pgfpathlineto{\pgfqpoint{1.880870in}{1.096497in}}%
\pgfpathlineto{\pgfqpoint{1.882793in}{1.083456in}}%
\pgfpathlineto{\pgfqpoint{1.884716in}{1.089138in}}%
\pgfpathlineto{\pgfqpoint{1.886640in}{1.088315in}}%
\pgfpathlineto{\pgfqpoint{1.888563in}{1.080460in}}%
\pgfpathlineto{\pgfqpoint{1.890487in}{1.087005in}}%
\pgfpathlineto{\pgfqpoint{1.892410in}{1.081262in}}%
\pgfpathlineto{\pgfqpoint{1.894333in}{1.081357in}}%
\pgfpathlineto{\pgfqpoint{1.896257in}{1.079033in}}%
\pgfpathlineto{\pgfqpoint{1.898180in}{1.072762in}}%
\pgfpathlineto{\pgfqpoint{1.900104in}{1.072429in}}%
\pgfpathlineto{\pgfqpoint{1.902027in}{1.059825in}}%
\pgfpathlineto{\pgfqpoint{1.905874in}{1.050021in}}%
\pgfpathlineto{\pgfqpoint{1.907797in}{1.041060in}}%
\pgfpathlineto{\pgfqpoint{1.911644in}{1.045123in}}%
\pgfpathlineto{\pgfqpoint{1.913567in}{1.041244in}}%
\pgfpathlineto{\pgfqpoint{1.915491in}{1.044568in}}%
\pgfpathlineto{\pgfqpoint{1.917414in}{1.053641in}}%
\pgfpathlineto{\pgfqpoint{1.919338in}{1.056944in}}%
\pgfpathlineto{\pgfqpoint{1.921261in}{1.075082in}}%
\pgfpathlineto{\pgfqpoint{1.923185in}{1.064666in}}%
\pgfpathlineto{\pgfqpoint{1.925108in}{1.063476in}}%
\pgfpathlineto{\pgfqpoint{1.927031in}{1.052173in}}%
\pgfpathlineto{\pgfqpoint{1.928955in}{1.060983in}}%
\pgfpathlineto{\pgfqpoint{1.930878in}{1.061798in}}%
\pgfpathlineto{\pgfqpoint{1.932802in}{1.060157in}}%
\pgfpathlineto{\pgfqpoint{1.936648in}{1.052498in}}%
\pgfpathlineto{\pgfqpoint{1.938572in}{1.056116in}}%
\pgfpathlineto{\pgfqpoint{1.940495in}{1.056862in}}%
\pgfpathlineto{\pgfqpoint{1.942419in}{1.060810in}}%
\pgfpathlineto{\pgfqpoint{1.944342in}{1.056073in}}%
\pgfpathlineto{\pgfqpoint{1.948189in}{1.074085in}}%
\pgfpathlineto{\pgfqpoint{1.950112in}{1.070840in}}%
\pgfpathlineto{\pgfqpoint{1.955883in}{1.080823in}}%
\pgfpathlineto{\pgfqpoint{1.959729in}{1.079132in}}%
\pgfpathlineto{\pgfqpoint{1.961653in}{1.093260in}}%
\pgfpathlineto{\pgfqpoint{1.969346in}{1.060622in}}%
\pgfpathlineto{\pgfqpoint{1.971270in}{1.053855in}}%
\pgfpathlineto{\pgfqpoint{1.973193in}{1.064289in}}%
\pgfpathlineto{\pgfqpoint{1.975117in}{1.049056in}}%
\pgfpathlineto{\pgfqpoint{1.977040in}{1.051613in}}%
\pgfpathlineto{\pgfqpoint{1.978963in}{1.036987in}}%
\pgfpathlineto{\pgfqpoint{1.982810in}{1.040315in}}%
\pgfpathlineto{\pgfqpoint{1.984734in}{1.042740in}}%
\pgfpathlineto{\pgfqpoint{1.986657in}{1.039139in}}%
\pgfpathlineto{\pgfqpoint{1.988581in}{1.042005in}}%
\pgfpathlineto{\pgfqpoint{1.990504in}{1.038974in}}%
\pgfpathlineto{\pgfqpoint{1.992427in}{1.033459in}}%
\pgfpathlineto{\pgfqpoint{1.994351in}{1.023855in}}%
\pgfpathlineto{\pgfqpoint{1.998198in}{1.037840in}}%
\pgfpathlineto{\pgfqpoint{2.000121in}{1.034713in}}%
\pgfpathlineto{\pgfqpoint{2.002044in}{1.035684in}}%
\pgfpathlineto{\pgfqpoint{2.003968in}{1.035217in}}%
\pgfpathlineto{\pgfqpoint{2.005891in}{1.032758in}}%
\pgfpathlineto{\pgfqpoint{2.007815in}{1.035041in}}%
\pgfpathlineto{\pgfqpoint{2.009738in}{1.021285in}}%
\pgfpathlineto{\pgfqpoint{2.011661in}{1.015598in}}%
\pgfpathlineto{\pgfqpoint{2.015508in}{1.014140in}}%
\pgfpathlineto{\pgfqpoint{2.017432in}{1.030032in}}%
\pgfpathlineto{\pgfqpoint{2.021279in}{1.007973in}}%
\pgfpathlineto{\pgfqpoint{2.023202in}{1.020448in}}%
\pgfpathlineto{\pgfqpoint{2.025125in}{1.016361in}}%
\pgfpathlineto{\pgfqpoint{2.032819in}{1.049987in}}%
\pgfpathlineto{\pgfqpoint{2.034742in}{1.039533in}}%
\pgfpathlineto{\pgfqpoint{2.038589in}{1.046591in}}%
\pgfpathlineto{\pgfqpoint{2.042436in}{1.032005in}}%
\pgfpathlineto{\pgfqpoint{2.044359in}{1.022325in}}%
\pgfpathlineto{\pgfqpoint{2.046283in}{1.034060in}}%
\pgfpathlineto{\pgfqpoint{2.048206in}{1.033870in}}%
\pgfpathlineto{\pgfqpoint{2.050130in}{1.039990in}}%
\pgfpathlineto{\pgfqpoint{2.052053in}{1.032732in}}%
\pgfpathlineto{\pgfqpoint{2.053976in}{1.041969in}}%
\pgfpathlineto{\pgfqpoint{2.055900in}{1.038830in}}%
\pgfpathlineto{\pgfqpoint{2.057823in}{1.044355in}}%
\pgfpathlineto{\pgfqpoint{2.059747in}{1.041435in}}%
\pgfpathlineto{\pgfqpoint{2.061670in}{1.044528in}}%
\pgfpathlineto{\pgfqpoint{2.063594in}{1.044622in}}%
\pgfpathlineto{\pgfqpoint{2.065517in}{1.036271in}}%
\pgfpathlineto{\pgfqpoint{2.069364in}{1.048687in}}%
\pgfpathlineto{\pgfqpoint{2.071287in}{1.045703in}}%
\pgfpathlineto{\pgfqpoint{2.075134in}{1.034959in}}%
\pgfpathlineto{\pgfqpoint{2.077057in}{1.040554in}}%
\pgfpathlineto{\pgfqpoint{2.078981in}{1.037500in}}%
\pgfpathlineto{\pgfqpoint{2.080904in}{1.030287in}}%
\pgfpathlineto{\pgfqpoint{2.082828in}{1.031704in}}%
\pgfpathlineto{\pgfqpoint{2.084751in}{1.031508in}}%
\pgfpathlineto{\pgfqpoint{2.088598in}{1.005767in}}%
\pgfpathlineto{\pgfqpoint{2.090521in}{1.005602in}}%
\pgfpathlineto{\pgfqpoint{2.092445in}{0.996936in}}%
\pgfpathlineto{\pgfqpoint{2.094368in}{1.002061in}}%
\pgfpathlineto{\pgfqpoint{2.096292in}{1.002936in}}%
\pgfpathlineto{\pgfqpoint{2.098215in}{0.999281in}}%
\pgfpathlineto{\pgfqpoint{2.100138in}{1.001271in}}%
\pgfpathlineto{\pgfqpoint{2.102062in}{1.006529in}}%
\pgfpathlineto{\pgfqpoint{2.103985in}{0.990690in}}%
\pgfpathlineto{\pgfqpoint{2.105909in}{0.984700in}}%
\pgfpathlineto{\pgfqpoint{2.109755in}{0.992653in}}%
\pgfpathlineto{\pgfqpoint{2.111679in}{0.997244in}}%
\pgfpathlineto{\pgfqpoint{2.113602in}{0.992194in}}%
\pgfpathlineto{\pgfqpoint{2.115526in}{0.995225in}}%
\pgfpathlineto{\pgfqpoint{2.117449in}{0.976848in}}%
\pgfpathlineto{\pgfqpoint{2.119372in}{0.983309in}}%
\pgfpathlineto{\pgfqpoint{2.123219in}{0.975294in}}%
\pgfpathlineto{\pgfqpoint{2.125143in}{0.979502in}}%
\pgfpathlineto{\pgfqpoint{2.127066in}{0.969784in}}%
\pgfpathlineto{\pgfqpoint{2.128990in}{0.982825in}}%
\pgfpathlineto{\pgfqpoint{2.130913in}{0.983350in}}%
\pgfpathlineto{\pgfqpoint{2.132836in}{0.970540in}}%
\pgfpathlineto{\pgfqpoint{2.134760in}{0.970002in}}%
\pgfpathlineto{\pgfqpoint{2.136683in}{0.966888in}}%
\pgfpathlineto{\pgfqpoint{2.138607in}{0.952232in}}%
\pgfpathlineto{\pgfqpoint{2.142453in}{0.951511in}}%
\pgfpathlineto{\pgfqpoint{2.146300in}{0.944543in}}%
\pgfpathlineto{\pgfqpoint{2.148224in}{0.950695in}}%
\pgfpathlineto{\pgfqpoint{2.150147in}{0.934638in}}%
\pgfpathlineto{\pgfqpoint{2.153994in}{0.939584in}}%
\pgfpathlineto{\pgfqpoint{2.155917in}{0.928693in}}%
\pgfpathlineto{\pgfqpoint{2.161688in}{0.923917in}}%
\pgfpathlineto{\pgfqpoint{2.163611in}{0.926037in}}%
\pgfpathlineto{\pgfqpoint{2.169381in}{0.894619in}}%
\pgfpathlineto{\pgfqpoint{2.171305in}{0.884970in}}%
\pgfpathlineto{\pgfqpoint{2.173228in}{0.889908in}}%
\pgfpathlineto{\pgfqpoint{2.175151in}{0.888386in}}%
\pgfpathlineto{\pgfqpoint{2.177075in}{0.884497in}}%
\pgfpathlineto{\pgfqpoint{2.178998in}{0.889711in}}%
\pgfpathlineto{\pgfqpoint{2.180922in}{0.880593in}}%
\pgfpathlineto{\pgfqpoint{2.182845in}{0.891635in}}%
\pgfpathlineto{\pgfqpoint{2.184768in}{0.891878in}}%
\pgfpathlineto{\pgfqpoint{2.188615in}{0.873956in}}%
\pgfpathlineto{\pgfqpoint{2.194386in}{0.888821in}}%
\pgfpathlineto{\pgfqpoint{2.196309in}{0.888772in}}%
\pgfpathlineto{\pgfqpoint{2.198232in}{0.879290in}}%
\pgfpathlineto{\pgfqpoint{2.202079in}{0.891974in}}%
\pgfpathlineto{\pgfqpoint{2.205926in}{0.877981in}}%
\pgfpathlineto{\pgfqpoint{2.207849in}{0.878751in}}%
\pgfpathlineto{\pgfqpoint{2.209773in}{0.877649in}}%
\pgfpathlineto{\pgfqpoint{2.211696in}{0.871523in}}%
\pgfpathlineto{\pgfqpoint{2.213620in}{0.877250in}}%
\pgfpathlineto{\pgfqpoint{2.217466in}{0.864946in}}%
\pgfpathlineto{\pgfqpoint{2.219390in}{0.863937in}}%
\pgfpathlineto{\pgfqpoint{2.221313in}{0.860669in}}%
\pgfpathlineto{\pgfqpoint{2.223237in}{0.842512in}}%
\pgfpathlineto{\pgfqpoint{2.227083in}{0.854022in}}%
\pgfpathlineto{\pgfqpoint{2.229007in}{0.847066in}}%
\pgfpathlineto{\pgfqpoint{2.230930in}{0.844681in}}%
\pgfpathlineto{\pgfqpoint{2.238624in}{0.872408in}}%
\pgfpathlineto{\pgfqpoint{2.240547in}{0.872953in}}%
\pgfpathlineto{\pgfqpoint{2.242471in}{0.875763in}}%
\pgfpathlineto{\pgfqpoint{2.244394in}{0.883265in}}%
\pgfpathlineto{\pgfqpoint{2.246318in}{0.879495in}}%
\pgfpathlineto{\pgfqpoint{2.248241in}{0.870230in}}%
\pgfpathlineto{\pgfqpoint{2.254011in}{0.883329in}}%
\pgfpathlineto{\pgfqpoint{2.255935in}{0.876928in}}%
\pgfpathlineto{\pgfqpoint{2.257858in}{0.879749in}}%
\pgfpathlineto{\pgfqpoint{2.259781in}{0.877146in}}%
\pgfpathlineto{\pgfqpoint{2.261705in}{0.878216in}}%
\pgfpathlineto{\pgfqpoint{2.263628in}{0.885416in}}%
\pgfpathlineto{\pgfqpoint{2.265552in}{0.882417in}}%
\pgfpathlineto{\pgfqpoint{2.267475in}{0.869324in}}%
\pgfpathlineto{\pgfqpoint{2.269399in}{0.865574in}}%
\pgfpathlineto{\pgfqpoint{2.271322in}{0.855934in}}%
\pgfpathlineto{\pgfqpoint{2.273245in}{0.857346in}}%
\pgfpathlineto{\pgfqpoint{2.275169in}{0.854644in}}%
\pgfpathlineto{\pgfqpoint{2.277092in}{0.847239in}}%
\pgfpathlineto{\pgfqpoint{2.279016in}{0.844956in}}%
\pgfpathlineto{\pgfqpoint{2.280939in}{0.848633in}}%
\pgfpathlineto{\pgfqpoint{2.282862in}{0.846463in}}%
\pgfpathlineto{\pgfqpoint{2.284786in}{0.842322in}}%
\pgfpathlineto{\pgfqpoint{2.286709in}{0.853469in}}%
\pgfpathlineto{\pgfqpoint{2.288633in}{0.854260in}}%
\pgfpathlineto{\pgfqpoint{2.292479in}{0.869091in}}%
\pgfpathlineto{\pgfqpoint{2.296326in}{0.867070in}}%
\pgfpathlineto{\pgfqpoint{2.298250in}{0.862751in}}%
\pgfpathlineto{\pgfqpoint{2.307867in}{0.885657in}}%
\pgfpathlineto{\pgfqpoint{2.309790in}{0.906533in}}%
\pgfpathlineto{\pgfqpoint{2.311714in}{0.899169in}}%
\pgfpathlineto{\pgfqpoint{2.313637in}{0.896240in}}%
\pgfpathlineto{\pgfqpoint{2.315560in}{0.898984in}}%
\pgfpathlineto{\pgfqpoint{2.317484in}{0.905088in}}%
\pgfpathlineto{\pgfqpoint{2.319407in}{0.901741in}}%
\pgfpathlineto{\pgfqpoint{2.321331in}{0.902144in}}%
\pgfpathlineto{\pgfqpoint{2.323254in}{0.900635in}}%
\pgfpathlineto{\pgfqpoint{2.325177in}{0.906115in}}%
\pgfpathlineto{\pgfqpoint{2.327101in}{0.903440in}}%
\pgfpathlineto{\pgfqpoint{2.329024in}{0.897139in}}%
\pgfpathlineto{\pgfqpoint{2.330948in}{0.899662in}}%
\pgfpathlineto{\pgfqpoint{2.332871in}{0.894451in}}%
\pgfpathlineto{\pgfqpoint{2.334795in}{0.903613in}}%
\pgfpathlineto{\pgfqpoint{2.336718in}{0.897689in}}%
\pgfpathlineto{\pgfqpoint{2.338641in}{0.900247in}}%
\pgfpathlineto{\pgfqpoint{2.340565in}{0.906586in}}%
\pgfpathlineto{\pgfqpoint{2.342488in}{0.920498in}}%
\pgfpathlineto{\pgfqpoint{2.344412in}{0.914882in}}%
\pgfpathlineto{\pgfqpoint{2.346335in}{0.923346in}}%
\pgfpathlineto{\pgfqpoint{2.352105in}{0.899923in}}%
\pgfpathlineto{\pgfqpoint{2.355952in}{0.908892in}}%
\pgfpathlineto{\pgfqpoint{2.359799in}{0.942578in}}%
\pgfpathlineto{\pgfqpoint{2.361722in}{0.940722in}}%
\pgfpathlineto{\pgfqpoint{2.363646in}{0.943331in}}%
\pgfpathlineto{\pgfqpoint{2.369416in}{0.930944in}}%
\pgfpathlineto{\pgfqpoint{2.371339in}{0.930805in}}%
\pgfpathlineto{\pgfqpoint{2.373263in}{0.927531in}}%
\pgfpathlineto{\pgfqpoint{2.377110in}{0.917711in}}%
\pgfpathlineto{\pgfqpoint{2.379033in}{0.924647in}}%
\pgfpathlineto{\pgfqpoint{2.380956in}{0.912332in}}%
\pgfpathlineto{\pgfqpoint{2.382880in}{0.908174in}}%
\pgfpathlineto{\pgfqpoint{2.384803in}{0.907177in}}%
\pgfpathlineto{\pgfqpoint{2.386727in}{0.899442in}}%
\pgfpathlineto{\pgfqpoint{2.388650in}{0.902309in}}%
\pgfpathlineto{\pgfqpoint{2.390573in}{0.902676in}}%
\pgfpathlineto{\pgfqpoint{2.392497in}{0.892551in}}%
\pgfpathlineto{\pgfqpoint{2.396344in}{0.922223in}}%
\pgfpathlineto{\pgfqpoint{2.398267in}{0.916144in}}%
\pgfpathlineto{\pgfqpoint{2.400190in}{0.914151in}}%
\pgfpathlineto{\pgfqpoint{2.402114in}{0.919718in}}%
\pgfpathlineto{\pgfqpoint{2.404037in}{0.908010in}}%
\pgfpathlineto{\pgfqpoint{2.405961in}{0.908302in}}%
\pgfpathlineto{\pgfqpoint{2.407884in}{0.901376in}}%
\pgfpathlineto{\pgfqpoint{2.409808in}{0.909580in}}%
\pgfpathlineto{\pgfqpoint{2.411731in}{0.922671in}}%
\pgfpathlineto{\pgfqpoint{2.417501in}{0.936860in}}%
\pgfpathlineto{\pgfqpoint{2.419425in}{0.923159in}}%
\pgfpathlineto{\pgfqpoint{2.421348in}{0.920779in}}%
\pgfpathlineto{\pgfqpoint{2.423271in}{0.916518in}}%
\pgfpathlineto{\pgfqpoint{2.425195in}{0.928043in}}%
\pgfpathlineto{\pgfqpoint{2.427118in}{0.928048in}}%
\pgfpathlineto{\pgfqpoint{2.429042in}{0.924923in}}%
\pgfpathlineto{\pgfqpoint{2.430965in}{0.916621in}}%
\pgfpathlineto{\pgfqpoint{2.432888in}{0.901493in}}%
\pgfpathlineto{\pgfqpoint{2.434812in}{0.907596in}}%
\pgfpathlineto{\pgfqpoint{2.436735in}{0.899996in}}%
\pgfpathlineto{\pgfqpoint{2.440582in}{0.893121in}}%
\pgfpathlineto{\pgfqpoint{2.442506in}{0.880416in}}%
\pgfpathlineto{\pgfqpoint{2.444429in}{0.886361in}}%
\pgfpathlineto{\pgfqpoint{2.448276in}{0.905356in}}%
\pgfpathlineto{\pgfqpoint{2.450199in}{0.903119in}}%
\pgfpathlineto{\pgfqpoint{2.452123in}{0.902701in}}%
\pgfpathlineto{\pgfqpoint{2.454046in}{0.907335in}}%
\pgfpathlineto{\pgfqpoint{2.455969in}{0.906586in}}%
\pgfpathlineto{\pgfqpoint{2.457893in}{0.913213in}}%
\pgfpathlineto{\pgfqpoint{2.459816in}{0.901235in}}%
\pgfpathlineto{\pgfqpoint{2.461740in}{0.905472in}}%
\pgfpathlineto{\pgfqpoint{2.463663in}{0.906075in}}%
\pgfpathlineto{\pgfqpoint{2.465586in}{0.910989in}}%
\pgfpathlineto{\pgfqpoint{2.469433in}{0.907623in}}%
\pgfpathlineto{\pgfqpoint{2.471357in}{0.908446in}}%
\pgfpathlineto{\pgfqpoint{2.473280in}{0.923436in}}%
\pgfpathlineto{\pgfqpoint{2.475204in}{0.919292in}}%
\pgfpathlineto{\pgfqpoint{2.477127in}{0.910014in}}%
\pgfpathlineto{\pgfqpoint{2.479050in}{0.907043in}}%
\pgfpathlineto{\pgfqpoint{2.480974in}{0.912573in}}%
\pgfpathlineto{\pgfqpoint{2.484821in}{0.892315in}}%
\pgfpathlineto{\pgfqpoint{2.486744in}{0.893901in}}%
\pgfpathlineto{\pgfqpoint{2.488667in}{0.884570in}}%
\pgfpathlineto{\pgfqpoint{2.492514in}{0.877324in}}%
\pgfpathlineto{\pgfqpoint{2.494438in}{0.879295in}}%
\pgfpathlineto{\pgfqpoint{2.496361in}{0.866800in}}%
\pgfpathlineto{\pgfqpoint{2.498284in}{0.862124in}}%
\pgfpathlineto{\pgfqpoint{2.500208in}{0.872393in}}%
\pgfpathlineto{\pgfqpoint{2.502131in}{0.875846in}}%
\pgfpathlineto{\pgfqpoint{2.504055in}{0.865847in}}%
\pgfpathlineto{\pgfqpoint{2.505978in}{0.866396in}}%
\pgfpathlineto{\pgfqpoint{2.507901in}{0.869631in}}%
\pgfpathlineto{\pgfqpoint{2.509825in}{0.866039in}}%
\pgfpathlineto{\pgfqpoint{2.511748in}{0.844422in}}%
\pgfpathlineto{\pgfqpoint{2.513672in}{0.849981in}}%
\pgfpathlineto{\pgfqpoint{2.515595in}{0.852007in}}%
\pgfpathlineto{\pgfqpoint{2.517519in}{0.845839in}}%
\pgfpathlineto{\pgfqpoint{2.519442in}{0.846121in}}%
\pgfpathlineto{\pgfqpoint{2.521365in}{0.847902in}}%
\pgfpathlineto{\pgfqpoint{2.525212in}{0.834412in}}%
\pgfpathlineto{\pgfqpoint{2.527136in}{0.822287in}}%
\pgfpathlineto{\pgfqpoint{2.529059in}{0.822184in}}%
\pgfpathlineto{\pgfqpoint{2.530982in}{0.828289in}}%
\pgfpathlineto{\pgfqpoint{2.532906in}{0.817141in}}%
\pgfpathlineto{\pgfqpoint{2.534829in}{0.816210in}}%
\pgfpathlineto{\pgfqpoint{2.538676in}{0.823146in}}%
\pgfpathlineto{\pgfqpoint{2.540599in}{0.821934in}}%
\pgfpathlineto{\pgfqpoint{2.544446in}{0.812047in}}%
\pgfpathlineto{\pgfqpoint{2.548293in}{0.792918in}}%
\pgfpathlineto{\pgfqpoint{2.550217in}{0.793172in}}%
\pgfpathlineto{\pgfqpoint{2.552140in}{0.805981in}}%
\pgfpathlineto{\pgfqpoint{2.554063in}{0.809417in}}%
\pgfpathlineto{\pgfqpoint{2.555987in}{0.805725in}}%
\pgfpathlineto{\pgfqpoint{2.557910in}{0.809219in}}%
\pgfpathlineto{\pgfqpoint{2.559834in}{0.807803in}}%
\pgfpathlineto{\pgfqpoint{2.561757in}{0.808135in}}%
\pgfpathlineto{\pgfqpoint{2.563680in}{0.810284in}}%
\pgfpathlineto{\pgfqpoint{2.565604in}{0.806840in}}%
\pgfpathlineto{\pgfqpoint{2.567527in}{0.822475in}}%
\pgfpathlineto{\pgfqpoint{2.571374in}{0.812638in}}%
\pgfpathlineto{\pgfqpoint{2.573297in}{0.811002in}}%
\pgfpathlineto{\pgfqpoint{2.575221in}{0.803793in}}%
\pgfpathlineto{\pgfqpoint{2.577144in}{0.791518in}}%
\pgfpathlineto{\pgfqpoint{2.579068in}{0.798421in}}%
\pgfpathlineto{\pgfqpoint{2.580991in}{0.790068in}}%
\pgfpathlineto{\pgfqpoint{2.582915in}{0.789543in}}%
\pgfpathlineto{\pgfqpoint{2.584838in}{0.790969in}}%
\pgfpathlineto{\pgfqpoint{2.586761in}{0.797309in}}%
\pgfpathlineto{\pgfqpoint{2.588685in}{0.790225in}}%
\pgfpathlineto{\pgfqpoint{2.592532in}{0.794214in}}%
\pgfpathlineto{\pgfqpoint{2.594455in}{0.785059in}}%
\pgfpathlineto{\pgfqpoint{2.596378in}{0.791116in}}%
\pgfpathlineto{\pgfqpoint{2.598302in}{0.775877in}}%
\pgfpathlineto{\pgfqpoint{2.602149in}{0.769976in}}%
\pgfpathlineto{\pgfqpoint{2.604072in}{0.760409in}}%
\pgfpathlineto{\pgfqpoint{2.605995in}{0.769464in}}%
\pgfpathlineto{\pgfqpoint{2.607919in}{0.762057in}}%
\pgfpathlineto{\pgfqpoint{2.609842in}{0.762724in}}%
\pgfpathlineto{\pgfqpoint{2.611766in}{0.762010in}}%
\pgfpathlineto{\pgfqpoint{2.613689in}{0.755455in}}%
\pgfpathlineto{\pgfqpoint{2.615613in}{0.763293in}}%
\pgfpathlineto{\pgfqpoint{2.617536in}{0.760725in}}%
\pgfpathlineto{\pgfqpoint{2.619459in}{0.772694in}}%
\pgfpathlineto{\pgfqpoint{2.621383in}{0.763605in}}%
\pgfpathlineto{\pgfqpoint{2.623306in}{0.765785in}}%
\pgfpathlineto{\pgfqpoint{2.625230in}{0.762227in}}%
\pgfpathlineto{\pgfqpoint{2.627153in}{0.771098in}}%
\pgfpathlineto{\pgfqpoint{2.629076in}{0.768257in}}%
\pgfpathlineto{\pgfqpoint{2.631000in}{0.756909in}}%
\pgfpathlineto{\pgfqpoint{2.632923in}{0.760720in}}%
\pgfpathlineto{\pgfqpoint{2.638693in}{0.760610in}}%
\pgfpathlineto{\pgfqpoint{2.640617in}{0.761963in}}%
\pgfpathlineto{\pgfqpoint{2.642540in}{0.760751in}}%
\pgfpathlineto{\pgfqpoint{2.644464in}{0.757253in}}%
\pgfpathlineto{\pgfqpoint{2.646387in}{0.761444in}}%
\pgfpathlineto{\pgfqpoint{2.652157in}{0.779383in}}%
\pgfpathlineto{\pgfqpoint{2.656004in}{0.787324in}}%
\pgfpathlineto{\pgfqpoint{2.657928in}{0.798939in}}%
\pgfpathlineto{\pgfqpoint{2.659851in}{0.801357in}}%
\pgfpathlineto{\pgfqpoint{2.661774in}{0.791716in}}%
\pgfpathlineto{\pgfqpoint{2.663698in}{0.792907in}}%
\pgfpathlineto{\pgfqpoint{2.667545in}{0.786429in}}%
\pgfpathlineto{\pgfqpoint{2.669468in}{0.785584in}}%
\pgfpathlineto{\pgfqpoint{2.671391in}{0.780455in}}%
\pgfpathlineto{\pgfqpoint{2.673315in}{0.781140in}}%
\pgfpathlineto{\pgfqpoint{2.675238in}{0.791798in}}%
\pgfpathlineto{\pgfqpoint{2.677162in}{0.783832in}}%
\pgfpathlineto{\pgfqpoint{2.679085in}{0.797701in}}%
\pgfpathlineto{\pgfqpoint{2.681008in}{0.802312in}}%
\pgfpathlineto{\pgfqpoint{2.682932in}{0.790230in}}%
\pgfpathlineto{\pgfqpoint{2.686779in}{0.796838in}}%
\pgfpathlineto{\pgfqpoint{2.688702in}{0.795955in}}%
\pgfpathlineto{\pgfqpoint{2.690626in}{0.798290in}}%
\pgfpathlineto{\pgfqpoint{2.692549in}{0.790150in}}%
\pgfpathlineto{\pgfqpoint{2.694472in}{0.798485in}}%
\pgfpathlineto{\pgfqpoint{2.696396in}{0.796834in}}%
\pgfpathlineto{\pgfqpoint{2.702166in}{0.789210in}}%
\pgfpathlineto{\pgfqpoint{2.704089in}{0.779194in}}%
\pgfpathlineto{\pgfqpoint{2.706013in}{0.777071in}}%
\pgfpathlineto{\pgfqpoint{2.707936in}{0.789532in}}%
\pgfpathlineto{\pgfqpoint{2.709860in}{0.792755in}}%
\pgfpathlineto{\pgfqpoint{2.711783in}{0.791110in}}%
\pgfpathlineto{\pgfqpoint{2.715630in}{0.801763in}}%
\pgfpathlineto{\pgfqpoint{2.717553in}{0.790522in}}%
\pgfpathlineto{\pgfqpoint{2.721400in}{0.782207in}}%
\pgfpathlineto{\pgfqpoint{2.723324in}{0.788910in}}%
\pgfpathlineto{\pgfqpoint{2.725247in}{0.786041in}}%
\pgfpathlineto{\pgfqpoint{2.729094in}{0.795428in}}%
\pgfpathlineto{\pgfqpoint{2.731017in}{0.800335in}}%
\pgfpathlineto{\pgfqpoint{2.732941in}{0.814205in}}%
\pgfpathlineto{\pgfqpoint{2.734864in}{0.809568in}}%
\pgfpathlineto{\pgfqpoint{2.736787in}{0.812535in}}%
\pgfpathlineto{\pgfqpoint{2.738711in}{0.804562in}}%
\pgfpathlineto{\pgfqpoint{2.740634in}{0.805963in}}%
\pgfpathlineto{\pgfqpoint{2.742558in}{0.802288in}}%
\pgfpathlineto{\pgfqpoint{2.744481in}{0.803691in}}%
\pgfpathlineto{\pgfqpoint{2.746404in}{0.793756in}}%
\pgfpathlineto{\pgfqpoint{2.748328in}{0.800740in}}%
\pgfpathlineto{\pgfqpoint{2.750251in}{0.803017in}}%
\pgfpathlineto{\pgfqpoint{2.752175in}{0.803005in}}%
\pgfpathlineto{\pgfqpoint{2.754098in}{0.790023in}}%
\pgfpathlineto{\pgfqpoint{2.756022in}{0.793519in}}%
\pgfpathlineto{\pgfqpoint{2.759868in}{0.790551in}}%
\pgfpathlineto{\pgfqpoint{2.761792in}{0.797196in}}%
\pgfpathlineto{\pgfqpoint{2.763715in}{0.815803in}}%
\pgfpathlineto{\pgfqpoint{2.767562in}{0.801768in}}%
\pgfpathlineto{\pgfqpoint{2.767562in}{0.801768in}}%
\pgfusepath{stroke}%
\end{pgfscope}%
\begin{pgfscope}%
\pgfpathrectangle{\pgfqpoint{0.750000in}{0.660000in}}{\pgfqpoint{2.113636in}{2.100000in}}%
\pgfusepath{clip}%
\pgfsetroundcap%
\pgfsetroundjoin%
\pgfsetlinewidth{0.602250pt}%
\definecolor{currentstroke}{rgb}{1.000000,0.498039,0.000000}%
\pgfsetstrokecolor{currentstroke}%
\pgfsetdash{}{0pt}%
\pgfpathmoveto{\pgfqpoint{0.846074in}{1.626543in}}%
\pgfpathlineto{\pgfqpoint{0.847998in}{1.616626in}}%
\pgfpathlineto{\pgfqpoint{0.849921in}{1.632092in}}%
\pgfpathlineto{\pgfqpoint{0.853768in}{1.632399in}}%
\pgfpathlineto{\pgfqpoint{0.855691in}{1.626298in}}%
\pgfpathlineto{\pgfqpoint{0.859538in}{1.647665in}}%
\pgfpathlineto{\pgfqpoint{0.865308in}{1.628374in}}%
\pgfpathlineto{\pgfqpoint{0.867232in}{1.630387in}}%
\pgfpathlineto{\pgfqpoint{0.873002in}{1.614111in}}%
\pgfpathlineto{\pgfqpoint{0.874926in}{1.616644in}}%
\pgfpathlineto{\pgfqpoint{0.876849in}{1.612041in}}%
\pgfpathlineto{\pgfqpoint{0.878772in}{1.596857in}}%
\pgfpathlineto{\pgfqpoint{0.880696in}{1.594429in}}%
\pgfpathlineto{\pgfqpoint{0.882619in}{1.594515in}}%
\pgfpathlineto{\pgfqpoint{0.886466in}{1.600458in}}%
\pgfpathlineto{\pgfqpoint{0.888389in}{1.586042in}}%
\pgfpathlineto{\pgfqpoint{0.890313in}{1.595015in}}%
\pgfpathlineto{\pgfqpoint{0.892236in}{1.594694in}}%
\pgfpathlineto{\pgfqpoint{0.896083in}{1.609662in}}%
\pgfpathlineto{\pgfqpoint{0.899930in}{1.612684in}}%
\pgfpathlineto{\pgfqpoint{0.901853in}{1.607069in}}%
\pgfpathlineto{\pgfqpoint{0.905700in}{1.611656in}}%
\pgfpathlineto{\pgfqpoint{0.907624in}{1.613306in}}%
\pgfpathlineto{\pgfqpoint{0.909547in}{1.604144in}}%
\pgfpathlineto{\pgfqpoint{0.911470in}{1.603681in}}%
\pgfpathlineto{\pgfqpoint{0.913394in}{1.599872in}}%
\pgfpathlineto{\pgfqpoint{0.917241in}{1.602716in}}%
\pgfpathlineto{\pgfqpoint{0.919164in}{1.604870in}}%
\pgfpathlineto{\pgfqpoint{0.921087in}{1.593687in}}%
\pgfpathlineto{\pgfqpoint{0.923011in}{1.599590in}}%
\pgfpathlineto{\pgfqpoint{0.924934in}{1.594186in}}%
\pgfpathlineto{\pgfqpoint{0.926858in}{1.592780in}}%
\pgfpathlineto{\pgfqpoint{0.928781in}{1.585089in}}%
\pgfpathlineto{\pgfqpoint{0.930704in}{1.591513in}}%
\pgfpathlineto{\pgfqpoint{0.932628in}{1.586501in}}%
\pgfpathlineto{\pgfqpoint{0.934551in}{1.587783in}}%
\pgfpathlineto{\pgfqpoint{0.938398in}{1.573331in}}%
\pgfpathlineto{\pgfqpoint{0.940322in}{1.572996in}}%
\pgfpathlineto{\pgfqpoint{0.942245in}{1.569561in}}%
\pgfpathlineto{\pgfqpoint{0.944168in}{1.580219in}}%
\pgfpathlineto{\pgfqpoint{0.946092in}{1.573876in}}%
\pgfpathlineto{\pgfqpoint{0.949939in}{1.575225in}}%
\pgfpathlineto{\pgfqpoint{0.951862in}{1.570353in}}%
\pgfpathlineto{\pgfqpoint{0.959556in}{1.589064in}}%
\pgfpathlineto{\pgfqpoint{0.961479in}{1.591836in}}%
\pgfpathlineto{\pgfqpoint{0.963402in}{1.585881in}}%
\pgfpathlineto{\pgfqpoint{0.967249in}{1.589044in}}%
\pgfpathlineto{\pgfqpoint{0.973020in}{1.592525in}}%
\pgfpathlineto{\pgfqpoint{0.974943in}{1.604690in}}%
\pgfpathlineto{\pgfqpoint{0.976866in}{1.604625in}}%
\pgfpathlineto{\pgfqpoint{0.978790in}{1.606994in}}%
\pgfpathlineto{\pgfqpoint{0.982637in}{1.630231in}}%
\pgfpathlineto{\pgfqpoint{0.984560in}{1.625699in}}%
\pgfpathlineto{\pgfqpoint{0.986483in}{1.631584in}}%
\pgfpathlineto{\pgfqpoint{0.988407in}{1.630256in}}%
\pgfpathlineto{\pgfqpoint{0.992254in}{1.644010in}}%
\pgfpathlineto{\pgfqpoint{0.994177in}{1.647493in}}%
\pgfpathlineto{\pgfqpoint{0.996100in}{1.654026in}}%
\pgfpathlineto{\pgfqpoint{0.998024in}{1.650382in}}%
\pgfpathlineto{\pgfqpoint{0.999947in}{1.657001in}}%
\pgfpathlineto{\pgfqpoint{1.001871in}{1.644373in}}%
\pgfpathlineto{\pgfqpoint{1.005717in}{1.652857in}}%
\pgfpathlineto{\pgfqpoint{1.007641in}{1.649750in}}%
\pgfpathlineto{\pgfqpoint{1.009564in}{1.657271in}}%
\pgfpathlineto{\pgfqpoint{1.011488in}{1.658452in}}%
\pgfpathlineto{\pgfqpoint{1.013411in}{1.656401in}}%
\pgfpathlineto{\pgfqpoint{1.015335in}{1.647909in}}%
\pgfpathlineto{\pgfqpoint{1.017258in}{1.655190in}}%
\pgfpathlineto{\pgfqpoint{1.019181in}{1.652914in}}%
\pgfpathlineto{\pgfqpoint{1.024952in}{1.634592in}}%
\pgfpathlineto{\pgfqpoint{1.026875in}{1.638357in}}%
\pgfpathlineto{\pgfqpoint{1.028798in}{1.636553in}}%
\pgfpathlineto{\pgfqpoint{1.030722in}{1.630598in}}%
\pgfpathlineto{\pgfqpoint{1.032645in}{1.630332in}}%
\pgfpathlineto{\pgfqpoint{1.034569in}{1.626891in}}%
\pgfpathlineto{\pgfqpoint{1.036492in}{1.633646in}}%
\pgfpathlineto{\pgfqpoint{1.038415in}{1.635429in}}%
\pgfpathlineto{\pgfqpoint{1.040339in}{1.623216in}}%
\pgfpathlineto{\pgfqpoint{1.048033in}{1.633121in}}%
\pgfpathlineto{\pgfqpoint{1.049956in}{1.620531in}}%
\pgfpathlineto{\pgfqpoint{1.051879in}{1.618036in}}%
\pgfpathlineto{\pgfqpoint{1.053803in}{1.604286in}}%
\pgfpathlineto{\pgfqpoint{1.055726in}{1.609361in}}%
\pgfpathlineto{\pgfqpoint{1.057650in}{1.607305in}}%
\pgfpathlineto{\pgfqpoint{1.061496in}{1.610365in}}%
\pgfpathlineto{\pgfqpoint{1.063420in}{1.606535in}}%
\pgfpathlineto{\pgfqpoint{1.065343in}{1.606925in}}%
\pgfpathlineto{\pgfqpoint{1.067267in}{1.608800in}}%
\pgfpathlineto{\pgfqpoint{1.069190in}{1.601632in}}%
\pgfpathlineto{\pgfqpoint{1.071113in}{1.602411in}}%
\pgfpathlineto{\pgfqpoint{1.074960in}{1.591911in}}%
\pgfpathlineto{\pgfqpoint{1.076884in}{1.596302in}}%
\pgfpathlineto{\pgfqpoint{1.078807in}{1.594217in}}%
\pgfpathlineto{\pgfqpoint{1.080731in}{1.594277in}}%
\pgfpathlineto{\pgfqpoint{1.082654in}{1.573123in}}%
\pgfpathlineto{\pgfqpoint{1.084577in}{1.567229in}}%
\pgfpathlineto{\pgfqpoint{1.086501in}{1.569802in}}%
\pgfpathlineto{\pgfqpoint{1.088424in}{1.577799in}}%
\pgfpathlineto{\pgfqpoint{1.090348in}{1.573661in}}%
\pgfpathlineto{\pgfqpoint{1.092271in}{1.584560in}}%
\pgfpathlineto{\pgfqpoint{1.096118in}{1.582533in}}%
\pgfpathlineto{\pgfqpoint{1.098041in}{1.583209in}}%
\pgfpathlineto{\pgfqpoint{1.099965in}{1.577129in}}%
\pgfpathlineto{\pgfqpoint{1.101888in}{1.581008in}}%
\pgfpathlineto{\pgfqpoint{1.103811in}{1.589385in}}%
\pgfpathlineto{\pgfqpoint{1.107658in}{1.562594in}}%
\pgfpathlineto{\pgfqpoint{1.109582in}{1.561512in}}%
\pgfpathlineto{\pgfqpoint{1.111505in}{1.555342in}}%
\pgfpathlineto{\pgfqpoint{1.113429in}{1.554105in}}%
\pgfpathlineto{\pgfqpoint{1.115352in}{1.555381in}}%
\pgfpathlineto{\pgfqpoint{1.117275in}{1.550468in}}%
\pgfpathlineto{\pgfqpoint{1.121122in}{1.521481in}}%
\pgfpathlineto{\pgfqpoint{1.123046in}{1.532687in}}%
\pgfpathlineto{\pgfqpoint{1.124969in}{1.531636in}}%
\pgfpathlineto{\pgfqpoint{1.126892in}{1.525889in}}%
\pgfpathlineto{\pgfqpoint{1.130739in}{1.498937in}}%
\pgfpathlineto{\pgfqpoint{1.132663in}{1.497677in}}%
\pgfpathlineto{\pgfqpoint{1.136509in}{1.484036in}}%
\pgfpathlineto{\pgfqpoint{1.138433in}{1.497464in}}%
\pgfpathlineto{\pgfqpoint{1.140356in}{1.487902in}}%
\pgfpathlineto{\pgfqpoint{1.142280in}{1.493194in}}%
\pgfpathlineto{\pgfqpoint{1.144203in}{1.494681in}}%
\pgfpathlineto{\pgfqpoint{1.146126in}{1.491501in}}%
\pgfpathlineto{\pgfqpoint{1.148050in}{1.490932in}}%
\pgfpathlineto{\pgfqpoint{1.149973in}{1.498854in}}%
\pgfpathlineto{\pgfqpoint{1.151897in}{1.489260in}}%
\pgfpathlineto{\pgfqpoint{1.153820in}{1.488196in}}%
\pgfpathlineto{\pgfqpoint{1.155744in}{1.491693in}}%
\pgfpathlineto{\pgfqpoint{1.157667in}{1.492572in}}%
\pgfpathlineto{\pgfqpoint{1.159590in}{1.489037in}}%
\pgfpathlineto{\pgfqpoint{1.161514in}{1.465640in}}%
\pgfpathlineto{\pgfqpoint{1.163437in}{1.464108in}}%
\pgfpathlineto{\pgfqpoint{1.169207in}{1.496616in}}%
\pgfpathlineto{\pgfqpoint{1.173054in}{1.476886in}}%
\pgfpathlineto{\pgfqpoint{1.174978in}{1.475886in}}%
\pgfpathlineto{\pgfqpoint{1.176901in}{1.477627in}}%
\pgfpathlineto{\pgfqpoint{1.178824in}{1.482360in}}%
\pgfpathlineto{\pgfqpoint{1.182671in}{1.470643in}}%
\pgfpathlineto{\pgfqpoint{1.186518in}{1.500198in}}%
\pgfpathlineto{\pgfqpoint{1.188442in}{1.492712in}}%
\pgfpathlineto{\pgfqpoint{1.190365in}{1.493885in}}%
\pgfpathlineto{\pgfqpoint{1.192288in}{1.487565in}}%
\pgfpathlineto{\pgfqpoint{1.194212in}{1.486662in}}%
\pgfpathlineto{\pgfqpoint{1.196135in}{1.484507in}}%
\pgfpathlineto{\pgfqpoint{1.198059in}{1.474044in}}%
\pgfpathlineto{\pgfqpoint{1.199982in}{1.455791in}}%
\pgfpathlineto{\pgfqpoint{1.201905in}{1.470444in}}%
\pgfpathlineto{\pgfqpoint{1.203829in}{1.472857in}}%
\pgfpathlineto{\pgfqpoint{1.205752in}{1.460448in}}%
\pgfpathlineto{\pgfqpoint{1.207676in}{1.458512in}}%
\pgfpathlineto{\pgfqpoint{1.211522in}{1.464187in}}%
\pgfpathlineto{\pgfqpoint{1.213446in}{1.475267in}}%
\pgfpathlineto{\pgfqpoint{1.215369in}{1.472817in}}%
\pgfpathlineto{\pgfqpoint{1.217293in}{1.474396in}}%
\pgfpathlineto{\pgfqpoint{1.224986in}{1.450180in}}%
\pgfpathlineto{\pgfqpoint{1.226910in}{1.435455in}}%
\pgfpathlineto{\pgfqpoint{1.230757in}{1.438760in}}%
\pgfpathlineto{\pgfqpoint{1.232680in}{1.443410in}}%
\pgfpathlineto{\pgfqpoint{1.234603in}{1.440407in}}%
\pgfpathlineto{\pgfqpoint{1.236527in}{1.440320in}}%
\pgfpathlineto{\pgfqpoint{1.238450in}{1.427800in}}%
\pgfpathlineto{\pgfqpoint{1.240374in}{1.431346in}}%
\pgfpathlineto{\pgfqpoint{1.242297in}{1.422852in}}%
\pgfpathlineto{\pgfqpoint{1.244220in}{1.434610in}}%
\pgfpathlineto{\pgfqpoint{1.246144in}{1.436289in}}%
\pgfpathlineto{\pgfqpoint{1.248067in}{1.434386in}}%
\pgfpathlineto{\pgfqpoint{1.249991in}{1.437235in}}%
\pgfpathlineto{\pgfqpoint{1.251914in}{1.421647in}}%
\pgfpathlineto{\pgfqpoint{1.253838in}{1.428683in}}%
\pgfpathlineto{\pgfqpoint{1.255761in}{1.424168in}}%
\pgfpathlineto{\pgfqpoint{1.257684in}{1.426073in}}%
\pgfpathlineto{\pgfqpoint{1.259608in}{1.415375in}}%
\pgfpathlineto{\pgfqpoint{1.261531in}{1.413617in}}%
\pgfpathlineto{\pgfqpoint{1.265378in}{1.420712in}}%
\pgfpathlineto{\pgfqpoint{1.267301in}{1.428164in}}%
\pgfpathlineto{\pgfqpoint{1.269225in}{1.424812in}}%
\pgfpathlineto{\pgfqpoint{1.271148in}{1.427291in}}%
\pgfpathlineto{\pgfqpoint{1.274995in}{1.403545in}}%
\pgfpathlineto{\pgfqpoint{1.276918in}{1.415515in}}%
\pgfpathlineto{\pgfqpoint{1.280765in}{1.421420in}}%
\pgfpathlineto{\pgfqpoint{1.282689in}{1.408757in}}%
\pgfpathlineto{\pgfqpoint{1.284612in}{1.416815in}}%
\pgfpathlineto{\pgfqpoint{1.288459in}{1.407348in}}%
\pgfpathlineto{\pgfqpoint{1.290382in}{1.399334in}}%
\pgfpathlineto{\pgfqpoint{1.292306in}{1.397261in}}%
\pgfpathlineto{\pgfqpoint{1.299999in}{1.410324in}}%
\pgfpathlineto{\pgfqpoint{1.301923in}{1.402064in}}%
\pgfpathlineto{\pgfqpoint{1.303846in}{1.410721in}}%
\pgfpathlineto{\pgfqpoint{1.307693in}{1.408385in}}%
\pgfpathlineto{\pgfqpoint{1.309616in}{1.402198in}}%
\pgfpathlineto{\pgfqpoint{1.311540in}{1.404572in}}%
\pgfpathlineto{\pgfqpoint{1.313463in}{1.402464in}}%
\pgfpathlineto{\pgfqpoint{1.315387in}{1.398196in}}%
\pgfpathlineto{\pgfqpoint{1.317310in}{1.413449in}}%
\pgfpathlineto{\pgfqpoint{1.319233in}{1.410741in}}%
\pgfpathlineto{\pgfqpoint{1.323080in}{1.396341in}}%
\pgfpathlineto{\pgfqpoint{1.325004in}{1.380259in}}%
\pgfpathlineto{\pgfqpoint{1.326927in}{1.380703in}}%
\pgfpathlineto{\pgfqpoint{1.328851in}{1.392681in}}%
\pgfpathlineto{\pgfqpoint{1.330774in}{1.387363in}}%
\pgfpathlineto{\pgfqpoint{1.334621in}{1.396079in}}%
\pgfpathlineto{\pgfqpoint{1.336544in}{1.398354in}}%
\pgfpathlineto{\pgfqpoint{1.338468in}{1.394410in}}%
\pgfpathlineto{\pgfqpoint{1.340391in}{1.409898in}}%
\pgfpathlineto{\pgfqpoint{1.342314in}{1.406404in}}%
\pgfpathlineto{\pgfqpoint{1.344238in}{1.392555in}}%
\pgfpathlineto{\pgfqpoint{1.346161in}{1.394238in}}%
\pgfpathlineto{\pgfqpoint{1.351931in}{1.371001in}}%
\pgfpathlineto{\pgfqpoint{1.353855in}{1.377875in}}%
\pgfpathlineto{\pgfqpoint{1.357702in}{1.372334in}}%
\pgfpathlineto{\pgfqpoint{1.359625in}{1.384464in}}%
\pgfpathlineto{\pgfqpoint{1.361549in}{1.379282in}}%
\pgfpathlineto{\pgfqpoint{1.363472in}{1.386558in}}%
\pgfpathlineto{\pgfqpoint{1.367319in}{1.389898in}}%
\pgfpathlineto{\pgfqpoint{1.369242in}{1.389762in}}%
\pgfpathlineto{\pgfqpoint{1.371166in}{1.399140in}}%
\pgfpathlineto{\pgfqpoint{1.373089in}{1.399202in}}%
\pgfpathlineto{\pgfqpoint{1.376936in}{1.387237in}}%
\pgfpathlineto{\pgfqpoint{1.378859in}{1.389084in}}%
\pgfpathlineto{\pgfqpoint{1.380783in}{1.384514in}}%
\pgfpathlineto{\pgfqpoint{1.384629in}{1.366266in}}%
\pgfpathlineto{\pgfqpoint{1.386553in}{1.372096in}}%
\pgfpathlineto{\pgfqpoint{1.388476in}{1.367751in}}%
\pgfpathlineto{\pgfqpoint{1.390400in}{1.372428in}}%
\pgfpathlineto{\pgfqpoint{1.398093in}{1.349970in}}%
\pgfpathlineto{\pgfqpoint{1.400017in}{1.353762in}}%
\pgfpathlineto{\pgfqpoint{1.403864in}{1.373903in}}%
\pgfpathlineto{\pgfqpoint{1.405787in}{1.366309in}}%
\pgfpathlineto{\pgfqpoint{1.407710in}{1.376617in}}%
\pgfpathlineto{\pgfqpoint{1.411557in}{1.359696in}}%
\pgfpathlineto{\pgfqpoint{1.413481in}{1.367312in}}%
\pgfpathlineto{\pgfqpoint{1.415404in}{1.368885in}}%
\pgfpathlineto{\pgfqpoint{1.419251in}{1.367141in}}%
\pgfpathlineto{\pgfqpoint{1.421174in}{1.375188in}}%
\pgfpathlineto{\pgfqpoint{1.423098in}{1.374194in}}%
\pgfpathlineto{\pgfqpoint{1.425021in}{1.371010in}}%
\pgfpathlineto{\pgfqpoint{1.426945in}{1.389971in}}%
\pgfpathlineto{\pgfqpoint{1.428868in}{1.397266in}}%
\pgfpathlineto{\pgfqpoint{1.430791in}{1.392187in}}%
\pgfpathlineto{\pgfqpoint{1.432715in}{1.395507in}}%
\pgfpathlineto{\pgfqpoint{1.434638in}{1.388480in}}%
\pgfpathlineto{\pgfqpoint{1.436562in}{1.388385in}}%
\pgfpathlineto{\pgfqpoint{1.438485in}{1.385913in}}%
\pgfpathlineto{\pgfqpoint{1.440408in}{1.395970in}}%
\pgfpathlineto{\pgfqpoint{1.442332in}{1.395186in}}%
\pgfpathlineto{\pgfqpoint{1.444255in}{1.395746in}}%
\pgfpathlineto{\pgfqpoint{1.446179in}{1.397790in}}%
\pgfpathlineto{\pgfqpoint{1.448102in}{1.390659in}}%
\pgfpathlineto{\pgfqpoint{1.450025in}{1.389914in}}%
\pgfpathlineto{\pgfqpoint{1.451949in}{1.387103in}}%
\pgfpathlineto{\pgfqpoint{1.453872in}{1.390704in}}%
\pgfpathlineto{\pgfqpoint{1.463489in}{1.380395in}}%
\pgfpathlineto{\pgfqpoint{1.465413in}{1.396814in}}%
\pgfpathlineto{\pgfqpoint{1.467336in}{1.391251in}}%
\pgfpathlineto{\pgfqpoint{1.471183in}{1.388581in}}%
\pgfpathlineto{\pgfqpoint{1.473106in}{1.400657in}}%
\pgfpathlineto{\pgfqpoint{1.475030in}{1.403220in}}%
\pgfpathlineto{\pgfqpoint{1.476953in}{1.402740in}}%
\pgfpathlineto{\pgfqpoint{1.478877in}{1.394669in}}%
\pgfpathlineto{\pgfqpoint{1.480800in}{1.381140in}}%
\pgfpathlineto{\pgfqpoint{1.482723in}{1.380347in}}%
\pgfpathlineto{\pgfqpoint{1.486570in}{1.396373in}}%
\pgfpathlineto{\pgfqpoint{1.488494in}{1.394012in}}%
\pgfpathlineto{\pgfqpoint{1.490417in}{1.385201in}}%
\pgfpathlineto{\pgfqpoint{1.494264in}{1.406490in}}%
\pgfpathlineto{\pgfqpoint{1.496187in}{1.406477in}}%
\pgfpathlineto{\pgfqpoint{1.500034in}{1.411788in}}%
\pgfpathlineto{\pgfqpoint{1.501958in}{1.416612in}}%
\pgfpathlineto{\pgfqpoint{1.503881in}{1.408252in}}%
\pgfpathlineto{\pgfqpoint{1.507728in}{1.424096in}}%
\pgfpathlineto{\pgfqpoint{1.509651in}{1.421871in}}%
\pgfpathlineto{\pgfqpoint{1.511575in}{1.428133in}}%
\pgfpathlineto{\pgfqpoint{1.513498in}{1.415297in}}%
\pgfpathlineto{\pgfqpoint{1.515421in}{1.414602in}}%
\pgfpathlineto{\pgfqpoint{1.519268in}{1.394104in}}%
\pgfpathlineto{\pgfqpoint{1.521192in}{1.403387in}}%
\pgfpathlineto{\pgfqpoint{1.525038in}{1.397304in}}%
\pgfpathlineto{\pgfqpoint{1.526962in}{1.412323in}}%
\pgfpathlineto{\pgfqpoint{1.528885in}{1.404740in}}%
\pgfpathlineto{\pgfqpoint{1.530809in}{1.410593in}}%
\pgfpathlineto{\pgfqpoint{1.534656in}{1.405002in}}%
\pgfpathlineto{\pgfqpoint{1.536579in}{1.406784in}}%
\pgfpathlineto{\pgfqpoint{1.538502in}{1.396827in}}%
\pgfpathlineto{\pgfqpoint{1.546196in}{1.401013in}}%
\pgfpathlineto{\pgfqpoint{1.548119in}{1.414902in}}%
\pgfpathlineto{\pgfqpoint{1.550043in}{1.418803in}}%
\pgfpathlineto{\pgfqpoint{1.551966in}{1.408026in}}%
\pgfpathlineto{\pgfqpoint{1.553890in}{1.408476in}}%
\pgfpathlineto{\pgfqpoint{1.557736in}{1.404641in}}%
\pgfpathlineto{\pgfqpoint{1.559660in}{1.420879in}}%
\pgfpathlineto{\pgfqpoint{1.561583in}{1.413891in}}%
\pgfpathlineto{\pgfqpoint{1.563507in}{1.420074in}}%
\pgfpathlineto{\pgfqpoint{1.565430in}{1.432159in}}%
\pgfpathlineto{\pgfqpoint{1.567354in}{1.435845in}}%
\pgfpathlineto{\pgfqpoint{1.569277in}{1.443261in}}%
\pgfpathlineto{\pgfqpoint{1.575047in}{1.427254in}}%
\pgfpathlineto{\pgfqpoint{1.576971in}{1.438322in}}%
\pgfpathlineto{\pgfqpoint{1.578894in}{1.437605in}}%
\pgfpathlineto{\pgfqpoint{1.580817in}{1.438946in}}%
\pgfpathlineto{\pgfqpoint{1.582741in}{1.432709in}}%
\pgfpathlineto{\pgfqpoint{1.584664in}{1.418503in}}%
\pgfpathlineto{\pgfqpoint{1.586588in}{1.415965in}}%
\pgfpathlineto{\pgfqpoint{1.588511in}{1.418070in}}%
\pgfpathlineto{\pgfqpoint{1.590434in}{1.426313in}}%
\pgfpathlineto{\pgfqpoint{1.592358in}{1.416672in}}%
\pgfpathlineto{\pgfqpoint{1.596205in}{1.411683in}}%
\pgfpathlineto{\pgfqpoint{1.600051in}{1.424111in}}%
\pgfpathlineto{\pgfqpoint{1.601975in}{1.427296in}}%
\pgfpathlineto{\pgfqpoint{1.603898in}{1.437494in}}%
\pgfpathlineto{\pgfqpoint{1.607745in}{1.432789in}}%
\pgfpathlineto{\pgfqpoint{1.609669in}{1.424459in}}%
\pgfpathlineto{\pgfqpoint{1.611592in}{1.429580in}}%
\pgfpathlineto{\pgfqpoint{1.613515in}{1.428308in}}%
\pgfpathlineto{\pgfqpoint{1.615439in}{1.423157in}}%
\pgfpathlineto{\pgfqpoint{1.617362in}{1.431205in}}%
\pgfpathlineto{\pgfqpoint{1.619286in}{1.430901in}}%
\pgfpathlineto{\pgfqpoint{1.621209in}{1.426456in}}%
\pgfpathlineto{\pgfqpoint{1.623132in}{1.416962in}}%
\pgfpathlineto{\pgfqpoint{1.625056in}{1.429246in}}%
\pgfpathlineto{\pgfqpoint{1.626979in}{1.429294in}}%
\pgfpathlineto{\pgfqpoint{1.628903in}{1.440672in}}%
\pgfpathlineto{\pgfqpoint{1.630826in}{1.441458in}}%
\pgfpathlineto{\pgfqpoint{1.632749in}{1.446909in}}%
\pgfpathlineto{\pgfqpoint{1.634673in}{1.444931in}}%
\pgfpathlineto{\pgfqpoint{1.636596in}{1.448136in}}%
\pgfpathlineto{\pgfqpoint{1.638520in}{1.448518in}}%
\pgfpathlineto{\pgfqpoint{1.640443in}{1.450732in}}%
\pgfpathlineto{\pgfqpoint{1.642367in}{1.449423in}}%
\pgfpathlineto{\pgfqpoint{1.644290in}{1.455130in}}%
\pgfpathlineto{\pgfqpoint{1.648137in}{1.430882in}}%
\pgfpathlineto{\pgfqpoint{1.650060in}{1.435952in}}%
\pgfpathlineto{\pgfqpoint{1.651984in}{1.426518in}}%
\pgfpathlineto{\pgfqpoint{1.655830in}{1.429005in}}%
\pgfpathlineto{\pgfqpoint{1.657754in}{1.428851in}}%
\pgfpathlineto{\pgfqpoint{1.661601in}{1.415319in}}%
\pgfpathlineto{\pgfqpoint{1.663524in}{1.429957in}}%
\pgfpathlineto{\pgfqpoint{1.665447in}{1.434689in}}%
\pgfpathlineto{\pgfqpoint{1.667371in}{1.433952in}}%
\pgfpathlineto{\pgfqpoint{1.669294in}{1.431088in}}%
\pgfpathlineto{\pgfqpoint{1.671218in}{1.412876in}}%
\pgfpathlineto{\pgfqpoint{1.673141in}{1.419125in}}%
\pgfpathlineto{\pgfqpoint{1.675065in}{1.434588in}}%
\pgfpathlineto{\pgfqpoint{1.676988in}{1.427522in}}%
\pgfpathlineto{\pgfqpoint{1.680835in}{1.437554in}}%
\pgfpathlineto{\pgfqpoint{1.682758in}{1.436489in}}%
\pgfpathlineto{\pgfqpoint{1.684682in}{1.440835in}}%
\pgfpathlineto{\pgfqpoint{1.686605in}{1.420053in}}%
\pgfpathlineto{\pgfqpoint{1.690452in}{1.428242in}}%
\pgfpathlineto{\pgfqpoint{1.692375in}{1.413141in}}%
\pgfpathlineto{\pgfqpoint{1.694299in}{1.418367in}}%
\pgfpathlineto{\pgfqpoint{1.696222in}{1.416447in}}%
\pgfpathlineto{\pgfqpoint{1.698145in}{1.403759in}}%
\pgfpathlineto{\pgfqpoint{1.700069in}{1.413296in}}%
\pgfpathlineto{\pgfqpoint{1.703916in}{1.406822in}}%
\pgfpathlineto{\pgfqpoint{1.705839in}{1.410631in}}%
\pgfpathlineto{\pgfqpoint{1.707763in}{1.400858in}}%
\pgfpathlineto{\pgfqpoint{1.711609in}{1.418505in}}%
\pgfpathlineto{\pgfqpoint{1.713533in}{1.418437in}}%
\pgfpathlineto{\pgfqpoint{1.715456in}{1.421309in}}%
\pgfpathlineto{\pgfqpoint{1.721226in}{1.404666in}}%
\pgfpathlineto{\pgfqpoint{1.725073in}{1.387456in}}%
\pgfpathlineto{\pgfqpoint{1.726997in}{1.391184in}}%
\pgfpathlineto{\pgfqpoint{1.728920in}{1.402662in}}%
\pgfpathlineto{\pgfqpoint{1.730843in}{1.398263in}}%
\pgfpathlineto{\pgfqpoint{1.732767in}{1.389487in}}%
\pgfpathlineto{\pgfqpoint{1.734690in}{1.388562in}}%
\pgfpathlineto{\pgfqpoint{1.736614in}{1.389804in}}%
\pgfpathlineto{\pgfqpoint{1.740461in}{1.384142in}}%
\pgfpathlineto{\pgfqpoint{1.744307in}{1.388439in}}%
\pgfpathlineto{\pgfqpoint{1.748154in}{1.376769in}}%
\pgfpathlineto{\pgfqpoint{1.750078in}{1.384032in}}%
\pgfpathlineto{\pgfqpoint{1.753924in}{1.389242in}}%
\pgfpathlineto{\pgfqpoint{1.757771in}{1.400586in}}%
\pgfpathlineto{\pgfqpoint{1.759695in}{1.398737in}}%
\pgfpathlineto{\pgfqpoint{1.761618in}{1.389260in}}%
\pgfpathlineto{\pgfqpoint{1.763541in}{1.386769in}}%
\pgfpathlineto{\pgfqpoint{1.765465in}{1.394021in}}%
\pgfpathlineto{\pgfqpoint{1.769312in}{1.387629in}}%
\pgfpathlineto{\pgfqpoint{1.771235in}{1.390231in}}%
\pgfpathlineto{\pgfqpoint{1.773158in}{1.389466in}}%
\pgfpathlineto{\pgfqpoint{1.775082in}{1.395067in}}%
\pgfpathlineto{\pgfqpoint{1.780852in}{1.360416in}}%
\pgfpathlineto{\pgfqpoint{1.782776in}{1.354075in}}%
\pgfpathlineto{\pgfqpoint{1.784699in}{1.354184in}}%
\pgfpathlineto{\pgfqpoint{1.786622in}{1.360404in}}%
\pgfpathlineto{\pgfqpoint{1.790469in}{1.359391in}}%
\pgfpathlineto{\pgfqpoint{1.792393in}{1.348314in}}%
\pgfpathlineto{\pgfqpoint{1.794316in}{1.357626in}}%
\pgfpathlineto{\pgfqpoint{1.798163in}{1.364984in}}%
\pgfpathlineto{\pgfqpoint{1.800086in}{1.357072in}}%
\pgfpathlineto{\pgfqpoint{1.802010in}{1.359044in}}%
\pgfpathlineto{\pgfqpoint{1.803933in}{1.344680in}}%
\pgfpathlineto{\pgfqpoint{1.805856in}{1.365879in}}%
\pgfpathlineto{\pgfqpoint{1.807780in}{1.364031in}}%
\pgfpathlineto{\pgfqpoint{1.809703in}{1.355113in}}%
\pgfpathlineto{\pgfqpoint{1.813550in}{1.376464in}}%
\pgfpathlineto{\pgfqpoint{1.815474in}{1.358637in}}%
\pgfpathlineto{\pgfqpoint{1.819320in}{1.366135in}}%
\pgfpathlineto{\pgfqpoint{1.821244in}{1.363552in}}%
\pgfpathlineto{\pgfqpoint{1.823167in}{1.378214in}}%
\pgfpathlineto{\pgfqpoint{1.827014in}{1.357661in}}%
\pgfpathlineto{\pgfqpoint{1.828937in}{1.357455in}}%
\pgfpathlineto{\pgfqpoint{1.830861in}{1.361190in}}%
\pgfpathlineto{\pgfqpoint{1.832784in}{1.377130in}}%
\pgfpathlineto{\pgfqpoint{1.834708in}{1.381105in}}%
\pgfpathlineto{\pgfqpoint{1.838554in}{1.401779in}}%
\pgfpathlineto{\pgfqpoint{1.840478in}{1.400674in}}%
\pgfpathlineto{\pgfqpoint{1.842401in}{1.393961in}}%
\pgfpathlineto{\pgfqpoint{1.844325in}{1.391282in}}%
\pgfpathlineto{\pgfqpoint{1.846248in}{1.391283in}}%
\pgfpathlineto{\pgfqpoint{1.848172in}{1.396746in}}%
\pgfpathlineto{\pgfqpoint{1.850095in}{1.384607in}}%
\pgfpathlineto{\pgfqpoint{1.852018in}{1.364014in}}%
\pgfpathlineto{\pgfqpoint{1.853942in}{1.366291in}}%
\pgfpathlineto{\pgfqpoint{1.855865in}{1.352525in}}%
\pgfpathlineto{\pgfqpoint{1.857789in}{1.351300in}}%
\pgfpathlineto{\pgfqpoint{1.861635in}{1.338478in}}%
\pgfpathlineto{\pgfqpoint{1.863559in}{1.341873in}}%
\pgfpathlineto{\pgfqpoint{1.865482in}{1.314095in}}%
\pgfpathlineto{\pgfqpoint{1.867406in}{1.306222in}}%
\pgfpathlineto{\pgfqpoint{1.869329in}{1.306471in}}%
\pgfpathlineto{\pgfqpoint{1.871252in}{1.311476in}}%
\pgfpathlineto{\pgfqpoint{1.873176in}{1.326137in}}%
\pgfpathlineto{\pgfqpoint{1.875099in}{1.329267in}}%
\pgfpathlineto{\pgfqpoint{1.877023in}{1.336286in}}%
\pgfpathlineto{\pgfqpoint{1.878946in}{1.326849in}}%
\pgfpathlineto{\pgfqpoint{1.880870in}{1.324449in}}%
\pgfpathlineto{\pgfqpoint{1.882793in}{1.319733in}}%
\pgfpathlineto{\pgfqpoint{1.884716in}{1.330189in}}%
\pgfpathlineto{\pgfqpoint{1.886640in}{1.327966in}}%
\pgfpathlineto{\pgfqpoint{1.890487in}{1.332505in}}%
\pgfpathlineto{\pgfqpoint{1.892410in}{1.349419in}}%
\pgfpathlineto{\pgfqpoint{1.896257in}{1.359446in}}%
\pgfpathlineto{\pgfqpoint{1.900104in}{1.355406in}}%
\pgfpathlineto{\pgfqpoint{1.902027in}{1.351063in}}%
\pgfpathlineto{\pgfqpoint{1.903950in}{1.341584in}}%
\pgfpathlineto{\pgfqpoint{1.905874in}{1.339925in}}%
\pgfpathlineto{\pgfqpoint{1.907797in}{1.342332in}}%
\pgfpathlineto{\pgfqpoint{1.909721in}{1.331050in}}%
\pgfpathlineto{\pgfqpoint{1.911644in}{1.330214in}}%
\pgfpathlineto{\pgfqpoint{1.913567in}{1.333094in}}%
\pgfpathlineto{\pgfqpoint{1.915491in}{1.330879in}}%
\pgfpathlineto{\pgfqpoint{1.917414in}{1.337275in}}%
\pgfpathlineto{\pgfqpoint{1.919338in}{1.337565in}}%
\pgfpathlineto{\pgfqpoint{1.921261in}{1.350914in}}%
\pgfpathlineto{\pgfqpoint{1.923185in}{1.352257in}}%
\pgfpathlineto{\pgfqpoint{1.925108in}{1.350839in}}%
\pgfpathlineto{\pgfqpoint{1.927031in}{1.341528in}}%
\pgfpathlineto{\pgfqpoint{1.928955in}{1.339197in}}%
\pgfpathlineto{\pgfqpoint{1.930878in}{1.343606in}}%
\pgfpathlineto{\pgfqpoint{1.932802in}{1.336655in}}%
\pgfpathlineto{\pgfqpoint{1.934725in}{1.335971in}}%
\pgfpathlineto{\pgfqpoint{1.936648in}{1.345004in}}%
\pgfpathlineto{\pgfqpoint{1.938572in}{1.346629in}}%
\pgfpathlineto{\pgfqpoint{1.940495in}{1.342197in}}%
\pgfpathlineto{\pgfqpoint{1.942419in}{1.342850in}}%
\pgfpathlineto{\pgfqpoint{1.944342in}{1.345968in}}%
\pgfpathlineto{\pgfqpoint{1.948189in}{1.337757in}}%
\pgfpathlineto{\pgfqpoint{1.950112in}{1.339252in}}%
\pgfpathlineto{\pgfqpoint{1.952036in}{1.336889in}}%
\pgfpathlineto{\pgfqpoint{1.953959in}{1.317637in}}%
\pgfpathlineto{\pgfqpoint{1.955883in}{1.328777in}}%
\pgfpathlineto{\pgfqpoint{1.959729in}{1.324892in}}%
\pgfpathlineto{\pgfqpoint{1.961653in}{1.334644in}}%
\pgfpathlineto{\pgfqpoint{1.963576in}{1.327172in}}%
\pgfpathlineto{\pgfqpoint{1.965500in}{1.325257in}}%
\pgfpathlineto{\pgfqpoint{1.969346in}{1.311196in}}%
\pgfpathlineto{\pgfqpoint{1.971270in}{1.317777in}}%
\pgfpathlineto{\pgfqpoint{1.973193in}{1.308070in}}%
\pgfpathlineto{\pgfqpoint{1.975117in}{1.309376in}}%
\pgfpathlineto{\pgfqpoint{1.980887in}{1.322649in}}%
\pgfpathlineto{\pgfqpoint{1.982810in}{1.317364in}}%
\pgfpathlineto{\pgfqpoint{1.986657in}{1.336489in}}%
\pgfpathlineto{\pgfqpoint{1.988581in}{1.342230in}}%
\pgfpathlineto{\pgfqpoint{1.990504in}{1.343370in}}%
\pgfpathlineto{\pgfqpoint{1.992427in}{1.364421in}}%
\pgfpathlineto{\pgfqpoint{1.996274in}{1.370188in}}%
\pgfpathlineto{\pgfqpoint{1.998198in}{1.364629in}}%
\pgfpathlineto{\pgfqpoint{2.000121in}{1.365614in}}%
\pgfpathlineto{\pgfqpoint{2.003968in}{1.353429in}}%
\pgfpathlineto{\pgfqpoint{2.005891in}{1.352574in}}%
\pgfpathlineto{\pgfqpoint{2.007815in}{1.355457in}}%
\pgfpathlineto{\pgfqpoint{2.009738in}{1.338873in}}%
\pgfpathlineto{\pgfqpoint{2.011661in}{1.340072in}}%
\pgfpathlineto{\pgfqpoint{2.015508in}{1.333580in}}%
\pgfpathlineto{\pgfqpoint{2.017432in}{1.325579in}}%
\pgfpathlineto{\pgfqpoint{2.019355in}{1.326451in}}%
\pgfpathlineto{\pgfqpoint{2.021279in}{1.324881in}}%
\pgfpathlineto{\pgfqpoint{2.023202in}{1.328372in}}%
\pgfpathlineto{\pgfqpoint{2.028972in}{1.325593in}}%
\pgfpathlineto{\pgfqpoint{2.032819in}{1.308295in}}%
\pgfpathlineto{\pgfqpoint{2.038589in}{1.329542in}}%
\pgfpathlineto{\pgfqpoint{2.040513in}{1.329723in}}%
\pgfpathlineto{\pgfqpoint{2.042436in}{1.334506in}}%
\pgfpathlineto{\pgfqpoint{2.048206in}{1.330098in}}%
\pgfpathlineto{\pgfqpoint{2.052053in}{1.316963in}}%
\pgfpathlineto{\pgfqpoint{2.057823in}{1.331941in}}%
\pgfpathlineto{\pgfqpoint{2.059747in}{1.336469in}}%
\pgfpathlineto{\pgfqpoint{2.061670in}{1.336943in}}%
\pgfpathlineto{\pgfqpoint{2.063594in}{1.334702in}}%
\pgfpathlineto{\pgfqpoint{2.067440in}{1.313930in}}%
\pgfpathlineto{\pgfqpoint{2.069364in}{1.311127in}}%
\pgfpathlineto{\pgfqpoint{2.071287in}{1.306321in}}%
\pgfpathlineto{\pgfqpoint{2.075134in}{1.307573in}}%
\pgfpathlineto{\pgfqpoint{2.077057in}{1.312732in}}%
\pgfpathlineto{\pgfqpoint{2.078981in}{1.301711in}}%
\pgfpathlineto{\pgfqpoint{2.080904in}{1.298096in}}%
\pgfpathlineto{\pgfqpoint{2.082828in}{1.288317in}}%
\pgfpathlineto{\pgfqpoint{2.084751in}{1.292006in}}%
\pgfpathlineto{\pgfqpoint{2.086674in}{1.292378in}}%
\pgfpathlineto{\pgfqpoint{2.088598in}{1.297009in}}%
\pgfpathlineto{\pgfqpoint{2.090521in}{1.292456in}}%
\pgfpathlineto{\pgfqpoint{2.094368in}{1.272908in}}%
\pgfpathlineto{\pgfqpoint{2.096292in}{1.278664in}}%
\pgfpathlineto{\pgfqpoint{2.098215in}{1.276509in}}%
\pgfpathlineto{\pgfqpoint{2.100138in}{1.269115in}}%
\pgfpathlineto{\pgfqpoint{2.102062in}{1.279132in}}%
\pgfpathlineto{\pgfqpoint{2.103985in}{1.266994in}}%
\pgfpathlineto{\pgfqpoint{2.105909in}{1.276108in}}%
\pgfpathlineto{\pgfqpoint{2.109755in}{1.262500in}}%
\pgfpathlineto{\pgfqpoint{2.113602in}{1.246510in}}%
\pgfpathlineto{\pgfqpoint{2.115526in}{1.250729in}}%
\pgfpathlineto{\pgfqpoint{2.117449in}{1.238322in}}%
\pgfpathlineto{\pgfqpoint{2.121296in}{1.266004in}}%
\pgfpathlineto{\pgfqpoint{2.123219in}{1.260121in}}%
\pgfpathlineto{\pgfqpoint{2.127066in}{1.278248in}}%
\pgfpathlineto{\pgfqpoint{2.128990in}{1.270271in}}%
\pgfpathlineto{\pgfqpoint{2.130913in}{1.272421in}}%
\pgfpathlineto{\pgfqpoint{2.132836in}{1.284103in}}%
\pgfpathlineto{\pgfqpoint{2.134760in}{1.282895in}}%
\pgfpathlineto{\pgfqpoint{2.136683in}{1.286729in}}%
\pgfpathlineto{\pgfqpoint{2.140530in}{1.282794in}}%
\pgfpathlineto{\pgfqpoint{2.150147in}{1.312932in}}%
\pgfpathlineto{\pgfqpoint{2.152070in}{1.312815in}}%
\pgfpathlineto{\pgfqpoint{2.153994in}{1.316866in}}%
\pgfpathlineto{\pgfqpoint{2.155917in}{1.312002in}}%
\pgfpathlineto{\pgfqpoint{2.157841in}{1.318670in}}%
\pgfpathlineto{\pgfqpoint{2.159764in}{1.305634in}}%
\pgfpathlineto{\pgfqpoint{2.163611in}{1.316512in}}%
\pgfpathlineto{\pgfqpoint{2.165534in}{1.321608in}}%
\pgfpathlineto{\pgfqpoint{2.167458in}{1.318398in}}%
\pgfpathlineto{\pgfqpoint{2.169381in}{1.325359in}}%
\pgfpathlineto{\pgfqpoint{2.171305in}{1.322488in}}%
\pgfpathlineto{\pgfqpoint{2.173228in}{1.322121in}}%
\pgfpathlineto{\pgfqpoint{2.175151in}{1.327512in}}%
\pgfpathlineto{\pgfqpoint{2.178998in}{1.296483in}}%
\pgfpathlineto{\pgfqpoint{2.180922in}{1.299462in}}%
\pgfpathlineto{\pgfqpoint{2.182845in}{1.311326in}}%
\pgfpathlineto{\pgfqpoint{2.184768in}{1.310626in}}%
\pgfpathlineto{\pgfqpoint{2.186692in}{1.320306in}}%
\pgfpathlineto{\pgfqpoint{2.190539in}{1.299774in}}%
\pgfpathlineto{\pgfqpoint{2.192462in}{1.301443in}}%
\pgfpathlineto{\pgfqpoint{2.194386in}{1.301099in}}%
\pgfpathlineto{\pgfqpoint{2.196309in}{1.297453in}}%
\pgfpathlineto{\pgfqpoint{2.198232in}{1.282084in}}%
\pgfpathlineto{\pgfqpoint{2.200156in}{1.282083in}}%
\pgfpathlineto{\pgfqpoint{2.204003in}{1.270938in}}%
\pgfpathlineto{\pgfqpoint{2.207849in}{1.249288in}}%
\pgfpathlineto{\pgfqpoint{2.209773in}{1.255678in}}%
\pgfpathlineto{\pgfqpoint{2.211696in}{1.242259in}}%
\pgfpathlineto{\pgfqpoint{2.213620in}{1.245652in}}%
\pgfpathlineto{\pgfqpoint{2.217466in}{1.236197in}}%
\pgfpathlineto{\pgfqpoint{2.219390in}{1.229670in}}%
\pgfpathlineto{\pgfqpoint{2.221313in}{1.235118in}}%
\pgfpathlineto{\pgfqpoint{2.223237in}{1.234254in}}%
\pgfpathlineto{\pgfqpoint{2.225160in}{1.238077in}}%
\pgfpathlineto{\pgfqpoint{2.227083in}{1.237612in}}%
\pgfpathlineto{\pgfqpoint{2.229007in}{1.242161in}}%
\pgfpathlineto{\pgfqpoint{2.230930in}{1.250080in}}%
\pgfpathlineto{\pgfqpoint{2.234777in}{1.246007in}}%
\pgfpathlineto{\pgfqpoint{2.236701in}{1.240073in}}%
\pgfpathlineto{\pgfqpoint{2.238624in}{1.226900in}}%
\pgfpathlineto{\pgfqpoint{2.240547in}{1.228189in}}%
\pgfpathlineto{\pgfqpoint{2.242471in}{1.216182in}}%
\pgfpathlineto{\pgfqpoint{2.244394in}{1.196151in}}%
\pgfpathlineto{\pgfqpoint{2.246318in}{1.206299in}}%
\pgfpathlineto{\pgfqpoint{2.250164in}{1.192417in}}%
\pgfpathlineto{\pgfqpoint{2.252088in}{1.190066in}}%
\pgfpathlineto{\pgfqpoint{2.254011in}{1.174801in}}%
\pgfpathlineto{\pgfqpoint{2.255935in}{1.180079in}}%
\pgfpathlineto{\pgfqpoint{2.257858in}{1.176595in}}%
\pgfpathlineto{\pgfqpoint{2.259781in}{1.178072in}}%
\pgfpathlineto{\pgfqpoint{2.261705in}{1.169424in}}%
\pgfpathlineto{\pgfqpoint{2.263628in}{1.169442in}}%
\pgfpathlineto{\pgfqpoint{2.267475in}{1.166486in}}%
\pgfpathlineto{\pgfqpoint{2.271322in}{1.184147in}}%
\pgfpathlineto{\pgfqpoint{2.273245in}{1.182841in}}%
\pgfpathlineto{\pgfqpoint{2.277092in}{1.183648in}}%
\pgfpathlineto{\pgfqpoint{2.280939in}{1.181379in}}%
\pgfpathlineto{\pgfqpoint{2.284786in}{1.188790in}}%
\pgfpathlineto{\pgfqpoint{2.286709in}{1.188121in}}%
\pgfpathlineto{\pgfqpoint{2.288633in}{1.185417in}}%
\pgfpathlineto{\pgfqpoint{2.290556in}{1.186393in}}%
\pgfpathlineto{\pgfqpoint{2.292479in}{1.177574in}}%
\pgfpathlineto{\pgfqpoint{2.294403in}{1.192185in}}%
\pgfpathlineto{\pgfqpoint{2.298250in}{1.199303in}}%
\pgfpathlineto{\pgfqpoint{2.300173in}{1.198450in}}%
\pgfpathlineto{\pgfqpoint{2.304020in}{1.192054in}}%
\pgfpathlineto{\pgfqpoint{2.305943in}{1.192572in}}%
\pgfpathlineto{\pgfqpoint{2.307867in}{1.183243in}}%
\pgfpathlineto{\pgfqpoint{2.309790in}{1.191205in}}%
\pgfpathlineto{\pgfqpoint{2.311714in}{1.188334in}}%
\pgfpathlineto{\pgfqpoint{2.313637in}{1.192929in}}%
\pgfpathlineto{\pgfqpoint{2.315560in}{1.193689in}}%
\pgfpathlineto{\pgfqpoint{2.317484in}{1.200268in}}%
\pgfpathlineto{\pgfqpoint{2.321331in}{1.203219in}}%
\pgfpathlineto{\pgfqpoint{2.327101in}{1.182355in}}%
\pgfpathlineto{\pgfqpoint{2.329024in}{1.190292in}}%
\pgfpathlineto{\pgfqpoint{2.330948in}{1.183906in}}%
\pgfpathlineto{\pgfqpoint{2.332871in}{1.186920in}}%
\pgfpathlineto{\pgfqpoint{2.334795in}{1.183788in}}%
\pgfpathlineto{\pgfqpoint{2.338641in}{1.186782in}}%
\pgfpathlineto{\pgfqpoint{2.340565in}{1.183867in}}%
\pgfpathlineto{\pgfqpoint{2.342488in}{1.175236in}}%
\pgfpathlineto{\pgfqpoint{2.344412in}{1.189644in}}%
\pgfpathlineto{\pgfqpoint{2.346335in}{1.188603in}}%
\pgfpathlineto{\pgfqpoint{2.348258in}{1.189298in}}%
\pgfpathlineto{\pgfqpoint{2.352105in}{1.198604in}}%
\pgfpathlineto{\pgfqpoint{2.354029in}{1.197281in}}%
\pgfpathlineto{\pgfqpoint{2.355952in}{1.186667in}}%
\pgfpathlineto{\pgfqpoint{2.357875in}{1.206249in}}%
\pgfpathlineto{\pgfqpoint{2.359799in}{1.211235in}}%
\pgfpathlineto{\pgfqpoint{2.361722in}{1.225691in}}%
\pgfpathlineto{\pgfqpoint{2.365569in}{1.205836in}}%
\pgfpathlineto{\pgfqpoint{2.369416in}{1.214743in}}%
\pgfpathlineto{\pgfqpoint{2.371339in}{1.215200in}}%
\pgfpathlineto{\pgfqpoint{2.373263in}{1.218831in}}%
\pgfpathlineto{\pgfqpoint{2.377110in}{1.219480in}}%
\pgfpathlineto{\pgfqpoint{2.379033in}{1.221742in}}%
\pgfpathlineto{\pgfqpoint{2.380956in}{1.217239in}}%
\pgfpathlineto{\pgfqpoint{2.384803in}{1.223437in}}%
\pgfpathlineto{\pgfqpoint{2.386727in}{1.219535in}}%
\pgfpathlineto{\pgfqpoint{2.388650in}{1.218207in}}%
\pgfpathlineto{\pgfqpoint{2.392497in}{1.209778in}}%
\pgfpathlineto{\pgfqpoint{2.394420in}{1.207539in}}%
\pgfpathlineto{\pgfqpoint{2.396344in}{1.214258in}}%
\pgfpathlineto{\pgfqpoint{2.402114in}{1.221231in}}%
\pgfpathlineto{\pgfqpoint{2.404037in}{1.222052in}}%
\pgfpathlineto{\pgfqpoint{2.405961in}{1.221145in}}%
\pgfpathlineto{\pgfqpoint{2.407884in}{1.225939in}}%
\pgfpathlineto{\pgfqpoint{2.409808in}{1.227498in}}%
\pgfpathlineto{\pgfqpoint{2.411731in}{1.224515in}}%
\pgfpathlineto{\pgfqpoint{2.419425in}{1.201436in}}%
\pgfpathlineto{\pgfqpoint{2.421348in}{1.205638in}}%
\pgfpathlineto{\pgfqpoint{2.423271in}{1.199534in}}%
\pgfpathlineto{\pgfqpoint{2.425195in}{1.187853in}}%
\pgfpathlineto{\pgfqpoint{2.427118in}{1.193422in}}%
\pgfpathlineto{\pgfqpoint{2.429042in}{1.189377in}}%
\pgfpathlineto{\pgfqpoint{2.430965in}{1.188579in}}%
\pgfpathlineto{\pgfqpoint{2.432888in}{1.173667in}}%
\pgfpathlineto{\pgfqpoint{2.434812in}{1.169950in}}%
\pgfpathlineto{\pgfqpoint{2.436735in}{1.174949in}}%
\pgfpathlineto{\pgfqpoint{2.438659in}{1.161273in}}%
\pgfpathlineto{\pgfqpoint{2.440582in}{1.170099in}}%
\pgfpathlineto{\pgfqpoint{2.442506in}{1.171643in}}%
\pgfpathlineto{\pgfqpoint{2.446352in}{1.184119in}}%
\pgfpathlineto{\pgfqpoint{2.448276in}{1.168401in}}%
\pgfpathlineto{\pgfqpoint{2.450199in}{1.168181in}}%
\pgfpathlineto{\pgfqpoint{2.452123in}{1.166139in}}%
\pgfpathlineto{\pgfqpoint{2.455969in}{1.166088in}}%
\pgfpathlineto{\pgfqpoint{2.459816in}{1.147931in}}%
\pgfpathlineto{\pgfqpoint{2.467510in}{1.112940in}}%
\pgfpathlineto{\pgfqpoint{2.469433in}{1.119836in}}%
\pgfpathlineto{\pgfqpoint{2.471357in}{1.122470in}}%
\pgfpathlineto{\pgfqpoint{2.473280in}{1.130037in}}%
\pgfpathlineto{\pgfqpoint{2.477127in}{1.133851in}}%
\pgfpathlineto{\pgfqpoint{2.479050in}{1.143823in}}%
\pgfpathlineto{\pgfqpoint{2.480974in}{1.140182in}}%
\pgfpathlineto{\pgfqpoint{2.482897in}{1.140797in}}%
\pgfpathlineto{\pgfqpoint{2.486744in}{1.127943in}}%
\pgfpathlineto{\pgfqpoint{2.488667in}{1.134697in}}%
\pgfpathlineto{\pgfqpoint{2.490591in}{1.119196in}}%
\pgfpathlineto{\pgfqpoint{2.492514in}{1.113705in}}%
\pgfpathlineto{\pgfqpoint{2.494438in}{1.115535in}}%
\pgfpathlineto{\pgfqpoint{2.496361in}{1.109430in}}%
\pgfpathlineto{\pgfqpoint{2.498284in}{1.107546in}}%
\pgfpathlineto{\pgfqpoint{2.500208in}{1.103031in}}%
\pgfpathlineto{\pgfqpoint{2.502131in}{1.113232in}}%
\pgfpathlineto{\pgfqpoint{2.504055in}{1.112342in}}%
\pgfpathlineto{\pgfqpoint{2.507901in}{1.120093in}}%
\pgfpathlineto{\pgfqpoint{2.509825in}{1.122417in}}%
\pgfpathlineto{\pgfqpoint{2.511748in}{1.132125in}}%
\pgfpathlineto{\pgfqpoint{2.515595in}{1.108879in}}%
\pgfpathlineto{\pgfqpoint{2.519442in}{1.121448in}}%
\pgfpathlineto{\pgfqpoint{2.521365in}{1.108290in}}%
\pgfpathlineto{\pgfqpoint{2.525212in}{1.096956in}}%
\pgfpathlineto{\pgfqpoint{2.527136in}{1.102321in}}%
\pgfpathlineto{\pgfqpoint{2.529059in}{1.103077in}}%
\pgfpathlineto{\pgfqpoint{2.530982in}{1.105872in}}%
\pgfpathlineto{\pgfqpoint{2.532906in}{1.091169in}}%
\pgfpathlineto{\pgfqpoint{2.534829in}{1.098533in}}%
\pgfpathlineto{\pgfqpoint{2.538676in}{1.095797in}}%
\pgfpathlineto{\pgfqpoint{2.542523in}{1.090070in}}%
\pgfpathlineto{\pgfqpoint{2.544446in}{1.092398in}}%
\pgfpathlineto{\pgfqpoint{2.548293in}{1.064116in}}%
\pgfpathlineto{\pgfqpoint{2.550217in}{1.060638in}}%
\pgfpathlineto{\pgfqpoint{2.554063in}{1.058164in}}%
\pgfpathlineto{\pgfqpoint{2.555987in}{1.051415in}}%
\pgfpathlineto{\pgfqpoint{2.557910in}{1.057885in}}%
\pgfpathlineto{\pgfqpoint{2.559834in}{1.070762in}}%
\pgfpathlineto{\pgfqpoint{2.561757in}{1.067681in}}%
\pgfpathlineto{\pgfqpoint{2.563680in}{1.084570in}}%
\pgfpathlineto{\pgfqpoint{2.565604in}{1.087881in}}%
\pgfpathlineto{\pgfqpoint{2.569451in}{1.076292in}}%
\pgfpathlineto{\pgfqpoint{2.571374in}{1.080502in}}%
\pgfpathlineto{\pgfqpoint{2.573297in}{1.076116in}}%
\pgfpathlineto{\pgfqpoint{2.577144in}{1.086055in}}%
\pgfpathlineto{\pgfqpoint{2.580991in}{1.068388in}}%
\pgfpathlineto{\pgfqpoint{2.582915in}{1.067724in}}%
\pgfpathlineto{\pgfqpoint{2.584838in}{1.071312in}}%
\pgfpathlineto{\pgfqpoint{2.586761in}{1.066239in}}%
\pgfpathlineto{\pgfqpoint{2.588685in}{1.066337in}}%
\pgfpathlineto{\pgfqpoint{2.590608in}{1.072056in}}%
\pgfpathlineto{\pgfqpoint{2.592532in}{1.065944in}}%
\pgfpathlineto{\pgfqpoint{2.594455in}{1.087176in}}%
\pgfpathlineto{\pgfqpoint{2.598302in}{1.087513in}}%
\pgfpathlineto{\pgfqpoint{2.604072in}{1.111425in}}%
\pgfpathlineto{\pgfqpoint{2.607919in}{1.111827in}}%
\pgfpathlineto{\pgfqpoint{2.609842in}{1.112338in}}%
\pgfpathlineto{\pgfqpoint{2.613689in}{1.132945in}}%
\pgfpathlineto{\pgfqpoint{2.615613in}{1.134022in}}%
\pgfpathlineto{\pgfqpoint{2.617536in}{1.144410in}}%
\pgfpathlineto{\pgfqpoint{2.621383in}{1.136412in}}%
\pgfpathlineto{\pgfqpoint{2.623306in}{1.129986in}}%
\pgfpathlineto{\pgfqpoint{2.625230in}{1.129950in}}%
\pgfpathlineto{\pgfqpoint{2.627153in}{1.121873in}}%
\pgfpathlineto{\pgfqpoint{2.631000in}{1.137350in}}%
\pgfpathlineto{\pgfqpoint{2.632923in}{1.149115in}}%
\pgfpathlineto{\pgfqpoint{2.634847in}{1.150418in}}%
\pgfpathlineto{\pgfqpoint{2.638693in}{1.173745in}}%
\pgfpathlineto{\pgfqpoint{2.642540in}{1.169756in}}%
\pgfpathlineto{\pgfqpoint{2.644464in}{1.176896in}}%
\pgfpathlineto{\pgfqpoint{2.646387in}{1.176689in}}%
\pgfpathlineto{\pgfqpoint{2.648311in}{1.167514in}}%
\pgfpathlineto{\pgfqpoint{2.650234in}{1.168790in}}%
\pgfpathlineto{\pgfqpoint{2.652157in}{1.176908in}}%
\pgfpathlineto{\pgfqpoint{2.654081in}{1.174235in}}%
\pgfpathlineto{\pgfqpoint{2.656004in}{1.174796in}}%
\pgfpathlineto{\pgfqpoint{2.657928in}{1.173833in}}%
\pgfpathlineto{\pgfqpoint{2.659851in}{1.176836in}}%
\pgfpathlineto{\pgfqpoint{2.663698in}{1.171299in}}%
\pgfpathlineto{\pgfqpoint{2.665621in}{1.169868in}}%
\pgfpathlineto{\pgfqpoint{2.669468in}{1.155153in}}%
\pgfpathlineto{\pgfqpoint{2.671391in}{1.167843in}}%
\pgfpathlineto{\pgfqpoint{2.673315in}{1.161290in}}%
\pgfpathlineto{\pgfqpoint{2.675238in}{1.167369in}}%
\pgfpathlineto{\pgfqpoint{2.677162in}{1.163330in}}%
\pgfpathlineto{\pgfqpoint{2.679085in}{1.152802in}}%
\pgfpathlineto{\pgfqpoint{2.681008in}{1.155805in}}%
\pgfpathlineto{\pgfqpoint{2.682932in}{1.161219in}}%
\pgfpathlineto{\pgfqpoint{2.684855in}{1.159127in}}%
\pgfpathlineto{\pgfqpoint{2.686779in}{1.162202in}}%
\pgfpathlineto{\pgfqpoint{2.688702in}{1.158768in}}%
\pgfpathlineto{\pgfqpoint{2.692549in}{1.144508in}}%
\pgfpathlineto{\pgfqpoint{2.694472in}{1.142439in}}%
\pgfpathlineto{\pgfqpoint{2.696396in}{1.129570in}}%
\pgfpathlineto{\pgfqpoint{2.702166in}{1.144915in}}%
\pgfpathlineto{\pgfqpoint{2.706013in}{1.136882in}}%
\pgfpathlineto{\pgfqpoint{2.707936in}{1.138970in}}%
\pgfpathlineto{\pgfqpoint{2.709860in}{1.127798in}}%
\pgfpathlineto{\pgfqpoint{2.711783in}{1.127388in}}%
\pgfpathlineto{\pgfqpoint{2.713706in}{1.123195in}}%
\pgfpathlineto{\pgfqpoint{2.715630in}{1.125389in}}%
\pgfpathlineto{\pgfqpoint{2.717553in}{1.123275in}}%
\pgfpathlineto{\pgfqpoint{2.719477in}{1.126872in}}%
\pgfpathlineto{\pgfqpoint{2.721400in}{1.124372in}}%
\pgfpathlineto{\pgfqpoint{2.723324in}{1.125959in}}%
\pgfpathlineto{\pgfqpoint{2.725247in}{1.130240in}}%
\pgfpathlineto{\pgfqpoint{2.727170in}{1.124619in}}%
\pgfpathlineto{\pgfqpoint{2.729094in}{1.128193in}}%
\pgfpathlineto{\pgfqpoint{2.731017in}{1.134234in}}%
\pgfpathlineto{\pgfqpoint{2.732941in}{1.136205in}}%
\pgfpathlineto{\pgfqpoint{2.734864in}{1.133120in}}%
\pgfpathlineto{\pgfqpoint{2.736787in}{1.138042in}}%
\pgfpathlineto{\pgfqpoint{2.738711in}{1.138234in}}%
\pgfpathlineto{\pgfqpoint{2.740634in}{1.135600in}}%
\pgfpathlineto{\pgfqpoint{2.744481in}{1.144544in}}%
\pgfpathlineto{\pgfqpoint{2.746404in}{1.138335in}}%
\pgfpathlineto{\pgfqpoint{2.750251in}{1.153184in}}%
\pgfpathlineto{\pgfqpoint{2.752175in}{1.155646in}}%
\pgfpathlineto{\pgfqpoint{2.756022in}{1.139541in}}%
\pgfpathlineto{\pgfqpoint{2.759868in}{1.131504in}}%
\pgfpathlineto{\pgfqpoint{2.761792in}{1.131902in}}%
\pgfpathlineto{\pgfqpoint{2.763715in}{1.122344in}}%
\pgfpathlineto{\pgfqpoint{2.767562in}{1.115560in}}%
\pgfpathlineto{\pgfqpoint{2.767562in}{1.115560in}}%
\pgfusepath{stroke}%
\end{pgfscope}%
\begin{pgfscope}%
\pgfpathrectangle{\pgfqpoint{0.750000in}{0.660000in}}{\pgfqpoint{2.113636in}{2.100000in}}%
\pgfusepath{clip}%
\pgfsetroundcap%
\pgfsetroundjoin%
\pgfsetlinewidth{0.602250pt}%
\definecolor{currentstroke}{rgb}{0.301961,0.686275,0.290196}%
\pgfsetstrokecolor{currentstroke}%
\pgfsetdash{}{0pt}%
\pgfpathmoveto{\pgfqpoint{0.846074in}{1.687494in}}%
\pgfpathlineto{\pgfqpoint{0.847998in}{1.688734in}}%
\pgfpathlineto{\pgfqpoint{0.849921in}{1.696788in}}%
\pgfpathlineto{\pgfqpoint{0.855691in}{1.684296in}}%
\pgfpathlineto{\pgfqpoint{0.857615in}{1.704101in}}%
\pgfpathlineto{\pgfqpoint{0.859538in}{1.697581in}}%
\pgfpathlineto{\pgfqpoint{0.861462in}{1.686706in}}%
\pgfpathlineto{\pgfqpoint{0.865308in}{1.686610in}}%
\pgfpathlineto{\pgfqpoint{0.869155in}{1.661319in}}%
\pgfpathlineto{\pgfqpoint{0.871079in}{1.670844in}}%
\pgfpathlineto{\pgfqpoint{0.873002in}{1.674483in}}%
\pgfpathlineto{\pgfqpoint{0.874926in}{1.664136in}}%
\pgfpathlineto{\pgfqpoint{0.876849in}{1.673502in}}%
\pgfpathlineto{\pgfqpoint{0.880696in}{1.669248in}}%
\pgfpathlineto{\pgfqpoint{0.882619in}{1.662658in}}%
\pgfpathlineto{\pgfqpoint{0.884543in}{1.652055in}}%
\pgfpathlineto{\pgfqpoint{0.886466in}{1.657035in}}%
\pgfpathlineto{\pgfqpoint{0.890313in}{1.677986in}}%
\pgfpathlineto{\pgfqpoint{0.892236in}{1.663695in}}%
\pgfpathlineto{\pgfqpoint{0.894160in}{1.672124in}}%
\pgfpathlineto{\pgfqpoint{0.896083in}{1.690830in}}%
\pgfpathlineto{\pgfqpoint{0.899930in}{1.664783in}}%
\pgfpathlineto{\pgfqpoint{0.905700in}{1.662000in}}%
\pgfpathlineto{\pgfqpoint{0.907624in}{1.655768in}}%
\pgfpathlineto{\pgfqpoint{0.909547in}{1.666847in}}%
\pgfpathlineto{\pgfqpoint{0.913394in}{1.657240in}}%
\pgfpathlineto{\pgfqpoint{0.915317in}{1.646462in}}%
\pgfpathlineto{\pgfqpoint{0.917241in}{1.658220in}}%
\pgfpathlineto{\pgfqpoint{0.919164in}{1.652281in}}%
\pgfpathlineto{\pgfqpoint{0.921087in}{1.653732in}}%
\pgfpathlineto{\pgfqpoint{0.923011in}{1.636136in}}%
\pgfpathlineto{\pgfqpoint{0.924934in}{1.638774in}}%
\pgfpathlineto{\pgfqpoint{0.926858in}{1.629267in}}%
\pgfpathlineto{\pgfqpoint{0.930704in}{1.634152in}}%
\pgfpathlineto{\pgfqpoint{0.932628in}{1.624782in}}%
\pgfpathlineto{\pgfqpoint{0.934551in}{1.624210in}}%
\pgfpathlineto{\pgfqpoint{0.936475in}{1.637907in}}%
\pgfpathlineto{\pgfqpoint{0.938398in}{1.639459in}}%
\pgfpathlineto{\pgfqpoint{0.940322in}{1.652130in}}%
\pgfpathlineto{\pgfqpoint{0.942245in}{1.654345in}}%
\pgfpathlineto{\pgfqpoint{0.944168in}{1.643395in}}%
\pgfpathlineto{\pgfqpoint{0.946092in}{1.649965in}}%
\pgfpathlineto{\pgfqpoint{0.948015in}{1.650966in}}%
\pgfpathlineto{\pgfqpoint{0.949939in}{1.655655in}}%
\pgfpathlineto{\pgfqpoint{0.951862in}{1.650229in}}%
\pgfpathlineto{\pgfqpoint{0.955709in}{1.646982in}}%
\pgfpathlineto{\pgfqpoint{0.957632in}{1.651076in}}%
\pgfpathlineto{\pgfqpoint{0.959556in}{1.647723in}}%
\pgfpathlineto{\pgfqpoint{0.961479in}{1.651581in}}%
\pgfpathlineto{\pgfqpoint{0.963402in}{1.672728in}}%
\pgfpathlineto{\pgfqpoint{0.967249in}{1.672834in}}%
\pgfpathlineto{\pgfqpoint{0.969173in}{1.677412in}}%
\pgfpathlineto{\pgfqpoint{0.971096in}{1.686192in}}%
\pgfpathlineto{\pgfqpoint{0.973020in}{1.688771in}}%
\pgfpathlineto{\pgfqpoint{0.976866in}{1.701434in}}%
\pgfpathlineto{\pgfqpoint{0.980713in}{1.682428in}}%
\pgfpathlineto{\pgfqpoint{0.986483in}{1.698631in}}%
\pgfpathlineto{\pgfqpoint{0.988407in}{1.701380in}}%
\pgfpathlineto{\pgfqpoint{0.990330in}{1.692249in}}%
\pgfpathlineto{\pgfqpoint{0.992254in}{1.689625in}}%
\pgfpathlineto{\pgfqpoint{0.994177in}{1.689607in}}%
\pgfpathlineto{\pgfqpoint{0.996100in}{1.696646in}}%
\pgfpathlineto{\pgfqpoint{0.998024in}{1.708206in}}%
\pgfpathlineto{\pgfqpoint{0.999947in}{1.707502in}}%
\pgfpathlineto{\pgfqpoint{1.001871in}{1.703185in}}%
\pgfpathlineto{\pgfqpoint{1.003794in}{1.706024in}}%
\pgfpathlineto{\pgfqpoint{1.005717in}{1.687988in}}%
\pgfpathlineto{\pgfqpoint{1.007641in}{1.689877in}}%
\pgfpathlineto{\pgfqpoint{1.009564in}{1.694433in}}%
\pgfpathlineto{\pgfqpoint{1.011488in}{1.693303in}}%
\pgfpathlineto{\pgfqpoint{1.015335in}{1.676727in}}%
\pgfpathlineto{\pgfqpoint{1.023028in}{1.656918in}}%
\pgfpathlineto{\pgfqpoint{1.024952in}{1.661682in}}%
\pgfpathlineto{\pgfqpoint{1.026875in}{1.669871in}}%
\pgfpathlineto{\pgfqpoint{1.028798in}{1.662173in}}%
\pgfpathlineto{\pgfqpoint{1.030722in}{1.664528in}}%
\pgfpathlineto{\pgfqpoint{1.032645in}{1.664504in}}%
\pgfpathlineto{\pgfqpoint{1.036492in}{1.660192in}}%
\pgfpathlineto{\pgfqpoint{1.040339in}{1.672688in}}%
\pgfpathlineto{\pgfqpoint{1.044186in}{1.662635in}}%
\pgfpathlineto{\pgfqpoint{1.046109in}{1.665891in}}%
\pgfpathlineto{\pgfqpoint{1.048033in}{1.678330in}}%
\pgfpathlineto{\pgfqpoint{1.049956in}{1.677356in}}%
\pgfpathlineto{\pgfqpoint{1.051879in}{1.674224in}}%
\pgfpathlineto{\pgfqpoint{1.055726in}{1.656068in}}%
\pgfpathlineto{\pgfqpoint{1.057650in}{1.652810in}}%
\pgfpathlineto{\pgfqpoint{1.059573in}{1.643422in}}%
\pgfpathlineto{\pgfqpoint{1.061496in}{1.641922in}}%
\pgfpathlineto{\pgfqpoint{1.063420in}{1.645011in}}%
\pgfpathlineto{\pgfqpoint{1.067267in}{1.641846in}}%
\pgfpathlineto{\pgfqpoint{1.069190in}{1.629129in}}%
\pgfpathlineto{\pgfqpoint{1.071113in}{1.627350in}}%
\pgfpathlineto{\pgfqpoint{1.074960in}{1.611134in}}%
\pgfpathlineto{\pgfqpoint{1.078807in}{1.624002in}}%
\pgfpathlineto{\pgfqpoint{1.080731in}{1.628614in}}%
\pgfpathlineto{\pgfqpoint{1.082654in}{1.628280in}}%
\pgfpathlineto{\pgfqpoint{1.084577in}{1.624979in}}%
\pgfpathlineto{\pgfqpoint{1.086501in}{1.626310in}}%
\pgfpathlineto{\pgfqpoint{1.088424in}{1.622500in}}%
\pgfpathlineto{\pgfqpoint{1.090348in}{1.608867in}}%
\pgfpathlineto{\pgfqpoint{1.092271in}{1.621324in}}%
\pgfpathlineto{\pgfqpoint{1.094194in}{1.600994in}}%
\pgfpathlineto{\pgfqpoint{1.096118in}{1.605148in}}%
\pgfpathlineto{\pgfqpoint{1.098041in}{1.622401in}}%
\pgfpathlineto{\pgfqpoint{1.101888in}{1.614948in}}%
\pgfpathlineto{\pgfqpoint{1.103811in}{1.607409in}}%
\pgfpathlineto{\pgfqpoint{1.105735in}{1.592315in}}%
\pgfpathlineto{\pgfqpoint{1.107658in}{1.591923in}}%
\pgfpathlineto{\pgfqpoint{1.109582in}{1.573564in}}%
\pgfpathlineto{\pgfqpoint{1.111505in}{1.572552in}}%
\pgfpathlineto{\pgfqpoint{1.113429in}{1.573267in}}%
\pgfpathlineto{\pgfqpoint{1.117275in}{1.585654in}}%
\pgfpathlineto{\pgfqpoint{1.119199in}{1.573754in}}%
\pgfpathlineto{\pgfqpoint{1.121122in}{1.581567in}}%
\pgfpathlineto{\pgfqpoint{1.123046in}{1.576664in}}%
\pgfpathlineto{\pgfqpoint{1.126892in}{1.590927in}}%
\pgfpathlineto{\pgfqpoint{1.128816in}{1.583780in}}%
\pgfpathlineto{\pgfqpoint{1.130739in}{1.585832in}}%
\pgfpathlineto{\pgfqpoint{1.132663in}{1.578455in}}%
\pgfpathlineto{\pgfqpoint{1.134586in}{1.563965in}}%
\pgfpathlineto{\pgfqpoint{1.136509in}{1.561668in}}%
\pgfpathlineto{\pgfqpoint{1.138433in}{1.569399in}}%
\pgfpathlineto{\pgfqpoint{1.140356in}{1.559770in}}%
\pgfpathlineto{\pgfqpoint{1.142280in}{1.559860in}}%
\pgfpathlineto{\pgfqpoint{1.144203in}{1.556006in}}%
\pgfpathlineto{\pgfqpoint{1.146126in}{1.567964in}}%
\pgfpathlineto{\pgfqpoint{1.149973in}{1.564170in}}%
\pgfpathlineto{\pgfqpoint{1.151897in}{1.552908in}}%
\pgfpathlineto{\pgfqpoint{1.155744in}{1.573674in}}%
\pgfpathlineto{\pgfqpoint{1.157667in}{1.571721in}}%
\pgfpathlineto{\pgfqpoint{1.159590in}{1.577080in}}%
\pgfpathlineto{\pgfqpoint{1.161514in}{1.577743in}}%
\pgfpathlineto{\pgfqpoint{1.165361in}{1.571250in}}%
\pgfpathlineto{\pgfqpoint{1.167284in}{1.561739in}}%
\pgfpathlineto{\pgfqpoint{1.169207in}{1.570700in}}%
\pgfpathlineto{\pgfqpoint{1.171131in}{1.564790in}}%
\pgfpathlineto{\pgfqpoint{1.174978in}{1.575078in}}%
\pgfpathlineto{\pgfqpoint{1.176901in}{1.570474in}}%
\pgfpathlineto{\pgfqpoint{1.178824in}{1.582922in}}%
\pgfpathlineto{\pgfqpoint{1.182671in}{1.577595in}}%
\pgfpathlineto{\pgfqpoint{1.184595in}{1.592487in}}%
\pgfpathlineto{\pgfqpoint{1.188442in}{1.595319in}}%
\pgfpathlineto{\pgfqpoint{1.190365in}{1.591050in}}%
\pgfpathlineto{\pgfqpoint{1.192288in}{1.582537in}}%
\pgfpathlineto{\pgfqpoint{1.194212in}{1.581391in}}%
\pgfpathlineto{\pgfqpoint{1.196135in}{1.578630in}}%
\pgfpathlineto{\pgfqpoint{1.198059in}{1.570727in}}%
\pgfpathlineto{\pgfqpoint{1.199982in}{1.569294in}}%
\pgfpathlineto{\pgfqpoint{1.203829in}{1.539766in}}%
\pgfpathlineto{\pgfqpoint{1.207676in}{1.530753in}}%
\pgfpathlineto{\pgfqpoint{1.209599in}{1.538518in}}%
\pgfpathlineto{\pgfqpoint{1.211522in}{1.552232in}}%
\pgfpathlineto{\pgfqpoint{1.213446in}{1.556402in}}%
\pgfpathlineto{\pgfqpoint{1.215369in}{1.566255in}}%
\pgfpathlineto{\pgfqpoint{1.217293in}{1.562149in}}%
\pgfpathlineto{\pgfqpoint{1.219216in}{1.565979in}}%
\pgfpathlineto{\pgfqpoint{1.221140in}{1.553610in}}%
\pgfpathlineto{\pgfqpoint{1.223063in}{1.553757in}}%
\pgfpathlineto{\pgfqpoint{1.230757in}{1.538152in}}%
\pgfpathlineto{\pgfqpoint{1.232680in}{1.557082in}}%
\pgfpathlineto{\pgfqpoint{1.234603in}{1.558637in}}%
\pgfpathlineto{\pgfqpoint{1.238450in}{1.537251in}}%
\pgfpathlineto{\pgfqpoint{1.240374in}{1.547034in}}%
\pgfpathlineto{\pgfqpoint{1.242297in}{1.549327in}}%
\pgfpathlineto{\pgfqpoint{1.244220in}{1.547609in}}%
\pgfpathlineto{\pgfqpoint{1.246144in}{1.549754in}}%
\pgfpathlineto{\pgfqpoint{1.249991in}{1.556424in}}%
\pgfpathlineto{\pgfqpoint{1.251914in}{1.557674in}}%
\pgfpathlineto{\pgfqpoint{1.253838in}{1.548943in}}%
\pgfpathlineto{\pgfqpoint{1.255761in}{1.546205in}}%
\pgfpathlineto{\pgfqpoint{1.257684in}{1.529480in}}%
\pgfpathlineto{\pgfqpoint{1.259608in}{1.542371in}}%
\pgfpathlineto{\pgfqpoint{1.261531in}{1.546506in}}%
\pgfpathlineto{\pgfqpoint{1.263455in}{1.540574in}}%
\pgfpathlineto{\pgfqpoint{1.271148in}{1.557857in}}%
\pgfpathlineto{\pgfqpoint{1.274995in}{1.547327in}}%
\pgfpathlineto{\pgfqpoint{1.276918in}{1.554354in}}%
\pgfpathlineto{\pgfqpoint{1.278842in}{1.555753in}}%
\pgfpathlineto{\pgfqpoint{1.280765in}{1.552417in}}%
\pgfpathlineto{\pgfqpoint{1.282689in}{1.555951in}}%
\pgfpathlineto{\pgfqpoint{1.290382in}{1.532713in}}%
\pgfpathlineto{\pgfqpoint{1.292306in}{1.538674in}}%
\pgfpathlineto{\pgfqpoint{1.294229in}{1.529595in}}%
\pgfpathlineto{\pgfqpoint{1.298076in}{1.524317in}}%
\pgfpathlineto{\pgfqpoint{1.301923in}{1.527326in}}%
\pgfpathlineto{\pgfqpoint{1.303846in}{1.523980in}}%
\pgfpathlineto{\pgfqpoint{1.305770in}{1.537996in}}%
\pgfpathlineto{\pgfqpoint{1.307693in}{1.539862in}}%
\pgfpathlineto{\pgfqpoint{1.311540in}{1.525162in}}%
\pgfpathlineto{\pgfqpoint{1.313463in}{1.537764in}}%
\pgfpathlineto{\pgfqpoint{1.315387in}{1.542087in}}%
\pgfpathlineto{\pgfqpoint{1.321157in}{1.569668in}}%
\pgfpathlineto{\pgfqpoint{1.323080in}{1.562776in}}%
\pgfpathlineto{\pgfqpoint{1.325004in}{1.562479in}}%
\pgfpathlineto{\pgfqpoint{1.326927in}{1.553247in}}%
\pgfpathlineto{\pgfqpoint{1.328851in}{1.555490in}}%
\pgfpathlineto{\pgfqpoint{1.332697in}{1.529260in}}%
\pgfpathlineto{\pgfqpoint{1.334621in}{1.521993in}}%
\pgfpathlineto{\pgfqpoint{1.336544in}{1.527745in}}%
\pgfpathlineto{\pgfqpoint{1.338468in}{1.524775in}}%
\pgfpathlineto{\pgfqpoint{1.340391in}{1.529021in}}%
\pgfpathlineto{\pgfqpoint{1.342314in}{1.538684in}}%
\pgfpathlineto{\pgfqpoint{1.346161in}{1.535766in}}%
\pgfpathlineto{\pgfqpoint{1.350008in}{1.552830in}}%
\pgfpathlineto{\pgfqpoint{1.351931in}{1.545843in}}%
\pgfpathlineto{\pgfqpoint{1.353855in}{1.549961in}}%
\pgfpathlineto{\pgfqpoint{1.355778in}{1.543592in}}%
\pgfpathlineto{\pgfqpoint{1.361549in}{1.543282in}}%
\pgfpathlineto{\pgfqpoint{1.365395in}{1.526454in}}%
\pgfpathlineto{\pgfqpoint{1.367319in}{1.532602in}}%
\pgfpathlineto{\pgfqpoint{1.369242in}{1.532878in}}%
\pgfpathlineto{\pgfqpoint{1.371166in}{1.530593in}}%
\pgfpathlineto{\pgfqpoint{1.373089in}{1.523748in}}%
\pgfpathlineto{\pgfqpoint{1.375012in}{1.523428in}}%
\pgfpathlineto{\pgfqpoint{1.376936in}{1.518496in}}%
\pgfpathlineto{\pgfqpoint{1.378859in}{1.520444in}}%
\pgfpathlineto{\pgfqpoint{1.380783in}{1.514498in}}%
\pgfpathlineto{\pgfqpoint{1.382706in}{1.515679in}}%
\pgfpathlineto{\pgfqpoint{1.384629in}{1.510810in}}%
\pgfpathlineto{\pgfqpoint{1.386553in}{1.493620in}}%
\pgfpathlineto{\pgfqpoint{1.390400in}{1.498273in}}%
\pgfpathlineto{\pgfqpoint{1.392323in}{1.495931in}}%
\pgfpathlineto{\pgfqpoint{1.394247in}{1.502716in}}%
\pgfpathlineto{\pgfqpoint{1.401940in}{1.480024in}}%
\pgfpathlineto{\pgfqpoint{1.403864in}{1.470943in}}%
\pgfpathlineto{\pgfqpoint{1.405787in}{1.467737in}}%
\pgfpathlineto{\pgfqpoint{1.407710in}{1.459587in}}%
\pgfpathlineto{\pgfqpoint{1.409634in}{1.459705in}}%
\pgfpathlineto{\pgfqpoint{1.415404in}{1.477342in}}%
\pgfpathlineto{\pgfqpoint{1.417327in}{1.468272in}}%
\pgfpathlineto{\pgfqpoint{1.419251in}{1.483892in}}%
\pgfpathlineto{\pgfqpoint{1.421174in}{1.485331in}}%
\pgfpathlineto{\pgfqpoint{1.423098in}{1.483917in}}%
\pgfpathlineto{\pgfqpoint{1.425021in}{1.498237in}}%
\pgfpathlineto{\pgfqpoint{1.426945in}{1.489477in}}%
\pgfpathlineto{\pgfqpoint{1.428868in}{1.490587in}}%
\pgfpathlineto{\pgfqpoint{1.430791in}{1.484223in}}%
\pgfpathlineto{\pgfqpoint{1.434638in}{1.494966in}}%
\pgfpathlineto{\pgfqpoint{1.436562in}{1.503643in}}%
\pgfpathlineto{\pgfqpoint{1.438485in}{1.503116in}}%
\pgfpathlineto{\pgfqpoint{1.442332in}{1.515023in}}%
\pgfpathlineto{\pgfqpoint{1.444255in}{1.508525in}}%
\pgfpathlineto{\pgfqpoint{1.446179in}{1.513407in}}%
\pgfpathlineto{\pgfqpoint{1.450025in}{1.508146in}}%
\pgfpathlineto{\pgfqpoint{1.451949in}{1.505337in}}%
\pgfpathlineto{\pgfqpoint{1.453872in}{1.509390in}}%
\pgfpathlineto{\pgfqpoint{1.455796in}{1.516284in}}%
\pgfpathlineto{\pgfqpoint{1.457719in}{1.513844in}}%
\pgfpathlineto{\pgfqpoint{1.459642in}{1.524258in}}%
\pgfpathlineto{\pgfqpoint{1.461566in}{1.523117in}}%
\pgfpathlineto{\pgfqpoint{1.465413in}{1.517159in}}%
\pgfpathlineto{\pgfqpoint{1.467336in}{1.506010in}}%
\pgfpathlineto{\pgfqpoint{1.469260in}{1.519402in}}%
\pgfpathlineto{\pgfqpoint{1.471183in}{1.522777in}}%
\pgfpathlineto{\pgfqpoint{1.473106in}{1.523027in}}%
\pgfpathlineto{\pgfqpoint{1.475030in}{1.516652in}}%
\pgfpathlineto{\pgfqpoint{1.476953in}{1.520454in}}%
\pgfpathlineto{\pgfqpoint{1.478877in}{1.533062in}}%
\pgfpathlineto{\pgfqpoint{1.480800in}{1.529766in}}%
\pgfpathlineto{\pgfqpoint{1.482723in}{1.520400in}}%
\pgfpathlineto{\pgfqpoint{1.484647in}{1.523862in}}%
\pgfpathlineto{\pgfqpoint{1.486570in}{1.533762in}}%
\pgfpathlineto{\pgfqpoint{1.488494in}{1.536640in}}%
\pgfpathlineto{\pgfqpoint{1.490417in}{1.544701in}}%
\pgfpathlineto{\pgfqpoint{1.494264in}{1.527942in}}%
\pgfpathlineto{\pgfqpoint{1.496187in}{1.526576in}}%
\pgfpathlineto{\pgfqpoint{1.498111in}{1.543085in}}%
\pgfpathlineto{\pgfqpoint{1.500034in}{1.549618in}}%
\pgfpathlineto{\pgfqpoint{1.501958in}{1.548238in}}%
\pgfpathlineto{\pgfqpoint{1.503881in}{1.533591in}}%
\pgfpathlineto{\pgfqpoint{1.505804in}{1.528850in}}%
\pgfpathlineto{\pgfqpoint{1.507728in}{1.532986in}}%
\pgfpathlineto{\pgfqpoint{1.509651in}{1.542218in}}%
\pgfpathlineto{\pgfqpoint{1.513498in}{1.542456in}}%
\pgfpathlineto{\pgfqpoint{1.517345in}{1.560947in}}%
\pgfpathlineto{\pgfqpoint{1.521192in}{1.547463in}}%
\pgfpathlineto{\pgfqpoint{1.523115in}{1.546885in}}%
\pgfpathlineto{\pgfqpoint{1.525038in}{1.555831in}}%
\pgfpathlineto{\pgfqpoint{1.526962in}{1.554210in}}%
\pgfpathlineto{\pgfqpoint{1.530809in}{1.569128in}}%
\pgfpathlineto{\pgfqpoint{1.532732in}{1.572105in}}%
\pgfpathlineto{\pgfqpoint{1.536579in}{1.555705in}}%
\pgfpathlineto{\pgfqpoint{1.538502in}{1.556949in}}%
\pgfpathlineto{\pgfqpoint{1.546196in}{1.554116in}}%
\pgfpathlineto{\pgfqpoint{1.548119in}{1.563592in}}%
\pgfpathlineto{\pgfqpoint{1.550043in}{1.564458in}}%
\pgfpathlineto{\pgfqpoint{1.551966in}{1.560177in}}%
\pgfpathlineto{\pgfqpoint{1.553890in}{1.574821in}}%
\pgfpathlineto{\pgfqpoint{1.555813in}{1.567528in}}%
\pgfpathlineto{\pgfqpoint{1.557736in}{1.573479in}}%
\pgfpathlineto{\pgfqpoint{1.561583in}{1.590835in}}%
\pgfpathlineto{\pgfqpoint{1.563507in}{1.594560in}}%
\pgfpathlineto{\pgfqpoint{1.565430in}{1.583162in}}%
\pgfpathlineto{\pgfqpoint{1.567354in}{1.584318in}}%
\pgfpathlineto{\pgfqpoint{1.569277in}{1.578039in}}%
\pgfpathlineto{\pgfqpoint{1.573124in}{1.587354in}}%
\pgfpathlineto{\pgfqpoint{1.575047in}{1.591750in}}%
\pgfpathlineto{\pgfqpoint{1.576971in}{1.588088in}}%
\pgfpathlineto{\pgfqpoint{1.578894in}{1.588320in}}%
\pgfpathlineto{\pgfqpoint{1.580817in}{1.585469in}}%
\pgfpathlineto{\pgfqpoint{1.582741in}{1.594195in}}%
\pgfpathlineto{\pgfqpoint{1.584664in}{1.590651in}}%
\pgfpathlineto{\pgfqpoint{1.586588in}{1.591636in}}%
\pgfpathlineto{\pgfqpoint{1.590434in}{1.587196in}}%
\pgfpathlineto{\pgfqpoint{1.592358in}{1.589311in}}%
\pgfpathlineto{\pgfqpoint{1.596205in}{1.577740in}}%
\pgfpathlineto{\pgfqpoint{1.598128in}{1.578232in}}%
\pgfpathlineto{\pgfqpoint{1.600051in}{1.580464in}}%
\pgfpathlineto{\pgfqpoint{1.603898in}{1.591955in}}%
\pgfpathlineto{\pgfqpoint{1.607745in}{1.579598in}}%
\pgfpathlineto{\pgfqpoint{1.609669in}{1.579258in}}%
\pgfpathlineto{\pgfqpoint{1.611592in}{1.572798in}}%
\pgfpathlineto{\pgfqpoint{1.613515in}{1.557094in}}%
\pgfpathlineto{\pgfqpoint{1.615439in}{1.554866in}}%
\pgfpathlineto{\pgfqpoint{1.617362in}{1.559243in}}%
\pgfpathlineto{\pgfqpoint{1.619286in}{1.560312in}}%
\pgfpathlineto{\pgfqpoint{1.621209in}{1.557938in}}%
\pgfpathlineto{\pgfqpoint{1.623132in}{1.560186in}}%
\pgfpathlineto{\pgfqpoint{1.625056in}{1.565888in}}%
\pgfpathlineto{\pgfqpoint{1.626979in}{1.558871in}}%
\pgfpathlineto{\pgfqpoint{1.628903in}{1.565806in}}%
\pgfpathlineto{\pgfqpoint{1.630826in}{1.561880in}}%
\pgfpathlineto{\pgfqpoint{1.632749in}{1.549398in}}%
\pgfpathlineto{\pgfqpoint{1.636596in}{1.539329in}}%
\pgfpathlineto{\pgfqpoint{1.638520in}{1.538400in}}%
\pgfpathlineto{\pgfqpoint{1.640443in}{1.547952in}}%
\pgfpathlineto{\pgfqpoint{1.642367in}{1.549871in}}%
\pgfpathlineto{\pgfqpoint{1.644290in}{1.558993in}}%
\pgfpathlineto{\pgfqpoint{1.648137in}{1.560380in}}%
\pgfpathlineto{\pgfqpoint{1.651984in}{1.576149in}}%
\pgfpathlineto{\pgfqpoint{1.653907in}{1.569877in}}%
\pgfpathlineto{\pgfqpoint{1.655830in}{1.574029in}}%
\pgfpathlineto{\pgfqpoint{1.657754in}{1.573069in}}%
\pgfpathlineto{\pgfqpoint{1.659677in}{1.564368in}}%
\pgfpathlineto{\pgfqpoint{1.661601in}{1.564459in}}%
\pgfpathlineto{\pgfqpoint{1.663524in}{1.570606in}}%
\pgfpathlineto{\pgfqpoint{1.667371in}{1.557566in}}%
\pgfpathlineto{\pgfqpoint{1.669294in}{1.563276in}}%
\pgfpathlineto{\pgfqpoint{1.671218in}{1.564687in}}%
\pgfpathlineto{\pgfqpoint{1.673141in}{1.548012in}}%
\pgfpathlineto{\pgfqpoint{1.675065in}{1.562766in}}%
\pgfpathlineto{\pgfqpoint{1.676988in}{1.552397in}}%
\pgfpathlineto{\pgfqpoint{1.678911in}{1.555183in}}%
\pgfpathlineto{\pgfqpoint{1.680835in}{1.554737in}}%
\pgfpathlineto{\pgfqpoint{1.682758in}{1.563359in}}%
\pgfpathlineto{\pgfqpoint{1.684682in}{1.555567in}}%
\pgfpathlineto{\pgfqpoint{1.686605in}{1.563275in}}%
\pgfpathlineto{\pgfqpoint{1.688528in}{1.553812in}}%
\pgfpathlineto{\pgfqpoint{1.690452in}{1.554941in}}%
\pgfpathlineto{\pgfqpoint{1.692375in}{1.552723in}}%
\pgfpathlineto{\pgfqpoint{1.694299in}{1.564253in}}%
\pgfpathlineto{\pgfqpoint{1.696222in}{1.564401in}}%
\pgfpathlineto{\pgfqpoint{1.698145in}{1.556737in}}%
\pgfpathlineto{\pgfqpoint{1.701992in}{1.560395in}}%
\pgfpathlineto{\pgfqpoint{1.705839in}{1.547868in}}%
\pgfpathlineto{\pgfqpoint{1.707763in}{1.539510in}}%
\pgfpathlineto{\pgfqpoint{1.717380in}{1.539007in}}%
\pgfpathlineto{\pgfqpoint{1.719303in}{1.537299in}}%
\pgfpathlineto{\pgfqpoint{1.721226in}{1.538970in}}%
\pgfpathlineto{\pgfqpoint{1.723150in}{1.519613in}}%
\pgfpathlineto{\pgfqpoint{1.726997in}{1.522162in}}%
\pgfpathlineto{\pgfqpoint{1.728920in}{1.515185in}}%
\pgfpathlineto{\pgfqpoint{1.730843in}{1.521122in}}%
\pgfpathlineto{\pgfqpoint{1.732767in}{1.518345in}}%
\pgfpathlineto{\pgfqpoint{1.734690in}{1.519345in}}%
\pgfpathlineto{\pgfqpoint{1.736614in}{1.517036in}}%
\pgfpathlineto{\pgfqpoint{1.738537in}{1.517555in}}%
\pgfpathlineto{\pgfqpoint{1.742384in}{1.523804in}}%
\pgfpathlineto{\pgfqpoint{1.744307in}{1.541960in}}%
\pgfpathlineto{\pgfqpoint{1.746231in}{1.533209in}}%
\pgfpathlineto{\pgfqpoint{1.748154in}{1.538648in}}%
\pgfpathlineto{\pgfqpoint{1.750078in}{1.539033in}}%
\pgfpathlineto{\pgfqpoint{1.752001in}{1.510217in}}%
\pgfpathlineto{\pgfqpoint{1.755848in}{1.533419in}}%
\pgfpathlineto{\pgfqpoint{1.757771in}{1.529123in}}%
\pgfpathlineto{\pgfqpoint{1.759695in}{1.529655in}}%
\pgfpathlineto{\pgfqpoint{1.761618in}{1.522181in}}%
\pgfpathlineto{\pgfqpoint{1.765465in}{1.530000in}}%
\pgfpathlineto{\pgfqpoint{1.767388in}{1.525224in}}%
\pgfpathlineto{\pgfqpoint{1.769312in}{1.535224in}}%
\pgfpathlineto{\pgfqpoint{1.771235in}{1.523378in}}%
\pgfpathlineto{\pgfqpoint{1.773158in}{1.534966in}}%
\pgfpathlineto{\pgfqpoint{1.775082in}{1.539658in}}%
\pgfpathlineto{\pgfqpoint{1.778929in}{1.556917in}}%
\pgfpathlineto{\pgfqpoint{1.780852in}{1.563266in}}%
\pgfpathlineto{\pgfqpoint{1.782776in}{1.565657in}}%
\pgfpathlineto{\pgfqpoint{1.784699in}{1.571048in}}%
\pgfpathlineto{\pgfqpoint{1.786622in}{1.571491in}}%
\pgfpathlineto{\pgfqpoint{1.788546in}{1.570208in}}%
\pgfpathlineto{\pgfqpoint{1.790469in}{1.573181in}}%
\pgfpathlineto{\pgfqpoint{1.792393in}{1.567295in}}%
\pgfpathlineto{\pgfqpoint{1.794316in}{1.568279in}}%
\pgfpathlineto{\pgfqpoint{1.796239in}{1.565878in}}%
\pgfpathlineto{\pgfqpoint{1.798163in}{1.566820in}}%
\pgfpathlineto{\pgfqpoint{1.800086in}{1.571509in}}%
\pgfpathlineto{\pgfqpoint{1.802010in}{1.571595in}}%
\pgfpathlineto{\pgfqpoint{1.803933in}{1.581381in}}%
\pgfpathlineto{\pgfqpoint{1.805856in}{1.579159in}}%
\pgfpathlineto{\pgfqpoint{1.811627in}{1.565681in}}%
\pgfpathlineto{\pgfqpoint{1.813550in}{1.567927in}}%
\pgfpathlineto{\pgfqpoint{1.815474in}{1.574123in}}%
\pgfpathlineto{\pgfqpoint{1.819320in}{1.572734in}}%
\pgfpathlineto{\pgfqpoint{1.821244in}{1.586544in}}%
\pgfpathlineto{\pgfqpoint{1.823167in}{1.587584in}}%
\pgfpathlineto{\pgfqpoint{1.825091in}{1.594816in}}%
\pgfpathlineto{\pgfqpoint{1.827014in}{1.581580in}}%
\pgfpathlineto{\pgfqpoint{1.828937in}{1.592227in}}%
\pgfpathlineto{\pgfqpoint{1.830861in}{1.589437in}}%
\pgfpathlineto{\pgfqpoint{1.832784in}{1.602235in}}%
\pgfpathlineto{\pgfqpoint{1.836631in}{1.593275in}}%
\pgfpathlineto{\pgfqpoint{1.840478in}{1.618009in}}%
\pgfpathlineto{\pgfqpoint{1.842401in}{1.604262in}}%
\pgfpathlineto{\pgfqpoint{1.844325in}{1.616824in}}%
\pgfpathlineto{\pgfqpoint{1.846248in}{1.611618in}}%
\pgfpathlineto{\pgfqpoint{1.848172in}{1.611424in}}%
\pgfpathlineto{\pgfqpoint{1.850095in}{1.608492in}}%
\pgfpathlineto{\pgfqpoint{1.852018in}{1.603434in}}%
\pgfpathlineto{\pgfqpoint{1.853942in}{1.602596in}}%
\pgfpathlineto{\pgfqpoint{1.855865in}{1.605504in}}%
\pgfpathlineto{\pgfqpoint{1.861635in}{1.579329in}}%
\pgfpathlineto{\pgfqpoint{1.863559in}{1.580776in}}%
\pgfpathlineto{\pgfqpoint{1.865482in}{1.590557in}}%
\pgfpathlineto{\pgfqpoint{1.867406in}{1.590266in}}%
\pgfpathlineto{\pgfqpoint{1.873176in}{1.569266in}}%
\pgfpathlineto{\pgfqpoint{1.875099in}{1.566269in}}%
\pgfpathlineto{\pgfqpoint{1.877023in}{1.574893in}}%
\pgfpathlineto{\pgfqpoint{1.878946in}{1.576630in}}%
\pgfpathlineto{\pgfqpoint{1.880870in}{1.570266in}}%
\pgfpathlineto{\pgfqpoint{1.882793in}{1.572486in}}%
\pgfpathlineto{\pgfqpoint{1.884716in}{1.563144in}}%
\pgfpathlineto{\pgfqpoint{1.888563in}{1.536985in}}%
\pgfpathlineto{\pgfqpoint{1.890487in}{1.540770in}}%
\pgfpathlineto{\pgfqpoint{1.892410in}{1.535195in}}%
\pgfpathlineto{\pgfqpoint{1.894333in}{1.525280in}}%
\pgfpathlineto{\pgfqpoint{1.896257in}{1.524124in}}%
\pgfpathlineto{\pgfqpoint{1.898180in}{1.521241in}}%
\pgfpathlineto{\pgfqpoint{1.900104in}{1.524294in}}%
\pgfpathlineto{\pgfqpoint{1.903950in}{1.520451in}}%
\pgfpathlineto{\pgfqpoint{1.905874in}{1.530279in}}%
\pgfpathlineto{\pgfqpoint{1.907797in}{1.528126in}}%
\pgfpathlineto{\pgfqpoint{1.909721in}{1.515943in}}%
\pgfpathlineto{\pgfqpoint{1.913567in}{1.530826in}}%
\pgfpathlineto{\pgfqpoint{1.915491in}{1.532969in}}%
\pgfpathlineto{\pgfqpoint{1.917414in}{1.519684in}}%
\pgfpathlineto{\pgfqpoint{1.919338in}{1.524556in}}%
\pgfpathlineto{\pgfqpoint{1.921261in}{1.516508in}}%
\pgfpathlineto{\pgfqpoint{1.923185in}{1.515907in}}%
\pgfpathlineto{\pgfqpoint{1.925108in}{1.516754in}}%
\pgfpathlineto{\pgfqpoint{1.927031in}{1.521455in}}%
\pgfpathlineto{\pgfqpoint{1.930878in}{1.499926in}}%
\pgfpathlineto{\pgfqpoint{1.932802in}{1.500760in}}%
\pgfpathlineto{\pgfqpoint{1.936648in}{1.496152in}}%
\pgfpathlineto{\pgfqpoint{1.938572in}{1.498692in}}%
\pgfpathlineto{\pgfqpoint{1.940495in}{1.512537in}}%
\pgfpathlineto{\pgfqpoint{1.942419in}{1.499414in}}%
\pgfpathlineto{\pgfqpoint{1.944342in}{1.504581in}}%
\pgfpathlineto{\pgfqpoint{1.946265in}{1.515819in}}%
\pgfpathlineto{\pgfqpoint{1.948189in}{1.505140in}}%
\pgfpathlineto{\pgfqpoint{1.950112in}{1.507530in}}%
\pgfpathlineto{\pgfqpoint{1.952036in}{1.504171in}}%
\pgfpathlineto{\pgfqpoint{1.953959in}{1.513912in}}%
\pgfpathlineto{\pgfqpoint{1.955883in}{1.508084in}}%
\pgfpathlineto{\pgfqpoint{1.957806in}{1.516099in}}%
\pgfpathlineto{\pgfqpoint{1.959729in}{1.503834in}}%
\pgfpathlineto{\pgfqpoint{1.961653in}{1.504025in}}%
\pgfpathlineto{\pgfqpoint{1.963576in}{1.521020in}}%
\pgfpathlineto{\pgfqpoint{1.965500in}{1.517884in}}%
\pgfpathlineto{\pgfqpoint{1.967423in}{1.511564in}}%
\pgfpathlineto{\pgfqpoint{1.969346in}{1.512389in}}%
\pgfpathlineto{\pgfqpoint{1.971270in}{1.514461in}}%
\pgfpathlineto{\pgfqpoint{1.975117in}{1.498594in}}%
\pgfpathlineto{\pgfqpoint{1.977040in}{1.496106in}}%
\pgfpathlineto{\pgfqpoint{1.978963in}{1.487826in}}%
\pgfpathlineto{\pgfqpoint{1.980887in}{1.486887in}}%
\pgfpathlineto{\pgfqpoint{1.982810in}{1.496642in}}%
\pgfpathlineto{\pgfqpoint{1.984734in}{1.496814in}}%
\pgfpathlineto{\pgfqpoint{1.986657in}{1.499327in}}%
\pgfpathlineto{\pgfqpoint{1.988581in}{1.495841in}}%
\pgfpathlineto{\pgfqpoint{1.990504in}{1.503380in}}%
\pgfpathlineto{\pgfqpoint{1.994351in}{1.507733in}}%
\pgfpathlineto{\pgfqpoint{1.998198in}{1.497504in}}%
\pgfpathlineto{\pgfqpoint{2.000121in}{1.501607in}}%
\pgfpathlineto{\pgfqpoint{2.003968in}{1.477096in}}%
\pgfpathlineto{\pgfqpoint{2.011661in}{1.497321in}}%
\pgfpathlineto{\pgfqpoint{2.013585in}{1.493046in}}%
\pgfpathlineto{\pgfqpoint{2.015508in}{1.494889in}}%
\pgfpathlineto{\pgfqpoint{2.017432in}{1.494385in}}%
\pgfpathlineto{\pgfqpoint{2.019355in}{1.489643in}}%
\pgfpathlineto{\pgfqpoint{2.025125in}{1.522241in}}%
\pgfpathlineto{\pgfqpoint{2.027049in}{1.513820in}}%
\pgfpathlineto{\pgfqpoint{2.028972in}{1.511714in}}%
\pgfpathlineto{\pgfqpoint{2.030896in}{1.515541in}}%
\pgfpathlineto{\pgfqpoint{2.032819in}{1.530832in}}%
\pgfpathlineto{\pgfqpoint{2.034742in}{1.529328in}}%
\pgfpathlineto{\pgfqpoint{2.036666in}{1.526056in}}%
\pgfpathlineto{\pgfqpoint{2.038589in}{1.548114in}}%
\pgfpathlineto{\pgfqpoint{2.040513in}{1.547179in}}%
\pgfpathlineto{\pgfqpoint{2.042436in}{1.541686in}}%
\pgfpathlineto{\pgfqpoint{2.044359in}{1.540997in}}%
\pgfpathlineto{\pgfqpoint{2.046283in}{1.535008in}}%
\pgfpathlineto{\pgfqpoint{2.048206in}{1.543505in}}%
\pgfpathlineto{\pgfqpoint{2.050130in}{1.537945in}}%
\pgfpathlineto{\pgfqpoint{2.052053in}{1.541554in}}%
\pgfpathlineto{\pgfqpoint{2.053976in}{1.541135in}}%
\pgfpathlineto{\pgfqpoint{2.057823in}{1.556099in}}%
\pgfpathlineto{\pgfqpoint{2.059747in}{1.553378in}}%
\pgfpathlineto{\pgfqpoint{2.063594in}{1.539413in}}%
\pgfpathlineto{\pgfqpoint{2.065517in}{1.537629in}}%
\pgfpathlineto{\pgfqpoint{2.067440in}{1.542592in}}%
\pgfpathlineto{\pgfqpoint{2.071287in}{1.537818in}}%
\pgfpathlineto{\pgfqpoint{2.075134in}{1.547614in}}%
\pgfpathlineto{\pgfqpoint{2.077057in}{1.548851in}}%
\pgfpathlineto{\pgfqpoint{2.080904in}{1.529753in}}%
\pgfpathlineto{\pgfqpoint{2.082828in}{1.529247in}}%
\pgfpathlineto{\pgfqpoint{2.086674in}{1.514529in}}%
\pgfpathlineto{\pgfqpoint{2.088598in}{1.514005in}}%
\pgfpathlineto{\pgfqpoint{2.090521in}{1.515406in}}%
\pgfpathlineto{\pgfqpoint{2.092445in}{1.527572in}}%
\pgfpathlineto{\pgfqpoint{2.094368in}{1.532436in}}%
\pgfpathlineto{\pgfqpoint{2.096292in}{1.527803in}}%
\pgfpathlineto{\pgfqpoint{2.100138in}{1.510770in}}%
\pgfpathlineto{\pgfqpoint{2.102062in}{1.517115in}}%
\pgfpathlineto{\pgfqpoint{2.103985in}{1.507612in}}%
\pgfpathlineto{\pgfqpoint{2.105909in}{1.518874in}}%
\pgfpathlineto{\pgfqpoint{2.107832in}{1.521630in}}%
\pgfpathlineto{\pgfqpoint{2.109755in}{1.515641in}}%
\pgfpathlineto{\pgfqpoint{2.111679in}{1.516514in}}%
\pgfpathlineto{\pgfqpoint{2.115526in}{1.503219in}}%
\pgfpathlineto{\pgfqpoint{2.117449in}{1.501018in}}%
\pgfpathlineto{\pgfqpoint{2.123219in}{1.532166in}}%
\pgfpathlineto{\pgfqpoint{2.125143in}{1.535781in}}%
\pgfpathlineto{\pgfqpoint{2.127066in}{1.536481in}}%
\pgfpathlineto{\pgfqpoint{2.128990in}{1.529739in}}%
\pgfpathlineto{\pgfqpoint{2.130913in}{1.550538in}}%
\pgfpathlineto{\pgfqpoint{2.132836in}{1.542812in}}%
\pgfpathlineto{\pgfqpoint{2.134760in}{1.552622in}}%
\pgfpathlineto{\pgfqpoint{2.136683in}{1.544284in}}%
\pgfpathlineto{\pgfqpoint{2.138607in}{1.544285in}}%
\pgfpathlineto{\pgfqpoint{2.140530in}{1.539549in}}%
\pgfpathlineto{\pgfqpoint{2.142453in}{1.551148in}}%
\pgfpathlineto{\pgfqpoint{2.144377in}{1.546414in}}%
\pgfpathlineto{\pgfqpoint{2.146300in}{1.545024in}}%
\pgfpathlineto{\pgfqpoint{2.150147in}{1.535841in}}%
\pgfpathlineto{\pgfqpoint{2.153994in}{1.502943in}}%
\pgfpathlineto{\pgfqpoint{2.157841in}{1.515116in}}%
\pgfpathlineto{\pgfqpoint{2.159764in}{1.515346in}}%
\pgfpathlineto{\pgfqpoint{2.161688in}{1.513715in}}%
\pgfpathlineto{\pgfqpoint{2.163611in}{1.504467in}}%
\pgfpathlineto{\pgfqpoint{2.165534in}{1.514143in}}%
\pgfpathlineto{\pgfqpoint{2.169381in}{1.503042in}}%
\pgfpathlineto{\pgfqpoint{2.171305in}{1.515166in}}%
\pgfpathlineto{\pgfqpoint{2.173228in}{1.518711in}}%
\pgfpathlineto{\pgfqpoint{2.175151in}{1.509749in}}%
\pgfpathlineto{\pgfqpoint{2.177075in}{1.522478in}}%
\pgfpathlineto{\pgfqpoint{2.180922in}{1.503144in}}%
\pgfpathlineto{\pgfqpoint{2.182845in}{1.506417in}}%
\pgfpathlineto{\pgfqpoint{2.184768in}{1.506431in}}%
\pgfpathlineto{\pgfqpoint{2.186692in}{1.507912in}}%
\pgfpathlineto{\pgfqpoint{2.192462in}{1.482599in}}%
\pgfpathlineto{\pgfqpoint{2.194386in}{1.476599in}}%
\pgfpathlineto{\pgfqpoint{2.196309in}{1.474914in}}%
\pgfpathlineto{\pgfqpoint{2.198232in}{1.469764in}}%
\pgfpathlineto{\pgfqpoint{2.200156in}{1.469426in}}%
\pgfpathlineto{\pgfqpoint{2.202079in}{1.472910in}}%
\pgfpathlineto{\pgfqpoint{2.204003in}{1.470462in}}%
\pgfpathlineto{\pgfqpoint{2.207849in}{1.459231in}}%
\pgfpathlineto{\pgfqpoint{2.211696in}{1.455267in}}%
\pgfpathlineto{\pgfqpoint{2.213620in}{1.456564in}}%
\pgfpathlineto{\pgfqpoint{2.217466in}{1.456870in}}%
\pgfpathlineto{\pgfqpoint{2.219390in}{1.462796in}}%
\pgfpathlineto{\pgfqpoint{2.221313in}{1.461031in}}%
\pgfpathlineto{\pgfqpoint{2.225160in}{1.445607in}}%
\pgfpathlineto{\pgfqpoint{2.229007in}{1.440731in}}%
\pgfpathlineto{\pgfqpoint{2.230930in}{1.451634in}}%
\pgfpathlineto{\pgfqpoint{2.232854in}{1.448578in}}%
\pgfpathlineto{\pgfqpoint{2.234777in}{1.449803in}}%
\pgfpathlineto{\pgfqpoint{2.236701in}{1.445155in}}%
\pgfpathlineto{\pgfqpoint{2.238624in}{1.446126in}}%
\pgfpathlineto{\pgfqpoint{2.240547in}{1.448994in}}%
\pgfpathlineto{\pgfqpoint{2.242471in}{1.435716in}}%
\pgfpathlineto{\pgfqpoint{2.246318in}{1.445760in}}%
\pgfpathlineto{\pgfqpoint{2.250164in}{1.463986in}}%
\pgfpathlineto{\pgfqpoint{2.254011in}{1.456782in}}%
\pgfpathlineto{\pgfqpoint{2.255935in}{1.460789in}}%
\pgfpathlineto{\pgfqpoint{2.257858in}{1.457207in}}%
\pgfpathlineto{\pgfqpoint{2.259781in}{1.450515in}}%
\pgfpathlineto{\pgfqpoint{2.263628in}{1.463099in}}%
\pgfpathlineto{\pgfqpoint{2.265552in}{1.455251in}}%
\pgfpathlineto{\pgfqpoint{2.267475in}{1.460436in}}%
\pgfpathlineto{\pgfqpoint{2.269399in}{1.443651in}}%
\pgfpathlineto{\pgfqpoint{2.271322in}{1.437016in}}%
\pgfpathlineto{\pgfqpoint{2.273245in}{1.436007in}}%
\pgfpathlineto{\pgfqpoint{2.275169in}{1.439037in}}%
\pgfpathlineto{\pgfqpoint{2.277092in}{1.428741in}}%
\pgfpathlineto{\pgfqpoint{2.280939in}{1.439262in}}%
\pgfpathlineto{\pgfqpoint{2.282862in}{1.450271in}}%
\pgfpathlineto{\pgfqpoint{2.284786in}{1.445828in}}%
\pgfpathlineto{\pgfqpoint{2.288633in}{1.453181in}}%
\pgfpathlineto{\pgfqpoint{2.290556in}{1.435813in}}%
\pgfpathlineto{\pgfqpoint{2.294403in}{1.424158in}}%
\pgfpathlineto{\pgfqpoint{2.296326in}{1.426501in}}%
\pgfpathlineto{\pgfqpoint{2.298250in}{1.417604in}}%
\pgfpathlineto{\pgfqpoint{2.300173in}{1.424064in}}%
\pgfpathlineto{\pgfqpoint{2.302097in}{1.426048in}}%
\pgfpathlineto{\pgfqpoint{2.304020in}{1.432885in}}%
\pgfpathlineto{\pgfqpoint{2.305943in}{1.432087in}}%
\pgfpathlineto{\pgfqpoint{2.307867in}{1.436639in}}%
\pgfpathlineto{\pgfqpoint{2.309790in}{1.420278in}}%
\pgfpathlineto{\pgfqpoint{2.311714in}{1.421155in}}%
\pgfpathlineto{\pgfqpoint{2.313637in}{1.414497in}}%
\pgfpathlineto{\pgfqpoint{2.317484in}{1.441559in}}%
\pgfpathlineto{\pgfqpoint{2.319407in}{1.437620in}}%
\pgfpathlineto{\pgfqpoint{2.321331in}{1.437164in}}%
\pgfpathlineto{\pgfqpoint{2.323254in}{1.433692in}}%
\pgfpathlineto{\pgfqpoint{2.325177in}{1.440280in}}%
\pgfpathlineto{\pgfqpoint{2.327101in}{1.452928in}}%
\pgfpathlineto{\pgfqpoint{2.329024in}{1.452895in}}%
\pgfpathlineto{\pgfqpoint{2.330948in}{1.435563in}}%
\pgfpathlineto{\pgfqpoint{2.332871in}{1.434847in}}%
\pgfpathlineto{\pgfqpoint{2.334795in}{1.446713in}}%
\pgfpathlineto{\pgfqpoint{2.336718in}{1.447682in}}%
\pgfpathlineto{\pgfqpoint{2.340565in}{1.461700in}}%
\pgfpathlineto{\pgfqpoint{2.348258in}{1.443850in}}%
\pgfpathlineto{\pgfqpoint{2.352105in}{1.423659in}}%
\pgfpathlineto{\pgfqpoint{2.354029in}{1.443738in}}%
\pgfpathlineto{\pgfqpoint{2.355952in}{1.450807in}}%
\pgfpathlineto{\pgfqpoint{2.359799in}{1.441638in}}%
\pgfpathlineto{\pgfqpoint{2.361722in}{1.451271in}}%
\pgfpathlineto{\pgfqpoint{2.363646in}{1.441692in}}%
\pgfpathlineto{\pgfqpoint{2.365569in}{1.448334in}}%
\pgfpathlineto{\pgfqpoint{2.367492in}{1.450274in}}%
\pgfpathlineto{\pgfqpoint{2.371339in}{1.469486in}}%
\pgfpathlineto{\pgfqpoint{2.373263in}{1.468742in}}%
\pgfpathlineto{\pgfqpoint{2.375186in}{1.466317in}}%
\pgfpathlineto{\pgfqpoint{2.380956in}{1.447189in}}%
\pgfpathlineto{\pgfqpoint{2.382880in}{1.448550in}}%
\pgfpathlineto{\pgfqpoint{2.384803in}{1.452447in}}%
\pgfpathlineto{\pgfqpoint{2.386727in}{1.450904in}}%
\pgfpathlineto{\pgfqpoint{2.388650in}{1.444559in}}%
\pgfpathlineto{\pgfqpoint{2.390573in}{1.444902in}}%
\pgfpathlineto{\pgfqpoint{2.392497in}{1.448527in}}%
\pgfpathlineto{\pgfqpoint{2.394420in}{1.433958in}}%
\pgfpathlineto{\pgfqpoint{2.396344in}{1.432191in}}%
\pgfpathlineto{\pgfqpoint{2.398267in}{1.419063in}}%
\pgfpathlineto{\pgfqpoint{2.400190in}{1.426951in}}%
\pgfpathlineto{\pgfqpoint{2.402114in}{1.425298in}}%
\pgfpathlineto{\pgfqpoint{2.405961in}{1.439316in}}%
\pgfpathlineto{\pgfqpoint{2.411731in}{1.446292in}}%
\pgfpathlineto{\pgfqpoint{2.415578in}{1.409485in}}%
\pgfpathlineto{\pgfqpoint{2.417501in}{1.415079in}}%
\pgfpathlineto{\pgfqpoint{2.419425in}{1.408564in}}%
\pgfpathlineto{\pgfqpoint{2.421348in}{1.412444in}}%
\pgfpathlineto{\pgfqpoint{2.425195in}{1.431867in}}%
\pgfpathlineto{\pgfqpoint{2.427118in}{1.418696in}}%
\pgfpathlineto{\pgfqpoint{2.429042in}{1.418040in}}%
\pgfpathlineto{\pgfqpoint{2.430965in}{1.426288in}}%
\pgfpathlineto{\pgfqpoint{2.432888in}{1.413456in}}%
\pgfpathlineto{\pgfqpoint{2.436735in}{1.423564in}}%
\pgfpathlineto{\pgfqpoint{2.438659in}{1.426443in}}%
\pgfpathlineto{\pgfqpoint{2.440582in}{1.417885in}}%
\pgfpathlineto{\pgfqpoint{2.442506in}{1.419460in}}%
\pgfpathlineto{\pgfqpoint{2.444429in}{1.412275in}}%
\pgfpathlineto{\pgfqpoint{2.446352in}{1.416748in}}%
\pgfpathlineto{\pgfqpoint{2.448276in}{1.413998in}}%
\pgfpathlineto{\pgfqpoint{2.450199in}{1.424883in}}%
\pgfpathlineto{\pgfqpoint{2.452123in}{1.425591in}}%
\pgfpathlineto{\pgfqpoint{2.454046in}{1.428398in}}%
\pgfpathlineto{\pgfqpoint{2.455969in}{1.427800in}}%
\pgfpathlineto{\pgfqpoint{2.457893in}{1.437158in}}%
\pgfpathlineto{\pgfqpoint{2.461740in}{1.423899in}}%
\pgfpathlineto{\pgfqpoint{2.467510in}{1.445073in}}%
\pgfpathlineto{\pgfqpoint{2.469433in}{1.446222in}}%
\pgfpathlineto{\pgfqpoint{2.471357in}{1.444352in}}%
\pgfpathlineto{\pgfqpoint{2.473280in}{1.451958in}}%
\pgfpathlineto{\pgfqpoint{2.475204in}{1.450870in}}%
\pgfpathlineto{\pgfqpoint{2.477127in}{1.455010in}}%
\pgfpathlineto{\pgfqpoint{2.480974in}{1.442920in}}%
\pgfpathlineto{\pgfqpoint{2.482897in}{1.459849in}}%
\pgfpathlineto{\pgfqpoint{2.484821in}{1.457699in}}%
\pgfpathlineto{\pgfqpoint{2.486744in}{1.466484in}}%
\pgfpathlineto{\pgfqpoint{2.490591in}{1.465519in}}%
\pgfpathlineto{\pgfqpoint{2.492514in}{1.461060in}}%
\pgfpathlineto{\pgfqpoint{2.498284in}{1.475199in}}%
\pgfpathlineto{\pgfqpoint{2.500208in}{1.473780in}}%
\pgfpathlineto{\pgfqpoint{2.502131in}{1.478110in}}%
\pgfpathlineto{\pgfqpoint{2.504055in}{1.467988in}}%
\pgfpathlineto{\pgfqpoint{2.505978in}{1.483679in}}%
\pgfpathlineto{\pgfqpoint{2.511748in}{1.496011in}}%
\pgfpathlineto{\pgfqpoint{2.513672in}{1.486508in}}%
\pgfpathlineto{\pgfqpoint{2.515595in}{1.510336in}}%
\pgfpathlineto{\pgfqpoint{2.517519in}{1.512778in}}%
\pgfpathlineto{\pgfqpoint{2.519442in}{1.511865in}}%
\pgfpathlineto{\pgfqpoint{2.523289in}{1.495928in}}%
\pgfpathlineto{\pgfqpoint{2.527136in}{1.501261in}}%
\pgfpathlineto{\pgfqpoint{2.530982in}{1.478568in}}%
\pgfpathlineto{\pgfqpoint{2.532906in}{1.475290in}}%
\pgfpathlineto{\pgfqpoint{2.534829in}{1.476930in}}%
\pgfpathlineto{\pgfqpoint{2.536753in}{1.476425in}}%
\pgfpathlineto{\pgfqpoint{2.538676in}{1.479588in}}%
\pgfpathlineto{\pgfqpoint{2.540599in}{1.478236in}}%
\pgfpathlineto{\pgfqpoint{2.544446in}{1.467322in}}%
\pgfpathlineto{\pgfqpoint{2.546370in}{1.467610in}}%
\pgfpathlineto{\pgfqpoint{2.550217in}{1.477656in}}%
\pgfpathlineto{\pgfqpoint{2.552140in}{1.480322in}}%
\pgfpathlineto{\pgfqpoint{2.555987in}{1.473598in}}%
\pgfpathlineto{\pgfqpoint{2.557910in}{1.477647in}}%
\pgfpathlineto{\pgfqpoint{2.559834in}{1.456880in}}%
\pgfpathlineto{\pgfqpoint{2.563680in}{1.456350in}}%
\pgfpathlineto{\pgfqpoint{2.565604in}{1.441023in}}%
\pgfpathlineto{\pgfqpoint{2.569451in}{1.469051in}}%
\pgfpathlineto{\pgfqpoint{2.571374in}{1.465375in}}%
\pgfpathlineto{\pgfqpoint{2.573297in}{1.469816in}}%
\pgfpathlineto{\pgfqpoint{2.575221in}{1.461236in}}%
\pgfpathlineto{\pgfqpoint{2.577144in}{1.466780in}}%
\pgfpathlineto{\pgfqpoint{2.579068in}{1.463520in}}%
\pgfpathlineto{\pgfqpoint{2.580991in}{1.465948in}}%
\pgfpathlineto{\pgfqpoint{2.582915in}{1.448745in}}%
\pgfpathlineto{\pgfqpoint{2.584838in}{1.461388in}}%
\pgfpathlineto{\pgfqpoint{2.586761in}{1.451579in}}%
\pgfpathlineto{\pgfqpoint{2.588685in}{1.453603in}}%
\pgfpathlineto{\pgfqpoint{2.594455in}{1.423653in}}%
\pgfpathlineto{\pgfqpoint{2.596378in}{1.425923in}}%
\pgfpathlineto{\pgfqpoint{2.598302in}{1.435518in}}%
\pgfpathlineto{\pgfqpoint{2.602149in}{1.441553in}}%
\pgfpathlineto{\pgfqpoint{2.605995in}{1.426329in}}%
\pgfpathlineto{\pgfqpoint{2.607919in}{1.417615in}}%
\pgfpathlineto{\pgfqpoint{2.609842in}{1.415247in}}%
\pgfpathlineto{\pgfqpoint{2.611766in}{1.400015in}}%
\pgfpathlineto{\pgfqpoint{2.613689in}{1.404235in}}%
\pgfpathlineto{\pgfqpoint{2.615613in}{1.402243in}}%
\pgfpathlineto{\pgfqpoint{2.619459in}{1.414098in}}%
\pgfpathlineto{\pgfqpoint{2.623306in}{1.396277in}}%
\pgfpathlineto{\pgfqpoint{2.625230in}{1.398844in}}%
\pgfpathlineto{\pgfqpoint{2.627153in}{1.399363in}}%
\pgfpathlineto{\pgfqpoint{2.629076in}{1.403511in}}%
\pgfpathlineto{\pgfqpoint{2.632923in}{1.387001in}}%
\pgfpathlineto{\pgfqpoint{2.634847in}{1.386771in}}%
\pgfpathlineto{\pgfqpoint{2.638693in}{1.389855in}}%
\pgfpathlineto{\pgfqpoint{2.642540in}{1.381488in}}%
\pgfpathlineto{\pgfqpoint{2.644464in}{1.381626in}}%
\pgfpathlineto{\pgfqpoint{2.648311in}{1.374845in}}%
\pgfpathlineto{\pgfqpoint{2.650234in}{1.384250in}}%
\pgfpathlineto{\pgfqpoint{2.652157in}{1.379428in}}%
\pgfpathlineto{\pgfqpoint{2.654081in}{1.388690in}}%
\pgfpathlineto{\pgfqpoint{2.656004in}{1.407375in}}%
\pgfpathlineto{\pgfqpoint{2.657928in}{1.407609in}}%
\pgfpathlineto{\pgfqpoint{2.659851in}{1.410022in}}%
\pgfpathlineto{\pgfqpoint{2.661774in}{1.402957in}}%
\pgfpathlineto{\pgfqpoint{2.663698in}{1.400560in}}%
\pgfpathlineto{\pgfqpoint{2.665621in}{1.409798in}}%
\pgfpathlineto{\pgfqpoint{2.669468in}{1.406582in}}%
\pgfpathlineto{\pgfqpoint{2.677162in}{1.384342in}}%
\pgfpathlineto{\pgfqpoint{2.681008in}{1.360805in}}%
\pgfpathlineto{\pgfqpoint{2.682932in}{1.356915in}}%
\pgfpathlineto{\pgfqpoint{2.684855in}{1.361836in}}%
\pgfpathlineto{\pgfqpoint{2.686779in}{1.346802in}}%
\pgfpathlineto{\pgfqpoint{2.688702in}{1.344651in}}%
\pgfpathlineto{\pgfqpoint{2.690626in}{1.348668in}}%
\pgfpathlineto{\pgfqpoint{2.692549in}{1.349331in}}%
\pgfpathlineto{\pgfqpoint{2.694472in}{1.351664in}}%
\pgfpathlineto{\pgfqpoint{2.698319in}{1.328976in}}%
\pgfpathlineto{\pgfqpoint{2.700243in}{1.326172in}}%
\pgfpathlineto{\pgfqpoint{2.702166in}{1.307917in}}%
\pgfpathlineto{\pgfqpoint{2.711783in}{1.285603in}}%
\pgfpathlineto{\pgfqpoint{2.715630in}{1.265468in}}%
\pgfpathlineto{\pgfqpoint{2.717553in}{1.268793in}}%
\pgfpathlineto{\pgfqpoint{2.719477in}{1.267147in}}%
\pgfpathlineto{\pgfqpoint{2.721400in}{1.269516in}}%
\pgfpathlineto{\pgfqpoint{2.723324in}{1.255774in}}%
\pgfpathlineto{\pgfqpoint{2.725247in}{1.257533in}}%
\pgfpathlineto{\pgfqpoint{2.727170in}{1.254215in}}%
\pgfpathlineto{\pgfqpoint{2.731017in}{1.233585in}}%
\pgfpathlineto{\pgfqpoint{2.732941in}{1.247504in}}%
\pgfpathlineto{\pgfqpoint{2.736787in}{1.226993in}}%
\pgfpathlineto{\pgfqpoint{2.738711in}{1.238069in}}%
\pgfpathlineto{\pgfqpoint{2.740634in}{1.227978in}}%
\pgfpathlineto{\pgfqpoint{2.742558in}{1.233081in}}%
\pgfpathlineto{\pgfqpoint{2.744481in}{1.234432in}}%
\pgfpathlineto{\pgfqpoint{2.746404in}{1.239030in}}%
\pgfpathlineto{\pgfqpoint{2.750251in}{1.251404in}}%
\pgfpathlineto{\pgfqpoint{2.752175in}{1.256549in}}%
\pgfpathlineto{\pgfqpoint{2.754098in}{1.268336in}}%
\pgfpathlineto{\pgfqpoint{2.756022in}{1.255929in}}%
\pgfpathlineto{\pgfqpoint{2.757945in}{1.254619in}}%
\pgfpathlineto{\pgfqpoint{2.759868in}{1.257631in}}%
\pgfpathlineto{\pgfqpoint{2.761792in}{1.263021in}}%
\pgfpathlineto{\pgfqpoint{2.765639in}{1.234500in}}%
\pgfpathlineto{\pgfqpoint{2.767562in}{1.245348in}}%
\pgfpathlineto{\pgfqpoint{2.767562in}{1.245348in}}%
\pgfusepath{stroke}%
\end{pgfscope}%
\begin{pgfscope}%
\pgfpathrectangle{\pgfqpoint{0.750000in}{0.660000in}}{\pgfqpoint{2.113636in}{2.100000in}}%
\pgfusepath{clip}%
\pgfsetroundcap%
\pgfsetroundjoin%
\pgfsetlinewidth{0.602250pt}%
\definecolor{currentstroke}{rgb}{0.968627,0.505882,0.749020}%
\pgfsetstrokecolor{currentstroke}%
\pgfsetdash{}{0pt}%
\pgfpathmoveto{\pgfqpoint{0.846074in}{1.748444in}}%
\pgfpathlineto{\pgfqpoint{0.847998in}{1.745149in}}%
\pgfpathlineto{\pgfqpoint{0.849921in}{1.748510in}}%
\pgfpathlineto{\pgfqpoint{0.851845in}{1.746707in}}%
\pgfpathlineto{\pgfqpoint{0.853768in}{1.737003in}}%
\pgfpathlineto{\pgfqpoint{0.855691in}{1.734876in}}%
\pgfpathlineto{\pgfqpoint{0.859538in}{1.740589in}}%
\pgfpathlineto{\pgfqpoint{0.861462in}{1.727486in}}%
\pgfpathlineto{\pgfqpoint{0.863385in}{1.735098in}}%
\pgfpathlineto{\pgfqpoint{0.867232in}{1.732089in}}%
\pgfpathlineto{\pgfqpoint{0.869155in}{1.726114in}}%
\pgfpathlineto{\pgfqpoint{0.871079in}{1.730319in}}%
\pgfpathlineto{\pgfqpoint{0.873002in}{1.727415in}}%
\pgfpathlineto{\pgfqpoint{0.874926in}{1.744852in}}%
\pgfpathlineto{\pgfqpoint{0.876849in}{1.741107in}}%
\pgfpathlineto{\pgfqpoint{0.878772in}{1.739880in}}%
\pgfpathlineto{\pgfqpoint{0.880696in}{1.744095in}}%
\pgfpathlineto{\pgfqpoint{0.882619in}{1.751119in}}%
\pgfpathlineto{\pgfqpoint{0.884543in}{1.735126in}}%
\pgfpathlineto{\pgfqpoint{0.886466in}{1.737682in}}%
\pgfpathlineto{\pgfqpoint{0.890313in}{1.733331in}}%
\pgfpathlineto{\pgfqpoint{0.892236in}{1.721584in}}%
\pgfpathlineto{\pgfqpoint{0.894160in}{1.732522in}}%
\pgfpathlineto{\pgfqpoint{0.896083in}{1.733752in}}%
\pgfpathlineto{\pgfqpoint{0.898006in}{1.739426in}}%
\pgfpathlineto{\pgfqpoint{0.899930in}{1.741144in}}%
\pgfpathlineto{\pgfqpoint{0.903777in}{1.723838in}}%
\pgfpathlineto{\pgfqpoint{0.905700in}{1.735366in}}%
\pgfpathlineto{\pgfqpoint{0.907624in}{1.735169in}}%
\pgfpathlineto{\pgfqpoint{0.909547in}{1.726260in}}%
\pgfpathlineto{\pgfqpoint{0.911470in}{1.734800in}}%
\pgfpathlineto{\pgfqpoint{0.913394in}{1.732913in}}%
\pgfpathlineto{\pgfqpoint{0.915317in}{1.737871in}}%
\pgfpathlineto{\pgfqpoint{0.921087in}{1.725324in}}%
\pgfpathlineto{\pgfqpoint{0.923011in}{1.716762in}}%
\pgfpathlineto{\pgfqpoint{0.924934in}{1.713858in}}%
\pgfpathlineto{\pgfqpoint{0.926858in}{1.715277in}}%
\pgfpathlineto{\pgfqpoint{0.928781in}{1.711040in}}%
\pgfpathlineto{\pgfqpoint{0.930704in}{1.720093in}}%
\pgfpathlineto{\pgfqpoint{0.932628in}{1.715902in}}%
\pgfpathlineto{\pgfqpoint{0.936475in}{1.703779in}}%
\pgfpathlineto{\pgfqpoint{0.938398in}{1.718848in}}%
\pgfpathlineto{\pgfqpoint{0.940322in}{1.702911in}}%
\pgfpathlineto{\pgfqpoint{0.944168in}{1.696547in}}%
\pgfpathlineto{\pgfqpoint{0.946092in}{1.698701in}}%
\pgfpathlineto{\pgfqpoint{0.948015in}{1.699072in}}%
\pgfpathlineto{\pgfqpoint{0.953785in}{1.716504in}}%
\pgfpathlineto{\pgfqpoint{0.955709in}{1.719301in}}%
\pgfpathlineto{\pgfqpoint{0.959556in}{1.740137in}}%
\pgfpathlineto{\pgfqpoint{0.961479in}{1.741527in}}%
\pgfpathlineto{\pgfqpoint{0.965326in}{1.751359in}}%
\pgfpathlineto{\pgfqpoint{0.967249in}{1.753035in}}%
\pgfpathlineto{\pgfqpoint{0.969173in}{1.762993in}}%
\pgfpathlineto{\pgfqpoint{0.971096in}{1.766332in}}%
\pgfpathlineto{\pgfqpoint{0.973020in}{1.751445in}}%
\pgfpathlineto{\pgfqpoint{0.974943in}{1.752095in}}%
\pgfpathlineto{\pgfqpoint{0.976866in}{1.740122in}}%
\pgfpathlineto{\pgfqpoint{0.978790in}{1.745307in}}%
\pgfpathlineto{\pgfqpoint{0.980713in}{1.746137in}}%
\pgfpathlineto{\pgfqpoint{0.982637in}{1.757271in}}%
\pgfpathlineto{\pgfqpoint{0.984560in}{1.753170in}}%
\pgfpathlineto{\pgfqpoint{0.986483in}{1.755288in}}%
\pgfpathlineto{\pgfqpoint{0.990330in}{1.751606in}}%
\pgfpathlineto{\pgfqpoint{0.992254in}{1.764893in}}%
\pgfpathlineto{\pgfqpoint{0.994177in}{1.762373in}}%
\pgfpathlineto{\pgfqpoint{0.998024in}{1.771998in}}%
\pgfpathlineto{\pgfqpoint{0.999947in}{1.773126in}}%
\pgfpathlineto{\pgfqpoint{1.003794in}{1.783681in}}%
\pgfpathlineto{\pgfqpoint{1.005717in}{1.715536in}}%
\pgfpathlineto{\pgfqpoint{1.007641in}{1.711803in}}%
\pgfpathlineto{\pgfqpoint{1.009564in}{1.719734in}}%
\pgfpathlineto{\pgfqpoint{1.011488in}{1.718837in}}%
\pgfpathlineto{\pgfqpoint{1.013411in}{1.725592in}}%
\pgfpathlineto{\pgfqpoint{1.017258in}{1.715141in}}%
\pgfpathlineto{\pgfqpoint{1.021105in}{1.728119in}}%
\pgfpathlineto{\pgfqpoint{1.023028in}{1.715956in}}%
\pgfpathlineto{\pgfqpoint{1.024952in}{1.714758in}}%
\pgfpathlineto{\pgfqpoint{1.026875in}{1.704186in}}%
\pgfpathlineto{\pgfqpoint{1.028798in}{1.700457in}}%
\pgfpathlineto{\pgfqpoint{1.032645in}{1.678705in}}%
\pgfpathlineto{\pgfqpoint{1.034569in}{1.702245in}}%
\pgfpathlineto{\pgfqpoint{1.036492in}{1.708531in}}%
\pgfpathlineto{\pgfqpoint{1.038415in}{1.709028in}}%
\pgfpathlineto{\pgfqpoint{1.040339in}{1.699403in}}%
\pgfpathlineto{\pgfqpoint{1.048033in}{1.741302in}}%
\pgfpathlineto{\pgfqpoint{1.051879in}{1.731166in}}%
\pgfpathlineto{\pgfqpoint{1.053803in}{1.730577in}}%
\pgfpathlineto{\pgfqpoint{1.055726in}{1.725015in}}%
\pgfpathlineto{\pgfqpoint{1.057650in}{1.730657in}}%
\pgfpathlineto{\pgfqpoint{1.059573in}{1.731154in}}%
\pgfpathlineto{\pgfqpoint{1.061496in}{1.733007in}}%
\pgfpathlineto{\pgfqpoint{1.063420in}{1.725239in}}%
\pgfpathlineto{\pgfqpoint{1.065343in}{1.723451in}}%
\pgfpathlineto{\pgfqpoint{1.067267in}{1.724363in}}%
\pgfpathlineto{\pgfqpoint{1.069190in}{1.734787in}}%
\pgfpathlineto{\pgfqpoint{1.071113in}{1.726957in}}%
\pgfpathlineto{\pgfqpoint{1.073037in}{1.730103in}}%
\pgfpathlineto{\pgfqpoint{1.074960in}{1.726398in}}%
\pgfpathlineto{\pgfqpoint{1.076884in}{1.728211in}}%
\pgfpathlineto{\pgfqpoint{1.078807in}{1.722786in}}%
\pgfpathlineto{\pgfqpoint{1.088424in}{1.772477in}}%
\pgfpathlineto{\pgfqpoint{1.090348in}{1.777530in}}%
\pgfpathlineto{\pgfqpoint{1.092271in}{1.771850in}}%
\pgfpathlineto{\pgfqpoint{1.094194in}{1.772767in}}%
\pgfpathlineto{\pgfqpoint{1.096118in}{1.759929in}}%
\pgfpathlineto{\pgfqpoint{1.098041in}{1.764099in}}%
\pgfpathlineto{\pgfqpoint{1.099965in}{1.765562in}}%
\pgfpathlineto{\pgfqpoint{1.101888in}{1.765282in}}%
\pgfpathlineto{\pgfqpoint{1.103811in}{1.757879in}}%
\pgfpathlineto{\pgfqpoint{1.105735in}{1.767776in}}%
\pgfpathlineto{\pgfqpoint{1.107658in}{1.766636in}}%
\pgfpathlineto{\pgfqpoint{1.109582in}{1.754034in}}%
\pgfpathlineto{\pgfqpoint{1.111505in}{1.756293in}}%
\pgfpathlineto{\pgfqpoint{1.115352in}{1.728928in}}%
\pgfpathlineto{\pgfqpoint{1.117275in}{1.746959in}}%
\pgfpathlineto{\pgfqpoint{1.123046in}{1.763260in}}%
\pgfpathlineto{\pgfqpoint{1.126892in}{1.750859in}}%
\pgfpathlineto{\pgfqpoint{1.128816in}{1.754754in}}%
\pgfpathlineto{\pgfqpoint{1.130739in}{1.754603in}}%
\pgfpathlineto{\pgfqpoint{1.132663in}{1.752018in}}%
\pgfpathlineto{\pgfqpoint{1.134586in}{1.742193in}}%
\pgfpathlineto{\pgfqpoint{1.136509in}{1.760577in}}%
\pgfpathlineto{\pgfqpoint{1.138433in}{1.759606in}}%
\pgfpathlineto{\pgfqpoint{1.140356in}{1.748203in}}%
\pgfpathlineto{\pgfqpoint{1.142280in}{1.746786in}}%
\pgfpathlineto{\pgfqpoint{1.144203in}{1.747484in}}%
\pgfpathlineto{\pgfqpoint{1.146126in}{1.736429in}}%
\pgfpathlineto{\pgfqpoint{1.148050in}{1.745709in}}%
\pgfpathlineto{\pgfqpoint{1.153820in}{1.732151in}}%
\pgfpathlineto{\pgfqpoint{1.157667in}{1.728496in}}%
\pgfpathlineto{\pgfqpoint{1.159590in}{1.739804in}}%
\pgfpathlineto{\pgfqpoint{1.161514in}{1.738500in}}%
\pgfpathlineto{\pgfqpoint{1.163437in}{1.743664in}}%
\pgfpathlineto{\pgfqpoint{1.165361in}{1.732413in}}%
\pgfpathlineto{\pgfqpoint{1.167284in}{1.732366in}}%
\pgfpathlineto{\pgfqpoint{1.169207in}{1.752784in}}%
\pgfpathlineto{\pgfqpoint{1.171131in}{1.752438in}}%
\pgfpathlineto{\pgfqpoint{1.173054in}{1.754332in}}%
\pgfpathlineto{\pgfqpoint{1.174978in}{1.749571in}}%
\pgfpathlineto{\pgfqpoint{1.180748in}{1.765045in}}%
\pgfpathlineto{\pgfqpoint{1.182671in}{1.787897in}}%
\pgfpathlineto{\pgfqpoint{1.186518in}{1.795628in}}%
\pgfpathlineto{\pgfqpoint{1.192288in}{1.783123in}}%
\pgfpathlineto{\pgfqpoint{1.194212in}{1.780050in}}%
\pgfpathlineto{\pgfqpoint{1.196135in}{1.786189in}}%
\pgfpathlineto{\pgfqpoint{1.198059in}{1.775623in}}%
\pgfpathlineto{\pgfqpoint{1.199982in}{1.789399in}}%
\pgfpathlineto{\pgfqpoint{1.201905in}{1.784534in}}%
\pgfpathlineto{\pgfqpoint{1.203829in}{1.782979in}}%
\pgfpathlineto{\pgfqpoint{1.207676in}{1.794544in}}%
\pgfpathlineto{\pgfqpoint{1.209599in}{1.793700in}}%
\pgfpathlineto{\pgfqpoint{1.213446in}{1.773919in}}%
\pgfpathlineto{\pgfqpoint{1.215369in}{1.790459in}}%
\pgfpathlineto{\pgfqpoint{1.219216in}{1.796222in}}%
\pgfpathlineto{\pgfqpoint{1.223063in}{1.795400in}}%
\pgfpathlineto{\pgfqpoint{1.224986in}{1.793016in}}%
\pgfpathlineto{\pgfqpoint{1.226910in}{1.801301in}}%
\pgfpathlineto{\pgfqpoint{1.228833in}{1.785364in}}%
\pgfpathlineto{\pgfqpoint{1.230757in}{1.785703in}}%
\pgfpathlineto{\pgfqpoint{1.232680in}{1.791384in}}%
\pgfpathlineto{\pgfqpoint{1.234603in}{1.800793in}}%
\pgfpathlineto{\pgfqpoint{1.236527in}{1.798626in}}%
\pgfpathlineto{\pgfqpoint{1.238450in}{1.794610in}}%
\pgfpathlineto{\pgfqpoint{1.240374in}{1.796663in}}%
\pgfpathlineto{\pgfqpoint{1.242297in}{1.795674in}}%
\pgfpathlineto{\pgfqpoint{1.244220in}{1.776249in}}%
\pgfpathlineto{\pgfqpoint{1.246144in}{1.774106in}}%
\pgfpathlineto{\pgfqpoint{1.248067in}{1.779326in}}%
\pgfpathlineto{\pgfqpoint{1.249991in}{1.774714in}}%
\pgfpathlineto{\pgfqpoint{1.255761in}{1.819455in}}%
\pgfpathlineto{\pgfqpoint{1.257684in}{1.814614in}}%
\pgfpathlineto{\pgfqpoint{1.259608in}{1.825097in}}%
\pgfpathlineto{\pgfqpoint{1.265378in}{1.792889in}}%
\pgfpathlineto{\pgfqpoint{1.269225in}{1.805155in}}%
\pgfpathlineto{\pgfqpoint{1.271148in}{1.805287in}}%
\pgfpathlineto{\pgfqpoint{1.273072in}{1.821583in}}%
\pgfpathlineto{\pgfqpoint{1.274995in}{1.826154in}}%
\pgfpathlineto{\pgfqpoint{1.276918in}{1.818834in}}%
\pgfpathlineto{\pgfqpoint{1.278842in}{1.821476in}}%
\pgfpathlineto{\pgfqpoint{1.280765in}{1.820860in}}%
\pgfpathlineto{\pgfqpoint{1.290382in}{1.797328in}}%
\pgfpathlineto{\pgfqpoint{1.294229in}{1.778315in}}%
\pgfpathlineto{\pgfqpoint{1.296153in}{1.774396in}}%
\pgfpathlineto{\pgfqpoint{1.305770in}{1.729348in}}%
\pgfpathlineto{\pgfqpoint{1.307693in}{1.707812in}}%
\pgfpathlineto{\pgfqpoint{1.311540in}{1.705405in}}%
\pgfpathlineto{\pgfqpoint{1.313463in}{1.714113in}}%
\pgfpathlineto{\pgfqpoint{1.315387in}{1.713813in}}%
\pgfpathlineto{\pgfqpoint{1.317310in}{1.709799in}}%
\pgfpathlineto{\pgfqpoint{1.319233in}{1.719702in}}%
\pgfpathlineto{\pgfqpoint{1.321157in}{1.722753in}}%
\pgfpathlineto{\pgfqpoint{1.323080in}{1.731991in}}%
\pgfpathlineto{\pgfqpoint{1.326927in}{1.736170in}}%
\pgfpathlineto{\pgfqpoint{1.328851in}{1.735689in}}%
\pgfpathlineto{\pgfqpoint{1.330774in}{1.738473in}}%
\pgfpathlineto{\pgfqpoint{1.332697in}{1.731691in}}%
\pgfpathlineto{\pgfqpoint{1.334621in}{1.729630in}}%
\pgfpathlineto{\pgfqpoint{1.336544in}{1.736525in}}%
\pgfpathlineto{\pgfqpoint{1.340391in}{1.742831in}}%
\pgfpathlineto{\pgfqpoint{1.344238in}{1.755707in}}%
\pgfpathlineto{\pgfqpoint{1.346161in}{1.755842in}}%
\pgfpathlineto{\pgfqpoint{1.348085in}{1.754733in}}%
\pgfpathlineto{\pgfqpoint{1.350008in}{1.758560in}}%
\pgfpathlineto{\pgfqpoint{1.351931in}{1.748319in}}%
\pgfpathlineto{\pgfqpoint{1.353855in}{1.758932in}}%
\pgfpathlineto{\pgfqpoint{1.355778in}{1.750307in}}%
\pgfpathlineto{\pgfqpoint{1.357702in}{1.759124in}}%
\pgfpathlineto{\pgfqpoint{1.359625in}{1.761069in}}%
\pgfpathlineto{\pgfqpoint{1.361549in}{1.742672in}}%
\pgfpathlineto{\pgfqpoint{1.365395in}{1.741725in}}%
\pgfpathlineto{\pgfqpoint{1.369242in}{1.750932in}}%
\pgfpathlineto{\pgfqpoint{1.375012in}{1.734435in}}%
\pgfpathlineto{\pgfqpoint{1.376936in}{1.740026in}}%
\pgfpathlineto{\pgfqpoint{1.378859in}{1.739013in}}%
\pgfpathlineto{\pgfqpoint{1.384629in}{1.723093in}}%
\pgfpathlineto{\pgfqpoint{1.386553in}{1.725362in}}%
\pgfpathlineto{\pgfqpoint{1.388476in}{1.723156in}}%
\pgfpathlineto{\pgfqpoint{1.390400in}{1.725971in}}%
\pgfpathlineto{\pgfqpoint{1.392323in}{1.725783in}}%
\pgfpathlineto{\pgfqpoint{1.394247in}{1.719003in}}%
\pgfpathlineto{\pgfqpoint{1.396170in}{1.722300in}}%
\pgfpathlineto{\pgfqpoint{1.398093in}{1.720335in}}%
\pgfpathlineto{\pgfqpoint{1.401940in}{1.727724in}}%
\pgfpathlineto{\pgfqpoint{1.403864in}{1.739152in}}%
\pgfpathlineto{\pgfqpoint{1.405787in}{1.741093in}}%
\pgfpathlineto{\pgfqpoint{1.409634in}{1.758352in}}%
\pgfpathlineto{\pgfqpoint{1.413481in}{1.755024in}}%
\pgfpathlineto{\pgfqpoint{1.415404in}{1.738436in}}%
\pgfpathlineto{\pgfqpoint{1.417327in}{1.741146in}}%
\pgfpathlineto{\pgfqpoint{1.419251in}{1.756779in}}%
\pgfpathlineto{\pgfqpoint{1.421174in}{1.743741in}}%
\pgfpathlineto{\pgfqpoint{1.423098in}{1.748008in}}%
\pgfpathlineto{\pgfqpoint{1.425021in}{1.748274in}}%
\pgfpathlineto{\pgfqpoint{1.430791in}{1.734165in}}%
\pgfpathlineto{\pgfqpoint{1.432715in}{1.726440in}}%
\pgfpathlineto{\pgfqpoint{1.434638in}{1.727012in}}%
\pgfpathlineto{\pgfqpoint{1.436562in}{1.725749in}}%
\pgfpathlineto{\pgfqpoint{1.440408in}{1.709000in}}%
\pgfpathlineto{\pgfqpoint{1.444255in}{1.721860in}}%
\pgfpathlineto{\pgfqpoint{1.446179in}{1.732432in}}%
\pgfpathlineto{\pgfqpoint{1.448102in}{1.736324in}}%
\pgfpathlineto{\pgfqpoint{1.451949in}{1.737326in}}%
\pgfpathlineto{\pgfqpoint{1.453872in}{1.745755in}}%
\pgfpathlineto{\pgfqpoint{1.457719in}{1.733299in}}%
\pgfpathlineto{\pgfqpoint{1.459642in}{1.722717in}}%
\pgfpathlineto{\pgfqpoint{1.461566in}{1.734006in}}%
\pgfpathlineto{\pgfqpoint{1.463489in}{1.723426in}}%
\pgfpathlineto{\pgfqpoint{1.467336in}{1.734857in}}%
\pgfpathlineto{\pgfqpoint{1.469260in}{1.746073in}}%
\pgfpathlineto{\pgfqpoint{1.471183in}{1.746496in}}%
\pgfpathlineto{\pgfqpoint{1.473106in}{1.739472in}}%
\pgfpathlineto{\pgfqpoint{1.475030in}{1.747850in}}%
\pgfpathlineto{\pgfqpoint{1.476953in}{1.738890in}}%
\pgfpathlineto{\pgfqpoint{1.478877in}{1.749066in}}%
\pgfpathlineto{\pgfqpoint{1.480800in}{1.741768in}}%
\pgfpathlineto{\pgfqpoint{1.482723in}{1.756950in}}%
\pgfpathlineto{\pgfqpoint{1.484647in}{1.758253in}}%
\pgfpathlineto{\pgfqpoint{1.486570in}{1.764049in}}%
\pgfpathlineto{\pgfqpoint{1.488494in}{1.762011in}}%
\pgfpathlineto{\pgfqpoint{1.490417in}{1.746359in}}%
\pgfpathlineto{\pgfqpoint{1.494264in}{1.750756in}}%
\pgfpathlineto{\pgfqpoint{1.501958in}{1.722112in}}%
\pgfpathlineto{\pgfqpoint{1.507728in}{1.731728in}}%
\pgfpathlineto{\pgfqpoint{1.509651in}{1.734255in}}%
\pgfpathlineto{\pgfqpoint{1.511575in}{1.730783in}}%
\pgfpathlineto{\pgfqpoint{1.515421in}{1.742812in}}%
\pgfpathlineto{\pgfqpoint{1.517345in}{1.741975in}}%
\pgfpathlineto{\pgfqpoint{1.519268in}{1.746406in}}%
\pgfpathlineto{\pgfqpoint{1.521192in}{1.755873in}}%
\pgfpathlineto{\pgfqpoint{1.523115in}{1.728669in}}%
\pgfpathlineto{\pgfqpoint{1.525038in}{1.727877in}}%
\pgfpathlineto{\pgfqpoint{1.526962in}{1.719362in}}%
\pgfpathlineto{\pgfqpoint{1.528885in}{1.716262in}}%
\pgfpathlineto{\pgfqpoint{1.532732in}{1.717593in}}%
\pgfpathlineto{\pgfqpoint{1.534656in}{1.723459in}}%
\pgfpathlineto{\pgfqpoint{1.536579in}{1.713165in}}%
\pgfpathlineto{\pgfqpoint{1.538502in}{1.715312in}}%
\pgfpathlineto{\pgfqpoint{1.540426in}{1.712504in}}%
\pgfpathlineto{\pgfqpoint{1.544273in}{1.717918in}}%
\pgfpathlineto{\pgfqpoint{1.548119in}{1.717878in}}%
\pgfpathlineto{\pgfqpoint{1.550043in}{1.722654in}}%
\pgfpathlineto{\pgfqpoint{1.551966in}{1.717385in}}%
\pgfpathlineto{\pgfqpoint{1.553890in}{1.707189in}}%
\pgfpathlineto{\pgfqpoint{1.555813in}{1.710004in}}%
\pgfpathlineto{\pgfqpoint{1.557736in}{1.704088in}}%
\pgfpathlineto{\pgfqpoint{1.559660in}{1.712310in}}%
\pgfpathlineto{\pgfqpoint{1.561583in}{1.704255in}}%
\pgfpathlineto{\pgfqpoint{1.563507in}{1.701018in}}%
\pgfpathlineto{\pgfqpoint{1.565430in}{1.693667in}}%
\pgfpathlineto{\pgfqpoint{1.567354in}{1.700153in}}%
\pgfpathlineto{\pgfqpoint{1.573124in}{1.679580in}}%
\pgfpathlineto{\pgfqpoint{1.575047in}{1.686956in}}%
\pgfpathlineto{\pgfqpoint{1.576971in}{1.683101in}}%
\pgfpathlineto{\pgfqpoint{1.580817in}{1.689475in}}%
\pgfpathlineto{\pgfqpoint{1.582741in}{1.691488in}}%
\pgfpathlineto{\pgfqpoint{1.584664in}{1.698472in}}%
\pgfpathlineto{\pgfqpoint{1.586588in}{1.697611in}}%
\pgfpathlineto{\pgfqpoint{1.588511in}{1.698656in}}%
\pgfpathlineto{\pgfqpoint{1.590434in}{1.690684in}}%
\pgfpathlineto{\pgfqpoint{1.594281in}{1.713345in}}%
\pgfpathlineto{\pgfqpoint{1.596205in}{1.703165in}}%
\pgfpathlineto{\pgfqpoint{1.600051in}{1.705320in}}%
\pgfpathlineto{\pgfqpoint{1.601975in}{1.711489in}}%
\pgfpathlineto{\pgfqpoint{1.603898in}{1.705322in}}%
\pgfpathlineto{\pgfqpoint{1.605822in}{1.711186in}}%
\pgfpathlineto{\pgfqpoint{1.607745in}{1.699868in}}%
\pgfpathlineto{\pgfqpoint{1.609669in}{1.706067in}}%
\pgfpathlineto{\pgfqpoint{1.613515in}{1.701273in}}%
\pgfpathlineto{\pgfqpoint{1.617362in}{1.688632in}}%
\pgfpathlineto{\pgfqpoint{1.619286in}{1.694258in}}%
\pgfpathlineto{\pgfqpoint{1.621209in}{1.693495in}}%
\pgfpathlineto{\pgfqpoint{1.623132in}{1.677463in}}%
\pgfpathlineto{\pgfqpoint{1.625056in}{1.676050in}}%
\pgfpathlineto{\pgfqpoint{1.626979in}{1.683574in}}%
\pgfpathlineto{\pgfqpoint{1.628903in}{1.682615in}}%
\pgfpathlineto{\pgfqpoint{1.632749in}{1.665196in}}%
\pgfpathlineto{\pgfqpoint{1.634673in}{1.673874in}}%
\pgfpathlineto{\pgfqpoint{1.636596in}{1.651724in}}%
\pgfpathlineto{\pgfqpoint{1.638520in}{1.652610in}}%
\pgfpathlineto{\pgfqpoint{1.640443in}{1.648966in}}%
\pgfpathlineto{\pgfqpoint{1.642367in}{1.651456in}}%
\pgfpathlineto{\pgfqpoint{1.644290in}{1.660259in}}%
\pgfpathlineto{\pgfqpoint{1.646213in}{1.657119in}}%
\pgfpathlineto{\pgfqpoint{1.648137in}{1.649809in}}%
\pgfpathlineto{\pgfqpoint{1.650060in}{1.637532in}}%
\pgfpathlineto{\pgfqpoint{1.651984in}{1.655347in}}%
\pgfpathlineto{\pgfqpoint{1.655830in}{1.648840in}}%
\pgfpathlineto{\pgfqpoint{1.657754in}{1.668561in}}%
\pgfpathlineto{\pgfqpoint{1.659677in}{1.674095in}}%
\pgfpathlineto{\pgfqpoint{1.661601in}{1.672260in}}%
\pgfpathlineto{\pgfqpoint{1.663524in}{1.675073in}}%
\pgfpathlineto{\pgfqpoint{1.665447in}{1.692667in}}%
\pgfpathlineto{\pgfqpoint{1.667371in}{1.687697in}}%
\pgfpathlineto{\pgfqpoint{1.669294in}{1.686323in}}%
\pgfpathlineto{\pgfqpoint{1.673141in}{1.698593in}}%
\pgfpathlineto{\pgfqpoint{1.675065in}{1.693239in}}%
\pgfpathlineto{\pgfqpoint{1.676988in}{1.678975in}}%
\pgfpathlineto{\pgfqpoint{1.680835in}{1.675417in}}%
\pgfpathlineto{\pgfqpoint{1.682758in}{1.667430in}}%
\pgfpathlineto{\pgfqpoint{1.684682in}{1.678027in}}%
\pgfpathlineto{\pgfqpoint{1.686605in}{1.675614in}}%
\pgfpathlineto{\pgfqpoint{1.688528in}{1.679786in}}%
\pgfpathlineto{\pgfqpoint{1.690452in}{1.675668in}}%
\pgfpathlineto{\pgfqpoint{1.692375in}{1.675542in}}%
\pgfpathlineto{\pgfqpoint{1.694299in}{1.688997in}}%
\pgfpathlineto{\pgfqpoint{1.698145in}{1.672434in}}%
\pgfpathlineto{\pgfqpoint{1.700069in}{1.672523in}}%
\pgfpathlineto{\pgfqpoint{1.701992in}{1.668290in}}%
\pgfpathlineto{\pgfqpoint{1.703916in}{1.668601in}}%
\pgfpathlineto{\pgfqpoint{1.705839in}{1.654302in}}%
\pgfpathlineto{\pgfqpoint{1.707763in}{1.661130in}}%
\pgfpathlineto{\pgfqpoint{1.709686in}{1.659099in}}%
\pgfpathlineto{\pgfqpoint{1.711609in}{1.667430in}}%
\pgfpathlineto{\pgfqpoint{1.713533in}{1.685248in}}%
\pgfpathlineto{\pgfqpoint{1.721226in}{1.697711in}}%
\pgfpathlineto{\pgfqpoint{1.723150in}{1.707348in}}%
\pgfpathlineto{\pgfqpoint{1.725073in}{1.704985in}}%
\pgfpathlineto{\pgfqpoint{1.726997in}{1.713013in}}%
\pgfpathlineto{\pgfqpoint{1.728920in}{1.709238in}}%
\pgfpathlineto{\pgfqpoint{1.732767in}{1.707092in}}%
\pgfpathlineto{\pgfqpoint{1.734690in}{1.707219in}}%
\pgfpathlineto{\pgfqpoint{1.736614in}{1.694937in}}%
\pgfpathlineto{\pgfqpoint{1.742384in}{1.687922in}}%
\pgfpathlineto{\pgfqpoint{1.744307in}{1.688970in}}%
\pgfpathlineto{\pgfqpoint{1.748154in}{1.687327in}}%
\pgfpathlineto{\pgfqpoint{1.750078in}{1.682951in}}%
\pgfpathlineto{\pgfqpoint{1.755848in}{1.652774in}}%
\pgfpathlineto{\pgfqpoint{1.757771in}{1.650916in}}%
\pgfpathlineto{\pgfqpoint{1.761618in}{1.659316in}}%
\pgfpathlineto{\pgfqpoint{1.763541in}{1.648384in}}%
\pgfpathlineto{\pgfqpoint{1.767388in}{1.659529in}}%
\pgfpathlineto{\pgfqpoint{1.769312in}{1.662047in}}%
\pgfpathlineto{\pgfqpoint{1.771235in}{1.662023in}}%
\pgfpathlineto{\pgfqpoint{1.775082in}{1.654410in}}%
\pgfpathlineto{\pgfqpoint{1.777005in}{1.659062in}}%
\pgfpathlineto{\pgfqpoint{1.778929in}{1.652409in}}%
\pgfpathlineto{\pgfqpoint{1.780852in}{1.652584in}}%
\pgfpathlineto{\pgfqpoint{1.782776in}{1.643302in}}%
\pgfpathlineto{\pgfqpoint{1.788546in}{1.671132in}}%
\pgfpathlineto{\pgfqpoint{1.792393in}{1.669622in}}%
\pgfpathlineto{\pgfqpoint{1.794316in}{1.666653in}}%
\pgfpathlineto{\pgfqpoint{1.796239in}{1.670619in}}%
\pgfpathlineto{\pgfqpoint{1.800086in}{1.691115in}}%
\pgfpathlineto{\pgfqpoint{1.802010in}{1.685057in}}%
\pgfpathlineto{\pgfqpoint{1.803933in}{1.684267in}}%
\pgfpathlineto{\pgfqpoint{1.805856in}{1.681217in}}%
\pgfpathlineto{\pgfqpoint{1.807780in}{1.683783in}}%
\pgfpathlineto{\pgfqpoint{1.809703in}{1.693383in}}%
\pgfpathlineto{\pgfqpoint{1.811627in}{1.691835in}}%
\pgfpathlineto{\pgfqpoint{1.817397in}{1.675124in}}%
\pgfpathlineto{\pgfqpoint{1.819320in}{1.660653in}}%
\pgfpathlineto{\pgfqpoint{1.823167in}{1.657211in}}%
\pgfpathlineto{\pgfqpoint{1.825091in}{1.673757in}}%
\pgfpathlineto{\pgfqpoint{1.827014in}{1.672379in}}%
\pgfpathlineto{\pgfqpoint{1.832784in}{1.700006in}}%
\pgfpathlineto{\pgfqpoint{1.836631in}{1.671575in}}%
\pgfpathlineto{\pgfqpoint{1.838554in}{1.659713in}}%
\pgfpathlineto{\pgfqpoint{1.840478in}{1.662530in}}%
\pgfpathlineto{\pgfqpoint{1.842401in}{1.671390in}}%
\pgfpathlineto{\pgfqpoint{1.844325in}{1.667683in}}%
\pgfpathlineto{\pgfqpoint{1.846248in}{1.667884in}}%
\pgfpathlineto{\pgfqpoint{1.848172in}{1.690789in}}%
\pgfpathlineto{\pgfqpoint{1.850095in}{1.688175in}}%
\pgfpathlineto{\pgfqpoint{1.852018in}{1.701076in}}%
\pgfpathlineto{\pgfqpoint{1.853942in}{1.702587in}}%
\pgfpathlineto{\pgfqpoint{1.855865in}{1.705998in}}%
\pgfpathlineto{\pgfqpoint{1.857789in}{1.703344in}}%
\pgfpathlineto{\pgfqpoint{1.859712in}{1.711094in}}%
\pgfpathlineto{\pgfqpoint{1.861635in}{1.704634in}}%
\pgfpathlineto{\pgfqpoint{1.863559in}{1.703861in}}%
\pgfpathlineto{\pgfqpoint{1.865482in}{1.693407in}}%
\pgfpathlineto{\pgfqpoint{1.867406in}{1.707140in}}%
\pgfpathlineto{\pgfqpoint{1.873176in}{1.716452in}}%
\pgfpathlineto{\pgfqpoint{1.875099in}{1.715795in}}%
\pgfpathlineto{\pgfqpoint{1.877023in}{1.732664in}}%
\pgfpathlineto{\pgfqpoint{1.878946in}{1.728195in}}%
\pgfpathlineto{\pgfqpoint{1.880870in}{1.726736in}}%
\pgfpathlineto{\pgfqpoint{1.882793in}{1.734865in}}%
\pgfpathlineto{\pgfqpoint{1.884716in}{1.727983in}}%
\pgfpathlineto{\pgfqpoint{1.886640in}{1.715271in}}%
\pgfpathlineto{\pgfqpoint{1.888563in}{1.713173in}}%
\pgfpathlineto{\pgfqpoint{1.892410in}{1.705791in}}%
\pgfpathlineto{\pgfqpoint{1.894333in}{1.716920in}}%
\pgfpathlineto{\pgfqpoint{1.896257in}{1.719943in}}%
\pgfpathlineto{\pgfqpoint{1.898180in}{1.725259in}}%
\pgfpathlineto{\pgfqpoint{1.900104in}{1.709106in}}%
\pgfpathlineto{\pgfqpoint{1.902027in}{1.711352in}}%
\pgfpathlineto{\pgfqpoint{1.905874in}{1.729980in}}%
\pgfpathlineto{\pgfqpoint{1.907797in}{1.728058in}}%
\pgfpathlineto{\pgfqpoint{1.911644in}{1.751174in}}%
\pgfpathlineto{\pgfqpoint{1.915491in}{1.742813in}}%
\pgfpathlineto{\pgfqpoint{1.917414in}{1.744639in}}%
\pgfpathlineto{\pgfqpoint{1.919338in}{1.734338in}}%
\pgfpathlineto{\pgfqpoint{1.923185in}{1.735185in}}%
\pgfpathlineto{\pgfqpoint{1.927031in}{1.713917in}}%
\pgfpathlineto{\pgfqpoint{1.928955in}{1.711647in}}%
\pgfpathlineto{\pgfqpoint{1.930878in}{1.714865in}}%
\pgfpathlineto{\pgfqpoint{1.932802in}{1.715368in}}%
\pgfpathlineto{\pgfqpoint{1.936648in}{1.722283in}}%
\pgfpathlineto{\pgfqpoint{1.938572in}{1.707904in}}%
\pgfpathlineto{\pgfqpoint{1.940495in}{1.710597in}}%
\pgfpathlineto{\pgfqpoint{1.942419in}{1.705172in}}%
\pgfpathlineto{\pgfqpoint{1.944342in}{1.710209in}}%
\pgfpathlineto{\pgfqpoint{1.948189in}{1.687820in}}%
\pgfpathlineto{\pgfqpoint{1.952036in}{1.684043in}}%
\pgfpathlineto{\pgfqpoint{1.953959in}{1.680241in}}%
\pgfpathlineto{\pgfqpoint{1.955883in}{1.682875in}}%
\pgfpathlineto{\pgfqpoint{1.957806in}{1.674305in}}%
\pgfpathlineto{\pgfqpoint{1.959729in}{1.672913in}}%
\pgfpathlineto{\pgfqpoint{1.961653in}{1.674333in}}%
\pgfpathlineto{\pgfqpoint{1.963576in}{1.678484in}}%
\pgfpathlineto{\pgfqpoint{1.965500in}{1.686617in}}%
\pgfpathlineto{\pgfqpoint{1.967423in}{1.687265in}}%
\pgfpathlineto{\pgfqpoint{1.969346in}{1.683661in}}%
\pgfpathlineto{\pgfqpoint{1.977040in}{1.656455in}}%
\pgfpathlineto{\pgfqpoint{1.978963in}{1.663278in}}%
\pgfpathlineto{\pgfqpoint{1.980887in}{1.660605in}}%
\pgfpathlineto{\pgfqpoint{1.982810in}{1.668538in}}%
\pgfpathlineto{\pgfqpoint{1.984734in}{1.659216in}}%
\pgfpathlineto{\pgfqpoint{1.986657in}{1.662049in}}%
\pgfpathlineto{\pgfqpoint{1.988581in}{1.661190in}}%
\pgfpathlineto{\pgfqpoint{1.990504in}{1.651132in}}%
\pgfpathlineto{\pgfqpoint{1.992427in}{1.654430in}}%
\pgfpathlineto{\pgfqpoint{1.994351in}{1.654949in}}%
\pgfpathlineto{\pgfqpoint{1.996274in}{1.660239in}}%
\pgfpathlineto{\pgfqpoint{1.998198in}{1.659056in}}%
\pgfpathlineto{\pgfqpoint{2.000121in}{1.675291in}}%
\pgfpathlineto{\pgfqpoint{2.002044in}{1.680518in}}%
\pgfpathlineto{\pgfqpoint{2.003968in}{1.677312in}}%
\pgfpathlineto{\pgfqpoint{2.005891in}{1.666776in}}%
\pgfpathlineto{\pgfqpoint{2.007815in}{1.673905in}}%
\pgfpathlineto{\pgfqpoint{2.009738in}{1.673927in}}%
\pgfpathlineto{\pgfqpoint{2.013585in}{1.666506in}}%
\pgfpathlineto{\pgfqpoint{2.017432in}{1.683948in}}%
\pgfpathlineto{\pgfqpoint{2.019355in}{1.679849in}}%
\pgfpathlineto{\pgfqpoint{2.021279in}{1.698284in}}%
\pgfpathlineto{\pgfqpoint{2.023202in}{1.701360in}}%
\pgfpathlineto{\pgfqpoint{2.025125in}{1.720831in}}%
\pgfpathlineto{\pgfqpoint{2.027049in}{1.719173in}}%
\pgfpathlineto{\pgfqpoint{2.028972in}{1.714273in}}%
\pgfpathlineto{\pgfqpoint{2.030896in}{1.722430in}}%
\pgfpathlineto{\pgfqpoint{2.034742in}{1.726537in}}%
\pgfpathlineto{\pgfqpoint{2.036666in}{1.735239in}}%
\pgfpathlineto{\pgfqpoint{2.038589in}{1.731401in}}%
\pgfpathlineto{\pgfqpoint{2.040513in}{1.736935in}}%
\pgfpathlineto{\pgfqpoint{2.042436in}{1.736442in}}%
\pgfpathlineto{\pgfqpoint{2.044359in}{1.731318in}}%
\pgfpathlineto{\pgfqpoint{2.048206in}{1.727142in}}%
\pgfpathlineto{\pgfqpoint{2.050130in}{1.728815in}}%
\pgfpathlineto{\pgfqpoint{2.052053in}{1.723499in}}%
\pgfpathlineto{\pgfqpoint{2.055900in}{1.726784in}}%
\pgfpathlineto{\pgfqpoint{2.061670in}{1.716582in}}%
\pgfpathlineto{\pgfqpoint{2.063594in}{1.717729in}}%
\pgfpathlineto{\pgfqpoint{2.065517in}{1.716734in}}%
\pgfpathlineto{\pgfqpoint{2.067440in}{1.711919in}}%
\pgfpathlineto{\pgfqpoint{2.069364in}{1.701692in}}%
\pgfpathlineto{\pgfqpoint{2.071287in}{1.702555in}}%
\pgfpathlineto{\pgfqpoint{2.075134in}{1.711789in}}%
\pgfpathlineto{\pgfqpoint{2.077057in}{1.711822in}}%
\pgfpathlineto{\pgfqpoint{2.078981in}{1.707976in}}%
\pgfpathlineto{\pgfqpoint{2.080904in}{1.708660in}}%
\pgfpathlineto{\pgfqpoint{2.082828in}{1.720361in}}%
\pgfpathlineto{\pgfqpoint{2.084751in}{1.714842in}}%
\pgfpathlineto{\pgfqpoint{2.086674in}{1.723283in}}%
\pgfpathlineto{\pgfqpoint{2.088598in}{1.723395in}}%
\pgfpathlineto{\pgfqpoint{2.090521in}{1.713823in}}%
\pgfpathlineto{\pgfqpoint{2.092445in}{1.712083in}}%
\pgfpathlineto{\pgfqpoint{2.094368in}{1.726363in}}%
\pgfpathlineto{\pgfqpoint{2.096292in}{1.721070in}}%
\pgfpathlineto{\pgfqpoint{2.098215in}{1.726816in}}%
\pgfpathlineto{\pgfqpoint{2.100138in}{1.719448in}}%
\pgfpathlineto{\pgfqpoint{2.103985in}{1.716834in}}%
\pgfpathlineto{\pgfqpoint{2.107832in}{1.722233in}}%
\pgfpathlineto{\pgfqpoint{2.109755in}{1.724023in}}%
\pgfpathlineto{\pgfqpoint{2.111679in}{1.733071in}}%
\pgfpathlineto{\pgfqpoint{2.113602in}{1.732853in}}%
\pgfpathlineto{\pgfqpoint{2.115526in}{1.730764in}}%
\pgfpathlineto{\pgfqpoint{2.117449in}{1.734226in}}%
\pgfpathlineto{\pgfqpoint{2.119372in}{1.734779in}}%
\pgfpathlineto{\pgfqpoint{2.121296in}{1.723264in}}%
\pgfpathlineto{\pgfqpoint{2.123219in}{1.718549in}}%
\pgfpathlineto{\pgfqpoint{2.125143in}{1.719184in}}%
\pgfpathlineto{\pgfqpoint{2.127066in}{1.715594in}}%
\pgfpathlineto{\pgfqpoint{2.128990in}{1.714575in}}%
\pgfpathlineto{\pgfqpoint{2.134760in}{1.702513in}}%
\pgfpathlineto{\pgfqpoint{2.136683in}{1.705309in}}%
\pgfpathlineto{\pgfqpoint{2.138607in}{1.713789in}}%
\pgfpathlineto{\pgfqpoint{2.142453in}{1.700873in}}%
\pgfpathlineto{\pgfqpoint{2.144377in}{1.712628in}}%
\pgfpathlineto{\pgfqpoint{2.146300in}{1.710783in}}%
\pgfpathlineto{\pgfqpoint{2.148224in}{1.713216in}}%
\pgfpathlineto{\pgfqpoint{2.150147in}{1.709873in}}%
\pgfpathlineto{\pgfqpoint{2.152070in}{1.704131in}}%
\pgfpathlineto{\pgfqpoint{2.153994in}{1.708597in}}%
\pgfpathlineto{\pgfqpoint{2.155917in}{1.705701in}}%
\pgfpathlineto{\pgfqpoint{2.157841in}{1.705941in}}%
\pgfpathlineto{\pgfqpoint{2.159764in}{1.691017in}}%
\pgfpathlineto{\pgfqpoint{2.161688in}{1.690414in}}%
\pgfpathlineto{\pgfqpoint{2.167458in}{1.698381in}}%
\pgfpathlineto{\pgfqpoint{2.169381in}{1.694514in}}%
\pgfpathlineto{\pgfqpoint{2.171305in}{1.682578in}}%
\pgfpathlineto{\pgfqpoint{2.173228in}{1.699826in}}%
\pgfpathlineto{\pgfqpoint{2.175151in}{1.701912in}}%
\pgfpathlineto{\pgfqpoint{2.177075in}{1.712612in}}%
\pgfpathlineto{\pgfqpoint{2.180922in}{1.710856in}}%
\pgfpathlineto{\pgfqpoint{2.182845in}{1.700168in}}%
\pgfpathlineto{\pgfqpoint{2.184768in}{1.704337in}}%
\pgfpathlineto{\pgfqpoint{2.186692in}{1.697884in}}%
\pgfpathlineto{\pgfqpoint{2.188615in}{1.701764in}}%
\pgfpathlineto{\pgfqpoint{2.190539in}{1.711761in}}%
\pgfpathlineto{\pgfqpoint{2.192462in}{1.710446in}}%
\pgfpathlineto{\pgfqpoint{2.194386in}{1.712563in}}%
\pgfpathlineto{\pgfqpoint{2.196309in}{1.707923in}}%
\pgfpathlineto{\pgfqpoint{2.200156in}{1.716150in}}%
\pgfpathlineto{\pgfqpoint{2.202079in}{1.714243in}}%
\pgfpathlineto{\pgfqpoint{2.204003in}{1.716469in}}%
\pgfpathlineto{\pgfqpoint{2.205926in}{1.715175in}}%
\pgfpathlineto{\pgfqpoint{2.207849in}{1.708109in}}%
\pgfpathlineto{\pgfqpoint{2.213620in}{1.719102in}}%
\pgfpathlineto{\pgfqpoint{2.215543in}{1.714241in}}%
\pgfpathlineto{\pgfqpoint{2.217466in}{1.720827in}}%
\pgfpathlineto{\pgfqpoint{2.219390in}{1.716289in}}%
\pgfpathlineto{\pgfqpoint{2.221313in}{1.708037in}}%
\pgfpathlineto{\pgfqpoint{2.223237in}{1.710129in}}%
\pgfpathlineto{\pgfqpoint{2.225160in}{1.719156in}}%
\pgfpathlineto{\pgfqpoint{2.227083in}{1.733639in}}%
\pgfpathlineto{\pgfqpoint{2.229007in}{1.734747in}}%
\pgfpathlineto{\pgfqpoint{2.230930in}{1.731723in}}%
\pgfpathlineto{\pgfqpoint{2.232854in}{1.733857in}}%
\pgfpathlineto{\pgfqpoint{2.238624in}{1.712617in}}%
\pgfpathlineto{\pgfqpoint{2.242471in}{1.726261in}}%
\pgfpathlineto{\pgfqpoint{2.244394in}{1.727290in}}%
\pgfpathlineto{\pgfqpoint{2.246318in}{1.723747in}}%
\pgfpathlineto{\pgfqpoint{2.248241in}{1.724319in}}%
\pgfpathlineto{\pgfqpoint{2.250164in}{1.717770in}}%
\pgfpathlineto{\pgfqpoint{2.252088in}{1.721358in}}%
\pgfpathlineto{\pgfqpoint{2.254011in}{1.716755in}}%
\pgfpathlineto{\pgfqpoint{2.255935in}{1.718771in}}%
\pgfpathlineto{\pgfqpoint{2.257858in}{1.718650in}}%
\pgfpathlineto{\pgfqpoint{2.259781in}{1.736574in}}%
\pgfpathlineto{\pgfqpoint{2.263628in}{1.725491in}}%
\pgfpathlineto{\pgfqpoint{2.265552in}{1.730205in}}%
\pgfpathlineto{\pgfqpoint{2.267475in}{1.742696in}}%
\pgfpathlineto{\pgfqpoint{2.269399in}{1.746540in}}%
\pgfpathlineto{\pgfqpoint{2.271322in}{1.737939in}}%
\pgfpathlineto{\pgfqpoint{2.273245in}{1.755604in}}%
\pgfpathlineto{\pgfqpoint{2.277092in}{1.743613in}}%
\pgfpathlineto{\pgfqpoint{2.279016in}{1.745326in}}%
\pgfpathlineto{\pgfqpoint{2.280939in}{1.727290in}}%
\pgfpathlineto{\pgfqpoint{2.282862in}{1.723645in}}%
\pgfpathlineto{\pgfqpoint{2.284786in}{1.733798in}}%
\pgfpathlineto{\pgfqpoint{2.286709in}{1.736260in}}%
\pgfpathlineto{\pgfqpoint{2.288633in}{1.735878in}}%
\pgfpathlineto{\pgfqpoint{2.292479in}{1.696981in}}%
\pgfpathlineto{\pgfqpoint{2.294403in}{1.698608in}}%
\pgfpathlineto{\pgfqpoint{2.298250in}{1.693184in}}%
\pgfpathlineto{\pgfqpoint{2.300173in}{1.695521in}}%
\pgfpathlineto{\pgfqpoint{2.302097in}{1.707413in}}%
\pgfpathlineto{\pgfqpoint{2.304020in}{1.707551in}}%
\pgfpathlineto{\pgfqpoint{2.305943in}{1.703361in}}%
\pgfpathlineto{\pgfqpoint{2.307867in}{1.696267in}}%
\pgfpathlineto{\pgfqpoint{2.309790in}{1.708761in}}%
\pgfpathlineto{\pgfqpoint{2.311714in}{1.704180in}}%
\pgfpathlineto{\pgfqpoint{2.313637in}{1.707004in}}%
\pgfpathlineto{\pgfqpoint{2.315560in}{1.705114in}}%
\pgfpathlineto{\pgfqpoint{2.319407in}{1.691495in}}%
\pgfpathlineto{\pgfqpoint{2.321331in}{1.690017in}}%
\pgfpathlineto{\pgfqpoint{2.325177in}{1.699694in}}%
\pgfpathlineto{\pgfqpoint{2.329024in}{1.694767in}}%
\pgfpathlineto{\pgfqpoint{2.330948in}{1.695264in}}%
\pgfpathlineto{\pgfqpoint{2.332871in}{1.698484in}}%
\pgfpathlineto{\pgfqpoint{2.336718in}{1.686829in}}%
\pgfpathlineto{\pgfqpoint{2.338641in}{1.675124in}}%
\pgfpathlineto{\pgfqpoint{2.340565in}{1.687814in}}%
\pgfpathlineto{\pgfqpoint{2.342488in}{1.678529in}}%
\pgfpathlineto{\pgfqpoint{2.344412in}{1.679997in}}%
\pgfpathlineto{\pgfqpoint{2.346335in}{1.669267in}}%
\pgfpathlineto{\pgfqpoint{2.348258in}{1.678200in}}%
\pgfpathlineto{\pgfqpoint{2.350182in}{1.669145in}}%
\pgfpathlineto{\pgfqpoint{2.352105in}{1.669020in}}%
\pgfpathlineto{\pgfqpoint{2.354029in}{1.672254in}}%
\pgfpathlineto{\pgfqpoint{2.355952in}{1.683132in}}%
\pgfpathlineto{\pgfqpoint{2.359799in}{1.686288in}}%
\pgfpathlineto{\pgfqpoint{2.363646in}{1.702258in}}%
\pgfpathlineto{\pgfqpoint{2.365569in}{1.703592in}}%
\pgfpathlineto{\pgfqpoint{2.367492in}{1.709221in}}%
\pgfpathlineto{\pgfqpoint{2.369416in}{1.707298in}}%
\pgfpathlineto{\pgfqpoint{2.373263in}{1.732325in}}%
\pgfpathlineto{\pgfqpoint{2.375186in}{1.737079in}}%
\pgfpathlineto{\pgfqpoint{2.379033in}{1.730497in}}%
\pgfpathlineto{\pgfqpoint{2.382880in}{1.743139in}}%
\pgfpathlineto{\pgfqpoint{2.384803in}{1.734469in}}%
\pgfpathlineto{\pgfqpoint{2.388650in}{1.767800in}}%
\pgfpathlineto{\pgfqpoint{2.390573in}{1.744479in}}%
\pgfpathlineto{\pgfqpoint{2.392497in}{1.745911in}}%
\pgfpathlineto{\pgfqpoint{2.394420in}{1.742629in}}%
\pgfpathlineto{\pgfqpoint{2.396344in}{1.728655in}}%
\pgfpathlineto{\pgfqpoint{2.398267in}{1.735729in}}%
\pgfpathlineto{\pgfqpoint{2.400190in}{1.729439in}}%
\pgfpathlineto{\pgfqpoint{2.402114in}{1.714661in}}%
\pgfpathlineto{\pgfqpoint{2.404037in}{1.715683in}}%
\pgfpathlineto{\pgfqpoint{2.405961in}{1.708551in}}%
\pgfpathlineto{\pgfqpoint{2.409808in}{1.716277in}}%
\pgfpathlineto{\pgfqpoint{2.415578in}{1.736691in}}%
\pgfpathlineto{\pgfqpoint{2.417501in}{1.741123in}}%
\pgfpathlineto{\pgfqpoint{2.419425in}{1.736153in}}%
\pgfpathlineto{\pgfqpoint{2.421348in}{1.739754in}}%
\pgfpathlineto{\pgfqpoint{2.423271in}{1.753256in}}%
\pgfpathlineto{\pgfqpoint{2.425195in}{1.753774in}}%
\pgfpathlineto{\pgfqpoint{2.427118in}{1.748770in}}%
\pgfpathlineto{\pgfqpoint{2.429042in}{1.747792in}}%
\pgfpathlineto{\pgfqpoint{2.430965in}{1.748066in}}%
\pgfpathlineto{\pgfqpoint{2.432888in}{1.741006in}}%
\pgfpathlineto{\pgfqpoint{2.434812in}{1.740063in}}%
\pgfpathlineto{\pgfqpoint{2.436735in}{1.731845in}}%
\pgfpathlineto{\pgfqpoint{2.440582in}{1.737389in}}%
\pgfpathlineto{\pgfqpoint{2.442506in}{1.730921in}}%
\pgfpathlineto{\pgfqpoint{2.444429in}{1.746299in}}%
\pgfpathlineto{\pgfqpoint{2.446352in}{1.751695in}}%
\pgfpathlineto{\pgfqpoint{2.448276in}{1.749142in}}%
\pgfpathlineto{\pgfqpoint{2.450199in}{1.755203in}}%
\pgfpathlineto{\pgfqpoint{2.454046in}{1.729782in}}%
\pgfpathlineto{\pgfqpoint{2.455969in}{1.730388in}}%
\pgfpathlineto{\pgfqpoint{2.457893in}{1.726264in}}%
\pgfpathlineto{\pgfqpoint{2.459816in}{1.711547in}}%
\pgfpathlineto{\pgfqpoint{2.461740in}{1.720622in}}%
\pgfpathlineto{\pgfqpoint{2.465586in}{1.696109in}}%
\pgfpathlineto{\pgfqpoint{2.467510in}{1.700834in}}%
\pgfpathlineto{\pgfqpoint{2.469433in}{1.685360in}}%
\pgfpathlineto{\pgfqpoint{2.471357in}{1.690967in}}%
\pgfpathlineto{\pgfqpoint{2.473280in}{1.689892in}}%
\pgfpathlineto{\pgfqpoint{2.479050in}{1.666599in}}%
\pgfpathlineto{\pgfqpoint{2.480974in}{1.664373in}}%
\pgfpathlineto{\pgfqpoint{2.482897in}{1.667437in}}%
\pgfpathlineto{\pgfqpoint{2.484821in}{1.678730in}}%
\pgfpathlineto{\pgfqpoint{2.486744in}{1.678857in}}%
\pgfpathlineto{\pgfqpoint{2.488667in}{1.680791in}}%
\pgfpathlineto{\pgfqpoint{2.494438in}{1.672736in}}%
\pgfpathlineto{\pgfqpoint{2.496361in}{1.659986in}}%
\pgfpathlineto{\pgfqpoint{2.498284in}{1.665785in}}%
\pgfpathlineto{\pgfqpoint{2.500208in}{1.665251in}}%
\pgfpathlineto{\pgfqpoint{2.502131in}{1.660715in}}%
\pgfpathlineto{\pgfqpoint{2.505978in}{1.658367in}}%
\pgfpathlineto{\pgfqpoint{2.507901in}{1.661638in}}%
\pgfpathlineto{\pgfqpoint{2.509825in}{1.654980in}}%
\pgfpathlineto{\pgfqpoint{2.511748in}{1.658246in}}%
\pgfpathlineto{\pgfqpoint{2.513672in}{1.650942in}}%
\pgfpathlineto{\pgfqpoint{2.515595in}{1.651601in}}%
\pgfpathlineto{\pgfqpoint{2.517519in}{1.655063in}}%
\pgfpathlineto{\pgfqpoint{2.519442in}{1.645939in}}%
\pgfpathlineto{\pgfqpoint{2.521365in}{1.648715in}}%
\pgfpathlineto{\pgfqpoint{2.523289in}{1.655382in}}%
\pgfpathlineto{\pgfqpoint{2.525212in}{1.667539in}}%
\pgfpathlineto{\pgfqpoint{2.529059in}{1.655166in}}%
\pgfpathlineto{\pgfqpoint{2.532906in}{1.644083in}}%
\pgfpathlineto{\pgfqpoint{2.534829in}{1.650993in}}%
\pgfpathlineto{\pgfqpoint{2.536753in}{1.649630in}}%
\pgfpathlineto{\pgfqpoint{2.538676in}{1.644380in}}%
\pgfpathlineto{\pgfqpoint{2.540599in}{1.631196in}}%
\pgfpathlineto{\pgfqpoint{2.542523in}{1.643618in}}%
\pgfpathlineto{\pgfqpoint{2.546370in}{1.645704in}}%
\pgfpathlineto{\pgfqpoint{2.548293in}{1.653024in}}%
\pgfpathlineto{\pgfqpoint{2.550217in}{1.647043in}}%
\pgfpathlineto{\pgfqpoint{2.552140in}{1.652595in}}%
\pgfpathlineto{\pgfqpoint{2.554063in}{1.645663in}}%
\pgfpathlineto{\pgfqpoint{2.555987in}{1.656541in}}%
\pgfpathlineto{\pgfqpoint{2.557910in}{1.657648in}}%
\pgfpathlineto{\pgfqpoint{2.559834in}{1.652512in}}%
\pgfpathlineto{\pgfqpoint{2.563680in}{1.662309in}}%
\pgfpathlineto{\pgfqpoint{2.565604in}{1.661582in}}%
\pgfpathlineto{\pgfqpoint{2.567527in}{1.668577in}}%
\pgfpathlineto{\pgfqpoint{2.569451in}{1.669141in}}%
\pgfpathlineto{\pgfqpoint{2.573297in}{1.655923in}}%
\pgfpathlineto{\pgfqpoint{2.575221in}{1.656657in}}%
\pgfpathlineto{\pgfqpoint{2.577144in}{1.652127in}}%
\pgfpathlineto{\pgfqpoint{2.579068in}{1.669727in}}%
\pgfpathlineto{\pgfqpoint{2.580991in}{1.673498in}}%
\pgfpathlineto{\pgfqpoint{2.582915in}{1.683615in}}%
\pgfpathlineto{\pgfqpoint{2.584838in}{1.683665in}}%
\pgfpathlineto{\pgfqpoint{2.586761in}{1.680296in}}%
\pgfpathlineto{\pgfqpoint{2.592532in}{1.703909in}}%
\pgfpathlineto{\pgfqpoint{2.594455in}{1.702980in}}%
\pgfpathlineto{\pgfqpoint{2.596378in}{1.718103in}}%
\pgfpathlineto{\pgfqpoint{2.598302in}{1.719983in}}%
\pgfpathlineto{\pgfqpoint{2.600225in}{1.729116in}}%
\pgfpathlineto{\pgfqpoint{2.604072in}{1.720592in}}%
\pgfpathlineto{\pgfqpoint{2.605995in}{1.725962in}}%
\pgfpathlineto{\pgfqpoint{2.607919in}{1.727858in}}%
\pgfpathlineto{\pgfqpoint{2.611766in}{1.709295in}}%
\pgfpathlineto{\pgfqpoint{2.613689in}{1.718042in}}%
\pgfpathlineto{\pgfqpoint{2.617536in}{1.721512in}}%
\pgfpathlineto{\pgfqpoint{2.619459in}{1.740622in}}%
\pgfpathlineto{\pgfqpoint{2.621383in}{1.740966in}}%
\pgfpathlineto{\pgfqpoint{2.625230in}{1.731394in}}%
\pgfpathlineto{\pgfqpoint{2.627153in}{1.740573in}}%
\pgfpathlineto{\pgfqpoint{2.629076in}{1.737907in}}%
\pgfpathlineto{\pgfqpoint{2.631000in}{1.727758in}}%
\pgfpathlineto{\pgfqpoint{2.632923in}{1.735848in}}%
\pgfpathlineto{\pgfqpoint{2.634847in}{1.729032in}}%
\pgfpathlineto{\pgfqpoint{2.638693in}{1.738319in}}%
\pgfpathlineto{\pgfqpoint{2.640617in}{1.725947in}}%
\pgfpathlineto{\pgfqpoint{2.642540in}{1.731634in}}%
\pgfpathlineto{\pgfqpoint{2.644464in}{1.726672in}}%
\pgfpathlineto{\pgfqpoint{2.646387in}{1.732674in}}%
\pgfpathlineto{\pgfqpoint{2.648311in}{1.732050in}}%
\pgfpathlineto{\pgfqpoint{2.650234in}{1.721481in}}%
\pgfpathlineto{\pgfqpoint{2.652157in}{1.719318in}}%
\pgfpathlineto{\pgfqpoint{2.656004in}{1.718430in}}%
\pgfpathlineto{\pgfqpoint{2.657928in}{1.727191in}}%
\pgfpathlineto{\pgfqpoint{2.659851in}{1.725313in}}%
\pgfpathlineto{\pgfqpoint{2.663698in}{1.745534in}}%
\pgfpathlineto{\pgfqpoint{2.665621in}{1.738873in}}%
\pgfpathlineto{\pgfqpoint{2.667545in}{1.740750in}}%
\pgfpathlineto{\pgfqpoint{2.669468in}{1.746071in}}%
\pgfpathlineto{\pgfqpoint{2.671391in}{1.745054in}}%
\pgfpathlineto{\pgfqpoint{2.673315in}{1.739259in}}%
\pgfpathlineto{\pgfqpoint{2.675238in}{1.744278in}}%
\pgfpathlineto{\pgfqpoint{2.677162in}{1.754349in}}%
\pgfpathlineto{\pgfqpoint{2.679085in}{1.745040in}}%
\pgfpathlineto{\pgfqpoint{2.681008in}{1.728348in}}%
\pgfpathlineto{\pgfqpoint{2.682932in}{1.722871in}}%
\pgfpathlineto{\pgfqpoint{2.684855in}{1.722571in}}%
\pgfpathlineto{\pgfqpoint{2.686779in}{1.710206in}}%
\pgfpathlineto{\pgfqpoint{2.688702in}{1.714256in}}%
\pgfpathlineto{\pgfqpoint{2.690626in}{1.712992in}}%
\pgfpathlineto{\pgfqpoint{2.692549in}{1.730611in}}%
\pgfpathlineto{\pgfqpoint{2.694472in}{1.731607in}}%
\pgfpathlineto{\pgfqpoint{2.696396in}{1.734042in}}%
\pgfpathlineto{\pgfqpoint{2.698319in}{1.730105in}}%
\pgfpathlineto{\pgfqpoint{2.702166in}{1.731724in}}%
\pgfpathlineto{\pgfqpoint{2.704089in}{1.717372in}}%
\pgfpathlineto{\pgfqpoint{2.706013in}{1.713098in}}%
\pgfpathlineto{\pgfqpoint{2.707936in}{1.723526in}}%
\pgfpathlineto{\pgfqpoint{2.709860in}{1.724079in}}%
\pgfpathlineto{\pgfqpoint{2.711783in}{1.735874in}}%
\pgfpathlineto{\pgfqpoint{2.713706in}{1.727710in}}%
\pgfpathlineto{\pgfqpoint{2.715630in}{1.741081in}}%
\pgfpathlineto{\pgfqpoint{2.717553in}{1.743413in}}%
\pgfpathlineto{\pgfqpoint{2.719477in}{1.756502in}}%
\pgfpathlineto{\pgfqpoint{2.721400in}{1.754418in}}%
\pgfpathlineto{\pgfqpoint{2.723324in}{1.769375in}}%
\pgfpathlineto{\pgfqpoint{2.727170in}{1.779550in}}%
\pgfpathlineto{\pgfqpoint{2.729094in}{1.778489in}}%
\pgfpathlineto{\pgfqpoint{2.731017in}{1.762282in}}%
\pgfpathlineto{\pgfqpoint{2.732941in}{1.761762in}}%
\pgfpathlineto{\pgfqpoint{2.734864in}{1.757071in}}%
\pgfpathlineto{\pgfqpoint{2.736787in}{1.759282in}}%
\pgfpathlineto{\pgfqpoint{2.738711in}{1.755369in}}%
\pgfpathlineto{\pgfqpoint{2.740634in}{1.760324in}}%
\pgfpathlineto{\pgfqpoint{2.742558in}{1.770843in}}%
\pgfpathlineto{\pgfqpoint{2.744481in}{1.770446in}}%
\pgfpathlineto{\pgfqpoint{2.746404in}{1.773407in}}%
\pgfpathlineto{\pgfqpoint{2.748328in}{1.768203in}}%
\pgfpathlineto{\pgfqpoint{2.750251in}{1.773920in}}%
\pgfpathlineto{\pgfqpoint{2.754098in}{1.776958in}}%
\pgfpathlineto{\pgfqpoint{2.756022in}{1.780354in}}%
\pgfpathlineto{\pgfqpoint{2.757945in}{1.787331in}}%
\pgfpathlineto{\pgfqpoint{2.759868in}{1.785226in}}%
\pgfpathlineto{\pgfqpoint{2.761792in}{1.796617in}}%
\pgfpathlineto{\pgfqpoint{2.763715in}{1.790898in}}%
\pgfpathlineto{\pgfqpoint{2.767562in}{1.796872in}}%
\pgfpathlineto{\pgfqpoint{2.767562in}{1.796872in}}%
\pgfusepath{stroke}%
\end{pgfscope}%
\begin{pgfscope}%
\pgfpathrectangle{\pgfqpoint{0.750000in}{0.660000in}}{\pgfqpoint{2.113636in}{2.100000in}}%
\pgfusepath{clip}%
\pgfsetroundcap%
\pgfsetroundjoin%
\pgfsetlinewidth{0.602250pt}%
\definecolor{currentstroke}{rgb}{0.650980,0.337255,0.156863}%
\pgfsetstrokecolor{currentstroke}%
\pgfsetdash{}{0pt}%
\pgfpathmoveto{\pgfqpoint{0.846074in}{1.809395in}}%
\pgfpathlineto{\pgfqpoint{0.847998in}{1.810754in}}%
\pgfpathlineto{\pgfqpoint{0.851845in}{1.804843in}}%
\pgfpathlineto{\pgfqpoint{0.853768in}{1.790353in}}%
\pgfpathlineto{\pgfqpoint{0.857615in}{1.796775in}}%
\pgfpathlineto{\pgfqpoint{0.859538in}{1.791930in}}%
\pgfpathlineto{\pgfqpoint{0.861462in}{1.792747in}}%
\pgfpathlineto{\pgfqpoint{0.867232in}{1.780399in}}%
\pgfpathlineto{\pgfqpoint{0.869155in}{1.778502in}}%
\pgfpathlineto{\pgfqpoint{0.871079in}{1.781494in}}%
\pgfpathlineto{\pgfqpoint{0.873002in}{1.779766in}}%
\pgfpathlineto{\pgfqpoint{0.874926in}{1.775038in}}%
\pgfpathlineto{\pgfqpoint{0.878772in}{1.784772in}}%
\pgfpathlineto{\pgfqpoint{0.882619in}{1.775648in}}%
\pgfpathlineto{\pgfqpoint{0.884543in}{1.775434in}}%
\pgfpathlineto{\pgfqpoint{0.886466in}{1.781517in}}%
\pgfpathlineto{\pgfqpoint{0.888389in}{1.782852in}}%
\pgfpathlineto{\pgfqpoint{0.890313in}{1.777701in}}%
\pgfpathlineto{\pgfqpoint{0.894160in}{1.787012in}}%
\pgfpathlineto{\pgfqpoint{0.896083in}{1.784200in}}%
\pgfpathlineto{\pgfqpoint{0.903777in}{1.766583in}}%
\pgfpathlineto{\pgfqpoint{0.909547in}{1.798454in}}%
\pgfpathlineto{\pgfqpoint{0.911470in}{1.811141in}}%
\pgfpathlineto{\pgfqpoint{0.915317in}{1.803709in}}%
\pgfpathlineto{\pgfqpoint{0.917241in}{1.804941in}}%
\pgfpathlineto{\pgfqpoint{0.921087in}{1.812959in}}%
\pgfpathlineto{\pgfqpoint{0.923011in}{1.817700in}}%
\pgfpathlineto{\pgfqpoint{0.924934in}{1.811750in}}%
\pgfpathlineto{\pgfqpoint{0.926858in}{1.810259in}}%
\pgfpathlineto{\pgfqpoint{0.932628in}{1.792535in}}%
\pgfpathlineto{\pgfqpoint{0.934551in}{1.794486in}}%
\pgfpathlineto{\pgfqpoint{0.938398in}{1.773971in}}%
\pgfpathlineto{\pgfqpoint{0.940322in}{1.779073in}}%
\pgfpathlineto{\pgfqpoint{0.942245in}{1.791123in}}%
\pgfpathlineto{\pgfqpoint{0.946092in}{1.792723in}}%
\pgfpathlineto{\pgfqpoint{0.948015in}{1.803501in}}%
\pgfpathlineto{\pgfqpoint{0.949939in}{1.792393in}}%
\pgfpathlineto{\pgfqpoint{0.951862in}{1.803127in}}%
\pgfpathlineto{\pgfqpoint{0.953785in}{1.804900in}}%
\pgfpathlineto{\pgfqpoint{0.955709in}{1.800597in}}%
\pgfpathlineto{\pgfqpoint{0.957632in}{1.787366in}}%
\pgfpathlineto{\pgfqpoint{0.959556in}{1.782816in}}%
\pgfpathlineto{\pgfqpoint{0.961479in}{1.781533in}}%
\pgfpathlineto{\pgfqpoint{0.963402in}{1.785072in}}%
\pgfpathlineto{\pgfqpoint{0.965326in}{1.803565in}}%
\pgfpathlineto{\pgfqpoint{0.969173in}{1.801006in}}%
\pgfpathlineto{\pgfqpoint{0.971096in}{1.789682in}}%
\pgfpathlineto{\pgfqpoint{0.973020in}{1.801066in}}%
\pgfpathlineto{\pgfqpoint{0.978790in}{1.809530in}}%
\pgfpathlineto{\pgfqpoint{0.980713in}{1.809045in}}%
\pgfpathlineto{\pgfqpoint{0.982637in}{1.806228in}}%
\pgfpathlineto{\pgfqpoint{0.986483in}{1.795941in}}%
\pgfpathlineto{\pgfqpoint{0.988407in}{1.779866in}}%
\pgfpathlineto{\pgfqpoint{0.990330in}{1.792616in}}%
\pgfpathlineto{\pgfqpoint{0.992254in}{1.788668in}}%
\pgfpathlineto{\pgfqpoint{0.996100in}{1.789171in}}%
\pgfpathlineto{\pgfqpoint{0.998024in}{1.793400in}}%
\pgfpathlineto{\pgfqpoint{0.999947in}{1.793336in}}%
\pgfpathlineto{\pgfqpoint{1.001871in}{1.797099in}}%
\pgfpathlineto{\pgfqpoint{1.003794in}{1.784402in}}%
\pgfpathlineto{\pgfqpoint{1.005717in}{1.850261in}}%
\pgfpathlineto{\pgfqpoint{1.007641in}{1.858556in}}%
\pgfpathlineto{\pgfqpoint{1.009564in}{1.854133in}}%
\pgfpathlineto{\pgfqpoint{1.011488in}{1.857546in}}%
\pgfpathlineto{\pgfqpoint{1.013411in}{1.866559in}}%
\pgfpathlineto{\pgfqpoint{1.015335in}{1.860126in}}%
\pgfpathlineto{\pgfqpoint{1.017258in}{1.858434in}}%
\pgfpathlineto{\pgfqpoint{1.019181in}{1.854616in}}%
\pgfpathlineto{\pgfqpoint{1.023028in}{1.854337in}}%
\pgfpathlineto{\pgfqpoint{1.024952in}{1.852665in}}%
\pgfpathlineto{\pgfqpoint{1.026875in}{1.841089in}}%
\pgfpathlineto{\pgfqpoint{1.028798in}{1.851449in}}%
\pgfpathlineto{\pgfqpoint{1.034569in}{1.830079in}}%
\pgfpathlineto{\pgfqpoint{1.036492in}{1.825779in}}%
\pgfpathlineto{\pgfqpoint{1.038415in}{1.829156in}}%
\pgfpathlineto{\pgfqpoint{1.040339in}{1.822824in}}%
\pgfpathlineto{\pgfqpoint{1.048033in}{1.855737in}}%
\pgfpathlineto{\pgfqpoint{1.049956in}{1.857097in}}%
\pgfpathlineto{\pgfqpoint{1.051879in}{1.863980in}}%
\pgfpathlineto{\pgfqpoint{1.053803in}{1.864229in}}%
\pgfpathlineto{\pgfqpoint{1.055726in}{1.840571in}}%
\pgfpathlineto{\pgfqpoint{1.061496in}{1.852021in}}%
\pgfpathlineto{\pgfqpoint{1.063420in}{1.844097in}}%
\pgfpathlineto{\pgfqpoint{1.067267in}{1.854328in}}%
\pgfpathlineto{\pgfqpoint{1.069190in}{1.848374in}}%
\pgfpathlineto{\pgfqpoint{1.071113in}{1.835728in}}%
\pgfpathlineto{\pgfqpoint{1.074960in}{1.853200in}}%
\pgfpathlineto{\pgfqpoint{1.078807in}{1.860723in}}%
\pgfpathlineto{\pgfqpoint{1.080731in}{1.859994in}}%
\pgfpathlineto{\pgfqpoint{1.082654in}{1.855294in}}%
\pgfpathlineto{\pgfqpoint{1.084577in}{1.853556in}}%
\pgfpathlineto{\pgfqpoint{1.086501in}{1.853856in}}%
\pgfpathlineto{\pgfqpoint{1.088424in}{1.856824in}}%
\pgfpathlineto{\pgfqpoint{1.090348in}{1.865941in}}%
\pgfpathlineto{\pgfqpoint{1.092271in}{1.856477in}}%
\pgfpathlineto{\pgfqpoint{1.094194in}{1.854310in}}%
\pgfpathlineto{\pgfqpoint{1.096118in}{1.855970in}}%
\pgfpathlineto{\pgfqpoint{1.099965in}{1.830227in}}%
\pgfpathlineto{\pgfqpoint{1.101888in}{1.831926in}}%
\pgfpathlineto{\pgfqpoint{1.103811in}{1.830991in}}%
\pgfpathlineto{\pgfqpoint{1.105735in}{1.820952in}}%
\pgfpathlineto{\pgfqpoint{1.107658in}{1.825550in}}%
\pgfpathlineto{\pgfqpoint{1.109582in}{1.824067in}}%
\pgfpathlineto{\pgfqpoint{1.113429in}{1.826705in}}%
\pgfpathlineto{\pgfqpoint{1.115352in}{1.844347in}}%
\pgfpathlineto{\pgfqpoint{1.117275in}{1.833700in}}%
\pgfpathlineto{\pgfqpoint{1.119199in}{1.831520in}}%
\pgfpathlineto{\pgfqpoint{1.121122in}{1.836952in}}%
\pgfpathlineto{\pgfqpoint{1.123046in}{1.825089in}}%
\pgfpathlineto{\pgfqpoint{1.124969in}{1.821212in}}%
\pgfpathlineto{\pgfqpoint{1.130739in}{1.835992in}}%
\pgfpathlineto{\pgfqpoint{1.132663in}{1.826185in}}%
\pgfpathlineto{\pgfqpoint{1.134586in}{1.831767in}}%
\pgfpathlineto{\pgfqpoint{1.136509in}{1.812815in}}%
\pgfpathlineto{\pgfqpoint{1.138433in}{1.815615in}}%
\pgfpathlineto{\pgfqpoint{1.140356in}{1.806331in}}%
\pgfpathlineto{\pgfqpoint{1.142280in}{1.805661in}}%
\pgfpathlineto{\pgfqpoint{1.144203in}{1.816528in}}%
\pgfpathlineto{\pgfqpoint{1.146126in}{1.816173in}}%
\pgfpathlineto{\pgfqpoint{1.148050in}{1.832450in}}%
\pgfpathlineto{\pgfqpoint{1.149973in}{1.836133in}}%
\pgfpathlineto{\pgfqpoint{1.153820in}{1.795218in}}%
\pgfpathlineto{\pgfqpoint{1.157667in}{1.785043in}}%
\pgfpathlineto{\pgfqpoint{1.159590in}{1.771636in}}%
\pgfpathlineto{\pgfqpoint{1.161514in}{1.782319in}}%
\pgfpathlineto{\pgfqpoint{1.163437in}{1.783329in}}%
\pgfpathlineto{\pgfqpoint{1.165361in}{1.780901in}}%
\pgfpathlineto{\pgfqpoint{1.167284in}{1.790691in}}%
\pgfpathlineto{\pgfqpoint{1.169207in}{1.791751in}}%
\pgfpathlineto{\pgfqpoint{1.171131in}{1.795639in}}%
\pgfpathlineto{\pgfqpoint{1.174978in}{1.778371in}}%
\pgfpathlineto{\pgfqpoint{1.182671in}{1.814265in}}%
\pgfpathlineto{\pgfqpoint{1.184595in}{1.817993in}}%
\pgfpathlineto{\pgfqpoint{1.186518in}{1.825296in}}%
\pgfpathlineto{\pgfqpoint{1.188442in}{1.824030in}}%
\pgfpathlineto{\pgfqpoint{1.190365in}{1.807895in}}%
\pgfpathlineto{\pgfqpoint{1.196135in}{1.830916in}}%
\pgfpathlineto{\pgfqpoint{1.199982in}{1.845276in}}%
\pgfpathlineto{\pgfqpoint{1.201905in}{1.848495in}}%
\pgfpathlineto{\pgfqpoint{1.203829in}{1.842647in}}%
\pgfpathlineto{\pgfqpoint{1.207676in}{1.861921in}}%
\pgfpathlineto{\pgfqpoint{1.209599in}{1.864413in}}%
\pgfpathlineto{\pgfqpoint{1.211522in}{1.864404in}}%
\pgfpathlineto{\pgfqpoint{1.213446in}{1.868159in}}%
\pgfpathlineto{\pgfqpoint{1.215369in}{1.865591in}}%
\pgfpathlineto{\pgfqpoint{1.217293in}{1.855784in}}%
\pgfpathlineto{\pgfqpoint{1.219216in}{1.856045in}}%
\pgfpathlineto{\pgfqpoint{1.223063in}{1.835210in}}%
\pgfpathlineto{\pgfqpoint{1.224986in}{1.839547in}}%
\pgfpathlineto{\pgfqpoint{1.226910in}{1.838480in}}%
\pgfpathlineto{\pgfqpoint{1.228833in}{1.851947in}}%
\pgfpathlineto{\pgfqpoint{1.234603in}{1.841511in}}%
\pgfpathlineto{\pgfqpoint{1.236527in}{1.848034in}}%
\pgfpathlineto{\pgfqpoint{1.240374in}{1.833209in}}%
\pgfpathlineto{\pgfqpoint{1.242297in}{1.835226in}}%
\pgfpathlineto{\pgfqpoint{1.248067in}{1.846026in}}%
\pgfpathlineto{\pgfqpoint{1.249991in}{1.844438in}}%
\pgfpathlineto{\pgfqpoint{1.251914in}{1.827681in}}%
\pgfpathlineto{\pgfqpoint{1.253838in}{1.833346in}}%
\pgfpathlineto{\pgfqpoint{1.255761in}{1.843848in}}%
\pgfpathlineto{\pgfqpoint{1.261531in}{1.847617in}}%
\pgfpathlineto{\pgfqpoint{1.265378in}{1.856285in}}%
\pgfpathlineto{\pgfqpoint{1.267301in}{1.856921in}}%
\pgfpathlineto{\pgfqpoint{1.269225in}{1.854115in}}%
\pgfpathlineto{\pgfqpoint{1.271148in}{1.859625in}}%
\pgfpathlineto{\pgfqpoint{1.273072in}{1.859692in}}%
\pgfpathlineto{\pgfqpoint{1.276918in}{1.870272in}}%
\pgfpathlineto{\pgfqpoint{1.280765in}{1.872369in}}%
\pgfpathlineto{\pgfqpoint{1.282689in}{1.864155in}}%
\pgfpathlineto{\pgfqpoint{1.284612in}{1.863948in}}%
\pgfpathlineto{\pgfqpoint{1.288459in}{1.860182in}}%
\pgfpathlineto{\pgfqpoint{1.290382in}{1.870915in}}%
\pgfpathlineto{\pgfqpoint{1.292306in}{1.870239in}}%
\pgfpathlineto{\pgfqpoint{1.296153in}{1.881817in}}%
\pgfpathlineto{\pgfqpoint{1.298076in}{1.875267in}}%
\pgfpathlineto{\pgfqpoint{1.299999in}{1.882269in}}%
\pgfpathlineto{\pgfqpoint{1.301923in}{1.884634in}}%
\pgfpathlineto{\pgfqpoint{1.305770in}{1.871903in}}%
\pgfpathlineto{\pgfqpoint{1.311540in}{1.902766in}}%
\pgfpathlineto{\pgfqpoint{1.317310in}{1.882593in}}%
\pgfpathlineto{\pgfqpoint{1.319233in}{1.885942in}}%
\pgfpathlineto{\pgfqpoint{1.321157in}{1.877439in}}%
\pgfpathlineto{\pgfqpoint{1.323080in}{1.896305in}}%
\pgfpathlineto{\pgfqpoint{1.325004in}{1.888475in}}%
\pgfpathlineto{\pgfqpoint{1.326927in}{1.896500in}}%
\pgfpathlineto{\pgfqpoint{1.330774in}{1.889620in}}%
\pgfpathlineto{\pgfqpoint{1.334621in}{1.904385in}}%
\pgfpathlineto{\pgfqpoint{1.338468in}{1.897023in}}%
\pgfpathlineto{\pgfqpoint{1.340391in}{1.911177in}}%
\pgfpathlineto{\pgfqpoint{1.342314in}{1.916315in}}%
\pgfpathlineto{\pgfqpoint{1.344238in}{1.909508in}}%
\pgfpathlineto{\pgfqpoint{1.351931in}{1.934325in}}%
\pgfpathlineto{\pgfqpoint{1.353855in}{1.932727in}}%
\pgfpathlineto{\pgfqpoint{1.355778in}{1.927277in}}%
\pgfpathlineto{\pgfqpoint{1.357702in}{1.917085in}}%
\pgfpathlineto{\pgfqpoint{1.359625in}{1.927441in}}%
\pgfpathlineto{\pgfqpoint{1.361549in}{1.927870in}}%
\pgfpathlineto{\pgfqpoint{1.363472in}{1.930559in}}%
\pgfpathlineto{\pgfqpoint{1.367319in}{1.944588in}}%
\pgfpathlineto{\pgfqpoint{1.369242in}{1.942837in}}%
\pgfpathlineto{\pgfqpoint{1.371166in}{1.931348in}}%
\pgfpathlineto{\pgfqpoint{1.373089in}{1.938202in}}%
\pgfpathlineto{\pgfqpoint{1.375012in}{1.937545in}}%
\pgfpathlineto{\pgfqpoint{1.376936in}{1.938795in}}%
\pgfpathlineto{\pgfqpoint{1.382706in}{1.925660in}}%
\pgfpathlineto{\pgfqpoint{1.384629in}{1.913442in}}%
\pgfpathlineto{\pgfqpoint{1.386553in}{1.913092in}}%
\pgfpathlineto{\pgfqpoint{1.388476in}{1.893284in}}%
\pgfpathlineto{\pgfqpoint{1.390400in}{1.900014in}}%
\pgfpathlineto{\pgfqpoint{1.394247in}{1.896861in}}%
\pgfpathlineto{\pgfqpoint{1.396170in}{1.898415in}}%
\pgfpathlineto{\pgfqpoint{1.403864in}{1.891130in}}%
\pgfpathlineto{\pgfqpoint{1.405787in}{1.871625in}}%
\pgfpathlineto{\pgfqpoint{1.409634in}{1.866900in}}%
\pgfpathlineto{\pgfqpoint{1.411557in}{1.866365in}}%
\pgfpathlineto{\pgfqpoint{1.413481in}{1.853215in}}%
\pgfpathlineto{\pgfqpoint{1.415404in}{1.858110in}}%
\pgfpathlineto{\pgfqpoint{1.417327in}{1.859773in}}%
\pgfpathlineto{\pgfqpoint{1.419251in}{1.864090in}}%
\pgfpathlineto{\pgfqpoint{1.421174in}{1.855540in}}%
\pgfpathlineto{\pgfqpoint{1.425021in}{1.859790in}}%
\pgfpathlineto{\pgfqpoint{1.426945in}{1.869765in}}%
\pgfpathlineto{\pgfqpoint{1.428868in}{1.870822in}}%
\pgfpathlineto{\pgfqpoint{1.434638in}{1.832402in}}%
\pgfpathlineto{\pgfqpoint{1.436562in}{1.837227in}}%
\pgfpathlineto{\pgfqpoint{1.438485in}{1.845128in}}%
\pgfpathlineto{\pgfqpoint{1.440408in}{1.845051in}}%
\pgfpathlineto{\pgfqpoint{1.442332in}{1.838442in}}%
\pgfpathlineto{\pgfqpoint{1.446179in}{1.855368in}}%
\pgfpathlineto{\pgfqpoint{1.448102in}{1.857123in}}%
\pgfpathlineto{\pgfqpoint{1.450025in}{1.868163in}}%
\pgfpathlineto{\pgfqpoint{1.453872in}{1.864261in}}%
\pgfpathlineto{\pgfqpoint{1.455796in}{1.884117in}}%
\pgfpathlineto{\pgfqpoint{1.457719in}{1.887820in}}%
\pgfpathlineto{\pgfqpoint{1.459642in}{1.870173in}}%
\pgfpathlineto{\pgfqpoint{1.461566in}{1.886738in}}%
\pgfpathlineto{\pgfqpoint{1.463489in}{1.884594in}}%
\pgfpathlineto{\pgfqpoint{1.465413in}{1.898757in}}%
\pgfpathlineto{\pgfqpoint{1.467336in}{1.895460in}}%
\pgfpathlineto{\pgfqpoint{1.469260in}{1.906184in}}%
\pgfpathlineto{\pgfqpoint{1.471183in}{1.907983in}}%
\pgfpathlineto{\pgfqpoint{1.473106in}{1.907505in}}%
\pgfpathlineto{\pgfqpoint{1.475030in}{1.919246in}}%
\pgfpathlineto{\pgfqpoint{1.476953in}{1.903926in}}%
\pgfpathlineto{\pgfqpoint{1.478877in}{1.907869in}}%
\pgfpathlineto{\pgfqpoint{1.480800in}{1.914680in}}%
\pgfpathlineto{\pgfqpoint{1.482723in}{1.909900in}}%
\pgfpathlineto{\pgfqpoint{1.484647in}{1.923608in}}%
\pgfpathlineto{\pgfqpoint{1.486570in}{1.929101in}}%
\pgfpathlineto{\pgfqpoint{1.490417in}{1.897035in}}%
\pgfpathlineto{\pgfqpoint{1.492340in}{1.908065in}}%
\pgfpathlineto{\pgfqpoint{1.496187in}{1.914730in}}%
\pgfpathlineto{\pgfqpoint{1.498111in}{1.914703in}}%
\pgfpathlineto{\pgfqpoint{1.500034in}{1.919156in}}%
\pgfpathlineto{\pgfqpoint{1.503881in}{1.897864in}}%
\pgfpathlineto{\pgfqpoint{1.505804in}{1.913681in}}%
\pgfpathlineto{\pgfqpoint{1.507728in}{1.898736in}}%
\pgfpathlineto{\pgfqpoint{1.509651in}{1.895125in}}%
\pgfpathlineto{\pgfqpoint{1.511575in}{1.900790in}}%
\pgfpathlineto{\pgfqpoint{1.513498in}{1.900730in}}%
\pgfpathlineto{\pgfqpoint{1.515421in}{1.892858in}}%
\pgfpathlineto{\pgfqpoint{1.519268in}{1.904626in}}%
\pgfpathlineto{\pgfqpoint{1.521192in}{1.902389in}}%
\pgfpathlineto{\pgfqpoint{1.523115in}{1.910686in}}%
\pgfpathlineto{\pgfqpoint{1.525038in}{1.908614in}}%
\pgfpathlineto{\pgfqpoint{1.526962in}{1.903585in}}%
\pgfpathlineto{\pgfqpoint{1.528885in}{1.918131in}}%
\pgfpathlineto{\pgfqpoint{1.530809in}{1.923007in}}%
\pgfpathlineto{\pgfqpoint{1.532732in}{1.915783in}}%
\pgfpathlineto{\pgfqpoint{1.534656in}{1.916784in}}%
\pgfpathlineto{\pgfqpoint{1.536579in}{1.900707in}}%
\pgfpathlineto{\pgfqpoint{1.538502in}{1.914662in}}%
\pgfpathlineto{\pgfqpoint{1.540426in}{1.906274in}}%
\pgfpathlineto{\pgfqpoint{1.542349in}{1.905528in}}%
\pgfpathlineto{\pgfqpoint{1.544273in}{1.907967in}}%
\pgfpathlineto{\pgfqpoint{1.546196in}{1.900644in}}%
\pgfpathlineto{\pgfqpoint{1.548119in}{1.901379in}}%
\pgfpathlineto{\pgfqpoint{1.551966in}{1.894835in}}%
\pgfpathlineto{\pgfqpoint{1.553890in}{1.898461in}}%
\pgfpathlineto{\pgfqpoint{1.555813in}{1.888981in}}%
\pgfpathlineto{\pgfqpoint{1.557736in}{1.900654in}}%
\pgfpathlineto{\pgfqpoint{1.559660in}{1.890536in}}%
\pgfpathlineto{\pgfqpoint{1.561583in}{1.891871in}}%
\pgfpathlineto{\pgfqpoint{1.563507in}{1.882556in}}%
\pgfpathlineto{\pgfqpoint{1.565430in}{1.888536in}}%
\pgfpathlineto{\pgfqpoint{1.567354in}{1.885338in}}%
\pgfpathlineto{\pgfqpoint{1.571200in}{1.872295in}}%
\pgfpathlineto{\pgfqpoint{1.575047in}{1.868763in}}%
\pgfpathlineto{\pgfqpoint{1.576971in}{1.861554in}}%
\pgfpathlineto{\pgfqpoint{1.578894in}{1.873777in}}%
\pgfpathlineto{\pgfqpoint{1.580817in}{1.874502in}}%
\pgfpathlineto{\pgfqpoint{1.582741in}{1.864552in}}%
\pgfpathlineto{\pgfqpoint{1.584664in}{1.847044in}}%
\pgfpathlineto{\pgfqpoint{1.588511in}{1.841607in}}%
\pgfpathlineto{\pgfqpoint{1.590434in}{1.863515in}}%
\pgfpathlineto{\pgfqpoint{1.592358in}{1.869449in}}%
\pgfpathlineto{\pgfqpoint{1.594281in}{1.881364in}}%
\pgfpathlineto{\pgfqpoint{1.596205in}{1.871035in}}%
\pgfpathlineto{\pgfqpoint{1.598128in}{1.869448in}}%
\pgfpathlineto{\pgfqpoint{1.600051in}{1.871514in}}%
\pgfpathlineto{\pgfqpoint{1.601975in}{1.869751in}}%
\pgfpathlineto{\pgfqpoint{1.603898in}{1.870008in}}%
\pgfpathlineto{\pgfqpoint{1.605822in}{1.871683in}}%
\pgfpathlineto{\pgfqpoint{1.607745in}{1.891952in}}%
\pgfpathlineto{\pgfqpoint{1.609669in}{1.898979in}}%
\pgfpathlineto{\pgfqpoint{1.611592in}{1.910421in}}%
\pgfpathlineto{\pgfqpoint{1.613515in}{1.912801in}}%
\pgfpathlineto{\pgfqpoint{1.615439in}{1.901003in}}%
\pgfpathlineto{\pgfqpoint{1.617362in}{1.913212in}}%
\pgfpathlineto{\pgfqpoint{1.621209in}{1.915503in}}%
\pgfpathlineto{\pgfqpoint{1.623132in}{1.910317in}}%
\pgfpathlineto{\pgfqpoint{1.625056in}{1.921193in}}%
\pgfpathlineto{\pgfqpoint{1.626979in}{1.918931in}}%
\pgfpathlineto{\pgfqpoint{1.628903in}{1.931382in}}%
\pgfpathlineto{\pgfqpoint{1.630826in}{1.931905in}}%
\pgfpathlineto{\pgfqpoint{1.632749in}{1.935936in}}%
\pgfpathlineto{\pgfqpoint{1.634673in}{1.937275in}}%
\pgfpathlineto{\pgfqpoint{1.636596in}{1.943479in}}%
\pgfpathlineto{\pgfqpoint{1.638520in}{1.929351in}}%
\pgfpathlineto{\pgfqpoint{1.644290in}{1.939603in}}%
\pgfpathlineto{\pgfqpoint{1.646213in}{1.958751in}}%
\pgfpathlineto{\pgfqpoint{1.650060in}{1.944169in}}%
\pgfpathlineto{\pgfqpoint{1.653907in}{1.930942in}}%
\pgfpathlineto{\pgfqpoint{1.655830in}{1.932595in}}%
\pgfpathlineto{\pgfqpoint{1.661601in}{1.958261in}}%
\pgfpathlineto{\pgfqpoint{1.663524in}{1.952444in}}%
\pgfpathlineto{\pgfqpoint{1.665447in}{1.956317in}}%
\pgfpathlineto{\pgfqpoint{1.667371in}{1.955960in}}%
\pgfpathlineto{\pgfqpoint{1.669294in}{1.972243in}}%
\pgfpathlineto{\pgfqpoint{1.671218in}{1.973467in}}%
\pgfpathlineto{\pgfqpoint{1.673141in}{1.979869in}}%
\pgfpathlineto{\pgfqpoint{1.675065in}{1.967109in}}%
\pgfpathlineto{\pgfqpoint{1.676988in}{1.964770in}}%
\pgfpathlineto{\pgfqpoint{1.678911in}{1.974888in}}%
\pgfpathlineto{\pgfqpoint{1.680835in}{1.973134in}}%
\pgfpathlineto{\pgfqpoint{1.682758in}{1.988745in}}%
\pgfpathlineto{\pgfqpoint{1.684682in}{1.990625in}}%
\pgfpathlineto{\pgfqpoint{1.686605in}{1.988883in}}%
\pgfpathlineto{\pgfqpoint{1.690452in}{1.969229in}}%
\pgfpathlineto{\pgfqpoint{1.694299in}{1.975813in}}%
\pgfpathlineto{\pgfqpoint{1.696222in}{1.968099in}}%
\pgfpathlineto{\pgfqpoint{1.701992in}{1.989083in}}%
\pgfpathlineto{\pgfqpoint{1.703916in}{1.987668in}}%
\pgfpathlineto{\pgfqpoint{1.705839in}{1.974290in}}%
\pgfpathlineto{\pgfqpoint{1.707763in}{1.972184in}}%
\pgfpathlineto{\pgfqpoint{1.709686in}{1.968264in}}%
\pgfpathlineto{\pgfqpoint{1.711609in}{1.959065in}}%
\pgfpathlineto{\pgfqpoint{1.715456in}{1.954806in}}%
\pgfpathlineto{\pgfqpoint{1.717380in}{1.955566in}}%
\pgfpathlineto{\pgfqpoint{1.719303in}{1.944525in}}%
\pgfpathlineto{\pgfqpoint{1.723150in}{1.959228in}}%
\pgfpathlineto{\pgfqpoint{1.725073in}{1.954238in}}%
\pgfpathlineto{\pgfqpoint{1.726997in}{1.960825in}}%
\pgfpathlineto{\pgfqpoint{1.728920in}{1.958655in}}%
\pgfpathlineto{\pgfqpoint{1.730843in}{1.952223in}}%
\pgfpathlineto{\pgfqpoint{1.732767in}{1.965313in}}%
\pgfpathlineto{\pgfqpoint{1.734690in}{1.964798in}}%
\pgfpathlineto{\pgfqpoint{1.736614in}{1.966818in}}%
\pgfpathlineto{\pgfqpoint{1.738537in}{1.970989in}}%
\pgfpathlineto{\pgfqpoint{1.740461in}{1.978519in}}%
\pgfpathlineto{\pgfqpoint{1.742384in}{1.981301in}}%
\pgfpathlineto{\pgfqpoint{1.744307in}{1.972168in}}%
\pgfpathlineto{\pgfqpoint{1.746231in}{1.984110in}}%
\pgfpathlineto{\pgfqpoint{1.748154in}{1.983492in}}%
\pgfpathlineto{\pgfqpoint{1.750078in}{1.968776in}}%
\pgfpathlineto{\pgfqpoint{1.752001in}{1.966040in}}%
\pgfpathlineto{\pgfqpoint{1.753924in}{1.967852in}}%
\pgfpathlineto{\pgfqpoint{1.755848in}{1.961043in}}%
\pgfpathlineto{\pgfqpoint{1.757771in}{1.963538in}}%
\pgfpathlineto{\pgfqpoint{1.759695in}{1.977960in}}%
\pgfpathlineto{\pgfqpoint{1.761618in}{1.976867in}}%
\pgfpathlineto{\pgfqpoint{1.763541in}{1.979600in}}%
\pgfpathlineto{\pgfqpoint{1.765465in}{1.979691in}}%
\pgfpathlineto{\pgfqpoint{1.767388in}{1.978507in}}%
\pgfpathlineto{\pgfqpoint{1.769312in}{1.992063in}}%
\pgfpathlineto{\pgfqpoint{1.773158in}{1.990900in}}%
\pgfpathlineto{\pgfqpoint{1.777005in}{1.963024in}}%
\pgfpathlineto{\pgfqpoint{1.778929in}{1.965947in}}%
\pgfpathlineto{\pgfqpoint{1.780852in}{1.962735in}}%
\pgfpathlineto{\pgfqpoint{1.786622in}{1.982376in}}%
\pgfpathlineto{\pgfqpoint{1.790469in}{1.966735in}}%
\pgfpathlineto{\pgfqpoint{1.792393in}{1.970929in}}%
\pgfpathlineto{\pgfqpoint{1.794316in}{1.966743in}}%
\pgfpathlineto{\pgfqpoint{1.796239in}{1.969746in}}%
\pgfpathlineto{\pgfqpoint{1.798163in}{1.968731in}}%
\pgfpathlineto{\pgfqpoint{1.800086in}{1.972189in}}%
\pgfpathlineto{\pgfqpoint{1.803933in}{1.950226in}}%
\pgfpathlineto{\pgfqpoint{1.807780in}{1.965259in}}%
\pgfpathlineto{\pgfqpoint{1.809703in}{1.958070in}}%
\pgfpathlineto{\pgfqpoint{1.813550in}{1.969047in}}%
\pgfpathlineto{\pgfqpoint{1.819320in}{1.931028in}}%
\pgfpathlineto{\pgfqpoint{1.821244in}{1.943346in}}%
\pgfpathlineto{\pgfqpoint{1.823167in}{1.940982in}}%
\pgfpathlineto{\pgfqpoint{1.827014in}{1.920296in}}%
\pgfpathlineto{\pgfqpoint{1.830861in}{1.925765in}}%
\pgfpathlineto{\pgfqpoint{1.832784in}{1.920121in}}%
\pgfpathlineto{\pgfqpoint{1.834708in}{1.918454in}}%
\pgfpathlineto{\pgfqpoint{1.836631in}{1.923086in}}%
\pgfpathlineto{\pgfqpoint{1.840478in}{1.926137in}}%
\pgfpathlineto{\pgfqpoint{1.842401in}{1.933836in}}%
\pgfpathlineto{\pgfqpoint{1.844325in}{1.927641in}}%
\pgfpathlineto{\pgfqpoint{1.846248in}{1.931531in}}%
\pgfpathlineto{\pgfqpoint{1.848172in}{1.942529in}}%
\pgfpathlineto{\pgfqpoint{1.850095in}{1.943041in}}%
\pgfpathlineto{\pgfqpoint{1.852018in}{1.941666in}}%
\pgfpathlineto{\pgfqpoint{1.855865in}{1.928215in}}%
\pgfpathlineto{\pgfqpoint{1.859712in}{1.934184in}}%
\pgfpathlineto{\pgfqpoint{1.863559in}{1.937125in}}%
\pgfpathlineto{\pgfqpoint{1.865482in}{1.933263in}}%
\pgfpathlineto{\pgfqpoint{1.867406in}{1.915412in}}%
\pgfpathlineto{\pgfqpoint{1.869329in}{1.914258in}}%
\pgfpathlineto{\pgfqpoint{1.871252in}{1.923153in}}%
\pgfpathlineto{\pgfqpoint{1.878946in}{1.884508in}}%
\pgfpathlineto{\pgfqpoint{1.880870in}{1.885546in}}%
\pgfpathlineto{\pgfqpoint{1.882793in}{1.874167in}}%
\pgfpathlineto{\pgfqpoint{1.886640in}{1.878616in}}%
\pgfpathlineto{\pgfqpoint{1.888563in}{1.874339in}}%
\pgfpathlineto{\pgfqpoint{1.890487in}{1.883190in}}%
\pgfpathlineto{\pgfqpoint{1.892410in}{1.884134in}}%
\pgfpathlineto{\pgfqpoint{1.894333in}{1.894408in}}%
\pgfpathlineto{\pgfqpoint{1.898180in}{1.893551in}}%
\pgfpathlineto{\pgfqpoint{1.900104in}{1.895853in}}%
\pgfpathlineto{\pgfqpoint{1.902027in}{1.901503in}}%
\pgfpathlineto{\pgfqpoint{1.903950in}{1.895634in}}%
\pgfpathlineto{\pgfqpoint{1.905874in}{1.898079in}}%
\pgfpathlineto{\pgfqpoint{1.907797in}{1.894212in}}%
\pgfpathlineto{\pgfqpoint{1.913567in}{1.857766in}}%
\pgfpathlineto{\pgfqpoint{1.915491in}{1.852142in}}%
\pgfpathlineto{\pgfqpoint{1.919338in}{1.850869in}}%
\pgfpathlineto{\pgfqpoint{1.921261in}{1.843789in}}%
\pgfpathlineto{\pgfqpoint{1.925108in}{1.839364in}}%
\pgfpathlineto{\pgfqpoint{1.927031in}{1.840756in}}%
\pgfpathlineto{\pgfqpoint{1.928955in}{1.829366in}}%
\pgfpathlineto{\pgfqpoint{1.932802in}{1.836152in}}%
\pgfpathlineto{\pgfqpoint{1.934725in}{1.838557in}}%
\pgfpathlineto{\pgfqpoint{1.936648in}{1.838220in}}%
\pgfpathlineto{\pgfqpoint{1.938572in}{1.845935in}}%
\pgfpathlineto{\pgfqpoint{1.942419in}{1.834527in}}%
\pgfpathlineto{\pgfqpoint{1.944342in}{1.838140in}}%
\pgfpathlineto{\pgfqpoint{1.946265in}{1.825876in}}%
\pgfpathlineto{\pgfqpoint{1.948189in}{1.825823in}}%
\pgfpathlineto{\pgfqpoint{1.950112in}{1.820435in}}%
\pgfpathlineto{\pgfqpoint{1.952036in}{1.841187in}}%
\pgfpathlineto{\pgfqpoint{1.953959in}{1.838070in}}%
\pgfpathlineto{\pgfqpoint{1.957806in}{1.853927in}}%
\pgfpathlineto{\pgfqpoint{1.959729in}{1.855563in}}%
\pgfpathlineto{\pgfqpoint{1.961653in}{1.850683in}}%
\pgfpathlineto{\pgfqpoint{1.963576in}{1.850898in}}%
\pgfpathlineto{\pgfqpoint{1.965500in}{1.845574in}}%
\pgfpathlineto{\pgfqpoint{1.967423in}{1.848740in}}%
\pgfpathlineto{\pgfqpoint{1.969346in}{1.857644in}}%
\pgfpathlineto{\pgfqpoint{1.971270in}{1.860732in}}%
\pgfpathlineto{\pgfqpoint{1.973193in}{1.858712in}}%
\pgfpathlineto{\pgfqpoint{1.975117in}{1.875416in}}%
\pgfpathlineto{\pgfqpoint{1.980887in}{1.852337in}}%
\pgfpathlineto{\pgfqpoint{1.984734in}{1.864954in}}%
\pgfpathlineto{\pgfqpoint{1.986657in}{1.862618in}}%
\pgfpathlineto{\pgfqpoint{1.988581in}{1.865271in}}%
\pgfpathlineto{\pgfqpoint{1.990504in}{1.865897in}}%
\pgfpathlineto{\pgfqpoint{1.992427in}{1.849813in}}%
\pgfpathlineto{\pgfqpoint{1.994351in}{1.846942in}}%
\pgfpathlineto{\pgfqpoint{1.996274in}{1.849741in}}%
\pgfpathlineto{\pgfqpoint{1.998198in}{1.842574in}}%
\pgfpathlineto{\pgfqpoint{2.002044in}{1.837765in}}%
\pgfpathlineto{\pgfqpoint{2.003968in}{1.846042in}}%
\pgfpathlineto{\pgfqpoint{2.005891in}{1.848464in}}%
\pgfpathlineto{\pgfqpoint{2.007815in}{1.833907in}}%
\pgfpathlineto{\pgfqpoint{2.011661in}{1.838143in}}%
\pgfpathlineto{\pgfqpoint{2.013585in}{1.852882in}}%
\pgfpathlineto{\pgfqpoint{2.017432in}{1.848603in}}%
\pgfpathlineto{\pgfqpoint{2.019355in}{1.856668in}}%
\pgfpathlineto{\pgfqpoint{2.021279in}{1.851144in}}%
\pgfpathlineto{\pgfqpoint{2.023202in}{1.856222in}}%
\pgfpathlineto{\pgfqpoint{2.025125in}{1.854473in}}%
\pgfpathlineto{\pgfqpoint{2.027049in}{1.856756in}}%
\pgfpathlineto{\pgfqpoint{2.028972in}{1.852828in}}%
\pgfpathlineto{\pgfqpoint{2.030896in}{1.852300in}}%
\pgfpathlineto{\pgfqpoint{2.032819in}{1.858723in}}%
\pgfpathlineto{\pgfqpoint{2.034742in}{1.849042in}}%
\pgfpathlineto{\pgfqpoint{2.036666in}{1.846279in}}%
\pgfpathlineto{\pgfqpoint{2.038589in}{1.837956in}}%
\pgfpathlineto{\pgfqpoint{2.040513in}{1.847883in}}%
\pgfpathlineto{\pgfqpoint{2.042436in}{1.849242in}}%
\pgfpathlineto{\pgfqpoint{2.044359in}{1.840732in}}%
\pgfpathlineto{\pgfqpoint{2.046283in}{1.838969in}}%
\pgfpathlineto{\pgfqpoint{2.050130in}{1.852890in}}%
\pgfpathlineto{\pgfqpoint{2.055900in}{1.861296in}}%
\pgfpathlineto{\pgfqpoint{2.057823in}{1.847841in}}%
\pgfpathlineto{\pgfqpoint{2.059747in}{1.855853in}}%
\pgfpathlineto{\pgfqpoint{2.063594in}{1.859911in}}%
\pgfpathlineto{\pgfqpoint{2.069364in}{1.887954in}}%
\pgfpathlineto{\pgfqpoint{2.071287in}{1.891716in}}%
\pgfpathlineto{\pgfqpoint{2.073211in}{1.885224in}}%
\pgfpathlineto{\pgfqpoint{2.077057in}{1.911724in}}%
\pgfpathlineto{\pgfqpoint{2.080904in}{1.912777in}}%
\pgfpathlineto{\pgfqpoint{2.084751in}{1.919392in}}%
\pgfpathlineto{\pgfqpoint{2.086674in}{1.929194in}}%
\pgfpathlineto{\pgfqpoint{2.088598in}{1.927726in}}%
\pgfpathlineto{\pgfqpoint{2.090521in}{1.918193in}}%
\pgfpathlineto{\pgfqpoint{2.094368in}{1.915340in}}%
\pgfpathlineto{\pgfqpoint{2.096292in}{1.916334in}}%
\pgfpathlineto{\pgfqpoint{2.098215in}{1.911440in}}%
\pgfpathlineto{\pgfqpoint{2.100138in}{1.921389in}}%
\pgfpathlineto{\pgfqpoint{2.102062in}{1.897592in}}%
\pgfpathlineto{\pgfqpoint{2.103985in}{1.896674in}}%
\pgfpathlineto{\pgfqpoint{2.105909in}{1.888595in}}%
\pgfpathlineto{\pgfqpoint{2.107832in}{1.899945in}}%
\pgfpathlineto{\pgfqpoint{2.109755in}{1.891598in}}%
\pgfpathlineto{\pgfqpoint{2.111679in}{1.908980in}}%
\pgfpathlineto{\pgfqpoint{2.113602in}{1.912180in}}%
\pgfpathlineto{\pgfqpoint{2.117449in}{1.889626in}}%
\pgfpathlineto{\pgfqpoint{2.119372in}{1.904301in}}%
\pgfpathlineto{\pgfqpoint{2.121296in}{1.893793in}}%
\pgfpathlineto{\pgfqpoint{2.123219in}{1.912790in}}%
\pgfpathlineto{\pgfqpoint{2.128990in}{1.894857in}}%
\pgfpathlineto{\pgfqpoint{2.130913in}{1.894278in}}%
\pgfpathlineto{\pgfqpoint{2.132836in}{1.888938in}}%
\pgfpathlineto{\pgfqpoint{2.134760in}{1.878385in}}%
\pgfpathlineto{\pgfqpoint{2.136683in}{1.887684in}}%
\pgfpathlineto{\pgfqpoint{2.138607in}{1.891313in}}%
\pgfpathlineto{\pgfqpoint{2.140530in}{1.906531in}}%
\pgfpathlineto{\pgfqpoint{2.142453in}{1.902756in}}%
\pgfpathlineto{\pgfqpoint{2.146300in}{1.900818in}}%
\pgfpathlineto{\pgfqpoint{2.148224in}{1.899359in}}%
\pgfpathlineto{\pgfqpoint{2.150147in}{1.904162in}}%
\pgfpathlineto{\pgfqpoint{2.152070in}{1.901062in}}%
\pgfpathlineto{\pgfqpoint{2.153994in}{1.894118in}}%
\pgfpathlineto{\pgfqpoint{2.157841in}{1.916488in}}%
\pgfpathlineto{\pgfqpoint{2.159764in}{1.906159in}}%
\pgfpathlineto{\pgfqpoint{2.161688in}{1.911532in}}%
\pgfpathlineto{\pgfqpoint{2.163611in}{1.911837in}}%
\pgfpathlineto{\pgfqpoint{2.165534in}{1.910184in}}%
\pgfpathlineto{\pgfqpoint{2.167458in}{1.897294in}}%
\pgfpathlineto{\pgfqpoint{2.169381in}{1.892243in}}%
\pgfpathlineto{\pgfqpoint{2.171305in}{1.895137in}}%
\pgfpathlineto{\pgfqpoint{2.173228in}{1.903290in}}%
\pgfpathlineto{\pgfqpoint{2.175151in}{1.900897in}}%
\pgfpathlineto{\pgfqpoint{2.177075in}{1.907896in}}%
\pgfpathlineto{\pgfqpoint{2.180922in}{1.932452in}}%
\pgfpathlineto{\pgfqpoint{2.182845in}{1.921182in}}%
\pgfpathlineto{\pgfqpoint{2.184768in}{1.929638in}}%
\pgfpathlineto{\pgfqpoint{2.186692in}{1.929987in}}%
\pgfpathlineto{\pgfqpoint{2.188615in}{1.934120in}}%
\pgfpathlineto{\pgfqpoint{2.190539in}{1.941134in}}%
\pgfpathlineto{\pgfqpoint{2.192462in}{1.942881in}}%
\pgfpathlineto{\pgfqpoint{2.194386in}{1.949427in}}%
\pgfpathlineto{\pgfqpoint{2.196309in}{1.943640in}}%
\pgfpathlineto{\pgfqpoint{2.198232in}{1.957833in}}%
\pgfpathlineto{\pgfqpoint{2.200156in}{1.948080in}}%
\pgfpathlineto{\pgfqpoint{2.202079in}{1.947996in}}%
\pgfpathlineto{\pgfqpoint{2.205926in}{1.967563in}}%
\pgfpathlineto{\pgfqpoint{2.207849in}{1.963410in}}%
\pgfpathlineto{\pgfqpoint{2.209773in}{1.963775in}}%
\pgfpathlineto{\pgfqpoint{2.211696in}{1.971030in}}%
\pgfpathlineto{\pgfqpoint{2.213620in}{1.966318in}}%
\pgfpathlineto{\pgfqpoint{2.215543in}{1.955763in}}%
\pgfpathlineto{\pgfqpoint{2.219390in}{1.957247in}}%
\pgfpathlineto{\pgfqpoint{2.221313in}{1.943344in}}%
\pgfpathlineto{\pgfqpoint{2.230930in}{1.939553in}}%
\pgfpathlineto{\pgfqpoint{2.232854in}{1.934222in}}%
\pgfpathlineto{\pgfqpoint{2.234777in}{1.935121in}}%
\pgfpathlineto{\pgfqpoint{2.236701in}{1.940901in}}%
\pgfpathlineto{\pgfqpoint{2.242471in}{1.971180in}}%
\pgfpathlineto{\pgfqpoint{2.248241in}{1.936257in}}%
\pgfpathlineto{\pgfqpoint{2.254011in}{1.938034in}}%
\pgfpathlineto{\pgfqpoint{2.255935in}{1.941632in}}%
\pgfpathlineto{\pgfqpoint{2.257858in}{1.942740in}}%
\pgfpathlineto{\pgfqpoint{2.259781in}{1.933231in}}%
\pgfpathlineto{\pgfqpoint{2.261705in}{1.930398in}}%
\pgfpathlineto{\pgfqpoint{2.263628in}{1.939448in}}%
\pgfpathlineto{\pgfqpoint{2.265552in}{1.937173in}}%
\pgfpathlineto{\pgfqpoint{2.267475in}{1.954751in}}%
\pgfpathlineto{\pgfqpoint{2.269399in}{1.952040in}}%
\pgfpathlineto{\pgfqpoint{2.271322in}{1.968727in}}%
\pgfpathlineto{\pgfqpoint{2.273245in}{1.974118in}}%
\pgfpathlineto{\pgfqpoint{2.275169in}{1.973487in}}%
\pgfpathlineto{\pgfqpoint{2.277092in}{1.980622in}}%
\pgfpathlineto{\pgfqpoint{2.279016in}{1.980034in}}%
\pgfpathlineto{\pgfqpoint{2.282862in}{1.976085in}}%
\pgfpathlineto{\pgfqpoint{2.284786in}{1.970265in}}%
\pgfpathlineto{\pgfqpoint{2.290556in}{1.981527in}}%
\pgfpathlineto{\pgfqpoint{2.294403in}{1.967213in}}%
\pgfpathlineto{\pgfqpoint{2.298250in}{1.984067in}}%
\pgfpathlineto{\pgfqpoint{2.300173in}{1.972313in}}%
\pgfpathlineto{\pgfqpoint{2.302097in}{1.976534in}}%
\pgfpathlineto{\pgfqpoint{2.305943in}{1.960920in}}%
\pgfpathlineto{\pgfqpoint{2.307867in}{1.961468in}}%
\pgfpathlineto{\pgfqpoint{2.309790in}{1.956110in}}%
\pgfpathlineto{\pgfqpoint{2.313637in}{1.955801in}}%
\pgfpathlineto{\pgfqpoint{2.317484in}{1.966737in}}%
\pgfpathlineto{\pgfqpoint{2.321331in}{1.950589in}}%
\pgfpathlineto{\pgfqpoint{2.327101in}{1.966146in}}%
\pgfpathlineto{\pgfqpoint{2.329024in}{1.966480in}}%
\pgfpathlineto{\pgfqpoint{2.332871in}{1.989270in}}%
\pgfpathlineto{\pgfqpoint{2.334795in}{1.980969in}}%
\pgfpathlineto{\pgfqpoint{2.336718in}{1.991964in}}%
\pgfpathlineto{\pgfqpoint{2.338641in}{1.990465in}}%
\pgfpathlineto{\pgfqpoint{2.340565in}{1.995960in}}%
\pgfpathlineto{\pgfqpoint{2.342488in}{2.007045in}}%
\pgfpathlineto{\pgfqpoint{2.344412in}{2.000736in}}%
\pgfpathlineto{\pgfqpoint{2.346335in}{2.008586in}}%
\pgfpathlineto{\pgfqpoint{2.348258in}{2.008663in}}%
\pgfpathlineto{\pgfqpoint{2.350182in}{2.004080in}}%
\pgfpathlineto{\pgfqpoint{2.352105in}{2.010245in}}%
\pgfpathlineto{\pgfqpoint{2.354029in}{2.007476in}}%
\pgfpathlineto{\pgfqpoint{2.355952in}{2.009492in}}%
\pgfpathlineto{\pgfqpoint{2.357875in}{2.016341in}}%
\pgfpathlineto{\pgfqpoint{2.359799in}{2.016951in}}%
\pgfpathlineto{\pgfqpoint{2.361722in}{2.014694in}}%
\pgfpathlineto{\pgfqpoint{2.363646in}{2.032432in}}%
\pgfpathlineto{\pgfqpoint{2.367492in}{2.035410in}}%
\pgfpathlineto{\pgfqpoint{2.371339in}{2.031704in}}%
\pgfpathlineto{\pgfqpoint{2.373263in}{2.032009in}}%
\pgfpathlineto{\pgfqpoint{2.375186in}{2.033898in}}%
\pgfpathlineto{\pgfqpoint{2.377110in}{2.031015in}}%
\pgfpathlineto{\pgfqpoint{2.379033in}{2.034309in}}%
\pgfpathlineto{\pgfqpoint{2.380956in}{2.027833in}}%
\pgfpathlineto{\pgfqpoint{2.382880in}{2.037097in}}%
\pgfpathlineto{\pgfqpoint{2.384803in}{2.033039in}}%
\pgfpathlineto{\pgfqpoint{2.386727in}{2.024648in}}%
\pgfpathlineto{\pgfqpoint{2.388650in}{2.029516in}}%
\pgfpathlineto{\pgfqpoint{2.392497in}{2.029507in}}%
\pgfpathlineto{\pgfqpoint{2.396344in}{2.013636in}}%
\pgfpathlineto{\pgfqpoint{2.398267in}{2.013434in}}%
\pgfpathlineto{\pgfqpoint{2.400190in}{2.018653in}}%
\pgfpathlineto{\pgfqpoint{2.402114in}{2.014429in}}%
\pgfpathlineto{\pgfqpoint{2.404037in}{2.001459in}}%
\pgfpathlineto{\pgfqpoint{2.405961in}{1.999118in}}%
\pgfpathlineto{\pgfqpoint{2.409808in}{2.016803in}}%
\pgfpathlineto{\pgfqpoint{2.411731in}{2.010062in}}%
\pgfpathlineto{\pgfqpoint{2.413654in}{2.010953in}}%
\pgfpathlineto{\pgfqpoint{2.417501in}{2.018148in}}%
\pgfpathlineto{\pgfqpoint{2.419425in}{2.006550in}}%
\pgfpathlineto{\pgfqpoint{2.421348in}{2.005315in}}%
\pgfpathlineto{\pgfqpoint{2.423271in}{2.007436in}}%
\pgfpathlineto{\pgfqpoint{2.425195in}{2.005692in}}%
\pgfpathlineto{\pgfqpoint{2.427118in}{1.990633in}}%
\pgfpathlineto{\pgfqpoint{2.429042in}{1.999661in}}%
\pgfpathlineto{\pgfqpoint{2.432888in}{2.002345in}}%
\pgfpathlineto{\pgfqpoint{2.434812in}{1.996013in}}%
\pgfpathlineto{\pgfqpoint{2.436735in}{1.993537in}}%
\pgfpathlineto{\pgfqpoint{2.438659in}{1.997098in}}%
\pgfpathlineto{\pgfqpoint{2.440582in}{1.985149in}}%
\pgfpathlineto{\pgfqpoint{2.442506in}{1.986546in}}%
\pgfpathlineto{\pgfqpoint{2.444429in}{1.984156in}}%
\pgfpathlineto{\pgfqpoint{2.448276in}{1.967666in}}%
\pgfpathlineto{\pgfqpoint{2.450199in}{1.969304in}}%
\pgfpathlineto{\pgfqpoint{2.455969in}{1.939759in}}%
\pgfpathlineto{\pgfqpoint{2.457893in}{1.933855in}}%
\pgfpathlineto{\pgfqpoint{2.459816in}{1.939441in}}%
\pgfpathlineto{\pgfqpoint{2.461740in}{1.937156in}}%
\pgfpathlineto{\pgfqpoint{2.463663in}{1.938430in}}%
\pgfpathlineto{\pgfqpoint{2.465586in}{1.935017in}}%
\pgfpathlineto{\pgfqpoint{2.467510in}{1.953274in}}%
\pgfpathlineto{\pgfqpoint{2.469433in}{1.946763in}}%
\pgfpathlineto{\pgfqpoint{2.471357in}{1.945786in}}%
\pgfpathlineto{\pgfqpoint{2.473280in}{1.947144in}}%
\pgfpathlineto{\pgfqpoint{2.475204in}{1.959773in}}%
\pgfpathlineto{\pgfqpoint{2.477127in}{1.960065in}}%
\pgfpathlineto{\pgfqpoint{2.479050in}{1.948989in}}%
\pgfpathlineto{\pgfqpoint{2.480974in}{1.950160in}}%
\pgfpathlineto{\pgfqpoint{2.482897in}{1.947219in}}%
\pgfpathlineto{\pgfqpoint{2.484821in}{1.953708in}}%
\pgfpathlineto{\pgfqpoint{2.486744in}{1.946518in}}%
\pgfpathlineto{\pgfqpoint{2.488667in}{1.947585in}}%
\pgfpathlineto{\pgfqpoint{2.492514in}{1.937964in}}%
\pgfpathlineto{\pgfqpoint{2.494438in}{1.926997in}}%
\pgfpathlineto{\pgfqpoint{2.496361in}{1.934079in}}%
\pgfpathlineto{\pgfqpoint{2.500208in}{1.970412in}}%
\pgfpathlineto{\pgfqpoint{2.502131in}{1.964022in}}%
\pgfpathlineto{\pgfqpoint{2.505978in}{1.988551in}}%
\pgfpathlineto{\pgfqpoint{2.507901in}{1.988807in}}%
\pgfpathlineto{\pgfqpoint{2.509825in}{1.979278in}}%
\pgfpathlineto{\pgfqpoint{2.511748in}{1.975542in}}%
\pgfpathlineto{\pgfqpoint{2.513672in}{1.987237in}}%
\pgfpathlineto{\pgfqpoint{2.515595in}{1.984452in}}%
\pgfpathlineto{\pgfqpoint{2.517519in}{1.988205in}}%
\pgfpathlineto{\pgfqpoint{2.519442in}{1.984812in}}%
\pgfpathlineto{\pgfqpoint{2.521365in}{1.978936in}}%
\pgfpathlineto{\pgfqpoint{2.523289in}{1.987359in}}%
\pgfpathlineto{\pgfqpoint{2.525212in}{1.989415in}}%
\pgfpathlineto{\pgfqpoint{2.527136in}{1.995937in}}%
\pgfpathlineto{\pgfqpoint{2.529059in}{1.997572in}}%
\pgfpathlineto{\pgfqpoint{2.534829in}{2.021546in}}%
\pgfpathlineto{\pgfqpoint{2.536753in}{2.019487in}}%
\pgfpathlineto{\pgfqpoint{2.544446in}{2.042056in}}%
\pgfpathlineto{\pgfqpoint{2.548293in}{2.022675in}}%
\pgfpathlineto{\pgfqpoint{2.550217in}{2.026186in}}%
\pgfpathlineto{\pgfqpoint{2.552140in}{2.021421in}}%
\pgfpathlineto{\pgfqpoint{2.554063in}{2.004988in}}%
\pgfpathlineto{\pgfqpoint{2.555987in}{2.004381in}}%
\pgfpathlineto{\pgfqpoint{2.557910in}{1.996842in}}%
\pgfpathlineto{\pgfqpoint{2.563680in}{2.020994in}}%
\pgfpathlineto{\pgfqpoint{2.565604in}{2.022343in}}%
\pgfpathlineto{\pgfqpoint{2.567527in}{2.025683in}}%
\pgfpathlineto{\pgfqpoint{2.571374in}{2.010613in}}%
\pgfpathlineto{\pgfqpoint{2.573297in}{2.025143in}}%
\pgfpathlineto{\pgfqpoint{2.575221in}{2.027409in}}%
\pgfpathlineto{\pgfqpoint{2.577144in}{2.027634in}}%
\pgfpathlineto{\pgfqpoint{2.579068in}{2.034022in}}%
\pgfpathlineto{\pgfqpoint{2.580991in}{2.026567in}}%
\pgfpathlineto{\pgfqpoint{2.582915in}{2.025463in}}%
\pgfpathlineto{\pgfqpoint{2.584838in}{2.031555in}}%
\pgfpathlineto{\pgfqpoint{2.590608in}{2.058912in}}%
\pgfpathlineto{\pgfqpoint{2.592532in}{2.060141in}}%
\pgfpathlineto{\pgfqpoint{2.594455in}{2.065227in}}%
\pgfpathlineto{\pgfqpoint{2.596378in}{2.064208in}}%
\pgfpathlineto{\pgfqpoint{2.598302in}{2.074432in}}%
\pgfpathlineto{\pgfqpoint{2.602149in}{2.058494in}}%
\pgfpathlineto{\pgfqpoint{2.604072in}{2.060670in}}%
\pgfpathlineto{\pgfqpoint{2.605995in}{2.051066in}}%
\pgfpathlineto{\pgfqpoint{2.607919in}{2.051350in}}%
\pgfpathlineto{\pgfqpoint{2.609842in}{2.060384in}}%
\pgfpathlineto{\pgfqpoint{2.611766in}{2.053863in}}%
\pgfpathlineto{\pgfqpoint{2.613689in}{2.057317in}}%
\pgfpathlineto{\pgfqpoint{2.615613in}{2.049731in}}%
\pgfpathlineto{\pgfqpoint{2.619459in}{2.050452in}}%
\pgfpathlineto{\pgfqpoint{2.621383in}{2.052268in}}%
\pgfpathlineto{\pgfqpoint{2.623306in}{2.059291in}}%
\pgfpathlineto{\pgfqpoint{2.625230in}{2.059868in}}%
\pgfpathlineto{\pgfqpoint{2.631000in}{2.080288in}}%
\pgfpathlineto{\pgfqpoint{2.632923in}{2.067574in}}%
\pgfpathlineto{\pgfqpoint{2.638693in}{2.085426in}}%
\pgfpathlineto{\pgfqpoint{2.640617in}{2.090170in}}%
\pgfpathlineto{\pgfqpoint{2.642540in}{2.085011in}}%
\pgfpathlineto{\pgfqpoint{2.644464in}{2.083976in}}%
\pgfpathlineto{\pgfqpoint{2.648311in}{2.066673in}}%
\pgfpathlineto{\pgfqpoint{2.650234in}{2.076414in}}%
\pgfpathlineto{\pgfqpoint{2.652157in}{2.075725in}}%
\pgfpathlineto{\pgfqpoint{2.654081in}{2.077553in}}%
\pgfpathlineto{\pgfqpoint{2.656004in}{2.058383in}}%
\pgfpathlineto{\pgfqpoint{2.657928in}{2.066454in}}%
\pgfpathlineto{\pgfqpoint{2.659851in}{2.060020in}}%
\pgfpathlineto{\pgfqpoint{2.661774in}{2.077283in}}%
\pgfpathlineto{\pgfqpoint{2.665621in}{2.090104in}}%
\pgfpathlineto{\pgfqpoint{2.667545in}{2.107695in}}%
\pgfpathlineto{\pgfqpoint{2.669468in}{2.103833in}}%
\pgfpathlineto{\pgfqpoint{2.671391in}{2.105762in}}%
\pgfpathlineto{\pgfqpoint{2.673315in}{2.113198in}}%
\pgfpathlineto{\pgfqpoint{2.675238in}{2.112267in}}%
\pgfpathlineto{\pgfqpoint{2.677162in}{2.115490in}}%
\pgfpathlineto{\pgfqpoint{2.679085in}{2.111287in}}%
\pgfpathlineto{\pgfqpoint{2.681008in}{2.110745in}}%
\pgfpathlineto{\pgfqpoint{2.682932in}{2.113887in}}%
\pgfpathlineto{\pgfqpoint{2.684855in}{2.120873in}}%
\pgfpathlineto{\pgfqpoint{2.690626in}{2.126761in}}%
\pgfpathlineto{\pgfqpoint{2.696396in}{2.161019in}}%
\pgfpathlineto{\pgfqpoint{2.698319in}{2.162472in}}%
\pgfpathlineto{\pgfqpoint{2.700243in}{2.155512in}}%
\pgfpathlineto{\pgfqpoint{2.702166in}{2.154385in}}%
\pgfpathlineto{\pgfqpoint{2.704089in}{2.157435in}}%
\pgfpathlineto{\pgfqpoint{2.706013in}{2.138202in}}%
\pgfpathlineto{\pgfqpoint{2.707936in}{2.139737in}}%
\pgfpathlineto{\pgfqpoint{2.709860in}{2.148875in}}%
\pgfpathlineto{\pgfqpoint{2.711783in}{2.141422in}}%
\pgfpathlineto{\pgfqpoint{2.713706in}{2.141583in}}%
\pgfpathlineto{\pgfqpoint{2.717553in}{2.128545in}}%
\pgfpathlineto{\pgfqpoint{2.719477in}{2.137616in}}%
\pgfpathlineto{\pgfqpoint{2.721400in}{2.140555in}}%
\pgfpathlineto{\pgfqpoint{2.723324in}{2.130029in}}%
\pgfpathlineto{\pgfqpoint{2.725247in}{2.129439in}}%
\pgfpathlineto{\pgfqpoint{2.727170in}{2.127594in}}%
\pgfpathlineto{\pgfqpoint{2.729094in}{2.119379in}}%
\pgfpathlineto{\pgfqpoint{2.731017in}{2.116393in}}%
\pgfpathlineto{\pgfqpoint{2.732941in}{2.128160in}}%
\pgfpathlineto{\pgfqpoint{2.736787in}{2.121597in}}%
\pgfpathlineto{\pgfqpoint{2.742558in}{2.125923in}}%
\pgfpathlineto{\pgfqpoint{2.744481in}{2.121155in}}%
\pgfpathlineto{\pgfqpoint{2.748328in}{2.098280in}}%
\pgfpathlineto{\pgfqpoint{2.750251in}{2.093503in}}%
\pgfpathlineto{\pgfqpoint{2.752175in}{2.102009in}}%
\pgfpathlineto{\pgfqpoint{2.754098in}{2.105234in}}%
\pgfpathlineto{\pgfqpoint{2.756022in}{2.103264in}}%
\pgfpathlineto{\pgfqpoint{2.757945in}{2.109603in}}%
\pgfpathlineto{\pgfqpoint{2.759868in}{2.104736in}}%
\pgfpathlineto{\pgfqpoint{2.761792in}{2.107747in}}%
\pgfpathlineto{\pgfqpoint{2.763715in}{2.115594in}}%
\pgfpathlineto{\pgfqpoint{2.765639in}{2.113917in}}%
\pgfpathlineto{\pgfqpoint{2.767562in}{2.104614in}}%
\pgfpathlineto{\pgfqpoint{2.767562in}{2.104614in}}%
\pgfusepath{stroke}%
\end{pgfscope}%
\begin{pgfscope}%
\pgfpathrectangle{\pgfqpoint{0.750000in}{0.660000in}}{\pgfqpoint{2.113636in}{2.100000in}}%
\pgfusepath{clip}%
\pgfsetroundcap%
\pgfsetroundjoin%
\pgfsetlinewidth{0.602250pt}%
\definecolor{currentstroke}{rgb}{0.596078,0.305882,0.639216}%
\pgfsetstrokecolor{currentstroke}%
\pgfsetdash{}{0pt}%
\pgfpathmoveto{\pgfqpoint{0.846074in}{1.870346in}}%
\pgfpathlineto{\pgfqpoint{0.847998in}{1.856991in}}%
\pgfpathlineto{\pgfqpoint{0.849921in}{1.858208in}}%
\pgfpathlineto{\pgfqpoint{0.851845in}{1.852384in}}%
\pgfpathlineto{\pgfqpoint{0.853768in}{1.851145in}}%
\pgfpathlineto{\pgfqpoint{0.855691in}{1.844511in}}%
\pgfpathlineto{\pgfqpoint{0.857615in}{1.853098in}}%
\pgfpathlineto{\pgfqpoint{0.859538in}{1.841074in}}%
\pgfpathlineto{\pgfqpoint{0.861462in}{1.853443in}}%
\pgfpathlineto{\pgfqpoint{0.863385in}{1.843328in}}%
\pgfpathlineto{\pgfqpoint{0.865308in}{1.841069in}}%
\pgfpathlineto{\pgfqpoint{0.867232in}{1.844956in}}%
\pgfpathlineto{\pgfqpoint{0.869155in}{1.845116in}}%
\pgfpathlineto{\pgfqpoint{0.871079in}{1.838314in}}%
\pgfpathlineto{\pgfqpoint{0.873002in}{1.840331in}}%
\pgfpathlineto{\pgfqpoint{0.876849in}{1.861661in}}%
\pgfpathlineto{\pgfqpoint{0.878772in}{1.873612in}}%
\pgfpathlineto{\pgfqpoint{0.880696in}{1.861695in}}%
\pgfpathlineto{\pgfqpoint{0.882619in}{1.878559in}}%
\pgfpathlineto{\pgfqpoint{0.884543in}{1.874133in}}%
\pgfpathlineto{\pgfqpoint{0.888389in}{1.876450in}}%
\pgfpathlineto{\pgfqpoint{0.890313in}{1.872970in}}%
\pgfpathlineto{\pgfqpoint{0.896083in}{1.898797in}}%
\pgfpathlineto{\pgfqpoint{0.898006in}{1.901615in}}%
\pgfpathlineto{\pgfqpoint{0.901853in}{1.894396in}}%
\pgfpathlineto{\pgfqpoint{0.903777in}{1.880117in}}%
\pgfpathlineto{\pgfqpoint{0.905700in}{1.882128in}}%
\pgfpathlineto{\pgfqpoint{0.909547in}{1.874218in}}%
\pgfpathlineto{\pgfqpoint{0.911470in}{1.879500in}}%
\pgfpathlineto{\pgfqpoint{0.913394in}{1.879825in}}%
\pgfpathlineto{\pgfqpoint{0.915317in}{1.877785in}}%
\pgfpathlineto{\pgfqpoint{0.917241in}{1.891134in}}%
\pgfpathlineto{\pgfqpoint{0.919164in}{1.880078in}}%
\pgfpathlineto{\pgfqpoint{0.921087in}{1.890852in}}%
\pgfpathlineto{\pgfqpoint{0.923011in}{1.894278in}}%
\pgfpathlineto{\pgfqpoint{0.924934in}{1.906060in}}%
\pgfpathlineto{\pgfqpoint{0.926858in}{1.904195in}}%
\pgfpathlineto{\pgfqpoint{0.928781in}{1.889101in}}%
\pgfpathlineto{\pgfqpoint{0.930704in}{1.892704in}}%
\pgfpathlineto{\pgfqpoint{0.932628in}{1.883902in}}%
\pgfpathlineto{\pgfqpoint{0.934551in}{1.889027in}}%
\pgfpathlineto{\pgfqpoint{0.936475in}{1.884924in}}%
\pgfpathlineto{\pgfqpoint{0.940322in}{1.892671in}}%
\pgfpathlineto{\pgfqpoint{0.942245in}{1.886422in}}%
\pgfpathlineto{\pgfqpoint{0.946092in}{1.902171in}}%
\pgfpathlineto{\pgfqpoint{0.948015in}{1.879898in}}%
\pgfpathlineto{\pgfqpoint{0.951862in}{1.870150in}}%
\pgfpathlineto{\pgfqpoint{0.953785in}{1.877672in}}%
\pgfpathlineto{\pgfqpoint{0.955709in}{1.880481in}}%
\pgfpathlineto{\pgfqpoint{0.957632in}{1.896893in}}%
\pgfpathlineto{\pgfqpoint{0.959556in}{1.895970in}}%
\pgfpathlineto{\pgfqpoint{0.961479in}{1.907009in}}%
\pgfpathlineto{\pgfqpoint{0.963402in}{1.908375in}}%
\pgfpathlineto{\pgfqpoint{0.967249in}{1.898410in}}%
\pgfpathlineto{\pgfqpoint{0.969173in}{1.893576in}}%
\pgfpathlineto{\pgfqpoint{0.974943in}{1.917748in}}%
\pgfpathlineto{\pgfqpoint{0.978790in}{1.914774in}}%
\pgfpathlineto{\pgfqpoint{0.982637in}{1.901029in}}%
\pgfpathlineto{\pgfqpoint{0.984560in}{1.897873in}}%
\pgfpathlineto{\pgfqpoint{0.986483in}{1.883428in}}%
\pgfpathlineto{\pgfqpoint{0.990330in}{1.876636in}}%
\pgfpathlineto{\pgfqpoint{0.992254in}{1.869786in}}%
\pgfpathlineto{\pgfqpoint{0.994177in}{1.894684in}}%
\pgfpathlineto{\pgfqpoint{0.996100in}{1.897470in}}%
\pgfpathlineto{\pgfqpoint{0.998024in}{1.902937in}}%
\pgfpathlineto{\pgfqpoint{0.999947in}{1.902408in}}%
\pgfpathlineto{\pgfqpoint{1.001871in}{1.890991in}}%
\pgfpathlineto{\pgfqpoint{1.003794in}{1.891388in}}%
\pgfpathlineto{\pgfqpoint{1.011488in}{1.901233in}}%
\pgfpathlineto{\pgfqpoint{1.013411in}{1.899356in}}%
\pgfpathlineto{\pgfqpoint{1.019181in}{1.903355in}}%
\pgfpathlineto{\pgfqpoint{1.021105in}{1.898990in}}%
\pgfpathlineto{\pgfqpoint{1.023028in}{1.891353in}}%
\pgfpathlineto{\pgfqpoint{1.024952in}{1.890596in}}%
\pgfpathlineto{\pgfqpoint{1.030722in}{1.873526in}}%
\pgfpathlineto{\pgfqpoint{1.032645in}{1.871609in}}%
\pgfpathlineto{\pgfqpoint{1.034569in}{1.891259in}}%
\pgfpathlineto{\pgfqpoint{1.038415in}{1.904747in}}%
\pgfpathlineto{\pgfqpoint{1.040339in}{1.897497in}}%
\pgfpathlineto{\pgfqpoint{1.042262in}{1.904229in}}%
\pgfpathlineto{\pgfqpoint{1.044186in}{1.902650in}}%
\pgfpathlineto{\pgfqpoint{1.046109in}{1.908263in}}%
\pgfpathlineto{\pgfqpoint{1.049956in}{1.899678in}}%
\pgfpathlineto{\pgfqpoint{1.051879in}{1.903512in}}%
\pgfpathlineto{\pgfqpoint{1.055726in}{1.891032in}}%
\pgfpathlineto{\pgfqpoint{1.057650in}{1.898393in}}%
\pgfpathlineto{\pgfqpoint{1.059573in}{1.900020in}}%
\pgfpathlineto{\pgfqpoint{1.061496in}{1.905918in}}%
\pgfpathlineto{\pgfqpoint{1.063420in}{1.902309in}}%
\pgfpathlineto{\pgfqpoint{1.069190in}{1.921346in}}%
\pgfpathlineto{\pgfqpoint{1.071113in}{1.922419in}}%
\pgfpathlineto{\pgfqpoint{1.073037in}{1.925818in}}%
\pgfpathlineto{\pgfqpoint{1.074960in}{1.925980in}}%
\pgfpathlineto{\pgfqpoint{1.076884in}{1.928273in}}%
\pgfpathlineto{\pgfqpoint{1.078807in}{1.928501in}}%
\pgfpathlineto{\pgfqpoint{1.080731in}{1.938877in}}%
\pgfpathlineto{\pgfqpoint{1.082654in}{1.933628in}}%
\pgfpathlineto{\pgfqpoint{1.088424in}{1.935225in}}%
\pgfpathlineto{\pgfqpoint{1.090348in}{1.941345in}}%
\pgfpathlineto{\pgfqpoint{1.092271in}{1.941731in}}%
\pgfpathlineto{\pgfqpoint{1.094194in}{1.946844in}}%
\pgfpathlineto{\pgfqpoint{1.098041in}{1.962858in}}%
\pgfpathlineto{\pgfqpoint{1.099965in}{1.965751in}}%
\pgfpathlineto{\pgfqpoint{1.101888in}{1.984049in}}%
\pgfpathlineto{\pgfqpoint{1.103811in}{1.977538in}}%
\pgfpathlineto{\pgfqpoint{1.105735in}{1.984373in}}%
\pgfpathlineto{\pgfqpoint{1.109582in}{1.981664in}}%
\pgfpathlineto{\pgfqpoint{1.113429in}{2.000758in}}%
\pgfpathlineto{\pgfqpoint{1.115352in}{1.999166in}}%
\pgfpathlineto{\pgfqpoint{1.117275in}{1.994155in}}%
\pgfpathlineto{\pgfqpoint{1.121122in}{1.960645in}}%
\pgfpathlineto{\pgfqpoint{1.123046in}{1.953176in}}%
\pgfpathlineto{\pgfqpoint{1.126892in}{1.961386in}}%
\pgfpathlineto{\pgfqpoint{1.130739in}{1.961123in}}%
\pgfpathlineto{\pgfqpoint{1.132663in}{1.951222in}}%
\pgfpathlineto{\pgfqpoint{1.134586in}{1.956792in}}%
\pgfpathlineto{\pgfqpoint{1.136509in}{1.955788in}}%
\pgfpathlineto{\pgfqpoint{1.138433in}{1.960286in}}%
\pgfpathlineto{\pgfqpoint{1.140356in}{1.969070in}}%
\pgfpathlineto{\pgfqpoint{1.142280in}{1.971707in}}%
\pgfpathlineto{\pgfqpoint{1.144203in}{1.962682in}}%
\pgfpathlineto{\pgfqpoint{1.148050in}{1.978932in}}%
\pgfpathlineto{\pgfqpoint{1.149973in}{1.979301in}}%
\pgfpathlineto{\pgfqpoint{1.151897in}{1.977533in}}%
\pgfpathlineto{\pgfqpoint{1.155744in}{1.966780in}}%
\pgfpathlineto{\pgfqpoint{1.159590in}{1.985877in}}%
\pgfpathlineto{\pgfqpoint{1.161514in}{1.986323in}}%
\pgfpathlineto{\pgfqpoint{1.163437in}{1.992124in}}%
\pgfpathlineto{\pgfqpoint{1.165361in}{2.012611in}}%
\pgfpathlineto{\pgfqpoint{1.167284in}{2.012829in}}%
\pgfpathlineto{\pgfqpoint{1.169207in}{2.017159in}}%
\pgfpathlineto{\pgfqpoint{1.171131in}{2.028173in}}%
\pgfpathlineto{\pgfqpoint{1.174978in}{2.006094in}}%
\pgfpathlineto{\pgfqpoint{1.176901in}{2.008047in}}%
\pgfpathlineto{\pgfqpoint{1.180748in}{2.028348in}}%
\pgfpathlineto{\pgfqpoint{1.182671in}{2.024509in}}%
\pgfpathlineto{\pgfqpoint{1.186518in}{2.036032in}}%
\pgfpathlineto{\pgfqpoint{1.192288in}{2.024071in}}%
\pgfpathlineto{\pgfqpoint{1.194212in}{2.028114in}}%
\pgfpathlineto{\pgfqpoint{1.198059in}{2.047168in}}%
\pgfpathlineto{\pgfqpoint{1.199982in}{2.044976in}}%
\pgfpathlineto{\pgfqpoint{1.201905in}{2.037595in}}%
\pgfpathlineto{\pgfqpoint{1.203829in}{2.036979in}}%
\pgfpathlineto{\pgfqpoint{1.205752in}{2.041334in}}%
\pgfpathlineto{\pgfqpoint{1.207676in}{2.038299in}}%
\pgfpathlineto{\pgfqpoint{1.211522in}{2.052744in}}%
\pgfpathlineto{\pgfqpoint{1.213446in}{2.054153in}}%
\pgfpathlineto{\pgfqpoint{1.215369in}{2.058407in}}%
\pgfpathlineto{\pgfqpoint{1.219216in}{2.060093in}}%
\pgfpathlineto{\pgfqpoint{1.223063in}{2.079113in}}%
\pgfpathlineto{\pgfqpoint{1.224986in}{2.078877in}}%
\pgfpathlineto{\pgfqpoint{1.226910in}{2.088339in}}%
\pgfpathlineto{\pgfqpoint{1.228833in}{2.078875in}}%
\pgfpathlineto{\pgfqpoint{1.230757in}{2.086595in}}%
\pgfpathlineto{\pgfqpoint{1.232680in}{2.072643in}}%
\pgfpathlineto{\pgfqpoint{1.234603in}{2.066836in}}%
\pgfpathlineto{\pgfqpoint{1.236527in}{2.069267in}}%
\pgfpathlineto{\pgfqpoint{1.240374in}{2.093142in}}%
\pgfpathlineto{\pgfqpoint{1.242297in}{2.092445in}}%
\pgfpathlineto{\pgfqpoint{1.244220in}{2.090436in}}%
\pgfpathlineto{\pgfqpoint{1.246144in}{2.094091in}}%
\pgfpathlineto{\pgfqpoint{1.249991in}{2.087454in}}%
\pgfpathlineto{\pgfqpoint{1.251914in}{2.075194in}}%
\pgfpathlineto{\pgfqpoint{1.253838in}{2.080371in}}%
\pgfpathlineto{\pgfqpoint{1.255761in}{2.075418in}}%
\pgfpathlineto{\pgfqpoint{1.257684in}{2.079946in}}%
\pgfpathlineto{\pgfqpoint{1.261531in}{2.067002in}}%
\pgfpathlineto{\pgfqpoint{1.263455in}{2.075588in}}%
\pgfpathlineto{\pgfqpoint{1.265378in}{2.070382in}}%
\pgfpathlineto{\pgfqpoint{1.267301in}{2.083016in}}%
\pgfpathlineto{\pgfqpoint{1.269225in}{2.079640in}}%
\pgfpathlineto{\pgfqpoint{1.271148in}{2.081379in}}%
\pgfpathlineto{\pgfqpoint{1.273072in}{2.085241in}}%
\pgfpathlineto{\pgfqpoint{1.274995in}{2.092961in}}%
\pgfpathlineto{\pgfqpoint{1.276918in}{2.095636in}}%
\pgfpathlineto{\pgfqpoint{1.278842in}{2.094302in}}%
\pgfpathlineto{\pgfqpoint{1.280765in}{2.088259in}}%
\pgfpathlineto{\pgfqpoint{1.282689in}{2.095456in}}%
\pgfpathlineto{\pgfqpoint{1.284612in}{2.097613in}}%
\pgfpathlineto{\pgfqpoint{1.286536in}{2.101889in}}%
\pgfpathlineto{\pgfqpoint{1.288459in}{2.096472in}}%
\pgfpathlineto{\pgfqpoint{1.290382in}{2.095811in}}%
\pgfpathlineto{\pgfqpoint{1.292306in}{2.100874in}}%
\pgfpathlineto{\pgfqpoint{1.294229in}{2.099348in}}%
\pgfpathlineto{\pgfqpoint{1.296153in}{2.113477in}}%
\pgfpathlineto{\pgfqpoint{1.299999in}{2.110395in}}%
\pgfpathlineto{\pgfqpoint{1.301923in}{2.128718in}}%
\pgfpathlineto{\pgfqpoint{1.305770in}{2.126629in}}%
\pgfpathlineto{\pgfqpoint{1.309616in}{2.121029in}}%
\pgfpathlineto{\pgfqpoint{1.311540in}{2.124070in}}%
\pgfpathlineto{\pgfqpoint{1.313463in}{2.123841in}}%
\pgfpathlineto{\pgfqpoint{1.315387in}{2.118155in}}%
\pgfpathlineto{\pgfqpoint{1.317310in}{2.120897in}}%
\pgfpathlineto{\pgfqpoint{1.319233in}{2.113573in}}%
\pgfpathlineto{\pgfqpoint{1.321157in}{2.115853in}}%
\pgfpathlineto{\pgfqpoint{1.323080in}{2.131517in}}%
\pgfpathlineto{\pgfqpoint{1.326927in}{2.126667in}}%
\pgfpathlineto{\pgfqpoint{1.330774in}{2.150534in}}%
\pgfpathlineto{\pgfqpoint{1.332697in}{2.149613in}}%
\pgfpathlineto{\pgfqpoint{1.334621in}{2.150449in}}%
\pgfpathlineto{\pgfqpoint{1.336544in}{2.154436in}}%
\pgfpathlineto{\pgfqpoint{1.338468in}{2.148159in}}%
\pgfpathlineto{\pgfqpoint{1.340391in}{2.135702in}}%
\pgfpathlineto{\pgfqpoint{1.342314in}{2.134938in}}%
\pgfpathlineto{\pgfqpoint{1.344238in}{2.145302in}}%
\pgfpathlineto{\pgfqpoint{1.346161in}{2.139140in}}%
\pgfpathlineto{\pgfqpoint{1.348085in}{2.143935in}}%
\pgfpathlineto{\pgfqpoint{1.350008in}{2.139435in}}%
\pgfpathlineto{\pgfqpoint{1.353855in}{2.160436in}}%
\pgfpathlineto{\pgfqpoint{1.355778in}{2.169465in}}%
\pgfpathlineto{\pgfqpoint{1.361549in}{2.172488in}}%
\pgfpathlineto{\pgfqpoint{1.369242in}{2.192884in}}%
\pgfpathlineto{\pgfqpoint{1.371166in}{2.184532in}}%
\pgfpathlineto{\pgfqpoint{1.376936in}{2.176411in}}%
\pgfpathlineto{\pgfqpoint{1.380783in}{2.146799in}}%
\pgfpathlineto{\pgfqpoint{1.382706in}{2.151993in}}%
\pgfpathlineto{\pgfqpoint{1.384629in}{2.153557in}}%
\pgfpathlineto{\pgfqpoint{1.386553in}{2.159381in}}%
\pgfpathlineto{\pgfqpoint{1.390400in}{2.139876in}}%
\pgfpathlineto{\pgfqpoint{1.394247in}{2.160782in}}%
\pgfpathlineto{\pgfqpoint{1.398093in}{2.154427in}}%
\pgfpathlineto{\pgfqpoint{1.400017in}{2.144255in}}%
\pgfpathlineto{\pgfqpoint{1.401940in}{2.165942in}}%
\pgfpathlineto{\pgfqpoint{1.403864in}{2.166129in}}%
\pgfpathlineto{\pgfqpoint{1.405787in}{2.163493in}}%
\pgfpathlineto{\pgfqpoint{1.407710in}{2.158313in}}%
\pgfpathlineto{\pgfqpoint{1.409634in}{2.164983in}}%
\pgfpathlineto{\pgfqpoint{1.411557in}{2.164489in}}%
\pgfpathlineto{\pgfqpoint{1.415404in}{2.170902in}}%
\pgfpathlineto{\pgfqpoint{1.417327in}{2.177134in}}%
\pgfpathlineto{\pgfqpoint{1.419251in}{2.175728in}}%
\pgfpathlineto{\pgfqpoint{1.423098in}{2.153057in}}%
\pgfpathlineto{\pgfqpoint{1.425021in}{2.152835in}}%
\pgfpathlineto{\pgfqpoint{1.428868in}{2.145869in}}%
\pgfpathlineto{\pgfqpoint{1.430791in}{2.134320in}}%
\pgfpathlineto{\pgfqpoint{1.434638in}{2.152003in}}%
\pgfpathlineto{\pgfqpoint{1.436562in}{2.149377in}}%
\pgfpathlineto{\pgfqpoint{1.438485in}{2.137708in}}%
\pgfpathlineto{\pgfqpoint{1.444255in}{2.153377in}}%
\pgfpathlineto{\pgfqpoint{1.446179in}{2.138989in}}%
\pgfpathlineto{\pgfqpoint{1.448102in}{2.141145in}}%
\pgfpathlineto{\pgfqpoint{1.450025in}{2.134206in}}%
\pgfpathlineto{\pgfqpoint{1.451949in}{2.139688in}}%
\pgfpathlineto{\pgfqpoint{1.453872in}{2.132119in}}%
\pgfpathlineto{\pgfqpoint{1.455796in}{2.135178in}}%
\pgfpathlineto{\pgfqpoint{1.457719in}{2.123689in}}%
\pgfpathlineto{\pgfqpoint{1.459642in}{2.123775in}}%
\pgfpathlineto{\pgfqpoint{1.461566in}{2.140858in}}%
\pgfpathlineto{\pgfqpoint{1.463489in}{2.147049in}}%
\pgfpathlineto{\pgfqpoint{1.465413in}{2.141798in}}%
\pgfpathlineto{\pgfqpoint{1.469260in}{2.156415in}}%
\pgfpathlineto{\pgfqpoint{1.475030in}{2.180991in}}%
\pgfpathlineto{\pgfqpoint{1.476953in}{2.175957in}}%
\pgfpathlineto{\pgfqpoint{1.478877in}{2.186524in}}%
\pgfpathlineto{\pgfqpoint{1.480800in}{2.180489in}}%
\pgfpathlineto{\pgfqpoint{1.482723in}{2.185879in}}%
\pgfpathlineto{\pgfqpoint{1.486570in}{2.157348in}}%
\pgfpathlineto{\pgfqpoint{1.488494in}{2.150446in}}%
\pgfpathlineto{\pgfqpoint{1.490417in}{2.149990in}}%
\pgfpathlineto{\pgfqpoint{1.492340in}{2.141294in}}%
\pgfpathlineto{\pgfqpoint{1.494264in}{2.155343in}}%
\pgfpathlineto{\pgfqpoint{1.498111in}{2.141958in}}%
\pgfpathlineto{\pgfqpoint{1.500034in}{2.149002in}}%
\pgfpathlineto{\pgfqpoint{1.501958in}{2.146937in}}%
\pgfpathlineto{\pgfqpoint{1.503881in}{2.149261in}}%
\pgfpathlineto{\pgfqpoint{1.505804in}{2.145862in}}%
\pgfpathlineto{\pgfqpoint{1.507728in}{2.134160in}}%
\pgfpathlineto{\pgfqpoint{1.511575in}{2.135834in}}%
\pgfpathlineto{\pgfqpoint{1.513498in}{2.130886in}}%
\pgfpathlineto{\pgfqpoint{1.515421in}{2.132996in}}%
\pgfpathlineto{\pgfqpoint{1.517345in}{2.140311in}}%
\pgfpathlineto{\pgfqpoint{1.519268in}{2.132504in}}%
\pgfpathlineto{\pgfqpoint{1.521192in}{2.135912in}}%
\pgfpathlineto{\pgfqpoint{1.525038in}{2.110093in}}%
\pgfpathlineto{\pgfqpoint{1.526962in}{2.117059in}}%
\pgfpathlineto{\pgfqpoint{1.528885in}{2.118437in}}%
\pgfpathlineto{\pgfqpoint{1.530809in}{2.117516in}}%
\pgfpathlineto{\pgfqpoint{1.532732in}{2.122902in}}%
\pgfpathlineto{\pgfqpoint{1.534656in}{2.140425in}}%
\pgfpathlineto{\pgfqpoint{1.536579in}{2.136102in}}%
\pgfpathlineto{\pgfqpoint{1.540426in}{2.120383in}}%
\pgfpathlineto{\pgfqpoint{1.542349in}{2.125116in}}%
\pgfpathlineto{\pgfqpoint{1.544273in}{2.123704in}}%
\pgfpathlineto{\pgfqpoint{1.546196in}{2.101363in}}%
\pgfpathlineto{\pgfqpoint{1.548119in}{2.097595in}}%
\pgfpathlineto{\pgfqpoint{1.550043in}{2.096288in}}%
\pgfpathlineto{\pgfqpoint{1.553890in}{2.113191in}}%
\pgfpathlineto{\pgfqpoint{1.557736in}{2.111130in}}%
\pgfpathlineto{\pgfqpoint{1.559660in}{2.118424in}}%
\pgfpathlineto{\pgfqpoint{1.561583in}{2.115964in}}%
\pgfpathlineto{\pgfqpoint{1.563507in}{2.118508in}}%
\pgfpathlineto{\pgfqpoint{1.565430in}{2.118550in}}%
\pgfpathlineto{\pgfqpoint{1.567354in}{2.116832in}}%
\pgfpathlineto{\pgfqpoint{1.569277in}{2.104398in}}%
\pgfpathlineto{\pgfqpoint{1.573124in}{2.120938in}}%
\pgfpathlineto{\pgfqpoint{1.576971in}{2.106309in}}%
\pgfpathlineto{\pgfqpoint{1.578894in}{2.119972in}}%
\pgfpathlineto{\pgfqpoint{1.580817in}{2.115738in}}%
\pgfpathlineto{\pgfqpoint{1.584664in}{2.098412in}}%
\pgfpathlineto{\pgfqpoint{1.586588in}{2.091825in}}%
\pgfpathlineto{\pgfqpoint{1.588511in}{2.097035in}}%
\pgfpathlineto{\pgfqpoint{1.590434in}{2.096754in}}%
\pgfpathlineto{\pgfqpoint{1.592358in}{2.090219in}}%
\pgfpathlineto{\pgfqpoint{1.594281in}{2.089472in}}%
\pgfpathlineto{\pgfqpoint{1.596205in}{2.084284in}}%
\pgfpathlineto{\pgfqpoint{1.601975in}{2.082581in}}%
\pgfpathlineto{\pgfqpoint{1.603898in}{2.086281in}}%
\pgfpathlineto{\pgfqpoint{1.605822in}{2.082491in}}%
\pgfpathlineto{\pgfqpoint{1.607745in}{2.082353in}}%
\pgfpathlineto{\pgfqpoint{1.609669in}{2.074877in}}%
\pgfpathlineto{\pgfqpoint{1.611592in}{2.086885in}}%
\pgfpathlineto{\pgfqpoint{1.613515in}{2.083803in}}%
\pgfpathlineto{\pgfqpoint{1.615439in}{2.088088in}}%
\pgfpathlineto{\pgfqpoint{1.617362in}{2.089112in}}%
\pgfpathlineto{\pgfqpoint{1.619286in}{2.074350in}}%
\pgfpathlineto{\pgfqpoint{1.621209in}{2.074335in}}%
\pgfpathlineto{\pgfqpoint{1.623132in}{2.080974in}}%
\pgfpathlineto{\pgfqpoint{1.625056in}{2.073388in}}%
\pgfpathlineto{\pgfqpoint{1.626979in}{2.081838in}}%
\pgfpathlineto{\pgfqpoint{1.628903in}{2.065214in}}%
\pgfpathlineto{\pgfqpoint{1.630826in}{2.062287in}}%
\pgfpathlineto{\pgfqpoint{1.632749in}{2.062998in}}%
\pgfpathlineto{\pgfqpoint{1.634673in}{2.066401in}}%
\pgfpathlineto{\pgfqpoint{1.636596in}{2.074768in}}%
\pgfpathlineto{\pgfqpoint{1.638520in}{2.075507in}}%
\pgfpathlineto{\pgfqpoint{1.640443in}{2.067178in}}%
\pgfpathlineto{\pgfqpoint{1.642367in}{2.068079in}}%
\pgfpathlineto{\pgfqpoint{1.646213in}{2.047858in}}%
\pgfpathlineto{\pgfqpoint{1.648137in}{2.056500in}}%
\pgfpathlineto{\pgfqpoint{1.650060in}{2.055231in}}%
\pgfpathlineto{\pgfqpoint{1.653907in}{2.034962in}}%
\pgfpathlineto{\pgfqpoint{1.655830in}{2.036169in}}%
\pgfpathlineto{\pgfqpoint{1.657754in}{2.038695in}}%
\pgfpathlineto{\pgfqpoint{1.659677in}{2.033481in}}%
\pgfpathlineto{\pgfqpoint{1.661601in}{2.032678in}}%
\pgfpathlineto{\pgfqpoint{1.665447in}{2.024075in}}%
\pgfpathlineto{\pgfqpoint{1.667371in}{2.024850in}}%
\pgfpathlineto{\pgfqpoint{1.669294in}{2.035216in}}%
\pgfpathlineto{\pgfqpoint{1.671218in}{2.035512in}}%
\pgfpathlineto{\pgfqpoint{1.678911in}{2.066181in}}%
\pgfpathlineto{\pgfqpoint{1.680835in}{2.066349in}}%
\pgfpathlineto{\pgfqpoint{1.682758in}{2.060541in}}%
\pgfpathlineto{\pgfqpoint{1.684682in}{2.061390in}}%
\pgfpathlineto{\pgfqpoint{1.686605in}{2.071090in}}%
\pgfpathlineto{\pgfqpoint{1.690452in}{2.076781in}}%
\pgfpathlineto{\pgfqpoint{1.692375in}{2.074583in}}%
\pgfpathlineto{\pgfqpoint{1.694299in}{2.078771in}}%
\pgfpathlineto{\pgfqpoint{1.698145in}{2.098276in}}%
\pgfpathlineto{\pgfqpoint{1.700069in}{2.092990in}}%
\pgfpathlineto{\pgfqpoint{1.701992in}{2.102600in}}%
\pgfpathlineto{\pgfqpoint{1.703916in}{2.098465in}}%
\pgfpathlineto{\pgfqpoint{1.705839in}{2.104589in}}%
\pgfpathlineto{\pgfqpoint{1.707763in}{2.098712in}}%
\pgfpathlineto{\pgfqpoint{1.709686in}{2.110030in}}%
\pgfpathlineto{\pgfqpoint{1.711609in}{2.108532in}}%
\pgfpathlineto{\pgfqpoint{1.713533in}{2.116824in}}%
\pgfpathlineto{\pgfqpoint{1.719303in}{2.091022in}}%
\pgfpathlineto{\pgfqpoint{1.723150in}{2.093767in}}%
\pgfpathlineto{\pgfqpoint{1.725073in}{2.098545in}}%
\pgfpathlineto{\pgfqpoint{1.726997in}{2.097627in}}%
\pgfpathlineto{\pgfqpoint{1.728920in}{2.100234in}}%
\pgfpathlineto{\pgfqpoint{1.730843in}{2.105222in}}%
\pgfpathlineto{\pgfqpoint{1.732767in}{2.114740in}}%
\pgfpathlineto{\pgfqpoint{1.734690in}{2.113113in}}%
\pgfpathlineto{\pgfqpoint{1.736614in}{2.121841in}}%
\pgfpathlineto{\pgfqpoint{1.740461in}{2.113663in}}%
\pgfpathlineto{\pgfqpoint{1.742384in}{2.116742in}}%
\pgfpathlineto{\pgfqpoint{1.744307in}{2.117079in}}%
\pgfpathlineto{\pgfqpoint{1.746231in}{2.103930in}}%
\pgfpathlineto{\pgfqpoint{1.748154in}{2.113337in}}%
\pgfpathlineto{\pgfqpoint{1.750078in}{2.128883in}}%
\pgfpathlineto{\pgfqpoint{1.752001in}{2.125250in}}%
\pgfpathlineto{\pgfqpoint{1.753924in}{2.126804in}}%
\pgfpathlineto{\pgfqpoint{1.755848in}{2.130340in}}%
\pgfpathlineto{\pgfqpoint{1.757771in}{2.136283in}}%
\pgfpathlineto{\pgfqpoint{1.761618in}{2.127149in}}%
\pgfpathlineto{\pgfqpoint{1.765465in}{2.138478in}}%
\pgfpathlineto{\pgfqpoint{1.767388in}{2.122664in}}%
\pgfpathlineto{\pgfqpoint{1.769312in}{2.116507in}}%
\pgfpathlineto{\pgfqpoint{1.773158in}{2.119383in}}%
\pgfpathlineto{\pgfqpoint{1.775082in}{2.115743in}}%
\pgfpathlineto{\pgfqpoint{1.778929in}{2.123517in}}%
\pgfpathlineto{\pgfqpoint{1.780852in}{2.122476in}}%
\pgfpathlineto{\pgfqpoint{1.782776in}{2.127774in}}%
\pgfpathlineto{\pgfqpoint{1.784699in}{2.125213in}}%
\pgfpathlineto{\pgfqpoint{1.788546in}{2.134782in}}%
\pgfpathlineto{\pgfqpoint{1.790469in}{2.150105in}}%
\pgfpathlineto{\pgfqpoint{1.792393in}{2.150893in}}%
\pgfpathlineto{\pgfqpoint{1.796239in}{2.136289in}}%
\pgfpathlineto{\pgfqpoint{1.798163in}{2.118119in}}%
\pgfpathlineto{\pgfqpoint{1.802010in}{2.133101in}}%
\pgfpathlineto{\pgfqpoint{1.803933in}{2.126352in}}%
\pgfpathlineto{\pgfqpoint{1.807780in}{2.122276in}}%
\pgfpathlineto{\pgfqpoint{1.809703in}{2.126853in}}%
\pgfpathlineto{\pgfqpoint{1.811627in}{2.127173in}}%
\pgfpathlineto{\pgfqpoint{1.815474in}{2.134687in}}%
\pgfpathlineto{\pgfqpoint{1.817397in}{2.134846in}}%
\pgfpathlineto{\pgfqpoint{1.819320in}{2.137532in}}%
\pgfpathlineto{\pgfqpoint{1.821244in}{2.137689in}}%
\pgfpathlineto{\pgfqpoint{1.823167in}{2.141959in}}%
\pgfpathlineto{\pgfqpoint{1.825091in}{2.137002in}}%
\pgfpathlineto{\pgfqpoint{1.827014in}{2.135565in}}%
\pgfpathlineto{\pgfqpoint{1.828937in}{2.130876in}}%
\pgfpathlineto{\pgfqpoint{1.830861in}{2.139854in}}%
\pgfpathlineto{\pgfqpoint{1.832784in}{2.138273in}}%
\pgfpathlineto{\pgfqpoint{1.834708in}{2.139599in}}%
\pgfpathlineto{\pgfqpoint{1.836631in}{2.127520in}}%
\pgfpathlineto{\pgfqpoint{1.838554in}{2.127510in}}%
\pgfpathlineto{\pgfqpoint{1.840478in}{2.131408in}}%
\pgfpathlineto{\pgfqpoint{1.842401in}{2.129733in}}%
\pgfpathlineto{\pgfqpoint{1.844325in}{2.139379in}}%
\pgfpathlineto{\pgfqpoint{1.848172in}{2.142876in}}%
\pgfpathlineto{\pgfqpoint{1.850095in}{2.130456in}}%
\pgfpathlineto{\pgfqpoint{1.852018in}{2.128693in}}%
\pgfpathlineto{\pgfqpoint{1.853942in}{2.135237in}}%
\pgfpathlineto{\pgfqpoint{1.855865in}{2.135999in}}%
\pgfpathlineto{\pgfqpoint{1.857789in}{2.140051in}}%
\pgfpathlineto{\pgfqpoint{1.865482in}{2.194660in}}%
\pgfpathlineto{\pgfqpoint{1.867406in}{2.194314in}}%
\pgfpathlineto{\pgfqpoint{1.869329in}{2.195144in}}%
\pgfpathlineto{\pgfqpoint{1.871252in}{2.192233in}}%
\pgfpathlineto{\pgfqpoint{1.873176in}{2.193421in}}%
\pgfpathlineto{\pgfqpoint{1.875099in}{2.203740in}}%
\pgfpathlineto{\pgfqpoint{1.877023in}{2.192806in}}%
\pgfpathlineto{\pgfqpoint{1.878946in}{2.197417in}}%
\pgfpathlineto{\pgfqpoint{1.880870in}{2.186626in}}%
\pgfpathlineto{\pgfqpoint{1.882793in}{2.188620in}}%
\pgfpathlineto{\pgfqpoint{1.884716in}{2.188696in}}%
\pgfpathlineto{\pgfqpoint{1.886640in}{2.204576in}}%
\pgfpathlineto{\pgfqpoint{1.888563in}{2.204559in}}%
\pgfpathlineto{\pgfqpoint{1.890487in}{2.190945in}}%
\pgfpathlineto{\pgfqpoint{1.892410in}{2.199846in}}%
\pgfpathlineto{\pgfqpoint{1.894333in}{2.186177in}}%
\pgfpathlineto{\pgfqpoint{1.896257in}{2.198654in}}%
\pgfpathlineto{\pgfqpoint{1.898180in}{2.195968in}}%
\pgfpathlineto{\pgfqpoint{1.900104in}{2.197325in}}%
\pgfpathlineto{\pgfqpoint{1.902027in}{2.184975in}}%
\pgfpathlineto{\pgfqpoint{1.903950in}{2.192586in}}%
\pgfpathlineto{\pgfqpoint{1.905874in}{2.188403in}}%
\pgfpathlineto{\pgfqpoint{1.907797in}{2.191838in}}%
\pgfpathlineto{\pgfqpoint{1.909721in}{2.179716in}}%
\pgfpathlineto{\pgfqpoint{1.911644in}{2.183486in}}%
\pgfpathlineto{\pgfqpoint{1.913567in}{2.179293in}}%
\pgfpathlineto{\pgfqpoint{1.915491in}{2.180710in}}%
\pgfpathlineto{\pgfqpoint{1.917414in}{2.193101in}}%
\pgfpathlineto{\pgfqpoint{1.919338in}{2.188362in}}%
\pgfpathlineto{\pgfqpoint{1.921261in}{2.190283in}}%
\pgfpathlineto{\pgfqpoint{1.923185in}{2.180608in}}%
\pgfpathlineto{\pgfqpoint{1.925108in}{2.184368in}}%
\pgfpathlineto{\pgfqpoint{1.930878in}{2.156248in}}%
\pgfpathlineto{\pgfqpoint{1.932802in}{2.161617in}}%
\pgfpathlineto{\pgfqpoint{1.934725in}{2.158380in}}%
\pgfpathlineto{\pgfqpoint{1.938572in}{2.146398in}}%
\pgfpathlineto{\pgfqpoint{1.940495in}{2.133685in}}%
\pgfpathlineto{\pgfqpoint{1.942419in}{2.134447in}}%
\pgfpathlineto{\pgfqpoint{1.946265in}{2.145152in}}%
\pgfpathlineto{\pgfqpoint{1.948189in}{2.141314in}}%
\pgfpathlineto{\pgfqpoint{1.950112in}{2.144646in}}%
\pgfpathlineto{\pgfqpoint{1.953959in}{2.144779in}}%
\pgfpathlineto{\pgfqpoint{1.957806in}{2.131094in}}%
\pgfpathlineto{\pgfqpoint{1.963576in}{2.153465in}}%
\pgfpathlineto{\pgfqpoint{1.965500in}{2.153211in}}%
\pgfpathlineto{\pgfqpoint{1.967423in}{2.154868in}}%
\pgfpathlineto{\pgfqpoint{1.969346in}{2.144279in}}%
\pgfpathlineto{\pgfqpoint{1.971270in}{2.140683in}}%
\pgfpathlineto{\pgfqpoint{1.978963in}{2.142553in}}%
\pgfpathlineto{\pgfqpoint{1.980887in}{2.140255in}}%
\pgfpathlineto{\pgfqpoint{1.982810in}{2.151016in}}%
\pgfpathlineto{\pgfqpoint{1.986657in}{2.154392in}}%
\pgfpathlineto{\pgfqpoint{1.988581in}{2.148536in}}%
\pgfpathlineto{\pgfqpoint{1.990504in}{2.151406in}}%
\pgfpathlineto{\pgfqpoint{1.992427in}{2.165013in}}%
\pgfpathlineto{\pgfqpoint{1.994351in}{2.162757in}}%
\pgfpathlineto{\pgfqpoint{1.996274in}{2.163239in}}%
\pgfpathlineto{\pgfqpoint{2.000121in}{2.185176in}}%
\pgfpathlineto{\pgfqpoint{2.002044in}{2.184643in}}%
\pgfpathlineto{\pgfqpoint{2.003968in}{2.172722in}}%
\pgfpathlineto{\pgfqpoint{2.005891in}{2.174279in}}%
\pgfpathlineto{\pgfqpoint{2.007815in}{2.171133in}}%
\pgfpathlineto{\pgfqpoint{2.009738in}{2.172061in}}%
\pgfpathlineto{\pgfqpoint{2.011661in}{2.174511in}}%
\pgfpathlineto{\pgfqpoint{2.013585in}{2.173501in}}%
\pgfpathlineto{\pgfqpoint{2.015508in}{2.170092in}}%
\pgfpathlineto{\pgfqpoint{2.019355in}{2.179870in}}%
\pgfpathlineto{\pgfqpoint{2.021279in}{2.179813in}}%
\pgfpathlineto{\pgfqpoint{2.023202in}{2.185737in}}%
\pgfpathlineto{\pgfqpoint{2.025125in}{2.187689in}}%
\pgfpathlineto{\pgfqpoint{2.027049in}{2.193582in}}%
\pgfpathlineto{\pgfqpoint{2.028972in}{2.186188in}}%
\pgfpathlineto{\pgfqpoint{2.030896in}{2.192008in}}%
\pgfpathlineto{\pgfqpoint{2.034742in}{2.194411in}}%
\pgfpathlineto{\pgfqpoint{2.036666in}{2.188523in}}%
\pgfpathlineto{\pgfqpoint{2.038589in}{2.196906in}}%
\pgfpathlineto{\pgfqpoint{2.040513in}{2.194542in}}%
\pgfpathlineto{\pgfqpoint{2.046283in}{2.163206in}}%
\pgfpathlineto{\pgfqpoint{2.050130in}{2.184510in}}%
\pgfpathlineto{\pgfqpoint{2.057823in}{2.208723in}}%
\pgfpathlineto{\pgfqpoint{2.061670in}{2.211747in}}%
\pgfpathlineto{\pgfqpoint{2.063594in}{2.211256in}}%
\pgfpathlineto{\pgfqpoint{2.065517in}{2.209039in}}%
\pgfpathlineto{\pgfqpoint{2.067440in}{2.212854in}}%
\pgfpathlineto{\pgfqpoint{2.069364in}{2.208541in}}%
\pgfpathlineto{\pgfqpoint{2.073211in}{2.209303in}}%
\pgfpathlineto{\pgfqpoint{2.075134in}{2.213030in}}%
\pgfpathlineto{\pgfqpoint{2.077057in}{2.211150in}}%
\pgfpathlineto{\pgfqpoint{2.078981in}{2.214808in}}%
\pgfpathlineto{\pgfqpoint{2.080904in}{2.214008in}}%
\pgfpathlineto{\pgfqpoint{2.082828in}{2.209787in}}%
\pgfpathlineto{\pgfqpoint{2.084751in}{2.209041in}}%
\pgfpathlineto{\pgfqpoint{2.086674in}{2.214432in}}%
\pgfpathlineto{\pgfqpoint{2.090521in}{2.229881in}}%
\pgfpathlineto{\pgfqpoint{2.092445in}{2.229730in}}%
\pgfpathlineto{\pgfqpoint{2.094368in}{2.239471in}}%
\pgfpathlineto{\pgfqpoint{2.096292in}{2.237602in}}%
\pgfpathlineto{\pgfqpoint{2.098215in}{2.232753in}}%
\pgfpathlineto{\pgfqpoint{2.102062in}{2.252978in}}%
\pgfpathlineto{\pgfqpoint{2.103985in}{2.243210in}}%
\pgfpathlineto{\pgfqpoint{2.109755in}{2.234096in}}%
\pgfpathlineto{\pgfqpoint{2.111679in}{2.224393in}}%
\pgfpathlineto{\pgfqpoint{2.115526in}{2.234936in}}%
\pgfpathlineto{\pgfqpoint{2.117449in}{2.242162in}}%
\pgfpathlineto{\pgfqpoint{2.121296in}{2.239541in}}%
\pgfpathlineto{\pgfqpoint{2.125143in}{2.226399in}}%
\pgfpathlineto{\pgfqpoint{2.127066in}{2.225302in}}%
\pgfpathlineto{\pgfqpoint{2.128990in}{2.227407in}}%
\pgfpathlineto{\pgfqpoint{2.130913in}{2.236213in}}%
\pgfpathlineto{\pgfqpoint{2.132836in}{2.239134in}}%
\pgfpathlineto{\pgfqpoint{2.134760in}{2.232357in}}%
\pgfpathlineto{\pgfqpoint{2.136683in}{2.215815in}}%
\pgfpathlineto{\pgfqpoint{2.138607in}{2.229765in}}%
\pgfpathlineto{\pgfqpoint{2.140530in}{2.230110in}}%
\pgfpathlineto{\pgfqpoint{2.144377in}{2.223868in}}%
\pgfpathlineto{\pgfqpoint{2.146300in}{2.225891in}}%
\pgfpathlineto{\pgfqpoint{2.148224in}{2.224794in}}%
\pgfpathlineto{\pgfqpoint{2.153994in}{2.235329in}}%
\pgfpathlineto{\pgfqpoint{2.157841in}{2.225821in}}%
\pgfpathlineto{\pgfqpoint{2.159764in}{2.224558in}}%
\pgfpathlineto{\pgfqpoint{2.163611in}{2.231855in}}%
\pgfpathlineto{\pgfqpoint{2.165534in}{2.233637in}}%
\pgfpathlineto{\pgfqpoint{2.167458in}{2.229409in}}%
\pgfpathlineto{\pgfqpoint{2.169381in}{2.232802in}}%
\pgfpathlineto{\pgfqpoint{2.173228in}{2.244638in}}%
\pgfpathlineto{\pgfqpoint{2.175151in}{2.247670in}}%
\pgfpathlineto{\pgfqpoint{2.178998in}{2.246592in}}%
\pgfpathlineto{\pgfqpoint{2.180922in}{2.243959in}}%
\pgfpathlineto{\pgfqpoint{2.184768in}{2.227037in}}%
\pgfpathlineto{\pgfqpoint{2.188615in}{2.214746in}}%
\pgfpathlineto{\pgfqpoint{2.190539in}{2.216085in}}%
\pgfpathlineto{\pgfqpoint{2.192462in}{2.207208in}}%
\pgfpathlineto{\pgfqpoint{2.194386in}{2.207859in}}%
\pgfpathlineto{\pgfqpoint{2.196309in}{2.219151in}}%
\pgfpathlineto{\pgfqpoint{2.204003in}{2.229354in}}%
\pgfpathlineto{\pgfqpoint{2.205926in}{2.224311in}}%
\pgfpathlineto{\pgfqpoint{2.207849in}{2.213446in}}%
\pgfpathlineto{\pgfqpoint{2.209773in}{2.227675in}}%
\pgfpathlineto{\pgfqpoint{2.211696in}{2.224897in}}%
\pgfpathlineto{\pgfqpoint{2.213620in}{2.201588in}}%
\pgfpathlineto{\pgfqpoint{2.215543in}{2.221948in}}%
\pgfpathlineto{\pgfqpoint{2.217466in}{2.220637in}}%
\pgfpathlineto{\pgfqpoint{2.219390in}{2.214160in}}%
\pgfpathlineto{\pgfqpoint{2.221313in}{2.224964in}}%
\pgfpathlineto{\pgfqpoint{2.223237in}{2.220901in}}%
\pgfpathlineto{\pgfqpoint{2.227083in}{2.232229in}}%
\pgfpathlineto{\pgfqpoint{2.229007in}{2.228215in}}%
\pgfpathlineto{\pgfqpoint{2.232854in}{2.241668in}}%
\pgfpathlineto{\pgfqpoint{2.236701in}{2.235828in}}%
\pgfpathlineto{\pgfqpoint{2.240547in}{2.241202in}}%
\pgfpathlineto{\pgfqpoint{2.242471in}{2.239390in}}%
\pgfpathlineto{\pgfqpoint{2.244394in}{2.250439in}}%
\pgfpathlineto{\pgfqpoint{2.246318in}{2.252286in}}%
\pgfpathlineto{\pgfqpoint{2.248241in}{2.251105in}}%
\pgfpathlineto{\pgfqpoint{2.252088in}{2.243583in}}%
\pgfpathlineto{\pgfqpoint{2.254011in}{2.245653in}}%
\pgfpathlineto{\pgfqpoint{2.255935in}{2.252067in}}%
\pgfpathlineto{\pgfqpoint{2.257858in}{2.252236in}}%
\pgfpathlineto{\pgfqpoint{2.259781in}{2.254573in}}%
\pgfpathlineto{\pgfqpoint{2.263628in}{2.244494in}}%
\pgfpathlineto{\pgfqpoint{2.265552in}{2.252854in}}%
\pgfpathlineto{\pgfqpoint{2.267475in}{2.254115in}}%
\pgfpathlineto{\pgfqpoint{2.269399in}{2.252749in}}%
\pgfpathlineto{\pgfqpoint{2.271322in}{2.261051in}}%
\pgfpathlineto{\pgfqpoint{2.275169in}{2.256479in}}%
\pgfpathlineto{\pgfqpoint{2.277092in}{2.252256in}}%
\pgfpathlineto{\pgfqpoint{2.280939in}{2.273350in}}%
\pgfpathlineto{\pgfqpoint{2.282862in}{2.264595in}}%
\pgfpathlineto{\pgfqpoint{2.284786in}{2.269201in}}%
\pgfpathlineto{\pgfqpoint{2.288633in}{2.261184in}}%
\pgfpathlineto{\pgfqpoint{2.292479in}{2.266866in}}%
\pgfpathlineto{\pgfqpoint{2.294403in}{2.266517in}}%
\pgfpathlineto{\pgfqpoint{2.296326in}{2.263937in}}%
\pgfpathlineto{\pgfqpoint{2.298250in}{2.263360in}}%
\pgfpathlineto{\pgfqpoint{2.300173in}{2.260567in}}%
\pgfpathlineto{\pgfqpoint{2.304020in}{2.242644in}}%
\pgfpathlineto{\pgfqpoint{2.309790in}{2.235913in}}%
\pgfpathlineto{\pgfqpoint{2.311714in}{2.240286in}}%
\pgfpathlineto{\pgfqpoint{2.313637in}{2.265315in}}%
\pgfpathlineto{\pgfqpoint{2.315560in}{2.267056in}}%
\pgfpathlineto{\pgfqpoint{2.317484in}{2.264292in}}%
\pgfpathlineto{\pgfqpoint{2.319407in}{2.270288in}}%
\pgfpathlineto{\pgfqpoint{2.321331in}{2.261487in}}%
\pgfpathlineto{\pgfqpoint{2.323254in}{2.269517in}}%
\pgfpathlineto{\pgfqpoint{2.327101in}{2.271105in}}%
\pgfpathlineto{\pgfqpoint{2.329024in}{2.255460in}}%
\pgfpathlineto{\pgfqpoint{2.330948in}{2.255919in}}%
\pgfpathlineto{\pgfqpoint{2.332871in}{2.258435in}}%
\pgfpathlineto{\pgfqpoint{2.334795in}{2.266603in}}%
\pgfpathlineto{\pgfqpoint{2.336718in}{2.269339in}}%
\pgfpathlineto{\pgfqpoint{2.338641in}{2.274904in}}%
\pgfpathlineto{\pgfqpoint{2.340565in}{2.275142in}}%
\pgfpathlineto{\pgfqpoint{2.342488in}{2.273664in}}%
\pgfpathlineto{\pgfqpoint{2.344412in}{2.280898in}}%
\pgfpathlineto{\pgfqpoint{2.350182in}{2.268812in}}%
\pgfpathlineto{\pgfqpoint{2.352105in}{2.277780in}}%
\pgfpathlineto{\pgfqpoint{2.355952in}{2.263722in}}%
\pgfpathlineto{\pgfqpoint{2.357875in}{2.273857in}}%
\pgfpathlineto{\pgfqpoint{2.361722in}{2.280111in}}%
\pgfpathlineto{\pgfqpoint{2.365569in}{2.272247in}}%
\pgfpathlineto{\pgfqpoint{2.367492in}{2.289367in}}%
\pgfpathlineto{\pgfqpoint{2.369416in}{2.293144in}}%
\pgfpathlineto{\pgfqpoint{2.371339in}{2.301326in}}%
\pgfpathlineto{\pgfqpoint{2.373263in}{2.287953in}}%
\pgfpathlineto{\pgfqpoint{2.377110in}{2.305839in}}%
\pgfpathlineto{\pgfqpoint{2.379033in}{2.303676in}}%
\pgfpathlineto{\pgfqpoint{2.380956in}{2.310375in}}%
\pgfpathlineto{\pgfqpoint{2.382880in}{2.308948in}}%
\pgfpathlineto{\pgfqpoint{2.384803in}{2.309544in}}%
\pgfpathlineto{\pgfqpoint{2.386727in}{2.318246in}}%
\pgfpathlineto{\pgfqpoint{2.390573in}{2.321813in}}%
\pgfpathlineto{\pgfqpoint{2.398267in}{2.348995in}}%
\pgfpathlineto{\pgfqpoint{2.400190in}{2.345955in}}%
\pgfpathlineto{\pgfqpoint{2.402114in}{2.348906in}}%
\pgfpathlineto{\pgfqpoint{2.404037in}{2.341374in}}%
\pgfpathlineto{\pgfqpoint{2.405961in}{2.344913in}}%
\pgfpathlineto{\pgfqpoint{2.409808in}{2.338730in}}%
\pgfpathlineto{\pgfqpoint{2.413654in}{2.350352in}}%
\pgfpathlineto{\pgfqpoint{2.415578in}{2.353282in}}%
\pgfpathlineto{\pgfqpoint{2.417501in}{2.346751in}}%
\pgfpathlineto{\pgfqpoint{2.419425in}{2.347747in}}%
\pgfpathlineto{\pgfqpoint{2.421348in}{2.350660in}}%
\pgfpathlineto{\pgfqpoint{2.423271in}{2.361917in}}%
\pgfpathlineto{\pgfqpoint{2.425195in}{2.360452in}}%
\pgfpathlineto{\pgfqpoint{2.429042in}{2.338020in}}%
\pgfpathlineto{\pgfqpoint{2.432888in}{2.334264in}}%
\pgfpathlineto{\pgfqpoint{2.436735in}{2.314465in}}%
\pgfpathlineto{\pgfqpoint{2.438659in}{2.312480in}}%
\pgfpathlineto{\pgfqpoint{2.440582in}{2.315962in}}%
\pgfpathlineto{\pgfqpoint{2.444429in}{2.333437in}}%
\pgfpathlineto{\pgfqpoint{2.446352in}{2.335904in}}%
\pgfpathlineto{\pgfqpoint{2.448276in}{2.327499in}}%
\pgfpathlineto{\pgfqpoint{2.450199in}{2.328576in}}%
\pgfpathlineto{\pgfqpoint{2.452123in}{2.331643in}}%
\pgfpathlineto{\pgfqpoint{2.454046in}{2.337056in}}%
\pgfpathlineto{\pgfqpoint{2.455969in}{2.334079in}}%
\pgfpathlineto{\pgfqpoint{2.459816in}{2.320184in}}%
\pgfpathlineto{\pgfqpoint{2.463663in}{2.326195in}}%
\pgfpathlineto{\pgfqpoint{2.465586in}{2.312784in}}%
\pgfpathlineto{\pgfqpoint{2.467510in}{2.313475in}}%
\pgfpathlineto{\pgfqpoint{2.471357in}{2.328239in}}%
\pgfpathlineto{\pgfqpoint{2.473280in}{2.314503in}}%
\pgfpathlineto{\pgfqpoint{2.475204in}{2.318874in}}%
\pgfpathlineto{\pgfqpoint{2.479050in}{2.314966in}}%
\pgfpathlineto{\pgfqpoint{2.480974in}{2.317779in}}%
\pgfpathlineto{\pgfqpoint{2.482897in}{2.318348in}}%
\pgfpathlineto{\pgfqpoint{2.484821in}{2.308468in}}%
\pgfpathlineto{\pgfqpoint{2.488667in}{2.316836in}}%
\pgfpathlineto{\pgfqpoint{2.490591in}{2.306123in}}%
\pgfpathlineto{\pgfqpoint{2.492514in}{2.304751in}}%
\pgfpathlineto{\pgfqpoint{2.494438in}{2.310678in}}%
\pgfpathlineto{\pgfqpoint{2.496361in}{2.322049in}}%
\pgfpathlineto{\pgfqpoint{2.498284in}{2.324042in}}%
\pgfpathlineto{\pgfqpoint{2.500208in}{2.322627in}}%
\pgfpathlineto{\pgfqpoint{2.505978in}{2.312283in}}%
\pgfpathlineto{\pgfqpoint{2.507901in}{2.322247in}}%
\pgfpathlineto{\pgfqpoint{2.513672in}{2.307683in}}%
\pgfpathlineto{\pgfqpoint{2.515595in}{2.311402in}}%
\pgfpathlineto{\pgfqpoint{2.521365in}{2.329444in}}%
\pgfpathlineto{\pgfqpoint{2.523289in}{2.328998in}}%
\pgfpathlineto{\pgfqpoint{2.525212in}{2.321141in}}%
\pgfpathlineto{\pgfqpoint{2.527136in}{2.322811in}}%
\pgfpathlineto{\pgfqpoint{2.530982in}{2.317307in}}%
\pgfpathlineto{\pgfqpoint{2.532906in}{2.318994in}}%
\pgfpathlineto{\pgfqpoint{2.538676in}{2.351504in}}%
\pgfpathlineto{\pgfqpoint{2.540599in}{2.352031in}}%
\pgfpathlineto{\pgfqpoint{2.544446in}{2.338198in}}%
\pgfpathlineto{\pgfqpoint{2.546370in}{2.331991in}}%
\pgfpathlineto{\pgfqpoint{2.548293in}{2.321943in}}%
\pgfpathlineto{\pgfqpoint{2.552140in}{2.341118in}}%
\pgfpathlineto{\pgfqpoint{2.554063in}{2.329375in}}%
\pgfpathlineto{\pgfqpoint{2.559834in}{2.355640in}}%
\pgfpathlineto{\pgfqpoint{2.561757in}{2.374355in}}%
\pgfpathlineto{\pgfqpoint{2.563680in}{2.374499in}}%
\pgfpathlineto{\pgfqpoint{2.571374in}{2.356666in}}%
\pgfpathlineto{\pgfqpoint{2.577144in}{2.338984in}}%
\pgfpathlineto{\pgfqpoint{2.580991in}{2.354925in}}%
\pgfpathlineto{\pgfqpoint{2.582915in}{2.343789in}}%
\pgfpathlineto{\pgfqpoint{2.584838in}{2.342422in}}%
\pgfpathlineto{\pgfqpoint{2.588685in}{2.332489in}}%
\pgfpathlineto{\pgfqpoint{2.590608in}{2.351367in}}%
\pgfpathlineto{\pgfqpoint{2.592532in}{2.339035in}}%
\pgfpathlineto{\pgfqpoint{2.594455in}{2.338555in}}%
\pgfpathlineto{\pgfqpoint{2.596378in}{2.330712in}}%
\pgfpathlineto{\pgfqpoint{2.598302in}{2.331739in}}%
\pgfpathlineto{\pgfqpoint{2.600225in}{2.348852in}}%
\pgfpathlineto{\pgfqpoint{2.602149in}{2.340167in}}%
\pgfpathlineto{\pgfqpoint{2.604072in}{2.342794in}}%
\pgfpathlineto{\pgfqpoint{2.605995in}{2.343300in}}%
\pgfpathlineto{\pgfqpoint{2.607919in}{2.344972in}}%
\pgfpathlineto{\pgfqpoint{2.609842in}{2.348962in}}%
\pgfpathlineto{\pgfqpoint{2.611766in}{2.357997in}}%
\pgfpathlineto{\pgfqpoint{2.613689in}{2.352882in}}%
\pgfpathlineto{\pgfqpoint{2.615613in}{2.361419in}}%
\pgfpathlineto{\pgfqpoint{2.617536in}{2.350035in}}%
\pgfpathlineto{\pgfqpoint{2.621383in}{2.354211in}}%
\pgfpathlineto{\pgfqpoint{2.623306in}{2.359585in}}%
\pgfpathlineto{\pgfqpoint{2.625230in}{2.351628in}}%
\pgfpathlineto{\pgfqpoint{2.627153in}{2.350442in}}%
\pgfpathlineto{\pgfqpoint{2.629076in}{2.343999in}}%
\pgfpathlineto{\pgfqpoint{2.631000in}{2.342477in}}%
\pgfpathlineto{\pgfqpoint{2.632923in}{2.351084in}}%
\pgfpathlineto{\pgfqpoint{2.634847in}{2.352797in}}%
\pgfpathlineto{\pgfqpoint{2.636770in}{2.357974in}}%
\pgfpathlineto{\pgfqpoint{2.640617in}{2.349517in}}%
\pgfpathlineto{\pgfqpoint{2.642540in}{2.362341in}}%
\pgfpathlineto{\pgfqpoint{2.644464in}{2.361164in}}%
\pgfpathlineto{\pgfqpoint{2.646387in}{2.362045in}}%
\pgfpathlineto{\pgfqpoint{2.650234in}{2.359279in}}%
\pgfpathlineto{\pgfqpoint{2.652157in}{2.348815in}}%
\pgfpathlineto{\pgfqpoint{2.654081in}{2.366210in}}%
\pgfpathlineto{\pgfqpoint{2.657928in}{2.365476in}}%
\pgfpathlineto{\pgfqpoint{2.663698in}{2.341871in}}%
\pgfpathlineto{\pgfqpoint{2.665621in}{2.344219in}}%
\pgfpathlineto{\pgfqpoint{2.667545in}{2.339121in}}%
\pgfpathlineto{\pgfqpoint{2.669468in}{2.337409in}}%
\pgfpathlineto{\pgfqpoint{2.671391in}{2.346452in}}%
\pgfpathlineto{\pgfqpoint{2.673315in}{2.340884in}}%
\pgfpathlineto{\pgfqpoint{2.675238in}{2.342022in}}%
\pgfpathlineto{\pgfqpoint{2.677162in}{2.334999in}}%
\pgfpathlineto{\pgfqpoint{2.679085in}{2.340979in}}%
\pgfpathlineto{\pgfqpoint{2.681008in}{2.336502in}}%
\pgfpathlineto{\pgfqpoint{2.692549in}{2.344888in}}%
\pgfpathlineto{\pgfqpoint{2.696396in}{2.368820in}}%
\pgfpathlineto{\pgfqpoint{2.698319in}{2.368956in}}%
\pgfpathlineto{\pgfqpoint{2.700243in}{2.376438in}}%
\pgfpathlineto{\pgfqpoint{2.704089in}{2.356854in}}%
\pgfpathlineto{\pgfqpoint{2.706013in}{2.366759in}}%
\pgfpathlineto{\pgfqpoint{2.707936in}{2.366378in}}%
\pgfpathlineto{\pgfqpoint{2.709860in}{2.369876in}}%
\pgfpathlineto{\pgfqpoint{2.711783in}{2.362978in}}%
\pgfpathlineto{\pgfqpoint{2.713706in}{2.361145in}}%
\pgfpathlineto{\pgfqpoint{2.715630in}{2.355812in}}%
\pgfpathlineto{\pgfqpoint{2.717553in}{2.360161in}}%
\pgfpathlineto{\pgfqpoint{2.719477in}{2.353865in}}%
\pgfpathlineto{\pgfqpoint{2.721400in}{2.352373in}}%
\pgfpathlineto{\pgfqpoint{2.723324in}{2.364331in}}%
\pgfpathlineto{\pgfqpoint{2.725247in}{2.360324in}}%
\pgfpathlineto{\pgfqpoint{2.731017in}{2.382473in}}%
\pgfpathlineto{\pgfqpoint{2.732941in}{2.376244in}}%
\pgfpathlineto{\pgfqpoint{2.734864in}{2.386624in}}%
\pgfpathlineto{\pgfqpoint{2.738711in}{2.385068in}}%
\pgfpathlineto{\pgfqpoint{2.740634in}{2.375224in}}%
\pgfpathlineto{\pgfqpoint{2.742558in}{2.385879in}}%
\pgfpathlineto{\pgfqpoint{2.746404in}{2.372583in}}%
\pgfpathlineto{\pgfqpoint{2.750251in}{2.367276in}}%
\pgfpathlineto{\pgfqpoint{2.752175in}{2.370924in}}%
\pgfpathlineto{\pgfqpoint{2.754098in}{2.362995in}}%
\pgfpathlineto{\pgfqpoint{2.756022in}{2.369679in}}%
\pgfpathlineto{\pgfqpoint{2.757945in}{2.363336in}}%
\pgfpathlineto{\pgfqpoint{2.759868in}{2.363579in}}%
\pgfpathlineto{\pgfqpoint{2.761792in}{2.369932in}}%
\pgfpathlineto{\pgfqpoint{2.763715in}{2.368265in}}%
\pgfpathlineto{\pgfqpoint{2.765639in}{2.369885in}}%
\pgfpathlineto{\pgfqpoint{2.767562in}{2.360274in}}%
\pgfpathlineto{\pgfqpoint{2.767562in}{2.360274in}}%
\pgfusepath{stroke}%
\end{pgfscope}%
\begin{pgfscope}%
\pgfpathrectangle{\pgfqpoint{0.750000in}{0.660000in}}{\pgfqpoint{2.113636in}{2.100000in}}%
\pgfusepath{clip}%
\pgfsetroundcap%
\pgfsetroundjoin%
\pgfsetlinewidth{0.602250pt}%
\definecolor{currentstroke}{rgb}{0.600000,0.600000,0.600000}%
\pgfsetstrokecolor{currentstroke}%
\pgfsetdash{}{0pt}%
\pgfpathmoveto{\pgfqpoint{0.846074in}{1.931296in}}%
\pgfpathlineto{\pgfqpoint{0.847998in}{1.935729in}}%
\pgfpathlineto{\pgfqpoint{0.849921in}{1.931751in}}%
\pgfpathlineto{\pgfqpoint{0.851845in}{1.939471in}}%
\pgfpathlineto{\pgfqpoint{0.853768in}{1.939008in}}%
\pgfpathlineto{\pgfqpoint{0.855691in}{1.933644in}}%
\pgfpathlineto{\pgfqpoint{0.857615in}{1.943840in}}%
\pgfpathlineto{\pgfqpoint{0.859538in}{1.945102in}}%
\pgfpathlineto{\pgfqpoint{0.861462in}{1.951774in}}%
\pgfpathlineto{\pgfqpoint{0.863385in}{1.953430in}}%
\pgfpathlineto{\pgfqpoint{0.865308in}{1.947634in}}%
\pgfpathlineto{\pgfqpoint{0.867232in}{1.949692in}}%
\pgfpathlineto{\pgfqpoint{0.869155in}{1.956179in}}%
\pgfpathlineto{\pgfqpoint{0.871079in}{1.955011in}}%
\pgfpathlineto{\pgfqpoint{0.873002in}{1.955946in}}%
\pgfpathlineto{\pgfqpoint{0.874926in}{1.946521in}}%
\pgfpathlineto{\pgfqpoint{0.876849in}{1.931245in}}%
\pgfpathlineto{\pgfqpoint{0.878772in}{1.939803in}}%
\pgfpathlineto{\pgfqpoint{0.880696in}{1.939429in}}%
\pgfpathlineto{\pgfqpoint{0.882619in}{1.948695in}}%
\pgfpathlineto{\pgfqpoint{0.886466in}{1.947030in}}%
\pgfpathlineto{\pgfqpoint{0.888389in}{1.950685in}}%
\pgfpathlineto{\pgfqpoint{0.890313in}{1.957831in}}%
\pgfpathlineto{\pgfqpoint{0.892236in}{1.944444in}}%
\pgfpathlineto{\pgfqpoint{0.896083in}{1.948681in}}%
\pgfpathlineto{\pgfqpoint{0.898006in}{1.948902in}}%
\pgfpathlineto{\pgfqpoint{0.899930in}{1.954131in}}%
\pgfpathlineto{\pgfqpoint{0.901853in}{1.941426in}}%
\pgfpathlineto{\pgfqpoint{0.909547in}{1.970051in}}%
\pgfpathlineto{\pgfqpoint{0.911470in}{1.960148in}}%
\pgfpathlineto{\pgfqpoint{0.915317in}{1.961735in}}%
\pgfpathlineto{\pgfqpoint{0.919164in}{1.983839in}}%
\pgfpathlineto{\pgfqpoint{0.921087in}{1.985206in}}%
\pgfpathlineto{\pgfqpoint{0.923011in}{1.989882in}}%
\pgfpathlineto{\pgfqpoint{0.924934in}{1.977104in}}%
\pgfpathlineto{\pgfqpoint{0.926858in}{1.986670in}}%
\pgfpathlineto{\pgfqpoint{0.928781in}{1.990184in}}%
\pgfpathlineto{\pgfqpoint{0.930704in}{1.998667in}}%
\pgfpathlineto{\pgfqpoint{0.932628in}{1.991741in}}%
\pgfpathlineto{\pgfqpoint{0.934551in}{1.993442in}}%
\pgfpathlineto{\pgfqpoint{0.940322in}{2.004932in}}%
\pgfpathlineto{\pgfqpoint{0.944168in}{2.008458in}}%
\pgfpathlineto{\pgfqpoint{0.946092in}{2.007402in}}%
\pgfpathlineto{\pgfqpoint{0.948015in}{2.018954in}}%
\pgfpathlineto{\pgfqpoint{0.951862in}{2.018685in}}%
\pgfpathlineto{\pgfqpoint{0.953785in}{2.016687in}}%
\pgfpathlineto{\pgfqpoint{0.955709in}{2.023447in}}%
\pgfpathlineto{\pgfqpoint{0.959556in}{2.020467in}}%
\pgfpathlineto{\pgfqpoint{0.963402in}{1.992712in}}%
\pgfpathlineto{\pgfqpoint{0.965326in}{1.989982in}}%
\pgfpathlineto{\pgfqpoint{0.967249in}{1.997071in}}%
\pgfpathlineto{\pgfqpoint{0.969173in}{1.987928in}}%
\pgfpathlineto{\pgfqpoint{0.971096in}{1.989354in}}%
\pgfpathlineto{\pgfqpoint{0.973020in}{1.994112in}}%
\pgfpathlineto{\pgfqpoint{0.978790in}{2.024791in}}%
\pgfpathlineto{\pgfqpoint{0.980713in}{2.030906in}}%
\pgfpathlineto{\pgfqpoint{0.982637in}{2.027909in}}%
\pgfpathlineto{\pgfqpoint{0.988407in}{2.041660in}}%
\pgfpathlineto{\pgfqpoint{0.990330in}{2.050345in}}%
\pgfpathlineto{\pgfqpoint{0.992254in}{2.048981in}}%
\pgfpathlineto{\pgfqpoint{0.994177in}{2.057230in}}%
\pgfpathlineto{\pgfqpoint{0.996100in}{2.059931in}}%
\pgfpathlineto{\pgfqpoint{0.998024in}{2.060644in}}%
\pgfpathlineto{\pgfqpoint{0.999947in}{2.058538in}}%
\pgfpathlineto{\pgfqpoint{1.001871in}{2.064415in}}%
\pgfpathlineto{\pgfqpoint{1.003794in}{2.058053in}}%
\pgfpathlineto{\pgfqpoint{1.005717in}{2.058268in}}%
\pgfpathlineto{\pgfqpoint{1.007641in}{2.053011in}}%
\pgfpathlineto{\pgfqpoint{1.009564in}{2.054073in}}%
\pgfpathlineto{\pgfqpoint{1.011488in}{2.041325in}}%
\pgfpathlineto{\pgfqpoint{1.013411in}{2.041863in}}%
\pgfpathlineto{\pgfqpoint{1.015335in}{2.038047in}}%
\pgfpathlineto{\pgfqpoint{1.017258in}{2.028855in}}%
\pgfpathlineto{\pgfqpoint{1.019181in}{2.040139in}}%
\pgfpathlineto{\pgfqpoint{1.023028in}{2.045144in}}%
\pgfpathlineto{\pgfqpoint{1.024952in}{2.057493in}}%
\pgfpathlineto{\pgfqpoint{1.026875in}{2.058042in}}%
\pgfpathlineto{\pgfqpoint{1.028798in}{2.050920in}}%
\pgfpathlineto{\pgfqpoint{1.034569in}{2.054164in}}%
\pgfpathlineto{\pgfqpoint{1.038415in}{2.044004in}}%
\pgfpathlineto{\pgfqpoint{1.040339in}{2.048062in}}%
\pgfpathlineto{\pgfqpoint{1.042262in}{2.057844in}}%
\pgfpathlineto{\pgfqpoint{1.044186in}{2.057816in}}%
\pgfpathlineto{\pgfqpoint{1.046109in}{2.054657in}}%
\pgfpathlineto{\pgfqpoint{1.048033in}{2.055181in}}%
\pgfpathlineto{\pgfqpoint{1.049956in}{2.057872in}}%
\pgfpathlineto{\pgfqpoint{1.051879in}{2.055354in}}%
\pgfpathlineto{\pgfqpoint{1.053803in}{2.050667in}}%
\pgfpathlineto{\pgfqpoint{1.055726in}{2.049580in}}%
\pgfpathlineto{\pgfqpoint{1.057650in}{2.046496in}}%
\pgfpathlineto{\pgfqpoint{1.061496in}{2.054493in}}%
\pgfpathlineto{\pgfqpoint{1.063420in}{2.064310in}}%
\pgfpathlineto{\pgfqpoint{1.065343in}{2.056800in}}%
\pgfpathlineto{\pgfqpoint{1.067267in}{2.056681in}}%
\pgfpathlineto{\pgfqpoint{1.069190in}{2.059692in}}%
\pgfpathlineto{\pgfqpoint{1.073037in}{2.069577in}}%
\pgfpathlineto{\pgfqpoint{1.076884in}{2.093018in}}%
\pgfpathlineto{\pgfqpoint{1.080731in}{2.079502in}}%
\pgfpathlineto{\pgfqpoint{1.082654in}{2.078449in}}%
\pgfpathlineto{\pgfqpoint{1.084577in}{2.085666in}}%
\pgfpathlineto{\pgfqpoint{1.086501in}{2.085382in}}%
\pgfpathlineto{\pgfqpoint{1.088424in}{2.099215in}}%
\pgfpathlineto{\pgfqpoint{1.090348in}{2.087858in}}%
\pgfpathlineto{\pgfqpoint{1.092271in}{2.089270in}}%
\pgfpathlineto{\pgfqpoint{1.094194in}{2.098136in}}%
\pgfpathlineto{\pgfqpoint{1.099965in}{2.072611in}}%
\pgfpathlineto{\pgfqpoint{1.101888in}{2.071733in}}%
\pgfpathlineto{\pgfqpoint{1.105735in}{2.064586in}}%
\pgfpathlineto{\pgfqpoint{1.107658in}{2.058822in}}%
\pgfpathlineto{\pgfqpoint{1.109582in}{2.074576in}}%
\pgfpathlineto{\pgfqpoint{1.111505in}{2.075814in}}%
\pgfpathlineto{\pgfqpoint{1.113429in}{2.068098in}}%
\pgfpathlineto{\pgfqpoint{1.115352in}{2.066087in}}%
\pgfpathlineto{\pgfqpoint{1.117275in}{2.070063in}}%
\pgfpathlineto{\pgfqpoint{1.119199in}{2.069592in}}%
\pgfpathlineto{\pgfqpoint{1.123046in}{2.079015in}}%
\pgfpathlineto{\pgfqpoint{1.124969in}{2.070796in}}%
\pgfpathlineto{\pgfqpoint{1.126892in}{2.070925in}}%
\pgfpathlineto{\pgfqpoint{1.130739in}{2.089646in}}%
\pgfpathlineto{\pgfqpoint{1.132663in}{2.079947in}}%
\pgfpathlineto{\pgfqpoint{1.134586in}{2.080355in}}%
\pgfpathlineto{\pgfqpoint{1.136509in}{2.088923in}}%
\pgfpathlineto{\pgfqpoint{1.138433in}{2.090709in}}%
\pgfpathlineto{\pgfqpoint{1.144203in}{2.118932in}}%
\pgfpathlineto{\pgfqpoint{1.146126in}{2.122536in}}%
\pgfpathlineto{\pgfqpoint{1.148050in}{2.132673in}}%
\pgfpathlineto{\pgfqpoint{1.149973in}{2.133610in}}%
\pgfpathlineto{\pgfqpoint{1.151897in}{2.143206in}}%
\pgfpathlineto{\pgfqpoint{1.153820in}{2.137603in}}%
\pgfpathlineto{\pgfqpoint{1.155744in}{2.140595in}}%
\pgfpathlineto{\pgfqpoint{1.159590in}{2.123859in}}%
\pgfpathlineto{\pgfqpoint{1.163437in}{2.123428in}}%
\pgfpathlineto{\pgfqpoint{1.165361in}{2.115028in}}%
\pgfpathlineto{\pgfqpoint{1.167284in}{2.125651in}}%
\pgfpathlineto{\pgfqpoint{1.171131in}{2.107719in}}%
\pgfpathlineto{\pgfqpoint{1.173054in}{2.094290in}}%
\pgfpathlineto{\pgfqpoint{1.174978in}{2.101996in}}%
\pgfpathlineto{\pgfqpoint{1.176901in}{2.091084in}}%
\pgfpathlineto{\pgfqpoint{1.178824in}{2.097491in}}%
\pgfpathlineto{\pgfqpoint{1.180748in}{2.094708in}}%
\pgfpathlineto{\pgfqpoint{1.182671in}{2.100425in}}%
\pgfpathlineto{\pgfqpoint{1.184595in}{2.101095in}}%
\pgfpathlineto{\pgfqpoint{1.186518in}{2.109769in}}%
\pgfpathlineto{\pgfqpoint{1.188442in}{2.112362in}}%
\pgfpathlineto{\pgfqpoint{1.190365in}{2.105751in}}%
\pgfpathlineto{\pgfqpoint{1.192288in}{2.110694in}}%
\pgfpathlineto{\pgfqpoint{1.196135in}{2.102453in}}%
\pgfpathlineto{\pgfqpoint{1.198059in}{2.095211in}}%
\pgfpathlineto{\pgfqpoint{1.201905in}{2.073102in}}%
\pgfpathlineto{\pgfqpoint{1.207676in}{2.122645in}}%
\pgfpathlineto{\pgfqpoint{1.209599in}{2.132510in}}%
\pgfpathlineto{\pgfqpoint{1.211522in}{2.131794in}}%
\pgfpathlineto{\pgfqpoint{1.213446in}{2.149445in}}%
\pgfpathlineto{\pgfqpoint{1.215369in}{2.143447in}}%
\pgfpathlineto{\pgfqpoint{1.217293in}{2.155314in}}%
\pgfpathlineto{\pgfqpoint{1.219216in}{2.160001in}}%
\pgfpathlineto{\pgfqpoint{1.221140in}{2.169398in}}%
\pgfpathlineto{\pgfqpoint{1.223063in}{2.166160in}}%
\pgfpathlineto{\pgfqpoint{1.224986in}{2.175231in}}%
\pgfpathlineto{\pgfqpoint{1.226910in}{2.177399in}}%
\pgfpathlineto{\pgfqpoint{1.232680in}{2.223601in}}%
\pgfpathlineto{\pgfqpoint{1.234603in}{2.232266in}}%
\pgfpathlineto{\pgfqpoint{1.236527in}{2.232196in}}%
\pgfpathlineto{\pgfqpoint{1.238450in}{2.234247in}}%
\pgfpathlineto{\pgfqpoint{1.240374in}{2.227832in}}%
\pgfpathlineto{\pgfqpoint{1.242297in}{2.238488in}}%
\pgfpathlineto{\pgfqpoint{1.244220in}{2.233972in}}%
\pgfpathlineto{\pgfqpoint{1.246144in}{2.226532in}}%
\pgfpathlineto{\pgfqpoint{1.248067in}{2.228557in}}%
\pgfpathlineto{\pgfqpoint{1.249991in}{2.241564in}}%
\pgfpathlineto{\pgfqpoint{1.255761in}{2.211935in}}%
\pgfpathlineto{\pgfqpoint{1.257684in}{2.214147in}}%
\pgfpathlineto{\pgfqpoint{1.259608in}{2.212259in}}%
\pgfpathlineto{\pgfqpoint{1.261531in}{2.223618in}}%
\pgfpathlineto{\pgfqpoint{1.265378in}{2.203427in}}%
\pgfpathlineto{\pgfqpoint{1.269225in}{2.208618in}}%
\pgfpathlineto{\pgfqpoint{1.271148in}{2.223247in}}%
\pgfpathlineto{\pgfqpoint{1.274995in}{2.229987in}}%
\pgfpathlineto{\pgfqpoint{1.276918in}{2.228905in}}%
\pgfpathlineto{\pgfqpoint{1.278842in}{2.225649in}}%
\pgfpathlineto{\pgfqpoint{1.280765in}{2.233980in}}%
\pgfpathlineto{\pgfqpoint{1.284612in}{2.256648in}}%
\pgfpathlineto{\pgfqpoint{1.286536in}{2.259033in}}%
\pgfpathlineto{\pgfqpoint{1.288459in}{2.257391in}}%
\pgfpathlineto{\pgfqpoint{1.292306in}{2.246633in}}%
\pgfpathlineto{\pgfqpoint{1.294229in}{2.249252in}}%
\pgfpathlineto{\pgfqpoint{1.296153in}{2.244164in}}%
\pgfpathlineto{\pgfqpoint{1.298076in}{2.255878in}}%
\pgfpathlineto{\pgfqpoint{1.299999in}{2.257461in}}%
\pgfpathlineto{\pgfqpoint{1.303846in}{2.251970in}}%
\pgfpathlineto{\pgfqpoint{1.307693in}{2.265515in}}%
\pgfpathlineto{\pgfqpoint{1.309616in}{2.264081in}}%
\pgfpathlineto{\pgfqpoint{1.311540in}{2.257192in}}%
\pgfpathlineto{\pgfqpoint{1.313463in}{2.245413in}}%
\pgfpathlineto{\pgfqpoint{1.315387in}{2.255624in}}%
\pgfpathlineto{\pgfqpoint{1.317310in}{2.249604in}}%
\pgfpathlineto{\pgfqpoint{1.319233in}{2.255449in}}%
\pgfpathlineto{\pgfqpoint{1.321157in}{2.250566in}}%
\pgfpathlineto{\pgfqpoint{1.323080in}{2.267411in}}%
\pgfpathlineto{\pgfqpoint{1.325004in}{2.260087in}}%
\pgfpathlineto{\pgfqpoint{1.328851in}{2.259061in}}%
\pgfpathlineto{\pgfqpoint{1.332697in}{2.244713in}}%
\pgfpathlineto{\pgfqpoint{1.334621in}{2.251321in}}%
\pgfpathlineto{\pgfqpoint{1.336544in}{2.249946in}}%
\pgfpathlineto{\pgfqpoint{1.338468in}{2.246823in}}%
\pgfpathlineto{\pgfqpoint{1.340391in}{2.251774in}}%
\pgfpathlineto{\pgfqpoint{1.342314in}{2.249914in}}%
\pgfpathlineto{\pgfqpoint{1.344238in}{2.255203in}}%
\pgfpathlineto{\pgfqpoint{1.346161in}{2.252488in}}%
\pgfpathlineto{\pgfqpoint{1.348085in}{2.247106in}}%
\pgfpathlineto{\pgfqpoint{1.351931in}{2.234317in}}%
\pgfpathlineto{\pgfqpoint{1.353855in}{2.226605in}}%
\pgfpathlineto{\pgfqpoint{1.355778in}{2.238407in}}%
\pgfpathlineto{\pgfqpoint{1.357702in}{2.243285in}}%
\pgfpathlineto{\pgfqpoint{1.361549in}{2.210680in}}%
\pgfpathlineto{\pgfqpoint{1.363472in}{2.204570in}}%
\pgfpathlineto{\pgfqpoint{1.365395in}{2.204752in}}%
\pgfpathlineto{\pgfqpoint{1.367319in}{2.214099in}}%
\pgfpathlineto{\pgfqpoint{1.371166in}{2.216724in}}%
\pgfpathlineto{\pgfqpoint{1.373089in}{2.209221in}}%
\pgfpathlineto{\pgfqpoint{1.375012in}{2.210807in}}%
\pgfpathlineto{\pgfqpoint{1.376936in}{2.216269in}}%
\pgfpathlineto{\pgfqpoint{1.378859in}{2.216143in}}%
\pgfpathlineto{\pgfqpoint{1.380783in}{2.213155in}}%
\pgfpathlineto{\pgfqpoint{1.384629in}{2.225735in}}%
\pgfpathlineto{\pgfqpoint{1.386553in}{2.224579in}}%
\pgfpathlineto{\pgfqpoint{1.388476in}{2.229782in}}%
\pgfpathlineto{\pgfqpoint{1.390400in}{2.242835in}}%
\pgfpathlineto{\pgfqpoint{1.392323in}{2.246241in}}%
\pgfpathlineto{\pgfqpoint{1.394247in}{2.240729in}}%
\pgfpathlineto{\pgfqpoint{1.396170in}{2.245612in}}%
\pgfpathlineto{\pgfqpoint{1.398093in}{2.261945in}}%
\pgfpathlineto{\pgfqpoint{1.401940in}{2.244807in}}%
\pgfpathlineto{\pgfqpoint{1.403864in}{2.250715in}}%
\pgfpathlineto{\pgfqpoint{1.405787in}{2.262081in}}%
\pgfpathlineto{\pgfqpoint{1.407710in}{2.260729in}}%
\pgfpathlineto{\pgfqpoint{1.409634in}{2.251558in}}%
\pgfpathlineto{\pgfqpoint{1.411557in}{2.258991in}}%
\pgfpathlineto{\pgfqpoint{1.413481in}{2.253972in}}%
\pgfpathlineto{\pgfqpoint{1.419251in}{2.270332in}}%
\pgfpathlineto{\pgfqpoint{1.421174in}{2.270013in}}%
\pgfpathlineto{\pgfqpoint{1.425021in}{2.250594in}}%
\pgfpathlineto{\pgfqpoint{1.426945in}{2.249825in}}%
\pgfpathlineto{\pgfqpoint{1.428868in}{2.239970in}}%
\pgfpathlineto{\pgfqpoint{1.430791in}{2.241205in}}%
\pgfpathlineto{\pgfqpoint{1.432715in}{2.244545in}}%
\pgfpathlineto{\pgfqpoint{1.434638in}{2.240166in}}%
\pgfpathlineto{\pgfqpoint{1.438485in}{2.222258in}}%
\pgfpathlineto{\pgfqpoint{1.440408in}{2.231116in}}%
\pgfpathlineto{\pgfqpoint{1.442332in}{2.233135in}}%
\pgfpathlineto{\pgfqpoint{1.444255in}{2.229684in}}%
\pgfpathlineto{\pgfqpoint{1.446179in}{2.240567in}}%
\pgfpathlineto{\pgfqpoint{1.448102in}{2.224399in}}%
\pgfpathlineto{\pgfqpoint{1.450025in}{2.219566in}}%
\pgfpathlineto{\pgfqpoint{1.451949in}{2.218362in}}%
\pgfpathlineto{\pgfqpoint{1.453872in}{2.211274in}}%
\pgfpathlineto{\pgfqpoint{1.459642in}{2.243230in}}%
\pgfpathlineto{\pgfqpoint{1.461566in}{2.233830in}}%
\pgfpathlineto{\pgfqpoint{1.467336in}{2.244493in}}%
\pgfpathlineto{\pgfqpoint{1.471183in}{2.265883in}}%
\pgfpathlineto{\pgfqpoint{1.475030in}{2.271497in}}%
\pgfpathlineto{\pgfqpoint{1.476953in}{2.274591in}}%
\pgfpathlineto{\pgfqpoint{1.478877in}{2.257694in}}%
\pgfpathlineto{\pgfqpoint{1.482723in}{2.272467in}}%
\pgfpathlineto{\pgfqpoint{1.484647in}{2.267344in}}%
\pgfpathlineto{\pgfqpoint{1.486570in}{2.254950in}}%
\pgfpathlineto{\pgfqpoint{1.488494in}{2.252179in}}%
\pgfpathlineto{\pgfqpoint{1.490417in}{2.257221in}}%
\pgfpathlineto{\pgfqpoint{1.492340in}{2.269360in}}%
\pgfpathlineto{\pgfqpoint{1.494264in}{2.274101in}}%
\pgfpathlineto{\pgfqpoint{1.496187in}{2.286317in}}%
\pgfpathlineto{\pgfqpoint{1.498111in}{2.282507in}}%
\pgfpathlineto{\pgfqpoint{1.503881in}{2.295035in}}%
\pgfpathlineto{\pgfqpoint{1.505804in}{2.279943in}}%
\pgfpathlineto{\pgfqpoint{1.509651in}{2.272618in}}%
\pgfpathlineto{\pgfqpoint{1.513498in}{2.254224in}}%
\pgfpathlineto{\pgfqpoint{1.515421in}{2.263146in}}%
\pgfpathlineto{\pgfqpoint{1.517345in}{2.277988in}}%
\pgfpathlineto{\pgfqpoint{1.519268in}{2.283447in}}%
\pgfpathlineto{\pgfqpoint{1.521192in}{2.273598in}}%
\pgfpathlineto{\pgfqpoint{1.523115in}{2.273274in}}%
\pgfpathlineto{\pgfqpoint{1.526962in}{2.268378in}}%
\pgfpathlineto{\pgfqpoint{1.528885in}{2.275732in}}%
\pgfpathlineto{\pgfqpoint{1.530809in}{2.271597in}}%
\pgfpathlineto{\pgfqpoint{1.532732in}{2.277447in}}%
\pgfpathlineto{\pgfqpoint{1.536579in}{2.270547in}}%
\pgfpathlineto{\pgfqpoint{1.540426in}{2.283522in}}%
\pgfpathlineto{\pgfqpoint{1.542349in}{2.276555in}}%
\pgfpathlineto{\pgfqpoint{1.544273in}{2.279193in}}%
\pgfpathlineto{\pgfqpoint{1.546196in}{2.274351in}}%
\pgfpathlineto{\pgfqpoint{1.548119in}{2.286417in}}%
\pgfpathlineto{\pgfqpoint{1.550043in}{2.289104in}}%
\pgfpathlineto{\pgfqpoint{1.553890in}{2.275573in}}%
\pgfpathlineto{\pgfqpoint{1.557736in}{2.258303in}}%
\pgfpathlineto{\pgfqpoint{1.559660in}{2.262001in}}%
\pgfpathlineto{\pgfqpoint{1.561583in}{2.251220in}}%
\pgfpathlineto{\pgfqpoint{1.563507in}{2.249758in}}%
\pgfpathlineto{\pgfqpoint{1.565430in}{2.250787in}}%
\pgfpathlineto{\pgfqpoint{1.569277in}{2.244805in}}%
\pgfpathlineto{\pgfqpoint{1.571200in}{2.245960in}}%
\pgfpathlineto{\pgfqpoint{1.573124in}{2.242444in}}%
\pgfpathlineto{\pgfqpoint{1.575047in}{2.247198in}}%
\pgfpathlineto{\pgfqpoint{1.578894in}{2.238974in}}%
\pgfpathlineto{\pgfqpoint{1.580817in}{2.238569in}}%
\pgfpathlineto{\pgfqpoint{1.582741in}{2.248970in}}%
\pgfpathlineto{\pgfqpoint{1.584664in}{2.240298in}}%
\pgfpathlineto{\pgfqpoint{1.592358in}{2.258487in}}%
\pgfpathlineto{\pgfqpoint{1.594281in}{2.255086in}}%
\pgfpathlineto{\pgfqpoint{1.596205in}{2.255311in}}%
\pgfpathlineto{\pgfqpoint{1.598128in}{2.266385in}}%
\pgfpathlineto{\pgfqpoint{1.600051in}{2.263763in}}%
\pgfpathlineto{\pgfqpoint{1.603898in}{2.262664in}}%
\pgfpathlineto{\pgfqpoint{1.605822in}{2.250060in}}%
\pgfpathlineto{\pgfqpoint{1.607745in}{2.252395in}}%
\pgfpathlineto{\pgfqpoint{1.613515in}{2.269826in}}%
\pgfpathlineto{\pgfqpoint{1.617362in}{2.264804in}}%
\pgfpathlineto{\pgfqpoint{1.621209in}{2.270592in}}%
\pgfpathlineto{\pgfqpoint{1.625056in}{2.291491in}}%
\pgfpathlineto{\pgfqpoint{1.626979in}{2.285625in}}%
\pgfpathlineto{\pgfqpoint{1.628903in}{2.283342in}}%
\pgfpathlineto{\pgfqpoint{1.630826in}{2.284535in}}%
\pgfpathlineto{\pgfqpoint{1.632749in}{2.287428in}}%
\pgfpathlineto{\pgfqpoint{1.634673in}{2.286339in}}%
\pgfpathlineto{\pgfqpoint{1.636596in}{2.294230in}}%
\pgfpathlineto{\pgfqpoint{1.638520in}{2.291387in}}%
\pgfpathlineto{\pgfqpoint{1.640443in}{2.305787in}}%
\pgfpathlineto{\pgfqpoint{1.642367in}{2.302633in}}%
\pgfpathlineto{\pgfqpoint{1.644290in}{2.290471in}}%
\pgfpathlineto{\pgfqpoint{1.648137in}{2.326053in}}%
\pgfpathlineto{\pgfqpoint{1.650060in}{2.321880in}}%
\pgfpathlineto{\pgfqpoint{1.653907in}{2.338697in}}%
\pgfpathlineto{\pgfqpoint{1.655830in}{2.338105in}}%
\pgfpathlineto{\pgfqpoint{1.657754in}{2.345003in}}%
\pgfpathlineto{\pgfqpoint{1.659677in}{2.334925in}}%
\pgfpathlineto{\pgfqpoint{1.661601in}{2.333133in}}%
\pgfpathlineto{\pgfqpoint{1.663524in}{2.334622in}}%
\pgfpathlineto{\pgfqpoint{1.665447in}{2.328578in}}%
\pgfpathlineto{\pgfqpoint{1.667371in}{2.314454in}}%
\pgfpathlineto{\pgfqpoint{1.669294in}{2.314664in}}%
\pgfpathlineto{\pgfqpoint{1.675065in}{2.346519in}}%
\pgfpathlineto{\pgfqpoint{1.676988in}{2.343177in}}%
\pgfpathlineto{\pgfqpoint{1.678911in}{2.337121in}}%
\pgfpathlineto{\pgfqpoint{1.680835in}{2.336411in}}%
\pgfpathlineto{\pgfqpoint{1.682758in}{2.328965in}}%
\pgfpathlineto{\pgfqpoint{1.684682in}{2.330786in}}%
\pgfpathlineto{\pgfqpoint{1.686605in}{2.327480in}}%
\pgfpathlineto{\pgfqpoint{1.688528in}{2.336662in}}%
\pgfpathlineto{\pgfqpoint{1.690452in}{2.332510in}}%
\pgfpathlineto{\pgfqpoint{1.694299in}{2.346288in}}%
\pgfpathlineto{\pgfqpoint{1.696222in}{2.352213in}}%
\pgfpathlineto{\pgfqpoint{1.698145in}{2.330368in}}%
\pgfpathlineto{\pgfqpoint{1.703916in}{2.317744in}}%
\pgfpathlineto{\pgfqpoint{1.705839in}{2.318249in}}%
\pgfpathlineto{\pgfqpoint{1.707763in}{2.312895in}}%
\pgfpathlineto{\pgfqpoint{1.709686in}{2.316987in}}%
\pgfpathlineto{\pgfqpoint{1.711609in}{2.308761in}}%
\pgfpathlineto{\pgfqpoint{1.715456in}{2.287151in}}%
\pgfpathlineto{\pgfqpoint{1.717380in}{2.301400in}}%
\pgfpathlineto{\pgfqpoint{1.719303in}{2.302445in}}%
\pgfpathlineto{\pgfqpoint{1.721226in}{2.299274in}}%
\pgfpathlineto{\pgfqpoint{1.723150in}{2.301980in}}%
\pgfpathlineto{\pgfqpoint{1.725073in}{2.298901in}}%
\pgfpathlineto{\pgfqpoint{1.726997in}{2.310118in}}%
\pgfpathlineto{\pgfqpoint{1.728920in}{2.305766in}}%
\pgfpathlineto{\pgfqpoint{1.730843in}{2.316232in}}%
\pgfpathlineto{\pgfqpoint{1.732767in}{2.319086in}}%
\pgfpathlineto{\pgfqpoint{1.734690in}{2.315196in}}%
\pgfpathlineto{\pgfqpoint{1.738537in}{2.316199in}}%
\pgfpathlineto{\pgfqpoint{1.742384in}{2.330054in}}%
\pgfpathlineto{\pgfqpoint{1.746231in}{2.318461in}}%
\pgfpathlineto{\pgfqpoint{1.748154in}{2.316156in}}%
\pgfpathlineto{\pgfqpoint{1.752001in}{2.328394in}}%
\pgfpathlineto{\pgfqpoint{1.753924in}{2.325192in}}%
\pgfpathlineto{\pgfqpoint{1.755848in}{2.330036in}}%
\pgfpathlineto{\pgfqpoint{1.757771in}{2.329377in}}%
\pgfpathlineto{\pgfqpoint{1.759695in}{2.325199in}}%
\pgfpathlineto{\pgfqpoint{1.763541in}{2.329615in}}%
\pgfpathlineto{\pgfqpoint{1.765465in}{2.331984in}}%
\pgfpathlineto{\pgfqpoint{1.769312in}{2.315921in}}%
\pgfpathlineto{\pgfqpoint{1.771235in}{2.330938in}}%
\pgfpathlineto{\pgfqpoint{1.775082in}{2.338773in}}%
\pgfpathlineto{\pgfqpoint{1.777005in}{2.350993in}}%
\pgfpathlineto{\pgfqpoint{1.778929in}{2.350493in}}%
\pgfpathlineto{\pgfqpoint{1.780852in}{2.347298in}}%
\pgfpathlineto{\pgfqpoint{1.782776in}{2.347181in}}%
\pgfpathlineto{\pgfqpoint{1.786622in}{2.356139in}}%
\pgfpathlineto{\pgfqpoint{1.788546in}{2.352362in}}%
\pgfpathlineto{\pgfqpoint{1.790469in}{2.361564in}}%
\pgfpathlineto{\pgfqpoint{1.794316in}{2.353572in}}%
\pgfpathlineto{\pgfqpoint{1.796239in}{2.355996in}}%
\pgfpathlineto{\pgfqpoint{1.798163in}{2.344538in}}%
\pgfpathlineto{\pgfqpoint{1.802010in}{2.347334in}}%
\pgfpathlineto{\pgfqpoint{1.803933in}{2.365764in}}%
\pgfpathlineto{\pgfqpoint{1.805856in}{2.365741in}}%
\pgfpathlineto{\pgfqpoint{1.807780in}{2.376188in}}%
\pgfpathlineto{\pgfqpoint{1.809703in}{2.380343in}}%
\pgfpathlineto{\pgfqpoint{1.815474in}{2.379229in}}%
\pgfpathlineto{\pgfqpoint{1.819320in}{2.389553in}}%
\pgfpathlineto{\pgfqpoint{1.821244in}{2.378213in}}%
\pgfpathlineto{\pgfqpoint{1.823167in}{2.382303in}}%
\pgfpathlineto{\pgfqpoint{1.827014in}{2.379532in}}%
\pgfpathlineto{\pgfqpoint{1.834708in}{2.414452in}}%
\pgfpathlineto{\pgfqpoint{1.836631in}{2.421147in}}%
\pgfpathlineto{\pgfqpoint{1.838554in}{2.418711in}}%
\pgfpathlineto{\pgfqpoint{1.842401in}{2.433302in}}%
\pgfpathlineto{\pgfqpoint{1.844325in}{2.445148in}}%
\pgfpathlineto{\pgfqpoint{1.848172in}{2.448550in}}%
\pgfpathlineto{\pgfqpoint{1.852018in}{2.441697in}}%
\pgfpathlineto{\pgfqpoint{1.853942in}{2.447815in}}%
\pgfpathlineto{\pgfqpoint{1.855865in}{2.460412in}}%
\pgfpathlineto{\pgfqpoint{1.857789in}{2.446960in}}%
\pgfpathlineto{\pgfqpoint{1.861635in}{2.467326in}}%
\pgfpathlineto{\pgfqpoint{1.863559in}{2.458195in}}%
\pgfpathlineto{\pgfqpoint{1.865482in}{2.457625in}}%
\pgfpathlineto{\pgfqpoint{1.867406in}{2.458889in}}%
\pgfpathlineto{\pgfqpoint{1.873176in}{2.438937in}}%
\pgfpathlineto{\pgfqpoint{1.875099in}{2.443261in}}%
\pgfpathlineto{\pgfqpoint{1.877023in}{2.439801in}}%
\pgfpathlineto{\pgfqpoint{1.878946in}{2.441555in}}%
\pgfpathlineto{\pgfqpoint{1.882793in}{2.442492in}}%
\pgfpathlineto{\pgfqpoint{1.884716in}{2.445368in}}%
\pgfpathlineto{\pgfqpoint{1.886640in}{2.437598in}}%
\pgfpathlineto{\pgfqpoint{1.888563in}{2.437312in}}%
\pgfpathlineto{\pgfqpoint{1.890487in}{2.441904in}}%
\pgfpathlineto{\pgfqpoint{1.894333in}{2.429893in}}%
\pgfpathlineto{\pgfqpoint{1.896257in}{2.441170in}}%
\pgfpathlineto{\pgfqpoint{1.898180in}{2.441190in}}%
\pgfpathlineto{\pgfqpoint{1.900104in}{2.427935in}}%
\pgfpathlineto{\pgfqpoint{1.902027in}{2.438755in}}%
\pgfpathlineto{\pgfqpoint{1.905874in}{2.442957in}}%
\pgfpathlineto{\pgfqpoint{1.907797in}{2.445616in}}%
\pgfpathlineto{\pgfqpoint{1.909721in}{2.441637in}}%
\pgfpathlineto{\pgfqpoint{1.913567in}{2.450916in}}%
\pgfpathlineto{\pgfqpoint{1.915491in}{2.451580in}}%
\pgfpathlineto{\pgfqpoint{1.919338in}{2.468834in}}%
\pgfpathlineto{\pgfqpoint{1.921261in}{2.469968in}}%
\pgfpathlineto{\pgfqpoint{1.923185in}{2.481845in}}%
\pgfpathlineto{\pgfqpoint{1.927031in}{2.479896in}}%
\pgfpathlineto{\pgfqpoint{1.928955in}{2.483951in}}%
\pgfpathlineto{\pgfqpoint{1.930878in}{2.473956in}}%
\pgfpathlineto{\pgfqpoint{1.932802in}{2.478376in}}%
\pgfpathlineto{\pgfqpoint{1.934725in}{2.478240in}}%
\pgfpathlineto{\pgfqpoint{1.936648in}{2.474366in}}%
\pgfpathlineto{\pgfqpoint{1.938572in}{2.481332in}}%
\pgfpathlineto{\pgfqpoint{1.940495in}{2.476954in}}%
\pgfpathlineto{\pgfqpoint{1.942419in}{2.478481in}}%
\pgfpathlineto{\pgfqpoint{1.944342in}{2.473470in}}%
\pgfpathlineto{\pgfqpoint{1.946265in}{2.482878in}}%
\pgfpathlineto{\pgfqpoint{1.948189in}{2.477493in}}%
\pgfpathlineto{\pgfqpoint{1.950112in}{2.475816in}}%
\pgfpathlineto{\pgfqpoint{1.952036in}{2.482160in}}%
\pgfpathlineto{\pgfqpoint{1.953959in}{2.472594in}}%
\pgfpathlineto{\pgfqpoint{1.957806in}{2.490296in}}%
\pgfpathlineto{\pgfqpoint{1.959729in}{2.495786in}}%
\pgfpathlineto{\pgfqpoint{1.961653in}{2.487029in}}%
\pgfpathlineto{\pgfqpoint{1.963576in}{2.490570in}}%
\pgfpathlineto{\pgfqpoint{1.967423in}{2.473898in}}%
\pgfpathlineto{\pgfqpoint{1.969346in}{2.484063in}}%
\pgfpathlineto{\pgfqpoint{1.971270in}{2.486752in}}%
\pgfpathlineto{\pgfqpoint{1.973193in}{2.497906in}}%
\pgfpathlineto{\pgfqpoint{1.975117in}{2.490591in}}%
\pgfpathlineto{\pgfqpoint{1.978963in}{2.496323in}}%
\pgfpathlineto{\pgfqpoint{1.984734in}{2.525904in}}%
\pgfpathlineto{\pgfqpoint{1.986657in}{2.522335in}}%
\pgfpathlineto{\pgfqpoint{1.988581in}{2.534636in}}%
\pgfpathlineto{\pgfqpoint{1.990504in}{2.536977in}}%
\pgfpathlineto{\pgfqpoint{1.992427in}{2.532232in}}%
\pgfpathlineto{\pgfqpoint{2.000121in}{2.538127in}}%
\pgfpathlineto{\pgfqpoint{2.003968in}{2.524381in}}%
\pgfpathlineto{\pgfqpoint{2.005891in}{2.529013in}}%
\pgfpathlineto{\pgfqpoint{2.007815in}{2.522197in}}%
\pgfpathlineto{\pgfqpoint{2.009738in}{2.532559in}}%
\pgfpathlineto{\pgfqpoint{2.011661in}{2.529664in}}%
\pgfpathlineto{\pgfqpoint{2.013585in}{2.533432in}}%
\pgfpathlineto{\pgfqpoint{2.015508in}{2.524972in}}%
\pgfpathlineto{\pgfqpoint{2.019355in}{2.535567in}}%
\pgfpathlineto{\pgfqpoint{2.021279in}{2.550713in}}%
\pgfpathlineto{\pgfqpoint{2.023202in}{2.549038in}}%
\pgfpathlineto{\pgfqpoint{2.027049in}{2.548319in}}%
\pgfpathlineto{\pgfqpoint{2.028972in}{2.545762in}}%
\pgfpathlineto{\pgfqpoint{2.032819in}{2.547555in}}%
\pgfpathlineto{\pgfqpoint{2.034742in}{2.546856in}}%
\pgfpathlineto{\pgfqpoint{2.036666in}{2.540860in}}%
\pgfpathlineto{\pgfqpoint{2.038589in}{2.549408in}}%
\pgfpathlineto{\pgfqpoint{2.040513in}{2.546627in}}%
\pgfpathlineto{\pgfqpoint{2.042436in}{2.539021in}}%
\pgfpathlineto{\pgfqpoint{2.044359in}{2.548677in}}%
\pgfpathlineto{\pgfqpoint{2.048206in}{2.541851in}}%
\pgfpathlineto{\pgfqpoint{2.050130in}{2.540155in}}%
\pgfpathlineto{\pgfqpoint{2.053976in}{2.522985in}}%
\pgfpathlineto{\pgfqpoint{2.055900in}{2.528323in}}%
\pgfpathlineto{\pgfqpoint{2.057823in}{2.521665in}}%
\pgfpathlineto{\pgfqpoint{2.059747in}{2.525915in}}%
\pgfpathlineto{\pgfqpoint{2.061670in}{2.525834in}}%
\pgfpathlineto{\pgfqpoint{2.065517in}{2.542408in}}%
\pgfpathlineto{\pgfqpoint{2.067440in}{2.540648in}}%
\pgfpathlineto{\pgfqpoint{2.069364in}{2.528598in}}%
\pgfpathlineto{\pgfqpoint{2.071287in}{2.531779in}}%
\pgfpathlineto{\pgfqpoint{2.073211in}{2.530295in}}%
\pgfpathlineto{\pgfqpoint{2.077057in}{2.511472in}}%
\pgfpathlineto{\pgfqpoint{2.082828in}{2.526501in}}%
\pgfpathlineto{\pgfqpoint{2.084751in}{2.532668in}}%
\pgfpathlineto{\pgfqpoint{2.088598in}{2.511247in}}%
\pgfpathlineto{\pgfqpoint{2.090521in}{2.500322in}}%
\pgfpathlineto{\pgfqpoint{2.094368in}{2.505444in}}%
\pgfpathlineto{\pgfqpoint{2.098215in}{2.517045in}}%
\pgfpathlineto{\pgfqpoint{2.100138in}{2.520591in}}%
\pgfpathlineto{\pgfqpoint{2.102062in}{2.533183in}}%
\pgfpathlineto{\pgfqpoint{2.103985in}{2.533386in}}%
\pgfpathlineto{\pgfqpoint{2.105909in}{2.532111in}}%
\pgfpathlineto{\pgfqpoint{2.107832in}{2.524826in}}%
\pgfpathlineto{\pgfqpoint{2.109755in}{2.541127in}}%
\pgfpathlineto{\pgfqpoint{2.111679in}{2.541722in}}%
\pgfpathlineto{\pgfqpoint{2.117449in}{2.554100in}}%
\pgfpathlineto{\pgfqpoint{2.121296in}{2.558716in}}%
\pgfpathlineto{\pgfqpoint{2.123219in}{2.568769in}}%
\pgfpathlineto{\pgfqpoint{2.127066in}{2.574852in}}%
\pgfpathlineto{\pgfqpoint{2.128990in}{2.574289in}}%
\pgfpathlineto{\pgfqpoint{2.132836in}{2.564786in}}%
\pgfpathlineto{\pgfqpoint{2.134760in}{2.574753in}}%
\pgfpathlineto{\pgfqpoint{2.140530in}{2.560061in}}%
\pgfpathlineto{\pgfqpoint{2.146300in}{2.535338in}}%
\pgfpathlineto{\pgfqpoint{2.150147in}{2.538281in}}%
\pgfpathlineto{\pgfqpoint{2.152070in}{2.537602in}}%
\pgfpathlineto{\pgfqpoint{2.153994in}{2.523861in}}%
\pgfpathlineto{\pgfqpoint{2.157841in}{2.527863in}}%
\pgfpathlineto{\pgfqpoint{2.159764in}{2.537689in}}%
\pgfpathlineto{\pgfqpoint{2.161688in}{2.526679in}}%
\pgfpathlineto{\pgfqpoint{2.163611in}{2.526936in}}%
\pgfpathlineto{\pgfqpoint{2.165534in}{2.523599in}}%
\pgfpathlineto{\pgfqpoint{2.167458in}{2.526596in}}%
\pgfpathlineto{\pgfqpoint{2.169381in}{2.522345in}}%
\pgfpathlineto{\pgfqpoint{2.171305in}{2.521755in}}%
\pgfpathlineto{\pgfqpoint{2.177075in}{2.515756in}}%
\pgfpathlineto{\pgfqpoint{2.178998in}{2.516060in}}%
\pgfpathlineto{\pgfqpoint{2.180922in}{2.513944in}}%
\pgfpathlineto{\pgfqpoint{2.182845in}{2.499228in}}%
\pgfpathlineto{\pgfqpoint{2.186692in}{2.511368in}}%
\pgfpathlineto{\pgfqpoint{2.192462in}{2.482606in}}%
\pgfpathlineto{\pgfqpoint{2.196309in}{2.475403in}}%
\pgfpathlineto{\pgfqpoint{2.198232in}{2.478549in}}%
\pgfpathlineto{\pgfqpoint{2.202079in}{2.498390in}}%
\pgfpathlineto{\pgfqpoint{2.204003in}{2.496058in}}%
\pgfpathlineto{\pgfqpoint{2.211696in}{2.539474in}}%
\pgfpathlineto{\pgfqpoint{2.213620in}{2.540408in}}%
\pgfpathlineto{\pgfqpoint{2.217466in}{2.528515in}}%
\pgfpathlineto{\pgfqpoint{2.219390in}{2.536178in}}%
\pgfpathlineto{\pgfqpoint{2.223237in}{2.569914in}}%
\pgfpathlineto{\pgfqpoint{2.225160in}{2.569234in}}%
\pgfpathlineto{\pgfqpoint{2.227083in}{2.574681in}}%
\pgfpathlineto{\pgfqpoint{2.229007in}{2.584853in}}%
\pgfpathlineto{\pgfqpoint{2.230930in}{2.577407in}}%
\pgfpathlineto{\pgfqpoint{2.232854in}{2.587051in}}%
\pgfpathlineto{\pgfqpoint{2.234777in}{2.588887in}}%
\pgfpathlineto{\pgfqpoint{2.236701in}{2.587702in}}%
\pgfpathlineto{\pgfqpoint{2.238624in}{2.588234in}}%
\pgfpathlineto{\pgfqpoint{2.240547in}{2.587478in}}%
\pgfpathlineto{\pgfqpoint{2.242471in}{2.582378in}}%
\pgfpathlineto{\pgfqpoint{2.244394in}{2.581666in}}%
\pgfpathlineto{\pgfqpoint{2.246318in}{2.587666in}}%
\pgfpathlineto{\pgfqpoint{2.248241in}{2.576860in}}%
\pgfpathlineto{\pgfqpoint{2.250164in}{2.585188in}}%
\pgfpathlineto{\pgfqpoint{2.252088in}{2.586061in}}%
\pgfpathlineto{\pgfqpoint{2.255935in}{2.600625in}}%
\pgfpathlineto{\pgfqpoint{2.257858in}{2.602626in}}%
\pgfpathlineto{\pgfqpoint{2.259781in}{2.595456in}}%
\pgfpathlineto{\pgfqpoint{2.261705in}{2.581789in}}%
\pgfpathlineto{\pgfqpoint{2.263628in}{2.579270in}}%
\pgfpathlineto{\pgfqpoint{2.265552in}{2.588819in}}%
\pgfpathlineto{\pgfqpoint{2.269399in}{2.595963in}}%
\pgfpathlineto{\pgfqpoint{2.271322in}{2.586908in}}%
\pgfpathlineto{\pgfqpoint{2.273245in}{2.587583in}}%
\pgfpathlineto{\pgfqpoint{2.277092in}{2.582759in}}%
\pgfpathlineto{\pgfqpoint{2.279016in}{2.585995in}}%
\pgfpathlineto{\pgfqpoint{2.280939in}{2.581503in}}%
\pgfpathlineto{\pgfqpoint{2.282862in}{2.580536in}}%
\pgfpathlineto{\pgfqpoint{2.284786in}{2.567583in}}%
\pgfpathlineto{\pgfqpoint{2.286709in}{2.562758in}}%
\pgfpathlineto{\pgfqpoint{2.290556in}{2.544397in}}%
\pgfpathlineto{\pgfqpoint{2.292479in}{2.554947in}}%
\pgfpathlineto{\pgfqpoint{2.294403in}{2.553853in}}%
\pgfpathlineto{\pgfqpoint{2.296326in}{2.549866in}}%
\pgfpathlineto{\pgfqpoint{2.298250in}{2.550264in}}%
\pgfpathlineto{\pgfqpoint{2.300173in}{2.552606in}}%
\pgfpathlineto{\pgfqpoint{2.304020in}{2.532242in}}%
\pgfpathlineto{\pgfqpoint{2.305943in}{2.531007in}}%
\pgfpathlineto{\pgfqpoint{2.307867in}{2.539089in}}%
\pgfpathlineto{\pgfqpoint{2.313637in}{2.526299in}}%
\pgfpathlineto{\pgfqpoint{2.315560in}{2.512257in}}%
\pgfpathlineto{\pgfqpoint{2.317484in}{2.513878in}}%
\pgfpathlineto{\pgfqpoint{2.319407in}{2.518191in}}%
\pgfpathlineto{\pgfqpoint{2.321331in}{2.516856in}}%
\pgfpathlineto{\pgfqpoint{2.323254in}{2.517872in}}%
\pgfpathlineto{\pgfqpoint{2.325177in}{2.522147in}}%
\pgfpathlineto{\pgfqpoint{2.327101in}{2.521301in}}%
\pgfpathlineto{\pgfqpoint{2.332871in}{2.548571in}}%
\pgfpathlineto{\pgfqpoint{2.334795in}{2.543586in}}%
\pgfpathlineto{\pgfqpoint{2.336718in}{2.545153in}}%
\pgfpathlineto{\pgfqpoint{2.338641in}{2.549313in}}%
\pgfpathlineto{\pgfqpoint{2.340565in}{2.545885in}}%
\pgfpathlineto{\pgfqpoint{2.344412in}{2.555494in}}%
\pgfpathlineto{\pgfqpoint{2.346335in}{2.545843in}}%
\pgfpathlineto{\pgfqpoint{2.348258in}{2.549300in}}%
\pgfpathlineto{\pgfqpoint{2.350182in}{2.556265in}}%
\pgfpathlineto{\pgfqpoint{2.352105in}{2.552053in}}%
\pgfpathlineto{\pgfqpoint{2.354029in}{2.556666in}}%
\pgfpathlineto{\pgfqpoint{2.355952in}{2.551975in}}%
\pgfpathlineto{\pgfqpoint{2.357875in}{2.553644in}}%
\pgfpathlineto{\pgfqpoint{2.359799in}{2.553129in}}%
\pgfpathlineto{\pgfqpoint{2.361722in}{2.568023in}}%
\pgfpathlineto{\pgfqpoint{2.363646in}{2.567698in}}%
\pgfpathlineto{\pgfqpoint{2.365569in}{2.576054in}}%
\pgfpathlineto{\pgfqpoint{2.367492in}{2.578778in}}%
\pgfpathlineto{\pgfqpoint{2.369416in}{2.571932in}}%
\pgfpathlineto{\pgfqpoint{2.371339in}{2.572908in}}%
\pgfpathlineto{\pgfqpoint{2.373263in}{2.582673in}}%
\pgfpathlineto{\pgfqpoint{2.375186in}{2.575882in}}%
\pgfpathlineto{\pgfqpoint{2.377110in}{2.559722in}}%
\pgfpathlineto{\pgfqpoint{2.380956in}{2.546391in}}%
\pgfpathlineto{\pgfqpoint{2.382880in}{2.548536in}}%
\pgfpathlineto{\pgfqpoint{2.384803in}{2.555302in}}%
\pgfpathlineto{\pgfqpoint{2.386727in}{2.554103in}}%
\pgfpathlineto{\pgfqpoint{2.388650in}{2.555778in}}%
\pgfpathlineto{\pgfqpoint{2.390573in}{2.559001in}}%
\pgfpathlineto{\pgfqpoint{2.392497in}{2.551683in}}%
\pgfpathlineto{\pgfqpoint{2.394420in}{2.563342in}}%
\pgfpathlineto{\pgfqpoint{2.398267in}{2.544447in}}%
\pgfpathlineto{\pgfqpoint{2.400190in}{2.543593in}}%
\pgfpathlineto{\pgfqpoint{2.402114in}{2.548115in}}%
\pgfpathlineto{\pgfqpoint{2.404037in}{2.548819in}}%
\pgfpathlineto{\pgfqpoint{2.405961in}{2.557784in}}%
\pgfpathlineto{\pgfqpoint{2.407884in}{2.560829in}}%
\pgfpathlineto{\pgfqpoint{2.409808in}{2.554086in}}%
\pgfpathlineto{\pgfqpoint{2.411731in}{2.558843in}}%
\pgfpathlineto{\pgfqpoint{2.413654in}{2.556954in}}%
\pgfpathlineto{\pgfqpoint{2.417501in}{2.557395in}}%
\pgfpathlineto{\pgfqpoint{2.421348in}{2.563053in}}%
\pgfpathlineto{\pgfqpoint{2.423271in}{2.570088in}}%
\pgfpathlineto{\pgfqpoint{2.425195in}{2.566480in}}%
\pgfpathlineto{\pgfqpoint{2.427118in}{2.568939in}}%
\pgfpathlineto{\pgfqpoint{2.429042in}{2.567074in}}%
\pgfpathlineto{\pgfqpoint{2.430965in}{2.568361in}}%
\pgfpathlineto{\pgfqpoint{2.434812in}{2.561351in}}%
\pgfpathlineto{\pgfqpoint{2.436735in}{2.564051in}}%
\pgfpathlineto{\pgfqpoint{2.440582in}{2.581703in}}%
\pgfpathlineto{\pgfqpoint{2.442506in}{2.581741in}}%
\pgfpathlineto{\pgfqpoint{2.446352in}{2.589280in}}%
\pgfpathlineto{\pgfqpoint{2.448276in}{2.588580in}}%
\pgfpathlineto{\pgfqpoint{2.450199in}{2.593093in}}%
\pgfpathlineto{\pgfqpoint{2.452123in}{2.587759in}}%
\pgfpathlineto{\pgfqpoint{2.454046in}{2.589961in}}%
\pgfpathlineto{\pgfqpoint{2.455969in}{2.596914in}}%
\pgfpathlineto{\pgfqpoint{2.457893in}{2.596112in}}%
\pgfpathlineto{\pgfqpoint{2.459816in}{2.599579in}}%
\pgfpathlineto{\pgfqpoint{2.461740in}{2.609540in}}%
\pgfpathlineto{\pgfqpoint{2.463663in}{2.607192in}}%
\pgfpathlineto{\pgfqpoint{2.467510in}{2.625254in}}%
\pgfpathlineto{\pgfqpoint{2.469433in}{2.619767in}}%
\pgfpathlineto{\pgfqpoint{2.471357in}{2.618987in}}%
\pgfpathlineto{\pgfqpoint{2.473280in}{2.613065in}}%
\pgfpathlineto{\pgfqpoint{2.475204in}{2.614125in}}%
\pgfpathlineto{\pgfqpoint{2.479050in}{2.625379in}}%
\pgfpathlineto{\pgfqpoint{2.480974in}{2.610130in}}%
\pgfpathlineto{\pgfqpoint{2.484821in}{2.624634in}}%
\pgfpathlineto{\pgfqpoint{2.486744in}{2.628185in}}%
\pgfpathlineto{\pgfqpoint{2.488667in}{2.649448in}}%
\pgfpathlineto{\pgfqpoint{2.490591in}{2.654730in}}%
\pgfpathlineto{\pgfqpoint{2.492514in}{2.649522in}}%
\pgfpathlineto{\pgfqpoint{2.494438in}{2.651117in}}%
\pgfpathlineto{\pgfqpoint{2.496361in}{2.643263in}}%
\pgfpathlineto{\pgfqpoint{2.498284in}{2.643920in}}%
\pgfpathlineto{\pgfqpoint{2.500208in}{2.647517in}}%
\pgfpathlineto{\pgfqpoint{2.504055in}{2.664545in}}%
\pgfpathlineto{\pgfqpoint{2.507901in}{2.641878in}}%
\pgfpathlineto{\pgfqpoint{2.513672in}{2.617106in}}%
\pgfpathlineto{\pgfqpoint{2.515595in}{2.621348in}}%
\pgfpathlineto{\pgfqpoint{2.519442in}{2.632739in}}%
\pgfpathlineto{\pgfqpoint{2.521365in}{2.635229in}}%
\pgfpathlineto{\pgfqpoint{2.523289in}{2.640509in}}%
\pgfpathlineto{\pgfqpoint{2.525212in}{2.640129in}}%
\pgfpathlineto{\pgfqpoint{2.527136in}{2.646016in}}%
\pgfpathlineto{\pgfqpoint{2.529059in}{2.642208in}}%
\pgfpathlineto{\pgfqpoint{2.530982in}{2.628128in}}%
\pgfpathlineto{\pgfqpoint{2.532906in}{2.638114in}}%
\pgfpathlineto{\pgfqpoint{2.534829in}{2.632280in}}%
\pgfpathlineto{\pgfqpoint{2.536753in}{2.616312in}}%
\pgfpathlineto{\pgfqpoint{2.540599in}{2.618439in}}%
\pgfpathlineto{\pgfqpoint{2.542523in}{2.630437in}}%
\pgfpathlineto{\pgfqpoint{2.544446in}{2.631054in}}%
\pgfpathlineto{\pgfqpoint{2.548293in}{2.646469in}}%
\pgfpathlineto{\pgfqpoint{2.552140in}{2.636371in}}%
\pgfpathlineto{\pgfqpoint{2.554063in}{2.628009in}}%
\pgfpathlineto{\pgfqpoint{2.555987in}{2.626506in}}%
\pgfpathlineto{\pgfqpoint{2.557910in}{2.610785in}}%
\pgfpathlineto{\pgfqpoint{2.559834in}{2.617925in}}%
\pgfpathlineto{\pgfqpoint{2.561757in}{2.610955in}}%
\pgfpathlineto{\pgfqpoint{2.563680in}{2.622482in}}%
\pgfpathlineto{\pgfqpoint{2.565604in}{2.624133in}}%
\pgfpathlineto{\pgfqpoint{2.569451in}{2.605444in}}%
\pgfpathlineto{\pgfqpoint{2.571374in}{2.597381in}}%
\pgfpathlineto{\pgfqpoint{2.573297in}{2.598835in}}%
\pgfpathlineto{\pgfqpoint{2.575221in}{2.594128in}}%
\pgfpathlineto{\pgfqpoint{2.577144in}{2.594783in}}%
\pgfpathlineto{\pgfqpoint{2.579068in}{2.590477in}}%
\pgfpathlineto{\pgfqpoint{2.580991in}{2.594077in}}%
\pgfpathlineto{\pgfqpoint{2.582915in}{2.585551in}}%
\pgfpathlineto{\pgfqpoint{2.584838in}{2.583787in}}%
\pgfpathlineto{\pgfqpoint{2.586761in}{2.584485in}}%
\pgfpathlineto{\pgfqpoint{2.588685in}{2.581619in}}%
\pgfpathlineto{\pgfqpoint{2.590608in}{2.583139in}}%
\pgfpathlineto{\pgfqpoint{2.594455in}{2.581611in}}%
\pgfpathlineto{\pgfqpoint{2.596378in}{2.575753in}}%
\pgfpathlineto{\pgfqpoint{2.598302in}{2.581041in}}%
\pgfpathlineto{\pgfqpoint{2.602149in}{2.600550in}}%
\pgfpathlineto{\pgfqpoint{2.604072in}{2.601311in}}%
\pgfpathlineto{\pgfqpoint{2.605995in}{2.593195in}}%
\pgfpathlineto{\pgfqpoint{2.609842in}{2.603900in}}%
\pgfpathlineto{\pgfqpoint{2.611766in}{2.602322in}}%
\pgfpathlineto{\pgfqpoint{2.613689in}{2.607131in}}%
\pgfpathlineto{\pgfqpoint{2.617536in}{2.587947in}}%
\pgfpathlineto{\pgfqpoint{2.619459in}{2.591046in}}%
\pgfpathlineto{\pgfqpoint{2.621383in}{2.584116in}}%
\pgfpathlineto{\pgfqpoint{2.623306in}{2.582331in}}%
\pgfpathlineto{\pgfqpoint{2.625230in}{2.592234in}}%
\pgfpathlineto{\pgfqpoint{2.629076in}{2.589537in}}%
\pgfpathlineto{\pgfqpoint{2.632923in}{2.581083in}}%
\pgfpathlineto{\pgfqpoint{2.634847in}{2.586835in}}%
\pgfpathlineto{\pgfqpoint{2.636770in}{2.585171in}}%
\pgfpathlineto{\pgfqpoint{2.640617in}{2.610887in}}%
\pgfpathlineto{\pgfqpoint{2.642540in}{2.615297in}}%
\pgfpathlineto{\pgfqpoint{2.644464in}{2.612197in}}%
\pgfpathlineto{\pgfqpoint{2.646387in}{2.606007in}}%
\pgfpathlineto{\pgfqpoint{2.650234in}{2.616613in}}%
\pgfpathlineto{\pgfqpoint{2.652157in}{2.616062in}}%
\pgfpathlineto{\pgfqpoint{2.654081in}{2.627912in}}%
\pgfpathlineto{\pgfqpoint{2.656004in}{2.620766in}}%
\pgfpathlineto{\pgfqpoint{2.657928in}{2.621199in}}%
\pgfpathlineto{\pgfqpoint{2.659851in}{2.625996in}}%
\pgfpathlineto{\pgfqpoint{2.665621in}{2.617909in}}%
\pgfpathlineto{\pgfqpoint{2.669468in}{2.622957in}}%
\pgfpathlineto{\pgfqpoint{2.671391in}{2.613083in}}%
\pgfpathlineto{\pgfqpoint{2.673315in}{2.610931in}}%
\pgfpathlineto{\pgfqpoint{2.675238in}{2.620492in}}%
\pgfpathlineto{\pgfqpoint{2.677162in}{2.616638in}}%
\pgfpathlineto{\pgfqpoint{2.679085in}{2.618604in}}%
\pgfpathlineto{\pgfqpoint{2.682932in}{2.625927in}}%
\pgfpathlineto{\pgfqpoint{2.684855in}{2.609723in}}%
\pgfpathlineto{\pgfqpoint{2.686779in}{2.612499in}}%
\pgfpathlineto{\pgfqpoint{2.688702in}{2.624911in}}%
\pgfpathlineto{\pgfqpoint{2.690626in}{2.625921in}}%
\pgfpathlineto{\pgfqpoint{2.692549in}{2.615759in}}%
\pgfpathlineto{\pgfqpoint{2.694472in}{2.614917in}}%
\pgfpathlineto{\pgfqpoint{2.696396in}{2.617051in}}%
\pgfpathlineto{\pgfqpoint{2.698319in}{2.621397in}}%
\pgfpathlineto{\pgfqpoint{2.700243in}{2.607067in}}%
\pgfpathlineto{\pgfqpoint{2.704089in}{2.615724in}}%
\pgfpathlineto{\pgfqpoint{2.706013in}{2.604196in}}%
\pgfpathlineto{\pgfqpoint{2.709860in}{2.616992in}}%
\pgfpathlineto{\pgfqpoint{2.711783in}{2.618795in}}%
\pgfpathlineto{\pgfqpoint{2.715630in}{2.627998in}}%
\pgfpathlineto{\pgfqpoint{2.717553in}{2.617854in}}%
\pgfpathlineto{\pgfqpoint{2.719477in}{2.624306in}}%
\pgfpathlineto{\pgfqpoint{2.721400in}{2.636408in}}%
\pgfpathlineto{\pgfqpoint{2.723324in}{2.629886in}}%
\pgfpathlineto{\pgfqpoint{2.727170in}{2.633644in}}%
\pgfpathlineto{\pgfqpoint{2.731017in}{2.614764in}}%
\pgfpathlineto{\pgfqpoint{2.736787in}{2.626947in}}%
\pgfpathlineto{\pgfqpoint{2.738711in}{2.623698in}}%
\pgfpathlineto{\pgfqpoint{2.740634in}{2.624491in}}%
\pgfpathlineto{\pgfqpoint{2.742558in}{2.628979in}}%
\pgfpathlineto{\pgfqpoint{2.744481in}{2.623387in}}%
\pgfpathlineto{\pgfqpoint{2.748328in}{2.646981in}}%
\pgfpathlineto{\pgfqpoint{2.752175in}{2.632506in}}%
\pgfpathlineto{\pgfqpoint{2.754098in}{2.631676in}}%
\pgfpathlineto{\pgfqpoint{2.759868in}{2.618852in}}%
\pgfpathlineto{\pgfqpoint{2.761792in}{2.615902in}}%
\pgfpathlineto{\pgfqpoint{2.763715in}{2.620094in}}%
\pgfpathlineto{\pgfqpoint{2.765639in}{2.610932in}}%
\pgfpathlineto{\pgfqpoint{2.767562in}{2.609944in}}%
\pgfpathlineto{\pgfqpoint{2.767562in}{2.609944in}}%
\pgfusepath{stroke}%
\end{pgfscope}%
\begin{pgfscope}%
\pgfsetrectcap%
\pgfsetmiterjoin%
\pgfsetlinewidth{0.000000pt}%
\definecolor{currentstroke}{rgb}{1.000000,1.000000,1.000000}%
\pgfsetstrokecolor{currentstroke}%
\pgfsetdash{}{0pt}%
\pgfpathmoveto{\pgfqpoint{0.750000in}{0.660000in}}%
\pgfpathlineto{\pgfqpoint{0.750000in}{2.760000in}}%
\pgfusepath{}%
\end{pgfscope}%
\begin{pgfscope}%
\pgfsetrectcap%
\pgfsetmiterjoin%
\pgfsetlinewidth{0.000000pt}%
\definecolor{currentstroke}{rgb}{1.000000,1.000000,1.000000}%
\pgfsetstrokecolor{currentstroke}%
\pgfsetdash{}{0pt}%
\pgfpathmoveto{\pgfqpoint{2.863636in}{0.660000in}}%
\pgfpathlineto{\pgfqpoint{2.863636in}{2.760000in}}%
\pgfusepath{}%
\end{pgfscope}%
\begin{pgfscope}%
\pgfsetrectcap%
\pgfsetmiterjoin%
\pgfsetlinewidth{0.000000pt}%
\definecolor{currentstroke}{rgb}{1.000000,1.000000,1.000000}%
\pgfsetstrokecolor{currentstroke}%
\pgfsetdash{}{0pt}%
\pgfpathmoveto{\pgfqpoint{0.750000in}{0.660000in}}%
\pgfpathlineto{\pgfqpoint{2.863636in}{0.660000in}}%
\pgfusepath{}%
\end{pgfscope}%
\begin{pgfscope}%
\pgfsetrectcap%
\pgfsetmiterjoin%
\pgfsetlinewidth{0.000000pt}%
\definecolor{currentstroke}{rgb}{1.000000,1.000000,1.000000}%
\pgfsetstrokecolor{currentstroke}%
\pgfsetdash{}{0pt}%
\pgfpathmoveto{\pgfqpoint{0.750000in}{2.760000in}}%
\pgfpathlineto{\pgfqpoint{2.863636in}{2.760000in}}%
\pgfusepath{}%
\end{pgfscope}%
\begin{pgfscope}%
\pgfsetbuttcap%
\pgfsetmiterjoin%
\definecolor{currentfill}{rgb}{0.917647,0.917647,0.949020}%
\pgfsetfillcolor{currentfill}%
\pgfsetlinewidth{0.000000pt}%
\definecolor{currentstroke}{rgb}{0.000000,0.000000,0.000000}%
\pgfsetstrokecolor{currentstroke}%
\pgfsetstrokeopacity{0.000000}%
\pgfsetdash{}{0pt}%
\pgfpathmoveto{\pgfqpoint{3.286364in}{0.660000in}}%
\pgfpathlineto{\pgfqpoint{5.400000in}{0.660000in}}%
\pgfpathlineto{\pgfqpoint{5.400000in}{2.760000in}}%
\pgfpathlineto{\pgfqpoint{3.286364in}{2.760000in}}%
\pgfpathlineto{\pgfqpoint{3.286364in}{0.660000in}}%
\pgfpathclose%
\pgfusepath{fill}%
\end{pgfscope}%
\begin{pgfscope}%
\pgfpathrectangle{\pgfqpoint{3.286364in}{0.660000in}}{\pgfqpoint{2.113636in}{2.100000in}}%
\pgfusepath{clip}%
\pgfsetroundcap%
\pgfsetroundjoin%
\pgfsetlinewidth{1.003750pt}%
\definecolor{currentstroke}{rgb}{1.000000,1.000000,1.000000}%
\pgfsetstrokecolor{currentstroke}%
\pgfsetdash{}{0pt}%
\pgfpathmoveto{\pgfqpoint{3.382438in}{0.660000in}}%
\pgfpathlineto{\pgfqpoint{3.382438in}{2.760000in}}%
\pgfusepath{stroke}%
\end{pgfscope}%
\begin{pgfscope}%
\definecolor{textcolor}{rgb}{0.150000,0.150000,0.150000}%
\pgfsetstrokecolor{textcolor}%
\pgfsetfillcolor{textcolor}%
\pgftext[x=3.382438in,y=0.562778in,,top]{\color{textcolor}\rmfamily\fontsize{10.000000}{12.000000}\selectfont \(\displaystyle {0.0}\)}%
\end{pgfscope}%
\begin{pgfscope}%
\pgfpathrectangle{\pgfqpoint{3.286364in}{0.660000in}}{\pgfqpoint{2.113636in}{2.100000in}}%
\pgfusepath{clip}%
\pgfsetroundcap%
\pgfsetroundjoin%
\pgfsetlinewidth{1.003750pt}%
\definecolor{currentstroke}{rgb}{1.000000,1.000000,1.000000}%
\pgfsetstrokecolor{currentstroke}%
\pgfsetdash{}{0pt}%
\pgfpathmoveto{\pgfqpoint{3.862810in}{0.660000in}}%
\pgfpathlineto{\pgfqpoint{3.862810in}{2.760000in}}%
\pgfusepath{stroke}%
\end{pgfscope}%
\begin{pgfscope}%
\definecolor{textcolor}{rgb}{0.150000,0.150000,0.150000}%
\pgfsetstrokecolor{textcolor}%
\pgfsetfillcolor{textcolor}%
\pgftext[x=3.862810in,y=0.562778in,,top]{\color{textcolor}\rmfamily\fontsize{10.000000}{12.000000}\selectfont \(\displaystyle {2.5}\)}%
\end{pgfscope}%
\begin{pgfscope}%
\pgfpathrectangle{\pgfqpoint{3.286364in}{0.660000in}}{\pgfqpoint{2.113636in}{2.100000in}}%
\pgfusepath{clip}%
\pgfsetroundcap%
\pgfsetroundjoin%
\pgfsetlinewidth{1.003750pt}%
\definecolor{currentstroke}{rgb}{1.000000,1.000000,1.000000}%
\pgfsetstrokecolor{currentstroke}%
\pgfsetdash{}{0pt}%
\pgfpathmoveto{\pgfqpoint{4.343182in}{0.660000in}}%
\pgfpathlineto{\pgfqpoint{4.343182in}{2.760000in}}%
\pgfusepath{stroke}%
\end{pgfscope}%
\begin{pgfscope}%
\definecolor{textcolor}{rgb}{0.150000,0.150000,0.150000}%
\pgfsetstrokecolor{textcolor}%
\pgfsetfillcolor{textcolor}%
\pgftext[x=4.343182in,y=0.562778in,,top]{\color{textcolor}\rmfamily\fontsize{10.000000}{12.000000}\selectfont \(\displaystyle {5.0}\)}%
\end{pgfscope}%
\begin{pgfscope}%
\pgfpathrectangle{\pgfqpoint{3.286364in}{0.660000in}}{\pgfqpoint{2.113636in}{2.100000in}}%
\pgfusepath{clip}%
\pgfsetroundcap%
\pgfsetroundjoin%
\pgfsetlinewidth{1.003750pt}%
\definecolor{currentstroke}{rgb}{1.000000,1.000000,1.000000}%
\pgfsetstrokecolor{currentstroke}%
\pgfsetdash{}{0pt}%
\pgfpathmoveto{\pgfqpoint{4.823554in}{0.660000in}}%
\pgfpathlineto{\pgfqpoint{4.823554in}{2.760000in}}%
\pgfusepath{stroke}%
\end{pgfscope}%
\begin{pgfscope}%
\definecolor{textcolor}{rgb}{0.150000,0.150000,0.150000}%
\pgfsetstrokecolor{textcolor}%
\pgfsetfillcolor{textcolor}%
\pgftext[x=4.823554in,y=0.562778in,,top]{\color{textcolor}\rmfamily\fontsize{10.000000}{12.000000}\selectfont \(\displaystyle {7.5}\)}%
\end{pgfscope}%
\begin{pgfscope}%
\pgfpathrectangle{\pgfqpoint{3.286364in}{0.660000in}}{\pgfqpoint{2.113636in}{2.100000in}}%
\pgfusepath{clip}%
\pgfsetroundcap%
\pgfsetroundjoin%
\pgfsetlinewidth{1.003750pt}%
\definecolor{currentstroke}{rgb}{1.000000,1.000000,1.000000}%
\pgfsetstrokecolor{currentstroke}%
\pgfsetdash{}{0pt}%
\pgfpathmoveto{\pgfqpoint{5.303926in}{0.660000in}}%
\pgfpathlineto{\pgfqpoint{5.303926in}{2.760000in}}%
\pgfusepath{stroke}%
\end{pgfscope}%
\begin{pgfscope}%
\definecolor{textcolor}{rgb}{0.150000,0.150000,0.150000}%
\pgfsetstrokecolor{textcolor}%
\pgfsetfillcolor{textcolor}%
\pgftext[x=5.303926in,y=0.562778in,,top]{\color{textcolor}\rmfamily\fontsize{10.000000}{12.000000}\selectfont \(\displaystyle {10.0}\)}%
\end{pgfscope}%
\begin{pgfscope}%
\definecolor{textcolor}{rgb}{0.150000,0.150000,0.150000}%
\pgfsetstrokecolor{textcolor}%
\pgfsetfillcolor{textcolor}%
\pgftext[x=4.343182in,y=0.383766in,,top]{\color{textcolor}\rmfamily\fontsize{11.000000}{13.200000}\selectfont time (\(\displaystyle t\))}%
\end{pgfscope}%
\begin{pgfscope}%
\pgfpathrectangle{\pgfqpoint{3.286364in}{0.660000in}}{\pgfqpoint{2.113636in}{2.100000in}}%
\pgfusepath{clip}%
\pgfsetroundcap%
\pgfsetroundjoin%
\pgfsetlinewidth{1.003750pt}%
\definecolor{currentstroke}{rgb}{1.000000,1.000000,1.000000}%
\pgfsetstrokecolor{currentstroke}%
\pgfsetdash{}{0pt}%
\pgfpathmoveto{\pgfqpoint{3.286364in}{1.164005in}}%
\pgfpathlineto{\pgfqpoint{5.400000in}{1.164005in}}%
\pgfusepath{stroke}%
\end{pgfscope}%
\begin{pgfscope}%
\definecolor{textcolor}{rgb}{0.150000,0.150000,0.150000}%
\pgfsetstrokecolor{textcolor}%
\pgfsetfillcolor{textcolor}%
\pgftext[x=2.942227in, y=1.115780in, left, base]{\color{textcolor}\rmfamily\fontsize{10.000000}{12.000000}\selectfont \(\displaystyle {\ensuremath{-}10}\)}%
\end{pgfscope}%
\begin{pgfscope}%
\pgfpathrectangle{\pgfqpoint{3.286364in}{0.660000in}}{\pgfqpoint{2.113636in}{2.100000in}}%
\pgfusepath{clip}%
\pgfsetroundcap%
\pgfsetroundjoin%
\pgfsetlinewidth{1.003750pt}%
\definecolor{currentstroke}{rgb}{1.000000,1.000000,1.000000}%
\pgfsetstrokecolor{currentstroke}%
\pgfsetdash{}{0pt}%
\pgfpathmoveto{\pgfqpoint{3.286364in}{1.752994in}}%
\pgfpathlineto{\pgfqpoint{5.400000in}{1.752994in}}%
\pgfusepath{stroke}%
\end{pgfscope}%
\begin{pgfscope}%
\definecolor{textcolor}{rgb}{0.150000,0.150000,0.150000}%
\pgfsetstrokecolor{textcolor}%
\pgfsetfillcolor{textcolor}%
\pgftext[x=3.119697in, y=1.704769in, left, base]{\color{textcolor}\rmfamily\fontsize{10.000000}{12.000000}\selectfont \(\displaystyle {0}\)}%
\end{pgfscope}%
\begin{pgfscope}%
\pgfpathrectangle{\pgfqpoint{3.286364in}{0.660000in}}{\pgfqpoint{2.113636in}{2.100000in}}%
\pgfusepath{clip}%
\pgfsetroundcap%
\pgfsetroundjoin%
\pgfsetlinewidth{1.003750pt}%
\definecolor{currentstroke}{rgb}{1.000000,1.000000,1.000000}%
\pgfsetstrokecolor{currentstroke}%
\pgfsetdash{}{0pt}%
\pgfpathmoveto{\pgfqpoint{3.286364in}{2.341984in}}%
\pgfpathlineto{\pgfqpoint{5.400000in}{2.341984in}}%
\pgfusepath{stroke}%
\end{pgfscope}%
\begin{pgfscope}%
\definecolor{textcolor}{rgb}{0.150000,0.150000,0.150000}%
\pgfsetstrokecolor{textcolor}%
\pgfsetfillcolor{textcolor}%
\pgftext[x=3.050252in, y=2.293759in, left, base]{\color{textcolor}\rmfamily\fontsize{10.000000}{12.000000}\selectfont \(\displaystyle {10}\)}%
\end{pgfscope}%
\begin{pgfscope}%
\definecolor{textcolor}{rgb}{0.150000,0.150000,0.150000}%
\pgfsetstrokecolor{textcolor}%
\pgfsetfillcolor{textcolor}%
\pgftext[x=2.886672in,y=1.710000in,,bottom,rotate=90.000000]{\color{textcolor}\rmfamily\fontsize{11.000000}{13.200000}\selectfont Position}%
\end{pgfscope}%
\begin{pgfscope}%
\pgfpathrectangle{\pgfqpoint{3.286364in}{0.660000in}}{\pgfqpoint{2.113636in}{2.100000in}}%
\pgfusepath{clip}%
\pgfsetroundcap%
\pgfsetroundjoin%
\pgfsetlinewidth{0.602250pt}%
\definecolor{currentstroke}{rgb}{0.215686,0.494118,0.721569}%
\pgfsetstrokecolor{currentstroke}%
\pgfsetdash{}{0pt}%
\pgfpathmoveto{\pgfqpoint{3.382438in}{1.605747in}}%
\pgfpathlineto{\pgfqpoint{3.383015in}{1.603342in}}%
\pgfpathlineto{\pgfqpoint{3.383591in}{1.604046in}}%
\pgfpathlineto{\pgfqpoint{3.383783in}{1.605869in}}%
\pgfpathlineto{\pgfqpoint{3.383975in}{1.603694in}}%
\pgfpathlineto{\pgfqpoint{3.384744in}{1.604390in}}%
\pgfpathlineto{\pgfqpoint{3.384936in}{1.602925in}}%
\pgfpathlineto{\pgfqpoint{3.385513in}{1.605638in}}%
\pgfpathlineto{\pgfqpoint{3.385705in}{1.607997in}}%
\pgfpathlineto{\pgfqpoint{3.386474in}{1.603890in}}%
\pgfpathlineto{\pgfqpoint{3.386858in}{1.599935in}}%
\pgfpathlineto{\pgfqpoint{3.387627in}{1.602179in}}%
\pgfpathlineto{\pgfqpoint{3.387819in}{1.602728in}}%
\pgfpathlineto{\pgfqpoint{3.388780in}{1.597005in}}%
\pgfpathlineto{\pgfqpoint{3.388972in}{1.598531in}}%
\pgfpathlineto{\pgfqpoint{3.389164in}{1.598464in}}%
\pgfpathlineto{\pgfqpoint{3.390125in}{1.600388in}}%
\pgfpathlineto{\pgfqpoint{3.389548in}{1.596843in}}%
\pgfpathlineto{\pgfqpoint{3.390317in}{1.599006in}}%
\pgfpathlineto{\pgfqpoint{3.390893in}{1.594648in}}%
\pgfpathlineto{\pgfqpoint{3.391278in}{1.596687in}}%
\pgfpathlineto{\pgfqpoint{3.393007in}{1.613052in}}%
\pgfpathlineto{\pgfqpoint{3.393392in}{1.612058in}}%
\pgfpathlineto{\pgfqpoint{3.394929in}{1.603847in}}%
\pgfpathlineto{\pgfqpoint{3.395121in}{1.606611in}}%
\pgfpathlineto{\pgfqpoint{3.396274in}{1.615562in}}%
\pgfpathlineto{\pgfqpoint{3.396658in}{1.612811in}}%
\pgfpathlineto{\pgfqpoint{3.397619in}{1.607715in}}%
\pgfpathlineto{\pgfqpoint{3.397811in}{1.608687in}}%
\pgfpathlineto{\pgfqpoint{3.398004in}{1.611294in}}%
\pgfpathlineto{\pgfqpoint{3.398772in}{1.609842in}}%
\pgfpathlineto{\pgfqpoint{3.399541in}{1.605248in}}%
\pgfpathlineto{\pgfqpoint{3.399925in}{1.607382in}}%
\pgfpathlineto{\pgfqpoint{3.402808in}{1.596985in}}%
\pgfpathlineto{\pgfqpoint{3.403192in}{1.600971in}}%
\pgfpathlineto{\pgfqpoint{3.403576in}{1.597100in}}%
\pgfpathlineto{\pgfqpoint{3.404922in}{1.582598in}}%
\pgfpathlineto{\pgfqpoint{3.405114in}{1.583790in}}%
\pgfpathlineto{\pgfqpoint{3.406267in}{1.588792in}}%
\pgfpathlineto{\pgfqpoint{3.406459in}{1.586051in}}%
\pgfpathlineto{\pgfqpoint{3.407612in}{1.581665in}}%
\pgfpathlineto{\pgfqpoint{3.407804in}{1.582081in}}%
\pgfpathlineto{\pgfqpoint{3.409342in}{1.593551in}}%
\pgfpathlineto{\pgfqpoint{3.410687in}{1.581734in}}%
\pgfpathlineto{\pgfqpoint{3.411071in}{1.584442in}}%
\pgfpathlineto{\pgfqpoint{3.411263in}{1.583821in}}%
\pgfpathlineto{\pgfqpoint{3.411455in}{1.586857in}}%
\pgfpathlineto{\pgfqpoint{3.411840in}{1.590225in}}%
\pgfpathlineto{\pgfqpoint{3.412416in}{1.586297in}}%
\pgfpathlineto{\pgfqpoint{3.412608in}{1.587958in}}%
\pgfpathlineto{\pgfqpoint{3.412993in}{1.592002in}}%
\pgfpathlineto{\pgfqpoint{3.413377in}{1.587319in}}%
\pgfpathlineto{\pgfqpoint{3.414146in}{1.583180in}}%
\pgfpathlineto{\pgfqpoint{3.414722in}{1.583841in}}%
\pgfpathlineto{\pgfqpoint{3.418566in}{1.555129in}}%
\pgfpathlineto{\pgfqpoint{3.419526in}{1.562223in}}%
\pgfpathlineto{\pgfqpoint{3.420103in}{1.561925in}}%
\pgfpathlineto{\pgfqpoint{3.420295in}{1.560221in}}%
\pgfpathlineto{\pgfqpoint{3.420679in}{1.566152in}}%
\pgfpathlineto{\pgfqpoint{3.420872in}{1.564609in}}%
\pgfpathlineto{\pgfqpoint{3.421448in}{1.567172in}}%
\pgfpathlineto{\pgfqpoint{3.421640in}{1.564000in}}%
\pgfpathlineto{\pgfqpoint{3.421832in}{1.564691in}}%
\pgfpathlineto{\pgfqpoint{3.423946in}{1.545278in}}%
\pgfpathlineto{\pgfqpoint{3.425676in}{1.554609in}}%
\pgfpathlineto{\pgfqpoint{3.426829in}{1.558283in}}%
\pgfpathlineto{\pgfqpoint{3.427021in}{1.556615in}}%
\pgfpathlineto{\pgfqpoint{3.428366in}{1.547202in}}%
\pgfpathlineto{\pgfqpoint{3.428558in}{1.547797in}}%
\pgfpathlineto{\pgfqpoint{3.429711in}{1.554264in}}%
\pgfpathlineto{\pgfqpoint{3.429904in}{1.551337in}}%
\pgfpathlineto{\pgfqpoint{3.430864in}{1.545310in}}%
\pgfpathlineto{\pgfqpoint{3.431249in}{1.547579in}}%
\pgfpathlineto{\pgfqpoint{3.431441in}{1.547668in}}%
\pgfpathlineto{\pgfqpoint{3.433170in}{1.537910in}}%
\pgfpathlineto{\pgfqpoint{3.434131in}{1.543842in}}%
\pgfpathlineto{\pgfqpoint{3.434516in}{1.541759in}}%
\pgfpathlineto{\pgfqpoint{3.434708in}{1.541655in}}%
\pgfpathlineto{\pgfqpoint{3.434900in}{1.542305in}}%
\pgfpathlineto{\pgfqpoint{3.437398in}{1.547913in}}%
\pgfpathlineto{\pgfqpoint{3.438743in}{1.540734in}}%
\pgfpathlineto{\pgfqpoint{3.439320in}{1.536001in}}%
\pgfpathlineto{\pgfqpoint{3.439704in}{1.539870in}}%
\pgfpathlineto{\pgfqpoint{3.439896in}{1.543110in}}%
\pgfpathlineto{\pgfqpoint{3.440857in}{1.541122in}}%
\pgfpathlineto{\pgfqpoint{3.443163in}{1.530861in}}%
\pgfpathlineto{\pgfqpoint{3.443355in}{1.533244in}}%
\pgfpathlineto{\pgfqpoint{3.444508in}{1.539032in}}%
\pgfpathlineto{\pgfqpoint{3.445085in}{1.541459in}}%
\pgfpathlineto{\pgfqpoint{3.445469in}{1.538938in}}%
\pgfpathlineto{\pgfqpoint{3.445661in}{1.540823in}}%
\pgfpathlineto{\pgfqpoint{3.445853in}{1.540905in}}%
\pgfpathlineto{\pgfqpoint{3.446046in}{1.542988in}}%
\pgfpathlineto{\pgfqpoint{3.446622in}{1.539141in}}%
\pgfpathlineto{\pgfqpoint{3.446814in}{1.541368in}}%
\pgfpathlineto{\pgfqpoint{3.447391in}{1.535996in}}%
\pgfpathlineto{\pgfqpoint{3.448159in}{1.538831in}}%
\pgfpathlineto{\pgfqpoint{3.448928in}{1.550775in}}%
\pgfpathlineto{\pgfqpoint{3.449505in}{1.550217in}}%
\pgfpathlineto{\pgfqpoint{3.450465in}{1.539486in}}%
\pgfpathlineto{\pgfqpoint{3.450850in}{1.542715in}}%
\pgfpathlineto{\pgfqpoint{3.451042in}{1.544865in}}%
\pgfpathlineto{\pgfqpoint{3.451426in}{1.540146in}}%
\pgfpathlineto{\pgfqpoint{3.451811in}{1.542055in}}%
\pgfpathlineto{\pgfqpoint{3.452387in}{1.536779in}}%
\pgfpathlineto{\pgfqpoint{3.453348in}{1.537437in}}%
\pgfpathlineto{\pgfqpoint{3.454309in}{1.541781in}}%
\pgfpathlineto{\pgfqpoint{3.454693in}{1.540256in}}%
\pgfpathlineto{\pgfqpoint{3.457768in}{1.522535in}}%
\pgfpathlineto{\pgfqpoint{3.457960in}{1.525351in}}%
\pgfpathlineto{\pgfqpoint{3.459497in}{1.535504in}}%
\pgfpathlineto{\pgfqpoint{3.459882in}{1.532502in}}%
\pgfpathlineto{\pgfqpoint{3.460266in}{1.537337in}}%
\pgfpathlineto{\pgfqpoint{3.460650in}{1.533684in}}%
\pgfpathlineto{\pgfqpoint{3.461035in}{1.535188in}}%
\pgfpathlineto{\pgfqpoint{3.461419in}{1.532365in}}%
\pgfpathlineto{\pgfqpoint{3.462572in}{1.527449in}}%
\pgfpathlineto{\pgfqpoint{3.462764in}{1.530227in}}%
\pgfpathlineto{\pgfqpoint{3.462956in}{1.531240in}}%
\pgfpathlineto{\pgfqpoint{3.463341in}{1.529238in}}%
\pgfpathlineto{\pgfqpoint{3.463533in}{1.527592in}}%
\pgfpathlineto{\pgfqpoint{3.463725in}{1.531907in}}%
\pgfpathlineto{\pgfqpoint{3.463917in}{1.531450in}}%
\pgfpathlineto{\pgfqpoint{3.464302in}{1.529961in}}%
\pgfpathlineto{\pgfqpoint{3.464878in}{1.532245in}}%
\pgfpathlineto{\pgfqpoint{3.465455in}{1.533374in}}%
\pgfpathlineto{\pgfqpoint{3.466031in}{1.530565in}}%
\pgfpathlineto{\pgfqpoint{3.466608in}{1.532509in}}%
\pgfpathlineto{\pgfqpoint{3.467568in}{1.526578in}}%
\pgfpathlineto{\pgfqpoint{3.467184in}{1.532848in}}%
\pgfpathlineto{\pgfqpoint{3.467761in}{1.527860in}}%
\pgfpathlineto{\pgfqpoint{3.469106in}{1.535590in}}%
\pgfpathlineto{\pgfqpoint{3.469490in}{1.538045in}}%
\pgfpathlineto{\pgfqpoint{3.469874in}{1.534639in}}%
\pgfpathlineto{\pgfqpoint{3.470259in}{1.535110in}}%
\pgfpathlineto{\pgfqpoint{3.471412in}{1.530221in}}%
\pgfpathlineto{\pgfqpoint{3.471796in}{1.525284in}}%
\pgfpathlineto{\pgfqpoint{3.472565in}{1.527133in}}%
\pgfpathlineto{\pgfqpoint{3.472949in}{1.525497in}}%
\pgfpathlineto{\pgfqpoint{3.474294in}{1.521342in}}%
\pgfpathlineto{\pgfqpoint{3.474679in}{1.517640in}}%
\pgfpathlineto{\pgfqpoint{3.475063in}{1.519732in}}%
\pgfpathlineto{\pgfqpoint{3.476600in}{1.531729in}}%
\pgfpathlineto{\pgfqpoint{3.476792in}{1.528806in}}%
\pgfpathlineto{\pgfqpoint{3.477177in}{1.524923in}}%
\pgfpathlineto{\pgfqpoint{3.477946in}{1.528182in}}%
\pgfpathlineto{\pgfqpoint{3.478906in}{1.529402in}}%
\pgfpathlineto{\pgfqpoint{3.480059in}{1.539708in}}%
\pgfpathlineto{\pgfqpoint{3.480444in}{1.537899in}}%
\pgfpathlineto{\pgfqpoint{3.481789in}{1.528821in}}%
\pgfpathlineto{\pgfqpoint{3.481981in}{1.531204in}}%
\pgfpathlineto{\pgfqpoint{3.482173in}{1.532253in}}%
\pgfpathlineto{\pgfqpoint{3.482365in}{1.530835in}}%
\pgfpathlineto{\pgfqpoint{3.483518in}{1.524465in}}%
\pgfpathlineto{\pgfqpoint{3.483903in}{1.527881in}}%
\pgfpathlineto{\pgfqpoint{3.484287in}{1.522959in}}%
\pgfpathlineto{\pgfqpoint{3.484479in}{1.523010in}}%
\pgfpathlineto{\pgfqpoint{3.485632in}{1.517393in}}%
\pgfpathlineto{\pgfqpoint{3.485824in}{1.520672in}}%
\pgfpathlineto{\pgfqpoint{3.486401in}{1.522480in}}%
\pgfpathlineto{\pgfqpoint{3.486593in}{1.520055in}}%
\pgfpathlineto{\pgfqpoint{3.486785in}{1.520746in}}%
\pgfpathlineto{\pgfqpoint{3.488707in}{1.508361in}}%
\pgfpathlineto{\pgfqpoint{3.488899in}{1.509451in}}%
\pgfpathlineto{\pgfqpoint{3.489091in}{1.505831in}}%
\pgfpathlineto{\pgfqpoint{3.489283in}{1.506225in}}%
\pgfpathlineto{\pgfqpoint{3.489668in}{1.504217in}}%
\pgfpathlineto{\pgfqpoint{3.489860in}{1.505483in}}%
\pgfpathlineto{\pgfqpoint{3.490629in}{1.515704in}}%
\pgfpathlineto{\pgfqpoint{3.491397in}{1.511711in}}%
\pgfpathlineto{\pgfqpoint{3.492935in}{1.517663in}}%
\pgfpathlineto{\pgfqpoint{3.493895in}{1.513290in}}%
\pgfpathlineto{\pgfqpoint{3.494280in}{1.514720in}}%
\pgfpathlineto{\pgfqpoint{3.494472in}{1.514537in}}%
\pgfpathlineto{\pgfqpoint{3.494664in}{1.514969in}}%
\pgfpathlineto{\pgfqpoint{3.494856in}{1.511028in}}%
\pgfpathlineto{\pgfqpoint{3.495817in}{1.512266in}}%
\pgfpathlineto{\pgfqpoint{3.497547in}{1.501954in}}%
\pgfpathlineto{\pgfqpoint{3.497739in}{1.503035in}}%
\pgfpathlineto{\pgfqpoint{3.498892in}{1.509773in}}%
\pgfpathlineto{\pgfqpoint{3.499084in}{1.506689in}}%
\pgfpathlineto{\pgfqpoint{3.500621in}{1.497738in}}%
\pgfpathlineto{\pgfqpoint{3.500813in}{1.497733in}}%
\pgfpathlineto{\pgfqpoint{3.502735in}{1.481750in}}%
\pgfpathlineto{\pgfqpoint{3.502927in}{1.481534in}}%
\pgfpathlineto{\pgfqpoint{3.503120in}{1.478767in}}%
\pgfpathlineto{\pgfqpoint{3.503696in}{1.483151in}}%
\pgfpathlineto{\pgfqpoint{3.503888in}{1.482640in}}%
\pgfpathlineto{\pgfqpoint{3.505426in}{1.484995in}}%
\pgfpathlineto{\pgfqpoint{3.506579in}{1.481122in}}%
\pgfpathlineto{\pgfqpoint{3.506771in}{1.482486in}}%
\pgfpathlineto{\pgfqpoint{3.510422in}{1.500685in}}%
\pgfpathlineto{\pgfqpoint{3.511383in}{1.496567in}}%
\pgfpathlineto{\pgfqpoint{3.512728in}{1.511056in}}%
\pgfpathlineto{\pgfqpoint{3.512920in}{1.510636in}}%
\pgfpathlineto{\pgfqpoint{3.513112in}{1.510154in}}%
\pgfpathlineto{\pgfqpoint{3.514073in}{1.504863in}}%
\pgfpathlineto{\pgfqpoint{3.514457in}{1.505056in}}%
\pgfpathlineto{\pgfqpoint{3.515803in}{1.511979in}}%
\pgfpathlineto{\pgfqpoint{3.515995in}{1.509604in}}%
\pgfpathlineto{\pgfqpoint{3.516379in}{1.507496in}}%
\pgfpathlineto{\pgfqpoint{3.516763in}{1.511316in}}%
\pgfpathlineto{\pgfqpoint{3.517340in}{1.511782in}}%
\pgfpathlineto{\pgfqpoint{3.519262in}{1.523604in}}%
\pgfpathlineto{\pgfqpoint{3.519838in}{1.522967in}}%
\pgfpathlineto{\pgfqpoint{3.520222in}{1.522329in}}%
\pgfpathlineto{\pgfqpoint{3.520799in}{1.526891in}}%
\pgfpathlineto{\pgfqpoint{3.521375in}{1.526317in}}%
\pgfpathlineto{\pgfqpoint{3.522528in}{1.515932in}}%
\pgfpathlineto{\pgfqpoint{3.522913in}{1.517902in}}%
\pgfpathlineto{\pgfqpoint{3.523297in}{1.518956in}}%
\pgfpathlineto{\pgfqpoint{3.524642in}{1.509840in}}%
\pgfpathlineto{\pgfqpoint{3.525411in}{1.511568in}}%
\pgfpathlineto{\pgfqpoint{3.525027in}{1.508319in}}%
\pgfpathlineto{\pgfqpoint{3.525795in}{1.510680in}}%
\pgfpathlineto{\pgfqpoint{3.526756in}{1.506924in}}%
\pgfpathlineto{\pgfqpoint{3.526948in}{1.508068in}}%
\pgfpathlineto{\pgfqpoint{3.527141in}{1.508270in}}%
\pgfpathlineto{\pgfqpoint{3.528486in}{1.512013in}}%
\pgfpathlineto{\pgfqpoint{3.529062in}{1.518592in}}%
\pgfpathlineto{\pgfqpoint{3.529831in}{1.515144in}}%
\pgfpathlineto{\pgfqpoint{3.530023in}{1.515444in}}%
\pgfpathlineto{\pgfqpoint{3.530407in}{1.510879in}}%
\pgfpathlineto{\pgfqpoint{3.530984in}{1.515599in}}%
\pgfpathlineto{\pgfqpoint{3.531176in}{1.516175in}}%
\pgfpathlineto{\pgfqpoint{3.531368in}{1.513622in}}%
\pgfpathlineto{\pgfqpoint{3.531753in}{1.507866in}}%
\pgfpathlineto{\pgfqpoint{3.532521in}{1.511039in}}%
\pgfpathlineto{\pgfqpoint{3.533290in}{1.516420in}}%
\pgfpathlineto{\pgfqpoint{3.533482in}{1.512823in}}%
\pgfpathlineto{\pgfqpoint{3.534635in}{1.507210in}}%
\pgfpathlineto{\pgfqpoint{3.534059in}{1.513828in}}%
\pgfpathlineto{\pgfqpoint{3.535019in}{1.508370in}}%
\pgfpathlineto{\pgfqpoint{3.536365in}{1.517002in}}%
\pgfpathlineto{\pgfqpoint{3.536749in}{1.514265in}}%
\pgfpathlineto{\pgfqpoint{3.537325in}{1.516067in}}%
\pgfpathlineto{\pgfqpoint{3.538478in}{1.508927in}}%
\pgfpathlineto{\pgfqpoint{3.538671in}{1.509010in}}%
\pgfpathlineto{\pgfqpoint{3.538863in}{1.512228in}}%
\pgfpathlineto{\pgfqpoint{3.539247in}{1.508470in}}%
\pgfpathlineto{\pgfqpoint{3.539824in}{1.510604in}}%
\pgfpathlineto{\pgfqpoint{3.540016in}{1.509486in}}%
\pgfpathlineto{\pgfqpoint{3.540400in}{1.513048in}}%
\pgfpathlineto{\pgfqpoint{3.542130in}{1.520428in}}%
\pgfpathlineto{\pgfqpoint{3.542322in}{1.518867in}}%
\pgfpathlineto{\pgfqpoint{3.543283in}{1.511780in}}%
\pgfpathlineto{\pgfqpoint{3.543667in}{1.514729in}}%
\pgfpathlineto{\pgfqpoint{3.543859in}{1.515321in}}%
\pgfpathlineto{\pgfqpoint{3.545012in}{1.508842in}}%
\pgfpathlineto{\pgfqpoint{3.545204in}{1.509027in}}%
\pgfpathlineto{\pgfqpoint{3.547702in}{1.497369in}}%
\pgfpathlineto{\pgfqpoint{3.548087in}{1.500079in}}%
\pgfpathlineto{\pgfqpoint{3.549240in}{1.506356in}}%
\pgfpathlineto{\pgfqpoint{3.550393in}{1.502972in}}%
\pgfpathlineto{\pgfqpoint{3.550585in}{1.505998in}}%
\pgfpathlineto{\pgfqpoint{3.551162in}{1.502682in}}%
\pgfpathlineto{\pgfqpoint{3.551354in}{1.503126in}}%
\pgfpathlineto{\pgfqpoint{3.551546in}{1.502765in}}%
\pgfpathlineto{\pgfqpoint{3.552122in}{1.498562in}}%
\pgfpathlineto{\pgfqpoint{3.552699in}{1.498996in}}%
\pgfpathlineto{\pgfqpoint{3.553852in}{1.501105in}}%
\pgfpathlineto{\pgfqpoint{3.554428in}{1.506859in}}%
\pgfpathlineto{\pgfqpoint{3.555005in}{1.503562in}}%
\pgfpathlineto{\pgfqpoint{3.555197in}{1.500113in}}%
\pgfpathlineto{\pgfqpoint{3.555581in}{1.505013in}}%
\pgfpathlineto{\pgfqpoint{3.555774in}{1.504417in}}%
\pgfpathlineto{\pgfqpoint{3.556734in}{1.509096in}}%
\pgfpathlineto{\pgfqpoint{3.556927in}{1.506972in}}%
\pgfpathlineto{\pgfqpoint{3.557503in}{1.508404in}}%
\pgfpathlineto{\pgfqpoint{3.558080in}{1.503001in}}%
\pgfpathlineto{\pgfqpoint{3.559425in}{1.508682in}}%
\pgfpathlineto{\pgfqpoint{3.560386in}{1.511439in}}%
\pgfpathlineto{\pgfqpoint{3.560578in}{1.509578in}}%
\pgfpathlineto{\pgfqpoint{3.560770in}{1.502335in}}%
\pgfpathlineto{\pgfqpoint{3.561731in}{1.504061in}}%
\pgfpathlineto{\pgfqpoint{3.561923in}{1.506640in}}%
\pgfpathlineto{\pgfqpoint{3.562499in}{1.502204in}}%
\pgfpathlineto{\pgfqpoint{3.563076in}{1.499445in}}%
\pgfpathlineto{\pgfqpoint{3.563652in}{1.501694in}}%
\pgfpathlineto{\pgfqpoint{3.566151in}{1.488966in}}%
\pgfpathlineto{\pgfqpoint{3.566535in}{1.490300in}}%
\pgfpathlineto{\pgfqpoint{3.567304in}{1.495065in}}%
\pgfpathlineto{\pgfqpoint{3.567688in}{1.493527in}}%
\pgfpathlineto{\pgfqpoint{3.568841in}{1.488913in}}%
\pgfpathlineto{\pgfqpoint{3.569033in}{1.489773in}}%
\pgfpathlineto{\pgfqpoint{3.569610in}{1.487088in}}%
\pgfpathlineto{\pgfqpoint{3.570955in}{1.500582in}}%
\pgfpathlineto{\pgfqpoint{3.571147in}{1.500148in}}%
\pgfpathlineto{\pgfqpoint{3.571339in}{1.499505in}}%
\pgfpathlineto{\pgfqpoint{3.571531in}{1.501380in}}%
\pgfpathlineto{\pgfqpoint{3.571723in}{1.500641in}}%
\pgfpathlineto{\pgfqpoint{3.571916in}{1.501667in}}%
\pgfpathlineto{\pgfqpoint{3.572108in}{1.498827in}}%
\pgfpathlineto{\pgfqpoint{3.572300in}{1.499496in}}%
\pgfpathlineto{\pgfqpoint{3.572492in}{1.497508in}}%
\pgfpathlineto{\pgfqpoint{3.573069in}{1.502191in}}%
\pgfpathlineto{\pgfqpoint{3.573261in}{1.502318in}}%
\pgfpathlineto{\pgfqpoint{3.574798in}{1.513258in}}%
\pgfpathlineto{\pgfqpoint{3.574990in}{1.510993in}}%
\pgfpathlineto{\pgfqpoint{3.575567in}{1.514335in}}%
\pgfpathlineto{\pgfqpoint{3.575759in}{1.515815in}}%
\pgfpathlineto{\pgfqpoint{3.576143in}{1.510396in}}%
\pgfpathlineto{\pgfqpoint{3.576336in}{1.511040in}}%
\pgfpathlineto{\pgfqpoint{3.576720in}{1.508765in}}%
\pgfpathlineto{\pgfqpoint{3.576912in}{1.508535in}}%
\pgfpathlineto{\pgfqpoint{3.577873in}{1.513075in}}%
\pgfpathlineto{\pgfqpoint{3.578449in}{1.512363in}}%
\pgfpathlineto{\pgfqpoint{3.578642in}{1.512717in}}%
\pgfpathlineto{\pgfqpoint{3.579218in}{1.519167in}}%
\pgfpathlineto{\pgfqpoint{3.579795in}{1.518201in}}%
\pgfpathlineto{\pgfqpoint{3.579987in}{1.516078in}}%
\pgfpathlineto{\pgfqpoint{3.580755in}{1.519032in}}%
\pgfpathlineto{\pgfqpoint{3.580948in}{1.517334in}}%
\pgfpathlineto{\pgfqpoint{3.581140in}{1.518774in}}%
\pgfpathlineto{\pgfqpoint{3.581524in}{1.517355in}}%
\pgfpathlineto{\pgfqpoint{3.582677in}{1.509813in}}%
\pgfpathlineto{\pgfqpoint{3.582869in}{1.511249in}}%
\pgfpathlineto{\pgfqpoint{3.583061in}{1.512688in}}%
\pgfpathlineto{\pgfqpoint{3.583446in}{1.509688in}}%
\pgfpathlineto{\pgfqpoint{3.583638in}{1.510854in}}%
\pgfpathlineto{\pgfqpoint{3.584214in}{1.512455in}}%
\pgfpathlineto{\pgfqpoint{3.585752in}{1.503175in}}%
\pgfpathlineto{\pgfqpoint{3.585944in}{1.504669in}}%
\pgfpathlineto{\pgfqpoint{3.586328in}{1.501750in}}%
\pgfpathlineto{\pgfqpoint{3.586905in}{1.504437in}}%
\pgfpathlineto{\pgfqpoint{3.587866in}{1.506203in}}%
\pgfpathlineto{\pgfqpoint{3.588058in}{1.504012in}}%
\pgfpathlineto{\pgfqpoint{3.588250in}{1.506625in}}%
\pgfpathlineto{\pgfqpoint{3.588826in}{1.505024in}}%
\pgfpathlineto{\pgfqpoint{3.589595in}{1.503188in}}%
\pgfpathlineto{\pgfqpoint{3.590364in}{1.511942in}}%
\pgfpathlineto{\pgfqpoint{3.590748in}{1.512643in}}%
\pgfpathlineto{\pgfqpoint{3.591709in}{1.505747in}}%
\pgfpathlineto{\pgfqpoint{3.592285in}{1.501543in}}%
\pgfpathlineto{\pgfqpoint{3.592862in}{1.504558in}}%
\pgfpathlineto{\pgfqpoint{3.594399in}{1.500428in}}%
\pgfpathlineto{\pgfqpoint{3.593246in}{1.505689in}}%
\pgfpathlineto{\pgfqpoint{3.594591in}{1.501143in}}%
\pgfpathlineto{\pgfqpoint{3.594784in}{1.503240in}}%
\pgfpathlineto{\pgfqpoint{3.595360in}{1.499666in}}%
\pgfpathlineto{\pgfqpoint{3.595552in}{1.500496in}}%
\pgfpathlineto{\pgfqpoint{3.597858in}{1.489706in}}%
\pgfpathlineto{\pgfqpoint{3.598435in}{1.490502in}}%
\pgfpathlineto{\pgfqpoint{3.598627in}{1.489302in}}%
\pgfpathlineto{\pgfqpoint{3.601317in}{1.467637in}}%
\pgfpathlineto{\pgfqpoint{3.601894in}{1.470608in}}%
\pgfpathlineto{\pgfqpoint{3.602470in}{1.468715in}}%
\pgfpathlineto{\pgfqpoint{3.602663in}{1.468769in}}%
\pgfpathlineto{\pgfqpoint{3.603047in}{1.471825in}}%
\pgfpathlineto{\pgfqpoint{3.603431in}{1.466562in}}%
\pgfpathlineto{\pgfqpoint{3.604969in}{1.458158in}}%
\pgfpathlineto{\pgfqpoint{3.606698in}{1.468885in}}%
\pgfpathlineto{\pgfqpoint{3.605545in}{1.457762in}}%
\pgfpathlineto{\pgfqpoint{3.607275in}{1.465469in}}%
\pgfpathlineto{\pgfqpoint{3.607851in}{1.465911in}}%
\pgfpathlineto{\pgfqpoint{3.608043in}{1.463891in}}%
\pgfpathlineto{\pgfqpoint{3.608235in}{1.467610in}}%
\pgfpathlineto{\pgfqpoint{3.608620in}{1.462671in}}%
\pgfpathlineto{\pgfqpoint{3.609196in}{1.465683in}}%
\pgfpathlineto{\pgfqpoint{3.609388in}{1.465350in}}%
\pgfpathlineto{\pgfqpoint{3.609773in}{1.459393in}}%
\pgfpathlineto{\pgfqpoint{3.610541in}{1.463613in}}%
\pgfpathlineto{\pgfqpoint{3.610734in}{1.464354in}}%
\pgfpathlineto{\pgfqpoint{3.610926in}{1.461505in}}%
\pgfpathlineto{\pgfqpoint{3.611118in}{1.462089in}}%
\pgfpathlineto{\pgfqpoint{3.611310in}{1.461401in}}%
\pgfpathlineto{\pgfqpoint{3.611694in}{1.467340in}}%
\pgfpathlineto{\pgfqpoint{3.612463in}{1.464757in}}%
\pgfpathlineto{\pgfqpoint{3.613232in}{1.466244in}}%
\pgfpathlineto{\pgfqpoint{3.613616in}{1.462666in}}%
\pgfpathlineto{\pgfqpoint{3.615153in}{1.468624in}}%
\pgfpathlineto{\pgfqpoint{3.616306in}{1.463311in}}%
\pgfpathlineto{\pgfqpoint{3.616883in}{1.464928in}}%
\pgfpathlineto{\pgfqpoint{3.617075in}{1.466963in}}%
\pgfpathlineto{\pgfqpoint{3.617844in}{1.463715in}}%
\pgfpathlineto{\pgfqpoint{3.618036in}{1.466072in}}%
\pgfpathlineto{\pgfqpoint{3.619573in}{1.460787in}}%
\pgfpathlineto{\pgfqpoint{3.619958in}{1.463096in}}%
\pgfpathlineto{\pgfqpoint{3.621303in}{1.468738in}}%
\pgfpathlineto{\pgfqpoint{3.622264in}{1.471131in}}%
\pgfpathlineto{\pgfqpoint{3.621687in}{1.468610in}}%
\pgfpathlineto{\pgfqpoint{3.622648in}{1.470810in}}%
\pgfpathlineto{\pgfqpoint{3.624185in}{1.462154in}}%
\pgfpathlineto{\pgfqpoint{3.623224in}{1.471791in}}%
\pgfpathlineto{\pgfqpoint{3.624570in}{1.464817in}}%
\pgfpathlineto{\pgfqpoint{3.625146in}{1.470368in}}%
\pgfpathlineto{\pgfqpoint{3.625915in}{1.468921in}}%
\pgfpathlineto{\pgfqpoint{3.626876in}{1.462706in}}%
\pgfpathlineto{\pgfqpoint{3.627068in}{1.463767in}}%
\pgfpathlineto{\pgfqpoint{3.627260in}{1.466721in}}%
\pgfpathlineto{\pgfqpoint{3.627452in}{1.462591in}}%
\pgfpathlineto{\pgfqpoint{3.628029in}{1.464065in}}%
\pgfpathlineto{\pgfqpoint{3.629182in}{1.456525in}}%
\pgfpathlineto{\pgfqpoint{3.629374in}{1.457487in}}%
\pgfpathlineto{\pgfqpoint{3.629566in}{1.456008in}}%
\pgfpathlineto{\pgfqpoint{3.629950in}{1.459301in}}%
\pgfpathlineto{\pgfqpoint{3.630335in}{1.458711in}}%
\pgfpathlineto{\pgfqpoint{3.631680in}{1.456263in}}%
\pgfpathlineto{\pgfqpoint{3.630719in}{1.459453in}}%
\pgfpathlineto{\pgfqpoint{3.632064in}{1.456842in}}%
\pgfpathlineto{\pgfqpoint{3.632256in}{1.459401in}}%
\pgfpathlineto{\pgfqpoint{3.632641in}{1.455880in}}%
\pgfpathlineto{\pgfqpoint{3.633025in}{1.456234in}}%
\pgfpathlineto{\pgfqpoint{3.634755in}{1.445902in}}%
\pgfpathlineto{\pgfqpoint{3.635331in}{1.443329in}}%
\pgfpathlineto{\pgfqpoint{3.635523in}{1.445587in}}%
\pgfpathlineto{\pgfqpoint{3.636100in}{1.450960in}}%
\pgfpathlineto{\pgfqpoint{3.636676in}{1.447712in}}%
\pgfpathlineto{\pgfqpoint{3.637445in}{1.445711in}}%
\pgfpathlineto{\pgfqpoint{3.637637in}{1.446927in}}%
\pgfpathlineto{\pgfqpoint{3.638214in}{1.450091in}}%
\pgfpathlineto{\pgfqpoint{3.638598in}{1.446611in}}%
\pgfpathlineto{\pgfqpoint{3.638982in}{1.445965in}}%
\pgfpathlineto{\pgfqpoint{3.639367in}{1.450490in}}%
\pgfpathlineto{\pgfqpoint{3.639943in}{1.448112in}}%
\pgfpathlineto{\pgfqpoint{3.640520in}{1.443590in}}%
\pgfpathlineto{\pgfqpoint{3.640904in}{1.448104in}}%
\pgfpathlineto{\pgfqpoint{3.642057in}{1.453718in}}%
\pgfpathlineto{\pgfqpoint{3.641288in}{1.446974in}}%
\pgfpathlineto{\pgfqpoint{3.642249in}{1.451599in}}%
\pgfpathlineto{\pgfqpoint{3.642441in}{1.448970in}}%
\pgfpathlineto{\pgfqpoint{3.642826in}{1.451741in}}%
\pgfpathlineto{\pgfqpoint{3.643210in}{1.449846in}}%
\pgfpathlineto{\pgfqpoint{3.644363in}{1.454756in}}%
\pgfpathlineto{\pgfqpoint{3.643594in}{1.449774in}}%
\pgfpathlineto{\pgfqpoint{3.644555in}{1.453989in}}%
\pgfpathlineto{\pgfqpoint{3.644747in}{1.452783in}}%
\pgfpathlineto{\pgfqpoint{3.645324in}{1.456125in}}%
\pgfpathlineto{\pgfqpoint{3.645516in}{1.455793in}}%
\pgfpathlineto{\pgfqpoint{3.645900in}{1.460941in}}%
\pgfpathlineto{\pgfqpoint{3.646477in}{1.455371in}}%
\pgfpathlineto{\pgfqpoint{3.646669in}{1.456793in}}%
\pgfpathlineto{\pgfqpoint{3.647245in}{1.455971in}}%
\pgfpathlineto{\pgfqpoint{3.647438in}{1.457113in}}%
\pgfpathlineto{\pgfqpoint{3.648014in}{1.459556in}}%
\pgfpathlineto{\pgfqpoint{3.648398in}{1.456436in}}%
\pgfpathlineto{\pgfqpoint{3.649552in}{1.450825in}}%
\pgfpathlineto{\pgfqpoint{3.649744in}{1.451580in}}%
\pgfpathlineto{\pgfqpoint{3.650512in}{1.450705in}}%
\pgfpathlineto{\pgfqpoint{3.651089in}{1.447184in}}%
\pgfpathlineto{\pgfqpoint{3.651665in}{1.449035in}}%
\pgfpathlineto{\pgfqpoint{3.651858in}{1.450462in}}%
\pgfpathlineto{\pgfqpoint{3.652242in}{1.445583in}}%
\pgfpathlineto{\pgfqpoint{3.653203in}{1.438212in}}%
\pgfpathlineto{\pgfqpoint{3.653779in}{1.439831in}}%
\pgfpathlineto{\pgfqpoint{3.654164in}{1.441487in}}%
\pgfpathlineto{\pgfqpoint{3.654548in}{1.439397in}}%
\pgfpathlineto{\pgfqpoint{3.654740in}{1.439088in}}%
\pgfpathlineto{\pgfqpoint{3.654932in}{1.439287in}}%
\pgfpathlineto{\pgfqpoint{3.655509in}{1.442264in}}%
\pgfpathlineto{\pgfqpoint{3.655893in}{1.441631in}}%
\pgfpathlineto{\pgfqpoint{3.656662in}{1.443809in}}%
\pgfpathlineto{\pgfqpoint{3.657238in}{1.437452in}}%
\pgfpathlineto{\pgfqpoint{3.657623in}{1.437919in}}%
\pgfpathlineto{\pgfqpoint{3.658583in}{1.425135in}}%
\pgfpathlineto{\pgfqpoint{3.658968in}{1.425304in}}%
\pgfpathlineto{\pgfqpoint{3.659929in}{1.437492in}}%
\pgfpathlineto{\pgfqpoint{3.660313in}{1.431866in}}%
\pgfpathlineto{\pgfqpoint{3.660697in}{1.433006in}}%
\pgfpathlineto{\pgfqpoint{3.661466in}{1.440443in}}%
\pgfpathlineto{\pgfqpoint{3.661850in}{1.436744in}}%
\pgfpathlineto{\pgfqpoint{3.662042in}{1.435219in}}%
\pgfpathlineto{\pgfqpoint{3.662427in}{1.439136in}}%
\pgfpathlineto{\pgfqpoint{3.663964in}{1.452056in}}%
\pgfpathlineto{\pgfqpoint{3.664733in}{1.447931in}}%
\pgfpathlineto{\pgfqpoint{3.665117in}{1.444159in}}%
\pgfpathlineto{\pgfqpoint{3.665694in}{1.448326in}}%
\pgfpathlineto{\pgfqpoint{3.666847in}{1.451124in}}%
\pgfpathlineto{\pgfqpoint{3.666270in}{1.448151in}}%
\pgfpathlineto{\pgfqpoint{3.667039in}{1.449588in}}%
\pgfpathlineto{\pgfqpoint{3.667807in}{1.447754in}}%
\pgfpathlineto{\pgfqpoint{3.668960in}{1.440479in}}%
\pgfpathlineto{\pgfqpoint{3.668192in}{1.449573in}}%
\pgfpathlineto{\pgfqpoint{3.669345in}{1.442385in}}%
\pgfpathlineto{\pgfqpoint{3.669537in}{1.445484in}}%
\pgfpathlineto{\pgfqpoint{3.670113in}{1.439582in}}%
\pgfpathlineto{\pgfqpoint{3.670306in}{1.440753in}}%
\pgfpathlineto{\pgfqpoint{3.670882in}{1.436017in}}%
\pgfpathlineto{\pgfqpoint{3.671459in}{1.437033in}}%
\pgfpathlineto{\pgfqpoint{3.672996in}{1.446558in}}%
\pgfpathlineto{\pgfqpoint{3.673188in}{1.444714in}}%
\pgfpathlineto{\pgfqpoint{3.673380in}{1.450343in}}%
\pgfpathlineto{\pgfqpoint{3.673573in}{1.455035in}}%
\pgfpathlineto{\pgfqpoint{3.674341in}{1.447226in}}%
\pgfpathlineto{\pgfqpoint{3.675686in}{1.439018in}}%
\pgfpathlineto{\pgfqpoint{3.676839in}{1.443796in}}%
\pgfpathlineto{\pgfqpoint{3.677032in}{1.443769in}}%
\pgfpathlineto{\pgfqpoint{3.677608in}{1.440572in}}%
\pgfpathlineto{\pgfqpoint{3.677416in}{1.444045in}}%
\pgfpathlineto{\pgfqpoint{3.678185in}{1.442201in}}%
\pgfpathlineto{\pgfqpoint{3.678377in}{1.443572in}}%
\pgfpathlineto{\pgfqpoint{3.678761in}{1.439748in}}%
\pgfpathlineto{\pgfqpoint{3.679145in}{1.439224in}}%
\pgfpathlineto{\pgfqpoint{3.679722in}{1.433275in}}%
\pgfpathlineto{\pgfqpoint{3.680106in}{1.436160in}}%
\pgfpathlineto{\pgfqpoint{3.680298in}{1.439262in}}%
\pgfpathlineto{\pgfqpoint{3.680683in}{1.432831in}}%
\pgfpathlineto{\pgfqpoint{3.681067in}{1.437372in}}%
\pgfpathlineto{\pgfqpoint{3.681644in}{1.437514in}}%
\pgfpathlineto{\pgfqpoint{3.683373in}{1.427260in}}%
\pgfpathlineto{\pgfqpoint{3.683565in}{1.428533in}}%
\pgfpathlineto{\pgfqpoint{3.683757in}{1.425934in}}%
\pgfpathlineto{\pgfqpoint{3.683950in}{1.426738in}}%
\pgfpathlineto{\pgfqpoint{3.684142in}{1.424056in}}%
\pgfpathlineto{\pgfqpoint{3.684718in}{1.430373in}}%
\pgfpathlineto{\pgfqpoint{3.684910in}{1.431539in}}%
\pgfpathlineto{\pgfqpoint{3.685295in}{1.427197in}}%
\pgfpathlineto{\pgfqpoint{3.685487in}{1.428663in}}%
\pgfpathlineto{\pgfqpoint{3.685679in}{1.428355in}}%
\pgfpathlineto{\pgfqpoint{3.685871in}{1.428677in}}%
\pgfpathlineto{\pgfqpoint{3.687216in}{1.434917in}}%
\pgfpathlineto{\pgfqpoint{3.687409in}{1.434472in}}%
\pgfpathlineto{\pgfqpoint{3.687793in}{1.436157in}}%
\pgfpathlineto{\pgfqpoint{3.688369in}{1.432646in}}%
\pgfpathlineto{\pgfqpoint{3.688946in}{1.438273in}}%
\pgfpathlineto{\pgfqpoint{3.689330in}{1.435294in}}%
\pgfpathlineto{\pgfqpoint{3.689715in}{1.431771in}}%
\pgfpathlineto{\pgfqpoint{3.689907in}{1.436330in}}%
\pgfpathlineto{\pgfqpoint{3.690483in}{1.434217in}}%
\pgfpathlineto{\pgfqpoint{3.691828in}{1.430710in}}%
\pgfpathlineto{\pgfqpoint{3.693366in}{1.439890in}}%
\pgfpathlineto{\pgfqpoint{3.693942in}{1.438249in}}%
\pgfpathlineto{\pgfqpoint{3.694134in}{1.440028in}}%
\pgfpathlineto{\pgfqpoint{3.694519in}{1.445497in}}%
\pgfpathlineto{\pgfqpoint{3.695095in}{1.439604in}}%
\pgfpathlineto{\pgfqpoint{3.695480in}{1.440876in}}%
\pgfpathlineto{\pgfqpoint{3.695672in}{1.439525in}}%
\pgfpathlineto{\pgfqpoint{3.697209in}{1.450881in}}%
\pgfpathlineto{\pgfqpoint{3.697593in}{1.449903in}}%
\pgfpathlineto{\pgfqpoint{3.698939in}{1.444974in}}%
\pgfpathlineto{\pgfqpoint{3.699323in}{1.448386in}}%
\pgfpathlineto{\pgfqpoint{3.699900in}{1.444435in}}%
\pgfpathlineto{\pgfqpoint{3.700092in}{1.445326in}}%
\pgfpathlineto{\pgfqpoint{3.700476in}{1.443716in}}%
\pgfpathlineto{\pgfqpoint{3.702206in}{1.435646in}}%
\pgfpathlineto{\pgfqpoint{3.703743in}{1.443350in}}%
\pgfpathlineto{\pgfqpoint{3.704319in}{1.439963in}}%
\pgfpathlineto{\pgfqpoint{3.705280in}{1.436508in}}%
\pgfpathlineto{\pgfqpoint{3.705472in}{1.438521in}}%
\pgfpathlineto{\pgfqpoint{3.706625in}{1.446989in}}%
\pgfpathlineto{\pgfqpoint{3.707202in}{1.446554in}}%
\pgfpathlineto{\pgfqpoint{3.709316in}{1.453551in}}%
\pgfpathlineto{\pgfqpoint{3.709508in}{1.451724in}}%
\pgfpathlineto{\pgfqpoint{3.709892in}{1.457777in}}%
\pgfpathlineto{\pgfqpoint{3.710084in}{1.456560in}}%
\pgfpathlineto{\pgfqpoint{3.711237in}{1.451790in}}%
\pgfpathlineto{\pgfqpoint{3.712006in}{1.452523in}}%
\pgfpathlineto{\pgfqpoint{3.712198in}{1.453195in}}%
\pgfpathlineto{\pgfqpoint{3.712390in}{1.449700in}}%
\pgfpathlineto{\pgfqpoint{3.713351in}{1.451643in}}%
\pgfpathlineto{\pgfqpoint{3.713736in}{1.448793in}}%
\pgfpathlineto{\pgfqpoint{3.713928in}{1.450252in}}%
\pgfpathlineto{\pgfqpoint{3.714120in}{1.455423in}}%
\pgfpathlineto{\pgfqpoint{3.714889in}{1.450516in}}%
\pgfpathlineto{\pgfqpoint{3.716810in}{1.443858in}}%
\pgfpathlineto{\pgfqpoint{3.717195in}{1.446966in}}%
\pgfpathlineto{\pgfqpoint{3.718540in}{1.452637in}}%
\pgfpathlineto{\pgfqpoint{3.718732in}{1.449226in}}%
\pgfpathlineto{\pgfqpoint{3.720077in}{1.445260in}}%
\pgfpathlineto{\pgfqpoint{3.720269in}{1.446153in}}%
\pgfpathlineto{\pgfqpoint{3.720461in}{1.443109in}}%
\pgfpathlineto{\pgfqpoint{3.721999in}{1.428591in}}%
\pgfpathlineto{\pgfqpoint{3.722191in}{1.431030in}}%
\pgfpathlineto{\pgfqpoint{3.723152in}{1.438141in}}%
\pgfpathlineto{\pgfqpoint{3.723536in}{1.436824in}}%
\pgfpathlineto{\pgfqpoint{3.724497in}{1.432428in}}%
\pgfpathlineto{\pgfqpoint{3.724689in}{1.432844in}}%
\pgfpathlineto{\pgfqpoint{3.724881in}{1.433757in}}%
\pgfpathlineto{\pgfqpoint{3.725074in}{1.430882in}}%
\pgfpathlineto{\pgfqpoint{3.725458in}{1.432203in}}%
\pgfpathlineto{\pgfqpoint{3.725650in}{1.430688in}}%
\pgfpathlineto{\pgfqpoint{3.725842in}{1.433404in}}%
\pgfpathlineto{\pgfqpoint{3.726034in}{1.436723in}}%
\pgfpathlineto{\pgfqpoint{3.726611in}{1.432040in}}%
\pgfpathlineto{\pgfqpoint{3.726803in}{1.432941in}}%
\pgfpathlineto{\pgfqpoint{3.728148in}{1.437019in}}%
\pgfpathlineto{\pgfqpoint{3.729686in}{1.426967in}}%
\pgfpathlineto{\pgfqpoint{3.729878in}{1.428415in}}%
\pgfpathlineto{\pgfqpoint{3.730070in}{1.428953in}}%
\pgfpathlineto{\pgfqpoint{3.730262in}{1.427853in}}%
\pgfpathlineto{\pgfqpoint{3.731223in}{1.419774in}}%
\pgfpathlineto{\pgfqpoint{3.732184in}{1.421975in}}%
\pgfpathlineto{\pgfqpoint{3.732376in}{1.421939in}}%
\pgfpathlineto{\pgfqpoint{3.732568in}{1.423044in}}%
\pgfpathlineto{\pgfqpoint{3.732760in}{1.421014in}}%
\pgfpathlineto{\pgfqpoint{3.732952in}{1.417746in}}%
\pgfpathlineto{\pgfqpoint{3.733529in}{1.423597in}}%
\pgfpathlineto{\pgfqpoint{3.733721in}{1.421702in}}%
\pgfpathlineto{\pgfqpoint{3.735835in}{1.416545in}}%
\pgfpathlineto{\pgfqpoint{3.736219in}{1.418068in}}%
\pgfpathlineto{\pgfqpoint{3.736604in}{1.421040in}}%
\pgfpathlineto{\pgfqpoint{3.737372in}{1.419943in}}%
\pgfpathlineto{\pgfqpoint{3.737564in}{1.420066in}}%
\pgfpathlineto{\pgfqpoint{3.739486in}{1.433837in}}%
\pgfpathlineto{\pgfqpoint{3.739678in}{1.433475in}}%
\pgfpathlineto{\pgfqpoint{3.740831in}{1.428981in}}%
\pgfpathlineto{\pgfqpoint{3.741023in}{1.432495in}}%
\pgfpathlineto{\pgfqpoint{3.741600in}{1.428911in}}%
\pgfpathlineto{\pgfqpoint{3.743522in}{1.412745in}}%
\pgfpathlineto{\pgfqpoint{3.744290in}{1.418731in}}%
\pgfpathlineto{\pgfqpoint{3.744867in}{1.416855in}}%
\pgfpathlineto{\pgfqpoint{3.746789in}{1.406757in}}%
\pgfpathlineto{\pgfqpoint{3.747749in}{1.410051in}}%
\pgfpathlineto{\pgfqpoint{3.747942in}{1.411350in}}%
\pgfpathlineto{\pgfqpoint{3.748134in}{1.408448in}}%
\pgfpathlineto{\pgfqpoint{3.748518in}{1.403613in}}%
\pgfpathlineto{\pgfqpoint{3.748902in}{1.410030in}}%
\pgfpathlineto{\pgfqpoint{3.749287in}{1.405897in}}%
\pgfpathlineto{\pgfqpoint{3.750440in}{1.402038in}}%
\pgfpathlineto{\pgfqpoint{3.749863in}{1.406072in}}%
\pgfpathlineto{\pgfqpoint{3.750632in}{1.402058in}}%
\pgfpathlineto{\pgfqpoint{3.750824in}{1.402314in}}%
\pgfpathlineto{\pgfqpoint{3.751593in}{1.409297in}}%
\pgfpathlineto{\pgfqpoint{3.752361in}{1.408491in}}%
\pgfpathlineto{\pgfqpoint{3.752554in}{1.408638in}}%
\pgfpathlineto{\pgfqpoint{3.755244in}{1.424246in}}%
\pgfpathlineto{\pgfqpoint{3.755820in}{1.422219in}}%
\pgfpathlineto{\pgfqpoint{3.756013in}{1.424348in}}%
\pgfpathlineto{\pgfqpoint{3.756397in}{1.427305in}}%
\pgfpathlineto{\pgfqpoint{3.757166in}{1.426368in}}%
\pgfpathlineto{\pgfqpoint{3.757742in}{1.423419in}}%
\pgfpathlineto{\pgfqpoint{3.758319in}{1.426192in}}%
\pgfpathlineto{\pgfqpoint{3.758703in}{1.429870in}}%
\pgfpathlineto{\pgfqpoint{3.759087in}{1.426314in}}%
\pgfpathlineto{\pgfqpoint{3.759856in}{1.419266in}}%
\pgfpathlineto{\pgfqpoint{3.760240in}{1.422472in}}%
\pgfpathlineto{\pgfqpoint{3.761393in}{1.434259in}}%
\pgfpathlineto{\pgfqpoint{3.761970in}{1.432815in}}%
\pgfpathlineto{\pgfqpoint{3.763315in}{1.429519in}}%
\pgfpathlineto{\pgfqpoint{3.763699in}{1.433789in}}%
\pgfpathlineto{\pgfqpoint{3.764276in}{1.431911in}}%
\pgfpathlineto{\pgfqpoint{3.766005in}{1.419484in}}%
\pgfpathlineto{\pgfqpoint{3.766390in}{1.420518in}}%
\pgfpathlineto{\pgfqpoint{3.768503in}{1.436302in}}%
\pgfpathlineto{\pgfqpoint{3.768696in}{1.436245in}}%
\pgfpathlineto{\pgfqpoint{3.768888in}{1.434948in}}%
\pgfpathlineto{\pgfqpoint{3.769080in}{1.439293in}}%
\pgfpathlineto{\pgfqpoint{3.769464in}{1.442356in}}%
\pgfpathlineto{\pgfqpoint{3.769849in}{1.437223in}}%
\pgfpathlineto{\pgfqpoint{3.770041in}{1.438291in}}%
\pgfpathlineto{\pgfqpoint{3.770617in}{1.436506in}}%
\pgfpathlineto{\pgfqpoint{3.771194in}{1.436750in}}%
\pgfpathlineto{\pgfqpoint{3.771770in}{1.436240in}}%
\pgfpathlineto{\pgfqpoint{3.772155in}{1.438536in}}%
\pgfpathlineto{\pgfqpoint{3.773884in}{1.430792in}}%
\pgfpathlineto{\pgfqpoint{3.774461in}{1.435400in}}%
\pgfpathlineto{\pgfqpoint{3.774845in}{1.434697in}}%
\pgfpathlineto{\pgfqpoint{3.775037in}{1.429358in}}%
\pgfpathlineto{\pgfqpoint{3.775806in}{1.435560in}}%
\pgfpathlineto{\pgfqpoint{3.777920in}{1.427039in}}%
\pgfpathlineto{\pgfqpoint{3.778304in}{1.428246in}}%
\pgfpathlineto{\pgfqpoint{3.778496in}{1.425948in}}%
\pgfpathlineto{\pgfqpoint{3.779073in}{1.427232in}}%
\pgfpathlineto{\pgfqpoint{3.780610in}{1.436079in}}%
\pgfpathlineto{\pgfqpoint{3.781571in}{1.431185in}}%
\pgfpathlineto{\pgfqpoint{3.781763in}{1.434404in}}%
\pgfpathlineto{\pgfqpoint{3.782916in}{1.442237in}}%
\pgfpathlineto{\pgfqpoint{3.783108in}{1.439288in}}%
\pgfpathlineto{\pgfqpoint{3.783300in}{1.438613in}}%
\pgfpathlineto{\pgfqpoint{3.783493in}{1.441034in}}%
\pgfpathlineto{\pgfqpoint{3.783685in}{1.439615in}}%
\pgfpathlineto{\pgfqpoint{3.784453in}{1.444103in}}%
\pgfpathlineto{\pgfqpoint{3.784646in}{1.441248in}}%
\pgfpathlineto{\pgfqpoint{3.785799in}{1.438235in}}%
\pgfpathlineto{\pgfqpoint{3.786375in}{1.442318in}}%
\pgfpathlineto{\pgfqpoint{3.786567in}{1.437429in}}%
\pgfpathlineto{\pgfqpoint{3.787336in}{1.432657in}}%
\pgfpathlineto{\pgfqpoint{3.787912in}{1.434046in}}%
\pgfpathlineto{\pgfqpoint{3.788105in}{1.433467in}}%
\pgfpathlineto{\pgfqpoint{3.788297in}{1.434920in}}%
\pgfpathlineto{\pgfqpoint{3.788873in}{1.439646in}}%
\pgfpathlineto{\pgfqpoint{3.789258in}{1.434584in}}%
\pgfpathlineto{\pgfqpoint{3.789450in}{1.435462in}}%
\pgfpathlineto{\pgfqpoint{3.790218in}{1.437984in}}%
\pgfpathlineto{\pgfqpoint{3.790603in}{1.436764in}}%
\pgfpathlineto{\pgfqpoint{3.790795in}{1.434332in}}%
\pgfpathlineto{\pgfqpoint{3.791564in}{1.438289in}}%
\pgfpathlineto{\pgfqpoint{3.793293in}{1.451281in}}%
\pgfpathlineto{\pgfqpoint{3.793485in}{1.450553in}}%
\pgfpathlineto{\pgfqpoint{3.793870in}{1.449036in}}%
\pgfpathlineto{\pgfqpoint{3.794062in}{1.451244in}}%
\pgfpathlineto{\pgfqpoint{3.794830in}{1.453708in}}%
\pgfpathlineto{\pgfqpoint{3.795023in}{1.453019in}}%
\pgfpathlineto{\pgfqpoint{3.796560in}{1.448315in}}%
\pgfpathlineto{\pgfqpoint{3.796944in}{1.448099in}}%
\pgfpathlineto{\pgfqpoint{3.797521in}{1.449397in}}%
\pgfpathlineto{\pgfqpoint{3.798674in}{1.444968in}}%
\pgfpathlineto{\pgfqpoint{3.798866in}{1.445772in}}%
\pgfpathlineto{\pgfqpoint{3.799058in}{1.445325in}}%
\pgfpathlineto{\pgfqpoint{3.799250in}{1.446225in}}%
\pgfpathlineto{\pgfqpoint{3.799827in}{1.445911in}}%
\pgfpathlineto{\pgfqpoint{3.800596in}{1.444114in}}%
\pgfpathlineto{\pgfqpoint{3.801172in}{1.450555in}}%
\pgfpathlineto{\pgfqpoint{3.801364in}{1.451148in}}%
\pgfpathlineto{\pgfqpoint{3.801556in}{1.449876in}}%
\pgfpathlineto{\pgfqpoint{3.801941in}{1.446374in}}%
\pgfpathlineto{\pgfqpoint{3.802325in}{1.451598in}}%
\pgfpathlineto{\pgfqpoint{3.802902in}{1.453929in}}%
\pgfpathlineto{\pgfqpoint{3.802709in}{1.450366in}}%
\pgfpathlineto{\pgfqpoint{3.803094in}{1.453400in}}%
\pgfpathlineto{\pgfqpoint{3.804247in}{1.457319in}}%
\pgfpathlineto{\pgfqpoint{3.804631in}{1.455287in}}%
\pgfpathlineto{\pgfqpoint{3.804823in}{1.459141in}}%
\pgfpathlineto{\pgfqpoint{3.805208in}{1.457115in}}%
\pgfpathlineto{\pgfqpoint{3.805400in}{1.457213in}}%
\pgfpathlineto{\pgfqpoint{3.805592in}{1.460842in}}%
\pgfpathlineto{\pgfqpoint{3.806168in}{1.457077in}}%
\pgfpathlineto{\pgfqpoint{3.806553in}{1.459837in}}%
\pgfpathlineto{\pgfqpoint{3.806745in}{1.459678in}}%
\pgfpathlineto{\pgfqpoint{3.807898in}{1.466584in}}%
\pgfpathlineto{\pgfqpoint{3.807321in}{1.458921in}}%
\pgfpathlineto{\pgfqpoint{3.808282in}{1.463253in}}%
\pgfpathlineto{\pgfqpoint{3.809051in}{1.457557in}}%
\pgfpathlineto{\pgfqpoint{3.809435in}{1.461074in}}%
\pgfpathlineto{\pgfqpoint{3.809627in}{1.460882in}}%
\pgfpathlineto{\pgfqpoint{3.810780in}{1.452431in}}%
\pgfpathlineto{\pgfqpoint{3.811165in}{1.452775in}}%
\pgfpathlineto{\pgfqpoint{3.811549in}{1.453274in}}%
\pgfpathlineto{\pgfqpoint{3.811741in}{1.451962in}}%
\pgfpathlineto{\pgfqpoint{3.812702in}{1.443371in}}%
\pgfpathlineto{\pgfqpoint{3.813471in}{1.444150in}}%
\pgfpathlineto{\pgfqpoint{3.814239in}{1.441603in}}%
\pgfpathlineto{\pgfqpoint{3.814432in}{1.441862in}}%
\pgfpathlineto{\pgfqpoint{3.815200in}{1.451665in}}%
\pgfpathlineto{\pgfqpoint{3.815969in}{1.447651in}}%
\pgfpathlineto{\pgfqpoint{3.817891in}{1.433492in}}%
\pgfpathlineto{\pgfqpoint{3.818083in}{1.433810in}}%
\pgfpathlineto{\pgfqpoint{3.818275in}{1.432541in}}%
\pgfpathlineto{\pgfqpoint{3.819620in}{1.426647in}}%
\pgfpathlineto{\pgfqpoint{3.818659in}{1.432776in}}%
\pgfpathlineto{\pgfqpoint{3.819812in}{1.427208in}}%
\pgfpathlineto{\pgfqpoint{3.820197in}{1.432962in}}%
\pgfpathlineto{\pgfqpoint{3.820773in}{1.426957in}}%
\pgfpathlineto{\pgfqpoint{3.821926in}{1.423904in}}%
\pgfpathlineto{\pgfqpoint{3.822695in}{1.424428in}}%
\pgfpathlineto{\pgfqpoint{3.822503in}{1.422184in}}%
\pgfpathlineto{\pgfqpoint{3.822887in}{1.423851in}}%
\pgfpathlineto{\pgfqpoint{3.824040in}{1.417011in}}%
\pgfpathlineto{\pgfqpoint{3.824424in}{1.418920in}}%
\pgfpathlineto{\pgfqpoint{3.824617in}{1.421740in}}%
\pgfpathlineto{\pgfqpoint{3.825385in}{1.418352in}}%
\pgfpathlineto{\pgfqpoint{3.826730in}{1.402288in}}%
\pgfpathlineto{\pgfqpoint{3.827883in}{1.402869in}}%
\pgfpathlineto{\pgfqpoint{3.828076in}{1.405419in}}%
\pgfpathlineto{\pgfqpoint{3.828652in}{1.401856in}}%
\pgfpathlineto{\pgfqpoint{3.828844in}{1.402193in}}%
\pgfpathlineto{\pgfqpoint{3.830189in}{1.394690in}}%
\pgfpathlineto{\pgfqpoint{3.831535in}{1.399541in}}%
\pgfpathlineto{\pgfqpoint{3.832495in}{1.394497in}}%
\pgfpathlineto{\pgfqpoint{3.833072in}{1.396130in}}%
\pgfpathlineto{\pgfqpoint{3.833264in}{1.397561in}}%
\pgfpathlineto{\pgfqpoint{3.833841in}{1.394535in}}%
\pgfpathlineto{\pgfqpoint{3.835378in}{1.391007in}}%
\pgfpathlineto{\pgfqpoint{3.835570in}{1.391632in}}%
\pgfpathlineto{\pgfqpoint{3.835762in}{1.389010in}}%
\pgfpathlineto{\pgfqpoint{3.835954in}{1.391058in}}%
\pgfpathlineto{\pgfqpoint{3.836147in}{1.389465in}}%
\pgfpathlineto{\pgfqpoint{3.836531in}{1.392131in}}%
\pgfpathlineto{\pgfqpoint{3.836723in}{1.392035in}}%
\pgfpathlineto{\pgfqpoint{3.837300in}{1.396442in}}%
\pgfpathlineto{\pgfqpoint{3.837876in}{1.394188in}}%
\pgfpathlineto{\pgfqpoint{3.838645in}{1.384450in}}%
\pgfpathlineto{\pgfqpoint{3.839221in}{1.386718in}}%
\pgfpathlineto{\pgfqpoint{3.839798in}{1.383509in}}%
\pgfpathlineto{\pgfqpoint{3.840759in}{1.385357in}}%
\pgfpathlineto{\pgfqpoint{3.841719in}{1.387870in}}%
\pgfpathlineto{\pgfqpoint{3.841335in}{1.384513in}}%
\pgfpathlineto{\pgfqpoint{3.841912in}{1.386985in}}%
\pgfpathlineto{\pgfqpoint{3.843065in}{1.381217in}}%
\pgfpathlineto{\pgfqpoint{3.843257in}{1.382339in}}%
\pgfpathlineto{\pgfqpoint{3.843833in}{1.385428in}}%
\pgfpathlineto{\pgfqpoint{3.845179in}{1.374769in}}%
\pgfpathlineto{\pgfqpoint{3.846524in}{1.378686in}}%
\pgfpathlineto{\pgfqpoint{3.846716in}{1.375457in}}%
\pgfpathlineto{\pgfqpoint{3.847485in}{1.380528in}}%
\pgfpathlineto{\pgfqpoint{3.847869in}{1.377739in}}%
\pgfpathlineto{\pgfqpoint{3.849214in}{1.368656in}}%
\pgfpathlineto{\pgfqpoint{3.850944in}{1.378085in}}%
\pgfpathlineto{\pgfqpoint{3.851136in}{1.378020in}}%
\pgfpathlineto{\pgfqpoint{3.852289in}{1.387867in}}%
\pgfpathlineto{\pgfqpoint{3.852673in}{1.387368in}}%
\pgfpathlineto{\pgfqpoint{3.852865in}{1.388848in}}%
\pgfpathlineto{\pgfqpoint{3.853250in}{1.385875in}}%
\pgfpathlineto{\pgfqpoint{3.853634in}{1.386625in}}%
\pgfpathlineto{\pgfqpoint{3.854018in}{1.386187in}}%
\pgfpathlineto{\pgfqpoint{3.854979in}{1.394417in}}%
\pgfpathlineto{\pgfqpoint{3.855171in}{1.396489in}}%
\pgfpathlineto{\pgfqpoint{3.855556in}{1.393214in}}%
\pgfpathlineto{\pgfqpoint{3.856132in}{1.386931in}}%
\pgfpathlineto{\pgfqpoint{3.856709in}{1.388788in}}%
\pgfpathlineto{\pgfqpoint{3.856901in}{1.389536in}}%
\pgfpathlineto{\pgfqpoint{3.857093in}{1.387619in}}%
\pgfpathlineto{\pgfqpoint{3.857669in}{1.384017in}}%
\pgfpathlineto{\pgfqpoint{3.858246in}{1.386639in}}%
\pgfpathlineto{\pgfqpoint{3.858822in}{1.388097in}}%
\pgfpathlineto{\pgfqpoint{3.859015in}{1.385767in}}%
\pgfpathlineto{\pgfqpoint{3.859591in}{1.388101in}}%
\pgfpathlineto{\pgfqpoint{3.860360in}{1.387235in}}%
\pgfpathlineto{\pgfqpoint{3.861321in}{1.384133in}}%
\pgfpathlineto{\pgfqpoint{3.861513in}{1.386494in}}%
\pgfpathlineto{\pgfqpoint{3.861897in}{1.386265in}}%
\pgfpathlineto{\pgfqpoint{3.862281in}{1.393959in}}%
\pgfpathlineto{\pgfqpoint{3.863434in}{1.391019in}}%
\pgfpathlineto{\pgfqpoint{3.864395in}{1.388905in}}%
\pgfpathlineto{\pgfqpoint{3.864587in}{1.390265in}}%
\pgfpathlineto{\pgfqpoint{3.865740in}{1.392996in}}%
\pgfpathlineto{\pgfqpoint{3.865356in}{1.389408in}}%
\pgfpathlineto{\pgfqpoint{3.865933in}{1.392394in}}%
\pgfpathlineto{\pgfqpoint{3.866317in}{1.389648in}}%
\pgfpathlineto{\pgfqpoint{3.866893in}{1.392606in}}%
\pgfpathlineto{\pgfqpoint{3.867086in}{1.392818in}}%
\pgfpathlineto{\pgfqpoint{3.867278in}{1.392015in}}%
\pgfpathlineto{\pgfqpoint{3.868815in}{1.380814in}}%
\pgfpathlineto{\pgfqpoint{3.869392in}{1.381815in}}%
\pgfpathlineto{\pgfqpoint{3.871313in}{1.389017in}}%
\pgfpathlineto{\pgfqpoint{3.871698in}{1.387599in}}%
\pgfpathlineto{\pgfqpoint{3.871890in}{1.385133in}}%
\pgfpathlineto{\pgfqpoint{3.872274in}{1.389645in}}%
\pgfpathlineto{\pgfqpoint{3.872851in}{1.385277in}}%
\pgfpathlineto{\pgfqpoint{3.874580in}{1.388577in}}%
\pgfpathlineto{\pgfqpoint{3.876694in}{1.374066in}}%
\pgfpathlineto{\pgfqpoint{3.877271in}{1.372474in}}%
\pgfpathlineto{\pgfqpoint{3.877463in}{1.375883in}}%
\pgfpathlineto{\pgfqpoint{3.877655in}{1.374059in}}%
\pgfpathlineto{\pgfqpoint{3.879769in}{1.384941in}}%
\pgfpathlineto{\pgfqpoint{3.880345in}{1.382112in}}%
\pgfpathlineto{\pgfqpoint{3.880537in}{1.385120in}}%
\pgfpathlineto{\pgfqpoint{3.880922in}{1.383806in}}%
\pgfpathlineto{\pgfqpoint{3.881306in}{1.386606in}}%
\pgfpathlineto{\pgfqpoint{3.881690in}{1.382417in}}%
\pgfpathlineto{\pgfqpoint{3.882075in}{1.381025in}}%
\pgfpathlineto{\pgfqpoint{3.882459in}{1.383842in}}%
\pgfpathlineto{\pgfqpoint{3.882843in}{1.381571in}}%
\pgfpathlineto{\pgfqpoint{3.883612in}{1.388341in}}%
\pgfpathlineto{\pgfqpoint{3.884189in}{1.387196in}}%
\pgfpathlineto{\pgfqpoint{3.885534in}{1.383459in}}%
\pgfpathlineto{\pgfqpoint{3.884957in}{1.387745in}}%
\pgfpathlineto{\pgfqpoint{3.886110in}{1.384007in}}%
\pgfpathlineto{\pgfqpoint{3.886495in}{1.376733in}}%
\pgfpathlineto{\pgfqpoint{3.887263in}{1.380142in}}%
\pgfpathlineto{\pgfqpoint{3.887840in}{1.383501in}}%
\pgfpathlineto{\pgfqpoint{3.888416in}{1.381967in}}%
\pgfpathlineto{\pgfqpoint{3.888608in}{1.382477in}}%
\pgfpathlineto{\pgfqpoint{3.888801in}{1.380977in}}%
\pgfpathlineto{\pgfqpoint{3.888993in}{1.379671in}}%
\pgfpathlineto{\pgfqpoint{3.889569in}{1.383173in}}%
\pgfpathlineto{\pgfqpoint{3.890338in}{1.388709in}}%
\pgfpathlineto{\pgfqpoint{3.890914in}{1.384225in}}%
\pgfpathlineto{\pgfqpoint{3.891491in}{1.383350in}}%
\pgfpathlineto{\pgfqpoint{3.891299in}{1.384613in}}%
\pgfpathlineto{\pgfqpoint{3.891875in}{1.383704in}}%
\pgfpathlineto{\pgfqpoint{3.893028in}{1.387192in}}%
\pgfpathlineto{\pgfqpoint{3.893413in}{1.385597in}}%
\pgfpathlineto{\pgfqpoint{3.893605in}{1.385053in}}%
\pgfpathlineto{\pgfqpoint{3.893797in}{1.386792in}}%
\pgfpathlineto{\pgfqpoint{3.893989in}{1.387346in}}%
\pgfpathlineto{\pgfqpoint{3.894181in}{1.384907in}}%
\pgfpathlineto{\pgfqpoint{3.894950in}{1.380392in}}%
\pgfpathlineto{\pgfqpoint{3.895334in}{1.383389in}}%
\pgfpathlineto{\pgfqpoint{3.895719in}{1.387864in}}%
\pgfpathlineto{\pgfqpoint{3.896487in}{1.384548in}}%
\pgfpathlineto{\pgfqpoint{3.897448in}{1.378436in}}%
\pgfpathlineto{\pgfqpoint{3.898217in}{1.379597in}}%
\pgfpathlineto{\pgfqpoint{3.898601in}{1.381867in}}%
\pgfpathlineto{\pgfqpoint{3.899178in}{1.378902in}}%
\pgfpathlineto{\pgfqpoint{3.901099in}{1.367237in}}%
\pgfpathlineto{\pgfqpoint{3.902445in}{1.374125in}}%
\pgfpathlineto{\pgfqpoint{3.902637in}{1.374022in}}%
\pgfpathlineto{\pgfqpoint{3.902829in}{1.370731in}}%
\pgfpathlineto{\pgfqpoint{3.903405in}{1.375675in}}%
\pgfpathlineto{\pgfqpoint{3.903598in}{1.373636in}}%
\pgfpathlineto{\pgfqpoint{3.904558in}{1.376929in}}%
\pgfpathlineto{\pgfqpoint{3.904174in}{1.373336in}}%
\pgfpathlineto{\pgfqpoint{3.904751in}{1.375180in}}%
\pgfpathlineto{\pgfqpoint{3.904943in}{1.376600in}}%
\pgfpathlineto{\pgfqpoint{3.905327in}{1.374409in}}%
\pgfpathlineto{\pgfqpoint{3.905519in}{1.375183in}}%
\pgfpathlineto{\pgfqpoint{3.906096in}{1.378539in}}%
\pgfpathlineto{\pgfqpoint{3.906864in}{1.372533in}}%
\pgfpathlineto{\pgfqpoint{3.907249in}{1.373365in}}%
\pgfpathlineto{\pgfqpoint{3.907441in}{1.371697in}}%
\pgfpathlineto{\pgfqpoint{3.908402in}{1.367387in}}%
\pgfpathlineto{\pgfqpoint{3.907825in}{1.372525in}}%
\pgfpathlineto{\pgfqpoint{3.908594in}{1.369170in}}%
\pgfpathlineto{\pgfqpoint{3.909170in}{1.372986in}}%
\pgfpathlineto{\pgfqpoint{3.909363in}{1.369761in}}%
\pgfpathlineto{\pgfqpoint{3.909555in}{1.367057in}}%
\pgfpathlineto{\pgfqpoint{3.910516in}{1.368502in}}%
\pgfpathlineto{\pgfqpoint{3.910708in}{1.372244in}}%
\pgfpathlineto{\pgfqpoint{3.911476in}{1.366755in}}%
\pgfpathlineto{\pgfqpoint{3.913014in}{1.372517in}}%
\pgfpathlineto{\pgfqpoint{3.913590in}{1.368428in}}%
\pgfpathlineto{\pgfqpoint{3.914167in}{1.370916in}}%
\pgfpathlineto{\pgfqpoint{3.915896in}{1.359017in}}%
\pgfpathlineto{\pgfqpoint{3.916281in}{1.360603in}}%
\pgfpathlineto{\pgfqpoint{3.917241in}{1.363399in}}%
\pgfpathlineto{\pgfqpoint{3.916857in}{1.359415in}}%
\pgfpathlineto{\pgfqpoint{3.918010in}{1.362892in}}%
\pgfpathlineto{\pgfqpoint{3.919548in}{1.352747in}}%
\pgfpathlineto{\pgfqpoint{3.919740in}{1.355000in}}%
\pgfpathlineto{\pgfqpoint{3.920701in}{1.354466in}}%
\pgfpathlineto{\pgfqpoint{3.921085in}{1.358027in}}%
\pgfpathlineto{\pgfqpoint{3.921277in}{1.358346in}}%
\pgfpathlineto{\pgfqpoint{3.921661in}{1.356133in}}%
\pgfpathlineto{\pgfqpoint{3.922238in}{1.359365in}}%
\pgfpathlineto{\pgfqpoint{3.922430in}{1.363719in}}%
\pgfpathlineto{\pgfqpoint{3.923199in}{1.359801in}}%
\pgfpathlineto{\pgfqpoint{3.923967in}{1.357030in}}%
\pgfpathlineto{\pgfqpoint{3.923583in}{1.360505in}}%
\pgfpathlineto{\pgfqpoint{3.924160in}{1.357519in}}%
\pgfpathlineto{\pgfqpoint{3.924736in}{1.362301in}}%
\pgfpathlineto{\pgfqpoint{3.925120in}{1.360013in}}%
\pgfpathlineto{\pgfqpoint{3.926081in}{1.356458in}}%
\pgfpathlineto{\pgfqpoint{3.926273in}{1.357980in}}%
\pgfpathlineto{\pgfqpoint{3.927234in}{1.351055in}}%
\pgfpathlineto{\pgfqpoint{3.928387in}{1.343676in}}%
\pgfpathlineto{\pgfqpoint{3.928579in}{1.343871in}}%
\pgfpathlineto{\pgfqpoint{3.929156in}{1.350186in}}%
\pgfpathlineto{\pgfqpoint{3.929925in}{1.347252in}}%
\pgfpathlineto{\pgfqpoint{3.931270in}{1.332826in}}%
\pgfpathlineto{\pgfqpoint{3.932038in}{1.336151in}}%
\pgfpathlineto{\pgfqpoint{3.932231in}{1.336335in}}%
\pgfpathlineto{\pgfqpoint{3.933191in}{1.328770in}}%
\pgfpathlineto{\pgfqpoint{3.933576in}{1.329618in}}%
\pgfpathlineto{\pgfqpoint{3.933960in}{1.329013in}}%
\pgfpathlineto{\pgfqpoint{3.934344in}{1.333270in}}%
\pgfpathlineto{\pgfqpoint{3.934537in}{1.334421in}}%
\pgfpathlineto{\pgfqpoint{3.934921in}{1.332726in}}%
\pgfpathlineto{\pgfqpoint{3.935305in}{1.333273in}}%
\pgfpathlineto{\pgfqpoint{3.935882in}{1.331417in}}%
\pgfpathlineto{\pgfqpoint{3.936074in}{1.332775in}}%
\pgfpathlineto{\pgfqpoint{3.936458in}{1.335205in}}%
\pgfpathlineto{\pgfqpoint{3.937035in}{1.332052in}}%
\pgfpathlineto{\pgfqpoint{3.937419in}{1.333173in}}%
\pgfpathlineto{\pgfqpoint{3.938956in}{1.346643in}}%
\pgfpathlineto{\pgfqpoint{3.937996in}{1.330869in}}%
\pgfpathlineto{\pgfqpoint{3.939533in}{1.343477in}}%
\pgfpathlineto{\pgfqpoint{3.940494in}{1.334075in}}%
\pgfpathlineto{\pgfqpoint{3.940878in}{1.339661in}}%
\pgfpathlineto{\pgfqpoint{3.941647in}{1.337468in}}%
\pgfpathlineto{\pgfqpoint{3.941262in}{1.340542in}}%
\pgfpathlineto{\pgfqpoint{3.941839in}{1.339863in}}%
\pgfpathlineto{\pgfqpoint{3.942031in}{1.340910in}}%
\pgfpathlineto{\pgfqpoint{3.942416in}{1.338555in}}%
\pgfpathlineto{\pgfqpoint{3.942800in}{1.340221in}}%
\pgfpathlineto{\pgfqpoint{3.944914in}{1.326131in}}%
\pgfpathlineto{\pgfqpoint{3.945106in}{1.326039in}}%
\pgfpathlineto{\pgfqpoint{3.945298in}{1.326511in}}%
\pgfpathlineto{\pgfqpoint{3.946643in}{1.323153in}}%
\pgfpathlineto{\pgfqpoint{3.946835in}{1.322791in}}%
\pgfpathlineto{\pgfqpoint{3.947028in}{1.323871in}}%
\pgfpathlineto{\pgfqpoint{3.948373in}{1.329879in}}%
\pgfpathlineto{\pgfqpoint{3.948565in}{1.329533in}}%
\pgfpathlineto{\pgfqpoint{3.948757in}{1.330035in}}%
\pgfpathlineto{\pgfqpoint{3.948949in}{1.331939in}}%
\pgfpathlineto{\pgfqpoint{3.949526in}{1.329816in}}%
\pgfpathlineto{\pgfqpoint{3.949718in}{1.325629in}}%
\pgfpathlineto{\pgfqpoint{3.950487in}{1.330573in}}%
\pgfpathlineto{\pgfqpoint{3.950679in}{1.328913in}}%
\pgfpathlineto{\pgfqpoint{3.950871in}{1.327426in}}%
\pgfpathlineto{\pgfqpoint{3.951447in}{1.330600in}}%
\pgfpathlineto{\pgfqpoint{3.951640in}{1.329289in}}%
\pgfpathlineto{\pgfqpoint{3.951832in}{1.329120in}}%
\pgfpathlineto{\pgfqpoint{3.952024in}{1.330826in}}%
\pgfpathlineto{\pgfqpoint{3.952408in}{1.326184in}}%
\pgfpathlineto{\pgfqpoint{3.952600in}{1.326170in}}%
\pgfpathlineto{\pgfqpoint{3.952793in}{1.326866in}}%
\pgfpathlineto{\pgfqpoint{3.952985in}{1.323405in}}%
\pgfpathlineto{\pgfqpoint{3.954906in}{1.309501in}}%
\pgfpathlineto{\pgfqpoint{3.955291in}{1.311228in}}%
\pgfpathlineto{\pgfqpoint{3.955483in}{1.308434in}}%
\pgfpathlineto{\pgfqpoint{3.955867in}{1.306276in}}%
\pgfpathlineto{\pgfqpoint{3.956252in}{1.310675in}}%
\pgfpathlineto{\pgfqpoint{3.957020in}{1.307826in}}%
\pgfpathlineto{\pgfqpoint{3.956636in}{1.311035in}}%
\pgfpathlineto{\pgfqpoint{3.957597in}{1.309333in}}%
\pgfpathlineto{\pgfqpoint{3.958173in}{1.315336in}}%
\pgfpathlineto{\pgfqpoint{3.958942in}{1.314478in}}%
\pgfpathlineto{\pgfqpoint{3.960479in}{1.323011in}}%
\pgfpathlineto{\pgfqpoint{3.961248in}{1.320586in}}%
\pgfpathlineto{\pgfqpoint{3.961632in}{1.318442in}}%
\pgfpathlineto{\pgfqpoint{3.962209in}{1.314675in}}%
\pgfpathlineto{\pgfqpoint{3.962593in}{1.318871in}}%
\pgfpathlineto{\pgfqpoint{3.962785in}{1.317303in}}%
\pgfpathlineto{\pgfqpoint{3.963170in}{1.320996in}}%
\pgfpathlineto{\pgfqpoint{3.963362in}{1.321753in}}%
\pgfpathlineto{\pgfqpoint{3.963554in}{1.320022in}}%
\pgfpathlineto{\pgfqpoint{3.964515in}{1.312515in}}%
\pgfpathlineto{\pgfqpoint{3.964899in}{1.313260in}}%
\pgfpathlineto{\pgfqpoint{3.965091in}{1.312720in}}%
\pgfpathlineto{\pgfqpoint{3.965283in}{1.314567in}}%
\pgfpathlineto{\pgfqpoint{3.965860in}{1.314152in}}%
\pgfpathlineto{\pgfqpoint{3.967013in}{1.320091in}}%
\pgfpathlineto{\pgfqpoint{3.967590in}{1.323714in}}%
\pgfpathlineto{\pgfqpoint{3.968166in}{1.320988in}}%
\pgfpathlineto{\pgfqpoint{3.970088in}{1.307821in}}%
\pgfpathlineto{\pgfqpoint{3.970280in}{1.310158in}}%
\pgfpathlineto{\pgfqpoint{3.970664in}{1.310096in}}%
\pgfpathlineto{\pgfqpoint{3.971049in}{1.314647in}}%
\pgfpathlineto{\pgfqpoint{3.971817in}{1.313124in}}%
\pgfpathlineto{\pgfqpoint{3.972009in}{1.309057in}}%
\pgfpathlineto{\pgfqpoint{3.972394in}{1.313854in}}%
\pgfpathlineto{\pgfqpoint{3.972970in}{1.310628in}}%
\pgfpathlineto{\pgfqpoint{3.973355in}{1.307706in}}%
\pgfpathlineto{\pgfqpoint{3.973739in}{1.302543in}}%
\pgfpathlineto{\pgfqpoint{3.974315in}{1.306227in}}%
\pgfpathlineto{\pgfqpoint{3.974892in}{1.309172in}}%
\pgfpathlineto{\pgfqpoint{3.975276in}{1.305052in}}%
\pgfpathlineto{\pgfqpoint{3.976429in}{1.300664in}}%
\pgfpathlineto{\pgfqpoint{3.977006in}{1.305248in}}%
\pgfpathlineto{\pgfqpoint{3.977390in}{1.301088in}}%
\pgfpathlineto{\pgfqpoint{3.977774in}{1.299451in}}%
\pgfpathlineto{\pgfqpoint{3.977967in}{1.302689in}}%
\pgfpathlineto{\pgfqpoint{3.979888in}{1.306548in}}%
\pgfpathlineto{\pgfqpoint{3.980849in}{1.300732in}}%
\pgfpathlineto{\pgfqpoint{3.981041in}{1.301875in}}%
\pgfpathlineto{\pgfqpoint{3.982002in}{1.312787in}}%
\pgfpathlineto{\pgfqpoint{3.982386in}{1.309861in}}%
\pgfpathlineto{\pgfqpoint{3.984692in}{1.296824in}}%
\pgfpathlineto{\pgfqpoint{3.984885in}{1.297745in}}%
\pgfpathlineto{\pgfqpoint{3.985077in}{1.301310in}}%
\pgfpathlineto{\pgfqpoint{3.985845in}{1.297613in}}%
\pgfpathlineto{\pgfqpoint{3.986038in}{1.298442in}}%
\pgfpathlineto{\pgfqpoint{3.986422in}{1.296055in}}%
\pgfpathlineto{\pgfqpoint{3.988344in}{1.286566in}}%
\pgfpathlineto{\pgfqpoint{3.988920in}{1.288110in}}%
\pgfpathlineto{\pgfqpoint{3.990265in}{1.295965in}}%
\pgfpathlineto{\pgfqpoint{3.990457in}{1.295465in}}%
\pgfpathlineto{\pgfqpoint{3.990650in}{1.297939in}}%
\pgfpathlineto{\pgfqpoint{3.991034in}{1.304465in}}%
\pgfpathlineto{\pgfqpoint{3.991418in}{1.296507in}}%
\pgfpathlineto{\pgfqpoint{3.991611in}{1.297336in}}%
\pgfpathlineto{\pgfqpoint{3.991803in}{1.294548in}}%
\pgfpathlineto{\pgfqpoint{3.992187in}{1.297720in}}%
\pgfpathlineto{\pgfqpoint{3.992764in}{1.296221in}}%
\pgfpathlineto{\pgfqpoint{3.993532in}{1.298331in}}%
\pgfpathlineto{\pgfqpoint{3.993148in}{1.295697in}}%
\pgfpathlineto{\pgfqpoint{3.993917in}{1.296405in}}%
\pgfpathlineto{\pgfqpoint{3.994109in}{1.296392in}}%
\pgfpathlineto{\pgfqpoint{3.994493in}{1.298380in}}%
\pgfpathlineto{\pgfqpoint{3.995070in}{1.296558in}}%
\pgfpathlineto{\pgfqpoint{3.995646in}{1.293730in}}%
\pgfpathlineto{\pgfqpoint{3.996223in}{1.294374in}}%
\pgfpathlineto{\pgfqpoint{3.996415in}{1.295651in}}%
\pgfpathlineto{\pgfqpoint{3.996607in}{1.293410in}}%
\pgfpathlineto{\pgfqpoint{3.996799in}{1.290720in}}%
\pgfpathlineto{\pgfqpoint{3.997183in}{1.297478in}}%
\pgfpathlineto{\pgfqpoint{3.997376in}{1.296057in}}%
\pgfpathlineto{\pgfqpoint{3.997760in}{1.295668in}}%
\pgfpathlineto{\pgfqpoint{3.998529in}{1.298131in}}%
\pgfpathlineto{\pgfqpoint{3.999105in}{1.295069in}}%
\pgfpathlineto{\pgfqpoint{3.999297in}{1.298699in}}%
\pgfpathlineto{\pgfqpoint{3.999489in}{1.298388in}}%
\pgfpathlineto{\pgfqpoint{3.999874in}{1.296515in}}%
\pgfpathlineto{\pgfqpoint{4.000066in}{1.298772in}}%
\pgfpathlineto{\pgfqpoint{4.000642in}{1.298619in}}%
\pgfpathlineto{\pgfqpoint{4.000835in}{1.298513in}}%
\pgfpathlineto{\pgfqpoint{4.002948in}{1.282709in}}%
\pgfpathlineto{\pgfqpoint{4.003717in}{1.288586in}}%
\pgfpathlineto{\pgfqpoint{4.004101in}{1.287157in}}%
\pgfpathlineto{\pgfqpoint{4.005447in}{1.277574in}}%
\pgfpathlineto{\pgfqpoint{4.005831in}{1.277603in}}%
\pgfpathlineto{\pgfqpoint{4.006407in}{1.275824in}}%
\pgfpathlineto{\pgfqpoint{4.006600in}{1.279368in}}%
\pgfpathlineto{\pgfqpoint{4.006984in}{1.277488in}}%
\pgfpathlineto{\pgfqpoint{4.007368in}{1.272718in}}%
\pgfpathlineto{\pgfqpoint{4.008137in}{1.276084in}}%
\pgfpathlineto{\pgfqpoint{4.008906in}{1.280746in}}%
\pgfpathlineto{\pgfqpoint{4.009098in}{1.276504in}}%
\pgfpathlineto{\pgfqpoint{4.009674in}{1.271285in}}%
\pgfpathlineto{\pgfqpoint{4.010251in}{1.275462in}}%
\pgfpathlineto{\pgfqpoint{4.011788in}{1.285807in}}%
\pgfpathlineto{\pgfqpoint{4.011980in}{1.287008in}}%
\pgfpathlineto{\pgfqpoint{4.012172in}{1.284690in}}%
\pgfpathlineto{\pgfqpoint{4.013133in}{1.285133in}}%
\pgfpathlineto{\pgfqpoint{4.013710in}{1.277290in}}%
\pgfpathlineto{\pgfqpoint{4.014094in}{1.281304in}}%
\pgfpathlineto{\pgfqpoint{4.014863in}{1.278860in}}%
\pgfpathlineto{\pgfqpoint{4.015055in}{1.277994in}}%
\pgfpathlineto{\pgfqpoint{4.015247in}{1.282204in}}%
\pgfpathlineto{\pgfqpoint{4.015824in}{1.286682in}}%
\pgfpathlineto{\pgfqpoint{4.016592in}{1.284858in}}%
\pgfpathlineto{\pgfqpoint{4.017169in}{1.285146in}}%
\pgfpathlineto{\pgfqpoint{4.018898in}{1.276824in}}%
\pgfpathlineto{\pgfqpoint{4.020051in}{1.273207in}}%
\pgfpathlineto{\pgfqpoint{4.019667in}{1.279102in}}%
\pgfpathlineto{\pgfqpoint{4.020244in}{1.273561in}}%
\pgfpathlineto{\pgfqpoint{4.020436in}{1.273157in}}%
\pgfpathlineto{\pgfqpoint{4.020628in}{1.274822in}}%
\pgfpathlineto{\pgfqpoint{4.021012in}{1.273778in}}%
\pgfpathlineto{\pgfqpoint{4.021204in}{1.274554in}}%
\pgfpathlineto{\pgfqpoint{4.021397in}{1.272771in}}%
\pgfpathlineto{\pgfqpoint{4.022165in}{1.273705in}}%
\pgfpathlineto{\pgfqpoint{4.023510in}{1.264007in}}%
\pgfpathlineto{\pgfqpoint{4.023703in}{1.263877in}}%
\pgfpathlineto{\pgfqpoint{4.024471in}{1.259938in}}%
\pgfpathlineto{\pgfqpoint{4.025048in}{1.260111in}}%
\pgfpathlineto{\pgfqpoint{4.027162in}{1.273450in}}%
\pgfpathlineto{\pgfqpoint{4.028122in}{1.281625in}}%
\pgfpathlineto{\pgfqpoint{4.028699in}{1.278812in}}%
\pgfpathlineto{\pgfqpoint{4.030236in}{1.268173in}}%
\pgfpathlineto{\pgfqpoint{4.032734in}{1.256844in}}%
\pgfpathlineto{\pgfqpoint{4.033503in}{1.260607in}}%
\pgfpathlineto{\pgfqpoint{4.033695in}{1.262808in}}%
\pgfpathlineto{\pgfqpoint{4.033887in}{1.259225in}}%
\pgfpathlineto{\pgfqpoint{4.034080in}{1.259571in}}%
\pgfpathlineto{\pgfqpoint{4.035233in}{1.254680in}}%
\pgfpathlineto{\pgfqpoint{4.035425in}{1.254885in}}%
\pgfpathlineto{\pgfqpoint{4.035617in}{1.254414in}}%
\pgfpathlineto{\pgfqpoint{4.035809in}{1.256848in}}%
\pgfpathlineto{\pgfqpoint{4.036962in}{1.267042in}}%
\pgfpathlineto{\pgfqpoint{4.037346in}{1.266653in}}%
\pgfpathlineto{\pgfqpoint{4.037539in}{1.266563in}}%
\pgfpathlineto{\pgfqpoint{4.038884in}{1.260705in}}%
\pgfpathlineto{\pgfqpoint{4.037923in}{1.267550in}}%
\pgfpathlineto{\pgfqpoint{4.039076in}{1.262236in}}%
\pgfpathlineto{\pgfqpoint{4.039460in}{1.265485in}}%
\pgfpathlineto{\pgfqpoint{4.039845in}{1.262165in}}%
\pgfpathlineto{\pgfqpoint{4.041382in}{1.254409in}}%
\pgfpathlineto{\pgfqpoint{4.041766in}{1.255781in}}%
\pgfpathlineto{\pgfqpoint{4.042151in}{1.259615in}}%
\pgfpathlineto{\pgfqpoint{4.042535in}{1.256282in}}%
\pgfpathlineto{\pgfqpoint{4.043880in}{1.253150in}}%
\pgfpathlineto{\pgfqpoint{4.044841in}{1.247283in}}%
\pgfpathlineto{\pgfqpoint{4.045418in}{1.248549in}}%
\pgfpathlineto{\pgfqpoint{4.046763in}{1.253834in}}%
\pgfpathlineto{\pgfqpoint{4.046955in}{1.254709in}}%
\pgfpathlineto{\pgfqpoint{4.047147in}{1.250605in}}%
\pgfpathlineto{\pgfqpoint{4.047339in}{1.251336in}}%
\pgfpathlineto{\pgfqpoint{4.047531in}{1.250565in}}%
\pgfpathlineto{\pgfqpoint{4.047916in}{1.253583in}}%
\pgfpathlineto{\pgfqpoint{4.048684in}{1.257499in}}%
\pgfpathlineto{\pgfqpoint{4.049069in}{1.257300in}}%
\pgfpathlineto{\pgfqpoint{4.049645in}{1.254716in}}%
\pgfpathlineto{\pgfqpoint{4.050990in}{1.249455in}}%
\pgfpathlineto{\pgfqpoint{4.050030in}{1.257581in}}%
\pgfpathlineto{\pgfqpoint{4.051183in}{1.249564in}}%
\pgfpathlineto{\pgfqpoint{4.051567in}{1.250321in}}%
\pgfpathlineto{\pgfqpoint{4.051759in}{1.251853in}}%
\pgfpathlineto{\pgfqpoint{4.052143in}{1.249827in}}%
\pgfpathlineto{\pgfqpoint{4.052336in}{1.251717in}}%
\pgfpathlineto{\pgfqpoint{4.052528in}{1.248981in}}%
\pgfpathlineto{\pgfqpoint{4.052912in}{1.252252in}}%
\pgfpathlineto{\pgfqpoint{4.053296in}{1.252246in}}%
\pgfpathlineto{\pgfqpoint{4.054449in}{1.258215in}}%
\pgfpathlineto{\pgfqpoint{4.054834in}{1.257919in}}%
\pgfpathlineto{\pgfqpoint{4.055410in}{1.259673in}}%
\pgfpathlineto{\pgfqpoint{4.055987in}{1.254951in}}%
\pgfpathlineto{\pgfqpoint{4.056371in}{1.257716in}}%
\pgfpathlineto{\pgfqpoint{4.056948in}{1.254672in}}%
\pgfpathlineto{\pgfqpoint{4.058677in}{1.245052in}}%
\pgfpathlineto{\pgfqpoint{4.059061in}{1.243060in}}%
\pgfpathlineto{\pgfqpoint{4.059638in}{1.235716in}}%
\pgfpathlineto{\pgfqpoint{4.060407in}{1.237511in}}%
\pgfpathlineto{\pgfqpoint{4.060791in}{1.238557in}}%
\pgfpathlineto{\pgfqpoint{4.060983in}{1.236606in}}%
\pgfpathlineto{\pgfqpoint{4.061367in}{1.237644in}}%
\pgfpathlineto{\pgfqpoint{4.061752in}{1.236713in}}%
\pgfpathlineto{\pgfqpoint{4.063097in}{1.222172in}}%
\pgfpathlineto{\pgfqpoint{4.063481in}{1.227708in}}%
\pgfpathlineto{\pgfqpoint{4.063673in}{1.227878in}}%
\pgfpathlineto{\pgfqpoint{4.064058in}{1.230425in}}%
\pgfpathlineto{\pgfqpoint{4.064442in}{1.226821in}}%
\pgfpathlineto{\pgfqpoint{4.065403in}{1.223620in}}%
\pgfpathlineto{\pgfqpoint{4.065595in}{1.225763in}}%
\pgfpathlineto{\pgfqpoint{4.066364in}{1.223891in}}%
\pgfpathlineto{\pgfqpoint{4.065980in}{1.228419in}}%
\pgfpathlineto{\pgfqpoint{4.066748in}{1.225403in}}%
\pgfpathlineto{\pgfqpoint{4.067133in}{1.227525in}}%
\pgfpathlineto{\pgfqpoint{4.067517in}{1.226787in}}%
\pgfpathlineto{\pgfqpoint{4.068670in}{1.219364in}}%
\pgfpathlineto{\pgfqpoint{4.068862in}{1.222166in}}%
\pgfpathlineto{\pgfqpoint{4.069054in}{1.221864in}}%
\pgfpathlineto{\pgfqpoint{4.069246in}{1.223025in}}%
\pgfpathlineto{\pgfqpoint{4.069823in}{1.218565in}}%
\pgfpathlineto{\pgfqpoint{4.070207in}{1.224425in}}%
\pgfpathlineto{\pgfqpoint{4.071552in}{1.230141in}}%
\pgfpathlineto{\pgfqpoint{4.073090in}{1.223229in}}%
\pgfpathlineto{\pgfqpoint{4.073474in}{1.225303in}}%
\pgfpathlineto{\pgfqpoint{4.074243in}{1.236035in}}%
\pgfpathlineto{\pgfqpoint{4.074819in}{1.230854in}}%
\pgfpathlineto{\pgfqpoint{4.075204in}{1.229335in}}%
\pgfpathlineto{\pgfqpoint{4.077125in}{1.220206in}}%
\pgfpathlineto{\pgfqpoint{4.077317in}{1.220867in}}%
\pgfpathlineto{\pgfqpoint{4.077894in}{1.223801in}}%
\pgfpathlineto{\pgfqpoint{4.078086in}{1.222149in}}%
\pgfpathlineto{\pgfqpoint{4.080392in}{1.204509in}}%
\pgfpathlineto{\pgfqpoint{4.080584in}{1.204830in}}%
\pgfpathlineto{\pgfqpoint{4.081161in}{1.212373in}}%
\pgfpathlineto{\pgfqpoint{4.081929in}{1.211555in}}%
\pgfpathlineto{\pgfqpoint{4.082122in}{1.213687in}}%
\pgfpathlineto{\pgfqpoint{4.082890in}{1.211137in}}%
\pgfpathlineto{\pgfqpoint{4.083082in}{1.210866in}}%
\pgfpathlineto{\pgfqpoint{4.083275in}{1.211202in}}%
\pgfpathlineto{\pgfqpoint{4.083467in}{1.206608in}}%
\pgfpathlineto{\pgfqpoint{4.084428in}{1.208505in}}%
\pgfpathlineto{\pgfqpoint{4.087502in}{1.218442in}}%
\pgfpathlineto{\pgfqpoint{4.087694in}{1.216181in}}%
\pgfpathlineto{\pgfqpoint{4.088463in}{1.209126in}}%
\pgfpathlineto{\pgfqpoint{4.089040in}{1.210576in}}%
\pgfpathlineto{\pgfqpoint{4.090577in}{1.215575in}}%
\pgfpathlineto{\pgfqpoint{4.092114in}{1.220948in}}%
\pgfpathlineto{\pgfqpoint{4.092307in}{1.219195in}}%
\pgfpathlineto{\pgfqpoint{4.092499in}{1.216819in}}%
\pgfpathlineto{\pgfqpoint{4.093267in}{1.217061in}}%
\pgfpathlineto{\pgfqpoint{4.093652in}{1.222322in}}%
\pgfpathlineto{\pgfqpoint{4.094420in}{1.219620in}}%
\pgfpathlineto{\pgfqpoint{4.094805in}{1.217590in}}%
\pgfpathlineto{\pgfqpoint{4.095189in}{1.221028in}}%
\pgfpathlineto{\pgfqpoint{4.095573in}{1.218972in}}%
\pgfpathlineto{\pgfqpoint{4.095766in}{1.219766in}}%
\pgfpathlineto{\pgfqpoint{4.095958in}{1.216801in}}%
\pgfpathlineto{\pgfqpoint{4.097111in}{1.213528in}}%
\pgfpathlineto{\pgfqpoint{4.097303in}{1.216309in}}%
\pgfpathlineto{\pgfqpoint{4.098264in}{1.215467in}}%
\pgfpathlineto{\pgfqpoint{4.098456in}{1.215833in}}%
\pgfpathlineto{\pgfqpoint{4.099225in}{1.221329in}}%
\pgfpathlineto{\pgfqpoint{4.099417in}{1.216839in}}%
\pgfpathlineto{\pgfqpoint{4.100185in}{1.215236in}}%
\pgfpathlineto{\pgfqpoint{4.100762in}{1.216022in}}%
\pgfpathlineto{\pgfqpoint{4.101146in}{1.216801in}}%
\pgfpathlineto{\pgfqpoint{4.101338in}{1.214556in}}%
\pgfpathlineto{\pgfqpoint{4.102299in}{1.209081in}}%
\pgfpathlineto{\pgfqpoint{4.102684in}{1.210173in}}%
\pgfpathlineto{\pgfqpoint{4.103260in}{1.213028in}}%
\pgfpathlineto{\pgfqpoint{4.103644in}{1.208882in}}%
\pgfpathlineto{\pgfqpoint{4.104029in}{1.210792in}}%
\pgfpathlineto{\pgfqpoint{4.104605in}{1.207829in}}%
\pgfpathlineto{\pgfqpoint{4.104797in}{1.207847in}}%
\pgfpathlineto{\pgfqpoint{4.106143in}{1.213957in}}%
\pgfpathlineto{\pgfqpoint{4.106719in}{1.209696in}}%
\pgfpathlineto{\pgfqpoint{4.107296in}{1.212685in}}%
\pgfpathlineto{\pgfqpoint{4.107680in}{1.208536in}}%
\pgfpathlineto{\pgfqpoint{4.108064in}{1.215089in}}%
\pgfpathlineto{\pgfqpoint{4.108449in}{1.218804in}}%
\pgfpathlineto{\pgfqpoint{4.109217in}{1.217971in}}%
\pgfpathlineto{\pgfqpoint{4.110178in}{1.212606in}}%
\pgfpathlineto{\pgfqpoint{4.110370in}{1.213786in}}%
\pgfpathlineto{\pgfqpoint{4.111523in}{1.221061in}}%
\pgfpathlineto{\pgfqpoint{4.111908in}{1.217031in}}%
\pgfpathlineto{\pgfqpoint{4.112484in}{1.220663in}}%
\pgfpathlineto{\pgfqpoint{4.112676in}{1.220651in}}%
\pgfpathlineto{\pgfqpoint{4.113253in}{1.213664in}}%
\pgfpathlineto{\pgfqpoint{4.113637in}{1.221447in}}%
\pgfpathlineto{\pgfqpoint{4.114214in}{1.223134in}}%
\pgfpathlineto{\pgfqpoint{4.114790in}{1.221895in}}%
\pgfpathlineto{\pgfqpoint{4.115559in}{1.218510in}}%
\pgfpathlineto{\pgfqpoint{4.115943in}{1.218910in}}%
\pgfpathlineto{\pgfqpoint{4.116520in}{1.222412in}}%
\pgfpathlineto{\pgfqpoint{4.116904in}{1.218538in}}%
\pgfpathlineto{\pgfqpoint{4.117288in}{1.216391in}}%
\pgfpathlineto{\pgfqpoint{4.117673in}{1.218958in}}%
\pgfpathlineto{\pgfqpoint{4.119018in}{1.221826in}}%
\pgfpathlineto{\pgfqpoint{4.119210in}{1.222404in}}%
\pgfpathlineto{\pgfqpoint{4.119979in}{1.213937in}}%
\pgfpathlineto{\pgfqpoint{4.120363in}{1.215700in}}%
\pgfpathlineto{\pgfqpoint{4.120747in}{1.222594in}}%
\pgfpathlineto{\pgfqpoint{4.121516in}{1.219996in}}%
\pgfpathlineto{\pgfqpoint{4.122093in}{1.215328in}}%
\pgfpathlineto{\pgfqpoint{4.122477in}{1.217589in}}%
\pgfpathlineto{\pgfqpoint{4.124591in}{1.239143in}}%
\pgfpathlineto{\pgfqpoint{4.124975in}{1.238336in}}%
\pgfpathlineto{\pgfqpoint{4.126512in}{1.231046in}}%
\pgfpathlineto{\pgfqpoint{4.126897in}{1.231503in}}%
\pgfpathlineto{\pgfqpoint{4.127089in}{1.231383in}}%
\pgfpathlineto{\pgfqpoint{4.128818in}{1.223769in}}%
\pgfpathlineto{\pgfqpoint{4.130740in}{1.230216in}}%
\pgfpathlineto{\pgfqpoint{4.130932in}{1.230631in}}%
\pgfpathlineto{\pgfqpoint{4.131124in}{1.230022in}}%
\pgfpathlineto{\pgfqpoint{4.132277in}{1.228264in}}%
\pgfpathlineto{\pgfqpoint{4.132470in}{1.230879in}}%
\pgfpathlineto{\pgfqpoint{4.133238in}{1.229580in}}%
\pgfpathlineto{\pgfqpoint{4.133623in}{1.226411in}}%
\pgfpathlineto{\pgfqpoint{4.134199in}{1.227755in}}%
\pgfpathlineto{\pgfqpoint{4.134391in}{1.231288in}}%
\pgfpathlineto{\pgfqpoint{4.135352in}{1.229001in}}%
\pgfpathlineto{\pgfqpoint{4.136697in}{1.223215in}}%
\pgfpathlineto{\pgfqpoint{4.138235in}{1.232097in}}%
\pgfpathlineto{\pgfqpoint{4.139964in}{1.225437in}}%
\pgfpathlineto{\pgfqpoint{4.138619in}{1.233895in}}%
\pgfpathlineto{\pgfqpoint{4.140156in}{1.228434in}}%
\pgfpathlineto{\pgfqpoint{4.140349in}{1.230339in}}%
\pgfpathlineto{\pgfqpoint{4.140733in}{1.225026in}}%
\pgfpathlineto{\pgfqpoint{4.140925in}{1.226955in}}%
\pgfpathlineto{\pgfqpoint{4.143231in}{1.217734in}}%
\pgfpathlineto{\pgfqpoint{4.143808in}{1.221456in}}%
\pgfpathlineto{\pgfqpoint{4.144192in}{1.220825in}}%
\pgfpathlineto{\pgfqpoint{4.145537in}{1.215245in}}%
\pgfpathlineto{\pgfqpoint{4.146114in}{1.217752in}}%
\pgfpathlineto{\pgfqpoint{4.146882in}{1.216649in}}%
\pgfpathlineto{\pgfqpoint{4.147651in}{1.213299in}}%
\pgfpathlineto{\pgfqpoint{4.147843in}{1.214344in}}%
\pgfpathlineto{\pgfqpoint{4.148612in}{1.213585in}}%
\pgfpathlineto{\pgfqpoint{4.149188in}{1.218607in}}%
\pgfpathlineto{\pgfqpoint{4.149765in}{1.216112in}}%
\pgfpathlineto{\pgfqpoint{4.149573in}{1.219347in}}%
\pgfpathlineto{\pgfqpoint{4.149957in}{1.218774in}}%
\pgfpathlineto{\pgfqpoint{4.150149in}{1.220739in}}%
\pgfpathlineto{\pgfqpoint{4.150918in}{1.217848in}}%
\pgfpathlineto{\pgfqpoint{4.151302in}{1.220128in}}%
\pgfpathlineto{\pgfqpoint{4.153800in}{1.208090in}}%
\pgfpathlineto{\pgfqpoint{4.153992in}{1.209809in}}%
\pgfpathlineto{\pgfqpoint{4.154377in}{1.205518in}}%
\pgfpathlineto{\pgfqpoint{4.154761in}{1.201645in}}%
\pgfpathlineto{\pgfqpoint{4.154953in}{1.197106in}}%
\pgfpathlineto{\pgfqpoint{4.155722in}{1.201994in}}%
\pgfpathlineto{\pgfqpoint{4.155914in}{1.200289in}}%
\pgfpathlineto{\pgfqpoint{4.156683in}{1.197707in}}%
\pgfpathlineto{\pgfqpoint{4.157067in}{1.198558in}}%
\pgfpathlineto{\pgfqpoint{4.157259in}{1.198691in}}%
\pgfpathlineto{\pgfqpoint{4.157451in}{1.198088in}}%
\pgfpathlineto{\pgfqpoint{4.159565in}{1.183974in}}%
\pgfpathlineto{\pgfqpoint{4.159950in}{1.185604in}}%
\pgfpathlineto{\pgfqpoint{4.160910in}{1.190393in}}%
\pgfpathlineto{\pgfqpoint{4.161295in}{1.190180in}}%
\pgfpathlineto{\pgfqpoint{4.162063in}{1.189388in}}%
\pgfpathlineto{\pgfqpoint{4.162832in}{1.187765in}}%
\pgfpathlineto{\pgfqpoint{4.163217in}{1.188075in}}%
\pgfpathlineto{\pgfqpoint{4.163985in}{1.187455in}}%
\pgfpathlineto{\pgfqpoint{4.164754in}{1.191714in}}%
\pgfpathlineto{\pgfqpoint{4.166291in}{1.196293in}}%
\pgfpathlineto{\pgfqpoint{4.167252in}{1.199279in}}%
\pgfpathlineto{\pgfqpoint{4.166868in}{1.195494in}}%
\pgfpathlineto{\pgfqpoint{4.167444in}{1.197108in}}%
\pgfpathlineto{\pgfqpoint{4.167636in}{1.197142in}}%
\pgfpathlineto{\pgfqpoint{4.168021in}{1.200686in}}%
\pgfpathlineto{\pgfqpoint{4.168405in}{1.195542in}}%
\pgfpathlineto{\pgfqpoint{4.170135in}{1.184163in}}%
\pgfpathlineto{\pgfqpoint{4.171480in}{1.196735in}}%
\pgfpathlineto{\pgfqpoint{4.172248in}{1.192624in}}%
\pgfpathlineto{\pgfqpoint{4.172441in}{1.190316in}}%
\pgfpathlineto{\pgfqpoint{4.173401in}{1.190719in}}%
\pgfpathlineto{\pgfqpoint{4.173786in}{1.193571in}}%
\pgfpathlineto{\pgfqpoint{4.173978in}{1.190761in}}%
\pgfpathlineto{\pgfqpoint{4.175707in}{1.184257in}}%
\pgfpathlineto{\pgfqpoint{4.174362in}{1.191178in}}%
\pgfpathlineto{\pgfqpoint{4.175900in}{1.186099in}}%
\pgfpathlineto{\pgfqpoint{4.177053in}{1.192594in}}%
\pgfpathlineto{\pgfqpoint{4.177437in}{1.192041in}}%
\pgfpathlineto{\pgfqpoint{4.177821in}{1.190679in}}%
\pgfpathlineto{\pgfqpoint{4.178398in}{1.194728in}}%
\pgfpathlineto{\pgfqpoint{4.178974in}{1.197014in}}%
\pgfpathlineto{\pgfqpoint{4.179359in}{1.194819in}}%
\pgfpathlineto{\pgfqpoint{4.181088in}{1.183629in}}%
\pgfpathlineto{\pgfqpoint{4.181472in}{1.186660in}}%
\pgfpathlineto{\pgfqpoint{4.182049in}{1.189308in}}%
\pgfpathlineto{\pgfqpoint{4.182433in}{1.186494in}}%
\pgfpathlineto{\pgfqpoint{4.183394in}{1.186981in}}%
\pgfpathlineto{\pgfqpoint{4.183778in}{1.182717in}}%
\pgfpathlineto{\pgfqpoint{4.184931in}{1.188506in}}%
\pgfpathlineto{\pgfqpoint{4.186277in}{1.180110in}}%
\pgfpathlineto{\pgfqpoint{4.187045in}{1.181168in}}%
\pgfpathlineto{\pgfqpoint{4.186661in}{1.178830in}}%
\pgfpathlineto{\pgfqpoint{4.187238in}{1.180807in}}%
\pgfpathlineto{\pgfqpoint{4.187430in}{1.180225in}}%
\pgfpathlineto{\pgfqpoint{4.187814in}{1.182544in}}%
\pgfpathlineto{\pgfqpoint{4.189351in}{1.187992in}}%
\pgfpathlineto{\pgfqpoint{4.189736in}{1.186684in}}%
\pgfpathlineto{\pgfqpoint{4.190312in}{1.187903in}}%
\pgfpathlineto{\pgfqpoint{4.190889in}{1.192327in}}%
\pgfpathlineto{\pgfqpoint{4.191273in}{1.189359in}}%
\pgfpathlineto{\pgfqpoint{4.192426in}{1.179607in}}%
\pgfpathlineto{\pgfqpoint{4.192618in}{1.182039in}}%
\pgfpathlineto{\pgfqpoint{4.193195in}{1.189253in}}%
\pgfpathlineto{\pgfqpoint{4.193771in}{1.185943in}}%
\pgfpathlineto{\pgfqpoint{4.194156in}{1.185251in}}%
\pgfpathlineto{\pgfqpoint{4.194732in}{1.187734in}}%
\pgfpathlineto{\pgfqpoint{4.194924in}{1.185071in}}%
\pgfpathlineto{\pgfqpoint{4.195501in}{1.187954in}}%
\pgfpathlineto{\pgfqpoint{4.196077in}{1.191214in}}%
\pgfpathlineto{\pgfqpoint{4.196654in}{1.190973in}}%
\pgfpathlineto{\pgfqpoint{4.197807in}{1.183045in}}%
\pgfpathlineto{\pgfqpoint{4.197999in}{1.183619in}}%
\pgfpathlineto{\pgfqpoint{4.198191in}{1.182951in}}%
\pgfpathlineto{\pgfqpoint{4.198575in}{1.185609in}}%
\pgfpathlineto{\pgfqpoint{4.198768in}{1.185078in}}%
\pgfpathlineto{\pgfqpoint{4.199152in}{1.186889in}}%
\pgfpathlineto{\pgfqpoint{4.200881in}{1.199225in}}%
\pgfpathlineto{\pgfqpoint{4.201458in}{1.195106in}}%
\pgfpathlineto{\pgfqpoint{4.201842in}{1.198172in}}%
\pgfpathlineto{\pgfqpoint{4.202227in}{1.199308in}}%
\pgfpathlineto{\pgfqpoint{4.202803in}{1.197354in}}%
\pgfpathlineto{\pgfqpoint{4.202995in}{1.197120in}}%
\pgfpathlineto{\pgfqpoint{4.203380in}{1.201962in}}%
\pgfpathlineto{\pgfqpoint{4.204148in}{1.199725in}}%
\pgfpathlineto{\pgfqpoint{4.204533in}{1.197091in}}%
\pgfpathlineto{\pgfqpoint{4.204917in}{1.200383in}}%
\pgfpathlineto{\pgfqpoint{4.205493in}{1.198167in}}%
\pgfpathlineto{\pgfqpoint{4.206839in}{1.211961in}}%
\pgfpathlineto{\pgfqpoint{4.207415in}{1.208444in}}%
\pgfpathlineto{\pgfqpoint{4.208568in}{1.202620in}}%
\pgfpathlineto{\pgfqpoint{4.208760in}{1.204702in}}%
\pgfpathlineto{\pgfqpoint{4.209145in}{1.201203in}}%
\pgfpathlineto{\pgfqpoint{4.209529in}{1.205069in}}%
\pgfpathlineto{\pgfqpoint{4.209913in}{1.203464in}}%
\pgfpathlineto{\pgfqpoint{4.210105in}{1.205003in}}%
\pgfpathlineto{\pgfqpoint{4.210490in}{1.201965in}}%
\pgfpathlineto{\pgfqpoint{4.212027in}{1.194256in}}%
\pgfpathlineto{\pgfqpoint{4.212219in}{1.195299in}}%
\pgfpathlineto{\pgfqpoint{4.212796in}{1.203693in}}%
\pgfpathlineto{\pgfqpoint{4.213757in}{1.203088in}}%
\pgfpathlineto{\pgfqpoint{4.214718in}{1.196896in}}%
\pgfpathlineto{\pgfqpoint{4.215102in}{1.199731in}}%
\pgfpathlineto{\pgfqpoint{4.215486in}{1.201702in}}%
\pgfpathlineto{\pgfqpoint{4.215678in}{1.201075in}}%
\pgfpathlineto{\pgfqpoint{4.215871in}{1.196924in}}%
\pgfpathlineto{\pgfqpoint{4.216639in}{1.199944in}}%
\pgfpathlineto{\pgfqpoint{4.217600in}{1.205945in}}%
\pgfpathlineto{\pgfqpoint{4.217984in}{1.204931in}}%
\pgfpathlineto{\pgfqpoint{4.219522in}{1.195526in}}%
\pgfpathlineto{\pgfqpoint{4.220290in}{1.198505in}}%
\pgfpathlineto{\pgfqpoint{4.220675in}{1.197632in}}%
\pgfpathlineto{\pgfqpoint{4.221636in}{1.191179in}}%
\pgfpathlineto{\pgfqpoint{4.222212in}{1.195162in}}%
\pgfpathlineto{\pgfqpoint{4.222404in}{1.195316in}}%
\pgfpathlineto{\pgfqpoint{4.222596in}{1.194331in}}%
\pgfpathlineto{\pgfqpoint{4.222789in}{1.193860in}}%
\pgfpathlineto{\pgfqpoint{4.222981in}{1.193977in}}%
\pgfpathlineto{\pgfqpoint{4.223365in}{1.198852in}}%
\pgfpathlineto{\pgfqpoint{4.224134in}{1.198247in}}%
\pgfpathlineto{\pgfqpoint{4.224326in}{1.196982in}}%
\pgfpathlineto{\pgfqpoint{4.224710in}{1.201317in}}%
\pgfpathlineto{\pgfqpoint{4.225095in}{1.199681in}}%
\pgfpathlineto{\pgfqpoint{4.225287in}{1.200638in}}%
\pgfpathlineto{\pgfqpoint{4.226440in}{1.211212in}}%
\pgfpathlineto{\pgfqpoint{4.227208in}{1.209756in}}%
\pgfpathlineto{\pgfqpoint{4.227401in}{1.209916in}}%
\pgfpathlineto{\pgfqpoint{4.227785in}{1.216500in}}%
\pgfpathlineto{\pgfqpoint{4.228554in}{1.216119in}}%
\pgfpathlineto{\pgfqpoint{4.229514in}{1.210558in}}%
\pgfpathlineto{\pgfqpoint{4.229899in}{1.213143in}}%
\pgfpathlineto{\pgfqpoint{4.230091in}{1.214764in}}%
\pgfpathlineto{\pgfqpoint{4.230667in}{1.210970in}}%
\pgfpathlineto{\pgfqpoint{4.232397in}{1.205455in}}%
\pgfpathlineto{\pgfqpoint{4.231052in}{1.212351in}}%
\pgfpathlineto{\pgfqpoint{4.232589in}{1.206334in}}%
\pgfpathlineto{\pgfqpoint{4.232781in}{1.207946in}}%
\pgfpathlineto{\pgfqpoint{4.233550in}{1.205897in}}%
\pgfpathlineto{\pgfqpoint{4.233934in}{1.209888in}}%
\pgfpathlineto{\pgfqpoint{4.235087in}{1.208871in}}%
\pgfpathlineto{\pgfqpoint{4.236625in}{1.199752in}}%
\pgfpathlineto{\pgfqpoint{4.237009in}{1.202494in}}%
\pgfpathlineto{\pgfqpoint{4.237201in}{1.202664in}}%
\pgfpathlineto{\pgfqpoint{4.237586in}{1.200362in}}%
\pgfpathlineto{\pgfqpoint{4.237970in}{1.201007in}}%
\pgfpathlineto{\pgfqpoint{4.238354in}{1.205759in}}%
\pgfpathlineto{\pgfqpoint{4.238739in}{1.200862in}}%
\pgfpathlineto{\pgfqpoint{4.239123in}{1.202387in}}%
\pgfpathlineto{\pgfqpoint{4.239315in}{1.203208in}}%
\pgfpathlineto{\pgfqpoint{4.239507in}{1.202125in}}%
\pgfpathlineto{\pgfqpoint{4.240852in}{1.192651in}}%
\pgfpathlineto{\pgfqpoint{4.241045in}{1.192694in}}%
\pgfpathlineto{\pgfqpoint{4.241813in}{1.196553in}}%
\pgfpathlineto{\pgfqpoint{4.241429in}{1.191156in}}%
\pgfpathlineto{\pgfqpoint{4.242390in}{1.195049in}}%
\pgfpathlineto{\pgfqpoint{4.243543in}{1.185622in}}%
\pgfpathlineto{\pgfqpoint{4.243927in}{1.186227in}}%
\pgfpathlineto{\pgfqpoint{4.244696in}{1.193301in}}%
\pgfpathlineto{\pgfqpoint{4.245272in}{1.190254in}}%
\pgfpathlineto{\pgfqpoint{4.245464in}{1.192133in}}%
\pgfpathlineto{\pgfqpoint{4.245657in}{1.187971in}}%
\pgfpathlineto{\pgfqpoint{4.247578in}{1.171814in}}%
\pgfpathlineto{\pgfqpoint{4.247963in}{1.169800in}}%
\pgfpathlineto{\pgfqpoint{4.248923in}{1.175980in}}%
\pgfpathlineto{\pgfqpoint{4.249884in}{1.169378in}}%
\pgfpathlineto{\pgfqpoint{4.250269in}{1.171259in}}%
\pgfpathlineto{\pgfqpoint{4.250461in}{1.171085in}}%
\pgfpathlineto{\pgfqpoint{4.250653in}{1.171856in}}%
\pgfpathlineto{\pgfqpoint{4.251229in}{1.175497in}}%
\pgfpathlineto{\pgfqpoint{4.251614in}{1.174212in}}%
\pgfpathlineto{\pgfqpoint{4.253151in}{1.167731in}}%
\pgfpathlineto{\pgfqpoint{4.254881in}{1.157969in}}%
\pgfpathlineto{\pgfqpoint{4.256802in}{1.177281in}}%
\pgfpathlineto{\pgfqpoint{4.256994in}{1.177156in}}%
\pgfpathlineto{\pgfqpoint{4.257379in}{1.172410in}}%
\pgfpathlineto{\pgfqpoint{4.257955in}{1.178411in}}%
\pgfpathlineto{\pgfqpoint{4.258147in}{1.181424in}}%
\pgfpathlineto{\pgfqpoint{4.258724in}{1.176796in}}%
\pgfpathlineto{\pgfqpoint{4.258916in}{1.177819in}}%
\pgfpathlineto{\pgfqpoint{4.259108in}{1.177548in}}%
\pgfpathlineto{\pgfqpoint{4.259877in}{1.181732in}}%
\pgfpathlineto{\pgfqpoint{4.260261in}{1.179662in}}%
\pgfpathlineto{\pgfqpoint{4.261799in}{1.173602in}}%
\pgfpathlineto{\pgfqpoint{4.261991in}{1.174651in}}%
\pgfpathlineto{\pgfqpoint{4.262567in}{1.177635in}}%
\pgfpathlineto{\pgfqpoint{4.262952in}{1.173934in}}%
\pgfpathlineto{\pgfqpoint{4.264105in}{1.168512in}}%
\pgfpathlineto{\pgfqpoint{4.264489in}{1.170449in}}%
\pgfpathlineto{\pgfqpoint{4.266411in}{1.176410in}}%
\pgfpathlineto{\pgfqpoint{4.266987in}{1.178217in}}%
\pgfpathlineto{\pgfqpoint{4.267564in}{1.173968in}}%
\pgfpathlineto{\pgfqpoint{4.268332in}{1.178841in}}%
\pgfpathlineto{\pgfqpoint{4.268717in}{1.177476in}}%
\pgfpathlineto{\pgfqpoint{4.270446in}{1.168073in}}%
\pgfpathlineto{\pgfqpoint{4.270638in}{1.171199in}}%
\pgfpathlineto{\pgfqpoint{4.271215in}{1.167615in}}%
\pgfpathlineto{\pgfqpoint{4.271407in}{1.168301in}}%
\pgfpathlineto{\pgfqpoint{4.272176in}{1.166786in}}%
\pgfpathlineto{\pgfqpoint{4.272368in}{1.168329in}}%
\pgfpathlineto{\pgfqpoint{4.272560in}{1.168464in}}%
\pgfpathlineto{\pgfqpoint{4.272944in}{1.171808in}}%
\pgfpathlineto{\pgfqpoint{4.273329in}{1.166237in}}%
\pgfpathlineto{\pgfqpoint{4.273521in}{1.166300in}}%
\pgfpathlineto{\pgfqpoint{4.273713in}{1.169544in}}%
\pgfpathlineto{\pgfqpoint{4.274290in}{1.165721in}}%
\pgfpathlineto{\pgfqpoint{4.274482in}{1.167241in}}%
\pgfpathlineto{\pgfqpoint{4.276980in}{1.146673in}}%
\pgfpathlineto{\pgfqpoint{4.277364in}{1.148772in}}%
\pgfpathlineto{\pgfqpoint{4.277749in}{1.147472in}}%
\pgfpathlineto{\pgfqpoint{4.279286in}{1.136259in}}%
\pgfpathlineto{\pgfqpoint{4.279478in}{1.133419in}}%
\pgfpathlineto{\pgfqpoint{4.280055in}{1.139000in}}%
\pgfpathlineto{\pgfqpoint{4.281015in}{1.142321in}}%
\pgfpathlineto{\pgfqpoint{4.281208in}{1.140056in}}%
\pgfpathlineto{\pgfqpoint{4.281400in}{1.139841in}}%
\pgfpathlineto{\pgfqpoint{4.281592in}{1.140453in}}%
\pgfpathlineto{\pgfqpoint{4.281976in}{1.139158in}}%
\pgfpathlineto{\pgfqpoint{4.283321in}{1.146233in}}%
\pgfpathlineto{\pgfqpoint{4.283514in}{1.146871in}}%
\pgfpathlineto{\pgfqpoint{4.283706in}{1.146009in}}%
\pgfpathlineto{\pgfqpoint{4.283898in}{1.146593in}}%
\pgfpathlineto{\pgfqpoint{4.284282in}{1.141858in}}%
\pgfpathlineto{\pgfqpoint{4.284667in}{1.149847in}}%
\pgfpathlineto{\pgfqpoint{4.286588in}{1.157951in}}%
\pgfpathlineto{\pgfqpoint{4.286781in}{1.157542in}}%
\pgfpathlineto{\pgfqpoint{4.288702in}{1.148307in}}%
\pgfpathlineto{\pgfqpoint{4.289471in}{1.153502in}}%
\pgfpathlineto{\pgfqpoint{4.290432in}{1.162103in}}%
\pgfpathlineto{\pgfqpoint{4.290624in}{1.160345in}}%
\pgfpathlineto{\pgfqpoint{4.291008in}{1.163431in}}%
\pgfpathlineto{\pgfqpoint{4.291585in}{1.162765in}}%
\pgfpathlineto{\pgfqpoint{4.292738in}{1.156370in}}%
\pgfpathlineto{\pgfqpoint{4.292930in}{1.156822in}}%
\pgfpathlineto{\pgfqpoint{4.293314in}{1.159414in}}%
\pgfpathlineto{\pgfqpoint{4.293506in}{1.156260in}}%
\pgfpathlineto{\pgfqpoint{4.293699in}{1.156971in}}%
\pgfpathlineto{\pgfqpoint{4.294467in}{1.153175in}}%
\pgfpathlineto{\pgfqpoint{4.294852in}{1.156175in}}%
\pgfpathlineto{\pgfqpoint{4.295044in}{1.156591in}}%
\pgfpathlineto{\pgfqpoint{4.296005in}{1.162732in}}%
\pgfpathlineto{\pgfqpoint{4.296965in}{1.155602in}}%
\pgfpathlineto{\pgfqpoint{4.297158in}{1.157321in}}%
\pgfpathlineto{\pgfqpoint{4.297926in}{1.155145in}}%
\pgfpathlineto{\pgfqpoint{4.298695in}{1.162436in}}%
\pgfpathlineto{\pgfqpoint{4.299079in}{1.164319in}}%
\pgfpathlineto{\pgfqpoint{4.299464in}{1.163255in}}%
\pgfpathlineto{\pgfqpoint{4.300809in}{1.155094in}}%
\pgfpathlineto{\pgfqpoint{4.301001in}{1.154524in}}%
\pgfpathlineto{\pgfqpoint{4.301193in}{1.156013in}}%
\pgfpathlineto{\pgfqpoint{4.301385in}{1.161098in}}%
\pgfpathlineto{\pgfqpoint{4.302154in}{1.155441in}}%
\pgfpathlineto{\pgfqpoint{4.303115in}{1.151370in}}%
\pgfpathlineto{\pgfqpoint{4.303883in}{1.159521in}}%
\pgfpathlineto{\pgfqpoint{4.304268in}{1.156173in}}%
\pgfpathlineto{\pgfqpoint{4.305613in}{1.152131in}}%
\pgfpathlineto{\pgfqpoint{4.305805in}{1.153615in}}%
\pgfpathlineto{\pgfqpoint{4.306574in}{1.157515in}}%
\pgfpathlineto{\pgfqpoint{4.306766in}{1.155398in}}%
\pgfpathlineto{\pgfqpoint{4.307342in}{1.152311in}}%
\pgfpathlineto{\pgfqpoint{4.307727in}{1.156491in}}%
\pgfpathlineto{\pgfqpoint{4.308111in}{1.152708in}}%
\pgfpathlineto{\pgfqpoint{4.308495in}{1.153119in}}%
\pgfpathlineto{\pgfqpoint{4.309264in}{1.151156in}}%
\pgfpathlineto{\pgfqpoint{4.309649in}{1.152557in}}%
\pgfpathlineto{\pgfqpoint{4.310033in}{1.149942in}}%
\pgfpathlineto{\pgfqpoint{4.310225in}{1.150821in}}%
\pgfpathlineto{\pgfqpoint{4.311570in}{1.155004in}}%
\pgfpathlineto{\pgfqpoint{4.312531in}{1.145684in}}%
\pgfpathlineto{\pgfqpoint{4.313108in}{1.145777in}}%
\pgfpathlineto{\pgfqpoint{4.313300in}{1.146646in}}%
\pgfpathlineto{\pgfqpoint{4.313492in}{1.144069in}}%
\pgfpathlineto{\pgfqpoint{4.313684in}{1.143196in}}%
\pgfpathlineto{\pgfqpoint{4.314068in}{1.143857in}}%
\pgfpathlineto{\pgfqpoint{4.315221in}{1.149386in}}%
\pgfpathlineto{\pgfqpoint{4.316951in}{1.140693in}}%
\pgfpathlineto{\pgfqpoint{4.317143in}{1.141246in}}%
\pgfpathlineto{\pgfqpoint{4.318296in}{1.137107in}}%
\pgfpathlineto{\pgfqpoint{4.317527in}{1.143649in}}%
\pgfpathlineto{\pgfqpoint{4.318488in}{1.137535in}}%
\pgfpathlineto{\pgfqpoint{4.318873in}{1.140927in}}%
\pgfpathlineto{\pgfqpoint{4.319257in}{1.137141in}}%
\pgfpathlineto{\pgfqpoint{4.319641in}{1.139503in}}%
\pgfpathlineto{\pgfqpoint{4.320986in}{1.134167in}}%
\pgfpathlineto{\pgfqpoint{4.321947in}{1.142909in}}%
\pgfpathlineto{\pgfqpoint{4.322716in}{1.140740in}}%
\pgfpathlineto{\pgfqpoint{4.323677in}{1.133481in}}%
\pgfpathlineto{\pgfqpoint{4.324445in}{1.133948in}}%
\pgfpathlineto{\pgfqpoint{4.325791in}{1.137477in}}%
\pgfpathlineto{\pgfqpoint{4.326175in}{1.136320in}}%
\pgfpathlineto{\pgfqpoint{4.326367in}{1.136715in}}%
\pgfpathlineto{\pgfqpoint{4.326559in}{1.135468in}}%
\pgfpathlineto{\pgfqpoint{4.327328in}{1.134477in}}%
\pgfpathlineto{\pgfqpoint{4.327712in}{1.134892in}}%
\pgfpathlineto{\pgfqpoint{4.327904in}{1.135563in}}%
\pgfpathlineto{\pgfqpoint{4.328289in}{1.133404in}}%
\pgfpathlineto{\pgfqpoint{4.328481in}{1.133890in}}%
\pgfpathlineto{\pgfqpoint{4.328673in}{1.131568in}}%
\pgfpathlineto{\pgfqpoint{4.329057in}{1.134153in}}%
\pgfpathlineto{\pgfqpoint{4.329442in}{1.131994in}}%
\pgfpathlineto{\pgfqpoint{4.329634in}{1.136071in}}%
\pgfpathlineto{\pgfqpoint{4.330595in}{1.134443in}}%
\pgfpathlineto{\pgfqpoint{4.331363in}{1.130039in}}%
\pgfpathlineto{\pgfqpoint{4.331748in}{1.131822in}}%
\pgfpathlineto{\pgfqpoint{4.333477in}{1.148784in}}%
\pgfpathlineto{\pgfqpoint{4.333670in}{1.148613in}}%
\pgfpathlineto{\pgfqpoint{4.333862in}{1.146440in}}%
\pgfpathlineto{\pgfqpoint{4.334438in}{1.146804in}}%
\pgfpathlineto{\pgfqpoint{4.334630in}{1.150638in}}%
\pgfpathlineto{\pgfqpoint{4.335399in}{1.147827in}}%
\pgfpathlineto{\pgfqpoint{4.337705in}{1.134597in}}%
\pgfpathlineto{\pgfqpoint{4.337897in}{1.135126in}}%
\pgfpathlineto{\pgfqpoint{4.339819in}{1.147174in}}%
\pgfpathlineto{\pgfqpoint{4.340011in}{1.145903in}}%
\pgfpathlineto{\pgfqpoint{4.340203in}{1.143326in}}%
\pgfpathlineto{\pgfqpoint{4.340780in}{1.149234in}}%
\pgfpathlineto{\pgfqpoint{4.341164in}{1.145363in}}%
\pgfpathlineto{\pgfqpoint{4.344815in}{1.132924in}}%
\pgfpathlineto{\pgfqpoint{4.341548in}{1.145866in}}%
\pgfpathlineto{\pgfqpoint{4.345392in}{1.134260in}}%
\pgfpathlineto{\pgfqpoint{4.345584in}{1.134315in}}%
\pgfpathlineto{\pgfqpoint{4.351349in}{1.098185in}}%
\pgfpathlineto{\pgfqpoint{4.351541in}{1.099168in}}%
\pgfpathlineto{\pgfqpoint{4.351925in}{1.103075in}}%
\pgfpathlineto{\pgfqpoint{4.352502in}{1.099760in}}%
\pgfpathlineto{\pgfqpoint{4.352694in}{1.099134in}}%
\pgfpathlineto{\pgfqpoint{4.353078in}{1.094019in}}%
\pgfpathlineto{\pgfqpoint{4.353655in}{1.097336in}}%
\pgfpathlineto{\pgfqpoint{4.354424in}{1.103065in}}%
\pgfpathlineto{\pgfqpoint{4.354808in}{1.101517in}}%
\pgfpathlineto{\pgfqpoint{4.356153in}{1.088859in}}%
\pgfpathlineto{\pgfqpoint{4.356345in}{1.091153in}}%
\pgfpathlineto{\pgfqpoint{4.358267in}{1.096952in}}%
\pgfpathlineto{\pgfqpoint{4.356730in}{1.089983in}}%
\pgfpathlineto{\pgfqpoint{4.358651in}{1.094388in}}%
\pgfpathlineto{\pgfqpoint{4.358844in}{1.093148in}}%
\pgfpathlineto{\pgfqpoint{4.359612in}{1.095101in}}%
\pgfpathlineto{\pgfqpoint{4.359804in}{1.095137in}}%
\pgfpathlineto{\pgfqpoint{4.361342in}{1.108444in}}%
\pgfpathlineto{\pgfqpoint{4.361534in}{1.107513in}}%
\pgfpathlineto{\pgfqpoint{4.361726in}{1.105376in}}%
\pgfpathlineto{\pgfqpoint{4.362110in}{1.108998in}}%
\pgfpathlineto{\pgfqpoint{4.362495in}{1.106653in}}%
\pgfpathlineto{\pgfqpoint{4.364224in}{1.115853in}}%
\pgfpathlineto{\pgfqpoint{4.364416in}{1.112478in}}%
\pgfpathlineto{\pgfqpoint{4.364993in}{1.117737in}}%
\pgfpathlineto{\pgfqpoint{4.365954in}{1.126648in}}%
\pgfpathlineto{\pgfqpoint{4.366338in}{1.125279in}}%
\pgfpathlineto{\pgfqpoint{4.366530in}{1.123028in}}%
\pgfpathlineto{\pgfqpoint{4.366915in}{1.125516in}}%
\pgfpathlineto{\pgfqpoint{4.367299in}{1.123506in}}%
\pgfpathlineto{\pgfqpoint{4.367491in}{1.126508in}}%
\pgfpathlineto{\pgfqpoint{4.367875in}{1.120553in}}%
\pgfpathlineto{\pgfqpoint{4.368068in}{1.122081in}}%
\pgfpathlineto{\pgfqpoint{4.368452in}{1.118696in}}%
\pgfpathlineto{\pgfqpoint{4.368836in}{1.123675in}}%
\pgfpathlineto{\pgfqpoint{4.369028in}{1.124967in}}%
\pgfpathlineto{\pgfqpoint{4.369797in}{1.122660in}}%
\pgfpathlineto{\pgfqpoint{4.371527in}{1.117242in}}%
\pgfpathlineto{\pgfqpoint{4.370181in}{1.124700in}}%
\pgfpathlineto{\pgfqpoint{4.371719in}{1.118130in}}%
\pgfpathlineto{\pgfqpoint{4.371911in}{1.118294in}}%
\pgfpathlineto{\pgfqpoint{4.372295in}{1.115437in}}%
\pgfpathlineto{\pgfqpoint{4.373064in}{1.115990in}}%
\pgfpathlineto{\pgfqpoint{4.374409in}{1.131961in}}%
\pgfpathlineto{\pgfqpoint{4.374793in}{1.130011in}}%
\pgfpathlineto{\pgfqpoint{4.376139in}{1.133688in}}%
\pgfpathlineto{\pgfqpoint{4.377676in}{1.124414in}}%
\pgfpathlineto{\pgfqpoint{4.377868in}{1.124791in}}%
\pgfpathlineto{\pgfqpoint{4.378252in}{1.122818in}}%
\pgfpathlineto{\pgfqpoint{4.379405in}{1.114215in}}%
\pgfpathlineto{\pgfqpoint{4.379982in}{1.115602in}}%
\pgfpathlineto{\pgfqpoint{4.380174in}{1.114418in}}%
\pgfpathlineto{\pgfqpoint{4.380751in}{1.116599in}}%
\pgfpathlineto{\pgfqpoint{4.381135in}{1.119769in}}%
\pgfpathlineto{\pgfqpoint{4.381711in}{1.117592in}}%
\pgfpathlineto{\pgfqpoint{4.382096in}{1.114858in}}%
\pgfpathlineto{\pgfqpoint{4.382288in}{1.117699in}}%
\pgfpathlineto{\pgfqpoint{4.382865in}{1.121698in}}%
\pgfpathlineto{\pgfqpoint{4.383441in}{1.120823in}}%
\pgfpathlineto{\pgfqpoint{4.385171in}{1.112268in}}%
\pgfpathlineto{\pgfqpoint{4.385363in}{1.108093in}}%
\pgfpathlineto{\pgfqpoint{4.386131in}{1.114507in}}%
\pgfpathlineto{\pgfqpoint{4.386324in}{1.114508in}}%
\pgfpathlineto{\pgfqpoint{4.387092in}{1.103292in}}%
\pgfpathlineto{\pgfqpoint{4.387669in}{1.106538in}}%
\pgfpathlineto{\pgfqpoint{4.387861in}{1.107004in}}%
\pgfpathlineto{\pgfqpoint{4.388053in}{1.104404in}}%
\pgfpathlineto{\pgfqpoint{4.388630in}{1.108977in}}%
\pgfpathlineto{\pgfqpoint{4.389014in}{1.115309in}}%
\pgfpathlineto{\pgfqpoint{4.389590in}{1.108432in}}%
\pgfpathlineto{\pgfqpoint{4.389783in}{1.108868in}}%
\pgfpathlineto{\pgfqpoint{4.389975in}{1.107456in}}%
\pgfpathlineto{\pgfqpoint{4.390167in}{1.105437in}}%
\pgfpathlineto{\pgfqpoint{4.390551in}{1.107879in}}%
\pgfpathlineto{\pgfqpoint{4.390743in}{1.111176in}}%
\pgfpathlineto{\pgfqpoint{4.391704in}{1.110023in}}%
\pgfpathlineto{\pgfqpoint{4.393242in}{1.099467in}}%
\pgfpathlineto{\pgfqpoint{4.393434in}{1.099925in}}%
\pgfpathlineto{\pgfqpoint{4.393818in}{1.104517in}}%
\pgfpathlineto{\pgfqpoint{4.394587in}{1.101487in}}%
\pgfpathlineto{\pgfqpoint{4.394971in}{1.106121in}}%
\pgfpathlineto{\pgfqpoint{4.395548in}{1.099697in}}%
\pgfpathlineto{\pgfqpoint{4.395740in}{1.102338in}}%
\pgfpathlineto{\pgfqpoint{4.395932in}{1.103619in}}%
\pgfpathlineto{\pgfqpoint{4.396701in}{1.101297in}}%
\pgfpathlineto{\pgfqpoint{4.397085in}{1.101886in}}%
\pgfpathlineto{\pgfqpoint{4.397661in}{1.098510in}}%
\pgfpathlineto{\pgfqpoint{4.398814in}{1.106725in}}%
\pgfpathlineto{\pgfqpoint{4.399007in}{1.102617in}}%
\pgfpathlineto{\pgfqpoint{4.399391in}{1.104842in}}%
\pgfpathlineto{\pgfqpoint{4.399775in}{1.099882in}}%
\pgfpathlineto{\pgfqpoint{4.400736in}{1.107324in}}%
\pgfpathlineto{\pgfqpoint{4.400928in}{1.105680in}}%
\pgfpathlineto{\pgfqpoint{4.402081in}{1.100213in}}%
\pgfpathlineto{\pgfqpoint{4.402273in}{1.101551in}}%
\pgfpathlineto{\pgfqpoint{4.403811in}{1.107459in}}%
\pgfpathlineto{\pgfqpoint{4.404003in}{1.105220in}}%
\pgfpathlineto{\pgfqpoint{4.406309in}{1.096797in}}%
\pgfpathlineto{\pgfqpoint{4.407270in}{1.102441in}}%
\pgfpathlineto{\pgfqpoint{4.407462in}{1.101226in}}%
\pgfpathlineto{\pgfqpoint{4.407846in}{1.097757in}}%
\pgfpathlineto{\pgfqpoint{4.408231in}{1.103529in}}%
\pgfpathlineto{\pgfqpoint{4.408999in}{1.107934in}}%
\pgfpathlineto{\pgfqpoint{4.409384in}{1.105228in}}%
\pgfpathlineto{\pgfqpoint{4.409576in}{1.104195in}}%
\pgfpathlineto{\pgfqpoint{4.409960in}{1.106360in}}%
\pgfpathlineto{\pgfqpoint{4.410345in}{1.109369in}}%
\pgfpathlineto{\pgfqpoint{4.410537in}{1.106159in}}%
\pgfpathlineto{\pgfqpoint{4.411113in}{1.100197in}}%
\pgfpathlineto{\pgfqpoint{4.411690in}{1.104232in}}%
\pgfpathlineto{\pgfqpoint{4.411882in}{1.107141in}}%
\pgfpathlineto{\pgfqpoint{4.412266in}{1.103247in}}%
\pgfpathlineto{\pgfqpoint{4.412651in}{1.104904in}}%
\pgfpathlineto{\pgfqpoint{4.413227in}{1.106960in}}%
\pgfpathlineto{\pgfqpoint{4.413996in}{1.100082in}}%
\pgfpathlineto{\pgfqpoint{4.415341in}{1.107103in}}%
\pgfpathlineto{\pgfqpoint{4.415533in}{1.105629in}}%
\pgfpathlineto{\pgfqpoint{4.416110in}{1.108447in}}%
\pgfpathlineto{\pgfqpoint{4.416878in}{1.109612in}}%
\pgfpathlineto{\pgfqpoint{4.417070in}{1.109104in}}%
\pgfpathlineto{\pgfqpoint{4.418031in}{1.101707in}}%
\pgfpathlineto{\pgfqpoint{4.418416in}{1.104803in}}%
\pgfpathlineto{\pgfqpoint{4.418608in}{1.106883in}}%
\pgfpathlineto{\pgfqpoint{4.419184in}{1.104307in}}%
\pgfpathlineto{\pgfqpoint{4.419376in}{1.101620in}}%
\pgfpathlineto{\pgfqpoint{4.419761in}{1.106848in}}%
\pgfpathlineto{\pgfqpoint{4.420337in}{1.102805in}}%
\pgfpathlineto{\pgfqpoint{4.421490in}{1.099101in}}%
\pgfpathlineto{\pgfqpoint{4.420722in}{1.105388in}}%
\pgfpathlineto{\pgfqpoint{4.421682in}{1.100439in}}%
\pgfpathlineto{\pgfqpoint{4.423220in}{1.107669in}}%
\pgfpathlineto{\pgfqpoint{4.424949in}{1.112212in}}%
\pgfpathlineto{\pgfqpoint{4.425334in}{1.113752in}}%
\pgfpathlineto{\pgfqpoint{4.425910in}{1.118365in}}%
\pgfpathlineto{\pgfqpoint{4.426487in}{1.116759in}}%
\pgfpathlineto{\pgfqpoint{4.427063in}{1.118451in}}%
\pgfpathlineto{\pgfqpoint{4.427832in}{1.113322in}}%
\pgfpathlineto{\pgfqpoint{4.428216in}{1.111716in}}%
\pgfpathlineto{\pgfqpoint{4.428793in}{1.113318in}}%
\pgfpathlineto{\pgfqpoint{4.428985in}{1.112513in}}%
\pgfpathlineto{\pgfqpoint{4.429369in}{1.114699in}}%
\pgfpathlineto{\pgfqpoint{4.429753in}{1.118787in}}%
\pgfpathlineto{\pgfqpoint{4.430330in}{1.116043in}}%
\pgfpathlineto{\pgfqpoint{4.430522in}{1.115334in}}%
\pgfpathlineto{\pgfqpoint{4.430906in}{1.116715in}}%
\pgfpathlineto{\pgfqpoint{4.431291in}{1.116620in}}%
\pgfpathlineto{\pgfqpoint{4.432252in}{1.121479in}}%
\pgfpathlineto{\pgfqpoint{4.433405in}{1.109685in}}%
\pgfpathlineto{\pgfqpoint{4.432636in}{1.122075in}}%
\pgfpathlineto{\pgfqpoint{4.433981in}{1.113031in}}%
\pgfpathlineto{\pgfqpoint{4.438209in}{1.083758in}}%
\pgfpathlineto{\pgfqpoint{4.438401in}{1.089586in}}%
\pgfpathlineto{\pgfqpoint{4.438593in}{1.090602in}}%
\pgfpathlineto{\pgfqpoint{4.438978in}{1.086969in}}%
\pgfpathlineto{\pgfqpoint{4.439938in}{1.085490in}}%
\pgfpathlineto{\pgfqpoint{4.439362in}{1.089332in}}%
\pgfpathlineto{\pgfqpoint{4.440131in}{1.085589in}}%
\pgfpathlineto{\pgfqpoint{4.441284in}{1.088738in}}%
\pgfpathlineto{\pgfqpoint{4.441668in}{1.085377in}}%
\pgfpathlineto{\pgfqpoint{4.442244in}{1.088242in}}%
\pgfpathlineto{\pgfqpoint{4.442437in}{1.090110in}}%
\pgfpathlineto{\pgfqpoint{4.442821in}{1.086045in}}%
\pgfpathlineto{\pgfqpoint{4.443013in}{1.086829in}}%
\pgfpathlineto{\pgfqpoint{4.443205in}{1.086182in}}%
\pgfpathlineto{\pgfqpoint{4.444550in}{1.094731in}}%
\pgfpathlineto{\pgfqpoint{4.445127in}{1.098218in}}%
\pgfpathlineto{\pgfqpoint{4.445703in}{1.095607in}}%
\pgfpathlineto{\pgfqpoint{4.445896in}{1.096018in}}%
\pgfpathlineto{\pgfqpoint{4.446280in}{1.093180in}}%
\pgfpathlineto{\pgfqpoint{4.446664in}{1.097083in}}%
\pgfpathlineto{\pgfqpoint{4.447049in}{1.101276in}}%
\pgfpathlineto{\pgfqpoint{4.447817in}{1.097863in}}%
\pgfpathlineto{\pgfqpoint{4.449162in}{1.101726in}}%
\pgfpathlineto{\pgfqpoint{4.449931in}{1.093310in}}%
\pgfpathlineto{\pgfqpoint{4.451084in}{1.094023in}}%
\pgfpathlineto{\pgfqpoint{4.451276in}{1.093039in}}%
\pgfpathlineto{\pgfqpoint{4.451468in}{1.093992in}}%
\pgfpathlineto{\pgfqpoint{4.452814in}{1.100977in}}%
\pgfpathlineto{\pgfqpoint{4.453006in}{1.099403in}}%
\pgfpathlineto{\pgfqpoint{4.453582in}{1.100873in}}%
\pgfpathlineto{\pgfqpoint{4.455120in}{1.112133in}}%
\pgfpathlineto{\pgfqpoint{4.455312in}{1.111459in}}%
\pgfpathlineto{\pgfqpoint{4.455696in}{1.112050in}}%
\pgfpathlineto{\pgfqpoint{4.456849in}{1.120110in}}%
\pgfpathlineto{\pgfqpoint{4.457041in}{1.119977in}}%
\pgfpathlineto{\pgfqpoint{4.457426in}{1.122038in}}%
\pgfpathlineto{\pgfqpoint{4.458579in}{1.116136in}}%
\pgfpathlineto{\pgfqpoint{4.459924in}{1.121774in}}%
\pgfpathlineto{\pgfqpoint{4.460500in}{1.116371in}}%
\pgfpathlineto{\pgfqpoint{4.461077in}{1.118382in}}%
\pgfpathlineto{\pgfqpoint{4.463191in}{1.127958in}}%
\pgfpathlineto{\pgfqpoint{4.461653in}{1.117382in}}%
\pgfpathlineto{\pgfqpoint{4.463383in}{1.127479in}}%
\pgfpathlineto{\pgfqpoint{4.463767in}{1.127851in}}%
\pgfpathlineto{\pgfqpoint{4.464536in}{1.121670in}}%
\pgfpathlineto{\pgfqpoint{4.464920in}{1.122529in}}%
\pgfpathlineto{\pgfqpoint{4.466073in}{1.112893in}}%
\pgfpathlineto{\pgfqpoint{4.467034in}{1.121577in}}%
\pgfpathlineto{\pgfqpoint{4.467611in}{1.120776in}}%
\pgfpathlineto{\pgfqpoint{4.467803in}{1.119409in}}%
\pgfpathlineto{\pgfqpoint{4.468571in}{1.121843in}}%
\pgfpathlineto{\pgfqpoint{4.469148in}{1.121154in}}%
\pgfpathlineto{\pgfqpoint{4.469340in}{1.122778in}}%
\pgfpathlineto{\pgfqpoint{4.469917in}{1.125036in}}%
\pgfpathlineto{\pgfqpoint{4.470493in}{1.120156in}}%
\pgfpathlineto{\pgfqpoint{4.470685in}{1.121742in}}%
\pgfpathlineto{\pgfqpoint{4.471262in}{1.118095in}}%
\pgfpathlineto{\pgfqpoint{4.473183in}{1.111270in}}%
\pgfpathlineto{\pgfqpoint{4.471646in}{1.118386in}}%
\pgfpathlineto{\pgfqpoint{4.473376in}{1.113249in}}%
\pgfpathlineto{\pgfqpoint{4.473760in}{1.117964in}}%
\pgfpathlineto{\pgfqpoint{4.474144in}{1.112709in}}%
\pgfpathlineto{\pgfqpoint{4.474721in}{1.106502in}}%
\pgfpathlineto{\pgfqpoint{4.475297in}{1.110397in}}%
\pgfpathlineto{\pgfqpoint{4.475489in}{1.109129in}}%
\pgfpathlineto{\pgfqpoint{4.475874in}{1.112315in}}%
\pgfpathlineto{\pgfqpoint{4.476066in}{1.112284in}}%
\pgfpathlineto{\pgfqpoint{4.476258in}{1.111924in}}%
\pgfpathlineto{\pgfqpoint{4.477411in}{1.104654in}}%
\pgfpathlineto{\pgfqpoint{4.477603in}{1.105171in}}%
\pgfpathlineto{\pgfqpoint{4.477795in}{1.106363in}}%
\pgfpathlineto{\pgfqpoint{4.478180in}{1.103054in}}%
\pgfpathlineto{\pgfqpoint{4.478756in}{1.105479in}}%
\pgfpathlineto{\pgfqpoint{4.479141in}{1.101610in}}%
\pgfpathlineto{\pgfqpoint{4.479909in}{1.104917in}}%
\pgfpathlineto{\pgfqpoint{4.480102in}{1.104678in}}%
\pgfpathlineto{\pgfqpoint{4.480486in}{1.103843in}}%
\pgfpathlineto{\pgfqpoint{4.481447in}{1.109163in}}%
\pgfpathlineto{\pgfqpoint{4.481831in}{1.111399in}}%
\pgfpathlineto{\pgfqpoint{4.482408in}{1.109424in}}%
\pgfpathlineto{\pgfqpoint{4.483368in}{1.098619in}}%
\pgfpathlineto{\pgfqpoint{4.483753in}{1.101050in}}%
\pgfpathlineto{\pgfqpoint{4.483945in}{1.102306in}}%
\pgfpathlineto{\pgfqpoint{4.484714in}{1.101610in}}%
\pgfpathlineto{\pgfqpoint{4.485098in}{1.097190in}}%
\pgfpathlineto{\pgfqpoint{4.486059in}{1.098413in}}%
\pgfpathlineto{\pgfqpoint{4.487020in}{1.100000in}}%
\pgfpathlineto{\pgfqpoint{4.487212in}{1.097929in}}%
\pgfpathlineto{\pgfqpoint{4.487788in}{1.101099in}}%
\pgfpathlineto{\pgfqpoint{4.488173in}{1.098096in}}%
\pgfpathlineto{\pgfqpoint{4.488365in}{1.099978in}}%
\pgfpathlineto{\pgfqpoint{4.488749in}{1.096838in}}%
\pgfpathlineto{\pgfqpoint{4.488941in}{1.097350in}}%
\pgfpathlineto{\pgfqpoint{4.489902in}{1.091208in}}%
\pgfpathlineto{\pgfqpoint{4.490286in}{1.092751in}}%
\pgfpathlineto{\pgfqpoint{4.490671in}{1.093294in}}%
\pgfpathlineto{\pgfqpoint{4.491247in}{1.083658in}}%
\pgfpathlineto{\pgfqpoint{4.493169in}{1.070570in}}%
\pgfpathlineto{\pgfqpoint{4.493361in}{1.072910in}}%
\pgfpathlineto{\pgfqpoint{4.493553in}{1.069873in}}%
\pgfpathlineto{\pgfqpoint{4.494130in}{1.071264in}}%
\pgfpathlineto{\pgfqpoint{4.495091in}{1.064962in}}%
\pgfpathlineto{\pgfqpoint{4.495283in}{1.065147in}}%
\pgfpathlineto{\pgfqpoint{4.496244in}{1.075644in}}%
\pgfpathlineto{\pgfqpoint{4.496628in}{1.073033in}}%
\pgfpathlineto{\pgfqpoint{4.497397in}{1.078250in}}%
\pgfpathlineto{\pgfqpoint{4.497973in}{1.076773in}}%
\pgfpathlineto{\pgfqpoint{4.498357in}{1.077738in}}%
\pgfpathlineto{\pgfqpoint{4.499318in}{1.073228in}}%
\pgfpathlineto{\pgfqpoint{4.499895in}{1.077496in}}%
\pgfpathlineto{\pgfqpoint{4.500471in}{1.075138in}}%
\pgfpathlineto{\pgfqpoint{4.500663in}{1.074847in}}%
\pgfpathlineto{\pgfqpoint{4.502009in}{1.066194in}}%
\pgfpathlineto{\pgfqpoint{4.502201in}{1.066432in}}%
\pgfpathlineto{\pgfqpoint{4.502585in}{1.068758in}}%
\pgfpathlineto{\pgfqpoint{4.502777in}{1.065134in}}%
\pgfpathlineto{\pgfqpoint{4.502969in}{1.067074in}}%
\pgfpathlineto{\pgfqpoint{4.503546in}{1.061816in}}%
\pgfpathlineto{\pgfqpoint{4.503930in}{1.065795in}}%
\pgfpathlineto{\pgfqpoint{4.504699in}{1.072265in}}%
\pgfpathlineto{\pgfqpoint{4.505276in}{1.070311in}}%
\pgfpathlineto{\pgfqpoint{4.505468in}{1.068937in}}%
\pgfpathlineto{\pgfqpoint{4.505852in}{1.071540in}}%
\pgfpathlineto{\pgfqpoint{4.506044in}{1.074476in}}%
\pgfpathlineto{\pgfqpoint{4.506813in}{1.070440in}}%
\pgfpathlineto{\pgfqpoint{4.507389in}{1.067716in}}%
\pgfpathlineto{\pgfqpoint{4.507774in}{1.068463in}}%
\pgfpathlineto{\pgfqpoint{4.508350in}{1.071944in}}%
\pgfpathlineto{\pgfqpoint{4.508927in}{1.071610in}}%
\pgfpathlineto{\pgfqpoint{4.509119in}{1.071387in}}%
\pgfpathlineto{\pgfqpoint{4.509311in}{1.071568in}}%
\pgfpathlineto{\pgfqpoint{4.510272in}{1.076887in}}%
\pgfpathlineto{\pgfqpoint{4.510848in}{1.074320in}}%
\pgfpathlineto{\pgfqpoint{4.511425in}{1.078805in}}%
\pgfpathlineto{\pgfqpoint{4.511617in}{1.074506in}}%
\pgfpathlineto{\pgfqpoint{4.513731in}{1.058987in}}%
\pgfpathlineto{\pgfqpoint{4.513923in}{1.061914in}}%
\pgfpathlineto{\pgfqpoint{4.514500in}{1.057967in}}%
\pgfpathlineto{\pgfqpoint{4.515460in}{1.049241in}}%
\pgfpathlineto{\pgfqpoint{4.515845in}{1.050658in}}%
\pgfpathlineto{\pgfqpoint{4.516421in}{1.052042in}}%
\pgfpathlineto{\pgfqpoint{4.516613in}{1.050547in}}%
\pgfpathlineto{\pgfqpoint{4.516806in}{1.049189in}}%
\pgfpathlineto{\pgfqpoint{4.517190in}{1.052175in}}%
\pgfpathlineto{\pgfqpoint{4.517574in}{1.056738in}}%
\pgfpathlineto{\pgfqpoint{4.518151in}{1.052481in}}%
\pgfpathlineto{\pgfqpoint{4.519688in}{1.042764in}}%
\pgfpathlineto{\pgfqpoint{4.522378in}{1.052313in}}%
\pgfpathlineto{\pgfqpoint{4.522763in}{1.047547in}}%
\pgfpathlineto{\pgfqpoint{4.523531in}{1.050226in}}%
\pgfpathlineto{\pgfqpoint{4.525069in}{1.044436in}}%
\pgfpathlineto{\pgfqpoint{4.525261in}{1.046307in}}%
\pgfpathlineto{\pgfqpoint{4.525837in}{1.050194in}}%
\pgfpathlineto{\pgfqpoint{4.526414in}{1.047457in}}%
\pgfpathlineto{\pgfqpoint{4.526798in}{1.041381in}}%
\pgfpathlineto{\pgfqpoint{4.527567in}{1.045284in}}%
\pgfpathlineto{\pgfqpoint{4.528912in}{1.054506in}}%
\pgfpathlineto{\pgfqpoint{4.529297in}{1.061482in}}%
\pgfpathlineto{\pgfqpoint{4.530065in}{1.057930in}}%
\pgfpathlineto{\pgfqpoint{4.531603in}{1.066324in}}%
\pgfpathlineto{\pgfqpoint{4.530642in}{1.057614in}}%
\pgfpathlineto{\pgfqpoint{4.531795in}{1.065746in}}%
\pgfpathlineto{\pgfqpoint{4.531987in}{1.063778in}}%
\pgfpathlineto{\pgfqpoint{4.532756in}{1.066633in}}%
\pgfpathlineto{\pgfqpoint{4.532948in}{1.064552in}}%
\pgfpathlineto{\pgfqpoint{4.533716in}{1.062793in}}%
\pgfpathlineto{\pgfqpoint{4.534293in}{1.066688in}}%
\pgfpathlineto{\pgfqpoint{4.534485in}{1.063142in}}%
\pgfpathlineto{\pgfqpoint{4.534869in}{1.067158in}}%
\pgfpathlineto{\pgfqpoint{4.535446in}{1.065622in}}%
\pgfpathlineto{\pgfqpoint{4.535638in}{1.065767in}}%
\pgfpathlineto{\pgfqpoint{4.535830in}{1.064924in}}%
\pgfpathlineto{\pgfqpoint{4.536983in}{1.059216in}}%
\pgfpathlineto{\pgfqpoint{4.537560in}{1.061096in}}%
\pgfpathlineto{\pgfqpoint{4.537752in}{1.063158in}}%
\pgfpathlineto{\pgfqpoint{4.537944in}{1.056535in}}%
\pgfpathlineto{\pgfqpoint{4.540058in}{1.045422in}}%
\pgfpathlineto{\pgfqpoint{4.540250in}{1.045672in}}%
\pgfpathlineto{\pgfqpoint{4.540634in}{1.044870in}}%
\pgfpathlineto{\pgfqpoint{4.541787in}{1.038016in}}%
\pgfpathlineto{\pgfqpoint{4.541980in}{1.040402in}}%
\pgfpathlineto{\pgfqpoint{4.542172in}{1.041224in}}%
\pgfpathlineto{\pgfqpoint{4.542364in}{1.037810in}}%
\pgfpathlineto{\pgfqpoint{4.542556in}{1.041022in}}%
\pgfpathlineto{\pgfqpoint{4.543901in}{1.033092in}}%
\pgfpathlineto{\pgfqpoint{4.544093in}{1.033293in}}%
\pgfpathlineto{\pgfqpoint{4.544862in}{1.034506in}}%
\pgfpathlineto{\pgfqpoint{4.545054in}{1.033519in}}%
\pgfpathlineto{\pgfqpoint{4.545439in}{1.027747in}}%
\pgfpathlineto{\pgfqpoint{4.546399in}{1.030781in}}%
\pgfpathlineto{\pgfqpoint{4.546592in}{1.030879in}}%
\pgfpathlineto{\pgfqpoint{4.547360in}{1.033862in}}%
\pgfpathlineto{\pgfqpoint{4.547745in}{1.031504in}}%
\pgfpathlineto{\pgfqpoint{4.547937in}{1.029869in}}%
\pgfpathlineto{\pgfqpoint{4.548898in}{1.030394in}}%
\pgfpathlineto{\pgfqpoint{4.549282in}{1.029573in}}%
\pgfpathlineto{\pgfqpoint{4.550435in}{1.036766in}}%
\pgfpathlineto{\pgfqpoint{4.550627in}{1.033702in}}%
\pgfpathlineto{\pgfqpoint{4.551204in}{1.037117in}}%
\pgfpathlineto{\pgfqpoint{4.551396in}{1.034394in}}%
\pgfpathlineto{\pgfqpoint{4.551588in}{1.037254in}}%
\pgfpathlineto{\pgfqpoint{4.551972in}{1.030971in}}%
\pgfpathlineto{\pgfqpoint{4.552357in}{1.035461in}}%
\pgfpathlineto{\pgfqpoint{4.552741in}{1.031543in}}%
\pgfpathlineto{\pgfqpoint{4.553318in}{1.036534in}}%
\pgfpathlineto{\pgfqpoint{4.553510in}{1.036920in}}%
\pgfpathlineto{\pgfqpoint{4.553702in}{1.034755in}}%
\pgfpathlineto{\pgfqpoint{4.554086in}{1.037217in}}%
\pgfpathlineto{\pgfqpoint{4.554471in}{1.036680in}}%
\pgfpathlineto{\pgfqpoint{4.554663in}{1.038538in}}%
\pgfpathlineto{\pgfqpoint{4.554855in}{1.036101in}}%
\pgfpathlineto{\pgfqpoint{4.555047in}{1.037155in}}%
\pgfpathlineto{\pgfqpoint{4.555239in}{1.032856in}}%
\pgfpathlineto{\pgfqpoint{4.555816in}{1.038348in}}%
\pgfpathlineto{\pgfqpoint{4.556008in}{1.038202in}}%
\pgfpathlineto{\pgfqpoint{4.556969in}{1.040491in}}%
\pgfpathlineto{\pgfqpoint{4.558506in}{1.050749in}}%
\pgfpathlineto{\pgfqpoint{4.559851in}{1.053767in}}%
\pgfpathlineto{\pgfqpoint{4.560620in}{1.048303in}}%
\pgfpathlineto{\pgfqpoint{4.561196in}{1.050322in}}%
\pgfpathlineto{\pgfqpoint{4.562542in}{1.058925in}}%
\pgfpathlineto{\pgfqpoint{4.562734in}{1.057628in}}%
\pgfpathlineto{\pgfqpoint{4.563887in}{1.063814in}}%
\pgfpathlineto{\pgfqpoint{4.563310in}{1.056337in}}%
\pgfpathlineto{\pgfqpoint{4.564079in}{1.061561in}}%
\pgfpathlineto{\pgfqpoint{4.564271in}{1.056255in}}%
\pgfpathlineto{\pgfqpoint{4.565040in}{1.060129in}}%
\pgfpathlineto{\pgfqpoint{4.565424in}{1.063672in}}%
\pgfpathlineto{\pgfqpoint{4.566385in}{1.063342in}}%
\pgfpathlineto{\pgfqpoint{4.566577in}{1.060899in}}%
\pgfpathlineto{\pgfqpoint{4.567154in}{1.065642in}}%
\pgfpathlineto{\pgfqpoint{4.567346in}{1.063181in}}%
\pgfpathlineto{\pgfqpoint{4.568499in}{1.074509in}}%
\pgfpathlineto{\pgfqpoint{4.568883in}{1.071285in}}%
\pgfpathlineto{\pgfqpoint{4.570036in}{1.067633in}}%
\pgfpathlineto{\pgfqpoint{4.570228in}{1.069529in}}%
\pgfpathlineto{\pgfqpoint{4.570613in}{1.070074in}}%
\pgfpathlineto{\pgfqpoint{4.570997in}{1.068786in}}%
\pgfpathlineto{\pgfqpoint{4.571573in}{1.065549in}}%
\pgfpathlineto{\pgfqpoint{4.571381in}{1.070421in}}%
\pgfpathlineto{\pgfqpoint{4.572342in}{1.067707in}}%
\pgfpathlineto{\pgfqpoint{4.572726in}{1.070921in}}%
\pgfpathlineto{\pgfqpoint{4.573303in}{1.068386in}}%
\pgfpathlineto{\pgfqpoint{4.573879in}{1.065450in}}%
\pgfpathlineto{\pgfqpoint{4.574456in}{1.066472in}}%
\pgfpathlineto{\pgfqpoint{4.575801in}{1.073893in}}%
\pgfpathlineto{\pgfqpoint{4.576185in}{1.072930in}}%
\pgfpathlineto{\pgfqpoint{4.576378in}{1.072934in}}%
\pgfpathlineto{\pgfqpoint{4.578107in}{1.080099in}}%
\pgfpathlineto{\pgfqpoint{4.578299in}{1.079617in}}%
\pgfpathlineto{\pgfqpoint{4.578492in}{1.077362in}}%
\pgfpathlineto{\pgfqpoint{4.579068in}{1.080408in}}%
\pgfpathlineto{\pgfqpoint{4.580029in}{1.079726in}}%
\pgfpathlineto{\pgfqpoint{4.580413in}{1.084967in}}%
\pgfpathlineto{\pgfqpoint{4.581566in}{1.078099in}}%
\pgfpathlineto{\pgfqpoint{4.581758in}{1.074426in}}%
\pgfpathlineto{\pgfqpoint{4.582335in}{1.082777in}}%
\pgfpathlineto{\pgfqpoint{4.582527in}{1.081531in}}%
\pgfpathlineto{\pgfqpoint{4.582719in}{1.082265in}}%
\pgfpathlineto{\pgfqpoint{4.583296in}{1.089989in}}%
\pgfpathlineto{\pgfqpoint{4.583872in}{1.085179in}}%
\pgfpathlineto{\pgfqpoint{4.584449in}{1.080346in}}%
\pgfpathlineto{\pgfqpoint{4.585217in}{1.082358in}}%
\pgfpathlineto{\pgfqpoint{4.585602in}{1.085744in}}%
\pgfpathlineto{\pgfqpoint{4.586178in}{1.083502in}}%
\pgfpathlineto{\pgfqpoint{4.586755in}{1.079860in}}%
\pgfpathlineto{\pgfqpoint{4.586563in}{1.083695in}}%
\pgfpathlineto{\pgfqpoint{4.587139in}{1.080617in}}%
\pgfpathlineto{\pgfqpoint{4.588292in}{1.085764in}}%
\pgfpathlineto{\pgfqpoint{4.588484in}{1.085409in}}%
\pgfpathlineto{\pgfqpoint{4.589829in}{1.093202in}}%
\pgfpathlineto{\pgfqpoint{4.590022in}{1.091462in}}%
\pgfpathlineto{\pgfqpoint{4.590406in}{1.091972in}}%
\pgfpathlineto{\pgfqpoint{4.591367in}{1.089475in}}%
\pgfpathlineto{\pgfqpoint{4.592904in}{1.095413in}}%
\pgfpathlineto{\pgfqpoint{4.593288in}{1.093628in}}%
\pgfpathlineto{\pgfqpoint{4.593865in}{1.094508in}}%
\pgfpathlineto{\pgfqpoint{4.594634in}{1.099483in}}%
\pgfpathlineto{\pgfqpoint{4.595018in}{1.096896in}}%
\pgfpathlineto{\pgfqpoint{4.597324in}{1.081754in}}%
\pgfpathlineto{\pgfqpoint{4.597516in}{1.084299in}}%
\pgfpathlineto{\pgfqpoint{4.597708in}{1.085716in}}%
\pgfpathlineto{\pgfqpoint{4.597900in}{1.083506in}}%
\pgfpathlineto{\pgfqpoint{4.598669in}{1.079190in}}%
\pgfpathlineto{\pgfqpoint{4.598861in}{1.084012in}}%
\pgfpathlineto{\pgfqpoint{4.599053in}{1.083058in}}%
\pgfpathlineto{\pgfqpoint{4.599438in}{1.085054in}}%
\pgfpathlineto{\pgfqpoint{4.601552in}{1.095766in}}%
\pgfpathlineto{\pgfqpoint{4.602320in}{1.090828in}}%
\pgfpathlineto{\pgfqpoint{4.602897in}{1.092752in}}%
\pgfpathlineto{\pgfqpoint{4.603473in}{1.094383in}}%
\pgfpathlineto{\pgfqpoint{4.603666in}{1.091718in}}%
\pgfpathlineto{\pgfqpoint{4.603858in}{1.089719in}}%
\pgfpathlineto{\pgfqpoint{4.604242in}{1.092509in}}%
\pgfpathlineto{\pgfqpoint{4.604626in}{1.091829in}}%
\pgfpathlineto{\pgfqpoint{4.605011in}{1.098971in}}%
\pgfpathlineto{\pgfqpoint{4.606356in}{1.097532in}}%
\pgfpathlineto{\pgfqpoint{4.606932in}{1.094720in}}%
\pgfpathlineto{\pgfqpoint{4.607317in}{1.096551in}}%
\pgfpathlineto{\pgfqpoint{4.607893in}{1.099205in}}%
\pgfpathlineto{\pgfqpoint{4.608278in}{1.098782in}}%
\pgfpathlineto{\pgfqpoint{4.609623in}{1.090625in}}%
\pgfpathlineto{\pgfqpoint{4.610199in}{1.091561in}}%
\pgfpathlineto{\pgfqpoint{4.611160in}{1.087058in}}%
\pgfpathlineto{\pgfqpoint{4.613082in}{1.079969in}}%
\pgfpathlineto{\pgfqpoint{4.613274in}{1.078621in}}%
\pgfpathlineto{\pgfqpoint{4.613850in}{1.080871in}}%
\pgfpathlineto{\pgfqpoint{4.614043in}{1.080305in}}%
\pgfpathlineto{\pgfqpoint{4.614619in}{1.083056in}}%
\pgfpathlineto{\pgfqpoint{4.614811in}{1.082283in}}%
\pgfpathlineto{\pgfqpoint{4.616156in}{1.089574in}}%
\pgfpathlineto{\pgfqpoint{4.616349in}{1.090343in}}%
\pgfpathlineto{\pgfqpoint{4.616541in}{1.087321in}}%
\pgfpathlineto{\pgfqpoint{4.616733in}{1.087659in}}%
\pgfpathlineto{\pgfqpoint{4.618078in}{1.080703in}}%
\pgfpathlineto{\pgfqpoint{4.619231in}{1.085528in}}%
\pgfpathlineto{\pgfqpoint{4.619615in}{1.084384in}}%
\pgfpathlineto{\pgfqpoint{4.620576in}{1.082238in}}%
\pgfpathlineto{\pgfqpoint{4.620192in}{1.085432in}}%
\pgfpathlineto{\pgfqpoint{4.620768in}{1.082999in}}%
\pgfpathlineto{\pgfqpoint{4.621921in}{1.088963in}}%
\pgfpathlineto{\pgfqpoint{4.621153in}{1.082572in}}%
\pgfpathlineto{\pgfqpoint{4.622498in}{1.088057in}}%
\pgfpathlineto{\pgfqpoint{4.622882in}{1.090301in}}%
\pgfpathlineto{\pgfqpoint{4.624035in}{1.081336in}}%
\pgfpathlineto{\pgfqpoint{4.624227in}{1.081814in}}%
\pgfpathlineto{\pgfqpoint{4.625380in}{1.091766in}}%
\pgfpathlineto{\pgfqpoint{4.626534in}{1.094812in}}%
\pgfpathlineto{\pgfqpoint{4.626726in}{1.091301in}}%
\pgfpathlineto{\pgfqpoint{4.627110in}{1.096294in}}%
\pgfpathlineto{\pgfqpoint{4.627494in}{1.091510in}}%
\pgfpathlineto{\pgfqpoint{4.627687in}{1.095372in}}%
\pgfpathlineto{\pgfqpoint{4.628263in}{1.089982in}}%
\pgfpathlineto{\pgfqpoint{4.628455in}{1.090798in}}%
\pgfpathlineto{\pgfqpoint{4.628647in}{1.090533in}}%
\pgfpathlineto{\pgfqpoint{4.629993in}{1.083654in}}%
\pgfpathlineto{\pgfqpoint{4.630377in}{1.084909in}}%
\pgfpathlineto{\pgfqpoint{4.630569in}{1.083956in}}%
\pgfpathlineto{\pgfqpoint{4.632106in}{1.074289in}}%
\pgfpathlineto{\pgfqpoint{4.632875in}{1.076710in}}%
\pgfpathlineto{\pgfqpoint{4.633067in}{1.072606in}}%
\pgfpathlineto{\pgfqpoint{4.635373in}{1.064405in}}%
\pgfpathlineto{\pgfqpoint{4.635565in}{1.065528in}}%
\pgfpathlineto{\pgfqpoint{4.636334in}{1.057992in}}%
\pgfpathlineto{\pgfqpoint{4.636718in}{1.060280in}}%
\pgfpathlineto{\pgfqpoint{4.636911in}{1.063008in}}%
\pgfpathlineto{\pgfqpoint{4.637295in}{1.057949in}}%
\pgfpathlineto{\pgfqpoint{4.638064in}{1.054494in}}%
\pgfpathlineto{\pgfqpoint{4.638448in}{1.056989in}}%
\pgfpathlineto{\pgfqpoint{4.638832in}{1.061247in}}%
\pgfpathlineto{\pgfqpoint{4.639217in}{1.058112in}}%
\pgfpathlineto{\pgfqpoint{4.639409in}{1.054213in}}%
\pgfpathlineto{\pgfqpoint{4.640370in}{1.055840in}}%
\pgfpathlineto{\pgfqpoint{4.640562in}{1.057659in}}%
\pgfpathlineto{\pgfqpoint{4.640946in}{1.051246in}}%
\pgfpathlineto{\pgfqpoint{4.641907in}{1.055876in}}%
\pgfpathlineto{\pgfqpoint{4.641523in}{1.050309in}}%
\pgfpathlineto{\pgfqpoint{4.642099in}{1.053534in}}%
\pgfpathlineto{\pgfqpoint{4.642291in}{1.051732in}}%
\pgfpathlineto{\pgfqpoint{4.642676in}{1.054573in}}%
\pgfpathlineto{\pgfqpoint{4.643252in}{1.052582in}}%
\pgfpathlineto{\pgfqpoint{4.643444in}{1.052504in}}%
\pgfpathlineto{\pgfqpoint{4.643829in}{1.056023in}}%
\pgfpathlineto{\pgfqpoint{4.644213in}{1.051130in}}%
\pgfpathlineto{\pgfqpoint{4.644405in}{1.052604in}}%
\pgfpathlineto{\pgfqpoint{4.644597in}{1.048434in}}%
\pgfpathlineto{\pgfqpoint{4.645174in}{1.053575in}}%
\pgfpathlineto{\pgfqpoint{4.645942in}{1.061358in}}%
\pgfpathlineto{\pgfqpoint{4.646327in}{1.058490in}}%
\pgfpathlineto{\pgfqpoint{4.647095in}{1.049492in}}%
\pgfpathlineto{\pgfqpoint{4.647864in}{1.050595in}}%
\pgfpathlineto{\pgfqpoint{4.648248in}{1.051952in}}%
\pgfpathlineto{\pgfqpoint{4.648441in}{1.051160in}}%
\pgfpathlineto{\pgfqpoint{4.648825in}{1.045156in}}%
\pgfpathlineto{\pgfqpoint{4.649594in}{1.047711in}}%
\pgfpathlineto{\pgfqpoint{4.649786in}{1.048742in}}%
\pgfpathlineto{\pgfqpoint{4.650170in}{1.045189in}}%
\pgfpathlineto{\pgfqpoint{4.650554in}{1.043754in}}%
\pgfpathlineto{\pgfqpoint{4.650939in}{1.046692in}}%
\pgfpathlineto{\pgfqpoint{4.651131in}{1.044572in}}%
\pgfpathlineto{\pgfqpoint{4.651900in}{1.049221in}}%
\pgfpathlineto{\pgfqpoint{4.652284in}{1.051431in}}%
\pgfpathlineto{\pgfqpoint{4.652476in}{1.047428in}}%
\pgfpathlineto{\pgfqpoint{4.653821in}{1.040524in}}%
\pgfpathlineto{\pgfqpoint{4.654014in}{1.041803in}}%
\pgfpathlineto{\pgfqpoint{4.654398in}{1.039433in}}%
\pgfpathlineto{\pgfqpoint{4.654590in}{1.040315in}}%
\pgfpathlineto{\pgfqpoint{4.654782in}{1.037988in}}%
\pgfpathlineto{\pgfqpoint{4.655359in}{1.042261in}}%
\pgfpathlineto{\pgfqpoint{4.655551in}{1.039679in}}%
\pgfpathlineto{\pgfqpoint{4.655935in}{1.043181in}}%
\pgfpathlineto{\pgfqpoint{4.656896in}{1.042560in}}%
\pgfpathlineto{\pgfqpoint{4.658049in}{1.028697in}}%
\pgfpathlineto{\pgfqpoint{4.658626in}{1.031127in}}%
\pgfpathlineto{\pgfqpoint{4.658818in}{1.031445in}}%
\pgfpathlineto{\pgfqpoint{4.659010in}{1.029880in}}%
\pgfpathlineto{\pgfqpoint{4.659586in}{1.030741in}}%
\pgfpathlineto{\pgfqpoint{4.660547in}{1.034402in}}%
\pgfpathlineto{\pgfqpoint{4.660163in}{1.030583in}}%
\pgfpathlineto{\pgfqpoint{4.660739in}{1.031487in}}%
\pgfpathlineto{\pgfqpoint{4.660932in}{1.032278in}}%
\pgfpathlineto{\pgfqpoint{4.661508in}{1.032070in}}%
\pgfpathlineto{\pgfqpoint{4.661700in}{1.029928in}}%
\pgfpathlineto{\pgfqpoint{4.662277in}{1.031202in}}%
\pgfpathlineto{\pgfqpoint{4.662469in}{1.034252in}}%
\pgfpathlineto{\pgfqpoint{4.662853in}{1.028867in}}%
\pgfpathlineto{\pgfqpoint{4.663238in}{1.030986in}}%
\pgfpathlineto{\pgfqpoint{4.664006in}{1.026657in}}%
\pgfpathlineto{\pgfqpoint{4.664198in}{1.029939in}}%
\pgfpathlineto{\pgfqpoint{4.665159in}{1.038647in}}%
\pgfpathlineto{\pgfqpoint{4.665351in}{1.037200in}}%
\pgfpathlineto{\pgfqpoint{4.665544in}{1.034006in}}%
\pgfpathlineto{\pgfqpoint{4.666120in}{1.038531in}}%
\pgfpathlineto{\pgfqpoint{4.666312in}{1.038420in}}%
\pgfpathlineto{\pgfqpoint{4.667850in}{1.045523in}}%
\pgfpathlineto{\pgfqpoint{4.668042in}{1.045377in}}%
\pgfpathlineto{\pgfqpoint{4.668234in}{1.039624in}}%
\pgfpathlineto{\pgfqpoint{4.669195in}{1.043723in}}%
\pgfpathlineto{\pgfqpoint{4.669387in}{1.043696in}}%
\pgfpathlineto{\pgfqpoint{4.669771in}{1.040981in}}%
\pgfpathlineto{\pgfqpoint{4.670156in}{1.045088in}}%
\pgfpathlineto{\pgfqpoint{4.670348in}{1.046959in}}%
\pgfpathlineto{\pgfqpoint{4.670732in}{1.045631in}}%
\pgfpathlineto{\pgfqpoint{4.671309in}{1.035794in}}%
\pgfpathlineto{\pgfqpoint{4.672077in}{1.037896in}}%
\pgfpathlineto{\pgfqpoint{4.673422in}{1.045094in}}%
\pgfpathlineto{\pgfqpoint{4.673615in}{1.042425in}}%
\pgfpathlineto{\pgfqpoint{4.674191in}{1.047310in}}%
\pgfpathlineto{\pgfqpoint{4.674383in}{1.047671in}}%
\pgfpathlineto{\pgfqpoint{4.674575in}{1.046827in}}%
\pgfpathlineto{\pgfqpoint{4.675344in}{1.043653in}}%
\pgfpathlineto{\pgfqpoint{4.675536in}{1.046742in}}%
\pgfpathlineto{\pgfqpoint{4.676305in}{1.054073in}}%
\pgfpathlineto{\pgfqpoint{4.676689in}{1.051964in}}%
\pgfpathlineto{\pgfqpoint{4.677266in}{1.053565in}}%
\pgfpathlineto{\pgfqpoint{4.678035in}{1.046942in}}%
\pgfpathlineto{\pgfqpoint{4.678227in}{1.048204in}}%
\pgfpathlineto{\pgfqpoint{4.678611in}{1.045290in}}%
\pgfpathlineto{\pgfqpoint{4.678995in}{1.047260in}}%
\pgfpathlineto{\pgfqpoint{4.680341in}{1.039956in}}%
\pgfpathlineto{\pgfqpoint{4.680725in}{1.045301in}}%
\pgfpathlineto{\pgfqpoint{4.681494in}{1.044393in}}%
\pgfpathlineto{\pgfqpoint{4.682070in}{1.042118in}}%
\pgfpathlineto{\pgfqpoint{4.682647in}{1.043964in}}%
\pgfpathlineto{\pgfqpoint{4.684184in}{1.051802in}}%
\pgfpathlineto{\pgfqpoint{4.684568in}{1.048083in}}%
\pgfpathlineto{\pgfqpoint{4.684953in}{1.045442in}}%
\pgfpathlineto{\pgfqpoint{4.685721in}{1.047743in}}%
\pgfpathlineto{\pgfqpoint{4.686298in}{1.050738in}}%
\pgfpathlineto{\pgfqpoint{4.686490in}{1.048898in}}%
\pgfpathlineto{\pgfqpoint{4.687835in}{1.036200in}}%
\pgfpathlineto{\pgfqpoint{4.688027in}{1.039576in}}%
\pgfpathlineto{\pgfqpoint{4.688604in}{1.038958in}}%
\pgfpathlineto{\pgfqpoint{4.689180in}{1.043044in}}%
\pgfpathlineto{\pgfqpoint{4.689757in}{1.039722in}}%
\pgfpathlineto{\pgfqpoint{4.691294in}{1.030388in}}%
\pgfpathlineto{\pgfqpoint{4.691678in}{1.028976in}}%
\pgfpathlineto{\pgfqpoint{4.691871in}{1.029539in}}%
\pgfpathlineto{\pgfqpoint{4.692063in}{1.027630in}}%
\pgfpathlineto{\pgfqpoint{4.692639in}{1.031637in}}%
\pgfpathlineto{\pgfqpoint{4.692831in}{1.029078in}}%
\pgfpathlineto{\pgfqpoint{4.693024in}{1.032064in}}%
\pgfpathlineto{\pgfqpoint{4.693408in}{1.027721in}}%
\pgfpathlineto{\pgfqpoint{4.694753in}{1.021008in}}%
\pgfpathlineto{\pgfqpoint{4.694945in}{1.020382in}}%
\pgfpathlineto{\pgfqpoint{4.695137in}{1.021244in}}%
\pgfpathlineto{\pgfqpoint{4.696098in}{1.024603in}}%
\pgfpathlineto{\pgfqpoint{4.697059in}{1.020554in}}%
\pgfpathlineto{\pgfqpoint{4.697251in}{1.020660in}}%
\pgfpathlineto{\pgfqpoint{4.697828in}{1.019696in}}%
\pgfpathlineto{\pgfqpoint{4.699173in}{1.026100in}}%
\pgfpathlineto{\pgfqpoint{4.699365in}{1.025416in}}%
\pgfpathlineto{\pgfqpoint{4.699557in}{1.027267in}}%
\pgfpathlineto{\pgfqpoint{4.699749in}{1.025816in}}%
\pgfpathlineto{\pgfqpoint{4.699942in}{1.028283in}}%
\pgfpathlineto{\pgfqpoint{4.700518in}{1.025362in}}%
\pgfpathlineto{\pgfqpoint{4.701479in}{1.021428in}}%
\pgfpathlineto{\pgfqpoint{4.701863in}{1.022798in}}%
\pgfpathlineto{\pgfqpoint{4.702056in}{1.022402in}}%
\pgfpathlineto{\pgfqpoint{4.702248in}{1.023071in}}%
\pgfpathlineto{\pgfqpoint{4.702440in}{1.026211in}}%
\pgfpathlineto{\pgfqpoint{4.703016in}{1.018395in}}%
\pgfpathlineto{\pgfqpoint{4.703209in}{1.020463in}}%
\pgfpathlineto{\pgfqpoint{4.703593in}{1.014349in}}%
\pgfpathlineto{\pgfqpoint{4.703785in}{1.014445in}}%
\pgfpathlineto{\pgfqpoint{4.703977in}{1.013701in}}%
\pgfpathlineto{\pgfqpoint{4.704362in}{1.013375in}}%
\pgfpathlineto{\pgfqpoint{4.705707in}{1.025451in}}%
\pgfpathlineto{\pgfqpoint{4.706283in}{1.020123in}}%
\pgfpathlineto{\pgfqpoint{4.706860in}{1.012232in}}%
\pgfpathlineto{\pgfqpoint{4.707436in}{1.012683in}}%
\pgfpathlineto{\pgfqpoint{4.708589in}{1.017913in}}%
\pgfpathlineto{\pgfqpoint{4.708781in}{1.015640in}}%
\pgfpathlineto{\pgfqpoint{4.708974in}{1.015448in}}%
\pgfpathlineto{\pgfqpoint{4.709166in}{1.016603in}}%
\pgfpathlineto{\pgfqpoint{4.710127in}{1.019229in}}%
\pgfpathlineto{\pgfqpoint{4.710319in}{1.017890in}}%
\pgfpathlineto{\pgfqpoint{4.710703in}{1.015333in}}%
\pgfpathlineto{\pgfqpoint{4.711280in}{1.017599in}}%
\pgfpathlineto{\pgfqpoint{4.712240in}{1.020018in}}%
\pgfpathlineto{\pgfqpoint{4.714162in}{1.005857in}}%
\pgfpathlineto{\pgfqpoint{4.714354in}{1.007215in}}%
\pgfpathlineto{\pgfqpoint{4.714931in}{1.004827in}}%
\pgfpathlineto{\pgfqpoint{4.715699in}{0.998151in}}%
\pgfpathlineto{\pgfqpoint{4.716276in}{0.999812in}}%
\pgfpathlineto{\pgfqpoint{4.716660in}{0.998009in}}%
\pgfpathlineto{\pgfqpoint{4.716852in}{1.001167in}}%
\pgfpathlineto{\pgfqpoint{4.717045in}{1.000492in}}%
\pgfpathlineto{\pgfqpoint{4.717429in}{1.000041in}}%
\pgfpathlineto{\pgfqpoint{4.717621in}{1.001655in}}%
\pgfpathlineto{\pgfqpoint{4.718005in}{0.999711in}}%
\pgfpathlineto{\pgfqpoint{4.718198in}{0.996836in}}%
\pgfpathlineto{\pgfqpoint{4.718390in}{1.000857in}}%
\pgfpathlineto{\pgfqpoint{4.718966in}{1.000704in}}%
\pgfpathlineto{\pgfqpoint{4.719158in}{0.999800in}}%
\pgfpathlineto{\pgfqpoint{4.719543in}{1.000349in}}%
\pgfpathlineto{\pgfqpoint{4.720888in}{1.009638in}}%
\pgfpathlineto{\pgfqpoint{4.721464in}{1.010949in}}%
\pgfpathlineto{\pgfqpoint{4.721849in}{1.007584in}}%
\pgfpathlineto{\pgfqpoint{4.722041in}{1.009996in}}%
\pgfpathlineto{\pgfqpoint{4.722425in}{1.005961in}}%
\pgfpathlineto{\pgfqpoint{4.723002in}{1.008338in}}%
\pgfpathlineto{\pgfqpoint{4.723578in}{1.010229in}}%
\pgfpathlineto{\pgfqpoint{4.723770in}{1.007086in}}%
\pgfpathlineto{\pgfqpoint{4.724347in}{1.004140in}}%
\pgfpathlineto{\pgfqpoint{4.724924in}{1.006617in}}%
\pgfpathlineto{\pgfqpoint{4.725500in}{1.009721in}}%
\pgfpathlineto{\pgfqpoint{4.725308in}{1.005891in}}%
\pgfpathlineto{\pgfqpoint{4.725884in}{1.006636in}}%
\pgfpathlineto{\pgfqpoint{4.726077in}{1.005478in}}%
\pgfpathlineto{\pgfqpoint{4.726461in}{1.008090in}}%
\pgfpathlineto{\pgfqpoint{4.726653in}{1.010966in}}%
\pgfpathlineto{\pgfqpoint{4.727422in}{1.007321in}}%
\pgfpathlineto{\pgfqpoint{4.727614in}{1.008451in}}%
\pgfpathlineto{\pgfqpoint{4.727806in}{1.006083in}}%
\pgfpathlineto{\pgfqpoint{4.729151in}{1.001233in}}%
\pgfpathlineto{\pgfqpoint{4.729343in}{0.999306in}}%
\pgfpathlineto{\pgfqpoint{4.729920in}{1.002622in}}%
\pgfpathlineto{\pgfqpoint{4.730112in}{1.001743in}}%
\pgfpathlineto{\pgfqpoint{4.730304in}{1.001385in}}%
\pgfpathlineto{\pgfqpoint{4.730689in}{1.002706in}}%
\pgfpathlineto{\pgfqpoint{4.731265in}{1.007443in}}%
\pgfpathlineto{\pgfqpoint{4.732034in}{1.005815in}}%
\pgfpathlineto{\pgfqpoint{4.733763in}{1.000356in}}%
\pgfpathlineto{\pgfqpoint{4.735108in}{1.009472in}}%
\pgfpathlineto{\pgfqpoint{4.735301in}{1.004557in}}%
\pgfpathlineto{\pgfqpoint{4.736069in}{1.012909in}}%
\pgfpathlineto{\pgfqpoint{4.736646in}{1.007892in}}%
\pgfpathlineto{\pgfqpoint{4.737030in}{1.014994in}}%
\pgfpathlineto{\pgfqpoint{4.737607in}{1.015229in}}%
\pgfpathlineto{\pgfqpoint{4.738760in}{1.007545in}}%
\pgfpathlineto{\pgfqpoint{4.738952in}{1.010645in}}%
\pgfpathlineto{\pgfqpoint{4.741258in}{1.015224in}}%
\pgfpathlineto{\pgfqpoint{4.742603in}{1.007421in}}%
\pgfpathlineto{\pgfqpoint{4.742795in}{1.010292in}}%
\pgfpathlineto{\pgfqpoint{4.743372in}{1.003332in}}%
\pgfpathlineto{\pgfqpoint{4.743564in}{1.006917in}}%
\pgfpathlineto{\pgfqpoint{4.744909in}{1.001632in}}%
\pgfpathlineto{\pgfqpoint{4.744140in}{1.009153in}}%
\pgfpathlineto{\pgfqpoint{4.745293in}{1.003122in}}%
\pgfpathlineto{\pgfqpoint{4.745678in}{1.003823in}}%
\pgfpathlineto{\pgfqpoint{4.745870in}{1.001646in}}%
\pgfpathlineto{\pgfqpoint{4.746062in}{1.001486in}}%
\pgfpathlineto{\pgfqpoint{4.746638in}{1.002352in}}%
\pgfpathlineto{\pgfqpoint{4.747599in}{0.994590in}}%
\pgfpathlineto{\pgfqpoint{4.747791in}{0.994059in}}%
\pgfpathlineto{\pgfqpoint{4.747984in}{0.989957in}}%
\pgfpathlineto{\pgfqpoint{4.748752in}{0.996239in}}%
\pgfpathlineto{\pgfqpoint{4.749137in}{0.994779in}}%
\pgfpathlineto{\pgfqpoint{4.749521in}{0.996200in}}%
\pgfpathlineto{\pgfqpoint{4.750290in}{1.003383in}}%
\pgfpathlineto{\pgfqpoint{4.751058in}{1.000401in}}%
\pgfpathlineto{\pgfqpoint{4.751635in}{1.001269in}}%
\pgfpathlineto{\pgfqpoint{4.752788in}{0.994312in}}%
\pgfpathlineto{\pgfqpoint{4.753172in}{0.995353in}}%
\pgfpathlineto{\pgfqpoint{4.754517in}{0.999643in}}%
\pgfpathlineto{\pgfqpoint{4.754902in}{1.001464in}}%
\pgfpathlineto{\pgfqpoint{4.755094in}{0.998748in}}%
\pgfpathlineto{\pgfqpoint{4.760475in}{0.971531in}}%
\pgfpathlineto{\pgfqpoint{4.760859in}{0.973930in}}%
\pgfpathlineto{\pgfqpoint{4.761243in}{0.975238in}}%
\pgfpathlineto{\pgfqpoint{4.762588in}{0.963077in}}%
\pgfpathlineto{\pgfqpoint{4.763549in}{0.972232in}}%
\pgfpathlineto{\pgfqpoint{4.763934in}{0.970386in}}%
\pgfpathlineto{\pgfqpoint{4.764318in}{0.971202in}}%
\pgfpathlineto{\pgfqpoint{4.765279in}{0.967919in}}%
\pgfpathlineto{\pgfqpoint{4.768738in}{0.984615in}}%
\pgfpathlineto{\pgfqpoint{4.769122in}{0.982340in}}%
\pgfpathlineto{\pgfqpoint{4.770083in}{0.976356in}}%
\pgfpathlineto{\pgfqpoint{4.770467in}{0.978013in}}%
\pgfpathlineto{\pgfqpoint{4.770659in}{0.977770in}}%
\pgfpathlineto{\pgfqpoint{4.771812in}{0.985165in}}%
\pgfpathlineto{\pgfqpoint{4.772389in}{0.984921in}}%
\pgfpathlineto{\pgfqpoint{4.772965in}{0.980114in}}%
\pgfpathlineto{\pgfqpoint{4.773542in}{0.982985in}}%
\pgfpathlineto{\pgfqpoint{4.773734in}{0.982993in}}%
\pgfpathlineto{\pgfqpoint{4.776040in}{0.962453in}}%
\pgfpathlineto{\pgfqpoint{4.776425in}{0.963012in}}%
\pgfpathlineto{\pgfqpoint{4.776809in}{0.966203in}}%
\pgfpathlineto{\pgfqpoint{4.777193in}{0.960426in}}%
\pgfpathlineto{\pgfqpoint{4.777770in}{0.960013in}}%
\pgfpathlineto{\pgfqpoint{4.777962in}{0.960996in}}%
\pgfpathlineto{\pgfqpoint{4.778346in}{0.961548in}}%
\pgfpathlineto{\pgfqpoint{4.779884in}{0.952604in}}%
\pgfpathlineto{\pgfqpoint{4.780076in}{0.954332in}}%
\pgfpathlineto{\pgfqpoint{4.780268in}{0.950598in}}%
\pgfpathlineto{\pgfqpoint{4.780844in}{0.953284in}}%
\pgfpathlineto{\pgfqpoint{4.781805in}{0.949817in}}%
\pgfpathlineto{\pgfqpoint{4.781997in}{0.952357in}}%
\pgfpathlineto{\pgfqpoint{4.782766in}{0.952158in}}%
\pgfpathlineto{\pgfqpoint{4.783535in}{0.955557in}}%
\pgfpathlineto{\pgfqpoint{4.784496in}{0.952009in}}%
\pgfpathlineto{\pgfqpoint{4.784688in}{0.952960in}}%
\pgfpathlineto{\pgfqpoint{4.785072in}{0.959059in}}%
\pgfpathlineto{\pgfqpoint{4.786033in}{0.957185in}}%
\pgfpathlineto{\pgfqpoint{4.786609in}{0.959824in}}%
\pgfpathlineto{\pgfqpoint{4.786994in}{0.959074in}}%
\pgfpathlineto{\pgfqpoint{4.787762in}{0.951401in}}%
\pgfpathlineto{\pgfqpoint{4.788339in}{0.955395in}}%
\pgfpathlineto{\pgfqpoint{4.788531in}{0.954270in}}%
\pgfpathlineto{\pgfqpoint{4.788723in}{0.956540in}}%
\pgfpathlineto{\pgfqpoint{4.789492in}{0.965038in}}%
\pgfpathlineto{\pgfqpoint{4.789876in}{0.960236in}}%
\pgfpathlineto{\pgfqpoint{4.790068in}{0.960056in}}%
\pgfpathlineto{\pgfqpoint{4.791029in}{0.964136in}}%
\pgfpathlineto{\pgfqpoint{4.791606in}{0.963576in}}%
\pgfpathlineto{\pgfqpoint{4.791990in}{0.959014in}}%
\pgfpathlineto{\pgfqpoint{4.792182in}{0.955892in}}%
\pgfpathlineto{\pgfqpoint{4.792567in}{0.960678in}}%
\pgfpathlineto{\pgfqpoint{4.792951in}{0.958881in}}%
\pgfpathlineto{\pgfqpoint{4.793143in}{0.958754in}}%
\pgfpathlineto{\pgfqpoint{4.793335in}{0.957255in}}%
\pgfpathlineto{\pgfqpoint{4.793720in}{0.960030in}}%
\pgfpathlineto{\pgfqpoint{4.794104in}{0.959414in}}%
\pgfpathlineto{\pgfqpoint{4.794488in}{0.957058in}}%
\pgfpathlineto{\pgfqpoint{4.794873in}{0.959502in}}%
\pgfpathlineto{\pgfqpoint{4.795449in}{0.962451in}}%
\pgfpathlineto{\pgfqpoint{4.795833in}{0.959664in}}%
\pgfpathlineto{\pgfqpoint{4.796026in}{0.959166in}}%
\pgfpathlineto{\pgfqpoint{4.796218in}{0.960521in}}%
\pgfpathlineto{\pgfqpoint{4.797755in}{0.970838in}}%
\pgfpathlineto{\pgfqpoint{4.798140in}{0.969972in}}%
\pgfpathlineto{\pgfqpoint{4.800253in}{0.980591in}}%
\pgfpathlineto{\pgfqpoint{4.800446in}{0.979826in}}%
\pgfpathlineto{\pgfqpoint{4.800638in}{0.976573in}}%
\pgfpathlineto{\pgfqpoint{4.801022in}{0.980356in}}%
\pgfpathlineto{\pgfqpoint{4.801406in}{0.979608in}}%
\pgfpathlineto{\pgfqpoint{4.801599in}{0.981076in}}%
\pgfpathlineto{\pgfqpoint{4.801983in}{0.977391in}}%
\pgfpathlineto{\pgfqpoint{4.802175in}{0.976159in}}%
\pgfpathlineto{\pgfqpoint{4.802367in}{0.977789in}}%
\pgfpathlineto{\pgfqpoint{4.803328in}{0.982215in}}%
\pgfpathlineto{\pgfqpoint{4.803520in}{0.980504in}}%
\pgfpathlineto{\pgfqpoint{4.803905in}{0.976561in}}%
\pgfpathlineto{\pgfqpoint{4.804289in}{0.982543in}}%
\pgfpathlineto{\pgfqpoint{4.805634in}{0.987500in}}%
\pgfpathlineto{\pgfqpoint{4.805826in}{0.986814in}}%
\pgfpathlineto{\pgfqpoint{4.806211in}{0.984461in}}%
\pgfpathlineto{\pgfqpoint{4.806979in}{0.986361in}}%
\pgfpathlineto{\pgfqpoint{4.807171in}{0.986517in}}%
\pgfpathlineto{\pgfqpoint{4.809093in}{0.975108in}}%
\pgfpathlineto{\pgfqpoint{4.809477in}{0.977502in}}%
\pgfpathlineto{\pgfqpoint{4.810438in}{0.973402in}}%
\pgfpathlineto{\pgfqpoint{4.811976in}{0.981855in}}%
\pgfpathlineto{\pgfqpoint{4.812360in}{0.980187in}}%
\pgfpathlineto{\pgfqpoint{4.813129in}{0.975548in}}%
\pgfpathlineto{\pgfqpoint{4.813513in}{0.977132in}}%
\pgfpathlineto{\pgfqpoint{4.814089in}{0.982559in}}%
\pgfpathlineto{\pgfqpoint{4.815242in}{0.987091in}}%
\pgfpathlineto{\pgfqpoint{4.815435in}{0.986408in}}%
\pgfpathlineto{\pgfqpoint{4.815627in}{0.987477in}}%
\pgfpathlineto{\pgfqpoint{4.815819in}{0.985122in}}%
\pgfpathlineto{\pgfqpoint{4.816395in}{0.985679in}}%
\pgfpathlineto{\pgfqpoint{4.816780in}{0.981947in}}%
\pgfpathlineto{\pgfqpoint{4.817548in}{0.984710in}}%
\pgfpathlineto{\pgfqpoint{4.818125in}{0.987752in}}%
\pgfpathlineto{\pgfqpoint{4.818701in}{0.986568in}}%
\pgfpathlineto{\pgfqpoint{4.818894in}{0.985131in}}%
\pgfpathlineto{\pgfqpoint{4.819278in}{0.988006in}}%
\pgfpathlineto{\pgfqpoint{4.819662in}{0.986069in}}%
\pgfpathlineto{\pgfqpoint{4.819854in}{0.986508in}}%
\pgfpathlineto{\pgfqpoint{4.820047in}{0.990792in}}%
\pgfpathlineto{\pgfqpoint{4.820623in}{0.984909in}}%
\pgfpathlineto{\pgfqpoint{4.820815in}{0.986176in}}%
\pgfpathlineto{\pgfqpoint{4.822545in}{0.991703in}}%
\pgfpathlineto{\pgfqpoint{4.823121in}{0.987095in}}%
\pgfpathlineto{\pgfqpoint{4.823698in}{0.989691in}}%
\pgfpathlineto{\pgfqpoint{4.823890in}{0.989979in}}%
\pgfpathlineto{\pgfqpoint{4.825235in}{0.984884in}}%
\pgfpathlineto{\pgfqpoint{4.826580in}{0.992284in}}%
\pgfpathlineto{\pgfqpoint{4.827157in}{0.990892in}}%
\pgfpathlineto{\pgfqpoint{4.828310in}{0.986118in}}%
\pgfpathlineto{\pgfqpoint{4.829079in}{0.986704in}}%
\pgfpathlineto{\pgfqpoint{4.829655in}{0.987455in}}%
\pgfpathlineto{\pgfqpoint{4.830039in}{0.991316in}}%
\pgfpathlineto{\pgfqpoint{4.830808in}{0.988353in}}%
\pgfpathlineto{\pgfqpoint{4.831769in}{0.984478in}}%
\pgfpathlineto{\pgfqpoint{4.831961in}{0.986918in}}%
\pgfpathlineto{\pgfqpoint{4.833498in}{0.975132in}}%
\pgfpathlineto{\pgfqpoint{4.835036in}{0.966358in}}%
\pgfpathlineto{\pgfqpoint{4.837534in}{0.955090in}}%
\pgfpathlineto{\pgfqpoint{4.837918in}{0.950953in}}%
\pgfpathlineto{\pgfqpoint{4.838879in}{0.954093in}}%
\pgfpathlineto{\pgfqpoint{4.840609in}{0.966191in}}%
\pgfpathlineto{\pgfqpoint{4.841569in}{0.961663in}}%
\pgfpathlineto{\pgfqpoint{4.842146in}{0.962726in}}%
\pgfpathlineto{\pgfqpoint{4.844452in}{0.975585in}}%
\pgfpathlineto{\pgfqpoint{4.845221in}{0.971460in}}%
\pgfpathlineto{\pgfqpoint{4.846566in}{0.967698in}}%
\pgfpathlineto{\pgfqpoint{4.846950in}{0.968427in}}%
\pgfpathlineto{\pgfqpoint{4.847142in}{0.966702in}}%
\pgfpathlineto{\pgfqpoint{4.847335in}{0.969116in}}%
\pgfpathlineto{\pgfqpoint{4.847911in}{0.964775in}}%
\pgfpathlineto{\pgfqpoint{4.848295in}{0.968449in}}%
\pgfpathlineto{\pgfqpoint{4.850601in}{0.957778in}}%
\pgfpathlineto{\pgfqpoint{4.850986in}{0.958022in}}%
\pgfpathlineto{\pgfqpoint{4.851178in}{0.959885in}}%
\pgfpathlineto{\pgfqpoint{4.851754in}{0.958889in}}%
\pgfpathlineto{\pgfqpoint{4.851947in}{0.955885in}}%
\pgfpathlineto{\pgfqpoint{4.852715in}{0.959837in}}%
\pgfpathlineto{\pgfqpoint{4.853676in}{0.966182in}}%
\pgfpathlineto{\pgfqpoint{4.854445in}{0.965907in}}%
\pgfpathlineto{\pgfqpoint{4.855213in}{0.959215in}}%
\pgfpathlineto{\pgfqpoint{4.855790in}{0.961261in}}%
\pgfpathlineto{\pgfqpoint{4.855982in}{0.962507in}}%
\pgfpathlineto{\pgfqpoint{4.856366in}{0.960245in}}%
\pgfpathlineto{\pgfqpoint{4.856559in}{0.961029in}}%
\pgfpathlineto{\pgfqpoint{4.856751in}{0.959908in}}%
\pgfpathlineto{\pgfqpoint{4.857135in}{0.963287in}}%
\pgfpathlineto{\pgfqpoint{4.857519in}{0.966891in}}%
\pgfpathlineto{\pgfqpoint{4.858288in}{0.964212in}}%
\pgfpathlineto{\pgfqpoint{4.858672in}{0.966775in}}%
\pgfpathlineto{\pgfqpoint{4.859057in}{0.962513in}}%
\pgfpathlineto{\pgfqpoint{4.859633in}{0.959592in}}%
\pgfpathlineto{\pgfqpoint{4.860210in}{0.960516in}}%
\pgfpathlineto{\pgfqpoint{4.860402in}{0.962113in}}%
\pgfpathlineto{\pgfqpoint{4.861171in}{0.961168in}}%
\pgfpathlineto{\pgfqpoint{4.862516in}{0.952322in}}%
\pgfpathlineto{\pgfqpoint{4.862708in}{0.953336in}}%
\pgfpathlineto{\pgfqpoint{4.862900in}{0.953624in}}%
\pgfpathlineto{\pgfqpoint{4.863092in}{0.952434in}}%
\pgfpathlineto{\pgfqpoint{4.864630in}{0.942858in}}%
\pgfpathlineto{\pgfqpoint{4.864822in}{0.943400in}}%
\pgfpathlineto{\pgfqpoint{4.866551in}{0.951511in}}%
\pgfpathlineto{\pgfqpoint{4.867320in}{0.946915in}}%
\pgfpathlineto{\pgfqpoint{4.867896in}{0.949341in}}%
\pgfpathlineto{\pgfqpoint{4.868857in}{0.947323in}}%
\pgfpathlineto{\pgfqpoint{4.870202in}{0.958118in}}%
\pgfpathlineto{\pgfqpoint{4.870395in}{0.956810in}}%
\pgfpathlineto{\pgfqpoint{4.872701in}{0.971335in}}%
\pgfpathlineto{\pgfqpoint{4.873085in}{0.964999in}}%
\pgfpathlineto{\pgfqpoint{4.874622in}{0.954904in}}%
\pgfpathlineto{\pgfqpoint{4.873469in}{0.966036in}}%
\pgfpathlineto{\pgfqpoint{4.875583in}{0.959076in}}%
\pgfpathlineto{\pgfqpoint{4.875775in}{0.959319in}}%
\pgfpathlineto{\pgfqpoint{4.876160in}{0.963372in}}%
\pgfpathlineto{\pgfqpoint{4.876736in}{0.958872in}}%
\pgfpathlineto{\pgfqpoint{4.876928in}{0.958274in}}%
\pgfpathlineto{\pgfqpoint{4.878466in}{0.970022in}}%
\pgfpathlineto{\pgfqpoint{4.878658in}{0.968936in}}%
\pgfpathlineto{\pgfqpoint{4.879619in}{0.969641in}}%
\pgfpathlineto{\pgfqpoint{4.880003in}{0.965102in}}%
\pgfpathlineto{\pgfqpoint{4.880195in}{0.965920in}}%
\pgfpathlineto{\pgfqpoint{4.880387in}{0.964937in}}%
\pgfpathlineto{\pgfqpoint{4.880580in}{0.965069in}}%
\pgfpathlineto{\pgfqpoint{4.881348in}{0.960512in}}%
\pgfpathlineto{\pgfqpoint{4.881733in}{0.962032in}}%
\pgfpathlineto{\pgfqpoint{4.881925in}{0.964011in}}%
\pgfpathlineto{\pgfqpoint{4.882501in}{0.959362in}}%
\pgfpathlineto{\pgfqpoint{4.882886in}{0.958559in}}%
\pgfpathlineto{\pgfqpoint{4.883654in}{0.956178in}}%
\pgfpathlineto{\pgfqpoint{4.884039in}{0.958024in}}%
\pgfpathlineto{\pgfqpoint{4.885192in}{0.960787in}}%
\pgfpathlineto{\pgfqpoint{4.885384in}{0.958390in}}%
\pgfpathlineto{\pgfqpoint{4.885960in}{0.963234in}}%
\pgfpathlineto{\pgfqpoint{4.886537in}{0.964372in}}%
\pgfpathlineto{\pgfqpoint{4.886345in}{0.962668in}}%
\pgfpathlineto{\pgfqpoint{4.886729in}{0.963019in}}%
\pgfpathlineto{\pgfqpoint{4.888074in}{0.952499in}}%
\pgfpathlineto{\pgfqpoint{4.888266in}{0.954495in}}%
\pgfpathlineto{\pgfqpoint{4.889227in}{0.961547in}}%
\pgfpathlineto{\pgfqpoint{4.889804in}{0.961429in}}%
\pgfpathlineto{\pgfqpoint{4.891533in}{0.952799in}}%
\pgfpathlineto{\pgfqpoint{4.891917in}{0.954859in}}%
\pgfpathlineto{\pgfqpoint{4.892302in}{0.949769in}}%
\pgfpathlineto{\pgfqpoint{4.892494in}{0.949316in}}%
\pgfpathlineto{\pgfqpoint{4.892686in}{0.951349in}}%
\pgfpathlineto{\pgfqpoint{4.893455in}{0.953577in}}%
\pgfpathlineto{\pgfqpoint{4.893647in}{0.953320in}}%
\pgfpathlineto{\pgfqpoint{4.894416in}{0.946834in}}%
\pgfpathlineto{\pgfqpoint{4.894800in}{0.949982in}}%
\pgfpathlineto{\pgfqpoint{4.895377in}{0.951599in}}%
\pgfpathlineto{\pgfqpoint{4.895761in}{0.949326in}}%
\pgfpathlineto{\pgfqpoint{4.895953in}{0.949319in}}%
\pgfpathlineto{\pgfqpoint{4.896722in}{0.954148in}}%
\pgfpathlineto{\pgfqpoint{4.897298in}{0.952860in}}%
\pgfpathlineto{\pgfqpoint{4.897683in}{0.954256in}}%
\pgfpathlineto{\pgfqpoint{4.898067in}{0.949887in}}%
\pgfpathlineto{\pgfqpoint{4.898451in}{0.954651in}}%
\pgfpathlineto{\pgfqpoint{4.898836in}{0.951752in}}%
\pgfpathlineto{\pgfqpoint{4.899412in}{0.949193in}}%
\pgfpathlineto{\pgfqpoint{4.899796in}{0.950805in}}%
\pgfpathlineto{\pgfqpoint{4.900373in}{0.949092in}}%
\pgfpathlineto{\pgfqpoint{4.901334in}{0.955610in}}%
\pgfpathlineto{\pgfqpoint{4.901526in}{0.956708in}}%
\pgfpathlineto{\pgfqpoint{4.901718in}{0.954162in}}%
\pgfpathlineto{\pgfqpoint{4.901910in}{0.952142in}}%
\pgfpathlineto{\pgfqpoint{4.902295in}{0.955385in}}%
\pgfpathlineto{\pgfqpoint{4.902487in}{0.953956in}}%
\pgfpathlineto{\pgfqpoint{4.903255in}{0.958308in}}%
\pgfpathlineto{\pgfqpoint{4.903640in}{0.954367in}}%
\pgfpathlineto{\pgfqpoint{4.904216in}{0.954620in}}%
\pgfpathlineto{\pgfqpoint{4.904601in}{0.952603in}}%
\pgfpathlineto{\pgfqpoint{4.904793in}{0.954550in}}%
\pgfpathlineto{\pgfqpoint{4.905177in}{0.950134in}}%
\pgfpathlineto{\pgfqpoint{4.905369in}{0.948717in}}%
\pgfpathlineto{\pgfqpoint{4.905754in}{0.952555in}}%
\pgfpathlineto{\pgfqpoint{4.907099in}{0.959843in}}%
\pgfpathlineto{\pgfqpoint{4.907291in}{0.959546in}}%
\pgfpathlineto{\pgfqpoint{4.907675in}{0.961129in}}%
\pgfpathlineto{\pgfqpoint{4.908060in}{0.957620in}}%
\pgfpathlineto{\pgfqpoint{4.909597in}{0.948333in}}%
\pgfpathlineto{\pgfqpoint{4.910366in}{0.950909in}}%
\pgfpathlineto{\pgfqpoint{4.910750in}{0.949786in}}%
\pgfpathlineto{\pgfqpoint{4.910942in}{0.950866in}}%
\pgfpathlineto{\pgfqpoint{4.911134in}{0.950540in}}%
\pgfpathlineto{\pgfqpoint{4.911711in}{0.944166in}}%
\pgfpathlineto{\pgfqpoint{4.912287in}{0.948085in}}%
\pgfpathlineto{\pgfqpoint{4.912479in}{0.947592in}}%
\pgfpathlineto{\pgfqpoint{4.913056in}{0.952988in}}%
\pgfpathlineto{\pgfqpoint{4.913825in}{0.952739in}}%
\pgfpathlineto{\pgfqpoint{4.914978in}{0.958010in}}%
\pgfpathlineto{\pgfqpoint{4.915362in}{0.956284in}}%
\pgfpathlineto{\pgfqpoint{4.916131in}{0.960534in}}%
\pgfpathlineto{\pgfqpoint{4.916323in}{0.955898in}}%
\pgfpathlineto{\pgfqpoint{4.916515in}{0.959355in}}%
\pgfpathlineto{\pgfqpoint{4.917476in}{0.955221in}}%
\pgfpathlineto{\pgfqpoint{4.917668in}{0.957389in}}%
\pgfpathlineto{\pgfqpoint{4.917860in}{0.959797in}}%
\pgfpathlineto{\pgfqpoint{4.918437in}{0.957667in}}%
\pgfpathlineto{\pgfqpoint{4.919013in}{0.953312in}}%
\pgfpathlineto{\pgfqpoint{4.919590in}{0.957038in}}%
\pgfpathlineto{\pgfqpoint{4.919782in}{0.957383in}}%
\pgfpathlineto{\pgfqpoint{4.920935in}{0.954467in}}%
\pgfpathlineto{\pgfqpoint{4.920358in}{0.959041in}}%
\pgfpathlineto{\pgfqpoint{4.921127in}{0.954610in}}%
\pgfpathlineto{\pgfqpoint{4.921511in}{0.955926in}}%
\pgfpathlineto{\pgfqpoint{4.923049in}{0.960961in}}%
\pgfpathlineto{\pgfqpoint{4.924394in}{0.956608in}}%
\pgfpathlineto{\pgfqpoint{4.925163in}{0.961331in}}%
\pgfpathlineto{\pgfqpoint{4.925355in}{0.958508in}}%
\pgfpathlineto{\pgfqpoint{4.926892in}{0.947757in}}%
\pgfpathlineto{\pgfqpoint{4.927084in}{0.949881in}}%
\pgfpathlineto{\pgfqpoint{4.927853in}{0.954127in}}%
\pgfpathlineto{\pgfqpoint{4.928237in}{0.952709in}}%
\pgfpathlineto{\pgfqpoint{4.928622in}{0.949383in}}%
\pgfpathlineto{\pgfqpoint{4.929198in}{0.954229in}}%
\pgfpathlineto{\pgfqpoint{4.929775in}{0.957841in}}%
\pgfpathlineto{\pgfqpoint{4.930351in}{0.955594in}}%
\pgfpathlineto{\pgfqpoint{4.930543in}{0.956238in}}%
\pgfpathlineto{\pgfqpoint{4.930735in}{0.954231in}}%
\pgfpathlineto{\pgfqpoint{4.931696in}{0.945033in}}%
\pgfpathlineto{\pgfqpoint{4.932273in}{0.946659in}}%
\pgfpathlineto{\pgfqpoint{4.932465in}{0.947375in}}%
\pgfpathlineto{\pgfqpoint{4.932657in}{0.945580in}}%
\pgfpathlineto{\pgfqpoint{4.933234in}{0.946131in}}%
\pgfpathlineto{\pgfqpoint{4.936308in}{0.928405in}}%
\pgfpathlineto{\pgfqpoint{4.936500in}{0.929138in}}%
\pgfpathlineto{\pgfqpoint{4.937461in}{0.933309in}}%
\pgfpathlineto{\pgfqpoint{4.937653in}{0.931305in}}%
\pgfpathlineto{\pgfqpoint{4.938038in}{0.931543in}}%
\pgfpathlineto{\pgfqpoint{4.938230in}{0.933337in}}%
\pgfpathlineto{\pgfqpoint{4.938614in}{0.930428in}}%
\pgfpathlineto{\pgfqpoint{4.938806in}{0.932208in}}%
\pgfpathlineto{\pgfqpoint{4.939191in}{0.923438in}}%
\pgfpathlineto{\pgfqpoint{4.940152in}{0.925451in}}%
\pgfpathlineto{\pgfqpoint{4.940344in}{0.922890in}}%
\pgfpathlineto{\pgfqpoint{4.940728in}{0.925670in}}%
\pgfpathlineto{\pgfqpoint{4.941112in}{0.924972in}}%
\pgfpathlineto{\pgfqpoint{4.941305in}{0.925536in}}%
\pgfpathlineto{\pgfqpoint{4.941497in}{0.924161in}}%
\pgfpathlineto{\pgfqpoint{4.941689in}{0.922160in}}%
\pgfpathlineto{\pgfqpoint{4.942265in}{0.927083in}}%
\pgfpathlineto{\pgfqpoint{4.942458in}{0.927014in}}%
\pgfpathlineto{\pgfqpoint{4.943803in}{0.922067in}}%
\pgfpathlineto{\pgfqpoint{4.943995in}{0.922453in}}%
\pgfpathlineto{\pgfqpoint{4.944187in}{0.923847in}}%
\pgfpathlineto{\pgfqpoint{4.944572in}{0.920749in}}%
\pgfpathlineto{\pgfqpoint{4.944764in}{0.921917in}}%
\pgfpathlineto{\pgfqpoint{4.946109in}{0.912655in}}%
\pgfpathlineto{\pgfqpoint{4.946301in}{0.914476in}}%
\pgfpathlineto{\pgfqpoint{4.946493in}{0.918659in}}%
\pgfpathlineto{\pgfqpoint{4.947454in}{0.917499in}}%
\pgfpathlineto{\pgfqpoint{4.948799in}{0.907773in}}%
\pgfpathlineto{\pgfqpoint{4.948991in}{0.908182in}}%
\pgfpathlineto{\pgfqpoint{4.949568in}{0.905047in}}%
\pgfpathlineto{\pgfqpoint{4.950144in}{0.906200in}}%
\pgfpathlineto{\pgfqpoint{4.950529in}{0.908087in}}%
\pgfpathlineto{\pgfqpoint{4.950913in}{0.905888in}}%
\pgfpathlineto{\pgfqpoint{4.952643in}{0.895809in}}%
\pgfpathlineto{\pgfqpoint{4.952835in}{0.895532in}}%
\pgfpathlineto{\pgfqpoint{4.953027in}{0.898224in}}%
\pgfpathlineto{\pgfqpoint{4.953411in}{0.893521in}}%
\pgfpathlineto{\pgfqpoint{4.953796in}{0.897653in}}%
\pgfpathlineto{\pgfqpoint{4.954756in}{0.892403in}}%
\pgfpathlineto{\pgfqpoint{4.955333in}{0.893764in}}%
\pgfpathlineto{\pgfqpoint{4.955717in}{0.892711in}}%
\pgfpathlineto{\pgfqpoint{4.957062in}{0.898439in}}%
\pgfpathlineto{\pgfqpoint{4.958600in}{0.905723in}}%
\pgfpathlineto{\pgfqpoint{4.958792in}{0.906974in}}%
\pgfpathlineto{\pgfqpoint{4.959368in}{0.903576in}}%
\pgfpathlineto{\pgfqpoint{4.959945in}{0.904523in}}%
\pgfpathlineto{\pgfqpoint{4.960329in}{0.901421in}}%
\pgfpathlineto{\pgfqpoint{4.960906in}{0.908357in}}%
\pgfpathlineto{\pgfqpoint{4.961482in}{0.905860in}}%
\pgfpathlineto{\pgfqpoint{4.963020in}{0.900988in}}%
\pgfpathlineto{\pgfqpoint{4.963212in}{0.900845in}}%
\pgfpathlineto{\pgfqpoint{4.963404in}{0.902559in}}%
\pgfpathlineto{\pgfqpoint{4.963596in}{0.899728in}}%
\pgfpathlineto{\pgfqpoint{4.964173in}{0.900499in}}%
\pgfpathlineto{\pgfqpoint{4.964365in}{0.898962in}}%
\pgfpathlineto{\pgfqpoint{4.964749in}{0.903067in}}%
\pgfpathlineto{\pgfqpoint{4.964941in}{0.905068in}}%
\pgfpathlineto{\pgfqpoint{4.965133in}{0.900215in}}%
\pgfpathlineto{\pgfqpoint{4.965326in}{0.896872in}}%
\pgfpathlineto{\pgfqpoint{4.965710in}{0.901883in}}%
\pgfpathlineto{\pgfqpoint{4.966094in}{0.900603in}}%
\pgfpathlineto{\pgfqpoint{4.966286in}{0.900219in}}%
\pgfpathlineto{\pgfqpoint{4.966671in}{0.901529in}}%
\pgfpathlineto{\pgfqpoint{4.966863in}{0.901708in}}%
\pgfpathlineto{\pgfqpoint{4.967055in}{0.900707in}}%
\pgfpathlineto{\pgfqpoint{4.968400in}{0.907805in}}%
\pgfpathlineto{\pgfqpoint{4.968592in}{0.909219in}}%
\pgfpathlineto{\pgfqpoint{4.968785in}{0.904509in}}%
\pgfpathlineto{\pgfqpoint{4.969169in}{0.906232in}}%
\pgfpathlineto{\pgfqpoint{4.971091in}{0.898034in}}%
\pgfpathlineto{\pgfqpoint{4.971283in}{0.900725in}}%
\pgfpathlineto{\pgfqpoint{4.972052in}{0.897792in}}%
\pgfpathlineto{\pgfqpoint{4.972244in}{0.898818in}}%
\pgfpathlineto{\pgfqpoint{4.972436in}{0.899397in}}%
\pgfpathlineto{\pgfqpoint{4.972628in}{0.896221in}}%
\pgfpathlineto{\pgfqpoint{4.973397in}{0.900253in}}%
\pgfpathlineto{\pgfqpoint{4.973589in}{0.898052in}}%
\pgfpathlineto{\pgfqpoint{4.973781in}{0.897906in}}%
\pgfpathlineto{\pgfqpoint{4.975126in}{0.903550in}}%
\pgfpathlineto{\pgfqpoint{4.976856in}{0.894169in}}%
\pgfpathlineto{\pgfqpoint{4.978201in}{0.887293in}}%
\pgfpathlineto{\pgfqpoint{4.978393in}{0.888597in}}%
\pgfpathlineto{\pgfqpoint{4.978585in}{0.890244in}}%
\pgfpathlineto{\pgfqpoint{4.978970in}{0.884162in}}%
\pgfpathlineto{\pgfqpoint{4.980699in}{0.878550in}}%
\pgfpathlineto{\pgfqpoint{4.981468in}{0.885407in}}%
\pgfpathlineto{\pgfqpoint{4.982429in}{0.885217in}}%
\pgfpathlineto{\pgfqpoint{4.982813in}{0.883765in}}%
\pgfpathlineto{\pgfqpoint{4.983005in}{0.884381in}}%
\pgfpathlineto{\pgfqpoint{4.983966in}{0.890807in}}%
\pgfpathlineto{\pgfqpoint{4.984158in}{0.887867in}}%
\pgfpathlineto{\pgfqpoint{4.984735in}{0.888318in}}%
\pgfpathlineto{\pgfqpoint{4.985695in}{0.892116in}}%
\pgfpathlineto{\pgfqpoint{4.985888in}{0.891369in}}%
\pgfpathlineto{\pgfqpoint{4.986656in}{0.891922in}}%
\pgfpathlineto{\pgfqpoint{4.987041in}{0.886899in}}%
\pgfpathlineto{\pgfqpoint{4.987425in}{0.891756in}}%
\pgfpathlineto{\pgfqpoint{4.988578in}{0.890838in}}%
\pgfpathlineto{\pgfqpoint{4.989347in}{0.886158in}}%
\pgfpathlineto{\pgfqpoint{4.989923in}{0.887029in}}%
\pgfpathlineto{\pgfqpoint{4.991268in}{0.878348in}}%
\pgfpathlineto{\pgfqpoint{4.991653in}{0.882081in}}%
\pgfpathlineto{\pgfqpoint{4.992229in}{0.878120in}}%
\pgfpathlineto{\pgfqpoint{4.992421in}{0.876029in}}%
\pgfpathlineto{\pgfqpoint{4.993190in}{0.879708in}}%
\pgfpathlineto{\pgfqpoint{4.993574in}{0.881539in}}%
\pgfpathlineto{\pgfqpoint{4.993767in}{0.880422in}}%
\pgfpathlineto{\pgfqpoint{4.994343in}{0.874831in}}%
\pgfpathlineto{\pgfqpoint{4.994727in}{0.878125in}}%
\pgfpathlineto{\pgfqpoint{4.995112in}{0.883072in}}%
\pgfpathlineto{\pgfqpoint{4.995880in}{0.880342in}}%
\pgfpathlineto{\pgfqpoint{4.996073in}{0.879793in}}%
\pgfpathlineto{\pgfqpoint{4.997610in}{0.865454in}}%
\pgfpathlineto{\pgfqpoint{4.997802in}{0.865600in}}%
\pgfpathlineto{\pgfqpoint{4.999532in}{0.873555in}}%
\pgfpathlineto{\pgfqpoint{5.000108in}{0.874060in}}%
\pgfpathlineto{\pgfqpoint{5.000492in}{0.872103in}}%
\pgfpathlineto{\pgfqpoint{5.001645in}{0.879240in}}%
\pgfpathlineto{\pgfqpoint{5.002030in}{0.878992in}}%
\pgfpathlineto{\pgfqpoint{5.002991in}{0.872812in}}%
\pgfpathlineto{\pgfqpoint{5.003567in}{0.876284in}}%
\pgfpathlineto{\pgfqpoint{5.003951in}{0.881170in}}%
\pgfpathlineto{\pgfqpoint{5.004528in}{0.875061in}}%
\pgfpathlineto{\pgfqpoint{5.005681in}{0.871512in}}%
\pgfpathlineto{\pgfqpoint{5.005873in}{0.871748in}}%
\pgfpathlineto{\pgfqpoint{5.006642in}{0.868864in}}%
\pgfpathlineto{\pgfqpoint{5.006257in}{0.871895in}}%
\pgfpathlineto{\pgfqpoint{5.006834in}{0.870532in}}%
\pgfpathlineto{\pgfqpoint{5.007987in}{0.878449in}}%
\pgfpathlineto{\pgfqpoint{5.008179in}{0.874503in}}%
\pgfpathlineto{\pgfqpoint{5.008756in}{0.875175in}}%
\pgfpathlineto{\pgfqpoint{5.009332in}{0.873921in}}%
\pgfpathlineto{\pgfqpoint{5.010293in}{0.863886in}}%
\pgfpathlineto{\pgfqpoint{5.010869in}{0.867855in}}%
\pgfpathlineto{\pgfqpoint{5.011254in}{0.866580in}}%
\pgfpathlineto{\pgfqpoint{5.011446in}{0.866990in}}%
\pgfpathlineto{\pgfqpoint{5.012215in}{0.863431in}}%
\pgfpathlineto{\pgfqpoint{5.012022in}{0.867224in}}%
\pgfpathlineto{\pgfqpoint{5.012599in}{0.864895in}}%
\pgfpathlineto{\pgfqpoint{5.012983in}{0.866662in}}%
\pgfpathlineto{\pgfqpoint{5.013752in}{0.860650in}}%
\pgfpathlineto{\pgfqpoint{5.014905in}{0.852225in}}%
\pgfpathlineto{\pgfqpoint{5.015097in}{0.856325in}}%
\pgfpathlineto{\pgfqpoint{5.015866in}{0.852107in}}%
\pgfpathlineto{\pgfqpoint{5.016058in}{0.852335in}}%
\pgfpathlineto{\pgfqpoint{5.016250in}{0.855079in}}%
\pgfpathlineto{\pgfqpoint{5.016827in}{0.849208in}}%
\pgfpathlineto{\pgfqpoint{5.017019in}{0.850980in}}%
\pgfpathlineto{\pgfqpoint{5.017403in}{0.845913in}}%
\pgfpathlineto{\pgfqpoint{5.018172in}{0.850431in}}%
\pgfpathlineto{\pgfqpoint{5.021631in}{0.833695in}}%
\pgfpathlineto{\pgfqpoint{5.022400in}{0.836236in}}%
\pgfpathlineto{\pgfqpoint{5.022592in}{0.837088in}}%
\pgfpathlineto{\pgfqpoint{5.022784in}{0.833515in}}%
\pgfpathlineto{\pgfqpoint{5.023553in}{0.831359in}}%
\pgfpathlineto{\pgfqpoint{5.023360in}{0.833818in}}%
\pgfpathlineto{\pgfqpoint{5.023745in}{0.831535in}}%
\pgfpathlineto{\pgfqpoint{5.023937in}{0.833593in}}%
\pgfpathlineto{\pgfqpoint{5.024129in}{0.828869in}}%
\pgfpathlineto{\pgfqpoint{5.024513in}{0.830535in}}%
\pgfpathlineto{\pgfqpoint{5.026051in}{0.826452in}}%
\pgfpathlineto{\pgfqpoint{5.026243in}{0.826702in}}%
\pgfpathlineto{\pgfqpoint{5.026435in}{0.827303in}}%
\pgfpathlineto{\pgfqpoint{5.026627in}{0.826119in}}%
\pgfpathlineto{\pgfqpoint{5.026819in}{0.823363in}}%
\pgfpathlineto{\pgfqpoint{5.027204in}{0.827124in}}%
\pgfpathlineto{\pgfqpoint{5.027780in}{0.824920in}}%
\pgfpathlineto{\pgfqpoint{5.027972in}{0.825048in}}%
\pgfpathlineto{\pgfqpoint{5.028741in}{0.829174in}}%
\pgfpathlineto{\pgfqpoint{5.029318in}{0.828661in}}%
\pgfpathlineto{\pgfqpoint{5.029702in}{0.824724in}}%
\pgfpathlineto{\pgfqpoint{5.030278in}{0.827652in}}%
\pgfpathlineto{\pgfqpoint{5.030471in}{0.830642in}}%
\pgfpathlineto{\pgfqpoint{5.031047in}{0.824699in}}%
\pgfpathlineto{\pgfqpoint{5.031239in}{0.824440in}}%
\pgfpathlineto{\pgfqpoint{5.032200in}{0.818559in}}%
\pgfpathlineto{\pgfqpoint{5.032392in}{0.821077in}}%
\pgfpathlineto{\pgfqpoint{5.033737in}{0.830607in}}%
\pgfpathlineto{\pgfqpoint{5.033930in}{0.828607in}}%
\pgfpathlineto{\pgfqpoint{5.035467in}{0.819466in}}%
\pgfpathlineto{\pgfqpoint{5.036236in}{0.819898in}}%
\pgfpathlineto{\pgfqpoint{5.036428in}{0.821647in}}%
\pgfpathlineto{\pgfqpoint{5.036812in}{0.819809in}}%
\pgfpathlineto{\pgfqpoint{5.037196in}{0.820352in}}%
\pgfpathlineto{\pgfqpoint{5.037389in}{0.819108in}}%
\pgfpathlineto{\pgfqpoint{5.037581in}{0.821381in}}%
\pgfpathlineto{\pgfqpoint{5.037773in}{0.820729in}}%
\pgfpathlineto{\pgfqpoint{5.037965in}{0.824306in}}%
\pgfpathlineto{\pgfqpoint{5.038734in}{0.820949in}}%
\pgfpathlineto{\pgfqpoint{5.038926in}{0.821237in}}%
\pgfpathlineto{\pgfqpoint{5.039118in}{0.820818in}}%
\pgfpathlineto{\pgfqpoint{5.040848in}{0.813425in}}%
\pgfpathlineto{\pgfqpoint{5.041616in}{0.817859in}}%
\pgfpathlineto{\pgfqpoint{5.042193in}{0.817682in}}%
\pgfpathlineto{\pgfqpoint{5.042577in}{0.820012in}}%
\pgfpathlineto{\pgfqpoint{5.043154in}{0.816977in}}%
\pgfpathlineto{\pgfqpoint{5.043346in}{0.818007in}}%
\pgfpathlineto{\pgfqpoint{5.044691in}{0.813710in}}%
\pgfpathlineto{\pgfqpoint{5.044883in}{0.814644in}}%
\pgfpathlineto{\pgfqpoint{5.045268in}{0.816113in}}%
\pgfpathlineto{\pgfqpoint{5.045844in}{0.814708in}}%
\pgfpathlineto{\pgfqpoint{5.046228in}{0.812602in}}%
\pgfpathlineto{\pgfqpoint{5.046613in}{0.815316in}}%
\pgfpathlineto{\pgfqpoint{5.047381in}{0.815621in}}%
\pgfpathlineto{\pgfqpoint{5.048342in}{0.823481in}}%
\pgfpathlineto{\pgfqpoint{5.048727in}{0.822533in}}%
\pgfpathlineto{\pgfqpoint{5.051609in}{0.803987in}}%
\pgfpathlineto{\pgfqpoint{5.052378in}{0.805931in}}%
\pgfpathlineto{\pgfqpoint{5.052762in}{0.804098in}}%
\pgfpathlineto{\pgfqpoint{5.053723in}{0.794408in}}%
\pgfpathlineto{\pgfqpoint{5.054684in}{0.796089in}}%
\pgfpathlineto{\pgfqpoint{5.056029in}{0.803980in}}%
\pgfpathlineto{\pgfqpoint{5.056221in}{0.800995in}}%
\pgfpathlineto{\pgfqpoint{5.056413in}{0.800024in}}%
\pgfpathlineto{\pgfqpoint{5.056605in}{0.803527in}}%
\pgfpathlineto{\pgfqpoint{5.056798in}{0.803029in}}%
\pgfpathlineto{\pgfqpoint{5.057374in}{0.806909in}}%
\pgfpathlineto{\pgfqpoint{5.057566in}{0.805266in}}%
\pgfpathlineto{\pgfqpoint{5.058527in}{0.811111in}}%
\pgfpathlineto{\pgfqpoint{5.058719in}{0.809976in}}%
\pgfpathlineto{\pgfqpoint{5.059296in}{0.813068in}}%
\pgfpathlineto{\pgfqpoint{5.059872in}{0.806686in}}%
\pgfpathlineto{\pgfqpoint{5.060257in}{0.805753in}}%
\pgfpathlineto{\pgfqpoint{5.060449in}{0.805393in}}%
\pgfpathlineto{\pgfqpoint{5.060641in}{0.802475in}}%
\pgfpathlineto{\pgfqpoint{5.061410in}{0.805360in}}%
\pgfpathlineto{\pgfqpoint{5.062178in}{0.805600in}}%
\pgfpathlineto{\pgfqpoint{5.062755in}{0.810171in}}%
\pgfpathlineto{\pgfqpoint{5.063331in}{0.808008in}}%
\pgfpathlineto{\pgfqpoint{5.063716in}{0.806399in}}%
\pgfpathlineto{\pgfqpoint{5.064100in}{0.808833in}}%
\pgfpathlineto{\pgfqpoint{5.064484in}{0.810852in}}%
\pgfpathlineto{\pgfqpoint{5.064676in}{0.809280in}}%
\pgfpathlineto{\pgfqpoint{5.066406in}{0.798834in}}%
\pgfpathlineto{\pgfqpoint{5.066790in}{0.799567in}}%
\pgfpathlineto{\pgfqpoint{5.067367in}{0.805333in}}%
\pgfpathlineto{\pgfqpoint{5.067943in}{0.802757in}}%
\pgfpathlineto{\pgfqpoint{5.068136in}{0.803115in}}%
\pgfpathlineto{\pgfqpoint{5.068328in}{0.802127in}}%
\pgfpathlineto{\pgfqpoint{5.068520in}{0.800212in}}%
\pgfpathlineto{\pgfqpoint{5.069096in}{0.803599in}}%
\pgfpathlineto{\pgfqpoint{5.069289in}{0.802545in}}%
\pgfpathlineto{\pgfqpoint{5.069673in}{0.802139in}}%
\pgfpathlineto{\pgfqpoint{5.070057in}{0.803090in}}%
\pgfpathlineto{\pgfqpoint{5.070634in}{0.799540in}}%
\pgfpathlineto{\pgfqpoint{5.071210in}{0.800696in}}%
\pgfpathlineto{\pgfqpoint{5.071402in}{0.802155in}}%
\pgfpathlineto{\pgfqpoint{5.071979in}{0.799131in}}%
\pgfpathlineto{\pgfqpoint{5.072171in}{0.800036in}}%
\pgfpathlineto{\pgfqpoint{5.072748in}{0.795596in}}%
\pgfpathlineto{\pgfqpoint{5.073132in}{0.799991in}}%
\pgfpathlineto{\pgfqpoint{5.073324in}{0.799895in}}%
\pgfpathlineto{\pgfqpoint{5.073516in}{0.797990in}}%
\pgfpathlineto{\pgfqpoint{5.074093in}{0.803322in}}%
\pgfpathlineto{\pgfqpoint{5.074861in}{0.806132in}}%
\pgfpathlineto{\pgfqpoint{5.075054in}{0.802476in}}%
\pgfpathlineto{\pgfqpoint{5.075438in}{0.803160in}}%
\pgfpathlineto{\pgfqpoint{5.076207in}{0.794809in}}%
\pgfpathlineto{\pgfqpoint{5.076591in}{0.800565in}}%
\pgfpathlineto{\pgfqpoint{5.076975in}{0.799677in}}%
\pgfpathlineto{\pgfqpoint{5.078705in}{0.806606in}}%
\pgfpathlineto{\pgfqpoint{5.079281in}{0.799137in}}%
\pgfpathlineto{\pgfqpoint{5.080050in}{0.800875in}}%
\pgfpathlineto{\pgfqpoint{5.080242in}{0.799602in}}%
\pgfpathlineto{\pgfqpoint{5.080626in}{0.803194in}}%
\pgfpathlineto{\pgfqpoint{5.080819in}{0.803939in}}%
\pgfpathlineto{\pgfqpoint{5.081011in}{0.801564in}}%
\pgfpathlineto{\pgfqpoint{5.084278in}{0.784208in}}%
\pgfpathlineto{\pgfqpoint{5.081395in}{0.804291in}}%
\pgfpathlineto{\pgfqpoint{5.084662in}{0.786580in}}%
\pgfpathlineto{\pgfqpoint{5.085238in}{0.790617in}}%
\pgfpathlineto{\pgfqpoint{5.085815in}{0.787549in}}%
\pgfpathlineto{\pgfqpoint{5.087352in}{0.782523in}}%
\pgfpathlineto{\pgfqpoint{5.086199in}{0.787895in}}%
\pgfpathlineto{\pgfqpoint{5.087544in}{0.783123in}}%
\pgfpathlineto{\pgfqpoint{5.088505in}{0.787761in}}%
\pgfpathlineto{\pgfqpoint{5.088697in}{0.785074in}}%
\pgfpathlineto{\pgfqpoint{5.089274in}{0.789093in}}%
\pgfpathlineto{\pgfqpoint{5.089658in}{0.784234in}}%
\pgfpathlineto{\pgfqpoint{5.090235in}{0.785534in}}%
\pgfpathlineto{\pgfqpoint{5.090619in}{0.784637in}}%
\pgfpathlineto{\pgfqpoint{5.090811in}{0.781838in}}%
\pgfpathlineto{\pgfqpoint{5.091388in}{0.788157in}}%
\pgfpathlineto{\pgfqpoint{5.091772in}{0.784037in}}%
\pgfpathlineto{\pgfqpoint{5.092733in}{0.785063in}}%
\pgfpathlineto{\pgfqpoint{5.093310in}{0.776590in}}%
\pgfpathlineto{\pgfqpoint{5.093886in}{0.775941in}}%
\pgfpathlineto{\pgfqpoint{5.094270in}{0.776742in}}%
\pgfpathlineto{\pgfqpoint{5.095808in}{0.783905in}}%
\pgfpathlineto{\pgfqpoint{5.096000in}{0.781348in}}%
\pgfpathlineto{\pgfqpoint{5.096576in}{0.784790in}}%
\pgfpathlineto{\pgfqpoint{5.096769in}{0.782392in}}%
\pgfpathlineto{\pgfqpoint{5.096961in}{0.778905in}}%
\pgfpathlineto{\pgfqpoint{5.097729in}{0.784372in}}%
\pgfpathlineto{\pgfqpoint{5.099075in}{0.790251in}}%
\pgfpathlineto{\pgfqpoint{5.100420in}{0.786095in}}%
\pgfpathlineto{\pgfqpoint{5.100804in}{0.790358in}}%
\pgfpathlineto{\pgfqpoint{5.101381in}{0.784071in}}%
\pgfpathlineto{\pgfqpoint{5.102918in}{0.794134in}}%
\pgfpathlineto{\pgfqpoint{5.103879in}{0.785518in}}%
\pgfpathlineto{\pgfqpoint{5.104071in}{0.790859in}}%
\pgfpathlineto{\pgfqpoint{5.105416in}{0.797609in}}%
\pgfpathlineto{\pgfqpoint{5.107146in}{0.791998in}}%
\pgfpathlineto{\pgfqpoint{5.108299in}{0.796220in}}%
\pgfpathlineto{\pgfqpoint{5.107722in}{0.791452in}}%
\pgfpathlineto{\pgfqpoint{5.108491in}{0.794882in}}%
\pgfpathlineto{\pgfqpoint{5.110220in}{0.789018in}}%
\pgfpathlineto{\pgfqpoint{5.109067in}{0.796653in}}%
\pgfpathlineto{\pgfqpoint{5.110605in}{0.790281in}}%
\pgfpathlineto{\pgfqpoint{5.111565in}{0.795997in}}%
\pgfpathlineto{\pgfqpoint{5.111758in}{0.794958in}}%
\pgfpathlineto{\pgfqpoint{5.111950in}{0.790777in}}%
\pgfpathlineto{\pgfqpoint{5.112718in}{0.794249in}}%
\pgfpathlineto{\pgfqpoint{5.113295in}{0.795265in}}%
\pgfpathlineto{\pgfqpoint{5.113487in}{0.795095in}}%
\pgfpathlineto{\pgfqpoint{5.113679in}{0.791878in}}%
\pgfpathlineto{\pgfqpoint{5.114448in}{0.793677in}}%
\pgfpathlineto{\pgfqpoint{5.115985in}{0.803016in}}%
\pgfpathlineto{\pgfqpoint{5.116178in}{0.800694in}}%
\pgfpathlineto{\pgfqpoint{5.116946in}{0.802512in}}%
\pgfpathlineto{\pgfqpoint{5.117715in}{0.792044in}}%
\pgfpathlineto{\pgfqpoint{5.118291in}{0.799773in}}%
\pgfpathlineto{\pgfqpoint{5.118868in}{0.794193in}}%
\pgfpathlineto{\pgfqpoint{5.119252in}{0.791481in}}%
\pgfpathlineto{\pgfqpoint{5.119444in}{0.789651in}}%
\pgfpathlineto{\pgfqpoint{5.120213in}{0.792591in}}%
\pgfpathlineto{\pgfqpoint{5.120597in}{0.791749in}}%
\pgfpathlineto{\pgfqpoint{5.120790in}{0.795112in}}%
\pgfpathlineto{\pgfqpoint{5.121558in}{0.792644in}}%
\pgfpathlineto{\pgfqpoint{5.122327in}{0.787324in}}%
\pgfpathlineto{\pgfqpoint{5.122519in}{0.792668in}}%
\pgfpathlineto{\pgfqpoint{5.124056in}{0.795707in}}%
\pgfpathlineto{\pgfqpoint{5.122903in}{0.791780in}}%
\pgfpathlineto{\pgfqpoint{5.124249in}{0.795499in}}%
\pgfpathlineto{\pgfqpoint{5.124441in}{0.794334in}}%
\pgfpathlineto{\pgfqpoint{5.124825in}{0.798159in}}%
\pgfpathlineto{\pgfqpoint{5.125209in}{0.795965in}}%
\pgfpathlineto{\pgfqpoint{5.126747in}{0.801264in}}%
\pgfpathlineto{\pgfqpoint{5.127515in}{0.796482in}}%
\pgfpathlineto{\pgfqpoint{5.127900in}{0.800347in}}%
\pgfpathlineto{\pgfqpoint{5.128861in}{0.802752in}}%
\pgfpathlineto{\pgfqpoint{5.129053in}{0.804800in}}%
\pgfpathlineto{\pgfqpoint{5.129437in}{0.800136in}}%
\pgfpathlineto{\pgfqpoint{5.129629in}{0.800640in}}%
\pgfpathlineto{\pgfqpoint{5.130014in}{0.798509in}}%
\pgfpathlineto{\pgfqpoint{5.130590in}{0.801845in}}%
\pgfpathlineto{\pgfqpoint{5.131743in}{0.808237in}}%
\pgfpathlineto{\pgfqpoint{5.131167in}{0.800745in}}%
\pgfpathlineto{\pgfqpoint{5.132320in}{0.806820in}}%
\pgfpathlineto{\pgfqpoint{5.133857in}{0.798525in}}%
\pgfpathlineto{\pgfqpoint{5.134049in}{0.798601in}}%
\pgfpathlineto{\pgfqpoint{5.134433in}{0.799307in}}%
\pgfpathlineto{\pgfqpoint{5.134818in}{0.797729in}}%
\pgfpathlineto{\pgfqpoint{5.135010in}{0.796000in}}%
\pgfpathlineto{\pgfqpoint{5.135202in}{0.800950in}}%
\pgfpathlineto{\pgfqpoint{5.135779in}{0.804874in}}%
\pgfpathlineto{\pgfqpoint{5.136163in}{0.800200in}}%
\pgfpathlineto{\pgfqpoint{5.136355in}{0.802513in}}%
\pgfpathlineto{\pgfqpoint{5.136932in}{0.796699in}}%
\pgfpathlineto{\pgfqpoint{5.138277in}{0.798363in}}%
\pgfpathlineto{\pgfqpoint{5.138853in}{0.802685in}}%
\pgfpathlineto{\pgfqpoint{5.139430in}{0.801827in}}%
\pgfpathlineto{\pgfqpoint{5.140775in}{0.797480in}}%
\pgfpathlineto{\pgfqpoint{5.141159in}{0.797798in}}%
\pgfpathlineto{\pgfqpoint{5.144042in}{0.809549in}}%
\pgfpathlineto{\pgfqpoint{5.144811in}{0.806500in}}%
\pgfpathlineto{\pgfqpoint{5.146540in}{0.819612in}}%
\pgfpathlineto{\pgfqpoint{5.147309in}{0.815988in}}%
\pgfpathlineto{\pgfqpoint{5.146924in}{0.820652in}}%
\pgfpathlineto{\pgfqpoint{5.147501in}{0.816669in}}%
\pgfpathlineto{\pgfqpoint{5.147693in}{0.820459in}}%
\pgfpathlineto{\pgfqpoint{5.148462in}{0.814644in}}%
\pgfpathlineto{\pgfqpoint{5.148846in}{0.813762in}}%
\pgfpathlineto{\pgfqpoint{5.149999in}{0.808044in}}%
\pgfpathlineto{\pgfqpoint{5.150191in}{0.808411in}}%
\pgfpathlineto{\pgfqpoint{5.152113in}{0.815072in}}%
\pgfpathlineto{\pgfqpoint{5.152305in}{0.815258in}}%
\pgfpathlineto{\pgfqpoint{5.153842in}{0.805541in}}%
\pgfpathlineto{\pgfqpoint{5.154035in}{0.807992in}}%
\pgfpathlineto{\pgfqpoint{5.155380in}{0.819084in}}%
\pgfpathlineto{\pgfqpoint{5.155764in}{0.818443in}}%
\pgfpathlineto{\pgfqpoint{5.156533in}{0.814005in}}%
\pgfpathlineto{\pgfqpoint{5.157109in}{0.817104in}}%
\pgfpathlineto{\pgfqpoint{5.157301in}{0.817886in}}%
\pgfpathlineto{\pgfqpoint{5.157494in}{0.816905in}}%
\pgfpathlineto{\pgfqpoint{5.159031in}{0.806175in}}%
\pgfpathlineto{\pgfqpoint{5.160568in}{0.812895in}}%
\pgfpathlineto{\pgfqpoint{5.160953in}{0.814122in}}%
\pgfpathlineto{\pgfqpoint{5.161337in}{0.809525in}}%
\pgfpathlineto{\pgfqpoint{5.161913in}{0.810897in}}%
\pgfpathlineto{\pgfqpoint{5.162682in}{0.819580in}}%
\pgfpathlineto{\pgfqpoint{5.163066in}{0.814933in}}%
\pgfpathlineto{\pgfqpoint{5.163643in}{0.816824in}}%
\pgfpathlineto{\pgfqpoint{5.163451in}{0.813737in}}%
\pgfpathlineto{\pgfqpoint{5.163835in}{0.815732in}}%
\pgfpathlineto{\pgfqpoint{5.165565in}{0.804274in}}%
\pgfpathlineto{\pgfqpoint{5.165757in}{0.804673in}}%
\pgfpathlineto{\pgfqpoint{5.165949in}{0.807356in}}%
\pgfpathlineto{\pgfqpoint{5.166718in}{0.803487in}}%
\pgfpathlineto{\pgfqpoint{5.168639in}{0.793582in}}%
\pgfpathlineto{\pgfqpoint{5.168832in}{0.795178in}}%
\pgfpathlineto{\pgfqpoint{5.169408in}{0.792592in}}%
\pgfpathlineto{\pgfqpoint{5.169600in}{0.794187in}}%
\pgfpathlineto{\pgfqpoint{5.172483in}{0.776588in}}%
\pgfpathlineto{\pgfqpoint{5.173059in}{0.778551in}}%
\pgfpathlineto{\pgfqpoint{5.173251in}{0.777203in}}%
\pgfpathlineto{\pgfqpoint{5.173828in}{0.773523in}}%
\pgfpathlineto{\pgfqpoint{5.174212in}{0.775715in}}%
\pgfpathlineto{\pgfqpoint{5.174597in}{0.778315in}}%
\pgfpathlineto{\pgfqpoint{5.174981in}{0.774990in}}%
\pgfpathlineto{\pgfqpoint{5.175557in}{0.770848in}}%
\pgfpathlineto{\pgfqpoint{5.175942in}{0.770928in}}%
\pgfpathlineto{\pgfqpoint{5.177287in}{0.780274in}}%
\pgfpathlineto{\pgfqpoint{5.177671in}{0.778859in}}%
\pgfpathlineto{\pgfqpoint{5.177863in}{0.779760in}}%
\pgfpathlineto{\pgfqpoint{5.178056in}{0.782399in}}%
\pgfpathlineto{\pgfqpoint{5.178440in}{0.778685in}}%
\pgfpathlineto{\pgfqpoint{5.179016in}{0.781107in}}%
\pgfpathlineto{\pgfqpoint{5.179209in}{0.778332in}}%
\pgfpathlineto{\pgfqpoint{5.179785in}{0.785569in}}%
\pgfpathlineto{\pgfqpoint{5.180169in}{0.783247in}}%
\pgfpathlineto{\pgfqpoint{5.180554in}{0.785748in}}%
\pgfpathlineto{\pgfqpoint{5.180938in}{0.784019in}}%
\pgfpathlineto{\pgfqpoint{5.181515in}{0.787995in}}%
\pgfpathlineto{\pgfqpoint{5.181707in}{0.786234in}}%
\pgfpathlineto{\pgfqpoint{5.181899in}{0.782574in}}%
\pgfpathlineto{\pgfqpoint{5.182668in}{0.788065in}}%
\pgfpathlineto{\pgfqpoint{5.182860in}{0.787876in}}%
\pgfpathlineto{\pgfqpoint{5.183052in}{0.784946in}}%
\pgfpathlineto{\pgfqpoint{5.183628in}{0.788952in}}%
\pgfpathlineto{\pgfqpoint{5.184013in}{0.786289in}}%
\pgfpathlineto{\pgfqpoint{5.184205in}{0.785330in}}%
\pgfpathlineto{\pgfqpoint{5.184781in}{0.787882in}}%
\pgfpathlineto{\pgfqpoint{5.185166in}{0.787486in}}%
\pgfpathlineto{\pgfqpoint{5.185358in}{0.789760in}}%
\pgfpathlineto{\pgfqpoint{5.185934in}{0.786279in}}%
\pgfpathlineto{\pgfqpoint{5.186127in}{0.787177in}}%
\pgfpathlineto{\pgfqpoint{5.186319in}{0.785532in}}%
\pgfpathlineto{\pgfqpoint{5.186895in}{0.787808in}}%
\pgfpathlineto{\pgfqpoint{5.187280in}{0.786261in}}%
\pgfpathlineto{\pgfqpoint{5.187472in}{0.785743in}}%
\pgfpathlineto{\pgfqpoint{5.187664in}{0.790133in}}%
\pgfpathlineto{\pgfqpoint{5.188048in}{0.783834in}}%
\pgfpathlineto{\pgfqpoint{5.188433in}{0.784453in}}%
\pgfpathlineto{\pgfqpoint{5.189009in}{0.793489in}}%
\pgfpathlineto{\pgfqpoint{5.190354in}{0.792053in}}%
\pgfpathlineto{\pgfqpoint{5.193045in}{0.800438in}}%
\pgfpathlineto{\pgfqpoint{5.193621in}{0.798553in}}%
\pgfpathlineto{\pgfqpoint{5.194390in}{0.796381in}}%
\pgfpathlineto{\pgfqpoint{5.194006in}{0.799299in}}%
\pgfpathlineto{\pgfqpoint{5.194582in}{0.798951in}}%
\pgfpathlineto{\pgfqpoint{5.195543in}{0.803481in}}%
\pgfpathlineto{\pgfqpoint{5.195735in}{0.801339in}}%
\pgfpathlineto{\pgfqpoint{5.197272in}{0.792475in}}%
\pgfpathlineto{\pgfqpoint{5.197465in}{0.793477in}}%
\pgfpathlineto{\pgfqpoint{5.197849in}{0.790743in}}%
\pgfpathlineto{\pgfqpoint{5.198618in}{0.792488in}}%
\pgfpathlineto{\pgfqpoint{5.199578in}{0.795340in}}%
\pgfpathlineto{\pgfqpoint{5.199002in}{0.791909in}}%
\pgfpathlineto{\pgfqpoint{5.199771in}{0.793230in}}%
\pgfpathlineto{\pgfqpoint{5.200347in}{0.793828in}}%
\pgfpathlineto{\pgfqpoint{5.201500in}{0.803191in}}%
\pgfpathlineto{\pgfqpoint{5.202269in}{0.802406in}}%
\pgfpathlineto{\pgfqpoint{5.203037in}{0.805960in}}%
\pgfpathlineto{\pgfqpoint{5.203614in}{0.805342in}}%
\pgfpathlineto{\pgfqpoint{5.204190in}{0.798618in}}%
\pgfpathlineto{\pgfqpoint{5.204767in}{0.800978in}}%
\pgfpathlineto{\pgfqpoint{5.205920in}{0.804127in}}%
\pgfpathlineto{\pgfqpoint{5.207842in}{0.796896in}}%
\pgfpathlineto{\pgfqpoint{5.208802in}{0.795329in}}%
\pgfpathlineto{\pgfqpoint{5.208418in}{0.798306in}}%
\pgfpathlineto{\pgfqpoint{5.208995in}{0.796486in}}%
\pgfpathlineto{\pgfqpoint{5.209379in}{0.799984in}}%
\pgfpathlineto{\pgfqpoint{5.209763in}{0.795671in}}%
\pgfpathlineto{\pgfqpoint{5.210148in}{0.795966in}}%
\pgfpathlineto{\pgfqpoint{5.211108in}{0.793739in}}%
\pgfpathlineto{\pgfqpoint{5.211685in}{0.797018in}}%
\pgfpathlineto{\pgfqpoint{5.211877in}{0.793248in}}%
\pgfpathlineto{\pgfqpoint{5.213607in}{0.779011in}}%
\pgfpathlineto{\pgfqpoint{5.213991in}{0.780708in}}%
\pgfpathlineto{\pgfqpoint{5.214568in}{0.782076in}}%
\pgfpathlineto{\pgfqpoint{5.215336in}{0.785882in}}%
\pgfpathlineto{\pgfqpoint{5.215144in}{0.780865in}}%
\pgfpathlineto{\pgfqpoint{5.215528in}{0.783764in}}%
\pgfpathlineto{\pgfqpoint{5.216681in}{0.777241in}}%
\pgfpathlineto{\pgfqpoint{5.215913in}{0.784367in}}%
\pgfpathlineto{\pgfqpoint{5.216874in}{0.778344in}}%
\pgfpathlineto{\pgfqpoint{5.217066in}{0.779233in}}%
\pgfpathlineto{\pgfqpoint{5.217258in}{0.777127in}}%
\pgfpathlineto{\pgfqpoint{5.218987in}{0.767173in}}%
\pgfpathlineto{\pgfqpoint{5.217834in}{0.777243in}}%
\pgfpathlineto{\pgfqpoint{5.219180in}{0.768061in}}%
\pgfpathlineto{\pgfqpoint{5.219948in}{0.768022in}}%
\pgfpathlineto{\pgfqpoint{5.220140in}{0.764986in}}%
\pgfpathlineto{\pgfqpoint{5.220717in}{0.768828in}}%
\pgfpathlineto{\pgfqpoint{5.220909in}{0.767628in}}%
\pgfpathlineto{\pgfqpoint{5.222831in}{0.780445in}}%
\pgfpathlineto{\pgfqpoint{5.223407in}{0.779883in}}%
\pgfpathlineto{\pgfqpoint{5.223984in}{0.777972in}}%
\pgfpathlineto{\pgfqpoint{5.223792in}{0.782583in}}%
\pgfpathlineto{\pgfqpoint{5.224176in}{0.781646in}}%
\pgfpathlineto{\pgfqpoint{5.224368in}{0.783127in}}%
\pgfpathlineto{\pgfqpoint{5.224560in}{0.779624in}}%
\pgfpathlineto{\pgfqpoint{5.224752in}{0.779930in}}%
\pgfpathlineto{\pgfqpoint{5.224945in}{0.777948in}}%
\pgfpathlineto{\pgfqpoint{5.225521in}{0.780860in}}%
\pgfpathlineto{\pgfqpoint{5.226674in}{0.789529in}}%
\pgfpathlineto{\pgfqpoint{5.226866in}{0.787579in}}%
\pgfpathlineto{\pgfqpoint{5.227251in}{0.782743in}}%
\pgfpathlineto{\pgfqpoint{5.228019in}{0.786030in}}%
\pgfpathlineto{\pgfqpoint{5.228404in}{0.784560in}}%
\pgfpathlineto{\pgfqpoint{5.228596in}{0.786249in}}%
\pgfpathlineto{\pgfqpoint{5.229364in}{0.786355in}}%
\pgfpathlineto{\pgfqpoint{5.229941in}{0.781491in}}%
\pgfpathlineto{\pgfqpoint{5.231286in}{0.775115in}}%
\pgfpathlineto{\pgfqpoint{5.230710in}{0.781831in}}%
\pgfpathlineto{\pgfqpoint{5.231670in}{0.775997in}}%
\pgfpathlineto{\pgfqpoint{5.232055in}{0.776314in}}%
\pgfpathlineto{\pgfqpoint{5.233208in}{0.770071in}}%
\pgfpathlineto{\pgfqpoint{5.233400in}{0.773167in}}%
\pgfpathlineto{\pgfqpoint{5.233592in}{0.772595in}}%
\pgfpathlineto{\pgfqpoint{5.235322in}{0.784762in}}%
\pgfpathlineto{\pgfqpoint{5.235706in}{0.787112in}}%
\pgfpathlineto{\pgfqpoint{5.236090in}{0.783202in}}%
\pgfpathlineto{\pgfqpoint{5.236282in}{0.784096in}}%
\pgfpathlineto{\pgfqpoint{5.239742in}{0.769012in}}%
\pgfpathlineto{\pgfqpoint{5.240126in}{0.771944in}}%
\pgfpathlineto{\pgfqpoint{5.240510in}{0.771622in}}%
\pgfpathlineto{\pgfqpoint{5.240702in}{0.773093in}}%
\pgfpathlineto{\pgfqpoint{5.240895in}{0.770525in}}%
\pgfpathlineto{\pgfqpoint{5.241471in}{0.775711in}}%
\pgfpathlineto{\pgfqpoint{5.242048in}{0.778636in}}%
\pgfpathlineto{\pgfqpoint{5.242240in}{0.778521in}}%
\pgfpathlineto{\pgfqpoint{5.242816in}{0.784555in}}%
\pgfpathlineto{\pgfqpoint{5.243393in}{0.779811in}}%
\pgfpathlineto{\pgfqpoint{5.244546in}{0.783038in}}%
\pgfpathlineto{\pgfqpoint{5.243777in}{0.777050in}}%
\pgfpathlineto{\pgfqpoint{5.244738in}{0.782534in}}%
\pgfpathlineto{\pgfqpoint{5.244930in}{0.781322in}}%
\pgfpathlineto{\pgfqpoint{5.245507in}{0.783468in}}%
\pgfpathlineto{\pgfqpoint{5.246083in}{0.787934in}}%
\pgfpathlineto{\pgfqpoint{5.246660in}{0.784493in}}%
\pgfpathlineto{\pgfqpoint{5.247620in}{0.784772in}}%
\pgfpathlineto{\pgfqpoint{5.247813in}{0.782442in}}%
\pgfpathlineto{\pgfqpoint{5.250119in}{0.795335in}}%
\pgfpathlineto{\pgfqpoint{5.250695in}{0.792328in}}%
\pgfpathlineto{\pgfqpoint{5.252425in}{0.787330in}}%
\pgfpathlineto{\pgfqpoint{5.253193in}{0.791181in}}%
\pgfpathlineto{\pgfqpoint{5.253962in}{0.792039in}}%
\pgfpathlineto{\pgfqpoint{5.253578in}{0.790235in}}%
\pgfpathlineto{\pgfqpoint{5.254154in}{0.790896in}}%
\pgfpathlineto{\pgfqpoint{5.254346in}{0.791095in}}%
\pgfpathlineto{\pgfqpoint{5.256268in}{0.770453in}}%
\pgfpathlineto{\pgfqpoint{5.257613in}{0.781756in}}%
\pgfpathlineto{\pgfqpoint{5.258382in}{0.781013in}}%
\pgfpathlineto{\pgfqpoint{5.258574in}{0.779860in}}%
\pgfpathlineto{\pgfqpoint{5.258958in}{0.782082in}}%
\pgfpathlineto{\pgfqpoint{5.260880in}{0.797578in}}%
\pgfpathlineto{\pgfqpoint{5.261072in}{0.796664in}}%
\pgfpathlineto{\pgfqpoint{5.261264in}{0.797453in}}%
\pgfpathlineto{\pgfqpoint{5.262225in}{0.801549in}}%
\pgfpathlineto{\pgfqpoint{5.262417in}{0.798073in}}%
\pgfpathlineto{\pgfqpoint{5.262610in}{0.799023in}}%
\pgfpathlineto{\pgfqpoint{5.262994in}{0.796883in}}%
\pgfpathlineto{\pgfqpoint{5.264531in}{0.790018in}}%
\pgfpathlineto{\pgfqpoint{5.266069in}{0.797312in}}%
\pgfpathlineto{\pgfqpoint{5.266261in}{0.797249in}}%
\pgfpathlineto{\pgfqpoint{5.267222in}{0.801946in}}%
\pgfpathlineto{\pgfqpoint{5.267414in}{0.798149in}}%
\pgfpathlineto{\pgfqpoint{5.267606in}{0.798554in}}%
\pgfpathlineto{\pgfqpoint{5.269143in}{0.787379in}}%
\pgfpathlineto{\pgfqpoint{5.269528in}{0.788807in}}%
\pgfpathlineto{\pgfqpoint{5.269912in}{0.787642in}}%
\pgfpathlineto{\pgfqpoint{5.271449in}{0.769852in}}%
\pgfpathlineto{\pgfqpoint{5.272026in}{0.770856in}}%
\pgfpathlineto{\pgfqpoint{5.273755in}{0.755455in}}%
\pgfpathlineto{\pgfqpoint{5.274140in}{0.759281in}}%
\pgfpathlineto{\pgfqpoint{5.274908in}{0.761187in}}%
\pgfpathlineto{\pgfqpoint{5.275100in}{0.757596in}}%
\pgfpathlineto{\pgfqpoint{5.275869in}{0.762845in}}%
\pgfpathlineto{\pgfqpoint{5.276446in}{0.760411in}}%
\pgfpathlineto{\pgfqpoint{5.276830in}{0.758921in}}%
\pgfpathlineto{\pgfqpoint{5.277214in}{0.761668in}}%
\pgfpathlineto{\pgfqpoint{5.277791in}{0.767538in}}%
\pgfpathlineto{\pgfqpoint{5.278367in}{0.763623in}}%
\pgfpathlineto{\pgfqpoint{5.278752in}{0.762181in}}%
\pgfpathlineto{\pgfqpoint{5.279136in}{0.763327in}}%
\pgfpathlineto{\pgfqpoint{5.280097in}{0.767947in}}%
\pgfpathlineto{\pgfqpoint{5.280289in}{0.765869in}}%
\pgfpathlineto{\pgfqpoint{5.280865in}{0.760291in}}%
\pgfpathlineto{\pgfqpoint{5.281250in}{0.765563in}}%
\pgfpathlineto{\pgfqpoint{5.281442in}{0.767281in}}%
\pgfpathlineto{\pgfqpoint{5.281826in}{0.762699in}}%
\pgfpathlineto{\pgfqpoint{5.282211in}{0.763000in}}%
\pgfpathlineto{\pgfqpoint{5.282595in}{0.762218in}}%
\pgfpathlineto{\pgfqpoint{5.283748in}{0.767010in}}%
\pgfpathlineto{\pgfqpoint{5.283940in}{0.766367in}}%
\pgfpathlineto{\pgfqpoint{5.284132in}{0.762913in}}%
\pgfpathlineto{\pgfqpoint{5.284901in}{0.765353in}}%
\pgfpathlineto{\pgfqpoint{5.286246in}{0.779086in}}%
\pgfpathlineto{\pgfqpoint{5.287015in}{0.777897in}}%
\pgfpathlineto{\pgfqpoint{5.287784in}{0.783319in}}%
\pgfpathlineto{\pgfqpoint{5.288168in}{0.778624in}}%
\pgfpathlineto{\pgfqpoint{5.288360in}{0.777695in}}%
\pgfpathlineto{\pgfqpoint{5.288744in}{0.779604in}}%
\pgfpathlineto{\pgfqpoint{5.289513in}{0.785726in}}%
\pgfpathlineto{\pgfqpoint{5.290090in}{0.785545in}}%
\pgfpathlineto{\pgfqpoint{5.290858in}{0.788861in}}%
\pgfpathlineto{\pgfqpoint{5.291627in}{0.788367in}}%
\pgfpathlineto{\pgfqpoint{5.292396in}{0.785115in}}%
\pgfpathlineto{\pgfqpoint{5.292588in}{0.786146in}}%
\pgfpathlineto{\pgfqpoint{5.292780in}{0.789658in}}%
\pgfpathlineto{\pgfqpoint{5.293549in}{0.785722in}}%
\pgfpathlineto{\pgfqpoint{5.293741in}{0.788445in}}%
\pgfpathlineto{\pgfqpoint{5.293933in}{0.786227in}}%
\pgfpathlineto{\pgfqpoint{5.294125in}{0.789144in}}%
\pgfpathlineto{\pgfqpoint{5.294702in}{0.786917in}}%
\pgfpathlineto{\pgfqpoint{5.295470in}{0.789099in}}%
\pgfpathlineto{\pgfqpoint{5.295662in}{0.787767in}}%
\pgfpathlineto{\pgfqpoint{5.296239in}{0.782378in}}%
\pgfpathlineto{\pgfqpoint{5.296623in}{0.786703in}}%
\pgfpathlineto{\pgfqpoint{5.297392in}{0.788733in}}%
\pgfpathlineto{\pgfqpoint{5.297008in}{0.785854in}}%
\pgfpathlineto{\pgfqpoint{5.297584in}{0.785846in}}%
\pgfpathlineto{\pgfqpoint{5.298737in}{0.780190in}}%
\pgfpathlineto{\pgfqpoint{5.298929in}{0.780554in}}%
\pgfpathlineto{\pgfqpoint{5.299121in}{0.784253in}}%
\pgfpathlineto{\pgfqpoint{5.299698in}{0.778823in}}%
\pgfpathlineto{\pgfqpoint{5.299890in}{0.779312in}}%
\pgfpathlineto{\pgfqpoint{5.300467in}{0.779577in}}%
\pgfpathlineto{\pgfqpoint{5.301235in}{0.769438in}}%
\pgfpathlineto{\pgfqpoint{5.302004in}{0.772572in}}%
\pgfpathlineto{\pgfqpoint{5.302196in}{0.772630in}}%
\pgfpathlineto{\pgfqpoint{5.302773in}{0.765620in}}%
\pgfpathlineto{\pgfqpoint{5.303349in}{0.768127in}}%
\pgfpathlineto{\pgfqpoint{5.303926in}{0.774019in}}%
\pgfpathlineto{\pgfqpoint{5.303926in}{0.774019in}}%
\pgfusepath{stroke}%
\end{pgfscope}%
\begin{pgfscope}%
\pgfpathrectangle{\pgfqpoint{3.286364in}{0.660000in}}{\pgfqpoint{2.113636in}{2.100000in}}%
\pgfusepath{clip}%
\pgfsetroundcap%
\pgfsetroundjoin%
\pgfsetlinewidth{0.602250pt}%
\definecolor{currentstroke}{rgb}{1.000000,0.498039,0.000000}%
\pgfsetstrokecolor{currentstroke}%
\pgfsetdash{}{0pt}%
\pgfpathmoveto{\pgfqpoint{3.382438in}{1.642559in}}%
\pgfpathlineto{\pgfqpoint{3.382822in}{1.645860in}}%
\pgfpathlineto{\pgfqpoint{3.383207in}{1.643435in}}%
\pgfpathlineto{\pgfqpoint{3.383591in}{1.637752in}}%
\pgfpathlineto{\pgfqpoint{3.384552in}{1.640417in}}%
\pgfpathlineto{\pgfqpoint{3.384936in}{1.641258in}}%
\pgfpathlineto{\pgfqpoint{3.385128in}{1.641695in}}%
\pgfpathlineto{\pgfqpoint{3.385321in}{1.639459in}}%
\pgfpathlineto{\pgfqpoint{3.385513in}{1.639540in}}%
\pgfpathlineto{\pgfqpoint{3.385897in}{1.636834in}}%
\pgfpathlineto{\pgfqpoint{3.386089in}{1.640731in}}%
\pgfpathlineto{\pgfqpoint{3.386281in}{1.639502in}}%
\pgfpathlineto{\pgfqpoint{3.387627in}{1.646729in}}%
\pgfpathlineto{\pgfqpoint{3.388011in}{1.643338in}}%
\pgfpathlineto{\pgfqpoint{3.390893in}{1.653303in}}%
\pgfpathlineto{\pgfqpoint{3.391278in}{1.651671in}}%
\pgfpathlineto{\pgfqpoint{3.391662in}{1.657245in}}%
\pgfpathlineto{\pgfqpoint{3.392431in}{1.653121in}}%
\pgfpathlineto{\pgfqpoint{3.392623in}{1.653395in}}%
\pgfpathlineto{\pgfqpoint{3.393776in}{1.657503in}}%
\pgfpathlineto{\pgfqpoint{3.393968in}{1.656510in}}%
\pgfpathlineto{\pgfqpoint{3.394352in}{1.657678in}}%
\pgfpathlineto{\pgfqpoint{3.396082in}{1.643763in}}%
\pgfpathlineto{\pgfqpoint{3.396274in}{1.645482in}}%
\pgfpathlineto{\pgfqpoint{3.396658in}{1.641908in}}%
\pgfpathlineto{\pgfqpoint{3.397043in}{1.643483in}}%
\pgfpathlineto{\pgfqpoint{3.398388in}{1.634186in}}%
\pgfpathlineto{\pgfqpoint{3.398772in}{1.634711in}}%
\pgfpathlineto{\pgfqpoint{3.398964in}{1.635053in}}%
\pgfpathlineto{\pgfqpoint{3.399157in}{1.634923in}}%
\pgfpathlineto{\pgfqpoint{3.399733in}{1.633064in}}%
\pgfpathlineto{\pgfqpoint{3.399925in}{1.635408in}}%
\pgfpathlineto{\pgfqpoint{3.400310in}{1.634465in}}%
\pgfpathlineto{\pgfqpoint{3.401463in}{1.640381in}}%
\pgfpathlineto{\pgfqpoint{3.401655in}{1.638994in}}%
\pgfpathlineto{\pgfqpoint{3.403192in}{1.629203in}}%
\pgfpathlineto{\pgfqpoint{3.404922in}{1.617523in}}%
\pgfpathlineto{\pgfqpoint{3.406843in}{1.632533in}}%
\pgfpathlineto{\pgfqpoint{3.407036in}{1.632037in}}%
\pgfpathlineto{\pgfqpoint{3.407420in}{1.639135in}}%
\pgfpathlineto{\pgfqpoint{3.408189in}{1.636279in}}%
\pgfpathlineto{\pgfqpoint{3.408381in}{1.635468in}}%
\pgfpathlineto{\pgfqpoint{3.408957in}{1.636944in}}%
\pgfpathlineto{\pgfqpoint{3.409342in}{1.638512in}}%
\pgfpathlineto{\pgfqpoint{3.409726in}{1.637874in}}%
\pgfpathlineto{\pgfqpoint{3.410879in}{1.644229in}}%
\pgfpathlineto{\pgfqpoint{3.411071in}{1.642400in}}%
\pgfpathlineto{\pgfqpoint{3.411455in}{1.646398in}}%
\pgfpathlineto{\pgfqpoint{3.411648in}{1.648255in}}%
\pgfpathlineto{\pgfqpoint{3.412032in}{1.646165in}}%
\pgfpathlineto{\pgfqpoint{3.412608in}{1.647436in}}%
\pgfpathlineto{\pgfqpoint{3.412801in}{1.645573in}}%
\pgfpathlineto{\pgfqpoint{3.413377in}{1.650807in}}%
\pgfpathlineto{\pgfqpoint{3.413761in}{1.645743in}}%
\pgfpathlineto{\pgfqpoint{3.413954in}{1.645916in}}%
\pgfpathlineto{\pgfqpoint{3.414338in}{1.652410in}}%
\pgfpathlineto{\pgfqpoint{3.415107in}{1.650025in}}%
\pgfpathlineto{\pgfqpoint{3.415299in}{1.649574in}}%
\pgfpathlineto{\pgfqpoint{3.416452in}{1.633141in}}%
\pgfpathlineto{\pgfqpoint{3.417028in}{1.637484in}}%
\pgfpathlineto{\pgfqpoint{3.417605in}{1.638286in}}%
\pgfpathlineto{\pgfqpoint{3.417797in}{1.637777in}}%
\pgfpathlineto{\pgfqpoint{3.418758in}{1.627688in}}%
\pgfpathlineto{\pgfqpoint{3.419142in}{1.627806in}}%
\pgfpathlineto{\pgfqpoint{3.419334in}{1.626832in}}%
\pgfpathlineto{\pgfqpoint{3.419719in}{1.629939in}}%
\pgfpathlineto{\pgfqpoint{3.420103in}{1.630170in}}%
\pgfpathlineto{\pgfqpoint{3.420487in}{1.626176in}}%
\pgfpathlineto{\pgfqpoint{3.421064in}{1.630625in}}%
\pgfpathlineto{\pgfqpoint{3.421640in}{1.628247in}}%
\pgfpathlineto{\pgfqpoint{3.422409in}{1.629102in}}%
\pgfpathlineto{\pgfqpoint{3.422601in}{1.630269in}}%
\pgfpathlineto{\pgfqpoint{3.422985in}{1.628701in}}%
\pgfpathlineto{\pgfqpoint{3.424138in}{1.623165in}}%
\pgfpathlineto{\pgfqpoint{3.424331in}{1.624109in}}%
\pgfpathlineto{\pgfqpoint{3.424523in}{1.624794in}}%
\pgfpathlineto{\pgfqpoint{3.424907in}{1.623460in}}%
\pgfpathlineto{\pgfqpoint{3.425291in}{1.620443in}}%
\pgfpathlineto{\pgfqpoint{3.425676in}{1.624208in}}%
\pgfpathlineto{\pgfqpoint{3.425868in}{1.626394in}}%
\pgfpathlineto{\pgfqpoint{3.426252in}{1.623450in}}%
\pgfpathlineto{\pgfqpoint{3.427405in}{1.613888in}}%
\pgfpathlineto{\pgfqpoint{3.427597in}{1.615473in}}%
\pgfpathlineto{\pgfqpoint{3.427790in}{1.617175in}}%
\pgfpathlineto{\pgfqpoint{3.428174in}{1.613638in}}%
\pgfpathlineto{\pgfqpoint{3.429327in}{1.610724in}}%
\pgfpathlineto{\pgfqpoint{3.428750in}{1.616220in}}%
\pgfpathlineto{\pgfqpoint{3.429519in}{1.612137in}}%
\pgfpathlineto{\pgfqpoint{3.431057in}{1.622845in}}%
\pgfpathlineto{\pgfqpoint{3.431249in}{1.619374in}}%
\pgfpathlineto{\pgfqpoint{3.431441in}{1.615931in}}%
\pgfpathlineto{\pgfqpoint{3.432017in}{1.622868in}}%
\pgfpathlineto{\pgfqpoint{3.432594in}{1.622304in}}%
\pgfpathlineto{\pgfqpoint{3.432786in}{1.622570in}}%
\pgfpathlineto{\pgfqpoint{3.433555in}{1.631607in}}%
\pgfpathlineto{\pgfqpoint{3.434131in}{1.628969in}}%
\pgfpathlineto{\pgfqpoint{3.435476in}{1.636207in}}%
\pgfpathlineto{\pgfqpoint{3.435669in}{1.634905in}}%
\pgfpathlineto{\pgfqpoint{3.436245in}{1.635435in}}%
\pgfpathlineto{\pgfqpoint{3.437014in}{1.630969in}}%
\pgfpathlineto{\pgfqpoint{3.437782in}{1.640673in}}%
\pgfpathlineto{\pgfqpoint{3.437975in}{1.634253in}}%
\pgfpathlineto{\pgfqpoint{3.438551in}{1.628024in}}%
\pgfpathlineto{\pgfqpoint{3.438935in}{1.634034in}}%
\pgfpathlineto{\pgfqpoint{3.439128in}{1.634151in}}%
\pgfpathlineto{\pgfqpoint{3.439320in}{1.633222in}}%
\pgfpathlineto{\pgfqpoint{3.440665in}{1.620756in}}%
\pgfpathlineto{\pgfqpoint{3.440857in}{1.621104in}}%
\pgfpathlineto{\pgfqpoint{3.441626in}{1.614265in}}%
\pgfpathlineto{\pgfqpoint{3.442587in}{1.610322in}}%
\pgfpathlineto{\pgfqpoint{3.442779in}{1.610474in}}%
\pgfpathlineto{\pgfqpoint{3.443355in}{1.620900in}}%
\pgfpathlineto{\pgfqpoint{3.444124in}{1.618790in}}%
\pgfpathlineto{\pgfqpoint{3.444316in}{1.620043in}}%
\pgfpathlineto{\pgfqpoint{3.444700in}{1.618409in}}%
\pgfpathlineto{\pgfqpoint{3.444893in}{1.618476in}}%
\pgfpathlineto{\pgfqpoint{3.445853in}{1.611473in}}%
\pgfpathlineto{\pgfqpoint{3.446238in}{1.613834in}}%
\pgfpathlineto{\pgfqpoint{3.446430in}{1.613621in}}%
\pgfpathlineto{\pgfqpoint{3.446622in}{1.613929in}}%
\pgfpathlineto{\pgfqpoint{3.446814in}{1.616995in}}%
\pgfpathlineto{\pgfqpoint{3.447583in}{1.612593in}}%
\pgfpathlineto{\pgfqpoint{3.448159in}{1.610887in}}%
\pgfpathlineto{\pgfqpoint{3.449120in}{1.616785in}}%
\pgfpathlineto{\pgfqpoint{3.449312in}{1.616270in}}%
\pgfpathlineto{\pgfqpoint{3.449697in}{1.618668in}}%
\pgfpathlineto{\pgfqpoint{3.450273in}{1.615900in}}%
\pgfpathlineto{\pgfqpoint{3.450465in}{1.614475in}}%
\pgfpathlineto{\pgfqpoint{3.451042in}{1.617896in}}%
\pgfpathlineto{\pgfqpoint{3.452771in}{1.625595in}}%
\pgfpathlineto{\pgfqpoint{3.453732in}{1.620276in}}%
\pgfpathlineto{\pgfqpoint{3.454117in}{1.621598in}}%
\pgfpathlineto{\pgfqpoint{3.454309in}{1.622514in}}%
\pgfpathlineto{\pgfqpoint{3.454501in}{1.620441in}}%
\pgfpathlineto{\pgfqpoint{3.454885in}{1.621293in}}%
\pgfpathlineto{\pgfqpoint{3.455654in}{1.619094in}}%
\pgfpathlineto{\pgfqpoint{3.455846in}{1.620688in}}%
\pgfpathlineto{\pgfqpoint{3.457191in}{1.628688in}}%
\pgfpathlineto{\pgfqpoint{3.457960in}{1.622231in}}%
\pgfpathlineto{\pgfqpoint{3.458344in}{1.624537in}}%
\pgfpathlineto{\pgfqpoint{3.459113in}{1.628073in}}%
\pgfpathlineto{\pgfqpoint{3.459690in}{1.627250in}}%
\pgfpathlineto{\pgfqpoint{3.459882in}{1.627138in}}%
\pgfpathlineto{\pgfqpoint{3.460074in}{1.622816in}}%
\pgfpathlineto{\pgfqpoint{3.460843in}{1.625974in}}%
\pgfpathlineto{\pgfqpoint{3.461803in}{1.633788in}}%
\pgfpathlineto{\pgfqpoint{3.462188in}{1.629651in}}%
\pgfpathlineto{\pgfqpoint{3.462572in}{1.626290in}}%
\pgfpathlineto{\pgfqpoint{3.463533in}{1.627635in}}%
\pgfpathlineto{\pgfqpoint{3.463725in}{1.627946in}}%
\pgfpathlineto{\pgfqpoint{3.463917in}{1.627533in}}%
\pgfpathlineto{\pgfqpoint{3.464494in}{1.629320in}}%
\pgfpathlineto{\pgfqpoint{3.465455in}{1.620885in}}%
\pgfpathlineto{\pgfqpoint{3.466223in}{1.620032in}}%
\pgfpathlineto{\pgfqpoint{3.466800in}{1.623761in}}%
\pgfpathlineto{\pgfqpoint{3.468721in}{1.610926in}}%
\pgfpathlineto{\pgfqpoint{3.469298in}{1.617231in}}%
\pgfpathlineto{\pgfqpoint{3.469874in}{1.615389in}}%
\pgfpathlineto{\pgfqpoint{3.470067in}{1.611744in}}%
\pgfpathlineto{\pgfqpoint{3.471027in}{1.612912in}}%
\pgfpathlineto{\pgfqpoint{3.471988in}{1.605392in}}%
\pgfpathlineto{\pgfqpoint{3.473526in}{1.606502in}}%
\pgfpathlineto{\pgfqpoint{3.473718in}{1.607923in}}%
\pgfpathlineto{\pgfqpoint{3.474294in}{1.606128in}}%
\pgfpathlineto{\pgfqpoint{3.475832in}{1.594458in}}%
\pgfpathlineto{\pgfqpoint{3.476024in}{1.595464in}}%
\pgfpathlineto{\pgfqpoint{3.476408in}{1.591784in}}%
\pgfpathlineto{\pgfqpoint{3.478522in}{1.576559in}}%
\pgfpathlineto{\pgfqpoint{3.478714in}{1.579602in}}%
\pgfpathlineto{\pgfqpoint{3.479483in}{1.574628in}}%
\pgfpathlineto{\pgfqpoint{3.480828in}{1.581485in}}%
\pgfpathlineto{\pgfqpoint{3.481212in}{1.581199in}}%
\pgfpathlineto{\pgfqpoint{3.482750in}{1.575089in}}%
\pgfpathlineto{\pgfqpoint{3.483711in}{1.580624in}}%
\pgfpathlineto{\pgfqpoint{3.483903in}{1.579214in}}%
\pgfpathlineto{\pgfqpoint{3.484479in}{1.576773in}}%
\pgfpathlineto{\pgfqpoint{3.484671in}{1.579478in}}%
\pgfpathlineto{\pgfqpoint{3.484864in}{1.578366in}}%
\pgfpathlineto{\pgfqpoint{3.485248in}{1.580786in}}%
\pgfpathlineto{\pgfqpoint{3.485440in}{1.577525in}}%
\pgfpathlineto{\pgfqpoint{3.485824in}{1.578816in}}%
\pgfpathlineto{\pgfqpoint{3.486017in}{1.578420in}}%
\pgfpathlineto{\pgfqpoint{3.486209in}{1.580251in}}%
\pgfpathlineto{\pgfqpoint{3.486977in}{1.582963in}}%
\pgfpathlineto{\pgfqpoint{3.486785in}{1.579157in}}%
\pgfpathlineto{\pgfqpoint{3.487554in}{1.582016in}}%
\pgfpathlineto{\pgfqpoint{3.487938in}{1.583404in}}%
\pgfpathlineto{\pgfqpoint{3.489283in}{1.577893in}}%
\pgfpathlineto{\pgfqpoint{3.491013in}{1.588542in}}%
\pgfpathlineto{\pgfqpoint{3.492550in}{1.580929in}}%
\pgfpathlineto{\pgfqpoint{3.494664in}{1.568854in}}%
\pgfpathlineto{\pgfqpoint{3.494856in}{1.570026in}}%
\pgfpathlineto{\pgfqpoint{3.495433in}{1.568629in}}%
\pgfpathlineto{\pgfqpoint{3.495817in}{1.569298in}}%
\pgfpathlineto{\pgfqpoint{3.496586in}{1.574856in}}%
\pgfpathlineto{\pgfqpoint{3.497354in}{1.571797in}}%
\pgfpathlineto{\pgfqpoint{3.497547in}{1.570950in}}%
\pgfpathlineto{\pgfqpoint{3.497739in}{1.573138in}}%
\pgfpathlineto{\pgfqpoint{3.497931in}{1.572154in}}%
\pgfpathlineto{\pgfqpoint{3.498123in}{1.574831in}}%
\pgfpathlineto{\pgfqpoint{3.498700in}{1.567867in}}%
\pgfpathlineto{\pgfqpoint{3.498892in}{1.565125in}}%
\pgfpathlineto{\pgfqpoint{3.499468in}{1.569671in}}%
\pgfpathlineto{\pgfqpoint{3.499853in}{1.570724in}}%
\pgfpathlineto{\pgfqpoint{3.500237in}{1.568365in}}%
\pgfpathlineto{\pgfqpoint{3.500621in}{1.566111in}}%
\pgfpathlineto{\pgfqpoint{3.502735in}{1.550109in}}%
\pgfpathlineto{\pgfqpoint{3.506386in}{1.536096in}}%
\pgfpathlineto{\pgfqpoint{3.506579in}{1.537359in}}%
\pgfpathlineto{\pgfqpoint{3.508692in}{1.543095in}}%
\pgfpathlineto{\pgfqpoint{3.509077in}{1.543789in}}%
\pgfpathlineto{\pgfqpoint{3.510422in}{1.540644in}}%
\pgfpathlineto{\pgfqpoint{3.511767in}{1.546399in}}%
\pgfpathlineto{\pgfqpoint{3.511959in}{1.545594in}}%
\pgfpathlineto{\pgfqpoint{3.514650in}{1.566565in}}%
\pgfpathlineto{\pgfqpoint{3.515034in}{1.563180in}}%
\pgfpathlineto{\pgfqpoint{3.515610in}{1.556061in}}%
\pgfpathlineto{\pgfqpoint{3.516187in}{1.558718in}}%
\pgfpathlineto{\pgfqpoint{3.516956in}{1.563379in}}%
\pgfpathlineto{\pgfqpoint{3.517532in}{1.561922in}}%
\pgfpathlineto{\pgfqpoint{3.518877in}{1.557941in}}%
\pgfpathlineto{\pgfqpoint{3.519262in}{1.557187in}}%
\pgfpathlineto{\pgfqpoint{3.519454in}{1.554515in}}%
\pgfpathlineto{\pgfqpoint{3.519646in}{1.558781in}}%
\pgfpathlineto{\pgfqpoint{3.520415in}{1.556166in}}%
\pgfpathlineto{\pgfqpoint{3.520607in}{1.555344in}}%
\pgfpathlineto{\pgfqpoint{3.520799in}{1.556384in}}%
\pgfpathlineto{\pgfqpoint{3.521952in}{1.562997in}}%
\pgfpathlineto{\pgfqpoint{3.523489in}{1.556338in}}%
\pgfpathlineto{\pgfqpoint{3.524834in}{1.560135in}}%
\pgfpathlineto{\pgfqpoint{3.524066in}{1.555410in}}%
\pgfpathlineto{\pgfqpoint{3.525219in}{1.557972in}}%
\pgfpathlineto{\pgfqpoint{3.525603in}{1.556786in}}%
\pgfpathlineto{\pgfqpoint{3.525795in}{1.560372in}}%
\pgfpathlineto{\pgfqpoint{3.526372in}{1.561918in}}%
\pgfpathlineto{\pgfqpoint{3.526564in}{1.560526in}}%
\pgfpathlineto{\pgfqpoint{3.527525in}{1.556178in}}%
\pgfpathlineto{\pgfqpoint{3.527909in}{1.556429in}}%
\pgfpathlineto{\pgfqpoint{3.528486in}{1.555429in}}%
\pgfpathlineto{\pgfqpoint{3.529639in}{1.565050in}}%
\pgfpathlineto{\pgfqpoint{3.530407in}{1.566219in}}%
\pgfpathlineto{\pgfqpoint{3.531368in}{1.560721in}}%
\pgfpathlineto{\pgfqpoint{3.531945in}{1.560572in}}%
\pgfpathlineto{\pgfqpoint{3.532906in}{1.565815in}}%
\pgfpathlineto{\pgfqpoint{3.534635in}{1.556795in}}%
\pgfpathlineto{\pgfqpoint{3.535019in}{1.561561in}}%
\pgfpathlineto{\pgfqpoint{3.535596in}{1.564275in}}%
\pgfpathlineto{\pgfqpoint{3.535980in}{1.561908in}}%
\pgfpathlineto{\pgfqpoint{3.537325in}{1.556676in}}%
\pgfpathlineto{\pgfqpoint{3.537518in}{1.555510in}}%
\pgfpathlineto{\pgfqpoint{3.537902in}{1.558932in}}%
\pgfpathlineto{\pgfqpoint{3.538094in}{1.558559in}}%
\pgfpathlineto{\pgfqpoint{3.538671in}{1.562269in}}%
\pgfpathlineto{\pgfqpoint{3.539439in}{1.559182in}}%
\pgfpathlineto{\pgfqpoint{3.540016in}{1.557094in}}%
\pgfpathlineto{\pgfqpoint{3.540592in}{1.558567in}}%
\pgfpathlineto{\pgfqpoint{3.542706in}{1.566856in}}%
\pgfpathlineto{\pgfqpoint{3.542898in}{1.571447in}}%
\pgfpathlineto{\pgfqpoint{3.543859in}{1.568579in}}%
\pgfpathlineto{\pgfqpoint{3.544051in}{1.570323in}}%
\pgfpathlineto{\pgfqpoint{3.544243in}{1.567250in}}%
\pgfpathlineto{\pgfqpoint{3.544628in}{1.568202in}}%
\pgfpathlineto{\pgfqpoint{3.545396in}{1.564052in}}%
\pgfpathlineto{\pgfqpoint{3.546165in}{1.564729in}}%
\pgfpathlineto{\pgfqpoint{3.546357in}{1.567477in}}%
\pgfpathlineto{\pgfqpoint{3.546934in}{1.563266in}}%
\pgfpathlineto{\pgfqpoint{3.547702in}{1.560142in}}%
\pgfpathlineto{\pgfqpoint{3.547895in}{1.561341in}}%
\pgfpathlineto{\pgfqpoint{3.549624in}{1.566472in}}%
\pgfpathlineto{\pgfqpoint{3.549816in}{1.566416in}}%
\pgfpathlineto{\pgfqpoint{3.550201in}{1.561202in}}%
\pgfpathlineto{\pgfqpoint{3.551354in}{1.561468in}}%
\pgfpathlineto{\pgfqpoint{3.551738in}{1.560449in}}%
\pgfpathlineto{\pgfqpoint{3.552507in}{1.563975in}}%
\pgfpathlineto{\pgfqpoint{3.553083in}{1.562712in}}%
\pgfpathlineto{\pgfqpoint{3.552891in}{1.565977in}}%
\pgfpathlineto{\pgfqpoint{3.553275in}{1.564403in}}%
\pgfpathlineto{\pgfqpoint{3.553660in}{1.566101in}}%
\pgfpathlineto{\pgfqpoint{3.553852in}{1.561889in}}%
\pgfpathlineto{\pgfqpoint{3.554044in}{1.565222in}}%
\pgfpathlineto{\pgfqpoint{3.554428in}{1.559486in}}%
\pgfpathlineto{\pgfqpoint{3.555005in}{1.567487in}}%
\pgfpathlineto{\pgfqpoint{3.558080in}{1.583982in}}%
\pgfpathlineto{\pgfqpoint{3.558272in}{1.583052in}}%
\pgfpathlineto{\pgfqpoint{3.560001in}{1.568836in}}%
\pgfpathlineto{\pgfqpoint{3.561154in}{1.565872in}}%
\pgfpathlineto{\pgfqpoint{3.563652in}{1.555021in}}%
\pgfpathlineto{\pgfqpoint{3.564613in}{1.555344in}}%
\pgfpathlineto{\pgfqpoint{3.565574in}{1.558601in}}%
\pgfpathlineto{\pgfqpoint{3.565766in}{1.556314in}}%
\pgfpathlineto{\pgfqpoint{3.565958in}{1.552422in}}%
\pgfpathlineto{\pgfqpoint{3.566919in}{1.554086in}}%
\pgfpathlineto{\pgfqpoint{3.567111in}{1.557390in}}%
\pgfpathlineto{\pgfqpoint{3.567880in}{1.551818in}}%
\pgfpathlineto{\pgfqpoint{3.568649in}{1.546709in}}%
\pgfpathlineto{\pgfqpoint{3.569225in}{1.547263in}}%
\pgfpathlineto{\pgfqpoint{3.569610in}{1.549028in}}%
\pgfpathlineto{\pgfqpoint{3.569994in}{1.545584in}}%
\pgfpathlineto{\pgfqpoint{3.570955in}{1.540063in}}%
\pgfpathlineto{\pgfqpoint{3.571531in}{1.540813in}}%
\pgfpathlineto{\pgfqpoint{3.572108in}{1.542423in}}%
\pgfpathlineto{\pgfqpoint{3.572492in}{1.542060in}}%
\pgfpathlineto{\pgfqpoint{3.573069in}{1.539783in}}%
\pgfpathlineto{\pgfqpoint{3.573261in}{1.542953in}}%
\pgfpathlineto{\pgfqpoint{3.575567in}{1.566139in}}%
\pgfpathlineto{\pgfqpoint{3.576336in}{1.561143in}}%
\pgfpathlineto{\pgfqpoint{3.577489in}{1.558289in}}%
\pgfpathlineto{\pgfqpoint{3.577681in}{1.560359in}}%
\pgfpathlineto{\pgfqpoint{3.578257in}{1.557769in}}%
\pgfpathlineto{\pgfqpoint{3.578449in}{1.558005in}}%
\pgfpathlineto{\pgfqpoint{3.580371in}{1.541790in}}%
\pgfpathlineto{\pgfqpoint{3.580563in}{1.542547in}}%
\pgfpathlineto{\pgfqpoint{3.581524in}{1.542009in}}%
\pgfpathlineto{\pgfqpoint{3.581716in}{1.545286in}}%
\pgfpathlineto{\pgfqpoint{3.582485in}{1.545946in}}%
\pgfpathlineto{\pgfqpoint{3.582869in}{1.542099in}}%
\pgfpathlineto{\pgfqpoint{3.583061in}{1.543903in}}%
\pgfpathlineto{\pgfqpoint{3.583638in}{1.539249in}}%
\pgfpathlineto{\pgfqpoint{3.584022in}{1.541863in}}%
\pgfpathlineto{\pgfqpoint{3.584983in}{1.537027in}}%
\pgfpathlineto{\pgfqpoint{3.586136in}{1.544061in}}%
\pgfpathlineto{\pgfqpoint{3.586328in}{1.542528in}}%
\pgfpathlineto{\pgfqpoint{3.586520in}{1.545691in}}%
\pgfpathlineto{\pgfqpoint{3.587097in}{1.538081in}}%
\pgfpathlineto{\pgfqpoint{3.587481in}{1.532583in}}%
\pgfpathlineto{\pgfqpoint{3.588442in}{1.535358in}}%
\pgfpathlineto{\pgfqpoint{3.590172in}{1.540599in}}%
\pgfpathlineto{\pgfqpoint{3.590364in}{1.539963in}}%
\pgfpathlineto{\pgfqpoint{3.591901in}{1.556965in}}%
\pgfpathlineto{\pgfqpoint{3.592093in}{1.557012in}}%
\pgfpathlineto{\pgfqpoint{3.592285in}{1.552896in}}%
\pgfpathlineto{\pgfqpoint{3.593054in}{1.558670in}}%
\pgfpathlineto{\pgfqpoint{3.593246in}{1.560156in}}%
\pgfpathlineto{\pgfqpoint{3.593823in}{1.557095in}}%
\pgfpathlineto{\pgfqpoint{3.594591in}{1.549813in}}%
\pgfpathlineto{\pgfqpoint{3.595552in}{1.551951in}}%
\pgfpathlineto{\pgfqpoint{3.596321in}{1.557509in}}%
\pgfpathlineto{\pgfqpoint{3.596897in}{1.553980in}}%
\pgfpathlineto{\pgfqpoint{3.597090in}{1.554886in}}%
\pgfpathlineto{\pgfqpoint{3.597666in}{1.552702in}}%
\pgfpathlineto{\pgfqpoint{3.598435in}{1.542260in}}%
\pgfpathlineto{\pgfqpoint{3.599011in}{1.544618in}}%
\pgfpathlineto{\pgfqpoint{3.599203in}{1.548938in}}%
\pgfpathlineto{\pgfqpoint{3.600164in}{1.545581in}}%
\pgfpathlineto{\pgfqpoint{3.600357in}{1.546569in}}%
\pgfpathlineto{\pgfqpoint{3.600549in}{1.542948in}}%
\pgfpathlineto{\pgfqpoint{3.600933in}{1.544330in}}%
\pgfpathlineto{\pgfqpoint{3.601125in}{1.544300in}}%
\pgfpathlineto{\pgfqpoint{3.601317in}{1.548069in}}%
\pgfpathlineto{\pgfqpoint{3.601894in}{1.541946in}}%
\pgfpathlineto{\pgfqpoint{3.602086in}{1.544089in}}%
\pgfpathlineto{\pgfqpoint{3.602278in}{1.543177in}}%
\pgfpathlineto{\pgfqpoint{3.603239in}{1.533868in}}%
\pgfpathlineto{\pgfqpoint{3.603623in}{1.534172in}}%
\pgfpathlineto{\pgfqpoint{3.604969in}{1.530970in}}%
\pgfpathlineto{\pgfqpoint{3.605161in}{1.531038in}}%
\pgfpathlineto{\pgfqpoint{3.605353in}{1.530820in}}%
\pgfpathlineto{\pgfqpoint{3.605929in}{1.525817in}}%
\pgfpathlineto{\pgfqpoint{3.606698in}{1.526995in}}%
\pgfpathlineto{\pgfqpoint{3.607082in}{1.528862in}}%
\pgfpathlineto{\pgfqpoint{3.607467in}{1.524668in}}%
\pgfpathlineto{\pgfqpoint{3.608620in}{1.521197in}}%
\pgfpathlineto{\pgfqpoint{3.608812in}{1.522441in}}%
\pgfpathlineto{\pgfqpoint{3.609196in}{1.518298in}}%
\pgfpathlineto{\pgfqpoint{3.610349in}{1.513852in}}%
\pgfpathlineto{\pgfqpoint{3.610541in}{1.517519in}}%
\pgfpathlineto{\pgfqpoint{3.611310in}{1.515229in}}%
\pgfpathlineto{\pgfqpoint{3.611694in}{1.512766in}}%
\pgfpathlineto{\pgfqpoint{3.612271in}{1.515644in}}%
\pgfpathlineto{\pgfqpoint{3.612463in}{1.514581in}}%
\pgfpathlineto{\pgfqpoint{3.612655in}{1.515330in}}%
\pgfpathlineto{\pgfqpoint{3.613040in}{1.514338in}}%
\pgfpathlineto{\pgfqpoint{3.614385in}{1.505548in}}%
\pgfpathlineto{\pgfqpoint{3.613424in}{1.514434in}}%
\pgfpathlineto{\pgfqpoint{3.614769in}{1.507657in}}%
\pgfpathlineto{\pgfqpoint{3.616883in}{1.523826in}}%
\pgfpathlineto{\pgfqpoint{3.617459in}{1.516768in}}%
\pgfpathlineto{\pgfqpoint{3.618036in}{1.521289in}}%
\pgfpathlineto{\pgfqpoint{3.618612in}{1.516692in}}%
\pgfpathlineto{\pgfqpoint{3.619381in}{1.519305in}}%
\pgfpathlineto{\pgfqpoint{3.619573in}{1.521596in}}%
\pgfpathlineto{\pgfqpoint{3.620342in}{1.517989in}}%
\pgfpathlineto{\pgfqpoint{3.622456in}{1.506682in}}%
\pgfpathlineto{\pgfqpoint{3.622840in}{1.510949in}}%
\pgfpathlineto{\pgfqpoint{3.623224in}{1.514220in}}%
\pgfpathlineto{\pgfqpoint{3.623801in}{1.510815in}}%
\pgfpathlineto{\pgfqpoint{3.623993in}{1.509988in}}%
\pgfpathlineto{\pgfqpoint{3.624185in}{1.513135in}}%
\pgfpathlineto{\pgfqpoint{3.624570in}{1.511647in}}%
\pgfpathlineto{\pgfqpoint{3.625146in}{1.515714in}}%
\pgfpathlineto{\pgfqpoint{3.625531in}{1.511569in}}%
\pgfpathlineto{\pgfqpoint{3.625915in}{1.504672in}}%
\pgfpathlineto{\pgfqpoint{3.626876in}{1.505929in}}%
\pgfpathlineto{\pgfqpoint{3.627452in}{1.506683in}}%
\pgfpathlineto{\pgfqpoint{3.628413in}{1.499361in}}%
\pgfpathlineto{\pgfqpoint{3.628990in}{1.500013in}}%
\pgfpathlineto{\pgfqpoint{3.629374in}{1.501205in}}%
\pgfpathlineto{\pgfqpoint{3.629758in}{1.502154in}}%
\pgfpathlineto{\pgfqpoint{3.629950in}{1.500527in}}%
\pgfpathlineto{\pgfqpoint{3.630911in}{1.494423in}}%
\pgfpathlineto{\pgfqpoint{3.631488in}{1.496008in}}%
\pgfpathlineto{\pgfqpoint{3.632064in}{1.499287in}}%
\pgfpathlineto{\pgfqpoint{3.632449in}{1.503569in}}%
\pgfpathlineto{\pgfqpoint{3.633025in}{1.501106in}}%
\pgfpathlineto{\pgfqpoint{3.634562in}{1.494256in}}%
\pgfpathlineto{\pgfqpoint{3.634755in}{1.493539in}}%
\pgfpathlineto{\pgfqpoint{3.634947in}{1.496071in}}%
\pgfpathlineto{\pgfqpoint{3.635139in}{1.498103in}}%
\pgfpathlineto{\pgfqpoint{3.635331in}{1.494730in}}%
\pgfpathlineto{\pgfqpoint{3.635908in}{1.497306in}}%
\pgfpathlineto{\pgfqpoint{3.636100in}{1.492803in}}%
\pgfpathlineto{\pgfqpoint{3.636868in}{1.498389in}}%
\pgfpathlineto{\pgfqpoint{3.637061in}{1.499027in}}%
\pgfpathlineto{\pgfqpoint{3.637253in}{1.497734in}}%
\pgfpathlineto{\pgfqpoint{3.637445in}{1.498391in}}%
\pgfpathlineto{\pgfqpoint{3.638790in}{1.486969in}}%
\pgfpathlineto{\pgfqpoint{3.639367in}{1.489375in}}%
\pgfpathlineto{\pgfqpoint{3.643210in}{1.460813in}}%
\pgfpathlineto{\pgfqpoint{3.643594in}{1.466248in}}%
\pgfpathlineto{\pgfqpoint{3.644171in}{1.470513in}}%
\pgfpathlineto{\pgfqpoint{3.644747in}{1.468809in}}%
\pgfpathlineto{\pgfqpoint{3.645324in}{1.464840in}}%
\pgfpathlineto{\pgfqpoint{3.645516in}{1.468834in}}%
\pgfpathlineto{\pgfqpoint{3.645708in}{1.471083in}}%
\pgfpathlineto{\pgfqpoint{3.646285in}{1.466322in}}%
\pgfpathlineto{\pgfqpoint{3.646861in}{1.465499in}}%
\pgfpathlineto{\pgfqpoint{3.646669in}{1.467233in}}%
\pgfpathlineto{\pgfqpoint{3.647245in}{1.466856in}}%
\pgfpathlineto{\pgfqpoint{3.647438in}{1.468889in}}%
\pgfpathlineto{\pgfqpoint{3.647630in}{1.464595in}}%
\pgfpathlineto{\pgfqpoint{3.647822in}{1.467269in}}%
\pgfpathlineto{\pgfqpoint{3.648014in}{1.463123in}}%
\pgfpathlineto{\pgfqpoint{3.648783in}{1.465416in}}%
\pgfpathlineto{\pgfqpoint{3.650128in}{1.472643in}}%
\pgfpathlineto{\pgfqpoint{3.650512in}{1.472986in}}%
\pgfpathlineto{\pgfqpoint{3.651858in}{1.467104in}}%
\pgfpathlineto{\pgfqpoint{3.652242in}{1.470283in}}%
\pgfpathlineto{\pgfqpoint{3.652626in}{1.466713in}}%
\pgfpathlineto{\pgfqpoint{3.653011in}{1.461648in}}%
\pgfpathlineto{\pgfqpoint{3.653779in}{1.464281in}}%
\pgfpathlineto{\pgfqpoint{3.654164in}{1.464907in}}%
\pgfpathlineto{\pgfqpoint{3.654932in}{1.463922in}}%
\pgfpathlineto{\pgfqpoint{3.655509in}{1.469413in}}%
\pgfpathlineto{\pgfqpoint{3.656470in}{1.465633in}}%
\pgfpathlineto{\pgfqpoint{3.656854in}{1.466916in}}%
\pgfpathlineto{\pgfqpoint{3.657238in}{1.468187in}}%
\pgfpathlineto{\pgfqpoint{3.657430in}{1.465121in}}%
\pgfpathlineto{\pgfqpoint{3.657815in}{1.467841in}}%
\pgfpathlineto{\pgfqpoint{3.658007in}{1.466153in}}%
\pgfpathlineto{\pgfqpoint{3.658199in}{1.470742in}}%
\pgfpathlineto{\pgfqpoint{3.658776in}{1.469403in}}%
\pgfpathlineto{\pgfqpoint{3.659352in}{1.473615in}}%
\pgfpathlineto{\pgfqpoint{3.659544in}{1.473310in}}%
\pgfpathlineto{\pgfqpoint{3.660697in}{1.480042in}}%
\pgfpathlineto{\pgfqpoint{3.660889in}{1.479823in}}%
\pgfpathlineto{\pgfqpoint{3.661466in}{1.482296in}}%
\pgfpathlineto{\pgfqpoint{3.662235in}{1.476080in}}%
\pgfpathlineto{\pgfqpoint{3.663003in}{1.477649in}}%
\pgfpathlineto{\pgfqpoint{3.662811in}{1.475718in}}%
\pgfpathlineto{\pgfqpoint{3.663388in}{1.476075in}}%
\pgfpathlineto{\pgfqpoint{3.663964in}{1.474283in}}%
\pgfpathlineto{\pgfqpoint{3.664925in}{1.471669in}}%
\pgfpathlineto{\pgfqpoint{3.665117in}{1.472534in}}%
\pgfpathlineto{\pgfqpoint{3.665309in}{1.475543in}}%
\pgfpathlineto{\pgfqpoint{3.666270in}{1.474682in}}%
\pgfpathlineto{\pgfqpoint{3.666462in}{1.473609in}}%
\pgfpathlineto{\pgfqpoint{3.666654in}{1.478231in}}%
\pgfpathlineto{\pgfqpoint{3.667039in}{1.481945in}}%
\pgfpathlineto{\pgfqpoint{3.667807in}{1.479562in}}%
\pgfpathlineto{\pgfqpoint{3.668384in}{1.477087in}}%
\pgfpathlineto{\pgfqpoint{3.668576in}{1.480279in}}%
\pgfpathlineto{\pgfqpoint{3.668960in}{1.479085in}}%
\pgfpathlineto{\pgfqpoint{3.669153in}{1.479339in}}%
\pgfpathlineto{\pgfqpoint{3.669345in}{1.481757in}}%
\pgfpathlineto{\pgfqpoint{3.669921in}{1.478378in}}%
\pgfpathlineto{\pgfqpoint{3.670113in}{1.479618in}}%
\pgfpathlineto{\pgfqpoint{3.670882in}{1.480610in}}%
\pgfpathlineto{\pgfqpoint{3.671843in}{1.474872in}}%
\pgfpathlineto{\pgfqpoint{3.672612in}{1.478616in}}%
\pgfpathlineto{\pgfqpoint{3.673188in}{1.477497in}}%
\pgfpathlineto{\pgfqpoint{3.673573in}{1.472664in}}%
\pgfpathlineto{\pgfqpoint{3.674149in}{1.478692in}}%
\pgfpathlineto{\pgfqpoint{3.674533in}{1.483472in}}%
\pgfpathlineto{\pgfqpoint{3.675302in}{1.481037in}}%
\pgfpathlineto{\pgfqpoint{3.676455in}{1.472269in}}%
\pgfpathlineto{\pgfqpoint{3.677032in}{1.473664in}}%
\pgfpathlineto{\pgfqpoint{3.677224in}{1.472927in}}%
\pgfpathlineto{\pgfqpoint{3.677608in}{1.473095in}}%
\pgfpathlineto{\pgfqpoint{3.678953in}{1.480778in}}%
\pgfpathlineto{\pgfqpoint{3.679914in}{1.481807in}}%
\pgfpathlineto{\pgfqpoint{3.680298in}{1.475730in}}%
\pgfpathlineto{\pgfqpoint{3.680491in}{1.475292in}}%
\pgfpathlineto{\pgfqpoint{3.680683in}{1.477327in}}%
\pgfpathlineto{\pgfqpoint{3.680875in}{1.476312in}}%
\pgfpathlineto{\pgfqpoint{3.681259in}{1.477396in}}%
\pgfpathlineto{\pgfqpoint{3.682797in}{1.466082in}}%
\pgfpathlineto{\pgfqpoint{3.682989in}{1.469661in}}%
\pgfpathlineto{\pgfqpoint{3.683565in}{1.465826in}}%
\pgfpathlineto{\pgfqpoint{3.684526in}{1.457596in}}%
\pgfpathlineto{\pgfqpoint{3.684910in}{1.459172in}}%
\pgfpathlineto{\pgfqpoint{3.685679in}{1.465243in}}%
\pgfpathlineto{\pgfqpoint{3.686063in}{1.464532in}}%
\pgfpathlineto{\pgfqpoint{3.686832in}{1.457632in}}%
\pgfpathlineto{\pgfqpoint{3.687216in}{1.460783in}}%
\pgfpathlineto{\pgfqpoint{3.687601in}{1.461981in}}%
\pgfpathlineto{\pgfqpoint{3.688177in}{1.454867in}}%
\pgfpathlineto{\pgfqpoint{3.688754in}{1.458214in}}%
\pgfpathlineto{\pgfqpoint{3.688946in}{1.458174in}}%
\pgfpathlineto{\pgfqpoint{3.690099in}{1.453231in}}%
\pgfpathlineto{\pgfqpoint{3.690291in}{1.454370in}}%
\pgfpathlineto{\pgfqpoint{3.692021in}{1.462417in}}%
\pgfpathlineto{\pgfqpoint{3.692981in}{1.458969in}}%
\pgfpathlineto{\pgfqpoint{3.692789in}{1.462735in}}%
\pgfpathlineto{\pgfqpoint{3.693558in}{1.460722in}}%
\pgfpathlineto{\pgfqpoint{3.695287in}{1.469187in}}%
\pgfpathlineto{\pgfqpoint{3.696248in}{1.467746in}}%
\pgfpathlineto{\pgfqpoint{3.695672in}{1.470010in}}%
\pgfpathlineto{\pgfqpoint{3.696440in}{1.468436in}}%
\pgfpathlineto{\pgfqpoint{3.696633in}{1.470243in}}%
\pgfpathlineto{\pgfqpoint{3.696825in}{1.466408in}}%
\pgfpathlineto{\pgfqpoint{3.697401in}{1.468625in}}%
\pgfpathlineto{\pgfqpoint{3.697593in}{1.465824in}}%
\pgfpathlineto{\pgfqpoint{3.698362in}{1.468435in}}%
\pgfpathlineto{\pgfqpoint{3.698747in}{1.471437in}}%
\pgfpathlineto{\pgfqpoint{3.698939in}{1.471198in}}%
\pgfpathlineto{\pgfqpoint{3.700284in}{1.480590in}}%
\pgfpathlineto{\pgfqpoint{3.700476in}{1.480118in}}%
\pgfpathlineto{\pgfqpoint{3.701629in}{1.485631in}}%
\pgfpathlineto{\pgfqpoint{3.701821in}{1.484088in}}%
\pgfpathlineto{\pgfqpoint{3.702013in}{1.483879in}}%
\pgfpathlineto{\pgfqpoint{3.702398in}{1.480527in}}%
\pgfpathlineto{\pgfqpoint{3.703359in}{1.480999in}}%
\pgfpathlineto{\pgfqpoint{3.704319in}{1.484328in}}%
\pgfpathlineto{\pgfqpoint{3.703743in}{1.480900in}}%
\pgfpathlineto{\pgfqpoint{3.704512in}{1.483677in}}%
\pgfpathlineto{\pgfqpoint{3.704896in}{1.478107in}}%
\pgfpathlineto{\pgfqpoint{3.705472in}{1.482650in}}%
\pgfpathlineto{\pgfqpoint{3.705665in}{1.483641in}}%
\pgfpathlineto{\pgfqpoint{3.706049in}{1.480194in}}%
\pgfpathlineto{\pgfqpoint{3.706241in}{1.482266in}}%
\pgfpathlineto{\pgfqpoint{3.706433in}{1.480649in}}%
\pgfpathlineto{\pgfqpoint{3.706818in}{1.485498in}}%
\pgfpathlineto{\pgfqpoint{3.707010in}{1.484396in}}%
\pgfpathlineto{\pgfqpoint{3.708163in}{1.495131in}}%
\pgfpathlineto{\pgfqpoint{3.708931in}{1.491093in}}%
\pgfpathlineto{\pgfqpoint{3.710277in}{1.489296in}}%
\pgfpathlineto{\pgfqpoint{3.711814in}{1.501169in}}%
\pgfpathlineto{\pgfqpoint{3.713351in}{1.496411in}}%
\pgfpathlineto{\pgfqpoint{3.713543in}{1.497730in}}%
\pgfpathlineto{\pgfqpoint{3.714889in}{1.509148in}}%
\pgfpathlineto{\pgfqpoint{3.715081in}{1.507360in}}%
\pgfpathlineto{\pgfqpoint{3.715273in}{1.505153in}}%
\pgfpathlineto{\pgfqpoint{3.716042in}{1.508803in}}%
\pgfpathlineto{\pgfqpoint{3.716234in}{1.506676in}}%
\pgfpathlineto{\pgfqpoint{3.718155in}{1.498344in}}%
\pgfpathlineto{\pgfqpoint{3.718540in}{1.498075in}}%
\pgfpathlineto{\pgfqpoint{3.720077in}{1.507290in}}%
\pgfpathlineto{\pgfqpoint{3.720654in}{1.513409in}}%
\pgfpathlineto{\pgfqpoint{3.721422in}{1.512525in}}%
\pgfpathlineto{\pgfqpoint{3.722960in}{1.504875in}}%
\pgfpathlineto{\pgfqpoint{3.724113in}{1.508414in}}%
\pgfpathlineto{\pgfqpoint{3.724497in}{1.511312in}}%
\pgfpathlineto{\pgfqpoint{3.725074in}{1.506936in}}%
\pgfpathlineto{\pgfqpoint{3.725842in}{1.497458in}}%
\pgfpathlineto{\pgfqpoint{3.726611in}{1.501349in}}%
\pgfpathlineto{\pgfqpoint{3.726995in}{1.504809in}}%
\pgfpathlineto{\pgfqpoint{3.727572in}{1.503830in}}%
\pgfpathlineto{\pgfqpoint{3.729878in}{1.485172in}}%
\pgfpathlineto{\pgfqpoint{3.730646in}{1.490588in}}%
\pgfpathlineto{\pgfqpoint{3.731415in}{1.487734in}}%
\pgfpathlineto{\pgfqpoint{3.732568in}{1.494857in}}%
\pgfpathlineto{\pgfqpoint{3.731799in}{1.487187in}}%
\pgfpathlineto{\pgfqpoint{3.732952in}{1.490537in}}%
\pgfpathlineto{\pgfqpoint{3.733913in}{1.485884in}}%
\pgfpathlineto{\pgfqpoint{3.734490in}{1.483343in}}%
\pgfpathlineto{\pgfqpoint{3.735066in}{1.485501in}}%
\pgfpathlineto{\pgfqpoint{3.735451in}{1.485923in}}%
\pgfpathlineto{\pgfqpoint{3.735643in}{1.484928in}}%
\pgfpathlineto{\pgfqpoint{3.736411in}{1.481437in}}%
\pgfpathlineto{\pgfqpoint{3.736604in}{1.484018in}}%
\pgfpathlineto{\pgfqpoint{3.736988in}{1.487001in}}%
\pgfpathlineto{\pgfqpoint{3.737564in}{1.483391in}}%
\pgfpathlineto{\pgfqpoint{3.737949in}{1.485448in}}%
\pgfpathlineto{\pgfqpoint{3.738333in}{1.483578in}}%
\pgfpathlineto{\pgfqpoint{3.739102in}{1.485152in}}%
\pgfpathlineto{\pgfqpoint{3.739294in}{1.489404in}}%
\pgfpathlineto{\pgfqpoint{3.740255in}{1.488617in}}%
\pgfpathlineto{\pgfqpoint{3.740639in}{1.485132in}}%
\pgfpathlineto{\pgfqpoint{3.741023in}{1.489115in}}%
\pgfpathlineto{\pgfqpoint{3.741408in}{1.486343in}}%
\pgfpathlineto{\pgfqpoint{3.741792in}{1.492809in}}%
\pgfpathlineto{\pgfqpoint{3.742369in}{1.486209in}}%
\pgfpathlineto{\pgfqpoint{3.742561in}{1.486021in}}%
\pgfpathlineto{\pgfqpoint{3.742753in}{1.487213in}}%
\pgfpathlineto{\pgfqpoint{3.743906in}{1.493698in}}%
\pgfpathlineto{\pgfqpoint{3.744290in}{1.491722in}}%
\pgfpathlineto{\pgfqpoint{3.744867in}{1.493676in}}%
\pgfpathlineto{\pgfqpoint{3.745059in}{1.495094in}}%
\pgfpathlineto{\pgfqpoint{3.745251in}{1.491461in}}%
\pgfpathlineto{\pgfqpoint{3.745828in}{1.492494in}}%
\pgfpathlineto{\pgfqpoint{3.746981in}{1.498094in}}%
\pgfpathlineto{\pgfqpoint{3.747557in}{1.495130in}}%
\pgfpathlineto{\pgfqpoint{3.749671in}{1.483547in}}%
\pgfpathlineto{\pgfqpoint{3.750248in}{1.485933in}}%
\pgfpathlineto{\pgfqpoint{3.750632in}{1.483517in}}%
\pgfpathlineto{\pgfqpoint{3.750824in}{1.486432in}}%
\pgfpathlineto{\pgfqpoint{3.752938in}{1.493381in}}%
\pgfpathlineto{\pgfqpoint{3.753130in}{1.492507in}}%
\pgfpathlineto{\pgfqpoint{3.754091in}{1.485520in}}%
\pgfpathlineto{\pgfqpoint{3.754475in}{1.487544in}}%
\pgfpathlineto{\pgfqpoint{3.754667in}{1.489052in}}%
\pgfpathlineto{\pgfqpoint{3.755052in}{1.483923in}}%
\pgfpathlineto{\pgfqpoint{3.755244in}{1.486618in}}%
\pgfpathlineto{\pgfqpoint{3.756013in}{1.483229in}}%
\pgfpathlineto{\pgfqpoint{3.756397in}{1.485865in}}%
\pgfpathlineto{\pgfqpoint{3.759856in}{1.468342in}}%
\pgfpathlineto{\pgfqpoint{3.761009in}{1.469272in}}%
\pgfpathlineto{\pgfqpoint{3.762354in}{1.476537in}}%
\pgfpathlineto{\pgfqpoint{3.763123in}{1.475593in}}%
\pgfpathlineto{\pgfqpoint{3.764276in}{1.474422in}}%
\pgfpathlineto{\pgfqpoint{3.764468in}{1.474729in}}%
\pgfpathlineto{\pgfqpoint{3.764660in}{1.472509in}}%
\pgfpathlineto{\pgfqpoint{3.765044in}{1.475749in}}%
\pgfpathlineto{\pgfqpoint{3.765429in}{1.475588in}}%
\pgfpathlineto{\pgfqpoint{3.765621in}{1.478363in}}%
\pgfpathlineto{\pgfqpoint{3.766197in}{1.473169in}}%
\pgfpathlineto{\pgfqpoint{3.766582in}{1.473015in}}%
\pgfpathlineto{\pgfqpoint{3.767735in}{1.476845in}}%
\pgfpathlineto{\pgfqpoint{3.767927in}{1.476819in}}%
\pgfpathlineto{\pgfqpoint{3.768503in}{1.479194in}}%
\pgfpathlineto{\pgfqpoint{3.768888in}{1.475039in}}%
\pgfpathlineto{\pgfqpoint{3.769464in}{1.477760in}}%
\pgfpathlineto{\pgfqpoint{3.770041in}{1.476138in}}%
\pgfpathlineto{\pgfqpoint{3.770233in}{1.475756in}}%
\pgfpathlineto{\pgfqpoint{3.770809in}{1.476956in}}%
\pgfpathlineto{\pgfqpoint{3.771002in}{1.477951in}}%
\pgfpathlineto{\pgfqpoint{3.771386in}{1.475472in}}%
\pgfpathlineto{\pgfqpoint{3.773116in}{1.468958in}}%
\pgfpathlineto{\pgfqpoint{3.773500in}{1.466049in}}%
\pgfpathlineto{\pgfqpoint{3.774269in}{1.468259in}}%
\pgfpathlineto{\pgfqpoint{3.775037in}{1.476817in}}%
\pgfpathlineto{\pgfqpoint{3.775614in}{1.470881in}}%
\pgfpathlineto{\pgfqpoint{3.775998in}{1.471410in}}%
\pgfpathlineto{\pgfqpoint{3.777151in}{1.475094in}}%
\pgfpathlineto{\pgfqpoint{3.777343in}{1.474631in}}%
\pgfpathlineto{\pgfqpoint{3.777920in}{1.468619in}}%
\pgfpathlineto{\pgfqpoint{3.778496in}{1.472241in}}%
\pgfpathlineto{\pgfqpoint{3.779073in}{1.473725in}}%
\pgfpathlineto{\pgfqpoint{3.780226in}{1.465804in}}%
\pgfpathlineto{\pgfqpoint{3.780994in}{1.471974in}}%
\pgfpathlineto{\pgfqpoint{3.781379in}{1.469315in}}%
\pgfpathlineto{\pgfqpoint{3.783108in}{1.464139in}}%
\pgfpathlineto{\pgfqpoint{3.783493in}{1.463222in}}%
\pgfpathlineto{\pgfqpoint{3.784261in}{1.468661in}}%
\pgfpathlineto{\pgfqpoint{3.784838in}{1.465205in}}%
\pgfpathlineto{\pgfqpoint{3.785414in}{1.465852in}}%
\pgfpathlineto{\pgfqpoint{3.785799in}{1.472158in}}%
\pgfpathlineto{\pgfqpoint{3.786759in}{1.471036in}}%
\pgfpathlineto{\pgfqpoint{3.786952in}{1.472715in}}%
\pgfpathlineto{\pgfqpoint{3.787528in}{1.469994in}}%
\pgfpathlineto{\pgfqpoint{3.787912in}{1.471615in}}%
\pgfpathlineto{\pgfqpoint{3.788297in}{1.468312in}}%
\pgfpathlineto{\pgfqpoint{3.788873in}{1.470964in}}%
\pgfpathlineto{\pgfqpoint{3.789450in}{1.471617in}}%
\pgfpathlineto{\pgfqpoint{3.789642in}{1.470966in}}%
\pgfpathlineto{\pgfqpoint{3.789834in}{1.470209in}}%
\pgfpathlineto{\pgfqpoint{3.790026in}{1.470717in}}%
\pgfpathlineto{\pgfqpoint{3.791179in}{1.475734in}}%
\pgfpathlineto{\pgfqpoint{3.791371in}{1.474994in}}%
\pgfpathlineto{\pgfqpoint{3.792140in}{1.479198in}}%
\pgfpathlineto{\pgfqpoint{3.792717in}{1.471167in}}%
\pgfpathlineto{\pgfqpoint{3.793485in}{1.472441in}}%
\pgfpathlineto{\pgfqpoint{3.793870in}{1.470839in}}%
\pgfpathlineto{\pgfqpoint{3.794254in}{1.473301in}}%
\pgfpathlineto{\pgfqpoint{3.794638in}{1.473754in}}%
\pgfpathlineto{\pgfqpoint{3.794830in}{1.474844in}}%
\pgfpathlineto{\pgfqpoint{3.795023in}{1.471666in}}%
\pgfpathlineto{\pgfqpoint{3.795215in}{1.467529in}}%
\pgfpathlineto{\pgfqpoint{3.795984in}{1.471559in}}%
\pgfpathlineto{\pgfqpoint{3.796752in}{1.467014in}}%
\pgfpathlineto{\pgfqpoint{3.797137in}{1.471398in}}%
\pgfpathlineto{\pgfqpoint{3.797329in}{1.473678in}}%
\pgfpathlineto{\pgfqpoint{3.797521in}{1.470540in}}%
\pgfpathlineto{\pgfqpoint{3.798097in}{1.470613in}}%
\pgfpathlineto{\pgfqpoint{3.798290in}{1.468957in}}%
\pgfpathlineto{\pgfqpoint{3.798674in}{1.470393in}}%
\pgfpathlineto{\pgfqpoint{3.798866in}{1.473680in}}%
\pgfpathlineto{\pgfqpoint{3.799635in}{1.468754in}}%
\pgfpathlineto{\pgfqpoint{3.800019in}{1.467188in}}%
\pgfpathlineto{\pgfqpoint{3.800403in}{1.469653in}}%
\pgfpathlineto{\pgfqpoint{3.800596in}{1.467679in}}%
\pgfpathlineto{\pgfqpoint{3.801556in}{1.474698in}}%
\pgfpathlineto{\pgfqpoint{3.801941in}{1.473663in}}%
\pgfpathlineto{\pgfqpoint{3.802709in}{1.470852in}}%
\pgfpathlineto{\pgfqpoint{3.802902in}{1.471589in}}%
\pgfpathlineto{\pgfqpoint{3.804631in}{1.479209in}}%
\pgfpathlineto{\pgfqpoint{3.804823in}{1.478340in}}%
\pgfpathlineto{\pgfqpoint{3.805400in}{1.480340in}}%
\pgfpathlineto{\pgfqpoint{3.805592in}{1.481120in}}%
\pgfpathlineto{\pgfqpoint{3.805784in}{1.478425in}}%
\pgfpathlineto{\pgfqpoint{3.805976in}{1.476020in}}%
\pgfpathlineto{\pgfqpoint{3.806361in}{1.482975in}}%
\pgfpathlineto{\pgfqpoint{3.806937in}{1.484739in}}%
\pgfpathlineto{\pgfqpoint{3.807321in}{1.482428in}}%
\pgfpathlineto{\pgfqpoint{3.807514in}{1.480946in}}%
\pgfpathlineto{\pgfqpoint{3.807706in}{1.484304in}}%
\pgfpathlineto{\pgfqpoint{3.807898in}{1.483093in}}%
\pgfpathlineto{\pgfqpoint{3.808667in}{1.488533in}}%
\pgfpathlineto{\pgfqpoint{3.808859in}{1.485267in}}%
\pgfpathlineto{\pgfqpoint{3.809435in}{1.486407in}}%
\pgfpathlineto{\pgfqpoint{3.810012in}{1.482878in}}%
\pgfpathlineto{\pgfqpoint{3.810396in}{1.486212in}}%
\pgfpathlineto{\pgfqpoint{3.810588in}{1.483003in}}%
\pgfpathlineto{\pgfqpoint{3.812318in}{1.475854in}}%
\pgfpathlineto{\pgfqpoint{3.812510in}{1.475305in}}%
\pgfpathlineto{\pgfqpoint{3.812702in}{1.476096in}}%
\pgfpathlineto{\pgfqpoint{3.813086in}{1.480127in}}%
\pgfpathlineto{\pgfqpoint{3.813663in}{1.479415in}}%
\pgfpathlineto{\pgfqpoint{3.815200in}{1.468414in}}%
\pgfpathlineto{\pgfqpoint{3.815392in}{1.466591in}}%
\pgfpathlineto{\pgfqpoint{3.815969in}{1.469786in}}%
\pgfpathlineto{\pgfqpoint{3.819812in}{1.490570in}}%
\pgfpathlineto{\pgfqpoint{3.820197in}{1.488862in}}%
\pgfpathlineto{\pgfqpoint{3.820389in}{1.487720in}}%
\pgfpathlineto{\pgfqpoint{3.820773in}{1.489722in}}%
\pgfpathlineto{\pgfqpoint{3.821926in}{1.493552in}}%
\pgfpathlineto{\pgfqpoint{3.822118in}{1.491064in}}%
\pgfpathlineto{\pgfqpoint{3.822695in}{1.495857in}}%
\pgfpathlineto{\pgfqpoint{3.822887in}{1.495344in}}%
\pgfpathlineto{\pgfqpoint{3.823079in}{1.495908in}}%
\pgfpathlineto{\pgfqpoint{3.823656in}{1.489656in}}%
\pgfpathlineto{\pgfqpoint{3.824232in}{1.491964in}}%
\pgfpathlineto{\pgfqpoint{3.825770in}{1.498461in}}%
\pgfpathlineto{\pgfqpoint{3.826346in}{1.494349in}}%
\pgfpathlineto{\pgfqpoint{3.826730in}{1.500253in}}%
\pgfpathlineto{\pgfqpoint{3.827691in}{1.502232in}}%
\pgfpathlineto{\pgfqpoint{3.827883in}{1.501189in}}%
\pgfpathlineto{\pgfqpoint{3.828460in}{1.495961in}}%
\pgfpathlineto{\pgfqpoint{3.829421in}{1.497405in}}%
\pgfpathlineto{\pgfqpoint{3.829997in}{1.503755in}}%
\pgfpathlineto{\pgfqpoint{3.830766in}{1.502911in}}%
\pgfpathlineto{\pgfqpoint{3.832495in}{1.494893in}}%
\pgfpathlineto{\pgfqpoint{3.832688in}{1.498559in}}%
\pgfpathlineto{\pgfqpoint{3.833456in}{1.496574in}}%
\pgfpathlineto{\pgfqpoint{3.834225in}{1.488624in}}%
\pgfpathlineto{\pgfqpoint{3.834994in}{1.492951in}}%
\pgfpathlineto{\pgfqpoint{3.835186in}{1.493166in}}%
\pgfpathlineto{\pgfqpoint{3.836339in}{1.496758in}}%
\pgfpathlineto{\pgfqpoint{3.837300in}{1.497406in}}%
\pgfpathlineto{\pgfqpoint{3.838068in}{1.490672in}}%
\pgfpathlineto{\pgfqpoint{3.838645in}{1.493163in}}%
\pgfpathlineto{\pgfqpoint{3.839029in}{1.490867in}}%
\pgfpathlineto{\pgfqpoint{3.839798in}{1.487816in}}%
\pgfpathlineto{\pgfqpoint{3.839606in}{1.492636in}}%
\pgfpathlineto{\pgfqpoint{3.840182in}{1.489727in}}%
\pgfpathlineto{\pgfqpoint{3.841527in}{1.493973in}}%
\pgfpathlineto{\pgfqpoint{3.841719in}{1.493765in}}%
\pgfpathlineto{\pgfqpoint{3.841912in}{1.494081in}}%
\pgfpathlineto{\pgfqpoint{3.842104in}{1.492596in}}%
\pgfpathlineto{\pgfqpoint{3.844025in}{1.480647in}}%
\pgfpathlineto{\pgfqpoint{3.842488in}{1.492701in}}%
\pgfpathlineto{\pgfqpoint{3.844218in}{1.484896in}}%
\pgfpathlineto{\pgfqpoint{3.844794in}{1.484024in}}%
\pgfpathlineto{\pgfqpoint{3.845563in}{1.484828in}}%
\pgfpathlineto{\pgfqpoint{3.846332in}{1.477913in}}%
\pgfpathlineto{\pgfqpoint{3.846524in}{1.480455in}}%
\pgfpathlineto{\pgfqpoint{3.846908in}{1.476588in}}%
\pgfpathlineto{\pgfqpoint{3.847485in}{1.478494in}}%
\pgfpathlineto{\pgfqpoint{3.847677in}{1.476872in}}%
\pgfpathlineto{\pgfqpoint{3.848061in}{1.481668in}}%
\pgfpathlineto{\pgfqpoint{3.848253in}{1.482475in}}%
\pgfpathlineto{\pgfqpoint{3.848445in}{1.479315in}}%
\pgfpathlineto{\pgfqpoint{3.848638in}{1.481423in}}%
\pgfpathlineto{\pgfqpoint{3.850175in}{1.474540in}}%
\pgfpathlineto{\pgfqpoint{3.851328in}{1.476156in}}%
\pgfpathlineto{\pgfqpoint{3.852673in}{1.483177in}}%
\pgfpathlineto{\pgfqpoint{3.853634in}{1.479821in}}%
\pgfpathlineto{\pgfqpoint{3.853057in}{1.483318in}}%
\pgfpathlineto{\pgfqpoint{3.853826in}{1.480931in}}%
\pgfpathlineto{\pgfqpoint{3.854210in}{1.484740in}}%
\pgfpathlineto{\pgfqpoint{3.854403in}{1.481758in}}%
\pgfpathlineto{\pgfqpoint{3.854595in}{1.478240in}}%
\pgfpathlineto{\pgfqpoint{3.855556in}{1.480789in}}%
\pgfpathlineto{\pgfqpoint{3.855748in}{1.480237in}}%
\pgfpathlineto{\pgfqpoint{3.855940in}{1.482840in}}%
\pgfpathlineto{\pgfqpoint{3.856132in}{1.481022in}}%
\pgfpathlineto{\pgfqpoint{3.856324in}{1.483284in}}%
\pgfpathlineto{\pgfqpoint{3.856709in}{1.478020in}}%
\pgfpathlineto{\pgfqpoint{3.857093in}{1.479694in}}%
\pgfpathlineto{\pgfqpoint{3.857477in}{1.480287in}}%
\pgfpathlineto{\pgfqpoint{3.858438in}{1.477893in}}%
\pgfpathlineto{\pgfqpoint{3.858630in}{1.477547in}}%
\pgfpathlineto{\pgfqpoint{3.858822in}{1.478333in}}%
\pgfpathlineto{\pgfqpoint{3.860360in}{1.486871in}}%
\pgfpathlineto{\pgfqpoint{3.860744in}{1.483823in}}%
\pgfpathlineto{\pgfqpoint{3.861321in}{1.484729in}}%
\pgfpathlineto{\pgfqpoint{3.862281in}{1.490425in}}%
\pgfpathlineto{\pgfqpoint{3.862858in}{1.490050in}}%
\pgfpathlineto{\pgfqpoint{3.863050in}{1.489789in}}%
\pgfpathlineto{\pgfqpoint{3.863627in}{1.496177in}}%
\pgfpathlineto{\pgfqpoint{3.864203in}{1.494858in}}%
\pgfpathlineto{\pgfqpoint{3.864972in}{1.493471in}}%
\pgfpathlineto{\pgfqpoint{3.865740in}{1.498739in}}%
\pgfpathlineto{\pgfqpoint{3.865933in}{1.496209in}}%
\pgfpathlineto{\pgfqpoint{3.867278in}{1.488760in}}%
\pgfpathlineto{\pgfqpoint{3.868046in}{1.493867in}}%
\pgfpathlineto{\pgfqpoint{3.868431in}{1.491757in}}%
\pgfpathlineto{\pgfqpoint{3.868815in}{1.492073in}}%
\pgfpathlineto{\pgfqpoint{3.870737in}{1.504756in}}%
\pgfpathlineto{\pgfqpoint{3.871506in}{1.503878in}}%
\pgfpathlineto{\pgfqpoint{3.871698in}{1.503428in}}%
\pgfpathlineto{\pgfqpoint{3.871890in}{1.507184in}}%
\pgfpathlineto{\pgfqpoint{3.872659in}{1.503760in}}%
\pgfpathlineto{\pgfqpoint{3.872851in}{1.503533in}}%
\pgfpathlineto{\pgfqpoint{3.873043in}{1.504193in}}%
\pgfpathlineto{\pgfqpoint{3.873619in}{1.508212in}}%
\pgfpathlineto{\pgfqpoint{3.874388in}{1.505805in}}%
\pgfpathlineto{\pgfqpoint{3.874580in}{1.504876in}}%
\pgfpathlineto{\pgfqpoint{3.874772in}{1.505498in}}%
\pgfpathlineto{\pgfqpoint{3.874965in}{1.508582in}}%
\pgfpathlineto{\pgfqpoint{3.875349in}{1.504358in}}%
\pgfpathlineto{\pgfqpoint{3.875925in}{1.507723in}}%
\pgfpathlineto{\pgfqpoint{3.876502in}{1.504063in}}%
\pgfpathlineto{\pgfqpoint{3.877078in}{1.505698in}}%
\pgfpathlineto{\pgfqpoint{3.877463in}{1.508245in}}%
\pgfpathlineto{\pgfqpoint{3.877847in}{1.504337in}}%
\pgfpathlineto{\pgfqpoint{3.879192in}{1.498861in}}%
\pgfpathlineto{\pgfqpoint{3.879384in}{1.500584in}}%
\pgfpathlineto{\pgfqpoint{3.880730in}{1.507350in}}%
\pgfpathlineto{\pgfqpoint{3.879961in}{1.500452in}}%
\pgfpathlineto{\pgfqpoint{3.881114in}{1.505595in}}%
\pgfpathlineto{\pgfqpoint{3.881883in}{1.496066in}}%
\pgfpathlineto{\pgfqpoint{3.882651in}{1.499928in}}%
\pgfpathlineto{\pgfqpoint{3.883228in}{1.495369in}}%
\pgfpathlineto{\pgfqpoint{3.884189in}{1.496079in}}%
\pgfpathlineto{\pgfqpoint{3.885342in}{1.505229in}}%
\pgfpathlineto{\pgfqpoint{3.885534in}{1.502955in}}%
\pgfpathlineto{\pgfqpoint{3.887263in}{1.492481in}}%
\pgfpathlineto{\pgfqpoint{3.888801in}{1.505876in}}%
\pgfpathlineto{\pgfqpoint{3.889185in}{1.502592in}}%
\pgfpathlineto{\pgfqpoint{3.890530in}{1.500434in}}%
\pgfpathlineto{\pgfqpoint{3.891299in}{1.506715in}}%
\pgfpathlineto{\pgfqpoint{3.891875in}{1.505659in}}%
\pgfpathlineto{\pgfqpoint{3.893605in}{1.495565in}}%
\pgfpathlineto{\pgfqpoint{3.893797in}{1.498802in}}%
\pgfpathlineto{\pgfqpoint{3.893989in}{1.499700in}}%
\pgfpathlineto{\pgfqpoint{3.894181in}{1.496654in}}%
\pgfpathlineto{\pgfqpoint{3.894374in}{1.497158in}}%
\pgfpathlineto{\pgfqpoint{3.895527in}{1.490292in}}%
\pgfpathlineto{\pgfqpoint{3.895911in}{1.491568in}}%
\pgfpathlineto{\pgfqpoint{3.897064in}{1.495030in}}%
\pgfpathlineto{\pgfqpoint{3.897256in}{1.494136in}}%
\pgfpathlineto{\pgfqpoint{3.898409in}{1.488545in}}%
\pgfpathlineto{\pgfqpoint{3.898601in}{1.490205in}}%
\pgfpathlineto{\pgfqpoint{3.900523in}{1.481450in}}%
\pgfpathlineto{\pgfqpoint{3.900907in}{1.483538in}}%
\pgfpathlineto{\pgfqpoint{3.901868in}{1.487908in}}%
\pgfpathlineto{\pgfqpoint{3.902060in}{1.485086in}}%
\pgfpathlineto{\pgfqpoint{3.902252in}{1.486673in}}%
\pgfpathlineto{\pgfqpoint{3.902445in}{1.483834in}}%
\pgfpathlineto{\pgfqpoint{3.902637in}{1.480130in}}%
\pgfpathlineto{\pgfqpoint{3.903405in}{1.485505in}}%
\pgfpathlineto{\pgfqpoint{3.903790in}{1.486791in}}%
\pgfpathlineto{\pgfqpoint{3.903982in}{1.489196in}}%
\pgfpathlineto{\pgfqpoint{3.904751in}{1.485894in}}%
\pgfpathlineto{\pgfqpoint{3.905135in}{1.483997in}}%
\pgfpathlineto{\pgfqpoint{3.905519in}{1.487204in}}%
\pgfpathlineto{\pgfqpoint{3.907249in}{1.502583in}}%
\pgfpathlineto{\pgfqpoint{3.907633in}{1.499235in}}%
\pgfpathlineto{\pgfqpoint{3.908978in}{1.494156in}}%
\pgfpathlineto{\pgfqpoint{3.909555in}{1.493600in}}%
\pgfpathlineto{\pgfqpoint{3.909747in}{1.495028in}}%
\pgfpathlineto{\pgfqpoint{3.910900in}{1.489942in}}%
\pgfpathlineto{\pgfqpoint{3.911092in}{1.490725in}}%
\pgfpathlineto{\pgfqpoint{3.911284in}{1.491863in}}%
\pgfpathlineto{\pgfqpoint{3.911476in}{1.488629in}}%
\pgfpathlineto{\pgfqpoint{3.911861in}{1.490115in}}%
\pgfpathlineto{\pgfqpoint{3.913206in}{1.486354in}}%
\pgfpathlineto{\pgfqpoint{3.913782in}{1.487786in}}%
\pgfpathlineto{\pgfqpoint{3.914935in}{1.480863in}}%
\pgfpathlineto{\pgfqpoint{3.916473in}{1.493114in}}%
\pgfpathlineto{\pgfqpoint{3.916665in}{1.491217in}}%
\pgfpathlineto{\pgfqpoint{3.917818in}{1.484823in}}%
\pgfpathlineto{\pgfqpoint{3.918010in}{1.487177in}}%
\pgfpathlineto{\pgfqpoint{3.918395in}{1.488343in}}%
\pgfpathlineto{\pgfqpoint{3.918587in}{1.487846in}}%
\pgfpathlineto{\pgfqpoint{3.920124in}{1.474812in}}%
\pgfpathlineto{\pgfqpoint{3.920316in}{1.475753in}}%
\pgfpathlineto{\pgfqpoint{3.921854in}{1.482171in}}%
\pgfpathlineto{\pgfqpoint{3.922430in}{1.479217in}}%
\pgfpathlineto{\pgfqpoint{3.923391in}{1.476608in}}%
\pgfpathlineto{\pgfqpoint{3.923775in}{1.477063in}}%
\pgfpathlineto{\pgfqpoint{3.923967in}{1.476562in}}%
\pgfpathlineto{\pgfqpoint{3.924544in}{1.477435in}}%
\pgfpathlineto{\pgfqpoint{3.924736in}{1.481079in}}%
\pgfpathlineto{\pgfqpoint{3.925313in}{1.474481in}}%
\pgfpathlineto{\pgfqpoint{3.925505in}{1.475032in}}%
\pgfpathlineto{\pgfqpoint{3.926273in}{1.477397in}}%
\pgfpathlineto{\pgfqpoint{3.926466in}{1.474380in}}%
\pgfpathlineto{\pgfqpoint{3.926658in}{1.475507in}}%
\pgfpathlineto{\pgfqpoint{3.926850in}{1.473997in}}%
\pgfpathlineto{\pgfqpoint{3.927811in}{1.474339in}}%
\pgfpathlineto{\pgfqpoint{3.928003in}{1.475113in}}%
\pgfpathlineto{\pgfqpoint{3.928387in}{1.472689in}}%
\pgfpathlineto{\pgfqpoint{3.928579in}{1.473920in}}%
\pgfpathlineto{\pgfqpoint{3.928772in}{1.472116in}}%
\pgfpathlineto{\pgfqpoint{3.929348in}{1.476064in}}%
\pgfpathlineto{\pgfqpoint{3.929732in}{1.472504in}}%
\pgfpathlineto{\pgfqpoint{3.929925in}{1.472953in}}%
\pgfpathlineto{\pgfqpoint{3.930117in}{1.477308in}}%
\pgfpathlineto{\pgfqpoint{3.930885in}{1.469710in}}%
\pgfpathlineto{\pgfqpoint{3.931078in}{1.473923in}}%
\pgfpathlineto{\pgfqpoint{3.932807in}{1.465362in}}%
\pgfpathlineto{\pgfqpoint{3.932999in}{1.465948in}}%
\pgfpathlineto{\pgfqpoint{3.933191in}{1.466505in}}%
\pgfpathlineto{\pgfqpoint{3.933576in}{1.464643in}}%
\pgfpathlineto{\pgfqpoint{3.933960in}{1.463145in}}%
\pgfpathlineto{\pgfqpoint{3.934152in}{1.463970in}}%
\pgfpathlineto{\pgfqpoint{3.935497in}{1.468418in}}%
\pgfpathlineto{\pgfqpoint{3.936074in}{1.467390in}}%
\pgfpathlineto{\pgfqpoint{3.936650in}{1.473111in}}%
\pgfpathlineto{\pgfqpoint{3.937611in}{1.462175in}}%
\pgfpathlineto{\pgfqpoint{3.937996in}{1.464870in}}%
\pgfpathlineto{\pgfqpoint{3.938764in}{1.471173in}}%
\pgfpathlineto{\pgfqpoint{3.939533in}{1.470728in}}%
\pgfpathlineto{\pgfqpoint{3.940686in}{1.461337in}}%
\pgfpathlineto{\pgfqpoint{3.941070in}{1.462014in}}%
\pgfpathlineto{\pgfqpoint{3.941262in}{1.464918in}}%
\pgfpathlineto{\pgfqpoint{3.941647in}{1.461502in}}%
\pgfpathlineto{\pgfqpoint{3.942031in}{1.461914in}}%
\pgfpathlineto{\pgfqpoint{3.942223in}{1.462334in}}%
\pgfpathlineto{\pgfqpoint{3.943953in}{1.478010in}}%
\pgfpathlineto{\pgfqpoint{3.944337in}{1.478468in}}%
\pgfpathlineto{\pgfqpoint{3.944722in}{1.477086in}}%
\pgfpathlineto{\pgfqpoint{3.946259in}{1.462582in}}%
\pgfpathlineto{\pgfqpoint{3.947028in}{1.467187in}}%
\pgfpathlineto{\pgfqpoint{3.947220in}{1.464265in}}%
\pgfpathlineto{\pgfqpoint{3.947988in}{1.460829in}}%
\pgfpathlineto{\pgfqpoint{3.948181in}{1.463899in}}%
\pgfpathlineto{\pgfqpoint{3.949526in}{1.467035in}}%
\pgfpathlineto{\pgfqpoint{3.951640in}{1.480552in}}%
\pgfpathlineto{\pgfqpoint{3.951832in}{1.478415in}}%
\pgfpathlineto{\pgfqpoint{3.952408in}{1.480369in}}%
\pgfpathlineto{\pgfqpoint{3.952600in}{1.478550in}}%
\pgfpathlineto{\pgfqpoint{3.954522in}{1.466475in}}%
\pgfpathlineto{\pgfqpoint{3.954714in}{1.466835in}}%
\pgfpathlineto{\pgfqpoint{3.955483in}{1.473724in}}%
\pgfpathlineto{\pgfqpoint{3.956059in}{1.471850in}}%
\pgfpathlineto{\pgfqpoint{3.957597in}{1.477503in}}%
\pgfpathlineto{\pgfqpoint{3.957789in}{1.477123in}}%
\pgfpathlineto{\pgfqpoint{3.958173in}{1.477847in}}%
\pgfpathlineto{\pgfqpoint{3.959711in}{1.487059in}}%
\pgfpathlineto{\pgfqpoint{3.959903in}{1.487292in}}%
\pgfpathlineto{\pgfqpoint{3.962593in}{1.469806in}}%
\pgfpathlineto{\pgfqpoint{3.962785in}{1.471754in}}%
\pgfpathlineto{\pgfqpoint{3.963554in}{1.480797in}}%
\pgfpathlineto{\pgfqpoint{3.964323in}{1.477751in}}%
\pgfpathlineto{\pgfqpoint{3.966052in}{1.466050in}}%
\pgfpathlineto{\pgfqpoint{3.966244in}{1.467814in}}%
\pgfpathlineto{\pgfqpoint{3.966629in}{1.470332in}}%
\pgfpathlineto{\pgfqpoint{3.967205in}{1.466666in}}%
\pgfpathlineto{\pgfqpoint{3.967590in}{1.467646in}}%
\pgfpathlineto{\pgfqpoint{3.967782in}{1.463787in}}%
\pgfpathlineto{\pgfqpoint{3.968166in}{1.468523in}}%
\pgfpathlineto{\pgfqpoint{3.968743in}{1.465984in}}%
\pgfpathlineto{\pgfqpoint{3.969127in}{1.466910in}}%
\pgfpathlineto{\pgfqpoint{3.969319in}{1.464891in}}%
\pgfpathlineto{\pgfqpoint{3.969511in}{1.466022in}}%
\pgfpathlineto{\pgfqpoint{3.969896in}{1.463188in}}%
\pgfpathlineto{\pgfqpoint{3.970088in}{1.463682in}}%
\pgfpathlineto{\pgfqpoint{3.970280in}{1.468382in}}%
\pgfpathlineto{\pgfqpoint{3.971049in}{1.463051in}}%
\pgfpathlineto{\pgfqpoint{3.971817in}{1.465987in}}%
\pgfpathlineto{\pgfqpoint{3.971433in}{1.462573in}}%
\pgfpathlineto{\pgfqpoint{3.972202in}{1.465019in}}%
\pgfpathlineto{\pgfqpoint{3.973547in}{1.461360in}}%
\pgfpathlineto{\pgfqpoint{3.973739in}{1.461217in}}%
\pgfpathlineto{\pgfqpoint{3.974892in}{1.454668in}}%
\pgfpathlineto{\pgfqpoint{3.975276in}{1.457226in}}%
\pgfpathlineto{\pgfqpoint{3.975468in}{1.456988in}}%
\pgfpathlineto{\pgfqpoint{3.977006in}{1.449641in}}%
\pgfpathlineto{\pgfqpoint{3.977967in}{1.454516in}}%
\pgfpathlineto{\pgfqpoint{3.978159in}{1.452251in}}%
\pgfpathlineto{\pgfqpoint{3.978351in}{1.448761in}}%
\pgfpathlineto{\pgfqpoint{3.979120in}{1.453908in}}%
\pgfpathlineto{\pgfqpoint{3.979312in}{1.451364in}}%
\pgfpathlineto{\pgfqpoint{3.980465in}{1.457711in}}%
\pgfpathlineto{\pgfqpoint{3.980657in}{1.457695in}}%
\pgfpathlineto{\pgfqpoint{3.981233in}{1.458325in}}%
\pgfpathlineto{\pgfqpoint{3.982002in}{1.454826in}}%
\pgfpathlineto{\pgfqpoint{3.982194in}{1.459239in}}%
\pgfpathlineto{\pgfqpoint{3.982963in}{1.454656in}}%
\pgfpathlineto{\pgfqpoint{3.983155in}{1.456317in}}%
\pgfpathlineto{\pgfqpoint{3.983539in}{1.455342in}}%
\pgfpathlineto{\pgfqpoint{3.983732in}{1.456659in}}%
\pgfpathlineto{\pgfqpoint{3.984116in}{1.460923in}}%
\pgfpathlineto{\pgfqpoint{3.984692in}{1.455960in}}%
\pgfpathlineto{\pgfqpoint{3.984885in}{1.457962in}}%
\pgfpathlineto{\pgfqpoint{3.985461in}{1.456906in}}%
\pgfpathlineto{\pgfqpoint{3.986422in}{1.461344in}}%
\pgfpathlineto{\pgfqpoint{3.986806in}{1.460588in}}%
\pgfpathlineto{\pgfqpoint{3.986998in}{1.458939in}}%
\pgfpathlineto{\pgfqpoint{3.987575in}{1.462591in}}%
\pgfpathlineto{\pgfqpoint{3.987959in}{1.463854in}}%
\pgfpathlineto{\pgfqpoint{3.988728in}{1.459442in}}%
\pgfpathlineto{\pgfqpoint{3.989112in}{1.461550in}}%
\pgfpathlineto{\pgfqpoint{3.990073in}{1.457840in}}%
\pgfpathlineto{\pgfqpoint{3.991803in}{1.449629in}}%
\pgfpathlineto{\pgfqpoint{3.991995in}{1.449985in}}%
\pgfpathlineto{\pgfqpoint{3.992764in}{1.451337in}}%
\pgfpathlineto{\pgfqpoint{3.993724in}{1.452030in}}%
\pgfpathlineto{\pgfqpoint{3.994109in}{1.446264in}}%
\pgfpathlineto{\pgfqpoint{3.994301in}{1.445811in}}%
\pgfpathlineto{\pgfqpoint{3.994493in}{1.447721in}}%
\pgfpathlineto{\pgfqpoint{3.995262in}{1.449748in}}%
\pgfpathlineto{\pgfqpoint{3.995454in}{1.449341in}}%
\pgfpathlineto{\pgfqpoint{3.996030in}{1.440744in}}%
\pgfpathlineto{\pgfqpoint{3.996799in}{1.441382in}}%
\pgfpathlineto{\pgfqpoint{3.997183in}{1.446711in}}%
\pgfpathlineto{\pgfqpoint{3.997952in}{1.442018in}}%
\pgfpathlineto{\pgfqpoint{3.998144in}{1.442270in}}%
\pgfpathlineto{\pgfqpoint{3.998336in}{1.441215in}}%
\pgfpathlineto{\pgfqpoint{3.998529in}{1.439925in}}%
\pgfpathlineto{\pgfqpoint{3.998721in}{1.441880in}}%
\pgfpathlineto{\pgfqpoint{3.999297in}{1.441547in}}%
\pgfpathlineto{\pgfqpoint{3.999489in}{1.442788in}}%
\pgfpathlineto{\pgfqpoint{3.999874in}{1.439837in}}%
\pgfpathlineto{\pgfqpoint{4.000066in}{1.441058in}}%
\pgfpathlineto{\pgfqpoint{4.002372in}{1.424643in}}%
\pgfpathlineto{\pgfqpoint{4.003141in}{1.425367in}}%
\pgfpathlineto{\pgfqpoint{4.003525in}{1.423730in}}%
\pgfpathlineto{\pgfqpoint{4.004294in}{1.417779in}}%
\pgfpathlineto{\pgfqpoint{4.004870in}{1.420817in}}%
\pgfpathlineto{\pgfqpoint{4.005062in}{1.420988in}}%
\pgfpathlineto{\pgfqpoint{4.005447in}{1.415289in}}%
\pgfpathlineto{\pgfqpoint{4.005831in}{1.421244in}}%
\pgfpathlineto{\pgfqpoint{4.006215in}{1.419194in}}%
\pgfpathlineto{\pgfqpoint{4.006984in}{1.416668in}}%
\pgfpathlineto{\pgfqpoint{4.006600in}{1.420505in}}%
\pgfpathlineto{\pgfqpoint{4.007176in}{1.418070in}}%
\pgfpathlineto{\pgfqpoint{4.008521in}{1.423430in}}%
\pgfpathlineto{\pgfqpoint{4.008713in}{1.423117in}}%
\pgfpathlineto{\pgfqpoint{4.008906in}{1.424727in}}%
\pgfpathlineto{\pgfqpoint{4.009482in}{1.420674in}}%
\pgfpathlineto{\pgfqpoint{4.009674in}{1.422791in}}%
\pgfpathlineto{\pgfqpoint{4.010635in}{1.418059in}}%
\pgfpathlineto{\pgfqpoint{4.011019in}{1.420680in}}%
\pgfpathlineto{\pgfqpoint{4.011404in}{1.421292in}}%
\pgfpathlineto{\pgfqpoint{4.012365in}{1.412170in}}%
\pgfpathlineto{\pgfqpoint{4.012557in}{1.415395in}}%
\pgfpathlineto{\pgfqpoint{4.013710in}{1.423810in}}%
\pgfpathlineto{\pgfqpoint{4.013902in}{1.423336in}}%
\pgfpathlineto{\pgfqpoint{4.014094in}{1.423733in}}%
\pgfpathlineto{\pgfqpoint{4.014478in}{1.423283in}}%
\pgfpathlineto{\pgfqpoint{4.015247in}{1.412155in}}%
\pgfpathlineto{\pgfqpoint{4.016016in}{1.413171in}}%
\pgfpathlineto{\pgfqpoint{4.016592in}{1.411010in}}%
\pgfpathlineto{\pgfqpoint{4.016785in}{1.412944in}}%
\pgfpathlineto{\pgfqpoint{4.016977in}{1.415173in}}%
\pgfpathlineto{\pgfqpoint{4.017361in}{1.408688in}}%
\pgfpathlineto{\pgfqpoint{4.018514in}{1.403262in}}%
\pgfpathlineto{\pgfqpoint{4.018706in}{1.404996in}}%
\pgfpathlineto{\pgfqpoint{4.019091in}{1.406344in}}%
\pgfpathlineto{\pgfqpoint{4.019283in}{1.403521in}}%
\pgfpathlineto{\pgfqpoint{4.019667in}{1.405419in}}%
\pgfpathlineto{\pgfqpoint{4.020628in}{1.401100in}}%
\pgfpathlineto{\pgfqpoint{4.022550in}{1.409181in}}%
\pgfpathlineto{\pgfqpoint{4.022742in}{1.407594in}}%
\pgfpathlineto{\pgfqpoint{4.023126in}{1.410563in}}%
\pgfpathlineto{\pgfqpoint{4.023510in}{1.409577in}}%
\pgfpathlineto{\pgfqpoint{4.023895in}{1.411849in}}%
\pgfpathlineto{\pgfqpoint{4.024279in}{1.409043in}}%
\pgfpathlineto{\pgfqpoint{4.024856in}{1.411075in}}%
\pgfpathlineto{\pgfqpoint{4.025048in}{1.411630in}}%
\pgfpathlineto{\pgfqpoint{4.025816in}{1.404587in}}%
\pgfpathlineto{\pgfqpoint{4.026393in}{1.407067in}}%
\pgfpathlineto{\pgfqpoint{4.028122in}{1.398107in}}%
\pgfpathlineto{\pgfqpoint{4.029275in}{1.409644in}}%
\pgfpathlineto{\pgfqpoint{4.029660in}{1.408274in}}%
\pgfpathlineto{\pgfqpoint{4.029852in}{1.408323in}}%
\pgfpathlineto{\pgfqpoint{4.030621in}{1.412352in}}%
\pgfpathlineto{\pgfqpoint{4.031197in}{1.410815in}}%
\pgfpathlineto{\pgfqpoint{4.031389in}{1.410023in}}%
\pgfpathlineto{\pgfqpoint{4.031774in}{1.412600in}}%
\pgfpathlineto{\pgfqpoint{4.031966in}{1.414259in}}%
\pgfpathlineto{\pgfqpoint{4.032350in}{1.413082in}}%
\pgfpathlineto{\pgfqpoint{4.033503in}{1.404985in}}%
\pgfpathlineto{\pgfqpoint{4.033695in}{1.405625in}}%
\pgfpathlineto{\pgfqpoint{4.033887in}{1.406336in}}%
\pgfpathlineto{\pgfqpoint{4.034464in}{1.404479in}}%
\pgfpathlineto{\pgfqpoint{4.034848in}{1.401082in}}%
\pgfpathlineto{\pgfqpoint{4.035233in}{1.405685in}}%
\pgfpathlineto{\pgfqpoint{4.036001in}{1.402770in}}%
\pgfpathlineto{\pgfqpoint{4.036962in}{1.406659in}}%
\pgfpathlineto{\pgfqpoint{4.037154in}{1.405659in}}%
\pgfpathlineto{\pgfqpoint{4.037346in}{1.400894in}}%
\pgfpathlineto{\pgfqpoint{4.038307in}{1.402522in}}%
\pgfpathlineto{\pgfqpoint{4.039076in}{1.407142in}}%
\pgfpathlineto{\pgfqpoint{4.039652in}{1.405727in}}%
\pgfpathlineto{\pgfqpoint{4.040037in}{1.403866in}}%
\pgfpathlineto{\pgfqpoint{4.040421in}{1.406048in}}%
\pgfpathlineto{\pgfqpoint{4.040613in}{1.407180in}}%
\pgfpathlineto{\pgfqpoint{4.040806in}{1.405639in}}%
\pgfpathlineto{\pgfqpoint{4.040998in}{1.405764in}}%
\pgfpathlineto{\pgfqpoint{4.042151in}{1.399163in}}%
\pgfpathlineto{\pgfqpoint{4.042343in}{1.401851in}}%
\pgfpathlineto{\pgfqpoint{4.042919in}{1.396858in}}%
\pgfpathlineto{\pgfqpoint{4.043112in}{1.397811in}}%
\pgfpathlineto{\pgfqpoint{4.043304in}{1.398836in}}%
\pgfpathlineto{\pgfqpoint{4.043880in}{1.396543in}}%
\pgfpathlineto{\pgfqpoint{4.044265in}{1.394853in}}%
\pgfpathlineto{\pgfqpoint{4.044457in}{1.397323in}}%
\pgfpathlineto{\pgfqpoint{4.044649in}{1.396938in}}%
\pgfpathlineto{\pgfqpoint{4.046571in}{1.415843in}}%
\pgfpathlineto{\pgfqpoint{4.046955in}{1.411888in}}%
\pgfpathlineto{\pgfqpoint{4.048300in}{1.404804in}}%
\pgfpathlineto{\pgfqpoint{4.048684in}{1.404785in}}%
\pgfpathlineto{\pgfqpoint{4.049261in}{1.403216in}}%
\pgfpathlineto{\pgfqpoint{4.049645in}{1.406034in}}%
\pgfpathlineto{\pgfqpoint{4.050414in}{1.404662in}}%
\pgfpathlineto{\pgfqpoint{4.050798in}{1.408486in}}%
\pgfpathlineto{\pgfqpoint{4.052336in}{1.398695in}}%
\pgfpathlineto{\pgfqpoint{4.053104in}{1.402810in}}%
\pgfpathlineto{\pgfqpoint{4.053296in}{1.399912in}}%
\pgfpathlineto{\pgfqpoint{4.054449in}{1.389473in}}%
\pgfpathlineto{\pgfqpoint{4.054642in}{1.391116in}}%
\pgfpathlineto{\pgfqpoint{4.055602in}{1.396820in}}%
\pgfpathlineto{\pgfqpoint{4.056179in}{1.394560in}}%
\pgfpathlineto{\pgfqpoint{4.057332in}{1.389627in}}%
\pgfpathlineto{\pgfqpoint{4.057524in}{1.390918in}}%
\pgfpathlineto{\pgfqpoint{4.057716in}{1.392747in}}%
\pgfpathlineto{\pgfqpoint{4.058485in}{1.389649in}}%
\pgfpathlineto{\pgfqpoint{4.058677in}{1.389975in}}%
\pgfpathlineto{\pgfqpoint{4.058869in}{1.388157in}}%
\pgfpathlineto{\pgfqpoint{4.059061in}{1.389269in}}%
\pgfpathlineto{\pgfqpoint{4.060022in}{1.383833in}}%
\pgfpathlineto{\pgfqpoint{4.060214in}{1.385419in}}%
\pgfpathlineto{\pgfqpoint{4.061175in}{1.390585in}}%
\pgfpathlineto{\pgfqpoint{4.061367in}{1.386886in}}%
\pgfpathlineto{\pgfqpoint{4.062520in}{1.381403in}}%
\pgfpathlineto{\pgfqpoint{4.062905in}{1.382740in}}%
\pgfpathlineto{\pgfqpoint{4.063673in}{1.381999in}}%
\pgfpathlineto{\pgfqpoint{4.064058in}{1.385618in}}%
\pgfpathlineto{\pgfqpoint{4.066172in}{1.373424in}}%
\pgfpathlineto{\pgfqpoint{4.066748in}{1.373745in}}%
\pgfpathlineto{\pgfqpoint{4.069439in}{1.357285in}}%
\pgfpathlineto{\pgfqpoint{4.070592in}{1.343926in}}%
\pgfpathlineto{\pgfqpoint{4.070784in}{1.347424in}}%
\pgfpathlineto{\pgfqpoint{4.071552in}{1.354618in}}%
\pgfpathlineto{\pgfqpoint{4.072321in}{1.352156in}}%
\pgfpathlineto{\pgfqpoint{4.072705in}{1.350163in}}%
\pgfpathlineto{\pgfqpoint{4.073474in}{1.354219in}}%
\pgfpathlineto{\pgfqpoint{4.073858in}{1.351820in}}%
\pgfpathlineto{\pgfqpoint{4.074243in}{1.354006in}}%
\pgfpathlineto{\pgfqpoint{4.076164in}{1.373409in}}%
\pgfpathlineto{\pgfqpoint{4.076933in}{1.370059in}}%
\pgfpathlineto{\pgfqpoint{4.078663in}{1.350177in}}%
\pgfpathlineto{\pgfqpoint{4.079047in}{1.353967in}}%
\pgfpathlineto{\pgfqpoint{4.079623in}{1.351039in}}%
\pgfpathlineto{\pgfqpoint{4.080200in}{1.353740in}}%
\pgfpathlineto{\pgfqpoint{4.080392in}{1.355057in}}%
\pgfpathlineto{\pgfqpoint{4.080776in}{1.351652in}}%
\pgfpathlineto{\pgfqpoint{4.081161in}{1.349522in}}%
\pgfpathlineto{\pgfqpoint{4.081353in}{1.355082in}}%
\pgfpathlineto{\pgfqpoint{4.081545in}{1.352986in}}%
\pgfpathlineto{\pgfqpoint{4.082122in}{1.351912in}}%
\pgfpathlineto{\pgfqpoint{4.083467in}{1.360213in}}%
\pgfpathlineto{\pgfqpoint{4.084235in}{1.357596in}}%
\pgfpathlineto{\pgfqpoint{4.084428in}{1.358647in}}%
\pgfpathlineto{\pgfqpoint{4.085004in}{1.366415in}}%
\pgfpathlineto{\pgfqpoint{4.085581in}{1.362266in}}%
\pgfpathlineto{\pgfqpoint{4.086734in}{1.364281in}}%
\pgfpathlineto{\pgfqpoint{4.088079in}{1.358093in}}%
\pgfpathlineto{\pgfqpoint{4.088848in}{1.364953in}}%
\pgfpathlineto{\pgfqpoint{4.089424in}{1.360443in}}%
\pgfpathlineto{\pgfqpoint{4.089616in}{1.360774in}}%
\pgfpathlineto{\pgfqpoint{4.090961in}{1.367842in}}%
\pgfpathlineto{\pgfqpoint{4.091154in}{1.368092in}}%
\pgfpathlineto{\pgfqpoint{4.092307in}{1.376129in}}%
\pgfpathlineto{\pgfqpoint{4.092499in}{1.376091in}}%
\pgfpathlineto{\pgfqpoint{4.094036in}{1.362441in}}%
\pgfpathlineto{\pgfqpoint{4.095189in}{1.367428in}}%
\pgfpathlineto{\pgfqpoint{4.095381in}{1.366251in}}%
\pgfpathlineto{\pgfqpoint{4.096919in}{1.360549in}}%
\pgfpathlineto{\pgfqpoint{4.098264in}{1.366477in}}%
\pgfpathlineto{\pgfqpoint{4.097303in}{1.359220in}}%
\pgfpathlineto{\pgfqpoint{4.098456in}{1.365714in}}%
\pgfpathlineto{\pgfqpoint{4.098648in}{1.361906in}}%
\pgfpathlineto{\pgfqpoint{4.099609in}{1.364008in}}%
\pgfpathlineto{\pgfqpoint{4.100954in}{1.360247in}}%
\pgfpathlineto{\pgfqpoint{4.101146in}{1.360342in}}%
\pgfpathlineto{\pgfqpoint{4.101915in}{1.365195in}}%
\pgfpathlineto{\pgfqpoint{4.102299in}{1.363089in}}%
\pgfpathlineto{\pgfqpoint{4.103837in}{1.355600in}}%
\pgfpathlineto{\pgfqpoint{4.104413in}{1.358584in}}%
\pgfpathlineto{\pgfqpoint{4.105374in}{1.364497in}}%
\pgfpathlineto{\pgfqpoint{4.105758in}{1.360662in}}%
\pgfpathlineto{\pgfqpoint{4.105950in}{1.360531in}}%
\pgfpathlineto{\pgfqpoint{4.106527in}{1.356709in}}%
\pgfpathlineto{\pgfqpoint{4.106911in}{1.362552in}}%
\pgfpathlineto{\pgfqpoint{4.107103in}{1.363338in}}%
\pgfpathlineto{\pgfqpoint{4.107296in}{1.360788in}}%
\pgfpathlineto{\pgfqpoint{4.107872in}{1.361620in}}%
\pgfpathlineto{\pgfqpoint{4.108833in}{1.355442in}}%
\pgfpathlineto{\pgfqpoint{4.109602in}{1.348507in}}%
\pgfpathlineto{\pgfqpoint{4.109986in}{1.350318in}}%
\pgfpathlineto{\pgfqpoint{4.110178in}{1.354348in}}%
\pgfpathlineto{\pgfqpoint{4.111139in}{1.353130in}}%
\pgfpathlineto{\pgfqpoint{4.112100in}{1.357473in}}%
\pgfpathlineto{\pgfqpoint{4.112292in}{1.356440in}}%
\pgfpathlineto{\pgfqpoint{4.113253in}{1.351223in}}%
\pgfpathlineto{\pgfqpoint{4.113445in}{1.353874in}}%
\pgfpathlineto{\pgfqpoint{4.114214in}{1.359703in}}%
\pgfpathlineto{\pgfqpoint{4.114790in}{1.356004in}}%
\pgfpathlineto{\pgfqpoint{4.115175in}{1.354894in}}%
\pgfpathlineto{\pgfqpoint{4.115559in}{1.356869in}}%
\pgfpathlineto{\pgfqpoint{4.116135in}{1.353790in}}%
\pgfpathlineto{\pgfqpoint{4.116712in}{1.355849in}}%
\pgfpathlineto{\pgfqpoint{4.116904in}{1.356265in}}%
\pgfpathlineto{\pgfqpoint{4.117096in}{1.355209in}}%
\pgfpathlineto{\pgfqpoint{4.117673in}{1.353993in}}%
\pgfpathlineto{\pgfqpoint{4.118057in}{1.355020in}}%
\pgfpathlineto{\pgfqpoint{4.118634in}{1.363032in}}%
\pgfpathlineto{\pgfqpoint{4.119402in}{1.358314in}}%
\pgfpathlineto{\pgfqpoint{4.119594in}{1.356049in}}%
\pgfpathlineto{\pgfqpoint{4.120363in}{1.359288in}}%
\pgfpathlineto{\pgfqpoint{4.120555in}{1.356746in}}%
\pgfpathlineto{\pgfqpoint{4.120940in}{1.358657in}}%
\pgfpathlineto{\pgfqpoint{4.121708in}{1.357822in}}%
\pgfpathlineto{\pgfqpoint{4.123053in}{1.349100in}}%
\pgfpathlineto{\pgfqpoint{4.124591in}{1.352441in}}%
\pgfpathlineto{\pgfqpoint{4.125167in}{1.354401in}}%
\pgfpathlineto{\pgfqpoint{4.126705in}{1.345658in}}%
\pgfpathlineto{\pgfqpoint{4.127473in}{1.350725in}}%
\pgfpathlineto{\pgfqpoint{4.128242in}{1.348972in}}%
\pgfpathlineto{\pgfqpoint{4.129203in}{1.344493in}}%
\pgfpathlineto{\pgfqpoint{4.129587in}{1.344723in}}%
\pgfpathlineto{\pgfqpoint{4.129971in}{1.354684in}}%
\pgfpathlineto{\pgfqpoint{4.130932in}{1.351393in}}%
\pgfpathlineto{\pgfqpoint{4.131701in}{1.353117in}}%
\pgfpathlineto{\pgfqpoint{4.132662in}{1.358108in}}%
\pgfpathlineto{\pgfqpoint{4.132277in}{1.352461in}}%
\pgfpathlineto{\pgfqpoint{4.132854in}{1.354008in}}%
\pgfpathlineto{\pgfqpoint{4.133046in}{1.352204in}}%
\pgfpathlineto{\pgfqpoint{4.133623in}{1.356508in}}%
\pgfpathlineto{\pgfqpoint{4.135352in}{1.365907in}}%
\pgfpathlineto{\pgfqpoint{4.135544in}{1.365446in}}%
\pgfpathlineto{\pgfqpoint{4.138043in}{1.373524in}}%
\pgfpathlineto{\pgfqpoint{4.139003in}{1.371280in}}%
\pgfpathlineto{\pgfqpoint{4.139196in}{1.370816in}}%
\pgfpathlineto{\pgfqpoint{4.140541in}{1.375966in}}%
\pgfpathlineto{\pgfqpoint{4.142655in}{1.363208in}}%
\pgfpathlineto{\pgfqpoint{4.143039in}{1.364201in}}%
\pgfpathlineto{\pgfqpoint{4.143231in}{1.363376in}}%
\pgfpathlineto{\pgfqpoint{4.143423in}{1.365721in}}%
\pgfpathlineto{\pgfqpoint{4.143808in}{1.365118in}}%
\pgfpathlineto{\pgfqpoint{4.144000in}{1.365598in}}%
\pgfpathlineto{\pgfqpoint{4.145729in}{1.352874in}}%
\pgfpathlineto{\pgfqpoint{4.145921in}{1.353976in}}%
\pgfpathlineto{\pgfqpoint{4.146306in}{1.350577in}}%
\pgfpathlineto{\pgfqpoint{4.147459in}{1.348633in}}%
\pgfpathlineto{\pgfqpoint{4.147651in}{1.349482in}}%
\pgfpathlineto{\pgfqpoint{4.148035in}{1.347452in}}%
\pgfpathlineto{\pgfqpoint{4.148996in}{1.342355in}}%
\pgfpathlineto{\pgfqpoint{4.149573in}{1.343646in}}%
\pgfpathlineto{\pgfqpoint{4.149957in}{1.349946in}}%
\pgfpathlineto{\pgfqpoint{4.150726in}{1.345549in}}%
\pgfpathlineto{\pgfqpoint{4.150918in}{1.344753in}}%
\pgfpathlineto{\pgfqpoint{4.151302in}{1.345767in}}%
\pgfpathlineto{\pgfqpoint{4.151494in}{1.345636in}}%
\pgfpathlineto{\pgfqpoint{4.151686in}{1.347433in}}%
\pgfpathlineto{\pgfqpoint{4.152071in}{1.341423in}}%
\pgfpathlineto{\pgfqpoint{4.153224in}{1.333974in}}%
\pgfpathlineto{\pgfqpoint{4.153416in}{1.336120in}}%
\pgfpathlineto{\pgfqpoint{4.154185in}{1.337323in}}%
\pgfpathlineto{\pgfqpoint{4.154761in}{1.331712in}}%
\pgfpathlineto{\pgfqpoint{4.155530in}{1.334086in}}%
\pgfpathlineto{\pgfqpoint{4.155722in}{1.332391in}}%
\pgfpathlineto{\pgfqpoint{4.156106in}{1.330218in}}%
\pgfpathlineto{\pgfqpoint{4.156491in}{1.333811in}}%
\pgfpathlineto{\pgfqpoint{4.156875in}{1.337048in}}%
\pgfpathlineto{\pgfqpoint{4.157451in}{1.335022in}}%
\pgfpathlineto{\pgfqpoint{4.158797in}{1.326616in}}%
\pgfpathlineto{\pgfqpoint{4.159181in}{1.327756in}}%
\pgfpathlineto{\pgfqpoint{4.159565in}{1.327089in}}%
\pgfpathlineto{\pgfqpoint{4.160910in}{1.336366in}}%
\pgfpathlineto{\pgfqpoint{4.161679in}{1.330723in}}%
\pgfpathlineto{\pgfqpoint{4.162448in}{1.332870in}}%
\pgfpathlineto{\pgfqpoint{4.162640in}{1.334980in}}%
\pgfpathlineto{\pgfqpoint{4.163217in}{1.332722in}}%
\pgfpathlineto{\pgfqpoint{4.163601in}{1.334172in}}%
\pgfpathlineto{\pgfqpoint{4.164177in}{1.329324in}}%
\pgfpathlineto{\pgfqpoint{4.164370in}{1.333474in}}%
\pgfpathlineto{\pgfqpoint{4.164562in}{1.336600in}}%
\pgfpathlineto{\pgfqpoint{4.165138in}{1.333007in}}%
\pgfpathlineto{\pgfqpoint{4.166676in}{1.323751in}}%
\pgfpathlineto{\pgfqpoint{4.168213in}{1.328497in}}%
\pgfpathlineto{\pgfqpoint{4.169558in}{1.321506in}}%
\pgfpathlineto{\pgfqpoint{4.169750in}{1.321104in}}%
\pgfpathlineto{\pgfqpoint{4.169942in}{1.323459in}}%
\pgfpathlineto{\pgfqpoint{4.170135in}{1.321026in}}%
\pgfpathlineto{\pgfqpoint{4.170903in}{1.322922in}}%
\pgfpathlineto{\pgfqpoint{4.171095in}{1.321487in}}%
\pgfpathlineto{\pgfqpoint{4.171480in}{1.326542in}}%
\pgfpathlineto{\pgfqpoint{4.171672in}{1.326707in}}%
\pgfpathlineto{\pgfqpoint{4.173594in}{1.317884in}}%
\pgfpathlineto{\pgfqpoint{4.173786in}{1.316460in}}%
\pgfpathlineto{\pgfqpoint{4.174362in}{1.319416in}}%
\pgfpathlineto{\pgfqpoint{4.174554in}{1.318590in}}%
\pgfpathlineto{\pgfqpoint{4.175131in}{1.319715in}}%
\pgfpathlineto{\pgfqpoint{4.175323in}{1.316275in}}%
\pgfpathlineto{\pgfqpoint{4.176668in}{1.322330in}}%
\pgfpathlineto{\pgfqpoint{4.177629in}{1.317575in}}%
\pgfpathlineto{\pgfqpoint{4.178013in}{1.318756in}}%
\pgfpathlineto{\pgfqpoint{4.178782in}{1.315904in}}%
\pgfpathlineto{\pgfqpoint{4.179359in}{1.321572in}}%
\pgfpathlineto{\pgfqpoint{4.179743in}{1.318189in}}%
\pgfpathlineto{\pgfqpoint{4.180319in}{1.323485in}}%
\pgfpathlineto{\pgfqpoint{4.180512in}{1.327423in}}%
\pgfpathlineto{\pgfqpoint{4.181280in}{1.325827in}}%
\pgfpathlineto{\pgfqpoint{4.181665in}{1.318110in}}%
\pgfpathlineto{\pgfqpoint{4.182241in}{1.323053in}}%
\pgfpathlineto{\pgfqpoint{4.182818in}{1.329390in}}%
\pgfpathlineto{\pgfqpoint{4.183202in}{1.323400in}}%
\pgfpathlineto{\pgfqpoint{4.183394in}{1.318301in}}%
\pgfpathlineto{\pgfqpoint{4.183971in}{1.324825in}}%
\pgfpathlineto{\pgfqpoint{4.184355in}{1.319695in}}%
\pgfpathlineto{\pgfqpoint{4.187814in}{1.301276in}}%
\pgfpathlineto{\pgfqpoint{4.184739in}{1.320284in}}%
\pgfpathlineto{\pgfqpoint{4.188198in}{1.303695in}}%
\pgfpathlineto{\pgfqpoint{4.188583in}{1.307765in}}%
\pgfpathlineto{\pgfqpoint{4.189159in}{1.305323in}}%
\pgfpathlineto{\pgfqpoint{4.192426in}{1.280889in}}%
\pgfpathlineto{\pgfqpoint{4.193387in}{1.288697in}}%
\pgfpathlineto{\pgfqpoint{4.194156in}{1.283921in}}%
\pgfpathlineto{\pgfqpoint{4.195501in}{1.274009in}}%
\pgfpathlineto{\pgfqpoint{4.196077in}{1.279289in}}%
\pgfpathlineto{\pgfqpoint{4.196654in}{1.278990in}}%
\pgfpathlineto{\pgfqpoint{4.196846in}{1.279517in}}%
\pgfpathlineto{\pgfqpoint{4.197615in}{1.271638in}}%
\pgfpathlineto{\pgfqpoint{4.198191in}{1.273705in}}%
\pgfpathlineto{\pgfqpoint{4.198768in}{1.277150in}}%
\pgfpathlineto{\pgfqpoint{4.199152in}{1.273209in}}%
\pgfpathlineto{\pgfqpoint{4.199536in}{1.272473in}}%
\pgfpathlineto{\pgfqpoint{4.199728in}{1.272963in}}%
\pgfpathlineto{\pgfqpoint{4.200305in}{1.275506in}}%
\pgfpathlineto{\pgfqpoint{4.200497in}{1.273506in}}%
\pgfpathlineto{\pgfqpoint{4.201842in}{1.265871in}}%
\pgfpathlineto{\pgfqpoint{4.202034in}{1.265883in}}%
\pgfpathlineto{\pgfqpoint{4.202611in}{1.267218in}}%
\pgfpathlineto{\pgfqpoint{4.203572in}{1.271020in}}%
\pgfpathlineto{\pgfqpoint{4.203764in}{1.267268in}}%
\pgfpathlineto{\pgfqpoint{4.204533in}{1.269739in}}%
\pgfpathlineto{\pgfqpoint{4.204725in}{1.269044in}}%
\pgfpathlineto{\pgfqpoint{4.205878in}{1.264733in}}%
\pgfpathlineto{\pgfqpoint{4.207223in}{1.271373in}}%
\pgfpathlineto{\pgfqpoint{4.207415in}{1.274283in}}%
\pgfpathlineto{\pgfqpoint{4.207799in}{1.269308in}}%
\pgfpathlineto{\pgfqpoint{4.208184in}{1.271368in}}%
\pgfpathlineto{\pgfqpoint{4.209529in}{1.262651in}}%
\pgfpathlineto{\pgfqpoint{4.209721in}{1.263927in}}%
\pgfpathlineto{\pgfqpoint{4.210298in}{1.266804in}}%
\pgfpathlineto{\pgfqpoint{4.210874in}{1.265943in}}%
\pgfpathlineto{\pgfqpoint{4.212027in}{1.251788in}}%
\pgfpathlineto{\pgfqpoint{4.212604in}{1.252620in}}%
\pgfpathlineto{\pgfqpoint{4.213565in}{1.257592in}}%
\pgfpathlineto{\pgfqpoint{4.213949in}{1.256998in}}%
\pgfpathlineto{\pgfqpoint{4.216063in}{1.246766in}}%
\pgfpathlineto{\pgfqpoint{4.216639in}{1.249493in}}%
\pgfpathlineto{\pgfqpoint{4.216831in}{1.252284in}}%
\pgfpathlineto{\pgfqpoint{4.217600in}{1.249143in}}%
\pgfpathlineto{\pgfqpoint{4.218177in}{1.250615in}}%
\pgfpathlineto{\pgfqpoint{4.219714in}{1.259210in}}%
\pgfpathlineto{\pgfqpoint{4.219906in}{1.259127in}}%
\pgfpathlineto{\pgfqpoint{4.220098in}{1.257218in}}%
\pgfpathlineto{\pgfqpoint{4.220483in}{1.262011in}}%
\pgfpathlineto{\pgfqpoint{4.222404in}{1.271123in}}%
\pgfpathlineto{\pgfqpoint{4.222596in}{1.270754in}}%
\pgfpathlineto{\pgfqpoint{4.223749in}{1.264352in}}%
\pgfpathlineto{\pgfqpoint{4.222981in}{1.270936in}}%
\pgfpathlineto{\pgfqpoint{4.224134in}{1.265367in}}%
\pgfpathlineto{\pgfqpoint{4.224326in}{1.267210in}}%
\pgfpathlineto{\pgfqpoint{4.224710in}{1.261813in}}%
\pgfpathlineto{\pgfqpoint{4.224902in}{1.261513in}}%
\pgfpathlineto{\pgfqpoint{4.226248in}{1.265878in}}%
\pgfpathlineto{\pgfqpoint{4.226824in}{1.264951in}}%
\pgfpathlineto{\pgfqpoint{4.226632in}{1.266722in}}%
\pgfpathlineto{\pgfqpoint{4.227016in}{1.265918in}}%
\pgfpathlineto{\pgfqpoint{4.227208in}{1.268675in}}%
\pgfpathlineto{\pgfqpoint{4.227401in}{1.265753in}}%
\pgfpathlineto{\pgfqpoint{4.227977in}{1.266345in}}%
\pgfpathlineto{\pgfqpoint{4.228169in}{1.264410in}}%
\pgfpathlineto{\pgfqpoint{4.228938in}{1.265507in}}%
\pgfpathlineto{\pgfqpoint{4.229130in}{1.266857in}}%
\pgfpathlineto{\pgfqpoint{4.229514in}{1.263464in}}%
\pgfpathlineto{\pgfqpoint{4.229707in}{1.262235in}}%
\pgfpathlineto{\pgfqpoint{4.230091in}{1.265383in}}%
\pgfpathlineto{\pgfqpoint{4.230283in}{1.264429in}}%
\pgfpathlineto{\pgfqpoint{4.230475in}{1.264526in}}%
\pgfpathlineto{\pgfqpoint{4.231436in}{1.271807in}}%
\pgfpathlineto{\pgfqpoint{4.232013in}{1.271163in}}%
\pgfpathlineto{\pgfqpoint{4.233358in}{1.265305in}}%
\pgfpathlineto{\pgfqpoint{4.234511in}{1.273584in}}%
\pgfpathlineto{\pgfqpoint{4.234703in}{1.272261in}}%
\pgfpathlineto{\pgfqpoint{4.235856in}{1.268528in}}%
\pgfpathlineto{\pgfqpoint{4.236433in}{1.268245in}}%
\pgfpathlineto{\pgfqpoint{4.237201in}{1.273932in}}%
\pgfpathlineto{\pgfqpoint{4.238162in}{1.269790in}}%
\pgfpathlineto{\pgfqpoint{4.238354in}{1.271434in}}%
\pgfpathlineto{\pgfqpoint{4.238546in}{1.271309in}}%
\pgfpathlineto{\pgfqpoint{4.238739in}{1.268601in}}%
\pgfpathlineto{\pgfqpoint{4.239315in}{1.274366in}}%
\pgfpathlineto{\pgfqpoint{4.239507in}{1.272580in}}%
\pgfpathlineto{\pgfqpoint{4.239699in}{1.271254in}}%
\pgfpathlineto{\pgfqpoint{4.239892in}{1.274344in}}%
\pgfpathlineto{\pgfqpoint{4.240084in}{1.273879in}}%
\pgfpathlineto{\pgfqpoint{4.241237in}{1.280017in}}%
\pgfpathlineto{\pgfqpoint{4.241429in}{1.279060in}}%
\pgfpathlineto{\pgfqpoint{4.242198in}{1.281545in}}%
\pgfpathlineto{\pgfqpoint{4.242774in}{1.281423in}}%
\pgfpathlineto{\pgfqpoint{4.242966in}{1.280171in}}%
\pgfpathlineto{\pgfqpoint{4.243351in}{1.282574in}}%
\pgfpathlineto{\pgfqpoint{4.243543in}{1.282464in}}%
\pgfpathlineto{\pgfqpoint{4.244119in}{1.280886in}}%
\pgfpathlineto{\pgfqpoint{4.244504in}{1.283428in}}%
\pgfpathlineto{\pgfqpoint{4.246041in}{1.275507in}}%
\pgfpathlineto{\pgfqpoint{4.246425in}{1.281261in}}%
\pgfpathlineto{\pgfqpoint{4.247194in}{1.280196in}}%
\pgfpathlineto{\pgfqpoint{4.247386in}{1.278015in}}%
\pgfpathlineto{\pgfqpoint{4.247963in}{1.282145in}}%
\pgfpathlineto{\pgfqpoint{4.248347in}{1.281171in}}%
\pgfpathlineto{\pgfqpoint{4.248731in}{1.283100in}}%
\pgfpathlineto{\pgfqpoint{4.248923in}{1.279405in}}%
\pgfpathlineto{\pgfqpoint{4.249692in}{1.283921in}}%
\pgfpathlineto{\pgfqpoint{4.249884in}{1.284458in}}%
\pgfpathlineto{\pgfqpoint{4.250461in}{1.278474in}}%
\pgfpathlineto{\pgfqpoint{4.251037in}{1.282343in}}%
\pgfpathlineto{\pgfqpoint{4.251229in}{1.283069in}}%
\pgfpathlineto{\pgfqpoint{4.251614in}{1.281024in}}%
\pgfpathlineto{\pgfqpoint{4.251806in}{1.281140in}}%
\pgfpathlineto{\pgfqpoint{4.252767in}{1.282497in}}%
\pgfpathlineto{\pgfqpoint{4.253535in}{1.275043in}}%
\pgfpathlineto{\pgfqpoint{4.255457in}{1.284610in}}%
\pgfpathlineto{\pgfqpoint{4.254112in}{1.274198in}}%
\pgfpathlineto{\pgfqpoint{4.255841in}{1.283289in}}%
\pgfpathlineto{\pgfqpoint{4.256802in}{1.280738in}}%
\pgfpathlineto{\pgfqpoint{4.257187in}{1.281386in}}%
\pgfpathlineto{\pgfqpoint{4.257571in}{1.281683in}}%
\pgfpathlineto{\pgfqpoint{4.259685in}{1.267246in}}%
\pgfpathlineto{\pgfqpoint{4.260069in}{1.268897in}}%
\pgfpathlineto{\pgfqpoint{4.261414in}{1.260107in}}%
\pgfpathlineto{\pgfqpoint{4.261607in}{1.260301in}}%
\pgfpathlineto{\pgfqpoint{4.261799in}{1.260804in}}%
\pgfpathlineto{\pgfqpoint{4.261991in}{1.260670in}}%
\pgfpathlineto{\pgfqpoint{4.263913in}{1.243421in}}%
\pgfpathlineto{\pgfqpoint{4.264297in}{1.247116in}}%
\pgfpathlineto{\pgfqpoint{4.264681in}{1.248240in}}%
\pgfpathlineto{\pgfqpoint{4.265066in}{1.246427in}}%
\pgfpathlineto{\pgfqpoint{4.266411in}{1.240852in}}%
\pgfpathlineto{\pgfqpoint{4.266603in}{1.242691in}}%
\pgfpathlineto{\pgfqpoint{4.267179in}{1.238961in}}%
\pgfpathlineto{\pgfqpoint{4.267372in}{1.240363in}}%
\pgfpathlineto{\pgfqpoint{4.267948in}{1.244667in}}%
\pgfpathlineto{\pgfqpoint{4.268140in}{1.242108in}}%
\pgfpathlineto{\pgfqpoint{4.269678in}{1.234569in}}%
\pgfpathlineto{\pgfqpoint{4.270638in}{1.235414in}}%
\pgfpathlineto{\pgfqpoint{4.271215in}{1.229371in}}%
\pgfpathlineto{\pgfqpoint{4.271599in}{1.229175in}}%
\pgfpathlineto{\pgfqpoint{4.273713in}{1.240768in}}%
\pgfpathlineto{\pgfqpoint{4.274482in}{1.236588in}}%
\pgfpathlineto{\pgfqpoint{4.274866in}{1.239460in}}%
\pgfpathlineto{\pgfqpoint{4.276211in}{1.251063in}}%
\pgfpathlineto{\pgfqpoint{4.276788in}{1.247759in}}%
\pgfpathlineto{\pgfqpoint{4.277556in}{1.240442in}}%
\pgfpathlineto{\pgfqpoint{4.278133in}{1.241829in}}%
\pgfpathlineto{\pgfqpoint{4.278517in}{1.235915in}}%
\pgfpathlineto{\pgfqpoint{4.279286in}{1.239129in}}%
\pgfpathlineto{\pgfqpoint{4.279862in}{1.242255in}}%
\pgfpathlineto{\pgfqpoint{4.280247in}{1.238120in}}%
\pgfpathlineto{\pgfqpoint{4.280631in}{1.241085in}}%
\pgfpathlineto{\pgfqpoint{4.281592in}{1.237988in}}%
\pgfpathlineto{\pgfqpoint{4.282553in}{1.230797in}}%
\pgfpathlineto{\pgfqpoint{4.282745in}{1.233578in}}%
\pgfpathlineto{\pgfqpoint{4.282937in}{1.236307in}}%
\pgfpathlineto{\pgfqpoint{4.283514in}{1.234680in}}%
\pgfpathlineto{\pgfqpoint{4.284090in}{1.226439in}}%
\pgfpathlineto{\pgfqpoint{4.284475in}{1.230745in}}%
\pgfpathlineto{\pgfqpoint{4.285628in}{1.237049in}}%
\pgfpathlineto{\pgfqpoint{4.285820in}{1.234143in}}%
\pgfpathlineto{\pgfqpoint{4.286588in}{1.237141in}}%
\pgfpathlineto{\pgfqpoint{4.287357in}{1.240689in}}%
\pgfpathlineto{\pgfqpoint{4.288126in}{1.239329in}}%
\pgfpathlineto{\pgfqpoint{4.289087in}{1.231371in}}%
\pgfpathlineto{\pgfqpoint{4.289471in}{1.232075in}}%
\pgfpathlineto{\pgfqpoint{4.290432in}{1.234899in}}%
\pgfpathlineto{\pgfqpoint{4.290047in}{1.231333in}}%
\pgfpathlineto{\pgfqpoint{4.290624in}{1.233535in}}%
\pgfpathlineto{\pgfqpoint{4.292161in}{1.225489in}}%
\pgfpathlineto{\pgfqpoint{4.292353in}{1.226963in}}%
\pgfpathlineto{\pgfqpoint{4.292738in}{1.224210in}}%
\pgfpathlineto{\pgfqpoint{4.292930in}{1.222083in}}%
\pgfpathlineto{\pgfqpoint{4.293699in}{1.225560in}}%
\pgfpathlineto{\pgfqpoint{4.293891in}{1.225798in}}%
\pgfpathlineto{\pgfqpoint{4.294659in}{1.219174in}}%
\pgfpathlineto{\pgfqpoint{4.295044in}{1.225335in}}%
\pgfpathlineto{\pgfqpoint{4.295236in}{1.224144in}}%
\pgfpathlineto{\pgfqpoint{4.295620in}{1.228241in}}%
\pgfpathlineto{\pgfqpoint{4.296773in}{1.232309in}}%
\pgfpathlineto{\pgfqpoint{4.297542in}{1.232879in}}%
\pgfpathlineto{\pgfqpoint{4.298311in}{1.223675in}}%
\pgfpathlineto{\pgfqpoint{4.298695in}{1.225854in}}%
\pgfpathlineto{\pgfqpoint{4.299079in}{1.222687in}}%
\pgfpathlineto{\pgfqpoint{4.301193in}{1.209723in}}%
\pgfpathlineto{\pgfqpoint{4.301962in}{1.212432in}}%
\pgfpathlineto{\pgfqpoint{4.302730in}{1.215274in}}%
\pgfpathlineto{\pgfqpoint{4.302923in}{1.211689in}}%
\pgfpathlineto{\pgfqpoint{4.303307in}{1.211189in}}%
\pgfpathlineto{\pgfqpoint{4.303499in}{1.213974in}}%
\pgfpathlineto{\pgfqpoint{4.304268in}{1.209279in}}%
\pgfpathlineto{\pgfqpoint{4.304460in}{1.208611in}}%
\pgfpathlineto{\pgfqpoint{4.304652in}{1.211399in}}%
\pgfpathlineto{\pgfqpoint{4.305036in}{1.213064in}}%
\pgfpathlineto{\pgfqpoint{4.305421in}{1.217219in}}%
\pgfpathlineto{\pgfqpoint{4.306189in}{1.216251in}}%
\pgfpathlineto{\pgfqpoint{4.306382in}{1.214315in}}%
\pgfpathlineto{\pgfqpoint{4.306766in}{1.218783in}}%
\pgfpathlineto{\pgfqpoint{4.308880in}{1.233709in}}%
\pgfpathlineto{\pgfqpoint{4.309264in}{1.230604in}}%
\pgfpathlineto{\pgfqpoint{4.309649in}{1.234182in}}%
\pgfpathlineto{\pgfqpoint{4.310033in}{1.233045in}}%
\pgfpathlineto{\pgfqpoint{4.310417in}{1.227539in}}%
\pgfpathlineto{\pgfqpoint{4.310994in}{1.231827in}}%
\pgfpathlineto{\pgfqpoint{4.311762in}{1.237829in}}%
\pgfpathlineto{\pgfqpoint{4.312339in}{1.233982in}}%
\pgfpathlineto{\pgfqpoint{4.312531in}{1.233890in}}%
\pgfpathlineto{\pgfqpoint{4.313876in}{1.227726in}}%
\pgfpathlineto{\pgfqpoint{4.314068in}{1.229685in}}%
\pgfpathlineto{\pgfqpoint{4.314453in}{1.232419in}}%
\pgfpathlineto{\pgfqpoint{4.314645in}{1.228452in}}%
\pgfpathlineto{\pgfqpoint{4.315029in}{1.229491in}}%
\pgfpathlineto{\pgfqpoint{4.315606in}{1.228379in}}%
\pgfpathlineto{\pgfqpoint{4.315990in}{1.223152in}}%
\pgfpathlineto{\pgfqpoint{4.316759in}{1.226194in}}%
\pgfpathlineto{\pgfqpoint{4.317143in}{1.225600in}}%
\pgfpathlineto{\pgfqpoint{4.317335in}{1.227005in}}%
\pgfpathlineto{\pgfqpoint{4.317527in}{1.226801in}}%
\pgfpathlineto{\pgfqpoint{4.318488in}{1.230868in}}%
\pgfpathlineto{\pgfqpoint{4.318680in}{1.230581in}}%
\pgfpathlineto{\pgfqpoint{4.318873in}{1.227705in}}%
\pgfpathlineto{\pgfqpoint{4.319449in}{1.234495in}}%
\pgfpathlineto{\pgfqpoint{4.320218in}{1.231170in}}%
\pgfpathlineto{\pgfqpoint{4.320410in}{1.235274in}}%
\pgfpathlineto{\pgfqpoint{4.320602in}{1.233952in}}%
\pgfpathlineto{\pgfqpoint{4.321179in}{1.232830in}}%
\pgfpathlineto{\pgfqpoint{4.321947in}{1.241357in}}%
\pgfpathlineto{\pgfqpoint{4.322332in}{1.238640in}}%
\pgfpathlineto{\pgfqpoint{4.323100in}{1.240074in}}%
\pgfpathlineto{\pgfqpoint{4.324830in}{1.246712in}}%
\pgfpathlineto{\pgfqpoint{4.325214in}{1.247028in}}%
\pgfpathlineto{\pgfqpoint{4.326944in}{1.237515in}}%
\pgfpathlineto{\pgfqpoint{4.327136in}{1.239756in}}%
\pgfpathlineto{\pgfqpoint{4.327712in}{1.237121in}}%
\pgfpathlineto{\pgfqpoint{4.328289in}{1.230279in}}%
\pgfpathlineto{\pgfqpoint{4.328865in}{1.235822in}}%
\pgfpathlineto{\pgfqpoint{4.329442in}{1.234101in}}%
\pgfpathlineto{\pgfqpoint{4.329634in}{1.236380in}}%
\pgfpathlineto{\pgfqpoint{4.330403in}{1.244973in}}%
\pgfpathlineto{\pgfqpoint{4.331171in}{1.244290in}}%
\pgfpathlineto{\pgfqpoint{4.332516in}{1.239665in}}%
\pgfpathlineto{\pgfqpoint{4.333093in}{1.246527in}}%
\pgfpathlineto{\pgfqpoint{4.333670in}{1.244487in}}%
\pgfpathlineto{\pgfqpoint{4.335591in}{1.230998in}}%
\pgfpathlineto{\pgfqpoint{4.335783in}{1.231438in}}%
\pgfpathlineto{\pgfqpoint{4.336744in}{1.238107in}}%
\pgfpathlineto{\pgfqpoint{4.337513in}{1.237334in}}%
\pgfpathlineto{\pgfqpoint{4.337897in}{1.237714in}}%
\pgfpathlineto{\pgfqpoint{4.338089in}{1.237530in}}%
\pgfpathlineto{\pgfqpoint{4.339627in}{1.227897in}}%
\pgfpathlineto{\pgfqpoint{4.339819in}{1.227476in}}%
\pgfpathlineto{\pgfqpoint{4.340395in}{1.231891in}}%
\pgfpathlineto{\pgfqpoint{4.340972in}{1.228982in}}%
\pgfpathlineto{\pgfqpoint{4.342317in}{1.221668in}}%
\pgfpathlineto{\pgfqpoint{4.342701in}{1.226386in}}%
\pgfpathlineto{\pgfqpoint{4.344431in}{1.231257in}}%
\pgfpathlineto{\pgfqpoint{4.345584in}{1.235065in}}%
\pgfpathlineto{\pgfqpoint{4.345776in}{1.234847in}}%
\pgfpathlineto{\pgfqpoint{4.346737in}{1.226013in}}%
\pgfpathlineto{\pgfqpoint{4.347313in}{1.227510in}}%
\pgfpathlineto{\pgfqpoint{4.347698in}{1.224042in}}%
\pgfpathlineto{\pgfqpoint{4.348466in}{1.225895in}}%
\pgfpathlineto{\pgfqpoint{4.348659in}{1.227452in}}%
\pgfpathlineto{\pgfqpoint{4.348851in}{1.223190in}}%
\pgfpathlineto{\pgfqpoint{4.349235in}{1.223730in}}%
\pgfpathlineto{\pgfqpoint{4.349427in}{1.220754in}}%
\pgfpathlineto{\pgfqpoint{4.350004in}{1.224334in}}%
\pgfpathlineto{\pgfqpoint{4.350196in}{1.221948in}}%
\pgfpathlineto{\pgfqpoint{4.350580in}{1.227461in}}%
\pgfpathlineto{\pgfqpoint{4.351157in}{1.221654in}}%
\pgfpathlineto{\pgfqpoint{4.351349in}{1.221373in}}%
\pgfpathlineto{\pgfqpoint{4.352694in}{1.227056in}}%
\pgfpathlineto{\pgfqpoint{4.353078in}{1.224686in}}%
\pgfpathlineto{\pgfqpoint{4.353271in}{1.228580in}}%
\pgfpathlineto{\pgfqpoint{4.353463in}{1.228554in}}%
\pgfpathlineto{\pgfqpoint{4.355192in}{1.240834in}}%
\pgfpathlineto{\pgfqpoint{4.355769in}{1.230765in}}%
\pgfpathlineto{\pgfqpoint{4.356537in}{1.231383in}}%
\pgfpathlineto{\pgfqpoint{4.357691in}{1.239742in}}%
\pgfpathlineto{\pgfqpoint{4.357883in}{1.235312in}}%
\pgfpathlineto{\pgfqpoint{4.358075in}{1.234917in}}%
\pgfpathlineto{\pgfqpoint{4.358267in}{1.236212in}}%
\pgfpathlineto{\pgfqpoint{4.359612in}{1.239345in}}%
\pgfpathlineto{\pgfqpoint{4.359804in}{1.237046in}}%
\pgfpathlineto{\pgfqpoint{4.360381in}{1.241553in}}%
\pgfpathlineto{\pgfqpoint{4.360573in}{1.242694in}}%
\pgfpathlineto{\pgfqpoint{4.361150in}{1.239517in}}%
\pgfpathlineto{\pgfqpoint{4.362110in}{1.232396in}}%
\pgfpathlineto{\pgfqpoint{4.362495in}{1.236060in}}%
\pgfpathlineto{\pgfqpoint{4.362687in}{1.237109in}}%
\pgfpathlineto{\pgfqpoint{4.363071in}{1.233151in}}%
\pgfpathlineto{\pgfqpoint{4.364609in}{1.225266in}}%
\pgfpathlineto{\pgfqpoint{4.364801in}{1.227002in}}%
\pgfpathlineto{\pgfqpoint{4.364993in}{1.230323in}}%
\pgfpathlineto{\pgfqpoint{4.365569in}{1.223124in}}%
\pgfpathlineto{\pgfqpoint{4.365762in}{1.222563in}}%
\pgfpathlineto{\pgfqpoint{4.367491in}{1.235421in}}%
\pgfpathlineto{\pgfqpoint{4.367875in}{1.237188in}}%
\pgfpathlineto{\pgfqpoint{4.368644in}{1.236166in}}%
\pgfpathlineto{\pgfqpoint{4.368836in}{1.236470in}}%
\pgfpathlineto{\pgfqpoint{4.369028in}{1.235893in}}%
\pgfpathlineto{\pgfqpoint{4.369989in}{1.233182in}}%
\pgfpathlineto{\pgfqpoint{4.369605in}{1.238087in}}%
\pgfpathlineto{\pgfqpoint{4.370181in}{1.235222in}}%
\pgfpathlineto{\pgfqpoint{4.370566in}{1.233814in}}%
\pgfpathlineto{\pgfqpoint{4.370950in}{1.236569in}}%
\pgfpathlineto{\pgfqpoint{4.371142in}{1.236665in}}%
\pgfpathlineto{\pgfqpoint{4.371334in}{1.234722in}}%
\pgfpathlineto{\pgfqpoint{4.372103in}{1.237978in}}%
\pgfpathlineto{\pgfqpoint{4.374217in}{1.226321in}}%
\pgfpathlineto{\pgfqpoint{4.372487in}{1.238862in}}%
\pgfpathlineto{\pgfqpoint{4.374793in}{1.229930in}}%
\pgfpathlineto{\pgfqpoint{4.375178in}{1.231784in}}%
\pgfpathlineto{\pgfqpoint{4.375754in}{1.229571in}}%
\pgfpathlineto{\pgfqpoint{4.376331in}{1.225772in}}%
\pgfpathlineto{\pgfqpoint{4.376523in}{1.228799in}}%
\pgfpathlineto{\pgfqpoint{4.378829in}{1.255833in}}%
\pgfpathlineto{\pgfqpoint{4.380174in}{1.243354in}}%
\pgfpathlineto{\pgfqpoint{4.380558in}{1.244988in}}%
\pgfpathlineto{\pgfqpoint{4.381904in}{1.239302in}}%
\pgfpathlineto{\pgfqpoint{4.382096in}{1.240122in}}%
\pgfpathlineto{\pgfqpoint{4.383441in}{1.250077in}}%
\pgfpathlineto{\pgfqpoint{4.384018in}{1.250984in}}%
\pgfpathlineto{\pgfqpoint{4.384786in}{1.245765in}}%
\pgfpathlineto{\pgfqpoint{4.385171in}{1.244914in}}%
\pgfpathlineto{\pgfqpoint{4.385939in}{1.247320in}}%
\pgfpathlineto{\pgfqpoint{4.386131in}{1.244974in}}%
\pgfpathlineto{\pgfqpoint{4.386708in}{1.250219in}}%
\pgfpathlineto{\pgfqpoint{4.388630in}{1.260992in}}%
\pgfpathlineto{\pgfqpoint{4.389398in}{1.259639in}}%
\pgfpathlineto{\pgfqpoint{4.389590in}{1.259449in}}%
\pgfpathlineto{\pgfqpoint{4.389783in}{1.262175in}}%
\pgfpathlineto{\pgfqpoint{4.390167in}{1.255911in}}%
\pgfpathlineto{\pgfqpoint{4.390359in}{1.259721in}}%
\pgfpathlineto{\pgfqpoint{4.391128in}{1.253287in}}%
\pgfpathlineto{\pgfqpoint{4.391704in}{1.255568in}}%
\pgfpathlineto{\pgfqpoint{4.392089in}{1.253542in}}%
\pgfpathlineto{\pgfqpoint{4.392281in}{1.258901in}}%
\pgfpathlineto{\pgfqpoint{4.393434in}{1.264117in}}%
\pgfpathlineto{\pgfqpoint{4.393626in}{1.263219in}}%
\pgfpathlineto{\pgfqpoint{4.393818in}{1.262108in}}%
\pgfpathlineto{\pgfqpoint{4.394202in}{1.264053in}}%
\pgfpathlineto{\pgfqpoint{4.396124in}{1.274488in}}%
\pgfpathlineto{\pgfqpoint{4.396893in}{1.261828in}}%
\pgfpathlineto{\pgfqpoint{4.397661in}{1.264828in}}%
\pgfpathlineto{\pgfqpoint{4.398238in}{1.265635in}}%
\pgfpathlineto{\pgfqpoint{4.399007in}{1.260474in}}%
\pgfpathlineto{\pgfqpoint{4.399199in}{1.265078in}}%
\pgfpathlineto{\pgfqpoint{4.400160in}{1.263329in}}%
\pgfpathlineto{\pgfqpoint{4.400544in}{1.264385in}}%
\pgfpathlineto{\pgfqpoint{4.400736in}{1.263227in}}%
\pgfpathlineto{\pgfqpoint{4.401313in}{1.259634in}}%
\pgfpathlineto{\pgfqpoint{4.401889in}{1.262577in}}%
\pgfpathlineto{\pgfqpoint{4.402081in}{1.262315in}}%
\pgfpathlineto{\pgfqpoint{4.402273in}{1.263472in}}%
\pgfpathlineto{\pgfqpoint{4.403042in}{1.266690in}}%
\pgfpathlineto{\pgfqpoint{4.403234in}{1.264083in}}%
\pgfpathlineto{\pgfqpoint{4.405156in}{1.257265in}}%
\pgfpathlineto{\pgfqpoint{4.405732in}{1.261236in}}%
\pgfpathlineto{\pgfqpoint{4.406117in}{1.257134in}}%
\pgfpathlineto{\pgfqpoint{4.407078in}{1.247693in}}%
\pgfpathlineto{\pgfqpoint{4.408039in}{1.249978in}}%
\pgfpathlineto{\pgfqpoint{4.408999in}{1.257741in}}%
\pgfpathlineto{\pgfqpoint{4.409960in}{1.256517in}}%
\pgfpathlineto{\pgfqpoint{4.410152in}{1.256211in}}%
\pgfpathlineto{\pgfqpoint{4.410537in}{1.256612in}}%
\pgfpathlineto{\pgfqpoint{4.410729in}{1.258439in}}%
\pgfpathlineto{\pgfqpoint{4.411498in}{1.258078in}}%
\pgfpathlineto{\pgfqpoint{4.413611in}{1.245895in}}%
\pgfpathlineto{\pgfqpoint{4.413996in}{1.249322in}}%
\pgfpathlineto{\pgfqpoint{4.414572in}{1.245330in}}%
\pgfpathlineto{\pgfqpoint{4.414764in}{1.244293in}}%
\pgfpathlineto{\pgfqpoint{4.414957in}{1.247349in}}%
\pgfpathlineto{\pgfqpoint{4.415149in}{1.246305in}}%
\pgfpathlineto{\pgfqpoint{4.417263in}{1.259200in}}%
\pgfpathlineto{\pgfqpoint{4.417455in}{1.258881in}}%
\pgfpathlineto{\pgfqpoint{4.417647in}{1.258792in}}%
\pgfpathlineto{\pgfqpoint{4.419569in}{1.245828in}}%
\pgfpathlineto{\pgfqpoint{4.419953in}{1.246211in}}%
\pgfpathlineto{\pgfqpoint{4.420145in}{1.246301in}}%
\pgfpathlineto{\pgfqpoint{4.420337in}{1.245530in}}%
\pgfpathlineto{\pgfqpoint{4.423220in}{1.226910in}}%
\pgfpathlineto{\pgfqpoint{4.420722in}{1.245962in}}%
\pgfpathlineto{\pgfqpoint{4.423988in}{1.229176in}}%
\pgfpathlineto{\pgfqpoint{4.424757in}{1.235065in}}%
\pgfpathlineto{\pgfqpoint{4.425141in}{1.233463in}}%
\pgfpathlineto{\pgfqpoint{4.425334in}{1.232596in}}%
\pgfpathlineto{\pgfqpoint{4.425718in}{1.234964in}}%
\pgfpathlineto{\pgfqpoint{4.426679in}{1.241513in}}%
\pgfpathlineto{\pgfqpoint{4.427255in}{1.240277in}}%
\pgfpathlineto{\pgfqpoint{4.427640in}{1.240129in}}%
\pgfpathlineto{\pgfqpoint{4.429369in}{1.232448in}}%
\pgfpathlineto{\pgfqpoint{4.430714in}{1.237598in}}%
\pgfpathlineto{\pgfqpoint{4.431099in}{1.236381in}}%
\pgfpathlineto{\pgfqpoint{4.431483in}{1.240695in}}%
\pgfpathlineto{\pgfqpoint{4.431675in}{1.236004in}}%
\pgfpathlineto{\pgfqpoint{4.432636in}{1.238344in}}%
\pgfpathlineto{\pgfqpoint{4.434750in}{1.223625in}}%
\pgfpathlineto{\pgfqpoint{4.435711in}{1.224563in}}%
\pgfpathlineto{\pgfqpoint{4.435903in}{1.225143in}}%
\pgfpathlineto{\pgfqpoint{4.436287in}{1.223662in}}%
\pgfpathlineto{\pgfqpoint{4.436479in}{1.223893in}}%
\pgfpathlineto{\pgfqpoint{4.438209in}{1.212166in}}%
\pgfpathlineto{\pgfqpoint{4.438401in}{1.213231in}}%
\pgfpathlineto{\pgfqpoint{4.438785in}{1.210657in}}%
\pgfpathlineto{\pgfqpoint{4.438978in}{1.209794in}}%
\pgfpathlineto{\pgfqpoint{4.439170in}{1.211763in}}%
\pgfpathlineto{\pgfqpoint{4.440131in}{1.214883in}}%
\pgfpathlineto{\pgfqpoint{4.439746in}{1.210657in}}%
\pgfpathlineto{\pgfqpoint{4.440323in}{1.213835in}}%
\pgfpathlineto{\pgfqpoint{4.441860in}{1.206632in}}%
\pgfpathlineto{\pgfqpoint{4.442821in}{1.211457in}}%
\pgfpathlineto{\pgfqpoint{4.443013in}{1.207982in}}%
\pgfpathlineto{\pgfqpoint{4.444166in}{1.201285in}}%
\pgfpathlineto{\pgfqpoint{4.444358in}{1.202400in}}%
\pgfpathlineto{\pgfqpoint{4.445896in}{1.210494in}}%
\pgfpathlineto{\pgfqpoint{4.446088in}{1.207813in}}%
\pgfpathlineto{\pgfqpoint{4.446664in}{1.212614in}}%
\pgfpathlineto{\pgfqpoint{4.446856in}{1.212532in}}%
\pgfpathlineto{\pgfqpoint{4.447625in}{1.219541in}}%
\pgfpathlineto{\pgfqpoint{4.448202in}{1.218566in}}%
\pgfpathlineto{\pgfqpoint{4.449547in}{1.228492in}}%
\pgfpathlineto{\pgfqpoint{4.449931in}{1.227999in}}%
\pgfpathlineto{\pgfqpoint{4.451084in}{1.223878in}}%
\pgfpathlineto{\pgfqpoint{4.451276in}{1.226970in}}%
\pgfpathlineto{\pgfqpoint{4.451853in}{1.223040in}}%
\pgfpathlineto{\pgfqpoint{4.452045in}{1.224774in}}%
\pgfpathlineto{\pgfqpoint{4.453390in}{1.219430in}}%
\pgfpathlineto{\pgfqpoint{4.454735in}{1.227818in}}%
\pgfpathlineto{\pgfqpoint{4.455120in}{1.225977in}}%
\pgfpathlineto{\pgfqpoint{4.455312in}{1.224907in}}%
\pgfpathlineto{\pgfqpoint{4.455696in}{1.228621in}}%
\pgfpathlineto{\pgfqpoint{4.455888in}{1.226946in}}%
\pgfpathlineto{\pgfqpoint{4.456273in}{1.228257in}}%
\pgfpathlineto{\pgfqpoint{4.456465in}{1.226538in}}%
\pgfpathlineto{\pgfqpoint{4.456849in}{1.226679in}}%
\pgfpathlineto{\pgfqpoint{4.457618in}{1.221770in}}%
\pgfpathlineto{\pgfqpoint{4.458194in}{1.224051in}}%
\pgfpathlineto{\pgfqpoint{4.458387in}{1.223912in}}%
\pgfpathlineto{\pgfqpoint{4.458963in}{1.220940in}}%
\pgfpathlineto{\pgfqpoint{4.459540in}{1.223041in}}%
\pgfpathlineto{\pgfqpoint{4.460116in}{1.223905in}}%
\pgfpathlineto{\pgfqpoint{4.461077in}{1.217459in}}%
\pgfpathlineto{\pgfqpoint{4.461269in}{1.213129in}}%
\pgfpathlineto{\pgfqpoint{4.461653in}{1.217971in}}%
\pgfpathlineto{\pgfqpoint{4.462230in}{1.215052in}}%
\pgfpathlineto{\pgfqpoint{4.463191in}{1.213082in}}%
\pgfpathlineto{\pgfqpoint{4.462614in}{1.217704in}}%
\pgfpathlineto{\pgfqpoint{4.463383in}{1.214692in}}%
\pgfpathlineto{\pgfqpoint{4.464920in}{1.225655in}}%
\pgfpathlineto{\pgfqpoint{4.465112in}{1.224059in}}%
\pgfpathlineto{\pgfqpoint{4.465305in}{1.224976in}}%
\pgfpathlineto{\pgfqpoint{4.465689in}{1.223371in}}%
\pgfpathlineto{\pgfqpoint{4.466265in}{1.219795in}}%
\pgfpathlineto{\pgfqpoint{4.466458in}{1.224373in}}%
\pgfpathlineto{\pgfqpoint{4.466650in}{1.223520in}}%
\pgfpathlineto{\pgfqpoint{4.467034in}{1.222113in}}%
\pgfpathlineto{\pgfqpoint{4.467418in}{1.225882in}}%
\pgfpathlineto{\pgfqpoint{4.467611in}{1.228273in}}%
\pgfpathlineto{\pgfqpoint{4.468571in}{1.226562in}}%
\pgfpathlineto{\pgfqpoint{4.469532in}{1.222943in}}%
\pgfpathlineto{\pgfqpoint{4.470493in}{1.232328in}}%
\pgfpathlineto{\pgfqpoint{4.470877in}{1.231774in}}%
\pgfpathlineto{\pgfqpoint{4.471262in}{1.232596in}}%
\pgfpathlineto{\pgfqpoint{4.472799in}{1.238907in}}%
\pgfpathlineto{\pgfqpoint{4.472991in}{1.236279in}}%
\pgfpathlineto{\pgfqpoint{4.473568in}{1.241199in}}%
\pgfpathlineto{\pgfqpoint{4.474144in}{1.246227in}}%
\pgfpathlineto{\pgfqpoint{4.474721in}{1.243043in}}%
\pgfpathlineto{\pgfqpoint{4.475105in}{1.239001in}}%
\pgfpathlineto{\pgfqpoint{4.475489in}{1.244564in}}%
\pgfpathlineto{\pgfqpoint{4.475682in}{1.247204in}}%
\pgfpathlineto{\pgfqpoint{4.476066in}{1.243721in}}%
\pgfpathlineto{\pgfqpoint{4.476258in}{1.244845in}}%
\pgfpathlineto{\pgfqpoint{4.476642in}{1.237185in}}%
\pgfpathlineto{\pgfqpoint{4.477603in}{1.240864in}}%
\pgfpathlineto{\pgfqpoint{4.477795in}{1.242004in}}%
\pgfpathlineto{\pgfqpoint{4.477988in}{1.240369in}}%
\pgfpathlineto{\pgfqpoint{4.478180in}{1.236252in}}%
\pgfpathlineto{\pgfqpoint{4.478756in}{1.244803in}}%
\pgfpathlineto{\pgfqpoint{4.478948in}{1.242829in}}%
\pgfpathlineto{\pgfqpoint{4.480486in}{1.235126in}}%
\pgfpathlineto{\pgfqpoint{4.480678in}{1.235158in}}%
\pgfpathlineto{\pgfqpoint{4.480870in}{1.233384in}}%
\pgfpathlineto{\pgfqpoint{4.481255in}{1.237445in}}%
\pgfpathlineto{\pgfqpoint{4.481447in}{1.238537in}}%
\pgfpathlineto{\pgfqpoint{4.481831in}{1.234515in}}%
\pgfpathlineto{\pgfqpoint{4.482792in}{1.229312in}}%
\pgfpathlineto{\pgfqpoint{4.482984in}{1.231977in}}%
\pgfpathlineto{\pgfqpoint{4.483368in}{1.233288in}}%
\pgfpathlineto{\pgfqpoint{4.483753in}{1.231670in}}%
\pgfpathlineto{\pgfqpoint{4.484137in}{1.234562in}}%
\pgfpathlineto{\pgfqpoint{4.484906in}{1.232180in}}%
\pgfpathlineto{\pgfqpoint{4.485098in}{1.232232in}}%
\pgfpathlineto{\pgfqpoint{4.486251in}{1.227438in}}%
\pgfpathlineto{\pgfqpoint{4.486635in}{1.228013in}}%
\pgfpathlineto{\pgfqpoint{4.487980in}{1.231352in}}%
\pgfpathlineto{\pgfqpoint{4.488173in}{1.230057in}}%
\pgfpathlineto{\pgfqpoint{4.488365in}{1.233405in}}%
\pgfpathlineto{\pgfqpoint{4.488557in}{1.237384in}}%
\pgfpathlineto{\pgfqpoint{4.489326in}{1.232519in}}%
\pgfpathlineto{\pgfqpoint{4.489710in}{1.232515in}}%
\pgfpathlineto{\pgfqpoint{4.490671in}{1.224817in}}%
\pgfpathlineto{\pgfqpoint{4.490863in}{1.227116in}}%
\pgfpathlineto{\pgfqpoint{4.492400in}{1.232155in}}%
\pgfpathlineto{\pgfqpoint{4.493361in}{1.222087in}}%
\pgfpathlineto{\pgfqpoint{4.494322in}{1.226599in}}%
\pgfpathlineto{\pgfqpoint{4.495283in}{1.229789in}}%
\pgfpathlineto{\pgfqpoint{4.495475in}{1.229581in}}%
\pgfpathlineto{\pgfqpoint{4.496051in}{1.224634in}}%
\pgfpathlineto{\pgfqpoint{4.496436in}{1.228323in}}%
\pgfpathlineto{\pgfqpoint{4.497204in}{1.234713in}}%
\pgfpathlineto{\pgfqpoint{4.497397in}{1.233479in}}%
\pgfpathlineto{\pgfqpoint{4.497589in}{1.227907in}}%
\pgfpathlineto{\pgfqpoint{4.498550in}{1.229889in}}%
\pgfpathlineto{\pgfqpoint{4.498742in}{1.229454in}}%
\pgfpathlineto{\pgfqpoint{4.498934in}{1.230514in}}%
\pgfpathlineto{\pgfqpoint{4.499510in}{1.236150in}}%
\pgfpathlineto{\pgfqpoint{4.500087in}{1.235584in}}%
\pgfpathlineto{\pgfqpoint{4.500471in}{1.232126in}}%
\pgfpathlineto{\pgfqpoint{4.501048in}{1.235082in}}%
\pgfpathlineto{\pgfqpoint{4.501240in}{1.235274in}}%
\pgfpathlineto{\pgfqpoint{4.501816in}{1.227059in}}%
\pgfpathlineto{\pgfqpoint{4.502585in}{1.231195in}}%
\pgfpathlineto{\pgfqpoint{4.503930in}{1.236015in}}%
\pgfpathlineto{\pgfqpoint{4.504122in}{1.235177in}}%
\pgfpathlineto{\pgfqpoint{4.504891in}{1.232142in}}%
\pgfpathlineto{\pgfqpoint{4.505276in}{1.233411in}}%
\pgfpathlineto{\pgfqpoint{4.506429in}{1.239198in}}%
\pgfpathlineto{\pgfqpoint{4.506813in}{1.236321in}}%
\pgfpathlineto{\pgfqpoint{4.507389in}{1.234377in}}%
\pgfpathlineto{\pgfqpoint{4.509119in}{1.243288in}}%
\pgfpathlineto{\pgfqpoint{4.509311in}{1.239541in}}%
\pgfpathlineto{\pgfqpoint{4.509888in}{1.244518in}}%
\pgfpathlineto{\pgfqpoint{4.510080in}{1.243614in}}%
\pgfpathlineto{\pgfqpoint{4.512386in}{1.263283in}}%
\pgfpathlineto{\pgfqpoint{4.513154in}{1.257391in}}%
\pgfpathlineto{\pgfqpoint{4.513731in}{1.258848in}}%
\pgfpathlineto{\pgfqpoint{4.515460in}{1.266885in}}%
\pgfpathlineto{\pgfqpoint{4.516998in}{1.256891in}}%
\pgfpathlineto{\pgfqpoint{4.518919in}{1.245822in}}%
\pgfpathlineto{\pgfqpoint{4.519112in}{1.247239in}}%
\pgfpathlineto{\pgfqpoint{4.519304in}{1.246689in}}%
\pgfpathlineto{\pgfqpoint{4.519496in}{1.247568in}}%
\pgfpathlineto{\pgfqpoint{4.520265in}{1.250666in}}%
\pgfpathlineto{\pgfqpoint{4.520649in}{1.250347in}}%
\pgfpathlineto{\pgfqpoint{4.521994in}{1.240200in}}%
\pgfpathlineto{\pgfqpoint{4.522378in}{1.244835in}}%
\pgfpathlineto{\pgfqpoint{4.522571in}{1.244716in}}%
\pgfpathlineto{\pgfqpoint{4.522763in}{1.245098in}}%
\pgfpathlineto{\pgfqpoint{4.523339in}{1.250076in}}%
\pgfpathlineto{\pgfqpoint{4.523916in}{1.246437in}}%
\pgfpathlineto{\pgfqpoint{4.524108in}{1.246664in}}%
\pgfpathlineto{\pgfqpoint{4.525069in}{1.256981in}}%
\pgfpathlineto{\pgfqpoint{4.525645in}{1.256001in}}%
\pgfpathlineto{\pgfqpoint{4.526414in}{1.250125in}}%
\pgfpathlineto{\pgfqpoint{4.526990in}{1.251258in}}%
\pgfpathlineto{\pgfqpoint{4.527183in}{1.253852in}}%
\pgfpathlineto{\pgfqpoint{4.527951in}{1.250160in}}%
\pgfpathlineto{\pgfqpoint{4.528336in}{1.250827in}}%
\pgfpathlineto{\pgfqpoint{4.529681in}{1.242169in}}%
\pgfpathlineto{\pgfqpoint{4.529873in}{1.244432in}}%
\pgfpathlineto{\pgfqpoint{4.530065in}{1.244854in}}%
\pgfpathlineto{\pgfqpoint{4.530257in}{1.243129in}}%
\pgfpathlineto{\pgfqpoint{4.531218in}{1.237994in}}%
\pgfpathlineto{\pgfqpoint{4.531987in}{1.239635in}}%
\pgfpathlineto{\pgfqpoint{4.533524in}{1.248996in}}%
\pgfpathlineto{\pgfqpoint{4.533909in}{1.247433in}}%
\pgfpathlineto{\pgfqpoint{4.534485in}{1.249147in}}%
\pgfpathlineto{\pgfqpoint{4.534677in}{1.247764in}}%
\pgfpathlineto{\pgfqpoint{4.535254in}{1.240357in}}%
\pgfpathlineto{\pgfqpoint{4.536022in}{1.240530in}}%
\pgfpathlineto{\pgfqpoint{4.536215in}{1.240605in}}%
\pgfpathlineto{\pgfqpoint{4.536599in}{1.244227in}}%
\pgfpathlineto{\pgfqpoint{4.537175in}{1.240890in}}%
\pgfpathlineto{\pgfqpoint{4.537368in}{1.239380in}}%
\pgfpathlineto{\pgfqpoint{4.537560in}{1.242581in}}%
\pgfpathlineto{\pgfqpoint{4.538136in}{1.245880in}}%
\pgfpathlineto{\pgfqpoint{4.538328in}{1.242076in}}%
\pgfpathlineto{\pgfqpoint{4.538713in}{1.244846in}}%
\pgfpathlineto{\pgfqpoint{4.539674in}{1.238686in}}%
\pgfpathlineto{\pgfqpoint{4.540058in}{1.239867in}}%
\pgfpathlineto{\pgfqpoint{4.541403in}{1.237979in}}%
\pgfpathlineto{\pgfqpoint{4.541595in}{1.240809in}}%
\pgfpathlineto{\pgfqpoint{4.541980in}{1.235217in}}%
\pgfpathlineto{\pgfqpoint{4.542172in}{1.236101in}}%
\pgfpathlineto{\pgfqpoint{4.542364in}{1.233669in}}%
\pgfpathlineto{\pgfqpoint{4.543133in}{1.235824in}}%
\pgfpathlineto{\pgfqpoint{4.543325in}{1.239701in}}%
\pgfpathlineto{\pgfqpoint{4.544093in}{1.232945in}}%
\pgfpathlineto{\pgfqpoint{4.544286in}{1.232897in}}%
\pgfpathlineto{\pgfqpoint{4.544862in}{1.238098in}}%
\pgfpathlineto{\pgfqpoint{4.545631in}{1.237439in}}%
\pgfpathlineto{\pgfqpoint{4.545823in}{1.237576in}}%
\pgfpathlineto{\pgfqpoint{4.547168in}{1.230969in}}%
\pgfpathlineto{\pgfqpoint{4.547937in}{1.234327in}}%
\pgfpathlineto{\pgfqpoint{4.548321in}{1.232063in}}%
\pgfpathlineto{\pgfqpoint{4.549474in}{1.224534in}}%
\pgfpathlineto{\pgfqpoint{4.549858in}{1.225980in}}%
\pgfpathlineto{\pgfqpoint{4.550051in}{1.225613in}}%
\pgfpathlineto{\pgfqpoint{4.550243in}{1.226521in}}%
\pgfpathlineto{\pgfqpoint{4.551204in}{1.229473in}}%
\pgfpathlineto{\pgfqpoint{4.551972in}{1.221773in}}%
\pgfpathlineto{\pgfqpoint{4.552357in}{1.224719in}}%
\pgfpathlineto{\pgfqpoint{4.553318in}{1.229894in}}%
\pgfpathlineto{\pgfqpoint{4.552741in}{1.222861in}}%
\pgfpathlineto{\pgfqpoint{4.553894in}{1.227805in}}%
\pgfpathlineto{\pgfqpoint{4.554086in}{1.224857in}}%
\pgfpathlineto{\pgfqpoint{4.554663in}{1.228696in}}%
\pgfpathlineto{\pgfqpoint{4.555431in}{1.235080in}}%
\pgfpathlineto{\pgfqpoint{4.555624in}{1.232551in}}%
\pgfpathlineto{\pgfqpoint{4.556777in}{1.225315in}}%
\pgfpathlineto{\pgfqpoint{4.556969in}{1.226095in}}%
\pgfpathlineto{\pgfqpoint{4.559275in}{1.231452in}}%
\pgfpathlineto{\pgfqpoint{4.561581in}{1.220073in}}%
\pgfpathlineto{\pgfqpoint{4.561773in}{1.223484in}}%
\pgfpathlineto{\pgfqpoint{4.562157in}{1.217433in}}%
\pgfpathlineto{\pgfqpoint{4.562542in}{1.218372in}}%
\pgfpathlineto{\pgfqpoint{4.564079in}{1.208116in}}%
\pgfpathlineto{\pgfqpoint{4.564463in}{1.208830in}}%
\pgfpathlineto{\pgfqpoint{4.564848in}{1.211378in}}%
\pgfpathlineto{\pgfqpoint{4.565232in}{1.207854in}}%
\pgfpathlineto{\pgfqpoint{4.565808in}{1.205415in}}%
\pgfpathlineto{\pgfqpoint{4.566001in}{1.209584in}}%
\pgfpathlineto{\pgfqpoint{4.566385in}{1.207016in}}%
\pgfpathlineto{\pgfqpoint{4.566769in}{1.205145in}}%
\pgfpathlineto{\pgfqpoint{4.567154in}{1.206833in}}%
\pgfpathlineto{\pgfqpoint{4.567346in}{1.209569in}}%
\pgfpathlineto{\pgfqpoint{4.567538in}{1.204165in}}%
\pgfpathlineto{\pgfqpoint{4.568114in}{1.206467in}}%
\pgfpathlineto{\pgfqpoint{4.568499in}{1.206956in}}%
\pgfpathlineto{\pgfqpoint{4.569652in}{1.196565in}}%
\pgfpathlineto{\pgfqpoint{4.570036in}{1.198906in}}%
\pgfpathlineto{\pgfqpoint{4.571766in}{1.184616in}}%
\pgfpathlineto{\pgfqpoint{4.572342in}{1.187755in}}%
\pgfpathlineto{\pgfqpoint{4.572726in}{1.184220in}}%
\pgfpathlineto{\pgfqpoint{4.572919in}{1.183350in}}%
\pgfpathlineto{\pgfqpoint{4.573495in}{1.185313in}}%
\pgfpathlineto{\pgfqpoint{4.573879in}{1.183974in}}%
\pgfpathlineto{\pgfqpoint{4.574264in}{1.186230in}}%
\pgfpathlineto{\pgfqpoint{4.575993in}{1.176429in}}%
\pgfpathlineto{\pgfqpoint{4.576185in}{1.179372in}}%
\pgfpathlineto{\pgfqpoint{4.576570in}{1.172590in}}%
\pgfpathlineto{\pgfqpoint{4.577723in}{1.164592in}}%
\pgfpathlineto{\pgfqpoint{4.577915in}{1.164682in}}%
\pgfpathlineto{\pgfqpoint{4.578299in}{1.168469in}}%
\pgfpathlineto{\pgfqpoint{4.578876in}{1.167139in}}%
\pgfpathlineto{\pgfqpoint{4.580029in}{1.160552in}}%
\pgfpathlineto{\pgfqpoint{4.580221in}{1.161272in}}%
\pgfpathlineto{\pgfqpoint{4.580605in}{1.163533in}}%
\pgfpathlineto{\pgfqpoint{4.580990in}{1.160126in}}%
\pgfpathlineto{\pgfqpoint{4.581182in}{1.156365in}}%
\pgfpathlineto{\pgfqpoint{4.581374in}{1.160506in}}%
\pgfpathlineto{\pgfqpoint{4.582143in}{1.157814in}}%
\pgfpathlineto{\pgfqpoint{4.583296in}{1.162756in}}%
\pgfpathlineto{\pgfqpoint{4.582527in}{1.157359in}}%
\pgfpathlineto{\pgfqpoint{4.583488in}{1.160770in}}%
\pgfpathlineto{\pgfqpoint{4.584641in}{1.157979in}}%
\pgfpathlineto{\pgfqpoint{4.585217in}{1.158198in}}%
\pgfpathlineto{\pgfqpoint{4.585794in}{1.159674in}}%
\pgfpathlineto{\pgfqpoint{4.586755in}{1.163577in}}%
\pgfpathlineto{\pgfqpoint{4.586947in}{1.162214in}}%
\pgfpathlineto{\pgfqpoint{4.587139in}{1.159440in}}%
\pgfpathlineto{\pgfqpoint{4.587331in}{1.164218in}}%
\pgfpathlineto{\pgfqpoint{4.587908in}{1.163264in}}%
\pgfpathlineto{\pgfqpoint{4.588100in}{1.163256in}}%
\pgfpathlineto{\pgfqpoint{4.588292in}{1.161664in}}%
\pgfpathlineto{\pgfqpoint{4.588676in}{1.163776in}}%
\pgfpathlineto{\pgfqpoint{4.589061in}{1.168062in}}%
\pgfpathlineto{\pgfqpoint{4.589445in}{1.164340in}}%
\pgfpathlineto{\pgfqpoint{4.589829in}{1.160712in}}%
\pgfpathlineto{\pgfqpoint{4.590598in}{1.163070in}}%
\pgfpathlineto{\pgfqpoint{4.591175in}{1.161309in}}%
\pgfpathlineto{\pgfqpoint{4.592904in}{1.151033in}}%
\pgfpathlineto{\pgfqpoint{4.593096in}{1.151778in}}%
\pgfpathlineto{\pgfqpoint{4.595594in}{1.162291in}}%
\pgfpathlineto{\pgfqpoint{4.595787in}{1.161168in}}%
\pgfpathlineto{\pgfqpoint{4.596747in}{1.161531in}}%
\pgfpathlineto{\pgfqpoint{4.596940in}{1.162214in}}%
\pgfpathlineto{\pgfqpoint{4.597324in}{1.159728in}}%
\pgfpathlineto{\pgfqpoint{4.598477in}{1.153566in}}%
\pgfpathlineto{\pgfqpoint{4.598861in}{1.156516in}}%
\pgfpathlineto{\pgfqpoint{4.599246in}{1.161431in}}%
\pgfpathlineto{\pgfqpoint{4.600014in}{1.159011in}}%
\pgfpathlineto{\pgfqpoint{4.600206in}{1.159137in}}%
\pgfpathlineto{\pgfqpoint{4.600399in}{1.156182in}}%
\pgfpathlineto{\pgfqpoint{4.601359in}{1.156491in}}%
\pgfpathlineto{\pgfqpoint{4.601552in}{1.156709in}}%
\pgfpathlineto{\pgfqpoint{4.601744in}{1.156326in}}%
\pgfpathlineto{\pgfqpoint{4.601936in}{1.155111in}}%
\pgfpathlineto{\pgfqpoint{4.602320in}{1.158445in}}%
\pgfpathlineto{\pgfqpoint{4.602705in}{1.156956in}}%
\pgfpathlineto{\pgfqpoint{4.603858in}{1.160913in}}%
\pgfpathlineto{\pgfqpoint{4.603281in}{1.155770in}}%
\pgfpathlineto{\pgfqpoint{4.604050in}{1.160467in}}%
\pgfpathlineto{\pgfqpoint{4.605587in}{1.154309in}}%
\pgfpathlineto{\pgfqpoint{4.605972in}{1.154648in}}%
\pgfpathlineto{\pgfqpoint{4.606740in}{1.157866in}}%
\pgfpathlineto{\pgfqpoint{4.606932in}{1.155613in}}%
\pgfpathlineto{\pgfqpoint{4.607125in}{1.151270in}}%
\pgfpathlineto{\pgfqpoint{4.607893in}{1.153305in}}%
\pgfpathlineto{\pgfqpoint{4.608085in}{1.154755in}}%
\pgfpathlineto{\pgfqpoint{4.608854in}{1.152192in}}%
\pgfpathlineto{\pgfqpoint{4.609046in}{1.153730in}}%
\pgfpathlineto{\pgfqpoint{4.609431in}{1.155385in}}%
\pgfpathlineto{\pgfqpoint{4.610007in}{1.151084in}}%
\pgfpathlineto{\pgfqpoint{4.610391in}{1.155278in}}%
\pgfpathlineto{\pgfqpoint{4.611160in}{1.152042in}}%
\pgfpathlineto{\pgfqpoint{4.611929in}{1.153197in}}%
\pgfpathlineto{\pgfqpoint{4.612313in}{1.156298in}}%
\pgfpathlineto{\pgfqpoint{4.612697in}{1.152924in}}%
\pgfpathlineto{\pgfqpoint{4.614043in}{1.144159in}}%
\pgfpathlineto{\pgfqpoint{4.614235in}{1.144843in}}%
\pgfpathlineto{\pgfqpoint{4.614619in}{1.145790in}}%
\pgfpathlineto{\pgfqpoint{4.614811in}{1.147356in}}%
\pgfpathlineto{\pgfqpoint{4.615196in}{1.142043in}}%
\pgfpathlineto{\pgfqpoint{4.616349in}{1.139464in}}%
\pgfpathlineto{\pgfqpoint{4.618078in}{1.145352in}}%
\pgfpathlineto{\pgfqpoint{4.618270in}{1.144371in}}%
\pgfpathlineto{\pgfqpoint{4.618847in}{1.141476in}}%
\pgfpathlineto{\pgfqpoint{4.619039in}{1.144163in}}%
\pgfpathlineto{\pgfqpoint{4.619615in}{1.143361in}}%
\pgfpathlineto{\pgfqpoint{4.620384in}{1.147326in}}%
\pgfpathlineto{\pgfqpoint{4.620576in}{1.148934in}}%
\pgfpathlineto{\pgfqpoint{4.621537in}{1.148475in}}%
\pgfpathlineto{\pgfqpoint{4.622690in}{1.145052in}}%
\pgfpathlineto{\pgfqpoint{4.622114in}{1.149672in}}%
\pgfpathlineto{\pgfqpoint{4.622882in}{1.146240in}}%
\pgfpathlineto{\pgfqpoint{4.623074in}{1.147565in}}%
\pgfpathlineto{\pgfqpoint{4.623459in}{1.144420in}}%
\pgfpathlineto{\pgfqpoint{4.623843in}{1.145564in}}%
\pgfpathlineto{\pgfqpoint{4.624612in}{1.147227in}}%
\pgfpathlineto{\pgfqpoint{4.626149in}{1.138067in}}%
\pgfpathlineto{\pgfqpoint{4.626341in}{1.138839in}}%
\pgfpathlineto{\pgfqpoint{4.626534in}{1.137199in}}%
\pgfpathlineto{\pgfqpoint{4.626726in}{1.134721in}}%
\pgfpathlineto{\pgfqpoint{4.627494in}{1.138760in}}%
\pgfpathlineto{\pgfqpoint{4.627687in}{1.139595in}}%
\pgfpathlineto{\pgfqpoint{4.627879in}{1.138476in}}%
\pgfpathlineto{\pgfqpoint{4.629224in}{1.130303in}}%
\pgfpathlineto{\pgfqpoint{4.629416in}{1.132256in}}%
\pgfpathlineto{\pgfqpoint{4.631722in}{1.122956in}}%
\pgfpathlineto{\pgfqpoint{4.631914in}{1.122549in}}%
\pgfpathlineto{\pgfqpoint{4.632299in}{1.124141in}}%
\pgfpathlineto{\pgfqpoint{4.633067in}{1.132033in}}%
\pgfpathlineto{\pgfqpoint{4.633836in}{1.129420in}}%
\pgfpathlineto{\pgfqpoint{4.634028in}{1.125994in}}%
\pgfpathlineto{\pgfqpoint{4.634412in}{1.130567in}}%
\pgfpathlineto{\pgfqpoint{4.634989in}{1.128396in}}%
\pgfpathlineto{\pgfqpoint{4.635373in}{1.128800in}}%
\pgfpathlineto{\pgfqpoint{4.636526in}{1.124426in}}%
\pgfpathlineto{\pgfqpoint{4.636911in}{1.127058in}}%
\pgfpathlineto{\pgfqpoint{4.637487in}{1.125232in}}%
\pgfpathlineto{\pgfqpoint{4.637679in}{1.122053in}}%
\pgfpathlineto{\pgfqpoint{4.638256in}{1.127529in}}%
\pgfpathlineto{\pgfqpoint{4.638448in}{1.127469in}}%
\pgfpathlineto{\pgfqpoint{4.638832in}{1.129364in}}%
\pgfpathlineto{\pgfqpoint{4.639217in}{1.128454in}}%
\pgfpathlineto{\pgfqpoint{4.640754in}{1.122299in}}%
\pgfpathlineto{\pgfqpoint{4.641715in}{1.113822in}}%
\pgfpathlineto{\pgfqpoint{4.642483in}{1.117061in}}%
\pgfpathlineto{\pgfqpoint{4.643252in}{1.118337in}}%
\pgfpathlineto{\pgfqpoint{4.642868in}{1.115029in}}%
\pgfpathlineto{\pgfqpoint{4.643444in}{1.116677in}}%
\pgfpathlineto{\pgfqpoint{4.644021in}{1.115180in}}%
\pgfpathlineto{\pgfqpoint{4.644405in}{1.117759in}}%
\pgfpathlineto{\pgfqpoint{4.646327in}{1.104803in}}%
\pgfpathlineto{\pgfqpoint{4.646903in}{1.109810in}}%
\pgfpathlineto{\pgfqpoint{4.647288in}{1.113122in}}%
\pgfpathlineto{\pgfqpoint{4.647480in}{1.109584in}}%
\pgfpathlineto{\pgfqpoint{4.647672in}{1.109728in}}%
\pgfpathlineto{\pgfqpoint{4.647864in}{1.106586in}}%
\pgfpathlineto{\pgfqpoint{4.648633in}{1.108746in}}%
\pgfpathlineto{\pgfqpoint{4.649978in}{1.112723in}}%
\pgfpathlineto{\pgfqpoint{4.650170in}{1.112105in}}%
\pgfpathlineto{\pgfqpoint{4.650362in}{1.112902in}}%
\pgfpathlineto{\pgfqpoint{4.650939in}{1.116879in}}%
\pgfpathlineto{\pgfqpoint{4.651323in}{1.110551in}}%
\pgfpathlineto{\pgfqpoint{4.651515in}{1.113335in}}%
\pgfpathlineto{\pgfqpoint{4.652092in}{1.110487in}}%
\pgfpathlineto{\pgfqpoint{4.652476in}{1.112921in}}%
\pgfpathlineto{\pgfqpoint{4.653437in}{1.102481in}}%
\pgfpathlineto{\pgfqpoint{4.653821in}{1.103617in}}%
\pgfpathlineto{\pgfqpoint{4.654206in}{1.106839in}}%
\pgfpathlineto{\pgfqpoint{4.654782in}{1.104478in}}%
\pgfpathlineto{\pgfqpoint{4.654974in}{1.103585in}}%
\pgfpathlineto{\pgfqpoint{4.655167in}{1.105262in}}%
\pgfpathlineto{\pgfqpoint{4.655743in}{1.105215in}}%
\pgfpathlineto{\pgfqpoint{4.655935in}{1.106376in}}%
\pgfpathlineto{\pgfqpoint{4.656127in}{1.103997in}}%
\pgfpathlineto{\pgfqpoint{4.656512in}{1.104267in}}%
\pgfpathlineto{\pgfqpoint{4.657857in}{1.096952in}}%
\pgfpathlineto{\pgfqpoint{4.659394in}{1.106587in}}%
\pgfpathlineto{\pgfqpoint{4.660163in}{1.107080in}}%
\pgfpathlineto{\pgfqpoint{4.660355in}{1.103730in}}%
\pgfpathlineto{\pgfqpoint{4.660547in}{1.107355in}}%
\pgfpathlineto{\pgfqpoint{4.661316in}{1.103919in}}%
\pgfpathlineto{\pgfqpoint{4.662277in}{1.101908in}}%
\pgfpathlineto{\pgfqpoint{4.662469in}{1.102629in}}%
\pgfpathlineto{\pgfqpoint{4.663238in}{1.110389in}}%
\pgfpathlineto{\pgfqpoint{4.663622in}{1.107602in}}%
\pgfpathlineto{\pgfqpoint{4.664391in}{1.101707in}}%
\pgfpathlineto{\pgfqpoint{4.664967in}{1.103721in}}%
\pgfpathlineto{\pgfqpoint{4.665159in}{1.105915in}}%
\pgfpathlineto{\pgfqpoint{4.665736in}{1.101804in}}%
\pgfpathlineto{\pgfqpoint{4.665928in}{1.101573in}}%
\pgfpathlineto{\pgfqpoint{4.666889in}{1.108139in}}%
\pgfpathlineto{\pgfqpoint{4.667273in}{1.107500in}}%
\pgfpathlineto{\pgfqpoint{4.668234in}{1.106645in}}%
\pgfpathlineto{\pgfqpoint{4.668426in}{1.104687in}}%
\pgfpathlineto{\pgfqpoint{4.669195in}{1.106778in}}%
\pgfpathlineto{\pgfqpoint{4.669771in}{1.112973in}}%
\pgfpathlineto{\pgfqpoint{4.670348in}{1.108746in}}%
\pgfpathlineto{\pgfqpoint{4.671501in}{1.115806in}}%
\pgfpathlineto{\pgfqpoint{4.672269in}{1.100691in}}%
\pgfpathlineto{\pgfqpoint{4.673230in}{1.104435in}}%
\pgfpathlineto{\pgfqpoint{4.673807in}{1.104658in}}%
\pgfpathlineto{\pgfqpoint{4.675536in}{1.097309in}}%
\pgfpathlineto{\pgfqpoint{4.675729in}{1.100281in}}%
\pgfpathlineto{\pgfqpoint{4.676497in}{1.096177in}}%
\pgfpathlineto{\pgfqpoint{4.676882in}{1.090380in}}%
\pgfpathlineto{\pgfqpoint{4.677650in}{1.091806in}}%
\pgfpathlineto{\pgfqpoint{4.678035in}{1.098391in}}%
\pgfpathlineto{\pgfqpoint{4.678803in}{1.094471in}}%
\pgfpathlineto{\pgfqpoint{4.679764in}{1.091367in}}%
\pgfpathlineto{\pgfqpoint{4.679956in}{1.092516in}}%
\pgfpathlineto{\pgfqpoint{4.680533in}{1.093248in}}%
\pgfpathlineto{\pgfqpoint{4.680725in}{1.091314in}}%
\pgfpathlineto{\pgfqpoint{4.680917in}{1.095116in}}%
\pgfpathlineto{\pgfqpoint{4.681878in}{1.092225in}}%
\pgfpathlineto{\pgfqpoint{4.683031in}{1.086521in}}%
\pgfpathlineto{\pgfqpoint{4.683415in}{1.087002in}}%
\pgfpathlineto{\pgfqpoint{4.683800in}{1.087574in}}%
\pgfpathlineto{\pgfqpoint{4.683992in}{1.089776in}}%
\pgfpathlineto{\pgfqpoint{4.684376in}{1.085887in}}%
\pgfpathlineto{\pgfqpoint{4.684760in}{1.088199in}}%
\pgfpathlineto{\pgfqpoint{4.684953in}{1.083343in}}%
\pgfpathlineto{\pgfqpoint{4.685721in}{1.089948in}}%
\pgfpathlineto{\pgfqpoint{4.685913in}{1.088215in}}%
\pgfpathlineto{\pgfqpoint{4.686106in}{1.092092in}}%
\pgfpathlineto{\pgfqpoint{4.686490in}{1.091067in}}%
\pgfpathlineto{\pgfqpoint{4.687259in}{1.093093in}}%
\pgfpathlineto{\pgfqpoint{4.687066in}{1.090423in}}%
\pgfpathlineto{\pgfqpoint{4.687643in}{1.092849in}}%
\pgfpathlineto{\pgfqpoint{4.687835in}{1.090796in}}%
\pgfpathlineto{\pgfqpoint{4.688027in}{1.092956in}}%
\pgfpathlineto{\pgfqpoint{4.688604in}{1.092411in}}%
\pgfpathlineto{\pgfqpoint{4.689565in}{1.091853in}}%
\pgfpathlineto{\pgfqpoint{4.689757in}{1.094113in}}%
\pgfpathlineto{\pgfqpoint{4.690718in}{1.092315in}}%
\pgfpathlineto{\pgfqpoint{4.690910in}{1.092936in}}%
\pgfpathlineto{\pgfqpoint{4.691871in}{1.090695in}}%
\pgfpathlineto{\pgfqpoint{4.691294in}{1.093876in}}%
\pgfpathlineto{\pgfqpoint{4.692063in}{1.092353in}}%
\pgfpathlineto{\pgfqpoint{4.692831in}{1.094225in}}%
\pgfpathlineto{\pgfqpoint{4.693216in}{1.090501in}}%
\pgfpathlineto{\pgfqpoint{4.693792in}{1.094895in}}%
\pgfpathlineto{\pgfqpoint{4.693984in}{1.095450in}}%
\pgfpathlineto{\pgfqpoint{4.694177in}{1.093912in}}%
\pgfpathlineto{\pgfqpoint{4.694369in}{1.092974in}}%
\pgfpathlineto{\pgfqpoint{4.694753in}{1.095852in}}%
\pgfpathlineto{\pgfqpoint{4.695137in}{1.096832in}}%
\pgfpathlineto{\pgfqpoint{4.696675in}{1.106821in}}%
\pgfpathlineto{\pgfqpoint{4.697059in}{1.105633in}}%
\pgfpathlineto{\pgfqpoint{4.698596in}{1.099407in}}%
\pgfpathlineto{\pgfqpoint{4.699365in}{1.102381in}}%
\pgfpathlineto{\pgfqpoint{4.699557in}{1.101928in}}%
\pgfpathlineto{\pgfqpoint{4.699942in}{1.101305in}}%
\pgfpathlineto{\pgfqpoint{4.700903in}{1.104889in}}%
\pgfpathlineto{\pgfqpoint{4.701287in}{1.102105in}}%
\pgfpathlineto{\pgfqpoint{4.702056in}{1.096678in}}%
\pgfpathlineto{\pgfqpoint{4.702440in}{1.098635in}}%
\pgfpathlineto{\pgfqpoint{4.702824in}{1.097620in}}%
\pgfpathlineto{\pgfqpoint{4.703593in}{1.091179in}}%
\pgfpathlineto{\pgfqpoint{4.704169in}{1.091944in}}%
\pgfpathlineto{\pgfqpoint{4.704746in}{1.089019in}}%
\pgfpathlineto{\pgfqpoint{4.704938in}{1.086007in}}%
\pgfpathlineto{\pgfqpoint{4.705707in}{1.089495in}}%
\pgfpathlineto{\pgfqpoint{4.705899in}{1.090130in}}%
\pgfpathlineto{\pgfqpoint{4.706283in}{1.088824in}}%
\pgfpathlineto{\pgfqpoint{4.709550in}{1.075509in}}%
\pgfpathlineto{\pgfqpoint{4.709742in}{1.077313in}}%
\pgfpathlineto{\pgfqpoint{4.710127in}{1.071446in}}%
\pgfpathlineto{\pgfqpoint{4.710511in}{1.072271in}}%
\pgfpathlineto{\pgfqpoint{4.710895in}{1.067806in}}%
\pgfpathlineto{\pgfqpoint{4.711087in}{1.066318in}}%
\pgfpathlineto{\pgfqpoint{4.711280in}{1.068671in}}%
\pgfpathlineto{\pgfqpoint{4.711856in}{1.076486in}}%
\pgfpathlineto{\pgfqpoint{4.712625in}{1.075078in}}%
\pgfpathlineto{\pgfqpoint{4.713009in}{1.075365in}}%
\pgfpathlineto{\pgfqpoint{4.714546in}{1.088098in}}%
\pgfpathlineto{\pgfqpoint{4.714931in}{1.087834in}}%
\pgfpathlineto{\pgfqpoint{4.715892in}{1.085347in}}%
\pgfpathlineto{\pgfqpoint{4.716084in}{1.086925in}}%
\pgfpathlineto{\pgfqpoint{4.716468in}{1.090325in}}%
\pgfpathlineto{\pgfqpoint{4.717237in}{1.088306in}}%
\pgfpathlineto{\pgfqpoint{4.718966in}{1.083239in}}%
\pgfpathlineto{\pgfqpoint{4.720311in}{1.089671in}}%
\pgfpathlineto{\pgfqpoint{4.721849in}{1.079934in}}%
\pgfpathlineto{\pgfqpoint{4.723002in}{1.085376in}}%
\pgfpathlineto{\pgfqpoint{4.722233in}{1.079552in}}%
\pgfpathlineto{\pgfqpoint{4.723194in}{1.084305in}}%
\pgfpathlineto{\pgfqpoint{4.724731in}{1.077226in}}%
\pgfpathlineto{\pgfqpoint{4.723963in}{1.084626in}}%
\pgfpathlineto{\pgfqpoint{4.724924in}{1.078184in}}%
\pgfpathlineto{\pgfqpoint{4.725500in}{1.077250in}}%
\pgfpathlineto{\pgfqpoint{4.726077in}{1.079447in}}%
\pgfpathlineto{\pgfqpoint{4.726461in}{1.076180in}}%
\pgfpathlineto{\pgfqpoint{4.727230in}{1.078524in}}%
\pgfpathlineto{\pgfqpoint{4.727422in}{1.078446in}}%
\pgfpathlineto{\pgfqpoint{4.727614in}{1.079020in}}%
\pgfpathlineto{\pgfqpoint{4.729151in}{1.071191in}}%
\pgfpathlineto{\pgfqpoint{4.729343in}{1.073188in}}%
\pgfpathlineto{\pgfqpoint{4.729920in}{1.068353in}}%
\pgfpathlineto{\pgfqpoint{4.730112in}{1.070004in}}%
\pgfpathlineto{\pgfqpoint{4.730496in}{1.068835in}}%
\pgfpathlineto{\pgfqpoint{4.730689in}{1.070550in}}%
\pgfpathlineto{\pgfqpoint{4.730881in}{1.071196in}}%
\pgfpathlineto{\pgfqpoint{4.731073in}{1.069375in}}%
\pgfpathlineto{\pgfqpoint{4.731457in}{1.069423in}}%
\pgfpathlineto{\pgfqpoint{4.731842in}{1.066953in}}%
\pgfpathlineto{\pgfqpoint{4.732034in}{1.062943in}}%
\pgfpathlineto{\pgfqpoint{4.732802in}{1.067118in}}%
\pgfpathlineto{\pgfqpoint{4.733379in}{1.066162in}}%
\pgfpathlineto{\pgfqpoint{4.733571in}{1.066728in}}%
\pgfpathlineto{\pgfqpoint{4.734340in}{1.069061in}}%
\pgfpathlineto{\pgfqpoint{4.734532in}{1.068842in}}%
\pgfpathlineto{\pgfqpoint{4.736838in}{1.050478in}}%
\pgfpathlineto{\pgfqpoint{4.737607in}{1.051191in}}%
\pgfpathlineto{\pgfqpoint{4.737799in}{1.047686in}}%
\pgfpathlineto{\pgfqpoint{4.737991in}{1.041962in}}%
\pgfpathlineto{\pgfqpoint{4.738952in}{1.045201in}}%
\pgfpathlineto{\pgfqpoint{4.740105in}{1.053287in}}%
\pgfpathlineto{\pgfqpoint{4.740489in}{1.052678in}}%
\pgfpathlineto{\pgfqpoint{4.741258in}{1.043203in}}%
\pgfpathlineto{\pgfqpoint{4.741834in}{1.046294in}}%
\pgfpathlineto{\pgfqpoint{4.742795in}{1.049769in}}%
\pgfpathlineto{\pgfqpoint{4.742987in}{1.049661in}}%
\pgfpathlineto{\pgfqpoint{4.743179in}{1.049095in}}%
\pgfpathlineto{\pgfqpoint{4.743372in}{1.051104in}}%
\pgfpathlineto{\pgfqpoint{4.743948in}{1.057219in}}%
\pgfpathlineto{\pgfqpoint{4.744525in}{1.052810in}}%
\pgfpathlineto{\pgfqpoint{4.744717in}{1.050485in}}%
\pgfpathlineto{\pgfqpoint{4.745293in}{1.052186in}}%
\pgfpathlineto{\pgfqpoint{4.746638in}{1.056831in}}%
\pgfpathlineto{\pgfqpoint{4.747599in}{1.063061in}}%
\pgfpathlineto{\pgfqpoint{4.747984in}{1.062776in}}%
\pgfpathlineto{\pgfqpoint{4.749521in}{1.056102in}}%
\pgfpathlineto{\pgfqpoint{4.750482in}{1.060559in}}%
\pgfpathlineto{\pgfqpoint{4.750674in}{1.059062in}}%
\pgfpathlineto{\pgfqpoint{4.751058in}{1.060039in}}%
\pgfpathlineto{\pgfqpoint{4.751251in}{1.057926in}}%
\pgfpathlineto{\pgfqpoint{4.751635in}{1.058966in}}%
\pgfpathlineto{\pgfqpoint{4.751827in}{1.057459in}}%
\pgfpathlineto{\pgfqpoint{4.752404in}{1.061162in}}%
\pgfpathlineto{\pgfqpoint{4.752788in}{1.065445in}}%
\pgfpathlineto{\pgfqpoint{4.753557in}{1.061502in}}%
\pgfpathlineto{\pgfqpoint{4.753941in}{1.063015in}}%
\pgfpathlineto{\pgfqpoint{4.757016in}{1.052026in}}%
\pgfpathlineto{\pgfqpoint{4.757400in}{1.052326in}}%
\pgfpathlineto{\pgfqpoint{4.757592in}{1.053621in}}%
\pgfpathlineto{\pgfqpoint{4.758169in}{1.051508in}}%
\pgfpathlineto{\pgfqpoint{4.759129in}{1.042978in}}%
\pgfpathlineto{\pgfqpoint{4.759514in}{1.045973in}}%
\pgfpathlineto{\pgfqpoint{4.759898in}{1.052372in}}%
\pgfpathlineto{\pgfqpoint{4.760667in}{1.049033in}}%
\pgfpathlineto{\pgfqpoint{4.761051in}{1.046029in}}%
\pgfpathlineto{\pgfqpoint{4.761243in}{1.049844in}}%
\pgfpathlineto{\pgfqpoint{4.761435in}{1.049369in}}%
\pgfpathlineto{\pgfqpoint{4.762012in}{1.055285in}}%
\pgfpathlineto{\pgfqpoint{4.762973in}{1.054158in}}%
\pgfpathlineto{\pgfqpoint{4.763741in}{1.048691in}}%
\pgfpathlineto{\pgfqpoint{4.764126in}{1.050675in}}%
\pgfpathlineto{\pgfqpoint{4.764510in}{1.052847in}}%
\pgfpathlineto{\pgfqpoint{4.765279in}{1.052285in}}%
\pgfpathlineto{\pgfqpoint{4.765471in}{1.052173in}}%
\pgfpathlineto{\pgfqpoint{4.765663in}{1.052747in}}%
\pgfpathlineto{\pgfqpoint{4.766047in}{1.056757in}}%
\pgfpathlineto{\pgfqpoint{4.766624in}{1.054792in}}%
\pgfpathlineto{\pgfqpoint{4.766816in}{1.052228in}}%
\pgfpathlineto{\pgfqpoint{4.767777in}{1.054003in}}%
\pgfpathlineto{\pgfqpoint{4.768353in}{1.054631in}}%
\pgfpathlineto{\pgfqpoint{4.769891in}{1.064284in}}%
\pgfpathlineto{\pgfqpoint{4.770083in}{1.063930in}}%
\pgfpathlineto{\pgfqpoint{4.770275in}{1.065711in}}%
\pgfpathlineto{\pgfqpoint{4.770467in}{1.069506in}}%
\pgfpathlineto{\pgfqpoint{4.771236in}{1.069045in}}%
\pgfpathlineto{\pgfqpoint{4.772773in}{1.059938in}}%
\pgfpathlineto{\pgfqpoint{4.774311in}{1.054823in}}%
\pgfpathlineto{\pgfqpoint{4.777193in}{1.077559in}}%
\pgfpathlineto{\pgfqpoint{4.777385in}{1.075666in}}%
\pgfpathlineto{\pgfqpoint{4.777578in}{1.075333in}}%
\pgfpathlineto{\pgfqpoint{4.780076in}{1.091866in}}%
\pgfpathlineto{\pgfqpoint{4.780268in}{1.091322in}}%
\pgfpathlineto{\pgfqpoint{4.781613in}{1.084479in}}%
\pgfpathlineto{\pgfqpoint{4.781997in}{1.085898in}}%
\pgfpathlineto{\pgfqpoint{4.782766in}{1.092334in}}%
\pgfpathlineto{\pgfqpoint{4.783150in}{1.089577in}}%
\pgfpathlineto{\pgfqpoint{4.785841in}{1.073918in}}%
\pgfpathlineto{\pgfqpoint{4.786417in}{1.074415in}}%
\pgfpathlineto{\pgfqpoint{4.788147in}{1.082364in}}%
\pgfpathlineto{\pgfqpoint{4.788339in}{1.082334in}}%
\pgfpathlineto{\pgfqpoint{4.789492in}{1.075438in}}%
\pgfpathlineto{\pgfqpoint{4.790453in}{1.083316in}}%
\pgfpathlineto{\pgfqpoint{4.790645in}{1.077727in}}%
\pgfpathlineto{\pgfqpoint{4.791990in}{1.072593in}}%
\pgfpathlineto{\pgfqpoint{4.792182in}{1.072308in}}%
\pgfpathlineto{\pgfqpoint{4.792374in}{1.074441in}}%
\pgfpathlineto{\pgfqpoint{4.792759in}{1.070901in}}%
\pgfpathlineto{\pgfqpoint{4.793143in}{1.071304in}}%
\pgfpathlineto{\pgfqpoint{4.793720in}{1.071706in}}%
\pgfpathlineto{\pgfqpoint{4.794296in}{1.068312in}}%
\pgfpathlineto{\pgfqpoint{4.794488in}{1.069099in}}%
\pgfpathlineto{\pgfqpoint{4.794680in}{1.067366in}}%
\pgfpathlineto{\pgfqpoint{4.794873in}{1.065225in}}%
\pgfpathlineto{\pgfqpoint{4.795257in}{1.069289in}}%
\pgfpathlineto{\pgfqpoint{4.795449in}{1.069064in}}%
\pgfpathlineto{\pgfqpoint{4.797371in}{1.081979in}}%
\pgfpathlineto{\pgfqpoint{4.797947in}{1.079723in}}%
\pgfpathlineto{\pgfqpoint{4.798140in}{1.078683in}}%
\pgfpathlineto{\pgfqpoint{4.798524in}{1.081475in}}%
\pgfpathlineto{\pgfqpoint{4.800253in}{1.087968in}}%
\pgfpathlineto{\pgfqpoint{4.800446in}{1.083423in}}%
\pgfpathlineto{\pgfqpoint{4.800830in}{1.088463in}}%
\pgfpathlineto{\pgfqpoint{4.801406in}{1.086141in}}%
\pgfpathlineto{\pgfqpoint{4.802175in}{1.093985in}}%
\pgfpathlineto{\pgfqpoint{4.802944in}{1.092748in}}%
\pgfpathlineto{\pgfqpoint{4.804865in}{1.083531in}}%
\pgfpathlineto{\pgfqpoint{4.805442in}{1.085064in}}%
\pgfpathlineto{\pgfqpoint{4.806211in}{1.079199in}}%
\pgfpathlineto{\pgfqpoint{4.806595in}{1.082722in}}%
\pgfpathlineto{\pgfqpoint{4.806787in}{1.082542in}}%
\pgfpathlineto{\pgfqpoint{4.808324in}{1.087455in}}%
\pgfpathlineto{\pgfqpoint{4.808517in}{1.086501in}}%
\pgfpathlineto{\pgfqpoint{4.808709in}{1.089799in}}%
\pgfpathlineto{\pgfqpoint{4.809093in}{1.088752in}}%
\pgfpathlineto{\pgfqpoint{4.809285in}{1.090027in}}%
\pgfpathlineto{\pgfqpoint{4.809862in}{1.086835in}}%
\pgfpathlineto{\pgfqpoint{4.811976in}{1.091949in}}%
\pgfpathlineto{\pgfqpoint{4.810630in}{1.086465in}}%
\pgfpathlineto{\pgfqpoint{4.812360in}{1.090769in}}%
\pgfpathlineto{\pgfqpoint{4.813129in}{1.089475in}}%
\pgfpathlineto{\pgfqpoint{4.812744in}{1.091210in}}%
\pgfpathlineto{\pgfqpoint{4.813321in}{1.090476in}}%
\pgfpathlineto{\pgfqpoint{4.815435in}{1.098139in}}%
\pgfpathlineto{\pgfqpoint{4.816011in}{1.096314in}}%
\pgfpathlineto{\pgfqpoint{4.815819in}{1.098932in}}%
\pgfpathlineto{\pgfqpoint{4.816203in}{1.098185in}}%
\pgfpathlineto{\pgfqpoint{4.817741in}{1.115497in}}%
\pgfpathlineto{\pgfqpoint{4.818125in}{1.112604in}}%
\pgfpathlineto{\pgfqpoint{4.818509in}{1.108536in}}%
\pgfpathlineto{\pgfqpoint{4.818894in}{1.113461in}}%
\pgfpathlineto{\pgfqpoint{4.819086in}{1.116866in}}%
\pgfpathlineto{\pgfqpoint{4.819662in}{1.112282in}}%
\pgfpathlineto{\pgfqpoint{4.820047in}{1.115186in}}%
\pgfpathlineto{\pgfqpoint{4.820815in}{1.110712in}}%
\pgfpathlineto{\pgfqpoint{4.821392in}{1.111964in}}%
\pgfpathlineto{\pgfqpoint{4.822353in}{1.113343in}}%
\pgfpathlineto{\pgfqpoint{4.821776in}{1.111139in}}%
\pgfpathlineto{\pgfqpoint{4.822545in}{1.112810in}}%
\pgfpathlineto{\pgfqpoint{4.823506in}{1.107102in}}%
\pgfpathlineto{\pgfqpoint{4.824467in}{1.109246in}}%
\pgfpathlineto{\pgfqpoint{4.825043in}{1.111322in}}%
\pgfpathlineto{\pgfqpoint{4.825427in}{1.110309in}}%
\pgfpathlineto{\pgfqpoint{4.826388in}{1.105329in}}%
\pgfpathlineto{\pgfqpoint{4.826580in}{1.106662in}}%
\pgfpathlineto{\pgfqpoint{4.826773in}{1.110430in}}%
\pgfpathlineto{\pgfqpoint{4.827541in}{1.105592in}}%
\pgfpathlineto{\pgfqpoint{4.827926in}{1.101092in}}%
\pgfpathlineto{\pgfqpoint{4.828502in}{1.106883in}}%
\pgfpathlineto{\pgfqpoint{4.828694in}{1.104423in}}%
\pgfpathlineto{\pgfqpoint{4.829463in}{1.100171in}}%
\pgfpathlineto{\pgfqpoint{4.829655in}{1.105124in}}%
\pgfpathlineto{\pgfqpoint{4.829847in}{1.103682in}}%
\pgfpathlineto{\pgfqpoint{4.830039in}{1.103607in}}%
\pgfpathlineto{\pgfqpoint{4.831000in}{1.109142in}}%
\pgfpathlineto{\pgfqpoint{4.831385in}{1.108122in}}%
\pgfpathlineto{\pgfqpoint{4.831961in}{1.110195in}}%
\pgfpathlineto{\pgfqpoint{4.833114in}{1.115087in}}%
\pgfpathlineto{\pgfqpoint{4.834075in}{1.107776in}}%
\pgfpathlineto{\pgfqpoint{4.834267in}{1.111066in}}%
\pgfpathlineto{\pgfqpoint{4.835420in}{1.118360in}}%
\pgfpathlineto{\pgfqpoint{4.835804in}{1.117933in}}%
\pgfpathlineto{\pgfqpoint{4.837150in}{1.105917in}}%
\pgfpathlineto{\pgfqpoint{4.837342in}{1.109392in}}%
\pgfpathlineto{\pgfqpoint{4.838879in}{1.119353in}}%
\pgfpathlineto{\pgfqpoint{4.839071in}{1.118667in}}%
\pgfpathlineto{\pgfqpoint{4.839456in}{1.118937in}}%
\pgfpathlineto{\pgfqpoint{4.839648in}{1.118527in}}%
\pgfpathlineto{\pgfqpoint{4.840224in}{1.123584in}}%
\pgfpathlineto{\pgfqpoint{4.840801in}{1.121449in}}%
\pgfpathlineto{\pgfqpoint{4.841377in}{1.121305in}}%
\pgfpathlineto{\pgfqpoint{4.841569in}{1.121745in}}%
\pgfpathlineto{\pgfqpoint{4.841762in}{1.123160in}}%
\pgfpathlineto{\pgfqpoint{4.842146in}{1.119859in}}%
\pgfpathlineto{\pgfqpoint{4.843491in}{1.113920in}}%
\pgfpathlineto{\pgfqpoint{4.845797in}{1.123524in}}%
\pgfpathlineto{\pgfqpoint{4.845989in}{1.122659in}}%
\pgfpathlineto{\pgfqpoint{4.846758in}{1.118418in}}%
\pgfpathlineto{\pgfqpoint{4.847142in}{1.121309in}}%
\pgfpathlineto{\pgfqpoint{4.847335in}{1.121806in}}%
\pgfpathlineto{\pgfqpoint{4.847527in}{1.119461in}}%
\pgfpathlineto{\pgfqpoint{4.847719in}{1.115710in}}%
\pgfpathlineto{\pgfqpoint{4.848295in}{1.125365in}}%
\pgfpathlineto{\pgfqpoint{4.850025in}{1.119894in}}%
\pgfpathlineto{\pgfqpoint{4.852715in}{1.141010in}}%
\pgfpathlineto{\pgfqpoint{4.852907in}{1.141117in}}%
\pgfpathlineto{\pgfqpoint{4.853100in}{1.144617in}}%
\pgfpathlineto{\pgfqpoint{4.853868in}{1.142760in}}%
\pgfpathlineto{\pgfqpoint{4.854829in}{1.136981in}}%
\pgfpathlineto{\pgfqpoint{4.855021in}{1.138401in}}%
\pgfpathlineto{\pgfqpoint{4.855213in}{1.140051in}}%
\pgfpathlineto{\pgfqpoint{4.855790in}{1.136981in}}%
\pgfpathlineto{\pgfqpoint{4.855982in}{1.137169in}}%
\pgfpathlineto{\pgfqpoint{4.860402in}{1.153629in}}%
\pgfpathlineto{\pgfqpoint{4.860594in}{1.151939in}}%
\pgfpathlineto{\pgfqpoint{4.860978in}{1.155958in}}%
\pgfpathlineto{\pgfqpoint{4.861363in}{1.154141in}}%
\pgfpathlineto{\pgfqpoint{4.862708in}{1.168218in}}%
\pgfpathlineto{\pgfqpoint{4.863284in}{1.166611in}}%
\pgfpathlineto{\pgfqpoint{4.863477in}{1.166255in}}%
\pgfpathlineto{\pgfqpoint{4.864822in}{1.160236in}}%
\pgfpathlineto{\pgfqpoint{4.865398in}{1.155002in}}%
\pgfpathlineto{\pgfqpoint{4.866167in}{1.157783in}}%
\pgfpathlineto{\pgfqpoint{4.866359in}{1.157879in}}%
\pgfpathlineto{\pgfqpoint{4.868857in}{1.147448in}}%
\pgfpathlineto{\pgfqpoint{4.869049in}{1.147969in}}%
\pgfpathlineto{\pgfqpoint{4.869242in}{1.148077in}}%
\pgfpathlineto{\pgfqpoint{4.869434in}{1.146952in}}%
\pgfpathlineto{\pgfqpoint{4.870010in}{1.145726in}}%
\pgfpathlineto{\pgfqpoint{4.871356in}{1.139473in}}%
\pgfpathlineto{\pgfqpoint{4.872509in}{1.144258in}}%
\pgfpathlineto{\pgfqpoint{4.872701in}{1.143561in}}%
\pgfpathlineto{\pgfqpoint{4.873662in}{1.139028in}}%
\pgfpathlineto{\pgfqpoint{4.874046in}{1.140324in}}%
\pgfpathlineto{\pgfqpoint{4.875968in}{1.151935in}}%
\pgfpathlineto{\pgfqpoint{4.876160in}{1.149013in}}%
\pgfpathlineto{\pgfqpoint{4.876352in}{1.145896in}}%
\pgfpathlineto{\pgfqpoint{4.877121in}{1.146982in}}%
\pgfpathlineto{\pgfqpoint{4.878466in}{1.153874in}}%
\pgfpathlineto{\pgfqpoint{4.878850in}{1.151693in}}%
\pgfpathlineto{\pgfqpoint{4.879234in}{1.155221in}}%
\pgfpathlineto{\pgfqpoint{4.879619in}{1.155097in}}%
\pgfpathlineto{\pgfqpoint{4.880387in}{1.161017in}}%
\pgfpathlineto{\pgfqpoint{4.880580in}{1.158423in}}%
\pgfpathlineto{\pgfqpoint{4.882117in}{1.146356in}}%
\pgfpathlineto{\pgfqpoint{4.882886in}{1.149588in}}%
\pgfpathlineto{\pgfqpoint{4.882501in}{1.146008in}}%
\pgfpathlineto{\pgfqpoint{4.883078in}{1.148019in}}%
\pgfpathlineto{\pgfqpoint{4.884039in}{1.144275in}}%
\pgfpathlineto{\pgfqpoint{4.884231in}{1.145560in}}%
\pgfpathlineto{\pgfqpoint{4.884423in}{1.146753in}}%
\pgfpathlineto{\pgfqpoint{4.884807in}{1.143215in}}%
\pgfpathlineto{\pgfqpoint{4.884999in}{1.143191in}}%
\pgfpathlineto{\pgfqpoint{4.886729in}{1.128855in}}%
\pgfpathlineto{\pgfqpoint{4.886921in}{1.131667in}}%
\pgfpathlineto{\pgfqpoint{4.887113in}{1.131968in}}%
\pgfpathlineto{\pgfqpoint{4.888074in}{1.136886in}}%
\pgfpathlineto{\pgfqpoint{4.888266in}{1.134794in}}%
\pgfpathlineto{\pgfqpoint{4.888651in}{1.134290in}}%
\pgfpathlineto{\pgfqpoint{4.888843in}{1.135738in}}%
\pgfpathlineto{\pgfqpoint{4.889035in}{1.137504in}}%
\pgfpathlineto{\pgfqpoint{4.889611in}{1.132594in}}%
\pgfpathlineto{\pgfqpoint{4.889804in}{1.134299in}}%
\pgfpathlineto{\pgfqpoint{4.890188in}{1.133621in}}%
\pgfpathlineto{\pgfqpoint{4.892110in}{1.125746in}}%
\pgfpathlineto{\pgfqpoint{4.892494in}{1.125035in}}%
\pgfpathlineto{\pgfqpoint{4.895953in}{1.132973in}}%
\pgfpathlineto{\pgfqpoint{4.896914in}{1.136927in}}%
\pgfpathlineto{\pgfqpoint{4.897106in}{1.136099in}}%
\pgfpathlineto{\pgfqpoint{4.897490in}{1.133767in}}%
\pgfpathlineto{\pgfqpoint{4.897875in}{1.134451in}}%
\pgfpathlineto{\pgfqpoint{4.898067in}{1.137686in}}%
\pgfpathlineto{\pgfqpoint{4.898836in}{1.136035in}}%
\pgfpathlineto{\pgfqpoint{4.899412in}{1.129817in}}%
\pgfpathlineto{\pgfqpoint{4.899989in}{1.130602in}}%
\pgfpathlineto{\pgfqpoint{4.900181in}{1.134121in}}%
\pgfpathlineto{\pgfqpoint{4.900757in}{1.125833in}}%
\pgfpathlineto{\pgfqpoint{4.900949in}{1.128041in}}%
\pgfpathlineto{\pgfqpoint{4.901142in}{1.128699in}}%
\pgfpathlineto{\pgfqpoint{4.901526in}{1.126534in}}%
\pgfpathlineto{\pgfqpoint{4.901910in}{1.125513in}}%
\pgfpathlineto{\pgfqpoint{4.902102in}{1.125970in}}%
\pgfpathlineto{\pgfqpoint{4.902679in}{1.127988in}}%
\pgfpathlineto{\pgfqpoint{4.903063in}{1.124969in}}%
\pgfpathlineto{\pgfqpoint{4.903255in}{1.124557in}}%
\pgfpathlineto{\pgfqpoint{4.903448in}{1.126074in}}%
\pgfpathlineto{\pgfqpoint{4.904216in}{1.131471in}}%
\pgfpathlineto{\pgfqpoint{4.904985in}{1.128644in}}%
\pgfpathlineto{\pgfqpoint{4.905754in}{1.124414in}}%
\pgfpathlineto{\pgfqpoint{4.906138in}{1.124851in}}%
\pgfpathlineto{\pgfqpoint{4.906522in}{1.123713in}}%
\pgfpathlineto{\pgfqpoint{4.907867in}{1.132123in}}%
\pgfpathlineto{\pgfqpoint{4.908444in}{1.136337in}}%
\pgfpathlineto{\pgfqpoint{4.908636in}{1.133105in}}%
\pgfpathlineto{\pgfqpoint{4.908828in}{1.129539in}}%
\pgfpathlineto{\pgfqpoint{4.909405in}{1.133475in}}%
\pgfpathlineto{\pgfqpoint{4.909597in}{1.133279in}}%
\pgfpathlineto{\pgfqpoint{4.910173in}{1.132887in}}%
\pgfpathlineto{\pgfqpoint{4.910750in}{1.137125in}}%
\pgfpathlineto{\pgfqpoint{4.910942in}{1.134829in}}%
\pgfpathlineto{\pgfqpoint{4.911326in}{1.139153in}}%
\pgfpathlineto{\pgfqpoint{4.912479in}{1.144427in}}%
\pgfpathlineto{\pgfqpoint{4.912672in}{1.143761in}}%
\pgfpathlineto{\pgfqpoint{4.915170in}{1.153772in}}%
\pgfpathlineto{\pgfqpoint{4.915362in}{1.152526in}}%
\pgfpathlineto{\pgfqpoint{4.915554in}{1.152538in}}%
\pgfpathlineto{\pgfqpoint{4.915746in}{1.149253in}}%
\pgfpathlineto{\pgfqpoint{4.916707in}{1.151031in}}%
\pgfpathlineto{\pgfqpoint{4.918244in}{1.156989in}}%
\pgfpathlineto{\pgfqpoint{4.917091in}{1.149835in}}%
\pgfpathlineto{\pgfqpoint{4.918437in}{1.156378in}}%
\pgfpathlineto{\pgfqpoint{4.919974in}{1.140243in}}%
\pgfpathlineto{\pgfqpoint{4.920166in}{1.141288in}}%
\pgfpathlineto{\pgfqpoint{4.920551in}{1.140588in}}%
\pgfpathlineto{\pgfqpoint{4.921896in}{1.135379in}}%
\pgfpathlineto{\pgfqpoint{4.922664in}{1.143122in}}%
\pgfpathlineto{\pgfqpoint{4.923049in}{1.138613in}}%
\pgfpathlineto{\pgfqpoint{4.923433in}{1.133857in}}%
\pgfpathlineto{\pgfqpoint{4.924010in}{1.139855in}}%
\pgfpathlineto{\pgfqpoint{4.924202in}{1.142397in}}%
\pgfpathlineto{\pgfqpoint{4.924778in}{1.138476in}}%
\pgfpathlineto{\pgfqpoint{4.927469in}{1.122529in}}%
\pgfpathlineto{\pgfqpoint{4.927853in}{1.123736in}}%
\pgfpathlineto{\pgfqpoint{4.928045in}{1.125430in}}%
\pgfpathlineto{\pgfqpoint{4.928429in}{1.122168in}}%
\pgfpathlineto{\pgfqpoint{4.930159in}{1.114071in}}%
\pgfpathlineto{\pgfqpoint{4.930351in}{1.114638in}}%
\pgfpathlineto{\pgfqpoint{4.930543in}{1.112252in}}%
\pgfpathlineto{\pgfqpoint{4.931120in}{1.109810in}}%
\pgfpathlineto{\pgfqpoint{4.931312in}{1.113398in}}%
\pgfpathlineto{\pgfqpoint{4.931696in}{1.120529in}}%
\pgfpathlineto{\pgfqpoint{4.932465in}{1.117436in}}%
\pgfpathlineto{\pgfqpoint{4.934002in}{1.107132in}}%
\pgfpathlineto{\pgfqpoint{4.934387in}{1.104324in}}%
\pgfpathlineto{\pgfqpoint{4.935155in}{1.105134in}}%
\pgfpathlineto{\pgfqpoint{4.935540in}{1.104553in}}%
\pgfpathlineto{\pgfqpoint{4.936308in}{1.108950in}}%
\pgfpathlineto{\pgfqpoint{4.936693in}{1.110042in}}%
\pgfpathlineto{\pgfqpoint{4.936885in}{1.111443in}}%
\pgfpathlineto{\pgfqpoint{4.937461in}{1.107752in}}%
\pgfpathlineto{\pgfqpoint{4.938999in}{1.097564in}}%
\pgfpathlineto{\pgfqpoint{4.940152in}{1.103359in}}%
\pgfpathlineto{\pgfqpoint{4.940536in}{1.102631in}}%
\pgfpathlineto{\pgfqpoint{4.941881in}{1.096378in}}%
\pgfpathlineto{\pgfqpoint{4.942650in}{1.096996in}}%
\pgfpathlineto{\pgfqpoint{4.943034in}{1.094968in}}%
\pgfpathlineto{\pgfqpoint{4.943418in}{1.099045in}}%
\pgfpathlineto{\pgfqpoint{4.943995in}{1.096396in}}%
\pgfpathlineto{\pgfqpoint{4.946109in}{1.086388in}}%
\pgfpathlineto{\pgfqpoint{4.946301in}{1.089856in}}%
\pgfpathlineto{\pgfqpoint{4.946493in}{1.090649in}}%
\pgfpathlineto{\pgfqpoint{4.946685in}{1.087903in}}%
\pgfpathlineto{\pgfqpoint{4.947262in}{1.090384in}}%
\pgfpathlineto{\pgfqpoint{4.949568in}{1.077070in}}%
\pgfpathlineto{\pgfqpoint{4.949760in}{1.078212in}}%
\pgfpathlineto{\pgfqpoint{4.950529in}{1.076855in}}%
\pgfpathlineto{\pgfqpoint{4.950721in}{1.077946in}}%
\pgfpathlineto{\pgfqpoint{4.952066in}{1.070478in}}%
\pgfpathlineto{\pgfqpoint{4.952835in}{1.072213in}}%
\pgfpathlineto{\pgfqpoint{4.953219in}{1.070544in}}%
\pgfpathlineto{\pgfqpoint{4.954180in}{1.074360in}}%
\pgfpathlineto{\pgfqpoint{4.954372in}{1.073219in}}%
\pgfpathlineto{\pgfqpoint{4.954756in}{1.074781in}}%
\pgfpathlineto{\pgfqpoint{4.955525in}{1.077486in}}%
\pgfpathlineto{\pgfqpoint{4.955909in}{1.076351in}}%
\pgfpathlineto{\pgfqpoint{4.956102in}{1.071696in}}%
\pgfpathlineto{\pgfqpoint{4.956294in}{1.076761in}}%
\pgfpathlineto{\pgfqpoint{4.957062in}{1.072073in}}%
\pgfpathlineto{\pgfqpoint{4.957639in}{1.077475in}}%
\pgfpathlineto{\pgfqpoint{4.958215in}{1.072766in}}%
\pgfpathlineto{\pgfqpoint{4.958408in}{1.072223in}}%
\pgfpathlineto{\pgfqpoint{4.958600in}{1.074638in}}%
\pgfpathlineto{\pgfqpoint{4.958984in}{1.075313in}}%
\pgfpathlineto{\pgfqpoint{4.959176in}{1.073415in}}%
\pgfpathlineto{\pgfqpoint{4.959945in}{1.068531in}}%
\pgfpathlineto{\pgfqpoint{4.959561in}{1.073683in}}%
\pgfpathlineto{\pgfqpoint{4.960137in}{1.071301in}}%
\pgfpathlineto{\pgfqpoint{4.960906in}{1.080140in}}%
\pgfpathlineto{\pgfqpoint{4.961482in}{1.076328in}}%
\pgfpathlineto{\pgfqpoint{4.962443in}{1.068966in}}%
\pgfpathlineto{\pgfqpoint{4.962635in}{1.072996in}}%
\pgfpathlineto{\pgfqpoint{4.963596in}{1.071520in}}%
\pgfpathlineto{\pgfqpoint{4.963212in}{1.073521in}}%
\pgfpathlineto{\pgfqpoint{4.963788in}{1.072167in}}%
\pgfpathlineto{\pgfqpoint{4.963980in}{1.072725in}}%
\pgfpathlineto{\pgfqpoint{4.964365in}{1.071094in}}%
\pgfpathlineto{\pgfqpoint{4.965902in}{1.064101in}}%
\pgfpathlineto{\pgfqpoint{4.966286in}{1.067112in}}%
\pgfpathlineto{\pgfqpoint{4.966863in}{1.063380in}}%
\pgfpathlineto{\pgfqpoint{4.967247in}{1.066835in}}%
\pgfpathlineto{\pgfqpoint{4.967632in}{1.065418in}}%
\pgfpathlineto{\pgfqpoint{4.967824in}{1.067860in}}%
\pgfpathlineto{\pgfqpoint{4.968208in}{1.067635in}}%
\pgfpathlineto{\pgfqpoint{4.968592in}{1.068970in}}%
\pgfpathlineto{\pgfqpoint{4.968977in}{1.066757in}}%
\pgfpathlineto{\pgfqpoint{4.969169in}{1.066713in}}%
\pgfpathlineto{\pgfqpoint{4.969361in}{1.065373in}}%
\pgfpathlineto{\pgfqpoint{4.969938in}{1.068155in}}%
\pgfpathlineto{\pgfqpoint{4.970130in}{1.068266in}}%
\pgfpathlineto{\pgfqpoint{4.970322in}{1.071978in}}%
\pgfpathlineto{\pgfqpoint{4.970899in}{1.066764in}}%
\pgfpathlineto{\pgfqpoint{4.972820in}{1.056477in}}%
\pgfpathlineto{\pgfqpoint{4.973012in}{1.056422in}}%
\pgfpathlineto{\pgfqpoint{4.973205in}{1.057473in}}%
\pgfpathlineto{\pgfqpoint{4.973397in}{1.058195in}}%
\pgfpathlineto{\pgfqpoint{4.973589in}{1.056222in}}%
\pgfpathlineto{\pgfqpoint{4.973781in}{1.056388in}}%
\pgfpathlineto{\pgfqpoint{4.975318in}{1.043973in}}%
\pgfpathlineto{\pgfqpoint{4.974165in}{1.057297in}}%
\pgfpathlineto{\pgfqpoint{4.976471in}{1.044519in}}%
\pgfpathlineto{\pgfqpoint{4.976664in}{1.047313in}}%
\pgfpathlineto{\pgfqpoint{4.977048in}{1.041651in}}%
\pgfpathlineto{\pgfqpoint{4.977432in}{1.043838in}}%
\pgfpathlineto{\pgfqpoint{4.977624in}{1.041724in}}%
\pgfpathlineto{\pgfqpoint{4.978009in}{1.047080in}}%
\pgfpathlineto{\pgfqpoint{4.978970in}{1.054512in}}%
\pgfpathlineto{\pgfqpoint{4.979162in}{1.052926in}}%
\pgfpathlineto{\pgfqpoint{4.979546in}{1.050816in}}%
\pgfpathlineto{\pgfqpoint{4.979930in}{1.054507in}}%
\pgfpathlineto{\pgfqpoint{4.980123in}{1.053558in}}%
\pgfpathlineto{\pgfqpoint{4.982044in}{1.056586in}}%
\pgfpathlineto{\pgfqpoint{4.982429in}{1.054771in}}%
\pgfpathlineto{\pgfqpoint{4.984158in}{1.043204in}}%
\pgfpathlineto{\pgfqpoint{4.984350in}{1.043826in}}%
\pgfpathlineto{\pgfqpoint{4.984542in}{1.042771in}}%
\pgfpathlineto{\pgfqpoint{4.985119in}{1.044858in}}%
\pgfpathlineto{\pgfqpoint{4.985695in}{1.048302in}}%
\pgfpathlineto{\pgfqpoint{4.986080in}{1.046403in}}%
\pgfpathlineto{\pgfqpoint{4.987425in}{1.040643in}}%
\pgfpathlineto{\pgfqpoint{4.988001in}{1.036646in}}%
\pgfpathlineto{\pgfqpoint{4.988578in}{1.039746in}}%
\pgfpathlineto{\pgfqpoint{4.988962in}{1.038190in}}%
\pgfpathlineto{\pgfqpoint{4.990307in}{1.044402in}}%
\pgfpathlineto{\pgfqpoint{4.990692in}{1.038277in}}%
\pgfpathlineto{\pgfqpoint{4.991460in}{1.041569in}}%
\pgfpathlineto{\pgfqpoint{4.992998in}{1.049110in}}%
\pgfpathlineto{\pgfqpoint{4.993190in}{1.048736in}}%
\pgfpathlineto{\pgfqpoint{4.995496in}{1.040600in}}%
\pgfpathlineto{\pgfqpoint{4.996841in}{1.031047in}}%
\pgfpathlineto{\pgfqpoint{4.997226in}{1.034116in}}%
\pgfpathlineto{\pgfqpoint{4.997418in}{1.035558in}}%
\pgfpathlineto{\pgfqpoint{4.997610in}{1.033778in}}%
\pgfpathlineto{\pgfqpoint{4.997994in}{1.034549in}}%
\pgfpathlineto{\pgfqpoint{4.998186in}{1.032765in}}%
\pgfpathlineto{\pgfqpoint{4.998763in}{1.037382in}}%
\pgfpathlineto{\pgfqpoint{4.999147in}{1.039362in}}%
\pgfpathlineto{\pgfqpoint{5.000108in}{1.031132in}}%
\pgfpathlineto{\pgfqpoint{5.000300in}{1.033036in}}%
\pgfpathlineto{\pgfqpoint{5.000877in}{1.035744in}}%
\pgfpathlineto{\pgfqpoint{5.001453in}{1.034595in}}%
\pgfpathlineto{\pgfqpoint{5.001838in}{1.033540in}}%
\pgfpathlineto{\pgfqpoint{5.002030in}{1.035928in}}%
\pgfpathlineto{\pgfqpoint{5.002222in}{1.039260in}}%
\pgfpathlineto{\pgfqpoint{5.002798in}{1.036600in}}%
\pgfpathlineto{\pgfqpoint{5.002991in}{1.032162in}}%
\pgfpathlineto{\pgfqpoint{5.003759in}{1.033054in}}%
\pgfpathlineto{\pgfqpoint{5.003951in}{1.035716in}}%
\pgfpathlineto{\pgfqpoint{5.004720in}{1.035121in}}%
\pgfpathlineto{\pgfqpoint{5.005681in}{1.031528in}}%
\pgfpathlineto{\pgfqpoint{5.006065in}{1.031882in}}%
\pgfpathlineto{\pgfqpoint{5.006450in}{1.033745in}}%
\pgfpathlineto{\pgfqpoint{5.006642in}{1.030669in}}%
\pgfpathlineto{\pgfqpoint{5.008563in}{1.019392in}}%
\pgfpathlineto{\pgfqpoint{5.008756in}{1.019550in}}%
\pgfpathlineto{\pgfqpoint{5.009524in}{1.023745in}}%
\pgfpathlineto{\pgfqpoint{5.010101in}{1.022095in}}%
\pgfpathlineto{\pgfqpoint{5.010293in}{1.021325in}}%
\pgfpathlineto{\pgfqpoint{5.010485in}{1.022950in}}%
\pgfpathlineto{\pgfqpoint{5.010869in}{1.021933in}}%
\pgfpathlineto{\pgfqpoint{5.011830in}{1.029057in}}%
\pgfpathlineto{\pgfqpoint{5.012022in}{1.028087in}}%
\pgfpathlineto{\pgfqpoint{5.012599in}{1.020266in}}%
\pgfpathlineto{\pgfqpoint{5.013175in}{1.020780in}}%
\pgfpathlineto{\pgfqpoint{5.013944in}{1.025486in}}%
\pgfpathlineto{\pgfqpoint{5.014136in}{1.028683in}}%
\pgfpathlineto{\pgfqpoint{5.014905in}{1.026611in}}%
\pgfpathlineto{\pgfqpoint{5.015481in}{1.028582in}}%
\pgfpathlineto{\pgfqpoint{5.016250in}{1.024312in}}%
\pgfpathlineto{\pgfqpoint{5.017595in}{1.031670in}}%
\pgfpathlineto{\pgfqpoint{5.017788in}{1.030951in}}%
\pgfpathlineto{\pgfqpoint{5.018172in}{1.026688in}}%
\pgfpathlineto{\pgfqpoint{5.018941in}{1.028977in}}%
\pgfpathlineto{\pgfqpoint{5.019325in}{1.025016in}}%
\pgfpathlineto{\pgfqpoint{5.019901in}{1.029173in}}%
\pgfpathlineto{\pgfqpoint{5.020094in}{1.029426in}}%
\pgfpathlineto{\pgfqpoint{5.021247in}{1.035135in}}%
\pgfpathlineto{\pgfqpoint{5.021439in}{1.033256in}}%
\pgfpathlineto{\pgfqpoint{5.021823in}{1.034750in}}%
\pgfpathlineto{\pgfqpoint{5.022207in}{1.031464in}}%
\pgfpathlineto{\pgfqpoint{5.022976in}{1.030725in}}%
\pgfpathlineto{\pgfqpoint{5.023168in}{1.031863in}}%
\pgfpathlineto{\pgfqpoint{5.023360in}{1.034924in}}%
\pgfpathlineto{\pgfqpoint{5.024129in}{1.029869in}}%
\pgfpathlineto{\pgfqpoint{5.026435in}{1.017610in}}%
\pgfpathlineto{\pgfqpoint{5.024706in}{1.030258in}}%
\pgfpathlineto{\pgfqpoint{5.027204in}{1.020761in}}%
\pgfpathlineto{\pgfqpoint{5.027972in}{1.029143in}}%
\pgfpathlineto{\pgfqpoint{5.028741in}{1.025613in}}%
\pgfpathlineto{\pgfqpoint{5.031239in}{1.035043in}}%
\pgfpathlineto{\pgfqpoint{5.031431in}{1.033766in}}%
\pgfpathlineto{\pgfqpoint{5.032008in}{1.034606in}}%
\pgfpathlineto{\pgfqpoint{5.032392in}{1.031616in}}%
\pgfpathlineto{\pgfqpoint{5.033161in}{1.036529in}}%
\pgfpathlineto{\pgfqpoint{5.033353in}{1.034507in}}%
\pgfpathlineto{\pgfqpoint{5.034698in}{1.028929in}}%
\pgfpathlineto{\pgfqpoint{5.036236in}{1.033549in}}%
\pgfpathlineto{\pgfqpoint{5.037965in}{1.023950in}}%
\pgfpathlineto{\pgfqpoint{5.038542in}{1.023456in}}%
\pgfpathlineto{\pgfqpoint{5.039118in}{1.029098in}}%
\pgfpathlineto{\pgfqpoint{5.039310in}{1.023200in}}%
\pgfpathlineto{\pgfqpoint{5.039502in}{1.024452in}}%
\pgfpathlineto{\pgfqpoint{5.041424in}{1.016513in}}%
\pgfpathlineto{\pgfqpoint{5.042001in}{1.010608in}}%
\pgfpathlineto{\pgfqpoint{5.042385in}{1.016711in}}%
\pgfpathlineto{\pgfqpoint{5.042577in}{1.017181in}}%
\pgfpathlineto{\pgfqpoint{5.042962in}{1.013792in}}%
\pgfpathlineto{\pgfqpoint{5.043538in}{1.015302in}}%
\pgfpathlineto{\pgfqpoint{5.044499in}{1.021397in}}%
\pgfpathlineto{\pgfqpoint{5.044883in}{1.020034in}}%
\pgfpathlineto{\pgfqpoint{5.045075in}{1.018677in}}%
\pgfpathlineto{\pgfqpoint{5.045460in}{1.022904in}}%
\pgfpathlineto{\pgfqpoint{5.045844in}{1.019786in}}%
\pgfpathlineto{\pgfqpoint{5.046036in}{1.021663in}}%
\pgfpathlineto{\pgfqpoint{5.046421in}{1.015912in}}%
\pgfpathlineto{\pgfqpoint{5.046613in}{1.016682in}}%
\pgfpathlineto{\pgfqpoint{5.047189in}{1.019034in}}%
\pgfpathlineto{\pgfqpoint{5.047766in}{1.021783in}}%
\pgfpathlineto{\pgfqpoint{5.048342in}{1.019422in}}%
\pgfpathlineto{\pgfqpoint{5.048534in}{1.019204in}}%
\pgfpathlineto{\pgfqpoint{5.048727in}{1.013537in}}%
\pgfpathlineto{\pgfqpoint{5.049495in}{1.018823in}}%
\pgfpathlineto{\pgfqpoint{5.051225in}{1.014573in}}%
\pgfpathlineto{\pgfqpoint{5.053915in}{1.030399in}}%
\pgfpathlineto{\pgfqpoint{5.054492in}{1.022596in}}%
\pgfpathlineto{\pgfqpoint{5.055068in}{1.023922in}}%
\pgfpathlineto{\pgfqpoint{5.055260in}{1.026867in}}%
\pgfpathlineto{\pgfqpoint{5.056221in}{1.026106in}}%
\pgfpathlineto{\pgfqpoint{5.056413in}{1.024619in}}%
\pgfpathlineto{\pgfqpoint{5.056990in}{1.027090in}}%
\pgfpathlineto{\pgfqpoint{5.057182in}{1.026227in}}%
\pgfpathlineto{\pgfqpoint{5.058335in}{1.032890in}}%
\pgfpathlineto{\pgfqpoint{5.058527in}{1.030055in}}%
\pgfpathlineto{\pgfqpoint{5.059680in}{1.020589in}}%
\pgfpathlineto{\pgfqpoint{5.059872in}{1.021302in}}%
\pgfpathlineto{\pgfqpoint{5.061025in}{1.028889in}}%
\pgfpathlineto{\pgfqpoint{5.061410in}{1.026047in}}%
\pgfpathlineto{\pgfqpoint{5.062370in}{1.023829in}}%
\pgfpathlineto{\pgfqpoint{5.062563in}{1.025551in}}%
\pgfpathlineto{\pgfqpoint{5.062755in}{1.027171in}}%
\pgfpathlineto{\pgfqpoint{5.063331in}{1.023952in}}%
\pgfpathlineto{\pgfqpoint{5.063716in}{1.026285in}}%
\pgfpathlineto{\pgfqpoint{5.065061in}{1.015958in}}%
\pgfpathlineto{\pgfqpoint{5.065829in}{1.017130in}}%
\pgfpathlineto{\pgfqpoint{5.067559in}{1.024634in}}%
\pgfpathlineto{\pgfqpoint{5.067751in}{1.024524in}}%
\pgfpathlineto{\pgfqpoint{5.067943in}{1.024830in}}%
\pgfpathlineto{\pgfqpoint{5.069096in}{1.032378in}}%
\pgfpathlineto{\pgfqpoint{5.069289in}{1.029865in}}%
\pgfpathlineto{\pgfqpoint{5.070634in}{1.019433in}}%
\pgfpathlineto{\pgfqpoint{5.070826in}{1.021025in}}%
\pgfpathlineto{\pgfqpoint{5.071402in}{1.015184in}}%
\pgfpathlineto{\pgfqpoint{5.071979in}{1.019344in}}%
\pgfpathlineto{\pgfqpoint{5.075246in}{1.032454in}}%
\pgfpathlineto{\pgfqpoint{5.075438in}{1.031791in}}%
\pgfpathlineto{\pgfqpoint{5.075630in}{1.030353in}}%
\pgfpathlineto{\pgfqpoint{5.075822in}{1.032285in}}%
\pgfpathlineto{\pgfqpoint{5.076207in}{1.031657in}}%
\pgfpathlineto{\pgfqpoint{5.076591in}{1.036165in}}%
\pgfpathlineto{\pgfqpoint{5.076975in}{1.031282in}}%
\pgfpathlineto{\pgfqpoint{5.077167in}{1.029392in}}%
\pgfpathlineto{\pgfqpoint{5.077360in}{1.034526in}}%
\pgfpathlineto{\pgfqpoint{5.077744in}{1.033278in}}%
\pgfpathlineto{\pgfqpoint{5.077936in}{1.036784in}}%
\pgfpathlineto{\pgfqpoint{5.078513in}{1.032827in}}%
\pgfpathlineto{\pgfqpoint{5.078897in}{1.035886in}}%
\pgfpathlineto{\pgfqpoint{5.079089in}{1.035231in}}%
\pgfpathlineto{\pgfqpoint{5.079281in}{1.036413in}}%
\pgfpathlineto{\pgfqpoint{5.079858in}{1.035802in}}%
\pgfpathlineto{\pgfqpoint{5.080050in}{1.036575in}}%
\pgfpathlineto{\pgfqpoint{5.080434in}{1.034954in}}%
\pgfpathlineto{\pgfqpoint{5.080626in}{1.034218in}}%
\pgfpathlineto{\pgfqpoint{5.081779in}{1.039482in}}%
\pgfpathlineto{\pgfqpoint{5.082932in}{1.034855in}}%
\pgfpathlineto{\pgfqpoint{5.083125in}{1.035696in}}%
\pgfpathlineto{\pgfqpoint{5.083317in}{1.034740in}}%
\pgfpathlineto{\pgfqpoint{5.083509in}{1.037107in}}%
\pgfpathlineto{\pgfqpoint{5.083701in}{1.038932in}}%
\pgfpathlineto{\pgfqpoint{5.084085in}{1.034162in}}%
\pgfpathlineto{\pgfqpoint{5.084470in}{1.037775in}}%
\pgfpathlineto{\pgfqpoint{5.085623in}{1.030898in}}%
\pgfpathlineto{\pgfqpoint{5.086199in}{1.034990in}}%
\pgfpathlineto{\pgfqpoint{5.090811in}{1.063114in}}%
\pgfpathlineto{\pgfqpoint{5.091004in}{1.062826in}}%
\pgfpathlineto{\pgfqpoint{5.091580in}{1.053575in}}%
\pgfpathlineto{\pgfqpoint{5.092157in}{1.057255in}}%
\pgfpathlineto{\pgfqpoint{5.092349in}{1.058552in}}%
\pgfpathlineto{\pgfqpoint{5.092733in}{1.055277in}}%
\pgfpathlineto{\pgfqpoint{5.095423in}{1.042759in}}%
\pgfpathlineto{\pgfqpoint{5.095808in}{1.050721in}}%
\pgfpathlineto{\pgfqpoint{5.096576in}{1.046754in}}%
\pgfpathlineto{\pgfqpoint{5.098114in}{1.039774in}}%
\pgfpathlineto{\pgfqpoint{5.099459in}{1.042827in}}%
\pgfpathlineto{\pgfqpoint{5.100035in}{1.039181in}}%
\pgfpathlineto{\pgfqpoint{5.100612in}{1.041061in}}%
\pgfpathlineto{\pgfqpoint{5.101573in}{1.040078in}}%
\pgfpathlineto{\pgfqpoint{5.102918in}{1.042807in}}%
\pgfpathlineto{\pgfqpoint{5.103687in}{1.037331in}}%
\pgfpathlineto{\pgfqpoint{5.104071in}{1.039073in}}%
\pgfpathlineto{\pgfqpoint{5.105032in}{1.037814in}}%
\pgfpathlineto{\pgfqpoint{5.105224in}{1.040819in}}%
\pgfpathlineto{\pgfqpoint{5.105800in}{1.035941in}}%
\pgfpathlineto{\pgfqpoint{5.108683in}{1.022106in}}%
\pgfpathlineto{\pgfqpoint{5.108875in}{1.019380in}}%
\pgfpathlineto{\pgfqpoint{5.109644in}{1.024229in}}%
\pgfpathlineto{\pgfqpoint{5.110028in}{1.025375in}}%
\pgfpathlineto{\pgfqpoint{5.110220in}{1.023468in}}%
\pgfpathlineto{\pgfqpoint{5.110989in}{1.018906in}}%
\pgfpathlineto{\pgfqpoint{5.111181in}{1.017795in}}%
\pgfpathlineto{\pgfqpoint{5.111565in}{1.019948in}}%
\pgfpathlineto{\pgfqpoint{5.111758in}{1.019387in}}%
\pgfpathlineto{\pgfqpoint{5.112334in}{1.023456in}}%
\pgfpathlineto{\pgfqpoint{5.113679in}{1.028354in}}%
\pgfpathlineto{\pgfqpoint{5.115217in}{1.022776in}}%
\pgfpathlineto{\pgfqpoint{5.115409in}{1.025521in}}%
\pgfpathlineto{\pgfqpoint{5.115793in}{1.019771in}}%
\pgfpathlineto{\pgfqpoint{5.115985in}{1.020138in}}%
\pgfpathlineto{\pgfqpoint{5.117331in}{1.014918in}}%
\pgfpathlineto{\pgfqpoint{5.117715in}{1.014529in}}%
\pgfpathlineto{\pgfqpoint{5.118484in}{1.017665in}}%
\pgfpathlineto{\pgfqpoint{5.119060in}{1.013783in}}%
\pgfpathlineto{\pgfqpoint{5.119444in}{1.017997in}}%
\pgfpathlineto{\pgfqpoint{5.119637in}{1.018638in}}%
\pgfpathlineto{\pgfqpoint{5.119829in}{1.014651in}}%
\pgfpathlineto{\pgfqpoint{5.120597in}{1.017201in}}%
\pgfpathlineto{\pgfqpoint{5.121174in}{1.021529in}}%
\pgfpathlineto{\pgfqpoint{5.121558in}{1.016369in}}%
\pgfpathlineto{\pgfqpoint{5.121750in}{1.015896in}}%
\pgfpathlineto{\pgfqpoint{5.122135in}{1.012233in}}%
\pgfpathlineto{\pgfqpoint{5.122903in}{1.013838in}}%
\pgfpathlineto{\pgfqpoint{5.123096in}{1.013961in}}%
\pgfpathlineto{\pgfqpoint{5.123864in}{1.012594in}}%
\pgfpathlineto{\pgfqpoint{5.123480in}{1.015745in}}%
\pgfpathlineto{\pgfqpoint{5.124249in}{1.013548in}}%
\pgfpathlineto{\pgfqpoint{5.124441in}{1.015497in}}%
\pgfpathlineto{\pgfqpoint{5.125017in}{1.010772in}}%
\pgfpathlineto{\pgfqpoint{5.126362in}{1.000533in}}%
\pgfpathlineto{\pgfqpoint{5.127131in}{1.007039in}}%
\pgfpathlineto{\pgfqpoint{5.127515in}{1.002252in}}%
\pgfpathlineto{\pgfqpoint{5.128284in}{1.000446in}}%
\pgfpathlineto{\pgfqpoint{5.128668in}{1.000505in}}%
\pgfpathlineto{\pgfqpoint{5.128861in}{1.002214in}}%
\pgfpathlineto{\pgfqpoint{5.129245in}{0.997429in}}%
\pgfpathlineto{\pgfqpoint{5.129437in}{0.995556in}}%
\pgfpathlineto{\pgfqpoint{5.130014in}{0.998788in}}%
\pgfpathlineto{\pgfqpoint{5.130206in}{0.998092in}}%
\pgfpathlineto{\pgfqpoint{5.130398in}{0.999333in}}%
\pgfpathlineto{\pgfqpoint{5.130590in}{0.996457in}}%
\pgfpathlineto{\pgfqpoint{5.131167in}{0.998482in}}%
\pgfpathlineto{\pgfqpoint{5.131359in}{0.994605in}}%
\pgfpathlineto{\pgfqpoint{5.132127in}{0.999143in}}%
\pgfpathlineto{\pgfqpoint{5.132896in}{1.003515in}}%
\pgfpathlineto{\pgfqpoint{5.133473in}{1.001930in}}%
\pgfpathlineto{\pgfqpoint{5.133665in}{1.000619in}}%
\pgfpathlineto{\pgfqpoint{5.134241in}{1.003630in}}%
\pgfpathlineto{\pgfqpoint{5.134433in}{1.004260in}}%
\pgfpathlineto{\pgfqpoint{5.134626in}{1.002834in}}%
\pgfpathlineto{\pgfqpoint{5.136932in}{0.988165in}}%
\pgfpathlineto{\pgfqpoint{5.138469in}{0.985273in}}%
\pgfpathlineto{\pgfqpoint{5.137316in}{0.989210in}}%
\pgfpathlineto{\pgfqpoint{5.138661in}{0.986406in}}%
\pgfpathlineto{\pgfqpoint{5.140006in}{0.995681in}}%
\pgfpathlineto{\pgfqpoint{5.140199in}{0.993578in}}%
\pgfpathlineto{\pgfqpoint{5.140583in}{0.996684in}}%
\pgfpathlineto{\pgfqpoint{5.140775in}{0.996058in}}%
\pgfpathlineto{\pgfqpoint{5.141736in}{1.003604in}}%
\pgfpathlineto{\pgfqpoint{5.142697in}{1.003111in}}%
\pgfpathlineto{\pgfqpoint{5.142889in}{1.000023in}}%
\pgfpathlineto{\pgfqpoint{5.143658in}{1.003499in}}%
\pgfpathlineto{\pgfqpoint{5.143850in}{1.002062in}}%
\pgfpathlineto{\pgfqpoint{5.144042in}{1.002212in}}%
\pgfpathlineto{\pgfqpoint{5.144234in}{1.001748in}}%
\pgfpathlineto{\pgfqpoint{5.145771in}{1.013526in}}%
\pgfpathlineto{\pgfqpoint{5.145964in}{1.009552in}}%
\pgfpathlineto{\pgfqpoint{5.146540in}{1.014951in}}%
\pgfpathlineto{\pgfqpoint{5.146924in}{1.011451in}}%
\pgfpathlineto{\pgfqpoint{5.147117in}{1.013213in}}%
\pgfpathlineto{\pgfqpoint{5.147501in}{1.009569in}}%
\pgfpathlineto{\pgfqpoint{5.147693in}{1.007257in}}%
\pgfpathlineto{\pgfqpoint{5.148270in}{1.010640in}}%
\pgfpathlineto{\pgfqpoint{5.148462in}{1.010631in}}%
\pgfpathlineto{\pgfqpoint{5.148846in}{1.010588in}}%
\pgfpathlineto{\pgfqpoint{5.149038in}{1.011358in}}%
\pgfpathlineto{\pgfqpoint{5.149230in}{1.011587in}}%
\pgfpathlineto{\pgfqpoint{5.149423in}{1.011278in}}%
\pgfpathlineto{\pgfqpoint{5.150576in}{1.020011in}}%
\pgfpathlineto{\pgfqpoint{5.150768in}{1.019012in}}%
\pgfpathlineto{\pgfqpoint{5.150960in}{1.016758in}}%
\pgfpathlineto{\pgfqpoint{5.151344in}{1.021838in}}%
\pgfpathlineto{\pgfqpoint{5.151729in}{1.017625in}}%
\pgfpathlineto{\pgfqpoint{5.152113in}{1.022287in}}%
\pgfpathlineto{\pgfqpoint{5.152689in}{1.015287in}}%
\pgfpathlineto{\pgfqpoint{5.153458in}{1.016653in}}%
\pgfpathlineto{\pgfqpoint{5.154035in}{1.021746in}}%
\pgfpathlineto{\pgfqpoint{5.154611in}{1.017317in}}%
\pgfpathlineto{\pgfqpoint{5.154803in}{1.017331in}}%
\pgfpathlineto{\pgfqpoint{5.154995in}{1.018713in}}%
\pgfpathlineto{\pgfqpoint{5.155572in}{1.016782in}}%
\pgfpathlineto{\pgfqpoint{5.157109in}{1.011562in}}%
\pgfpathlineto{\pgfqpoint{5.158262in}{1.023048in}}%
\pgfpathlineto{\pgfqpoint{5.158647in}{1.022610in}}%
\pgfpathlineto{\pgfqpoint{5.159800in}{1.013286in}}%
\pgfpathlineto{\pgfqpoint{5.160184in}{1.015618in}}%
\pgfpathlineto{\pgfqpoint{5.160568in}{1.010601in}}%
\pgfpathlineto{\pgfqpoint{5.160760in}{1.011352in}}%
\pgfpathlineto{\pgfqpoint{5.160953in}{1.006023in}}%
\pgfpathlineto{\pgfqpoint{5.161913in}{1.007298in}}%
\pgfpathlineto{\pgfqpoint{5.162298in}{1.007488in}}%
\pgfpathlineto{\pgfqpoint{5.163643in}{1.001485in}}%
\pgfpathlineto{\pgfqpoint{5.164027in}{1.006288in}}%
\pgfpathlineto{\pgfqpoint{5.164412in}{1.000775in}}%
\pgfpathlineto{\pgfqpoint{5.164604in}{1.002248in}}%
\pgfpathlineto{\pgfqpoint{5.165373in}{0.995581in}}%
\pgfpathlineto{\pgfqpoint{5.166141in}{0.996220in}}%
\pgfpathlineto{\pgfqpoint{5.166333in}{0.994895in}}%
\pgfpathlineto{\pgfqpoint{5.166910in}{0.996568in}}%
\pgfpathlineto{\pgfqpoint{5.168255in}{1.000230in}}%
\pgfpathlineto{\pgfqpoint{5.168639in}{0.997628in}}%
\pgfpathlineto{\pgfqpoint{5.169024in}{1.000855in}}%
\pgfpathlineto{\pgfqpoint{5.169216in}{1.002761in}}%
\pgfpathlineto{\pgfqpoint{5.169408in}{0.999006in}}%
\pgfpathlineto{\pgfqpoint{5.169792in}{0.999741in}}%
\pgfpathlineto{\pgfqpoint{5.170177in}{0.997241in}}%
\pgfpathlineto{\pgfqpoint{5.170369in}{0.998516in}}%
\pgfpathlineto{\pgfqpoint{5.171522in}{1.007863in}}%
\pgfpathlineto{\pgfqpoint{5.171714in}{1.007071in}}%
\pgfpathlineto{\pgfqpoint{5.171906in}{1.004254in}}%
\pgfpathlineto{\pgfqpoint{5.172675in}{1.008405in}}%
\pgfpathlineto{\pgfqpoint{5.173444in}{1.003681in}}%
\pgfpathlineto{\pgfqpoint{5.173636in}{1.005526in}}%
\pgfpathlineto{\pgfqpoint{5.173828in}{1.008244in}}%
\pgfpathlineto{\pgfqpoint{5.174597in}{1.003896in}}%
\pgfpathlineto{\pgfqpoint{5.177095in}{1.019027in}}%
\pgfpathlineto{\pgfqpoint{5.177863in}{1.018236in}}%
\pgfpathlineto{\pgfqpoint{5.178248in}{1.015758in}}%
\pgfpathlineto{\pgfqpoint{5.178824in}{1.018932in}}%
\pgfpathlineto{\pgfqpoint{5.179401in}{1.018383in}}%
\pgfpathlineto{\pgfqpoint{5.180169in}{1.019799in}}%
\pgfpathlineto{\pgfqpoint{5.180362in}{1.018853in}}%
\pgfpathlineto{\pgfqpoint{5.180554in}{1.020805in}}%
\pgfpathlineto{\pgfqpoint{5.180746in}{1.019496in}}%
\pgfpathlineto{\pgfqpoint{5.180938in}{1.022874in}}%
\pgfpathlineto{\pgfqpoint{5.181130in}{1.018729in}}%
\pgfpathlineto{\pgfqpoint{5.181707in}{1.020406in}}%
\pgfpathlineto{\pgfqpoint{5.181899in}{1.018819in}}%
\pgfpathlineto{\pgfqpoint{5.182091in}{1.021627in}}%
\pgfpathlineto{\pgfqpoint{5.182475in}{1.020660in}}%
\pgfpathlineto{\pgfqpoint{5.183628in}{1.026162in}}%
\pgfpathlineto{\pgfqpoint{5.183821in}{1.023577in}}%
\pgfpathlineto{\pgfqpoint{5.184013in}{1.023668in}}%
\pgfpathlineto{\pgfqpoint{5.185742in}{1.010896in}}%
\pgfpathlineto{\pgfqpoint{5.186127in}{1.015356in}}%
\pgfpathlineto{\pgfqpoint{5.186703in}{1.014894in}}%
\pgfpathlineto{\pgfqpoint{5.187280in}{1.017472in}}%
\pgfpathlineto{\pgfqpoint{5.187664in}{1.010358in}}%
\pgfpathlineto{\pgfqpoint{5.188240in}{1.014713in}}%
\pgfpathlineto{\pgfqpoint{5.189586in}{1.020793in}}%
\pgfpathlineto{\pgfqpoint{5.190547in}{1.017237in}}%
\pgfpathlineto{\pgfqpoint{5.190739in}{1.020041in}}%
\pgfpathlineto{\pgfqpoint{5.190931in}{1.020744in}}%
\pgfpathlineto{\pgfqpoint{5.191123in}{1.018897in}}%
\pgfpathlineto{\pgfqpoint{5.192276in}{1.011075in}}%
\pgfpathlineto{\pgfqpoint{5.192660in}{1.015694in}}%
\pgfpathlineto{\pgfqpoint{5.194774in}{1.023773in}}%
\pgfpathlineto{\pgfqpoint{5.195735in}{1.017683in}}%
\pgfpathlineto{\pgfqpoint{5.195927in}{1.020054in}}%
\pgfpathlineto{\pgfqpoint{5.196119in}{1.022663in}}%
\pgfpathlineto{\pgfqpoint{5.196696in}{1.017337in}}%
\pgfpathlineto{\pgfqpoint{5.196888in}{1.020023in}}%
\pgfpathlineto{\pgfqpoint{5.197080in}{1.018403in}}%
\pgfpathlineto{\pgfqpoint{5.197657in}{1.021622in}}%
\pgfpathlineto{\pgfqpoint{5.197849in}{1.022400in}}%
\pgfpathlineto{\pgfqpoint{5.198041in}{1.019598in}}%
\pgfpathlineto{\pgfqpoint{5.199002in}{1.015611in}}%
\pgfpathlineto{\pgfqpoint{5.198425in}{1.020828in}}%
\pgfpathlineto{\pgfqpoint{5.199386in}{1.017463in}}%
\pgfpathlineto{\pgfqpoint{5.200155in}{1.016299in}}%
\pgfpathlineto{\pgfqpoint{5.201500in}{1.009303in}}%
\pgfpathlineto{\pgfqpoint{5.201884in}{1.011427in}}%
\pgfpathlineto{\pgfqpoint{5.202653in}{1.008100in}}%
\pgfpathlineto{\pgfqpoint{5.203037in}{1.008939in}}%
\pgfpathlineto{\pgfqpoint{5.204190in}{1.012502in}}%
\pgfpathlineto{\pgfqpoint{5.204383in}{1.012385in}}%
\pgfpathlineto{\pgfqpoint{5.204575in}{1.010904in}}%
\pgfpathlineto{\pgfqpoint{5.204959in}{1.014680in}}%
\pgfpathlineto{\pgfqpoint{5.205151in}{1.014315in}}%
\pgfpathlineto{\pgfqpoint{5.205536in}{1.012924in}}%
\pgfpathlineto{\pgfqpoint{5.206112in}{1.008896in}}%
\pgfpathlineto{\pgfqpoint{5.206304in}{1.012538in}}%
\pgfpathlineto{\pgfqpoint{5.206496in}{1.014091in}}%
\pgfpathlineto{\pgfqpoint{5.207265in}{1.012796in}}%
\pgfpathlineto{\pgfqpoint{5.207457in}{1.011632in}}%
\pgfpathlineto{\pgfqpoint{5.207649in}{1.014918in}}%
\pgfpathlineto{\pgfqpoint{5.208226in}{1.018651in}}%
\pgfpathlineto{\pgfqpoint{5.208802in}{1.016536in}}%
\pgfpathlineto{\pgfqpoint{5.209763in}{1.006401in}}%
\pgfpathlineto{\pgfqpoint{5.210148in}{1.012414in}}%
\pgfpathlineto{\pgfqpoint{5.210340in}{1.011865in}}%
\pgfpathlineto{\pgfqpoint{5.210532in}{1.014311in}}%
\pgfpathlineto{\pgfqpoint{5.211301in}{1.023818in}}%
\pgfpathlineto{\pgfqpoint{5.211877in}{1.022850in}}%
\pgfpathlineto{\pgfqpoint{5.212069in}{1.025099in}}%
\pgfpathlineto{\pgfqpoint{5.212646in}{1.019222in}}%
\pgfpathlineto{\pgfqpoint{5.213607in}{1.015710in}}%
\pgfpathlineto{\pgfqpoint{5.215336in}{1.004591in}}%
\pgfpathlineto{\pgfqpoint{5.215528in}{1.005048in}}%
\pgfpathlineto{\pgfqpoint{5.215721in}{1.004534in}}%
\pgfpathlineto{\pgfqpoint{5.216105in}{1.004769in}}%
\pgfpathlineto{\pgfqpoint{5.217258in}{0.997369in}}%
\pgfpathlineto{\pgfqpoint{5.218411in}{0.992654in}}%
\pgfpathlineto{\pgfqpoint{5.218603in}{0.994815in}}%
\pgfpathlineto{\pgfqpoint{5.218987in}{0.998272in}}%
\pgfpathlineto{\pgfqpoint{5.219372in}{0.994473in}}%
\pgfpathlineto{\pgfqpoint{5.219564in}{0.993203in}}%
\pgfpathlineto{\pgfqpoint{5.220140in}{0.995894in}}%
\pgfpathlineto{\pgfqpoint{5.220525in}{0.993917in}}%
\pgfpathlineto{\pgfqpoint{5.221486in}{0.997888in}}%
\pgfpathlineto{\pgfqpoint{5.221870in}{0.994967in}}%
\pgfpathlineto{\pgfqpoint{5.222062in}{0.993678in}}%
\pgfpathlineto{\pgfqpoint{5.222254in}{0.996710in}}%
\pgfpathlineto{\pgfqpoint{5.222446in}{0.996730in}}%
\pgfpathlineto{\pgfqpoint{5.224368in}{1.003002in}}%
\pgfpathlineto{\pgfqpoint{5.224752in}{1.000675in}}%
\pgfpathlineto{\pgfqpoint{5.225137in}{1.003708in}}%
\pgfpathlineto{\pgfqpoint{5.226674in}{1.006867in}}%
\pgfpathlineto{\pgfqpoint{5.228980in}{0.999270in}}%
\pgfpathlineto{\pgfqpoint{5.229172in}{1.000620in}}%
\pgfpathlineto{\pgfqpoint{5.229557in}{1.002971in}}%
\pgfpathlineto{\pgfqpoint{5.229941in}{0.999433in}}%
\pgfpathlineto{\pgfqpoint{5.230325in}{0.998211in}}%
\pgfpathlineto{\pgfqpoint{5.230517in}{1.000799in}}%
\pgfpathlineto{\pgfqpoint{5.231286in}{1.009820in}}%
\pgfpathlineto{\pgfqpoint{5.231670in}{1.004864in}}%
\pgfpathlineto{\pgfqpoint{5.232055in}{1.003221in}}%
\pgfpathlineto{\pgfqpoint{5.232439in}{1.005550in}}%
\pgfpathlineto{\pgfqpoint{5.232823in}{1.008870in}}%
\pgfpathlineto{\pgfqpoint{5.233016in}{1.005291in}}%
\pgfpathlineto{\pgfqpoint{5.233976in}{0.999225in}}%
\pgfpathlineto{\pgfqpoint{5.234169in}{1.001882in}}%
\pgfpathlineto{\pgfqpoint{5.234361in}{1.003927in}}%
\pgfpathlineto{\pgfqpoint{5.234937in}{1.000368in}}%
\pgfpathlineto{\pgfqpoint{5.235129in}{1.001522in}}%
\pgfpathlineto{\pgfqpoint{5.235322in}{1.001826in}}%
\pgfpathlineto{\pgfqpoint{5.235514in}{1.000676in}}%
\pgfpathlineto{\pgfqpoint{5.235898in}{0.992804in}}%
\pgfpathlineto{\pgfqpoint{5.236859in}{0.994977in}}%
\pgfpathlineto{\pgfqpoint{5.237243in}{0.995213in}}%
\pgfpathlineto{\pgfqpoint{5.238204in}{0.992360in}}%
\pgfpathlineto{\pgfqpoint{5.238396in}{0.992394in}}%
\pgfpathlineto{\pgfqpoint{5.238781in}{0.996069in}}%
\pgfpathlineto{\pgfqpoint{5.239549in}{0.994047in}}%
\pgfpathlineto{\pgfqpoint{5.239742in}{0.993235in}}%
\pgfpathlineto{\pgfqpoint{5.240126in}{0.995252in}}%
\pgfpathlineto{\pgfqpoint{5.240510in}{0.994315in}}%
\pgfpathlineto{\pgfqpoint{5.240895in}{0.999569in}}%
\pgfpathlineto{\pgfqpoint{5.242048in}{0.999354in}}%
\pgfpathlineto{\pgfqpoint{5.242240in}{0.997065in}}%
\pgfpathlineto{\pgfqpoint{5.242816in}{1.000720in}}%
\pgfpathlineto{\pgfqpoint{5.243008in}{1.003477in}}%
\pgfpathlineto{\pgfqpoint{5.243969in}{1.001807in}}%
\pgfpathlineto{\pgfqpoint{5.244738in}{1.005897in}}%
\pgfpathlineto{\pgfqpoint{5.245507in}{0.998991in}}%
\pgfpathlineto{\pgfqpoint{5.245891in}{1.000040in}}%
\pgfpathlineto{\pgfqpoint{5.246275in}{1.000849in}}%
\pgfpathlineto{\pgfqpoint{5.246467in}{0.999534in}}%
\pgfpathlineto{\pgfqpoint{5.246660in}{0.998448in}}%
\pgfpathlineto{\pgfqpoint{5.246852in}{1.000567in}}%
\pgfpathlineto{\pgfqpoint{5.247044in}{1.002992in}}%
\pgfpathlineto{\pgfqpoint{5.247620in}{0.996748in}}%
\pgfpathlineto{\pgfqpoint{5.247813in}{0.999276in}}%
\pgfpathlineto{\pgfqpoint{5.249158in}{0.990184in}}%
\pgfpathlineto{\pgfqpoint{5.249350in}{0.990708in}}%
\pgfpathlineto{\pgfqpoint{5.249734in}{0.995170in}}%
\pgfpathlineto{\pgfqpoint{5.250503in}{0.991860in}}%
\pgfpathlineto{\pgfqpoint{5.251656in}{0.984801in}}%
\pgfpathlineto{\pgfqpoint{5.251848in}{0.987486in}}%
\pgfpathlineto{\pgfqpoint{5.252425in}{0.992701in}}%
\pgfpathlineto{\pgfqpoint{5.253385in}{0.991966in}}%
\pgfpathlineto{\pgfqpoint{5.253578in}{0.990573in}}%
\pgfpathlineto{\pgfqpoint{5.254154in}{0.992775in}}%
\pgfpathlineto{\pgfqpoint{5.255691in}{0.999644in}}%
\pgfpathlineto{\pgfqpoint{5.255884in}{0.997574in}}%
\pgfpathlineto{\pgfqpoint{5.257229in}{0.990842in}}%
\pgfpathlineto{\pgfqpoint{5.257805in}{0.993353in}}%
\pgfpathlineto{\pgfqpoint{5.257613in}{0.990665in}}%
\pgfpathlineto{\pgfqpoint{5.257997in}{0.991914in}}%
\pgfpathlineto{\pgfqpoint{5.259150in}{0.985203in}}%
\pgfpathlineto{\pgfqpoint{5.259343in}{0.988322in}}%
\pgfpathlineto{\pgfqpoint{5.260111in}{0.995405in}}%
\pgfpathlineto{\pgfqpoint{5.260688in}{0.989897in}}%
\pgfpathlineto{\pgfqpoint{5.261264in}{0.985394in}}%
\pgfpathlineto{\pgfqpoint{5.262033in}{0.987753in}}%
\pgfpathlineto{\pgfqpoint{5.262417in}{0.988636in}}%
\pgfpathlineto{\pgfqpoint{5.263955in}{0.995607in}}%
\pgfpathlineto{\pgfqpoint{5.264339in}{0.994289in}}%
\pgfpathlineto{\pgfqpoint{5.266261in}{0.983259in}}%
\pgfpathlineto{\pgfqpoint{5.267606in}{0.990657in}}%
\pgfpathlineto{\pgfqpoint{5.267798in}{0.989370in}}%
\pgfpathlineto{\pgfqpoint{5.267990in}{0.989665in}}%
\pgfpathlineto{\pgfqpoint{5.268182in}{0.993346in}}%
\pgfpathlineto{\pgfqpoint{5.268759in}{0.986910in}}%
\pgfpathlineto{\pgfqpoint{5.270104in}{0.980779in}}%
\pgfpathlineto{\pgfqpoint{5.269335in}{0.988365in}}%
\pgfpathlineto{\pgfqpoint{5.270488in}{0.981034in}}%
\pgfpathlineto{\pgfqpoint{5.271834in}{0.985821in}}%
\pgfpathlineto{\pgfqpoint{5.272987in}{0.982295in}}%
\pgfpathlineto{\pgfqpoint{5.272410in}{0.986647in}}%
\pgfpathlineto{\pgfqpoint{5.273179in}{0.984421in}}%
\pgfpathlineto{\pgfqpoint{5.274332in}{0.992609in}}%
\pgfpathlineto{\pgfqpoint{5.275100in}{0.991924in}}%
\pgfpathlineto{\pgfqpoint{5.275293in}{0.990523in}}%
\pgfpathlineto{\pgfqpoint{5.275677in}{0.994225in}}%
\pgfpathlineto{\pgfqpoint{5.276061in}{0.996571in}}%
\pgfpathlineto{\pgfqpoint{5.276253in}{0.993926in}}%
\pgfpathlineto{\pgfqpoint{5.276638in}{0.988679in}}%
\pgfpathlineto{\pgfqpoint{5.277406in}{0.989269in}}%
\pgfpathlineto{\pgfqpoint{5.277599in}{0.992263in}}%
\pgfpathlineto{\pgfqpoint{5.278175in}{0.986917in}}%
\pgfpathlineto{\pgfqpoint{5.278367in}{0.988507in}}%
\pgfpathlineto{\pgfqpoint{5.278559in}{0.988937in}}%
\pgfpathlineto{\pgfqpoint{5.278752in}{0.988144in}}%
\pgfpathlineto{\pgfqpoint{5.278944in}{0.985310in}}%
\pgfpathlineto{\pgfqpoint{5.279712in}{0.989618in}}%
\pgfpathlineto{\pgfqpoint{5.280289in}{0.991339in}}%
\pgfpathlineto{\pgfqpoint{5.280097in}{0.987709in}}%
\pgfpathlineto{\pgfqpoint{5.280481in}{0.990412in}}%
\pgfpathlineto{\pgfqpoint{5.280673in}{0.988223in}}%
\pgfpathlineto{\pgfqpoint{5.281058in}{0.994932in}}%
\pgfpathlineto{\pgfqpoint{5.281250in}{0.994841in}}%
\pgfpathlineto{\pgfqpoint{5.281634in}{0.992738in}}%
\pgfpathlineto{\pgfqpoint{5.282018in}{0.995503in}}%
\pgfpathlineto{\pgfqpoint{5.282211in}{0.998829in}}%
\pgfpathlineto{\pgfqpoint{5.283171in}{0.996777in}}%
\pgfpathlineto{\pgfqpoint{5.284709in}{1.002456in}}%
\pgfpathlineto{\pgfqpoint{5.286054in}{0.996494in}}%
\pgfpathlineto{\pgfqpoint{5.286631in}{1.000402in}}%
\pgfpathlineto{\pgfqpoint{5.287015in}{0.997803in}}%
\pgfpathlineto{\pgfqpoint{5.287399in}{0.996714in}}%
\pgfpathlineto{\pgfqpoint{5.287591in}{0.997162in}}%
\pgfpathlineto{\pgfqpoint{5.288744in}{1.004910in}}%
\pgfpathlineto{\pgfqpoint{5.288937in}{1.003527in}}%
\pgfpathlineto{\pgfqpoint{5.290090in}{0.993478in}}%
\pgfpathlineto{\pgfqpoint{5.290474in}{0.995740in}}%
\pgfpathlineto{\pgfqpoint{5.290858in}{0.999657in}}%
\pgfpathlineto{\pgfqpoint{5.291435in}{0.993181in}}%
\pgfpathlineto{\pgfqpoint{5.293933in}{0.984694in}}%
\pgfpathlineto{\pgfqpoint{5.296431in}{0.997972in}}%
\pgfpathlineto{\pgfqpoint{5.296623in}{0.996938in}}%
\pgfpathlineto{\pgfqpoint{5.296815in}{1.000749in}}%
\pgfpathlineto{\pgfqpoint{5.297392in}{0.998628in}}%
\pgfpathlineto{\pgfqpoint{5.297584in}{0.999067in}}%
\pgfpathlineto{\pgfqpoint{5.298545in}{0.992360in}}%
\pgfpathlineto{\pgfqpoint{5.298737in}{0.994042in}}%
\pgfpathlineto{\pgfqpoint{5.299121in}{0.992920in}}%
\pgfpathlineto{\pgfqpoint{5.299314in}{0.994338in}}%
\pgfpathlineto{\pgfqpoint{5.299506in}{0.993163in}}%
\pgfpathlineto{\pgfqpoint{5.299890in}{0.998522in}}%
\pgfpathlineto{\pgfqpoint{5.300659in}{0.996843in}}%
\pgfpathlineto{\pgfqpoint{5.301043in}{0.997572in}}%
\pgfpathlineto{\pgfqpoint{5.301620in}{1.000629in}}%
\pgfpathlineto{\pgfqpoint{5.302773in}{0.993881in}}%
\pgfpathlineto{\pgfqpoint{5.302965in}{0.994392in}}%
\pgfpathlineto{\pgfqpoint{5.303349in}{0.995320in}}%
\pgfpathlineto{\pgfqpoint{5.303541in}{0.994691in}}%
\pgfpathlineto{\pgfqpoint{5.303926in}{0.992106in}}%
\pgfpathlineto{\pgfqpoint{5.303926in}{0.992106in}}%
\pgfusepath{stroke}%
\end{pgfscope}%
\begin{pgfscope}%
\pgfpathrectangle{\pgfqpoint{3.286364in}{0.660000in}}{\pgfqpoint{2.113636in}{2.100000in}}%
\pgfusepath{clip}%
\pgfsetroundcap%
\pgfsetroundjoin%
\pgfsetlinewidth{0.602250pt}%
\definecolor{currentstroke}{rgb}{0.301961,0.686275,0.290196}%
\pgfsetstrokecolor{currentstroke}%
\pgfsetdash{}{0pt}%
\pgfpathmoveto{\pgfqpoint{3.382438in}{1.679371in}}%
\pgfpathlineto{\pgfqpoint{3.383207in}{1.678765in}}%
\pgfpathlineto{\pgfqpoint{3.383783in}{1.684299in}}%
\pgfpathlineto{\pgfqpoint{3.384168in}{1.691674in}}%
\pgfpathlineto{\pgfqpoint{3.384936in}{1.687467in}}%
\pgfpathlineto{\pgfqpoint{3.385897in}{1.683278in}}%
\pgfpathlineto{\pgfqpoint{3.386089in}{1.686329in}}%
\pgfpathlineto{\pgfqpoint{3.386666in}{1.691804in}}%
\pgfpathlineto{\pgfqpoint{3.387050in}{1.687412in}}%
\pgfpathlineto{\pgfqpoint{3.390509in}{1.675899in}}%
\pgfpathlineto{\pgfqpoint{3.390893in}{1.678572in}}%
\pgfpathlineto{\pgfqpoint{3.391278in}{1.680735in}}%
\pgfpathlineto{\pgfqpoint{3.392046in}{1.679096in}}%
\pgfpathlineto{\pgfqpoint{3.392431in}{1.681446in}}%
\pgfpathlineto{\pgfqpoint{3.393776in}{1.676033in}}%
\pgfpathlineto{\pgfqpoint{3.394352in}{1.677875in}}%
\pgfpathlineto{\pgfqpoint{3.394545in}{1.676426in}}%
\pgfpathlineto{\pgfqpoint{3.395505in}{1.671726in}}%
\pgfpathlineto{\pgfqpoint{3.395121in}{1.678038in}}%
\pgfpathlineto{\pgfqpoint{3.395698in}{1.673124in}}%
\pgfpathlineto{\pgfqpoint{3.395890in}{1.674972in}}%
\pgfpathlineto{\pgfqpoint{3.396274in}{1.670860in}}%
\pgfpathlineto{\pgfqpoint{3.398388in}{1.654701in}}%
\pgfpathlineto{\pgfqpoint{3.398772in}{1.653402in}}%
\pgfpathlineto{\pgfqpoint{3.399925in}{1.663212in}}%
\pgfpathlineto{\pgfqpoint{3.400886in}{1.656615in}}%
\pgfpathlineto{\pgfqpoint{3.401463in}{1.657287in}}%
\pgfpathlineto{\pgfqpoint{3.402808in}{1.664926in}}%
\pgfpathlineto{\pgfqpoint{3.403384in}{1.662765in}}%
\pgfpathlineto{\pgfqpoint{3.403576in}{1.665950in}}%
\pgfpathlineto{\pgfqpoint{3.404345in}{1.665704in}}%
\pgfpathlineto{\pgfqpoint{3.404922in}{1.670095in}}%
\pgfpathlineto{\pgfqpoint{3.405883in}{1.658830in}}%
\pgfpathlineto{\pgfqpoint{3.406267in}{1.663114in}}%
\pgfpathlineto{\pgfqpoint{3.406459in}{1.663892in}}%
\pgfpathlineto{\pgfqpoint{3.406651in}{1.663772in}}%
\pgfpathlineto{\pgfqpoint{3.407228in}{1.659031in}}%
\pgfpathlineto{\pgfqpoint{3.407804in}{1.661666in}}%
\pgfpathlineto{\pgfqpoint{3.408381in}{1.666040in}}%
\pgfpathlineto{\pgfqpoint{3.408765in}{1.661203in}}%
\pgfpathlineto{\pgfqpoint{3.409149in}{1.660334in}}%
\pgfpathlineto{\pgfqpoint{3.410687in}{1.672695in}}%
\pgfpathlineto{\pgfqpoint{3.410879in}{1.672480in}}%
\pgfpathlineto{\pgfqpoint{3.411455in}{1.680050in}}%
\pgfpathlineto{\pgfqpoint{3.412416in}{1.678297in}}%
\pgfpathlineto{\pgfqpoint{3.412608in}{1.674117in}}%
\pgfpathlineto{\pgfqpoint{3.413377in}{1.679699in}}%
\pgfpathlineto{\pgfqpoint{3.413954in}{1.680474in}}%
\pgfpathlineto{\pgfqpoint{3.414146in}{1.677113in}}%
\pgfpathlineto{\pgfqpoint{3.414722in}{1.683073in}}%
\pgfpathlineto{\pgfqpoint{3.416067in}{1.688730in}}%
\pgfpathlineto{\pgfqpoint{3.417605in}{1.682802in}}%
\pgfpathlineto{\pgfqpoint{3.417797in}{1.681227in}}%
\pgfpathlineto{\pgfqpoint{3.418373in}{1.683809in}}%
\pgfpathlineto{\pgfqpoint{3.418566in}{1.684184in}}%
\pgfpathlineto{\pgfqpoint{3.418758in}{1.683827in}}%
\pgfpathlineto{\pgfqpoint{3.419911in}{1.679878in}}%
\pgfpathlineto{\pgfqpoint{3.420872in}{1.688152in}}%
\pgfpathlineto{\pgfqpoint{3.421064in}{1.684298in}}%
\pgfpathlineto{\pgfqpoint{3.422601in}{1.679406in}}%
\pgfpathlineto{\pgfqpoint{3.423562in}{1.685283in}}%
\pgfpathlineto{\pgfqpoint{3.423946in}{1.683221in}}%
\pgfpathlineto{\pgfqpoint{3.425099in}{1.677344in}}%
\pgfpathlineto{\pgfqpoint{3.425291in}{1.681430in}}%
\pgfpathlineto{\pgfqpoint{3.426252in}{1.678274in}}%
\pgfpathlineto{\pgfqpoint{3.427021in}{1.680778in}}%
\pgfpathlineto{\pgfqpoint{3.427213in}{1.679512in}}%
\pgfpathlineto{\pgfqpoint{3.428558in}{1.675438in}}%
\pgfpathlineto{\pgfqpoint{3.428943in}{1.680190in}}%
\pgfpathlineto{\pgfqpoint{3.429519in}{1.673889in}}%
\pgfpathlineto{\pgfqpoint{3.430864in}{1.686046in}}%
\pgfpathlineto{\pgfqpoint{3.431441in}{1.683880in}}%
\pgfpathlineto{\pgfqpoint{3.433363in}{1.665863in}}%
\pgfpathlineto{\pgfqpoint{3.433555in}{1.666068in}}%
\pgfpathlineto{\pgfqpoint{3.433747in}{1.668620in}}%
\pgfpathlineto{\pgfqpoint{3.434708in}{1.667150in}}%
\pgfpathlineto{\pgfqpoint{3.435669in}{1.675255in}}%
\pgfpathlineto{\pgfqpoint{3.436437in}{1.669960in}}%
\pgfpathlineto{\pgfqpoint{3.437014in}{1.667174in}}%
\pgfpathlineto{\pgfqpoint{3.437206in}{1.668405in}}%
\pgfpathlineto{\pgfqpoint{3.437782in}{1.665760in}}%
\pgfpathlineto{\pgfqpoint{3.438359in}{1.673362in}}%
\pgfpathlineto{\pgfqpoint{3.439320in}{1.669638in}}%
\pgfpathlineto{\pgfqpoint{3.439704in}{1.671032in}}%
\pgfpathlineto{\pgfqpoint{3.441049in}{1.676447in}}%
\pgfpathlineto{\pgfqpoint{3.441626in}{1.673281in}}%
\pgfpathlineto{\pgfqpoint{3.441818in}{1.673451in}}%
\pgfpathlineto{\pgfqpoint{3.443163in}{1.679513in}}%
\pgfpathlineto{\pgfqpoint{3.443355in}{1.679204in}}%
\pgfpathlineto{\pgfqpoint{3.445085in}{1.673370in}}%
\pgfpathlineto{\pgfqpoint{3.443932in}{1.679752in}}%
\pgfpathlineto{\pgfqpoint{3.445277in}{1.673749in}}%
\pgfpathlineto{\pgfqpoint{3.445661in}{1.676274in}}%
\pgfpathlineto{\pgfqpoint{3.446622in}{1.674243in}}%
\pgfpathlineto{\pgfqpoint{3.447391in}{1.665162in}}%
\pgfpathlineto{\pgfqpoint{3.447775in}{1.669373in}}%
\pgfpathlineto{\pgfqpoint{3.448928in}{1.676744in}}%
\pgfpathlineto{\pgfqpoint{3.449120in}{1.674765in}}%
\pgfpathlineto{\pgfqpoint{3.449697in}{1.675182in}}%
\pgfpathlineto{\pgfqpoint{3.450465in}{1.680173in}}%
\pgfpathlineto{\pgfqpoint{3.450850in}{1.678035in}}%
\pgfpathlineto{\pgfqpoint{3.452003in}{1.670350in}}%
\pgfpathlineto{\pgfqpoint{3.452195in}{1.673766in}}%
\pgfpathlineto{\pgfqpoint{3.453156in}{1.679133in}}%
\pgfpathlineto{\pgfqpoint{3.453348in}{1.678040in}}%
\pgfpathlineto{\pgfqpoint{3.456999in}{1.665211in}}%
\pgfpathlineto{\pgfqpoint{3.457191in}{1.667414in}}%
\pgfpathlineto{\pgfqpoint{3.457384in}{1.664705in}}%
\pgfpathlineto{\pgfqpoint{3.457960in}{1.665409in}}%
\pgfpathlineto{\pgfqpoint{3.459113in}{1.661652in}}%
\pgfpathlineto{\pgfqpoint{3.459882in}{1.660893in}}%
\pgfpathlineto{\pgfqpoint{3.460843in}{1.668900in}}%
\pgfpathlineto{\pgfqpoint{3.462572in}{1.659846in}}%
\pgfpathlineto{\pgfqpoint{3.462764in}{1.661575in}}%
\pgfpathlineto{\pgfqpoint{3.463725in}{1.669921in}}%
\pgfpathlineto{\pgfqpoint{3.464109in}{1.669881in}}%
\pgfpathlineto{\pgfqpoint{3.464302in}{1.666396in}}%
\pgfpathlineto{\pgfqpoint{3.465070in}{1.672575in}}%
\pgfpathlineto{\pgfqpoint{3.465647in}{1.671620in}}%
\pgfpathlineto{\pgfqpoint{3.466992in}{1.676153in}}%
\pgfpathlineto{\pgfqpoint{3.468145in}{1.679731in}}%
\pgfpathlineto{\pgfqpoint{3.468529in}{1.678902in}}%
\pgfpathlineto{\pgfqpoint{3.468721in}{1.679671in}}%
\pgfpathlineto{\pgfqpoint{3.468914in}{1.679107in}}%
\pgfpathlineto{\pgfqpoint{3.470259in}{1.669843in}}%
\pgfpathlineto{\pgfqpoint{3.470451in}{1.670037in}}%
\pgfpathlineto{\pgfqpoint{3.470643in}{1.670193in}}%
\pgfpathlineto{\pgfqpoint{3.471220in}{1.674073in}}%
\pgfpathlineto{\pgfqpoint{3.471796in}{1.671446in}}%
\pgfpathlineto{\pgfqpoint{3.472180in}{1.672412in}}%
\pgfpathlineto{\pgfqpoint{3.473141in}{1.675722in}}%
\pgfpathlineto{\pgfqpoint{3.472565in}{1.671279in}}%
\pgfpathlineto{\pgfqpoint{3.473333in}{1.674917in}}%
\pgfpathlineto{\pgfqpoint{3.474486in}{1.666167in}}%
\pgfpathlineto{\pgfqpoint{3.474871in}{1.666488in}}%
\pgfpathlineto{\pgfqpoint{3.475255in}{1.667838in}}%
\pgfpathlineto{\pgfqpoint{3.475447in}{1.664767in}}%
\pgfpathlineto{\pgfqpoint{3.475639in}{1.665235in}}%
\pgfpathlineto{\pgfqpoint{3.475832in}{1.663110in}}%
\pgfpathlineto{\pgfqpoint{3.476216in}{1.668259in}}%
\pgfpathlineto{\pgfqpoint{3.476408in}{1.671162in}}%
\pgfpathlineto{\pgfqpoint{3.476985in}{1.667306in}}%
\pgfpathlineto{\pgfqpoint{3.478522in}{1.657345in}}%
\pgfpathlineto{\pgfqpoint{3.478906in}{1.658971in}}%
\pgfpathlineto{\pgfqpoint{3.479675in}{1.653975in}}%
\pgfpathlineto{\pgfqpoint{3.480059in}{1.646463in}}%
\pgfpathlineto{\pgfqpoint{3.480636in}{1.650479in}}%
\pgfpathlineto{\pgfqpoint{3.481212in}{1.657286in}}%
\pgfpathlineto{\pgfqpoint{3.481789in}{1.654930in}}%
\pgfpathlineto{\pgfqpoint{3.481981in}{1.652957in}}%
\pgfpathlineto{\pgfqpoint{3.482558in}{1.658169in}}%
\pgfpathlineto{\pgfqpoint{3.483326in}{1.652277in}}%
\pgfpathlineto{\pgfqpoint{3.482942in}{1.658426in}}%
\pgfpathlineto{\pgfqpoint{3.483711in}{1.654852in}}%
\pgfpathlineto{\pgfqpoint{3.483903in}{1.659020in}}%
\pgfpathlineto{\pgfqpoint{3.484671in}{1.656429in}}%
\pgfpathlineto{\pgfqpoint{3.484864in}{1.654848in}}%
\pgfpathlineto{\pgfqpoint{3.485248in}{1.658553in}}%
\pgfpathlineto{\pgfqpoint{3.485824in}{1.662975in}}%
\pgfpathlineto{\pgfqpoint{3.486209in}{1.660560in}}%
\pgfpathlineto{\pgfqpoint{3.486593in}{1.658593in}}%
\pgfpathlineto{\pgfqpoint{3.487170in}{1.660370in}}%
\pgfpathlineto{\pgfqpoint{3.488707in}{1.663234in}}%
\pgfpathlineto{\pgfqpoint{3.489091in}{1.661991in}}%
\pgfpathlineto{\pgfqpoint{3.490052in}{1.654814in}}%
\pgfpathlineto{\pgfqpoint{3.490436in}{1.654897in}}%
\pgfpathlineto{\pgfqpoint{3.491397in}{1.658766in}}%
\pgfpathlineto{\pgfqpoint{3.491589in}{1.655834in}}%
\pgfpathlineto{\pgfqpoint{3.492166in}{1.658274in}}%
\pgfpathlineto{\pgfqpoint{3.492358in}{1.659131in}}%
\pgfpathlineto{\pgfqpoint{3.492742in}{1.656877in}}%
\pgfpathlineto{\pgfqpoint{3.493511in}{1.649117in}}%
\pgfpathlineto{\pgfqpoint{3.494856in}{1.649863in}}%
\pgfpathlineto{\pgfqpoint{3.495625in}{1.651075in}}%
\pgfpathlineto{\pgfqpoint{3.495433in}{1.648651in}}%
\pgfpathlineto{\pgfqpoint{3.495817in}{1.650241in}}%
\pgfpathlineto{\pgfqpoint{3.496586in}{1.644174in}}%
\pgfpathlineto{\pgfqpoint{3.496778in}{1.648184in}}%
\pgfpathlineto{\pgfqpoint{3.497162in}{1.651712in}}%
\pgfpathlineto{\pgfqpoint{3.497739in}{1.646562in}}%
\pgfpathlineto{\pgfqpoint{3.497931in}{1.646443in}}%
\pgfpathlineto{\pgfqpoint{3.498315in}{1.643237in}}%
\pgfpathlineto{\pgfqpoint{3.499276in}{1.643726in}}%
\pgfpathlineto{\pgfqpoint{3.500237in}{1.645296in}}%
\pgfpathlineto{\pgfqpoint{3.499853in}{1.643284in}}%
\pgfpathlineto{\pgfqpoint{3.500429in}{1.644836in}}%
\pgfpathlineto{\pgfqpoint{3.500621in}{1.641244in}}%
\pgfpathlineto{\pgfqpoint{3.501390in}{1.643610in}}%
\pgfpathlineto{\pgfqpoint{3.501966in}{1.644783in}}%
\pgfpathlineto{\pgfqpoint{3.502159in}{1.644157in}}%
\pgfpathlineto{\pgfqpoint{3.502927in}{1.640076in}}%
\pgfpathlineto{\pgfqpoint{3.503120in}{1.643474in}}%
\pgfpathlineto{\pgfqpoint{3.503504in}{1.647865in}}%
\pgfpathlineto{\pgfqpoint{3.504080in}{1.643380in}}%
\pgfpathlineto{\pgfqpoint{3.504465in}{1.640966in}}%
\pgfpathlineto{\pgfqpoint{3.505618in}{1.639420in}}%
\pgfpathlineto{\pgfqpoint{3.505810in}{1.644141in}}%
\pgfpathlineto{\pgfqpoint{3.506579in}{1.636056in}}%
\pgfpathlineto{\pgfqpoint{3.507155in}{1.644558in}}%
\pgfpathlineto{\pgfqpoint{3.507924in}{1.638658in}}%
\pgfpathlineto{\pgfqpoint{3.508500in}{1.632417in}}%
\pgfpathlineto{\pgfqpoint{3.509461in}{1.633729in}}%
\pgfpathlineto{\pgfqpoint{3.509845in}{1.632062in}}%
\pgfpathlineto{\pgfqpoint{3.510038in}{1.633647in}}%
\pgfpathlineto{\pgfqpoint{3.510806in}{1.635899in}}%
\pgfpathlineto{\pgfqpoint{3.511191in}{1.633955in}}%
\pgfpathlineto{\pgfqpoint{3.512344in}{1.629680in}}%
\pgfpathlineto{\pgfqpoint{3.511575in}{1.634608in}}%
\pgfpathlineto{\pgfqpoint{3.512728in}{1.631726in}}%
\pgfpathlineto{\pgfqpoint{3.512920in}{1.632528in}}%
\pgfpathlineto{\pgfqpoint{3.513112in}{1.630319in}}%
\pgfpathlineto{\pgfqpoint{3.514457in}{1.620920in}}%
\pgfpathlineto{\pgfqpoint{3.514650in}{1.621850in}}%
\pgfpathlineto{\pgfqpoint{3.514842in}{1.624269in}}%
\pgfpathlineto{\pgfqpoint{3.515418in}{1.619617in}}%
\pgfpathlineto{\pgfqpoint{3.515610in}{1.621878in}}%
\pgfpathlineto{\pgfqpoint{3.515995in}{1.622297in}}%
\pgfpathlineto{\pgfqpoint{3.516956in}{1.615295in}}%
\pgfpathlineto{\pgfqpoint{3.517340in}{1.618041in}}%
\pgfpathlineto{\pgfqpoint{3.517532in}{1.620353in}}%
\pgfpathlineto{\pgfqpoint{3.518109in}{1.617684in}}%
\pgfpathlineto{\pgfqpoint{3.518301in}{1.618876in}}%
\pgfpathlineto{\pgfqpoint{3.519262in}{1.613811in}}%
\pgfpathlineto{\pgfqpoint{3.519454in}{1.616222in}}%
\pgfpathlineto{\pgfqpoint{3.520222in}{1.620907in}}%
\pgfpathlineto{\pgfqpoint{3.520415in}{1.618908in}}%
\pgfpathlineto{\pgfqpoint{3.520991in}{1.619218in}}%
\pgfpathlineto{\pgfqpoint{3.521760in}{1.610445in}}%
\pgfpathlineto{\pgfqpoint{3.521952in}{1.613645in}}%
\pgfpathlineto{\pgfqpoint{3.522336in}{1.609408in}}%
\pgfpathlineto{\pgfqpoint{3.522913in}{1.612578in}}%
\pgfpathlineto{\pgfqpoint{3.523297in}{1.612059in}}%
\pgfpathlineto{\pgfqpoint{3.523874in}{1.607287in}}%
\pgfpathlineto{\pgfqpoint{3.524258in}{1.610353in}}%
\pgfpathlineto{\pgfqpoint{3.525219in}{1.615066in}}%
\pgfpathlineto{\pgfqpoint{3.525603in}{1.612254in}}%
\pgfpathlineto{\pgfqpoint{3.526180in}{1.608992in}}%
\pgfpathlineto{\pgfqpoint{3.526564in}{1.611759in}}%
\pgfpathlineto{\pgfqpoint{3.526756in}{1.613397in}}%
\pgfpathlineto{\pgfqpoint{3.527333in}{1.612009in}}%
\pgfpathlineto{\pgfqpoint{3.527909in}{1.609390in}}%
\pgfpathlineto{\pgfqpoint{3.528294in}{1.612711in}}%
\pgfpathlineto{\pgfqpoint{3.528870in}{1.620065in}}%
\pgfpathlineto{\pgfqpoint{3.529447in}{1.616668in}}%
\pgfpathlineto{\pgfqpoint{3.529639in}{1.614102in}}%
\pgfpathlineto{\pgfqpoint{3.530023in}{1.620740in}}%
\pgfpathlineto{\pgfqpoint{3.530215in}{1.621761in}}%
\pgfpathlineto{\pgfqpoint{3.530407in}{1.619065in}}%
\pgfpathlineto{\pgfqpoint{3.530600in}{1.617428in}}%
\pgfpathlineto{\pgfqpoint{3.530792in}{1.621546in}}%
\pgfpathlineto{\pgfqpoint{3.530984in}{1.621189in}}%
\pgfpathlineto{\pgfqpoint{3.531560in}{1.624267in}}%
\pgfpathlineto{\pgfqpoint{3.531753in}{1.620375in}}%
\pgfpathlineto{\pgfqpoint{3.532906in}{1.611248in}}%
\pgfpathlineto{\pgfqpoint{3.533290in}{1.613090in}}%
\pgfpathlineto{\pgfqpoint{3.534251in}{1.617394in}}%
\pgfpathlineto{\pgfqpoint{3.534635in}{1.612076in}}%
\pgfpathlineto{\pgfqpoint{3.535404in}{1.614563in}}%
\pgfpathlineto{\pgfqpoint{3.536172in}{1.621442in}}%
\pgfpathlineto{\pgfqpoint{3.536557in}{1.618912in}}%
\pgfpathlineto{\pgfqpoint{3.536749in}{1.616757in}}%
\pgfpathlineto{\pgfqpoint{3.537133in}{1.619084in}}%
\pgfpathlineto{\pgfqpoint{3.538286in}{1.629334in}}%
\pgfpathlineto{\pgfqpoint{3.538478in}{1.629308in}}%
\pgfpathlineto{\pgfqpoint{3.538671in}{1.627723in}}%
\pgfpathlineto{\pgfqpoint{3.539055in}{1.631646in}}%
\pgfpathlineto{\pgfqpoint{3.540016in}{1.635835in}}%
\pgfpathlineto{\pgfqpoint{3.539824in}{1.631490in}}%
\pgfpathlineto{\pgfqpoint{3.540208in}{1.632016in}}%
\pgfpathlineto{\pgfqpoint{3.541169in}{1.623904in}}%
\pgfpathlineto{\pgfqpoint{3.541745in}{1.624296in}}%
\pgfpathlineto{\pgfqpoint{3.543090in}{1.633517in}}%
\pgfpathlineto{\pgfqpoint{3.543283in}{1.630959in}}%
\pgfpathlineto{\pgfqpoint{3.549240in}{1.595343in}}%
\pgfpathlineto{\pgfqpoint{3.550201in}{1.595595in}}%
\pgfpathlineto{\pgfqpoint{3.550777in}{1.601835in}}%
\pgfpathlineto{\pgfqpoint{3.551354in}{1.599141in}}%
\pgfpathlineto{\pgfqpoint{3.552315in}{1.594221in}}%
\pgfpathlineto{\pgfqpoint{3.552507in}{1.597233in}}%
\pgfpathlineto{\pgfqpoint{3.552699in}{1.596337in}}%
\pgfpathlineto{\pgfqpoint{3.553083in}{1.597976in}}%
\pgfpathlineto{\pgfqpoint{3.553468in}{1.603208in}}%
\pgfpathlineto{\pgfqpoint{3.554044in}{1.597069in}}%
\pgfpathlineto{\pgfqpoint{3.554428in}{1.596580in}}%
\pgfpathlineto{\pgfqpoint{3.555774in}{1.606784in}}%
\pgfpathlineto{\pgfqpoint{3.556350in}{1.612408in}}%
\pgfpathlineto{\pgfqpoint{3.556927in}{1.608339in}}%
\pgfpathlineto{\pgfqpoint{3.559040in}{1.600448in}}%
\pgfpathlineto{\pgfqpoint{3.559233in}{1.601341in}}%
\pgfpathlineto{\pgfqpoint{3.559425in}{1.595749in}}%
\pgfpathlineto{\pgfqpoint{3.560386in}{1.597232in}}%
\pgfpathlineto{\pgfqpoint{3.560962in}{1.592514in}}%
\pgfpathlineto{\pgfqpoint{3.561539in}{1.596271in}}%
\pgfpathlineto{\pgfqpoint{3.562115in}{1.600155in}}%
\pgfpathlineto{\pgfqpoint{3.562884in}{1.597524in}}%
\pgfpathlineto{\pgfqpoint{3.564229in}{1.589283in}}%
\pgfpathlineto{\pgfqpoint{3.565190in}{1.592113in}}%
\pgfpathlineto{\pgfqpoint{3.565382in}{1.592055in}}%
\pgfpathlineto{\pgfqpoint{3.566151in}{1.584965in}}%
\pgfpathlineto{\pgfqpoint{3.566535in}{1.588207in}}%
\pgfpathlineto{\pgfqpoint{3.566727in}{1.589353in}}%
\pgfpathlineto{\pgfqpoint{3.567111in}{1.586565in}}%
\pgfpathlineto{\pgfqpoint{3.567880in}{1.579744in}}%
\pgfpathlineto{\pgfqpoint{3.568264in}{1.583958in}}%
\pgfpathlineto{\pgfqpoint{3.569417in}{1.586417in}}%
\pgfpathlineto{\pgfqpoint{3.569033in}{1.583428in}}%
\pgfpathlineto{\pgfqpoint{3.569610in}{1.584693in}}%
\pgfpathlineto{\pgfqpoint{3.569802in}{1.584665in}}%
\pgfpathlineto{\pgfqpoint{3.571531in}{1.575805in}}%
\pgfpathlineto{\pgfqpoint{3.570378in}{1.588058in}}%
\pgfpathlineto{\pgfqpoint{3.571723in}{1.577001in}}%
\pgfpathlineto{\pgfqpoint{3.571916in}{1.576501in}}%
\pgfpathlineto{\pgfqpoint{3.572108in}{1.577422in}}%
\pgfpathlineto{\pgfqpoint{3.572492in}{1.579994in}}%
\pgfpathlineto{\pgfqpoint{3.572876in}{1.578063in}}%
\pgfpathlineto{\pgfqpoint{3.573261in}{1.572998in}}%
\pgfpathlineto{\pgfqpoint{3.573837in}{1.576435in}}%
\pgfpathlineto{\pgfqpoint{3.575182in}{1.585102in}}%
\pgfpathlineto{\pgfqpoint{3.575951in}{1.584103in}}%
\pgfpathlineto{\pgfqpoint{3.577296in}{1.582066in}}%
\pgfpathlineto{\pgfqpoint{3.577489in}{1.582305in}}%
\pgfpathlineto{\pgfqpoint{3.577681in}{1.581949in}}%
\pgfpathlineto{\pgfqpoint{3.578065in}{1.580239in}}%
\pgfpathlineto{\pgfqpoint{3.578642in}{1.580924in}}%
\pgfpathlineto{\pgfqpoint{3.578834in}{1.582263in}}%
\pgfpathlineto{\pgfqpoint{3.579410in}{1.580581in}}%
\pgfpathlineto{\pgfqpoint{3.580563in}{1.575879in}}%
\pgfpathlineto{\pgfqpoint{3.580755in}{1.577718in}}%
\pgfpathlineto{\pgfqpoint{3.581908in}{1.573863in}}%
\pgfpathlineto{\pgfqpoint{3.584407in}{1.566080in}}%
\pgfpathlineto{\pgfqpoint{3.586328in}{1.577244in}}%
\pgfpathlineto{\pgfqpoint{3.586520in}{1.576397in}}%
\pgfpathlineto{\pgfqpoint{3.586905in}{1.570099in}}%
\pgfpathlineto{\pgfqpoint{3.587673in}{1.574362in}}%
\pgfpathlineto{\pgfqpoint{3.588634in}{1.579739in}}%
\pgfpathlineto{\pgfqpoint{3.588250in}{1.572610in}}%
\pgfpathlineto{\pgfqpoint{3.589019in}{1.578996in}}%
\pgfpathlineto{\pgfqpoint{3.589979in}{1.580788in}}%
\pgfpathlineto{\pgfqpoint{3.590364in}{1.575780in}}%
\pgfpathlineto{\pgfqpoint{3.592093in}{1.584469in}}%
\pgfpathlineto{\pgfqpoint{3.592285in}{1.584108in}}%
\pgfpathlineto{\pgfqpoint{3.592862in}{1.580135in}}%
\pgfpathlineto{\pgfqpoint{3.593246in}{1.585463in}}%
\pgfpathlineto{\pgfqpoint{3.593438in}{1.585778in}}%
\pgfpathlineto{\pgfqpoint{3.593631in}{1.584441in}}%
\pgfpathlineto{\pgfqpoint{3.593823in}{1.583175in}}%
\pgfpathlineto{\pgfqpoint{3.594399in}{1.586180in}}%
\pgfpathlineto{\pgfqpoint{3.596129in}{1.597485in}}%
\pgfpathlineto{\pgfqpoint{3.596321in}{1.597831in}}%
\pgfpathlineto{\pgfqpoint{3.596513in}{1.594330in}}%
\pgfpathlineto{\pgfqpoint{3.597090in}{1.598262in}}%
\pgfpathlineto{\pgfqpoint{3.597282in}{1.596523in}}%
\pgfpathlineto{\pgfqpoint{3.598243in}{1.598094in}}%
\pgfpathlineto{\pgfqpoint{3.600164in}{1.589189in}}%
\pgfpathlineto{\pgfqpoint{3.600357in}{1.590911in}}%
\pgfpathlineto{\pgfqpoint{3.600549in}{1.587289in}}%
\pgfpathlineto{\pgfqpoint{3.601125in}{1.589944in}}%
\pgfpathlineto{\pgfqpoint{3.602663in}{1.586529in}}%
\pgfpathlineto{\pgfqpoint{3.602855in}{1.589621in}}%
\pgfpathlineto{\pgfqpoint{3.603239in}{1.585720in}}%
\pgfpathlineto{\pgfqpoint{3.603623in}{1.585942in}}%
\pgfpathlineto{\pgfqpoint{3.604776in}{1.577602in}}%
\pgfpathlineto{\pgfqpoint{3.605161in}{1.578948in}}%
\pgfpathlineto{\pgfqpoint{3.605353in}{1.580950in}}%
\pgfpathlineto{\pgfqpoint{3.605929in}{1.575642in}}%
\pgfpathlineto{\pgfqpoint{3.606314in}{1.569729in}}%
\pgfpathlineto{\pgfqpoint{3.606890in}{1.578312in}}%
\pgfpathlineto{\pgfqpoint{3.608428in}{1.567064in}}%
\pgfpathlineto{\pgfqpoint{3.608620in}{1.565205in}}%
\pgfpathlineto{\pgfqpoint{3.608812in}{1.568040in}}%
\pgfpathlineto{\pgfqpoint{3.610157in}{1.575033in}}%
\pgfpathlineto{\pgfqpoint{3.611694in}{1.581837in}}%
\pgfpathlineto{\pgfqpoint{3.611887in}{1.579469in}}%
\pgfpathlineto{\pgfqpoint{3.612271in}{1.583583in}}%
\pgfpathlineto{\pgfqpoint{3.612655in}{1.582942in}}%
\pgfpathlineto{\pgfqpoint{3.614385in}{1.576848in}}%
\pgfpathlineto{\pgfqpoint{3.614577in}{1.577159in}}%
\pgfpathlineto{\pgfqpoint{3.614961in}{1.575535in}}%
\pgfpathlineto{\pgfqpoint{3.615346in}{1.577384in}}%
\pgfpathlineto{\pgfqpoint{3.616306in}{1.580769in}}%
\pgfpathlineto{\pgfqpoint{3.616499in}{1.580716in}}%
\pgfpathlineto{\pgfqpoint{3.617844in}{1.570267in}}%
\pgfpathlineto{\pgfqpoint{3.618228in}{1.576091in}}%
\pgfpathlineto{\pgfqpoint{3.618612in}{1.574331in}}%
\pgfpathlineto{\pgfqpoint{3.619381in}{1.575320in}}%
\pgfpathlineto{\pgfqpoint{3.623609in}{1.599836in}}%
\pgfpathlineto{\pgfqpoint{3.623801in}{1.599801in}}%
\pgfpathlineto{\pgfqpoint{3.624570in}{1.595204in}}%
\pgfpathlineto{\pgfqpoint{3.624185in}{1.600251in}}%
\pgfpathlineto{\pgfqpoint{3.624954in}{1.598837in}}%
\pgfpathlineto{\pgfqpoint{3.625723in}{1.597579in}}%
\pgfpathlineto{\pgfqpoint{3.626107in}{1.601911in}}%
\pgfpathlineto{\pgfqpoint{3.626299in}{1.598727in}}%
\pgfpathlineto{\pgfqpoint{3.627260in}{1.600904in}}%
\pgfpathlineto{\pgfqpoint{3.628990in}{1.611944in}}%
\pgfpathlineto{\pgfqpoint{3.630335in}{1.605089in}}%
\pgfpathlineto{\pgfqpoint{3.630527in}{1.606066in}}%
\pgfpathlineto{\pgfqpoint{3.630911in}{1.604743in}}%
\pgfpathlineto{\pgfqpoint{3.631872in}{1.599923in}}%
\pgfpathlineto{\pgfqpoint{3.632064in}{1.603028in}}%
\pgfpathlineto{\pgfqpoint{3.632449in}{1.602050in}}%
\pgfpathlineto{\pgfqpoint{3.633602in}{1.605760in}}%
\pgfpathlineto{\pgfqpoint{3.634178in}{1.600410in}}%
\pgfpathlineto{\pgfqpoint{3.634755in}{1.603731in}}%
\pgfpathlineto{\pgfqpoint{3.634947in}{1.604365in}}%
\pgfpathlineto{\pgfqpoint{3.635523in}{1.597128in}}%
\pgfpathlineto{\pgfqpoint{3.636100in}{1.598946in}}%
\pgfpathlineto{\pgfqpoint{3.636676in}{1.604931in}}%
\pgfpathlineto{\pgfqpoint{3.637253in}{1.601432in}}%
\pgfpathlineto{\pgfqpoint{3.637445in}{1.601603in}}%
\pgfpathlineto{\pgfqpoint{3.638406in}{1.609921in}}%
\pgfpathlineto{\pgfqpoint{3.638790in}{1.609404in}}%
\pgfpathlineto{\pgfqpoint{3.638982in}{1.610505in}}%
\pgfpathlineto{\pgfqpoint{3.639174in}{1.607773in}}%
\pgfpathlineto{\pgfqpoint{3.639943in}{1.603658in}}%
\pgfpathlineto{\pgfqpoint{3.640327in}{1.605924in}}%
\pgfpathlineto{\pgfqpoint{3.640520in}{1.606094in}}%
\pgfpathlineto{\pgfqpoint{3.641480in}{1.609128in}}%
\pgfpathlineto{\pgfqpoint{3.641865in}{1.601987in}}%
\pgfpathlineto{\pgfqpoint{3.642633in}{1.603327in}}%
\pgfpathlineto{\pgfqpoint{3.642826in}{1.606788in}}%
\pgfpathlineto{\pgfqpoint{3.643402in}{1.599655in}}%
\pgfpathlineto{\pgfqpoint{3.643594in}{1.600505in}}%
\pgfpathlineto{\pgfqpoint{3.643786in}{1.598701in}}%
\pgfpathlineto{\pgfqpoint{3.643979in}{1.599372in}}%
\pgfpathlineto{\pgfqpoint{3.644171in}{1.596603in}}%
\pgfpathlineto{\pgfqpoint{3.644747in}{1.604240in}}%
\pgfpathlineto{\pgfqpoint{3.645132in}{1.606514in}}%
\pgfpathlineto{\pgfqpoint{3.645900in}{1.601805in}}%
\pgfpathlineto{\pgfqpoint{3.646092in}{1.603713in}}%
\pgfpathlineto{\pgfqpoint{3.646477in}{1.599357in}}%
\pgfpathlineto{\pgfqpoint{3.646669in}{1.599730in}}%
\pgfpathlineto{\pgfqpoint{3.646861in}{1.599558in}}%
\pgfpathlineto{\pgfqpoint{3.647053in}{1.595647in}}%
\pgfpathlineto{\pgfqpoint{3.648014in}{1.598327in}}%
\pgfpathlineto{\pgfqpoint{3.648206in}{1.596544in}}%
\pgfpathlineto{\pgfqpoint{3.648783in}{1.600693in}}%
\pgfpathlineto{\pgfqpoint{3.648975in}{1.601402in}}%
\pgfpathlineto{\pgfqpoint{3.649359in}{1.599911in}}%
\pgfpathlineto{\pgfqpoint{3.650128in}{1.594892in}}%
\pgfpathlineto{\pgfqpoint{3.650512in}{1.597566in}}%
\pgfpathlineto{\pgfqpoint{3.650705in}{1.597412in}}%
\pgfpathlineto{\pgfqpoint{3.651858in}{1.592769in}}%
\pgfpathlineto{\pgfqpoint{3.652050in}{1.593374in}}%
\pgfpathlineto{\pgfqpoint{3.653395in}{1.599133in}}%
\pgfpathlineto{\pgfqpoint{3.653587in}{1.596568in}}%
\pgfpathlineto{\pgfqpoint{3.654548in}{1.589336in}}%
\pgfpathlineto{\pgfqpoint{3.654740in}{1.593506in}}%
\pgfpathlineto{\pgfqpoint{3.655124in}{1.594211in}}%
\pgfpathlineto{\pgfqpoint{3.656854in}{1.585103in}}%
\pgfpathlineto{\pgfqpoint{3.657046in}{1.587343in}}%
\pgfpathlineto{\pgfqpoint{3.657815in}{1.584832in}}%
\pgfpathlineto{\pgfqpoint{3.658391in}{1.582008in}}%
\pgfpathlineto{\pgfqpoint{3.658776in}{1.584693in}}%
\pgfpathlineto{\pgfqpoint{3.658968in}{1.584883in}}%
\pgfpathlineto{\pgfqpoint{3.660121in}{1.595263in}}%
\pgfpathlineto{\pgfqpoint{3.660313in}{1.594218in}}%
\pgfpathlineto{\pgfqpoint{3.661658in}{1.589265in}}%
\pgfpathlineto{\pgfqpoint{3.661850in}{1.590587in}}%
\pgfpathlineto{\pgfqpoint{3.662427in}{1.589041in}}%
\pgfpathlineto{\pgfqpoint{3.662619in}{1.589433in}}%
\pgfpathlineto{\pgfqpoint{3.663195in}{1.588856in}}%
\pgfpathlineto{\pgfqpoint{3.663388in}{1.590474in}}%
\pgfpathlineto{\pgfqpoint{3.663580in}{1.591255in}}%
\pgfpathlineto{\pgfqpoint{3.663772in}{1.589348in}}%
\pgfpathlineto{\pgfqpoint{3.665309in}{1.583183in}}%
\pgfpathlineto{\pgfqpoint{3.666847in}{1.575776in}}%
\pgfpathlineto{\pgfqpoint{3.667231in}{1.578455in}}%
\pgfpathlineto{\pgfqpoint{3.668960in}{1.585566in}}%
\pgfpathlineto{\pgfqpoint{3.670498in}{1.573918in}}%
\pgfpathlineto{\pgfqpoint{3.670882in}{1.577221in}}%
\pgfpathlineto{\pgfqpoint{3.671266in}{1.577619in}}%
\pgfpathlineto{\pgfqpoint{3.671459in}{1.574670in}}%
\pgfpathlineto{\pgfqpoint{3.672227in}{1.574752in}}%
\pgfpathlineto{\pgfqpoint{3.672419in}{1.577425in}}%
\pgfpathlineto{\pgfqpoint{3.673188in}{1.574589in}}%
\pgfpathlineto{\pgfqpoint{3.673380in}{1.576791in}}%
\pgfpathlineto{\pgfqpoint{3.673573in}{1.577140in}}%
\pgfpathlineto{\pgfqpoint{3.673765in}{1.576475in}}%
\pgfpathlineto{\pgfqpoint{3.673957in}{1.576797in}}%
\pgfpathlineto{\pgfqpoint{3.675110in}{1.567032in}}%
\pgfpathlineto{\pgfqpoint{3.675302in}{1.569166in}}%
\pgfpathlineto{\pgfqpoint{3.675494in}{1.570217in}}%
\pgfpathlineto{\pgfqpoint{3.675686in}{1.569328in}}%
\pgfpathlineto{\pgfqpoint{3.675879in}{1.564775in}}%
\pgfpathlineto{\pgfqpoint{3.676839in}{1.565632in}}%
\pgfpathlineto{\pgfqpoint{3.677032in}{1.566163in}}%
\pgfpathlineto{\pgfqpoint{3.677224in}{1.565447in}}%
\pgfpathlineto{\pgfqpoint{3.677800in}{1.562452in}}%
\pgfpathlineto{\pgfqpoint{3.678185in}{1.565547in}}%
\pgfpathlineto{\pgfqpoint{3.678377in}{1.566300in}}%
\pgfpathlineto{\pgfqpoint{3.678761in}{1.564248in}}%
\pgfpathlineto{\pgfqpoint{3.678953in}{1.565829in}}%
\pgfpathlineto{\pgfqpoint{3.679145in}{1.562493in}}%
\pgfpathlineto{\pgfqpoint{3.679914in}{1.567185in}}%
\pgfpathlineto{\pgfqpoint{3.681067in}{1.562097in}}%
\pgfpathlineto{\pgfqpoint{3.681644in}{1.558217in}}%
\pgfpathlineto{\pgfqpoint{3.682220in}{1.561060in}}%
\pgfpathlineto{\pgfqpoint{3.683181in}{1.559301in}}%
\pgfpathlineto{\pgfqpoint{3.684334in}{1.552232in}}%
\pgfpathlineto{\pgfqpoint{3.685679in}{1.557277in}}%
\pgfpathlineto{\pgfqpoint{3.685871in}{1.559100in}}%
\pgfpathlineto{\pgfqpoint{3.686256in}{1.552976in}}%
\pgfpathlineto{\pgfqpoint{3.686640in}{1.557923in}}%
\pgfpathlineto{\pgfqpoint{3.686832in}{1.557519in}}%
\pgfpathlineto{\pgfqpoint{3.687024in}{1.558028in}}%
\pgfpathlineto{\pgfqpoint{3.687985in}{1.567653in}}%
\pgfpathlineto{\pgfqpoint{3.689138in}{1.567271in}}%
\pgfpathlineto{\pgfqpoint{3.689330in}{1.568457in}}%
\pgfpathlineto{\pgfqpoint{3.689522in}{1.566756in}}%
\pgfpathlineto{\pgfqpoint{3.689907in}{1.567945in}}%
\pgfpathlineto{\pgfqpoint{3.690099in}{1.565508in}}%
\pgfpathlineto{\pgfqpoint{3.691060in}{1.566954in}}%
\pgfpathlineto{\pgfqpoint{3.692597in}{1.577077in}}%
\pgfpathlineto{\pgfqpoint{3.693174in}{1.575633in}}%
\pgfpathlineto{\pgfqpoint{3.693366in}{1.572788in}}%
\pgfpathlineto{\pgfqpoint{3.693558in}{1.576084in}}%
\pgfpathlineto{\pgfqpoint{3.694327in}{1.573476in}}%
\pgfpathlineto{\pgfqpoint{3.695095in}{1.581995in}}%
\pgfpathlineto{\pgfqpoint{3.695480in}{1.575571in}}%
\pgfpathlineto{\pgfqpoint{3.695672in}{1.574554in}}%
\pgfpathlineto{\pgfqpoint{3.696056in}{1.576918in}}%
\pgfpathlineto{\pgfqpoint{3.696633in}{1.578603in}}%
\pgfpathlineto{\pgfqpoint{3.696825in}{1.576464in}}%
\pgfpathlineto{\pgfqpoint{3.697017in}{1.575009in}}%
\pgfpathlineto{\pgfqpoint{3.697593in}{1.578254in}}%
\pgfpathlineto{\pgfqpoint{3.699707in}{1.586584in}}%
\pgfpathlineto{\pgfqpoint{3.700092in}{1.584209in}}%
\pgfpathlineto{\pgfqpoint{3.700476in}{1.584778in}}%
\pgfpathlineto{\pgfqpoint{3.700668in}{1.583410in}}%
\pgfpathlineto{\pgfqpoint{3.700860in}{1.579730in}}%
\pgfpathlineto{\pgfqpoint{3.701821in}{1.581855in}}%
\pgfpathlineto{\pgfqpoint{3.703166in}{1.588718in}}%
\pgfpathlineto{\pgfqpoint{3.703359in}{1.588521in}}%
\pgfpathlineto{\pgfqpoint{3.704512in}{1.579763in}}%
\pgfpathlineto{\pgfqpoint{3.704704in}{1.583291in}}%
\pgfpathlineto{\pgfqpoint{3.704896in}{1.583047in}}%
\pgfpathlineto{\pgfqpoint{3.705665in}{1.577669in}}%
\pgfpathlineto{\pgfqpoint{3.705857in}{1.578421in}}%
\pgfpathlineto{\pgfqpoint{3.707202in}{1.591485in}}%
\pgfpathlineto{\pgfqpoint{3.707394in}{1.590890in}}%
\pgfpathlineto{\pgfqpoint{3.708355in}{1.595316in}}%
\pgfpathlineto{\pgfqpoint{3.708739in}{1.592229in}}%
\pgfpathlineto{\pgfqpoint{3.708931in}{1.591476in}}%
\pgfpathlineto{\pgfqpoint{3.709124in}{1.594540in}}%
\pgfpathlineto{\pgfqpoint{3.709316in}{1.593804in}}%
\pgfpathlineto{\pgfqpoint{3.712390in}{1.613089in}}%
\pgfpathlineto{\pgfqpoint{3.713543in}{1.608827in}}%
\pgfpathlineto{\pgfqpoint{3.713736in}{1.611080in}}%
\pgfpathlineto{\pgfqpoint{3.714312in}{1.605143in}}%
\pgfpathlineto{\pgfqpoint{3.714696in}{1.606819in}}%
\pgfpathlineto{\pgfqpoint{3.715081in}{1.604428in}}%
\pgfpathlineto{\pgfqpoint{3.715849in}{1.606076in}}%
\pgfpathlineto{\pgfqpoint{3.716234in}{1.600155in}}%
\pgfpathlineto{\pgfqpoint{3.716426in}{1.602549in}}%
\pgfpathlineto{\pgfqpoint{3.717002in}{1.598595in}}%
\pgfpathlineto{\pgfqpoint{3.717195in}{1.600756in}}%
\pgfpathlineto{\pgfqpoint{3.717387in}{1.600436in}}%
\pgfpathlineto{\pgfqpoint{3.717771in}{1.601696in}}%
\pgfpathlineto{\pgfqpoint{3.718155in}{1.601428in}}%
\pgfpathlineto{\pgfqpoint{3.718348in}{1.602590in}}%
\pgfpathlineto{\pgfqpoint{3.718540in}{1.603909in}}%
\pgfpathlineto{\pgfqpoint{3.719116in}{1.600492in}}%
\pgfpathlineto{\pgfqpoint{3.719693in}{1.604582in}}%
\pgfpathlineto{\pgfqpoint{3.720077in}{1.601296in}}%
\pgfpathlineto{\pgfqpoint{3.720269in}{1.599186in}}%
\pgfpathlineto{\pgfqpoint{3.720846in}{1.604467in}}%
\pgfpathlineto{\pgfqpoint{3.721038in}{1.604110in}}%
\pgfpathlineto{\pgfqpoint{3.721614in}{1.606665in}}%
\pgfpathlineto{\pgfqpoint{3.721999in}{1.602868in}}%
\pgfpathlineto{\pgfqpoint{3.722575in}{1.603885in}}%
\pgfpathlineto{\pgfqpoint{3.722768in}{1.602445in}}%
\pgfpathlineto{\pgfqpoint{3.723152in}{1.600396in}}%
\pgfpathlineto{\pgfqpoint{3.723728in}{1.602686in}}%
\pgfpathlineto{\pgfqpoint{3.723921in}{1.603007in}}%
\pgfpathlineto{\pgfqpoint{3.724305in}{1.606306in}}%
\pgfpathlineto{\pgfqpoint{3.724689in}{1.601375in}}%
\pgfpathlineto{\pgfqpoint{3.724881in}{1.601913in}}%
\pgfpathlineto{\pgfqpoint{3.725074in}{1.601615in}}%
\pgfpathlineto{\pgfqpoint{3.725266in}{1.598922in}}%
\pgfpathlineto{\pgfqpoint{3.725842in}{1.601307in}}%
\pgfpathlineto{\pgfqpoint{3.727187in}{1.607258in}}%
\pgfpathlineto{\pgfqpoint{3.727380in}{1.604029in}}%
\pgfpathlineto{\pgfqpoint{3.727956in}{1.609827in}}%
\pgfpathlineto{\pgfqpoint{3.728148in}{1.609707in}}%
\pgfpathlineto{\pgfqpoint{3.729493in}{1.612198in}}%
\pgfpathlineto{\pgfqpoint{3.729686in}{1.611770in}}%
\pgfpathlineto{\pgfqpoint{3.729878in}{1.614589in}}%
\pgfpathlineto{\pgfqpoint{3.730454in}{1.610049in}}%
\pgfpathlineto{\pgfqpoint{3.730646in}{1.611119in}}%
\pgfpathlineto{\pgfqpoint{3.731992in}{1.619376in}}%
\pgfpathlineto{\pgfqpoint{3.732568in}{1.614487in}}%
\pgfpathlineto{\pgfqpoint{3.733337in}{1.611254in}}%
\pgfpathlineto{\pgfqpoint{3.732952in}{1.615423in}}%
\pgfpathlineto{\pgfqpoint{3.733913in}{1.612626in}}%
\pgfpathlineto{\pgfqpoint{3.734682in}{1.612929in}}%
\pgfpathlineto{\pgfqpoint{3.736219in}{1.621505in}}%
\pgfpathlineto{\pgfqpoint{3.737180in}{1.617962in}}%
\pgfpathlineto{\pgfqpoint{3.737372in}{1.618448in}}%
\pgfpathlineto{\pgfqpoint{3.737757in}{1.616400in}}%
\pgfpathlineto{\pgfqpoint{3.738141in}{1.616060in}}%
\pgfpathlineto{\pgfqpoint{3.739678in}{1.625001in}}%
\pgfpathlineto{\pgfqpoint{3.740831in}{1.618866in}}%
\pgfpathlineto{\pgfqpoint{3.741023in}{1.620209in}}%
\pgfpathlineto{\pgfqpoint{3.741216in}{1.620422in}}%
\pgfpathlineto{\pgfqpoint{3.741408in}{1.614705in}}%
\pgfpathlineto{\pgfqpoint{3.742369in}{1.616146in}}%
\pgfpathlineto{\pgfqpoint{3.743714in}{1.600871in}}%
\pgfpathlineto{\pgfqpoint{3.745059in}{1.604584in}}%
\pgfpathlineto{\pgfqpoint{3.745635in}{1.603533in}}%
\pgfpathlineto{\pgfqpoint{3.746596in}{1.600585in}}%
\pgfpathlineto{\pgfqpoint{3.746789in}{1.600859in}}%
\pgfpathlineto{\pgfqpoint{3.747942in}{1.609823in}}%
\pgfpathlineto{\pgfqpoint{3.748710in}{1.608413in}}%
\pgfpathlineto{\pgfqpoint{3.748902in}{1.606731in}}%
\pgfpathlineto{\pgfqpoint{3.749479in}{1.610190in}}%
\pgfpathlineto{\pgfqpoint{3.749863in}{1.609462in}}%
\pgfpathlineto{\pgfqpoint{3.752361in}{1.616940in}}%
\pgfpathlineto{\pgfqpoint{3.752554in}{1.620167in}}%
\pgfpathlineto{\pgfqpoint{3.753130in}{1.614559in}}%
\pgfpathlineto{\pgfqpoint{3.753322in}{1.615698in}}%
\pgfpathlineto{\pgfqpoint{3.753514in}{1.615930in}}%
\pgfpathlineto{\pgfqpoint{3.754091in}{1.611251in}}%
\pgfpathlineto{\pgfqpoint{3.754667in}{1.615108in}}%
\pgfpathlineto{\pgfqpoint{3.755052in}{1.619072in}}%
\pgfpathlineto{\pgfqpoint{3.755436in}{1.614274in}}%
\pgfpathlineto{\pgfqpoint{3.755820in}{1.616261in}}%
\pgfpathlineto{\pgfqpoint{3.756205in}{1.611001in}}%
\pgfpathlineto{\pgfqpoint{3.757166in}{1.612559in}}%
\pgfpathlineto{\pgfqpoint{3.757742in}{1.615876in}}%
\pgfpathlineto{\pgfqpoint{3.758126in}{1.612033in}}%
\pgfpathlineto{\pgfqpoint{3.759087in}{1.603435in}}%
\pgfpathlineto{\pgfqpoint{3.760048in}{1.607997in}}%
\pgfpathlineto{\pgfqpoint{3.761393in}{1.609563in}}%
\pgfpathlineto{\pgfqpoint{3.762931in}{1.601143in}}%
\pgfpathlineto{\pgfqpoint{3.763507in}{1.601890in}}%
\pgfpathlineto{\pgfqpoint{3.763699in}{1.601733in}}%
\pgfpathlineto{\pgfqpoint{3.763891in}{1.602140in}}%
\pgfpathlineto{\pgfqpoint{3.764852in}{1.604319in}}%
\pgfpathlineto{\pgfqpoint{3.766582in}{1.591933in}}%
\pgfpathlineto{\pgfqpoint{3.766774in}{1.595633in}}%
\pgfpathlineto{\pgfqpoint{3.767543in}{1.592161in}}%
\pgfpathlineto{\pgfqpoint{3.767735in}{1.592361in}}%
\pgfpathlineto{\pgfqpoint{3.767927in}{1.592136in}}%
\pgfpathlineto{\pgfqpoint{3.768119in}{1.588902in}}%
\pgfpathlineto{\pgfqpoint{3.769080in}{1.591126in}}%
\pgfpathlineto{\pgfqpoint{3.771002in}{1.586215in}}%
\pgfpathlineto{\pgfqpoint{3.771194in}{1.589512in}}%
\pgfpathlineto{\pgfqpoint{3.771770in}{1.583391in}}%
\pgfpathlineto{\pgfqpoint{3.771963in}{1.583826in}}%
\pgfpathlineto{\pgfqpoint{3.774076in}{1.576722in}}%
\pgfpathlineto{\pgfqpoint{3.774461in}{1.579846in}}%
\pgfpathlineto{\pgfqpoint{3.775037in}{1.578789in}}%
\pgfpathlineto{\pgfqpoint{3.775422in}{1.575809in}}%
\pgfpathlineto{\pgfqpoint{3.775998in}{1.579110in}}%
\pgfpathlineto{\pgfqpoint{3.776190in}{1.578243in}}%
\pgfpathlineto{\pgfqpoint{3.777535in}{1.582028in}}%
\pgfpathlineto{\pgfqpoint{3.777728in}{1.580099in}}%
\pgfpathlineto{\pgfqpoint{3.778304in}{1.583705in}}%
\pgfpathlineto{\pgfqpoint{3.778688in}{1.580208in}}%
\pgfpathlineto{\pgfqpoint{3.778881in}{1.579963in}}%
\pgfpathlineto{\pgfqpoint{3.780418in}{1.592400in}}%
\pgfpathlineto{\pgfqpoint{3.780610in}{1.592262in}}%
\pgfpathlineto{\pgfqpoint{3.781379in}{1.589323in}}%
\pgfpathlineto{\pgfqpoint{3.781763in}{1.589561in}}%
\pgfpathlineto{\pgfqpoint{3.783300in}{1.594136in}}%
\pgfpathlineto{\pgfqpoint{3.783493in}{1.589366in}}%
\pgfpathlineto{\pgfqpoint{3.784261in}{1.593302in}}%
\pgfpathlineto{\pgfqpoint{3.785030in}{1.596702in}}%
\pgfpathlineto{\pgfqpoint{3.785414in}{1.593680in}}%
\pgfpathlineto{\pgfqpoint{3.785799in}{1.594946in}}%
\pgfpathlineto{\pgfqpoint{3.786952in}{1.600065in}}%
\pgfpathlineto{\pgfqpoint{3.787912in}{1.592171in}}%
\pgfpathlineto{\pgfqpoint{3.788105in}{1.594502in}}%
\pgfpathlineto{\pgfqpoint{3.789642in}{1.602948in}}%
\pgfpathlineto{\pgfqpoint{3.790411in}{1.601881in}}%
\pgfpathlineto{\pgfqpoint{3.790603in}{1.601488in}}%
\pgfpathlineto{\pgfqpoint{3.790795in}{1.602837in}}%
\pgfpathlineto{\pgfqpoint{3.790987in}{1.602477in}}%
\pgfpathlineto{\pgfqpoint{3.791179in}{1.603682in}}%
\pgfpathlineto{\pgfqpoint{3.791564in}{1.600543in}}%
\pgfpathlineto{\pgfqpoint{3.791756in}{1.602824in}}%
\pgfpathlineto{\pgfqpoint{3.791948in}{1.601306in}}%
\pgfpathlineto{\pgfqpoint{3.792524in}{1.605518in}}%
\pgfpathlineto{\pgfqpoint{3.792717in}{1.606288in}}%
\pgfpathlineto{\pgfqpoint{3.793293in}{1.604295in}}%
\pgfpathlineto{\pgfqpoint{3.793485in}{1.603549in}}%
\pgfpathlineto{\pgfqpoint{3.793870in}{1.605719in}}%
\pgfpathlineto{\pgfqpoint{3.794062in}{1.606080in}}%
\pgfpathlineto{\pgfqpoint{3.794254in}{1.604532in}}%
\pgfpathlineto{\pgfqpoint{3.794638in}{1.605048in}}%
\pgfpathlineto{\pgfqpoint{3.795215in}{1.605879in}}%
\pgfpathlineto{\pgfqpoint{3.796176in}{1.601271in}}%
\pgfpathlineto{\pgfqpoint{3.796752in}{1.606295in}}%
\pgfpathlineto{\pgfqpoint{3.797137in}{1.599248in}}%
\pgfpathlineto{\pgfqpoint{3.798674in}{1.592033in}}%
\pgfpathlineto{\pgfqpoint{3.799827in}{1.600000in}}%
\pgfpathlineto{\pgfqpoint{3.800019in}{1.599584in}}%
\pgfpathlineto{\pgfqpoint{3.800596in}{1.588594in}}%
\pgfpathlineto{\pgfqpoint{3.801364in}{1.593609in}}%
\pgfpathlineto{\pgfqpoint{3.802517in}{1.600335in}}%
\pgfpathlineto{\pgfqpoint{3.802709in}{1.599004in}}%
\pgfpathlineto{\pgfqpoint{3.803862in}{1.592677in}}%
\pgfpathlineto{\pgfqpoint{3.804055in}{1.595094in}}%
\pgfpathlineto{\pgfqpoint{3.804631in}{1.598133in}}%
\pgfpathlineto{\pgfqpoint{3.805015in}{1.596342in}}%
\pgfpathlineto{\pgfqpoint{3.806937in}{1.589302in}}%
\pgfpathlineto{\pgfqpoint{3.807129in}{1.588560in}}%
\pgfpathlineto{\pgfqpoint{3.807514in}{1.590880in}}%
\pgfpathlineto{\pgfqpoint{3.807706in}{1.589804in}}%
\pgfpathlineto{\pgfqpoint{3.807898in}{1.592031in}}%
\pgfpathlineto{\pgfqpoint{3.808282in}{1.588590in}}%
\pgfpathlineto{\pgfqpoint{3.808474in}{1.590373in}}%
\pgfpathlineto{\pgfqpoint{3.809243in}{1.583075in}}%
\pgfpathlineto{\pgfqpoint{3.809627in}{1.586233in}}%
\pgfpathlineto{\pgfqpoint{3.809820in}{1.585858in}}%
\pgfpathlineto{\pgfqpoint{3.810012in}{1.591260in}}%
\pgfpathlineto{\pgfqpoint{3.810780in}{1.584907in}}%
\pgfpathlineto{\pgfqpoint{3.812126in}{1.586496in}}%
\pgfpathlineto{\pgfqpoint{3.813663in}{1.579711in}}%
\pgfpathlineto{\pgfqpoint{3.812510in}{1.587465in}}%
\pgfpathlineto{\pgfqpoint{3.813855in}{1.580489in}}%
\pgfpathlineto{\pgfqpoint{3.815008in}{1.585560in}}%
\pgfpathlineto{\pgfqpoint{3.815200in}{1.583521in}}%
\pgfpathlineto{\pgfqpoint{3.816930in}{1.590222in}}%
\pgfpathlineto{\pgfqpoint{3.817122in}{1.588202in}}%
\pgfpathlineto{\pgfqpoint{3.817314in}{1.588315in}}%
\pgfpathlineto{\pgfqpoint{3.817506in}{1.586115in}}%
\pgfpathlineto{\pgfqpoint{3.817891in}{1.591270in}}%
\pgfpathlineto{\pgfqpoint{3.819236in}{1.601799in}}%
\pgfpathlineto{\pgfqpoint{3.819428in}{1.600541in}}%
\pgfpathlineto{\pgfqpoint{3.819620in}{1.597706in}}%
\pgfpathlineto{\pgfqpoint{3.820005in}{1.603638in}}%
\pgfpathlineto{\pgfqpoint{3.821350in}{1.613705in}}%
\pgfpathlineto{\pgfqpoint{3.821542in}{1.612302in}}%
\pgfpathlineto{\pgfqpoint{3.822118in}{1.602615in}}%
\pgfpathlineto{\pgfqpoint{3.822887in}{1.606792in}}%
\pgfpathlineto{\pgfqpoint{3.823079in}{1.607670in}}%
\pgfpathlineto{\pgfqpoint{3.823271in}{1.606657in}}%
\pgfpathlineto{\pgfqpoint{3.823848in}{1.597887in}}%
\pgfpathlineto{\pgfqpoint{3.824617in}{1.601865in}}%
\pgfpathlineto{\pgfqpoint{3.824809in}{1.602232in}}%
\pgfpathlineto{\pgfqpoint{3.825001in}{1.600826in}}%
\pgfpathlineto{\pgfqpoint{3.825962in}{1.598226in}}%
\pgfpathlineto{\pgfqpoint{3.825577in}{1.603133in}}%
\pgfpathlineto{\pgfqpoint{3.826154in}{1.599650in}}%
\pgfpathlineto{\pgfqpoint{3.826346in}{1.600852in}}%
\pgfpathlineto{\pgfqpoint{3.826923in}{1.599876in}}%
\pgfpathlineto{\pgfqpoint{3.827115in}{1.595760in}}%
\pgfpathlineto{\pgfqpoint{3.827883in}{1.597285in}}%
\pgfpathlineto{\pgfqpoint{3.828268in}{1.600836in}}%
\pgfpathlineto{\pgfqpoint{3.828460in}{1.597217in}}%
\pgfpathlineto{\pgfqpoint{3.829229in}{1.599931in}}%
\pgfpathlineto{\pgfqpoint{3.830382in}{1.593712in}}%
\pgfpathlineto{\pgfqpoint{3.829997in}{1.600236in}}%
\pgfpathlineto{\pgfqpoint{3.830766in}{1.595853in}}%
\pgfpathlineto{\pgfqpoint{3.830958in}{1.596622in}}%
\pgfpathlineto{\pgfqpoint{3.831150in}{1.595683in}}%
\pgfpathlineto{\pgfqpoint{3.831919in}{1.588574in}}%
\pgfpathlineto{\pgfqpoint{3.832688in}{1.589064in}}%
\pgfpathlineto{\pgfqpoint{3.833264in}{1.595125in}}%
\pgfpathlineto{\pgfqpoint{3.833841in}{1.590030in}}%
\pgfpathlineto{\pgfqpoint{3.834225in}{1.590480in}}%
\pgfpathlineto{\pgfqpoint{3.834417in}{1.589736in}}%
\pgfpathlineto{\pgfqpoint{3.835570in}{1.585334in}}%
\pgfpathlineto{\pgfqpoint{3.835954in}{1.587678in}}%
\pgfpathlineto{\pgfqpoint{3.836531in}{1.586192in}}%
\pgfpathlineto{\pgfqpoint{3.837300in}{1.584856in}}%
\pgfpathlineto{\pgfqpoint{3.837107in}{1.587307in}}%
\pgfpathlineto{\pgfqpoint{3.837492in}{1.585583in}}%
\pgfpathlineto{\pgfqpoint{3.837684in}{1.587280in}}%
\pgfpathlineto{\pgfqpoint{3.838068in}{1.583640in}}%
\pgfpathlineto{\pgfqpoint{3.838645in}{1.586005in}}%
\pgfpathlineto{\pgfqpoint{3.838837in}{1.583805in}}%
\pgfpathlineto{\pgfqpoint{3.839606in}{1.585289in}}%
\pgfpathlineto{\pgfqpoint{3.840951in}{1.590230in}}%
\pgfpathlineto{\pgfqpoint{3.841335in}{1.589715in}}%
\pgfpathlineto{\pgfqpoint{3.841527in}{1.590562in}}%
\pgfpathlineto{\pgfqpoint{3.842296in}{1.592954in}}%
\pgfpathlineto{\pgfqpoint{3.842104in}{1.590093in}}%
\pgfpathlineto{\pgfqpoint{3.842488in}{1.591194in}}%
\pgfpathlineto{\pgfqpoint{3.843257in}{1.586671in}}%
\pgfpathlineto{\pgfqpoint{3.843833in}{1.590167in}}%
\pgfpathlineto{\pgfqpoint{3.844602in}{1.594787in}}%
\pgfpathlineto{\pgfqpoint{3.845179in}{1.592538in}}%
\pgfpathlineto{\pgfqpoint{3.846524in}{1.583641in}}%
\pgfpathlineto{\pgfqpoint{3.846908in}{1.585310in}}%
\pgfpathlineto{\pgfqpoint{3.847100in}{1.586049in}}%
\pgfpathlineto{\pgfqpoint{3.847292in}{1.582605in}}%
\pgfpathlineto{\pgfqpoint{3.848445in}{1.570143in}}%
\pgfpathlineto{\pgfqpoint{3.849406in}{1.571313in}}%
\pgfpathlineto{\pgfqpoint{3.850367in}{1.576215in}}%
\pgfpathlineto{\pgfqpoint{3.850559in}{1.575168in}}%
\pgfpathlineto{\pgfqpoint{3.850944in}{1.572889in}}%
\pgfpathlineto{\pgfqpoint{3.851136in}{1.576427in}}%
\pgfpathlineto{\pgfqpoint{3.851520in}{1.575903in}}%
\pgfpathlineto{\pgfqpoint{3.851904in}{1.579828in}}%
\pgfpathlineto{\pgfqpoint{3.852481in}{1.577964in}}%
\pgfpathlineto{\pgfqpoint{3.852865in}{1.571857in}}%
\pgfpathlineto{\pgfqpoint{3.853634in}{1.574019in}}%
\pgfpathlineto{\pgfqpoint{3.854979in}{1.570629in}}%
\pgfpathlineto{\pgfqpoint{3.855363in}{1.571564in}}%
\pgfpathlineto{\pgfqpoint{3.858246in}{1.583803in}}%
\pgfpathlineto{\pgfqpoint{3.858630in}{1.581849in}}%
\pgfpathlineto{\pgfqpoint{3.859975in}{1.577780in}}%
\pgfpathlineto{\pgfqpoint{3.860552in}{1.580360in}}%
\pgfpathlineto{\pgfqpoint{3.860360in}{1.577450in}}%
\pgfpathlineto{\pgfqpoint{3.860936in}{1.579808in}}%
\pgfpathlineto{\pgfqpoint{3.861897in}{1.571414in}}%
\pgfpathlineto{\pgfqpoint{3.862281in}{1.575280in}}%
\pgfpathlineto{\pgfqpoint{3.862858in}{1.577676in}}%
\pgfpathlineto{\pgfqpoint{3.863242in}{1.574952in}}%
\pgfpathlineto{\pgfqpoint{3.863434in}{1.574143in}}%
\pgfpathlineto{\pgfqpoint{3.863627in}{1.576201in}}%
\pgfpathlineto{\pgfqpoint{3.864203in}{1.580314in}}%
\pgfpathlineto{\pgfqpoint{3.864780in}{1.580279in}}%
\pgfpathlineto{\pgfqpoint{3.866317in}{1.576111in}}%
\pgfpathlineto{\pgfqpoint{3.866701in}{1.577666in}}%
\pgfpathlineto{\pgfqpoint{3.867086in}{1.576313in}}%
\pgfpathlineto{\pgfqpoint{3.867470in}{1.571433in}}%
\pgfpathlineto{\pgfqpoint{3.867854in}{1.577172in}}%
\pgfpathlineto{\pgfqpoint{3.868046in}{1.580052in}}%
\pgfpathlineto{\pgfqpoint{3.868239in}{1.573818in}}%
\pgfpathlineto{\pgfqpoint{3.868623in}{1.574810in}}%
\pgfpathlineto{\pgfqpoint{3.869007in}{1.572005in}}%
\pgfpathlineto{\pgfqpoint{3.869200in}{1.575358in}}%
\pgfpathlineto{\pgfqpoint{3.869584in}{1.574022in}}%
\pgfpathlineto{\pgfqpoint{3.869968in}{1.576836in}}%
\pgfpathlineto{\pgfqpoint{3.870737in}{1.574800in}}%
\pgfpathlineto{\pgfqpoint{3.871121in}{1.574058in}}%
\pgfpathlineto{\pgfqpoint{3.872082in}{1.569480in}}%
\pgfpathlineto{\pgfqpoint{3.872274in}{1.570273in}}%
\pgfpathlineto{\pgfqpoint{3.872466in}{1.572949in}}%
\pgfpathlineto{\pgfqpoint{3.873235in}{1.569143in}}%
\pgfpathlineto{\pgfqpoint{3.873427in}{1.567275in}}%
\pgfpathlineto{\pgfqpoint{3.874196in}{1.570491in}}%
\pgfpathlineto{\pgfqpoint{3.875733in}{1.577798in}}%
\pgfpathlineto{\pgfqpoint{3.876310in}{1.577321in}}%
\pgfpathlineto{\pgfqpoint{3.876886in}{1.571507in}}%
\pgfpathlineto{\pgfqpoint{3.877463in}{1.573520in}}%
\pgfpathlineto{\pgfqpoint{3.879192in}{1.579040in}}%
\pgfpathlineto{\pgfqpoint{3.880922in}{1.570023in}}%
\pgfpathlineto{\pgfqpoint{3.881114in}{1.570748in}}%
\pgfpathlineto{\pgfqpoint{3.882843in}{1.583540in}}%
\pgfpathlineto{\pgfqpoint{3.883036in}{1.582671in}}%
\pgfpathlineto{\pgfqpoint{3.885534in}{1.562766in}}%
\pgfpathlineto{\pgfqpoint{3.883420in}{1.584719in}}%
\pgfpathlineto{\pgfqpoint{3.885726in}{1.563693in}}%
\pgfpathlineto{\pgfqpoint{3.885918in}{1.564478in}}%
\pgfpathlineto{\pgfqpoint{3.887263in}{1.556545in}}%
\pgfpathlineto{\pgfqpoint{3.888416in}{1.563361in}}%
\pgfpathlineto{\pgfqpoint{3.888608in}{1.559910in}}%
\pgfpathlineto{\pgfqpoint{3.889569in}{1.567936in}}%
\pgfpathlineto{\pgfqpoint{3.890338in}{1.564354in}}%
\pgfpathlineto{\pgfqpoint{3.891107in}{1.557859in}}%
\pgfpathlineto{\pgfqpoint{3.891491in}{1.560091in}}%
\pgfpathlineto{\pgfqpoint{3.892644in}{1.568087in}}%
\pgfpathlineto{\pgfqpoint{3.892836in}{1.567273in}}%
\pgfpathlineto{\pgfqpoint{3.893220in}{1.567740in}}%
\pgfpathlineto{\pgfqpoint{3.893605in}{1.573371in}}%
\pgfpathlineto{\pgfqpoint{3.894374in}{1.572425in}}%
\pgfpathlineto{\pgfqpoint{3.895527in}{1.560864in}}%
\pgfpathlineto{\pgfqpoint{3.895911in}{1.563469in}}%
\pgfpathlineto{\pgfqpoint{3.896103in}{1.562101in}}%
\pgfpathlineto{\pgfqpoint{3.896487in}{1.565796in}}%
\pgfpathlineto{\pgfqpoint{3.896872in}{1.567995in}}%
\pgfpathlineto{\pgfqpoint{3.897640in}{1.561715in}}%
\pgfpathlineto{\pgfqpoint{3.898025in}{1.566285in}}%
\pgfpathlineto{\pgfqpoint{3.898986in}{1.563036in}}%
\pgfpathlineto{\pgfqpoint{3.899370in}{1.564344in}}%
\pgfpathlineto{\pgfqpoint{3.899754in}{1.567226in}}%
\pgfpathlineto{\pgfqpoint{3.900139in}{1.562664in}}%
\pgfpathlineto{\pgfqpoint{3.900331in}{1.564545in}}%
\pgfpathlineto{\pgfqpoint{3.901868in}{1.560679in}}%
\pgfpathlineto{\pgfqpoint{3.902060in}{1.562766in}}%
\pgfpathlineto{\pgfqpoint{3.905327in}{1.577405in}}%
\pgfpathlineto{\pgfqpoint{3.905519in}{1.575203in}}%
\pgfpathlineto{\pgfqpoint{3.907249in}{1.562059in}}%
\pgfpathlineto{\pgfqpoint{3.908017in}{1.557018in}}%
\pgfpathlineto{\pgfqpoint{3.908402in}{1.558660in}}%
\pgfpathlineto{\pgfqpoint{3.909555in}{1.566953in}}%
\pgfpathlineto{\pgfqpoint{3.908786in}{1.557723in}}%
\pgfpathlineto{\pgfqpoint{3.909939in}{1.564550in}}%
\pgfpathlineto{\pgfqpoint{3.910516in}{1.561804in}}%
\pgfpathlineto{\pgfqpoint{3.911092in}{1.563907in}}%
\pgfpathlineto{\pgfqpoint{3.914359in}{1.576958in}}%
\pgfpathlineto{\pgfqpoint{3.915512in}{1.572094in}}%
\pgfpathlineto{\pgfqpoint{3.914935in}{1.577169in}}%
\pgfpathlineto{\pgfqpoint{3.915704in}{1.573617in}}%
\pgfpathlineto{\pgfqpoint{3.917049in}{1.576844in}}%
\pgfpathlineto{\pgfqpoint{3.917241in}{1.575949in}}%
\pgfpathlineto{\pgfqpoint{3.918395in}{1.568040in}}%
\pgfpathlineto{\pgfqpoint{3.918779in}{1.569769in}}%
\pgfpathlineto{\pgfqpoint{3.919932in}{1.577957in}}%
\pgfpathlineto{\pgfqpoint{3.920316in}{1.576921in}}%
\pgfpathlineto{\pgfqpoint{3.920508in}{1.574058in}}%
\pgfpathlineto{\pgfqpoint{3.921085in}{1.577108in}}%
\pgfpathlineto{\pgfqpoint{3.921277in}{1.575300in}}%
\pgfpathlineto{\pgfqpoint{3.922430in}{1.579250in}}%
\pgfpathlineto{\pgfqpoint{3.923583in}{1.576638in}}%
\pgfpathlineto{\pgfqpoint{3.924352in}{1.578082in}}%
\pgfpathlineto{\pgfqpoint{3.924160in}{1.575439in}}%
\pgfpathlineto{\pgfqpoint{3.924544in}{1.575964in}}%
\pgfpathlineto{\pgfqpoint{3.924928in}{1.574618in}}%
\pgfpathlineto{\pgfqpoint{3.925120in}{1.576394in}}%
\pgfpathlineto{\pgfqpoint{3.926658in}{1.583058in}}%
\pgfpathlineto{\pgfqpoint{3.927234in}{1.581353in}}%
\pgfpathlineto{\pgfqpoint{3.927619in}{1.584084in}}%
\pgfpathlineto{\pgfqpoint{3.929348in}{1.565965in}}%
\pgfpathlineto{\pgfqpoint{3.929540in}{1.566337in}}%
\pgfpathlineto{\pgfqpoint{3.930501in}{1.574607in}}%
\pgfpathlineto{\pgfqpoint{3.930885in}{1.573815in}}%
\pgfpathlineto{\pgfqpoint{3.931654in}{1.567875in}}%
\pgfpathlineto{\pgfqpoint{3.932231in}{1.569405in}}%
\pgfpathlineto{\pgfqpoint{3.932999in}{1.573168in}}%
\pgfpathlineto{\pgfqpoint{3.933576in}{1.571808in}}%
\pgfpathlineto{\pgfqpoint{3.933768in}{1.568681in}}%
\pgfpathlineto{\pgfqpoint{3.933960in}{1.572370in}}%
\pgfpathlineto{\pgfqpoint{3.934537in}{1.571745in}}%
\pgfpathlineto{\pgfqpoint{3.935305in}{1.575222in}}%
\pgfpathlineto{\pgfqpoint{3.935497in}{1.573376in}}%
\pgfpathlineto{\pgfqpoint{3.935690in}{1.570466in}}%
\pgfpathlineto{\pgfqpoint{3.936458in}{1.575400in}}%
\pgfpathlineto{\pgfqpoint{3.937611in}{1.571060in}}%
\pgfpathlineto{\pgfqpoint{3.937803in}{1.572231in}}%
\pgfpathlineto{\pgfqpoint{3.937996in}{1.572664in}}%
\pgfpathlineto{\pgfqpoint{3.938380in}{1.569554in}}%
\pgfpathlineto{\pgfqpoint{3.938764in}{1.572969in}}%
\pgfpathlineto{\pgfqpoint{3.939149in}{1.571013in}}%
\pgfpathlineto{\pgfqpoint{3.939917in}{1.576404in}}%
\pgfpathlineto{\pgfqpoint{3.940494in}{1.573903in}}%
\pgfpathlineto{\pgfqpoint{3.941262in}{1.570477in}}%
\pgfpathlineto{\pgfqpoint{3.941839in}{1.572089in}}%
\pgfpathlineto{\pgfqpoint{3.942031in}{1.572545in}}%
\pgfpathlineto{\pgfqpoint{3.942416in}{1.579391in}}%
\pgfpathlineto{\pgfqpoint{3.942992in}{1.573024in}}%
\pgfpathlineto{\pgfqpoint{3.944337in}{1.566267in}}%
\pgfpathlineto{\pgfqpoint{3.945106in}{1.569025in}}%
\pgfpathlineto{\pgfqpoint{3.945298in}{1.570489in}}%
\pgfpathlineto{\pgfqpoint{3.945682in}{1.565911in}}%
\pgfpathlineto{\pgfqpoint{3.945875in}{1.567322in}}%
\pgfpathlineto{\pgfqpoint{3.947028in}{1.562706in}}%
\pgfpathlineto{\pgfqpoint{3.947220in}{1.563836in}}%
\pgfpathlineto{\pgfqpoint{3.948565in}{1.568442in}}%
\pgfpathlineto{\pgfqpoint{3.947604in}{1.562851in}}%
\pgfpathlineto{\pgfqpoint{3.948757in}{1.568097in}}%
\pgfpathlineto{\pgfqpoint{3.948949in}{1.569044in}}%
\pgfpathlineto{\pgfqpoint{3.949334in}{1.566110in}}%
\pgfpathlineto{\pgfqpoint{3.949526in}{1.566060in}}%
\pgfpathlineto{\pgfqpoint{3.950487in}{1.562562in}}%
\pgfpathlineto{\pgfqpoint{3.950679in}{1.562723in}}%
\pgfpathlineto{\pgfqpoint{3.951255in}{1.564673in}}%
\pgfpathlineto{\pgfqpoint{3.951640in}{1.561389in}}%
\pgfpathlineto{\pgfqpoint{3.952024in}{1.559395in}}%
\pgfpathlineto{\pgfqpoint{3.952600in}{1.556375in}}%
\pgfpathlineto{\pgfqpoint{3.952985in}{1.557084in}}%
\pgfpathlineto{\pgfqpoint{3.954138in}{1.561420in}}%
\pgfpathlineto{\pgfqpoint{3.955099in}{1.553669in}}%
\pgfpathlineto{\pgfqpoint{3.955675in}{1.555272in}}%
\pgfpathlineto{\pgfqpoint{3.955867in}{1.555577in}}%
\pgfpathlineto{\pgfqpoint{3.956059in}{1.553958in}}%
\pgfpathlineto{\pgfqpoint{3.956252in}{1.555078in}}%
\pgfpathlineto{\pgfqpoint{3.956636in}{1.552130in}}%
\pgfpathlineto{\pgfqpoint{3.957405in}{1.553508in}}%
\pgfpathlineto{\pgfqpoint{3.957597in}{1.553638in}}%
\pgfpathlineto{\pgfqpoint{3.957981in}{1.550481in}}%
\pgfpathlineto{\pgfqpoint{3.958365in}{1.553591in}}%
\pgfpathlineto{\pgfqpoint{3.958942in}{1.557201in}}%
\pgfpathlineto{\pgfqpoint{3.959134in}{1.553729in}}%
\pgfpathlineto{\pgfqpoint{3.959326in}{1.550745in}}%
\pgfpathlineto{\pgfqpoint{3.959518in}{1.555240in}}%
\pgfpathlineto{\pgfqpoint{3.960095in}{1.554972in}}%
\pgfpathlineto{\pgfqpoint{3.960287in}{1.555332in}}%
\pgfpathlineto{\pgfqpoint{3.960479in}{1.555150in}}%
\pgfpathlineto{\pgfqpoint{3.961440in}{1.551569in}}%
\pgfpathlineto{\pgfqpoint{3.961632in}{1.552593in}}%
\pgfpathlineto{\pgfqpoint{3.961824in}{1.556569in}}%
\pgfpathlineto{\pgfqpoint{3.962593in}{1.552190in}}%
\pgfpathlineto{\pgfqpoint{3.962785in}{1.554521in}}%
\pgfpathlineto{\pgfqpoint{3.962977in}{1.553110in}}%
\pgfpathlineto{\pgfqpoint{3.963362in}{1.554655in}}%
\pgfpathlineto{\pgfqpoint{3.963746in}{1.559610in}}%
\pgfpathlineto{\pgfqpoint{3.964515in}{1.558579in}}%
\pgfpathlineto{\pgfqpoint{3.964899in}{1.556877in}}%
\pgfpathlineto{\pgfqpoint{3.965476in}{1.557518in}}%
\pgfpathlineto{\pgfqpoint{3.966052in}{1.565392in}}%
\pgfpathlineto{\pgfqpoint{3.967013in}{1.564997in}}%
\pgfpathlineto{\pgfqpoint{3.967205in}{1.565615in}}%
\pgfpathlineto{\pgfqpoint{3.967590in}{1.565158in}}%
\pgfpathlineto{\pgfqpoint{3.968743in}{1.560414in}}%
\pgfpathlineto{\pgfqpoint{3.968358in}{1.565359in}}%
\pgfpathlineto{\pgfqpoint{3.968935in}{1.560644in}}%
\pgfpathlineto{\pgfqpoint{3.969319in}{1.558083in}}%
\pgfpathlineto{\pgfqpoint{3.969896in}{1.560334in}}%
\pgfpathlineto{\pgfqpoint{3.971241in}{1.562970in}}%
\pgfpathlineto{\pgfqpoint{3.972586in}{1.553157in}}%
\pgfpathlineto{\pgfqpoint{3.972778in}{1.554689in}}%
\pgfpathlineto{\pgfqpoint{3.972970in}{1.556388in}}%
\pgfpathlineto{\pgfqpoint{3.973355in}{1.553160in}}%
\pgfpathlineto{\pgfqpoint{3.973547in}{1.553209in}}%
\pgfpathlineto{\pgfqpoint{3.973931in}{1.551098in}}%
\pgfpathlineto{\pgfqpoint{3.974123in}{1.553854in}}%
\pgfpathlineto{\pgfqpoint{3.974315in}{1.556181in}}%
\pgfpathlineto{\pgfqpoint{3.974700in}{1.552079in}}%
\pgfpathlineto{\pgfqpoint{3.975084in}{1.552536in}}%
\pgfpathlineto{\pgfqpoint{3.975276in}{1.551897in}}%
\pgfpathlineto{\pgfqpoint{3.975468in}{1.555917in}}%
\pgfpathlineto{\pgfqpoint{3.976045in}{1.550900in}}%
\pgfpathlineto{\pgfqpoint{3.976237in}{1.547624in}}%
\pgfpathlineto{\pgfqpoint{3.976621in}{1.554386in}}%
\pgfpathlineto{\pgfqpoint{3.977006in}{1.552395in}}%
\pgfpathlineto{\pgfqpoint{3.977198in}{1.552934in}}%
\pgfpathlineto{\pgfqpoint{3.977390in}{1.551708in}}%
\pgfpathlineto{\pgfqpoint{3.978159in}{1.550361in}}%
\pgfpathlineto{\pgfqpoint{3.978351in}{1.552222in}}%
\pgfpathlineto{\pgfqpoint{3.978543in}{1.553677in}}%
\pgfpathlineto{\pgfqpoint{3.978927in}{1.552146in}}%
\pgfpathlineto{\pgfqpoint{3.979504in}{1.545780in}}%
\pgfpathlineto{\pgfqpoint{3.980080in}{1.547597in}}%
\pgfpathlineto{\pgfqpoint{3.981233in}{1.552143in}}%
\pgfpathlineto{\pgfqpoint{3.980465in}{1.546884in}}%
\pgfpathlineto{\pgfqpoint{3.981810in}{1.550754in}}%
\pgfpathlineto{\pgfqpoint{3.982002in}{1.550737in}}%
\pgfpathlineto{\pgfqpoint{3.982386in}{1.557834in}}%
\pgfpathlineto{\pgfqpoint{3.983347in}{1.555301in}}%
\pgfpathlineto{\pgfqpoint{3.984308in}{1.561172in}}%
\pgfpathlineto{\pgfqpoint{3.984692in}{1.560762in}}%
\pgfpathlineto{\pgfqpoint{3.985269in}{1.559261in}}%
\pgfpathlineto{\pgfqpoint{3.985461in}{1.560282in}}%
\pgfpathlineto{\pgfqpoint{3.985845in}{1.562803in}}%
\pgfpathlineto{\pgfqpoint{3.986230in}{1.557959in}}%
\pgfpathlineto{\pgfqpoint{3.986422in}{1.558418in}}%
\pgfpathlineto{\pgfqpoint{3.986998in}{1.558625in}}%
\pgfpathlineto{\pgfqpoint{3.987959in}{1.551144in}}%
\pgfpathlineto{\pgfqpoint{3.989112in}{1.548253in}}%
\pgfpathlineto{\pgfqpoint{3.990457in}{1.555775in}}%
\pgfpathlineto{\pgfqpoint{3.990650in}{1.556400in}}%
\pgfpathlineto{\pgfqpoint{3.991034in}{1.552900in}}%
\pgfpathlineto{\pgfqpoint{3.991803in}{1.554512in}}%
\pgfpathlineto{\pgfqpoint{3.993532in}{1.563240in}}%
\pgfpathlineto{\pgfqpoint{3.994109in}{1.560267in}}%
\pgfpathlineto{\pgfqpoint{3.994685in}{1.560481in}}%
\pgfpathlineto{\pgfqpoint{3.995454in}{1.558715in}}%
\pgfpathlineto{\pgfqpoint{3.995646in}{1.560212in}}%
\pgfpathlineto{\pgfqpoint{3.996415in}{1.563090in}}%
\pgfpathlineto{\pgfqpoint{3.996799in}{1.562994in}}%
\pgfpathlineto{\pgfqpoint{3.999105in}{1.553303in}}%
\pgfpathlineto{\pgfqpoint{3.999682in}{1.552437in}}%
\pgfpathlineto{\pgfqpoint{4.000066in}{1.555016in}}%
\pgfpathlineto{\pgfqpoint{4.001603in}{1.549783in}}%
\pgfpathlineto{\pgfqpoint{4.001795in}{1.549532in}}%
\pgfpathlineto{\pgfqpoint{4.001988in}{1.550689in}}%
\pgfpathlineto{\pgfqpoint{4.002180in}{1.551172in}}%
\pgfpathlineto{\pgfqpoint{4.002372in}{1.549990in}}%
\pgfpathlineto{\pgfqpoint{4.002564in}{1.548390in}}%
\pgfpathlineto{\pgfqpoint{4.002756in}{1.549211in}}%
\pgfpathlineto{\pgfqpoint{4.003909in}{1.560031in}}%
\pgfpathlineto{\pgfqpoint{4.004101in}{1.557794in}}%
\pgfpathlineto{\pgfqpoint{4.004294in}{1.559123in}}%
\pgfpathlineto{\pgfqpoint{4.004870in}{1.556222in}}%
\pgfpathlineto{\pgfqpoint{4.005062in}{1.556204in}}%
\pgfpathlineto{\pgfqpoint{4.006215in}{1.560963in}}%
\pgfpathlineto{\pgfqpoint{4.007176in}{1.554981in}}%
\pgfpathlineto{\pgfqpoint{4.007753in}{1.555113in}}%
\pgfpathlineto{\pgfqpoint{4.008521in}{1.561192in}}%
\pgfpathlineto{\pgfqpoint{4.009674in}{1.558806in}}%
\pgfpathlineto{\pgfqpoint{4.010059in}{1.557948in}}%
\pgfpathlineto{\pgfqpoint{4.010251in}{1.560129in}}%
\pgfpathlineto{\pgfqpoint{4.010635in}{1.559172in}}%
\pgfpathlineto{\pgfqpoint{4.011788in}{1.565614in}}%
\pgfpathlineto{\pgfqpoint{4.011980in}{1.564981in}}%
\pgfpathlineto{\pgfqpoint{4.012941in}{1.567003in}}%
\pgfpathlineto{\pgfqpoint{4.013325in}{1.566314in}}%
\pgfpathlineto{\pgfqpoint{4.014094in}{1.570052in}}%
\pgfpathlineto{\pgfqpoint{4.014671in}{1.566976in}}%
\pgfpathlineto{\pgfqpoint{4.016785in}{1.561081in}}%
\pgfpathlineto{\pgfqpoint{4.015055in}{1.567908in}}%
\pgfpathlineto{\pgfqpoint{4.016977in}{1.562171in}}%
\pgfpathlineto{\pgfqpoint{4.018130in}{1.567662in}}%
\pgfpathlineto{\pgfqpoint{4.018514in}{1.566547in}}%
\pgfpathlineto{\pgfqpoint{4.019859in}{1.563326in}}%
\pgfpathlineto{\pgfqpoint{4.019091in}{1.567956in}}%
\pgfpathlineto{\pgfqpoint{4.020051in}{1.563823in}}%
\pgfpathlineto{\pgfqpoint{4.020628in}{1.566996in}}%
\pgfpathlineto{\pgfqpoint{4.020820in}{1.562720in}}%
\pgfpathlineto{\pgfqpoint{4.023126in}{1.548591in}}%
\pgfpathlineto{\pgfqpoint{4.023510in}{1.551163in}}%
\pgfpathlineto{\pgfqpoint{4.023895in}{1.544442in}}%
\pgfpathlineto{\pgfqpoint{4.024087in}{1.544267in}}%
\pgfpathlineto{\pgfqpoint{4.024471in}{1.540530in}}%
\pgfpathlineto{\pgfqpoint{4.025048in}{1.546577in}}%
\pgfpathlineto{\pgfqpoint{4.026777in}{1.540829in}}%
\pgfpathlineto{\pgfqpoint{4.026969in}{1.541760in}}%
\pgfpathlineto{\pgfqpoint{4.027354in}{1.543841in}}%
\pgfpathlineto{\pgfqpoint{4.027546in}{1.541965in}}%
\pgfpathlineto{\pgfqpoint{4.029083in}{1.530152in}}%
\pgfpathlineto{\pgfqpoint{4.029275in}{1.532980in}}%
\pgfpathlineto{\pgfqpoint{4.029660in}{1.529266in}}%
\pgfpathlineto{\pgfqpoint{4.030621in}{1.531613in}}%
\pgfpathlineto{\pgfqpoint{4.031005in}{1.539422in}}%
\pgfpathlineto{\pgfqpoint{4.031774in}{1.534887in}}%
\pgfpathlineto{\pgfqpoint{4.033119in}{1.525984in}}%
\pgfpathlineto{\pgfqpoint{4.033311in}{1.526223in}}%
\pgfpathlineto{\pgfqpoint{4.033503in}{1.523904in}}%
\pgfpathlineto{\pgfqpoint{4.034080in}{1.529242in}}%
\pgfpathlineto{\pgfqpoint{4.034656in}{1.534578in}}%
\pgfpathlineto{\pgfqpoint{4.035040in}{1.528870in}}%
\pgfpathlineto{\pgfqpoint{4.035233in}{1.530624in}}%
\pgfpathlineto{\pgfqpoint{4.037154in}{1.533715in}}%
\pgfpathlineto{\pgfqpoint{4.035617in}{1.530159in}}%
\pgfpathlineto{\pgfqpoint{4.037539in}{1.533331in}}%
\pgfpathlineto{\pgfqpoint{4.038692in}{1.526797in}}%
\pgfpathlineto{\pgfqpoint{4.037923in}{1.533768in}}%
\pgfpathlineto{\pgfqpoint{4.039460in}{1.530133in}}%
\pgfpathlineto{\pgfqpoint{4.040229in}{1.533038in}}%
\pgfpathlineto{\pgfqpoint{4.040421in}{1.530043in}}%
\pgfpathlineto{\pgfqpoint{4.040613in}{1.530092in}}%
\pgfpathlineto{\pgfqpoint{4.041382in}{1.526797in}}%
\pgfpathlineto{\pgfqpoint{4.041766in}{1.528655in}}%
\pgfpathlineto{\pgfqpoint{4.042727in}{1.522484in}}%
\pgfpathlineto{\pgfqpoint{4.043112in}{1.525217in}}%
\pgfpathlineto{\pgfqpoint{4.044457in}{1.528220in}}%
\pgfpathlineto{\pgfqpoint{4.044649in}{1.528522in}}%
\pgfpathlineto{\pgfqpoint{4.045802in}{1.518491in}}%
\pgfpathlineto{\pgfqpoint{4.046571in}{1.519791in}}%
\pgfpathlineto{\pgfqpoint{4.047724in}{1.517557in}}%
\pgfpathlineto{\pgfqpoint{4.048684in}{1.522764in}}%
\pgfpathlineto{\pgfqpoint{4.049069in}{1.521684in}}%
\pgfpathlineto{\pgfqpoint{4.049261in}{1.522155in}}%
\pgfpathlineto{\pgfqpoint{4.049453in}{1.519751in}}%
\pgfpathlineto{\pgfqpoint{4.049837in}{1.521826in}}%
\pgfpathlineto{\pgfqpoint{4.050030in}{1.520614in}}%
\pgfpathlineto{\pgfqpoint{4.050798in}{1.521811in}}%
\pgfpathlineto{\pgfqpoint{4.051375in}{1.523747in}}%
\pgfpathlineto{\pgfqpoint{4.052720in}{1.514550in}}%
\pgfpathlineto{\pgfqpoint{4.052912in}{1.514582in}}%
\pgfpathlineto{\pgfqpoint{4.053104in}{1.514588in}}%
\pgfpathlineto{\pgfqpoint{4.054065in}{1.524918in}}%
\pgfpathlineto{\pgfqpoint{4.054449in}{1.523970in}}%
\pgfpathlineto{\pgfqpoint{4.055026in}{1.525579in}}%
\pgfpathlineto{\pgfqpoint{4.055410in}{1.521522in}}%
\pgfpathlineto{\pgfqpoint{4.055987in}{1.525223in}}%
\pgfpathlineto{\pgfqpoint{4.056563in}{1.528116in}}%
\pgfpathlineto{\pgfqpoint{4.056948in}{1.523395in}}%
\pgfpathlineto{\pgfqpoint{4.057332in}{1.521434in}}%
\pgfpathlineto{\pgfqpoint{4.057524in}{1.522341in}}%
\pgfpathlineto{\pgfqpoint{4.058293in}{1.516728in}}%
\pgfpathlineto{\pgfqpoint{4.058677in}{1.521567in}}%
\pgfpathlineto{\pgfqpoint{4.059830in}{1.529863in}}%
\pgfpathlineto{\pgfqpoint{4.060022in}{1.528290in}}%
\pgfpathlineto{\pgfqpoint{4.061752in}{1.514182in}}%
\pgfpathlineto{\pgfqpoint{4.062713in}{1.522449in}}%
\pgfpathlineto{\pgfqpoint{4.063097in}{1.519720in}}%
\pgfpathlineto{\pgfqpoint{4.063481in}{1.519138in}}%
\pgfpathlineto{\pgfqpoint{4.063673in}{1.518204in}}%
\pgfpathlineto{\pgfqpoint{4.063866in}{1.521729in}}%
\pgfpathlineto{\pgfqpoint{4.065403in}{1.529412in}}%
\pgfpathlineto{\pgfqpoint{4.065595in}{1.527964in}}%
\pgfpathlineto{\pgfqpoint{4.066748in}{1.516767in}}%
\pgfpathlineto{\pgfqpoint{4.067517in}{1.518984in}}%
\pgfpathlineto{\pgfqpoint{4.068478in}{1.521031in}}%
\pgfpathlineto{\pgfqpoint{4.068670in}{1.520729in}}%
\pgfpathlineto{\pgfqpoint{4.069631in}{1.515551in}}%
\pgfpathlineto{\pgfqpoint{4.070015in}{1.516095in}}%
\pgfpathlineto{\pgfqpoint{4.070207in}{1.518583in}}%
\pgfpathlineto{\pgfqpoint{4.070592in}{1.514743in}}%
\pgfpathlineto{\pgfqpoint{4.071168in}{1.518212in}}%
\pgfpathlineto{\pgfqpoint{4.071360in}{1.517596in}}%
\pgfpathlineto{\pgfqpoint{4.071937in}{1.519366in}}%
\pgfpathlineto{\pgfqpoint{4.072129in}{1.522946in}}%
\pgfpathlineto{\pgfqpoint{4.072898in}{1.520420in}}%
\pgfpathlineto{\pgfqpoint{4.073282in}{1.518434in}}%
\pgfpathlineto{\pgfqpoint{4.074051in}{1.519873in}}%
\pgfpathlineto{\pgfqpoint{4.074819in}{1.523166in}}%
\pgfpathlineto{\pgfqpoint{4.075011in}{1.520809in}}%
\pgfpathlineto{\pgfqpoint{4.075588in}{1.518451in}}%
\pgfpathlineto{\pgfqpoint{4.075780in}{1.521129in}}%
\pgfpathlineto{\pgfqpoint{4.076741in}{1.524457in}}%
\pgfpathlineto{\pgfqpoint{4.077125in}{1.524083in}}%
\pgfpathlineto{\pgfqpoint{4.078663in}{1.516807in}}%
\pgfpathlineto{\pgfqpoint{4.078855in}{1.517992in}}%
\pgfpathlineto{\pgfqpoint{4.080200in}{1.528709in}}%
\pgfpathlineto{\pgfqpoint{4.080584in}{1.527245in}}%
\pgfpathlineto{\pgfqpoint{4.080969in}{1.524935in}}%
\pgfpathlineto{\pgfqpoint{4.081737in}{1.526905in}}%
\pgfpathlineto{\pgfqpoint{4.082506in}{1.530320in}}%
\pgfpathlineto{\pgfqpoint{4.083082in}{1.529225in}}%
\pgfpathlineto{\pgfqpoint{4.083275in}{1.528041in}}%
\pgfpathlineto{\pgfqpoint{4.083851in}{1.530089in}}%
\pgfpathlineto{\pgfqpoint{4.084620in}{1.535668in}}%
\pgfpathlineto{\pgfqpoint{4.085196in}{1.533257in}}%
\pgfpathlineto{\pgfqpoint{4.085388in}{1.532305in}}%
\pgfpathlineto{\pgfqpoint{4.085965in}{1.533751in}}%
\pgfpathlineto{\pgfqpoint{4.086157in}{1.532787in}}%
\pgfpathlineto{\pgfqpoint{4.086541in}{1.533174in}}%
\pgfpathlineto{\pgfqpoint{4.086734in}{1.531872in}}%
\pgfpathlineto{\pgfqpoint{4.088463in}{1.523092in}}%
\pgfpathlineto{\pgfqpoint{4.089232in}{1.522488in}}%
\pgfpathlineto{\pgfqpoint{4.089424in}{1.524902in}}%
\pgfpathlineto{\pgfqpoint{4.090001in}{1.518975in}}%
\pgfpathlineto{\pgfqpoint{4.090577in}{1.522945in}}%
\pgfpathlineto{\pgfqpoint{4.090769in}{1.523221in}}%
\pgfpathlineto{\pgfqpoint{4.090961in}{1.521258in}}%
\pgfpathlineto{\pgfqpoint{4.091346in}{1.525098in}}%
\pgfpathlineto{\pgfqpoint{4.092691in}{1.536199in}}%
\pgfpathlineto{\pgfqpoint{4.094036in}{1.532121in}}%
\pgfpathlineto{\pgfqpoint{4.096150in}{1.545018in}}%
\pgfpathlineto{\pgfqpoint{4.097303in}{1.540671in}}%
\pgfpathlineto{\pgfqpoint{4.097495in}{1.542119in}}%
\pgfpathlineto{\pgfqpoint{4.097879in}{1.545058in}}%
\pgfpathlineto{\pgfqpoint{4.098264in}{1.541218in}}%
\pgfpathlineto{\pgfqpoint{4.098648in}{1.544309in}}%
\pgfpathlineto{\pgfqpoint{4.099032in}{1.543408in}}%
\pgfpathlineto{\pgfqpoint{4.099993in}{1.534578in}}%
\pgfpathlineto{\pgfqpoint{4.100378in}{1.536583in}}%
\pgfpathlineto{\pgfqpoint{4.100762in}{1.535925in}}%
\pgfpathlineto{\pgfqpoint{4.100954in}{1.536216in}}%
\pgfpathlineto{\pgfqpoint{4.101146in}{1.535722in}}%
\pgfpathlineto{\pgfqpoint{4.101338in}{1.537145in}}%
\pgfpathlineto{\pgfqpoint{4.101531in}{1.540325in}}%
\pgfpathlineto{\pgfqpoint{4.102107in}{1.534415in}}%
\pgfpathlineto{\pgfqpoint{4.103837in}{1.517083in}}%
\pgfpathlineto{\pgfqpoint{4.104221in}{1.517447in}}%
\pgfpathlineto{\pgfqpoint{4.106143in}{1.513016in}}%
\pgfpathlineto{\pgfqpoint{4.104797in}{1.518185in}}%
\pgfpathlineto{\pgfqpoint{4.106527in}{1.515144in}}%
\pgfpathlineto{\pgfqpoint{4.106719in}{1.516377in}}%
\pgfpathlineto{\pgfqpoint{4.106911in}{1.513492in}}%
\pgfpathlineto{\pgfqpoint{4.108256in}{1.504901in}}%
\pgfpathlineto{\pgfqpoint{4.108641in}{1.504951in}}%
\pgfpathlineto{\pgfqpoint{4.109217in}{1.507188in}}%
\pgfpathlineto{\pgfqpoint{4.109602in}{1.502909in}}%
\pgfpathlineto{\pgfqpoint{4.110562in}{1.507264in}}%
\pgfpathlineto{\pgfqpoint{4.110755in}{1.506639in}}%
\pgfpathlineto{\pgfqpoint{4.111139in}{1.511409in}}%
\pgfpathlineto{\pgfqpoint{4.111908in}{1.508079in}}%
\pgfpathlineto{\pgfqpoint{4.112100in}{1.506946in}}%
\pgfpathlineto{\pgfqpoint{4.112292in}{1.508435in}}%
\pgfpathlineto{\pgfqpoint{4.113637in}{1.514856in}}%
\pgfpathlineto{\pgfqpoint{4.116135in}{1.502369in}}%
\pgfpathlineto{\pgfqpoint{4.116328in}{1.502045in}}%
\pgfpathlineto{\pgfqpoint{4.116520in}{1.503255in}}%
\pgfpathlineto{\pgfqpoint{4.116904in}{1.504970in}}%
\pgfpathlineto{\pgfqpoint{4.117288in}{1.502111in}}%
\pgfpathlineto{\pgfqpoint{4.117673in}{1.500314in}}%
\pgfpathlineto{\pgfqpoint{4.118826in}{1.510136in}}%
\pgfpathlineto{\pgfqpoint{4.119210in}{1.509890in}}%
\pgfpathlineto{\pgfqpoint{4.119594in}{1.508539in}}%
\pgfpathlineto{\pgfqpoint{4.119787in}{1.506889in}}%
\pgfpathlineto{\pgfqpoint{4.120171in}{1.508850in}}%
\pgfpathlineto{\pgfqpoint{4.120363in}{1.507299in}}%
\pgfpathlineto{\pgfqpoint{4.120555in}{1.510045in}}%
\pgfpathlineto{\pgfqpoint{4.121324in}{1.507045in}}%
\pgfpathlineto{\pgfqpoint{4.121708in}{1.507943in}}%
\pgfpathlineto{\pgfqpoint{4.121900in}{1.506502in}}%
\pgfpathlineto{\pgfqpoint{4.122093in}{1.503956in}}%
\pgfpathlineto{\pgfqpoint{4.123053in}{1.505912in}}%
\pgfpathlineto{\pgfqpoint{4.123246in}{1.506380in}}%
\pgfpathlineto{\pgfqpoint{4.123438in}{1.506196in}}%
\pgfpathlineto{\pgfqpoint{4.124014in}{1.503486in}}%
\pgfpathlineto{\pgfqpoint{4.124591in}{1.504366in}}%
\pgfpathlineto{\pgfqpoint{4.124783in}{1.504215in}}%
\pgfpathlineto{\pgfqpoint{4.124975in}{1.504690in}}%
\pgfpathlineto{\pgfqpoint{4.126320in}{1.483202in}}%
\pgfpathlineto{\pgfqpoint{4.126705in}{1.484739in}}%
\pgfpathlineto{\pgfqpoint{4.126897in}{1.484558in}}%
\pgfpathlineto{\pgfqpoint{4.127281in}{1.485255in}}%
\pgfpathlineto{\pgfqpoint{4.128050in}{1.480084in}}%
\pgfpathlineto{\pgfqpoint{4.128242in}{1.480840in}}%
\pgfpathlineto{\pgfqpoint{4.128434in}{1.479480in}}%
\pgfpathlineto{\pgfqpoint{4.128626in}{1.479466in}}%
\pgfpathlineto{\pgfqpoint{4.128818in}{1.477974in}}%
\pgfpathlineto{\pgfqpoint{4.129203in}{1.482686in}}%
\pgfpathlineto{\pgfqpoint{4.130548in}{1.489607in}}%
\pgfpathlineto{\pgfqpoint{4.132085in}{1.486727in}}%
\pgfpathlineto{\pgfqpoint{4.132470in}{1.488372in}}%
\pgfpathlineto{\pgfqpoint{4.134007in}{1.497276in}}%
\pgfpathlineto{\pgfqpoint{4.134199in}{1.496954in}}%
\pgfpathlineto{\pgfqpoint{4.136121in}{1.510179in}}%
\pgfpathlineto{\pgfqpoint{4.136313in}{1.508501in}}%
\pgfpathlineto{\pgfqpoint{4.136889in}{1.502439in}}%
\pgfpathlineto{\pgfqpoint{4.137466in}{1.505636in}}%
\pgfpathlineto{\pgfqpoint{4.138811in}{1.509016in}}%
\pgfpathlineto{\pgfqpoint{4.139772in}{1.510477in}}%
\pgfpathlineto{\pgfqpoint{4.140349in}{1.503571in}}%
\pgfpathlineto{\pgfqpoint{4.140541in}{1.503368in}}%
\pgfpathlineto{\pgfqpoint{4.141886in}{1.498132in}}%
\pgfpathlineto{\pgfqpoint{4.142655in}{1.490399in}}%
\pgfpathlineto{\pgfqpoint{4.143423in}{1.493107in}}%
\pgfpathlineto{\pgfqpoint{4.143615in}{1.494455in}}%
\pgfpathlineto{\pgfqpoint{4.144000in}{1.491575in}}%
\pgfpathlineto{\pgfqpoint{4.144192in}{1.492017in}}%
\pgfpathlineto{\pgfqpoint{4.145729in}{1.480278in}}%
\pgfpathlineto{\pgfqpoint{4.146690in}{1.482535in}}%
\pgfpathlineto{\pgfqpoint{4.147074in}{1.484200in}}%
\pgfpathlineto{\pgfqpoint{4.149380in}{1.494845in}}%
\pgfpathlineto{\pgfqpoint{4.149957in}{1.494476in}}%
\pgfpathlineto{\pgfqpoint{4.150341in}{1.495114in}}%
\pgfpathlineto{\pgfqpoint{4.151494in}{1.487393in}}%
\pgfpathlineto{\pgfqpoint{4.151686in}{1.488731in}}%
\pgfpathlineto{\pgfqpoint{4.154569in}{1.506419in}}%
\pgfpathlineto{\pgfqpoint{4.154761in}{1.506192in}}%
\pgfpathlineto{\pgfqpoint{4.155338in}{1.509041in}}%
\pgfpathlineto{\pgfqpoint{4.156106in}{1.503472in}}%
\pgfpathlineto{\pgfqpoint{4.156491in}{1.502644in}}%
\pgfpathlineto{\pgfqpoint{4.158028in}{1.512470in}}%
\pgfpathlineto{\pgfqpoint{4.158989in}{1.509265in}}%
\pgfpathlineto{\pgfqpoint{4.159181in}{1.512408in}}%
\pgfpathlineto{\pgfqpoint{4.159373in}{1.512297in}}%
\pgfpathlineto{\pgfqpoint{4.159565in}{1.513331in}}%
\pgfpathlineto{\pgfqpoint{4.160910in}{1.520059in}}%
\pgfpathlineto{\pgfqpoint{4.161679in}{1.524143in}}%
\pgfpathlineto{\pgfqpoint{4.162063in}{1.522279in}}%
\pgfpathlineto{\pgfqpoint{4.162256in}{1.522296in}}%
\pgfpathlineto{\pgfqpoint{4.162832in}{1.517468in}}%
\pgfpathlineto{\pgfqpoint{4.163217in}{1.519552in}}%
\pgfpathlineto{\pgfqpoint{4.163409in}{1.523345in}}%
\pgfpathlineto{\pgfqpoint{4.164370in}{1.520520in}}%
\pgfpathlineto{\pgfqpoint{4.164946in}{1.516545in}}%
\pgfpathlineto{\pgfqpoint{4.165138in}{1.513967in}}%
\pgfpathlineto{\pgfqpoint{4.165907in}{1.516727in}}%
\pgfpathlineto{\pgfqpoint{4.166099in}{1.516091in}}%
\pgfpathlineto{\pgfqpoint{4.166291in}{1.519010in}}%
\pgfpathlineto{\pgfqpoint{4.166483in}{1.519029in}}%
\pgfpathlineto{\pgfqpoint{4.168021in}{1.527088in}}%
\pgfpathlineto{\pgfqpoint{4.168405in}{1.525875in}}%
\pgfpathlineto{\pgfqpoint{4.168597in}{1.525969in}}%
\pgfpathlineto{\pgfqpoint{4.168789in}{1.524448in}}%
\pgfpathlineto{\pgfqpoint{4.169558in}{1.526874in}}%
\pgfpathlineto{\pgfqpoint{4.170135in}{1.524690in}}%
\pgfpathlineto{\pgfqpoint{4.171480in}{1.538938in}}%
\pgfpathlineto{\pgfqpoint{4.172633in}{1.535578in}}%
\pgfpathlineto{\pgfqpoint{4.172825in}{1.537821in}}%
\pgfpathlineto{\pgfqpoint{4.173209in}{1.532332in}}%
\pgfpathlineto{\pgfqpoint{4.174747in}{1.525198in}}%
\pgfpathlineto{\pgfqpoint{4.173978in}{1.533030in}}%
\pgfpathlineto{\pgfqpoint{4.174939in}{1.525949in}}%
\pgfpathlineto{\pgfqpoint{4.176092in}{1.524005in}}%
\pgfpathlineto{\pgfqpoint{4.176476in}{1.520852in}}%
\pgfpathlineto{\pgfqpoint{4.177053in}{1.523974in}}%
\pgfpathlineto{\pgfqpoint{4.177245in}{1.523794in}}%
\pgfpathlineto{\pgfqpoint{4.177437in}{1.525043in}}%
\pgfpathlineto{\pgfqpoint{4.178206in}{1.523265in}}%
\pgfpathlineto{\pgfqpoint{4.179166in}{1.530431in}}%
\pgfpathlineto{\pgfqpoint{4.180319in}{1.520302in}}%
\pgfpathlineto{\pgfqpoint{4.180704in}{1.524577in}}%
\pgfpathlineto{\pgfqpoint{4.182049in}{1.516079in}}%
\pgfpathlineto{\pgfqpoint{4.182625in}{1.517174in}}%
\pgfpathlineto{\pgfqpoint{4.183010in}{1.516229in}}%
\pgfpathlineto{\pgfqpoint{4.183202in}{1.520480in}}%
\pgfpathlineto{\pgfqpoint{4.184163in}{1.517677in}}%
\pgfpathlineto{\pgfqpoint{4.184547in}{1.518993in}}%
\pgfpathlineto{\pgfqpoint{4.184931in}{1.525117in}}%
\pgfpathlineto{\pgfqpoint{4.185700in}{1.523138in}}%
\pgfpathlineto{\pgfqpoint{4.187430in}{1.511678in}}%
\pgfpathlineto{\pgfqpoint{4.188583in}{1.516168in}}%
\pgfpathlineto{\pgfqpoint{4.188775in}{1.515988in}}%
\pgfpathlineto{\pgfqpoint{4.189351in}{1.515141in}}%
\pgfpathlineto{\pgfqpoint{4.189928in}{1.521688in}}%
\pgfpathlineto{\pgfqpoint{4.190504in}{1.519768in}}%
\pgfpathlineto{\pgfqpoint{4.193579in}{1.500966in}}%
\pgfpathlineto{\pgfqpoint{4.193963in}{1.505072in}}%
\pgfpathlineto{\pgfqpoint{4.194540in}{1.501660in}}%
\pgfpathlineto{\pgfqpoint{4.194924in}{1.500716in}}%
\pgfpathlineto{\pgfqpoint{4.195116in}{1.501178in}}%
\pgfpathlineto{\pgfqpoint{4.195501in}{1.504067in}}%
\pgfpathlineto{\pgfqpoint{4.195885in}{1.498957in}}%
\pgfpathlineto{\pgfqpoint{4.196077in}{1.499961in}}%
\pgfpathlineto{\pgfqpoint{4.196269in}{1.498480in}}%
\pgfpathlineto{\pgfqpoint{4.196654in}{1.500997in}}%
\pgfpathlineto{\pgfqpoint{4.197230in}{1.499687in}}%
\pgfpathlineto{\pgfqpoint{4.198191in}{1.494881in}}%
\pgfpathlineto{\pgfqpoint{4.198383in}{1.496185in}}%
\pgfpathlineto{\pgfqpoint{4.198575in}{1.498741in}}%
\pgfpathlineto{\pgfqpoint{4.199344in}{1.497321in}}%
\pgfpathlineto{\pgfqpoint{4.199536in}{1.493651in}}%
\pgfpathlineto{\pgfqpoint{4.200305in}{1.499363in}}%
\pgfpathlineto{\pgfqpoint{4.200497in}{1.499599in}}%
\pgfpathlineto{\pgfqpoint{4.201074in}{1.503185in}}%
\pgfpathlineto{\pgfqpoint{4.201842in}{1.496377in}}%
\pgfpathlineto{\pgfqpoint{4.202995in}{1.489057in}}%
\pgfpathlineto{\pgfqpoint{4.203187in}{1.491521in}}%
\pgfpathlineto{\pgfqpoint{4.204725in}{1.500430in}}%
\pgfpathlineto{\pgfqpoint{4.206646in}{1.490988in}}%
\pgfpathlineto{\pgfqpoint{4.207992in}{1.489312in}}%
\pgfpathlineto{\pgfqpoint{4.209145in}{1.496141in}}%
\pgfpathlineto{\pgfqpoint{4.209721in}{1.494811in}}%
\pgfpathlineto{\pgfqpoint{4.209913in}{1.491257in}}%
\pgfpathlineto{\pgfqpoint{4.210490in}{1.497394in}}%
\pgfpathlineto{\pgfqpoint{4.210682in}{1.496538in}}%
\pgfpathlineto{\pgfqpoint{4.211066in}{1.497254in}}%
\pgfpathlineto{\pgfqpoint{4.211451in}{1.497002in}}%
\pgfpathlineto{\pgfqpoint{4.211835in}{1.493769in}}%
\pgfpathlineto{\pgfqpoint{4.212219in}{1.497092in}}%
\pgfpathlineto{\pgfqpoint{4.212412in}{1.496490in}}%
\pgfpathlineto{\pgfqpoint{4.213372in}{1.500049in}}%
\pgfpathlineto{\pgfqpoint{4.213565in}{1.498523in}}%
\pgfpathlineto{\pgfqpoint{4.214525in}{1.495798in}}%
\pgfpathlineto{\pgfqpoint{4.215294in}{1.498511in}}%
\pgfpathlineto{\pgfqpoint{4.215486in}{1.497164in}}%
\pgfpathlineto{\pgfqpoint{4.215678in}{1.495706in}}%
\pgfpathlineto{\pgfqpoint{4.216447in}{1.497664in}}%
\pgfpathlineto{\pgfqpoint{4.217024in}{1.500524in}}%
\pgfpathlineto{\pgfqpoint{4.217408in}{1.496902in}}%
\pgfpathlineto{\pgfqpoint{4.217600in}{1.498202in}}%
\pgfpathlineto{\pgfqpoint{4.218945in}{1.492821in}}%
\pgfpathlineto{\pgfqpoint{4.217984in}{1.498656in}}%
\pgfpathlineto{\pgfqpoint{4.219137in}{1.493883in}}%
\pgfpathlineto{\pgfqpoint{4.220675in}{1.501863in}}%
\pgfpathlineto{\pgfqpoint{4.221251in}{1.498644in}}%
\pgfpathlineto{\pgfqpoint{4.221636in}{1.502326in}}%
\pgfpathlineto{\pgfqpoint{4.222020in}{1.502783in}}%
\pgfpathlineto{\pgfqpoint{4.223749in}{1.487642in}}%
\pgfpathlineto{\pgfqpoint{4.224134in}{1.488372in}}%
\pgfpathlineto{\pgfqpoint{4.226632in}{1.496825in}}%
\pgfpathlineto{\pgfqpoint{4.227401in}{1.492092in}}%
\pgfpathlineto{\pgfqpoint{4.227593in}{1.496264in}}%
\pgfpathlineto{\pgfqpoint{4.228938in}{1.501178in}}%
\pgfpathlineto{\pgfqpoint{4.229514in}{1.499520in}}%
\pgfpathlineto{\pgfqpoint{4.229899in}{1.503650in}}%
\pgfpathlineto{\pgfqpoint{4.231052in}{1.498605in}}%
\pgfpathlineto{\pgfqpoint{4.231244in}{1.500856in}}%
\pgfpathlineto{\pgfqpoint{4.231820in}{1.496494in}}%
\pgfpathlineto{\pgfqpoint{4.232013in}{1.498105in}}%
\pgfpathlineto{\pgfqpoint{4.232205in}{1.497051in}}%
\pgfpathlineto{\pgfqpoint{4.232589in}{1.499313in}}%
\pgfpathlineto{\pgfqpoint{4.232781in}{1.498846in}}%
\pgfpathlineto{\pgfqpoint{4.233550in}{1.501371in}}%
\pgfpathlineto{\pgfqpoint{4.233166in}{1.496952in}}%
\pgfpathlineto{\pgfqpoint{4.233934in}{1.500079in}}%
\pgfpathlineto{\pgfqpoint{4.234126in}{1.497992in}}%
\pgfpathlineto{\pgfqpoint{4.234895in}{1.498930in}}%
\pgfpathlineto{\pgfqpoint{4.235279in}{1.504984in}}%
\pgfpathlineto{\pgfqpoint{4.236048in}{1.501226in}}%
\pgfpathlineto{\pgfqpoint{4.237393in}{1.493428in}}%
\pgfpathlineto{\pgfqpoint{4.238162in}{1.496371in}}%
\pgfpathlineto{\pgfqpoint{4.237778in}{1.492540in}}%
\pgfpathlineto{\pgfqpoint{4.238546in}{1.495708in}}%
\pgfpathlineto{\pgfqpoint{4.239892in}{1.490491in}}%
\pgfpathlineto{\pgfqpoint{4.240852in}{1.494425in}}%
\pgfpathlineto{\pgfqpoint{4.241045in}{1.492467in}}%
\pgfpathlineto{\pgfqpoint{4.241237in}{1.492706in}}%
\pgfpathlineto{\pgfqpoint{4.241429in}{1.490767in}}%
\pgfpathlineto{\pgfqpoint{4.241813in}{1.494184in}}%
\pgfpathlineto{\pgfqpoint{4.242198in}{1.492689in}}%
\pgfpathlineto{\pgfqpoint{4.242774in}{1.491022in}}%
\pgfpathlineto{\pgfqpoint{4.243158in}{1.491764in}}%
\pgfpathlineto{\pgfqpoint{4.244504in}{1.483977in}}%
\pgfpathlineto{\pgfqpoint{4.245849in}{1.488644in}}%
\pgfpathlineto{\pgfqpoint{4.246233in}{1.487645in}}%
\pgfpathlineto{\pgfqpoint{4.247194in}{1.486087in}}%
\pgfpathlineto{\pgfqpoint{4.247386in}{1.487478in}}%
\pgfpathlineto{\pgfqpoint{4.247963in}{1.482992in}}%
\pgfpathlineto{\pgfqpoint{4.248155in}{1.481930in}}%
\pgfpathlineto{\pgfqpoint{4.248347in}{1.485432in}}%
\pgfpathlineto{\pgfqpoint{4.248731in}{1.484565in}}%
\pgfpathlineto{\pgfqpoint{4.250076in}{1.489539in}}%
\pgfpathlineto{\pgfqpoint{4.250269in}{1.487811in}}%
\pgfpathlineto{\pgfqpoint{4.252575in}{1.494334in}}%
\pgfpathlineto{\pgfqpoint{4.253343in}{1.492688in}}%
\pgfpathlineto{\pgfqpoint{4.253920in}{1.494510in}}%
\pgfpathlineto{\pgfqpoint{4.254688in}{1.489669in}}%
\pgfpathlineto{\pgfqpoint{4.254881in}{1.489599in}}%
\pgfpathlineto{\pgfqpoint{4.255457in}{1.485059in}}%
\pgfpathlineto{\pgfqpoint{4.256034in}{1.486536in}}%
\pgfpathlineto{\pgfqpoint{4.256226in}{1.488806in}}%
\pgfpathlineto{\pgfqpoint{4.256802in}{1.486940in}}%
\pgfpathlineto{\pgfqpoint{4.257763in}{1.478660in}}%
\pgfpathlineto{\pgfqpoint{4.258147in}{1.479992in}}%
\pgfpathlineto{\pgfqpoint{4.258532in}{1.478284in}}%
\pgfpathlineto{\pgfqpoint{4.259877in}{1.491885in}}%
\pgfpathlineto{\pgfqpoint{4.260069in}{1.488910in}}%
\pgfpathlineto{\pgfqpoint{4.260838in}{1.492943in}}%
\pgfpathlineto{\pgfqpoint{4.261222in}{1.491787in}}%
\pgfpathlineto{\pgfqpoint{4.261414in}{1.493549in}}%
\pgfpathlineto{\pgfqpoint{4.262183in}{1.492476in}}%
\pgfpathlineto{\pgfqpoint{4.262567in}{1.496705in}}%
\pgfpathlineto{\pgfqpoint{4.263720in}{1.486458in}}%
\pgfpathlineto{\pgfqpoint{4.264297in}{1.491939in}}%
\pgfpathlineto{\pgfqpoint{4.264681in}{1.494210in}}%
\pgfpathlineto{\pgfqpoint{4.264873in}{1.490151in}}%
\pgfpathlineto{\pgfqpoint{4.265258in}{1.492362in}}%
\pgfpathlineto{\pgfqpoint{4.265450in}{1.489563in}}%
\pgfpathlineto{\pgfqpoint{4.266411in}{1.491690in}}%
\pgfpathlineto{\pgfqpoint{4.267756in}{1.489424in}}%
\pgfpathlineto{\pgfqpoint{4.268332in}{1.497786in}}%
\pgfpathlineto{\pgfqpoint{4.268909in}{1.491226in}}%
\pgfpathlineto{\pgfqpoint{4.269101in}{1.490179in}}%
\pgfpathlineto{\pgfqpoint{4.269293in}{1.492776in}}%
\pgfpathlineto{\pgfqpoint{4.270638in}{1.499288in}}%
\pgfpathlineto{\pgfqpoint{4.271023in}{1.497339in}}%
\pgfpathlineto{\pgfqpoint{4.271407in}{1.499840in}}%
\pgfpathlineto{\pgfqpoint{4.271984in}{1.497792in}}%
\pgfpathlineto{\pgfqpoint{4.272176in}{1.499876in}}%
\pgfpathlineto{\pgfqpoint{4.272752in}{1.496045in}}%
\pgfpathlineto{\pgfqpoint{4.273329in}{1.493482in}}%
\pgfpathlineto{\pgfqpoint{4.273521in}{1.496794in}}%
\pgfpathlineto{\pgfqpoint{4.274097in}{1.496263in}}%
\pgfpathlineto{\pgfqpoint{4.275250in}{1.505076in}}%
\pgfpathlineto{\pgfqpoint{4.275443in}{1.503745in}}%
\pgfpathlineto{\pgfqpoint{4.277172in}{1.510601in}}%
\pgfpathlineto{\pgfqpoint{4.278517in}{1.505076in}}%
\pgfpathlineto{\pgfqpoint{4.279286in}{1.513920in}}%
\pgfpathlineto{\pgfqpoint{4.279670in}{1.511144in}}%
\pgfpathlineto{\pgfqpoint{4.281400in}{1.499645in}}%
\pgfpathlineto{\pgfqpoint{4.282553in}{1.506724in}}%
\pgfpathlineto{\pgfqpoint{4.283129in}{1.506026in}}%
\pgfpathlineto{\pgfqpoint{4.284090in}{1.503781in}}%
\pgfpathlineto{\pgfqpoint{4.283706in}{1.507300in}}%
\pgfpathlineto{\pgfqpoint{4.284282in}{1.504459in}}%
\pgfpathlineto{\pgfqpoint{4.284859in}{1.512799in}}%
\pgfpathlineto{\pgfqpoint{4.285628in}{1.509529in}}%
\pgfpathlineto{\pgfqpoint{4.285820in}{1.508334in}}%
\pgfpathlineto{\pgfqpoint{4.286012in}{1.511910in}}%
\pgfpathlineto{\pgfqpoint{4.286396in}{1.517446in}}%
\pgfpathlineto{\pgfqpoint{4.287165in}{1.517388in}}%
\pgfpathlineto{\pgfqpoint{4.289279in}{1.503200in}}%
\pgfpathlineto{\pgfqpoint{4.289663in}{1.506109in}}%
\pgfpathlineto{\pgfqpoint{4.289855in}{1.510050in}}%
\pgfpathlineto{\pgfqpoint{4.290624in}{1.503165in}}%
\pgfpathlineto{\pgfqpoint{4.290816in}{1.505499in}}%
\pgfpathlineto{\pgfqpoint{4.291777in}{1.503923in}}%
\pgfpathlineto{\pgfqpoint{4.291969in}{1.503068in}}%
\pgfpathlineto{\pgfqpoint{4.292161in}{1.505102in}}%
\pgfpathlineto{\pgfqpoint{4.292546in}{1.504441in}}%
\pgfpathlineto{\pgfqpoint{4.292930in}{1.504359in}}%
\pgfpathlineto{\pgfqpoint{4.294275in}{1.512447in}}%
\pgfpathlineto{\pgfqpoint{4.295428in}{1.504132in}}%
\pgfpathlineto{\pgfqpoint{4.295620in}{1.506127in}}%
\pgfpathlineto{\pgfqpoint{4.296197in}{1.503401in}}%
\pgfpathlineto{\pgfqpoint{4.296389in}{1.503384in}}%
\pgfpathlineto{\pgfqpoint{4.297158in}{1.497979in}}%
\pgfpathlineto{\pgfqpoint{4.297734in}{1.498234in}}%
\pgfpathlineto{\pgfqpoint{4.297926in}{1.502405in}}%
\pgfpathlineto{\pgfqpoint{4.298503in}{1.495411in}}%
\pgfpathlineto{\pgfqpoint{4.298695in}{1.498759in}}%
\pgfpathlineto{\pgfqpoint{4.300232in}{1.487065in}}%
\pgfpathlineto{\pgfqpoint{4.300424in}{1.488214in}}%
\pgfpathlineto{\pgfqpoint{4.300809in}{1.492937in}}%
\pgfpathlineto{\pgfqpoint{4.301193in}{1.486149in}}%
\pgfpathlineto{\pgfqpoint{4.301385in}{1.488480in}}%
\pgfpathlineto{\pgfqpoint{4.302154in}{1.482991in}}%
\pgfpathlineto{\pgfqpoint{4.302730in}{1.484861in}}%
\pgfpathlineto{\pgfqpoint{4.303115in}{1.485509in}}%
\pgfpathlineto{\pgfqpoint{4.303691in}{1.482166in}}%
\pgfpathlineto{\pgfqpoint{4.304076in}{1.482232in}}%
\pgfpathlineto{\pgfqpoint{4.305036in}{1.477009in}}%
\pgfpathlineto{\pgfqpoint{4.305805in}{1.478487in}}%
\pgfpathlineto{\pgfqpoint{4.305997in}{1.478348in}}%
\pgfpathlineto{\pgfqpoint{4.306574in}{1.476517in}}%
\pgfpathlineto{\pgfqpoint{4.306766in}{1.479207in}}%
\pgfpathlineto{\pgfqpoint{4.308495in}{1.490994in}}%
\pgfpathlineto{\pgfqpoint{4.309649in}{1.488676in}}%
\pgfpathlineto{\pgfqpoint{4.308880in}{1.492648in}}%
\pgfpathlineto{\pgfqpoint{4.309841in}{1.489113in}}%
\pgfpathlineto{\pgfqpoint{4.310033in}{1.490814in}}%
\pgfpathlineto{\pgfqpoint{4.310417in}{1.488240in}}%
\pgfpathlineto{\pgfqpoint{4.310609in}{1.489195in}}%
\pgfpathlineto{\pgfqpoint{4.312147in}{1.481886in}}%
\pgfpathlineto{\pgfqpoint{4.312339in}{1.482074in}}%
\pgfpathlineto{\pgfqpoint{4.314261in}{1.468906in}}%
\pgfpathlineto{\pgfqpoint{4.314453in}{1.469133in}}%
\pgfpathlineto{\pgfqpoint{4.314645in}{1.469846in}}%
\pgfpathlineto{\pgfqpoint{4.315029in}{1.468497in}}%
\pgfpathlineto{\pgfqpoint{4.315221in}{1.466179in}}%
\pgfpathlineto{\pgfqpoint{4.315606in}{1.472436in}}%
\pgfpathlineto{\pgfqpoint{4.315798in}{1.473341in}}%
\pgfpathlineto{\pgfqpoint{4.316182in}{1.470971in}}%
\pgfpathlineto{\pgfqpoint{4.316759in}{1.467595in}}%
\pgfpathlineto{\pgfqpoint{4.317143in}{1.472425in}}%
\pgfpathlineto{\pgfqpoint{4.317912in}{1.466205in}}%
\pgfpathlineto{\pgfqpoint{4.318296in}{1.470589in}}%
\pgfpathlineto{\pgfqpoint{4.319833in}{1.480419in}}%
\pgfpathlineto{\pgfqpoint{4.321179in}{1.466562in}}%
\pgfpathlineto{\pgfqpoint{4.321755in}{1.467291in}}%
\pgfpathlineto{\pgfqpoint{4.322524in}{1.467105in}}%
\pgfpathlineto{\pgfqpoint{4.323869in}{1.477704in}}%
\pgfpathlineto{\pgfqpoint{4.324253in}{1.474233in}}%
\pgfpathlineto{\pgfqpoint{4.325214in}{1.475220in}}%
\pgfpathlineto{\pgfqpoint{4.325598in}{1.477436in}}%
\pgfpathlineto{\pgfqpoint{4.326175in}{1.475415in}}%
\pgfpathlineto{\pgfqpoint{4.326559in}{1.474067in}}%
\pgfpathlineto{\pgfqpoint{4.326751in}{1.476583in}}%
\pgfpathlineto{\pgfqpoint{4.326944in}{1.475838in}}%
\pgfpathlineto{\pgfqpoint{4.327520in}{1.477676in}}%
\pgfpathlineto{\pgfqpoint{4.327904in}{1.474818in}}%
\pgfpathlineto{\pgfqpoint{4.328481in}{1.471578in}}%
\pgfpathlineto{\pgfqpoint{4.328673in}{1.475062in}}%
\pgfpathlineto{\pgfqpoint{4.329057in}{1.474065in}}%
\pgfpathlineto{\pgfqpoint{4.330979in}{1.486711in}}%
\pgfpathlineto{\pgfqpoint{4.331363in}{1.486067in}}%
\pgfpathlineto{\pgfqpoint{4.332709in}{1.498032in}}%
\pgfpathlineto{\pgfqpoint{4.333285in}{1.496400in}}%
\pgfpathlineto{\pgfqpoint{4.333477in}{1.495273in}}%
\pgfpathlineto{\pgfqpoint{4.333862in}{1.498276in}}%
\pgfpathlineto{\pgfqpoint{4.334246in}{1.501847in}}%
\pgfpathlineto{\pgfqpoint{4.334823in}{1.499471in}}%
\pgfpathlineto{\pgfqpoint{4.335015in}{1.497433in}}%
\pgfpathlineto{\pgfqpoint{4.335591in}{1.502450in}}%
\pgfpathlineto{\pgfqpoint{4.335783in}{1.502521in}}%
\pgfpathlineto{\pgfqpoint{4.335976in}{1.503841in}}%
\pgfpathlineto{\pgfqpoint{4.336360in}{1.501678in}}%
\pgfpathlineto{\pgfqpoint{4.337705in}{1.496918in}}%
\pgfpathlineto{\pgfqpoint{4.339435in}{1.511845in}}%
\pgfpathlineto{\pgfqpoint{4.339819in}{1.509599in}}%
\pgfpathlineto{\pgfqpoint{4.340780in}{1.503033in}}%
\pgfpathlineto{\pgfqpoint{4.340972in}{1.504692in}}%
\pgfpathlineto{\pgfqpoint{4.341356in}{1.509525in}}%
\pgfpathlineto{\pgfqpoint{4.342317in}{1.508270in}}%
\pgfpathlineto{\pgfqpoint{4.344623in}{1.492497in}}%
\pgfpathlineto{\pgfqpoint{4.345007in}{1.497538in}}%
\pgfpathlineto{\pgfqpoint{4.346160in}{1.501109in}}%
\pgfpathlineto{\pgfqpoint{4.346353in}{1.499521in}}%
\pgfpathlineto{\pgfqpoint{4.347313in}{1.494737in}}%
\pgfpathlineto{\pgfqpoint{4.347506in}{1.492644in}}%
\pgfpathlineto{\pgfqpoint{4.348274in}{1.493007in}}%
\pgfpathlineto{\pgfqpoint{4.349619in}{1.500855in}}%
\pgfpathlineto{\pgfqpoint{4.349812in}{1.499277in}}%
\pgfpathlineto{\pgfqpoint{4.350580in}{1.499376in}}%
\pgfpathlineto{\pgfqpoint{4.351733in}{1.502574in}}%
\pgfpathlineto{\pgfqpoint{4.352694in}{1.497756in}}%
\pgfpathlineto{\pgfqpoint{4.352886in}{1.498086in}}%
\pgfpathlineto{\pgfqpoint{4.354039in}{1.505339in}}%
\pgfpathlineto{\pgfqpoint{4.355577in}{1.493953in}}%
\pgfpathlineto{\pgfqpoint{4.356345in}{1.497252in}}%
\pgfpathlineto{\pgfqpoint{4.355961in}{1.492834in}}%
\pgfpathlineto{\pgfqpoint{4.356730in}{1.494566in}}%
\pgfpathlineto{\pgfqpoint{4.358844in}{1.490181in}}%
\pgfpathlineto{\pgfqpoint{4.359036in}{1.491848in}}%
\pgfpathlineto{\pgfqpoint{4.359420in}{1.487769in}}%
\pgfpathlineto{\pgfqpoint{4.359612in}{1.488326in}}%
\pgfpathlineto{\pgfqpoint{4.360957in}{1.480783in}}%
\pgfpathlineto{\pgfqpoint{4.361726in}{1.478980in}}%
\pgfpathlineto{\pgfqpoint{4.361534in}{1.480936in}}%
\pgfpathlineto{\pgfqpoint{4.361918in}{1.479843in}}%
\pgfpathlineto{\pgfqpoint{4.362687in}{1.486652in}}%
\pgfpathlineto{\pgfqpoint{4.363071in}{1.483490in}}%
\pgfpathlineto{\pgfqpoint{4.363456in}{1.482323in}}%
\pgfpathlineto{\pgfqpoint{4.364032in}{1.484050in}}%
\pgfpathlineto{\pgfqpoint{4.364224in}{1.483377in}}%
\pgfpathlineto{\pgfqpoint{4.364609in}{1.483263in}}%
\pgfpathlineto{\pgfqpoint{4.365377in}{1.483628in}}%
\pgfpathlineto{\pgfqpoint{4.365762in}{1.479041in}}%
\pgfpathlineto{\pgfqpoint{4.368260in}{1.489083in}}%
\pgfpathlineto{\pgfqpoint{4.368836in}{1.484424in}}%
\pgfpathlineto{\pgfqpoint{4.369605in}{1.485376in}}%
\pgfpathlineto{\pgfqpoint{4.369989in}{1.483459in}}%
\pgfpathlineto{\pgfqpoint{4.370181in}{1.484628in}}%
\pgfpathlineto{\pgfqpoint{4.370566in}{1.487688in}}%
\pgfpathlineto{\pgfqpoint{4.371142in}{1.483230in}}%
\pgfpathlineto{\pgfqpoint{4.371719in}{1.480194in}}%
\pgfpathlineto{\pgfqpoint{4.372295in}{1.481312in}}%
\pgfpathlineto{\pgfqpoint{4.374025in}{1.466260in}}%
\pgfpathlineto{\pgfqpoint{4.374601in}{1.470846in}}%
\pgfpathlineto{\pgfqpoint{4.375754in}{1.476046in}}%
\pgfpathlineto{\pgfqpoint{4.374986in}{1.468825in}}%
\pgfpathlineto{\pgfqpoint{4.376139in}{1.475655in}}%
\pgfpathlineto{\pgfqpoint{4.376523in}{1.472271in}}%
\pgfpathlineto{\pgfqpoint{4.377099in}{1.475239in}}%
\pgfpathlineto{\pgfqpoint{4.377292in}{1.475099in}}%
\pgfpathlineto{\pgfqpoint{4.377484in}{1.478042in}}%
\pgfpathlineto{\pgfqpoint{4.378252in}{1.474332in}}%
\pgfpathlineto{\pgfqpoint{4.379213in}{1.469855in}}%
\pgfpathlineto{\pgfqpoint{4.378637in}{1.476504in}}%
\pgfpathlineto{\pgfqpoint{4.379405in}{1.471051in}}%
\pgfpathlineto{\pgfqpoint{4.379598in}{1.472017in}}%
\pgfpathlineto{\pgfqpoint{4.379790in}{1.468248in}}%
\pgfpathlineto{\pgfqpoint{4.379982in}{1.467905in}}%
\pgfpathlineto{\pgfqpoint{4.381327in}{1.459916in}}%
\pgfpathlineto{\pgfqpoint{4.382096in}{1.464547in}}%
\pgfpathlineto{\pgfqpoint{4.382288in}{1.461944in}}%
\pgfpathlineto{\pgfqpoint{4.384018in}{1.453330in}}%
\pgfpathlineto{\pgfqpoint{4.384402in}{1.455263in}}%
\pgfpathlineto{\pgfqpoint{4.384594in}{1.451489in}}%
\pgfpathlineto{\pgfqpoint{4.385171in}{1.454277in}}%
\pgfpathlineto{\pgfqpoint{4.385363in}{1.454247in}}%
\pgfpathlineto{\pgfqpoint{4.385555in}{1.451684in}}%
\pgfpathlineto{\pgfqpoint{4.385939in}{1.455646in}}%
\pgfpathlineto{\pgfqpoint{4.386516in}{1.453546in}}%
\pgfpathlineto{\pgfqpoint{4.388053in}{1.448135in}}%
\pgfpathlineto{\pgfqpoint{4.389590in}{1.453983in}}%
\pgfpathlineto{\pgfqpoint{4.394395in}{1.425225in}}%
\pgfpathlineto{\pgfqpoint{4.394587in}{1.428946in}}%
\pgfpathlineto{\pgfqpoint{4.395355in}{1.427695in}}%
\pgfpathlineto{\pgfqpoint{4.396508in}{1.421362in}}%
\pgfpathlineto{\pgfqpoint{4.395740in}{1.430155in}}%
\pgfpathlineto{\pgfqpoint{4.396893in}{1.422559in}}%
\pgfpathlineto{\pgfqpoint{4.397277in}{1.421059in}}%
\pgfpathlineto{\pgfqpoint{4.398238in}{1.426713in}}%
\pgfpathlineto{\pgfqpoint{4.398622in}{1.426317in}}%
\pgfpathlineto{\pgfqpoint{4.401120in}{1.436448in}}%
\pgfpathlineto{\pgfqpoint{4.401505in}{1.435507in}}%
\pgfpathlineto{\pgfqpoint{4.403042in}{1.428623in}}%
\pgfpathlineto{\pgfqpoint{4.403234in}{1.431816in}}%
\pgfpathlineto{\pgfqpoint{4.404003in}{1.426953in}}%
\pgfpathlineto{\pgfqpoint{4.404195in}{1.424571in}}%
\pgfpathlineto{\pgfqpoint{4.404579in}{1.427590in}}%
\pgfpathlineto{\pgfqpoint{4.404964in}{1.425924in}}%
\pgfpathlineto{\pgfqpoint{4.405540in}{1.429730in}}%
\pgfpathlineto{\pgfqpoint{4.406117in}{1.428050in}}%
\pgfpathlineto{\pgfqpoint{4.406501in}{1.426116in}}%
\pgfpathlineto{\pgfqpoint{4.406886in}{1.430166in}}%
\pgfpathlineto{\pgfqpoint{4.407078in}{1.430580in}}%
\pgfpathlineto{\pgfqpoint{4.408423in}{1.419786in}}%
\pgfpathlineto{\pgfqpoint{4.408615in}{1.420443in}}%
\pgfpathlineto{\pgfqpoint{4.408807in}{1.422276in}}%
\pgfpathlineto{\pgfqpoint{4.409192in}{1.418025in}}%
\pgfpathlineto{\pgfqpoint{4.409384in}{1.418145in}}%
\pgfpathlineto{\pgfqpoint{4.409576in}{1.418183in}}%
\pgfpathlineto{\pgfqpoint{4.409960in}{1.416406in}}%
\pgfpathlineto{\pgfqpoint{4.410345in}{1.420586in}}%
\pgfpathlineto{\pgfqpoint{4.411498in}{1.424333in}}%
\pgfpathlineto{\pgfqpoint{4.410921in}{1.420101in}}%
\pgfpathlineto{\pgfqpoint{4.411690in}{1.424206in}}%
\pgfpathlineto{\pgfqpoint{4.411882in}{1.421976in}}%
\pgfpathlineto{\pgfqpoint{4.412266in}{1.425047in}}%
\pgfpathlineto{\pgfqpoint{4.412458in}{1.429650in}}%
\pgfpathlineto{\pgfqpoint{4.413419in}{1.428371in}}%
\pgfpathlineto{\pgfqpoint{4.414188in}{1.431533in}}%
\pgfpathlineto{\pgfqpoint{4.415725in}{1.436498in}}%
\pgfpathlineto{\pgfqpoint{4.416686in}{1.434423in}}%
\pgfpathlineto{\pgfqpoint{4.416878in}{1.435104in}}%
\pgfpathlineto{\pgfqpoint{4.417839in}{1.440278in}}%
\pgfpathlineto{\pgfqpoint{4.418416in}{1.436612in}}%
\pgfpathlineto{\pgfqpoint{4.418608in}{1.434164in}}%
\pgfpathlineto{\pgfqpoint{4.419184in}{1.437949in}}%
\pgfpathlineto{\pgfqpoint{4.420529in}{1.442484in}}%
\pgfpathlineto{\pgfqpoint{4.420722in}{1.441237in}}%
\pgfpathlineto{\pgfqpoint{4.422259in}{1.437557in}}%
\pgfpathlineto{\pgfqpoint{4.423028in}{1.446437in}}%
\pgfpathlineto{\pgfqpoint{4.423796in}{1.444303in}}%
\pgfpathlineto{\pgfqpoint{4.424181in}{1.441745in}}%
\pgfpathlineto{\pgfqpoint{4.424757in}{1.444771in}}%
\pgfpathlineto{\pgfqpoint{4.424949in}{1.444665in}}%
\pgfpathlineto{\pgfqpoint{4.425910in}{1.448178in}}%
\pgfpathlineto{\pgfqpoint{4.426102in}{1.447215in}}%
\pgfpathlineto{\pgfqpoint{4.427255in}{1.449947in}}%
\pgfpathlineto{\pgfqpoint{4.427447in}{1.448207in}}%
\pgfpathlineto{\pgfqpoint{4.428024in}{1.448468in}}%
\pgfpathlineto{\pgfqpoint{4.429946in}{1.439314in}}%
\pgfpathlineto{\pgfqpoint{4.431099in}{1.453433in}}%
\pgfpathlineto{\pgfqpoint{4.431483in}{1.450202in}}%
\pgfpathlineto{\pgfqpoint{4.431675in}{1.448219in}}%
\pgfpathlineto{\pgfqpoint{4.432060in}{1.454426in}}%
\pgfpathlineto{\pgfqpoint{4.432444in}{1.458067in}}%
\pgfpathlineto{\pgfqpoint{4.432828in}{1.454114in}}%
\pgfpathlineto{\pgfqpoint{4.433020in}{1.454337in}}%
\pgfpathlineto{\pgfqpoint{4.433213in}{1.452723in}}%
\pgfpathlineto{\pgfqpoint{4.433981in}{1.454021in}}%
\pgfpathlineto{\pgfqpoint{4.434558in}{1.456878in}}%
\pgfpathlineto{\pgfqpoint{4.435134in}{1.454916in}}%
\pgfpathlineto{\pgfqpoint{4.435326in}{1.453343in}}%
\pgfpathlineto{\pgfqpoint{4.435711in}{1.455754in}}%
\pgfpathlineto{\pgfqpoint{4.437248in}{1.465979in}}%
\pgfpathlineto{\pgfqpoint{4.437440in}{1.465803in}}%
\pgfpathlineto{\pgfqpoint{4.438209in}{1.456918in}}%
\pgfpathlineto{\pgfqpoint{4.438978in}{1.457516in}}%
\pgfpathlineto{\pgfqpoint{4.440515in}{1.467066in}}%
\pgfpathlineto{\pgfqpoint{4.440707in}{1.466801in}}%
\pgfpathlineto{\pgfqpoint{4.441091in}{1.470905in}}%
\pgfpathlineto{\pgfqpoint{4.441860in}{1.469322in}}%
\pgfpathlineto{\pgfqpoint{4.442052in}{1.466495in}}%
\pgfpathlineto{\pgfqpoint{4.443013in}{1.466985in}}%
\pgfpathlineto{\pgfqpoint{4.444358in}{1.473988in}}%
\pgfpathlineto{\pgfqpoint{4.444743in}{1.473561in}}%
\pgfpathlineto{\pgfqpoint{4.444935in}{1.471127in}}%
\pgfpathlineto{\pgfqpoint{4.445319in}{1.476761in}}%
\pgfpathlineto{\pgfqpoint{4.445511in}{1.476527in}}%
\pgfpathlineto{\pgfqpoint{4.445703in}{1.476959in}}%
\pgfpathlineto{\pgfqpoint{4.445896in}{1.484377in}}%
\pgfpathlineto{\pgfqpoint{4.446664in}{1.479236in}}%
\pgfpathlineto{\pgfqpoint{4.447817in}{1.466033in}}%
\pgfpathlineto{\pgfqpoint{4.448970in}{1.466464in}}%
\pgfpathlineto{\pgfqpoint{4.449547in}{1.467919in}}%
\pgfpathlineto{\pgfqpoint{4.449739in}{1.471361in}}%
\pgfpathlineto{\pgfqpoint{4.450508in}{1.468230in}}%
\pgfpathlineto{\pgfqpoint{4.451661in}{1.465228in}}%
\pgfpathlineto{\pgfqpoint{4.452045in}{1.465884in}}%
\pgfpathlineto{\pgfqpoint{4.452429in}{1.464668in}}%
\pgfpathlineto{\pgfqpoint{4.452814in}{1.468796in}}%
\pgfpathlineto{\pgfqpoint{4.454543in}{1.477686in}}%
\pgfpathlineto{\pgfqpoint{4.453198in}{1.467629in}}%
\pgfpathlineto{\pgfqpoint{4.454735in}{1.476407in}}%
\pgfpathlineto{\pgfqpoint{4.454927in}{1.473449in}}%
\pgfpathlineto{\pgfqpoint{4.455312in}{1.477980in}}%
\pgfpathlineto{\pgfqpoint{4.455696in}{1.477822in}}%
\pgfpathlineto{\pgfqpoint{4.455888in}{1.478040in}}%
\pgfpathlineto{\pgfqpoint{4.457041in}{1.469623in}}%
\pgfpathlineto{\pgfqpoint{4.457234in}{1.470976in}}%
\pgfpathlineto{\pgfqpoint{4.457618in}{1.473918in}}%
\pgfpathlineto{\pgfqpoint{4.458194in}{1.473249in}}%
\pgfpathlineto{\pgfqpoint{4.458579in}{1.466774in}}%
\pgfpathlineto{\pgfqpoint{4.459540in}{1.467775in}}%
\pgfpathlineto{\pgfqpoint{4.459924in}{1.469884in}}%
\pgfpathlineto{\pgfqpoint{4.460308in}{1.465733in}}%
\pgfpathlineto{\pgfqpoint{4.460885in}{1.467142in}}%
\pgfpathlineto{\pgfqpoint{4.461846in}{1.461832in}}%
\pgfpathlineto{\pgfqpoint{4.462038in}{1.463299in}}%
\pgfpathlineto{\pgfqpoint{4.462422in}{1.458564in}}%
\pgfpathlineto{\pgfqpoint{4.463575in}{1.454308in}}%
\pgfpathlineto{\pgfqpoint{4.463767in}{1.454631in}}%
\pgfpathlineto{\pgfqpoint{4.465881in}{1.466737in}}%
\pgfpathlineto{\pgfqpoint{4.464728in}{1.454259in}}%
\pgfpathlineto{\pgfqpoint{4.466073in}{1.465581in}}%
\pgfpathlineto{\pgfqpoint{4.466842in}{1.464202in}}%
\pgfpathlineto{\pgfqpoint{4.467226in}{1.469094in}}%
\pgfpathlineto{\pgfqpoint{4.467995in}{1.466310in}}%
\pgfpathlineto{\pgfqpoint{4.469340in}{1.462591in}}%
\pgfpathlineto{\pgfqpoint{4.470109in}{1.464269in}}%
\pgfpathlineto{\pgfqpoint{4.469917in}{1.461183in}}%
\pgfpathlineto{\pgfqpoint{4.470301in}{1.462021in}}%
\pgfpathlineto{\pgfqpoint{4.470877in}{1.459561in}}%
\pgfpathlineto{\pgfqpoint{4.471262in}{1.462319in}}%
\pgfpathlineto{\pgfqpoint{4.471646in}{1.463853in}}%
\pgfpathlineto{\pgfqpoint{4.472799in}{1.458606in}}%
\pgfpathlineto{\pgfqpoint{4.472991in}{1.459001in}}%
\pgfpathlineto{\pgfqpoint{4.473952in}{1.453869in}}%
\pgfpathlineto{\pgfqpoint{4.474336in}{1.455289in}}%
\pgfpathlineto{\pgfqpoint{4.474721in}{1.460756in}}%
\pgfpathlineto{\pgfqpoint{4.475105in}{1.449424in}}%
\pgfpathlineto{\pgfqpoint{4.475874in}{1.444863in}}%
\pgfpathlineto{\pgfqpoint{4.476066in}{1.448223in}}%
\pgfpathlineto{\pgfqpoint{4.476642in}{1.453577in}}%
\pgfpathlineto{\pgfqpoint{4.477219in}{1.450708in}}%
\pgfpathlineto{\pgfqpoint{4.477603in}{1.444054in}}%
\pgfpathlineto{\pgfqpoint{4.478180in}{1.451974in}}%
\pgfpathlineto{\pgfqpoint{4.478372in}{1.454001in}}%
\pgfpathlineto{\pgfqpoint{4.478948in}{1.451335in}}%
\pgfpathlineto{\pgfqpoint{4.480486in}{1.446473in}}%
\pgfpathlineto{\pgfqpoint{4.480870in}{1.450889in}}%
\pgfpathlineto{\pgfqpoint{4.481831in}{1.450068in}}%
\pgfpathlineto{\pgfqpoint{4.482215in}{1.447067in}}%
\pgfpathlineto{\pgfqpoint{4.482984in}{1.448878in}}%
\pgfpathlineto{\pgfqpoint{4.483368in}{1.448976in}}%
\pgfpathlineto{\pgfqpoint{4.483561in}{1.446791in}}%
\pgfpathlineto{\pgfqpoint{4.484137in}{1.451766in}}%
\pgfpathlineto{\pgfqpoint{4.484906in}{1.455318in}}%
\pgfpathlineto{\pgfqpoint{4.486059in}{1.467074in}}%
\pgfpathlineto{\pgfqpoint{4.486443in}{1.466238in}}%
\pgfpathlineto{\pgfqpoint{4.487596in}{1.459983in}}%
\pgfpathlineto{\pgfqpoint{4.487788in}{1.461131in}}%
\pgfpathlineto{\pgfqpoint{4.488173in}{1.463460in}}%
\pgfpathlineto{\pgfqpoint{4.488557in}{1.459905in}}%
\pgfpathlineto{\pgfqpoint{4.488749in}{1.460104in}}%
\pgfpathlineto{\pgfqpoint{4.488941in}{1.459408in}}%
\pgfpathlineto{\pgfqpoint{4.490479in}{1.468836in}}%
\pgfpathlineto{\pgfqpoint{4.491632in}{1.464736in}}%
\pgfpathlineto{\pgfqpoint{4.492016in}{1.468940in}}%
\pgfpathlineto{\pgfqpoint{4.492592in}{1.466534in}}%
\pgfpathlineto{\pgfqpoint{4.493169in}{1.463283in}}%
\pgfpathlineto{\pgfqpoint{4.493553in}{1.464857in}}%
\pgfpathlineto{\pgfqpoint{4.494514in}{1.468587in}}%
\pgfpathlineto{\pgfqpoint{4.494706in}{1.467098in}}%
\pgfpathlineto{\pgfqpoint{4.495091in}{1.466517in}}%
\pgfpathlineto{\pgfqpoint{4.496244in}{1.471389in}}%
\pgfpathlineto{\pgfqpoint{4.496436in}{1.469520in}}%
\pgfpathlineto{\pgfqpoint{4.496628in}{1.468847in}}%
\pgfpathlineto{\pgfqpoint{4.497012in}{1.471194in}}%
\pgfpathlineto{\pgfqpoint{4.497781in}{1.473535in}}%
\pgfpathlineto{\pgfqpoint{4.498165in}{1.472081in}}%
\pgfpathlineto{\pgfqpoint{4.498357in}{1.470577in}}%
\pgfpathlineto{\pgfqpoint{4.498550in}{1.475169in}}%
\pgfpathlineto{\pgfqpoint{4.498742in}{1.474473in}}%
\pgfpathlineto{\pgfqpoint{4.498934in}{1.476410in}}%
\pgfpathlineto{\pgfqpoint{4.499510in}{1.473455in}}%
\pgfpathlineto{\pgfqpoint{4.501816in}{1.461164in}}%
\pgfpathlineto{\pgfqpoint{4.502009in}{1.463354in}}%
\pgfpathlineto{\pgfqpoint{4.502969in}{1.461727in}}%
\pgfpathlineto{\pgfqpoint{4.503354in}{1.461360in}}%
\pgfpathlineto{\pgfqpoint{4.503930in}{1.463606in}}%
\pgfpathlineto{\pgfqpoint{4.506044in}{1.446746in}}%
\pgfpathlineto{\pgfqpoint{4.506236in}{1.448026in}}%
\pgfpathlineto{\pgfqpoint{4.506429in}{1.446262in}}%
\pgfpathlineto{\pgfqpoint{4.506621in}{1.446736in}}%
\pgfpathlineto{\pgfqpoint{4.506813in}{1.443016in}}%
\pgfpathlineto{\pgfqpoint{4.507582in}{1.448726in}}%
\pgfpathlineto{\pgfqpoint{4.508735in}{1.457068in}}%
\pgfpathlineto{\pgfqpoint{4.509311in}{1.455277in}}%
\pgfpathlineto{\pgfqpoint{4.510464in}{1.446556in}}%
\pgfpathlineto{\pgfqpoint{4.511617in}{1.453252in}}%
\pgfpathlineto{\pgfqpoint{4.511809in}{1.452633in}}%
\pgfpathlineto{\pgfqpoint{4.512386in}{1.453968in}}%
\pgfpathlineto{\pgfqpoint{4.512578in}{1.451359in}}%
\pgfpathlineto{\pgfqpoint{4.513539in}{1.448992in}}%
\pgfpathlineto{\pgfqpoint{4.512962in}{1.453638in}}%
\pgfpathlineto{\pgfqpoint{4.513731in}{1.450334in}}%
\pgfpathlineto{\pgfqpoint{4.515653in}{1.464733in}}%
\pgfpathlineto{\pgfqpoint{4.515845in}{1.461165in}}%
\pgfpathlineto{\pgfqpoint{4.516037in}{1.461172in}}%
\pgfpathlineto{\pgfqpoint{4.516229in}{1.459023in}}%
\pgfpathlineto{\pgfqpoint{4.516613in}{1.464477in}}%
\pgfpathlineto{\pgfqpoint{4.516806in}{1.463163in}}%
\pgfpathlineto{\pgfqpoint{4.516998in}{1.466339in}}%
\pgfpathlineto{\pgfqpoint{4.517382in}{1.462286in}}%
\pgfpathlineto{\pgfqpoint{4.517574in}{1.464361in}}%
\pgfpathlineto{\pgfqpoint{4.518727in}{1.456312in}}%
\pgfpathlineto{\pgfqpoint{4.518919in}{1.456791in}}%
\pgfpathlineto{\pgfqpoint{4.519112in}{1.458058in}}%
\pgfpathlineto{\pgfqpoint{4.519496in}{1.455821in}}%
\pgfpathlineto{\pgfqpoint{4.519688in}{1.453941in}}%
\pgfpathlineto{\pgfqpoint{4.520072in}{1.458071in}}%
\pgfpathlineto{\pgfqpoint{4.520841in}{1.456207in}}%
\pgfpathlineto{\pgfqpoint{4.521418in}{1.461511in}}%
\pgfpathlineto{\pgfqpoint{4.522186in}{1.462195in}}%
\pgfpathlineto{\pgfqpoint{4.522378in}{1.458723in}}%
\pgfpathlineto{\pgfqpoint{4.523916in}{1.449354in}}%
\pgfpathlineto{\pgfqpoint{4.524108in}{1.450456in}}%
\pgfpathlineto{\pgfqpoint{4.525453in}{1.442499in}}%
\pgfpathlineto{\pgfqpoint{4.526414in}{1.437857in}}%
\pgfpathlineto{\pgfqpoint{4.526798in}{1.439665in}}%
\pgfpathlineto{\pgfqpoint{4.527759in}{1.444050in}}%
\pgfpathlineto{\pgfqpoint{4.528143in}{1.443255in}}%
\pgfpathlineto{\pgfqpoint{4.529489in}{1.436154in}}%
\pgfpathlineto{\pgfqpoint{4.529873in}{1.438027in}}%
\pgfpathlineto{\pgfqpoint{4.530257in}{1.437014in}}%
\pgfpathlineto{\pgfqpoint{4.530450in}{1.438336in}}%
\pgfpathlineto{\pgfqpoint{4.530642in}{1.439526in}}%
\pgfpathlineto{\pgfqpoint{4.531026in}{1.436680in}}%
\pgfpathlineto{\pgfqpoint{4.532948in}{1.427778in}}%
\pgfpathlineto{\pgfqpoint{4.533140in}{1.430789in}}%
\pgfpathlineto{\pgfqpoint{4.533909in}{1.428901in}}%
\pgfpathlineto{\pgfqpoint{4.535062in}{1.421919in}}%
\pgfpathlineto{\pgfqpoint{4.535638in}{1.424558in}}%
\pgfpathlineto{\pgfqpoint{4.535830in}{1.428805in}}%
\pgfpathlineto{\pgfqpoint{4.536407in}{1.423833in}}%
\pgfpathlineto{\pgfqpoint{4.536791in}{1.426854in}}%
\pgfpathlineto{\pgfqpoint{4.536983in}{1.427251in}}%
\pgfpathlineto{\pgfqpoint{4.537175in}{1.425344in}}%
\pgfpathlineto{\pgfqpoint{4.537560in}{1.425851in}}%
\pgfpathlineto{\pgfqpoint{4.538328in}{1.429492in}}%
\pgfpathlineto{\pgfqpoint{4.538521in}{1.423984in}}%
\pgfpathlineto{\pgfqpoint{4.539289in}{1.418171in}}%
\pgfpathlineto{\pgfqpoint{4.539866in}{1.419508in}}%
\pgfpathlineto{\pgfqpoint{4.541787in}{1.431886in}}%
\pgfpathlineto{\pgfqpoint{4.542172in}{1.430949in}}%
\pgfpathlineto{\pgfqpoint{4.542556in}{1.430717in}}%
\pgfpathlineto{\pgfqpoint{4.543325in}{1.435975in}}%
\pgfpathlineto{\pgfqpoint{4.543901in}{1.434649in}}%
\pgfpathlineto{\pgfqpoint{4.546976in}{1.414077in}}%
\pgfpathlineto{\pgfqpoint{4.547745in}{1.417307in}}%
\pgfpathlineto{\pgfqpoint{4.548129in}{1.416532in}}%
\pgfpathlineto{\pgfqpoint{4.548321in}{1.414859in}}%
\pgfpathlineto{\pgfqpoint{4.548705in}{1.417715in}}%
\pgfpathlineto{\pgfqpoint{4.549474in}{1.432048in}}%
\pgfpathlineto{\pgfqpoint{4.550435in}{1.429407in}}%
\pgfpathlineto{\pgfqpoint{4.550819in}{1.427025in}}%
\pgfpathlineto{\pgfqpoint{4.551204in}{1.430935in}}%
\pgfpathlineto{\pgfqpoint{4.551780in}{1.428222in}}%
\pgfpathlineto{\pgfqpoint{4.552357in}{1.431892in}}%
\pgfpathlineto{\pgfqpoint{4.552933in}{1.429729in}}%
\pgfpathlineto{\pgfqpoint{4.554663in}{1.410482in}}%
\pgfpathlineto{\pgfqpoint{4.555431in}{1.416227in}}%
\pgfpathlineto{\pgfqpoint{4.556200in}{1.415090in}}%
\pgfpathlineto{\pgfqpoint{4.557545in}{1.407241in}}%
\pgfpathlineto{\pgfqpoint{4.557737in}{1.408738in}}%
\pgfpathlineto{\pgfqpoint{4.558314in}{1.412621in}}%
\pgfpathlineto{\pgfqpoint{4.558890in}{1.410266in}}%
\pgfpathlineto{\pgfqpoint{4.559083in}{1.407846in}}%
\pgfpathlineto{\pgfqpoint{4.559467in}{1.410806in}}%
\pgfpathlineto{\pgfqpoint{4.560043in}{1.409289in}}%
\pgfpathlineto{\pgfqpoint{4.560620in}{1.415633in}}%
\pgfpathlineto{\pgfqpoint{4.561196in}{1.411485in}}%
\pgfpathlineto{\pgfqpoint{4.564079in}{1.388253in}}%
\pgfpathlineto{\pgfqpoint{4.564655in}{1.394247in}}%
\pgfpathlineto{\pgfqpoint{4.566385in}{1.401388in}}%
\pgfpathlineto{\pgfqpoint{4.566769in}{1.402431in}}%
\pgfpathlineto{\pgfqpoint{4.567346in}{1.398832in}}%
\pgfpathlineto{\pgfqpoint{4.567730in}{1.393936in}}%
\pgfpathlineto{\pgfqpoint{4.568499in}{1.395220in}}%
\pgfpathlineto{\pgfqpoint{4.568691in}{1.395019in}}%
\pgfpathlineto{\pgfqpoint{4.570420in}{1.407947in}}%
\pgfpathlineto{\pgfqpoint{4.571958in}{1.404218in}}%
\pgfpathlineto{\pgfqpoint{4.572150in}{1.403842in}}%
\pgfpathlineto{\pgfqpoint{4.572342in}{1.405181in}}%
\pgfpathlineto{\pgfqpoint{4.572726in}{1.404817in}}%
\pgfpathlineto{\pgfqpoint{4.573111in}{1.406750in}}%
\pgfpathlineto{\pgfqpoint{4.574648in}{1.400852in}}%
\pgfpathlineto{\pgfqpoint{4.575609in}{1.398533in}}%
\pgfpathlineto{\pgfqpoint{4.575801in}{1.399224in}}%
\pgfpathlineto{\pgfqpoint{4.577531in}{1.405613in}}%
\pgfpathlineto{\pgfqpoint{4.576378in}{1.399026in}}%
\pgfpathlineto{\pgfqpoint{4.577723in}{1.404003in}}%
\pgfpathlineto{\pgfqpoint{4.577915in}{1.400694in}}%
\pgfpathlineto{\pgfqpoint{4.578492in}{1.408695in}}%
\pgfpathlineto{\pgfqpoint{4.579645in}{1.404772in}}%
\pgfpathlineto{\pgfqpoint{4.579837in}{1.406934in}}%
\pgfpathlineto{\pgfqpoint{4.580029in}{1.407953in}}%
\pgfpathlineto{\pgfqpoint{4.580221in}{1.404829in}}%
\pgfpathlineto{\pgfqpoint{4.580413in}{1.405771in}}%
\pgfpathlineto{\pgfqpoint{4.581374in}{1.401219in}}%
\pgfpathlineto{\pgfqpoint{4.581566in}{1.404866in}}%
\pgfpathlineto{\pgfqpoint{4.583296in}{1.413788in}}%
\pgfpathlineto{\pgfqpoint{4.583680in}{1.414807in}}%
\pgfpathlineto{\pgfqpoint{4.584833in}{1.421571in}}%
\pgfpathlineto{\pgfqpoint{4.585986in}{1.412521in}}%
\pgfpathlineto{\pgfqpoint{4.586178in}{1.413359in}}%
\pgfpathlineto{\pgfqpoint{4.586370in}{1.416370in}}%
\pgfpathlineto{\pgfqpoint{4.586563in}{1.411992in}}%
\pgfpathlineto{\pgfqpoint{4.587139in}{1.413458in}}%
\pgfpathlineto{\pgfqpoint{4.587523in}{1.408849in}}%
\pgfpathlineto{\pgfqpoint{4.588292in}{1.411151in}}%
\pgfpathlineto{\pgfqpoint{4.588484in}{1.414277in}}%
\pgfpathlineto{\pgfqpoint{4.589253in}{1.408776in}}%
\pgfpathlineto{\pgfqpoint{4.590598in}{1.401334in}}%
\pgfpathlineto{\pgfqpoint{4.592135in}{1.390252in}}%
\pgfpathlineto{\pgfqpoint{4.592328in}{1.390534in}}%
\pgfpathlineto{\pgfqpoint{4.594057in}{1.372810in}}%
\pgfpathlineto{\pgfqpoint{4.594826in}{1.376836in}}%
\pgfpathlineto{\pgfqpoint{4.595979in}{1.370681in}}%
\pgfpathlineto{\pgfqpoint{4.596555in}{1.372507in}}%
\pgfpathlineto{\pgfqpoint{4.597708in}{1.377974in}}%
\pgfpathlineto{\pgfqpoint{4.597132in}{1.371003in}}%
\pgfpathlineto{\pgfqpoint{4.597900in}{1.377699in}}%
\pgfpathlineto{\pgfqpoint{4.598093in}{1.377849in}}%
\pgfpathlineto{\pgfqpoint{4.598861in}{1.377596in}}%
\pgfpathlineto{\pgfqpoint{4.599630in}{1.383076in}}%
\pgfpathlineto{\pgfqpoint{4.601167in}{1.391109in}}%
\pgfpathlineto{\pgfqpoint{4.601359in}{1.392349in}}%
\pgfpathlineto{\pgfqpoint{4.601744in}{1.390059in}}%
\pgfpathlineto{\pgfqpoint{4.601936in}{1.390777in}}%
\pgfpathlineto{\pgfqpoint{4.602128in}{1.390024in}}%
\pgfpathlineto{\pgfqpoint{4.603089in}{1.381241in}}%
\pgfpathlineto{\pgfqpoint{4.603281in}{1.383904in}}%
\pgfpathlineto{\pgfqpoint{4.603473in}{1.385175in}}%
\pgfpathlineto{\pgfqpoint{4.603858in}{1.382012in}}%
\pgfpathlineto{\pgfqpoint{4.604050in}{1.379105in}}%
\pgfpathlineto{\pgfqpoint{4.604819in}{1.383306in}}%
\pgfpathlineto{\pgfqpoint{4.605972in}{1.384383in}}%
\pgfpathlineto{\pgfqpoint{4.606164in}{1.385893in}}%
\pgfpathlineto{\pgfqpoint{4.606548in}{1.384109in}}%
\pgfpathlineto{\pgfqpoint{4.606740in}{1.385453in}}%
\pgfpathlineto{\pgfqpoint{4.607317in}{1.381642in}}%
\pgfpathlineto{\pgfqpoint{4.607701in}{1.386609in}}%
\pgfpathlineto{\pgfqpoint{4.610007in}{1.376283in}}%
\pgfpathlineto{\pgfqpoint{4.610391in}{1.380978in}}%
\pgfpathlineto{\pgfqpoint{4.611160in}{1.377361in}}%
\pgfpathlineto{\pgfqpoint{4.611352in}{1.376656in}}%
\pgfpathlineto{\pgfqpoint{4.611737in}{1.379396in}}%
\pgfpathlineto{\pgfqpoint{4.613082in}{1.381948in}}%
\pgfpathlineto{\pgfqpoint{4.614427in}{1.385758in}}%
\pgfpathlineto{\pgfqpoint{4.614619in}{1.384283in}}%
\pgfpathlineto{\pgfqpoint{4.615964in}{1.375950in}}%
\pgfpathlineto{\pgfqpoint{4.616541in}{1.376399in}}%
\pgfpathlineto{\pgfqpoint{4.616733in}{1.378696in}}%
\pgfpathlineto{\pgfqpoint{4.617117in}{1.372121in}}%
\pgfpathlineto{\pgfqpoint{4.617502in}{1.377341in}}%
\pgfpathlineto{\pgfqpoint{4.617694in}{1.376725in}}%
\pgfpathlineto{\pgfqpoint{4.618078in}{1.378430in}}%
\pgfpathlineto{\pgfqpoint{4.618270in}{1.380867in}}%
\pgfpathlineto{\pgfqpoint{4.618655in}{1.374843in}}%
\pgfpathlineto{\pgfqpoint{4.618847in}{1.375299in}}%
\pgfpathlineto{\pgfqpoint{4.619231in}{1.375979in}}%
\pgfpathlineto{\pgfqpoint{4.619423in}{1.374630in}}%
\pgfpathlineto{\pgfqpoint{4.619615in}{1.371949in}}%
\pgfpathlineto{\pgfqpoint{4.620000in}{1.378615in}}%
\pgfpathlineto{\pgfqpoint{4.620192in}{1.377720in}}%
\pgfpathlineto{\pgfqpoint{4.620576in}{1.380137in}}%
\pgfpathlineto{\pgfqpoint{4.621153in}{1.379480in}}%
\pgfpathlineto{\pgfqpoint{4.622306in}{1.372999in}}%
\pgfpathlineto{\pgfqpoint{4.623651in}{1.378507in}}%
\pgfpathlineto{\pgfqpoint{4.623843in}{1.377968in}}%
\pgfpathlineto{\pgfqpoint{4.624420in}{1.386990in}}%
\pgfpathlineto{\pgfqpoint{4.624996in}{1.381386in}}%
\pgfpathlineto{\pgfqpoint{4.626534in}{1.371784in}}%
\pgfpathlineto{\pgfqpoint{4.626726in}{1.373198in}}%
\pgfpathlineto{\pgfqpoint{4.628071in}{1.377235in}}%
\pgfpathlineto{\pgfqpoint{4.628263in}{1.377086in}}%
\pgfpathlineto{\pgfqpoint{4.628840in}{1.371950in}}%
\pgfpathlineto{\pgfqpoint{4.629608in}{1.374559in}}%
\pgfpathlineto{\pgfqpoint{4.629993in}{1.374756in}}%
\pgfpathlineto{\pgfqpoint{4.631338in}{1.369063in}}%
\pgfpathlineto{\pgfqpoint{4.633067in}{1.377186in}}%
\pgfpathlineto{\pgfqpoint{4.633259in}{1.377174in}}%
\pgfpathlineto{\pgfqpoint{4.633644in}{1.378322in}}%
\pgfpathlineto{\pgfqpoint{4.634797in}{1.371837in}}%
\pgfpathlineto{\pgfqpoint{4.634989in}{1.375238in}}%
\pgfpathlineto{\pgfqpoint{4.635758in}{1.370833in}}%
\pgfpathlineto{\pgfqpoint{4.635950in}{1.370686in}}%
\pgfpathlineto{\pgfqpoint{4.637103in}{1.364034in}}%
\pgfpathlineto{\pgfqpoint{4.637487in}{1.365074in}}%
\pgfpathlineto{\pgfqpoint{4.638064in}{1.368684in}}%
\pgfpathlineto{\pgfqpoint{4.638640in}{1.365780in}}%
\pgfpathlineto{\pgfqpoint{4.639985in}{1.361236in}}%
\pgfpathlineto{\pgfqpoint{4.640177in}{1.362885in}}%
\pgfpathlineto{\pgfqpoint{4.640946in}{1.368331in}}%
\pgfpathlineto{\pgfqpoint{4.641523in}{1.368281in}}%
\pgfpathlineto{\pgfqpoint{4.642868in}{1.359001in}}%
\pgfpathlineto{\pgfqpoint{4.643444in}{1.360212in}}%
\pgfpathlineto{\pgfqpoint{4.645942in}{1.376476in}}%
\pgfpathlineto{\pgfqpoint{4.647288in}{1.372471in}}%
\pgfpathlineto{\pgfqpoint{4.647480in}{1.372005in}}%
\pgfpathlineto{\pgfqpoint{4.647672in}{1.374247in}}%
\pgfpathlineto{\pgfqpoint{4.647864in}{1.374636in}}%
\pgfpathlineto{\pgfqpoint{4.648056in}{1.373207in}}%
\pgfpathlineto{\pgfqpoint{4.649786in}{1.390127in}}%
\pgfpathlineto{\pgfqpoint{4.650554in}{1.392640in}}%
\pgfpathlineto{\pgfqpoint{4.650747in}{1.391415in}}%
\pgfpathlineto{\pgfqpoint{4.650939in}{1.390489in}}%
\pgfpathlineto{\pgfqpoint{4.651515in}{1.392537in}}%
\pgfpathlineto{\pgfqpoint{4.651708in}{1.392096in}}%
\pgfpathlineto{\pgfqpoint{4.651900in}{1.392462in}}%
\pgfpathlineto{\pgfqpoint{4.652092in}{1.390562in}}%
\pgfpathlineto{\pgfqpoint{4.652284in}{1.390957in}}%
\pgfpathlineto{\pgfqpoint{4.653245in}{1.386569in}}%
\pgfpathlineto{\pgfqpoint{4.653821in}{1.387615in}}%
\pgfpathlineto{\pgfqpoint{4.654590in}{1.387103in}}%
\pgfpathlineto{\pgfqpoint{4.655167in}{1.392315in}}%
\pgfpathlineto{\pgfqpoint{4.656127in}{1.389021in}}%
\pgfpathlineto{\pgfqpoint{4.655743in}{1.392908in}}%
\pgfpathlineto{\pgfqpoint{4.656320in}{1.389076in}}%
\pgfpathlineto{\pgfqpoint{4.656896in}{1.388760in}}%
\pgfpathlineto{\pgfqpoint{4.657665in}{1.391815in}}%
\pgfpathlineto{\pgfqpoint{4.659586in}{1.378387in}}%
\pgfpathlineto{\pgfqpoint{4.659779in}{1.379660in}}%
\pgfpathlineto{\pgfqpoint{4.659971in}{1.378657in}}%
\pgfpathlineto{\pgfqpoint{4.660355in}{1.373009in}}%
\pgfpathlineto{\pgfqpoint{4.660932in}{1.377804in}}%
\pgfpathlineto{\pgfqpoint{4.661892in}{1.382194in}}%
\pgfpathlineto{\pgfqpoint{4.662277in}{1.381635in}}%
\pgfpathlineto{\pgfqpoint{4.662853in}{1.376726in}}%
\pgfpathlineto{\pgfqpoint{4.663622in}{1.377100in}}%
\pgfpathlineto{\pgfqpoint{4.664583in}{1.383606in}}%
\pgfpathlineto{\pgfqpoint{4.664967in}{1.381087in}}%
\pgfpathlineto{\pgfqpoint{4.666504in}{1.372964in}}%
\pgfpathlineto{\pgfqpoint{4.666697in}{1.373009in}}%
\pgfpathlineto{\pgfqpoint{4.667850in}{1.378602in}}%
\pgfpathlineto{\pgfqpoint{4.668042in}{1.376064in}}%
\pgfpathlineto{\pgfqpoint{4.669003in}{1.371932in}}%
\pgfpathlineto{\pgfqpoint{4.668618in}{1.377463in}}%
\pgfpathlineto{\pgfqpoint{4.669387in}{1.372245in}}%
\pgfpathlineto{\pgfqpoint{4.669579in}{1.374892in}}%
\pgfpathlineto{\pgfqpoint{4.669963in}{1.371898in}}%
\pgfpathlineto{\pgfqpoint{4.670348in}{1.374391in}}%
\pgfpathlineto{\pgfqpoint{4.671693in}{1.366581in}}%
\pgfpathlineto{\pgfqpoint{4.671885in}{1.366916in}}%
\pgfpathlineto{\pgfqpoint{4.673230in}{1.379048in}}%
\pgfpathlineto{\pgfqpoint{4.673615in}{1.376948in}}%
\pgfpathlineto{\pgfqpoint{4.674960in}{1.388149in}}%
\pgfpathlineto{\pgfqpoint{4.676113in}{1.387698in}}%
\pgfpathlineto{\pgfqpoint{4.676497in}{1.383954in}}%
\pgfpathlineto{\pgfqpoint{4.676882in}{1.387980in}}%
\pgfpathlineto{\pgfqpoint{4.678227in}{1.391757in}}%
\pgfpathlineto{\pgfqpoint{4.678803in}{1.389500in}}%
\pgfpathlineto{\pgfqpoint{4.678611in}{1.393450in}}%
\pgfpathlineto{\pgfqpoint{4.678995in}{1.392265in}}%
\pgfpathlineto{\pgfqpoint{4.679572in}{1.396015in}}%
\pgfpathlineto{\pgfqpoint{4.680148in}{1.399795in}}%
\pgfpathlineto{\pgfqpoint{4.680725in}{1.398308in}}%
\pgfpathlineto{\pgfqpoint{4.680917in}{1.395320in}}%
\pgfpathlineto{\pgfqpoint{4.681301in}{1.399352in}}%
\pgfpathlineto{\pgfqpoint{4.681878in}{1.397085in}}%
\pgfpathlineto{\pgfqpoint{4.682262in}{1.397173in}}%
\pgfpathlineto{\pgfqpoint{4.682647in}{1.392416in}}%
\pgfpathlineto{\pgfqpoint{4.683415in}{1.392830in}}%
\pgfpathlineto{\pgfqpoint{4.683800in}{1.396543in}}%
\pgfpathlineto{\pgfqpoint{4.684953in}{1.396319in}}%
\pgfpathlineto{\pgfqpoint{4.685145in}{1.394292in}}%
\pgfpathlineto{\pgfqpoint{4.685721in}{1.397970in}}%
\pgfpathlineto{\pgfqpoint{4.685913in}{1.399742in}}%
\pgfpathlineto{\pgfqpoint{4.686298in}{1.396724in}}%
\pgfpathlineto{\pgfqpoint{4.686682in}{1.397289in}}%
\pgfpathlineto{\pgfqpoint{4.686874in}{1.397215in}}%
\pgfpathlineto{\pgfqpoint{4.687643in}{1.396963in}}%
\pgfpathlineto{\pgfqpoint{4.688219in}{1.400525in}}%
\pgfpathlineto{\pgfqpoint{4.688412in}{1.400459in}}%
\pgfpathlineto{\pgfqpoint{4.688604in}{1.399053in}}%
\pgfpathlineto{\pgfqpoint{4.689180in}{1.402990in}}%
\pgfpathlineto{\pgfqpoint{4.689372in}{1.401222in}}%
\pgfpathlineto{\pgfqpoint{4.689565in}{1.401019in}}%
\pgfpathlineto{\pgfqpoint{4.691294in}{1.391498in}}%
\pgfpathlineto{\pgfqpoint{4.692639in}{1.386883in}}%
\pgfpathlineto{\pgfqpoint{4.693024in}{1.388429in}}%
\pgfpathlineto{\pgfqpoint{4.694177in}{1.393949in}}%
\pgfpathlineto{\pgfqpoint{4.695522in}{1.381521in}}%
\pgfpathlineto{\pgfqpoint{4.695714in}{1.383295in}}%
\pgfpathlineto{\pgfqpoint{4.696675in}{1.396139in}}%
\pgfpathlineto{\pgfqpoint{4.697251in}{1.394699in}}%
\pgfpathlineto{\pgfqpoint{4.697443in}{1.395830in}}%
\pgfpathlineto{\pgfqpoint{4.697828in}{1.393793in}}%
\pgfpathlineto{\pgfqpoint{4.698596in}{1.394218in}}%
\pgfpathlineto{\pgfqpoint{4.698981in}{1.388651in}}%
\pgfpathlineto{\pgfqpoint{4.699557in}{1.393664in}}%
\pgfpathlineto{\pgfqpoint{4.700134in}{1.390212in}}%
\pgfpathlineto{\pgfqpoint{4.700326in}{1.389815in}}%
\pgfpathlineto{\pgfqpoint{4.700518in}{1.390889in}}%
\pgfpathlineto{\pgfqpoint{4.701671in}{1.396546in}}%
\pgfpathlineto{\pgfqpoint{4.701863in}{1.393770in}}%
\pgfpathlineto{\pgfqpoint{4.702632in}{1.394663in}}%
\pgfpathlineto{\pgfqpoint{4.703209in}{1.389906in}}%
\pgfpathlineto{\pgfqpoint{4.704746in}{1.399254in}}%
\pgfpathlineto{\pgfqpoint{4.704938in}{1.398702in}}%
\pgfpathlineto{\pgfqpoint{4.705515in}{1.395547in}}%
\pgfpathlineto{\pgfqpoint{4.705899in}{1.396485in}}%
\pgfpathlineto{\pgfqpoint{4.706860in}{1.405693in}}%
\pgfpathlineto{\pgfqpoint{4.707244in}{1.403622in}}%
\pgfpathlineto{\pgfqpoint{4.707436in}{1.403958in}}%
\pgfpathlineto{\pgfqpoint{4.707628in}{1.403009in}}%
\pgfpathlineto{\pgfqpoint{4.707821in}{1.402900in}}%
\pgfpathlineto{\pgfqpoint{4.708205in}{1.402488in}}%
\pgfpathlineto{\pgfqpoint{4.709358in}{1.392870in}}%
\pgfpathlineto{\pgfqpoint{4.709742in}{1.397537in}}%
\pgfpathlineto{\pgfqpoint{4.711472in}{1.391969in}}%
\pgfpathlineto{\pgfqpoint{4.712433in}{1.394342in}}%
\pgfpathlineto{\pgfqpoint{4.712817in}{1.394055in}}%
\pgfpathlineto{\pgfqpoint{4.713009in}{1.393132in}}%
\pgfpathlineto{\pgfqpoint{4.713586in}{1.393331in}}%
\pgfpathlineto{\pgfqpoint{4.714931in}{1.399756in}}%
\pgfpathlineto{\pgfqpoint{4.715123in}{1.397641in}}%
\pgfpathlineto{\pgfqpoint{4.715507in}{1.402119in}}%
\pgfpathlineto{\pgfqpoint{4.715699in}{1.400564in}}%
\pgfpathlineto{\pgfqpoint{4.716468in}{1.405059in}}%
\pgfpathlineto{\pgfqpoint{4.716852in}{1.402176in}}%
\pgfpathlineto{\pgfqpoint{4.717429in}{1.399439in}}%
\pgfpathlineto{\pgfqpoint{4.718582in}{1.388306in}}%
\pgfpathlineto{\pgfqpoint{4.718774in}{1.390501in}}%
\pgfpathlineto{\pgfqpoint{4.720311in}{1.396304in}}%
\pgfpathlineto{\pgfqpoint{4.721657in}{1.393599in}}%
\pgfpathlineto{\pgfqpoint{4.721849in}{1.394265in}}%
\pgfpathlineto{\pgfqpoint{4.723194in}{1.397279in}}%
\pgfpathlineto{\pgfqpoint{4.723386in}{1.396237in}}%
\pgfpathlineto{\pgfqpoint{4.723963in}{1.398349in}}%
\pgfpathlineto{\pgfqpoint{4.724347in}{1.404119in}}%
\pgfpathlineto{\pgfqpoint{4.725116in}{1.402326in}}%
\pgfpathlineto{\pgfqpoint{4.726653in}{1.393414in}}%
\pgfpathlineto{\pgfqpoint{4.726845in}{1.394241in}}%
\pgfpathlineto{\pgfqpoint{4.727422in}{1.392122in}}%
\pgfpathlineto{\pgfqpoint{4.727614in}{1.392565in}}%
\pgfpathlineto{\pgfqpoint{4.728383in}{1.381491in}}%
\pgfpathlineto{\pgfqpoint{4.728959in}{1.383337in}}%
\pgfpathlineto{\pgfqpoint{4.731265in}{1.398418in}}%
\pgfpathlineto{\pgfqpoint{4.731842in}{1.395133in}}%
\pgfpathlineto{\pgfqpoint{4.732226in}{1.393494in}}%
\pgfpathlineto{\pgfqpoint{4.732610in}{1.396300in}}%
\pgfpathlineto{\pgfqpoint{4.732802in}{1.397958in}}%
\pgfpathlineto{\pgfqpoint{4.733187in}{1.392654in}}%
\pgfpathlineto{\pgfqpoint{4.733955in}{1.394012in}}%
\pgfpathlineto{\pgfqpoint{4.734148in}{1.398111in}}%
\pgfpathlineto{\pgfqpoint{4.735108in}{1.396133in}}%
\pgfpathlineto{\pgfqpoint{4.735877in}{1.397574in}}%
\pgfpathlineto{\pgfqpoint{4.735685in}{1.395919in}}%
\pgfpathlineto{\pgfqpoint{4.736069in}{1.397351in}}%
\pgfpathlineto{\pgfqpoint{4.736261in}{1.395917in}}%
\pgfpathlineto{\pgfqpoint{4.736454in}{1.399239in}}%
\pgfpathlineto{\pgfqpoint{4.737030in}{1.407792in}}%
\pgfpathlineto{\pgfqpoint{4.737799in}{1.405831in}}%
\pgfpathlineto{\pgfqpoint{4.739528in}{1.411300in}}%
\pgfpathlineto{\pgfqpoint{4.738183in}{1.404264in}}%
\pgfpathlineto{\pgfqpoint{4.739720in}{1.411077in}}%
\pgfpathlineto{\pgfqpoint{4.741258in}{1.407689in}}%
\pgfpathlineto{\pgfqpoint{4.741450in}{1.402432in}}%
\pgfpathlineto{\pgfqpoint{4.742219in}{1.405436in}}%
\pgfpathlineto{\pgfqpoint{4.743948in}{1.415758in}}%
\pgfpathlineto{\pgfqpoint{4.744332in}{1.414675in}}%
\pgfpathlineto{\pgfqpoint{4.745485in}{1.411398in}}%
\pgfpathlineto{\pgfqpoint{4.745870in}{1.412849in}}%
\pgfpathlineto{\pgfqpoint{4.746831in}{1.416172in}}%
\pgfpathlineto{\pgfqpoint{4.747599in}{1.419532in}}%
\pgfpathlineto{\pgfqpoint{4.747791in}{1.417743in}}%
\pgfpathlineto{\pgfqpoint{4.749329in}{1.403342in}}%
\pgfpathlineto{\pgfqpoint{4.749905in}{1.406354in}}%
\pgfpathlineto{\pgfqpoint{4.751827in}{1.419431in}}%
\pgfpathlineto{\pgfqpoint{4.750290in}{1.405634in}}%
\pgfpathlineto{\pgfqpoint{4.752019in}{1.418680in}}%
\pgfpathlineto{\pgfqpoint{4.752404in}{1.420762in}}%
\pgfpathlineto{\pgfqpoint{4.752596in}{1.420583in}}%
\pgfpathlineto{\pgfqpoint{4.753941in}{1.415301in}}%
\pgfpathlineto{\pgfqpoint{4.754710in}{1.408319in}}%
\pgfpathlineto{\pgfqpoint{4.755670in}{1.411482in}}%
\pgfpathlineto{\pgfqpoint{4.756631in}{1.415547in}}%
\pgfpathlineto{\pgfqpoint{4.757016in}{1.413582in}}%
\pgfpathlineto{\pgfqpoint{4.757592in}{1.407578in}}%
\pgfpathlineto{\pgfqpoint{4.758361in}{1.408731in}}%
\pgfpathlineto{\pgfqpoint{4.758937in}{1.410263in}}%
\pgfpathlineto{\pgfqpoint{4.759322in}{1.408552in}}%
\pgfpathlineto{\pgfqpoint{4.760090in}{1.407195in}}%
\pgfpathlineto{\pgfqpoint{4.759898in}{1.409763in}}%
\pgfpathlineto{\pgfqpoint{4.760282in}{1.407571in}}%
\pgfpathlineto{\pgfqpoint{4.761243in}{1.415338in}}%
\pgfpathlineto{\pgfqpoint{4.762012in}{1.414220in}}%
\pgfpathlineto{\pgfqpoint{4.763741in}{1.408026in}}%
\pgfpathlineto{\pgfqpoint{4.765279in}{1.401238in}}%
\pgfpathlineto{\pgfqpoint{4.766816in}{1.411498in}}%
\pgfpathlineto{\pgfqpoint{4.768161in}{1.404622in}}%
\pgfpathlineto{\pgfqpoint{4.768353in}{1.405913in}}%
\pgfpathlineto{\pgfqpoint{4.768738in}{1.403030in}}%
\pgfpathlineto{\pgfqpoint{4.769122in}{1.406604in}}%
\pgfpathlineto{\pgfqpoint{4.770275in}{1.413135in}}%
\pgfpathlineto{\pgfqpoint{4.770852in}{1.404933in}}%
\pgfpathlineto{\pgfqpoint{4.771428in}{1.410248in}}%
\pgfpathlineto{\pgfqpoint{4.772581in}{1.415261in}}%
\pgfpathlineto{\pgfqpoint{4.772773in}{1.412871in}}%
\pgfpathlineto{\pgfqpoint{4.773926in}{1.417002in}}%
\pgfpathlineto{\pgfqpoint{4.774311in}{1.415188in}}%
\pgfpathlineto{\pgfqpoint{4.775272in}{1.412873in}}%
\pgfpathlineto{\pgfqpoint{4.775079in}{1.416000in}}%
\pgfpathlineto{\pgfqpoint{4.775464in}{1.413052in}}%
\pgfpathlineto{\pgfqpoint{4.776617in}{1.414359in}}%
\pgfpathlineto{\pgfqpoint{4.776809in}{1.414063in}}%
\pgfpathlineto{\pgfqpoint{4.777001in}{1.415769in}}%
\pgfpathlineto{\pgfqpoint{4.777385in}{1.415166in}}%
\pgfpathlineto{\pgfqpoint{4.777578in}{1.416607in}}%
\pgfpathlineto{\pgfqpoint{4.777770in}{1.413287in}}%
\pgfpathlineto{\pgfqpoint{4.778346in}{1.415051in}}%
\pgfpathlineto{\pgfqpoint{4.778538in}{1.412741in}}%
\pgfpathlineto{\pgfqpoint{4.779115in}{1.416581in}}%
\pgfpathlineto{\pgfqpoint{4.779307in}{1.414710in}}%
\pgfpathlineto{\pgfqpoint{4.780268in}{1.418316in}}%
\pgfpathlineto{\pgfqpoint{4.780652in}{1.416654in}}%
\pgfpathlineto{\pgfqpoint{4.781037in}{1.411811in}}%
\pgfpathlineto{\pgfqpoint{4.781421in}{1.417894in}}%
\pgfpathlineto{\pgfqpoint{4.781613in}{1.416462in}}%
\pgfpathlineto{\pgfqpoint{4.782190in}{1.417674in}}%
\pgfpathlineto{\pgfqpoint{4.783727in}{1.407609in}}%
\pgfpathlineto{\pgfqpoint{4.784496in}{1.403976in}}%
\pgfpathlineto{\pgfqpoint{4.784688in}{1.405441in}}%
\pgfpathlineto{\pgfqpoint{4.785456in}{1.408438in}}%
\pgfpathlineto{\pgfqpoint{4.785649in}{1.405305in}}%
\pgfpathlineto{\pgfqpoint{4.785841in}{1.406666in}}%
\pgfpathlineto{\pgfqpoint{4.788915in}{1.388165in}}%
\pgfpathlineto{\pgfqpoint{4.789300in}{1.390934in}}%
\pgfpathlineto{\pgfqpoint{4.789684in}{1.387087in}}%
\pgfpathlineto{\pgfqpoint{4.791029in}{1.376703in}}%
\pgfpathlineto{\pgfqpoint{4.791414in}{1.377472in}}%
\pgfpathlineto{\pgfqpoint{4.791990in}{1.376459in}}%
\pgfpathlineto{\pgfqpoint{4.792182in}{1.377821in}}%
\pgfpathlineto{\pgfqpoint{4.792374in}{1.379843in}}%
\pgfpathlineto{\pgfqpoint{4.792759in}{1.374751in}}%
\pgfpathlineto{\pgfqpoint{4.794296in}{1.362704in}}%
\pgfpathlineto{\pgfqpoint{4.795065in}{1.366129in}}%
\pgfpathlineto{\pgfqpoint{4.794680in}{1.361907in}}%
\pgfpathlineto{\pgfqpoint{4.795257in}{1.363955in}}%
\pgfpathlineto{\pgfqpoint{4.795833in}{1.359379in}}%
\pgfpathlineto{\pgfqpoint{4.796410in}{1.362867in}}%
\pgfpathlineto{\pgfqpoint{4.798524in}{1.375361in}}%
\pgfpathlineto{\pgfqpoint{4.798716in}{1.374311in}}%
\pgfpathlineto{\pgfqpoint{4.799100in}{1.369904in}}%
\pgfpathlineto{\pgfqpoint{4.799677in}{1.373029in}}%
\pgfpathlineto{\pgfqpoint{4.799869in}{1.375533in}}%
\pgfpathlineto{\pgfqpoint{4.800253in}{1.367277in}}%
\pgfpathlineto{\pgfqpoint{4.800446in}{1.364339in}}%
\pgfpathlineto{\pgfqpoint{4.801214in}{1.368194in}}%
\pgfpathlineto{\pgfqpoint{4.803136in}{1.358564in}}%
\pgfpathlineto{\pgfqpoint{4.804097in}{1.364581in}}%
\pgfpathlineto{\pgfqpoint{4.804289in}{1.363688in}}%
\pgfpathlineto{\pgfqpoint{4.805634in}{1.352842in}}%
\pgfpathlineto{\pgfqpoint{4.805826in}{1.353979in}}%
\pgfpathlineto{\pgfqpoint{4.806018in}{1.354022in}}%
\pgfpathlineto{\pgfqpoint{4.806403in}{1.357200in}}%
\pgfpathlineto{\pgfqpoint{4.806787in}{1.353451in}}%
\pgfpathlineto{\pgfqpoint{4.806979in}{1.351332in}}%
\pgfpathlineto{\pgfqpoint{4.807556in}{1.356382in}}%
\pgfpathlineto{\pgfqpoint{4.808709in}{1.360217in}}%
\pgfpathlineto{\pgfqpoint{4.809093in}{1.355554in}}%
\pgfpathlineto{\pgfqpoint{4.809862in}{1.356287in}}%
\pgfpathlineto{\pgfqpoint{4.810246in}{1.359174in}}%
\pgfpathlineto{\pgfqpoint{4.810630in}{1.355167in}}%
\pgfpathlineto{\pgfqpoint{4.814666in}{1.336147in}}%
\pgfpathlineto{\pgfqpoint{4.814858in}{1.336467in}}%
\pgfpathlineto{\pgfqpoint{4.815050in}{1.338287in}}%
\pgfpathlineto{\pgfqpoint{4.815435in}{1.335139in}}%
\pgfpathlineto{\pgfqpoint{4.815627in}{1.337383in}}%
\pgfpathlineto{\pgfqpoint{4.816972in}{1.332477in}}%
\pgfpathlineto{\pgfqpoint{4.817741in}{1.335888in}}%
\pgfpathlineto{\pgfqpoint{4.817933in}{1.333955in}}%
\pgfpathlineto{\pgfqpoint{4.818317in}{1.329921in}}%
\pgfpathlineto{\pgfqpoint{4.818894in}{1.334943in}}%
\pgfpathlineto{\pgfqpoint{4.819086in}{1.332832in}}%
\pgfpathlineto{\pgfqpoint{4.820239in}{1.341805in}}%
\pgfpathlineto{\pgfqpoint{4.821007in}{1.339102in}}%
\pgfpathlineto{\pgfqpoint{4.822929in}{1.328880in}}%
\pgfpathlineto{\pgfqpoint{4.823698in}{1.337233in}}%
\pgfpathlineto{\pgfqpoint{4.824274in}{1.334582in}}%
\pgfpathlineto{\pgfqpoint{4.825427in}{1.339287in}}%
\pgfpathlineto{\pgfqpoint{4.824851in}{1.334189in}}%
\pgfpathlineto{\pgfqpoint{4.825812in}{1.338932in}}%
\pgfpathlineto{\pgfqpoint{4.826196in}{1.336000in}}%
\pgfpathlineto{\pgfqpoint{4.826773in}{1.338381in}}%
\pgfpathlineto{\pgfqpoint{4.826965in}{1.339210in}}%
\pgfpathlineto{\pgfqpoint{4.827157in}{1.336135in}}%
\pgfpathlineto{\pgfqpoint{4.827349in}{1.333376in}}%
\pgfpathlineto{\pgfqpoint{4.827733in}{1.336917in}}%
\pgfpathlineto{\pgfqpoint{4.828118in}{1.335543in}}%
\pgfpathlineto{\pgfqpoint{4.828310in}{1.335945in}}%
\pgfpathlineto{\pgfqpoint{4.828502in}{1.333628in}}%
\pgfpathlineto{\pgfqpoint{4.829271in}{1.336657in}}%
\pgfpathlineto{\pgfqpoint{4.829463in}{1.337505in}}%
\pgfpathlineto{\pgfqpoint{4.829847in}{1.335226in}}%
\pgfpathlineto{\pgfqpoint{4.830232in}{1.332242in}}%
\pgfpathlineto{\pgfqpoint{4.830424in}{1.334070in}}%
\pgfpathlineto{\pgfqpoint{4.831192in}{1.342758in}}%
\pgfpathlineto{\pgfqpoint{4.831961in}{1.340928in}}%
\pgfpathlineto{\pgfqpoint{4.832153in}{1.340700in}}%
\pgfpathlineto{\pgfqpoint{4.833114in}{1.347515in}}%
\pgfpathlineto{\pgfqpoint{4.833498in}{1.344848in}}%
\pgfpathlineto{\pgfqpoint{4.835036in}{1.351634in}}%
\pgfpathlineto{\pgfqpoint{4.833883in}{1.344751in}}%
\pgfpathlineto{\pgfqpoint{4.835228in}{1.350200in}}%
\pgfpathlineto{\pgfqpoint{4.835612in}{1.346852in}}%
\pgfpathlineto{\pgfqpoint{4.836189in}{1.351408in}}%
\pgfpathlineto{\pgfqpoint{4.836381in}{1.353461in}}%
\pgfpathlineto{\pgfqpoint{4.836765in}{1.347757in}}%
\pgfpathlineto{\pgfqpoint{4.837726in}{1.340456in}}%
\pgfpathlineto{\pgfqpoint{4.837918in}{1.341496in}}%
\pgfpathlineto{\pgfqpoint{4.838110in}{1.344611in}}%
\pgfpathlineto{\pgfqpoint{4.838879in}{1.341196in}}%
\pgfpathlineto{\pgfqpoint{4.839071in}{1.342042in}}%
\pgfpathlineto{\pgfqpoint{4.839456in}{1.338977in}}%
\pgfpathlineto{\pgfqpoint{4.839648in}{1.339673in}}%
\pgfpathlineto{\pgfqpoint{4.839840in}{1.339769in}}%
\pgfpathlineto{\pgfqpoint{4.840032in}{1.341762in}}%
\pgfpathlineto{\pgfqpoint{4.840416in}{1.336324in}}%
\pgfpathlineto{\pgfqpoint{4.840801in}{1.338287in}}%
\pgfpathlineto{\pgfqpoint{4.840993in}{1.338993in}}%
\pgfpathlineto{\pgfqpoint{4.841185in}{1.337412in}}%
\pgfpathlineto{\pgfqpoint{4.841762in}{1.337886in}}%
\pgfpathlineto{\pgfqpoint{4.841954in}{1.338052in}}%
\pgfpathlineto{\pgfqpoint{4.843491in}{1.329720in}}%
\pgfpathlineto{\pgfqpoint{4.844260in}{1.333795in}}%
\pgfpathlineto{\pgfqpoint{4.844644in}{1.333606in}}%
\pgfpathlineto{\pgfqpoint{4.845221in}{1.330423in}}%
\pgfpathlineto{\pgfqpoint{4.845797in}{1.331766in}}%
\pgfpathlineto{\pgfqpoint{4.847911in}{1.341160in}}%
\pgfpathlineto{\pgfqpoint{4.850217in}{1.344572in}}%
\pgfpathlineto{\pgfqpoint{4.848295in}{1.340627in}}%
\pgfpathlineto{\pgfqpoint{4.850409in}{1.343887in}}%
\pgfpathlineto{\pgfqpoint{4.850601in}{1.341781in}}%
\pgfpathlineto{\pgfqpoint{4.850986in}{1.346965in}}%
\pgfpathlineto{\pgfqpoint{4.851370in}{1.345568in}}%
\pgfpathlineto{\pgfqpoint{4.851754in}{1.345387in}}%
\pgfpathlineto{\pgfqpoint{4.852331in}{1.343099in}}%
\pgfpathlineto{\pgfqpoint{4.852523in}{1.345155in}}%
\pgfpathlineto{\pgfqpoint{4.852715in}{1.348892in}}%
\pgfpathlineto{\pgfqpoint{4.853292in}{1.341435in}}%
\pgfpathlineto{\pgfqpoint{4.853676in}{1.346232in}}%
\pgfpathlineto{\pgfqpoint{4.854060in}{1.347965in}}%
\pgfpathlineto{\pgfqpoint{4.854253in}{1.345825in}}%
\pgfpathlineto{\pgfqpoint{4.855213in}{1.341971in}}%
\pgfpathlineto{\pgfqpoint{4.856174in}{1.349532in}}%
\pgfpathlineto{\pgfqpoint{4.856559in}{1.347285in}}%
\pgfpathlineto{\pgfqpoint{4.856751in}{1.347313in}}%
\pgfpathlineto{\pgfqpoint{4.858288in}{1.362246in}}%
\pgfpathlineto{\pgfqpoint{4.858480in}{1.360860in}}%
\pgfpathlineto{\pgfqpoint{4.858672in}{1.361375in}}%
\pgfpathlineto{\pgfqpoint{4.858865in}{1.359931in}}%
\pgfpathlineto{\pgfqpoint{4.859057in}{1.357416in}}%
\pgfpathlineto{\pgfqpoint{4.859249in}{1.360850in}}%
\pgfpathlineto{\pgfqpoint{4.859825in}{1.360526in}}%
\pgfpathlineto{\pgfqpoint{4.860594in}{1.363421in}}%
\pgfpathlineto{\pgfqpoint{4.861171in}{1.362310in}}%
\pgfpathlineto{\pgfqpoint{4.863284in}{1.355236in}}%
\pgfpathlineto{\pgfqpoint{4.861939in}{1.363826in}}%
\pgfpathlineto{\pgfqpoint{4.863477in}{1.355481in}}%
\pgfpathlineto{\pgfqpoint{4.864437in}{1.361587in}}%
\pgfpathlineto{\pgfqpoint{4.864822in}{1.359812in}}%
\pgfpathlineto{\pgfqpoint{4.865590in}{1.356428in}}%
\pgfpathlineto{\pgfqpoint{4.865206in}{1.361744in}}%
\pgfpathlineto{\pgfqpoint{4.865975in}{1.358350in}}%
\pgfpathlineto{\pgfqpoint{4.867896in}{1.368246in}}%
\pgfpathlineto{\pgfqpoint{4.868089in}{1.365695in}}%
\pgfpathlineto{\pgfqpoint{4.869242in}{1.361621in}}%
\pgfpathlineto{\pgfqpoint{4.869434in}{1.362370in}}%
\pgfpathlineto{\pgfqpoint{4.869626in}{1.364407in}}%
\pgfpathlineto{\pgfqpoint{4.870202in}{1.361205in}}%
\pgfpathlineto{\pgfqpoint{4.870395in}{1.361452in}}%
\pgfpathlineto{\pgfqpoint{4.871356in}{1.356171in}}%
\pgfpathlineto{\pgfqpoint{4.871548in}{1.356223in}}%
\pgfpathlineto{\pgfqpoint{4.872124in}{1.358906in}}%
\pgfpathlineto{\pgfqpoint{4.872509in}{1.357247in}}%
\pgfpathlineto{\pgfqpoint{4.872893in}{1.358555in}}%
\pgfpathlineto{\pgfqpoint{4.873469in}{1.355150in}}%
\pgfpathlineto{\pgfqpoint{4.873662in}{1.358080in}}%
\pgfpathlineto{\pgfqpoint{4.874238in}{1.355249in}}%
\pgfpathlineto{\pgfqpoint{4.875007in}{1.351586in}}%
\pgfpathlineto{\pgfqpoint{4.875391in}{1.354519in}}%
\pgfpathlineto{\pgfqpoint{4.876544in}{1.359224in}}%
\pgfpathlineto{\pgfqpoint{4.875775in}{1.353389in}}%
\pgfpathlineto{\pgfqpoint{4.877313in}{1.356203in}}%
\pgfpathlineto{\pgfqpoint{4.880003in}{1.339823in}}%
\pgfpathlineto{\pgfqpoint{4.880387in}{1.340142in}}%
\pgfpathlineto{\pgfqpoint{4.880772in}{1.337786in}}%
\pgfpathlineto{\pgfqpoint{4.881925in}{1.344301in}}%
\pgfpathlineto{\pgfqpoint{4.882693in}{1.337292in}}%
\pgfpathlineto{\pgfqpoint{4.883078in}{1.339762in}}%
\pgfpathlineto{\pgfqpoint{4.883462in}{1.341798in}}%
\pgfpathlineto{\pgfqpoint{4.884039in}{1.339122in}}%
\pgfpathlineto{\pgfqpoint{4.885384in}{1.335726in}}%
\pgfpathlineto{\pgfqpoint{4.886152in}{1.339926in}}%
\pgfpathlineto{\pgfqpoint{4.886537in}{1.339698in}}%
\pgfpathlineto{\pgfqpoint{4.889227in}{1.328464in}}%
\pgfpathlineto{\pgfqpoint{4.890572in}{1.340587in}}%
\pgfpathlineto{\pgfqpoint{4.890957in}{1.339091in}}%
\pgfpathlineto{\pgfqpoint{4.891917in}{1.337038in}}%
\pgfpathlineto{\pgfqpoint{4.892302in}{1.343396in}}%
\pgfpathlineto{\pgfqpoint{4.892878in}{1.339850in}}%
\pgfpathlineto{\pgfqpoint{4.893647in}{1.333971in}}%
\pgfpathlineto{\pgfqpoint{4.894223in}{1.334643in}}%
\pgfpathlineto{\pgfqpoint{4.894416in}{1.334089in}}%
\pgfpathlineto{\pgfqpoint{4.895761in}{1.341671in}}%
\pgfpathlineto{\pgfqpoint{4.897683in}{1.333089in}}%
\pgfpathlineto{\pgfqpoint{4.896337in}{1.341953in}}%
\pgfpathlineto{\pgfqpoint{4.897875in}{1.333179in}}%
\pgfpathlineto{\pgfqpoint{4.898836in}{1.335514in}}%
\pgfpathlineto{\pgfqpoint{4.898259in}{1.332320in}}%
\pgfpathlineto{\pgfqpoint{4.899028in}{1.334951in}}%
\pgfpathlineto{\pgfqpoint{4.900949in}{1.323413in}}%
\pgfpathlineto{\pgfqpoint{4.901718in}{1.324303in}}%
\pgfpathlineto{\pgfqpoint{4.901910in}{1.320744in}}%
\pgfpathlineto{\pgfqpoint{4.902487in}{1.324844in}}%
\pgfpathlineto{\pgfqpoint{4.902679in}{1.323173in}}%
\pgfpathlineto{\pgfqpoint{4.903832in}{1.327525in}}%
\pgfpathlineto{\pgfqpoint{4.904408in}{1.322860in}}%
\pgfpathlineto{\pgfqpoint{4.904985in}{1.325371in}}%
\pgfpathlineto{\pgfqpoint{4.905561in}{1.328591in}}%
\pgfpathlineto{\pgfqpoint{4.906138in}{1.325826in}}%
\pgfpathlineto{\pgfqpoint{4.906714in}{1.320832in}}%
\pgfpathlineto{\pgfqpoint{4.907483in}{1.322717in}}%
\pgfpathlineto{\pgfqpoint{4.907675in}{1.323685in}}%
\pgfpathlineto{\pgfqpoint{4.907867in}{1.321877in}}%
\pgfpathlineto{\pgfqpoint{4.908444in}{1.321932in}}%
\pgfpathlineto{\pgfqpoint{4.909981in}{1.315945in}}%
\pgfpathlineto{\pgfqpoint{4.910558in}{1.319078in}}%
\pgfpathlineto{\pgfqpoint{4.910942in}{1.315500in}}%
\pgfpathlineto{\pgfqpoint{4.912287in}{1.307179in}}%
\pgfpathlineto{\pgfqpoint{4.912672in}{1.309596in}}%
\pgfpathlineto{\pgfqpoint{4.914209in}{1.315654in}}%
\pgfpathlineto{\pgfqpoint{4.914401in}{1.310102in}}%
\pgfpathlineto{\pgfqpoint{4.915362in}{1.313973in}}%
\pgfpathlineto{\pgfqpoint{4.915554in}{1.310305in}}%
\pgfpathlineto{\pgfqpoint{4.916131in}{1.315332in}}%
\pgfpathlineto{\pgfqpoint{4.916515in}{1.311950in}}%
\pgfpathlineto{\pgfqpoint{4.916707in}{1.310793in}}%
\pgfpathlineto{\pgfqpoint{4.917091in}{1.314499in}}%
\pgfpathlineto{\pgfqpoint{4.918629in}{1.325838in}}%
\pgfpathlineto{\pgfqpoint{4.919782in}{1.329263in}}%
\pgfpathlineto{\pgfqpoint{4.919013in}{1.325509in}}%
\pgfpathlineto{\pgfqpoint{4.919974in}{1.326927in}}%
\pgfpathlineto{\pgfqpoint{4.923049in}{1.311494in}}%
\pgfpathlineto{\pgfqpoint{4.920743in}{1.328262in}}%
\pgfpathlineto{\pgfqpoint{4.923433in}{1.312035in}}%
\pgfpathlineto{\pgfqpoint{4.925547in}{1.324739in}}%
\pgfpathlineto{\pgfqpoint{4.925739in}{1.322530in}}%
\pgfpathlineto{\pgfqpoint{4.926316in}{1.323946in}}%
\pgfpathlineto{\pgfqpoint{4.926892in}{1.323258in}}%
\pgfpathlineto{\pgfqpoint{4.927084in}{1.322361in}}%
\pgfpathlineto{\pgfqpoint{4.927469in}{1.325720in}}%
\pgfpathlineto{\pgfqpoint{4.927661in}{1.324196in}}%
\pgfpathlineto{\pgfqpoint{4.929390in}{1.334975in}}%
\pgfpathlineto{\pgfqpoint{4.930735in}{1.331318in}}%
\pgfpathlineto{\pgfqpoint{4.931120in}{1.335783in}}%
\pgfpathlineto{\pgfqpoint{4.931888in}{1.332502in}}%
\pgfpathlineto{\pgfqpoint{4.932273in}{1.330944in}}%
\pgfpathlineto{\pgfqpoint{4.933234in}{1.336502in}}%
\pgfpathlineto{\pgfqpoint{4.935540in}{1.320946in}}%
\pgfpathlineto{\pgfqpoint{4.935732in}{1.321371in}}%
\pgfpathlineto{\pgfqpoint{4.935924in}{1.320373in}}%
\pgfpathlineto{\pgfqpoint{4.936500in}{1.311840in}}%
\pgfpathlineto{\pgfqpoint{4.937077in}{1.317837in}}%
\pgfpathlineto{\pgfqpoint{4.938614in}{1.314176in}}%
\pgfpathlineto{\pgfqpoint{4.938806in}{1.314251in}}%
\pgfpathlineto{\pgfqpoint{4.940152in}{1.327100in}}%
\pgfpathlineto{\pgfqpoint{4.940536in}{1.324244in}}%
\pgfpathlineto{\pgfqpoint{4.940920in}{1.323565in}}%
\pgfpathlineto{\pgfqpoint{4.941112in}{1.323992in}}%
\pgfpathlineto{\pgfqpoint{4.942073in}{1.327892in}}%
\pgfpathlineto{\pgfqpoint{4.942458in}{1.327129in}}%
\pgfpathlineto{\pgfqpoint{4.946493in}{1.298087in}}%
\pgfpathlineto{\pgfqpoint{4.946685in}{1.295322in}}%
\pgfpathlineto{\pgfqpoint{4.946878in}{1.298347in}}%
\pgfpathlineto{\pgfqpoint{4.947454in}{1.297477in}}%
\pgfpathlineto{\pgfqpoint{4.947838in}{1.301604in}}%
\pgfpathlineto{\pgfqpoint{4.948607in}{1.299202in}}%
\pgfpathlineto{\pgfqpoint{4.948799in}{1.299099in}}%
\pgfpathlineto{\pgfqpoint{4.948991in}{1.302900in}}%
\pgfpathlineto{\pgfqpoint{4.949568in}{1.297973in}}%
\pgfpathlineto{\pgfqpoint{4.949760in}{1.299250in}}%
\pgfpathlineto{\pgfqpoint{4.950721in}{1.295633in}}%
\pgfpathlineto{\pgfqpoint{4.950913in}{1.298938in}}%
\pgfpathlineto{\pgfqpoint{4.951490in}{1.294322in}}%
\pgfpathlineto{\pgfqpoint{4.951682in}{1.290659in}}%
\pgfpathlineto{\pgfqpoint{4.952450in}{1.295971in}}%
\pgfpathlineto{\pgfqpoint{4.952835in}{1.293917in}}%
\pgfpathlineto{\pgfqpoint{4.953411in}{1.296766in}}%
\pgfpathlineto{\pgfqpoint{4.953603in}{1.298359in}}%
\pgfpathlineto{\pgfqpoint{4.953988in}{1.295458in}}%
\pgfpathlineto{\pgfqpoint{4.954180in}{1.295512in}}%
\pgfpathlineto{\pgfqpoint{4.954372in}{1.294544in}}%
\pgfpathlineto{\pgfqpoint{4.954756in}{1.296654in}}%
\pgfpathlineto{\pgfqpoint{4.955141in}{1.298927in}}%
\pgfpathlineto{\pgfqpoint{4.955717in}{1.297644in}}%
\pgfpathlineto{\pgfqpoint{4.956294in}{1.297840in}}%
\pgfpathlineto{\pgfqpoint{4.956870in}{1.294582in}}%
\pgfpathlineto{\pgfqpoint{4.958215in}{1.297662in}}%
\pgfpathlineto{\pgfqpoint{4.959753in}{1.289285in}}%
\pgfpathlineto{\pgfqpoint{4.960137in}{1.287673in}}%
\pgfpathlineto{\pgfqpoint{4.960521in}{1.289147in}}%
\pgfpathlineto{\pgfqpoint{4.961098in}{1.284196in}}%
\pgfpathlineto{\pgfqpoint{4.961674in}{1.285875in}}%
\pgfpathlineto{\pgfqpoint{4.962635in}{1.291109in}}%
\pgfpathlineto{\pgfqpoint{4.962827in}{1.286908in}}%
\pgfpathlineto{\pgfqpoint{4.963212in}{1.284771in}}%
\pgfpathlineto{\pgfqpoint{4.963980in}{1.286558in}}%
\pgfpathlineto{\pgfqpoint{4.964365in}{1.291662in}}%
\pgfpathlineto{\pgfqpoint{4.965133in}{1.289668in}}%
\pgfpathlineto{\pgfqpoint{4.965518in}{1.289196in}}%
\pgfpathlineto{\pgfqpoint{4.966863in}{1.282131in}}%
\pgfpathlineto{\pgfqpoint{4.967439in}{1.285027in}}%
\pgfpathlineto{\pgfqpoint{4.967632in}{1.281536in}}%
\pgfpathlineto{\pgfqpoint{4.967824in}{1.282901in}}%
\pgfpathlineto{\pgfqpoint{4.968592in}{1.280752in}}%
\pgfpathlineto{\pgfqpoint{4.969938in}{1.290244in}}%
\pgfpathlineto{\pgfqpoint{4.971283in}{1.280587in}}%
\pgfpathlineto{\pgfqpoint{4.971667in}{1.282021in}}%
\pgfpathlineto{\pgfqpoint{4.973397in}{1.288824in}}%
\pgfpathlineto{\pgfqpoint{4.974550in}{1.285898in}}%
\pgfpathlineto{\pgfqpoint{4.975703in}{1.293220in}}%
\pgfpathlineto{\pgfqpoint{4.976087in}{1.292594in}}%
\pgfpathlineto{\pgfqpoint{4.976279in}{1.290085in}}%
\pgfpathlineto{\pgfqpoint{4.977048in}{1.294354in}}%
\pgfpathlineto{\pgfqpoint{4.978393in}{1.288079in}}%
\pgfpathlineto{\pgfqpoint{4.978585in}{1.290351in}}%
\pgfpathlineto{\pgfqpoint{4.978777in}{1.293085in}}%
\pgfpathlineto{\pgfqpoint{4.979354in}{1.291060in}}%
\pgfpathlineto{\pgfqpoint{4.980315in}{1.282155in}}%
\pgfpathlineto{\pgfqpoint{4.980699in}{1.284527in}}%
\pgfpathlineto{\pgfqpoint{4.981083in}{1.284006in}}%
\pgfpathlineto{\pgfqpoint{4.981276in}{1.285739in}}%
\pgfpathlineto{\pgfqpoint{4.981660in}{1.281926in}}%
\pgfpathlineto{\pgfqpoint{4.981852in}{1.282445in}}%
\pgfpathlineto{\pgfqpoint{4.982236in}{1.284532in}}%
\pgfpathlineto{\pgfqpoint{4.983389in}{1.276768in}}%
\pgfpathlineto{\pgfqpoint{4.983774in}{1.280783in}}%
\pgfpathlineto{\pgfqpoint{4.984350in}{1.279541in}}%
\pgfpathlineto{\pgfqpoint{4.985311in}{1.274078in}}%
\pgfpathlineto{\pgfqpoint{4.985503in}{1.278742in}}%
\pgfpathlineto{\pgfqpoint{4.986272in}{1.275642in}}%
\pgfpathlineto{\pgfqpoint{4.986848in}{1.276719in}}%
\pgfpathlineto{\pgfqpoint{4.988578in}{1.268988in}}%
\pgfpathlineto{\pgfqpoint{4.989731in}{1.275900in}}%
\pgfpathlineto{\pgfqpoint{4.990115in}{1.271900in}}%
\pgfpathlineto{\pgfqpoint{4.990500in}{1.272744in}}%
\pgfpathlineto{\pgfqpoint{4.994920in}{1.256238in}}%
\pgfpathlineto{\pgfqpoint{4.995304in}{1.259051in}}%
\pgfpathlineto{\pgfqpoint{4.996073in}{1.258672in}}%
\pgfpathlineto{\pgfqpoint{4.996841in}{1.255617in}}%
\pgfpathlineto{\pgfqpoint{4.997033in}{1.259431in}}%
\pgfpathlineto{\pgfqpoint{4.997610in}{1.256810in}}%
\pgfpathlineto{\pgfqpoint{4.997802in}{1.260509in}}%
\pgfpathlineto{\pgfqpoint{4.998379in}{1.255262in}}%
\pgfpathlineto{\pgfqpoint{4.998763in}{1.259852in}}%
\pgfpathlineto{\pgfqpoint{4.998955in}{1.259505in}}%
\pgfpathlineto{\pgfqpoint{4.999339in}{1.255206in}}%
\pgfpathlineto{\pgfqpoint{4.999916in}{1.261363in}}%
\pgfpathlineto{\pgfqpoint{5.000492in}{1.260927in}}%
\pgfpathlineto{\pgfqpoint{5.001645in}{1.266572in}}%
\pgfpathlineto{\pgfqpoint{5.002030in}{1.261503in}}%
\pgfpathlineto{\pgfqpoint{5.002606in}{1.262974in}}%
\pgfpathlineto{\pgfqpoint{5.004144in}{1.278405in}}%
\pgfpathlineto{\pgfqpoint{5.004336in}{1.275340in}}%
\pgfpathlineto{\pgfqpoint{5.005681in}{1.263489in}}%
\pgfpathlineto{\pgfqpoint{5.006065in}{1.265816in}}%
\pgfpathlineto{\pgfqpoint{5.006450in}{1.270022in}}%
\pgfpathlineto{\pgfqpoint{5.007026in}{1.266510in}}%
\pgfpathlineto{\pgfqpoint{5.009332in}{1.253133in}}%
\pgfpathlineto{\pgfqpoint{5.009716in}{1.254755in}}%
\pgfpathlineto{\pgfqpoint{5.010101in}{1.252627in}}%
\pgfpathlineto{\pgfqpoint{5.010293in}{1.251523in}}%
\pgfpathlineto{\pgfqpoint{5.010677in}{1.255043in}}%
\pgfpathlineto{\pgfqpoint{5.010869in}{1.254225in}}%
\pgfpathlineto{\pgfqpoint{5.011062in}{1.254048in}}%
\pgfpathlineto{\pgfqpoint{5.011638in}{1.250697in}}%
\pgfpathlineto{\pgfqpoint{5.012022in}{1.255514in}}%
\pgfpathlineto{\pgfqpoint{5.012983in}{1.254252in}}%
\pgfpathlineto{\pgfqpoint{5.013175in}{1.257952in}}%
\pgfpathlineto{\pgfqpoint{5.013944in}{1.253755in}}%
\pgfpathlineto{\pgfqpoint{5.015097in}{1.255339in}}%
\pgfpathlineto{\pgfqpoint{5.015289in}{1.252927in}}%
\pgfpathlineto{\pgfqpoint{5.016058in}{1.254935in}}%
\pgfpathlineto{\pgfqpoint{5.017211in}{1.263979in}}%
\pgfpathlineto{\pgfqpoint{5.017595in}{1.261607in}}%
\pgfpathlineto{\pgfqpoint{5.017788in}{1.257520in}}%
\pgfpathlineto{\pgfqpoint{5.018556in}{1.260537in}}%
\pgfpathlineto{\pgfqpoint{5.018941in}{1.264815in}}%
\pgfpathlineto{\pgfqpoint{5.019517in}{1.259590in}}%
\pgfpathlineto{\pgfqpoint{5.021054in}{1.251672in}}%
\pgfpathlineto{\pgfqpoint{5.021247in}{1.251865in}}%
\pgfpathlineto{\pgfqpoint{5.021631in}{1.254269in}}%
\pgfpathlineto{\pgfqpoint{5.022015in}{1.249696in}}%
\pgfpathlineto{\pgfqpoint{5.022207in}{1.248732in}}%
\pgfpathlineto{\pgfqpoint{5.022592in}{1.251768in}}%
\pgfpathlineto{\pgfqpoint{5.022976in}{1.252708in}}%
\pgfpathlineto{\pgfqpoint{5.023168in}{1.252521in}}%
\pgfpathlineto{\pgfqpoint{5.025090in}{1.239515in}}%
\pgfpathlineto{\pgfqpoint{5.025282in}{1.240424in}}%
\pgfpathlineto{\pgfqpoint{5.026435in}{1.244276in}}%
\pgfpathlineto{\pgfqpoint{5.026627in}{1.243973in}}%
\pgfpathlineto{\pgfqpoint{5.027012in}{1.240090in}}%
\pgfpathlineto{\pgfqpoint{5.027588in}{1.243654in}}%
\pgfpathlineto{\pgfqpoint{5.027972in}{1.244904in}}%
\pgfpathlineto{\pgfqpoint{5.028165in}{1.244365in}}%
\pgfpathlineto{\pgfqpoint{5.028357in}{1.240798in}}%
\pgfpathlineto{\pgfqpoint{5.029318in}{1.242072in}}%
\pgfpathlineto{\pgfqpoint{5.029510in}{1.243633in}}%
\pgfpathlineto{\pgfqpoint{5.029702in}{1.241710in}}%
\pgfpathlineto{\pgfqpoint{5.030086in}{1.242403in}}%
\pgfpathlineto{\pgfqpoint{5.030278in}{1.237330in}}%
\pgfpathlineto{\pgfqpoint{5.031047in}{1.243887in}}%
\pgfpathlineto{\pgfqpoint{5.032392in}{1.250737in}}%
\pgfpathlineto{\pgfqpoint{5.033353in}{1.252267in}}%
\pgfpathlineto{\pgfqpoint{5.033545in}{1.251420in}}%
\pgfpathlineto{\pgfqpoint{5.034506in}{1.250705in}}%
\pgfpathlineto{\pgfqpoint{5.034890in}{1.257156in}}%
\pgfpathlineto{\pgfqpoint{5.036236in}{1.254785in}}%
\pgfpathlineto{\pgfqpoint{5.036428in}{1.256423in}}%
\pgfpathlineto{\pgfqpoint{5.037389in}{1.258721in}}%
\pgfpathlineto{\pgfqpoint{5.038349in}{1.252445in}}%
\pgfpathlineto{\pgfqpoint{5.038542in}{1.255726in}}%
\pgfpathlineto{\pgfqpoint{5.040655in}{1.272435in}}%
\pgfpathlineto{\pgfqpoint{5.040848in}{1.268421in}}%
\pgfpathlineto{\pgfqpoint{5.041424in}{1.274892in}}%
\pgfpathlineto{\pgfqpoint{5.041616in}{1.276552in}}%
\pgfpathlineto{\pgfqpoint{5.042193in}{1.272335in}}%
\pgfpathlineto{\pgfqpoint{5.042385in}{1.274678in}}%
\pgfpathlineto{\pgfqpoint{5.042962in}{1.273249in}}%
\pgfpathlineto{\pgfqpoint{5.043154in}{1.276328in}}%
\pgfpathlineto{\pgfqpoint{5.043346in}{1.275834in}}%
\pgfpathlineto{\pgfqpoint{5.043922in}{1.276953in}}%
\pgfpathlineto{\pgfqpoint{5.044115in}{1.277694in}}%
\pgfpathlineto{\pgfqpoint{5.044307in}{1.276848in}}%
\pgfpathlineto{\pgfqpoint{5.044499in}{1.277203in}}%
\pgfpathlineto{\pgfqpoint{5.044691in}{1.275072in}}%
\pgfpathlineto{\pgfqpoint{5.045268in}{1.278421in}}%
\pgfpathlineto{\pgfqpoint{5.045460in}{1.278945in}}%
\pgfpathlineto{\pgfqpoint{5.045652in}{1.277977in}}%
\pgfpathlineto{\pgfqpoint{5.046997in}{1.274474in}}%
\pgfpathlineto{\pgfqpoint{5.047766in}{1.276026in}}%
\pgfpathlineto{\pgfqpoint{5.047381in}{1.273610in}}%
\pgfpathlineto{\pgfqpoint{5.048150in}{1.275029in}}%
\pgfpathlineto{\pgfqpoint{5.048342in}{1.273217in}}%
\pgfpathlineto{\pgfqpoint{5.048534in}{1.276908in}}%
\pgfpathlineto{\pgfqpoint{5.049111in}{1.275365in}}%
\pgfpathlineto{\pgfqpoint{5.049495in}{1.275536in}}%
\pgfpathlineto{\pgfqpoint{5.051225in}{1.286744in}}%
\pgfpathlineto{\pgfqpoint{5.051417in}{1.285259in}}%
\pgfpathlineto{\pgfqpoint{5.052954in}{1.292792in}}%
\pgfpathlineto{\pgfqpoint{5.053146in}{1.292556in}}%
\pgfpathlineto{\pgfqpoint{5.054299in}{1.279647in}}%
\pgfpathlineto{\pgfqpoint{5.054876in}{1.282412in}}%
\pgfpathlineto{\pgfqpoint{5.056029in}{1.289737in}}%
\pgfpathlineto{\pgfqpoint{5.056413in}{1.287819in}}%
\pgfpathlineto{\pgfqpoint{5.057951in}{1.280932in}}%
\pgfpathlineto{\pgfqpoint{5.058143in}{1.281077in}}%
\pgfpathlineto{\pgfqpoint{5.059104in}{1.272479in}}%
\pgfpathlineto{\pgfqpoint{5.059296in}{1.274253in}}%
\pgfpathlineto{\pgfqpoint{5.060257in}{1.281374in}}%
\pgfpathlineto{\pgfqpoint{5.060449in}{1.280895in}}%
\pgfpathlineto{\pgfqpoint{5.061217in}{1.277967in}}%
\pgfpathlineto{\pgfqpoint{5.061602in}{1.278947in}}%
\pgfpathlineto{\pgfqpoint{5.062178in}{1.283867in}}%
\pgfpathlineto{\pgfqpoint{5.062947in}{1.280865in}}%
\pgfpathlineto{\pgfqpoint{5.063139in}{1.279948in}}%
\pgfpathlineto{\pgfqpoint{5.063523in}{1.282978in}}%
\pgfpathlineto{\pgfqpoint{5.063716in}{1.282606in}}%
\pgfpathlineto{\pgfqpoint{5.063908in}{1.282450in}}%
\pgfpathlineto{\pgfqpoint{5.065445in}{1.269543in}}%
\pgfpathlineto{\pgfqpoint{5.065637in}{1.271927in}}%
\pgfpathlineto{\pgfqpoint{5.066214in}{1.270244in}}%
\pgfpathlineto{\pgfqpoint{5.066406in}{1.273630in}}%
\pgfpathlineto{\pgfqpoint{5.066598in}{1.272859in}}%
\pgfpathlineto{\pgfqpoint{5.067175in}{1.270063in}}%
\pgfpathlineto{\pgfqpoint{5.067559in}{1.271874in}}%
\pgfpathlineto{\pgfqpoint{5.069481in}{1.285070in}}%
\pgfpathlineto{\pgfqpoint{5.071210in}{1.291946in}}%
\pgfpathlineto{\pgfqpoint{5.071402in}{1.290616in}}%
\pgfpathlineto{\pgfqpoint{5.072171in}{1.292437in}}%
\pgfpathlineto{\pgfqpoint{5.075630in}{1.303653in}}%
\pgfpathlineto{\pgfqpoint{5.076399in}{1.305802in}}%
\pgfpathlineto{\pgfqpoint{5.076591in}{1.300324in}}%
\pgfpathlineto{\pgfqpoint{5.077552in}{1.306455in}}%
\pgfpathlineto{\pgfqpoint{5.077744in}{1.304394in}}%
\pgfpathlineto{\pgfqpoint{5.080626in}{1.289610in}}%
\pgfpathlineto{\pgfqpoint{5.081011in}{1.290856in}}%
\pgfpathlineto{\pgfqpoint{5.081587in}{1.294189in}}%
\pgfpathlineto{\pgfqpoint{5.081395in}{1.290745in}}%
\pgfpathlineto{\pgfqpoint{5.081972in}{1.291511in}}%
\pgfpathlineto{\pgfqpoint{5.082548in}{1.289719in}}%
\pgfpathlineto{\pgfqpoint{5.082932in}{1.290515in}}%
\pgfpathlineto{\pgfqpoint{5.083701in}{1.296088in}}%
\pgfpathlineto{\pgfqpoint{5.084085in}{1.291846in}}%
\pgfpathlineto{\pgfqpoint{5.085046in}{1.288432in}}%
\pgfpathlineto{\pgfqpoint{5.085238in}{1.290916in}}%
\pgfpathlineto{\pgfqpoint{5.085431in}{1.293108in}}%
\pgfpathlineto{\pgfqpoint{5.085815in}{1.289279in}}%
\pgfpathlineto{\pgfqpoint{5.086007in}{1.289717in}}%
\pgfpathlineto{\pgfqpoint{5.087160in}{1.281797in}}%
\pgfpathlineto{\pgfqpoint{5.087544in}{1.285568in}}%
\pgfpathlineto{\pgfqpoint{5.087929in}{1.287731in}}%
\pgfpathlineto{\pgfqpoint{5.088313in}{1.285495in}}%
\pgfpathlineto{\pgfqpoint{5.088505in}{1.284608in}}%
\pgfpathlineto{\pgfqpoint{5.088890in}{1.285802in}}%
\pgfpathlineto{\pgfqpoint{5.089274in}{1.292392in}}%
\pgfpathlineto{\pgfqpoint{5.090235in}{1.291176in}}%
\pgfpathlineto{\pgfqpoint{5.090427in}{1.292211in}}%
\pgfpathlineto{\pgfqpoint{5.090811in}{1.289591in}}%
\pgfpathlineto{\pgfqpoint{5.091004in}{1.288595in}}%
\pgfpathlineto{\pgfqpoint{5.091196in}{1.291019in}}%
\pgfpathlineto{\pgfqpoint{5.091388in}{1.290327in}}%
\pgfpathlineto{\pgfqpoint{5.093694in}{1.305426in}}%
\pgfpathlineto{\pgfqpoint{5.093886in}{1.300862in}}%
\pgfpathlineto{\pgfqpoint{5.094078in}{1.298260in}}%
\pgfpathlineto{\pgfqpoint{5.095039in}{1.299932in}}%
\pgfpathlineto{\pgfqpoint{5.095616in}{1.298632in}}%
\pgfpathlineto{\pgfqpoint{5.096192in}{1.301594in}}%
\pgfpathlineto{\pgfqpoint{5.097537in}{1.290342in}}%
\pgfpathlineto{\pgfqpoint{5.098498in}{1.292939in}}%
\pgfpathlineto{\pgfqpoint{5.098690in}{1.293583in}}%
\pgfpathlineto{\pgfqpoint{5.098882in}{1.293446in}}%
\pgfpathlineto{\pgfqpoint{5.099075in}{1.290332in}}%
\pgfpathlineto{\pgfqpoint{5.099459in}{1.296748in}}%
\pgfpathlineto{\pgfqpoint{5.101188in}{1.307268in}}%
\pgfpathlineto{\pgfqpoint{5.101573in}{1.305063in}}%
\pgfpathlineto{\pgfqpoint{5.101957in}{1.308645in}}%
\pgfpathlineto{\pgfqpoint{5.102149in}{1.307201in}}%
\pgfpathlineto{\pgfqpoint{5.103110in}{1.307387in}}%
\pgfpathlineto{\pgfqpoint{5.103494in}{1.305007in}}%
\pgfpathlineto{\pgfqpoint{5.103879in}{1.308551in}}%
\pgfpathlineto{\pgfqpoint{5.104071in}{1.305862in}}%
\pgfpathlineto{\pgfqpoint{5.104840in}{1.309255in}}%
\pgfpathlineto{\pgfqpoint{5.105032in}{1.307826in}}%
\pgfpathlineto{\pgfqpoint{5.105993in}{1.314443in}}%
\pgfpathlineto{\pgfqpoint{5.106185in}{1.314202in}}%
\pgfpathlineto{\pgfqpoint{5.106569in}{1.310406in}}%
\pgfpathlineto{\pgfqpoint{5.107146in}{1.314756in}}%
\pgfpathlineto{\pgfqpoint{5.107338in}{1.312045in}}%
\pgfpathlineto{\pgfqpoint{5.109067in}{1.303588in}}%
\pgfpathlineto{\pgfqpoint{5.109259in}{1.304382in}}%
\pgfpathlineto{\pgfqpoint{5.109452in}{1.305041in}}%
\pgfpathlineto{\pgfqpoint{5.109836in}{1.303377in}}%
\pgfpathlineto{\pgfqpoint{5.110028in}{1.302031in}}%
\pgfpathlineto{\pgfqpoint{5.110412in}{1.303812in}}%
\pgfpathlineto{\pgfqpoint{5.110989in}{1.307640in}}%
\pgfpathlineto{\pgfqpoint{5.111373in}{1.305567in}}%
\pgfpathlineto{\pgfqpoint{5.111565in}{1.302264in}}%
\pgfpathlineto{\pgfqpoint{5.112142in}{1.306243in}}%
\pgfpathlineto{\pgfqpoint{5.112334in}{1.309205in}}%
\pgfpathlineto{\pgfqpoint{5.112911in}{1.301902in}}%
\pgfpathlineto{\pgfqpoint{5.113295in}{1.304387in}}%
\pgfpathlineto{\pgfqpoint{5.113487in}{1.303385in}}%
\pgfpathlineto{\pgfqpoint{5.113679in}{1.308982in}}%
\pgfpathlineto{\pgfqpoint{5.114640in}{1.307299in}}%
\pgfpathlineto{\pgfqpoint{5.116754in}{1.319482in}}%
\pgfpathlineto{\pgfqpoint{5.115024in}{1.306762in}}%
\pgfpathlineto{\pgfqpoint{5.117331in}{1.315349in}}%
\pgfpathlineto{\pgfqpoint{5.117715in}{1.309792in}}%
\pgfpathlineto{\pgfqpoint{5.118484in}{1.312808in}}%
\pgfpathlineto{\pgfqpoint{5.120021in}{1.310339in}}%
\pgfpathlineto{\pgfqpoint{5.120213in}{1.309720in}}%
\pgfpathlineto{\pgfqpoint{5.120405in}{1.310715in}}%
\pgfpathlineto{\pgfqpoint{5.120790in}{1.315164in}}%
\pgfpathlineto{\pgfqpoint{5.121366in}{1.308756in}}%
\pgfpathlineto{\pgfqpoint{5.121558in}{1.310045in}}%
\pgfpathlineto{\pgfqpoint{5.121943in}{1.306260in}}%
\pgfpathlineto{\pgfqpoint{5.122327in}{1.309171in}}%
\pgfpathlineto{\pgfqpoint{5.123480in}{1.305734in}}%
\pgfpathlineto{\pgfqpoint{5.124825in}{1.292468in}}%
\pgfpathlineto{\pgfqpoint{5.125209in}{1.294607in}}%
\pgfpathlineto{\pgfqpoint{5.126362in}{1.299008in}}%
\pgfpathlineto{\pgfqpoint{5.126555in}{1.297575in}}%
\pgfpathlineto{\pgfqpoint{5.127900in}{1.302157in}}%
\pgfpathlineto{\pgfqpoint{5.128284in}{1.301913in}}%
\pgfpathlineto{\pgfqpoint{5.129821in}{1.294275in}}%
\pgfpathlineto{\pgfqpoint{5.130398in}{1.298667in}}%
\pgfpathlineto{\pgfqpoint{5.131551in}{1.297600in}}%
\pgfpathlineto{\pgfqpoint{5.131935in}{1.294268in}}%
\pgfpathlineto{\pgfqpoint{5.132320in}{1.297408in}}%
\pgfpathlineto{\pgfqpoint{5.132704in}{1.301067in}}%
\pgfpathlineto{\pgfqpoint{5.133088in}{1.294335in}}%
\pgfpathlineto{\pgfqpoint{5.133473in}{1.298178in}}%
\pgfpathlineto{\pgfqpoint{5.134626in}{1.301591in}}%
\pgfpathlineto{\pgfqpoint{5.134818in}{1.300737in}}%
\pgfpathlineto{\pgfqpoint{5.135010in}{1.300895in}}%
\pgfpathlineto{\pgfqpoint{5.135971in}{1.310072in}}%
\pgfpathlineto{\pgfqpoint{5.136355in}{1.307135in}}%
\pgfpathlineto{\pgfqpoint{5.137316in}{1.301936in}}%
\pgfpathlineto{\pgfqpoint{5.137508in}{1.302198in}}%
\pgfpathlineto{\pgfqpoint{5.138277in}{1.303588in}}%
\pgfpathlineto{\pgfqpoint{5.138085in}{1.300075in}}%
\pgfpathlineto{\pgfqpoint{5.138469in}{1.302616in}}%
\pgfpathlineto{\pgfqpoint{5.140391in}{1.290199in}}%
\pgfpathlineto{\pgfqpoint{5.141736in}{1.302578in}}%
\pgfpathlineto{\pgfqpoint{5.142120in}{1.301950in}}%
\pgfpathlineto{\pgfqpoint{5.142889in}{1.301259in}}%
\pgfpathlineto{\pgfqpoint{5.143081in}{1.303290in}}%
\pgfpathlineto{\pgfqpoint{5.143658in}{1.301564in}}%
\pgfpathlineto{\pgfqpoint{5.143465in}{1.304544in}}%
\pgfpathlineto{\pgfqpoint{5.144042in}{1.304142in}}%
\pgfpathlineto{\pgfqpoint{5.144234in}{1.305715in}}%
\pgfpathlineto{\pgfqpoint{5.144811in}{1.302046in}}%
\pgfpathlineto{\pgfqpoint{5.145003in}{1.303838in}}%
\pgfpathlineto{\pgfqpoint{5.145195in}{1.303501in}}%
\pgfpathlineto{\pgfqpoint{5.145387in}{1.304015in}}%
\pgfpathlineto{\pgfqpoint{5.145964in}{1.311363in}}%
\pgfpathlineto{\pgfqpoint{5.147117in}{1.310800in}}%
\pgfpathlineto{\pgfqpoint{5.148462in}{1.300911in}}%
\pgfpathlineto{\pgfqpoint{5.148654in}{1.303378in}}%
\pgfpathlineto{\pgfqpoint{5.149615in}{1.303090in}}%
\pgfpathlineto{\pgfqpoint{5.149999in}{1.304261in}}%
\pgfpathlineto{\pgfqpoint{5.151152in}{1.311828in}}%
\pgfpathlineto{\pgfqpoint{5.151344in}{1.310385in}}%
\pgfpathlineto{\pgfqpoint{5.151536in}{1.309934in}}%
\pgfpathlineto{\pgfqpoint{5.151729in}{1.311517in}}%
\pgfpathlineto{\pgfqpoint{5.152305in}{1.316906in}}%
\pgfpathlineto{\pgfqpoint{5.152689in}{1.312557in}}%
\pgfpathlineto{\pgfqpoint{5.154227in}{1.298382in}}%
\pgfpathlineto{\pgfqpoint{5.154419in}{1.298321in}}%
\pgfpathlineto{\pgfqpoint{5.155956in}{1.291183in}}%
\pgfpathlineto{\pgfqpoint{5.156725in}{1.293538in}}%
\pgfpathlineto{\pgfqpoint{5.157109in}{1.292119in}}%
\pgfpathlineto{\pgfqpoint{5.157686in}{1.290840in}}%
\pgfpathlineto{\pgfqpoint{5.157494in}{1.292695in}}%
\pgfpathlineto{\pgfqpoint{5.158070in}{1.291555in}}%
\pgfpathlineto{\pgfqpoint{5.159031in}{1.297141in}}%
\pgfpathlineto{\pgfqpoint{5.159415in}{1.295262in}}%
\pgfpathlineto{\pgfqpoint{5.159607in}{1.295321in}}%
\pgfpathlineto{\pgfqpoint{5.159800in}{1.298370in}}%
\pgfpathlineto{\pgfqpoint{5.160568in}{1.293650in}}%
\pgfpathlineto{\pgfqpoint{5.161913in}{1.287464in}}%
\pgfpathlineto{\pgfqpoint{5.162106in}{1.291066in}}%
\pgfpathlineto{\pgfqpoint{5.162874in}{1.286049in}}%
\pgfpathlineto{\pgfqpoint{5.163643in}{1.291345in}}%
\pgfpathlineto{\pgfqpoint{5.164220in}{1.287595in}}%
\pgfpathlineto{\pgfqpoint{5.165180in}{1.282749in}}%
\pgfpathlineto{\pgfqpoint{5.165373in}{1.284133in}}%
\pgfpathlineto{\pgfqpoint{5.165757in}{1.287642in}}%
\pgfpathlineto{\pgfqpoint{5.166333in}{1.283497in}}%
\pgfpathlineto{\pgfqpoint{5.166910in}{1.284535in}}%
\pgfpathlineto{\pgfqpoint{5.167102in}{1.286736in}}%
\pgfpathlineto{\pgfqpoint{5.167679in}{1.280826in}}%
\pgfpathlineto{\pgfqpoint{5.168063in}{1.281133in}}%
\pgfpathlineto{\pgfqpoint{5.168639in}{1.279526in}}%
\pgfpathlineto{\pgfqpoint{5.169408in}{1.284741in}}%
\pgfpathlineto{\pgfqpoint{5.169792in}{1.281816in}}%
\pgfpathlineto{\pgfqpoint{5.170177in}{1.276487in}}%
\pgfpathlineto{\pgfqpoint{5.170753in}{1.282148in}}%
\pgfpathlineto{\pgfqpoint{5.171522in}{1.285677in}}%
\pgfpathlineto{\pgfqpoint{5.171906in}{1.285427in}}%
\pgfpathlineto{\pgfqpoint{5.172098in}{1.282868in}}%
\pgfpathlineto{\pgfqpoint{5.172675in}{1.289659in}}%
\pgfpathlineto{\pgfqpoint{5.172867in}{1.289671in}}%
\pgfpathlineto{\pgfqpoint{5.173059in}{1.287572in}}%
\pgfpathlineto{\pgfqpoint{5.173636in}{1.290592in}}%
\pgfpathlineto{\pgfqpoint{5.174981in}{1.294299in}}%
\pgfpathlineto{\pgfqpoint{5.176326in}{1.288488in}}%
\pgfpathlineto{\pgfqpoint{5.176518in}{1.290130in}}%
\pgfpathlineto{\pgfqpoint{5.177095in}{1.285830in}}%
\pgfpathlineto{\pgfqpoint{5.177287in}{1.285774in}}%
\pgfpathlineto{\pgfqpoint{5.178248in}{1.295068in}}%
\pgfpathlineto{\pgfqpoint{5.179016in}{1.292867in}}%
\pgfpathlineto{\pgfqpoint{5.179401in}{1.287149in}}%
\pgfpathlineto{\pgfqpoint{5.179977in}{1.291506in}}%
\pgfpathlineto{\pgfqpoint{5.180938in}{1.298940in}}%
\pgfpathlineto{\pgfqpoint{5.181322in}{1.297530in}}%
\pgfpathlineto{\pgfqpoint{5.183628in}{1.310397in}}%
\pgfpathlineto{\pgfqpoint{5.185166in}{1.296121in}}%
\pgfpathlineto{\pgfqpoint{5.186703in}{1.304794in}}%
\pgfpathlineto{\pgfqpoint{5.186895in}{1.302123in}}%
\pgfpathlineto{\pgfqpoint{5.187664in}{1.297141in}}%
\pgfpathlineto{\pgfqpoint{5.187856in}{1.298455in}}%
\pgfpathlineto{\pgfqpoint{5.188625in}{1.302565in}}%
\pgfpathlineto{\pgfqpoint{5.189009in}{1.300281in}}%
\pgfpathlineto{\pgfqpoint{5.189201in}{1.299028in}}%
\pgfpathlineto{\pgfqpoint{5.189394in}{1.302510in}}%
\pgfpathlineto{\pgfqpoint{5.190931in}{1.314697in}}%
\pgfpathlineto{\pgfqpoint{5.191315in}{1.312163in}}%
\pgfpathlineto{\pgfqpoint{5.191507in}{1.308939in}}%
\pgfpathlineto{\pgfqpoint{5.192276in}{1.314466in}}%
\pgfpathlineto{\pgfqpoint{5.194006in}{1.305751in}}%
\pgfpathlineto{\pgfqpoint{5.194198in}{1.305892in}}%
\pgfpathlineto{\pgfqpoint{5.194774in}{1.307635in}}%
\pgfpathlineto{\pgfqpoint{5.194582in}{1.305299in}}%
\pgfpathlineto{\pgfqpoint{5.195159in}{1.305905in}}%
\pgfpathlineto{\pgfqpoint{5.197080in}{1.293687in}}%
\pgfpathlineto{\pgfqpoint{5.197272in}{1.293436in}}%
\pgfpathlineto{\pgfqpoint{5.197465in}{1.293975in}}%
\pgfpathlineto{\pgfqpoint{5.199386in}{1.300912in}}%
\pgfpathlineto{\pgfqpoint{5.199771in}{1.299522in}}%
\pgfpathlineto{\pgfqpoint{5.201116in}{1.306651in}}%
\pgfpathlineto{\pgfqpoint{5.201308in}{1.307448in}}%
\pgfpathlineto{\pgfqpoint{5.201692in}{1.305635in}}%
\pgfpathlineto{\pgfqpoint{5.201884in}{1.305634in}}%
\pgfpathlineto{\pgfqpoint{5.202269in}{1.301728in}}%
\pgfpathlineto{\pgfqpoint{5.203422in}{1.303763in}}%
\pgfpathlineto{\pgfqpoint{5.204575in}{1.307131in}}%
\pgfpathlineto{\pgfqpoint{5.205920in}{1.300923in}}%
\pgfpathlineto{\pgfqpoint{5.206689in}{1.298073in}}%
\pgfpathlineto{\pgfqpoint{5.206881in}{1.300351in}}%
\pgfpathlineto{\pgfqpoint{5.208610in}{1.307714in}}%
\pgfpathlineto{\pgfqpoint{5.208995in}{1.307596in}}%
\pgfpathlineto{\pgfqpoint{5.209763in}{1.301756in}}%
\pgfpathlineto{\pgfqpoint{5.210532in}{1.298651in}}%
\pgfpathlineto{\pgfqpoint{5.210724in}{1.300949in}}%
\pgfpathlineto{\pgfqpoint{5.212261in}{1.314621in}}%
\pgfpathlineto{\pgfqpoint{5.212838in}{1.313218in}}%
\pgfpathlineto{\pgfqpoint{5.213991in}{1.303770in}}%
\pgfpathlineto{\pgfqpoint{5.214568in}{1.307248in}}%
\pgfpathlineto{\pgfqpoint{5.214952in}{1.309120in}}%
\pgfpathlineto{\pgfqpoint{5.215528in}{1.303459in}}%
\pgfpathlineto{\pgfqpoint{5.216105in}{1.304458in}}%
\pgfpathlineto{\pgfqpoint{5.216874in}{1.309641in}}%
\pgfpathlineto{\pgfqpoint{5.217642in}{1.308940in}}%
\pgfpathlineto{\pgfqpoint{5.217834in}{1.309367in}}%
\pgfpathlineto{\pgfqpoint{5.218027in}{1.307344in}}%
\pgfpathlineto{\pgfqpoint{5.218219in}{1.307375in}}%
\pgfpathlineto{\pgfqpoint{5.218987in}{1.303745in}}%
\pgfpathlineto{\pgfqpoint{5.219372in}{1.307094in}}%
\pgfpathlineto{\pgfqpoint{5.221678in}{1.298690in}}%
\pgfpathlineto{\pgfqpoint{5.221870in}{1.302183in}}%
\pgfpathlineto{\pgfqpoint{5.222639in}{1.299851in}}%
\pgfpathlineto{\pgfqpoint{5.223023in}{1.300466in}}%
\pgfpathlineto{\pgfqpoint{5.223599in}{1.293736in}}%
\pgfpathlineto{\pgfqpoint{5.224752in}{1.287633in}}%
\pgfpathlineto{\pgfqpoint{5.225713in}{1.295351in}}%
\pgfpathlineto{\pgfqpoint{5.226098in}{1.293457in}}%
\pgfpathlineto{\pgfqpoint{5.226290in}{1.294534in}}%
\pgfpathlineto{\pgfqpoint{5.226482in}{1.292951in}}%
\pgfpathlineto{\pgfqpoint{5.227251in}{1.285783in}}%
\pgfpathlineto{\pgfqpoint{5.227827in}{1.289258in}}%
\pgfpathlineto{\pgfqpoint{5.228019in}{1.289333in}}%
\pgfpathlineto{\pgfqpoint{5.228788in}{1.293440in}}%
\pgfpathlineto{\pgfqpoint{5.229172in}{1.293193in}}%
\pgfpathlineto{\pgfqpoint{5.229364in}{1.290818in}}%
\pgfpathlineto{\pgfqpoint{5.229941in}{1.294041in}}%
\pgfpathlineto{\pgfqpoint{5.230133in}{1.291636in}}%
\pgfpathlineto{\pgfqpoint{5.230710in}{1.295296in}}%
\pgfpathlineto{\pgfqpoint{5.231094in}{1.294074in}}%
\pgfpathlineto{\pgfqpoint{5.232055in}{1.291531in}}%
\pgfpathlineto{\pgfqpoint{5.232247in}{1.291591in}}%
\pgfpathlineto{\pgfqpoint{5.233016in}{1.296492in}}%
\pgfpathlineto{\pgfqpoint{5.233208in}{1.293400in}}%
\pgfpathlineto{\pgfqpoint{5.233784in}{1.291949in}}%
\pgfpathlineto{\pgfqpoint{5.233976in}{1.293688in}}%
\pgfpathlineto{\pgfqpoint{5.234169in}{1.292988in}}%
\pgfpathlineto{\pgfqpoint{5.234937in}{1.298164in}}%
\pgfpathlineto{\pgfqpoint{5.235129in}{1.293896in}}%
\pgfpathlineto{\pgfqpoint{5.236090in}{1.285290in}}%
\pgfpathlineto{\pgfqpoint{5.236475in}{1.288482in}}%
\pgfpathlineto{\pgfqpoint{5.236859in}{1.292623in}}%
\pgfpathlineto{\pgfqpoint{5.237628in}{1.291887in}}%
\pgfpathlineto{\pgfqpoint{5.238396in}{1.287201in}}%
\pgfpathlineto{\pgfqpoint{5.238781in}{1.281255in}}%
\pgfpathlineto{\pgfqpoint{5.239357in}{1.285609in}}%
\pgfpathlineto{\pgfqpoint{5.239549in}{1.287304in}}%
\pgfpathlineto{\pgfqpoint{5.239934in}{1.282994in}}%
\pgfpathlineto{\pgfqpoint{5.240126in}{1.282440in}}%
\pgfpathlineto{\pgfqpoint{5.240318in}{1.283876in}}%
\pgfpathlineto{\pgfqpoint{5.240510in}{1.283939in}}%
\pgfpathlineto{\pgfqpoint{5.240895in}{1.283176in}}%
\pgfpathlineto{\pgfqpoint{5.244161in}{1.298047in}}%
\pgfpathlineto{\pgfqpoint{5.244546in}{1.295716in}}%
\pgfpathlineto{\pgfqpoint{5.244930in}{1.298545in}}%
\pgfpathlineto{\pgfqpoint{5.245122in}{1.298772in}}%
\pgfpathlineto{\pgfqpoint{5.245314in}{1.301507in}}%
\pgfpathlineto{\pgfqpoint{5.245699in}{1.297766in}}%
\pgfpathlineto{\pgfqpoint{5.246083in}{1.299364in}}%
\pgfpathlineto{\pgfqpoint{5.246275in}{1.299081in}}%
\pgfpathlineto{\pgfqpoint{5.246852in}{1.295733in}}%
\pgfpathlineto{\pgfqpoint{5.247044in}{1.300569in}}%
\pgfpathlineto{\pgfqpoint{5.247236in}{1.298407in}}%
\pgfpathlineto{\pgfqpoint{5.248581in}{1.305834in}}%
\pgfpathlineto{\pgfqpoint{5.248966in}{1.305181in}}%
\pgfpathlineto{\pgfqpoint{5.249158in}{1.305475in}}%
\pgfpathlineto{\pgfqpoint{5.249926in}{1.304097in}}%
\pgfpathlineto{\pgfqpoint{5.250503in}{1.309874in}}%
\pgfpathlineto{\pgfqpoint{5.250695in}{1.309742in}}%
\pgfpathlineto{\pgfqpoint{5.251079in}{1.312327in}}%
\pgfpathlineto{\pgfqpoint{5.251272in}{1.308504in}}%
\pgfpathlineto{\pgfqpoint{5.251656in}{1.309638in}}%
\pgfpathlineto{\pgfqpoint{5.251848in}{1.309320in}}%
\pgfpathlineto{\pgfqpoint{5.253770in}{1.322391in}}%
\pgfpathlineto{\pgfqpoint{5.254538in}{1.314915in}}%
\pgfpathlineto{\pgfqpoint{5.255115in}{1.316737in}}%
\pgfpathlineto{\pgfqpoint{5.256460in}{1.322555in}}%
\pgfpathlineto{\pgfqpoint{5.256844in}{1.319804in}}%
\pgfpathlineto{\pgfqpoint{5.257229in}{1.314531in}}%
\pgfpathlineto{\pgfqpoint{5.257805in}{1.317616in}}%
\pgfpathlineto{\pgfqpoint{5.259150in}{1.325050in}}%
\pgfpathlineto{\pgfqpoint{5.259343in}{1.324314in}}%
\pgfpathlineto{\pgfqpoint{5.259535in}{1.326078in}}%
\pgfpathlineto{\pgfqpoint{5.260688in}{1.333888in}}%
\pgfpathlineto{\pgfqpoint{5.261072in}{1.333687in}}%
\pgfpathlineto{\pgfqpoint{5.261264in}{1.330948in}}%
\pgfpathlineto{\pgfqpoint{5.261841in}{1.336971in}}%
\pgfpathlineto{\pgfqpoint{5.262225in}{1.332850in}}%
\pgfpathlineto{\pgfqpoint{5.263570in}{1.326286in}}%
\pgfpathlineto{\pgfqpoint{5.263763in}{1.327827in}}%
\pgfpathlineto{\pgfqpoint{5.265684in}{1.335986in}}%
\pgfpathlineto{\pgfqpoint{5.265876in}{1.332919in}}%
\pgfpathlineto{\pgfqpoint{5.266837in}{1.326610in}}%
\pgfpathlineto{\pgfqpoint{5.267222in}{1.328547in}}%
\pgfpathlineto{\pgfqpoint{5.267414in}{1.329284in}}%
\pgfpathlineto{\pgfqpoint{5.267606in}{1.327541in}}%
\pgfpathlineto{\pgfqpoint{5.267798in}{1.328872in}}%
\pgfpathlineto{\pgfqpoint{5.268182in}{1.331074in}}%
\pgfpathlineto{\pgfqpoint{5.269143in}{1.325601in}}%
\pgfpathlineto{\pgfqpoint{5.269335in}{1.323588in}}%
\pgfpathlineto{\pgfqpoint{5.270104in}{1.326748in}}%
\pgfpathlineto{\pgfqpoint{5.270296in}{1.327148in}}%
\pgfpathlineto{\pgfqpoint{5.270873in}{1.322128in}}%
\pgfpathlineto{\pgfqpoint{5.271257in}{1.325265in}}%
\pgfpathlineto{\pgfqpoint{5.272987in}{1.332684in}}%
\pgfpathlineto{\pgfqpoint{5.273371in}{1.331172in}}%
\pgfpathlineto{\pgfqpoint{5.273563in}{1.332849in}}%
\pgfpathlineto{\pgfqpoint{5.273947in}{1.331336in}}%
\pgfpathlineto{\pgfqpoint{5.274332in}{1.333527in}}%
\pgfpathlineto{\pgfqpoint{5.274716in}{1.330723in}}%
\pgfpathlineto{\pgfqpoint{5.275100in}{1.332204in}}%
\pgfpathlineto{\pgfqpoint{5.277791in}{1.318994in}}%
\pgfpathlineto{\pgfqpoint{5.277983in}{1.320289in}}%
\pgfpathlineto{\pgfqpoint{5.278752in}{1.322721in}}%
\pgfpathlineto{\pgfqpoint{5.279328in}{1.322203in}}%
\pgfpathlineto{\pgfqpoint{5.279905in}{1.320250in}}%
\pgfpathlineto{\pgfqpoint{5.280289in}{1.321796in}}%
\pgfpathlineto{\pgfqpoint{5.280481in}{1.322674in}}%
\pgfpathlineto{\pgfqpoint{5.280865in}{1.321105in}}%
\pgfpathlineto{\pgfqpoint{5.281058in}{1.322416in}}%
\pgfpathlineto{\pgfqpoint{5.281826in}{1.314657in}}%
\pgfpathlineto{\pgfqpoint{5.282018in}{1.318992in}}%
\pgfpathlineto{\pgfqpoint{5.282211in}{1.322836in}}%
\pgfpathlineto{\pgfqpoint{5.282979in}{1.320477in}}%
\pgfpathlineto{\pgfqpoint{5.283171in}{1.319667in}}%
\pgfpathlineto{\pgfqpoint{5.283556in}{1.322397in}}%
\pgfpathlineto{\pgfqpoint{5.283748in}{1.323257in}}%
\pgfpathlineto{\pgfqpoint{5.284132in}{1.319967in}}%
\pgfpathlineto{\pgfqpoint{5.284324in}{1.316575in}}%
\pgfpathlineto{\pgfqpoint{5.284709in}{1.321892in}}%
\pgfpathlineto{\pgfqpoint{5.284901in}{1.320948in}}%
\pgfpathlineto{\pgfqpoint{5.287015in}{1.338898in}}%
\pgfpathlineto{\pgfqpoint{5.288168in}{1.333393in}}%
\pgfpathlineto{\pgfqpoint{5.288744in}{1.334917in}}%
\pgfpathlineto{\pgfqpoint{5.289705in}{1.336589in}}%
\pgfpathlineto{\pgfqpoint{5.289513in}{1.334319in}}%
\pgfpathlineto{\pgfqpoint{5.289897in}{1.336356in}}%
\pgfpathlineto{\pgfqpoint{5.290666in}{1.328173in}}%
\pgfpathlineto{\pgfqpoint{5.291243in}{1.331632in}}%
\pgfpathlineto{\pgfqpoint{5.293164in}{1.345502in}}%
\pgfpathlineto{\pgfqpoint{5.293741in}{1.340033in}}%
\pgfpathlineto{\pgfqpoint{5.294509in}{1.341352in}}%
\pgfpathlineto{\pgfqpoint{5.295662in}{1.346360in}}%
\pgfpathlineto{\pgfqpoint{5.294894in}{1.340524in}}%
\pgfpathlineto{\pgfqpoint{5.295855in}{1.345804in}}%
\pgfpathlineto{\pgfqpoint{5.296239in}{1.344908in}}%
\pgfpathlineto{\pgfqpoint{5.297200in}{1.338282in}}%
\pgfpathlineto{\pgfqpoint{5.297392in}{1.342326in}}%
\pgfpathlineto{\pgfqpoint{5.297968in}{1.337719in}}%
\pgfpathlineto{\pgfqpoint{5.298737in}{1.338188in}}%
\pgfpathlineto{\pgfqpoint{5.300467in}{1.358018in}}%
\pgfpathlineto{\pgfqpoint{5.301235in}{1.352985in}}%
\pgfpathlineto{\pgfqpoint{5.301620in}{1.354323in}}%
\pgfpathlineto{\pgfqpoint{5.302388in}{1.360869in}}%
\pgfpathlineto{\pgfqpoint{5.302773in}{1.358383in}}%
\pgfpathlineto{\pgfqpoint{5.303157in}{1.358341in}}%
\pgfpathlineto{\pgfqpoint{5.303926in}{1.349124in}}%
\pgfpathlineto{\pgfqpoint{5.303926in}{1.349124in}}%
\pgfusepath{stroke}%
\end{pgfscope}%
\begin{pgfscope}%
\pgfpathrectangle{\pgfqpoint{3.286364in}{0.660000in}}{\pgfqpoint{2.113636in}{2.100000in}}%
\pgfusepath{clip}%
\pgfsetroundcap%
\pgfsetroundjoin%
\pgfsetlinewidth{0.602250pt}%
\definecolor{currentstroke}{rgb}{0.968627,0.505882,0.749020}%
\pgfsetstrokecolor{currentstroke}%
\pgfsetdash{}{0pt}%
\pgfpathmoveto{\pgfqpoint{3.382438in}{1.716183in}}%
\pgfpathlineto{\pgfqpoint{3.382822in}{1.719658in}}%
\pgfpathlineto{\pgfqpoint{3.383399in}{1.715941in}}%
\pgfpathlineto{\pgfqpoint{3.383591in}{1.717582in}}%
\pgfpathlineto{\pgfqpoint{3.383783in}{1.718212in}}%
\pgfpathlineto{\pgfqpoint{3.383975in}{1.716402in}}%
\pgfpathlineto{\pgfqpoint{3.384360in}{1.713103in}}%
\pgfpathlineto{\pgfqpoint{3.385128in}{1.715786in}}%
\pgfpathlineto{\pgfqpoint{3.385321in}{1.716225in}}%
\pgfpathlineto{\pgfqpoint{3.385705in}{1.715340in}}%
\pgfpathlineto{\pgfqpoint{3.387434in}{1.706813in}}%
\pgfpathlineto{\pgfqpoint{3.387819in}{1.708457in}}%
\pgfpathlineto{\pgfqpoint{3.388395in}{1.713161in}}%
\pgfpathlineto{\pgfqpoint{3.388587in}{1.707577in}}%
\pgfpathlineto{\pgfqpoint{3.388780in}{1.708565in}}%
\pgfpathlineto{\pgfqpoint{3.388972in}{1.708208in}}%
\pgfpathlineto{\pgfqpoint{3.389164in}{1.708781in}}%
\pgfpathlineto{\pgfqpoint{3.389548in}{1.711188in}}%
\pgfpathlineto{\pgfqpoint{3.389933in}{1.706809in}}%
\pgfpathlineto{\pgfqpoint{3.390701in}{1.702573in}}%
\pgfpathlineto{\pgfqpoint{3.391278in}{1.703749in}}%
\pgfpathlineto{\pgfqpoint{3.392815in}{1.711342in}}%
\pgfpathlineto{\pgfqpoint{3.394160in}{1.700825in}}%
\pgfpathlineto{\pgfqpoint{3.394545in}{1.702794in}}%
\pgfpathlineto{\pgfqpoint{3.394737in}{1.704047in}}%
\pgfpathlineto{\pgfqpoint{3.395121in}{1.699605in}}%
\pgfpathlineto{\pgfqpoint{3.395313in}{1.701224in}}%
\pgfpathlineto{\pgfqpoint{3.395890in}{1.698313in}}%
\pgfpathlineto{\pgfqpoint{3.396658in}{1.698970in}}%
\pgfpathlineto{\pgfqpoint{3.396851in}{1.698993in}}%
\pgfpathlineto{\pgfqpoint{3.397619in}{1.697395in}}%
\pgfpathlineto{\pgfqpoint{3.397427in}{1.699391in}}%
\pgfpathlineto{\pgfqpoint{3.397811in}{1.698787in}}%
\pgfpathlineto{\pgfqpoint{3.398964in}{1.710636in}}%
\pgfpathlineto{\pgfqpoint{3.399349in}{1.708425in}}%
\pgfpathlineto{\pgfqpoint{3.400694in}{1.700264in}}%
\pgfpathlineto{\pgfqpoint{3.400886in}{1.701230in}}%
\pgfpathlineto{\pgfqpoint{3.403192in}{1.693550in}}%
\pgfpathlineto{\pgfqpoint{3.401463in}{1.702264in}}%
\pgfpathlineto{\pgfqpoint{3.403576in}{1.696529in}}%
\pgfpathlineto{\pgfqpoint{3.403769in}{1.697399in}}%
\pgfpathlineto{\pgfqpoint{3.403961in}{1.695228in}}%
\pgfpathlineto{\pgfqpoint{3.404153in}{1.696409in}}%
\pgfpathlineto{\pgfqpoint{3.405114in}{1.691298in}}%
\pgfpathlineto{\pgfqpoint{3.405498in}{1.693096in}}%
\pgfpathlineto{\pgfqpoint{3.406075in}{1.690107in}}%
\pgfpathlineto{\pgfqpoint{3.406267in}{1.693892in}}%
\pgfpathlineto{\pgfqpoint{3.407804in}{1.702578in}}%
\pgfpathlineto{\pgfqpoint{3.408189in}{1.705952in}}%
\pgfpathlineto{\pgfqpoint{3.408573in}{1.701673in}}%
\pgfpathlineto{\pgfqpoint{3.409149in}{1.704286in}}%
\pgfpathlineto{\pgfqpoint{3.409342in}{1.703490in}}%
\pgfpathlineto{\pgfqpoint{3.409726in}{1.706230in}}%
\pgfpathlineto{\pgfqpoint{3.410110in}{1.705741in}}%
\pgfpathlineto{\pgfqpoint{3.411071in}{1.709325in}}%
\pgfpathlineto{\pgfqpoint{3.411648in}{1.710383in}}%
\pgfpathlineto{\pgfqpoint{3.413377in}{1.702114in}}%
\pgfpathlineto{\pgfqpoint{3.413569in}{1.705662in}}%
\pgfpathlineto{\pgfqpoint{3.414338in}{1.702537in}}%
\pgfpathlineto{\pgfqpoint{3.415299in}{1.704634in}}%
\pgfpathlineto{\pgfqpoint{3.415491in}{1.705937in}}%
\pgfpathlineto{\pgfqpoint{3.415683in}{1.701158in}}%
\pgfpathlineto{\pgfqpoint{3.416067in}{1.702570in}}%
\pgfpathlineto{\pgfqpoint{3.416452in}{1.698987in}}%
\pgfpathlineto{\pgfqpoint{3.417028in}{1.702590in}}%
\pgfpathlineto{\pgfqpoint{3.417989in}{1.710880in}}%
\pgfpathlineto{\pgfqpoint{3.418373in}{1.707379in}}%
\pgfpathlineto{\pgfqpoint{3.419334in}{1.700638in}}%
\pgfpathlineto{\pgfqpoint{3.419911in}{1.702352in}}%
\pgfpathlineto{\pgfqpoint{3.420103in}{1.705050in}}%
\pgfpathlineto{\pgfqpoint{3.420872in}{1.701517in}}%
\pgfpathlineto{\pgfqpoint{3.421064in}{1.701143in}}%
\pgfpathlineto{\pgfqpoint{3.422409in}{1.709852in}}%
\pgfpathlineto{\pgfqpoint{3.423946in}{1.701319in}}%
\pgfpathlineto{\pgfqpoint{3.425099in}{1.706363in}}%
\pgfpathlineto{\pgfqpoint{3.425291in}{1.706296in}}%
\pgfpathlineto{\pgfqpoint{3.425484in}{1.706513in}}%
\pgfpathlineto{\pgfqpoint{3.425676in}{1.705206in}}%
\pgfpathlineto{\pgfqpoint{3.427213in}{1.695133in}}%
\pgfpathlineto{\pgfqpoint{3.428366in}{1.705318in}}%
\pgfpathlineto{\pgfqpoint{3.428943in}{1.703728in}}%
\pgfpathlineto{\pgfqpoint{3.429135in}{1.702278in}}%
\pgfpathlineto{\pgfqpoint{3.429519in}{1.705310in}}%
\pgfpathlineto{\pgfqpoint{3.429711in}{1.706447in}}%
\pgfpathlineto{\pgfqpoint{3.430288in}{1.703449in}}%
\pgfpathlineto{\pgfqpoint{3.430480in}{1.705538in}}%
\pgfpathlineto{\pgfqpoint{3.431057in}{1.707090in}}%
\pgfpathlineto{\pgfqpoint{3.431633in}{1.703325in}}%
\pgfpathlineto{\pgfqpoint{3.432594in}{1.713081in}}%
\pgfpathlineto{\pgfqpoint{3.432978in}{1.708825in}}%
\pgfpathlineto{\pgfqpoint{3.434131in}{1.703025in}}%
\pgfpathlineto{\pgfqpoint{3.434323in}{1.704344in}}%
\pgfpathlineto{\pgfqpoint{3.434516in}{1.704189in}}%
\pgfpathlineto{\pgfqpoint{3.434708in}{1.701250in}}%
\pgfpathlineto{\pgfqpoint{3.435284in}{1.707678in}}%
\pgfpathlineto{\pgfqpoint{3.435476in}{1.705750in}}%
\pgfpathlineto{\pgfqpoint{3.436629in}{1.711078in}}%
\pgfpathlineto{\pgfqpoint{3.436822in}{1.710512in}}%
\pgfpathlineto{\pgfqpoint{3.437398in}{1.706432in}}%
\pgfpathlineto{\pgfqpoint{3.437782in}{1.707976in}}%
\pgfpathlineto{\pgfqpoint{3.438551in}{1.714626in}}%
\pgfpathlineto{\pgfqpoint{3.438935in}{1.711422in}}%
\pgfpathlineto{\pgfqpoint{3.439512in}{1.711358in}}%
\pgfpathlineto{\pgfqpoint{3.439896in}{1.710070in}}%
\pgfpathlineto{\pgfqpoint{3.440281in}{1.712876in}}%
\pgfpathlineto{\pgfqpoint{3.440857in}{1.714165in}}%
\pgfpathlineto{\pgfqpoint{3.441818in}{1.702598in}}%
\pgfpathlineto{\pgfqpoint{3.442587in}{1.709062in}}%
\pgfpathlineto{\pgfqpoint{3.443163in}{1.707715in}}%
\pgfpathlineto{\pgfqpoint{3.443547in}{1.703686in}}%
\pgfpathlineto{\pgfqpoint{3.444124in}{1.707086in}}%
\pgfpathlineto{\pgfqpoint{3.445085in}{1.716092in}}%
\pgfpathlineto{\pgfqpoint{3.445661in}{1.714888in}}%
\pgfpathlineto{\pgfqpoint{3.446814in}{1.708035in}}%
\pgfpathlineto{\pgfqpoint{3.447391in}{1.712117in}}%
\pgfpathlineto{\pgfqpoint{3.447583in}{1.712710in}}%
\pgfpathlineto{\pgfqpoint{3.447775in}{1.710495in}}%
\pgfpathlineto{\pgfqpoint{3.449505in}{1.701029in}}%
\pgfpathlineto{\pgfqpoint{3.449697in}{1.700893in}}%
\pgfpathlineto{\pgfqpoint{3.449889in}{1.701392in}}%
\pgfpathlineto{\pgfqpoint{3.451042in}{1.702994in}}%
\pgfpathlineto{\pgfqpoint{3.450273in}{1.700108in}}%
\pgfpathlineto{\pgfqpoint{3.451234in}{1.702511in}}%
\pgfpathlineto{\pgfqpoint{3.452195in}{1.696365in}}%
\pgfpathlineto{\pgfqpoint{3.452964in}{1.696615in}}%
\pgfpathlineto{\pgfqpoint{3.454309in}{1.706432in}}%
\pgfpathlineto{\pgfqpoint{3.454501in}{1.706155in}}%
\pgfpathlineto{\pgfqpoint{3.455462in}{1.707100in}}%
\pgfpathlineto{\pgfqpoint{3.456999in}{1.688055in}}%
\pgfpathlineto{\pgfqpoint{3.457191in}{1.688149in}}%
\pgfpathlineto{\pgfqpoint{3.457384in}{1.687709in}}%
\pgfpathlineto{\pgfqpoint{3.458921in}{1.695145in}}%
\pgfpathlineto{\pgfqpoint{3.459113in}{1.694379in}}%
\pgfpathlineto{\pgfqpoint{3.459305in}{1.697357in}}%
\pgfpathlineto{\pgfqpoint{3.462572in}{1.712736in}}%
\pgfpathlineto{\pgfqpoint{3.463725in}{1.701036in}}%
\pgfpathlineto{\pgfqpoint{3.464302in}{1.702649in}}%
\pgfpathlineto{\pgfqpoint{3.464686in}{1.705935in}}%
\pgfpathlineto{\pgfqpoint{3.465070in}{1.701967in}}%
\pgfpathlineto{\pgfqpoint{3.465455in}{1.702947in}}%
\pgfpathlineto{\pgfqpoint{3.466223in}{1.697824in}}%
\pgfpathlineto{\pgfqpoint{3.467184in}{1.698524in}}%
\pgfpathlineto{\pgfqpoint{3.467376in}{1.698681in}}%
\pgfpathlineto{\pgfqpoint{3.468914in}{1.706118in}}%
\pgfpathlineto{\pgfqpoint{3.469106in}{1.704397in}}%
\pgfpathlineto{\pgfqpoint{3.469682in}{1.698388in}}%
\pgfpathlineto{\pgfqpoint{3.470259in}{1.702266in}}%
\pgfpathlineto{\pgfqpoint{3.471220in}{1.705566in}}%
\pgfpathlineto{\pgfqpoint{3.471412in}{1.704004in}}%
\pgfpathlineto{\pgfqpoint{3.471604in}{1.704734in}}%
\pgfpathlineto{\pgfqpoint{3.471988in}{1.701928in}}%
\pgfpathlineto{\pgfqpoint{3.472757in}{1.699285in}}%
\pgfpathlineto{\pgfqpoint{3.472949in}{1.702246in}}%
\pgfpathlineto{\pgfqpoint{3.473333in}{1.704351in}}%
\pgfpathlineto{\pgfqpoint{3.473718in}{1.701519in}}%
\pgfpathlineto{\pgfqpoint{3.473910in}{1.699493in}}%
\pgfpathlineto{\pgfqpoint{3.474294in}{1.704967in}}%
\pgfpathlineto{\pgfqpoint{3.474679in}{1.700575in}}%
\pgfpathlineto{\pgfqpoint{3.475063in}{1.704659in}}%
\pgfpathlineto{\pgfqpoint{3.475639in}{1.699513in}}%
\pgfpathlineto{\pgfqpoint{3.476216in}{1.700452in}}%
\pgfpathlineto{\pgfqpoint{3.476792in}{1.697191in}}%
\pgfpathlineto{\pgfqpoint{3.477946in}{1.689671in}}%
\pgfpathlineto{\pgfqpoint{3.478330in}{1.693800in}}%
\pgfpathlineto{\pgfqpoint{3.478906in}{1.690680in}}%
\pgfpathlineto{\pgfqpoint{3.479099in}{1.690078in}}%
\pgfpathlineto{\pgfqpoint{3.479291in}{1.691787in}}%
\pgfpathlineto{\pgfqpoint{3.479675in}{1.691591in}}%
\pgfpathlineto{\pgfqpoint{3.479867in}{1.692004in}}%
\pgfpathlineto{\pgfqpoint{3.480059in}{1.689941in}}%
\pgfpathlineto{\pgfqpoint{3.481981in}{1.683867in}}%
\pgfpathlineto{\pgfqpoint{3.482365in}{1.679350in}}%
\pgfpathlineto{\pgfqpoint{3.483518in}{1.680123in}}%
\pgfpathlineto{\pgfqpoint{3.483711in}{1.682384in}}%
\pgfpathlineto{\pgfqpoint{3.484479in}{1.680781in}}%
\pgfpathlineto{\pgfqpoint{3.485440in}{1.674930in}}%
\pgfpathlineto{\pgfqpoint{3.486017in}{1.675495in}}%
\pgfpathlineto{\pgfqpoint{3.486401in}{1.678562in}}%
\pgfpathlineto{\pgfqpoint{3.487362in}{1.686752in}}%
\pgfpathlineto{\pgfqpoint{3.487746in}{1.683380in}}%
\pgfpathlineto{\pgfqpoint{3.488707in}{1.678207in}}%
\pgfpathlineto{\pgfqpoint{3.488323in}{1.684630in}}%
\pgfpathlineto{\pgfqpoint{3.489283in}{1.678930in}}%
\pgfpathlineto{\pgfqpoint{3.489668in}{1.681318in}}%
\pgfpathlineto{\pgfqpoint{3.490052in}{1.679201in}}%
\pgfpathlineto{\pgfqpoint{3.490821in}{1.674759in}}%
\pgfpathlineto{\pgfqpoint{3.491205in}{1.678101in}}%
\pgfpathlineto{\pgfqpoint{3.491397in}{1.678578in}}%
\pgfpathlineto{\pgfqpoint{3.491782in}{1.677359in}}%
\pgfpathlineto{\pgfqpoint{3.492550in}{1.678715in}}%
\pgfpathlineto{\pgfqpoint{3.493127in}{1.673247in}}%
\pgfpathlineto{\pgfqpoint{3.493319in}{1.674366in}}%
\pgfpathlineto{\pgfqpoint{3.494664in}{1.665888in}}%
\pgfpathlineto{\pgfqpoint{3.494856in}{1.664547in}}%
\pgfpathlineto{\pgfqpoint{3.495241in}{1.669670in}}%
\pgfpathlineto{\pgfqpoint{3.495433in}{1.667587in}}%
\pgfpathlineto{\pgfqpoint{3.495817in}{1.666137in}}%
\pgfpathlineto{\pgfqpoint{3.496394in}{1.668002in}}%
\pgfpathlineto{\pgfqpoint{3.496586in}{1.667558in}}%
\pgfpathlineto{\pgfqpoint{3.496970in}{1.667140in}}%
\pgfpathlineto{\pgfqpoint{3.497162in}{1.667329in}}%
\pgfpathlineto{\pgfqpoint{3.497931in}{1.667164in}}%
\pgfpathlineto{\pgfqpoint{3.498700in}{1.670343in}}%
\pgfpathlineto{\pgfqpoint{3.499276in}{1.666126in}}%
\pgfpathlineto{\pgfqpoint{3.499660in}{1.670379in}}%
\pgfpathlineto{\pgfqpoint{3.501774in}{1.681635in}}%
\pgfpathlineto{\pgfqpoint{3.502159in}{1.679871in}}%
\pgfpathlineto{\pgfqpoint{3.502735in}{1.682239in}}%
\pgfpathlineto{\pgfqpoint{3.502927in}{1.685382in}}%
\pgfpathlineto{\pgfqpoint{3.503696in}{1.681690in}}%
\pgfpathlineto{\pgfqpoint{3.503888in}{1.682750in}}%
\pgfpathlineto{\pgfqpoint{3.504273in}{1.680374in}}%
\pgfpathlineto{\pgfqpoint{3.504657in}{1.681672in}}%
\pgfpathlineto{\pgfqpoint{3.504849in}{1.680893in}}%
\pgfpathlineto{\pgfqpoint{3.505618in}{1.685286in}}%
\pgfpathlineto{\pgfqpoint{3.506002in}{1.684717in}}%
\pgfpathlineto{\pgfqpoint{3.506963in}{1.677575in}}%
\pgfpathlineto{\pgfqpoint{3.507539in}{1.680982in}}%
\pgfpathlineto{\pgfqpoint{3.508692in}{1.685636in}}%
\pgfpathlineto{\pgfqpoint{3.511767in}{1.671692in}}%
\pgfpathlineto{\pgfqpoint{3.511959in}{1.672684in}}%
\pgfpathlineto{\pgfqpoint{3.512728in}{1.674380in}}%
\pgfpathlineto{\pgfqpoint{3.512536in}{1.672177in}}%
\pgfpathlineto{\pgfqpoint{3.512920in}{1.673536in}}%
\pgfpathlineto{\pgfqpoint{3.513112in}{1.671319in}}%
\pgfpathlineto{\pgfqpoint{3.513689in}{1.674750in}}%
\pgfpathlineto{\pgfqpoint{3.514457in}{1.677954in}}%
\pgfpathlineto{\pgfqpoint{3.514842in}{1.675555in}}%
\pgfpathlineto{\pgfqpoint{3.515418in}{1.676911in}}%
\pgfpathlineto{\pgfqpoint{3.515610in}{1.675665in}}%
\pgfpathlineto{\pgfqpoint{3.515995in}{1.672489in}}%
\pgfpathlineto{\pgfqpoint{3.516763in}{1.673010in}}%
\pgfpathlineto{\pgfqpoint{3.518109in}{1.682085in}}%
\pgfpathlineto{\pgfqpoint{3.518685in}{1.676793in}}%
\pgfpathlineto{\pgfqpoint{3.519069in}{1.680850in}}%
\pgfpathlineto{\pgfqpoint{3.520030in}{1.683024in}}%
\pgfpathlineto{\pgfqpoint{3.520222in}{1.682547in}}%
\pgfpathlineto{\pgfqpoint{3.520415in}{1.678187in}}%
\pgfpathlineto{\pgfqpoint{3.520991in}{1.686290in}}%
\pgfpathlineto{\pgfqpoint{3.521183in}{1.683228in}}%
\pgfpathlineto{\pgfqpoint{3.521375in}{1.683082in}}%
\pgfpathlineto{\pgfqpoint{3.522913in}{1.674754in}}%
\pgfpathlineto{\pgfqpoint{3.523105in}{1.676577in}}%
\pgfpathlineto{\pgfqpoint{3.523874in}{1.682370in}}%
\pgfpathlineto{\pgfqpoint{3.524066in}{1.686666in}}%
\pgfpathlineto{\pgfqpoint{3.524450in}{1.680119in}}%
\pgfpathlineto{\pgfqpoint{3.525027in}{1.683312in}}%
\pgfpathlineto{\pgfqpoint{3.525795in}{1.684867in}}%
\pgfpathlineto{\pgfqpoint{3.525603in}{1.682062in}}%
\pgfpathlineto{\pgfqpoint{3.525987in}{1.683677in}}%
\pgfpathlineto{\pgfqpoint{3.526180in}{1.682163in}}%
\pgfpathlineto{\pgfqpoint{3.526756in}{1.684512in}}%
\pgfpathlineto{\pgfqpoint{3.527141in}{1.683329in}}%
\pgfpathlineto{\pgfqpoint{3.529831in}{1.671789in}}%
\pgfpathlineto{\pgfqpoint{3.531176in}{1.678627in}}%
\pgfpathlineto{\pgfqpoint{3.531560in}{1.677087in}}%
\pgfpathlineto{\pgfqpoint{3.531753in}{1.677290in}}%
\pgfpathlineto{\pgfqpoint{3.531945in}{1.676395in}}%
\pgfpathlineto{\pgfqpoint{3.532137in}{1.676504in}}%
\pgfpathlineto{\pgfqpoint{3.533098in}{1.670497in}}%
\pgfpathlineto{\pgfqpoint{3.533482in}{1.670538in}}%
\pgfpathlineto{\pgfqpoint{3.534443in}{1.663443in}}%
\pgfpathlineto{\pgfqpoint{3.534635in}{1.661572in}}%
\pgfpathlineto{\pgfqpoint{3.535212in}{1.666119in}}%
\pgfpathlineto{\pgfqpoint{3.535596in}{1.667804in}}%
\pgfpathlineto{\pgfqpoint{3.535980in}{1.665143in}}%
\pgfpathlineto{\pgfqpoint{3.536557in}{1.661903in}}%
\pgfpathlineto{\pgfqpoint{3.536941in}{1.663520in}}%
\pgfpathlineto{\pgfqpoint{3.537133in}{1.665353in}}%
\pgfpathlineto{\pgfqpoint{3.537518in}{1.660791in}}%
\pgfpathlineto{\pgfqpoint{3.537710in}{1.661307in}}%
\pgfpathlineto{\pgfqpoint{3.540016in}{1.639870in}}%
\pgfpathlineto{\pgfqpoint{3.540208in}{1.641717in}}%
\pgfpathlineto{\pgfqpoint{3.540592in}{1.642768in}}%
\pgfpathlineto{\pgfqpoint{3.540784in}{1.641771in}}%
\pgfpathlineto{\pgfqpoint{3.541937in}{1.636262in}}%
\pgfpathlineto{\pgfqpoint{3.543283in}{1.647298in}}%
\pgfpathlineto{\pgfqpoint{3.543475in}{1.646466in}}%
\pgfpathlineto{\pgfqpoint{3.543667in}{1.645458in}}%
\pgfpathlineto{\pgfqpoint{3.543859in}{1.648675in}}%
\pgfpathlineto{\pgfqpoint{3.545012in}{1.655909in}}%
\pgfpathlineto{\pgfqpoint{3.545204in}{1.654064in}}%
\pgfpathlineto{\pgfqpoint{3.545973in}{1.654777in}}%
\pgfpathlineto{\pgfqpoint{3.546549in}{1.652885in}}%
\pgfpathlineto{\pgfqpoint{3.547126in}{1.655393in}}%
\pgfpathlineto{\pgfqpoint{3.547702in}{1.656351in}}%
\pgfpathlineto{\pgfqpoint{3.547895in}{1.654353in}}%
\pgfpathlineto{\pgfqpoint{3.548855in}{1.646104in}}%
\pgfpathlineto{\pgfqpoint{3.549624in}{1.648321in}}%
\pgfpathlineto{\pgfqpoint{3.550201in}{1.651595in}}%
\pgfpathlineto{\pgfqpoint{3.550969in}{1.650046in}}%
\pgfpathlineto{\pgfqpoint{3.551354in}{1.651472in}}%
\pgfpathlineto{\pgfqpoint{3.552507in}{1.647451in}}%
\pgfpathlineto{\pgfqpoint{3.552891in}{1.648922in}}%
\pgfpathlineto{\pgfqpoint{3.553275in}{1.653881in}}%
\pgfpathlineto{\pgfqpoint{3.554044in}{1.649885in}}%
\pgfpathlineto{\pgfqpoint{3.554236in}{1.649186in}}%
\pgfpathlineto{\pgfqpoint{3.554428in}{1.651964in}}%
\pgfpathlineto{\pgfqpoint{3.555389in}{1.659840in}}%
\pgfpathlineto{\pgfqpoint{3.556350in}{1.658578in}}%
\pgfpathlineto{\pgfqpoint{3.556542in}{1.657114in}}%
\pgfpathlineto{\pgfqpoint{3.557311in}{1.659768in}}%
\pgfpathlineto{\pgfqpoint{3.558464in}{1.656272in}}%
\pgfpathlineto{\pgfqpoint{3.560386in}{1.673001in}}%
\pgfpathlineto{\pgfqpoint{3.560770in}{1.672195in}}%
\pgfpathlineto{\pgfqpoint{3.561346in}{1.674147in}}%
\pgfpathlineto{\pgfqpoint{3.562884in}{1.666813in}}%
\pgfpathlineto{\pgfqpoint{3.563268in}{1.671462in}}%
\pgfpathlineto{\pgfqpoint{3.563845in}{1.665583in}}%
\pgfpathlineto{\pgfqpoint{3.564998in}{1.662008in}}%
\pgfpathlineto{\pgfqpoint{3.565190in}{1.662731in}}%
\pgfpathlineto{\pgfqpoint{3.566151in}{1.667856in}}%
\pgfpathlineto{\pgfqpoint{3.566727in}{1.664925in}}%
\pgfpathlineto{\pgfqpoint{3.567496in}{1.666189in}}%
\pgfpathlineto{\pgfqpoint{3.567880in}{1.665556in}}%
\pgfpathlineto{\pgfqpoint{3.569994in}{1.655718in}}%
\pgfpathlineto{\pgfqpoint{3.570186in}{1.657284in}}%
\pgfpathlineto{\pgfqpoint{3.570763in}{1.653715in}}%
\pgfpathlineto{\pgfqpoint{3.570955in}{1.655492in}}%
\pgfpathlineto{\pgfqpoint{3.571147in}{1.653025in}}%
\pgfpathlineto{\pgfqpoint{3.571916in}{1.656132in}}%
\pgfpathlineto{\pgfqpoint{3.572108in}{1.656367in}}%
\pgfpathlineto{\pgfqpoint{3.574222in}{1.635895in}}%
\pgfpathlineto{\pgfqpoint{3.574414in}{1.637985in}}%
\pgfpathlineto{\pgfqpoint{3.575759in}{1.641210in}}%
\pgfpathlineto{\pgfqpoint{3.576720in}{1.636161in}}%
\pgfpathlineto{\pgfqpoint{3.577104in}{1.636672in}}%
\pgfpathlineto{\pgfqpoint{3.577296in}{1.636768in}}%
\pgfpathlineto{\pgfqpoint{3.578834in}{1.645245in}}%
\pgfpathlineto{\pgfqpoint{3.580563in}{1.635292in}}%
\pgfpathlineto{\pgfqpoint{3.580948in}{1.638149in}}%
\pgfpathlineto{\pgfqpoint{3.581524in}{1.634520in}}%
\pgfpathlineto{\pgfqpoint{3.582485in}{1.628818in}}%
\pgfpathlineto{\pgfqpoint{3.582677in}{1.631954in}}%
\pgfpathlineto{\pgfqpoint{3.584599in}{1.650064in}}%
\pgfpathlineto{\pgfqpoint{3.584791in}{1.649880in}}%
\pgfpathlineto{\pgfqpoint{3.584983in}{1.652313in}}%
\pgfpathlineto{\pgfqpoint{3.585560in}{1.646267in}}%
\pgfpathlineto{\pgfqpoint{3.585752in}{1.645984in}}%
\pgfpathlineto{\pgfqpoint{3.587289in}{1.655117in}}%
\pgfpathlineto{\pgfqpoint{3.588634in}{1.648373in}}%
\pgfpathlineto{\pgfqpoint{3.589211in}{1.649073in}}%
\pgfpathlineto{\pgfqpoint{3.590172in}{1.643841in}}%
\pgfpathlineto{\pgfqpoint{3.590556in}{1.644067in}}%
\pgfpathlineto{\pgfqpoint{3.594207in}{1.657654in}}%
\pgfpathlineto{\pgfqpoint{3.594784in}{1.650271in}}%
\pgfpathlineto{\pgfqpoint{3.595360in}{1.657040in}}%
\pgfpathlineto{\pgfqpoint{3.595552in}{1.657414in}}%
\pgfpathlineto{\pgfqpoint{3.596897in}{1.650816in}}%
\pgfpathlineto{\pgfqpoint{3.597090in}{1.650014in}}%
\pgfpathlineto{\pgfqpoint{3.597474in}{1.652649in}}%
\pgfpathlineto{\pgfqpoint{3.598819in}{1.663213in}}%
\pgfpathlineto{\pgfqpoint{3.599011in}{1.662833in}}%
\pgfpathlineto{\pgfqpoint{3.599588in}{1.663481in}}%
\pgfpathlineto{\pgfqpoint{3.599396in}{1.661680in}}%
\pgfpathlineto{\pgfqpoint{3.599780in}{1.661958in}}%
\pgfpathlineto{\pgfqpoint{3.600357in}{1.658604in}}%
\pgfpathlineto{\pgfqpoint{3.600741in}{1.663326in}}%
\pgfpathlineto{\pgfqpoint{3.601702in}{1.668110in}}%
\pgfpathlineto{\pgfqpoint{3.602086in}{1.663898in}}%
\pgfpathlineto{\pgfqpoint{3.602663in}{1.661622in}}%
\pgfpathlineto{\pgfqpoint{3.604200in}{1.647770in}}%
\pgfpathlineto{\pgfqpoint{3.604584in}{1.652399in}}%
\pgfpathlineto{\pgfqpoint{3.605545in}{1.649788in}}%
\pgfpathlineto{\pgfqpoint{3.606506in}{1.646701in}}%
\pgfpathlineto{\pgfqpoint{3.605929in}{1.650962in}}%
\pgfpathlineto{\pgfqpoint{3.606698in}{1.648533in}}%
\pgfpathlineto{\pgfqpoint{3.607082in}{1.650667in}}%
\pgfpathlineto{\pgfqpoint{3.607275in}{1.650327in}}%
\pgfpathlineto{\pgfqpoint{3.607467in}{1.646184in}}%
\pgfpathlineto{\pgfqpoint{3.608043in}{1.652944in}}%
\pgfpathlineto{\pgfqpoint{3.608235in}{1.649033in}}%
\pgfpathlineto{\pgfqpoint{3.609773in}{1.654610in}}%
\pgfpathlineto{\pgfqpoint{3.611887in}{1.664316in}}%
\pgfpathlineto{\pgfqpoint{3.612079in}{1.664581in}}%
\pgfpathlineto{\pgfqpoint{3.612847in}{1.668420in}}%
\pgfpathlineto{\pgfqpoint{3.613040in}{1.665630in}}%
\pgfpathlineto{\pgfqpoint{3.613232in}{1.664309in}}%
\pgfpathlineto{\pgfqpoint{3.613616in}{1.665856in}}%
\pgfpathlineto{\pgfqpoint{3.614000in}{1.665765in}}%
\pgfpathlineto{\pgfqpoint{3.615922in}{1.678790in}}%
\pgfpathlineto{\pgfqpoint{3.616883in}{1.672983in}}%
\pgfpathlineto{\pgfqpoint{3.617267in}{1.674525in}}%
\pgfpathlineto{\pgfqpoint{3.618228in}{1.678682in}}%
\pgfpathlineto{\pgfqpoint{3.618420in}{1.676390in}}%
\pgfpathlineto{\pgfqpoint{3.618612in}{1.675387in}}%
\pgfpathlineto{\pgfqpoint{3.618805in}{1.676432in}}%
\pgfpathlineto{\pgfqpoint{3.618997in}{1.680285in}}%
\pgfpathlineto{\pgfqpoint{3.619573in}{1.676219in}}%
\pgfpathlineto{\pgfqpoint{3.619765in}{1.676964in}}%
\pgfpathlineto{\pgfqpoint{3.620726in}{1.668268in}}%
\pgfpathlineto{\pgfqpoint{3.621303in}{1.669898in}}%
\pgfpathlineto{\pgfqpoint{3.621879in}{1.673740in}}%
\pgfpathlineto{\pgfqpoint{3.622264in}{1.667663in}}%
\pgfpathlineto{\pgfqpoint{3.622456in}{1.667088in}}%
\pgfpathlineto{\pgfqpoint{3.622648in}{1.668689in}}%
\pgfpathlineto{\pgfqpoint{3.622840in}{1.668213in}}%
\pgfpathlineto{\pgfqpoint{3.623032in}{1.670546in}}%
\pgfpathlineto{\pgfqpoint{3.623609in}{1.665486in}}%
\pgfpathlineto{\pgfqpoint{3.623801in}{1.669295in}}%
\pgfpathlineto{\pgfqpoint{3.624185in}{1.670726in}}%
\pgfpathlineto{\pgfqpoint{3.624377in}{1.668606in}}%
\pgfpathlineto{\pgfqpoint{3.625531in}{1.671293in}}%
\pgfpathlineto{\pgfqpoint{3.625915in}{1.672497in}}%
\pgfpathlineto{\pgfqpoint{3.626107in}{1.670804in}}%
\pgfpathlineto{\pgfqpoint{3.626876in}{1.666068in}}%
\pgfpathlineto{\pgfqpoint{3.627260in}{1.667538in}}%
\pgfpathlineto{\pgfqpoint{3.628605in}{1.673312in}}%
\pgfpathlineto{\pgfqpoint{3.629950in}{1.663754in}}%
\pgfpathlineto{\pgfqpoint{3.630527in}{1.667281in}}%
\pgfpathlineto{\pgfqpoint{3.632449in}{1.677175in}}%
\pgfpathlineto{\pgfqpoint{3.632641in}{1.673046in}}%
\pgfpathlineto{\pgfqpoint{3.633025in}{1.668920in}}%
\pgfpathlineto{\pgfqpoint{3.633602in}{1.670246in}}%
\pgfpathlineto{\pgfqpoint{3.635331in}{1.684878in}}%
\pgfpathlineto{\pgfqpoint{3.635523in}{1.683613in}}%
\pgfpathlineto{\pgfqpoint{3.635715in}{1.683931in}}%
\pgfpathlineto{\pgfqpoint{3.635908in}{1.681019in}}%
\pgfpathlineto{\pgfqpoint{3.636292in}{1.686052in}}%
\pgfpathlineto{\pgfqpoint{3.636484in}{1.686002in}}%
\pgfpathlineto{\pgfqpoint{3.637829in}{1.691713in}}%
\pgfpathlineto{\pgfqpoint{3.638214in}{1.686292in}}%
\pgfpathlineto{\pgfqpoint{3.638982in}{1.690204in}}%
\pgfpathlineto{\pgfqpoint{3.641288in}{1.676836in}}%
\pgfpathlineto{\pgfqpoint{3.643018in}{1.669952in}}%
\pgfpathlineto{\pgfqpoint{3.643210in}{1.671090in}}%
\pgfpathlineto{\pgfqpoint{3.644363in}{1.681571in}}%
\pgfpathlineto{\pgfqpoint{3.644939in}{1.678779in}}%
\pgfpathlineto{\pgfqpoint{3.645132in}{1.673465in}}%
\pgfpathlineto{\pgfqpoint{3.645900in}{1.676228in}}%
\pgfpathlineto{\pgfqpoint{3.647245in}{1.683747in}}%
\pgfpathlineto{\pgfqpoint{3.648591in}{1.679380in}}%
\pgfpathlineto{\pgfqpoint{3.649744in}{1.680852in}}%
\pgfpathlineto{\pgfqpoint{3.650897in}{1.676100in}}%
\pgfpathlineto{\pgfqpoint{3.651089in}{1.679218in}}%
\pgfpathlineto{\pgfqpoint{3.651281in}{1.679382in}}%
\pgfpathlineto{\pgfqpoint{3.651473in}{1.682139in}}%
\pgfpathlineto{\pgfqpoint{3.652050in}{1.679151in}}%
\pgfpathlineto{\pgfqpoint{3.652434in}{1.680247in}}%
\pgfpathlineto{\pgfqpoint{3.652818in}{1.682180in}}%
\pgfpathlineto{\pgfqpoint{3.653203in}{1.680054in}}%
\pgfpathlineto{\pgfqpoint{3.653587in}{1.680991in}}%
\pgfpathlineto{\pgfqpoint{3.656277in}{1.674049in}}%
\pgfpathlineto{\pgfqpoint{3.656470in}{1.675544in}}%
\pgfpathlineto{\pgfqpoint{3.657815in}{1.678927in}}%
\pgfpathlineto{\pgfqpoint{3.658007in}{1.678675in}}%
\pgfpathlineto{\pgfqpoint{3.658583in}{1.684569in}}%
\pgfpathlineto{\pgfqpoint{3.659352in}{1.683982in}}%
\pgfpathlineto{\pgfqpoint{3.659736in}{1.684684in}}%
\pgfpathlineto{\pgfqpoint{3.660121in}{1.684423in}}%
\pgfpathlineto{\pgfqpoint{3.660313in}{1.686223in}}%
\pgfpathlineto{\pgfqpoint{3.660697in}{1.680166in}}%
\pgfpathlineto{\pgfqpoint{3.661466in}{1.684695in}}%
\pgfpathlineto{\pgfqpoint{3.663195in}{1.677413in}}%
\pgfpathlineto{\pgfqpoint{3.663388in}{1.681364in}}%
\pgfpathlineto{\pgfqpoint{3.664156in}{1.678703in}}%
\pgfpathlineto{\pgfqpoint{3.664733in}{1.676729in}}%
\pgfpathlineto{\pgfqpoint{3.664925in}{1.679116in}}%
\pgfpathlineto{\pgfqpoint{3.665117in}{1.679751in}}%
\pgfpathlineto{\pgfqpoint{3.665501in}{1.678601in}}%
\pgfpathlineto{\pgfqpoint{3.665694in}{1.678839in}}%
\pgfpathlineto{\pgfqpoint{3.667423in}{1.670087in}}%
\pgfpathlineto{\pgfqpoint{3.667807in}{1.672322in}}%
\pgfpathlineto{\pgfqpoint{3.669345in}{1.683300in}}%
\pgfpathlineto{\pgfqpoint{3.669729in}{1.681362in}}%
\pgfpathlineto{\pgfqpoint{3.671459in}{1.672539in}}%
\pgfpathlineto{\pgfqpoint{3.672996in}{1.679495in}}%
\pgfpathlineto{\pgfqpoint{3.673188in}{1.679018in}}%
\pgfpathlineto{\pgfqpoint{3.674726in}{1.673869in}}%
\pgfpathlineto{\pgfqpoint{3.675110in}{1.679838in}}%
\pgfpathlineto{\pgfqpoint{3.675879in}{1.676027in}}%
\pgfpathlineto{\pgfqpoint{3.676071in}{1.676969in}}%
\pgfpathlineto{\pgfqpoint{3.676455in}{1.674170in}}%
\pgfpathlineto{\pgfqpoint{3.677032in}{1.674529in}}%
\pgfpathlineto{\pgfqpoint{3.677608in}{1.672000in}}%
\pgfpathlineto{\pgfqpoint{3.678185in}{1.673913in}}%
\pgfpathlineto{\pgfqpoint{3.678377in}{1.672034in}}%
\pgfpathlineto{\pgfqpoint{3.678569in}{1.667699in}}%
\pgfpathlineto{\pgfqpoint{3.679338in}{1.672655in}}%
\pgfpathlineto{\pgfqpoint{3.679530in}{1.677599in}}%
\pgfpathlineto{\pgfqpoint{3.680298in}{1.675332in}}%
\pgfpathlineto{\pgfqpoint{3.680683in}{1.671029in}}%
\pgfpathlineto{\pgfqpoint{3.681451in}{1.674478in}}%
\pgfpathlineto{\pgfqpoint{3.681836in}{1.672703in}}%
\pgfpathlineto{\pgfqpoint{3.682220in}{1.675707in}}%
\pgfpathlineto{\pgfqpoint{3.683757in}{1.682811in}}%
\pgfpathlineto{\pgfqpoint{3.682797in}{1.675345in}}%
\pgfpathlineto{\pgfqpoint{3.683950in}{1.681789in}}%
\pgfpathlineto{\pgfqpoint{3.684142in}{1.678799in}}%
\pgfpathlineto{\pgfqpoint{3.684910in}{1.683178in}}%
\pgfpathlineto{\pgfqpoint{3.686448in}{1.694884in}}%
\pgfpathlineto{\pgfqpoint{3.686640in}{1.692347in}}%
\pgfpathlineto{\pgfqpoint{3.688177in}{1.684044in}}%
\pgfpathlineto{\pgfqpoint{3.688562in}{1.684415in}}%
\pgfpathlineto{\pgfqpoint{3.689907in}{1.695251in}}%
\pgfpathlineto{\pgfqpoint{3.690483in}{1.690695in}}%
\pgfpathlineto{\pgfqpoint{3.691060in}{1.693803in}}%
\pgfpathlineto{\pgfqpoint{3.691444in}{1.694738in}}%
\pgfpathlineto{\pgfqpoint{3.691636in}{1.692402in}}%
\pgfpathlineto{\pgfqpoint{3.692789in}{1.689807in}}%
\pgfpathlineto{\pgfqpoint{3.693558in}{1.692095in}}%
\pgfpathlineto{\pgfqpoint{3.693942in}{1.691516in}}%
\pgfpathlineto{\pgfqpoint{3.695095in}{1.687453in}}%
\pgfpathlineto{\pgfqpoint{3.695672in}{1.693484in}}%
\pgfpathlineto{\pgfqpoint{3.696248in}{1.689340in}}%
\pgfpathlineto{\pgfqpoint{3.696440in}{1.689267in}}%
\pgfpathlineto{\pgfqpoint{3.696633in}{1.691621in}}%
\pgfpathlineto{\pgfqpoint{3.697017in}{1.683957in}}%
\pgfpathlineto{\pgfqpoint{3.697209in}{1.683912in}}%
\pgfpathlineto{\pgfqpoint{3.697978in}{1.680141in}}%
\pgfpathlineto{\pgfqpoint{3.698170in}{1.683852in}}%
\pgfpathlineto{\pgfqpoint{3.699131in}{1.686743in}}%
\pgfpathlineto{\pgfqpoint{3.700284in}{1.672963in}}%
\pgfpathlineto{\pgfqpoint{3.700860in}{1.676621in}}%
\pgfpathlineto{\pgfqpoint{3.701437in}{1.680673in}}%
\pgfpathlineto{\pgfqpoint{3.701629in}{1.679691in}}%
\pgfpathlineto{\pgfqpoint{3.703359in}{1.690260in}}%
\pgfpathlineto{\pgfqpoint{3.703551in}{1.689813in}}%
\pgfpathlineto{\pgfqpoint{3.705088in}{1.678583in}}%
\pgfpathlineto{\pgfqpoint{3.705472in}{1.679396in}}%
\pgfpathlineto{\pgfqpoint{3.705857in}{1.678365in}}%
\pgfpathlineto{\pgfqpoint{3.706433in}{1.675737in}}%
\pgfpathlineto{\pgfqpoint{3.707010in}{1.676792in}}%
\pgfpathlineto{\pgfqpoint{3.707202in}{1.676242in}}%
\pgfpathlineto{\pgfqpoint{3.707394in}{1.678912in}}%
\pgfpathlineto{\pgfqpoint{3.707778in}{1.680800in}}%
\pgfpathlineto{\pgfqpoint{3.707971in}{1.682156in}}%
\pgfpathlineto{\pgfqpoint{3.708355in}{1.681070in}}%
\pgfpathlineto{\pgfqpoint{3.708547in}{1.676358in}}%
\pgfpathlineto{\pgfqpoint{3.709508in}{1.678293in}}%
\pgfpathlineto{\pgfqpoint{3.712006in}{1.694414in}}%
\pgfpathlineto{\pgfqpoint{3.712390in}{1.692954in}}%
\pgfpathlineto{\pgfqpoint{3.712583in}{1.692908in}}%
\pgfpathlineto{\pgfqpoint{3.713736in}{1.696941in}}%
\pgfpathlineto{\pgfqpoint{3.713928in}{1.696801in}}%
\pgfpathlineto{\pgfqpoint{3.714889in}{1.692383in}}%
\pgfpathlineto{\pgfqpoint{3.714504in}{1.696880in}}%
\pgfpathlineto{\pgfqpoint{3.715081in}{1.693897in}}%
\pgfpathlineto{\pgfqpoint{3.716234in}{1.700302in}}%
\pgfpathlineto{\pgfqpoint{3.716618in}{1.695471in}}%
\pgfpathlineto{\pgfqpoint{3.717387in}{1.698073in}}%
\pgfpathlineto{\pgfqpoint{3.718540in}{1.702314in}}%
\pgfpathlineto{\pgfqpoint{3.718732in}{1.704860in}}%
\pgfpathlineto{\pgfqpoint{3.719308in}{1.699151in}}%
\pgfpathlineto{\pgfqpoint{3.719501in}{1.701131in}}%
\pgfpathlineto{\pgfqpoint{3.720654in}{1.703743in}}%
\pgfpathlineto{\pgfqpoint{3.720846in}{1.701446in}}%
\pgfpathlineto{\pgfqpoint{3.721230in}{1.707031in}}%
\pgfpathlineto{\pgfqpoint{3.721422in}{1.706756in}}%
\pgfpathlineto{\pgfqpoint{3.721614in}{1.708150in}}%
\pgfpathlineto{\pgfqpoint{3.721807in}{1.706300in}}%
\pgfpathlineto{\pgfqpoint{3.722191in}{1.700336in}}%
\pgfpathlineto{\pgfqpoint{3.722960in}{1.704852in}}%
\pgfpathlineto{\pgfqpoint{3.724113in}{1.697282in}}%
\pgfpathlineto{\pgfqpoint{3.724497in}{1.700558in}}%
\pgfpathlineto{\pgfqpoint{3.725650in}{1.704215in}}%
\pgfpathlineto{\pgfqpoint{3.724881in}{1.699728in}}%
\pgfpathlineto{\pgfqpoint{3.725842in}{1.703749in}}%
\pgfpathlineto{\pgfqpoint{3.726227in}{1.698622in}}%
\pgfpathlineto{\pgfqpoint{3.726803in}{1.704262in}}%
\pgfpathlineto{\pgfqpoint{3.726995in}{1.703261in}}%
\pgfpathlineto{\pgfqpoint{3.727380in}{1.706690in}}%
\pgfpathlineto{\pgfqpoint{3.727572in}{1.706801in}}%
\pgfpathlineto{\pgfqpoint{3.727764in}{1.708861in}}%
\pgfpathlineto{\pgfqpoint{3.728340in}{1.705636in}}%
\pgfpathlineto{\pgfqpoint{3.728725in}{1.705057in}}%
\pgfpathlineto{\pgfqpoint{3.728917in}{1.706572in}}%
\pgfpathlineto{\pgfqpoint{3.730262in}{1.712411in}}%
\pgfpathlineto{\pgfqpoint{3.730454in}{1.711771in}}%
\pgfpathlineto{\pgfqpoint{3.731223in}{1.708836in}}%
\pgfpathlineto{\pgfqpoint{3.731415in}{1.710105in}}%
\pgfpathlineto{\pgfqpoint{3.731799in}{1.714799in}}%
\pgfpathlineto{\pgfqpoint{3.732568in}{1.713967in}}%
\pgfpathlineto{\pgfqpoint{3.736027in}{1.699368in}}%
\pgfpathlineto{\pgfqpoint{3.733337in}{1.714428in}}%
\pgfpathlineto{\pgfqpoint{3.736411in}{1.702585in}}%
\pgfpathlineto{\pgfqpoint{3.737564in}{1.708289in}}%
\pgfpathlineto{\pgfqpoint{3.737757in}{1.707333in}}%
\pgfpathlineto{\pgfqpoint{3.737949in}{1.706170in}}%
\pgfpathlineto{\pgfqpoint{3.738333in}{1.709697in}}%
\pgfpathlineto{\pgfqpoint{3.738525in}{1.710291in}}%
\pgfpathlineto{\pgfqpoint{3.739486in}{1.695250in}}%
\pgfpathlineto{\pgfqpoint{3.740255in}{1.697204in}}%
\pgfpathlineto{\pgfqpoint{3.741023in}{1.693013in}}%
\pgfpathlineto{\pgfqpoint{3.741408in}{1.695627in}}%
\pgfpathlineto{\pgfqpoint{3.742369in}{1.699418in}}%
\pgfpathlineto{\pgfqpoint{3.742561in}{1.696730in}}%
\pgfpathlineto{\pgfqpoint{3.743329in}{1.692410in}}%
\pgfpathlineto{\pgfqpoint{3.743714in}{1.694475in}}%
\pgfpathlineto{\pgfqpoint{3.744098in}{1.692615in}}%
\pgfpathlineto{\pgfqpoint{3.744482in}{1.693530in}}%
\pgfpathlineto{\pgfqpoint{3.745251in}{1.689411in}}%
\pgfpathlineto{\pgfqpoint{3.745443in}{1.693190in}}%
\pgfpathlineto{\pgfqpoint{3.746212in}{1.688907in}}%
\pgfpathlineto{\pgfqpoint{3.746404in}{1.689989in}}%
\pgfpathlineto{\pgfqpoint{3.746596in}{1.687460in}}%
\pgfpathlineto{\pgfqpoint{3.746981in}{1.680251in}}%
\pgfpathlineto{\pgfqpoint{3.748134in}{1.681384in}}%
\pgfpathlineto{\pgfqpoint{3.748902in}{1.687748in}}%
\pgfpathlineto{\pgfqpoint{3.749287in}{1.684261in}}%
\pgfpathlineto{\pgfqpoint{3.749479in}{1.685432in}}%
\pgfpathlineto{\pgfqpoint{3.749863in}{1.682095in}}%
\pgfpathlineto{\pgfqpoint{3.750248in}{1.684423in}}%
\pgfpathlineto{\pgfqpoint{3.750440in}{1.682188in}}%
\pgfpathlineto{\pgfqpoint{3.751016in}{1.687202in}}%
\pgfpathlineto{\pgfqpoint{3.751401in}{1.689726in}}%
\pgfpathlineto{\pgfqpoint{3.752169in}{1.688743in}}%
\pgfpathlineto{\pgfqpoint{3.752361in}{1.688778in}}%
\pgfpathlineto{\pgfqpoint{3.752554in}{1.686906in}}%
\pgfpathlineto{\pgfqpoint{3.753130in}{1.689830in}}%
\pgfpathlineto{\pgfqpoint{3.753707in}{1.695055in}}%
\pgfpathlineto{\pgfqpoint{3.754283in}{1.692070in}}%
\pgfpathlineto{\pgfqpoint{3.754667in}{1.690942in}}%
\pgfpathlineto{\pgfqpoint{3.755244in}{1.692085in}}%
\pgfpathlineto{\pgfqpoint{3.756589in}{1.700978in}}%
\pgfpathlineto{\pgfqpoint{3.756973in}{1.698176in}}%
\pgfpathlineto{\pgfqpoint{3.757550in}{1.698908in}}%
\pgfpathlineto{\pgfqpoint{3.758703in}{1.691082in}}%
\pgfpathlineto{\pgfqpoint{3.759664in}{1.683511in}}%
\pgfpathlineto{\pgfqpoint{3.759856in}{1.686161in}}%
\pgfpathlineto{\pgfqpoint{3.760048in}{1.687941in}}%
\pgfpathlineto{\pgfqpoint{3.760432in}{1.681996in}}%
\pgfpathlineto{\pgfqpoint{3.761393in}{1.679980in}}%
\pgfpathlineto{\pgfqpoint{3.761585in}{1.681709in}}%
\pgfpathlineto{\pgfqpoint{3.762354in}{1.686833in}}%
\pgfpathlineto{\pgfqpoint{3.762738in}{1.685372in}}%
\pgfpathlineto{\pgfqpoint{3.763123in}{1.680041in}}%
\pgfpathlineto{\pgfqpoint{3.763891in}{1.683476in}}%
\pgfpathlineto{\pgfqpoint{3.764852in}{1.690167in}}%
\pgfpathlineto{\pgfqpoint{3.765237in}{1.688646in}}%
\pgfpathlineto{\pgfqpoint{3.766005in}{1.680708in}}%
\pgfpathlineto{\pgfqpoint{3.766582in}{1.682462in}}%
\pgfpathlineto{\pgfqpoint{3.766774in}{1.686911in}}%
\pgfpathlineto{\pgfqpoint{3.767735in}{1.683485in}}%
\pgfpathlineto{\pgfqpoint{3.770233in}{1.672595in}}%
\pgfpathlineto{\pgfqpoint{3.768311in}{1.684298in}}%
\pgfpathlineto{\pgfqpoint{3.770425in}{1.672797in}}%
\pgfpathlineto{\pgfqpoint{3.771963in}{1.685715in}}%
\pgfpathlineto{\pgfqpoint{3.772155in}{1.686542in}}%
\pgfpathlineto{\pgfqpoint{3.772539in}{1.684318in}}%
\pgfpathlineto{\pgfqpoint{3.772923in}{1.685270in}}%
\pgfpathlineto{\pgfqpoint{3.773692in}{1.683036in}}%
\pgfpathlineto{\pgfqpoint{3.773308in}{1.685564in}}%
\pgfpathlineto{\pgfqpoint{3.773884in}{1.684449in}}%
\pgfpathlineto{\pgfqpoint{3.774461in}{1.690092in}}%
\pgfpathlineto{\pgfqpoint{3.775229in}{1.688958in}}%
\pgfpathlineto{\pgfqpoint{3.775998in}{1.685931in}}%
\pgfpathlineto{\pgfqpoint{3.776190in}{1.687664in}}%
\pgfpathlineto{\pgfqpoint{3.777728in}{1.692109in}}%
\pgfpathlineto{\pgfqpoint{3.778496in}{1.688386in}}%
\pgfpathlineto{\pgfqpoint{3.778688in}{1.691861in}}%
\pgfpathlineto{\pgfqpoint{3.779073in}{1.693095in}}%
\pgfpathlineto{\pgfqpoint{3.779457in}{1.690983in}}%
\pgfpathlineto{\pgfqpoint{3.779649in}{1.690611in}}%
\pgfpathlineto{\pgfqpoint{3.779841in}{1.691115in}}%
\pgfpathlineto{\pgfqpoint{3.780226in}{1.690082in}}%
\pgfpathlineto{\pgfqpoint{3.781187in}{1.694821in}}%
\pgfpathlineto{\pgfqpoint{3.781379in}{1.691551in}}%
\pgfpathlineto{\pgfqpoint{3.782147in}{1.696147in}}%
\pgfpathlineto{\pgfqpoint{3.783108in}{1.691737in}}%
\pgfpathlineto{\pgfqpoint{3.782532in}{1.696964in}}%
\pgfpathlineto{\pgfqpoint{3.783493in}{1.694309in}}%
\pgfpathlineto{\pgfqpoint{3.783685in}{1.695615in}}%
\pgfpathlineto{\pgfqpoint{3.784069in}{1.692928in}}%
\pgfpathlineto{\pgfqpoint{3.784261in}{1.693500in}}%
\pgfpathlineto{\pgfqpoint{3.784646in}{1.691484in}}%
\pgfpathlineto{\pgfqpoint{3.785222in}{1.692075in}}%
\pgfpathlineto{\pgfqpoint{3.786183in}{1.701582in}}%
\pgfpathlineto{\pgfqpoint{3.786567in}{1.701382in}}%
\pgfpathlineto{\pgfqpoint{3.786759in}{1.696591in}}%
\pgfpathlineto{\pgfqpoint{3.787528in}{1.702827in}}%
\pgfpathlineto{\pgfqpoint{3.788489in}{1.696184in}}%
\pgfpathlineto{\pgfqpoint{3.788873in}{1.700420in}}%
\pgfpathlineto{\pgfqpoint{3.789450in}{1.703315in}}%
\pgfpathlineto{\pgfqpoint{3.789642in}{1.699862in}}%
\pgfpathlineto{\pgfqpoint{3.789834in}{1.700404in}}%
\pgfpathlineto{\pgfqpoint{3.790411in}{1.695718in}}%
\pgfpathlineto{\pgfqpoint{3.790987in}{1.697776in}}%
\pgfpathlineto{\pgfqpoint{3.791179in}{1.697938in}}%
\pgfpathlineto{\pgfqpoint{3.792717in}{1.685734in}}%
\pgfpathlineto{\pgfqpoint{3.792909in}{1.686987in}}%
\pgfpathlineto{\pgfqpoint{3.793101in}{1.684719in}}%
\pgfpathlineto{\pgfqpoint{3.793485in}{1.685274in}}%
\pgfpathlineto{\pgfqpoint{3.793677in}{1.684356in}}%
\pgfpathlineto{\pgfqpoint{3.793870in}{1.686950in}}%
\pgfpathlineto{\pgfqpoint{3.795215in}{1.700873in}}%
\pgfpathlineto{\pgfqpoint{3.795599in}{1.699285in}}%
\pgfpathlineto{\pgfqpoint{3.799443in}{1.672171in}}%
\pgfpathlineto{\pgfqpoint{3.799635in}{1.673039in}}%
\pgfpathlineto{\pgfqpoint{3.800019in}{1.670141in}}%
\pgfpathlineto{\pgfqpoint{3.800403in}{1.670166in}}%
\pgfpathlineto{\pgfqpoint{3.800596in}{1.669347in}}%
\pgfpathlineto{\pgfqpoint{3.801364in}{1.670572in}}%
\pgfpathlineto{\pgfqpoint{3.801941in}{1.664627in}}%
\pgfpathlineto{\pgfqpoint{3.802133in}{1.667650in}}%
\pgfpathlineto{\pgfqpoint{3.802902in}{1.665177in}}%
\pgfpathlineto{\pgfqpoint{3.805208in}{1.654345in}}%
\pgfpathlineto{\pgfqpoint{3.807514in}{1.645576in}}%
\pgfpathlineto{\pgfqpoint{3.805784in}{1.654761in}}%
\pgfpathlineto{\pgfqpoint{3.807706in}{1.649325in}}%
\pgfpathlineto{\pgfqpoint{3.808474in}{1.651559in}}%
\pgfpathlineto{\pgfqpoint{3.808667in}{1.649539in}}%
\pgfpathlineto{\pgfqpoint{3.810204in}{1.641097in}}%
\pgfpathlineto{\pgfqpoint{3.810396in}{1.641341in}}%
\pgfpathlineto{\pgfqpoint{3.810780in}{1.639200in}}%
\pgfpathlineto{\pgfqpoint{3.811165in}{1.640453in}}%
\pgfpathlineto{\pgfqpoint{3.811357in}{1.643545in}}%
\pgfpathlineto{\pgfqpoint{3.812126in}{1.642551in}}%
\pgfpathlineto{\pgfqpoint{3.812702in}{1.637427in}}%
\pgfpathlineto{\pgfqpoint{3.813279in}{1.638839in}}%
\pgfpathlineto{\pgfqpoint{3.814047in}{1.643886in}}%
\pgfpathlineto{\pgfqpoint{3.814432in}{1.642965in}}%
\pgfpathlineto{\pgfqpoint{3.815008in}{1.639509in}}%
\pgfpathlineto{\pgfqpoint{3.815585in}{1.641941in}}%
\pgfpathlineto{\pgfqpoint{3.816930in}{1.645696in}}%
\pgfpathlineto{\pgfqpoint{3.817122in}{1.647562in}}%
\pgfpathlineto{\pgfqpoint{3.817314in}{1.645464in}}%
\pgfpathlineto{\pgfqpoint{3.817891in}{1.645668in}}%
\pgfpathlineto{\pgfqpoint{3.819044in}{1.642418in}}%
\pgfpathlineto{\pgfqpoint{3.819236in}{1.643359in}}%
\pgfpathlineto{\pgfqpoint{3.819620in}{1.640269in}}%
\pgfpathlineto{\pgfqpoint{3.819812in}{1.641685in}}%
\pgfpathlineto{\pgfqpoint{3.820005in}{1.639938in}}%
\pgfpathlineto{\pgfqpoint{3.820389in}{1.642436in}}%
\pgfpathlineto{\pgfqpoint{3.820773in}{1.640986in}}%
\pgfpathlineto{\pgfqpoint{3.821926in}{1.647456in}}%
\pgfpathlineto{\pgfqpoint{3.822118in}{1.647396in}}%
\pgfpathlineto{\pgfqpoint{3.822311in}{1.647864in}}%
\pgfpathlineto{\pgfqpoint{3.822503in}{1.645637in}}%
\pgfpathlineto{\pgfqpoint{3.826154in}{1.633629in}}%
\pgfpathlineto{\pgfqpoint{3.826346in}{1.635095in}}%
\pgfpathlineto{\pgfqpoint{3.827115in}{1.638081in}}%
\pgfpathlineto{\pgfqpoint{3.827307in}{1.636050in}}%
\pgfpathlineto{\pgfqpoint{3.827499in}{1.634969in}}%
\pgfpathlineto{\pgfqpoint{3.827883in}{1.637123in}}%
\pgfpathlineto{\pgfqpoint{3.828460in}{1.644951in}}%
\pgfpathlineto{\pgfqpoint{3.829421in}{1.643325in}}%
\pgfpathlineto{\pgfqpoint{3.829805in}{1.639492in}}%
\pgfpathlineto{\pgfqpoint{3.830382in}{1.640542in}}%
\pgfpathlineto{\pgfqpoint{3.830958in}{1.635947in}}%
\pgfpathlineto{\pgfqpoint{3.833456in}{1.650347in}}%
\pgfpathlineto{\pgfqpoint{3.833648in}{1.649998in}}%
\pgfpathlineto{\pgfqpoint{3.833841in}{1.652688in}}%
\pgfpathlineto{\pgfqpoint{3.834417in}{1.646708in}}%
\pgfpathlineto{\pgfqpoint{3.834801in}{1.643378in}}%
\pgfpathlineto{\pgfqpoint{3.835378in}{1.646827in}}%
\pgfpathlineto{\pgfqpoint{3.835570in}{1.646068in}}%
\pgfpathlineto{\pgfqpoint{3.837107in}{1.635698in}}%
\pgfpathlineto{\pgfqpoint{3.837684in}{1.636552in}}%
\pgfpathlineto{\pgfqpoint{3.838837in}{1.652440in}}%
\pgfpathlineto{\pgfqpoint{3.839221in}{1.648664in}}%
\pgfpathlineto{\pgfqpoint{3.839606in}{1.649532in}}%
\pgfpathlineto{\pgfqpoint{3.839798in}{1.654457in}}%
\pgfpathlineto{\pgfqpoint{3.840374in}{1.645890in}}%
\pgfpathlineto{\pgfqpoint{3.840566in}{1.645124in}}%
\pgfpathlineto{\pgfqpoint{3.840759in}{1.646041in}}%
\pgfpathlineto{\pgfqpoint{3.840951in}{1.646059in}}%
\pgfpathlineto{\pgfqpoint{3.841719in}{1.653848in}}%
\pgfpathlineto{\pgfqpoint{3.842104in}{1.650810in}}%
\pgfpathlineto{\pgfqpoint{3.844794in}{1.637861in}}%
\pgfpathlineto{\pgfqpoint{3.845371in}{1.639684in}}%
\pgfpathlineto{\pgfqpoint{3.845755in}{1.637205in}}%
\pgfpathlineto{\pgfqpoint{3.847100in}{1.634399in}}%
\pgfpathlineto{\pgfqpoint{3.848830in}{1.627061in}}%
\pgfpathlineto{\pgfqpoint{3.849022in}{1.627256in}}%
\pgfpathlineto{\pgfqpoint{3.849406in}{1.632703in}}%
\pgfpathlineto{\pgfqpoint{3.849983in}{1.626798in}}%
\pgfpathlineto{\pgfqpoint{3.850175in}{1.624395in}}%
\pgfpathlineto{\pgfqpoint{3.850751in}{1.627240in}}%
\pgfpathlineto{\pgfqpoint{3.850944in}{1.626361in}}%
\pgfpathlineto{\pgfqpoint{3.851328in}{1.627863in}}%
\pgfpathlineto{\pgfqpoint{3.851520in}{1.627242in}}%
\pgfpathlineto{\pgfqpoint{3.852673in}{1.634911in}}%
\pgfpathlineto{\pgfqpoint{3.853057in}{1.633932in}}%
\pgfpathlineto{\pgfqpoint{3.853442in}{1.626274in}}%
\pgfpathlineto{\pgfqpoint{3.854210in}{1.629384in}}%
\pgfpathlineto{\pgfqpoint{3.854403in}{1.629654in}}%
\pgfpathlineto{\pgfqpoint{3.854979in}{1.632901in}}%
\pgfpathlineto{\pgfqpoint{3.855363in}{1.632234in}}%
\pgfpathlineto{\pgfqpoint{3.856709in}{1.625566in}}%
\pgfpathlineto{\pgfqpoint{3.858054in}{1.618965in}}%
\pgfpathlineto{\pgfqpoint{3.858438in}{1.620645in}}%
\pgfpathlineto{\pgfqpoint{3.859015in}{1.629011in}}%
\pgfpathlineto{\pgfqpoint{3.859783in}{1.623237in}}%
\pgfpathlineto{\pgfqpoint{3.860552in}{1.629326in}}%
\pgfpathlineto{\pgfqpoint{3.862089in}{1.641714in}}%
\pgfpathlineto{\pgfqpoint{3.862281in}{1.639163in}}%
\pgfpathlineto{\pgfqpoint{3.862666in}{1.644948in}}%
\pgfpathlineto{\pgfqpoint{3.862858in}{1.644317in}}%
\pgfpathlineto{\pgfqpoint{3.863242in}{1.643806in}}%
\pgfpathlineto{\pgfqpoint{3.863434in}{1.639807in}}%
\pgfpathlineto{\pgfqpoint{3.864203in}{1.642908in}}%
\pgfpathlineto{\pgfqpoint{3.865356in}{1.648476in}}%
\pgfpathlineto{\pgfqpoint{3.864587in}{1.640817in}}%
\pgfpathlineto{\pgfqpoint{3.865740in}{1.648155in}}%
\pgfpathlineto{\pgfqpoint{3.865933in}{1.645837in}}%
\pgfpathlineto{\pgfqpoint{3.866701in}{1.646992in}}%
\pgfpathlineto{\pgfqpoint{3.866893in}{1.647844in}}%
\pgfpathlineto{\pgfqpoint{3.867086in}{1.645816in}}%
\pgfpathlineto{\pgfqpoint{3.867278in}{1.645816in}}%
\pgfpathlineto{\pgfqpoint{3.867662in}{1.641034in}}%
\pgfpathlineto{\pgfqpoint{3.868815in}{1.641805in}}%
\pgfpathlineto{\pgfqpoint{3.869584in}{1.646737in}}%
\pgfpathlineto{\pgfqpoint{3.869968in}{1.643204in}}%
\pgfpathlineto{\pgfqpoint{3.871121in}{1.653148in}}%
\pgfpathlineto{\pgfqpoint{3.871698in}{1.649125in}}%
\pgfpathlineto{\pgfqpoint{3.873427in}{1.640661in}}%
\pgfpathlineto{\pgfqpoint{3.877655in}{1.662688in}}%
\pgfpathlineto{\pgfqpoint{3.877847in}{1.661744in}}%
\pgfpathlineto{\pgfqpoint{3.878424in}{1.654342in}}%
\pgfpathlineto{\pgfqpoint{3.879384in}{1.656519in}}%
\pgfpathlineto{\pgfqpoint{3.879961in}{1.656390in}}%
\pgfpathlineto{\pgfqpoint{3.881114in}{1.663995in}}%
\pgfpathlineto{\pgfqpoint{3.882651in}{1.652424in}}%
\pgfpathlineto{\pgfqpoint{3.883036in}{1.653512in}}%
\pgfpathlineto{\pgfqpoint{3.883420in}{1.652141in}}%
\pgfpathlineto{\pgfqpoint{3.884189in}{1.648402in}}%
\pgfpathlineto{\pgfqpoint{3.884573in}{1.651347in}}%
\pgfpathlineto{\pgfqpoint{3.884765in}{1.651442in}}%
\pgfpathlineto{\pgfqpoint{3.884957in}{1.648750in}}%
\pgfpathlineto{\pgfqpoint{3.885534in}{1.654094in}}%
\pgfpathlineto{\pgfqpoint{3.886302in}{1.657868in}}%
\pgfpathlineto{\pgfqpoint{3.886687in}{1.656036in}}%
\pgfpathlineto{\pgfqpoint{3.886879in}{1.653544in}}%
\pgfpathlineto{\pgfqpoint{3.887455in}{1.656546in}}%
\pgfpathlineto{\pgfqpoint{3.887648in}{1.655855in}}%
\pgfpathlineto{\pgfqpoint{3.888032in}{1.657786in}}%
\pgfpathlineto{\pgfqpoint{3.888416in}{1.654738in}}%
\pgfpathlineto{\pgfqpoint{3.888608in}{1.657232in}}%
\pgfpathlineto{\pgfqpoint{3.889954in}{1.654009in}}%
\pgfpathlineto{\pgfqpoint{3.890146in}{1.652957in}}%
\pgfpathlineto{\pgfqpoint{3.890338in}{1.656504in}}%
\pgfpathlineto{\pgfqpoint{3.890914in}{1.654583in}}%
\pgfpathlineto{\pgfqpoint{3.891107in}{1.654028in}}%
\pgfpathlineto{\pgfqpoint{3.891491in}{1.655910in}}%
\pgfpathlineto{\pgfqpoint{3.892836in}{1.657974in}}%
\pgfpathlineto{\pgfqpoint{3.893028in}{1.657897in}}%
\pgfpathlineto{\pgfqpoint{3.894181in}{1.667514in}}%
\pgfpathlineto{\pgfqpoint{3.894566in}{1.666055in}}%
\pgfpathlineto{\pgfqpoint{3.894758in}{1.664415in}}%
\pgfpathlineto{\pgfqpoint{3.895142in}{1.668425in}}%
\pgfpathlineto{\pgfqpoint{3.895719in}{1.670460in}}%
\pgfpathlineto{\pgfqpoint{3.895911in}{1.668110in}}%
\pgfpathlineto{\pgfqpoint{3.896487in}{1.670261in}}%
\pgfpathlineto{\pgfqpoint{3.897448in}{1.672669in}}%
\pgfpathlineto{\pgfqpoint{3.896872in}{1.670067in}}%
\pgfpathlineto{\pgfqpoint{3.897833in}{1.671777in}}%
\pgfpathlineto{\pgfqpoint{3.898409in}{1.666419in}}%
\pgfpathlineto{\pgfqpoint{3.899178in}{1.668620in}}%
\pgfpathlineto{\pgfqpoint{3.900139in}{1.672732in}}%
\pgfpathlineto{\pgfqpoint{3.900331in}{1.670525in}}%
\pgfpathlineto{\pgfqpoint{3.901484in}{1.663223in}}%
\pgfpathlineto{\pgfqpoint{3.901676in}{1.665611in}}%
\pgfpathlineto{\pgfqpoint{3.903021in}{1.672736in}}%
\pgfpathlineto{\pgfqpoint{3.904174in}{1.664751in}}%
\pgfpathlineto{\pgfqpoint{3.904366in}{1.666130in}}%
\pgfpathlineto{\pgfqpoint{3.904558in}{1.667003in}}%
\pgfpathlineto{\pgfqpoint{3.905135in}{1.664628in}}%
\pgfpathlineto{\pgfqpoint{3.905327in}{1.661980in}}%
\pgfpathlineto{\pgfqpoint{3.905711in}{1.670464in}}%
\pgfpathlineto{\pgfqpoint{3.905904in}{1.667621in}}%
\pgfpathlineto{\pgfqpoint{3.906672in}{1.668881in}}%
\pgfpathlineto{\pgfqpoint{3.906288in}{1.667301in}}%
\pgfpathlineto{\pgfqpoint{3.906864in}{1.667936in}}%
\pgfpathlineto{\pgfqpoint{3.907057in}{1.664753in}}%
\pgfpathlineto{\pgfqpoint{3.907633in}{1.669449in}}%
\pgfpathlineto{\pgfqpoint{3.907825in}{1.669202in}}%
\pgfpathlineto{\pgfqpoint{3.908017in}{1.668887in}}%
\pgfpathlineto{\pgfqpoint{3.908210in}{1.669129in}}%
\pgfpathlineto{\pgfqpoint{3.909363in}{1.675635in}}%
\pgfpathlineto{\pgfqpoint{3.909555in}{1.674191in}}%
\pgfpathlineto{\pgfqpoint{3.909747in}{1.674158in}}%
\pgfpathlineto{\pgfqpoint{3.910900in}{1.665902in}}%
\pgfpathlineto{\pgfqpoint{3.911284in}{1.667205in}}%
\pgfpathlineto{\pgfqpoint{3.912629in}{1.672266in}}%
\pgfpathlineto{\pgfqpoint{3.913398in}{1.678770in}}%
\pgfpathlineto{\pgfqpoint{3.913782in}{1.674716in}}%
\pgfpathlineto{\pgfqpoint{3.915128in}{1.671718in}}%
\pgfpathlineto{\pgfqpoint{3.915512in}{1.673954in}}%
\pgfpathlineto{\pgfqpoint{3.916088in}{1.671453in}}%
\pgfpathlineto{\pgfqpoint{3.916281in}{1.673188in}}%
\pgfpathlineto{\pgfqpoint{3.918202in}{1.665122in}}%
\pgfpathlineto{\pgfqpoint{3.918395in}{1.665983in}}%
\pgfpathlineto{\pgfqpoint{3.918971in}{1.663969in}}%
\pgfpathlineto{\pgfqpoint{3.919932in}{1.654230in}}%
\pgfpathlineto{\pgfqpoint{3.920701in}{1.657737in}}%
\pgfpathlineto{\pgfqpoint{3.920893in}{1.657301in}}%
\pgfpathlineto{\pgfqpoint{3.921085in}{1.658890in}}%
\pgfpathlineto{\pgfqpoint{3.921277in}{1.658645in}}%
\pgfpathlineto{\pgfqpoint{3.921854in}{1.660546in}}%
\pgfpathlineto{\pgfqpoint{3.922046in}{1.659721in}}%
\pgfpathlineto{\pgfqpoint{3.923007in}{1.655127in}}%
\pgfpathlineto{\pgfqpoint{3.923391in}{1.656311in}}%
\pgfpathlineto{\pgfqpoint{3.923967in}{1.653931in}}%
\pgfpathlineto{\pgfqpoint{3.924160in}{1.656836in}}%
\pgfpathlineto{\pgfqpoint{3.924352in}{1.659779in}}%
\pgfpathlineto{\pgfqpoint{3.924928in}{1.656577in}}%
\pgfpathlineto{\pgfqpoint{3.925313in}{1.657491in}}%
\pgfpathlineto{\pgfqpoint{3.926466in}{1.651539in}}%
\pgfpathlineto{\pgfqpoint{3.926081in}{1.657649in}}%
\pgfpathlineto{\pgfqpoint{3.926850in}{1.652513in}}%
\pgfpathlineto{\pgfqpoint{3.927811in}{1.653890in}}%
\pgfpathlineto{\pgfqpoint{3.927426in}{1.650615in}}%
\pgfpathlineto{\pgfqpoint{3.928003in}{1.652719in}}%
\pgfpathlineto{\pgfqpoint{3.929156in}{1.648943in}}%
\pgfpathlineto{\pgfqpoint{3.929732in}{1.656167in}}%
\pgfpathlineto{\pgfqpoint{3.930501in}{1.654370in}}%
\pgfpathlineto{\pgfqpoint{3.930693in}{1.654616in}}%
\pgfpathlineto{\pgfqpoint{3.930885in}{1.653384in}}%
\pgfpathlineto{\pgfqpoint{3.931078in}{1.653959in}}%
\pgfpathlineto{\pgfqpoint{3.932038in}{1.646938in}}%
\pgfpathlineto{\pgfqpoint{3.932615in}{1.650561in}}%
\pgfpathlineto{\pgfqpoint{3.932807in}{1.652556in}}%
\pgfpathlineto{\pgfqpoint{3.933576in}{1.649473in}}%
\pgfpathlineto{\pgfqpoint{3.934537in}{1.646995in}}%
\pgfpathlineto{\pgfqpoint{3.934729in}{1.650953in}}%
\pgfpathlineto{\pgfqpoint{3.935305in}{1.644540in}}%
\pgfpathlineto{\pgfqpoint{3.937035in}{1.636086in}}%
\pgfpathlineto{\pgfqpoint{3.937996in}{1.639658in}}%
\pgfpathlineto{\pgfqpoint{3.937419in}{1.635900in}}%
\pgfpathlineto{\pgfqpoint{3.938188in}{1.638980in}}%
\pgfpathlineto{\pgfqpoint{3.938956in}{1.637798in}}%
\pgfpathlineto{\pgfqpoint{3.939533in}{1.635598in}}%
\pgfpathlineto{\pgfqpoint{3.939917in}{1.637591in}}%
\pgfpathlineto{\pgfqpoint{3.940686in}{1.639566in}}%
\pgfpathlineto{\pgfqpoint{3.940878in}{1.637097in}}%
\pgfpathlineto{\pgfqpoint{3.941070in}{1.638730in}}%
\pgfpathlineto{\pgfqpoint{3.942223in}{1.633529in}}%
\pgfpathlineto{\pgfqpoint{3.942608in}{1.634745in}}%
\pgfpathlineto{\pgfqpoint{3.944529in}{1.645431in}}%
\pgfpathlineto{\pgfqpoint{3.944722in}{1.642239in}}%
\pgfpathlineto{\pgfqpoint{3.945490in}{1.646717in}}%
\pgfpathlineto{\pgfqpoint{3.946259in}{1.653444in}}%
\pgfpathlineto{\pgfqpoint{3.946835in}{1.649130in}}%
\pgfpathlineto{\pgfqpoint{3.947988in}{1.647346in}}%
\pgfpathlineto{\pgfqpoint{3.948181in}{1.648566in}}%
\pgfpathlineto{\pgfqpoint{3.948373in}{1.646215in}}%
\pgfpathlineto{\pgfqpoint{3.948757in}{1.652272in}}%
\pgfpathlineto{\pgfqpoint{3.948949in}{1.654622in}}%
\pgfpathlineto{\pgfqpoint{3.949526in}{1.650653in}}%
\pgfpathlineto{\pgfqpoint{3.949718in}{1.650898in}}%
\pgfpathlineto{\pgfqpoint{3.950871in}{1.639347in}}%
\pgfpathlineto{\pgfqpoint{3.951447in}{1.643020in}}%
\pgfpathlineto{\pgfqpoint{3.952216in}{1.642506in}}%
\pgfpathlineto{\pgfqpoint{3.953177in}{1.649026in}}%
\pgfpathlineto{\pgfqpoint{3.953369in}{1.649192in}}%
\pgfpathlineto{\pgfqpoint{3.954138in}{1.646037in}}%
\pgfpathlineto{\pgfqpoint{3.954522in}{1.648411in}}%
\pgfpathlineto{\pgfqpoint{3.954714in}{1.649353in}}%
\pgfpathlineto{\pgfqpoint{3.954906in}{1.647205in}}%
\pgfpathlineto{\pgfqpoint{3.955291in}{1.648264in}}%
\pgfpathlineto{\pgfqpoint{3.955675in}{1.646992in}}%
\pgfpathlineto{\pgfqpoint{3.956059in}{1.649465in}}%
\pgfpathlineto{\pgfqpoint{3.957789in}{1.653845in}}%
\pgfpathlineto{\pgfqpoint{3.958173in}{1.652636in}}%
\pgfpathlineto{\pgfqpoint{3.958558in}{1.649280in}}%
\pgfpathlineto{\pgfqpoint{3.959326in}{1.651794in}}%
\pgfpathlineto{\pgfqpoint{3.959518in}{1.652569in}}%
\pgfpathlineto{\pgfqpoint{3.959711in}{1.651610in}}%
\pgfpathlineto{\pgfqpoint{3.960095in}{1.651693in}}%
\pgfpathlineto{\pgfqpoint{3.960479in}{1.648447in}}%
\pgfpathlineto{\pgfqpoint{3.961056in}{1.650171in}}%
\pgfpathlineto{\pgfqpoint{3.961248in}{1.652272in}}%
\pgfpathlineto{\pgfqpoint{3.961824in}{1.647192in}}%
\pgfpathlineto{\pgfqpoint{3.962017in}{1.647627in}}%
\pgfpathlineto{\pgfqpoint{3.963938in}{1.668478in}}%
\pgfpathlineto{\pgfqpoint{3.965668in}{1.658855in}}%
\pgfpathlineto{\pgfqpoint{3.965860in}{1.659727in}}%
\pgfpathlineto{\pgfqpoint{3.966629in}{1.667260in}}%
\pgfpathlineto{\pgfqpoint{3.967205in}{1.666812in}}%
\pgfpathlineto{\pgfqpoint{3.969127in}{1.651624in}}%
\pgfpathlineto{\pgfqpoint{3.970088in}{1.655491in}}%
\pgfpathlineto{\pgfqpoint{3.970472in}{1.655402in}}%
\pgfpathlineto{\pgfqpoint{3.971241in}{1.648693in}}%
\pgfpathlineto{\pgfqpoint{3.971625in}{1.652878in}}%
\pgfpathlineto{\pgfqpoint{3.972009in}{1.651252in}}%
\pgfpathlineto{\pgfqpoint{3.972202in}{1.651833in}}%
\pgfpathlineto{\pgfqpoint{3.972394in}{1.650711in}}%
\pgfpathlineto{\pgfqpoint{3.972778in}{1.652126in}}%
\pgfpathlineto{\pgfqpoint{3.973162in}{1.654352in}}%
\pgfpathlineto{\pgfqpoint{3.973739in}{1.650585in}}%
\pgfpathlineto{\pgfqpoint{3.974892in}{1.641386in}}%
\pgfpathlineto{\pgfqpoint{3.975084in}{1.642796in}}%
\pgfpathlineto{\pgfqpoint{3.975276in}{1.641182in}}%
\pgfpathlineto{\pgfqpoint{3.975468in}{1.645692in}}%
\pgfpathlineto{\pgfqpoint{3.976237in}{1.641416in}}%
\pgfpathlineto{\pgfqpoint{3.977967in}{1.652057in}}%
\pgfpathlineto{\pgfqpoint{3.978927in}{1.648312in}}%
\pgfpathlineto{\pgfqpoint{3.978351in}{1.652868in}}%
\pgfpathlineto{\pgfqpoint{3.979120in}{1.650138in}}%
\pgfpathlineto{\pgfqpoint{3.979504in}{1.653724in}}%
\pgfpathlineto{\pgfqpoint{3.980273in}{1.651514in}}%
\pgfpathlineto{\pgfqpoint{3.982194in}{1.643706in}}%
\pgfpathlineto{\pgfqpoint{3.980657in}{1.651870in}}%
\pgfpathlineto{\pgfqpoint{3.982771in}{1.645469in}}%
\pgfpathlineto{\pgfqpoint{3.983539in}{1.651335in}}%
\pgfpathlineto{\pgfqpoint{3.984116in}{1.651156in}}%
\pgfpathlineto{\pgfqpoint{3.985269in}{1.646000in}}%
\pgfpathlineto{\pgfqpoint{3.985653in}{1.647029in}}%
\pgfpathlineto{\pgfqpoint{3.986806in}{1.653016in}}%
\pgfpathlineto{\pgfqpoint{3.986998in}{1.649766in}}%
\pgfpathlineto{\pgfqpoint{3.987383in}{1.650386in}}%
\pgfpathlineto{\pgfqpoint{3.988728in}{1.647593in}}%
\pgfpathlineto{\pgfqpoint{3.989304in}{1.642326in}}%
\pgfpathlineto{\pgfqpoint{3.989881in}{1.645252in}}%
\pgfpathlineto{\pgfqpoint{3.990265in}{1.645996in}}%
\pgfpathlineto{\pgfqpoint{3.991034in}{1.642482in}}%
\pgfpathlineto{\pgfqpoint{3.991418in}{1.644875in}}%
\pgfpathlineto{\pgfqpoint{3.992956in}{1.652899in}}%
\pgfpathlineto{\pgfqpoint{3.993148in}{1.652562in}}%
\pgfpathlineto{\pgfqpoint{3.994685in}{1.670198in}}%
\pgfpathlineto{\pgfqpoint{3.995262in}{1.669958in}}%
\pgfpathlineto{\pgfqpoint{3.996030in}{1.667169in}}%
\pgfpathlineto{\pgfqpoint{3.996223in}{1.671062in}}%
\pgfpathlineto{\pgfqpoint{3.996991in}{1.665879in}}%
\pgfpathlineto{\pgfqpoint{3.998721in}{1.659081in}}%
\pgfpathlineto{\pgfqpoint{3.998913in}{1.660120in}}%
\pgfpathlineto{\pgfqpoint{3.999297in}{1.656901in}}%
\pgfpathlineto{\pgfqpoint{3.999489in}{1.655847in}}%
\pgfpathlineto{\pgfqpoint{3.999874in}{1.657930in}}%
\pgfpathlineto{\pgfqpoint{4.000066in}{1.657663in}}%
\pgfpathlineto{\pgfqpoint{4.001603in}{1.666125in}}%
\pgfpathlineto{\pgfqpoint{4.002372in}{1.662513in}}%
\pgfpathlineto{\pgfqpoint{4.002756in}{1.663472in}}%
\pgfpathlineto{\pgfqpoint{4.002948in}{1.664580in}}%
\pgfpathlineto{\pgfqpoint{4.003333in}{1.661396in}}%
\pgfpathlineto{\pgfqpoint{4.005254in}{1.644310in}}%
\pgfpathlineto{\pgfqpoint{4.005447in}{1.644492in}}%
\pgfpathlineto{\pgfqpoint{4.005831in}{1.646316in}}%
\pgfpathlineto{\pgfqpoint{4.006407in}{1.643895in}}%
\pgfpathlineto{\pgfqpoint{4.006984in}{1.640214in}}%
\pgfpathlineto{\pgfqpoint{4.007368in}{1.643232in}}%
\pgfpathlineto{\pgfqpoint{4.008329in}{1.647201in}}%
\pgfpathlineto{\pgfqpoint{4.007753in}{1.641961in}}%
\pgfpathlineto{\pgfqpoint{4.008521in}{1.643913in}}%
\pgfpathlineto{\pgfqpoint{4.012749in}{1.615712in}}%
\pgfpathlineto{\pgfqpoint{4.013133in}{1.617458in}}%
\pgfpathlineto{\pgfqpoint{4.013710in}{1.613500in}}%
\pgfpathlineto{\pgfqpoint{4.014094in}{1.616634in}}%
\pgfpathlineto{\pgfqpoint{4.015824in}{1.625923in}}%
\pgfpathlineto{\pgfqpoint{4.016016in}{1.625630in}}%
\pgfpathlineto{\pgfqpoint{4.017361in}{1.619695in}}%
\pgfpathlineto{\pgfqpoint{4.017553in}{1.621143in}}%
\pgfpathlineto{\pgfqpoint{4.017745in}{1.617643in}}%
\pgfpathlineto{\pgfqpoint{4.017938in}{1.619910in}}%
\pgfpathlineto{\pgfqpoint{4.018898in}{1.615954in}}%
\pgfpathlineto{\pgfqpoint{4.019091in}{1.616066in}}%
\pgfpathlineto{\pgfqpoint{4.020436in}{1.626942in}}%
\pgfpathlineto{\pgfqpoint{4.021012in}{1.623832in}}%
\pgfpathlineto{\pgfqpoint{4.021204in}{1.623695in}}%
\pgfpathlineto{\pgfqpoint{4.021589in}{1.626769in}}%
\pgfpathlineto{\pgfqpoint{4.021781in}{1.623459in}}%
\pgfpathlineto{\pgfqpoint{4.021973in}{1.623443in}}%
\pgfpathlineto{\pgfqpoint{4.022550in}{1.617965in}}%
\pgfpathlineto{\pgfqpoint{4.023126in}{1.619814in}}%
\pgfpathlineto{\pgfqpoint{4.023318in}{1.621594in}}%
\pgfpathlineto{\pgfqpoint{4.023510in}{1.618848in}}%
\pgfpathlineto{\pgfqpoint{4.024087in}{1.620213in}}%
\pgfpathlineto{\pgfqpoint{4.026009in}{1.600750in}}%
\pgfpathlineto{\pgfqpoint{4.026585in}{1.598076in}}%
\pgfpathlineto{\pgfqpoint{4.027354in}{1.603316in}}%
\pgfpathlineto{\pgfqpoint{4.028699in}{1.596465in}}%
\pgfpathlineto{\pgfqpoint{4.029083in}{1.599903in}}%
\pgfpathlineto{\pgfqpoint{4.029468in}{1.605739in}}%
\pgfpathlineto{\pgfqpoint{4.030236in}{1.605106in}}%
\pgfpathlineto{\pgfqpoint{4.031966in}{1.608739in}}%
\pgfpathlineto{\pgfqpoint{4.032158in}{1.607822in}}%
\pgfpathlineto{\pgfqpoint{4.032542in}{1.608764in}}%
\pgfpathlineto{\pgfqpoint{4.032734in}{1.608340in}}%
\pgfpathlineto{\pgfqpoint{4.033119in}{1.611571in}}%
\pgfpathlineto{\pgfqpoint{4.033695in}{1.606265in}}%
\pgfpathlineto{\pgfqpoint{4.035233in}{1.594085in}}%
\pgfpathlineto{\pgfqpoint{4.036001in}{1.590505in}}%
\pgfpathlineto{\pgfqpoint{4.036193in}{1.593391in}}%
\pgfpathlineto{\pgfqpoint{4.037346in}{1.597702in}}%
\pgfpathlineto{\pgfqpoint{4.036770in}{1.592533in}}%
\pgfpathlineto{\pgfqpoint{4.037539in}{1.597289in}}%
\pgfpathlineto{\pgfqpoint{4.037923in}{1.595262in}}%
\pgfpathlineto{\pgfqpoint{4.038115in}{1.595868in}}%
\pgfpathlineto{\pgfqpoint{4.038499in}{1.592896in}}%
\pgfpathlineto{\pgfqpoint{4.039268in}{1.594015in}}%
\pgfpathlineto{\pgfqpoint{4.039845in}{1.593036in}}%
\pgfpathlineto{\pgfqpoint{4.040613in}{1.596335in}}%
\pgfpathlineto{\pgfqpoint{4.040998in}{1.597236in}}%
\pgfpathlineto{\pgfqpoint{4.041766in}{1.597034in}}%
\pgfpathlineto{\pgfqpoint{4.042343in}{1.601421in}}%
\pgfpathlineto{\pgfqpoint{4.042535in}{1.598768in}}%
\pgfpathlineto{\pgfqpoint{4.043304in}{1.602787in}}%
\pgfpathlineto{\pgfqpoint{4.043496in}{1.606333in}}%
\pgfpathlineto{\pgfqpoint{4.044265in}{1.602946in}}%
\pgfpathlineto{\pgfqpoint{4.044457in}{1.603185in}}%
\pgfpathlineto{\pgfqpoint{4.045033in}{1.609919in}}%
\pgfpathlineto{\pgfqpoint{4.045610in}{1.604640in}}%
\pgfpathlineto{\pgfqpoint{4.046186in}{1.600190in}}%
\pgfpathlineto{\pgfqpoint{4.046763in}{1.602657in}}%
\pgfpathlineto{\pgfqpoint{4.048300in}{1.607574in}}%
\pgfpathlineto{\pgfqpoint{4.049069in}{1.602544in}}%
\pgfpathlineto{\pgfqpoint{4.049645in}{1.604224in}}%
\pgfpathlineto{\pgfqpoint{4.050030in}{1.603998in}}%
\pgfpathlineto{\pgfqpoint{4.050606in}{1.607111in}}%
\pgfpathlineto{\pgfqpoint{4.051759in}{1.598773in}}%
\pgfpathlineto{\pgfqpoint{4.051951in}{1.602460in}}%
\pgfpathlineto{\pgfqpoint{4.052143in}{1.601573in}}%
\pgfpathlineto{\pgfqpoint{4.052528in}{1.604517in}}%
\pgfpathlineto{\pgfqpoint{4.052912in}{1.611010in}}%
\pgfpathlineto{\pgfqpoint{4.053489in}{1.605766in}}%
\pgfpathlineto{\pgfqpoint{4.055026in}{1.598218in}}%
\pgfpathlineto{\pgfqpoint{4.055410in}{1.599643in}}%
\pgfpathlineto{\pgfqpoint{4.056948in}{1.607952in}}%
\pgfpathlineto{\pgfqpoint{4.060599in}{1.619115in}}%
\pgfpathlineto{\pgfqpoint{4.060791in}{1.617039in}}%
\pgfpathlineto{\pgfqpoint{4.061175in}{1.620851in}}%
\pgfpathlineto{\pgfqpoint{4.061560in}{1.619968in}}%
\pgfpathlineto{\pgfqpoint{4.061752in}{1.621685in}}%
\pgfpathlineto{\pgfqpoint{4.062136in}{1.618310in}}%
\pgfpathlineto{\pgfqpoint{4.062520in}{1.619546in}}%
\pgfpathlineto{\pgfqpoint{4.063673in}{1.615213in}}%
\pgfpathlineto{\pgfqpoint{4.064058in}{1.616360in}}%
\pgfpathlineto{\pgfqpoint{4.064634in}{1.620417in}}%
\pgfpathlineto{\pgfqpoint{4.065211in}{1.617152in}}%
\pgfpathlineto{\pgfqpoint{4.066172in}{1.610182in}}%
\pgfpathlineto{\pgfqpoint{4.066748in}{1.612119in}}%
\pgfpathlineto{\pgfqpoint{4.067517in}{1.619888in}}%
\pgfpathlineto{\pgfqpoint{4.067901in}{1.618561in}}%
\pgfpathlineto{\pgfqpoint{4.070399in}{1.595647in}}%
\pgfpathlineto{\pgfqpoint{4.070592in}{1.594635in}}%
\pgfpathlineto{\pgfqpoint{4.070784in}{1.596253in}}%
\pgfpathlineto{\pgfqpoint{4.070976in}{1.595701in}}%
\pgfpathlineto{\pgfqpoint{4.071168in}{1.599202in}}%
\pgfpathlineto{\pgfqpoint{4.071937in}{1.598846in}}%
\pgfpathlineto{\pgfqpoint{4.072321in}{1.593615in}}%
\pgfpathlineto{\pgfqpoint{4.073090in}{1.597248in}}%
\pgfpathlineto{\pgfqpoint{4.073282in}{1.597464in}}%
\pgfpathlineto{\pgfqpoint{4.073474in}{1.596573in}}%
\pgfpathlineto{\pgfqpoint{4.074627in}{1.602897in}}%
\pgfpathlineto{\pgfqpoint{4.075204in}{1.603706in}}%
\pgfpathlineto{\pgfqpoint{4.075588in}{1.602427in}}%
\pgfpathlineto{\pgfqpoint{4.077125in}{1.608626in}}%
\pgfpathlineto{\pgfqpoint{4.079623in}{1.594007in}}%
\pgfpathlineto{\pgfqpoint{4.077510in}{1.610004in}}%
\pgfpathlineto{\pgfqpoint{4.079816in}{1.596467in}}%
\pgfpathlineto{\pgfqpoint{4.080008in}{1.596207in}}%
\pgfpathlineto{\pgfqpoint{4.080200in}{1.597122in}}%
\pgfpathlineto{\pgfqpoint{4.080392in}{1.597099in}}%
\pgfpathlineto{\pgfqpoint{4.080584in}{1.597350in}}%
\pgfpathlineto{\pgfqpoint{4.080776in}{1.595257in}}%
\pgfpathlineto{\pgfqpoint{4.081353in}{1.597473in}}%
\pgfpathlineto{\pgfqpoint{4.081545in}{1.600368in}}%
\pgfpathlineto{\pgfqpoint{4.081737in}{1.596836in}}%
\pgfpathlineto{\pgfqpoint{4.082314in}{1.597114in}}%
\pgfpathlineto{\pgfqpoint{4.082506in}{1.596569in}}%
\pgfpathlineto{\pgfqpoint{4.083659in}{1.605046in}}%
\pgfpathlineto{\pgfqpoint{4.083851in}{1.601101in}}%
\pgfpathlineto{\pgfqpoint{4.084043in}{1.603389in}}%
\pgfpathlineto{\pgfqpoint{4.084620in}{1.599747in}}%
\pgfpathlineto{\pgfqpoint{4.084812in}{1.599764in}}%
\pgfpathlineto{\pgfqpoint{4.085004in}{1.599642in}}%
\pgfpathlineto{\pgfqpoint{4.085196in}{1.600154in}}%
\pgfpathlineto{\pgfqpoint{4.086157in}{1.594632in}}%
\pgfpathlineto{\pgfqpoint{4.086541in}{1.595005in}}%
\pgfpathlineto{\pgfqpoint{4.086734in}{1.595424in}}%
\pgfpathlineto{\pgfqpoint{4.086926in}{1.593275in}}%
\pgfpathlineto{\pgfqpoint{4.087118in}{1.593304in}}%
\pgfpathlineto{\pgfqpoint{4.087887in}{1.586149in}}%
\pgfpathlineto{\pgfqpoint{4.088463in}{1.587760in}}%
\pgfpathlineto{\pgfqpoint{4.089808in}{1.593993in}}%
\pgfpathlineto{\pgfqpoint{4.088848in}{1.586179in}}%
\pgfpathlineto{\pgfqpoint{4.090001in}{1.590915in}}%
\pgfpathlineto{\pgfqpoint{4.091346in}{1.583830in}}%
\pgfpathlineto{\pgfqpoint{4.091538in}{1.584472in}}%
\pgfpathlineto{\pgfqpoint{4.091730in}{1.585556in}}%
\pgfpathlineto{\pgfqpoint{4.092307in}{1.584162in}}%
\pgfpathlineto{\pgfqpoint{4.093652in}{1.573608in}}%
\pgfpathlineto{\pgfqpoint{4.094228in}{1.574317in}}%
\pgfpathlineto{\pgfqpoint{4.095958in}{1.586593in}}%
\pgfpathlineto{\pgfqpoint{4.096726in}{1.583856in}}%
\pgfpathlineto{\pgfqpoint{4.097111in}{1.581753in}}%
\pgfpathlineto{\pgfqpoint{4.097495in}{1.585687in}}%
\pgfpathlineto{\pgfqpoint{4.098072in}{1.588673in}}%
\pgfpathlineto{\pgfqpoint{4.098840in}{1.594181in}}%
\pgfpathlineto{\pgfqpoint{4.099225in}{1.591148in}}%
\pgfpathlineto{\pgfqpoint{4.099417in}{1.590485in}}%
\pgfpathlineto{\pgfqpoint{4.099609in}{1.592417in}}%
\pgfpathlineto{\pgfqpoint{4.099801in}{1.595765in}}%
\pgfpathlineto{\pgfqpoint{4.100570in}{1.591782in}}%
\pgfpathlineto{\pgfqpoint{4.101338in}{1.593846in}}%
\pgfpathlineto{\pgfqpoint{4.101531in}{1.590545in}}%
\pgfpathlineto{\pgfqpoint{4.102299in}{1.595712in}}%
\pgfpathlineto{\pgfqpoint{4.102684in}{1.596753in}}%
\pgfpathlineto{\pgfqpoint{4.103068in}{1.594054in}}%
\pgfpathlineto{\pgfqpoint{4.103452in}{1.589198in}}%
\pgfpathlineto{\pgfqpoint{4.103837in}{1.595080in}}%
\pgfpathlineto{\pgfqpoint{4.104221in}{1.592108in}}%
\pgfpathlineto{\pgfqpoint{4.104413in}{1.592337in}}%
\pgfpathlineto{\pgfqpoint{4.106911in}{1.578080in}}%
\pgfpathlineto{\pgfqpoint{4.107103in}{1.578421in}}%
\pgfpathlineto{\pgfqpoint{4.107680in}{1.577606in}}%
\pgfpathlineto{\pgfqpoint{4.108833in}{1.584550in}}%
\pgfpathlineto{\pgfqpoint{4.109025in}{1.583536in}}%
\pgfpathlineto{\pgfqpoint{4.109409in}{1.581542in}}%
\pgfpathlineto{\pgfqpoint{4.110178in}{1.585380in}}%
\pgfpathlineto{\pgfqpoint{4.112100in}{1.575714in}}%
\pgfpathlineto{\pgfqpoint{4.112292in}{1.576817in}}%
\pgfpathlineto{\pgfqpoint{4.112868in}{1.580000in}}%
\pgfpathlineto{\pgfqpoint{4.113253in}{1.577639in}}%
\pgfpathlineto{\pgfqpoint{4.114790in}{1.569828in}}%
\pgfpathlineto{\pgfqpoint{4.114982in}{1.571006in}}%
\pgfpathlineto{\pgfqpoint{4.115175in}{1.569717in}}%
\pgfpathlineto{\pgfqpoint{4.115559in}{1.570315in}}%
\pgfpathlineto{\pgfqpoint{4.115943in}{1.567494in}}%
\pgfpathlineto{\pgfqpoint{4.116328in}{1.570688in}}%
\pgfpathlineto{\pgfqpoint{4.117673in}{1.576372in}}%
\pgfpathlineto{\pgfqpoint{4.118826in}{1.571387in}}%
\pgfpathlineto{\pgfqpoint{4.119018in}{1.572262in}}%
\pgfpathlineto{\pgfqpoint{4.119402in}{1.573901in}}%
\pgfpathlineto{\pgfqpoint{4.122285in}{1.589148in}}%
\pgfpathlineto{\pgfqpoint{4.123246in}{1.582098in}}%
\pgfpathlineto{\pgfqpoint{4.123630in}{1.584853in}}%
\pgfpathlineto{\pgfqpoint{4.124399in}{1.590729in}}%
\pgfpathlineto{\pgfqpoint{4.124783in}{1.588133in}}%
\pgfpathlineto{\pgfqpoint{4.126512in}{1.579192in}}%
\pgfpathlineto{\pgfqpoint{4.126897in}{1.583335in}}%
\pgfpathlineto{\pgfqpoint{4.127665in}{1.582574in}}%
\pgfpathlineto{\pgfqpoint{4.127858in}{1.581365in}}%
\pgfpathlineto{\pgfqpoint{4.128242in}{1.584008in}}%
\pgfpathlineto{\pgfqpoint{4.128434in}{1.584000in}}%
\pgfpathlineto{\pgfqpoint{4.129011in}{1.588365in}}%
\pgfpathlineto{\pgfqpoint{4.129203in}{1.589939in}}%
\pgfpathlineto{\pgfqpoint{4.129587in}{1.586385in}}%
\pgfpathlineto{\pgfqpoint{4.130164in}{1.589470in}}%
\pgfpathlineto{\pgfqpoint{4.131317in}{1.582019in}}%
\pgfpathlineto{\pgfqpoint{4.131701in}{1.585000in}}%
\pgfpathlineto{\pgfqpoint{4.131893in}{1.583540in}}%
\pgfpathlineto{\pgfqpoint{4.132085in}{1.588193in}}%
\pgfpathlineto{\pgfqpoint{4.132662in}{1.585221in}}%
\pgfpathlineto{\pgfqpoint{4.133623in}{1.585080in}}%
\pgfpathlineto{\pgfqpoint{4.134199in}{1.590334in}}%
\pgfpathlineto{\pgfqpoint{4.134776in}{1.584941in}}%
\pgfpathlineto{\pgfqpoint{4.135352in}{1.589423in}}%
\pgfpathlineto{\pgfqpoint{4.136121in}{1.592844in}}%
\pgfpathlineto{\pgfqpoint{4.136889in}{1.592059in}}%
\pgfpathlineto{\pgfqpoint{4.137850in}{1.589891in}}%
\pgfpathlineto{\pgfqpoint{4.138043in}{1.591666in}}%
\pgfpathlineto{\pgfqpoint{4.138811in}{1.592828in}}%
\pgfpathlineto{\pgfqpoint{4.138619in}{1.591295in}}%
\pgfpathlineto{\pgfqpoint{4.139003in}{1.592609in}}%
\pgfpathlineto{\pgfqpoint{4.140156in}{1.582871in}}%
\pgfpathlineto{\pgfqpoint{4.140349in}{1.583370in}}%
\pgfpathlineto{\pgfqpoint{4.141694in}{1.592422in}}%
\pgfpathlineto{\pgfqpoint{4.142078in}{1.591498in}}%
\pgfpathlineto{\pgfqpoint{4.142847in}{1.584646in}}%
\pgfpathlineto{\pgfqpoint{4.143039in}{1.588475in}}%
\pgfpathlineto{\pgfqpoint{4.143231in}{1.592281in}}%
\pgfpathlineto{\pgfqpoint{4.144192in}{1.589483in}}%
\pgfpathlineto{\pgfqpoint{4.144576in}{1.590654in}}%
\pgfpathlineto{\pgfqpoint{4.145729in}{1.583334in}}%
\pgfpathlineto{\pgfqpoint{4.146306in}{1.584387in}}%
\pgfpathlineto{\pgfqpoint{4.146498in}{1.583369in}}%
\pgfpathlineto{\pgfqpoint{4.147843in}{1.570511in}}%
\pgfpathlineto{\pgfqpoint{4.148420in}{1.572144in}}%
\pgfpathlineto{\pgfqpoint{4.149380in}{1.583471in}}%
\pgfpathlineto{\pgfqpoint{4.149957in}{1.580683in}}%
\pgfpathlineto{\pgfqpoint{4.150726in}{1.586345in}}%
\pgfpathlineto{\pgfqpoint{4.151110in}{1.582104in}}%
\pgfpathlineto{\pgfqpoint{4.151879in}{1.578037in}}%
\pgfpathlineto{\pgfqpoint{4.152263in}{1.580352in}}%
\pgfpathlineto{\pgfqpoint{4.153224in}{1.580168in}}%
\pgfpathlineto{\pgfqpoint{4.153992in}{1.587810in}}%
\pgfpathlineto{\pgfqpoint{4.155530in}{1.575194in}}%
\pgfpathlineto{\pgfqpoint{4.156875in}{1.579773in}}%
\pgfpathlineto{\pgfqpoint{4.156106in}{1.573327in}}%
\pgfpathlineto{\pgfqpoint{4.157067in}{1.578717in}}%
\pgfpathlineto{\pgfqpoint{4.159757in}{1.561631in}}%
\pgfpathlineto{\pgfqpoint{4.160142in}{1.562430in}}%
\pgfpathlineto{\pgfqpoint{4.161487in}{1.556634in}}%
\pgfpathlineto{\pgfqpoint{4.162640in}{1.547512in}}%
\pgfpathlineto{\pgfqpoint{4.163024in}{1.548790in}}%
\pgfpathlineto{\pgfqpoint{4.163217in}{1.550117in}}%
\pgfpathlineto{\pgfqpoint{4.163985in}{1.548345in}}%
\pgfpathlineto{\pgfqpoint{4.164177in}{1.548377in}}%
\pgfpathlineto{\pgfqpoint{4.165138in}{1.558447in}}%
\pgfpathlineto{\pgfqpoint{4.165523in}{1.552923in}}%
\pgfpathlineto{\pgfqpoint{4.166676in}{1.548942in}}%
\pgfpathlineto{\pgfqpoint{4.165907in}{1.555540in}}%
\pgfpathlineto{\pgfqpoint{4.166868in}{1.549680in}}%
\pgfpathlineto{\pgfqpoint{4.167252in}{1.548838in}}%
\pgfpathlineto{\pgfqpoint{4.167636in}{1.549770in}}%
\pgfpathlineto{\pgfqpoint{4.168213in}{1.553020in}}%
\pgfpathlineto{\pgfqpoint{4.168597in}{1.552157in}}%
\pgfpathlineto{\pgfqpoint{4.168789in}{1.546457in}}%
\pgfpathlineto{\pgfqpoint{4.169558in}{1.550308in}}%
\pgfpathlineto{\pgfqpoint{4.171480in}{1.560165in}}%
\pgfpathlineto{\pgfqpoint{4.171672in}{1.558965in}}%
\pgfpathlineto{\pgfqpoint{4.173209in}{1.550799in}}%
\pgfpathlineto{\pgfqpoint{4.174939in}{1.564042in}}%
\pgfpathlineto{\pgfqpoint{4.175515in}{1.560314in}}%
\pgfpathlineto{\pgfqpoint{4.175900in}{1.561508in}}%
\pgfpathlineto{\pgfqpoint{4.177053in}{1.566301in}}%
\pgfpathlineto{\pgfqpoint{4.177245in}{1.564251in}}%
\pgfpathlineto{\pgfqpoint{4.177821in}{1.563253in}}%
\pgfpathlineto{\pgfqpoint{4.178206in}{1.566154in}}%
\pgfpathlineto{\pgfqpoint{4.179166in}{1.562000in}}%
\pgfpathlineto{\pgfqpoint{4.178782in}{1.569184in}}%
\pgfpathlineto{\pgfqpoint{4.179359in}{1.562017in}}%
\pgfpathlineto{\pgfqpoint{4.179551in}{1.563311in}}%
\pgfpathlineto{\pgfqpoint{4.179935in}{1.559715in}}%
\pgfpathlineto{\pgfqpoint{4.180127in}{1.559431in}}%
\pgfpathlineto{\pgfqpoint{4.180319in}{1.560227in}}%
\pgfpathlineto{\pgfqpoint{4.180512in}{1.559705in}}%
\pgfpathlineto{\pgfqpoint{4.182625in}{1.572343in}}%
\pgfpathlineto{\pgfqpoint{4.183586in}{1.564514in}}%
\pgfpathlineto{\pgfqpoint{4.184163in}{1.566029in}}%
\pgfpathlineto{\pgfqpoint{4.185124in}{1.577340in}}%
\pgfpathlineto{\pgfqpoint{4.185892in}{1.573678in}}%
\pgfpathlineto{\pgfqpoint{4.186469in}{1.574111in}}%
\pgfpathlineto{\pgfqpoint{4.187045in}{1.568008in}}%
\pgfpathlineto{\pgfqpoint{4.187238in}{1.569794in}}%
\pgfpathlineto{\pgfqpoint{4.187622in}{1.564699in}}%
\pgfpathlineto{\pgfqpoint{4.187814in}{1.565622in}}%
\pgfpathlineto{\pgfqpoint{4.188198in}{1.567500in}}%
\pgfpathlineto{\pgfqpoint{4.188391in}{1.567256in}}%
\pgfpathlineto{\pgfqpoint{4.188967in}{1.573898in}}%
\pgfpathlineto{\pgfqpoint{4.189544in}{1.570240in}}%
\pgfpathlineto{\pgfqpoint{4.189736in}{1.569396in}}%
\pgfpathlineto{\pgfqpoint{4.189928in}{1.571396in}}%
\pgfpathlineto{\pgfqpoint{4.190504in}{1.574178in}}%
\pgfpathlineto{\pgfqpoint{4.190889in}{1.570827in}}%
\pgfpathlineto{\pgfqpoint{4.191081in}{1.572820in}}%
\pgfpathlineto{\pgfqpoint{4.191850in}{1.568952in}}%
\pgfpathlineto{\pgfqpoint{4.192042in}{1.570461in}}%
\pgfpathlineto{\pgfqpoint{4.192234in}{1.572506in}}%
\pgfpathlineto{\pgfqpoint{4.193003in}{1.569614in}}%
\pgfpathlineto{\pgfqpoint{4.193195in}{1.571128in}}%
\pgfpathlineto{\pgfqpoint{4.193963in}{1.566904in}}%
\pgfpathlineto{\pgfqpoint{4.194348in}{1.569667in}}%
\pgfpathlineto{\pgfqpoint{4.194540in}{1.571280in}}%
\pgfpathlineto{\pgfqpoint{4.194924in}{1.570382in}}%
\pgfpathlineto{\pgfqpoint{4.195309in}{1.563387in}}%
\pgfpathlineto{\pgfqpoint{4.196077in}{1.566593in}}%
\pgfpathlineto{\pgfqpoint{4.196269in}{1.565452in}}%
\pgfpathlineto{\pgfqpoint{4.196654in}{1.568784in}}%
\pgfpathlineto{\pgfqpoint{4.197038in}{1.566715in}}%
\pgfpathlineto{\pgfqpoint{4.197230in}{1.568542in}}%
\pgfpathlineto{\pgfqpoint{4.197807in}{1.566235in}}%
\pgfpathlineto{\pgfqpoint{4.198191in}{1.560969in}}%
\pgfpathlineto{\pgfqpoint{4.199152in}{1.561237in}}%
\pgfpathlineto{\pgfqpoint{4.199921in}{1.565040in}}%
\pgfpathlineto{\pgfqpoint{4.199536in}{1.560758in}}%
\pgfpathlineto{\pgfqpoint{4.200497in}{1.563780in}}%
\pgfpathlineto{\pgfqpoint{4.200689in}{1.563852in}}%
\pgfpathlineto{\pgfqpoint{4.202419in}{1.552832in}}%
\pgfpathlineto{\pgfqpoint{4.202611in}{1.556251in}}%
\pgfpathlineto{\pgfqpoint{4.202803in}{1.555761in}}%
\pgfpathlineto{\pgfqpoint{4.203187in}{1.557513in}}%
\pgfpathlineto{\pgfqpoint{4.203380in}{1.559864in}}%
\pgfpathlineto{\pgfqpoint{4.204340in}{1.558647in}}%
\pgfpathlineto{\pgfqpoint{4.204533in}{1.558896in}}%
\pgfpathlineto{\pgfqpoint{4.204725in}{1.558163in}}%
\pgfpathlineto{\pgfqpoint{4.204917in}{1.558104in}}%
\pgfpathlineto{\pgfqpoint{4.207031in}{1.546412in}}%
\pgfpathlineto{\pgfqpoint{4.207415in}{1.547083in}}%
\pgfpathlineto{\pgfqpoint{4.210682in}{1.566514in}}%
\pgfpathlineto{\pgfqpoint{4.213757in}{1.553056in}}%
\pgfpathlineto{\pgfqpoint{4.211451in}{1.567768in}}%
\pgfpathlineto{\pgfqpoint{4.213949in}{1.555002in}}%
\pgfpathlineto{\pgfqpoint{4.214333in}{1.556434in}}%
\pgfpathlineto{\pgfqpoint{4.215871in}{1.546618in}}%
\pgfpathlineto{\pgfqpoint{4.216063in}{1.546854in}}%
\pgfpathlineto{\pgfqpoint{4.216255in}{1.549123in}}%
\pgfpathlineto{\pgfqpoint{4.216639in}{1.545537in}}%
\pgfpathlineto{\pgfqpoint{4.217024in}{1.545978in}}%
\pgfpathlineto{\pgfqpoint{4.217216in}{1.547086in}}%
\pgfpathlineto{\pgfqpoint{4.217600in}{1.544029in}}%
\pgfpathlineto{\pgfqpoint{4.218177in}{1.540829in}}%
\pgfpathlineto{\pgfqpoint{4.218753in}{1.542425in}}%
\pgfpathlineto{\pgfqpoint{4.219330in}{1.548119in}}%
\pgfpathlineto{\pgfqpoint{4.219906in}{1.547345in}}%
\pgfpathlineto{\pgfqpoint{4.220098in}{1.545456in}}%
\pgfpathlineto{\pgfqpoint{4.220483in}{1.549546in}}%
\pgfpathlineto{\pgfqpoint{4.220675in}{1.548401in}}%
\pgfpathlineto{\pgfqpoint{4.223365in}{1.563428in}}%
\pgfpathlineto{\pgfqpoint{4.223557in}{1.563578in}}%
\pgfpathlineto{\pgfqpoint{4.224710in}{1.558083in}}%
\pgfpathlineto{\pgfqpoint{4.225095in}{1.560322in}}%
\pgfpathlineto{\pgfqpoint{4.225479in}{1.555150in}}%
\pgfpathlineto{\pgfqpoint{4.225671in}{1.554919in}}%
\pgfpathlineto{\pgfqpoint{4.226824in}{1.559193in}}%
\pgfpathlineto{\pgfqpoint{4.226055in}{1.554082in}}%
\pgfpathlineto{\pgfqpoint{4.227016in}{1.559040in}}%
\pgfpathlineto{\pgfqpoint{4.227593in}{1.555779in}}%
\pgfpathlineto{\pgfqpoint{4.227785in}{1.557354in}}%
\pgfpathlineto{\pgfqpoint{4.228361in}{1.560914in}}%
\pgfpathlineto{\pgfqpoint{4.228746in}{1.557440in}}%
\pgfpathlineto{\pgfqpoint{4.229130in}{1.556612in}}%
\pgfpathlineto{\pgfqpoint{4.229322in}{1.558383in}}%
\pgfpathlineto{\pgfqpoint{4.230283in}{1.563300in}}%
\pgfpathlineto{\pgfqpoint{4.229707in}{1.555829in}}%
\pgfpathlineto{\pgfqpoint{4.230860in}{1.561307in}}%
\pgfpathlineto{\pgfqpoint{4.231052in}{1.558874in}}%
\pgfpathlineto{\pgfqpoint{4.232013in}{1.560733in}}%
\pgfpathlineto{\pgfqpoint{4.235279in}{1.544330in}}%
\pgfpathlineto{\pgfqpoint{4.236625in}{1.549541in}}%
\pgfpathlineto{\pgfqpoint{4.237009in}{1.545486in}}%
\pgfpathlineto{\pgfqpoint{4.237778in}{1.547260in}}%
\pgfpathlineto{\pgfqpoint{4.238354in}{1.553745in}}%
\pgfpathlineto{\pgfqpoint{4.238739in}{1.551847in}}%
\pgfpathlineto{\pgfqpoint{4.240084in}{1.543527in}}%
\pgfpathlineto{\pgfqpoint{4.240660in}{1.542672in}}%
\pgfpathlineto{\pgfqpoint{4.244504in}{1.566742in}}%
\pgfpathlineto{\pgfqpoint{4.244888in}{1.568381in}}%
\pgfpathlineto{\pgfqpoint{4.245080in}{1.568048in}}%
\pgfpathlineto{\pgfqpoint{4.246617in}{1.573040in}}%
\pgfpathlineto{\pgfqpoint{4.247002in}{1.572430in}}%
\pgfpathlineto{\pgfqpoint{4.247578in}{1.574568in}}%
\pgfpathlineto{\pgfqpoint{4.248347in}{1.570784in}}%
\pgfpathlineto{\pgfqpoint{4.247963in}{1.575200in}}%
\pgfpathlineto{\pgfqpoint{4.248539in}{1.573541in}}%
\pgfpathlineto{\pgfqpoint{4.251806in}{1.581904in}}%
\pgfpathlineto{\pgfqpoint{4.251998in}{1.577483in}}%
\pgfpathlineto{\pgfqpoint{4.252959in}{1.579242in}}%
\pgfpathlineto{\pgfqpoint{4.253151in}{1.578720in}}%
\pgfpathlineto{\pgfqpoint{4.253343in}{1.581148in}}%
\pgfpathlineto{\pgfqpoint{4.253535in}{1.579845in}}%
\pgfpathlineto{\pgfqpoint{4.254496in}{1.583354in}}%
\pgfpathlineto{\pgfqpoint{4.254688in}{1.580651in}}%
\pgfpathlineto{\pgfqpoint{4.256994in}{1.588847in}}%
\pgfpathlineto{\pgfqpoint{4.257187in}{1.587132in}}%
\pgfpathlineto{\pgfqpoint{4.257379in}{1.586627in}}%
\pgfpathlineto{\pgfqpoint{4.257763in}{1.587993in}}%
\pgfpathlineto{\pgfqpoint{4.258147in}{1.587469in}}%
\pgfpathlineto{\pgfqpoint{4.258340in}{1.587564in}}%
\pgfpathlineto{\pgfqpoint{4.258916in}{1.588419in}}%
\pgfpathlineto{\pgfqpoint{4.260069in}{1.578380in}}%
\pgfpathlineto{\pgfqpoint{4.261414in}{1.586212in}}%
\pgfpathlineto{\pgfqpoint{4.261607in}{1.584065in}}%
\pgfpathlineto{\pgfqpoint{4.261991in}{1.590188in}}%
\pgfpathlineto{\pgfqpoint{4.263528in}{1.596747in}}%
\pgfpathlineto{\pgfqpoint{4.263720in}{1.596161in}}%
\pgfpathlineto{\pgfqpoint{4.264873in}{1.592870in}}%
\pgfpathlineto{\pgfqpoint{4.264105in}{1.597251in}}%
\pgfpathlineto{\pgfqpoint{4.265258in}{1.594689in}}%
\pgfpathlineto{\pgfqpoint{4.266411in}{1.602385in}}%
\pgfpathlineto{\pgfqpoint{4.266987in}{1.601655in}}%
\pgfpathlineto{\pgfqpoint{4.267756in}{1.596072in}}%
\pgfpathlineto{\pgfqpoint{4.268140in}{1.597279in}}%
\pgfpathlineto{\pgfqpoint{4.268909in}{1.596916in}}%
\pgfpathlineto{\pgfqpoint{4.271023in}{1.586175in}}%
\pgfpathlineto{\pgfqpoint{4.271215in}{1.586687in}}%
\pgfpathlineto{\pgfqpoint{4.272176in}{1.589750in}}%
\pgfpathlineto{\pgfqpoint{4.271984in}{1.586309in}}%
\pgfpathlineto{\pgfqpoint{4.272368in}{1.587715in}}%
\pgfpathlineto{\pgfqpoint{4.272752in}{1.586277in}}%
\pgfpathlineto{\pgfqpoint{4.273521in}{1.589601in}}%
\pgfpathlineto{\pgfqpoint{4.273713in}{1.588123in}}%
\pgfpathlineto{\pgfqpoint{4.274290in}{1.591343in}}%
\pgfpathlineto{\pgfqpoint{4.274866in}{1.593291in}}%
\pgfpathlineto{\pgfqpoint{4.274674in}{1.590416in}}%
\pgfpathlineto{\pgfqpoint{4.275058in}{1.591809in}}%
\pgfpathlineto{\pgfqpoint{4.276211in}{1.588637in}}%
\pgfpathlineto{\pgfqpoint{4.276596in}{1.588478in}}%
\pgfpathlineto{\pgfqpoint{4.279478in}{1.602134in}}%
\pgfpathlineto{\pgfqpoint{4.280631in}{1.597362in}}%
\pgfpathlineto{\pgfqpoint{4.280823in}{1.598773in}}%
\pgfpathlineto{\pgfqpoint{4.281400in}{1.597253in}}%
\pgfpathlineto{\pgfqpoint{4.281208in}{1.601373in}}%
\pgfpathlineto{\pgfqpoint{4.281784in}{1.598201in}}%
\pgfpathlineto{\pgfqpoint{4.283514in}{1.606325in}}%
\pgfpathlineto{\pgfqpoint{4.283898in}{1.612306in}}%
\pgfpathlineto{\pgfqpoint{4.284475in}{1.610297in}}%
\pgfpathlineto{\pgfqpoint{4.286012in}{1.596178in}}%
\pgfpathlineto{\pgfqpoint{4.286781in}{1.598554in}}%
\pgfpathlineto{\pgfqpoint{4.286973in}{1.596276in}}%
\pgfpathlineto{\pgfqpoint{4.287549in}{1.592768in}}%
\pgfpathlineto{\pgfqpoint{4.288318in}{1.593176in}}%
\pgfpathlineto{\pgfqpoint{4.288702in}{1.594070in}}%
\pgfpathlineto{\pgfqpoint{4.288894in}{1.591475in}}%
\pgfpathlineto{\pgfqpoint{4.289471in}{1.597266in}}%
\pgfpathlineto{\pgfqpoint{4.289663in}{1.595082in}}%
\pgfpathlineto{\pgfqpoint{4.289855in}{1.594927in}}%
\pgfpathlineto{\pgfqpoint{4.290047in}{1.595604in}}%
\pgfpathlineto{\pgfqpoint{4.290624in}{1.596314in}}%
\pgfpathlineto{\pgfqpoint{4.290816in}{1.595655in}}%
\pgfpathlineto{\pgfqpoint{4.291200in}{1.591236in}}%
\pgfpathlineto{\pgfqpoint{4.291777in}{1.597692in}}%
\pgfpathlineto{\pgfqpoint{4.292161in}{1.596755in}}%
\pgfpathlineto{\pgfqpoint{4.292546in}{1.600097in}}%
\pgfpathlineto{\pgfqpoint{4.293891in}{1.588780in}}%
\pgfpathlineto{\pgfqpoint{4.294083in}{1.590595in}}%
\pgfpathlineto{\pgfqpoint{4.295428in}{1.581629in}}%
\pgfpathlineto{\pgfqpoint{4.295812in}{1.586029in}}%
\pgfpathlineto{\pgfqpoint{4.298118in}{1.604029in}}%
\pgfpathlineto{\pgfqpoint{4.298503in}{1.605243in}}%
\pgfpathlineto{\pgfqpoint{4.299656in}{1.598145in}}%
\pgfpathlineto{\pgfqpoint{4.300232in}{1.604792in}}%
\pgfpathlineto{\pgfqpoint{4.300809in}{1.603212in}}%
\pgfpathlineto{\pgfqpoint{4.302538in}{1.596395in}}%
\pgfpathlineto{\pgfqpoint{4.304268in}{1.605323in}}%
\pgfpathlineto{\pgfqpoint{4.305421in}{1.599139in}}%
\pgfpathlineto{\pgfqpoint{4.305613in}{1.600702in}}%
\pgfpathlineto{\pgfqpoint{4.305997in}{1.605992in}}%
\pgfpathlineto{\pgfqpoint{4.306574in}{1.605063in}}%
\pgfpathlineto{\pgfqpoint{4.306766in}{1.598950in}}%
\pgfpathlineto{\pgfqpoint{4.307727in}{1.602314in}}%
\pgfpathlineto{\pgfqpoint{4.308111in}{1.602879in}}%
\pgfpathlineto{\pgfqpoint{4.308303in}{1.601480in}}%
\pgfpathlineto{\pgfqpoint{4.309841in}{1.611681in}}%
\pgfpathlineto{\pgfqpoint{4.310994in}{1.601222in}}%
\pgfpathlineto{\pgfqpoint{4.311570in}{1.601666in}}%
\pgfpathlineto{\pgfqpoint{4.311955in}{1.604157in}}%
\pgfpathlineto{\pgfqpoint{4.312339in}{1.599414in}}%
\pgfpathlineto{\pgfqpoint{4.312531in}{1.599453in}}%
\pgfpathlineto{\pgfqpoint{4.312915in}{1.605529in}}%
\pgfpathlineto{\pgfqpoint{4.313492in}{1.602053in}}%
\pgfpathlineto{\pgfqpoint{4.314261in}{1.595206in}}%
\pgfpathlineto{\pgfqpoint{4.314837in}{1.596382in}}%
\pgfpathlineto{\pgfqpoint{4.315606in}{1.598688in}}%
\pgfpathlineto{\pgfqpoint{4.315798in}{1.596191in}}%
\pgfpathlineto{\pgfqpoint{4.315990in}{1.593853in}}%
\pgfpathlineto{\pgfqpoint{4.316567in}{1.598582in}}%
\pgfpathlineto{\pgfqpoint{4.317335in}{1.596336in}}%
\pgfpathlineto{\pgfqpoint{4.317912in}{1.601136in}}%
\pgfpathlineto{\pgfqpoint{4.318296in}{1.601327in}}%
\pgfpathlineto{\pgfqpoint{4.319257in}{1.596562in}}%
\pgfpathlineto{\pgfqpoint{4.320026in}{1.601064in}}%
\pgfpathlineto{\pgfqpoint{4.320602in}{1.597358in}}%
\pgfpathlineto{\pgfqpoint{4.324253in}{1.576163in}}%
\pgfpathlineto{\pgfqpoint{4.325214in}{1.579230in}}%
\pgfpathlineto{\pgfqpoint{4.326559in}{1.582586in}}%
\pgfpathlineto{\pgfqpoint{4.328673in}{1.572689in}}%
\pgfpathlineto{\pgfqpoint{4.329250in}{1.574920in}}%
\pgfpathlineto{\pgfqpoint{4.330210in}{1.586049in}}%
\pgfpathlineto{\pgfqpoint{4.330595in}{1.585200in}}%
\pgfpathlineto{\pgfqpoint{4.330979in}{1.578902in}}%
\pgfpathlineto{\pgfqpoint{4.331940in}{1.580976in}}%
\pgfpathlineto{\pgfqpoint{4.332324in}{1.584341in}}%
\pgfpathlineto{\pgfqpoint{4.332901in}{1.580556in}}%
\pgfpathlineto{\pgfqpoint{4.333093in}{1.580324in}}%
\pgfpathlineto{\pgfqpoint{4.335015in}{1.590738in}}%
\pgfpathlineto{\pgfqpoint{4.335783in}{1.593026in}}%
\pgfpathlineto{\pgfqpoint{4.336168in}{1.592318in}}%
\pgfpathlineto{\pgfqpoint{4.337513in}{1.588229in}}%
\pgfpathlineto{\pgfqpoint{4.340203in}{1.601298in}}%
\pgfpathlineto{\pgfqpoint{4.340780in}{1.597588in}}%
\pgfpathlineto{\pgfqpoint{4.341741in}{1.593274in}}%
\pgfpathlineto{\pgfqpoint{4.342317in}{1.593764in}}%
\pgfpathlineto{\pgfqpoint{4.342509in}{1.593967in}}%
\pgfpathlineto{\pgfqpoint{4.343662in}{1.587623in}}%
\pgfpathlineto{\pgfqpoint{4.343854in}{1.589811in}}%
\pgfpathlineto{\pgfqpoint{4.344623in}{1.593974in}}%
\pgfpathlineto{\pgfqpoint{4.345007in}{1.593257in}}%
\pgfpathlineto{\pgfqpoint{4.346545in}{1.582673in}}%
\pgfpathlineto{\pgfqpoint{4.346737in}{1.584089in}}%
\pgfpathlineto{\pgfqpoint{4.347121in}{1.581018in}}%
\pgfpathlineto{\pgfqpoint{4.347313in}{1.582063in}}%
\pgfpathlineto{\pgfqpoint{4.347506in}{1.586311in}}%
\pgfpathlineto{\pgfqpoint{4.348274in}{1.580442in}}%
\pgfpathlineto{\pgfqpoint{4.350580in}{1.595752in}}%
\pgfpathlineto{\pgfqpoint{4.352310in}{1.581553in}}%
\pgfpathlineto{\pgfqpoint{4.352694in}{1.582852in}}%
\pgfpathlineto{\pgfqpoint{4.352886in}{1.580500in}}%
\pgfpathlineto{\pgfqpoint{4.353271in}{1.582150in}}%
\pgfpathlineto{\pgfqpoint{4.353463in}{1.580658in}}%
\pgfpathlineto{\pgfqpoint{4.353847in}{1.582300in}}%
\pgfpathlineto{\pgfqpoint{4.355192in}{1.590799in}}%
\pgfpathlineto{\pgfqpoint{4.355961in}{1.585830in}}%
\pgfpathlineto{\pgfqpoint{4.357498in}{1.581834in}}%
\pgfpathlineto{\pgfqpoint{4.357883in}{1.583384in}}%
\pgfpathlineto{\pgfqpoint{4.359612in}{1.597095in}}%
\pgfpathlineto{\pgfqpoint{4.359804in}{1.596809in}}%
\pgfpathlineto{\pgfqpoint{4.359997in}{1.594216in}}%
\pgfpathlineto{\pgfqpoint{4.360381in}{1.601701in}}%
\pgfpathlineto{\pgfqpoint{4.360573in}{1.601200in}}%
\pgfpathlineto{\pgfqpoint{4.361726in}{1.607769in}}%
\pgfpathlineto{\pgfqpoint{4.361150in}{1.600668in}}%
\pgfpathlineto{\pgfqpoint{4.361918in}{1.605365in}}%
\pgfpathlineto{\pgfqpoint{4.362110in}{1.603098in}}%
\pgfpathlineto{\pgfqpoint{4.362687in}{1.607192in}}%
\pgfpathlineto{\pgfqpoint{4.364032in}{1.615495in}}%
\pgfpathlineto{\pgfqpoint{4.364416in}{1.617004in}}%
\pgfpathlineto{\pgfqpoint{4.364801in}{1.615393in}}%
\pgfpathlineto{\pgfqpoint{4.365762in}{1.612511in}}%
\pgfpathlineto{\pgfqpoint{4.365185in}{1.615601in}}%
\pgfpathlineto{\pgfqpoint{4.365954in}{1.614939in}}%
\pgfpathlineto{\pgfqpoint{4.366722in}{1.625066in}}%
\pgfpathlineto{\pgfqpoint{4.367299in}{1.621139in}}%
\pgfpathlineto{\pgfqpoint{4.367683in}{1.619974in}}%
\pgfpathlineto{\pgfqpoint{4.368068in}{1.620760in}}%
\pgfpathlineto{\pgfqpoint{4.368452in}{1.623936in}}%
\pgfpathlineto{\pgfqpoint{4.368836in}{1.621220in}}%
\pgfpathlineto{\pgfqpoint{4.369028in}{1.617492in}}%
\pgfpathlineto{\pgfqpoint{4.369989in}{1.619186in}}%
\pgfpathlineto{\pgfqpoint{4.370374in}{1.616952in}}%
\pgfpathlineto{\pgfqpoint{4.373064in}{1.631625in}}%
\pgfpathlineto{\pgfqpoint{4.373256in}{1.626956in}}%
\pgfpathlineto{\pgfqpoint{4.374025in}{1.629281in}}%
\pgfpathlineto{\pgfqpoint{4.374409in}{1.631784in}}%
\pgfpathlineto{\pgfqpoint{4.374793in}{1.629146in}}%
\pgfpathlineto{\pgfqpoint{4.374986in}{1.630899in}}%
\pgfpathlineto{\pgfqpoint{4.375946in}{1.628273in}}%
\pgfpathlineto{\pgfqpoint{4.377484in}{1.637314in}}%
\pgfpathlineto{\pgfqpoint{4.378637in}{1.630164in}}%
\pgfpathlineto{\pgfqpoint{4.378829in}{1.633177in}}%
\pgfpathlineto{\pgfqpoint{4.379021in}{1.634483in}}%
\pgfpathlineto{\pgfqpoint{4.379405in}{1.630276in}}%
\pgfpathlineto{\pgfqpoint{4.379790in}{1.632431in}}%
\pgfpathlineto{\pgfqpoint{4.379982in}{1.632760in}}%
\pgfpathlineto{\pgfqpoint{4.380943in}{1.623162in}}%
\pgfpathlineto{\pgfqpoint{4.381135in}{1.628308in}}%
\pgfpathlineto{\pgfqpoint{4.381904in}{1.626756in}}%
\pgfpathlineto{\pgfqpoint{4.382096in}{1.627589in}}%
\pgfpathlineto{\pgfqpoint{4.382288in}{1.631644in}}%
\pgfpathlineto{\pgfqpoint{4.383057in}{1.628133in}}%
\pgfpathlineto{\pgfqpoint{4.383633in}{1.628681in}}%
\pgfpathlineto{\pgfqpoint{4.385171in}{1.619972in}}%
\pgfpathlineto{\pgfqpoint{4.385555in}{1.615345in}}%
\pgfpathlineto{\pgfqpoint{4.385747in}{1.619609in}}%
\pgfpathlineto{\pgfqpoint{4.386708in}{1.624066in}}%
\pgfpathlineto{\pgfqpoint{4.387092in}{1.621853in}}%
\pgfpathlineto{\pgfqpoint{4.387477in}{1.622420in}}%
\pgfpathlineto{\pgfqpoint{4.388053in}{1.621410in}}%
\pgfpathlineto{\pgfqpoint{4.391320in}{1.639833in}}%
\pgfpathlineto{\pgfqpoint{4.388437in}{1.620055in}}%
\pgfpathlineto{\pgfqpoint{4.392089in}{1.635763in}}%
\pgfpathlineto{\pgfqpoint{4.392665in}{1.632397in}}%
\pgfpathlineto{\pgfqpoint{4.392857in}{1.637276in}}%
\pgfpathlineto{\pgfqpoint{4.393049in}{1.639350in}}%
\pgfpathlineto{\pgfqpoint{4.393626in}{1.636888in}}%
\pgfpathlineto{\pgfqpoint{4.394779in}{1.632002in}}%
\pgfpathlineto{\pgfqpoint{4.396124in}{1.637336in}}%
\pgfpathlineto{\pgfqpoint{4.396701in}{1.635988in}}%
\pgfpathlineto{\pgfqpoint{4.397277in}{1.636857in}}%
\pgfpathlineto{\pgfqpoint{4.397469in}{1.637681in}}%
\pgfpathlineto{\pgfqpoint{4.397661in}{1.635593in}}%
\pgfpathlineto{\pgfqpoint{4.397854in}{1.636753in}}%
\pgfpathlineto{\pgfqpoint{4.399391in}{1.627830in}}%
\pgfpathlineto{\pgfqpoint{4.399583in}{1.630988in}}%
\pgfpathlineto{\pgfqpoint{4.400352in}{1.625901in}}%
\pgfpathlineto{\pgfqpoint{4.400736in}{1.620993in}}%
\pgfpathlineto{\pgfqpoint{4.401120in}{1.627600in}}%
\pgfpathlineto{\pgfqpoint{4.401313in}{1.628897in}}%
\pgfpathlineto{\pgfqpoint{4.401697in}{1.624751in}}%
\pgfpathlineto{\pgfqpoint{4.402081in}{1.626580in}}%
\pgfpathlineto{\pgfqpoint{4.402466in}{1.624449in}}%
\pgfpathlineto{\pgfqpoint{4.402850in}{1.619502in}}%
\pgfpathlineto{\pgfqpoint{4.403426in}{1.625235in}}%
\pgfpathlineto{\pgfqpoint{4.403619in}{1.622586in}}%
\pgfpathlineto{\pgfqpoint{4.403811in}{1.622917in}}%
\pgfpathlineto{\pgfqpoint{4.404003in}{1.621926in}}%
\pgfpathlineto{\pgfqpoint{4.404195in}{1.618758in}}%
\pgfpathlineto{\pgfqpoint{4.404772in}{1.623658in}}%
\pgfpathlineto{\pgfqpoint{4.406117in}{1.633621in}}%
\pgfpathlineto{\pgfqpoint{4.406309in}{1.632361in}}%
\pgfpathlineto{\pgfqpoint{4.407078in}{1.630194in}}%
\pgfpathlineto{\pgfqpoint{4.407270in}{1.632628in}}%
\pgfpathlineto{\pgfqpoint{4.408039in}{1.632144in}}%
\pgfpathlineto{\pgfqpoint{4.408807in}{1.638803in}}%
\pgfpathlineto{\pgfqpoint{4.409192in}{1.636243in}}%
\pgfpathlineto{\pgfqpoint{4.409576in}{1.640136in}}%
\pgfpathlineto{\pgfqpoint{4.410537in}{1.638338in}}%
\pgfpathlineto{\pgfqpoint{4.410152in}{1.640980in}}%
\pgfpathlineto{\pgfqpoint{4.410729in}{1.640083in}}%
\pgfpathlineto{\pgfqpoint{4.410921in}{1.641091in}}%
\pgfpathlineto{\pgfqpoint{4.411305in}{1.638817in}}%
\pgfpathlineto{\pgfqpoint{4.412074in}{1.636368in}}%
\pgfpathlineto{\pgfqpoint{4.412458in}{1.638384in}}%
\pgfpathlineto{\pgfqpoint{4.412843in}{1.636639in}}%
\pgfpathlineto{\pgfqpoint{4.414380in}{1.641121in}}%
\pgfpathlineto{\pgfqpoint{4.414572in}{1.639942in}}%
\pgfpathlineto{\pgfqpoint{4.414957in}{1.642313in}}%
\pgfpathlineto{\pgfqpoint{4.415725in}{1.652352in}}%
\pgfpathlineto{\pgfqpoint{4.416302in}{1.650637in}}%
\pgfpathlineto{\pgfqpoint{4.417070in}{1.643298in}}%
\pgfpathlineto{\pgfqpoint{4.417455in}{1.647746in}}%
\pgfpathlineto{\pgfqpoint{4.417647in}{1.649183in}}%
\pgfpathlineto{\pgfqpoint{4.418223in}{1.645244in}}%
\pgfpathlineto{\pgfqpoint{4.418416in}{1.648293in}}%
\pgfpathlineto{\pgfqpoint{4.418800in}{1.644080in}}%
\pgfpathlineto{\pgfqpoint{4.419376in}{1.647765in}}%
\pgfpathlineto{\pgfqpoint{4.419569in}{1.648881in}}%
\pgfpathlineto{\pgfqpoint{4.419953in}{1.645177in}}%
\pgfpathlineto{\pgfqpoint{4.420145in}{1.646721in}}%
\pgfpathlineto{\pgfqpoint{4.420337in}{1.646159in}}%
\pgfpathlineto{\pgfqpoint{4.420529in}{1.647066in}}%
\pgfpathlineto{\pgfqpoint{4.421490in}{1.650541in}}%
\pgfpathlineto{\pgfqpoint{4.421875in}{1.650127in}}%
\pgfpathlineto{\pgfqpoint{4.422067in}{1.649941in}}%
\pgfpathlineto{\pgfqpoint{4.423220in}{1.644330in}}%
\pgfpathlineto{\pgfqpoint{4.423604in}{1.648068in}}%
\pgfpathlineto{\pgfqpoint{4.424373in}{1.645581in}}%
\pgfpathlineto{\pgfqpoint{4.424949in}{1.645454in}}%
\pgfpathlineto{\pgfqpoint{4.425910in}{1.656704in}}%
\pgfpathlineto{\pgfqpoint{4.426679in}{1.662843in}}%
\pgfpathlineto{\pgfqpoint{4.427063in}{1.660953in}}%
\pgfpathlineto{\pgfqpoint{4.428600in}{1.652938in}}%
\pgfpathlineto{\pgfqpoint{4.429561in}{1.657918in}}%
\pgfpathlineto{\pgfqpoint{4.429946in}{1.655892in}}%
\pgfpathlineto{\pgfqpoint{4.430138in}{1.654909in}}%
\pgfpathlineto{\pgfqpoint{4.430714in}{1.657690in}}%
\pgfpathlineto{\pgfqpoint{4.431483in}{1.656862in}}%
\pgfpathlineto{\pgfqpoint{4.432636in}{1.664238in}}%
\pgfpathlineto{\pgfqpoint{4.432828in}{1.660982in}}%
\pgfpathlineto{\pgfqpoint{4.435519in}{1.642038in}}%
\pgfpathlineto{\pgfqpoint{4.435711in}{1.643545in}}%
\pgfpathlineto{\pgfqpoint{4.435903in}{1.643190in}}%
\pgfpathlineto{\pgfqpoint{4.436864in}{1.648261in}}%
\pgfpathlineto{\pgfqpoint{4.437056in}{1.647977in}}%
\pgfpathlineto{\pgfqpoint{4.438209in}{1.642004in}}%
\pgfpathlineto{\pgfqpoint{4.438401in}{1.644097in}}%
\pgfpathlineto{\pgfqpoint{4.438785in}{1.646126in}}%
\pgfpathlineto{\pgfqpoint{4.439170in}{1.643762in}}%
\pgfpathlineto{\pgfqpoint{4.439554in}{1.645876in}}%
\pgfpathlineto{\pgfqpoint{4.440899in}{1.641198in}}%
\pgfpathlineto{\pgfqpoint{4.439938in}{1.646474in}}%
\pgfpathlineto{\pgfqpoint{4.441091in}{1.641585in}}%
\pgfpathlineto{\pgfqpoint{4.444166in}{1.626564in}}%
\pgfpathlineto{\pgfqpoint{4.444550in}{1.627536in}}%
\pgfpathlineto{\pgfqpoint{4.446472in}{1.638350in}}%
\pgfpathlineto{\pgfqpoint{4.447241in}{1.637816in}}%
\pgfpathlineto{\pgfqpoint{4.448009in}{1.634180in}}%
\pgfpathlineto{\pgfqpoint{4.448202in}{1.635904in}}%
\pgfpathlineto{\pgfqpoint{4.448394in}{1.638664in}}%
\pgfpathlineto{\pgfqpoint{4.448778in}{1.635524in}}%
\pgfpathlineto{\pgfqpoint{4.449162in}{1.636485in}}%
\pgfpathlineto{\pgfqpoint{4.450508in}{1.630545in}}%
\pgfpathlineto{\pgfqpoint{4.449739in}{1.638578in}}%
\pgfpathlineto{\pgfqpoint{4.450700in}{1.631231in}}%
\pgfpathlineto{\pgfqpoint{4.451084in}{1.634066in}}%
\pgfpathlineto{\pgfqpoint{4.451661in}{1.630650in}}%
\pgfpathlineto{\pgfqpoint{4.452237in}{1.628669in}}%
\pgfpathlineto{\pgfqpoint{4.453774in}{1.621459in}}%
\pgfpathlineto{\pgfqpoint{4.453967in}{1.622089in}}%
\pgfpathlineto{\pgfqpoint{4.455312in}{1.608608in}}%
\pgfpathlineto{\pgfqpoint{4.455504in}{1.609684in}}%
\pgfpathlineto{\pgfqpoint{4.456465in}{1.614042in}}%
\pgfpathlineto{\pgfqpoint{4.458002in}{1.621370in}}%
\pgfpathlineto{\pgfqpoint{4.458194in}{1.619779in}}%
\pgfpathlineto{\pgfqpoint{4.458579in}{1.621515in}}%
\pgfpathlineto{\pgfqpoint{4.458963in}{1.621221in}}%
\pgfpathlineto{\pgfqpoint{4.459155in}{1.621459in}}%
\pgfpathlineto{\pgfqpoint{4.459732in}{1.626855in}}%
\pgfpathlineto{\pgfqpoint{4.460308in}{1.624315in}}%
\pgfpathlineto{\pgfqpoint{4.460885in}{1.619035in}}%
\pgfpathlineto{\pgfqpoint{4.461653in}{1.620096in}}%
\pgfpathlineto{\pgfqpoint{4.462038in}{1.621404in}}%
\pgfpathlineto{\pgfqpoint{4.462230in}{1.619551in}}%
\pgfpathlineto{\pgfqpoint{4.462422in}{1.620270in}}%
\pgfpathlineto{\pgfqpoint{4.463767in}{1.605105in}}%
\pgfpathlineto{\pgfqpoint{4.464728in}{1.610420in}}%
\pgfpathlineto{\pgfqpoint{4.464920in}{1.612124in}}%
\pgfpathlineto{\pgfqpoint{4.465305in}{1.609035in}}%
\pgfpathlineto{\pgfqpoint{4.465497in}{1.611559in}}%
\pgfpathlineto{\pgfqpoint{4.466073in}{1.605683in}}%
\pgfpathlineto{\pgfqpoint{4.466650in}{1.608588in}}%
\pgfpathlineto{\pgfqpoint{4.467034in}{1.607062in}}%
\pgfpathlineto{\pgfqpoint{4.467226in}{1.608420in}}%
\pgfpathlineto{\pgfqpoint{4.467611in}{1.610099in}}%
\pgfpathlineto{\pgfqpoint{4.468187in}{1.608130in}}%
\pgfpathlineto{\pgfqpoint{4.468571in}{1.605397in}}%
\pgfpathlineto{\pgfqpoint{4.469340in}{1.607118in}}%
\pgfpathlineto{\pgfqpoint{4.469532in}{1.605902in}}%
\pgfpathlineto{\pgfqpoint{4.469724in}{1.609357in}}%
\pgfpathlineto{\pgfqpoint{4.470109in}{1.608189in}}%
\pgfpathlineto{\pgfqpoint{4.471838in}{1.613956in}}%
\pgfpathlineto{\pgfqpoint{4.472030in}{1.613903in}}%
\pgfpathlineto{\pgfqpoint{4.473376in}{1.606968in}}%
\pgfpathlineto{\pgfqpoint{4.473568in}{1.610072in}}%
\pgfpathlineto{\pgfqpoint{4.473952in}{1.610342in}}%
\pgfpathlineto{\pgfqpoint{4.475105in}{1.613622in}}%
\pgfpathlineto{\pgfqpoint{4.475297in}{1.612691in}}%
\pgfpathlineto{\pgfqpoint{4.475489in}{1.614800in}}%
\pgfpathlineto{\pgfqpoint{4.475874in}{1.614126in}}%
\pgfpathlineto{\pgfqpoint{4.476258in}{1.617422in}}%
\pgfpathlineto{\pgfqpoint{4.476835in}{1.613493in}}%
\pgfpathlineto{\pgfqpoint{4.477219in}{1.615214in}}%
\pgfpathlineto{\pgfqpoint{4.477603in}{1.613411in}}%
\pgfpathlineto{\pgfqpoint{4.477795in}{1.615584in}}%
\pgfpathlineto{\pgfqpoint{4.478372in}{1.617031in}}%
\pgfpathlineto{\pgfqpoint{4.478756in}{1.614650in}}%
\pgfpathlineto{\pgfqpoint{4.479717in}{1.607648in}}%
\pgfpathlineto{\pgfqpoint{4.479909in}{1.610305in}}%
\pgfpathlineto{\pgfqpoint{4.480294in}{1.608831in}}%
\pgfpathlineto{\pgfqpoint{4.480678in}{1.609511in}}%
\pgfpathlineto{\pgfqpoint{4.481255in}{1.614201in}}%
\pgfpathlineto{\pgfqpoint{4.481831in}{1.611067in}}%
\pgfpathlineto{\pgfqpoint{4.482023in}{1.611278in}}%
\pgfpathlineto{\pgfqpoint{4.482215in}{1.610103in}}%
\pgfpathlineto{\pgfqpoint{4.483368in}{1.605633in}}%
\pgfpathlineto{\pgfqpoint{4.484714in}{1.611151in}}%
\pgfpathlineto{\pgfqpoint{4.485098in}{1.610088in}}%
\pgfpathlineto{\pgfqpoint{4.485674in}{1.611741in}}%
\pgfpathlineto{\pgfqpoint{4.485867in}{1.612914in}}%
\pgfpathlineto{\pgfqpoint{4.486251in}{1.611171in}}%
\pgfpathlineto{\pgfqpoint{4.487020in}{1.604845in}}%
\pgfpathlineto{\pgfqpoint{4.487788in}{1.606256in}}%
\pgfpathlineto{\pgfqpoint{4.488749in}{1.602323in}}%
\pgfpathlineto{\pgfqpoint{4.488365in}{1.606989in}}%
\pgfpathlineto{\pgfqpoint{4.488941in}{1.604327in}}%
\pgfpathlineto{\pgfqpoint{4.490094in}{1.596453in}}%
\pgfpathlineto{\pgfqpoint{4.490863in}{1.605019in}}%
\pgfpathlineto{\pgfqpoint{4.491439in}{1.601564in}}%
\pgfpathlineto{\pgfqpoint{4.491632in}{1.599590in}}%
\pgfpathlineto{\pgfqpoint{4.492016in}{1.602381in}}%
\pgfpathlineto{\pgfqpoint{4.492400in}{1.600623in}}%
\pgfpathlineto{\pgfqpoint{4.492977in}{1.599937in}}%
\pgfpathlineto{\pgfqpoint{4.493938in}{1.608066in}}%
\pgfpathlineto{\pgfqpoint{4.495859in}{1.600753in}}%
\pgfpathlineto{\pgfqpoint{4.496244in}{1.608555in}}%
\pgfpathlineto{\pgfqpoint{4.497204in}{1.606522in}}%
\pgfpathlineto{\pgfqpoint{4.498165in}{1.610250in}}%
\pgfpathlineto{\pgfqpoint{4.498357in}{1.607159in}}%
\pgfpathlineto{\pgfqpoint{4.498550in}{1.607696in}}%
\pgfpathlineto{\pgfqpoint{4.498742in}{1.604414in}}%
\pgfpathlineto{\pgfqpoint{4.499318in}{1.608095in}}%
\pgfpathlineto{\pgfqpoint{4.499510in}{1.607563in}}%
\pgfpathlineto{\pgfqpoint{4.500087in}{1.611202in}}%
\pgfpathlineto{\pgfqpoint{4.500471in}{1.606037in}}%
\pgfpathlineto{\pgfqpoint{4.500663in}{1.604778in}}%
\pgfpathlineto{\pgfqpoint{4.500856in}{1.610697in}}%
\pgfpathlineto{\pgfqpoint{4.501048in}{1.608029in}}%
\pgfpathlineto{\pgfqpoint{4.501624in}{1.611097in}}%
\pgfpathlineto{\pgfqpoint{4.501816in}{1.607553in}}%
\pgfpathlineto{\pgfqpoint{4.502009in}{1.608798in}}%
\pgfpathlineto{\pgfqpoint{4.502201in}{1.605978in}}%
\pgfpathlineto{\pgfqpoint{4.502777in}{1.610345in}}%
\pgfpathlineto{\pgfqpoint{4.502969in}{1.609796in}}%
\pgfpathlineto{\pgfqpoint{4.505083in}{1.619276in}}%
\pgfpathlineto{\pgfqpoint{4.503930in}{1.609400in}}%
\pgfpathlineto{\pgfqpoint{4.505276in}{1.616972in}}%
\pgfpathlineto{\pgfqpoint{4.506429in}{1.611082in}}%
\pgfpathlineto{\pgfqpoint{4.506621in}{1.612993in}}%
\pgfpathlineto{\pgfqpoint{4.507005in}{1.608982in}}%
\pgfpathlineto{\pgfqpoint{4.507197in}{1.610398in}}%
\pgfpathlineto{\pgfqpoint{4.508350in}{1.601143in}}%
\pgfpathlineto{\pgfqpoint{4.508542in}{1.603067in}}%
\pgfpathlineto{\pgfqpoint{4.509119in}{1.606408in}}%
\pgfpathlineto{\pgfqpoint{4.508927in}{1.602615in}}%
\pgfpathlineto{\pgfqpoint{4.509311in}{1.604290in}}%
\pgfpathlineto{\pgfqpoint{4.510080in}{1.594893in}}%
\pgfpathlineto{\pgfqpoint{4.510464in}{1.596305in}}%
\pgfpathlineto{\pgfqpoint{4.510656in}{1.598162in}}%
\pgfpathlineto{\pgfqpoint{4.511425in}{1.595073in}}%
\pgfpathlineto{\pgfqpoint{4.511617in}{1.594467in}}%
\pgfpathlineto{\pgfqpoint{4.511809in}{1.597231in}}%
\pgfpathlineto{\pgfqpoint{4.512001in}{1.596315in}}%
\pgfpathlineto{\pgfqpoint{4.514307in}{1.612729in}}%
\pgfpathlineto{\pgfqpoint{4.515268in}{1.611073in}}%
\pgfpathlineto{\pgfqpoint{4.516421in}{1.602395in}}%
\pgfpathlineto{\pgfqpoint{4.517959in}{1.612135in}}%
\pgfpathlineto{\pgfqpoint{4.518343in}{1.609478in}}%
\pgfpathlineto{\pgfqpoint{4.519112in}{1.610315in}}%
\pgfpathlineto{\pgfqpoint{4.521225in}{1.620499in}}%
\pgfpathlineto{\pgfqpoint{4.521994in}{1.610883in}}%
\pgfpathlineto{\pgfqpoint{4.522763in}{1.615843in}}%
\pgfpathlineto{\pgfqpoint{4.523147in}{1.620453in}}%
\pgfpathlineto{\pgfqpoint{4.523531in}{1.613954in}}%
\pgfpathlineto{\pgfqpoint{4.523724in}{1.616455in}}%
\pgfpathlineto{\pgfqpoint{4.525261in}{1.608268in}}%
\pgfpathlineto{\pgfqpoint{4.527375in}{1.618339in}}%
\pgfpathlineto{\pgfqpoint{4.527567in}{1.615331in}}%
\pgfpathlineto{\pgfqpoint{4.528143in}{1.622832in}}%
\pgfpathlineto{\pgfqpoint{4.528336in}{1.622646in}}%
\pgfpathlineto{\pgfqpoint{4.529297in}{1.629693in}}%
\pgfpathlineto{\pgfqpoint{4.529681in}{1.628266in}}%
\pgfpathlineto{\pgfqpoint{4.530642in}{1.631484in}}%
\pgfpathlineto{\pgfqpoint{4.530834in}{1.628041in}}%
\pgfpathlineto{\pgfqpoint{4.531603in}{1.631071in}}%
\pgfpathlineto{\pgfqpoint{4.531795in}{1.630786in}}%
\pgfpathlineto{\pgfqpoint{4.532563in}{1.627398in}}%
\pgfpathlineto{\pgfqpoint{4.532756in}{1.629212in}}%
\pgfpathlineto{\pgfqpoint{4.533140in}{1.633707in}}%
\pgfpathlineto{\pgfqpoint{4.533716in}{1.630986in}}%
\pgfpathlineto{\pgfqpoint{4.535638in}{1.619400in}}%
\pgfpathlineto{\pgfqpoint{4.536599in}{1.626303in}}%
\pgfpathlineto{\pgfqpoint{4.537368in}{1.625741in}}%
\pgfpathlineto{\pgfqpoint{4.537560in}{1.623840in}}%
\pgfpathlineto{\pgfqpoint{4.537752in}{1.628004in}}%
\pgfpathlineto{\pgfqpoint{4.538136in}{1.627821in}}%
\pgfpathlineto{\pgfqpoint{4.538328in}{1.628043in}}%
\pgfpathlineto{\pgfqpoint{4.538713in}{1.624290in}}%
\pgfpathlineto{\pgfqpoint{4.539481in}{1.625290in}}%
\pgfpathlineto{\pgfqpoint{4.540058in}{1.627130in}}%
\pgfpathlineto{\pgfqpoint{4.540250in}{1.626680in}}%
\pgfpathlineto{\pgfqpoint{4.541980in}{1.634141in}}%
\pgfpathlineto{\pgfqpoint{4.542556in}{1.631468in}}%
\pgfpathlineto{\pgfqpoint{4.542940in}{1.626481in}}%
\pgfpathlineto{\pgfqpoint{4.543517in}{1.630640in}}%
\pgfpathlineto{\pgfqpoint{4.544478in}{1.635327in}}%
\pgfpathlineto{\pgfqpoint{4.544093in}{1.629834in}}%
\pgfpathlineto{\pgfqpoint{4.544670in}{1.632084in}}%
\pgfpathlineto{\pgfqpoint{4.545823in}{1.626838in}}%
\pgfpathlineto{\pgfqpoint{4.545246in}{1.633106in}}%
\pgfpathlineto{\pgfqpoint{4.546207in}{1.630476in}}%
\pgfpathlineto{\pgfqpoint{4.546399in}{1.630612in}}%
\pgfpathlineto{\pgfqpoint{4.548129in}{1.644036in}}%
\pgfpathlineto{\pgfqpoint{4.549858in}{1.632501in}}%
\pgfpathlineto{\pgfqpoint{4.551780in}{1.645439in}}%
\pgfpathlineto{\pgfqpoint{4.552741in}{1.640398in}}%
\pgfpathlineto{\pgfqpoint{4.553125in}{1.643213in}}%
\pgfpathlineto{\pgfqpoint{4.553318in}{1.645352in}}%
\pgfpathlineto{\pgfqpoint{4.553510in}{1.642617in}}%
\pgfpathlineto{\pgfqpoint{4.553702in}{1.643638in}}%
\pgfpathlineto{\pgfqpoint{4.555047in}{1.632511in}}%
\pgfpathlineto{\pgfqpoint{4.555816in}{1.634064in}}%
\pgfpathlineto{\pgfqpoint{4.556392in}{1.626595in}}%
\pgfpathlineto{\pgfqpoint{4.557161in}{1.629659in}}%
\pgfpathlineto{\pgfqpoint{4.557353in}{1.629599in}}%
\pgfpathlineto{\pgfqpoint{4.558698in}{1.622820in}}%
\pgfpathlineto{\pgfqpoint{4.558890in}{1.623093in}}%
\pgfpathlineto{\pgfqpoint{4.559083in}{1.624158in}}%
\pgfpathlineto{\pgfqpoint{4.559275in}{1.623117in}}%
\pgfpathlineto{\pgfqpoint{4.560043in}{1.613810in}}%
\pgfpathlineto{\pgfqpoint{4.560620in}{1.615965in}}%
\pgfpathlineto{\pgfqpoint{4.561004in}{1.617288in}}%
\pgfpathlineto{\pgfqpoint{4.562157in}{1.623116in}}%
\pgfpathlineto{\pgfqpoint{4.562349in}{1.622304in}}%
\pgfpathlineto{\pgfqpoint{4.562734in}{1.622369in}}%
\pgfpathlineto{\pgfqpoint{4.562926in}{1.620895in}}%
\pgfpathlineto{\pgfqpoint{4.563118in}{1.624562in}}%
\pgfpathlineto{\pgfqpoint{4.563695in}{1.622071in}}%
\pgfpathlineto{\pgfqpoint{4.564271in}{1.622706in}}%
\pgfpathlineto{\pgfqpoint{4.564655in}{1.621888in}}%
\pgfpathlineto{\pgfqpoint{4.565232in}{1.620766in}}%
\pgfpathlineto{\pgfqpoint{4.565424in}{1.624302in}}%
\pgfpathlineto{\pgfqpoint{4.566001in}{1.619147in}}%
\pgfpathlineto{\pgfqpoint{4.566769in}{1.612884in}}%
\pgfpathlineto{\pgfqpoint{4.567346in}{1.614168in}}%
\pgfpathlineto{\pgfqpoint{4.567730in}{1.619480in}}%
\pgfpathlineto{\pgfqpoint{4.568691in}{1.617548in}}%
\pgfpathlineto{\pgfqpoint{4.569075in}{1.615657in}}%
\pgfpathlineto{\pgfqpoint{4.569460in}{1.618150in}}%
\pgfpathlineto{\pgfqpoint{4.569844in}{1.625441in}}%
\pgfpathlineto{\pgfqpoint{4.570613in}{1.622043in}}%
\pgfpathlineto{\pgfqpoint{4.571381in}{1.617937in}}%
\pgfpathlineto{\pgfqpoint{4.571958in}{1.618518in}}%
\pgfpathlineto{\pgfqpoint{4.572919in}{1.623066in}}%
\pgfpathlineto{\pgfqpoint{4.573303in}{1.621411in}}%
\pgfpathlineto{\pgfqpoint{4.573879in}{1.621991in}}%
\pgfpathlineto{\pgfqpoint{4.575417in}{1.609276in}}%
\pgfpathlineto{\pgfqpoint{4.575609in}{1.610511in}}%
\pgfpathlineto{\pgfqpoint{4.576185in}{1.616818in}}%
\pgfpathlineto{\pgfqpoint{4.577146in}{1.616413in}}%
\pgfpathlineto{\pgfqpoint{4.577338in}{1.616914in}}%
\pgfpathlineto{\pgfqpoint{4.577723in}{1.615890in}}%
\pgfpathlineto{\pgfqpoint{4.578684in}{1.609349in}}%
\pgfpathlineto{\pgfqpoint{4.579068in}{1.610299in}}%
\pgfpathlineto{\pgfqpoint{4.579260in}{1.612363in}}%
\pgfpathlineto{\pgfqpoint{4.579837in}{1.606969in}}%
\pgfpathlineto{\pgfqpoint{4.581374in}{1.595969in}}%
\pgfpathlineto{\pgfqpoint{4.581758in}{1.596547in}}%
\pgfpathlineto{\pgfqpoint{4.582527in}{1.599019in}}%
\pgfpathlineto{\pgfqpoint{4.582719in}{1.597471in}}%
\pgfpathlineto{\pgfqpoint{4.583872in}{1.595432in}}%
\pgfpathlineto{\pgfqpoint{4.584449in}{1.597094in}}%
\pgfpathlineto{\pgfqpoint{4.585602in}{1.606004in}}%
\pgfpathlineto{\pgfqpoint{4.586178in}{1.602320in}}%
\pgfpathlineto{\pgfqpoint{4.588484in}{1.612591in}}%
\pgfpathlineto{\pgfqpoint{4.588676in}{1.610855in}}%
\pgfpathlineto{\pgfqpoint{4.589061in}{1.608302in}}%
\pgfpathlineto{\pgfqpoint{4.589253in}{1.611198in}}%
\pgfpathlineto{\pgfqpoint{4.590214in}{1.618868in}}%
\pgfpathlineto{\pgfqpoint{4.590598in}{1.615667in}}%
\pgfpathlineto{\pgfqpoint{4.592520in}{1.619698in}}%
\pgfpathlineto{\pgfqpoint{4.592904in}{1.616850in}}%
\pgfpathlineto{\pgfqpoint{4.593481in}{1.618983in}}%
\pgfpathlineto{\pgfqpoint{4.595979in}{1.632811in}}%
\pgfpathlineto{\pgfqpoint{4.596171in}{1.630781in}}%
\pgfpathlineto{\pgfqpoint{4.596555in}{1.637219in}}%
\pgfpathlineto{\pgfqpoint{4.596747in}{1.634761in}}%
\pgfpathlineto{\pgfqpoint{4.598669in}{1.647845in}}%
\pgfpathlineto{\pgfqpoint{4.599822in}{1.646737in}}%
\pgfpathlineto{\pgfqpoint{4.601552in}{1.637213in}}%
\pgfpathlineto{\pgfqpoint{4.602128in}{1.637977in}}%
\pgfpathlineto{\pgfqpoint{4.602320in}{1.638277in}}%
\pgfpathlineto{\pgfqpoint{4.602513in}{1.637413in}}%
\pgfpathlineto{\pgfqpoint{4.603666in}{1.626430in}}%
\pgfpathlineto{\pgfqpoint{4.604050in}{1.628402in}}%
\pgfpathlineto{\pgfqpoint{4.605972in}{1.617491in}}%
\pgfpathlineto{\pgfqpoint{4.606356in}{1.618361in}}%
\pgfpathlineto{\pgfqpoint{4.608278in}{1.613801in}}%
\pgfpathlineto{\pgfqpoint{4.608662in}{1.615978in}}%
\pgfpathlineto{\pgfqpoint{4.609046in}{1.618271in}}%
\pgfpathlineto{\pgfqpoint{4.609238in}{1.615030in}}%
\pgfpathlineto{\pgfqpoint{4.610391in}{1.609860in}}%
\pgfpathlineto{\pgfqpoint{4.610584in}{1.612705in}}%
\pgfpathlineto{\pgfqpoint{4.610776in}{1.614329in}}%
\pgfpathlineto{\pgfqpoint{4.611160in}{1.611038in}}%
\pgfpathlineto{\pgfqpoint{4.613082in}{1.595220in}}%
\pgfpathlineto{\pgfqpoint{4.613274in}{1.596712in}}%
\pgfpathlineto{\pgfqpoint{4.613466in}{1.597434in}}%
\pgfpathlineto{\pgfqpoint{4.613658in}{1.594737in}}%
\pgfpathlineto{\pgfqpoint{4.613850in}{1.594914in}}%
\pgfpathlineto{\pgfqpoint{4.614619in}{1.590230in}}%
\pgfpathlineto{\pgfqpoint{4.615196in}{1.592929in}}%
\pgfpathlineto{\pgfqpoint{4.615388in}{1.595574in}}%
\pgfpathlineto{\pgfqpoint{4.615964in}{1.589172in}}%
\pgfpathlineto{\pgfqpoint{4.616156in}{1.589405in}}%
\pgfpathlineto{\pgfqpoint{4.616349in}{1.588819in}}%
\pgfpathlineto{\pgfqpoint{4.616733in}{1.594529in}}%
\pgfpathlineto{\pgfqpoint{4.617502in}{1.589599in}}%
\pgfpathlineto{\pgfqpoint{4.618078in}{1.582180in}}%
\pgfpathlineto{\pgfqpoint{4.618655in}{1.588384in}}%
\pgfpathlineto{\pgfqpoint{4.619808in}{1.579040in}}%
\pgfpathlineto{\pgfqpoint{4.620000in}{1.575680in}}%
\pgfpathlineto{\pgfqpoint{4.620961in}{1.576813in}}%
\pgfpathlineto{\pgfqpoint{4.622306in}{1.587931in}}%
\pgfpathlineto{\pgfqpoint{4.622690in}{1.587198in}}%
\pgfpathlineto{\pgfqpoint{4.622882in}{1.586232in}}%
\pgfpathlineto{\pgfqpoint{4.623267in}{1.587628in}}%
\pgfpathlineto{\pgfqpoint{4.623651in}{1.587405in}}%
\pgfpathlineto{\pgfqpoint{4.623843in}{1.587578in}}%
\pgfpathlineto{\pgfqpoint{4.624804in}{1.587427in}}%
\pgfpathlineto{\pgfqpoint{4.625188in}{1.590682in}}%
\pgfpathlineto{\pgfqpoint{4.626341in}{1.586435in}}%
\pgfpathlineto{\pgfqpoint{4.626534in}{1.588681in}}%
\pgfpathlineto{\pgfqpoint{4.627110in}{1.586188in}}%
\pgfpathlineto{\pgfqpoint{4.627494in}{1.587214in}}%
\pgfpathlineto{\pgfqpoint{4.628263in}{1.583036in}}%
\pgfpathlineto{\pgfqpoint{4.628840in}{1.584875in}}%
\pgfpathlineto{\pgfqpoint{4.629032in}{1.586293in}}%
\pgfpathlineto{\pgfqpoint{4.629224in}{1.582327in}}%
\pgfpathlineto{\pgfqpoint{4.629416in}{1.582519in}}%
\pgfpathlineto{\pgfqpoint{4.629608in}{1.581370in}}%
\pgfpathlineto{\pgfqpoint{4.630185in}{1.583354in}}%
\pgfpathlineto{\pgfqpoint{4.630377in}{1.583283in}}%
\pgfpathlineto{\pgfqpoint{4.631146in}{1.590968in}}%
\pgfpathlineto{\pgfqpoint{4.632299in}{1.590421in}}%
\pgfpathlineto{\pgfqpoint{4.633644in}{1.586408in}}%
\pgfpathlineto{\pgfqpoint{4.633836in}{1.586917in}}%
\pgfpathlineto{\pgfqpoint{4.634028in}{1.584985in}}%
\pgfpathlineto{\pgfqpoint{4.634797in}{1.580548in}}%
\pgfpathlineto{\pgfqpoint{4.635373in}{1.582121in}}%
\pgfpathlineto{\pgfqpoint{4.635950in}{1.583401in}}%
\pgfpathlineto{\pgfqpoint{4.635758in}{1.580324in}}%
\pgfpathlineto{\pgfqpoint{4.636142in}{1.580397in}}%
\pgfpathlineto{\pgfqpoint{4.636334in}{1.576444in}}%
\pgfpathlineto{\pgfqpoint{4.636911in}{1.580453in}}%
\pgfpathlineto{\pgfqpoint{4.637103in}{1.584764in}}%
\pgfpathlineto{\pgfqpoint{4.638064in}{1.583970in}}%
\pgfpathlineto{\pgfqpoint{4.638640in}{1.580623in}}%
\pgfpathlineto{\pgfqpoint{4.639024in}{1.584041in}}%
\pgfpathlineto{\pgfqpoint{4.639217in}{1.581199in}}%
\pgfpathlineto{\pgfqpoint{4.639793in}{1.585996in}}%
\pgfpathlineto{\pgfqpoint{4.640177in}{1.581668in}}%
\pgfpathlineto{\pgfqpoint{4.641330in}{1.574194in}}%
\pgfpathlineto{\pgfqpoint{4.642483in}{1.566270in}}%
\pgfpathlineto{\pgfqpoint{4.642868in}{1.568109in}}%
\pgfpathlineto{\pgfqpoint{4.643444in}{1.573941in}}%
\pgfpathlineto{\pgfqpoint{4.644021in}{1.568527in}}%
\pgfpathlineto{\pgfqpoint{4.644213in}{1.568335in}}%
\pgfpathlineto{\pgfqpoint{4.644405in}{1.571329in}}%
\pgfpathlineto{\pgfqpoint{4.644982in}{1.565706in}}%
\pgfpathlineto{\pgfqpoint{4.646135in}{1.563111in}}%
\pgfpathlineto{\pgfqpoint{4.645558in}{1.566339in}}%
\pgfpathlineto{\pgfqpoint{4.646327in}{1.565239in}}%
\pgfpathlineto{\pgfqpoint{4.646711in}{1.565464in}}%
\pgfpathlineto{\pgfqpoint{4.647480in}{1.562527in}}%
\pgfpathlineto{\pgfqpoint{4.648633in}{1.568213in}}%
\pgfpathlineto{\pgfqpoint{4.647864in}{1.561815in}}%
\pgfpathlineto{\pgfqpoint{4.648825in}{1.566718in}}%
\pgfpathlineto{\pgfqpoint{4.649209in}{1.569080in}}%
\pgfpathlineto{\pgfqpoint{4.651708in}{1.587848in}}%
\pgfpathlineto{\pgfqpoint{4.652861in}{1.579617in}}%
\pgfpathlineto{\pgfqpoint{4.653053in}{1.579868in}}%
\pgfpathlineto{\pgfqpoint{4.653437in}{1.583246in}}%
\pgfpathlineto{\pgfqpoint{4.654014in}{1.579705in}}%
\pgfpathlineto{\pgfqpoint{4.654206in}{1.582218in}}%
\pgfpathlineto{\pgfqpoint{4.654974in}{1.577668in}}%
\pgfpathlineto{\pgfqpoint{4.655359in}{1.581216in}}%
\pgfpathlineto{\pgfqpoint{4.655743in}{1.580610in}}%
\pgfpathlineto{\pgfqpoint{4.655935in}{1.581235in}}%
\pgfpathlineto{\pgfqpoint{4.656127in}{1.582175in}}%
\pgfpathlineto{\pgfqpoint{4.656512in}{1.580321in}}%
\pgfpathlineto{\pgfqpoint{4.656704in}{1.578322in}}%
\pgfpathlineto{\pgfqpoint{4.657280in}{1.583851in}}%
\pgfpathlineto{\pgfqpoint{4.657473in}{1.586428in}}%
\pgfpathlineto{\pgfqpoint{4.658049in}{1.581564in}}%
\pgfpathlineto{\pgfqpoint{4.658241in}{1.580230in}}%
\pgfpathlineto{\pgfqpoint{4.658626in}{1.581570in}}%
\pgfpathlineto{\pgfqpoint{4.658818in}{1.584324in}}%
\pgfpathlineto{\pgfqpoint{4.659394in}{1.577840in}}%
\pgfpathlineto{\pgfqpoint{4.659586in}{1.579177in}}%
\pgfpathlineto{\pgfqpoint{4.659971in}{1.574048in}}%
\pgfpathlineto{\pgfqpoint{4.660163in}{1.576122in}}%
\pgfpathlineto{\pgfqpoint{4.660355in}{1.575335in}}%
\pgfpathlineto{\pgfqpoint{4.660547in}{1.578794in}}%
\pgfpathlineto{\pgfqpoint{4.660739in}{1.576933in}}%
\pgfpathlineto{\pgfqpoint{4.661124in}{1.577850in}}%
\pgfpathlineto{\pgfqpoint{4.661316in}{1.575385in}}%
\pgfpathlineto{\pgfqpoint{4.661508in}{1.576520in}}%
\pgfpathlineto{\pgfqpoint{4.662085in}{1.576868in}}%
\pgfpathlineto{\pgfqpoint{4.662661in}{1.572066in}}%
\pgfpathlineto{\pgfqpoint{4.663238in}{1.576613in}}%
\pgfpathlineto{\pgfqpoint{4.663814in}{1.574536in}}%
\pgfpathlineto{\pgfqpoint{4.666120in}{1.583375in}}%
\pgfpathlineto{\pgfqpoint{4.666504in}{1.581514in}}%
\pgfpathlineto{\pgfqpoint{4.666697in}{1.584635in}}%
\pgfpathlineto{\pgfqpoint{4.666889in}{1.583762in}}%
\pgfpathlineto{\pgfqpoint{4.667465in}{1.588738in}}%
\pgfpathlineto{\pgfqpoint{4.667850in}{1.584868in}}%
\pgfpathlineto{\pgfqpoint{4.668810in}{1.579638in}}%
\pgfpathlineto{\pgfqpoint{4.668234in}{1.585693in}}%
\pgfpathlineto{\pgfqpoint{4.669387in}{1.581484in}}%
\pgfpathlineto{\pgfqpoint{4.669771in}{1.582386in}}%
\pgfpathlineto{\pgfqpoint{4.670348in}{1.581338in}}%
\pgfpathlineto{\pgfqpoint{4.670924in}{1.579659in}}%
\pgfpathlineto{\pgfqpoint{4.671309in}{1.584454in}}%
\pgfpathlineto{\pgfqpoint{4.671501in}{1.578682in}}%
\pgfpathlineto{\pgfqpoint{4.672077in}{1.581669in}}%
\pgfpathlineto{\pgfqpoint{4.672269in}{1.582172in}}%
\pgfpathlineto{\pgfqpoint{4.672654in}{1.577333in}}%
\pgfpathlineto{\pgfqpoint{4.673038in}{1.582657in}}%
\pgfpathlineto{\pgfqpoint{4.673422in}{1.581134in}}%
\pgfpathlineto{\pgfqpoint{4.673999in}{1.578075in}}%
\pgfpathlineto{\pgfqpoint{4.674768in}{1.579736in}}%
\pgfpathlineto{\pgfqpoint{4.676305in}{1.591049in}}%
\pgfpathlineto{\pgfqpoint{4.676497in}{1.590142in}}%
\pgfpathlineto{\pgfqpoint{4.677074in}{1.581579in}}%
\pgfpathlineto{\pgfqpoint{4.677842in}{1.585784in}}%
\pgfpathlineto{\pgfqpoint{4.678227in}{1.584424in}}%
\pgfpathlineto{\pgfqpoint{4.678419in}{1.585099in}}%
\pgfpathlineto{\pgfqpoint{4.679380in}{1.591006in}}%
\pgfpathlineto{\pgfqpoint{4.679572in}{1.587933in}}%
\pgfpathlineto{\pgfqpoint{4.680341in}{1.578833in}}%
\pgfpathlineto{\pgfqpoint{4.681109in}{1.579825in}}%
\pgfpathlineto{\pgfqpoint{4.682262in}{1.586276in}}%
\pgfpathlineto{\pgfqpoint{4.682839in}{1.585871in}}%
\pgfpathlineto{\pgfqpoint{4.683031in}{1.585863in}}%
\pgfpathlineto{\pgfqpoint{4.683415in}{1.583548in}}%
\pgfpathlineto{\pgfqpoint{4.683992in}{1.585387in}}%
\pgfpathlineto{\pgfqpoint{4.685145in}{1.590138in}}%
\pgfpathlineto{\pgfqpoint{4.687066in}{1.602420in}}%
\pgfpathlineto{\pgfqpoint{4.687643in}{1.596773in}}%
\pgfpathlineto{\pgfqpoint{4.688219in}{1.599419in}}%
\pgfpathlineto{\pgfqpoint{4.689757in}{1.605070in}}%
\pgfpathlineto{\pgfqpoint{4.688604in}{1.598957in}}%
\pgfpathlineto{\pgfqpoint{4.689949in}{1.603040in}}%
\pgfpathlineto{\pgfqpoint{4.691486in}{1.599132in}}%
\pgfpathlineto{\pgfqpoint{4.691871in}{1.600765in}}%
\pgfpathlineto{\pgfqpoint{4.692063in}{1.602703in}}%
\pgfpathlineto{\pgfqpoint{4.692831in}{1.599896in}}%
\pgfpathlineto{\pgfqpoint{4.693408in}{1.595712in}}%
\pgfpathlineto{\pgfqpoint{4.693984in}{1.598663in}}%
\pgfpathlineto{\pgfqpoint{4.694177in}{1.598502in}}%
\pgfpathlineto{\pgfqpoint{4.695330in}{1.602548in}}%
\pgfpathlineto{\pgfqpoint{4.696290in}{1.593321in}}%
\pgfpathlineto{\pgfqpoint{4.696867in}{1.594710in}}%
\pgfpathlineto{\pgfqpoint{4.697251in}{1.594689in}}%
\pgfpathlineto{\pgfqpoint{4.698789in}{1.587677in}}%
\pgfpathlineto{\pgfqpoint{4.699365in}{1.588274in}}%
\pgfpathlineto{\pgfqpoint{4.699749in}{1.589865in}}%
\pgfpathlineto{\pgfqpoint{4.701095in}{1.597375in}}%
\pgfpathlineto{\pgfqpoint{4.702632in}{1.594166in}}%
\pgfpathlineto{\pgfqpoint{4.702824in}{1.593916in}}%
\pgfpathlineto{\pgfqpoint{4.703209in}{1.594501in}}%
\pgfpathlineto{\pgfqpoint{4.704169in}{1.599932in}}%
\pgfpathlineto{\pgfqpoint{4.704362in}{1.597089in}}%
\pgfpathlineto{\pgfqpoint{4.705322in}{1.598763in}}%
\pgfpathlineto{\pgfqpoint{4.706475in}{1.592247in}}%
\pgfpathlineto{\pgfqpoint{4.706860in}{1.592918in}}%
\pgfpathlineto{\pgfqpoint{4.707052in}{1.592008in}}%
\pgfpathlineto{\pgfqpoint{4.707436in}{1.594688in}}%
\pgfpathlineto{\pgfqpoint{4.708013in}{1.592606in}}%
\pgfpathlineto{\pgfqpoint{4.708589in}{1.596190in}}%
\pgfpathlineto{\pgfqpoint{4.708974in}{1.593453in}}%
\pgfpathlineto{\pgfqpoint{4.709358in}{1.588972in}}%
\pgfpathlineto{\pgfqpoint{4.709934in}{1.592883in}}%
\pgfpathlineto{\pgfqpoint{4.710127in}{1.593316in}}%
\pgfpathlineto{\pgfqpoint{4.710703in}{1.595696in}}%
\pgfpathlineto{\pgfqpoint{4.711280in}{1.589156in}}%
\pgfpathlineto{\pgfqpoint{4.711856in}{1.591742in}}%
\pgfpathlineto{\pgfqpoint{4.712240in}{1.589544in}}%
\pgfpathlineto{\pgfqpoint{4.712433in}{1.588985in}}%
\pgfpathlineto{\pgfqpoint{4.712817in}{1.591040in}}%
\pgfpathlineto{\pgfqpoint{4.714162in}{1.599379in}}%
\pgfpathlineto{\pgfqpoint{4.714354in}{1.597181in}}%
\pgfpathlineto{\pgfqpoint{4.716660in}{1.583728in}}%
\pgfpathlineto{\pgfqpoint{4.717045in}{1.587234in}}%
\pgfpathlineto{\pgfqpoint{4.717813in}{1.586533in}}%
\pgfpathlineto{\pgfqpoint{4.718005in}{1.584739in}}%
\pgfpathlineto{\pgfqpoint{4.718198in}{1.587823in}}%
\pgfpathlineto{\pgfqpoint{4.718774in}{1.587006in}}%
\pgfpathlineto{\pgfqpoint{4.720119in}{1.590426in}}%
\pgfpathlineto{\pgfqpoint{4.720311in}{1.590155in}}%
\pgfpathlineto{\pgfqpoint{4.721657in}{1.583748in}}%
\pgfpathlineto{\pgfqpoint{4.722810in}{1.586814in}}%
\pgfpathlineto{\pgfqpoint{4.723194in}{1.583183in}}%
\pgfpathlineto{\pgfqpoint{4.723770in}{1.586287in}}%
\pgfpathlineto{\pgfqpoint{4.723963in}{1.588427in}}%
\pgfpathlineto{\pgfqpoint{4.724347in}{1.585154in}}%
\pgfpathlineto{\pgfqpoint{4.725884in}{1.575534in}}%
\pgfpathlineto{\pgfqpoint{4.727037in}{1.571556in}}%
\pgfpathlineto{\pgfqpoint{4.726461in}{1.577460in}}%
\pgfpathlineto{\pgfqpoint{4.727230in}{1.572291in}}%
\pgfpathlineto{\pgfqpoint{4.727614in}{1.574497in}}%
\pgfpathlineto{\pgfqpoint{4.727998in}{1.569075in}}%
\pgfpathlineto{\pgfqpoint{4.728383in}{1.567427in}}%
\pgfpathlineto{\pgfqpoint{4.728575in}{1.570265in}}%
\pgfpathlineto{\pgfqpoint{4.728959in}{1.571008in}}%
\pgfpathlineto{\pgfqpoint{4.729536in}{1.569821in}}%
\pgfpathlineto{\pgfqpoint{4.732610in}{1.551624in}}%
\pgfpathlineto{\pgfqpoint{4.729920in}{1.570238in}}%
\pgfpathlineto{\pgfqpoint{4.733379in}{1.557628in}}%
\pgfpathlineto{\pgfqpoint{4.733955in}{1.559544in}}%
\pgfpathlineto{\pgfqpoint{4.734148in}{1.559051in}}%
\pgfpathlineto{\pgfqpoint{4.734916in}{1.552223in}}%
\pgfpathlineto{\pgfqpoint{4.735301in}{1.557088in}}%
\pgfpathlineto{\pgfqpoint{4.736069in}{1.560711in}}%
\pgfpathlineto{\pgfqpoint{4.736646in}{1.560470in}}%
\pgfpathlineto{\pgfqpoint{4.736838in}{1.558830in}}%
\pgfpathlineto{\pgfqpoint{4.737030in}{1.565271in}}%
\pgfpathlineto{\pgfqpoint{4.738183in}{1.570954in}}%
\pgfpathlineto{\pgfqpoint{4.738567in}{1.570575in}}%
\pgfpathlineto{\pgfqpoint{4.739144in}{1.574027in}}%
\pgfpathlineto{\pgfqpoint{4.739528in}{1.571040in}}%
\pgfpathlineto{\pgfqpoint{4.739720in}{1.570572in}}%
\pgfpathlineto{\pgfqpoint{4.740105in}{1.574107in}}%
\pgfpathlineto{\pgfqpoint{4.740489in}{1.570033in}}%
\pgfpathlineto{\pgfqpoint{4.740681in}{1.570511in}}%
\pgfpathlineto{\pgfqpoint{4.741834in}{1.562120in}}%
\pgfpathlineto{\pgfqpoint{4.742026in}{1.563245in}}%
\pgfpathlineto{\pgfqpoint{4.745293in}{1.577410in}}%
\pgfpathlineto{\pgfqpoint{4.747215in}{1.562799in}}%
\pgfpathlineto{\pgfqpoint{4.747407in}{1.563190in}}%
\pgfpathlineto{\pgfqpoint{4.747599in}{1.563223in}}%
\pgfpathlineto{\pgfqpoint{4.747791in}{1.560284in}}%
\pgfpathlineto{\pgfqpoint{4.748176in}{1.564603in}}%
\pgfpathlineto{\pgfqpoint{4.748752in}{1.562300in}}%
\pgfpathlineto{\pgfqpoint{4.749137in}{1.560707in}}%
\pgfpathlineto{\pgfqpoint{4.749329in}{1.563631in}}%
\pgfpathlineto{\pgfqpoint{4.749521in}{1.562414in}}%
\pgfpathlineto{\pgfqpoint{4.749905in}{1.564860in}}%
\pgfpathlineto{\pgfqpoint{4.750098in}{1.561039in}}%
\pgfpathlineto{\pgfqpoint{4.750290in}{1.561214in}}%
\pgfpathlineto{\pgfqpoint{4.750482in}{1.559918in}}%
\pgfpathlineto{\pgfqpoint{4.750866in}{1.563194in}}%
\pgfpathlineto{\pgfqpoint{4.751635in}{1.569028in}}%
\pgfpathlineto{\pgfqpoint{4.751827in}{1.567007in}}%
\pgfpathlineto{\pgfqpoint{4.752211in}{1.557721in}}%
\pgfpathlineto{\pgfqpoint{4.752980in}{1.561588in}}%
\pgfpathlineto{\pgfqpoint{4.753172in}{1.562646in}}%
\pgfpathlineto{\pgfqpoint{4.753364in}{1.558955in}}%
\pgfpathlineto{\pgfqpoint{4.755094in}{1.552177in}}%
\pgfpathlineto{\pgfqpoint{4.755478in}{1.553510in}}%
\pgfpathlineto{\pgfqpoint{4.755670in}{1.554584in}}%
\pgfpathlineto{\pgfqpoint{4.755863in}{1.552051in}}%
\pgfpathlineto{\pgfqpoint{4.756055in}{1.552300in}}%
\pgfpathlineto{\pgfqpoint{4.756247in}{1.549657in}}%
\pgfpathlineto{\pgfqpoint{4.756631in}{1.554795in}}%
\pgfpathlineto{\pgfqpoint{4.756823in}{1.553470in}}%
\pgfpathlineto{\pgfqpoint{4.757976in}{1.555755in}}%
\pgfpathlineto{\pgfqpoint{4.758745in}{1.544742in}}%
\pgfpathlineto{\pgfqpoint{4.759706in}{1.547058in}}%
\pgfpathlineto{\pgfqpoint{4.761051in}{1.551836in}}%
\pgfpathlineto{\pgfqpoint{4.761628in}{1.548497in}}%
\pgfpathlineto{\pgfqpoint{4.761820in}{1.551918in}}%
\pgfpathlineto{\pgfqpoint{4.762012in}{1.551811in}}%
\pgfpathlineto{\pgfqpoint{4.762781in}{1.553273in}}%
\pgfpathlineto{\pgfqpoint{4.763165in}{1.551967in}}%
\pgfpathlineto{\pgfqpoint{4.763357in}{1.552301in}}%
\pgfpathlineto{\pgfqpoint{4.763549in}{1.551687in}}%
\pgfpathlineto{\pgfqpoint{4.763934in}{1.548963in}}%
\pgfpathlineto{\pgfqpoint{4.764510in}{1.550195in}}%
\pgfpathlineto{\pgfqpoint{4.765087in}{1.555806in}}%
\pgfpathlineto{\pgfqpoint{4.765471in}{1.549981in}}%
\pgfpathlineto{\pgfqpoint{4.765855in}{1.550267in}}%
\pgfpathlineto{\pgfqpoint{4.766624in}{1.555326in}}%
\pgfpathlineto{\pgfqpoint{4.767200in}{1.553351in}}%
\pgfpathlineto{\pgfqpoint{4.767393in}{1.552695in}}%
\pgfpathlineto{\pgfqpoint{4.768738in}{1.562495in}}%
\pgfpathlineto{\pgfqpoint{4.768930in}{1.562112in}}%
\pgfpathlineto{\pgfqpoint{4.769699in}{1.560358in}}%
\pgfpathlineto{\pgfqpoint{4.770083in}{1.566527in}}%
\pgfpathlineto{\pgfqpoint{4.770659in}{1.561126in}}%
\pgfpathlineto{\pgfqpoint{4.770852in}{1.561094in}}%
\pgfpathlineto{\pgfqpoint{4.772005in}{1.571637in}}%
\pgfpathlineto{\pgfqpoint{4.772197in}{1.570753in}}%
\pgfpathlineto{\pgfqpoint{4.772965in}{1.564951in}}%
\pgfpathlineto{\pgfqpoint{4.773350in}{1.565097in}}%
\pgfpathlineto{\pgfqpoint{4.773926in}{1.572661in}}%
\pgfpathlineto{\pgfqpoint{4.774887in}{1.571152in}}%
\pgfpathlineto{\pgfqpoint{4.775464in}{1.567593in}}%
\pgfpathlineto{\pgfqpoint{4.776040in}{1.558988in}}%
\pgfpathlineto{\pgfqpoint{4.777193in}{1.559981in}}%
\pgfpathlineto{\pgfqpoint{4.777962in}{1.570184in}}%
\pgfpathlineto{\pgfqpoint{4.778538in}{1.566865in}}%
\pgfpathlineto{\pgfqpoint{4.778923in}{1.567659in}}%
\pgfpathlineto{\pgfqpoint{4.779115in}{1.564775in}}%
\pgfpathlineto{\pgfqpoint{4.779884in}{1.569946in}}%
\pgfpathlineto{\pgfqpoint{4.780268in}{1.575851in}}%
\pgfpathlineto{\pgfqpoint{4.781229in}{1.574529in}}%
\pgfpathlineto{\pgfqpoint{4.782190in}{1.580908in}}%
\pgfpathlineto{\pgfqpoint{4.782382in}{1.577662in}}%
\pgfpathlineto{\pgfqpoint{4.783150in}{1.573797in}}%
\pgfpathlineto{\pgfqpoint{4.783343in}{1.577926in}}%
\pgfpathlineto{\pgfqpoint{4.784303in}{1.586459in}}%
\pgfpathlineto{\pgfqpoint{4.784688in}{1.581776in}}%
\pgfpathlineto{\pgfqpoint{4.784880in}{1.583131in}}%
\pgfpathlineto{\pgfqpoint{4.785264in}{1.579094in}}%
\pgfpathlineto{\pgfqpoint{4.785456in}{1.579059in}}%
\pgfpathlineto{\pgfqpoint{4.786994in}{1.574293in}}%
\pgfpathlineto{\pgfqpoint{4.787186in}{1.575179in}}%
\pgfpathlineto{\pgfqpoint{4.787570in}{1.572916in}}%
\pgfpathlineto{\pgfqpoint{4.787955in}{1.573781in}}%
\pgfpathlineto{\pgfqpoint{4.788723in}{1.570370in}}%
\pgfpathlineto{\pgfqpoint{4.789300in}{1.571812in}}%
\pgfpathlineto{\pgfqpoint{4.789684in}{1.569847in}}%
\pgfpathlineto{\pgfqpoint{4.789876in}{1.568928in}}%
\pgfpathlineto{\pgfqpoint{4.790453in}{1.571144in}}%
\pgfpathlineto{\pgfqpoint{4.790837in}{1.573804in}}%
\pgfpathlineto{\pgfqpoint{4.791414in}{1.570339in}}%
\pgfpathlineto{\pgfqpoint{4.791798in}{1.572509in}}%
\pgfpathlineto{\pgfqpoint{4.792182in}{1.571498in}}%
\pgfpathlineto{\pgfqpoint{4.793335in}{1.577987in}}%
\pgfpathlineto{\pgfqpoint{4.793527in}{1.576698in}}%
\pgfpathlineto{\pgfqpoint{4.793720in}{1.576183in}}%
\pgfpathlineto{\pgfqpoint{4.795065in}{1.569923in}}%
\pgfpathlineto{\pgfqpoint{4.795449in}{1.565306in}}%
\pgfpathlineto{\pgfqpoint{4.796026in}{1.568475in}}%
\pgfpathlineto{\pgfqpoint{4.797371in}{1.572534in}}%
\pgfpathlineto{\pgfqpoint{4.797563in}{1.571006in}}%
\pgfpathlineto{\pgfqpoint{4.798140in}{1.573043in}}%
\pgfpathlineto{\pgfqpoint{4.798332in}{1.572483in}}%
\pgfpathlineto{\pgfqpoint{4.799100in}{1.576882in}}%
\pgfpathlineto{\pgfqpoint{4.799485in}{1.576424in}}%
\pgfpathlineto{\pgfqpoint{4.799677in}{1.575481in}}%
\pgfpathlineto{\pgfqpoint{4.799869in}{1.578776in}}%
\pgfpathlineto{\pgfqpoint{4.800253in}{1.575703in}}%
\pgfpathlineto{\pgfqpoint{4.800638in}{1.580858in}}%
\pgfpathlineto{\pgfqpoint{4.801214in}{1.575565in}}%
\pgfpathlineto{\pgfqpoint{4.801599in}{1.575601in}}%
\pgfpathlineto{\pgfqpoint{4.802175in}{1.571789in}}%
\pgfpathlineto{\pgfqpoint{4.802559in}{1.576284in}}%
\pgfpathlineto{\pgfqpoint{4.802944in}{1.574088in}}%
\pgfpathlineto{\pgfqpoint{4.803136in}{1.576513in}}%
\pgfpathlineto{\pgfqpoint{4.803328in}{1.577875in}}%
\pgfpathlineto{\pgfqpoint{4.803712in}{1.574430in}}%
\pgfpathlineto{\pgfqpoint{4.804289in}{1.569761in}}%
\pgfpathlineto{\pgfqpoint{4.804481in}{1.573650in}}%
\pgfpathlineto{\pgfqpoint{4.804673in}{1.576961in}}%
\pgfpathlineto{\pgfqpoint{4.805250in}{1.574342in}}%
\pgfpathlineto{\pgfqpoint{4.805634in}{1.570073in}}%
\pgfpathlineto{\pgfqpoint{4.806018in}{1.575121in}}%
\pgfpathlineto{\pgfqpoint{4.806211in}{1.575002in}}%
\pgfpathlineto{\pgfqpoint{4.806979in}{1.578474in}}%
\pgfpathlineto{\pgfqpoint{4.807748in}{1.581369in}}%
\pgfpathlineto{\pgfqpoint{4.808132in}{1.579382in}}%
\pgfpathlineto{\pgfqpoint{4.808709in}{1.578463in}}%
\pgfpathlineto{\pgfqpoint{4.808901in}{1.575727in}}%
\pgfpathlineto{\pgfqpoint{4.809670in}{1.579046in}}%
\pgfpathlineto{\pgfqpoint{4.810054in}{1.580103in}}%
\pgfpathlineto{\pgfqpoint{4.810246in}{1.579763in}}%
\pgfpathlineto{\pgfqpoint{4.811207in}{1.574979in}}%
\pgfpathlineto{\pgfqpoint{4.811015in}{1.579991in}}%
\pgfpathlineto{\pgfqpoint{4.811399in}{1.576253in}}%
\pgfpathlineto{\pgfqpoint{4.813705in}{1.588157in}}%
\pgfpathlineto{\pgfqpoint{4.813897in}{1.583967in}}%
\pgfpathlineto{\pgfqpoint{4.814858in}{1.584345in}}%
\pgfpathlineto{\pgfqpoint{4.815627in}{1.582000in}}%
\pgfpathlineto{\pgfqpoint{4.815242in}{1.584993in}}%
\pgfpathlineto{\pgfqpoint{4.815819in}{1.584673in}}%
\pgfpathlineto{\pgfqpoint{4.816780in}{1.592594in}}%
\pgfpathlineto{\pgfqpoint{4.817548in}{1.592464in}}%
\pgfpathlineto{\pgfqpoint{4.817741in}{1.590555in}}%
\pgfpathlineto{\pgfqpoint{4.818317in}{1.592611in}}%
\pgfpathlineto{\pgfqpoint{4.818509in}{1.592044in}}%
\pgfpathlineto{\pgfqpoint{4.819278in}{1.597368in}}%
\pgfpathlineto{\pgfqpoint{4.819662in}{1.593813in}}%
\pgfpathlineto{\pgfqpoint{4.820431in}{1.587852in}}%
\pgfpathlineto{\pgfqpoint{4.821200in}{1.589048in}}%
\pgfpathlineto{\pgfqpoint{4.821584in}{1.591715in}}%
\pgfpathlineto{\pgfqpoint{4.821968in}{1.587287in}}%
\pgfpathlineto{\pgfqpoint{4.822929in}{1.583018in}}%
\pgfpathlineto{\pgfqpoint{4.823121in}{1.588323in}}%
\pgfpathlineto{\pgfqpoint{4.823890in}{1.584856in}}%
\pgfpathlineto{\pgfqpoint{4.824082in}{1.582881in}}%
\pgfpathlineto{\pgfqpoint{4.824851in}{1.583157in}}%
\pgfpathlineto{\pgfqpoint{4.825427in}{1.585495in}}%
\pgfpathlineto{\pgfqpoint{4.825620in}{1.583021in}}%
\pgfpathlineto{\pgfqpoint{4.825812in}{1.581141in}}%
\pgfpathlineto{\pgfqpoint{4.826196in}{1.584903in}}%
\pgfpathlineto{\pgfqpoint{4.826388in}{1.583694in}}%
\pgfpathlineto{\pgfqpoint{4.828310in}{1.596305in}}%
\pgfpathlineto{\pgfqpoint{4.828502in}{1.595362in}}%
\pgfpathlineto{\pgfqpoint{4.828694in}{1.590641in}}%
\pgfpathlineto{\pgfqpoint{4.829655in}{1.593757in}}%
\pgfpathlineto{\pgfqpoint{4.831000in}{1.589975in}}%
\pgfpathlineto{\pgfqpoint{4.832345in}{1.596195in}}%
\pgfpathlineto{\pgfqpoint{4.832538in}{1.595888in}}%
\pgfpathlineto{\pgfqpoint{4.832922in}{1.594509in}}%
\pgfpathlineto{\pgfqpoint{4.834459in}{1.589289in}}%
\pgfpathlineto{\pgfqpoint{4.834651in}{1.591086in}}%
\pgfpathlineto{\pgfqpoint{4.835036in}{1.585395in}}%
\pgfpathlineto{\pgfqpoint{4.835228in}{1.586729in}}%
\pgfpathlineto{\pgfqpoint{4.835804in}{1.583102in}}%
\pgfpathlineto{\pgfqpoint{4.836381in}{1.585797in}}%
\pgfpathlineto{\pgfqpoint{4.836957in}{1.591801in}}%
\pgfpathlineto{\pgfqpoint{4.837534in}{1.591142in}}%
\pgfpathlineto{\pgfqpoint{4.837726in}{1.590902in}}%
\pgfpathlineto{\pgfqpoint{4.838687in}{1.586291in}}%
\pgfpathlineto{\pgfqpoint{4.838879in}{1.589077in}}%
\pgfpathlineto{\pgfqpoint{4.839840in}{1.584636in}}%
\pgfpathlineto{\pgfqpoint{4.840032in}{1.585352in}}%
\pgfpathlineto{\pgfqpoint{4.841569in}{1.592180in}}%
\pgfpathlineto{\pgfqpoint{4.841954in}{1.589786in}}%
\pgfpathlineto{\pgfqpoint{4.842530in}{1.590881in}}%
\pgfpathlineto{\pgfqpoint{4.843875in}{1.597526in}}%
\pgfpathlineto{\pgfqpoint{4.844068in}{1.597609in}}%
\pgfpathlineto{\pgfqpoint{4.845028in}{1.600580in}}%
\pgfpathlineto{\pgfqpoint{4.844644in}{1.597004in}}%
\pgfpathlineto{\pgfqpoint{4.845221in}{1.598306in}}%
\pgfpathlineto{\pgfqpoint{4.846566in}{1.591730in}}%
\pgfpathlineto{\pgfqpoint{4.848103in}{1.599614in}}%
\pgfpathlineto{\pgfqpoint{4.848295in}{1.597132in}}%
\pgfpathlineto{\pgfqpoint{4.850794in}{1.607374in}}%
\pgfpathlineto{\pgfqpoint{4.848872in}{1.596261in}}%
\pgfpathlineto{\pgfqpoint{4.851370in}{1.604151in}}%
\pgfpathlineto{\pgfqpoint{4.851947in}{1.606007in}}%
\pgfpathlineto{\pgfqpoint{4.852331in}{1.603688in}}%
\pgfpathlineto{\pgfqpoint{4.853292in}{1.599895in}}%
\pgfpathlineto{\pgfqpoint{4.853676in}{1.600381in}}%
\pgfpathlineto{\pgfqpoint{4.854829in}{1.605337in}}%
\pgfpathlineto{\pgfqpoint{4.855213in}{1.603399in}}%
\pgfpathlineto{\pgfqpoint{4.855790in}{1.605137in}}%
\pgfpathlineto{\pgfqpoint{4.856559in}{1.608856in}}%
\pgfpathlineto{\pgfqpoint{4.857135in}{1.606319in}}%
\pgfpathlineto{\pgfqpoint{4.857519in}{1.606294in}}%
\pgfpathlineto{\pgfqpoint{4.857712in}{1.606853in}}%
\pgfpathlineto{\pgfqpoint{4.857904in}{1.608463in}}%
\pgfpathlineto{\pgfqpoint{4.858865in}{1.607161in}}%
\pgfpathlineto{\pgfqpoint{4.859633in}{1.601443in}}%
\pgfpathlineto{\pgfqpoint{4.860018in}{1.604782in}}%
\pgfpathlineto{\pgfqpoint{4.860210in}{1.607312in}}%
\pgfpathlineto{\pgfqpoint{4.860978in}{1.604224in}}%
\pgfpathlineto{\pgfqpoint{4.861363in}{1.604401in}}%
\pgfpathlineto{\pgfqpoint{4.862900in}{1.613150in}}%
\pgfpathlineto{\pgfqpoint{4.864630in}{1.620592in}}%
\pgfpathlineto{\pgfqpoint{4.864822in}{1.619163in}}%
\pgfpathlineto{\pgfqpoint{4.865014in}{1.617726in}}%
\pgfpathlineto{\pgfqpoint{4.865398in}{1.621366in}}%
\pgfpathlineto{\pgfqpoint{4.865590in}{1.619805in}}%
\pgfpathlineto{\pgfqpoint{4.867704in}{1.639850in}}%
\pgfpathlineto{\pgfqpoint{4.868857in}{1.633498in}}%
\pgfpathlineto{\pgfqpoint{4.868089in}{1.642500in}}%
\pgfpathlineto{\pgfqpoint{4.869434in}{1.634969in}}%
\pgfpathlineto{\pgfqpoint{4.870779in}{1.648238in}}%
\pgfpathlineto{\pgfqpoint{4.870971in}{1.645563in}}%
\pgfpathlineto{\pgfqpoint{4.872316in}{1.656679in}}%
\pgfpathlineto{\pgfqpoint{4.872509in}{1.654990in}}%
\pgfpathlineto{\pgfqpoint{4.875391in}{1.642447in}}%
\pgfpathlineto{\pgfqpoint{4.875775in}{1.642988in}}%
\pgfpathlineto{\pgfqpoint{4.876160in}{1.638881in}}%
\pgfpathlineto{\pgfqpoint{4.876928in}{1.640564in}}%
\pgfpathlineto{\pgfqpoint{4.877121in}{1.641630in}}%
\pgfpathlineto{\pgfqpoint{4.877313in}{1.637312in}}%
\pgfpathlineto{\pgfqpoint{4.877505in}{1.634144in}}%
\pgfpathlineto{\pgfqpoint{4.878274in}{1.638950in}}%
\pgfpathlineto{\pgfqpoint{4.878466in}{1.638494in}}%
\pgfpathlineto{\pgfqpoint{4.878850in}{1.640048in}}%
\pgfpathlineto{\pgfqpoint{4.879042in}{1.639672in}}%
\pgfpathlineto{\pgfqpoint{4.879619in}{1.640852in}}%
\pgfpathlineto{\pgfqpoint{4.879811in}{1.639242in}}%
\pgfpathlineto{\pgfqpoint{4.880580in}{1.636958in}}%
\pgfpathlineto{\pgfqpoint{4.880387in}{1.641452in}}%
\pgfpathlineto{\pgfqpoint{4.880772in}{1.638523in}}%
\pgfpathlineto{\pgfqpoint{4.881540in}{1.640223in}}%
\pgfpathlineto{\pgfqpoint{4.881733in}{1.638681in}}%
\pgfpathlineto{\pgfqpoint{4.883270in}{1.627805in}}%
\pgfpathlineto{\pgfqpoint{4.883462in}{1.627896in}}%
\pgfpathlineto{\pgfqpoint{4.883654in}{1.628537in}}%
\pgfpathlineto{\pgfqpoint{4.883846in}{1.625720in}}%
\pgfpathlineto{\pgfqpoint{4.884231in}{1.618327in}}%
\pgfpathlineto{\pgfqpoint{4.884999in}{1.622148in}}%
\pgfpathlineto{\pgfqpoint{4.886345in}{1.616583in}}%
\pgfpathlineto{\pgfqpoint{4.886921in}{1.609644in}}%
\pgfpathlineto{\pgfqpoint{4.887498in}{1.613777in}}%
\pgfpathlineto{\pgfqpoint{4.888074in}{1.615800in}}%
\pgfpathlineto{\pgfqpoint{4.888843in}{1.614890in}}%
\pgfpathlineto{\pgfqpoint{4.889227in}{1.611522in}}%
\pgfpathlineto{\pgfqpoint{4.889804in}{1.614500in}}%
\pgfpathlineto{\pgfqpoint{4.890188in}{1.615319in}}%
\pgfpathlineto{\pgfqpoint{4.890380in}{1.612159in}}%
\pgfpathlineto{\pgfqpoint{4.891341in}{1.614631in}}%
\pgfpathlineto{\pgfqpoint{4.891533in}{1.615114in}}%
\pgfpathlineto{\pgfqpoint{4.891917in}{1.613640in}}%
\pgfpathlineto{\pgfqpoint{4.892110in}{1.614041in}}%
\pgfpathlineto{\pgfqpoint{4.893263in}{1.607376in}}%
\pgfpathlineto{\pgfqpoint{4.893647in}{1.607516in}}%
\pgfpathlineto{\pgfqpoint{4.895377in}{1.615173in}}%
\pgfpathlineto{\pgfqpoint{4.899028in}{1.604809in}}%
\pgfpathlineto{\pgfqpoint{4.899220in}{1.606475in}}%
\pgfpathlineto{\pgfqpoint{4.899412in}{1.606593in}}%
\pgfpathlineto{\pgfqpoint{4.900757in}{1.597299in}}%
\pgfpathlineto{\pgfqpoint{4.902295in}{1.607119in}}%
\pgfpathlineto{\pgfqpoint{4.902679in}{1.606265in}}%
\pgfpathlineto{\pgfqpoint{4.902871in}{1.609763in}}%
\pgfpathlineto{\pgfqpoint{4.903832in}{1.607876in}}%
\pgfpathlineto{\pgfqpoint{4.905754in}{1.621516in}}%
\pgfpathlineto{\pgfqpoint{4.906138in}{1.620396in}}%
\pgfpathlineto{\pgfqpoint{4.907099in}{1.614750in}}%
\pgfpathlineto{\pgfqpoint{4.907291in}{1.614763in}}%
\pgfpathlineto{\pgfqpoint{4.907675in}{1.618068in}}%
\pgfpathlineto{\pgfqpoint{4.908252in}{1.614905in}}%
\pgfpathlineto{\pgfqpoint{4.909020in}{1.614270in}}%
\pgfpathlineto{\pgfqpoint{4.908636in}{1.615409in}}%
\pgfpathlineto{\pgfqpoint{4.909405in}{1.614675in}}%
\pgfpathlineto{\pgfqpoint{4.909981in}{1.615367in}}%
\pgfpathlineto{\pgfqpoint{4.910366in}{1.609442in}}%
\pgfpathlineto{\pgfqpoint{4.910942in}{1.612126in}}%
\pgfpathlineto{\pgfqpoint{4.911134in}{1.615738in}}%
\pgfpathlineto{\pgfqpoint{4.911519in}{1.610297in}}%
\pgfpathlineto{\pgfqpoint{4.911903in}{1.610921in}}%
\pgfpathlineto{\pgfqpoint{4.912479in}{1.606681in}}%
\pgfpathlineto{\pgfqpoint{4.912672in}{1.611898in}}%
\pgfpathlineto{\pgfqpoint{4.913248in}{1.614499in}}%
\pgfpathlineto{\pgfqpoint{4.913440in}{1.612706in}}%
\pgfpathlineto{\pgfqpoint{4.913632in}{1.609664in}}%
\pgfpathlineto{\pgfqpoint{4.914593in}{1.610317in}}%
\pgfpathlineto{\pgfqpoint{4.916899in}{1.615937in}}%
\pgfpathlineto{\pgfqpoint{4.914978in}{1.609878in}}%
\pgfpathlineto{\pgfqpoint{4.917091in}{1.615711in}}%
\pgfpathlineto{\pgfqpoint{4.917668in}{1.613077in}}%
\pgfpathlineto{\pgfqpoint{4.918437in}{1.609945in}}%
\pgfpathlineto{\pgfqpoint{4.919013in}{1.610748in}}%
\pgfpathlineto{\pgfqpoint{4.919205in}{1.610979in}}%
\pgfpathlineto{\pgfqpoint{4.920358in}{1.602311in}}%
\pgfpathlineto{\pgfqpoint{4.920551in}{1.604873in}}%
\pgfpathlineto{\pgfqpoint{4.920935in}{1.608817in}}%
\pgfpathlineto{\pgfqpoint{4.921511in}{1.606307in}}%
\pgfpathlineto{\pgfqpoint{4.921704in}{1.604938in}}%
\pgfpathlineto{\pgfqpoint{4.922280in}{1.607198in}}%
\pgfpathlineto{\pgfqpoint{4.922472in}{1.606742in}}%
\pgfpathlineto{\pgfqpoint{4.923049in}{1.604937in}}%
\pgfpathlineto{\pgfqpoint{4.923241in}{1.603218in}}%
\pgfpathlineto{\pgfqpoint{4.923817in}{1.607338in}}%
\pgfpathlineto{\pgfqpoint{4.924202in}{1.610738in}}%
\pgfpathlineto{\pgfqpoint{4.924778in}{1.606542in}}%
\pgfpathlineto{\pgfqpoint{4.924970in}{1.603704in}}%
\pgfpathlineto{\pgfqpoint{4.925547in}{1.607321in}}%
\pgfpathlineto{\pgfqpoint{4.925931in}{1.604737in}}%
\pgfpathlineto{\pgfqpoint{4.927276in}{1.610739in}}%
\pgfpathlineto{\pgfqpoint{4.928429in}{1.601484in}}%
\pgfpathlineto{\pgfqpoint{4.928622in}{1.602284in}}%
\pgfpathlineto{\pgfqpoint{4.929967in}{1.611024in}}%
\pgfpathlineto{\pgfqpoint{4.930543in}{1.608247in}}%
\pgfpathlineto{\pgfqpoint{4.930928in}{1.611264in}}%
\pgfpathlineto{\pgfqpoint{4.931120in}{1.611281in}}%
\pgfpathlineto{\pgfqpoint{4.931696in}{1.618731in}}%
\pgfpathlineto{\pgfqpoint{4.932657in}{1.616013in}}%
\pgfpathlineto{\pgfqpoint{4.933234in}{1.613089in}}%
\pgfpathlineto{\pgfqpoint{4.933618in}{1.614858in}}%
\pgfpathlineto{\pgfqpoint{4.933810in}{1.616505in}}%
\pgfpathlineto{\pgfqpoint{4.934387in}{1.613087in}}%
\pgfpathlineto{\pgfqpoint{4.935155in}{1.609764in}}%
\pgfpathlineto{\pgfqpoint{4.934771in}{1.613420in}}%
\pgfpathlineto{\pgfqpoint{4.935732in}{1.611597in}}%
\pgfpathlineto{\pgfqpoint{4.936308in}{1.612510in}}%
\pgfpathlineto{\pgfqpoint{4.936500in}{1.613072in}}%
\pgfpathlineto{\pgfqpoint{4.936885in}{1.611330in}}%
\pgfpathlineto{\pgfqpoint{4.937077in}{1.611564in}}%
\pgfpathlineto{\pgfqpoint{4.937461in}{1.605791in}}%
\pgfpathlineto{\pgfqpoint{4.938422in}{1.607053in}}%
\pgfpathlineto{\pgfqpoint{4.938806in}{1.603097in}}%
\pgfpathlineto{\pgfqpoint{4.939575in}{1.605731in}}%
\pgfpathlineto{\pgfqpoint{4.940152in}{1.608148in}}%
\pgfpathlineto{\pgfqpoint{4.939959in}{1.604022in}}%
\pgfpathlineto{\pgfqpoint{4.940344in}{1.608062in}}%
\pgfpathlineto{\pgfqpoint{4.940536in}{1.604933in}}%
\pgfpathlineto{\pgfqpoint{4.940920in}{1.609773in}}%
\pgfpathlineto{\pgfqpoint{4.941112in}{1.608800in}}%
\pgfpathlineto{\pgfqpoint{4.942458in}{1.618616in}}%
\pgfpathlineto{\pgfqpoint{4.943418in}{1.611235in}}%
\pgfpathlineto{\pgfqpoint{4.943611in}{1.612186in}}%
\pgfpathlineto{\pgfqpoint{4.944956in}{1.617524in}}%
\pgfpathlineto{\pgfqpoint{4.945148in}{1.617120in}}%
\pgfpathlineto{\pgfqpoint{4.945340in}{1.617771in}}%
\pgfpathlineto{\pgfqpoint{4.945532in}{1.617801in}}%
\pgfpathlineto{\pgfqpoint{4.945917in}{1.621641in}}%
\pgfpathlineto{\pgfqpoint{4.946493in}{1.616941in}}%
\pgfpathlineto{\pgfqpoint{4.946685in}{1.620034in}}%
\pgfpathlineto{\pgfqpoint{4.947070in}{1.619959in}}%
\pgfpathlineto{\pgfqpoint{4.948031in}{1.625376in}}%
\pgfpathlineto{\pgfqpoint{4.948223in}{1.622421in}}%
\pgfpathlineto{\pgfqpoint{4.949760in}{1.635832in}}%
\pgfpathlineto{\pgfqpoint{4.950337in}{1.634863in}}%
\pgfpathlineto{\pgfqpoint{4.951297in}{1.630652in}}%
\pgfpathlineto{\pgfqpoint{4.950721in}{1.634979in}}%
\pgfpathlineto{\pgfqpoint{4.951682in}{1.631897in}}%
\pgfpathlineto{\pgfqpoint{4.953411in}{1.642311in}}%
\pgfpathlineto{\pgfqpoint{4.954564in}{1.640554in}}%
\pgfpathlineto{\pgfqpoint{4.954756in}{1.639573in}}%
\pgfpathlineto{\pgfqpoint{4.954949in}{1.639964in}}%
\pgfpathlineto{\pgfqpoint{4.956294in}{1.644580in}}%
\pgfpathlineto{\pgfqpoint{4.956870in}{1.641923in}}%
\pgfpathlineto{\pgfqpoint{4.957062in}{1.645542in}}%
\pgfpathlineto{\pgfqpoint{4.957447in}{1.642736in}}%
\pgfpathlineto{\pgfqpoint{4.960137in}{1.653492in}}%
\pgfpathlineto{\pgfqpoint{4.960714in}{1.649633in}}%
\pgfpathlineto{\pgfqpoint{4.960906in}{1.649139in}}%
\pgfpathlineto{\pgfqpoint{4.961867in}{1.641065in}}%
\pgfpathlineto{\pgfqpoint{4.962251in}{1.643298in}}%
\pgfpathlineto{\pgfqpoint{4.964365in}{1.654771in}}%
\pgfpathlineto{\pgfqpoint{4.964941in}{1.653486in}}%
\pgfpathlineto{\pgfqpoint{4.965133in}{1.650916in}}%
\pgfpathlineto{\pgfqpoint{4.965902in}{1.652798in}}%
\pgfpathlineto{\pgfqpoint{4.966094in}{1.656184in}}%
\pgfpathlineto{\pgfqpoint{4.966479in}{1.652546in}}%
\pgfpathlineto{\pgfqpoint{4.967055in}{1.654529in}}%
\pgfpathlineto{\pgfqpoint{4.968016in}{1.656578in}}%
\pgfpathlineto{\pgfqpoint{4.969361in}{1.662807in}}%
\pgfpathlineto{\pgfqpoint{4.969746in}{1.659528in}}%
\pgfpathlineto{\pgfqpoint{4.970130in}{1.663164in}}%
\pgfpathlineto{\pgfqpoint{4.970322in}{1.664097in}}%
\pgfpathlineto{\pgfqpoint{4.970706in}{1.663798in}}%
\pgfpathlineto{\pgfqpoint{4.970899in}{1.661054in}}%
\pgfpathlineto{\pgfqpoint{4.971859in}{1.662991in}}%
\pgfpathlineto{\pgfqpoint{4.974550in}{1.672071in}}%
\pgfpathlineto{\pgfqpoint{4.974934in}{1.668524in}}%
\pgfpathlineto{\pgfqpoint{4.975126in}{1.666425in}}%
\pgfpathlineto{\pgfqpoint{4.975511in}{1.671157in}}%
\pgfpathlineto{\pgfqpoint{4.975895in}{1.679341in}}%
\pgfpathlineto{\pgfqpoint{4.976664in}{1.675733in}}%
\pgfpathlineto{\pgfqpoint{4.977817in}{1.670808in}}%
\pgfpathlineto{\pgfqpoint{4.978009in}{1.672744in}}%
\pgfpathlineto{\pgfqpoint{4.978201in}{1.670715in}}%
\pgfpathlineto{\pgfqpoint{4.978777in}{1.673613in}}%
\pgfpathlineto{\pgfqpoint{4.978970in}{1.672896in}}%
\pgfpathlineto{\pgfqpoint{4.980123in}{1.676995in}}%
\pgfpathlineto{\pgfqpoint{4.980507in}{1.671926in}}%
\pgfpathlineto{\pgfqpoint{4.981083in}{1.675014in}}%
\pgfpathlineto{\pgfqpoint{4.981276in}{1.676676in}}%
\pgfpathlineto{\pgfqpoint{4.981852in}{1.672957in}}%
\pgfpathlineto{\pgfqpoint{4.982044in}{1.672565in}}%
\pgfpathlineto{\pgfqpoint{4.983389in}{1.678809in}}%
\pgfpathlineto{\pgfqpoint{4.983582in}{1.678781in}}%
\pgfpathlineto{\pgfqpoint{4.983774in}{1.678743in}}%
\pgfpathlineto{\pgfqpoint{4.985119in}{1.671075in}}%
\pgfpathlineto{\pgfqpoint{4.985311in}{1.670946in}}%
\pgfpathlineto{\pgfqpoint{4.985503in}{1.669164in}}%
\pgfpathlineto{\pgfqpoint{4.986272in}{1.671311in}}%
\pgfpathlineto{\pgfqpoint{4.986464in}{1.671519in}}%
\pgfpathlineto{\pgfqpoint{4.986848in}{1.670317in}}%
\pgfpathlineto{\pgfqpoint{4.987041in}{1.670377in}}%
\pgfpathlineto{\pgfqpoint{4.987233in}{1.667405in}}%
\pgfpathlineto{\pgfqpoint{4.988001in}{1.672029in}}%
\pgfpathlineto{\pgfqpoint{4.988194in}{1.670473in}}%
\pgfpathlineto{\pgfqpoint{4.988770in}{1.673626in}}%
\pgfpathlineto{\pgfqpoint{4.989539in}{1.675648in}}%
\pgfpathlineto{\pgfqpoint{4.989154in}{1.672042in}}%
\pgfpathlineto{\pgfqpoint{4.989731in}{1.672887in}}%
\pgfpathlineto{\pgfqpoint{4.991076in}{1.680584in}}%
\pgfpathlineto{\pgfqpoint{4.991460in}{1.684359in}}%
\pgfpathlineto{\pgfqpoint{4.992229in}{1.682260in}}%
\pgfpathlineto{\pgfqpoint{4.992806in}{1.683198in}}%
\pgfpathlineto{\pgfqpoint{4.993190in}{1.684147in}}%
\pgfpathlineto{\pgfqpoint{4.994535in}{1.676644in}}%
\pgfpathlineto{\pgfqpoint{4.994727in}{1.677348in}}%
\pgfpathlineto{\pgfqpoint{4.996265in}{1.688540in}}%
\pgfpathlineto{\pgfqpoint{4.996457in}{1.687591in}}%
\pgfpathlineto{\pgfqpoint{4.996649in}{1.684571in}}%
\pgfpathlineto{\pgfqpoint{4.997418in}{1.689171in}}%
\pgfpathlineto{\pgfqpoint{4.997610in}{1.688291in}}%
\pgfpathlineto{\pgfqpoint{4.997994in}{1.690788in}}%
\pgfpathlineto{\pgfqpoint{4.998379in}{1.688884in}}%
\pgfpathlineto{\pgfqpoint{4.998763in}{1.690235in}}%
\pgfpathlineto{\pgfqpoint{4.999532in}{1.689788in}}%
\pgfpathlineto{\pgfqpoint{4.999916in}{1.685732in}}%
\pgfpathlineto{\pgfqpoint{5.000492in}{1.689553in}}%
\pgfpathlineto{\pgfqpoint{5.001838in}{1.701950in}}%
\pgfpathlineto{\pgfqpoint{5.002030in}{1.701192in}}%
\pgfpathlineto{\pgfqpoint{5.002991in}{1.690558in}}%
\pgfpathlineto{\pgfqpoint{5.003759in}{1.692506in}}%
\pgfpathlineto{\pgfqpoint{5.004720in}{1.696537in}}%
\pgfpathlineto{\pgfqpoint{5.004912in}{1.693549in}}%
\pgfpathlineto{\pgfqpoint{5.005104in}{1.693573in}}%
\pgfpathlineto{\pgfqpoint{5.005489in}{1.691574in}}%
\pgfpathlineto{\pgfqpoint{5.005873in}{1.694251in}}%
\pgfpathlineto{\pgfqpoint{5.006065in}{1.697046in}}%
\pgfpathlineto{\pgfqpoint{5.006642in}{1.693159in}}%
\pgfpathlineto{\pgfqpoint{5.007026in}{1.695365in}}%
\pgfpathlineto{\pgfqpoint{5.007795in}{1.700547in}}%
\pgfpathlineto{\pgfqpoint{5.008371in}{1.698444in}}%
\pgfpathlineto{\pgfqpoint{5.009524in}{1.691285in}}%
\pgfpathlineto{\pgfqpoint{5.009909in}{1.693687in}}%
\pgfpathlineto{\pgfqpoint{5.010869in}{1.700614in}}%
\pgfpathlineto{\pgfqpoint{5.011062in}{1.700243in}}%
\pgfpathlineto{\pgfqpoint{5.011254in}{1.698964in}}%
\pgfpathlineto{\pgfqpoint{5.011638in}{1.701937in}}%
\pgfpathlineto{\pgfqpoint{5.014136in}{1.713194in}}%
\pgfpathlineto{\pgfqpoint{5.012407in}{1.700979in}}%
\pgfpathlineto{\pgfqpoint{5.014713in}{1.710091in}}%
\pgfpathlineto{\pgfqpoint{5.015097in}{1.707313in}}%
\pgfpathlineto{\pgfqpoint{5.015481in}{1.711750in}}%
\pgfpathlineto{\pgfqpoint{5.017019in}{1.719093in}}%
\pgfpathlineto{\pgfqpoint{5.017403in}{1.716186in}}%
\pgfpathlineto{\pgfqpoint{5.017788in}{1.717169in}}%
\pgfpathlineto{\pgfqpoint{5.017980in}{1.715939in}}%
\pgfpathlineto{\pgfqpoint{5.018172in}{1.716277in}}%
\pgfpathlineto{\pgfqpoint{5.018364in}{1.713656in}}%
\pgfpathlineto{\pgfqpoint{5.018941in}{1.718497in}}%
\pgfpathlineto{\pgfqpoint{5.019133in}{1.715655in}}%
\pgfpathlineto{\pgfqpoint{5.019901in}{1.719574in}}%
\pgfpathlineto{\pgfqpoint{5.020478in}{1.719466in}}%
\pgfpathlineto{\pgfqpoint{5.021823in}{1.709256in}}%
\pgfpathlineto{\pgfqpoint{5.022207in}{1.710156in}}%
\pgfpathlineto{\pgfqpoint{5.024129in}{1.702470in}}%
\pgfpathlineto{\pgfqpoint{5.024513in}{1.702616in}}%
\pgfpathlineto{\pgfqpoint{5.025090in}{1.702404in}}%
\pgfpathlineto{\pgfqpoint{5.025859in}{1.705836in}}%
\pgfpathlineto{\pgfqpoint{5.026051in}{1.707115in}}%
\pgfpathlineto{\pgfqpoint{5.026627in}{1.703526in}}%
\pgfpathlineto{\pgfqpoint{5.027204in}{1.700760in}}%
\pgfpathlineto{\pgfqpoint{5.027588in}{1.703817in}}%
\pgfpathlineto{\pgfqpoint{5.027780in}{1.702867in}}%
\pgfpathlineto{\pgfqpoint{5.028549in}{1.705141in}}%
\pgfpathlineto{\pgfqpoint{5.028933in}{1.703272in}}%
\pgfpathlineto{\pgfqpoint{5.030663in}{1.692043in}}%
\pgfpathlineto{\pgfqpoint{5.032200in}{1.701316in}}%
\pgfpathlineto{\pgfqpoint{5.033161in}{1.696993in}}%
\pgfpathlineto{\pgfqpoint{5.033353in}{1.698307in}}%
\pgfpathlineto{\pgfqpoint{5.034890in}{1.701633in}}%
\pgfpathlineto{\pgfqpoint{5.035083in}{1.699711in}}%
\pgfpathlineto{\pgfqpoint{5.035275in}{1.704042in}}%
\pgfpathlineto{\pgfqpoint{5.036236in}{1.703110in}}%
\pgfpathlineto{\pgfqpoint{5.037389in}{1.711380in}}%
\pgfpathlineto{\pgfqpoint{5.037581in}{1.706940in}}%
\pgfpathlineto{\pgfqpoint{5.037773in}{1.705297in}}%
\pgfpathlineto{\pgfqpoint{5.038157in}{1.707428in}}%
\pgfpathlineto{\pgfqpoint{5.038542in}{1.705700in}}%
\pgfpathlineto{\pgfqpoint{5.039118in}{1.708708in}}%
\pgfpathlineto{\pgfqpoint{5.039310in}{1.705206in}}%
\pgfpathlineto{\pgfqpoint{5.039502in}{1.705533in}}%
\pgfpathlineto{\pgfqpoint{5.040079in}{1.701145in}}%
\pgfpathlineto{\pgfqpoint{5.040848in}{1.703751in}}%
\pgfpathlineto{\pgfqpoint{5.041616in}{1.707798in}}%
\pgfpathlineto{\pgfqpoint{5.041232in}{1.703645in}}%
\pgfpathlineto{\pgfqpoint{5.041808in}{1.705277in}}%
\pgfpathlineto{\pgfqpoint{5.043346in}{1.694815in}}%
\pgfpathlineto{\pgfqpoint{5.043730in}{1.697735in}}%
\pgfpathlineto{\pgfqpoint{5.044115in}{1.692707in}}%
\pgfpathlineto{\pgfqpoint{5.044307in}{1.693586in}}%
\pgfpathlineto{\pgfqpoint{5.044499in}{1.692936in}}%
\pgfpathlineto{\pgfqpoint{5.045075in}{1.694477in}}%
\pgfpathlineto{\pgfqpoint{5.045460in}{1.696077in}}%
\pgfpathlineto{\pgfqpoint{5.045844in}{1.695897in}}%
\pgfpathlineto{\pgfqpoint{5.047381in}{1.690117in}}%
\pgfpathlineto{\pgfqpoint{5.049111in}{1.683674in}}%
\pgfpathlineto{\pgfqpoint{5.049495in}{1.686691in}}%
\pgfpathlineto{\pgfqpoint{5.049687in}{1.682974in}}%
\pgfpathlineto{\pgfqpoint{5.051609in}{1.673038in}}%
\pgfpathlineto{\pgfqpoint{5.051801in}{1.674528in}}%
\pgfpathlineto{\pgfqpoint{5.052186in}{1.671477in}}%
\pgfpathlineto{\pgfqpoint{5.052378in}{1.671909in}}%
\pgfpathlineto{\pgfqpoint{5.052762in}{1.668449in}}%
\pgfpathlineto{\pgfqpoint{5.053339in}{1.671883in}}%
\pgfpathlineto{\pgfqpoint{5.053723in}{1.672775in}}%
\pgfpathlineto{\pgfqpoint{5.054107in}{1.675022in}}%
\pgfpathlineto{\pgfqpoint{5.054492in}{1.671476in}}%
\pgfpathlineto{\pgfqpoint{5.055452in}{1.665842in}}%
\pgfpathlineto{\pgfqpoint{5.056029in}{1.666774in}}%
\pgfpathlineto{\pgfqpoint{5.056221in}{1.669067in}}%
\pgfpathlineto{\pgfqpoint{5.056605in}{1.662707in}}%
\pgfpathlineto{\pgfqpoint{5.056798in}{1.660678in}}%
\pgfpathlineto{\pgfqpoint{5.057374in}{1.664723in}}%
\pgfpathlineto{\pgfqpoint{5.057566in}{1.664066in}}%
\pgfpathlineto{\pgfqpoint{5.057951in}{1.664297in}}%
\pgfpathlineto{\pgfqpoint{5.058143in}{1.663298in}}%
\pgfpathlineto{\pgfqpoint{5.059680in}{1.651772in}}%
\pgfpathlineto{\pgfqpoint{5.061025in}{1.657871in}}%
\pgfpathlineto{\pgfqpoint{5.061410in}{1.656697in}}%
\pgfpathlineto{\pgfqpoint{5.061602in}{1.656517in}}%
\pgfpathlineto{\pgfqpoint{5.062178in}{1.653944in}}%
\pgfpathlineto{\pgfqpoint{5.062563in}{1.656583in}}%
\pgfpathlineto{\pgfqpoint{5.062755in}{1.657027in}}%
\pgfpathlineto{\pgfqpoint{5.062947in}{1.655316in}}%
\pgfpathlineto{\pgfqpoint{5.065061in}{1.642509in}}%
\pgfpathlineto{\pgfqpoint{5.065253in}{1.644512in}}%
\pgfpathlineto{\pgfqpoint{5.065445in}{1.646414in}}%
\pgfpathlineto{\pgfqpoint{5.066022in}{1.643514in}}%
\pgfpathlineto{\pgfqpoint{5.067943in}{1.629129in}}%
\pgfpathlineto{\pgfqpoint{5.069481in}{1.623565in}}%
\pgfpathlineto{\pgfqpoint{5.068520in}{1.630123in}}%
\pgfpathlineto{\pgfqpoint{5.069673in}{1.624121in}}%
\pgfpathlineto{\pgfqpoint{5.070057in}{1.629015in}}%
\pgfpathlineto{\pgfqpoint{5.070634in}{1.622489in}}%
\pgfpathlineto{\pgfqpoint{5.070826in}{1.626312in}}%
\pgfpathlineto{\pgfqpoint{5.071210in}{1.624209in}}%
\pgfpathlineto{\pgfqpoint{5.072363in}{1.615939in}}%
\pgfpathlineto{\pgfqpoint{5.071595in}{1.624945in}}%
\pgfpathlineto{\pgfqpoint{5.072555in}{1.618934in}}%
\pgfpathlineto{\pgfqpoint{5.073132in}{1.621237in}}%
\pgfpathlineto{\pgfqpoint{5.074093in}{1.625988in}}%
\pgfpathlineto{\pgfqpoint{5.073516in}{1.621171in}}%
\pgfpathlineto{\pgfqpoint{5.074477in}{1.624791in}}%
\pgfpathlineto{\pgfqpoint{5.075438in}{1.619928in}}%
\pgfpathlineto{\pgfqpoint{5.075822in}{1.623959in}}%
\pgfpathlineto{\pgfqpoint{5.076399in}{1.626197in}}%
\pgfpathlineto{\pgfqpoint{5.077167in}{1.624454in}}%
\pgfpathlineto{\pgfqpoint{5.079089in}{1.620300in}}%
\pgfpathlineto{\pgfqpoint{5.079858in}{1.625685in}}%
\pgfpathlineto{\pgfqpoint{5.080242in}{1.625041in}}%
\pgfpathlineto{\pgfqpoint{5.080434in}{1.623216in}}%
\pgfpathlineto{\pgfqpoint{5.081011in}{1.628288in}}%
\pgfpathlineto{\pgfqpoint{5.081203in}{1.628160in}}%
\pgfpathlineto{\pgfqpoint{5.082740in}{1.617286in}}%
\pgfpathlineto{\pgfqpoint{5.082932in}{1.616524in}}%
\pgfpathlineto{\pgfqpoint{5.083125in}{1.617928in}}%
\pgfpathlineto{\pgfqpoint{5.083317in}{1.620212in}}%
\pgfpathlineto{\pgfqpoint{5.084085in}{1.618682in}}%
\pgfpathlineto{\pgfqpoint{5.084278in}{1.617915in}}%
\pgfpathlineto{\pgfqpoint{5.084470in}{1.619049in}}%
\pgfpathlineto{\pgfqpoint{5.084854in}{1.618357in}}%
\pgfpathlineto{\pgfqpoint{5.085238in}{1.623367in}}%
\pgfpathlineto{\pgfqpoint{5.086007in}{1.619114in}}%
\pgfpathlineto{\pgfqpoint{5.086968in}{1.605930in}}%
\pgfpathlineto{\pgfqpoint{5.087352in}{1.610701in}}%
\pgfpathlineto{\pgfqpoint{5.087544in}{1.612677in}}%
\pgfpathlineto{\pgfqpoint{5.087737in}{1.609533in}}%
\pgfpathlineto{\pgfqpoint{5.088505in}{1.612211in}}%
\pgfpathlineto{\pgfqpoint{5.088890in}{1.611515in}}%
\pgfpathlineto{\pgfqpoint{5.089466in}{1.612823in}}%
\pgfpathlineto{\pgfqpoint{5.090619in}{1.617820in}}%
\pgfpathlineto{\pgfqpoint{5.091388in}{1.616316in}}%
\pgfpathlineto{\pgfqpoint{5.091004in}{1.619988in}}%
\pgfpathlineto{\pgfqpoint{5.091580in}{1.618732in}}%
\pgfpathlineto{\pgfqpoint{5.091964in}{1.617831in}}%
\pgfpathlineto{\pgfqpoint{5.092349in}{1.619802in}}%
\pgfpathlineto{\pgfqpoint{5.093502in}{1.618836in}}%
\pgfpathlineto{\pgfqpoint{5.093886in}{1.617557in}}%
\pgfpathlineto{\pgfqpoint{5.094078in}{1.618880in}}%
\pgfpathlineto{\pgfqpoint{5.094847in}{1.620886in}}%
\pgfpathlineto{\pgfqpoint{5.095039in}{1.618546in}}%
\pgfpathlineto{\pgfqpoint{5.095231in}{1.618652in}}%
\pgfpathlineto{\pgfqpoint{5.096000in}{1.625955in}}%
\pgfpathlineto{\pgfqpoint{5.096384in}{1.621094in}}%
\pgfpathlineto{\pgfqpoint{5.096576in}{1.620827in}}%
\pgfpathlineto{\pgfqpoint{5.097345in}{1.631348in}}%
\pgfpathlineto{\pgfqpoint{5.097922in}{1.628415in}}%
\pgfpathlineto{\pgfqpoint{5.099267in}{1.620117in}}%
\pgfpathlineto{\pgfqpoint{5.100228in}{1.624271in}}%
\pgfpathlineto{\pgfqpoint{5.100420in}{1.622361in}}%
\pgfpathlineto{\pgfqpoint{5.103302in}{1.608171in}}%
\pgfpathlineto{\pgfqpoint{5.103494in}{1.611198in}}%
\pgfpathlineto{\pgfqpoint{5.103879in}{1.607179in}}%
\pgfpathlineto{\pgfqpoint{5.104263in}{1.609762in}}%
\pgfpathlineto{\pgfqpoint{5.104840in}{1.606904in}}%
\pgfpathlineto{\pgfqpoint{5.105032in}{1.609619in}}%
\pgfpathlineto{\pgfqpoint{5.105224in}{1.612691in}}%
\pgfpathlineto{\pgfqpoint{5.105800in}{1.605263in}}%
\pgfpathlineto{\pgfqpoint{5.106377in}{1.600897in}}%
\pgfpathlineto{\pgfqpoint{5.106953in}{1.603040in}}%
\pgfpathlineto{\pgfqpoint{5.108683in}{1.610796in}}%
\pgfpathlineto{\pgfqpoint{5.109452in}{1.614288in}}%
\pgfpathlineto{\pgfqpoint{5.109644in}{1.610378in}}%
\pgfpathlineto{\pgfqpoint{5.110220in}{1.602548in}}%
\pgfpathlineto{\pgfqpoint{5.110797in}{1.608145in}}%
\pgfpathlineto{\pgfqpoint{5.110989in}{1.611687in}}%
\pgfpathlineto{\pgfqpoint{5.111565in}{1.607330in}}%
\pgfpathlineto{\pgfqpoint{5.111758in}{1.608281in}}%
\pgfpathlineto{\pgfqpoint{5.112911in}{1.601547in}}%
\pgfpathlineto{\pgfqpoint{5.113103in}{1.602298in}}%
\pgfpathlineto{\pgfqpoint{5.113679in}{1.601555in}}%
\pgfpathlineto{\pgfqpoint{5.113871in}{1.604098in}}%
\pgfpathlineto{\pgfqpoint{5.114064in}{1.607763in}}%
\pgfpathlineto{\pgfqpoint{5.114832in}{1.606302in}}%
\pgfpathlineto{\pgfqpoint{5.117138in}{1.588399in}}%
\pgfpathlineto{\pgfqpoint{5.119060in}{1.595237in}}%
\pgfpathlineto{\pgfqpoint{5.120405in}{1.581311in}}%
\pgfpathlineto{\pgfqpoint{5.120982in}{1.581535in}}%
\pgfpathlineto{\pgfqpoint{5.123096in}{1.597889in}}%
\pgfpathlineto{\pgfqpoint{5.123672in}{1.596380in}}%
\pgfpathlineto{\pgfqpoint{5.124633in}{1.594486in}}%
\pgfpathlineto{\pgfqpoint{5.125402in}{1.593563in}}%
\pgfpathlineto{\pgfqpoint{5.126170in}{1.601802in}}%
\pgfpathlineto{\pgfqpoint{5.127900in}{1.592888in}}%
\pgfpathlineto{\pgfqpoint{5.128092in}{1.592673in}}%
\pgfpathlineto{\pgfqpoint{5.128284in}{1.593315in}}%
\pgfpathlineto{\pgfqpoint{5.129437in}{1.597252in}}%
\pgfpathlineto{\pgfqpoint{5.129629in}{1.595295in}}%
\pgfpathlineto{\pgfqpoint{5.129821in}{1.594950in}}%
\pgfpathlineto{\pgfqpoint{5.130014in}{1.595621in}}%
\pgfpathlineto{\pgfqpoint{5.131743in}{1.601383in}}%
\pgfpathlineto{\pgfqpoint{5.133088in}{1.596116in}}%
\pgfpathlineto{\pgfqpoint{5.132320in}{1.603653in}}%
\pgfpathlineto{\pgfqpoint{5.133280in}{1.597275in}}%
\pgfpathlineto{\pgfqpoint{5.134433in}{1.592841in}}%
\pgfpathlineto{\pgfqpoint{5.135394in}{1.588549in}}%
\pgfpathlineto{\pgfqpoint{5.134818in}{1.593642in}}%
\pgfpathlineto{\pgfqpoint{5.135586in}{1.591428in}}%
\pgfpathlineto{\pgfqpoint{5.136355in}{1.590732in}}%
\pgfpathlineto{\pgfqpoint{5.137700in}{1.579922in}}%
\pgfpathlineto{\pgfqpoint{5.138277in}{1.583359in}}%
\pgfpathlineto{\pgfqpoint{5.139430in}{1.586945in}}%
\pgfpathlineto{\pgfqpoint{5.139622in}{1.584871in}}%
\pgfpathlineto{\pgfqpoint{5.139814in}{1.581921in}}%
\pgfpathlineto{\pgfqpoint{5.140006in}{1.584936in}}%
\pgfpathlineto{\pgfqpoint{5.140391in}{1.583837in}}%
\pgfpathlineto{\pgfqpoint{5.140583in}{1.587516in}}%
\pgfpathlineto{\pgfqpoint{5.141352in}{1.582648in}}%
\pgfpathlineto{\pgfqpoint{5.141736in}{1.583679in}}%
\pgfpathlineto{\pgfqpoint{5.142505in}{1.579794in}}%
\pgfpathlineto{\pgfqpoint{5.143850in}{1.574390in}}%
\pgfpathlineto{\pgfqpoint{5.144042in}{1.574229in}}%
\pgfpathlineto{\pgfqpoint{5.144426in}{1.576012in}}%
\pgfpathlineto{\pgfqpoint{5.144811in}{1.571851in}}%
\pgfpathlineto{\pgfqpoint{5.145003in}{1.572191in}}%
\pgfpathlineto{\pgfqpoint{5.145195in}{1.571427in}}%
\pgfpathlineto{\pgfqpoint{5.147117in}{1.562758in}}%
\pgfpathlineto{\pgfqpoint{5.147309in}{1.562432in}}%
\pgfpathlineto{\pgfqpoint{5.149423in}{1.575196in}}%
\pgfpathlineto{\pgfqpoint{5.149615in}{1.573704in}}%
\pgfpathlineto{\pgfqpoint{5.150191in}{1.576649in}}%
\pgfpathlineto{\pgfqpoint{5.151921in}{1.583713in}}%
\pgfpathlineto{\pgfqpoint{5.152497in}{1.579002in}}%
\pgfpathlineto{\pgfqpoint{5.153074in}{1.579538in}}%
\pgfpathlineto{\pgfqpoint{5.153266in}{1.583963in}}%
\pgfpathlineto{\pgfqpoint{5.153842in}{1.578802in}}%
\pgfpathlineto{\pgfqpoint{5.154035in}{1.579167in}}%
\pgfpathlineto{\pgfqpoint{5.154611in}{1.578183in}}%
\pgfpathlineto{\pgfqpoint{5.154419in}{1.580578in}}%
\pgfpathlineto{\pgfqpoint{5.154803in}{1.578853in}}%
\pgfpathlineto{\pgfqpoint{5.155188in}{1.582545in}}%
\pgfpathlineto{\pgfqpoint{5.155764in}{1.578100in}}%
\pgfpathlineto{\pgfqpoint{5.155956in}{1.579546in}}%
\pgfpathlineto{\pgfqpoint{5.156725in}{1.577602in}}%
\pgfpathlineto{\pgfqpoint{5.157109in}{1.581592in}}%
\pgfpathlineto{\pgfqpoint{5.157494in}{1.581689in}}%
\pgfpathlineto{\pgfqpoint{5.158070in}{1.583076in}}%
\pgfpathlineto{\pgfqpoint{5.157878in}{1.581215in}}%
\pgfpathlineto{\pgfqpoint{5.158262in}{1.582471in}}%
\pgfpathlineto{\pgfqpoint{5.159607in}{1.574274in}}%
\pgfpathlineto{\pgfqpoint{5.160568in}{1.572596in}}%
\pgfpathlineto{\pgfqpoint{5.159992in}{1.575027in}}%
\pgfpathlineto{\pgfqpoint{5.160953in}{1.573676in}}%
\pgfpathlineto{\pgfqpoint{5.161721in}{1.578584in}}%
\pgfpathlineto{\pgfqpoint{5.162106in}{1.576628in}}%
\pgfpathlineto{\pgfqpoint{5.163835in}{1.569655in}}%
\pgfpathlineto{\pgfqpoint{5.164027in}{1.572019in}}%
\pgfpathlineto{\pgfqpoint{5.166141in}{1.582699in}}%
\pgfpathlineto{\pgfqpoint{5.166718in}{1.580831in}}%
\pgfpathlineto{\pgfqpoint{5.166910in}{1.581126in}}%
\pgfpathlineto{\pgfqpoint{5.167102in}{1.580343in}}%
\pgfpathlineto{\pgfqpoint{5.167294in}{1.578827in}}%
\pgfpathlineto{\pgfqpoint{5.167486in}{1.583145in}}%
\pgfpathlineto{\pgfqpoint{5.168255in}{1.579375in}}%
\pgfpathlineto{\pgfqpoint{5.168832in}{1.583199in}}%
\pgfpathlineto{\pgfqpoint{5.169216in}{1.579966in}}%
\pgfpathlineto{\pgfqpoint{5.169408in}{1.577913in}}%
\pgfpathlineto{\pgfqpoint{5.169792in}{1.584585in}}%
\pgfpathlineto{\pgfqpoint{5.170177in}{1.581036in}}%
\pgfpathlineto{\pgfqpoint{5.170753in}{1.584046in}}%
\pgfpathlineto{\pgfqpoint{5.171330in}{1.582403in}}%
\pgfpathlineto{\pgfqpoint{5.172291in}{1.587077in}}%
\pgfpathlineto{\pgfqpoint{5.171906in}{1.581895in}}%
\pgfpathlineto{\pgfqpoint{5.172483in}{1.584989in}}%
\pgfpathlineto{\pgfqpoint{5.173251in}{1.581201in}}%
\pgfpathlineto{\pgfqpoint{5.173636in}{1.583751in}}%
\pgfpathlineto{\pgfqpoint{5.173828in}{1.583848in}}%
\pgfpathlineto{\pgfqpoint{5.174981in}{1.578228in}}%
\pgfpathlineto{\pgfqpoint{5.175173in}{1.580055in}}%
\pgfpathlineto{\pgfqpoint{5.175365in}{1.580139in}}%
\pgfpathlineto{\pgfqpoint{5.175942in}{1.584744in}}%
\pgfpathlineto{\pgfqpoint{5.176518in}{1.581544in}}%
\pgfpathlineto{\pgfqpoint{5.177287in}{1.582341in}}%
\pgfpathlineto{\pgfqpoint{5.178824in}{1.589128in}}%
\pgfpathlineto{\pgfqpoint{5.179016in}{1.589280in}}%
\pgfpathlineto{\pgfqpoint{5.180362in}{1.593323in}}%
\pgfpathlineto{\pgfqpoint{5.180554in}{1.593143in}}%
\pgfpathlineto{\pgfqpoint{5.181130in}{1.581772in}}%
\pgfpathlineto{\pgfqpoint{5.181899in}{1.584218in}}%
\pgfpathlineto{\pgfqpoint{5.182091in}{1.584451in}}%
\pgfpathlineto{\pgfqpoint{5.182283in}{1.583535in}}%
\pgfpathlineto{\pgfqpoint{5.184205in}{1.574997in}}%
\pgfpathlineto{\pgfqpoint{5.185742in}{1.581289in}}%
\pgfpathlineto{\pgfqpoint{5.185934in}{1.580688in}}%
\pgfpathlineto{\pgfqpoint{5.186127in}{1.577590in}}%
\pgfpathlineto{\pgfqpoint{5.186703in}{1.581692in}}%
\pgfpathlineto{\pgfqpoint{5.187280in}{1.587387in}}%
\pgfpathlineto{\pgfqpoint{5.188048in}{1.585661in}}%
\pgfpathlineto{\pgfqpoint{5.188625in}{1.579663in}}%
\pgfpathlineto{\pgfqpoint{5.189201in}{1.582944in}}%
\pgfpathlineto{\pgfqpoint{5.189394in}{1.582009in}}%
\pgfpathlineto{\pgfqpoint{5.189970in}{1.589203in}}%
\pgfpathlineto{\pgfqpoint{5.190547in}{1.583277in}}%
\pgfpathlineto{\pgfqpoint{5.190931in}{1.583734in}}%
\pgfpathlineto{\pgfqpoint{5.191123in}{1.581989in}}%
\pgfpathlineto{\pgfqpoint{5.192468in}{1.577560in}}%
\pgfpathlineto{\pgfqpoint{5.193237in}{1.576874in}}%
\pgfpathlineto{\pgfqpoint{5.193621in}{1.580492in}}%
\pgfpathlineto{\pgfqpoint{5.195351in}{1.571086in}}%
\pgfpathlineto{\pgfqpoint{5.195543in}{1.571353in}}%
\pgfpathlineto{\pgfqpoint{5.195735in}{1.568623in}}%
\pgfpathlineto{\pgfqpoint{5.196504in}{1.571809in}}%
\pgfpathlineto{\pgfqpoint{5.196696in}{1.573830in}}%
\pgfpathlineto{\pgfqpoint{5.197080in}{1.569462in}}%
\pgfpathlineto{\pgfqpoint{5.197465in}{1.566541in}}%
\pgfpathlineto{\pgfqpoint{5.198233in}{1.567920in}}%
\pgfpathlineto{\pgfqpoint{5.199002in}{1.569702in}}%
\pgfpathlineto{\pgfqpoint{5.198618in}{1.567073in}}%
\pgfpathlineto{\pgfqpoint{5.199386in}{1.568294in}}%
\pgfpathlineto{\pgfqpoint{5.199771in}{1.565468in}}%
\pgfpathlineto{\pgfqpoint{5.200731in}{1.566431in}}%
\pgfpathlineto{\pgfqpoint{5.202269in}{1.571096in}}%
\pgfpathlineto{\pgfqpoint{5.203614in}{1.581542in}}%
\pgfpathlineto{\pgfqpoint{5.204383in}{1.578398in}}%
\pgfpathlineto{\pgfqpoint{5.206304in}{1.573638in}}%
\pgfpathlineto{\pgfqpoint{5.204767in}{1.579010in}}%
\pgfpathlineto{\pgfqpoint{5.206496in}{1.574059in}}%
\pgfpathlineto{\pgfqpoint{5.206881in}{1.572981in}}%
\pgfpathlineto{\pgfqpoint{5.208610in}{1.583568in}}%
\pgfpathlineto{\pgfqpoint{5.209571in}{1.576763in}}%
\pgfpathlineto{\pgfqpoint{5.209955in}{1.578596in}}%
\pgfpathlineto{\pgfqpoint{5.210340in}{1.578796in}}%
\pgfpathlineto{\pgfqpoint{5.210532in}{1.577334in}}%
\pgfpathlineto{\pgfqpoint{5.211108in}{1.580483in}}%
\pgfpathlineto{\pgfqpoint{5.211301in}{1.580514in}}%
\pgfpathlineto{\pgfqpoint{5.211493in}{1.579333in}}%
\pgfpathlineto{\pgfqpoint{5.212069in}{1.582627in}}%
\pgfpathlineto{\pgfqpoint{5.212454in}{1.581641in}}%
\pgfpathlineto{\pgfqpoint{5.213799in}{1.587785in}}%
\pgfpathlineto{\pgfqpoint{5.214760in}{1.580079in}}%
\pgfpathlineto{\pgfqpoint{5.216681in}{1.568330in}}%
\pgfpathlineto{\pgfqpoint{5.217066in}{1.570463in}}%
\pgfpathlineto{\pgfqpoint{5.217258in}{1.575079in}}%
\pgfpathlineto{\pgfqpoint{5.218219in}{1.574685in}}%
\pgfpathlineto{\pgfqpoint{5.218411in}{1.574466in}}%
\pgfpathlineto{\pgfqpoint{5.218603in}{1.577363in}}%
\pgfpathlineto{\pgfqpoint{5.219564in}{1.575593in}}%
\pgfpathlineto{\pgfqpoint{5.221101in}{1.579079in}}%
\pgfpathlineto{\pgfqpoint{5.222062in}{1.578107in}}%
\pgfpathlineto{\pgfqpoint{5.221486in}{1.580387in}}%
\pgfpathlineto{\pgfqpoint{5.222254in}{1.578227in}}%
\pgfpathlineto{\pgfqpoint{5.222446in}{1.582036in}}%
\pgfpathlineto{\pgfqpoint{5.223023in}{1.576286in}}%
\pgfpathlineto{\pgfqpoint{5.223407in}{1.571497in}}%
\pgfpathlineto{\pgfqpoint{5.224176in}{1.575052in}}%
\pgfpathlineto{\pgfqpoint{5.226098in}{1.589229in}}%
\pgfpathlineto{\pgfqpoint{5.227443in}{1.586975in}}%
\pgfpathlineto{\pgfqpoint{5.227635in}{1.586342in}}%
\pgfpathlineto{\pgfqpoint{5.227827in}{1.588443in}}%
\pgfpathlineto{\pgfqpoint{5.228019in}{1.589329in}}%
\pgfpathlineto{\pgfqpoint{5.228404in}{1.586076in}}%
\pgfpathlineto{\pgfqpoint{5.228596in}{1.586871in}}%
\pgfpathlineto{\pgfqpoint{5.228788in}{1.587285in}}%
\pgfpathlineto{\pgfqpoint{5.228980in}{1.586295in}}%
\pgfpathlineto{\pgfqpoint{5.229941in}{1.580663in}}%
\pgfpathlineto{\pgfqpoint{5.230517in}{1.581586in}}%
\pgfpathlineto{\pgfqpoint{5.232247in}{1.586048in}}%
\pgfpathlineto{\pgfqpoint{5.232631in}{1.582315in}}%
\pgfpathlineto{\pgfqpoint{5.233016in}{1.587169in}}%
\pgfpathlineto{\pgfqpoint{5.233784in}{1.590996in}}%
\pgfpathlineto{\pgfqpoint{5.234937in}{1.599663in}}%
\pgfpathlineto{\pgfqpoint{5.235129in}{1.599650in}}%
\pgfpathlineto{\pgfqpoint{5.235706in}{1.597945in}}%
\pgfpathlineto{\pgfqpoint{5.236475in}{1.606873in}}%
\pgfpathlineto{\pgfqpoint{5.237051in}{1.603639in}}%
\pgfpathlineto{\pgfqpoint{5.237243in}{1.603276in}}%
\pgfpathlineto{\pgfqpoint{5.237628in}{1.599174in}}%
\pgfpathlineto{\pgfqpoint{5.238204in}{1.601095in}}%
\pgfpathlineto{\pgfqpoint{5.239165in}{1.608128in}}%
\pgfpathlineto{\pgfqpoint{5.239549in}{1.606038in}}%
\pgfpathlineto{\pgfqpoint{5.239742in}{1.604694in}}%
\pgfpathlineto{\pgfqpoint{5.239934in}{1.606688in}}%
\pgfpathlineto{\pgfqpoint{5.240895in}{1.614365in}}%
\pgfpathlineto{\pgfqpoint{5.241087in}{1.609717in}}%
\pgfpathlineto{\pgfqpoint{5.241279in}{1.609534in}}%
\pgfpathlineto{\pgfqpoint{5.242624in}{1.600836in}}%
\pgfpathlineto{\pgfqpoint{5.242816in}{1.602818in}}%
\pgfpathlineto{\pgfqpoint{5.243393in}{1.598222in}}%
\pgfpathlineto{\pgfqpoint{5.243969in}{1.597468in}}%
\pgfpathlineto{\pgfqpoint{5.244354in}{1.599831in}}%
\pgfpathlineto{\pgfqpoint{5.245891in}{1.590451in}}%
\pgfpathlineto{\pgfqpoint{5.246275in}{1.591390in}}%
\pgfpathlineto{\pgfqpoint{5.246660in}{1.595722in}}%
\pgfpathlineto{\pgfqpoint{5.247236in}{1.591583in}}%
\pgfpathlineto{\pgfqpoint{5.247620in}{1.587925in}}%
\pgfpathlineto{\pgfqpoint{5.248389in}{1.588942in}}%
\pgfpathlineto{\pgfqpoint{5.248581in}{1.589223in}}%
\pgfpathlineto{\pgfqpoint{5.248773in}{1.586911in}}%
\pgfpathlineto{\pgfqpoint{5.249350in}{1.592568in}}%
\pgfpathlineto{\pgfqpoint{5.249542in}{1.593946in}}%
\pgfpathlineto{\pgfqpoint{5.249734in}{1.593536in}}%
\pgfpathlineto{\pgfqpoint{5.250887in}{1.585888in}}%
\pgfpathlineto{\pgfqpoint{5.251079in}{1.586156in}}%
\pgfpathlineto{\pgfqpoint{5.251272in}{1.587684in}}%
\pgfpathlineto{\pgfqpoint{5.251656in}{1.583875in}}%
\pgfpathlineto{\pgfqpoint{5.253001in}{1.578110in}}%
\pgfpathlineto{\pgfqpoint{5.253578in}{1.584109in}}%
\pgfpathlineto{\pgfqpoint{5.254154in}{1.583759in}}%
\pgfpathlineto{\pgfqpoint{5.254923in}{1.582283in}}%
\pgfpathlineto{\pgfqpoint{5.255115in}{1.583702in}}%
\pgfpathlineto{\pgfqpoint{5.255884in}{1.589946in}}%
\pgfpathlineto{\pgfqpoint{5.256268in}{1.586861in}}%
\pgfpathlineto{\pgfqpoint{5.258190in}{1.578881in}}%
\pgfpathlineto{\pgfqpoint{5.259150in}{1.587365in}}%
\pgfpathlineto{\pgfqpoint{5.260303in}{1.583591in}}%
\pgfpathlineto{\pgfqpoint{5.260688in}{1.581430in}}%
\pgfpathlineto{\pgfqpoint{5.261264in}{1.583599in}}%
\pgfpathlineto{\pgfqpoint{5.261456in}{1.584144in}}%
\pgfpathlineto{\pgfqpoint{5.261841in}{1.582811in}}%
\pgfpathlineto{\pgfqpoint{5.262610in}{1.583736in}}%
\pgfpathlineto{\pgfqpoint{5.263378in}{1.579694in}}%
\pgfpathlineto{\pgfqpoint{5.263570in}{1.581874in}}%
\pgfpathlineto{\pgfqpoint{5.264339in}{1.579393in}}%
\pgfpathlineto{\pgfqpoint{5.264531in}{1.581331in}}%
\pgfpathlineto{\pgfqpoint{5.265492in}{1.574683in}}%
\pgfpathlineto{\pgfqpoint{5.265876in}{1.575368in}}%
\pgfpathlineto{\pgfqpoint{5.266453in}{1.580542in}}%
\pgfpathlineto{\pgfqpoint{5.266261in}{1.575155in}}%
\pgfpathlineto{\pgfqpoint{5.267222in}{1.576436in}}%
\pgfpathlineto{\pgfqpoint{5.267414in}{1.575456in}}%
\pgfpathlineto{\pgfqpoint{5.267990in}{1.577381in}}%
\pgfpathlineto{\pgfqpoint{5.268759in}{1.579611in}}%
\pgfpathlineto{\pgfqpoint{5.268951in}{1.577085in}}%
\pgfpathlineto{\pgfqpoint{5.269143in}{1.577067in}}%
\pgfpathlineto{\pgfqpoint{5.270296in}{1.569593in}}%
\pgfpathlineto{\pgfqpoint{5.270681in}{1.573614in}}%
\pgfpathlineto{\pgfqpoint{5.271065in}{1.573977in}}%
\pgfpathlineto{\pgfqpoint{5.271257in}{1.572332in}}%
\pgfpathlineto{\pgfqpoint{5.271834in}{1.576429in}}%
\pgfpathlineto{\pgfqpoint{5.272794in}{1.583492in}}%
\pgfpathlineto{\pgfqpoint{5.273179in}{1.582093in}}%
\pgfpathlineto{\pgfqpoint{5.274332in}{1.578918in}}%
\pgfpathlineto{\pgfqpoint{5.274524in}{1.580001in}}%
\pgfpathlineto{\pgfqpoint{5.274908in}{1.576777in}}%
\pgfpathlineto{\pgfqpoint{5.275100in}{1.578002in}}%
\pgfpathlineto{\pgfqpoint{5.275293in}{1.576711in}}%
\pgfpathlineto{\pgfqpoint{5.275869in}{1.580348in}}%
\pgfpathlineto{\pgfqpoint{5.276061in}{1.581990in}}%
\pgfpathlineto{\pgfqpoint{5.276638in}{1.579320in}}%
\pgfpathlineto{\pgfqpoint{5.276830in}{1.580123in}}%
\pgfpathlineto{\pgfqpoint{5.279136in}{1.567260in}}%
\pgfpathlineto{\pgfqpoint{5.279328in}{1.567825in}}%
\pgfpathlineto{\pgfqpoint{5.280481in}{1.576339in}}%
\pgfpathlineto{\pgfqpoint{5.280673in}{1.575769in}}%
\pgfpathlineto{\pgfqpoint{5.280865in}{1.575834in}}%
\pgfpathlineto{\pgfqpoint{5.281442in}{1.572124in}}%
\pgfpathlineto{\pgfqpoint{5.282018in}{1.572633in}}%
\pgfpathlineto{\pgfqpoint{5.282403in}{1.570149in}}%
\pgfpathlineto{\pgfqpoint{5.282787in}{1.574846in}}%
\pgfpathlineto{\pgfqpoint{5.282979in}{1.571703in}}%
\pgfpathlineto{\pgfqpoint{5.283364in}{1.576640in}}%
\pgfpathlineto{\pgfqpoint{5.283748in}{1.575264in}}%
\pgfpathlineto{\pgfqpoint{5.285477in}{1.586177in}}%
\pgfpathlineto{\pgfqpoint{5.285862in}{1.590444in}}%
\pgfpathlineto{\pgfqpoint{5.286246in}{1.587229in}}%
\pgfpathlineto{\pgfqpoint{5.286438in}{1.583183in}}%
\pgfpathlineto{\pgfqpoint{5.287399in}{1.584940in}}%
\pgfpathlineto{\pgfqpoint{5.287591in}{1.584180in}}%
\pgfpathlineto{\pgfqpoint{5.287976in}{1.585860in}}%
\pgfpathlineto{\pgfqpoint{5.288168in}{1.587746in}}%
\pgfpathlineto{\pgfqpoint{5.288937in}{1.584832in}}%
\pgfpathlineto{\pgfqpoint{5.289129in}{1.587356in}}%
\pgfpathlineto{\pgfqpoint{5.289705in}{1.589225in}}%
\pgfpathlineto{\pgfqpoint{5.291050in}{1.582103in}}%
\pgfpathlineto{\pgfqpoint{5.291243in}{1.581749in}}%
\pgfpathlineto{\pgfqpoint{5.291627in}{1.575346in}}%
\pgfpathlineto{\pgfqpoint{5.292203in}{1.585713in}}%
\pgfpathlineto{\pgfqpoint{5.292396in}{1.584623in}}%
\pgfpathlineto{\pgfqpoint{5.292780in}{1.588647in}}%
\pgfpathlineto{\pgfqpoint{5.295086in}{1.608535in}}%
\pgfpathlineto{\pgfqpoint{5.295278in}{1.608400in}}%
\pgfpathlineto{\pgfqpoint{5.296623in}{1.597678in}}%
\pgfpathlineto{\pgfqpoint{5.297776in}{1.602762in}}%
\pgfpathlineto{\pgfqpoint{5.298353in}{1.601462in}}%
\pgfpathlineto{\pgfqpoint{5.298737in}{1.597247in}}%
\pgfpathlineto{\pgfqpoint{5.299314in}{1.601547in}}%
\pgfpathlineto{\pgfqpoint{5.300851in}{1.605545in}}%
\pgfpathlineto{\pgfqpoint{5.301043in}{1.605129in}}%
\pgfpathlineto{\pgfqpoint{5.302388in}{1.600499in}}%
\pgfpathlineto{\pgfqpoint{5.303541in}{1.592282in}}%
\pgfpathlineto{\pgfqpoint{5.303926in}{1.595869in}}%
\pgfusepath{stroke}%
\end{pgfscope}%
\begin{pgfscope}%
\pgfpathrectangle{\pgfqpoint{3.286364in}{0.660000in}}{\pgfqpoint{2.113636in}{2.100000in}}%
\pgfusepath{clip}%
\pgfsetroundcap%
\pgfsetroundjoin%
\pgfsetlinewidth{0.602250pt}%
\definecolor{currentstroke}{rgb}{0.650980,0.337255,0.156863}%
\pgfsetstrokecolor{currentstroke}%
\pgfsetdash{}{0pt}%
\pgfpathmoveto{\pgfqpoint{3.382438in}{1.752994in}}%
\pgfpathlineto{\pgfqpoint{3.382630in}{1.755767in}}%
\pgfpathlineto{\pgfqpoint{3.383015in}{1.751812in}}%
\pgfpathlineto{\pgfqpoint{3.383207in}{1.752589in}}%
\pgfpathlineto{\pgfqpoint{3.384360in}{1.742059in}}%
\pgfpathlineto{\pgfqpoint{3.384744in}{1.746361in}}%
\pgfpathlineto{\pgfqpoint{3.385321in}{1.744046in}}%
\pgfpathlineto{\pgfqpoint{3.386089in}{1.739674in}}%
\pgfpathlineto{\pgfqpoint{3.386281in}{1.742440in}}%
\pgfpathlineto{\pgfqpoint{3.387819in}{1.750429in}}%
\pgfpathlineto{\pgfqpoint{3.388780in}{1.744422in}}%
\pgfpathlineto{\pgfqpoint{3.389164in}{1.746261in}}%
\pgfpathlineto{\pgfqpoint{3.389356in}{1.745675in}}%
\pgfpathlineto{\pgfqpoint{3.389548in}{1.747925in}}%
\pgfpathlineto{\pgfqpoint{3.389933in}{1.747332in}}%
\pgfpathlineto{\pgfqpoint{3.390125in}{1.748136in}}%
\pgfpathlineto{\pgfqpoint{3.390317in}{1.745972in}}%
\pgfpathlineto{\pgfqpoint{3.392046in}{1.735556in}}%
\pgfpathlineto{\pgfqpoint{3.392623in}{1.735074in}}%
\pgfpathlineto{\pgfqpoint{3.394545in}{1.747186in}}%
\pgfpathlineto{\pgfqpoint{3.396658in}{1.735455in}}%
\pgfpathlineto{\pgfqpoint{3.397043in}{1.736293in}}%
\pgfpathlineto{\pgfqpoint{3.397427in}{1.742659in}}%
\pgfpathlineto{\pgfqpoint{3.398004in}{1.738717in}}%
\pgfpathlineto{\pgfqpoint{3.398964in}{1.735478in}}%
\pgfpathlineto{\pgfqpoint{3.399733in}{1.735349in}}%
\pgfpathlineto{\pgfqpoint{3.400310in}{1.740434in}}%
\pgfpathlineto{\pgfqpoint{3.400502in}{1.740195in}}%
\pgfpathlineto{\pgfqpoint{3.400886in}{1.742582in}}%
\pgfpathlineto{\pgfqpoint{3.401270in}{1.738190in}}%
\pgfpathlineto{\pgfqpoint{3.402039in}{1.739212in}}%
\pgfpathlineto{\pgfqpoint{3.402231in}{1.738607in}}%
\pgfpathlineto{\pgfqpoint{3.402423in}{1.737854in}}%
\pgfpathlineto{\pgfqpoint{3.402616in}{1.741090in}}%
\pgfpathlineto{\pgfqpoint{3.403000in}{1.739643in}}%
\pgfpathlineto{\pgfqpoint{3.403576in}{1.743771in}}%
\pgfpathlineto{\pgfqpoint{3.403961in}{1.741326in}}%
\pgfpathlineto{\pgfqpoint{3.406459in}{1.724439in}}%
\pgfpathlineto{\pgfqpoint{3.407036in}{1.725317in}}%
\pgfpathlineto{\pgfqpoint{3.407420in}{1.723805in}}%
\pgfpathlineto{\pgfqpoint{3.407804in}{1.723188in}}%
\pgfpathlineto{\pgfqpoint{3.408189in}{1.717394in}}%
\pgfpathlineto{\pgfqpoint{3.408957in}{1.721964in}}%
\pgfpathlineto{\pgfqpoint{3.409149in}{1.721094in}}%
\pgfpathlineto{\pgfqpoint{3.409534in}{1.729281in}}%
\pgfpathlineto{\pgfqpoint{3.410302in}{1.728812in}}%
\pgfpathlineto{\pgfqpoint{3.411648in}{1.721737in}}%
\pgfpathlineto{\pgfqpoint{3.412224in}{1.723758in}}%
\pgfpathlineto{\pgfqpoint{3.412416in}{1.722607in}}%
\pgfpathlineto{\pgfqpoint{3.413377in}{1.723795in}}%
\pgfpathlineto{\pgfqpoint{3.413569in}{1.720232in}}%
\pgfpathlineto{\pgfqpoint{3.414530in}{1.729125in}}%
\pgfpathlineto{\pgfqpoint{3.415107in}{1.726975in}}%
\pgfpathlineto{\pgfqpoint{3.415491in}{1.727938in}}%
\pgfpathlineto{\pgfqpoint{3.416067in}{1.731590in}}%
\pgfpathlineto{\pgfqpoint{3.416452in}{1.727567in}}%
\pgfpathlineto{\pgfqpoint{3.417220in}{1.721741in}}%
\pgfpathlineto{\pgfqpoint{3.418181in}{1.723290in}}%
\pgfpathlineto{\pgfqpoint{3.418566in}{1.728414in}}%
\pgfpathlineto{\pgfqpoint{3.419142in}{1.724586in}}%
\pgfpathlineto{\pgfqpoint{3.419526in}{1.722769in}}%
\pgfpathlineto{\pgfqpoint{3.419911in}{1.723158in}}%
\pgfpathlineto{\pgfqpoint{3.420103in}{1.725581in}}%
\pgfpathlineto{\pgfqpoint{3.420295in}{1.721264in}}%
\pgfpathlineto{\pgfqpoint{3.421064in}{1.724247in}}%
\pgfpathlineto{\pgfqpoint{3.421832in}{1.720674in}}%
\pgfpathlineto{\pgfqpoint{3.422025in}{1.721997in}}%
\pgfpathlineto{\pgfqpoint{3.423370in}{1.729086in}}%
\pgfpathlineto{\pgfqpoint{3.423754in}{1.730148in}}%
\pgfpathlineto{\pgfqpoint{3.424138in}{1.729387in}}%
\pgfpathlineto{\pgfqpoint{3.425099in}{1.733254in}}%
\pgfpathlineto{\pgfqpoint{3.426060in}{1.730416in}}%
\pgfpathlineto{\pgfqpoint{3.425676in}{1.733799in}}%
\pgfpathlineto{\pgfqpoint{3.426252in}{1.731994in}}%
\pgfpathlineto{\pgfqpoint{3.426829in}{1.735939in}}%
\pgfpathlineto{\pgfqpoint{3.427213in}{1.735362in}}%
\pgfpathlineto{\pgfqpoint{3.427982in}{1.729338in}}%
\pgfpathlineto{\pgfqpoint{3.428366in}{1.732190in}}%
\pgfpathlineto{\pgfqpoint{3.429711in}{1.738027in}}%
\pgfpathlineto{\pgfqpoint{3.428750in}{1.730734in}}%
\pgfpathlineto{\pgfqpoint{3.429904in}{1.736067in}}%
\pgfpathlineto{\pgfqpoint{3.430288in}{1.732992in}}%
\pgfpathlineto{\pgfqpoint{3.432210in}{1.721614in}}%
\pgfpathlineto{\pgfqpoint{3.432402in}{1.721984in}}%
\pgfpathlineto{\pgfqpoint{3.432786in}{1.723341in}}%
\pgfpathlineto{\pgfqpoint{3.432978in}{1.720540in}}%
\pgfpathlineto{\pgfqpoint{3.433747in}{1.715955in}}%
\pgfpathlineto{\pgfqpoint{3.434131in}{1.719801in}}%
\pgfpathlineto{\pgfqpoint{3.434323in}{1.721578in}}%
\pgfpathlineto{\pgfqpoint{3.434708in}{1.716655in}}%
\pgfpathlineto{\pgfqpoint{3.434900in}{1.716813in}}%
\pgfpathlineto{\pgfqpoint{3.436822in}{1.729521in}}%
\pgfpathlineto{\pgfqpoint{3.437206in}{1.726683in}}%
\pgfpathlineto{\pgfqpoint{3.437590in}{1.724127in}}%
\pgfpathlineto{\pgfqpoint{3.437782in}{1.724205in}}%
\pgfpathlineto{\pgfqpoint{3.439320in}{1.737412in}}%
\pgfpathlineto{\pgfqpoint{3.439896in}{1.740217in}}%
\pgfpathlineto{\pgfqpoint{3.440281in}{1.736256in}}%
\pgfpathlineto{\pgfqpoint{3.440473in}{1.736379in}}%
\pgfpathlineto{\pgfqpoint{3.440665in}{1.735675in}}%
\pgfpathlineto{\pgfqpoint{3.441049in}{1.744336in}}%
\pgfpathlineto{\pgfqpoint{3.441818in}{1.736682in}}%
\pgfpathlineto{\pgfqpoint{3.442010in}{1.737029in}}%
\pgfpathlineto{\pgfqpoint{3.442202in}{1.736744in}}%
\pgfpathlineto{\pgfqpoint{3.442587in}{1.731106in}}%
\pgfpathlineto{\pgfqpoint{3.443355in}{1.735444in}}%
\pgfpathlineto{\pgfqpoint{3.444700in}{1.745192in}}%
\pgfpathlineto{\pgfqpoint{3.444893in}{1.744716in}}%
\pgfpathlineto{\pgfqpoint{3.446238in}{1.739019in}}%
\pgfpathlineto{\pgfqpoint{3.446814in}{1.733522in}}%
\pgfpathlineto{\pgfqpoint{3.447583in}{1.736052in}}%
\pgfpathlineto{\pgfqpoint{3.449120in}{1.746421in}}%
\pgfpathlineto{\pgfqpoint{3.449505in}{1.743978in}}%
\pgfpathlineto{\pgfqpoint{3.449697in}{1.743599in}}%
\pgfpathlineto{\pgfqpoint{3.449889in}{1.744014in}}%
\pgfpathlineto{\pgfqpoint{3.450273in}{1.748484in}}%
\pgfpathlineto{\pgfqpoint{3.450850in}{1.745729in}}%
\pgfpathlineto{\pgfqpoint{3.451042in}{1.743830in}}%
\pgfpathlineto{\pgfqpoint{3.451426in}{1.747885in}}%
\pgfpathlineto{\pgfqpoint{3.451811in}{1.750869in}}%
\pgfpathlineto{\pgfqpoint{3.452003in}{1.746431in}}%
\pgfpathlineto{\pgfqpoint{3.452387in}{1.746698in}}%
\pgfpathlineto{\pgfqpoint{3.453540in}{1.754176in}}%
\pgfpathlineto{\pgfqpoint{3.454501in}{1.751636in}}%
\pgfpathlineto{\pgfqpoint{3.454693in}{1.749180in}}%
\pgfpathlineto{\pgfqpoint{3.455462in}{1.750738in}}%
\pgfpathlineto{\pgfqpoint{3.457191in}{1.763976in}}%
\pgfpathlineto{\pgfqpoint{3.457576in}{1.762491in}}%
\pgfpathlineto{\pgfqpoint{3.457768in}{1.763036in}}%
\pgfpathlineto{\pgfqpoint{3.457960in}{1.765279in}}%
\pgfpathlineto{\pgfqpoint{3.458537in}{1.762631in}}%
\pgfpathlineto{\pgfqpoint{3.458729in}{1.763939in}}%
\pgfpathlineto{\pgfqpoint{3.460266in}{1.754206in}}%
\pgfpathlineto{\pgfqpoint{3.461035in}{1.757973in}}%
\pgfpathlineto{\pgfqpoint{3.461419in}{1.757360in}}%
\pgfpathlineto{\pgfqpoint{3.461803in}{1.754610in}}%
\pgfpathlineto{\pgfqpoint{3.461996in}{1.758704in}}%
\pgfpathlineto{\pgfqpoint{3.462188in}{1.757229in}}%
\pgfpathlineto{\pgfqpoint{3.462572in}{1.764150in}}%
\pgfpathlineto{\pgfqpoint{3.463149in}{1.758802in}}%
\pgfpathlineto{\pgfqpoint{3.464302in}{1.756076in}}%
\pgfpathlineto{\pgfqpoint{3.464686in}{1.758305in}}%
\pgfpathlineto{\pgfqpoint{3.465070in}{1.754141in}}%
\pgfpathlineto{\pgfqpoint{3.466031in}{1.751773in}}%
\pgfpathlineto{\pgfqpoint{3.466223in}{1.752586in}}%
\pgfpathlineto{\pgfqpoint{3.466800in}{1.756905in}}%
\pgfpathlineto{\pgfqpoint{3.467184in}{1.752777in}}%
\pgfpathlineto{\pgfqpoint{3.467568in}{1.749166in}}%
\pgfpathlineto{\pgfqpoint{3.468145in}{1.753911in}}%
\pgfpathlineto{\pgfqpoint{3.468337in}{1.752351in}}%
\pgfpathlineto{\pgfqpoint{3.468529in}{1.754049in}}%
\pgfpathlineto{\pgfqpoint{3.468721in}{1.750554in}}%
\pgfpathlineto{\pgfqpoint{3.469298in}{1.750992in}}%
\pgfpathlineto{\pgfqpoint{3.469682in}{1.749440in}}%
\pgfpathlineto{\pgfqpoint{3.469874in}{1.752849in}}%
\pgfpathlineto{\pgfqpoint{3.470067in}{1.752423in}}%
\pgfpathlineto{\pgfqpoint{3.470451in}{1.752463in}}%
\pgfpathlineto{\pgfqpoint{3.470643in}{1.750256in}}%
\pgfpathlineto{\pgfqpoint{3.471604in}{1.751809in}}%
\pgfpathlineto{\pgfqpoint{3.471796in}{1.750956in}}%
\pgfpathlineto{\pgfqpoint{3.472373in}{1.752126in}}%
\pgfpathlineto{\pgfqpoint{3.473333in}{1.762740in}}%
\pgfpathlineto{\pgfqpoint{3.474102in}{1.756560in}}%
\pgfpathlineto{\pgfqpoint{3.475447in}{1.761473in}}%
\pgfpathlineto{\pgfqpoint{3.475639in}{1.760486in}}%
\pgfpathlineto{\pgfqpoint{3.476792in}{1.765750in}}%
\pgfpathlineto{\pgfqpoint{3.476985in}{1.765221in}}%
\pgfpathlineto{\pgfqpoint{3.477177in}{1.764677in}}%
\pgfpathlineto{\pgfqpoint{3.477561in}{1.766820in}}%
\pgfpathlineto{\pgfqpoint{3.479099in}{1.775459in}}%
\pgfpathlineto{\pgfqpoint{3.480444in}{1.774004in}}%
\pgfpathlineto{\pgfqpoint{3.480636in}{1.771991in}}%
\pgfpathlineto{\pgfqpoint{3.481020in}{1.777507in}}%
\pgfpathlineto{\pgfqpoint{3.481212in}{1.776770in}}%
\pgfpathlineto{\pgfqpoint{3.482365in}{1.781352in}}%
\pgfpathlineto{\pgfqpoint{3.481789in}{1.774152in}}%
\pgfpathlineto{\pgfqpoint{3.482558in}{1.780017in}}%
\pgfpathlineto{\pgfqpoint{3.482750in}{1.776756in}}%
\pgfpathlineto{\pgfqpoint{3.483326in}{1.780343in}}%
\pgfpathlineto{\pgfqpoint{3.483711in}{1.777884in}}%
\pgfpathlineto{\pgfqpoint{3.483903in}{1.778119in}}%
\pgfpathlineto{\pgfqpoint{3.485056in}{1.768608in}}%
\pgfpathlineto{\pgfqpoint{3.485248in}{1.769270in}}%
\pgfpathlineto{\pgfqpoint{3.486017in}{1.775358in}}%
\pgfpathlineto{\pgfqpoint{3.486593in}{1.771682in}}%
\pgfpathlineto{\pgfqpoint{3.487938in}{1.767133in}}%
\pgfpathlineto{\pgfqpoint{3.488515in}{1.771712in}}%
\pgfpathlineto{\pgfqpoint{3.488899in}{1.766031in}}%
\pgfpathlineto{\pgfqpoint{3.489668in}{1.756258in}}%
\pgfpathlineto{\pgfqpoint{3.490436in}{1.759274in}}%
\pgfpathlineto{\pgfqpoint{3.490629in}{1.757498in}}%
\pgfpathlineto{\pgfqpoint{3.491013in}{1.761365in}}%
\pgfpathlineto{\pgfqpoint{3.491205in}{1.763181in}}%
\pgfpathlineto{\pgfqpoint{3.491589in}{1.759057in}}%
\pgfpathlineto{\pgfqpoint{3.491974in}{1.762562in}}%
\pgfpathlineto{\pgfqpoint{3.492166in}{1.758814in}}%
\pgfpathlineto{\pgfqpoint{3.492742in}{1.760814in}}%
\pgfpathlineto{\pgfqpoint{3.494088in}{1.774239in}}%
\pgfpathlineto{\pgfqpoint{3.494280in}{1.772645in}}%
\pgfpathlineto{\pgfqpoint{3.494664in}{1.769986in}}%
\pgfpathlineto{\pgfqpoint{3.494856in}{1.769456in}}%
\pgfpathlineto{\pgfqpoint{3.496201in}{1.774487in}}%
\pgfpathlineto{\pgfqpoint{3.496970in}{1.779617in}}%
\pgfpathlineto{\pgfqpoint{3.497354in}{1.778018in}}%
\pgfpathlineto{\pgfqpoint{3.497931in}{1.778326in}}%
\pgfpathlineto{\pgfqpoint{3.498507in}{1.774759in}}%
\pgfpathlineto{\pgfqpoint{3.499660in}{1.786970in}}%
\pgfpathlineto{\pgfqpoint{3.500045in}{1.785732in}}%
\pgfpathlineto{\pgfqpoint{3.500237in}{1.785210in}}%
\pgfpathlineto{\pgfqpoint{3.500621in}{1.786162in}}%
\pgfpathlineto{\pgfqpoint{3.500813in}{1.788628in}}%
\pgfpathlineto{\pgfqpoint{3.501774in}{1.788044in}}%
\pgfpathlineto{\pgfqpoint{3.502351in}{1.787692in}}%
\pgfpathlineto{\pgfqpoint{3.503120in}{1.789679in}}%
\pgfpathlineto{\pgfqpoint{3.504465in}{1.783376in}}%
\pgfpathlineto{\pgfqpoint{3.504849in}{1.784386in}}%
\pgfpathlineto{\pgfqpoint{3.506002in}{1.786368in}}%
\pgfpathlineto{\pgfqpoint{3.506194in}{1.785958in}}%
\pgfpathlineto{\pgfqpoint{3.507539in}{1.781059in}}%
\pgfpathlineto{\pgfqpoint{3.507732in}{1.782081in}}%
\pgfpathlineto{\pgfqpoint{3.510998in}{1.763730in}}%
\pgfpathlineto{\pgfqpoint{3.508116in}{1.782284in}}%
\pgfpathlineto{\pgfqpoint{3.512151in}{1.766750in}}%
\pgfpathlineto{\pgfqpoint{3.513112in}{1.771529in}}%
\pgfpathlineto{\pgfqpoint{3.513304in}{1.770480in}}%
\pgfpathlineto{\pgfqpoint{3.513497in}{1.767074in}}%
\pgfpathlineto{\pgfqpoint{3.514265in}{1.767980in}}%
\pgfpathlineto{\pgfqpoint{3.514457in}{1.770100in}}%
\pgfpathlineto{\pgfqpoint{3.514842in}{1.764025in}}%
\pgfpathlineto{\pgfqpoint{3.515034in}{1.763912in}}%
\pgfpathlineto{\pgfqpoint{3.515418in}{1.758510in}}%
\pgfpathlineto{\pgfqpoint{3.516187in}{1.763394in}}%
\pgfpathlineto{\pgfqpoint{3.517532in}{1.771496in}}%
\pgfpathlineto{\pgfqpoint{3.516571in}{1.762990in}}%
\pgfpathlineto{\pgfqpoint{3.518109in}{1.769716in}}%
\pgfpathlineto{\pgfqpoint{3.518301in}{1.769657in}}%
\pgfpathlineto{\pgfqpoint{3.518685in}{1.772392in}}%
\pgfpathlineto{\pgfqpoint{3.518877in}{1.768930in}}%
\pgfpathlineto{\pgfqpoint{3.519262in}{1.771433in}}%
\pgfpathlineto{\pgfqpoint{3.520607in}{1.766505in}}%
\pgfpathlineto{\pgfqpoint{3.520799in}{1.769009in}}%
\pgfpathlineto{\pgfqpoint{3.521568in}{1.765982in}}%
\pgfpathlineto{\pgfqpoint{3.521760in}{1.767068in}}%
\pgfpathlineto{\pgfqpoint{3.521952in}{1.767092in}}%
\pgfpathlineto{\pgfqpoint{3.522144in}{1.764970in}}%
\pgfpathlineto{\pgfqpoint{3.522721in}{1.769249in}}%
\pgfpathlineto{\pgfqpoint{3.523681in}{1.777896in}}%
\pgfpathlineto{\pgfqpoint{3.524258in}{1.773483in}}%
\pgfpathlineto{\pgfqpoint{3.524642in}{1.772494in}}%
\pgfpathlineto{\pgfqpoint{3.525027in}{1.773728in}}%
\pgfpathlineto{\pgfqpoint{3.525987in}{1.782411in}}%
\pgfpathlineto{\pgfqpoint{3.526372in}{1.776462in}}%
\pgfpathlineto{\pgfqpoint{3.527141in}{1.779081in}}%
\pgfpathlineto{\pgfqpoint{3.527717in}{1.778034in}}%
\pgfpathlineto{\pgfqpoint{3.527909in}{1.777777in}}%
\pgfpathlineto{\pgfqpoint{3.529447in}{1.783961in}}%
\pgfpathlineto{\pgfqpoint{3.530600in}{1.779914in}}%
\pgfpathlineto{\pgfqpoint{3.530792in}{1.780510in}}%
\pgfpathlineto{\pgfqpoint{3.531753in}{1.787357in}}%
\pgfpathlineto{\pgfqpoint{3.531176in}{1.778968in}}%
\pgfpathlineto{\pgfqpoint{3.532521in}{1.783285in}}%
\pgfpathlineto{\pgfqpoint{3.532713in}{1.783274in}}%
\pgfpathlineto{\pgfqpoint{3.533290in}{1.786611in}}%
\pgfpathlineto{\pgfqpoint{3.533674in}{1.782260in}}%
\pgfpathlineto{\pgfqpoint{3.533866in}{1.782325in}}%
\pgfpathlineto{\pgfqpoint{3.535404in}{1.773716in}}%
\pgfpathlineto{\pgfqpoint{3.535596in}{1.774590in}}%
\pgfpathlineto{\pgfqpoint{3.537325in}{1.791977in}}%
\pgfpathlineto{\pgfqpoint{3.537518in}{1.791838in}}%
\pgfpathlineto{\pgfqpoint{3.538286in}{1.788264in}}%
\pgfpathlineto{\pgfqpoint{3.537902in}{1.792602in}}%
\pgfpathlineto{\pgfqpoint{3.538671in}{1.791461in}}%
\pgfpathlineto{\pgfqpoint{3.539631in}{1.792925in}}%
\pgfpathlineto{\pgfqpoint{3.540208in}{1.790399in}}%
\pgfpathlineto{\pgfqpoint{3.540016in}{1.793056in}}%
\pgfpathlineto{\pgfqpoint{3.540592in}{1.791805in}}%
\pgfpathlineto{\pgfqpoint{3.540784in}{1.792992in}}%
\pgfpathlineto{\pgfqpoint{3.540977in}{1.792234in}}%
\pgfpathlineto{\pgfqpoint{3.541361in}{1.784283in}}%
\pgfpathlineto{\pgfqpoint{3.542130in}{1.789043in}}%
\pgfpathlineto{\pgfqpoint{3.543283in}{1.797036in}}%
\pgfpathlineto{\pgfqpoint{3.543475in}{1.796494in}}%
\pgfpathlineto{\pgfqpoint{3.543667in}{1.796574in}}%
\pgfpathlineto{\pgfqpoint{3.544436in}{1.801624in}}%
\pgfpathlineto{\pgfqpoint{3.544820in}{1.801152in}}%
\pgfpathlineto{\pgfqpoint{3.546357in}{1.789056in}}%
\pgfpathlineto{\pgfqpoint{3.546742in}{1.792551in}}%
\pgfpathlineto{\pgfqpoint{3.547895in}{1.801689in}}%
\pgfpathlineto{\pgfqpoint{3.548471in}{1.801102in}}%
\pgfpathlineto{\pgfqpoint{3.549624in}{1.805269in}}%
\pgfpathlineto{\pgfqpoint{3.549816in}{1.803708in}}%
\pgfpathlineto{\pgfqpoint{3.550201in}{1.804530in}}%
\pgfpathlineto{\pgfqpoint{3.550393in}{1.805618in}}%
\pgfpathlineto{\pgfqpoint{3.550585in}{1.803588in}}%
\pgfpathlineto{\pgfqpoint{3.551162in}{1.804705in}}%
\pgfpathlineto{\pgfqpoint{3.551738in}{1.803739in}}%
\pgfpathlineto{\pgfqpoint{3.551930in}{1.804386in}}%
\pgfpathlineto{\pgfqpoint{3.553083in}{1.812260in}}%
\pgfpathlineto{\pgfqpoint{3.553275in}{1.810858in}}%
\pgfpathlineto{\pgfqpoint{3.555197in}{1.794623in}}%
\pgfpathlineto{\pgfqpoint{3.555389in}{1.797765in}}%
\pgfpathlineto{\pgfqpoint{3.556734in}{1.801037in}}%
\pgfpathlineto{\pgfqpoint{3.557695in}{1.793732in}}%
\pgfpathlineto{\pgfqpoint{3.558272in}{1.797775in}}%
\pgfpathlineto{\pgfqpoint{3.560001in}{1.804190in}}%
\pgfpathlineto{\pgfqpoint{3.560578in}{1.807662in}}%
\pgfpathlineto{\pgfqpoint{3.561154in}{1.804479in}}%
\pgfpathlineto{\pgfqpoint{3.561346in}{1.804091in}}%
\pgfpathlineto{\pgfqpoint{3.561923in}{1.808328in}}%
\pgfpathlineto{\pgfqpoint{3.562499in}{1.807335in}}%
\pgfpathlineto{\pgfqpoint{3.562692in}{1.802013in}}%
\pgfpathlineto{\pgfqpoint{3.563652in}{1.803028in}}%
\pgfpathlineto{\pgfqpoint{3.564037in}{1.802701in}}%
\pgfpathlineto{\pgfqpoint{3.564421in}{1.803944in}}%
\pgfpathlineto{\pgfqpoint{3.564805in}{1.801273in}}%
\pgfpathlineto{\pgfqpoint{3.565190in}{1.803460in}}%
\pgfpathlineto{\pgfqpoint{3.566727in}{1.816739in}}%
\pgfpathlineto{\pgfqpoint{3.566919in}{1.815336in}}%
\pgfpathlineto{\pgfqpoint{3.567304in}{1.818945in}}%
\pgfpathlineto{\pgfqpoint{3.569417in}{1.823819in}}%
\pgfpathlineto{\pgfqpoint{3.571147in}{1.814625in}}%
\pgfpathlineto{\pgfqpoint{3.573261in}{1.806103in}}%
\pgfpathlineto{\pgfqpoint{3.573645in}{1.807689in}}%
\pgfpathlineto{\pgfqpoint{3.573837in}{1.807779in}}%
\pgfpathlineto{\pgfqpoint{3.575759in}{1.791315in}}%
\pgfpathlineto{\pgfqpoint{3.576912in}{1.798409in}}%
\pgfpathlineto{\pgfqpoint{3.577296in}{1.798321in}}%
\pgfpathlineto{\pgfqpoint{3.577681in}{1.798086in}}%
\pgfpathlineto{\pgfqpoint{3.578642in}{1.789378in}}%
\pgfpathlineto{\pgfqpoint{3.579026in}{1.792129in}}%
\pgfpathlineto{\pgfqpoint{3.579602in}{1.799532in}}%
\pgfpathlineto{\pgfqpoint{3.580179in}{1.796630in}}%
\pgfpathlineto{\pgfqpoint{3.580371in}{1.794241in}}%
\pgfpathlineto{\pgfqpoint{3.580755in}{1.797249in}}%
\pgfpathlineto{\pgfqpoint{3.581140in}{1.795906in}}%
\pgfpathlineto{\pgfqpoint{3.581332in}{1.798518in}}%
\pgfpathlineto{\pgfqpoint{3.582101in}{1.794737in}}%
\pgfpathlineto{\pgfqpoint{3.582293in}{1.794671in}}%
\pgfpathlineto{\pgfqpoint{3.582485in}{1.795682in}}%
\pgfpathlineto{\pgfqpoint{3.583446in}{1.798624in}}%
\pgfpathlineto{\pgfqpoint{3.583638in}{1.797228in}}%
\pgfpathlineto{\pgfqpoint{3.583830in}{1.795758in}}%
\pgfpathlineto{\pgfqpoint{3.584791in}{1.796512in}}%
\pgfpathlineto{\pgfqpoint{3.585560in}{1.800049in}}%
\pgfpathlineto{\pgfqpoint{3.585944in}{1.798022in}}%
\pgfpathlineto{\pgfqpoint{3.586136in}{1.796545in}}%
\pgfpathlineto{\pgfqpoint{3.586520in}{1.800791in}}%
\pgfpathlineto{\pgfqpoint{3.586713in}{1.802248in}}%
\pgfpathlineto{\pgfqpoint{3.587289in}{1.799182in}}%
\pgfpathlineto{\pgfqpoint{3.587481in}{1.799963in}}%
\pgfpathlineto{\pgfqpoint{3.587673in}{1.800354in}}%
\pgfpathlineto{\pgfqpoint{3.588634in}{1.794902in}}%
\pgfpathlineto{\pgfqpoint{3.588826in}{1.795402in}}%
\pgfpathlineto{\pgfqpoint{3.589019in}{1.796512in}}%
\pgfpathlineto{\pgfqpoint{3.589403in}{1.792465in}}%
\pgfpathlineto{\pgfqpoint{3.590748in}{1.785459in}}%
\pgfpathlineto{\pgfqpoint{3.590940in}{1.787524in}}%
\pgfpathlineto{\pgfqpoint{3.591132in}{1.787060in}}%
\pgfpathlineto{\pgfqpoint{3.591325in}{1.788876in}}%
\pgfpathlineto{\pgfqpoint{3.591901in}{1.790911in}}%
\pgfpathlineto{\pgfqpoint{3.592093in}{1.789487in}}%
\pgfpathlineto{\pgfqpoint{3.592862in}{1.795374in}}%
\pgfpathlineto{\pgfqpoint{3.593438in}{1.794774in}}%
\pgfpathlineto{\pgfqpoint{3.596705in}{1.779657in}}%
\pgfpathlineto{\pgfqpoint{3.596897in}{1.781300in}}%
\pgfpathlineto{\pgfqpoint{3.597282in}{1.782181in}}%
\pgfpathlineto{\pgfqpoint{3.597474in}{1.780608in}}%
\pgfpathlineto{\pgfqpoint{3.597666in}{1.780865in}}%
\pgfpathlineto{\pgfqpoint{3.598627in}{1.774868in}}%
\pgfpathlineto{\pgfqpoint{3.599011in}{1.775637in}}%
\pgfpathlineto{\pgfqpoint{3.600741in}{1.759906in}}%
\pgfpathlineto{\pgfqpoint{3.601510in}{1.765448in}}%
\pgfpathlineto{\pgfqpoint{3.602086in}{1.762703in}}%
\pgfpathlineto{\pgfqpoint{3.602278in}{1.762798in}}%
\pgfpathlineto{\pgfqpoint{3.602663in}{1.770931in}}%
\pgfpathlineto{\pgfqpoint{3.603431in}{1.767569in}}%
\pgfpathlineto{\pgfqpoint{3.603623in}{1.768800in}}%
\pgfpathlineto{\pgfqpoint{3.604008in}{1.764735in}}%
\pgfpathlineto{\pgfqpoint{3.604200in}{1.765041in}}%
\pgfpathlineto{\pgfqpoint{3.605545in}{1.769990in}}%
\pgfpathlineto{\pgfqpoint{3.605929in}{1.769684in}}%
\pgfpathlineto{\pgfqpoint{3.607659in}{1.760937in}}%
\pgfpathlineto{\pgfqpoint{3.607851in}{1.761319in}}%
\pgfpathlineto{\pgfqpoint{3.608043in}{1.759742in}}%
\pgfpathlineto{\pgfqpoint{3.608812in}{1.753653in}}%
\pgfpathlineto{\pgfqpoint{3.609773in}{1.744508in}}%
\pgfpathlineto{\pgfqpoint{3.609965in}{1.746803in}}%
\pgfpathlineto{\pgfqpoint{3.610157in}{1.750093in}}%
\pgfpathlineto{\pgfqpoint{3.611118in}{1.748111in}}%
\pgfpathlineto{\pgfqpoint{3.612655in}{1.735721in}}%
\pgfpathlineto{\pgfqpoint{3.612847in}{1.737336in}}%
\pgfpathlineto{\pgfqpoint{3.613424in}{1.733263in}}%
\pgfpathlineto{\pgfqpoint{3.614577in}{1.737713in}}%
\pgfpathlineto{\pgfqpoint{3.614769in}{1.737233in}}%
\pgfpathlineto{\pgfqpoint{3.615538in}{1.737428in}}%
\pgfpathlineto{\pgfqpoint{3.616114in}{1.731245in}}%
\pgfpathlineto{\pgfqpoint{3.617267in}{1.739431in}}%
\pgfpathlineto{\pgfqpoint{3.617459in}{1.737806in}}%
\pgfpathlineto{\pgfqpoint{3.620150in}{1.724708in}}%
\pgfpathlineto{\pgfqpoint{3.620342in}{1.726350in}}%
\pgfpathlineto{\pgfqpoint{3.620534in}{1.726964in}}%
\pgfpathlineto{\pgfqpoint{3.620726in}{1.726778in}}%
\pgfpathlineto{\pgfqpoint{3.620918in}{1.722978in}}%
\pgfpathlineto{\pgfqpoint{3.621687in}{1.724508in}}%
\pgfpathlineto{\pgfqpoint{3.622071in}{1.731878in}}%
\pgfpathlineto{\pgfqpoint{3.622840in}{1.727035in}}%
\pgfpathlineto{\pgfqpoint{3.623032in}{1.728982in}}%
\pgfpathlineto{\pgfqpoint{3.623417in}{1.725133in}}%
\pgfpathlineto{\pgfqpoint{3.623801in}{1.725686in}}%
\pgfpathlineto{\pgfqpoint{3.624377in}{1.727961in}}%
\pgfpathlineto{\pgfqpoint{3.625915in}{1.716820in}}%
\pgfpathlineto{\pgfqpoint{3.626299in}{1.715151in}}%
\pgfpathlineto{\pgfqpoint{3.626491in}{1.717579in}}%
\pgfpathlineto{\pgfqpoint{3.627837in}{1.726147in}}%
\pgfpathlineto{\pgfqpoint{3.628029in}{1.723220in}}%
\pgfpathlineto{\pgfqpoint{3.628221in}{1.723334in}}%
\pgfpathlineto{\pgfqpoint{3.628413in}{1.722486in}}%
\pgfpathlineto{\pgfqpoint{3.629182in}{1.714891in}}%
\pgfpathlineto{\pgfqpoint{3.629758in}{1.715923in}}%
\pgfpathlineto{\pgfqpoint{3.630527in}{1.722156in}}%
\pgfpathlineto{\pgfqpoint{3.631103in}{1.720844in}}%
\pgfpathlineto{\pgfqpoint{3.631296in}{1.718704in}}%
\pgfpathlineto{\pgfqpoint{3.631680in}{1.722085in}}%
\pgfpathlineto{\pgfqpoint{3.632064in}{1.721040in}}%
\pgfpathlineto{\pgfqpoint{3.632256in}{1.720860in}}%
\pgfpathlineto{\pgfqpoint{3.632833in}{1.723926in}}%
\pgfpathlineto{\pgfqpoint{3.633409in}{1.723319in}}%
\pgfpathlineto{\pgfqpoint{3.634755in}{1.718401in}}%
\pgfpathlineto{\pgfqpoint{3.633794in}{1.724309in}}%
\pgfpathlineto{\pgfqpoint{3.634947in}{1.719284in}}%
\pgfpathlineto{\pgfqpoint{3.635139in}{1.721361in}}%
\pgfpathlineto{\pgfqpoint{3.635523in}{1.717429in}}%
\pgfpathlineto{\pgfqpoint{3.635715in}{1.717462in}}%
\pgfpathlineto{\pgfqpoint{3.637061in}{1.715618in}}%
\pgfpathlineto{\pgfqpoint{3.638790in}{1.722930in}}%
\pgfpathlineto{\pgfqpoint{3.639174in}{1.722666in}}%
\pgfpathlineto{\pgfqpoint{3.640327in}{1.719518in}}%
\pgfpathlineto{\pgfqpoint{3.640904in}{1.722550in}}%
\pgfpathlineto{\pgfqpoint{3.641288in}{1.719001in}}%
\pgfpathlineto{\pgfqpoint{3.642633in}{1.713386in}}%
\pgfpathlineto{\pgfqpoint{3.642826in}{1.713966in}}%
\pgfpathlineto{\pgfqpoint{3.643786in}{1.722252in}}%
\pgfpathlineto{\pgfqpoint{3.644171in}{1.720566in}}%
\pgfpathlineto{\pgfqpoint{3.644363in}{1.720124in}}%
\pgfpathlineto{\pgfqpoint{3.644555in}{1.722473in}}%
\pgfpathlineto{\pgfqpoint{3.645132in}{1.719191in}}%
\pgfpathlineto{\pgfqpoint{3.645516in}{1.720733in}}%
\pgfpathlineto{\pgfqpoint{3.645900in}{1.721512in}}%
\pgfpathlineto{\pgfqpoint{3.646477in}{1.725019in}}%
\pgfpathlineto{\pgfqpoint{3.646861in}{1.722247in}}%
\pgfpathlineto{\pgfqpoint{3.647438in}{1.723119in}}%
\pgfpathlineto{\pgfqpoint{3.648206in}{1.717427in}}%
\pgfpathlineto{\pgfqpoint{3.648398in}{1.716829in}}%
\pgfpathlineto{\pgfqpoint{3.648591in}{1.717827in}}%
\pgfpathlineto{\pgfqpoint{3.649936in}{1.725452in}}%
\pgfpathlineto{\pgfqpoint{3.650128in}{1.723518in}}%
\pgfpathlineto{\pgfqpoint{3.652050in}{1.717347in}}%
\pgfpathlineto{\pgfqpoint{3.652626in}{1.719232in}}%
\pgfpathlineto{\pgfqpoint{3.652818in}{1.724212in}}%
\pgfpathlineto{\pgfqpoint{3.653779in}{1.721612in}}%
\pgfpathlineto{\pgfqpoint{3.654740in}{1.725537in}}%
\pgfpathlineto{\pgfqpoint{3.656662in}{1.744274in}}%
\pgfpathlineto{\pgfqpoint{3.656854in}{1.742378in}}%
\pgfpathlineto{\pgfqpoint{3.657430in}{1.745470in}}%
\pgfpathlineto{\pgfqpoint{3.657623in}{1.748015in}}%
\pgfpathlineto{\pgfqpoint{3.658199in}{1.743591in}}%
\pgfpathlineto{\pgfqpoint{3.658391in}{1.746353in}}%
\pgfpathlineto{\pgfqpoint{3.658583in}{1.745711in}}%
\pgfpathlineto{\pgfqpoint{3.658776in}{1.747280in}}%
\pgfpathlineto{\pgfqpoint{3.658968in}{1.748517in}}%
\pgfpathlineto{\pgfqpoint{3.660121in}{1.735750in}}%
\pgfpathlineto{\pgfqpoint{3.660313in}{1.739260in}}%
\pgfpathlineto{\pgfqpoint{3.661466in}{1.742435in}}%
\pgfpathlineto{\pgfqpoint{3.661658in}{1.740226in}}%
\pgfpathlineto{\pgfqpoint{3.662042in}{1.749505in}}%
\pgfpathlineto{\pgfqpoint{3.663003in}{1.748194in}}%
\pgfpathlineto{\pgfqpoint{3.663388in}{1.747683in}}%
\pgfpathlineto{\pgfqpoint{3.664541in}{1.753742in}}%
\pgfpathlineto{\pgfqpoint{3.664925in}{1.753145in}}%
\pgfpathlineto{\pgfqpoint{3.666078in}{1.758502in}}%
\pgfpathlineto{\pgfqpoint{3.667039in}{1.751560in}}%
\pgfpathlineto{\pgfqpoint{3.667615in}{1.754189in}}%
\pgfpathlineto{\pgfqpoint{3.667807in}{1.754539in}}%
\pgfpathlineto{\pgfqpoint{3.668000in}{1.754104in}}%
\pgfpathlineto{\pgfqpoint{3.668768in}{1.751324in}}%
\pgfpathlineto{\pgfqpoint{3.668960in}{1.754687in}}%
\pgfpathlineto{\pgfqpoint{3.669345in}{1.752499in}}%
\pgfpathlineto{\pgfqpoint{3.670113in}{1.753773in}}%
\pgfpathlineto{\pgfqpoint{3.670498in}{1.752910in}}%
\pgfpathlineto{\pgfqpoint{3.671843in}{1.741407in}}%
\pgfpathlineto{\pgfqpoint{3.672227in}{1.743081in}}%
\pgfpathlineto{\pgfqpoint{3.672996in}{1.737091in}}%
\pgfpathlineto{\pgfqpoint{3.673380in}{1.737456in}}%
\pgfpathlineto{\pgfqpoint{3.673573in}{1.734558in}}%
\pgfpathlineto{\pgfqpoint{3.674341in}{1.736523in}}%
\pgfpathlineto{\pgfqpoint{3.675302in}{1.741710in}}%
\pgfpathlineto{\pgfqpoint{3.675494in}{1.738266in}}%
\pgfpathlineto{\pgfqpoint{3.676263in}{1.731943in}}%
\pgfpathlineto{\pgfqpoint{3.676647in}{1.735313in}}%
\pgfpathlineto{\pgfqpoint{3.677032in}{1.739038in}}%
\pgfpathlineto{\pgfqpoint{3.677608in}{1.735568in}}%
\pgfpathlineto{\pgfqpoint{3.677800in}{1.734902in}}%
\pgfpathlineto{\pgfqpoint{3.678185in}{1.736331in}}%
\pgfpathlineto{\pgfqpoint{3.678377in}{1.735137in}}%
\pgfpathlineto{\pgfqpoint{3.678569in}{1.737061in}}%
\pgfpathlineto{\pgfqpoint{3.679338in}{1.733775in}}%
\pgfpathlineto{\pgfqpoint{3.679722in}{1.736260in}}%
\pgfpathlineto{\pgfqpoint{3.680875in}{1.739273in}}%
\pgfpathlineto{\pgfqpoint{3.684142in}{1.725603in}}%
\pgfpathlineto{\pgfqpoint{3.684334in}{1.727172in}}%
\pgfpathlineto{\pgfqpoint{3.687216in}{1.740634in}}%
\pgfpathlineto{\pgfqpoint{3.687409in}{1.737508in}}%
\pgfpathlineto{\pgfqpoint{3.687601in}{1.735588in}}%
\pgfpathlineto{\pgfqpoint{3.688177in}{1.740684in}}%
\pgfpathlineto{\pgfqpoint{3.689330in}{1.746865in}}%
\pgfpathlineto{\pgfqpoint{3.689522in}{1.746398in}}%
\pgfpathlineto{\pgfqpoint{3.689715in}{1.744962in}}%
\pgfpathlineto{\pgfqpoint{3.690099in}{1.748639in}}%
\pgfpathlineto{\pgfqpoint{3.690483in}{1.753230in}}%
\pgfpathlineto{\pgfqpoint{3.690868in}{1.746408in}}%
\pgfpathlineto{\pgfqpoint{3.691060in}{1.747145in}}%
\pgfpathlineto{\pgfqpoint{3.691252in}{1.745981in}}%
\pgfpathlineto{\pgfqpoint{3.691444in}{1.746124in}}%
\pgfpathlineto{\pgfqpoint{3.691828in}{1.750851in}}%
\pgfpathlineto{\pgfqpoint{3.692405in}{1.744315in}}%
\pgfpathlineto{\pgfqpoint{3.692789in}{1.740590in}}%
\pgfpathlineto{\pgfqpoint{3.693366in}{1.744929in}}%
\pgfpathlineto{\pgfqpoint{3.693558in}{1.743507in}}%
\pgfpathlineto{\pgfqpoint{3.694134in}{1.741157in}}%
\pgfpathlineto{\pgfqpoint{3.694327in}{1.739354in}}%
\pgfpathlineto{\pgfqpoint{3.694711in}{1.741826in}}%
\pgfpathlineto{\pgfqpoint{3.694903in}{1.741484in}}%
\pgfpathlineto{\pgfqpoint{3.695095in}{1.742797in}}%
\pgfpathlineto{\pgfqpoint{3.695672in}{1.740113in}}%
\pgfpathlineto{\pgfqpoint{3.695864in}{1.740554in}}%
\pgfpathlineto{\pgfqpoint{3.698362in}{1.727783in}}%
\pgfpathlineto{\pgfqpoint{3.698554in}{1.729393in}}%
\pgfpathlineto{\pgfqpoint{3.698747in}{1.729489in}}%
\pgfpathlineto{\pgfqpoint{3.700284in}{1.739689in}}%
\pgfpathlineto{\pgfqpoint{3.701053in}{1.735067in}}%
\pgfpathlineto{\pgfqpoint{3.701437in}{1.738427in}}%
\pgfpathlineto{\pgfqpoint{3.701629in}{1.738361in}}%
\pgfpathlineto{\pgfqpoint{3.701821in}{1.739124in}}%
\pgfpathlineto{\pgfqpoint{3.702013in}{1.741621in}}%
\pgfpathlineto{\pgfqpoint{3.702782in}{1.740043in}}%
\pgfpathlineto{\pgfqpoint{3.704319in}{1.733246in}}%
\pgfpathlineto{\pgfqpoint{3.704896in}{1.740576in}}%
\pgfpathlineto{\pgfqpoint{3.705665in}{1.737115in}}%
\pgfpathlineto{\pgfqpoint{3.706625in}{1.733890in}}%
\pgfpathlineto{\pgfqpoint{3.707010in}{1.735785in}}%
\pgfpathlineto{\pgfqpoint{3.707202in}{1.736261in}}%
\pgfpathlineto{\pgfqpoint{3.707971in}{1.725562in}}%
\pgfpathlineto{\pgfqpoint{3.708355in}{1.728973in}}%
\pgfpathlineto{\pgfqpoint{3.709124in}{1.734506in}}%
\pgfpathlineto{\pgfqpoint{3.709508in}{1.730568in}}%
\pgfpathlineto{\pgfqpoint{3.710084in}{1.729095in}}%
\pgfpathlineto{\pgfqpoint{3.710853in}{1.735954in}}%
\pgfpathlineto{\pgfqpoint{3.711237in}{1.732182in}}%
\pgfpathlineto{\pgfqpoint{3.712775in}{1.719891in}}%
\pgfpathlineto{\pgfqpoint{3.713351in}{1.725330in}}%
\pgfpathlineto{\pgfqpoint{3.713736in}{1.724273in}}%
\pgfpathlineto{\pgfqpoint{3.713928in}{1.720985in}}%
\pgfpathlineto{\pgfqpoint{3.714696in}{1.726364in}}%
\pgfpathlineto{\pgfqpoint{3.715081in}{1.723700in}}%
\pgfpathlineto{\pgfqpoint{3.715465in}{1.729004in}}%
\pgfpathlineto{\pgfqpoint{3.716810in}{1.722915in}}%
\pgfpathlineto{\pgfqpoint{3.718155in}{1.716372in}}%
\pgfpathlineto{\pgfqpoint{3.718348in}{1.716772in}}%
\pgfpathlineto{\pgfqpoint{3.718924in}{1.715917in}}%
\pgfpathlineto{\pgfqpoint{3.719308in}{1.717718in}}%
\pgfpathlineto{\pgfqpoint{3.719885in}{1.714557in}}%
\pgfpathlineto{\pgfqpoint{3.720077in}{1.717975in}}%
\pgfpathlineto{\pgfqpoint{3.720461in}{1.721222in}}%
\pgfpathlineto{\pgfqpoint{3.721038in}{1.716652in}}%
\pgfpathlineto{\pgfqpoint{3.721230in}{1.714949in}}%
\pgfpathlineto{\pgfqpoint{3.721422in}{1.721626in}}%
\pgfpathlineto{\pgfqpoint{3.722191in}{1.724456in}}%
\pgfpathlineto{\pgfqpoint{3.722383in}{1.722563in}}%
\pgfpathlineto{\pgfqpoint{3.723344in}{1.717391in}}%
\pgfpathlineto{\pgfqpoint{3.723536in}{1.718929in}}%
\pgfpathlineto{\pgfqpoint{3.723921in}{1.721044in}}%
\pgfpathlineto{\pgfqpoint{3.724305in}{1.717956in}}%
\pgfpathlineto{\pgfqpoint{3.724689in}{1.714428in}}%
\pgfpathlineto{\pgfqpoint{3.725074in}{1.718480in}}%
\pgfpathlineto{\pgfqpoint{3.725266in}{1.717052in}}%
\pgfpathlineto{\pgfqpoint{3.726419in}{1.723528in}}%
\pgfpathlineto{\pgfqpoint{3.727187in}{1.716851in}}%
\pgfpathlineto{\pgfqpoint{3.727764in}{1.718535in}}%
\pgfpathlineto{\pgfqpoint{3.729301in}{1.727791in}}%
\pgfpathlineto{\pgfqpoint{3.728148in}{1.717568in}}%
\pgfpathlineto{\pgfqpoint{3.729493in}{1.725827in}}%
\pgfpathlineto{\pgfqpoint{3.730262in}{1.722495in}}%
\pgfpathlineto{\pgfqpoint{3.730454in}{1.725056in}}%
\pgfpathlineto{\pgfqpoint{3.732376in}{1.734164in}}%
\pgfpathlineto{\pgfqpoint{3.732568in}{1.731820in}}%
\pgfpathlineto{\pgfqpoint{3.732760in}{1.730107in}}%
\pgfpathlineto{\pgfqpoint{3.733145in}{1.734108in}}%
\pgfpathlineto{\pgfqpoint{3.733337in}{1.737542in}}%
\pgfpathlineto{\pgfqpoint{3.734105in}{1.734955in}}%
\pgfpathlineto{\pgfqpoint{3.734298in}{1.732637in}}%
\pgfpathlineto{\pgfqpoint{3.734874in}{1.737050in}}%
\pgfpathlineto{\pgfqpoint{3.735066in}{1.736461in}}%
\pgfpathlineto{\pgfqpoint{3.735258in}{1.736055in}}%
\pgfpathlineto{\pgfqpoint{3.735451in}{1.737515in}}%
\pgfpathlineto{\pgfqpoint{3.736796in}{1.745530in}}%
\pgfpathlineto{\pgfqpoint{3.736988in}{1.744183in}}%
\pgfpathlineto{\pgfqpoint{3.737949in}{1.742693in}}%
\pgfpathlineto{\pgfqpoint{3.739294in}{1.732714in}}%
\pgfpathlineto{\pgfqpoint{3.739486in}{1.733078in}}%
\pgfpathlineto{\pgfqpoint{3.739678in}{1.736116in}}%
\pgfpathlineto{\pgfqpoint{3.740255in}{1.729833in}}%
\pgfpathlineto{\pgfqpoint{3.740447in}{1.729644in}}%
\pgfpathlineto{\pgfqpoint{3.741984in}{1.739749in}}%
\pgfpathlineto{\pgfqpoint{3.742176in}{1.737724in}}%
\pgfpathlineto{\pgfqpoint{3.742753in}{1.735890in}}%
\pgfpathlineto{\pgfqpoint{3.743137in}{1.736298in}}%
\pgfpathlineto{\pgfqpoint{3.743714in}{1.743811in}}%
\pgfpathlineto{\pgfqpoint{3.744290in}{1.740087in}}%
\pgfpathlineto{\pgfqpoint{3.744675in}{1.736680in}}%
\pgfpathlineto{\pgfqpoint{3.744867in}{1.741015in}}%
\pgfpathlineto{\pgfqpoint{3.745059in}{1.742637in}}%
\pgfpathlineto{\pgfqpoint{3.745635in}{1.738703in}}%
\pgfpathlineto{\pgfqpoint{3.747173in}{1.726825in}}%
\pgfpathlineto{\pgfqpoint{3.747942in}{1.731058in}}%
\pgfpathlineto{\pgfqpoint{3.748518in}{1.728715in}}%
\pgfpathlineto{\pgfqpoint{3.749287in}{1.732390in}}%
\pgfpathlineto{\pgfqpoint{3.749671in}{1.729287in}}%
\pgfpathlineto{\pgfqpoint{3.750824in}{1.734766in}}%
\pgfpathlineto{\pgfqpoint{3.751208in}{1.737869in}}%
\pgfpathlineto{\pgfqpoint{3.751977in}{1.735424in}}%
\pgfpathlineto{\pgfqpoint{3.752746in}{1.729309in}}%
\pgfpathlineto{\pgfqpoint{3.753130in}{1.732009in}}%
\pgfpathlineto{\pgfqpoint{3.753707in}{1.736790in}}%
\pgfpathlineto{\pgfqpoint{3.754283in}{1.733177in}}%
\pgfpathlineto{\pgfqpoint{3.754475in}{1.732949in}}%
\pgfpathlineto{\pgfqpoint{3.754667in}{1.734401in}}%
\pgfpathlineto{\pgfqpoint{3.754860in}{1.733931in}}%
\pgfpathlineto{\pgfqpoint{3.755244in}{1.735850in}}%
\pgfpathlineto{\pgfqpoint{3.755628in}{1.736900in}}%
\pgfpathlineto{\pgfqpoint{3.755820in}{1.736524in}}%
\pgfpathlineto{\pgfqpoint{3.756781in}{1.728207in}}%
\pgfpathlineto{\pgfqpoint{3.757166in}{1.731464in}}%
\pgfpathlineto{\pgfqpoint{3.757358in}{1.732303in}}%
\pgfpathlineto{\pgfqpoint{3.757550in}{1.731474in}}%
\pgfpathlineto{\pgfqpoint{3.758703in}{1.725120in}}%
\pgfpathlineto{\pgfqpoint{3.758895in}{1.725348in}}%
\pgfpathlineto{\pgfqpoint{3.759087in}{1.726233in}}%
\pgfpathlineto{\pgfqpoint{3.759279in}{1.725049in}}%
\pgfpathlineto{\pgfqpoint{3.759472in}{1.725168in}}%
\pgfpathlineto{\pgfqpoint{3.759856in}{1.723143in}}%
\pgfpathlineto{\pgfqpoint{3.760432in}{1.725639in}}%
\pgfpathlineto{\pgfqpoint{3.761009in}{1.724476in}}%
\pgfpathlineto{\pgfqpoint{3.761970in}{1.731411in}}%
\pgfpathlineto{\pgfqpoint{3.763123in}{1.726331in}}%
\pgfpathlineto{\pgfqpoint{3.764276in}{1.734445in}}%
\pgfpathlineto{\pgfqpoint{3.764660in}{1.731985in}}%
\pgfpathlineto{\pgfqpoint{3.766197in}{1.719538in}}%
\pgfpathlineto{\pgfqpoint{3.766390in}{1.720611in}}%
\pgfpathlineto{\pgfqpoint{3.766774in}{1.717183in}}%
\pgfpathlineto{\pgfqpoint{3.766966in}{1.717768in}}%
\pgfpathlineto{\pgfqpoint{3.767735in}{1.723315in}}%
\pgfpathlineto{\pgfqpoint{3.768311in}{1.727489in}}%
\pgfpathlineto{\pgfqpoint{3.768503in}{1.723364in}}%
\pgfpathlineto{\pgfqpoint{3.769656in}{1.715406in}}%
\pgfpathlineto{\pgfqpoint{3.770041in}{1.719770in}}%
\pgfpathlineto{\pgfqpoint{3.770425in}{1.721860in}}%
\pgfpathlineto{\pgfqpoint{3.771002in}{1.719342in}}%
\pgfpathlineto{\pgfqpoint{3.771194in}{1.720380in}}%
\pgfpathlineto{\pgfqpoint{3.772347in}{1.711661in}}%
\pgfpathlineto{\pgfqpoint{3.772923in}{1.712420in}}%
\pgfpathlineto{\pgfqpoint{3.774269in}{1.720886in}}%
\pgfpathlineto{\pgfqpoint{3.774461in}{1.718929in}}%
\pgfpathlineto{\pgfqpoint{3.774845in}{1.718579in}}%
\pgfpathlineto{\pgfqpoint{3.775037in}{1.719654in}}%
\pgfpathlineto{\pgfqpoint{3.776382in}{1.714301in}}%
\pgfpathlineto{\pgfqpoint{3.776575in}{1.713888in}}%
\pgfpathlineto{\pgfqpoint{3.776767in}{1.714635in}}%
\pgfpathlineto{\pgfqpoint{3.777151in}{1.716710in}}%
\pgfpathlineto{\pgfqpoint{3.777535in}{1.712520in}}%
\pgfpathlineto{\pgfqpoint{3.778304in}{1.716990in}}%
\pgfpathlineto{\pgfqpoint{3.777920in}{1.711224in}}%
\pgfpathlineto{\pgfqpoint{3.778688in}{1.713562in}}%
\pgfpathlineto{\pgfqpoint{3.778881in}{1.712563in}}%
\pgfpathlineto{\pgfqpoint{3.779073in}{1.714296in}}%
\pgfpathlineto{\pgfqpoint{3.779265in}{1.717387in}}%
\pgfpathlineto{\pgfqpoint{3.780034in}{1.714742in}}%
\pgfpathlineto{\pgfqpoint{3.781379in}{1.710407in}}%
\pgfpathlineto{\pgfqpoint{3.781763in}{1.707458in}}%
\pgfpathlineto{\pgfqpoint{3.781955in}{1.710827in}}%
\pgfpathlineto{\pgfqpoint{3.782340in}{1.716791in}}%
\pgfpathlineto{\pgfqpoint{3.782916in}{1.713322in}}%
\pgfpathlineto{\pgfqpoint{3.784261in}{1.706141in}}%
\pgfpathlineto{\pgfqpoint{3.786183in}{1.716537in}}%
\pgfpathlineto{\pgfqpoint{3.786952in}{1.713302in}}%
\pgfpathlineto{\pgfqpoint{3.787528in}{1.714273in}}%
\pgfpathlineto{\pgfqpoint{3.788873in}{1.704511in}}%
\pgfpathlineto{\pgfqpoint{3.790411in}{1.715083in}}%
\pgfpathlineto{\pgfqpoint{3.790987in}{1.708807in}}%
\pgfpathlineto{\pgfqpoint{3.791948in}{1.711938in}}%
\pgfpathlineto{\pgfqpoint{3.792909in}{1.713204in}}%
\pgfpathlineto{\pgfqpoint{3.792717in}{1.711310in}}%
\pgfpathlineto{\pgfqpoint{3.793101in}{1.712682in}}%
\pgfpathlineto{\pgfqpoint{3.793485in}{1.711362in}}%
\pgfpathlineto{\pgfqpoint{3.793677in}{1.713093in}}%
\pgfpathlineto{\pgfqpoint{3.793870in}{1.712256in}}%
\pgfpathlineto{\pgfqpoint{3.795599in}{1.721969in}}%
\pgfpathlineto{\pgfqpoint{3.795791in}{1.720376in}}%
\pgfpathlineto{\pgfqpoint{3.796368in}{1.724735in}}%
\pgfpathlineto{\pgfqpoint{3.796560in}{1.724587in}}%
\pgfpathlineto{\pgfqpoint{3.796752in}{1.725835in}}%
\pgfpathlineto{\pgfqpoint{3.797521in}{1.733533in}}%
\pgfpathlineto{\pgfqpoint{3.798482in}{1.731562in}}%
\pgfpathlineto{\pgfqpoint{3.799827in}{1.725863in}}%
\pgfpathlineto{\pgfqpoint{3.800596in}{1.731761in}}%
\pgfpathlineto{\pgfqpoint{3.801172in}{1.727498in}}%
\pgfpathlineto{\pgfqpoint{3.801364in}{1.723479in}}%
\pgfpathlineto{\pgfqpoint{3.802325in}{1.724069in}}%
\pgfpathlineto{\pgfqpoint{3.803094in}{1.722433in}}%
\pgfpathlineto{\pgfqpoint{3.804439in}{1.733864in}}%
\pgfpathlineto{\pgfqpoint{3.804631in}{1.733195in}}%
\pgfpathlineto{\pgfqpoint{3.805400in}{1.730665in}}%
\pgfpathlineto{\pgfqpoint{3.805015in}{1.733679in}}%
\pgfpathlineto{\pgfqpoint{3.805592in}{1.732559in}}%
\pgfpathlineto{\pgfqpoint{3.805976in}{1.732707in}}%
\pgfpathlineto{\pgfqpoint{3.807129in}{1.725136in}}%
\pgfpathlineto{\pgfqpoint{3.807321in}{1.725022in}}%
\pgfpathlineto{\pgfqpoint{3.807514in}{1.725453in}}%
\pgfpathlineto{\pgfqpoint{3.807898in}{1.727956in}}%
\pgfpathlineto{\pgfqpoint{3.808090in}{1.724781in}}%
\pgfpathlineto{\pgfqpoint{3.808667in}{1.726519in}}%
\pgfpathlineto{\pgfqpoint{3.810012in}{1.721561in}}%
\pgfpathlineto{\pgfqpoint{3.810588in}{1.718889in}}%
\pgfpathlineto{\pgfqpoint{3.810973in}{1.719731in}}%
\pgfpathlineto{\pgfqpoint{3.811549in}{1.726611in}}%
\pgfpathlineto{\pgfqpoint{3.812318in}{1.725863in}}%
\pgfpathlineto{\pgfqpoint{3.815392in}{1.713118in}}%
\pgfpathlineto{\pgfqpoint{3.812894in}{1.726016in}}%
\pgfpathlineto{\pgfqpoint{3.815969in}{1.714738in}}%
\pgfpathlineto{\pgfqpoint{3.816161in}{1.716186in}}%
\pgfpathlineto{\pgfqpoint{3.816545in}{1.714958in}}%
\pgfpathlineto{\pgfqpoint{3.817122in}{1.709178in}}%
\pgfpathlineto{\pgfqpoint{3.817891in}{1.710056in}}%
\pgfpathlineto{\pgfqpoint{3.818851in}{1.715954in}}%
\pgfpathlineto{\pgfqpoint{3.818467in}{1.709772in}}%
\pgfpathlineto{\pgfqpoint{3.819236in}{1.712737in}}%
\pgfpathlineto{\pgfqpoint{3.819428in}{1.709248in}}%
\pgfpathlineto{\pgfqpoint{3.820005in}{1.716339in}}%
\pgfpathlineto{\pgfqpoint{3.820965in}{1.710813in}}%
\pgfpathlineto{\pgfqpoint{3.821734in}{1.713218in}}%
\pgfpathlineto{\pgfqpoint{3.822118in}{1.715599in}}%
\pgfpathlineto{\pgfqpoint{3.822311in}{1.715447in}}%
\pgfpathlineto{\pgfqpoint{3.822695in}{1.709453in}}%
\pgfpathlineto{\pgfqpoint{3.823656in}{1.710000in}}%
\pgfpathlineto{\pgfqpoint{3.825770in}{1.719290in}}%
\pgfpathlineto{\pgfqpoint{3.824232in}{1.708897in}}%
\pgfpathlineto{\pgfqpoint{3.825962in}{1.718720in}}%
\pgfpathlineto{\pgfqpoint{3.827883in}{1.708376in}}%
\pgfpathlineto{\pgfqpoint{3.828076in}{1.708147in}}%
\pgfpathlineto{\pgfqpoint{3.828268in}{1.710407in}}%
\pgfpathlineto{\pgfqpoint{3.828844in}{1.707807in}}%
\pgfpathlineto{\pgfqpoint{3.829229in}{1.708891in}}%
\pgfpathlineto{\pgfqpoint{3.829421in}{1.708770in}}%
\pgfpathlineto{\pgfqpoint{3.829805in}{1.703409in}}%
\pgfpathlineto{\pgfqpoint{3.830574in}{1.706944in}}%
\pgfpathlineto{\pgfqpoint{3.832303in}{1.716852in}}%
\pgfpathlineto{\pgfqpoint{3.833456in}{1.716215in}}%
\pgfpathlineto{\pgfqpoint{3.833841in}{1.713573in}}%
\pgfpathlineto{\pgfqpoint{3.834609in}{1.719010in}}%
\pgfpathlineto{\pgfqpoint{3.834994in}{1.717376in}}%
\pgfpathlineto{\pgfqpoint{3.835186in}{1.713896in}}%
\pgfpathlineto{\pgfqpoint{3.835762in}{1.719370in}}%
\pgfpathlineto{\pgfqpoint{3.835954in}{1.719037in}}%
\pgfpathlineto{\pgfqpoint{3.836147in}{1.722181in}}%
\pgfpathlineto{\pgfqpoint{3.836723in}{1.716343in}}%
\pgfpathlineto{\pgfqpoint{3.837107in}{1.715389in}}%
\pgfpathlineto{\pgfqpoint{3.838260in}{1.719654in}}%
\pgfpathlineto{\pgfqpoint{3.839029in}{1.712679in}}%
\pgfpathlineto{\pgfqpoint{3.839221in}{1.716338in}}%
\pgfpathlineto{\pgfqpoint{3.839606in}{1.720219in}}%
\pgfpathlineto{\pgfqpoint{3.840182in}{1.715163in}}%
\pgfpathlineto{\pgfqpoint{3.841912in}{1.696486in}}%
\pgfpathlineto{\pgfqpoint{3.843257in}{1.699442in}}%
\pgfpathlineto{\pgfqpoint{3.843641in}{1.703099in}}%
\pgfpathlineto{\pgfqpoint{3.844602in}{1.702910in}}%
\pgfpathlineto{\pgfqpoint{3.845179in}{1.701038in}}%
\pgfpathlineto{\pgfqpoint{3.844986in}{1.703354in}}%
\pgfpathlineto{\pgfqpoint{3.845755in}{1.702717in}}%
\pgfpathlineto{\pgfqpoint{3.846716in}{1.705227in}}%
\pgfpathlineto{\pgfqpoint{3.846908in}{1.704481in}}%
\pgfpathlineto{\pgfqpoint{3.847100in}{1.704804in}}%
\pgfpathlineto{\pgfqpoint{3.847485in}{1.703311in}}%
\pgfpathlineto{\pgfqpoint{3.847869in}{1.704270in}}%
\pgfpathlineto{\pgfqpoint{3.849406in}{1.695060in}}%
\pgfpathlineto{\pgfqpoint{3.849791in}{1.698616in}}%
\pgfpathlineto{\pgfqpoint{3.849983in}{1.694592in}}%
\pgfpathlineto{\pgfqpoint{3.850751in}{1.684550in}}%
\pgfpathlineto{\pgfqpoint{3.851328in}{1.687382in}}%
\pgfpathlineto{\pgfqpoint{3.852673in}{1.698804in}}%
\pgfpathlineto{\pgfqpoint{3.853057in}{1.696685in}}%
\pgfpathlineto{\pgfqpoint{3.853634in}{1.693079in}}%
\pgfpathlineto{\pgfqpoint{3.854210in}{1.695384in}}%
\pgfpathlineto{\pgfqpoint{3.854403in}{1.698825in}}%
\pgfpathlineto{\pgfqpoint{3.854979in}{1.690544in}}%
\pgfpathlineto{\pgfqpoint{3.856324in}{1.696210in}}%
\pgfpathlineto{\pgfqpoint{3.856709in}{1.692496in}}%
\pgfpathlineto{\pgfqpoint{3.857477in}{1.694847in}}%
\pgfpathlineto{\pgfqpoint{3.857862in}{1.695569in}}%
\pgfpathlineto{\pgfqpoint{3.858246in}{1.694097in}}%
\pgfpathlineto{\pgfqpoint{3.858822in}{1.690519in}}%
\pgfpathlineto{\pgfqpoint{3.859207in}{1.693150in}}%
\pgfpathlineto{\pgfqpoint{3.859591in}{1.695769in}}%
\pgfpathlineto{\pgfqpoint{3.860168in}{1.691778in}}%
\pgfpathlineto{\pgfqpoint{3.860360in}{1.693691in}}%
\pgfpathlineto{\pgfqpoint{3.860552in}{1.694026in}}%
\pgfpathlineto{\pgfqpoint{3.860744in}{1.691117in}}%
\pgfpathlineto{\pgfqpoint{3.861321in}{1.697705in}}%
\pgfpathlineto{\pgfqpoint{3.861513in}{1.695726in}}%
\pgfpathlineto{\pgfqpoint{3.861705in}{1.694832in}}%
\pgfpathlineto{\pgfqpoint{3.862089in}{1.696622in}}%
\pgfpathlineto{\pgfqpoint{3.862281in}{1.696563in}}%
\pgfpathlineto{\pgfqpoint{3.862666in}{1.696317in}}%
\pgfpathlineto{\pgfqpoint{3.863050in}{1.693432in}}%
\pgfpathlineto{\pgfqpoint{3.863819in}{1.695313in}}%
\pgfpathlineto{\pgfqpoint{3.864203in}{1.698643in}}%
\pgfpathlineto{\pgfqpoint{3.864587in}{1.698027in}}%
\pgfpathlineto{\pgfqpoint{3.865933in}{1.691799in}}%
\pgfpathlineto{\pgfqpoint{3.866125in}{1.692242in}}%
\pgfpathlineto{\pgfqpoint{3.866317in}{1.690609in}}%
\pgfpathlineto{\pgfqpoint{3.866893in}{1.689673in}}%
\pgfpathlineto{\pgfqpoint{3.868046in}{1.697137in}}%
\pgfpathlineto{\pgfqpoint{3.868239in}{1.696261in}}%
\pgfpathlineto{\pgfqpoint{3.868815in}{1.696125in}}%
\pgfpathlineto{\pgfqpoint{3.869392in}{1.707106in}}%
\pgfpathlineto{\pgfqpoint{3.870160in}{1.713130in}}%
\pgfpathlineto{\pgfqpoint{3.870545in}{1.711592in}}%
\pgfpathlineto{\pgfqpoint{3.870737in}{1.710450in}}%
\pgfpathlineto{\pgfqpoint{3.870929in}{1.714641in}}%
\pgfpathlineto{\pgfqpoint{3.871313in}{1.714234in}}%
\pgfpathlineto{\pgfqpoint{3.872851in}{1.721524in}}%
\pgfpathlineto{\pgfqpoint{3.874772in}{1.701405in}}%
\pgfpathlineto{\pgfqpoint{3.874965in}{1.701994in}}%
\pgfpathlineto{\pgfqpoint{3.875349in}{1.704160in}}%
\pgfpathlineto{\pgfqpoint{3.875733in}{1.701886in}}%
\pgfpathlineto{\pgfqpoint{3.875925in}{1.699943in}}%
\pgfpathlineto{\pgfqpoint{3.876310in}{1.703024in}}%
\pgfpathlineto{\pgfqpoint{3.876502in}{1.702717in}}%
\pgfpathlineto{\pgfqpoint{3.876886in}{1.702360in}}%
\pgfpathlineto{\pgfqpoint{3.877847in}{1.706163in}}%
\pgfpathlineto{\pgfqpoint{3.878231in}{1.702332in}}%
\pgfpathlineto{\pgfqpoint{3.878808in}{1.707758in}}%
\pgfpathlineto{\pgfqpoint{3.879577in}{1.710116in}}%
\pgfpathlineto{\pgfqpoint{3.879769in}{1.707201in}}%
\pgfpathlineto{\pgfqpoint{3.879961in}{1.709302in}}%
\pgfpathlineto{\pgfqpoint{3.882075in}{1.700468in}}%
\pgfpathlineto{\pgfqpoint{3.883996in}{1.693395in}}%
\pgfpathlineto{\pgfqpoint{3.882651in}{1.701796in}}%
\pgfpathlineto{\pgfqpoint{3.884189in}{1.695517in}}%
\pgfpathlineto{\pgfqpoint{3.884381in}{1.695701in}}%
\pgfpathlineto{\pgfqpoint{3.886495in}{1.681115in}}%
\pgfpathlineto{\pgfqpoint{3.886687in}{1.681823in}}%
\pgfpathlineto{\pgfqpoint{3.886879in}{1.683811in}}%
\pgfpathlineto{\pgfqpoint{3.887263in}{1.679102in}}%
\pgfpathlineto{\pgfqpoint{3.887648in}{1.680941in}}%
\pgfpathlineto{\pgfqpoint{3.887840in}{1.680058in}}%
\pgfpathlineto{\pgfqpoint{3.888032in}{1.684130in}}%
\pgfpathlineto{\pgfqpoint{3.888608in}{1.690610in}}%
\pgfpathlineto{\pgfqpoint{3.889377in}{1.687891in}}%
\pgfpathlineto{\pgfqpoint{3.890146in}{1.680531in}}%
\pgfpathlineto{\pgfqpoint{3.890530in}{1.683704in}}%
\pgfpathlineto{\pgfqpoint{3.892067in}{1.695921in}}%
\pgfpathlineto{\pgfqpoint{3.892260in}{1.693402in}}%
\pgfpathlineto{\pgfqpoint{3.893028in}{1.700477in}}%
\pgfpathlineto{\pgfqpoint{3.893605in}{1.696500in}}%
\pgfpathlineto{\pgfqpoint{3.893797in}{1.697119in}}%
\pgfpathlineto{\pgfqpoint{3.893989in}{1.695568in}}%
\pgfpathlineto{\pgfqpoint{3.894758in}{1.691698in}}%
\pgfpathlineto{\pgfqpoint{3.894950in}{1.696332in}}%
\pgfpathlineto{\pgfqpoint{3.895527in}{1.701251in}}%
\pgfpathlineto{\pgfqpoint{3.896295in}{1.699178in}}%
\pgfpathlineto{\pgfqpoint{3.896487in}{1.698060in}}%
\pgfpathlineto{\pgfqpoint{3.897064in}{1.701087in}}%
\pgfpathlineto{\pgfqpoint{3.897833in}{1.704921in}}%
\pgfpathlineto{\pgfqpoint{3.898217in}{1.701955in}}%
\pgfpathlineto{\pgfqpoint{3.899562in}{1.692493in}}%
\pgfpathlineto{\pgfqpoint{3.899754in}{1.693954in}}%
\pgfpathlineto{\pgfqpoint{3.900331in}{1.692686in}}%
\pgfpathlineto{\pgfqpoint{3.900523in}{1.693095in}}%
\pgfpathlineto{\pgfqpoint{3.903213in}{1.710840in}}%
\pgfpathlineto{\pgfqpoint{3.903790in}{1.709814in}}%
\pgfpathlineto{\pgfqpoint{3.904366in}{1.713233in}}%
\pgfpathlineto{\pgfqpoint{3.904558in}{1.711575in}}%
\pgfpathlineto{\pgfqpoint{3.905135in}{1.714190in}}%
\pgfpathlineto{\pgfqpoint{3.905327in}{1.714823in}}%
\pgfpathlineto{\pgfqpoint{3.905904in}{1.713219in}}%
\pgfpathlineto{\pgfqpoint{3.906096in}{1.714511in}}%
\pgfpathlineto{\pgfqpoint{3.907249in}{1.701499in}}%
\pgfpathlineto{\pgfqpoint{3.908402in}{1.702187in}}%
\pgfpathlineto{\pgfqpoint{3.909363in}{1.701230in}}%
\pgfpathlineto{\pgfqpoint{3.909555in}{1.705180in}}%
\pgfpathlineto{\pgfqpoint{3.909747in}{1.704205in}}%
\pgfpathlineto{\pgfqpoint{3.909939in}{1.704361in}}%
\pgfpathlineto{\pgfqpoint{3.911284in}{1.717604in}}%
\pgfpathlineto{\pgfqpoint{3.911476in}{1.715442in}}%
\pgfpathlineto{\pgfqpoint{3.912053in}{1.715742in}}%
\pgfpathlineto{\pgfqpoint{3.913014in}{1.719879in}}%
\pgfpathlineto{\pgfqpoint{3.913206in}{1.716914in}}%
\pgfpathlineto{\pgfqpoint{3.913590in}{1.716085in}}%
\pgfpathlineto{\pgfqpoint{3.913975in}{1.714002in}}%
\pgfpathlineto{\pgfqpoint{3.914359in}{1.715807in}}%
\pgfpathlineto{\pgfqpoint{3.915704in}{1.722997in}}%
\pgfpathlineto{\pgfqpoint{3.915896in}{1.722390in}}%
\pgfpathlineto{\pgfqpoint{3.916088in}{1.725237in}}%
\pgfpathlineto{\pgfqpoint{3.916857in}{1.730252in}}%
\pgfpathlineto{\pgfqpoint{3.917241in}{1.726923in}}%
\pgfpathlineto{\pgfqpoint{3.918202in}{1.727982in}}%
\pgfpathlineto{\pgfqpoint{3.918587in}{1.722020in}}%
\pgfpathlineto{\pgfqpoint{3.919548in}{1.722303in}}%
\pgfpathlineto{\pgfqpoint{3.919740in}{1.722835in}}%
\pgfpathlineto{\pgfqpoint{3.919932in}{1.721367in}}%
\pgfpathlineto{\pgfqpoint{3.921854in}{1.710232in}}%
\pgfpathlineto{\pgfqpoint{3.920701in}{1.721675in}}%
\pgfpathlineto{\pgfqpoint{3.922046in}{1.711520in}}%
\pgfpathlineto{\pgfqpoint{3.923391in}{1.720122in}}%
\pgfpathlineto{\pgfqpoint{3.923583in}{1.718705in}}%
\pgfpathlineto{\pgfqpoint{3.924544in}{1.712666in}}%
\pgfpathlineto{\pgfqpoint{3.924928in}{1.709480in}}%
\pgfpathlineto{\pgfqpoint{3.925120in}{1.711217in}}%
\pgfpathlineto{\pgfqpoint{3.925313in}{1.714523in}}%
\pgfpathlineto{\pgfqpoint{3.926081in}{1.710352in}}%
\pgfpathlineto{\pgfqpoint{3.927042in}{1.716368in}}%
\pgfpathlineto{\pgfqpoint{3.928387in}{1.726824in}}%
\pgfpathlineto{\pgfqpoint{3.929156in}{1.723498in}}%
\pgfpathlineto{\pgfqpoint{3.929925in}{1.723891in}}%
\pgfpathlineto{\pgfqpoint{3.930501in}{1.723282in}}%
\pgfpathlineto{\pgfqpoint{3.931462in}{1.727950in}}%
\pgfpathlineto{\pgfqpoint{3.931654in}{1.723621in}}%
\pgfpathlineto{\pgfqpoint{3.932615in}{1.724773in}}%
\pgfpathlineto{\pgfqpoint{3.933384in}{1.730749in}}%
\pgfpathlineto{\pgfqpoint{3.933576in}{1.728862in}}%
\pgfpathlineto{\pgfqpoint{3.934537in}{1.719034in}}%
\pgfpathlineto{\pgfqpoint{3.934729in}{1.724007in}}%
\pgfpathlineto{\pgfqpoint{3.935113in}{1.726151in}}%
\pgfpathlineto{\pgfqpoint{3.935305in}{1.723052in}}%
\pgfpathlineto{\pgfqpoint{3.935690in}{1.724060in}}%
\pgfpathlineto{\pgfqpoint{3.936650in}{1.722309in}}%
\pgfpathlineto{\pgfqpoint{3.936266in}{1.727537in}}%
\pgfpathlineto{\pgfqpoint{3.936843in}{1.723180in}}%
\pgfpathlineto{\pgfqpoint{3.938188in}{1.731877in}}%
\pgfpathlineto{\pgfqpoint{3.939149in}{1.719585in}}%
\pgfpathlineto{\pgfqpoint{3.939533in}{1.721456in}}%
\pgfpathlineto{\pgfqpoint{3.940686in}{1.726222in}}%
\pgfpathlineto{\pgfqpoint{3.940302in}{1.720091in}}%
\pgfpathlineto{\pgfqpoint{3.940878in}{1.725348in}}%
\pgfpathlineto{\pgfqpoint{3.941455in}{1.724186in}}%
\pgfpathlineto{\pgfqpoint{3.941839in}{1.725872in}}%
\pgfpathlineto{\pgfqpoint{3.942031in}{1.726136in}}%
\pgfpathlineto{\pgfqpoint{3.942416in}{1.732238in}}%
\pgfpathlineto{\pgfqpoint{3.943184in}{1.730648in}}%
\pgfpathlineto{\pgfqpoint{3.943569in}{1.728873in}}%
\pgfpathlineto{\pgfqpoint{3.943761in}{1.729814in}}%
\pgfpathlineto{\pgfqpoint{3.944337in}{1.733405in}}%
\pgfpathlineto{\pgfqpoint{3.944914in}{1.731480in}}%
\pgfpathlineto{\pgfqpoint{3.945875in}{1.726937in}}%
\pgfpathlineto{\pgfqpoint{3.946067in}{1.729311in}}%
\pgfpathlineto{\pgfqpoint{3.946451in}{1.732737in}}%
\pgfpathlineto{\pgfqpoint{3.946643in}{1.728353in}}%
\pgfpathlineto{\pgfqpoint{3.946835in}{1.729300in}}%
\pgfpathlineto{\pgfqpoint{3.947796in}{1.724823in}}%
\pgfpathlineto{\pgfqpoint{3.948181in}{1.725047in}}%
\pgfpathlineto{\pgfqpoint{3.948949in}{1.728016in}}%
\pgfpathlineto{\pgfqpoint{3.948757in}{1.724395in}}%
\pgfpathlineto{\pgfqpoint{3.949141in}{1.725560in}}%
\pgfpathlineto{\pgfqpoint{3.949718in}{1.721164in}}%
\pgfpathlineto{\pgfqpoint{3.950294in}{1.721450in}}%
\pgfpathlineto{\pgfqpoint{3.952024in}{1.732039in}}%
\pgfpathlineto{\pgfqpoint{3.952216in}{1.729034in}}%
\pgfpathlineto{\pgfqpoint{3.953177in}{1.730490in}}%
\pgfpathlineto{\pgfqpoint{3.955099in}{1.719365in}}%
\pgfpathlineto{\pgfqpoint{3.956252in}{1.709204in}}%
\pgfpathlineto{\pgfqpoint{3.957020in}{1.710472in}}%
\pgfpathlineto{\pgfqpoint{3.957405in}{1.712802in}}%
\pgfpathlineto{\pgfqpoint{3.957597in}{1.704979in}}%
\pgfpathlineto{\pgfqpoint{3.957789in}{1.707033in}}%
\pgfpathlineto{\pgfqpoint{3.959711in}{1.714479in}}%
\pgfpathlineto{\pgfqpoint{3.959903in}{1.712278in}}%
\pgfpathlineto{\pgfqpoint{3.960671in}{1.715887in}}%
\pgfpathlineto{\pgfqpoint{3.961248in}{1.715108in}}%
\pgfpathlineto{\pgfqpoint{3.962017in}{1.716907in}}%
\pgfpathlineto{\pgfqpoint{3.962209in}{1.716427in}}%
\pgfpathlineto{\pgfqpoint{3.963746in}{1.722945in}}%
\pgfpathlineto{\pgfqpoint{3.963938in}{1.722779in}}%
\pgfpathlineto{\pgfqpoint{3.964130in}{1.724763in}}%
\pgfpathlineto{\pgfqpoint{3.964707in}{1.719604in}}%
\pgfpathlineto{\pgfqpoint{3.967397in}{1.693041in}}%
\pgfpathlineto{\pgfqpoint{3.968166in}{1.695416in}}%
\pgfpathlineto{\pgfqpoint{3.968550in}{1.695681in}}%
\pgfpathlineto{\pgfqpoint{3.968743in}{1.691922in}}%
\pgfpathlineto{\pgfqpoint{3.969703in}{1.692468in}}%
\pgfpathlineto{\pgfqpoint{3.970088in}{1.693209in}}%
\pgfpathlineto{\pgfqpoint{3.970472in}{1.690719in}}%
\pgfpathlineto{\pgfqpoint{3.971433in}{1.687568in}}%
\pgfpathlineto{\pgfqpoint{3.971625in}{1.688283in}}%
\pgfpathlineto{\pgfqpoint{3.972394in}{1.694579in}}%
\pgfpathlineto{\pgfqpoint{3.973355in}{1.694370in}}%
\pgfpathlineto{\pgfqpoint{3.973931in}{1.689304in}}%
\pgfpathlineto{\pgfqpoint{3.974315in}{1.691839in}}%
\pgfpathlineto{\pgfqpoint{3.975084in}{1.702475in}}%
\pgfpathlineto{\pgfqpoint{3.975661in}{1.697648in}}%
\pgfpathlineto{\pgfqpoint{3.975853in}{1.697265in}}%
\pgfpathlineto{\pgfqpoint{3.977582in}{1.710337in}}%
\pgfpathlineto{\pgfqpoint{3.977774in}{1.711476in}}%
\pgfpathlineto{\pgfqpoint{3.978351in}{1.709168in}}%
\pgfpathlineto{\pgfqpoint{3.978543in}{1.709755in}}%
\pgfpathlineto{\pgfqpoint{3.981810in}{1.721165in}}%
\pgfpathlineto{\pgfqpoint{3.982002in}{1.719168in}}%
\pgfpathlineto{\pgfqpoint{3.982579in}{1.710983in}}%
\pgfpathlineto{\pgfqpoint{3.983155in}{1.714655in}}%
\pgfpathlineto{\pgfqpoint{3.984500in}{1.721571in}}%
\pgfpathlineto{\pgfqpoint{3.984692in}{1.719985in}}%
\pgfpathlineto{\pgfqpoint{3.985077in}{1.723970in}}%
\pgfpathlineto{\pgfqpoint{3.985653in}{1.727805in}}%
\pgfpathlineto{\pgfqpoint{3.986038in}{1.727366in}}%
\pgfpathlineto{\pgfqpoint{3.987575in}{1.713147in}}%
\pgfpathlineto{\pgfqpoint{3.987767in}{1.714796in}}%
\pgfpathlineto{\pgfqpoint{3.987959in}{1.715020in}}%
\pgfpathlineto{\pgfqpoint{3.988151in}{1.714197in}}%
\pgfpathlineto{\pgfqpoint{3.989689in}{1.726868in}}%
\pgfpathlineto{\pgfqpoint{3.990457in}{1.723348in}}%
\pgfpathlineto{\pgfqpoint{3.990650in}{1.725555in}}%
\pgfpathlineto{\pgfqpoint{3.990842in}{1.720162in}}%
\pgfpathlineto{\pgfqpoint{3.991611in}{1.722802in}}%
\pgfpathlineto{\pgfqpoint{3.992379in}{1.726454in}}%
\pgfpathlineto{\pgfqpoint{3.992187in}{1.722742in}}%
\pgfpathlineto{\pgfqpoint{3.992764in}{1.724213in}}%
\pgfpathlineto{\pgfqpoint{3.993148in}{1.726766in}}%
\pgfpathlineto{\pgfqpoint{3.993340in}{1.724933in}}%
\pgfpathlineto{\pgfqpoint{3.994109in}{1.713767in}}%
\pgfpathlineto{\pgfqpoint{3.994493in}{1.721557in}}%
\pgfpathlineto{\pgfqpoint{3.995454in}{1.720894in}}%
\pgfpathlineto{\pgfqpoint{3.995838in}{1.725737in}}%
\pgfpathlineto{\pgfqpoint{3.996030in}{1.721011in}}%
\pgfpathlineto{\pgfqpoint{3.996991in}{1.724289in}}%
\pgfpathlineto{\pgfqpoint{3.997183in}{1.724401in}}%
\pgfpathlineto{\pgfqpoint{3.997760in}{1.720073in}}%
\pgfpathlineto{\pgfqpoint{3.998336in}{1.723770in}}%
\pgfpathlineto{\pgfqpoint{3.998529in}{1.728504in}}%
\pgfpathlineto{\pgfqpoint{3.998913in}{1.721053in}}%
\pgfpathlineto{\pgfqpoint{3.999489in}{1.725094in}}%
\pgfpathlineto{\pgfqpoint{4.000258in}{1.727786in}}%
\pgfpathlineto{\pgfqpoint{4.001411in}{1.732224in}}%
\pgfpathlineto{\pgfqpoint{4.002180in}{1.727764in}}%
\pgfpathlineto{\pgfqpoint{4.002564in}{1.730077in}}%
\pgfpathlineto{\pgfqpoint{4.002948in}{1.732663in}}%
\pgfpathlineto{\pgfqpoint{4.003909in}{1.743822in}}%
\pgfpathlineto{\pgfqpoint{4.004294in}{1.741150in}}%
\pgfpathlineto{\pgfqpoint{4.004678in}{1.740227in}}%
\pgfpathlineto{\pgfqpoint{4.004870in}{1.741573in}}%
\pgfpathlineto{\pgfqpoint{4.005062in}{1.742114in}}%
\pgfpathlineto{\pgfqpoint{4.005254in}{1.739605in}}%
\pgfpathlineto{\pgfqpoint{4.005447in}{1.737776in}}%
\pgfpathlineto{\pgfqpoint{4.006023in}{1.740736in}}%
\pgfpathlineto{\pgfqpoint{4.006215in}{1.740684in}}%
\pgfpathlineto{\pgfqpoint{4.006407in}{1.741592in}}%
\pgfpathlineto{\pgfqpoint{4.006792in}{1.739500in}}%
\pgfpathlineto{\pgfqpoint{4.007176in}{1.738252in}}%
\pgfpathlineto{\pgfqpoint{4.007368in}{1.741931in}}%
\pgfpathlineto{\pgfqpoint{4.007753in}{1.738986in}}%
\pgfpathlineto{\pgfqpoint{4.008521in}{1.737934in}}%
\pgfpathlineto{\pgfqpoint{4.009098in}{1.745166in}}%
\pgfpathlineto{\pgfqpoint{4.010059in}{1.750923in}}%
\pgfpathlineto{\pgfqpoint{4.010251in}{1.748772in}}%
\pgfpathlineto{\pgfqpoint{4.010635in}{1.750027in}}%
\pgfpathlineto{\pgfqpoint{4.011404in}{1.745615in}}%
\pgfpathlineto{\pgfqpoint{4.011596in}{1.747794in}}%
\pgfpathlineto{\pgfqpoint{4.012172in}{1.744663in}}%
\pgfpathlineto{\pgfqpoint{4.013325in}{1.738041in}}%
\pgfpathlineto{\pgfqpoint{4.013902in}{1.744710in}}%
\pgfpathlineto{\pgfqpoint{4.014286in}{1.743344in}}%
\pgfpathlineto{\pgfqpoint{4.014478in}{1.738309in}}%
\pgfpathlineto{\pgfqpoint{4.015247in}{1.746499in}}%
\pgfpathlineto{\pgfqpoint{4.015824in}{1.749490in}}%
\pgfpathlineto{\pgfqpoint{4.016208in}{1.752574in}}%
\pgfpathlineto{\pgfqpoint{4.016785in}{1.748330in}}%
\pgfpathlineto{\pgfqpoint{4.019859in}{1.733706in}}%
\pgfpathlineto{\pgfqpoint{4.020628in}{1.736346in}}%
\pgfpathlineto{\pgfqpoint{4.020820in}{1.738019in}}%
\pgfpathlineto{\pgfqpoint{4.021397in}{1.737540in}}%
\pgfpathlineto{\pgfqpoint{4.022550in}{1.730854in}}%
\pgfpathlineto{\pgfqpoint{4.022742in}{1.733059in}}%
\pgfpathlineto{\pgfqpoint{4.025240in}{1.745416in}}%
\pgfpathlineto{\pgfqpoint{4.023126in}{1.732882in}}%
\pgfpathlineto{\pgfqpoint{4.025432in}{1.743284in}}%
\pgfpathlineto{\pgfqpoint{4.026201in}{1.736246in}}%
\pgfpathlineto{\pgfqpoint{4.026585in}{1.740310in}}%
\pgfpathlineto{\pgfqpoint{4.026777in}{1.740198in}}%
\pgfpathlineto{\pgfqpoint{4.026969in}{1.742186in}}%
\pgfpathlineto{\pgfqpoint{4.027354in}{1.738688in}}%
\pgfpathlineto{\pgfqpoint{4.027738in}{1.741728in}}%
\pgfpathlineto{\pgfqpoint{4.027930in}{1.738999in}}%
\pgfpathlineto{\pgfqpoint{4.028891in}{1.741052in}}%
\pgfpathlineto{\pgfqpoint{4.030044in}{1.747090in}}%
\pgfpathlineto{\pgfqpoint{4.030428in}{1.747014in}}%
\pgfpathlineto{\pgfqpoint{4.031389in}{1.749383in}}%
\pgfpathlineto{\pgfqpoint{4.032350in}{1.746818in}}%
\pgfpathlineto{\pgfqpoint{4.032542in}{1.748589in}}%
\pgfpathlineto{\pgfqpoint{4.032734in}{1.749020in}}%
\pgfpathlineto{\pgfqpoint{4.032927in}{1.747236in}}%
\pgfpathlineto{\pgfqpoint{4.033119in}{1.747473in}}%
\pgfpathlineto{\pgfqpoint{4.034080in}{1.739521in}}%
\pgfpathlineto{\pgfqpoint{4.035233in}{1.741090in}}%
\pgfpathlineto{\pgfqpoint{4.035425in}{1.741560in}}%
\pgfpathlineto{\pgfqpoint{4.035617in}{1.740319in}}%
\pgfpathlineto{\pgfqpoint{4.035809in}{1.738416in}}%
\pgfpathlineto{\pgfqpoint{4.036386in}{1.741945in}}%
\pgfpathlineto{\pgfqpoint{4.036578in}{1.743001in}}%
\pgfpathlineto{\pgfqpoint{4.037154in}{1.741735in}}%
\pgfpathlineto{\pgfqpoint{4.037539in}{1.738028in}}%
\pgfpathlineto{\pgfqpoint{4.038307in}{1.741009in}}%
\pgfpathlineto{\pgfqpoint{4.038692in}{1.740125in}}%
\pgfpathlineto{\pgfqpoint{4.040421in}{1.729774in}}%
\pgfpathlineto{\pgfqpoint{4.041766in}{1.735976in}}%
\pgfpathlineto{\pgfqpoint{4.042919in}{1.731514in}}%
\pgfpathlineto{\pgfqpoint{4.043304in}{1.733036in}}%
\pgfpathlineto{\pgfqpoint{4.043496in}{1.735664in}}%
\pgfpathlineto{\pgfqpoint{4.044072in}{1.732079in}}%
\pgfpathlineto{\pgfqpoint{4.045225in}{1.720110in}}%
\pgfpathlineto{\pgfqpoint{4.045418in}{1.722852in}}%
\pgfpathlineto{\pgfqpoint{4.045610in}{1.725425in}}%
\pgfpathlineto{\pgfqpoint{4.046186in}{1.719672in}}%
\pgfpathlineto{\pgfqpoint{4.047531in}{1.714460in}}%
\pgfpathlineto{\pgfqpoint{4.047724in}{1.716029in}}%
\pgfpathlineto{\pgfqpoint{4.048300in}{1.719146in}}%
\pgfpathlineto{\pgfqpoint{4.048684in}{1.717553in}}%
\pgfpathlineto{\pgfqpoint{4.048877in}{1.715995in}}%
\pgfpathlineto{\pgfqpoint{4.049261in}{1.720729in}}%
\pgfpathlineto{\pgfqpoint{4.049453in}{1.720215in}}%
\pgfpathlineto{\pgfqpoint{4.049645in}{1.719913in}}%
\pgfpathlineto{\pgfqpoint{4.050414in}{1.727321in}}%
\pgfpathlineto{\pgfqpoint{4.050606in}{1.724647in}}%
\pgfpathlineto{\pgfqpoint{4.050798in}{1.717730in}}%
\pgfpathlineto{\pgfqpoint{4.051759in}{1.719393in}}%
\pgfpathlineto{\pgfqpoint{4.051951in}{1.719061in}}%
\pgfpathlineto{\pgfqpoint{4.052143in}{1.720372in}}%
\pgfpathlineto{\pgfqpoint{4.053681in}{1.725687in}}%
\pgfpathlineto{\pgfqpoint{4.054449in}{1.728736in}}%
\pgfpathlineto{\pgfqpoint{4.054642in}{1.727432in}}%
\pgfpathlineto{\pgfqpoint{4.055987in}{1.717718in}}%
\pgfpathlineto{\pgfqpoint{4.056371in}{1.718108in}}%
\pgfpathlineto{\pgfqpoint{4.057140in}{1.720410in}}%
\pgfpathlineto{\pgfqpoint{4.056948in}{1.717346in}}%
\pgfpathlineto{\pgfqpoint{4.057332in}{1.718830in}}%
\pgfpathlineto{\pgfqpoint{4.057716in}{1.715603in}}%
\pgfpathlineto{\pgfqpoint{4.057908in}{1.719643in}}%
\pgfpathlineto{\pgfqpoint{4.058101in}{1.717976in}}%
\pgfpathlineto{\pgfqpoint{4.059638in}{1.724882in}}%
\pgfpathlineto{\pgfqpoint{4.060983in}{1.716116in}}%
\pgfpathlineto{\pgfqpoint{4.061944in}{1.722911in}}%
\pgfpathlineto{\pgfqpoint{4.062328in}{1.722264in}}%
\pgfpathlineto{\pgfqpoint{4.062520in}{1.722218in}}%
\pgfpathlineto{\pgfqpoint{4.062905in}{1.719112in}}%
\pgfpathlineto{\pgfqpoint{4.063289in}{1.725344in}}%
\pgfpathlineto{\pgfqpoint{4.063481in}{1.725519in}}%
\pgfpathlineto{\pgfqpoint{4.065595in}{1.739020in}}%
\pgfpathlineto{\pgfqpoint{4.066556in}{1.741158in}}%
\pgfpathlineto{\pgfqpoint{4.066748in}{1.740101in}}%
\pgfpathlineto{\pgfqpoint{4.066940in}{1.744246in}}%
\pgfpathlineto{\pgfqpoint{4.067709in}{1.740961in}}%
\pgfpathlineto{\pgfqpoint{4.069054in}{1.735695in}}%
\pgfpathlineto{\pgfqpoint{4.070592in}{1.727027in}}%
\pgfpathlineto{\pgfqpoint{4.073282in}{1.740105in}}%
\pgfpathlineto{\pgfqpoint{4.073474in}{1.740190in}}%
\pgfpathlineto{\pgfqpoint{4.075011in}{1.731421in}}%
\pgfpathlineto{\pgfqpoint{4.075396in}{1.734602in}}%
\pgfpathlineto{\pgfqpoint{4.075972in}{1.732999in}}%
\pgfpathlineto{\pgfqpoint{4.076357in}{1.730372in}}%
\pgfpathlineto{\pgfqpoint{4.076933in}{1.734120in}}%
\pgfpathlineto{\pgfqpoint{4.077702in}{1.730844in}}%
\pgfpathlineto{\pgfqpoint{4.077510in}{1.734355in}}%
\pgfpathlineto{\pgfqpoint{4.078278in}{1.730968in}}%
\pgfpathlineto{\pgfqpoint{4.078470in}{1.732549in}}%
\pgfpathlineto{\pgfqpoint{4.078855in}{1.727654in}}%
\pgfpathlineto{\pgfqpoint{4.079239in}{1.724120in}}%
\pgfpathlineto{\pgfqpoint{4.079623in}{1.729089in}}%
\pgfpathlineto{\pgfqpoint{4.079816in}{1.728077in}}%
\pgfpathlineto{\pgfqpoint{4.080200in}{1.728607in}}%
\pgfpathlineto{\pgfqpoint{4.080392in}{1.731624in}}%
\pgfpathlineto{\pgfqpoint{4.080969in}{1.729625in}}%
\pgfpathlineto{\pgfqpoint{4.081353in}{1.721847in}}%
\pgfpathlineto{\pgfqpoint{4.082314in}{1.722270in}}%
\pgfpathlineto{\pgfqpoint{4.082698in}{1.723929in}}%
\pgfpathlineto{\pgfqpoint{4.082890in}{1.721775in}}%
\pgfpathlineto{\pgfqpoint{4.083082in}{1.721884in}}%
\pgfpathlineto{\pgfqpoint{4.085581in}{1.705304in}}%
\pgfpathlineto{\pgfqpoint{4.085773in}{1.703766in}}%
\pgfpathlineto{\pgfqpoint{4.086157in}{1.706452in}}%
\pgfpathlineto{\pgfqpoint{4.086541in}{1.704494in}}%
\pgfpathlineto{\pgfqpoint{4.087118in}{1.708741in}}%
\pgfpathlineto{\pgfqpoint{4.087502in}{1.707104in}}%
\pgfpathlineto{\pgfqpoint{4.089040in}{1.697038in}}%
\pgfpathlineto{\pgfqpoint{4.089424in}{1.697877in}}%
\pgfpathlineto{\pgfqpoint{4.090577in}{1.693220in}}%
\pgfpathlineto{\pgfqpoint{4.090769in}{1.692855in}}%
\pgfpathlineto{\pgfqpoint{4.091538in}{1.699636in}}%
\pgfpathlineto{\pgfqpoint{4.091922in}{1.694577in}}%
\pgfpathlineto{\pgfqpoint{4.093075in}{1.697168in}}%
\pgfpathlineto{\pgfqpoint{4.092691in}{1.693922in}}%
\pgfpathlineto{\pgfqpoint{4.093267in}{1.695954in}}%
\pgfpathlineto{\pgfqpoint{4.094036in}{1.691184in}}%
\pgfpathlineto{\pgfqpoint{4.094420in}{1.693942in}}%
\pgfpathlineto{\pgfqpoint{4.094613in}{1.696312in}}%
\pgfpathlineto{\pgfqpoint{4.095381in}{1.693177in}}%
\pgfpathlineto{\pgfqpoint{4.097879in}{1.702171in}}%
\pgfpathlineto{\pgfqpoint{4.098072in}{1.699958in}}%
\pgfpathlineto{\pgfqpoint{4.099993in}{1.692575in}}%
\pgfpathlineto{\pgfqpoint{4.100185in}{1.694364in}}%
\pgfpathlineto{\pgfqpoint{4.101723in}{1.697845in}}%
\pgfpathlineto{\pgfqpoint{4.103260in}{1.693971in}}%
\pgfpathlineto{\pgfqpoint{4.102299in}{1.698332in}}%
\pgfpathlineto{\pgfqpoint{4.103837in}{1.695578in}}%
\pgfpathlineto{\pgfqpoint{4.104029in}{1.698270in}}%
\pgfpathlineto{\pgfqpoint{4.104605in}{1.696897in}}%
\pgfpathlineto{\pgfqpoint{4.105566in}{1.686479in}}%
\pgfpathlineto{\pgfqpoint{4.105950in}{1.687910in}}%
\pgfpathlineto{\pgfqpoint{4.106719in}{1.683742in}}%
\pgfpathlineto{\pgfqpoint{4.107103in}{1.686357in}}%
\pgfpathlineto{\pgfqpoint{4.107680in}{1.689563in}}%
\pgfpathlineto{\pgfqpoint{4.108449in}{1.691713in}}%
\pgfpathlineto{\pgfqpoint{4.108256in}{1.688938in}}%
\pgfpathlineto{\pgfqpoint{4.108833in}{1.689983in}}%
\pgfpathlineto{\pgfqpoint{4.109602in}{1.691614in}}%
\pgfpathlineto{\pgfqpoint{4.109409in}{1.688934in}}%
\pgfpathlineto{\pgfqpoint{4.110178in}{1.691399in}}%
\pgfpathlineto{\pgfqpoint{4.111523in}{1.684751in}}%
\pgfpathlineto{\pgfqpoint{4.112292in}{1.686443in}}%
\pgfpathlineto{\pgfqpoint{4.111908in}{1.684316in}}%
\pgfpathlineto{\pgfqpoint{4.112484in}{1.685813in}}%
\pgfpathlineto{\pgfqpoint{4.112676in}{1.684892in}}%
\pgfpathlineto{\pgfqpoint{4.112868in}{1.687186in}}%
\pgfpathlineto{\pgfqpoint{4.113253in}{1.685243in}}%
\pgfpathlineto{\pgfqpoint{4.114598in}{1.692523in}}%
\pgfpathlineto{\pgfqpoint{4.116135in}{1.684435in}}%
\pgfpathlineto{\pgfqpoint{4.116328in}{1.684171in}}%
\pgfpathlineto{\pgfqpoint{4.116712in}{1.680579in}}%
\pgfpathlineto{\pgfqpoint{4.117481in}{1.681439in}}%
\pgfpathlineto{\pgfqpoint{4.117673in}{1.682015in}}%
\pgfpathlineto{\pgfqpoint{4.117865in}{1.680618in}}%
\pgfpathlineto{\pgfqpoint{4.118057in}{1.680843in}}%
\pgfpathlineto{\pgfqpoint{4.118634in}{1.677469in}}%
\pgfpathlineto{\pgfqpoint{4.119018in}{1.680748in}}%
\pgfpathlineto{\pgfqpoint{4.119210in}{1.680346in}}%
\pgfpathlineto{\pgfqpoint{4.119402in}{1.681814in}}%
\pgfpathlineto{\pgfqpoint{4.120363in}{1.687018in}}%
\pgfpathlineto{\pgfqpoint{4.120747in}{1.683500in}}%
\pgfpathlineto{\pgfqpoint{4.121900in}{1.676795in}}%
\pgfpathlineto{\pgfqpoint{4.122093in}{1.680867in}}%
\pgfpathlineto{\pgfqpoint{4.123630in}{1.685170in}}%
\pgfpathlineto{\pgfqpoint{4.124591in}{1.677237in}}%
\pgfpathlineto{\pgfqpoint{4.124783in}{1.678584in}}%
\pgfpathlineto{\pgfqpoint{4.125552in}{1.686278in}}%
\pgfpathlineto{\pgfqpoint{4.126320in}{1.682594in}}%
\pgfpathlineto{\pgfqpoint{4.126705in}{1.682566in}}%
\pgfpathlineto{\pgfqpoint{4.127473in}{1.673706in}}%
\pgfpathlineto{\pgfqpoint{4.128434in}{1.674812in}}%
\pgfpathlineto{\pgfqpoint{4.129971in}{1.686280in}}%
\pgfpathlineto{\pgfqpoint{4.130164in}{1.685448in}}%
\pgfpathlineto{\pgfqpoint{4.130932in}{1.678308in}}%
\pgfpathlineto{\pgfqpoint{4.131509in}{1.681279in}}%
\pgfpathlineto{\pgfqpoint{4.132662in}{1.673261in}}%
\pgfpathlineto{\pgfqpoint{4.131893in}{1.681634in}}%
\pgfpathlineto{\pgfqpoint{4.132854in}{1.675489in}}%
\pgfpathlineto{\pgfqpoint{4.133430in}{1.678105in}}%
\pgfpathlineto{\pgfqpoint{4.134007in}{1.676953in}}%
\pgfpathlineto{\pgfqpoint{4.134391in}{1.669352in}}%
\pgfpathlineto{\pgfqpoint{4.135352in}{1.674106in}}%
\pgfpathlineto{\pgfqpoint{4.136313in}{1.671051in}}%
\pgfpathlineto{\pgfqpoint{4.135736in}{1.675932in}}%
\pgfpathlineto{\pgfqpoint{4.136505in}{1.672876in}}%
\pgfpathlineto{\pgfqpoint{4.137658in}{1.684043in}}%
\pgfpathlineto{\pgfqpoint{4.138043in}{1.680536in}}%
\pgfpathlineto{\pgfqpoint{4.139772in}{1.675258in}}%
\pgfpathlineto{\pgfqpoint{4.138427in}{1.681924in}}%
\pgfpathlineto{\pgfqpoint{4.139964in}{1.677715in}}%
\pgfpathlineto{\pgfqpoint{4.140733in}{1.680309in}}%
\pgfpathlineto{\pgfqpoint{4.141117in}{1.678300in}}%
\pgfpathlineto{\pgfqpoint{4.141502in}{1.677146in}}%
\pgfpathlineto{\pgfqpoint{4.141886in}{1.678716in}}%
\pgfpathlineto{\pgfqpoint{4.142078in}{1.677570in}}%
\pgfpathlineto{\pgfqpoint{4.142847in}{1.678915in}}%
\pgfpathlineto{\pgfqpoint{4.143039in}{1.678436in}}%
\pgfpathlineto{\pgfqpoint{4.143423in}{1.680112in}}%
\pgfpathlineto{\pgfqpoint{4.144384in}{1.687850in}}%
\pgfpathlineto{\pgfqpoint{4.144576in}{1.685498in}}%
\pgfpathlineto{\pgfqpoint{4.145921in}{1.675988in}}%
\pgfpathlineto{\pgfqpoint{4.146114in}{1.677271in}}%
\pgfpathlineto{\pgfqpoint{4.146498in}{1.674466in}}%
\pgfpathlineto{\pgfqpoint{4.146882in}{1.676469in}}%
\pgfpathlineto{\pgfqpoint{4.147074in}{1.679243in}}%
\pgfpathlineto{\pgfqpoint{4.147651in}{1.675123in}}%
\pgfpathlineto{\pgfqpoint{4.148035in}{1.678423in}}%
\pgfpathlineto{\pgfqpoint{4.148804in}{1.673844in}}%
\pgfpathlineto{\pgfqpoint{4.149765in}{1.665981in}}%
\pgfpathlineto{\pgfqpoint{4.149957in}{1.669784in}}%
\pgfpathlineto{\pgfqpoint{4.151494in}{1.674429in}}%
\pgfpathlineto{\pgfqpoint{4.151686in}{1.674106in}}%
\pgfpathlineto{\pgfqpoint{4.151879in}{1.669894in}}%
\pgfpathlineto{\pgfqpoint{4.152647in}{1.677156in}}%
\pgfpathlineto{\pgfqpoint{4.154569in}{1.682580in}}%
\pgfpathlineto{\pgfqpoint{4.154761in}{1.680952in}}%
\pgfpathlineto{\pgfqpoint{4.155145in}{1.680347in}}%
\pgfpathlineto{\pgfqpoint{4.156683in}{1.671610in}}%
\pgfpathlineto{\pgfqpoint{4.156875in}{1.672562in}}%
\pgfpathlineto{\pgfqpoint{4.157259in}{1.672376in}}%
\pgfpathlineto{\pgfqpoint{4.158028in}{1.675655in}}%
\pgfpathlineto{\pgfqpoint{4.158412in}{1.671340in}}%
\pgfpathlineto{\pgfqpoint{4.158797in}{1.676903in}}%
\pgfpathlineto{\pgfqpoint{4.159950in}{1.686824in}}%
\pgfpathlineto{\pgfqpoint{4.160718in}{1.686230in}}%
\pgfpathlineto{\pgfqpoint{4.161103in}{1.687742in}}%
\pgfpathlineto{\pgfqpoint{4.162063in}{1.682208in}}%
\pgfpathlineto{\pgfqpoint{4.163409in}{1.687024in}}%
\pgfpathlineto{\pgfqpoint{4.163793in}{1.683948in}}%
\pgfpathlineto{\pgfqpoint{4.163985in}{1.686797in}}%
\pgfpathlineto{\pgfqpoint{4.164946in}{1.690074in}}%
\pgfpathlineto{\pgfqpoint{4.164562in}{1.685811in}}%
\pgfpathlineto{\pgfqpoint{4.165138in}{1.688242in}}%
\pgfpathlineto{\pgfqpoint{4.165330in}{1.689496in}}%
\pgfpathlineto{\pgfqpoint{4.165907in}{1.687677in}}%
\pgfpathlineto{\pgfqpoint{4.166099in}{1.689104in}}%
\pgfpathlineto{\pgfqpoint{4.166676in}{1.685659in}}%
\pgfpathlineto{\pgfqpoint{4.167060in}{1.687616in}}%
\pgfpathlineto{\pgfqpoint{4.167444in}{1.691986in}}%
\pgfpathlineto{\pgfqpoint{4.168021in}{1.687518in}}%
\pgfpathlineto{\pgfqpoint{4.168213in}{1.688291in}}%
\pgfpathlineto{\pgfqpoint{4.168405in}{1.688411in}}%
\pgfpathlineto{\pgfqpoint{4.169366in}{1.684222in}}%
\pgfpathlineto{\pgfqpoint{4.169558in}{1.684904in}}%
\pgfpathlineto{\pgfqpoint{4.171288in}{1.692963in}}%
\pgfpathlineto{\pgfqpoint{4.171480in}{1.689502in}}%
\pgfpathlineto{\pgfqpoint{4.172056in}{1.683968in}}%
\pgfpathlineto{\pgfqpoint{4.172825in}{1.684465in}}%
\pgfpathlineto{\pgfqpoint{4.173786in}{1.687011in}}%
\pgfpathlineto{\pgfqpoint{4.175131in}{1.679438in}}%
\pgfpathlineto{\pgfqpoint{4.175515in}{1.676997in}}%
\pgfpathlineto{\pgfqpoint{4.176092in}{1.679908in}}%
\pgfpathlineto{\pgfqpoint{4.176284in}{1.679671in}}%
\pgfpathlineto{\pgfqpoint{4.176476in}{1.677290in}}%
\pgfpathlineto{\pgfqpoint{4.177245in}{1.680483in}}%
\pgfpathlineto{\pgfqpoint{4.177437in}{1.678467in}}%
\pgfpathlineto{\pgfqpoint{4.177821in}{1.678818in}}%
\pgfpathlineto{\pgfqpoint{4.178206in}{1.678213in}}%
\pgfpathlineto{\pgfqpoint{4.179551in}{1.687306in}}%
\pgfpathlineto{\pgfqpoint{4.180896in}{1.691622in}}%
\pgfpathlineto{\pgfqpoint{4.182818in}{1.685343in}}%
\pgfpathlineto{\pgfqpoint{4.183586in}{1.695080in}}%
\pgfpathlineto{\pgfqpoint{4.183971in}{1.687346in}}%
\pgfpathlineto{\pgfqpoint{4.184547in}{1.692374in}}%
\pgfpathlineto{\pgfqpoint{4.185316in}{1.689037in}}%
\pgfpathlineto{\pgfqpoint{4.185700in}{1.688014in}}%
\pgfpathlineto{\pgfqpoint{4.186084in}{1.687637in}}%
\pgfpathlineto{\pgfqpoint{4.187238in}{1.695294in}}%
\pgfpathlineto{\pgfqpoint{4.187430in}{1.694786in}}%
\pgfpathlineto{\pgfqpoint{4.187622in}{1.695058in}}%
\pgfpathlineto{\pgfqpoint{4.188775in}{1.700452in}}%
\pgfpathlineto{\pgfqpoint{4.188967in}{1.698539in}}%
\pgfpathlineto{\pgfqpoint{4.189544in}{1.701356in}}%
\pgfpathlineto{\pgfqpoint{4.189736in}{1.702168in}}%
\pgfpathlineto{\pgfqpoint{4.190120in}{1.701721in}}%
\pgfpathlineto{\pgfqpoint{4.190889in}{1.697677in}}%
\pgfpathlineto{\pgfqpoint{4.191273in}{1.700347in}}%
\pgfpathlineto{\pgfqpoint{4.191657in}{1.702831in}}%
\pgfpathlineto{\pgfqpoint{4.192042in}{1.699126in}}%
\pgfpathlineto{\pgfqpoint{4.193195in}{1.694673in}}%
\pgfpathlineto{\pgfqpoint{4.193579in}{1.694774in}}%
\pgfpathlineto{\pgfqpoint{4.194924in}{1.698690in}}%
\pgfpathlineto{\pgfqpoint{4.197615in}{1.677957in}}%
\pgfpathlineto{\pgfqpoint{4.198383in}{1.686804in}}%
\pgfpathlineto{\pgfqpoint{4.198960in}{1.684383in}}%
\pgfpathlineto{\pgfqpoint{4.199344in}{1.684766in}}%
\pgfpathlineto{\pgfqpoint{4.200305in}{1.694315in}}%
\pgfpathlineto{\pgfqpoint{4.199728in}{1.684332in}}%
\pgfpathlineto{\pgfqpoint{4.201458in}{1.692077in}}%
\pgfpathlineto{\pgfqpoint{4.201650in}{1.691594in}}%
\pgfpathlineto{\pgfqpoint{4.201842in}{1.693136in}}%
\pgfpathlineto{\pgfqpoint{4.204533in}{1.708671in}}%
\pgfpathlineto{\pgfqpoint{4.204725in}{1.705664in}}%
\pgfpathlineto{\pgfqpoint{4.205686in}{1.700484in}}%
\pgfpathlineto{\pgfqpoint{4.205878in}{1.702838in}}%
\pgfpathlineto{\pgfqpoint{4.206070in}{1.703071in}}%
\pgfpathlineto{\pgfqpoint{4.207607in}{1.698091in}}%
\pgfpathlineto{\pgfqpoint{4.208184in}{1.694546in}}%
\pgfpathlineto{\pgfqpoint{4.209913in}{1.679271in}}%
\pgfpathlineto{\pgfqpoint{4.210490in}{1.672950in}}%
\pgfpathlineto{\pgfqpoint{4.211066in}{1.678728in}}%
\pgfpathlineto{\pgfqpoint{4.211451in}{1.677754in}}%
\pgfpathlineto{\pgfqpoint{4.211835in}{1.681229in}}%
\pgfpathlineto{\pgfqpoint{4.212027in}{1.682965in}}%
\pgfpathlineto{\pgfqpoint{4.212412in}{1.678501in}}%
\pgfpathlineto{\pgfqpoint{4.212796in}{1.680572in}}%
\pgfpathlineto{\pgfqpoint{4.213180in}{1.679856in}}%
\pgfpathlineto{\pgfqpoint{4.213372in}{1.680990in}}%
\pgfpathlineto{\pgfqpoint{4.214333in}{1.687835in}}%
\pgfpathlineto{\pgfqpoint{4.214718in}{1.687077in}}%
\pgfpathlineto{\pgfqpoint{4.215294in}{1.687962in}}%
\pgfpathlineto{\pgfqpoint{4.215678in}{1.689315in}}%
\pgfpathlineto{\pgfqpoint{4.216063in}{1.688210in}}%
\pgfpathlineto{\pgfqpoint{4.217792in}{1.679117in}}%
\pgfpathlineto{\pgfqpoint{4.218177in}{1.678376in}}%
\pgfpathlineto{\pgfqpoint{4.218945in}{1.686782in}}%
\pgfpathlineto{\pgfqpoint{4.219906in}{1.693148in}}%
\pgfpathlineto{\pgfqpoint{4.219330in}{1.683956in}}%
\pgfpathlineto{\pgfqpoint{4.220483in}{1.691640in}}%
\pgfpathlineto{\pgfqpoint{4.221443in}{1.696856in}}%
\pgfpathlineto{\pgfqpoint{4.221636in}{1.695245in}}%
\pgfpathlineto{\pgfqpoint{4.223365in}{1.680965in}}%
\pgfpathlineto{\pgfqpoint{4.223557in}{1.684170in}}%
\pgfpathlineto{\pgfqpoint{4.225287in}{1.674705in}}%
\pgfpathlineto{\pgfqpoint{4.225671in}{1.676854in}}%
\pgfpathlineto{\pgfqpoint{4.226440in}{1.675323in}}%
\pgfpathlineto{\pgfqpoint{4.226632in}{1.679563in}}%
\pgfpathlineto{\pgfqpoint{4.227401in}{1.674377in}}%
\pgfpathlineto{\pgfqpoint{4.227785in}{1.677480in}}%
\pgfpathlineto{\pgfqpoint{4.227977in}{1.677504in}}%
\pgfpathlineto{\pgfqpoint{4.228938in}{1.687449in}}%
\pgfpathlineto{\pgfqpoint{4.229514in}{1.682858in}}%
\pgfpathlineto{\pgfqpoint{4.230091in}{1.683173in}}%
\pgfpathlineto{\pgfqpoint{4.230860in}{1.685462in}}%
\pgfpathlineto{\pgfqpoint{4.231244in}{1.680782in}}%
\pgfpathlineto{\pgfqpoint{4.231436in}{1.683534in}}%
\pgfpathlineto{\pgfqpoint{4.232013in}{1.679550in}}%
\pgfpathlineto{\pgfqpoint{4.232205in}{1.680658in}}%
\pgfpathlineto{\pgfqpoint{4.232589in}{1.684101in}}%
\pgfpathlineto{\pgfqpoint{4.233550in}{1.691447in}}%
\pgfpathlineto{\pgfqpoint{4.233934in}{1.690512in}}%
\pgfpathlineto{\pgfqpoint{4.234126in}{1.689054in}}%
\pgfpathlineto{\pgfqpoint{4.234511in}{1.691723in}}%
\pgfpathlineto{\pgfqpoint{4.235664in}{1.696506in}}%
\pgfpathlineto{\pgfqpoint{4.235856in}{1.695198in}}%
\pgfpathlineto{\pgfqpoint{4.237586in}{1.687452in}}%
\pgfpathlineto{\pgfqpoint{4.237970in}{1.688046in}}%
\pgfpathlineto{\pgfqpoint{4.238354in}{1.685669in}}%
\pgfpathlineto{\pgfqpoint{4.238931in}{1.693180in}}%
\pgfpathlineto{\pgfqpoint{4.239699in}{1.691935in}}%
\pgfpathlineto{\pgfqpoint{4.239892in}{1.689739in}}%
\pgfpathlineto{\pgfqpoint{4.240276in}{1.695075in}}%
\pgfpathlineto{\pgfqpoint{4.241237in}{1.699277in}}%
\pgfpathlineto{\pgfqpoint{4.241429in}{1.697847in}}%
\pgfpathlineto{\pgfqpoint{4.241621in}{1.695831in}}%
\pgfpathlineto{\pgfqpoint{4.242198in}{1.698934in}}%
\pgfpathlineto{\pgfqpoint{4.242390in}{1.697209in}}%
\pgfpathlineto{\pgfqpoint{4.243158in}{1.695439in}}%
\pgfpathlineto{\pgfqpoint{4.243735in}{1.700802in}}%
\pgfpathlineto{\pgfqpoint{4.243927in}{1.701210in}}%
\pgfpathlineto{\pgfqpoint{4.245080in}{1.714064in}}%
\pgfpathlineto{\pgfqpoint{4.245464in}{1.709889in}}%
\pgfpathlineto{\pgfqpoint{4.245849in}{1.705574in}}%
\pgfpathlineto{\pgfqpoint{4.246233in}{1.711069in}}%
\pgfpathlineto{\pgfqpoint{4.246810in}{1.714996in}}%
\pgfpathlineto{\pgfqpoint{4.247386in}{1.711622in}}%
\pgfpathlineto{\pgfqpoint{4.248155in}{1.706326in}}%
\pgfpathlineto{\pgfqpoint{4.248347in}{1.707496in}}%
\pgfpathlineto{\pgfqpoint{4.249116in}{1.716638in}}%
\pgfpathlineto{\pgfqpoint{4.249692in}{1.713439in}}%
\pgfpathlineto{\pgfqpoint{4.250845in}{1.708142in}}%
\pgfpathlineto{\pgfqpoint{4.251037in}{1.710820in}}%
\pgfpathlineto{\pgfqpoint{4.251422in}{1.713781in}}%
\pgfpathlineto{\pgfqpoint{4.251998in}{1.710452in}}%
\pgfpathlineto{\pgfqpoint{4.252190in}{1.711243in}}%
\pgfpathlineto{\pgfqpoint{4.252575in}{1.713066in}}%
\pgfpathlineto{\pgfqpoint{4.254112in}{1.705173in}}%
\pgfpathlineto{\pgfqpoint{4.254304in}{1.705504in}}%
\pgfpathlineto{\pgfqpoint{4.254496in}{1.704935in}}%
\pgfpathlineto{\pgfqpoint{4.254881in}{1.702082in}}%
\pgfpathlineto{\pgfqpoint{4.255841in}{1.702854in}}%
\pgfpathlineto{\pgfqpoint{4.256418in}{1.707165in}}%
\pgfpathlineto{\pgfqpoint{4.256610in}{1.705349in}}%
\pgfpathlineto{\pgfqpoint{4.256802in}{1.701714in}}%
\pgfpathlineto{\pgfqpoint{4.257187in}{1.707752in}}%
\pgfpathlineto{\pgfqpoint{4.257571in}{1.705988in}}%
\pgfpathlineto{\pgfqpoint{4.258340in}{1.705641in}}%
\pgfpathlineto{\pgfqpoint{4.258916in}{1.709164in}}%
\pgfpathlineto{\pgfqpoint{4.261607in}{1.698730in}}%
\pgfpathlineto{\pgfqpoint{4.262183in}{1.706002in}}%
\pgfpathlineto{\pgfqpoint{4.262760in}{1.702850in}}%
\pgfpathlineto{\pgfqpoint{4.264681in}{1.688708in}}%
\pgfpathlineto{\pgfqpoint{4.265258in}{1.693283in}}%
\pgfpathlineto{\pgfqpoint{4.265450in}{1.693773in}}%
\pgfpathlineto{\pgfqpoint{4.265642in}{1.692044in}}%
\pgfpathlineto{\pgfqpoint{4.265834in}{1.691550in}}%
\pgfpathlineto{\pgfqpoint{4.266026in}{1.692137in}}%
\pgfpathlineto{\pgfqpoint{4.267564in}{1.700500in}}%
\pgfpathlineto{\pgfqpoint{4.269293in}{1.687939in}}%
\pgfpathlineto{\pgfqpoint{4.270062in}{1.689987in}}%
\pgfpathlineto{\pgfqpoint{4.269870in}{1.687692in}}%
\pgfpathlineto{\pgfqpoint{4.270254in}{1.689319in}}%
\pgfpathlineto{\pgfqpoint{4.271215in}{1.687034in}}%
\pgfpathlineto{\pgfqpoint{4.271407in}{1.687888in}}%
\pgfpathlineto{\pgfqpoint{4.272560in}{1.685274in}}%
\pgfpathlineto{\pgfqpoint{4.272752in}{1.685609in}}%
\pgfpathlineto{\pgfqpoint{4.273521in}{1.682231in}}%
\pgfpathlineto{\pgfqpoint{4.273713in}{1.684792in}}%
\pgfpathlineto{\pgfqpoint{4.273905in}{1.688443in}}%
\pgfpathlineto{\pgfqpoint{4.274482in}{1.682424in}}%
\pgfpathlineto{\pgfqpoint{4.274866in}{1.683469in}}%
\pgfpathlineto{\pgfqpoint{4.275827in}{1.676845in}}%
\pgfpathlineto{\pgfqpoint{4.276019in}{1.678478in}}%
\pgfpathlineto{\pgfqpoint{4.276211in}{1.674697in}}%
\pgfpathlineto{\pgfqpoint{4.276403in}{1.675199in}}%
\pgfpathlineto{\pgfqpoint{4.277941in}{1.661759in}}%
\pgfpathlineto{\pgfqpoint{4.278517in}{1.662621in}}%
\pgfpathlineto{\pgfqpoint{4.279094in}{1.667476in}}%
\pgfpathlineto{\pgfqpoint{4.279478in}{1.664921in}}%
\pgfpathlineto{\pgfqpoint{4.279670in}{1.660165in}}%
\pgfpathlineto{\pgfqpoint{4.280055in}{1.665443in}}%
\pgfpathlineto{\pgfqpoint{4.280439in}{1.665076in}}%
\pgfpathlineto{\pgfqpoint{4.281592in}{1.669883in}}%
\pgfpathlineto{\pgfqpoint{4.281208in}{1.663831in}}%
\pgfpathlineto{\pgfqpoint{4.281784in}{1.667992in}}%
\pgfpathlineto{\pgfqpoint{4.282361in}{1.665231in}}%
\pgfpathlineto{\pgfqpoint{4.282745in}{1.667727in}}%
\pgfpathlineto{\pgfqpoint{4.283514in}{1.673187in}}%
\pgfpathlineto{\pgfqpoint{4.283898in}{1.672417in}}%
\pgfpathlineto{\pgfqpoint{4.284475in}{1.669603in}}%
\pgfpathlineto{\pgfqpoint{4.284282in}{1.672786in}}%
\pgfpathlineto{\pgfqpoint{4.284859in}{1.670384in}}%
\pgfpathlineto{\pgfqpoint{4.285820in}{1.678441in}}%
\pgfpathlineto{\pgfqpoint{4.286012in}{1.676128in}}%
\pgfpathlineto{\pgfqpoint{4.286204in}{1.675784in}}%
\pgfpathlineto{\pgfqpoint{4.286588in}{1.676018in}}%
\pgfpathlineto{\pgfqpoint{4.287934in}{1.684022in}}%
\pgfpathlineto{\pgfqpoint{4.288510in}{1.683566in}}%
\pgfpathlineto{\pgfqpoint{4.289087in}{1.684283in}}%
\pgfpathlineto{\pgfqpoint{4.290432in}{1.688144in}}%
\pgfpathlineto{\pgfqpoint{4.290624in}{1.687855in}}%
\pgfpathlineto{\pgfqpoint{4.290816in}{1.689880in}}%
\pgfpathlineto{\pgfqpoint{4.291393in}{1.685556in}}%
\pgfpathlineto{\pgfqpoint{4.292353in}{1.682063in}}%
\pgfpathlineto{\pgfqpoint{4.292738in}{1.687070in}}%
\pgfpathlineto{\pgfqpoint{4.293506in}{1.683656in}}%
\pgfpathlineto{\pgfqpoint{4.293699in}{1.684859in}}%
\pgfpathlineto{\pgfqpoint{4.293891in}{1.682748in}}%
\pgfpathlineto{\pgfqpoint{4.294083in}{1.679716in}}%
\pgfpathlineto{\pgfqpoint{4.294852in}{1.682736in}}%
\pgfpathlineto{\pgfqpoint{4.295236in}{1.680811in}}%
\pgfpathlineto{\pgfqpoint{4.295812in}{1.682427in}}%
\pgfpathlineto{\pgfqpoint{4.296197in}{1.686272in}}%
\pgfpathlineto{\pgfqpoint{4.296773in}{1.685876in}}%
\pgfpathlineto{\pgfqpoint{4.296965in}{1.682433in}}%
\pgfpathlineto{\pgfqpoint{4.297926in}{1.684696in}}%
\pgfpathlineto{\pgfqpoint{4.298118in}{1.684865in}}%
\pgfpathlineto{\pgfqpoint{4.299464in}{1.674680in}}%
\pgfpathlineto{\pgfqpoint{4.299848in}{1.675302in}}%
\pgfpathlineto{\pgfqpoint{4.300040in}{1.677145in}}%
\pgfpathlineto{\pgfqpoint{4.300617in}{1.674571in}}%
\pgfpathlineto{\pgfqpoint{4.301770in}{1.671865in}}%
\pgfpathlineto{\pgfqpoint{4.302730in}{1.670739in}}%
\pgfpathlineto{\pgfqpoint{4.302923in}{1.673799in}}%
\pgfpathlineto{\pgfqpoint{4.303115in}{1.672437in}}%
\pgfpathlineto{\pgfqpoint{4.303499in}{1.675312in}}%
\pgfpathlineto{\pgfqpoint{4.303691in}{1.678159in}}%
\pgfpathlineto{\pgfqpoint{4.304268in}{1.673773in}}%
\pgfpathlineto{\pgfqpoint{4.305229in}{1.669156in}}%
\pgfpathlineto{\pgfqpoint{4.305997in}{1.669576in}}%
\pgfpathlineto{\pgfqpoint{4.306189in}{1.670750in}}%
\pgfpathlineto{\pgfqpoint{4.306382in}{1.668939in}}%
\pgfpathlineto{\pgfqpoint{4.307342in}{1.663784in}}%
\pgfpathlineto{\pgfqpoint{4.307535in}{1.665159in}}%
\pgfpathlineto{\pgfqpoint{4.308111in}{1.667547in}}%
\pgfpathlineto{\pgfqpoint{4.309841in}{1.678451in}}%
\pgfpathlineto{\pgfqpoint{4.310225in}{1.679978in}}%
\pgfpathlineto{\pgfqpoint{4.311186in}{1.675970in}}%
\pgfpathlineto{\pgfqpoint{4.311378in}{1.679825in}}%
\pgfpathlineto{\pgfqpoint{4.311762in}{1.671103in}}%
\pgfpathlineto{\pgfqpoint{4.312147in}{1.674879in}}%
\pgfpathlineto{\pgfqpoint{4.312915in}{1.673430in}}%
\pgfpathlineto{\pgfqpoint{4.313492in}{1.679330in}}%
\pgfpathlineto{\pgfqpoint{4.314068in}{1.674044in}}%
\pgfpathlineto{\pgfqpoint{4.314645in}{1.677903in}}%
\pgfpathlineto{\pgfqpoint{4.316182in}{1.686240in}}%
\pgfpathlineto{\pgfqpoint{4.316567in}{1.686005in}}%
\pgfpathlineto{\pgfqpoint{4.317335in}{1.679786in}}%
\pgfpathlineto{\pgfqpoint{4.317720in}{1.685004in}}%
\pgfpathlineto{\pgfqpoint{4.318296in}{1.690485in}}%
\pgfpathlineto{\pgfqpoint{4.319065in}{1.689425in}}%
\pgfpathlineto{\pgfqpoint{4.319833in}{1.685787in}}%
\pgfpathlineto{\pgfqpoint{4.320218in}{1.687596in}}%
\pgfpathlineto{\pgfqpoint{4.320410in}{1.688350in}}%
\pgfpathlineto{\pgfqpoint{4.320602in}{1.686447in}}%
\pgfpathlineto{\pgfqpoint{4.320986in}{1.687089in}}%
\pgfpathlineto{\pgfqpoint{4.322908in}{1.682250in}}%
\pgfpathlineto{\pgfqpoint{4.324445in}{1.690512in}}%
\pgfpathlineto{\pgfqpoint{4.325022in}{1.685095in}}%
\pgfpathlineto{\pgfqpoint{4.325791in}{1.688229in}}%
\pgfpathlineto{\pgfqpoint{4.326751in}{1.690473in}}%
\pgfpathlineto{\pgfqpoint{4.326559in}{1.687460in}}%
\pgfpathlineto{\pgfqpoint{4.326944in}{1.688723in}}%
\pgfpathlineto{\pgfqpoint{4.327136in}{1.688532in}}%
\pgfpathlineto{\pgfqpoint{4.330018in}{1.704850in}}%
\pgfpathlineto{\pgfqpoint{4.330403in}{1.703170in}}%
\pgfpathlineto{\pgfqpoint{4.330595in}{1.706058in}}%
\pgfpathlineto{\pgfqpoint{4.330787in}{1.705300in}}%
\pgfpathlineto{\pgfqpoint{4.331171in}{1.708692in}}%
\pgfpathlineto{\pgfqpoint{4.331363in}{1.710934in}}%
\pgfpathlineto{\pgfqpoint{4.331940in}{1.705035in}}%
\pgfpathlineto{\pgfqpoint{4.333477in}{1.697495in}}%
\pgfpathlineto{\pgfqpoint{4.333670in}{1.700951in}}%
\pgfpathlineto{\pgfqpoint{4.334438in}{1.698760in}}%
\pgfpathlineto{\pgfqpoint{4.335015in}{1.696489in}}%
\pgfpathlineto{\pgfqpoint{4.335591in}{1.698038in}}%
\pgfpathlineto{\pgfqpoint{4.336936in}{1.703407in}}%
\pgfpathlineto{\pgfqpoint{4.337321in}{1.706100in}}%
\pgfpathlineto{\pgfqpoint{4.337705in}{1.702160in}}%
\pgfpathlineto{\pgfqpoint{4.337897in}{1.704208in}}%
\pgfpathlineto{\pgfqpoint{4.338858in}{1.699323in}}%
\pgfpathlineto{\pgfqpoint{4.339435in}{1.700441in}}%
\pgfpathlineto{\pgfqpoint{4.339819in}{1.701330in}}%
\pgfpathlineto{\pgfqpoint{4.340203in}{1.699918in}}%
\pgfpathlineto{\pgfqpoint{4.340395in}{1.700138in}}%
\pgfpathlineto{\pgfqpoint{4.340972in}{1.701367in}}%
\pgfpathlineto{\pgfqpoint{4.341933in}{1.705386in}}%
\pgfpathlineto{\pgfqpoint{4.342509in}{1.705193in}}%
\pgfpathlineto{\pgfqpoint{4.343086in}{1.701731in}}%
\pgfpathlineto{\pgfqpoint{4.343662in}{1.703664in}}%
\pgfpathlineto{\pgfqpoint{4.344047in}{1.704790in}}%
\pgfpathlineto{\pgfqpoint{4.344239in}{1.702577in}}%
\pgfpathlineto{\pgfqpoint{4.344431in}{1.703505in}}%
\pgfpathlineto{\pgfqpoint{4.350388in}{1.670294in}}%
\pgfpathlineto{\pgfqpoint{4.350580in}{1.671720in}}%
\pgfpathlineto{\pgfqpoint{4.350772in}{1.672246in}}%
\pgfpathlineto{\pgfqpoint{4.350965in}{1.669595in}}%
\pgfpathlineto{\pgfqpoint{4.351349in}{1.665179in}}%
\pgfpathlineto{\pgfqpoint{4.351925in}{1.668038in}}%
\pgfpathlineto{\pgfqpoint{4.353078in}{1.672915in}}%
\pgfpathlineto{\pgfqpoint{4.353847in}{1.670104in}}%
\pgfpathlineto{\pgfqpoint{4.354039in}{1.673800in}}%
\pgfpathlineto{\pgfqpoint{4.354231in}{1.673604in}}%
\pgfpathlineto{\pgfqpoint{4.354424in}{1.674845in}}%
\pgfpathlineto{\pgfqpoint{4.354616in}{1.674735in}}%
\pgfpathlineto{\pgfqpoint{4.355961in}{1.681193in}}%
\pgfpathlineto{\pgfqpoint{4.356345in}{1.679902in}}%
\pgfpathlineto{\pgfqpoint{4.357498in}{1.675149in}}%
\pgfpathlineto{\pgfqpoint{4.357114in}{1.680392in}}%
\pgfpathlineto{\pgfqpoint{4.357691in}{1.676459in}}%
\pgfpathlineto{\pgfqpoint{4.357883in}{1.676370in}}%
\pgfpathlineto{\pgfqpoint{4.358075in}{1.674578in}}%
\pgfpathlineto{\pgfqpoint{4.358267in}{1.679666in}}%
\pgfpathlineto{\pgfqpoint{4.358651in}{1.679341in}}%
\pgfpathlineto{\pgfqpoint{4.358844in}{1.679098in}}%
\pgfpathlineto{\pgfqpoint{4.359036in}{1.679779in}}%
\pgfpathlineto{\pgfqpoint{4.359420in}{1.684288in}}%
\pgfpathlineto{\pgfqpoint{4.359997in}{1.678937in}}%
\pgfpathlineto{\pgfqpoint{4.360189in}{1.677880in}}%
\pgfpathlineto{\pgfqpoint{4.360381in}{1.679406in}}%
\pgfpathlineto{\pgfqpoint{4.360573in}{1.679302in}}%
\pgfpathlineto{\pgfqpoint{4.361726in}{1.685845in}}%
\pgfpathlineto{\pgfqpoint{4.361150in}{1.678986in}}%
\pgfpathlineto{\pgfqpoint{4.362110in}{1.684781in}}%
\pgfpathlineto{\pgfqpoint{4.362879in}{1.680874in}}%
\pgfpathlineto{\pgfqpoint{4.363263in}{1.684099in}}%
\pgfpathlineto{\pgfqpoint{4.363840in}{1.682272in}}%
\pgfpathlineto{\pgfqpoint{4.365185in}{1.693994in}}%
\pgfpathlineto{\pgfqpoint{4.365377in}{1.692341in}}%
\pgfpathlineto{\pgfqpoint{4.365569in}{1.690363in}}%
\pgfpathlineto{\pgfqpoint{4.365762in}{1.693165in}}%
\pgfpathlineto{\pgfqpoint{4.365954in}{1.692355in}}%
\pgfpathlineto{\pgfqpoint{4.366722in}{1.701545in}}%
\pgfpathlineto{\pgfqpoint{4.367299in}{1.699930in}}%
\pgfpathlineto{\pgfqpoint{4.368068in}{1.702171in}}%
\pgfpathlineto{\pgfqpoint{4.368260in}{1.701881in}}%
\pgfpathlineto{\pgfqpoint{4.368836in}{1.701829in}}%
\pgfpathlineto{\pgfqpoint{4.369797in}{1.710424in}}%
\pgfpathlineto{\pgfqpoint{4.370181in}{1.708880in}}%
\pgfpathlineto{\pgfqpoint{4.370566in}{1.704799in}}%
\pgfpathlineto{\pgfqpoint{4.371334in}{1.707878in}}%
\pgfpathlineto{\pgfqpoint{4.371719in}{1.707912in}}%
\pgfpathlineto{\pgfqpoint{4.372295in}{1.703981in}}%
\pgfpathlineto{\pgfqpoint{4.372872in}{1.704377in}}%
\pgfpathlineto{\pgfqpoint{4.373833in}{1.708205in}}%
\pgfpathlineto{\pgfqpoint{4.373448in}{1.703932in}}%
\pgfpathlineto{\pgfqpoint{4.374025in}{1.707379in}}%
\pgfpathlineto{\pgfqpoint{4.375946in}{1.698419in}}%
\pgfpathlineto{\pgfqpoint{4.376907in}{1.703439in}}%
\pgfpathlineto{\pgfqpoint{4.377484in}{1.700900in}}%
\pgfpathlineto{\pgfqpoint{4.377868in}{1.697995in}}%
\pgfpathlineto{\pgfqpoint{4.378445in}{1.700389in}}%
\pgfpathlineto{\pgfqpoint{4.378637in}{1.700666in}}%
\pgfpathlineto{\pgfqpoint{4.378829in}{1.699466in}}%
\pgfpathlineto{\pgfqpoint{4.379021in}{1.696741in}}%
\pgfpathlineto{\pgfqpoint{4.379405in}{1.701400in}}%
\pgfpathlineto{\pgfqpoint{4.379790in}{1.698887in}}%
\pgfpathlineto{\pgfqpoint{4.382480in}{1.715587in}}%
\pgfpathlineto{\pgfqpoint{4.383249in}{1.715086in}}%
\pgfpathlineto{\pgfqpoint{4.383441in}{1.714636in}}%
\pgfpathlineto{\pgfqpoint{4.383633in}{1.716268in}}%
\pgfpathlineto{\pgfqpoint{4.383825in}{1.716329in}}%
\pgfpathlineto{\pgfqpoint{4.384210in}{1.720277in}}%
\pgfpathlineto{\pgfqpoint{4.385363in}{1.719421in}}%
\pgfpathlineto{\pgfqpoint{4.385939in}{1.714528in}}%
\pgfpathlineto{\pgfqpoint{4.386516in}{1.718808in}}%
\pgfpathlineto{\pgfqpoint{4.387092in}{1.723725in}}%
\pgfpathlineto{\pgfqpoint{4.387669in}{1.720968in}}%
\pgfpathlineto{\pgfqpoint{4.389206in}{1.709445in}}%
\pgfpathlineto{\pgfqpoint{4.389398in}{1.710975in}}%
\pgfpathlineto{\pgfqpoint{4.390551in}{1.716885in}}%
\pgfpathlineto{\pgfqpoint{4.390743in}{1.715425in}}%
\pgfpathlineto{\pgfqpoint{4.390936in}{1.716955in}}%
\pgfpathlineto{\pgfqpoint{4.391512in}{1.715618in}}%
\pgfpathlineto{\pgfqpoint{4.391704in}{1.712546in}}%
\pgfpathlineto{\pgfqpoint{4.392089in}{1.716775in}}%
\pgfpathlineto{\pgfqpoint{4.392473in}{1.712975in}}%
\pgfpathlineto{\pgfqpoint{4.393818in}{1.721084in}}%
\pgfpathlineto{\pgfqpoint{4.394010in}{1.720910in}}%
\pgfpathlineto{\pgfqpoint{4.395163in}{1.708735in}}%
\pgfpathlineto{\pgfqpoint{4.395548in}{1.713825in}}%
\pgfpathlineto{\pgfqpoint{4.396124in}{1.712619in}}%
\pgfpathlineto{\pgfqpoint{4.396508in}{1.714651in}}%
\pgfpathlineto{\pgfqpoint{4.398238in}{1.725463in}}%
\pgfpathlineto{\pgfqpoint{4.398814in}{1.724006in}}%
\pgfpathlineto{\pgfqpoint{4.399199in}{1.724964in}}%
\pgfpathlineto{\pgfqpoint{4.399391in}{1.727738in}}%
\pgfpathlineto{\pgfqpoint{4.399967in}{1.720406in}}%
\pgfpathlineto{\pgfqpoint{4.400736in}{1.723914in}}%
\pgfpathlineto{\pgfqpoint{4.401120in}{1.722109in}}%
\pgfpathlineto{\pgfqpoint{4.401697in}{1.718350in}}%
\pgfpathlineto{\pgfqpoint{4.402273in}{1.720644in}}%
\pgfpathlineto{\pgfqpoint{4.403426in}{1.724956in}}%
\pgfpathlineto{\pgfqpoint{4.403619in}{1.724007in}}%
\pgfpathlineto{\pgfqpoint{4.404195in}{1.719112in}}%
\pgfpathlineto{\pgfqpoint{4.404964in}{1.721131in}}%
\pgfpathlineto{\pgfqpoint{4.405925in}{1.726964in}}%
\pgfpathlineto{\pgfqpoint{4.406501in}{1.724764in}}%
\pgfpathlineto{\pgfqpoint{4.407654in}{1.720496in}}%
\pgfpathlineto{\pgfqpoint{4.408039in}{1.725998in}}%
\pgfpathlineto{\pgfqpoint{4.408807in}{1.721318in}}%
\pgfpathlineto{\pgfqpoint{4.408999in}{1.719119in}}%
\pgfpathlineto{\pgfqpoint{4.409576in}{1.725157in}}%
\pgfpathlineto{\pgfqpoint{4.409768in}{1.721388in}}%
\pgfpathlineto{\pgfqpoint{4.411498in}{1.728555in}}%
\pgfpathlineto{\pgfqpoint{4.412651in}{1.727916in}}%
\pgfpathlineto{\pgfqpoint{4.413035in}{1.725677in}}%
\pgfpathlineto{\pgfqpoint{4.413804in}{1.726987in}}%
\pgfpathlineto{\pgfqpoint{4.416110in}{1.738666in}}%
\pgfpathlineto{\pgfqpoint{4.416302in}{1.738355in}}%
\pgfpathlineto{\pgfqpoint{4.416686in}{1.731382in}}%
\pgfpathlineto{\pgfqpoint{4.417647in}{1.733412in}}%
\pgfpathlineto{\pgfqpoint{4.419184in}{1.720800in}}%
\pgfpathlineto{\pgfqpoint{4.418031in}{1.734850in}}%
\pgfpathlineto{\pgfqpoint{4.419953in}{1.721343in}}%
\pgfpathlineto{\pgfqpoint{4.420337in}{1.721113in}}%
\pgfpathlineto{\pgfqpoint{4.420529in}{1.722506in}}%
\pgfpathlineto{\pgfqpoint{4.420722in}{1.725392in}}%
\pgfpathlineto{\pgfqpoint{4.421298in}{1.718656in}}%
\pgfpathlineto{\pgfqpoint{4.421490in}{1.721622in}}%
\pgfpathlineto{\pgfqpoint{4.422067in}{1.723894in}}%
\pgfpathlineto{\pgfqpoint{4.423220in}{1.717635in}}%
\pgfpathlineto{\pgfqpoint{4.423412in}{1.717110in}}%
\pgfpathlineto{\pgfqpoint{4.423796in}{1.721762in}}%
\pgfpathlineto{\pgfqpoint{4.424565in}{1.720521in}}%
\pgfpathlineto{\pgfqpoint{4.424757in}{1.720056in}}%
\pgfpathlineto{\pgfqpoint{4.425141in}{1.721652in}}%
\pgfpathlineto{\pgfqpoint{4.425526in}{1.721289in}}%
\pgfpathlineto{\pgfqpoint{4.426679in}{1.726103in}}%
\pgfpathlineto{\pgfqpoint{4.427063in}{1.723124in}}%
\pgfpathlineto{\pgfqpoint{4.428024in}{1.718572in}}%
\pgfpathlineto{\pgfqpoint{4.428216in}{1.718802in}}%
\pgfpathlineto{\pgfqpoint{4.429369in}{1.729416in}}%
\pgfpathlineto{\pgfqpoint{4.430138in}{1.725895in}}%
\pgfpathlineto{\pgfqpoint{4.430330in}{1.724215in}}%
\pgfpathlineto{\pgfqpoint{4.430906in}{1.727438in}}%
\pgfpathlineto{\pgfqpoint{4.432060in}{1.736126in}}%
\pgfpathlineto{\pgfqpoint{4.432444in}{1.732961in}}%
\pgfpathlineto{\pgfqpoint{4.432828in}{1.729418in}}%
\pgfpathlineto{\pgfqpoint{4.433213in}{1.732329in}}%
\pgfpathlineto{\pgfqpoint{4.433405in}{1.733728in}}%
\pgfpathlineto{\pgfqpoint{4.433789in}{1.731014in}}%
\pgfpathlineto{\pgfqpoint{4.433981in}{1.732691in}}%
\pgfpathlineto{\pgfqpoint{4.434173in}{1.730382in}}%
\pgfpathlineto{\pgfqpoint{4.434750in}{1.734817in}}%
\pgfpathlineto{\pgfqpoint{4.434942in}{1.737243in}}%
\pgfpathlineto{\pgfqpoint{4.435326in}{1.733892in}}%
\pgfpathlineto{\pgfqpoint{4.435711in}{1.736355in}}%
\pgfpathlineto{\pgfqpoint{4.436864in}{1.726506in}}%
\pgfpathlineto{\pgfqpoint{4.437056in}{1.728598in}}%
\pgfpathlineto{\pgfqpoint{4.437632in}{1.733775in}}%
\pgfpathlineto{\pgfqpoint{4.438209in}{1.730969in}}%
\pgfpathlineto{\pgfqpoint{4.439938in}{1.726218in}}%
\pgfpathlineto{\pgfqpoint{4.440131in}{1.728272in}}%
\pgfpathlineto{\pgfqpoint{4.440707in}{1.724883in}}%
\pgfpathlineto{\pgfqpoint{4.441284in}{1.722414in}}%
\pgfpathlineto{\pgfqpoint{4.441476in}{1.725345in}}%
\pgfpathlineto{\pgfqpoint{4.442629in}{1.729420in}}%
\pgfpathlineto{\pgfqpoint{4.442821in}{1.728702in}}%
\pgfpathlineto{\pgfqpoint{4.443013in}{1.725472in}}%
\pgfpathlineto{\pgfqpoint{4.443590in}{1.728991in}}%
\pgfpathlineto{\pgfqpoint{4.443782in}{1.728906in}}%
\pgfpathlineto{\pgfqpoint{4.444358in}{1.726723in}}%
\pgfpathlineto{\pgfqpoint{4.444743in}{1.731316in}}%
\pgfpathlineto{\pgfqpoint{4.444935in}{1.729167in}}%
\pgfpathlineto{\pgfqpoint{4.445511in}{1.733450in}}%
\pgfpathlineto{\pgfqpoint{4.445703in}{1.733864in}}%
\pgfpathlineto{\pgfqpoint{4.446664in}{1.739166in}}%
\pgfpathlineto{\pgfqpoint{4.446856in}{1.739015in}}%
\pgfpathlineto{\pgfqpoint{4.447817in}{1.735765in}}%
\pgfpathlineto{\pgfqpoint{4.447241in}{1.739269in}}%
\pgfpathlineto{\pgfqpoint{4.448202in}{1.737183in}}%
\pgfpathlineto{\pgfqpoint{4.448394in}{1.737051in}}%
\pgfpathlineto{\pgfqpoint{4.448778in}{1.741744in}}%
\pgfpathlineto{\pgfqpoint{4.449162in}{1.735468in}}%
\pgfpathlineto{\pgfqpoint{4.449931in}{1.726170in}}%
\pgfpathlineto{\pgfqpoint{4.450315in}{1.732907in}}%
\pgfpathlineto{\pgfqpoint{4.450700in}{1.731745in}}%
\pgfpathlineto{\pgfqpoint{4.451276in}{1.728190in}}%
\pgfpathlineto{\pgfqpoint{4.451661in}{1.730249in}}%
\pgfpathlineto{\pgfqpoint{4.452237in}{1.735215in}}%
\pgfpathlineto{\pgfqpoint{4.452814in}{1.731988in}}%
\pgfpathlineto{\pgfqpoint{4.454159in}{1.720070in}}%
\pgfpathlineto{\pgfqpoint{4.454735in}{1.722918in}}%
\pgfpathlineto{\pgfqpoint{4.455696in}{1.727048in}}%
\pgfpathlineto{\pgfqpoint{4.455312in}{1.721876in}}%
\pgfpathlineto{\pgfqpoint{4.456081in}{1.725490in}}%
\pgfpathlineto{\pgfqpoint{4.456657in}{1.727648in}}%
\pgfpathlineto{\pgfqpoint{4.456849in}{1.725337in}}%
\pgfpathlineto{\pgfqpoint{4.457041in}{1.723343in}}%
\pgfpathlineto{\pgfqpoint{4.457618in}{1.726891in}}%
\pgfpathlineto{\pgfqpoint{4.457810in}{1.727038in}}%
\pgfpathlineto{\pgfqpoint{4.459732in}{1.720126in}}%
\pgfpathlineto{\pgfqpoint{4.460885in}{1.722561in}}%
\pgfpathlineto{\pgfqpoint{4.461269in}{1.720569in}}%
\pgfpathlineto{\pgfqpoint{4.461461in}{1.719445in}}%
\pgfpathlineto{\pgfqpoint{4.462038in}{1.722511in}}%
\pgfpathlineto{\pgfqpoint{4.463575in}{1.712677in}}%
\pgfpathlineto{\pgfqpoint{4.464344in}{1.716313in}}%
\pgfpathlineto{\pgfqpoint{4.464728in}{1.715114in}}%
\pgfpathlineto{\pgfqpoint{4.465305in}{1.717430in}}%
\pgfpathlineto{\pgfqpoint{4.465881in}{1.718260in}}%
\pgfpathlineto{\pgfqpoint{4.466650in}{1.708891in}}%
\pgfpathlineto{\pgfqpoint{4.466842in}{1.709880in}}%
\pgfpathlineto{\pgfqpoint{4.467034in}{1.707036in}}%
\pgfpathlineto{\pgfqpoint{4.467226in}{1.705887in}}%
\pgfpathlineto{\pgfqpoint{4.467418in}{1.707876in}}%
\pgfpathlineto{\pgfqpoint{4.468571in}{1.718269in}}%
\pgfpathlineto{\pgfqpoint{4.468764in}{1.717669in}}%
\pgfpathlineto{\pgfqpoint{4.470301in}{1.712104in}}%
\pgfpathlineto{\pgfqpoint{4.470493in}{1.714674in}}%
\pgfpathlineto{\pgfqpoint{4.470877in}{1.709391in}}%
\pgfpathlineto{\pgfqpoint{4.472030in}{1.699721in}}%
\pgfpathlineto{\pgfqpoint{4.472415in}{1.702338in}}%
\pgfpathlineto{\pgfqpoint{4.474913in}{1.718833in}}%
\pgfpathlineto{\pgfqpoint{4.475489in}{1.719330in}}%
\pgfpathlineto{\pgfqpoint{4.475682in}{1.717277in}}%
\pgfpathlineto{\pgfqpoint{4.476258in}{1.722111in}}%
\pgfpathlineto{\pgfqpoint{4.476835in}{1.726027in}}%
\pgfpathlineto{\pgfqpoint{4.477219in}{1.727979in}}%
\pgfpathlineto{\pgfqpoint{4.477603in}{1.725356in}}%
\pgfpathlineto{\pgfqpoint{4.477988in}{1.726754in}}%
\pgfpathlineto{\pgfqpoint{4.478180in}{1.726786in}}%
\pgfpathlineto{\pgfqpoint{4.479141in}{1.720588in}}%
\pgfpathlineto{\pgfqpoint{4.479333in}{1.721643in}}%
\pgfpathlineto{\pgfqpoint{4.481255in}{1.734127in}}%
\pgfpathlineto{\pgfqpoint{4.481447in}{1.735033in}}%
\pgfpathlineto{\pgfqpoint{4.481639in}{1.733252in}}%
\pgfpathlineto{\pgfqpoint{4.481831in}{1.733217in}}%
\pgfpathlineto{\pgfqpoint{4.482792in}{1.727674in}}%
\pgfpathlineto{\pgfqpoint{4.483176in}{1.730934in}}%
\pgfpathlineto{\pgfqpoint{4.483945in}{1.739895in}}%
\pgfpathlineto{\pgfqpoint{4.484521in}{1.736721in}}%
\pgfpathlineto{\pgfqpoint{4.484714in}{1.736213in}}%
\pgfpathlineto{\pgfqpoint{4.484906in}{1.737829in}}%
\pgfpathlineto{\pgfqpoint{4.486827in}{1.747863in}}%
\pgfpathlineto{\pgfqpoint{4.487020in}{1.746111in}}%
\pgfpathlineto{\pgfqpoint{4.489133in}{1.755698in}}%
\pgfpathlineto{\pgfqpoint{4.487404in}{1.745737in}}%
\pgfpathlineto{\pgfqpoint{4.489902in}{1.754272in}}%
\pgfpathlineto{\pgfqpoint{4.490094in}{1.752351in}}%
\pgfpathlineto{\pgfqpoint{4.490671in}{1.755937in}}%
\pgfpathlineto{\pgfqpoint{4.490863in}{1.754482in}}%
\pgfpathlineto{\pgfqpoint{4.491055in}{1.758772in}}%
\pgfpathlineto{\pgfqpoint{4.492016in}{1.756052in}}%
\pgfpathlineto{\pgfqpoint{4.492208in}{1.758925in}}%
\pgfpathlineto{\pgfqpoint{4.492785in}{1.754232in}}%
\pgfpathlineto{\pgfqpoint{4.492977in}{1.753571in}}%
\pgfpathlineto{\pgfqpoint{4.493169in}{1.754782in}}%
\pgfpathlineto{\pgfqpoint{4.493361in}{1.754575in}}%
\pgfpathlineto{\pgfqpoint{4.494706in}{1.760862in}}%
\pgfpathlineto{\pgfqpoint{4.495283in}{1.761907in}}%
\pgfpathlineto{\pgfqpoint{4.496820in}{1.770015in}}%
\pgfpathlineto{\pgfqpoint{4.497012in}{1.768841in}}%
\pgfpathlineto{\pgfqpoint{4.498357in}{1.761999in}}%
\pgfpathlineto{\pgfqpoint{4.499126in}{1.765193in}}%
\pgfpathlineto{\pgfqpoint{4.499510in}{1.763035in}}%
\pgfpathlineto{\pgfqpoint{4.499703in}{1.763155in}}%
\pgfpathlineto{\pgfqpoint{4.500856in}{1.772885in}}%
\pgfpathlineto{\pgfqpoint{4.501048in}{1.769367in}}%
\pgfpathlineto{\pgfqpoint{4.502201in}{1.762801in}}%
\pgfpathlineto{\pgfqpoint{4.502585in}{1.768670in}}%
\pgfpathlineto{\pgfqpoint{4.502777in}{1.762004in}}%
\pgfpathlineto{\pgfqpoint{4.503354in}{1.765847in}}%
\pgfpathlineto{\pgfqpoint{4.503738in}{1.768569in}}%
\pgfpathlineto{\pgfqpoint{4.503930in}{1.765649in}}%
\pgfpathlineto{\pgfqpoint{4.504315in}{1.762449in}}%
\pgfpathlineto{\pgfqpoint{4.504891in}{1.765527in}}%
\pgfpathlineto{\pgfqpoint{4.505083in}{1.767026in}}%
\pgfpathlineto{\pgfqpoint{4.505660in}{1.763628in}}%
\pgfpathlineto{\pgfqpoint{4.505852in}{1.765009in}}%
\pgfpathlineto{\pgfqpoint{4.506044in}{1.764335in}}%
\pgfpathlineto{\pgfqpoint{4.506429in}{1.764998in}}%
\pgfpathlineto{\pgfqpoint{4.506813in}{1.771151in}}%
\pgfpathlineto{\pgfqpoint{4.507582in}{1.767290in}}%
\pgfpathlineto{\pgfqpoint{4.508158in}{1.767550in}}%
\pgfpathlineto{\pgfqpoint{4.508927in}{1.763769in}}%
\pgfpathlineto{\pgfqpoint{4.509119in}{1.763647in}}%
\pgfpathlineto{\pgfqpoint{4.510656in}{1.755070in}}%
\pgfpathlineto{\pgfqpoint{4.512001in}{1.763837in}}%
\pgfpathlineto{\pgfqpoint{4.512194in}{1.760597in}}%
\pgfpathlineto{\pgfqpoint{4.512770in}{1.761260in}}%
\pgfpathlineto{\pgfqpoint{4.513347in}{1.753785in}}%
\pgfpathlineto{\pgfqpoint{4.513923in}{1.748527in}}%
\pgfpathlineto{\pgfqpoint{4.514115in}{1.754624in}}%
\pgfpathlineto{\pgfqpoint{4.514307in}{1.754081in}}%
\pgfpathlineto{\pgfqpoint{4.514500in}{1.758360in}}%
\pgfpathlineto{\pgfqpoint{4.515268in}{1.751560in}}%
\pgfpathlineto{\pgfqpoint{4.516613in}{1.742346in}}%
\pgfpathlineto{\pgfqpoint{4.516806in}{1.745854in}}%
\pgfpathlineto{\pgfqpoint{4.518151in}{1.753736in}}%
\pgfpathlineto{\pgfqpoint{4.519496in}{1.747594in}}%
\pgfpathlineto{\pgfqpoint{4.520457in}{1.746527in}}%
\pgfpathlineto{\pgfqpoint{4.521418in}{1.755839in}}%
\pgfpathlineto{\pgfqpoint{4.521610in}{1.754505in}}%
\pgfpathlineto{\pgfqpoint{4.521802in}{1.759431in}}%
\pgfpathlineto{\pgfqpoint{4.522186in}{1.757033in}}%
\pgfpathlineto{\pgfqpoint{4.522955in}{1.760205in}}%
\pgfpathlineto{\pgfqpoint{4.523916in}{1.759262in}}%
\pgfpathlineto{\pgfqpoint{4.526030in}{1.740301in}}%
\pgfpathlineto{\pgfqpoint{4.526606in}{1.744987in}}%
\pgfpathlineto{\pgfqpoint{4.526990in}{1.742028in}}%
\pgfpathlineto{\pgfqpoint{4.528528in}{1.734951in}}%
\pgfpathlineto{\pgfqpoint{4.529297in}{1.740513in}}%
\pgfpathlineto{\pgfqpoint{4.529681in}{1.737653in}}%
\pgfpathlineto{\pgfqpoint{4.530257in}{1.735473in}}%
\pgfpathlineto{\pgfqpoint{4.530834in}{1.737350in}}%
\pgfpathlineto{\pgfqpoint{4.531987in}{1.733226in}}%
\pgfpathlineto{\pgfqpoint{4.531603in}{1.738191in}}%
\pgfpathlineto{\pgfqpoint{4.532371in}{1.735934in}}%
\pgfpathlineto{\pgfqpoint{4.532563in}{1.738151in}}%
\pgfpathlineto{\pgfqpoint{4.533140in}{1.735208in}}%
\pgfpathlineto{\pgfqpoint{4.534677in}{1.721984in}}%
\pgfpathlineto{\pgfqpoint{4.535062in}{1.725165in}}%
\pgfpathlineto{\pgfqpoint{4.535254in}{1.725145in}}%
\pgfpathlineto{\pgfqpoint{4.536215in}{1.717541in}}%
\pgfpathlineto{\pgfqpoint{4.536791in}{1.718250in}}%
\pgfpathlineto{\pgfqpoint{4.537175in}{1.717889in}}%
\pgfpathlineto{\pgfqpoint{4.537752in}{1.718518in}}%
\pgfpathlineto{\pgfqpoint{4.537944in}{1.714162in}}%
\pgfpathlineto{\pgfqpoint{4.538713in}{1.719969in}}%
\pgfpathlineto{\pgfqpoint{4.539097in}{1.719534in}}%
\pgfpathlineto{\pgfqpoint{4.539289in}{1.721285in}}%
\pgfpathlineto{\pgfqpoint{4.539674in}{1.717583in}}%
\pgfpathlineto{\pgfqpoint{4.540058in}{1.720555in}}%
\pgfpathlineto{\pgfqpoint{4.541019in}{1.715343in}}%
\pgfpathlineto{\pgfqpoint{4.541403in}{1.717109in}}%
\pgfpathlineto{\pgfqpoint{4.541595in}{1.718259in}}%
\pgfpathlineto{\pgfqpoint{4.541980in}{1.715867in}}%
\pgfpathlineto{\pgfqpoint{4.542364in}{1.716016in}}%
\pgfpathlineto{\pgfqpoint{4.542940in}{1.713865in}}%
\pgfpathlineto{\pgfqpoint{4.544286in}{1.726406in}}%
\pgfpathlineto{\pgfqpoint{4.545439in}{1.724710in}}%
\pgfpathlineto{\pgfqpoint{4.546976in}{1.730379in}}%
\pgfpathlineto{\pgfqpoint{4.547168in}{1.729358in}}%
\pgfpathlineto{\pgfqpoint{4.547360in}{1.732286in}}%
\pgfpathlineto{\pgfqpoint{4.547552in}{1.731966in}}%
\pgfpathlineto{\pgfqpoint{4.548705in}{1.738723in}}%
\pgfpathlineto{\pgfqpoint{4.549090in}{1.736527in}}%
\pgfpathlineto{\pgfqpoint{4.549282in}{1.734657in}}%
\pgfpathlineto{\pgfqpoint{4.549666in}{1.740306in}}%
\pgfpathlineto{\pgfqpoint{4.550435in}{1.744299in}}%
\pgfpathlineto{\pgfqpoint{4.551011in}{1.743878in}}%
\pgfpathlineto{\pgfqpoint{4.552741in}{1.738186in}}%
\pgfpathlineto{\pgfqpoint{4.551588in}{1.745734in}}%
\pgfpathlineto{\pgfqpoint{4.552933in}{1.739704in}}%
\pgfpathlineto{\pgfqpoint{4.553510in}{1.742724in}}%
\pgfpathlineto{\pgfqpoint{4.554278in}{1.741347in}}%
\pgfpathlineto{\pgfqpoint{4.555239in}{1.740521in}}%
\pgfpathlineto{\pgfqpoint{4.554855in}{1.741913in}}%
\pgfpathlineto{\pgfqpoint{4.555431in}{1.740875in}}%
\pgfpathlineto{\pgfqpoint{4.555624in}{1.741381in}}%
\pgfpathlineto{\pgfqpoint{4.555816in}{1.739279in}}%
\pgfpathlineto{\pgfqpoint{4.556200in}{1.740440in}}%
\pgfpathlineto{\pgfqpoint{4.556584in}{1.741261in}}%
\pgfpathlineto{\pgfqpoint{4.559467in}{1.727717in}}%
\pgfpathlineto{\pgfqpoint{4.559659in}{1.723838in}}%
\pgfpathlineto{\pgfqpoint{4.560620in}{1.723920in}}%
\pgfpathlineto{\pgfqpoint{4.560812in}{1.723442in}}%
\pgfpathlineto{\pgfqpoint{4.562157in}{1.727966in}}%
\pgfpathlineto{\pgfqpoint{4.562349in}{1.728630in}}%
\pgfpathlineto{\pgfqpoint{4.562542in}{1.726309in}}%
\pgfpathlineto{\pgfqpoint{4.562734in}{1.726638in}}%
\pgfpathlineto{\pgfqpoint{4.563502in}{1.719616in}}%
\pgfpathlineto{\pgfqpoint{4.563887in}{1.724084in}}%
\pgfpathlineto{\pgfqpoint{4.564848in}{1.727646in}}%
\pgfpathlineto{\pgfqpoint{4.564271in}{1.723572in}}%
\pgfpathlineto{\pgfqpoint{4.565040in}{1.726057in}}%
\pgfpathlineto{\pgfqpoint{4.565424in}{1.721785in}}%
\pgfpathlineto{\pgfqpoint{4.566193in}{1.724921in}}%
\pgfpathlineto{\pgfqpoint{4.568114in}{1.732782in}}%
\pgfpathlineto{\pgfqpoint{4.568307in}{1.732900in}}%
\pgfpathlineto{\pgfqpoint{4.570228in}{1.748117in}}%
\pgfpathlineto{\pgfqpoint{4.570613in}{1.744685in}}%
\pgfpathlineto{\pgfqpoint{4.570805in}{1.742610in}}%
\pgfpathlineto{\pgfqpoint{4.571381in}{1.745795in}}%
\pgfpathlineto{\pgfqpoint{4.571573in}{1.745714in}}%
\pgfpathlineto{\pgfqpoint{4.571958in}{1.745845in}}%
\pgfpathlineto{\pgfqpoint{4.572534in}{1.740462in}}%
\pgfpathlineto{\pgfqpoint{4.573111in}{1.742067in}}%
\pgfpathlineto{\pgfqpoint{4.573687in}{1.738690in}}%
\pgfpathlineto{\pgfqpoint{4.575032in}{1.751382in}}%
\pgfpathlineto{\pgfqpoint{4.575417in}{1.750892in}}%
\pgfpathlineto{\pgfqpoint{4.576185in}{1.756389in}}%
\pgfpathlineto{\pgfqpoint{4.576954in}{1.755957in}}%
\pgfpathlineto{\pgfqpoint{4.578492in}{1.748214in}}%
\pgfpathlineto{\pgfqpoint{4.578684in}{1.749428in}}%
\pgfpathlineto{\pgfqpoint{4.579068in}{1.748508in}}%
\pgfpathlineto{\pgfqpoint{4.580413in}{1.742500in}}%
\pgfpathlineto{\pgfqpoint{4.581758in}{1.750427in}}%
\pgfpathlineto{\pgfqpoint{4.582143in}{1.753108in}}%
\pgfpathlineto{\pgfqpoint{4.582719in}{1.750423in}}%
\pgfpathlineto{\pgfqpoint{4.583872in}{1.746946in}}%
\pgfpathlineto{\pgfqpoint{4.584257in}{1.749356in}}%
\pgfpathlineto{\pgfqpoint{4.584449in}{1.749502in}}%
\pgfpathlineto{\pgfqpoint{4.585410in}{1.742026in}}%
\pgfpathlineto{\pgfqpoint{4.585602in}{1.745045in}}%
\pgfpathlineto{\pgfqpoint{4.587139in}{1.757460in}}%
\pgfpathlineto{\pgfqpoint{4.587331in}{1.760252in}}%
\pgfpathlineto{\pgfqpoint{4.587908in}{1.752642in}}%
\pgfpathlineto{\pgfqpoint{4.588676in}{1.754305in}}%
\pgfpathlineto{\pgfqpoint{4.588292in}{1.751251in}}%
\pgfpathlineto{\pgfqpoint{4.589061in}{1.753952in}}%
\pgfpathlineto{\pgfqpoint{4.590214in}{1.744874in}}%
\pgfpathlineto{\pgfqpoint{4.590790in}{1.745595in}}%
\pgfpathlineto{\pgfqpoint{4.590982in}{1.746719in}}%
\pgfpathlineto{\pgfqpoint{4.591367in}{1.742693in}}%
\pgfpathlineto{\pgfqpoint{4.592135in}{1.742421in}}%
\pgfpathlineto{\pgfqpoint{4.593673in}{1.751125in}}%
\pgfpathlineto{\pgfqpoint{4.593865in}{1.750319in}}%
\pgfpathlineto{\pgfqpoint{4.594057in}{1.748639in}}%
\pgfpathlineto{\pgfqpoint{4.594634in}{1.750870in}}%
\pgfpathlineto{\pgfqpoint{4.595979in}{1.754146in}}%
\pgfpathlineto{\pgfqpoint{4.596747in}{1.753497in}}%
\pgfpathlineto{\pgfqpoint{4.597324in}{1.755441in}}%
\pgfpathlineto{\pgfqpoint{4.599438in}{1.744829in}}%
\pgfpathlineto{\pgfqpoint{4.599630in}{1.747335in}}%
\pgfpathlineto{\pgfqpoint{4.600206in}{1.740571in}}%
\pgfpathlineto{\pgfqpoint{4.601744in}{1.734916in}}%
\pgfpathlineto{\pgfqpoint{4.601936in}{1.736788in}}%
\pgfpathlineto{\pgfqpoint{4.602705in}{1.740103in}}%
\pgfpathlineto{\pgfqpoint{4.603089in}{1.737830in}}%
\pgfpathlineto{\pgfqpoint{4.603281in}{1.737126in}}%
\pgfpathlineto{\pgfqpoint{4.603473in}{1.740426in}}%
\pgfpathlineto{\pgfqpoint{4.603666in}{1.740642in}}%
\pgfpathlineto{\pgfqpoint{4.605011in}{1.749191in}}%
\pgfpathlineto{\pgfqpoint{4.605203in}{1.748357in}}%
\pgfpathlineto{\pgfqpoint{4.605395in}{1.749367in}}%
\pgfpathlineto{\pgfqpoint{4.605779in}{1.746119in}}%
\pgfpathlineto{\pgfqpoint{4.605972in}{1.747640in}}%
\pgfpathlineto{\pgfqpoint{4.607125in}{1.743474in}}%
\pgfpathlineto{\pgfqpoint{4.607317in}{1.744244in}}%
\pgfpathlineto{\pgfqpoint{4.609238in}{1.751378in}}%
\pgfpathlineto{\pgfqpoint{4.610199in}{1.742749in}}%
\pgfpathlineto{\pgfqpoint{4.610391in}{1.747171in}}%
\pgfpathlineto{\pgfqpoint{4.610584in}{1.747622in}}%
\pgfpathlineto{\pgfqpoint{4.610776in}{1.744552in}}%
\pgfpathlineto{\pgfqpoint{4.611160in}{1.747948in}}%
\pgfpathlineto{\pgfqpoint{4.611544in}{1.746164in}}%
\pgfpathlineto{\pgfqpoint{4.613082in}{1.752360in}}%
\pgfpathlineto{\pgfqpoint{4.611929in}{1.745643in}}%
\pgfpathlineto{\pgfqpoint{4.613274in}{1.751118in}}%
\pgfpathlineto{\pgfqpoint{4.613658in}{1.750003in}}%
\pgfpathlineto{\pgfqpoint{4.613850in}{1.752647in}}%
\pgfpathlineto{\pgfqpoint{4.614043in}{1.750391in}}%
\pgfpathlineto{\pgfqpoint{4.614427in}{1.758172in}}%
\pgfpathlineto{\pgfqpoint{4.615196in}{1.757234in}}%
\pgfpathlineto{\pgfqpoint{4.616541in}{1.751649in}}%
\pgfpathlineto{\pgfqpoint{4.616925in}{1.752567in}}%
\pgfpathlineto{\pgfqpoint{4.617309in}{1.753679in}}%
\pgfpathlineto{\pgfqpoint{4.617502in}{1.753090in}}%
\pgfpathlineto{\pgfqpoint{4.617694in}{1.750920in}}%
\pgfpathlineto{\pgfqpoint{4.618078in}{1.758125in}}%
\pgfpathlineto{\pgfqpoint{4.619615in}{1.770010in}}%
\pgfpathlineto{\pgfqpoint{4.620000in}{1.765500in}}%
\pgfpathlineto{\pgfqpoint{4.621153in}{1.761211in}}%
\pgfpathlineto{\pgfqpoint{4.621345in}{1.763173in}}%
\pgfpathlineto{\pgfqpoint{4.622306in}{1.767342in}}%
\pgfpathlineto{\pgfqpoint{4.622498in}{1.765118in}}%
\pgfpathlineto{\pgfqpoint{4.624420in}{1.760373in}}%
\pgfpathlineto{\pgfqpoint{4.622882in}{1.765489in}}%
\pgfpathlineto{\pgfqpoint{4.624996in}{1.761834in}}%
\pgfpathlineto{\pgfqpoint{4.625573in}{1.767573in}}%
\pgfpathlineto{\pgfqpoint{4.626149in}{1.764489in}}%
\pgfpathlineto{\pgfqpoint{4.626534in}{1.761715in}}%
\pgfpathlineto{\pgfqpoint{4.626726in}{1.767065in}}%
\pgfpathlineto{\pgfqpoint{4.626918in}{1.765282in}}%
\pgfpathlineto{\pgfqpoint{4.628071in}{1.773728in}}%
\pgfpathlineto{\pgfqpoint{4.628455in}{1.770443in}}%
\pgfpathlineto{\pgfqpoint{4.628647in}{1.772018in}}%
\pgfpathlineto{\pgfqpoint{4.629032in}{1.768818in}}%
\pgfpathlineto{\pgfqpoint{4.629224in}{1.769062in}}%
\pgfpathlineto{\pgfqpoint{4.630569in}{1.761408in}}%
\pgfpathlineto{\pgfqpoint{4.630953in}{1.766228in}}%
\pgfpathlineto{\pgfqpoint{4.631146in}{1.766631in}}%
\pgfpathlineto{\pgfqpoint{4.631722in}{1.773657in}}%
\pgfpathlineto{\pgfqpoint{4.632299in}{1.767879in}}%
\pgfpathlineto{\pgfqpoint{4.633836in}{1.775059in}}%
\pgfpathlineto{\pgfqpoint{4.634797in}{1.770080in}}%
\pgfpathlineto{\pgfqpoint{4.634989in}{1.771369in}}%
\pgfpathlineto{\pgfqpoint{4.635758in}{1.777786in}}%
\pgfpathlineto{\pgfqpoint{4.636142in}{1.776740in}}%
\pgfpathlineto{\pgfqpoint{4.637679in}{1.771245in}}%
\pgfpathlineto{\pgfqpoint{4.636718in}{1.777352in}}%
\pgfpathlineto{\pgfqpoint{4.638064in}{1.771471in}}%
\pgfpathlineto{\pgfqpoint{4.638256in}{1.775624in}}%
\pgfpathlineto{\pgfqpoint{4.639024in}{1.769354in}}%
\pgfpathlineto{\pgfqpoint{4.639217in}{1.769353in}}%
\pgfpathlineto{\pgfqpoint{4.639601in}{1.760940in}}%
\pgfpathlineto{\pgfqpoint{4.640562in}{1.764680in}}%
\pgfpathlineto{\pgfqpoint{4.640754in}{1.764403in}}%
\pgfpathlineto{\pgfqpoint{4.640946in}{1.765144in}}%
\pgfpathlineto{\pgfqpoint{4.641138in}{1.769150in}}%
\pgfpathlineto{\pgfqpoint{4.641907in}{1.768013in}}%
\pgfpathlineto{\pgfqpoint{4.643444in}{1.758503in}}%
\pgfpathlineto{\pgfqpoint{4.645558in}{1.767110in}}%
\pgfpathlineto{\pgfqpoint{4.646903in}{1.761005in}}%
\pgfpathlineto{\pgfqpoint{4.646327in}{1.767563in}}%
\pgfpathlineto{\pgfqpoint{4.647095in}{1.761616in}}%
\pgfpathlineto{\pgfqpoint{4.648056in}{1.765672in}}%
\pgfpathlineto{\pgfqpoint{4.647480in}{1.760692in}}%
\pgfpathlineto{\pgfqpoint{4.648248in}{1.763983in}}%
\pgfpathlineto{\pgfqpoint{4.649786in}{1.755486in}}%
\pgfpathlineto{\pgfqpoint{4.650170in}{1.759479in}}%
\pgfpathlineto{\pgfqpoint{4.650554in}{1.757179in}}%
\pgfpathlineto{\pgfqpoint{4.651900in}{1.745390in}}%
\pgfpathlineto{\pgfqpoint{4.652092in}{1.748117in}}%
\pgfpathlineto{\pgfqpoint{4.652668in}{1.743068in}}%
\pgfpathlineto{\pgfqpoint{4.653629in}{1.738061in}}%
\pgfpathlineto{\pgfqpoint{4.653821in}{1.739019in}}%
\pgfpathlineto{\pgfqpoint{4.656127in}{1.759664in}}%
\pgfpathlineto{\pgfqpoint{4.656320in}{1.760432in}}%
\pgfpathlineto{\pgfqpoint{4.656512in}{1.758343in}}%
\pgfpathlineto{\pgfqpoint{4.656896in}{1.756224in}}%
\pgfpathlineto{\pgfqpoint{4.657280in}{1.759073in}}%
\pgfpathlineto{\pgfqpoint{4.657473in}{1.757641in}}%
\pgfpathlineto{\pgfqpoint{4.659779in}{1.766616in}}%
\pgfpathlineto{\pgfqpoint{4.657857in}{1.756880in}}%
\pgfpathlineto{\pgfqpoint{4.659971in}{1.766126in}}%
\pgfpathlineto{\pgfqpoint{4.661124in}{1.761822in}}%
\pgfpathlineto{\pgfqpoint{4.661316in}{1.763059in}}%
\pgfpathlineto{\pgfqpoint{4.661700in}{1.765660in}}%
\pgfpathlineto{\pgfqpoint{4.661892in}{1.762616in}}%
\pgfpathlineto{\pgfqpoint{4.662085in}{1.765338in}}%
\pgfpathlineto{\pgfqpoint{4.663238in}{1.753133in}}%
\pgfpathlineto{\pgfqpoint{4.663622in}{1.754811in}}%
\pgfpathlineto{\pgfqpoint{4.664198in}{1.757997in}}%
\pgfpathlineto{\pgfqpoint{4.664006in}{1.754497in}}%
\pgfpathlineto{\pgfqpoint{4.664583in}{1.754972in}}%
\pgfpathlineto{\pgfqpoint{4.665351in}{1.750721in}}%
\pgfpathlineto{\pgfqpoint{4.665928in}{1.752230in}}%
\pgfpathlineto{\pgfqpoint{4.666889in}{1.754826in}}%
\pgfpathlineto{\pgfqpoint{4.667657in}{1.747541in}}%
\pgfpathlineto{\pgfqpoint{4.668042in}{1.750434in}}%
\pgfpathlineto{\pgfqpoint{4.668234in}{1.751047in}}%
\pgfpathlineto{\pgfqpoint{4.668618in}{1.748959in}}%
\pgfpathlineto{\pgfqpoint{4.670540in}{1.736443in}}%
\pgfpathlineto{\pgfqpoint{4.670924in}{1.736052in}}%
\pgfpathlineto{\pgfqpoint{4.672269in}{1.742168in}}%
\pgfpathlineto{\pgfqpoint{4.673422in}{1.733783in}}%
\pgfpathlineto{\pgfqpoint{4.673807in}{1.736488in}}%
\pgfpathlineto{\pgfqpoint{4.674383in}{1.741116in}}%
\pgfpathlineto{\pgfqpoint{4.674191in}{1.736317in}}%
\pgfpathlineto{\pgfqpoint{4.675729in}{1.740786in}}%
\pgfpathlineto{\pgfqpoint{4.677266in}{1.732990in}}%
\pgfpathlineto{\pgfqpoint{4.677842in}{1.738479in}}%
\pgfpathlineto{\pgfqpoint{4.678227in}{1.732568in}}%
\pgfpathlineto{\pgfqpoint{4.678419in}{1.734321in}}%
\pgfpathlineto{\pgfqpoint{4.678611in}{1.734219in}}%
\pgfpathlineto{\pgfqpoint{4.678995in}{1.731021in}}%
\pgfpathlineto{\pgfqpoint{4.679956in}{1.739783in}}%
\pgfpathlineto{\pgfqpoint{4.680148in}{1.740678in}}%
\pgfpathlineto{\pgfqpoint{4.680533in}{1.737903in}}%
\pgfpathlineto{\pgfqpoint{4.682262in}{1.728797in}}%
\pgfpathlineto{\pgfqpoint{4.684184in}{1.721998in}}%
\pgfpathlineto{\pgfqpoint{4.685145in}{1.715437in}}%
\pgfpathlineto{\pgfqpoint{4.685337in}{1.718061in}}%
\pgfpathlineto{\pgfqpoint{4.685529in}{1.717846in}}%
\pgfpathlineto{\pgfqpoint{4.686490in}{1.712151in}}%
\pgfpathlineto{\pgfqpoint{4.686874in}{1.712196in}}%
\pgfpathlineto{\pgfqpoint{4.687259in}{1.711360in}}%
\pgfpathlineto{\pgfqpoint{4.687451in}{1.711910in}}%
\pgfpathlineto{\pgfqpoint{4.688027in}{1.711689in}}%
\pgfpathlineto{\pgfqpoint{4.688988in}{1.717103in}}%
\pgfpathlineto{\pgfqpoint{4.689757in}{1.719713in}}%
\pgfpathlineto{\pgfqpoint{4.689949in}{1.718397in}}%
\pgfpathlineto{\pgfqpoint{4.691294in}{1.710032in}}%
\pgfpathlineto{\pgfqpoint{4.691486in}{1.712945in}}%
\pgfpathlineto{\pgfqpoint{4.691678in}{1.712756in}}%
\pgfpathlineto{\pgfqpoint{4.691871in}{1.715301in}}%
\pgfpathlineto{\pgfqpoint{4.692447in}{1.711366in}}%
\pgfpathlineto{\pgfqpoint{4.693792in}{1.701477in}}%
\pgfpathlineto{\pgfqpoint{4.693984in}{1.702262in}}%
\pgfpathlineto{\pgfqpoint{4.694177in}{1.702819in}}%
\pgfpathlineto{\pgfqpoint{4.694369in}{1.702424in}}%
\pgfpathlineto{\pgfqpoint{4.694561in}{1.699021in}}%
\pgfpathlineto{\pgfqpoint{4.695522in}{1.699407in}}%
\pgfpathlineto{\pgfqpoint{4.696098in}{1.704256in}}%
\pgfpathlineto{\pgfqpoint{4.696867in}{1.702899in}}%
\pgfpathlineto{\pgfqpoint{4.697828in}{1.716761in}}%
\pgfpathlineto{\pgfqpoint{4.698981in}{1.712918in}}%
\pgfpathlineto{\pgfqpoint{4.699942in}{1.710144in}}%
\pgfpathlineto{\pgfqpoint{4.700134in}{1.711807in}}%
\pgfpathlineto{\pgfqpoint{4.700903in}{1.714874in}}%
\pgfpathlineto{\pgfqpoint{4.701287in}{1.714673in}}%
\pgfpathlineto{\pgfqpoint{4.703016in}{1.703307in}}%
\pgfpathlineto{\pgfqpoint{4.703785in}{1.708758in}}%
\pgfpathlineto{\pgfqpoint{4.704746in}{1.713017in}}%
\pgfpathlineto{\pgfqpoint{4.704169in}{1.707257in}}%
\pgfpathlineto{\pgfqpoint{4.704938in}{1.709944in}}%
\pgfpathlineto{\pgfqpoint{4.705899in}{1.706279in}}%
\pgfpathlineto{\pgfqpoint{4.706091in}{1.709161in}}%
\pgfpathlineto{\pgfqpoint{4.708013in}{1.715983in}}%
\pgfpathlineto{\pgfqpoint{4.709358in}{1.714688in}}%
\pgfpathlineto{\pgfqpoint{4.710319in}{1.706667in}}%
\pgfpathlineto{\pgfqpoint{4.710703in}{1.709267in}}%
\pgfpathlineto{\pgfqpoint{4.711856in}{1.715692in}}%
\pgfpathlineto{\pgfqpoint{4.712433in}{1.715309in}}%
\pgfpathlineto{\pgfqpoint{4.713009in}{1.714756in}}%
\pgfpathlineto{\pgfqpoint{4.713201in}{1.715926in}}%
\pgfpathlineto{\pgfqpoint{4.714739in}{1.731339in}}%
\pgfpathlineto{\pgfqpoint{4.714931in}{1.730577in}}%
\pgfpathlineto{\pgfqpoint{4.715123in}{1.732004in}}%
\pgfpathlineto{\pgfqpoint{4.715507in}{1.726897in}}%
\pgfpathlineto{\pgfqpoint{4.716660in}{1.719555in}}%
\pgfpathlineto{\pgfqpoint{4.716852in}{1.719999in}}%
\pgfpathlineto{\pgfqpoint{4.717237in}{1.723797in}}%
\pgfpathlineto{\pgfqpoint{4.718005in}{1.721835in}}%
\pgfpathlineto{\pgfqpoint{4.719351in}{1.717786in}}%
\pgfpathlineto{\pgfqpoint{4.720888in}{1.726430in}}%
\pgfpathlineto{\pgfqpoint{4.721080in}{1.724206in}}%
\pgfpathlineto{\pgfqpoint{4.723194in}{1.710300in}}%
\pgfpathlineto{\pgfqpoint{4.724155in}{1.718981in}}%
\pgfpathlineto{\pgfqpoint{4.724924in}{1.717834in}}%
\pgfpathlineto{\pgfqpoint{4.725116in}{1.715982in}}%
\pgfpathlineto{\pgfqpoint{4.725500in}{1.719996in}}%
\pgfpathlineto{\pgfqpoint{4.725884in}{1.721487in}}%
\pgfpathlineto{\pgfqpoint{4.726077in}{1.719132in}}%
\pgfpathlineto{\pgfqpoint{4.726461in}{1.721268in}}%
\pgfpathlineto{\pgfqpoint{4.728383in}{1.710506in}}%
\pgfpathlineto{\pgfqpoint{4.728575in}{1.712819in}}%
\pgfpathlineto{\pgfqpoint{4.728959in}{1.709967in}}%
\pgfpathlineto{\pgfqpoint{4.729343in}{1.710567in}}%
\pgfpathlineto{\pgfqpoint{4.729536in}{1.708409in}}%
\pgfpathlineto{\pgfqpoint{4.730112in}{1.711653in}}%
\pgfpathlineto{\pgfqpoint{4.730304in}{1.710694in}}%
\pgfpathlineto{\pgfqpoint{4.731265in}{1.717045in}}%
\pgfpathlineto{\pgfqpoint{4.731457in}{1.716164in}}%
\pgfpathlineto{\pgfqpoint{4.732418in}{1.710692in}}%
\pgfpathlineto{\pgfqpoint{4.732802in}{1.711603in}}%
\pgfpathlineto{\pgfqpoint{4.732995in}{1.714834in}}%
\pgfpathlineto{\pgfqpoint{4.733571in}{1.709062in}}%
\pgfpathlineto{\pgfqpoint{4.733763in}{1.709784in}}%
\pgfpathlineto{\pgfqpoint{4.735877in}{1.724952in}}%
\pgfpathlineto{\pgfqpoint{4.736069in}{1.721576in}}%
\pgfpathlineto{\pgfqpoint{4.736646in}{1.725795in}}%
\pgfpathlineto{\pgfqpoint{4.737030in}{1.723899in}}%
\pgfpathlineto{\pgfqpoint{4.737414in}{1.728179in}}%
\pgfpathlineto{\pgfqpoint{4.737991in}{1.722176in}}%
\pgfpathlineto{\pgfqpoint{4.738760in}{1.716950in}}%
\pgfpathlineto{\pgfqpoint{4.739336in}{1.719267in}}%
\pgfpathlineto{\pgfqpoint{4.739913in}{1.724347in}}%
\pgfpathlineto{\pgfqpoint{4.740489in}{1.722251in}}%
\pgfpathlineto{\pgfqpoint{4.740873in}{1.720997in}}%
\pgfpathlineto{\pgfqpoint{4.741066in}{1.723063in}}%
\pgfpathlineto{\pgfqpoint{4.741258in}{1.723085in}}%
\pgfpathlineto{\pgfqpoint{4.741450in}{1.724712in}}%
\pgfpathlineto{\pgfqpoint{4.741834in}{1.722084in}}%
\pgfpathlineto{\pgfqpoint{4.742026in}{1.722085in}}%
\pgfpathlineto{\pgfqpoint{4.742219in}{1.719002in}}%
\pgfpathlineto{\pgfqpoint{4.743179in}{1.719135in}}%
\pgfpathlineto{\pgfqpoint{4.744140in}{1.723724in}}%
\pgfpathlineto{\pgfqpoint{4.744332in}{1.721032in}}%
\pgfpathlineto{\pgfqpoint{4.744717in}{1.721221in}}%
\pgfpathlineto{\pgfqpoint{4.745870in}{1.715331in}}%
\pgfpathlineto{\pgfqpoint{4.746254in}{1.711509in}}%
\pgfpathlineto{\pgfqpoint{4.746831in}{1.715542in}}%
\pgfpathlineto{\pgfqpoint{4.748368in}{1.720892in}}%
\pgfpathlineto{\pgfqpoint{4.748560in}{1.720926in}}%
\pgfpathlineto{\pgfqpoint{4.749137in}{1.718514in}}%
\pgfpathlineto{\pgfqpoint{4.749713in}{1.719611in}}%
\pgfpathlineto{\pgfqpoint{4.750482in}{1.726004in}}%
\pgfpathlineto{\pgfqpoint{4.750866in}{1.722675in}}%
\pgfpathlineto{\pgfqpoint{4.751635in}{1.719643in}}%
\pgfpathlineto{\pgfqpoint{4.752211in}{1.720111in}}%
\pgfpathlineto{\pgfqpoint{4.752404in}{1.720597in}}%
\pgfpathlineto{\pgfqpoint{4.752596in}{1.719505in}}%
\pgfpathlineto{\pgfqpoint{4.752980in}{1.712198in}}%
\pgfpathlineto{\pgfqpoint{4.753749in}{1.715860in}}%
\pgfpathlineto{\pgfqpoint{4.753941in}{1.715634in}}%
\pgfpathlineto{\pgfqpoint{4.755094in}{1.709856in}}%
\pgfpathlineto{\pgfqpoint{4.755863in}{1.714539in}}%
\pgfpathlineto{\pgfqpoint{4.756439in}{1.713778in}}%
\pgfpathlineto{\pgfqpoint{4.757016in}{1.712650in}}%
\pgfpathlineto{\pgfqpoint{4.757208in}{1.715108in}}%
\pgfpathlineto{\pgfqpoint{4.757784in}{1.716491in}}%
\pgfpathlineto{\pgfqpoint{4.757976in}{1.714779in}}%
\pgfpathlineto{\pgfqpoint{4.759514in}{1.722709in}}%
\pgfpathlineto{\pgfqpoint{4.762012in}{1.728415in}}%
\pgfpathlineto{\pgfqpoint{4.760090in}{1.722426in}}%
\pgfpathlineto{\pgfqpoint{4.762396in}{1.727631in}}%
\pgfpathlineto{\pgfqpoint{4.763741in}{1.719692in}}%
\pgfpathlineto{\pgfqpoint{4.764318in}{1.721089in}}%
\pgfpathlineto{\pgfqpoint{4.765087in}{1.725510in}}%
\pgfpathlineto{\pgfqpoint{4.765471in}{1.722768in}}%
\pgfpathlineto{\pgfqpoint{4.766816in}{1.730335in}}%
\pgfpathlineto{\pgfqpoint{4.767200in}{1.728560in}}%
\pgfpathlineto{\pgfqpoint{4.767777in}{1.730086in}}%
\pgfpathlineto{\pgfqpoint{4.768930in}{1.720423in}}%
\pgfpathlineto{\pgfqpoint{4.769122in}{1.721403in}}%
\pgfpathlineto{\pgfqpoint{4.769314in}{1.718693in}}%
\pgfpathlineto{\pgfqpoint{4.769891in}{1.721035in}}%
\pgfpathlineto{\pgfqpoint{4.770467in}{1.716190in}}%
\pgfpathlineto{\pgfqpoint{4.771236in}{1.717143in}}%
\pgfpathlineto{\pgfqpoint{4.772581in}{1.721357in}}%
\pgfpathlineto{\pgfqpoint{4.772965in}{1.718286in}}%
\pgfpathlineto{\pgfqpoint{4.773158in}{1.722283in}}%
\pgfpathlineto{\pgfqpoint{4.773350in}{1.721763in}}%
\pgfpathlineto{\pgfqpoint{4.773734in}{1.725505in}}%
\pgfpathlineto{\pgfqpoint{4.774503in}{1.722390in}}%
\pgfpathlineto{\pgfqpoint{4.774887in}{1.724314in}}%
\pgfpathlineto{\pgfqpoint{4.775272in}{1.721531in}}%
\pgfpathlineto{\pgfqpoint{4.775656in}{1.721426in}}%
\pgfpathlineto{\pgfqpoint{4.775848in}{1.719975in}}%
\pgfpathlineto{\pgfqpoint{4.776040in}{1.722221in}}%
\pgfpathlineto{\pgfqpoint{4.776232in}{1.720871in}}%
\pgfpathlineto{\pgfqpoint{4.777193in}{1.728637in}}%
\pgfpathlineto{\pgfqpoint{4.777578in}{1.726434in}}%
\pgfpathlineto{\pgfqpoint{4.778346in}{1.731883in}}%
\pgfpathlineto{\pgfqpoint{4.779884in}{1.744035in}}%
\pgfpathlineto{\pgfqpoint{4.780076in}{1.741244in}}%
\pgfpathlineto{\pgfqpoint{4.780460in}{1.740375in}}%
\pgfpathlineto{\pgfqpoint{4.780652in}{1.742489in}}%
\pgfpathlineto{\pgfqpoint{4.781421in}{1.748856in}}%
\pgfpathlineto{\pgfqpoint{4.781997in}{1.746426in}}%
\pgfpathlineto{\pgfqpoint{4.782766in}{1.748423in}}%
\pgfpathlineto{\pgfqpoint{4.782574in}{1.745544in}}%
\pgfpathlineto{\pgfqpoint{4.782958in}{1.746204in}}%
\pgfpathlineto{\pgfqpoint{4.783343in}{1.737702in}}%
\pgfpathlineto{\pgfqpoint{4.784111in}{1.741651in}}%
\pgfpathlineto{\pgfqpoint{4.784303in}{1.741308in}}%
\pgfpathlineto{\pgfqpoint{4.784880in}{1.748499in}}%
\pgfpathlineto{\pgfqpoint{4.785264in}{1.741897in}}%
\pgfpathlineto{\pgfqpoint{4.785649in}{1.738973in}}%
\pgfpathlineto{\pgfqpoint{4.786225in}{1.740542in}}%
\pgfpathlineto{\pgfqpoint{4.788531in}{1.753560in}}%
\pgfpathlineto{\pgfqpoint{4.788915in}{1.755038in}}%
\pgfpathlineto{\pgfqpoint{4.790068in}{1.763465in}}%
\pgfpathlineto{\pgfqpoint{4.790645in}{1.755171in}}%
\pgfpathlineto{\pgfqpoint{4.791414in}{1.758568in}}%
\pgfpathlineto{\pgfqpoint{4.791990in}{1.754036in}}%
\pgfpathlineto{\pgfqpoint{4.792567in}{1.756444in}}%
\pgfpathlineto{\pgfqpoint{4.792759in}{1.757220in}}%
\pgfpathlineto{\pgfqpoint{4.793143in}{1.755734in}}%
\pgfpathlineto{\pgfqpoint{4.793335in}{1.750367in}}%
\pgfpathlineto{\pgfqpoint{4.793912in}{1.757451in}}%
\pgfpathlineto{\pgfqpoint{4.794104in}{1.760439in}}%
\pgfpathlineto{\pgfqpoint{4.794873in}{1.756435in}}%
\pgfpathlineto{\pgfqpoint{4.795065in}{1.756461in}}%
\pgfpathlineto{\pgfqpoint{4.796026in}{1.759649in}}%
\pgfpathlineto{\pgfqpoint{4.797179in}{1.753428in}}%
\pgfpathlineto{\pgfqpoint{4.797563in}{1.755884in}}%
\pgfpathlineto{\pgfqpoint{4.798908in}{1.747455in}}%
\pgfpathlineto{\pgfqpoint{4.800253in}{1.757925in}}%
\pgfpathlineto{\pgfqpoint{4.800830in}{1.756349in}}%
\pgfpathlineto{\pgfqpoint{4.801022in}{1.754972in}}%
\pgfpathlineto{\pgfqpoint{4.801214in}{1.756466in}}%
\pgfpathlineto{\pgfqpoint{4.802559in}{1.767401in}}%
\pgfpathlineto{\pgfqpoint{4.803328in}{1.762699in}}%
\pgfpathlineto{\pgfqpoint{4.802944in}{1.767756in}}%
\pgfpathlineto{\pgfqpoint{4.803905in}{1.765089in}}%
\pgfpathlineto{\pgfqpoint{4.804097in}{1.763579in}}%
\pgfpathlineto{\pgfqpoint{4.804673in}{1.765843in}}%
\pgfpathlineto{\pgfqpoint{4.804865in}{1.764742in}}%
\pgfpathlineto{\pgfqpoint{4.805058in}{1.766653in}}%
\pgfpathlineto{\pgfqpoint{4.805634in}{1.761727in}}%
\pgfpathlineto{\pgfqpoint{4.808132in}{1.754706in}}%
\pgfpathlineto{\pgfqpoint{4.806018in}{1.762507in}}%
\pgfpathlineto{\pgfqpoint{4.808324in}{1.756141in}}%
\pgfpathlineto{\pgfqpoint{4.808709in}{1.753898in}}%
\pgfpathlineto{\pgfqpoint{4.808901in}{1.755418in}}%
\pgfpathlineto{\pgfqpoint{4.810438in}{1.745483in}}%
\pgfpathlineto{\pgfqpoint{4.811976in}{1.751110in}}%
\pgfpathlineto{\pgfqpoint{4.812360in}{1.750604in}}%
\pgfpathlineto{\pgfqpoint{4.815242in}{1.738161in}}%
\pgfpathlineto{\pgfqpoint{4.815435in}{1.738269in}}%
\pgfpathlineto{\pgfqpoint{4.817548in}{1.749289in}}%
\pgfpathlineto{\pgfqpoint{4.817741in}{1.749357in}}%
\pgfpathlineto{\pgfqpoint{4.818125in}{1.745650in}}%
\pgfpathlineto{\pgfqpoint{4.818701in}{1.750636in}}%
\pgfpathlineto{\pgfqpoint{4.819278in}{1.752348in}}%
\pgfpathlineto{\pgfqpoint{4.819470in}{1.751737in}}%
\pgfpathlineto{\pgfqpoint{4.819662in}{1.753710in}}%
\pgfpathlineto{\pgfqpoint{4.820239in}{1.749245in}}%
\pgfpathlineto{\pgfqpoint{4.820431in}{1.748804in}}%
\pgfpathlineto{\pgfqpoint{4.820623in}{1.749040in}}%
\pgfpathlineto{\pgfqpoint{4.820815in}{1.751815in}}%
\pgfpathlineto{\pgfqpoint{4.821584in}{1.748087in}}%
\pgfpathlineto{\pgfqpoint{4.823506in}{1.736088in}}%
\pgfpathlineto{\pgfqpoint{4.823698in}{1.738443in}}%
\pgfpathlineto{\pgfqpoint{4.825427in}{1.750337in}}%
\pgfpathlineto{\pgfqpoint{4.826196in}{1.745182in}}%
\pgfpathlineto{\pgfqpoint{4.826965in}{1.746563in}}%
\pgfpathlineto{\pgfqpoint{4.828118in}{1.761428in}}%
\pgfpathlineto{\pgfqpoint{4.828694in}{1.758031in}}%
\pgfpathlineto{\pgfqpoint{4.828886in}{1.757756in}}%
\pgfpathlineto{\pgfqpoint{4.829271in}{1.754051in}}%
\pgfpathlineto{\pgfqpoint{4.829847in}{1.756492in}}%
\pgfpathlineto{\pgfqpoint{4.830039in}{1.758715in}}%
\pgfpathlineto{\pgfqpoint{4.830424in}{1.754068in}}%
\pgfpathlineto{\pgfqpoint{4.830808in}{1.756172in}}%
\pgfpathlineto{\pgfqpoint{4.831000in}{1.756033in}}%
\pgfpathlineto{\pgfqpoint{4.831769in}{1.748664in}}%
\pgfpathlineto{\pgfqpoint{4.832345in}{1.752137in}}%
\pgfpathlineto{\pgfqpoint{4.832730in}{1.751367in}}%
\pgfpathlineto{\pgfqpoint{4.833114in}{1.755165in}}%
\pgfpathlineto{\pgfqpoint{4.833498in}{1.748394in}}%
\pgfpathlineto{\pgfqpoint{4.834459in}{1.749077in}}%
\pgfpathlineto{\pgfqpoint{4.834651in}{1.749195in}}%
\pgfpathlineto{\pgfqpoint{4.834844in}{1.748685in}}%
\pgfpathlineto{\pgfqpoint{4.835228in}{1.755235in}}%
\pgfpathlineto{\pgfqpoint{4.835997in}{1.752647in}}%
\pgfpathlineto{\pgfqpoint{4.836189in}{1.751831in}}%
\pgfpathlineto{\pgfqpoint{4.836381in}{1.753318in}}%
\pgfpathlineto{\pgfqpoint{4.836573in}{1.752844in}}%
\pgfpathlineto{\pgfqpoint{4.838495in}{1.766188in}}%
\pgfpathlineto{\pgfqpoint{4.839648in}{1.757181in}}%
\pgfpathlineto{\pgfqpoint{4.840609in}{1.761871in}}%
\pgfpathlineto{\pgfqpoint{4.840993in}{1.761791in}}%
\pgfpathlineto{\pgfqpoint{4.842530in}{1.767552in}}%
\pgfpathlineto{\pgfqpoint{4.842722in}{1.766595in}}%
\pgfpathlineto{\pgfqpoint{4.842915in}{1.770746in}}%
\pgfpathlineto{\pgfqpoint{4.843107in}{1.770672in}}%
\pgfpathlineto{\pgfqpoint{4.844068in}{1.775945in}}%
\pgfpathlineto{\pgfqpoint{4.844644in}{1.775175in}}%
\pgfpathlineto{\pgfqpoint{4.845605in}{1.780441in}}%
\pgfpathlineto{\pgfqpoint{4.845797in}{1.777242in}}%
\pgfpathlineto{\pgfqpoint{4.846181in}{1.777810in}}%
\pgfpathlineto{\pgfqpoint{4.846950in}{1.772689in}}%
\pgfpathlineto{\pgfqpoint{4.847335in}{1.776549in}}%
\pgfpathlineto{\pgfqpoint{4.847719in}{1.779963in}}%
\pgfpathlineto{\pgfqpoint{4.847911in}{1.776311in}}%
\pgfpathlineto{\pgfqpoint{4.848103in}{1.776559in}}%
\pgfpathlineto{\pgfqpoint{4.848295in}{1.774799in}}%
\pgfpathlineto{\pgfqpoint{4.848488in}{1.779196in}}%
\pgfpathlineto{\pgfqpoint{4.848680in}{1.778811in}}%
\pgfpathlineto{\pgfqpoint{4.848872in}{1.779799in}}%
\pgfpathlineto{\pgfqpoint{4.849064in}{1.776436in}}%
\pgfpathlineto{\pgfqpoint{4.849641in}{1.773877in}}%
\pgfpathlineto{\pgfqpoint{4.850409in}{1.775379in}}%
\pgfpathlineto{\pgfqpoint{4.850794in}{1.779826in}}%
\pgfpathlineto{\pgfqpoint{4.851370in}{1.775892in}}%
\pgfpathlineto{\pgfqpoint{4.852331in}{1.772148in}}%
\pgfpathlineto{\pgfqpoint{4.852523in}{1.773202in}}%
\pgfpathlineto{\pgfqpoint{4.852715in}{1.774086in}}%
\pgfpathlineto{\pgfqpoint{4.852907in}{1.770296in}}%
\pgfpathlineto{\pgfqpoint{4.853292in}{1.771951in}}%
\pgfpathlineto{\pgfqpoint{4.854060in}{1.773846in}}%
\pgfpathlineto{\pgfqpoint{4.853676in}{1.771577in}}%
\pgfpathlineto{\pgfqpoint{4.854829in}{1.773008in}}%
\pgfpathlineto{\pgfqpoint{4.855598in}{1.773834in}}%
\pgfpathlineto{\pgfqpoint{4.855982in}{1.771353in}}%
\pgfpathlineto{\pgfqpoint{4.856943in}{1.775990in}}%
\pgfpathlineto{\pgfqpoint{4.857135in}{1.772243in}}%
\pgfpathlineto{\pgfqpoint{4.857712in}{1.774575in}}%
\pgfpathlineto{\pgfqpoint{4.857904in}{1.772782in}}%
\pgfpathlineto{\pgfqpoint{4.858096in}{1.770810in}}%
\pgfpathlineto{\pgfqpoint{4.858672in}{1.775616in}}%
\pgfpathlineto{\pgfqpoint{4.859057in}{1.780082in}}%
\pgfpathlineto{\pgfqpoint{4.859825in}{1.779362in}}%
\pgfpathlineto{\pgfqpoint{4.860978in}{1.767727in}}%
\pgfpathlineto{\pgfqpoint{4.861363in}{1.769739in}}%
\pgfpathlineto{\pgfqpoint{4.861939in}{1.773360in}}%
\pgfpathlineto{\pgfqpoint{4.862131in}{1.769971in}}%
\pgfpathlineto{\pgfqpoint{4.862324in}{1.767630in}}%
\pgfpathlineto{\pgfqpoint{4.862516in}{1.770155in}}%
\pgfpathlineto{\pgfqpoint{4.863284in}{1.769031in}}%
\pgfpathlineto{\pgfqpoint{4.863477in}{1.768101in}}%
\pgfpathlineto{\pgfqpoint{4.863669in}{1.771520in}}%
\pgfpathlineto{\pgfqpoint{4.864245in}{1.768700in}}%
\pgfpathlineto{\pgfqpoint{4.864437in}{1.769054in}}%
\pgfpathlineto{\pgfqpoint{4.864630in}{1.764809in}}%
\pgfpathlineto{\pgfqpoint{4.865398in}{1.769442in}}%
\pgfpathlineto{\pgfqpoint{4.865590in}{1.769899in}}%
\pgfpathlineto{\pgfqpoint{4.865783in}{1.767667in}}%
\pgfpathlineto{\pgfqpoint{4.866167in}{1.768489in}}%
\pgfpathlineto{\pgfqpoint{4.866551in}{1.767042in}}%
\pgfpathlineto{\pgfqpoint{4.866743in}{1.768046in}}%
\pgfpathlineto{\pgfqpoint{4.868281in}{1.778474in}}%
\pgfpathlineto{\pgfqpoint{4.867320in}{1.766430in}}%
\pgfpathlineto{\pgfqpoint{4.868473in}{1.777330in}}%
\pgfpathlineto{\pgfqpoint{4.868665in}{1.777860in}}%
\pgfpathlineto{\pgfqpoint{4.870779in}{1.759307in}}%
\pgfpathlineto{\pgfqpoint{4.871356in}{1.758714in}}%
\pgfpathlineto{\pgfqpoint{4.871548in}{1.761452in}}%
\pgfpathlineto{\pgfqpoint{4.872316in}{1.758373in}}%
\pgfpathlineto{\pgfqpoint{4.872509in}{1.759589in}}%
\pgfpathlineto{\pgfqpoint{4.873085in}{1.761715in}}%
\pgfpathlineto{\pgfqpoint{4.873469in}{1.758665in}}%
\pgfpathlineto{\pgfqpoint{4.873854in}{1.757349in}}%
\pgfpathlineto{\pgfqpoint{4.874046in}{1.758540in}}%
\pgfpathlineto{\pgfqpoint{4.875199in}{1.764037in}}%
\pgfpathlineto{\pgfqpoint{4.874622in}{1.758025in}}%
\pgfpathlineto{\pgfqpoint{4.875391in}{1.762700in}}%
\pgfpathlineto{\pgfqpoint{4.877889in}{1.774764in}}%
\pgfpathlineto{\pgfqpoint{4.878081in}{1.775674in}}%
\pgfpathlineto{\pgfqpoint{4.879042in}{1.775278in}}%
\pgfpathlineto{\pgfqpoint{4.880003in}{1.771564in}}%
\pgfpathlineto{\pgfqpoint{4.880387in}{1.774777in}}%
\pgfpathlineto{\pgfqpoint{4.880964in}{1.776494in}}%
\pgfpathlineto{\pgfqpoint{4.881348in}{1.775159in}}%
\pgfpathlineto{\pgfqpoint{4.883270in}{1.767184in}}%
\pgfpathlineto{\pgfqpoint{4.883846in}{1.767941in}}%
\pgfpathlineto{\pgfqpoint{4.884039in}{1.770335in}}%
\pgfpathlineto{\pgfqpoint{4.884615in}{1.766256in}}%
\pgfpathlineto{\pgfqpoint{4.884999in}{1.768969in}}%
\pgfpathlineto{\pgfqpoint{4.885576in}{1.767970in}}%
\pgfpathlineto{\pgfqpoint{4.885768in}{1.769819in}}%
\pgfpathlineto{\pgfqpoint{4.885960in}{1.768517in}}%
\pgfpathlineto{\pgfqpoint{4.886152in}{1.770907in}}%
\pgfpathlineto{\pgfqpoint{4.886729in}{1.766209in}}%
\pgfpathlineto{\pgfqpoint{4.887305in}{1.757130in}}%
\pgfpathlineto{\pgfqpoint{4.888651in}{1.759677in}}%
\pgfpathlineto{\pgfqpoint{4.889419in}{1.763635in}}%
\pgfpathlineto{\pgfqpoint{4.889996in}{1.763363in}}%
\pgfpathlineto{\pgfqpoint{4.890188in}{1.761682in}}%
\pgfpathlineto{\pgfqpoint{4.890380in}{1.764167in}}%
\pgfpathlineto{\pgfqpoint{4.890764in}{1.762983in}}%
\pgfpathlineto{\pgfqpoint{4.893070in}{1.781211in}}%
\pgfpathlineto{\pgfqpoint{4.893263in}{1.778094in}}%
\pgfpathlineto{\pgfqpoint{4.893647in}{1.780184in}}%
\pgfpathlineto{\pgfqpoint{4.893839in}{1.779174in}}%
\pgfpathlineto{\pgfqpoint{4.894992in}{1.773352in}}%
\pgfpathlineto{\pgfqpoint{4.895377in}{1.776957in}}%
\pgfpathlineto{\pgfqpoint{4.896145in}{1.775352in}}%
\pgfpathlineto{\pgfqpoint{4.896337in}{1.775570in}}%
\pgfpathlineto{\pgfqpoint{4.896530in}{1.778534in}}%
\pgfpathlineto{\pgfqpoint{4.897490in}{1.776271in}}%
\pgfpathlineto{\pgfqpoint{4.897875in}{1.774016in}}%
\pgfpathlineto{\pgfqpoint{4.898259in}{1.776994in}}%
\pgfpathlineto{\pgfqpoint{4.900373in}{1.784466in}}%
\pgfpathlineto{\pgfqpoint{4.900757in}{1.783661in}}%
\pgfpathlineto{\pgfqpoint{4.901142in}{1.781688in}}%
\pgfpathlineto{\pgfqpoint{4.901526in}{1.785418in}}%
\pgfpathlineto{\pgfqpoint{4.901718in}{1.785537in}}%
\pgfpathlineto{\pgfqpoint{4.901910in}{1.783723in}}%
\pgfpathlineto{\pgfqpoint{4.902102in}{1.787519in}}%
\pgfpathlineto{\pgfqpoint{4.903640in}{1.796262in}}%
\pgfpathlineto{\pgfqpoint{4.904024in}{1.797350in}}%
\pgfpathlineto{\pgfqpoint{4.904985in}{1.794613in}}%
\pgfpathlineto{\pgfqpoint{4.905177in}{1.796610in}}%
\pgfpathlineto{\pgfqpoint{4.905369in}{1.793918in}}%
\pgfpathlineto{\pgfqpoint{4.905754in}{1.794510in}}%
\pgfpathlineto{\pgfqpoint{4.908060in}{1.784415in}}%
\pgfpathlineto{\pgfqpoint{4.908828in}{1.784989in}}%
\pgfpathlineto{\pgfqpoint{4.910558in}{1.775237in}}%
\pgfpathlineto{\pgfqpoint{4.911711in}{1.780756in}}%
\pgfpathlineto{\pgfqpoint{4.911903in}{1.778591in}}%
\pgfpathlineto{\pgfqpoint{4.912287in}{1.775151in}}%
\pgfpathlineto{\pgfqpoint{4.912864in}{1.778732in}}%
\pgfpathlineto{\pgfqpoint{4.913825in}{1.780189in}}%
\pgfpathlineto{\pgfqpoint{4.913248in}{1.778293in}}%
\pgfpathlineto{\pgfqpoint{4.914017in}{1.779920in}}%
\pgfpathlineto{\pgfqpoint{4.915362in}{1.767756in}}%
\pgfpathlineto{\pgfqpoint{4.915554in}{1.767967in}}%
\pgfpathlineto{\pgfqpoint{4.916899in}{1.765130in}}%
\pgfpathlineto{\pgfqpoint{4.915938in}{1.769296in}}%
\pgfpathlineto{\pgfqpoint{4.917091in}{1.765669in}}%
\pgfpathlineto{\pgfqpoint{4.917284in}{1.765488in}}%
\pgfpathlineto{\pgfqpoint{4.918052in}{1.774095in}}%
\pgfpathlineto{\pgfqpoint{4.918629in}{1.771208in}}%
\pgfpathlineto{\pgfqpoint{4.921127in}{1.759262in}}%
\pgfpathlineto{\pgfqpoint{4.921319in}{1.761664in}}%
\pgfpathlineto{\pgfqpoint{4.922088in}{1.757860in}}%
\pgfpathlineto{\pgfqpoint{4.922472in}{1.754167in}}%
\pgfpathlineto{\pgfqpoint{4.922857in}{1.758436in}}%
\pgfpathlineto{\pgfqpoint{4.923049in}{1.756366in}}%
\pgfpathlineto{\pgfqpoint{4.925547in}{1.769847in}}%
\pgfpathlineto{\pgfqpoint{4.925931in}{1.769018in}}%
\pgfpathlineto{\pgfqpoint{4.927661in}{1.774533in}}%
\pgfpathlineto{\pgfqpoint{4.926316in}{1.767813in}}%
\pgfpathlineto{\pgfqpoint{4.928237in}{1.771541in}}%
\pgfpathlineto{\pgfqpoint{4.928429in}{1.768769in}}%
\pgfpathlineto{\pgfqpoint{4.929198in}{1.772052in}}%
\pgfpathlineto{\pgfqpoint{4.929390in}{1.770362in}}%
\pgfpathlineto{\pgfqpoint{4.929582in}{1.769982in}}%
\pgfpathlineto{\pgfqpoint{4.929967in}{1.771349in}}%
\pgfpathlineto{\pgfqpoint{4.932657in}{1.786526in}}%
\pgfpathlineto{\pgfqpoint{4.932849in}{1.786357in}}%
\pgfpathlineto{\pgfqpoint{4.933618in}{1.788857in}}%
\pgfpathlineto{\pgfqpoint{4.933810in}{1.786527in}}%
\pgfpathlineto{\pgfqpoint{4.934579in}{1.797760in}}%
\pgfpathlineto{\pgfqpoint{4.934963in}{1.794020in}}%
\pgfpathlineto{\pgfqpoint{4.936308in}{1.785764in}}%
\pgfpathlineto{\pgfqpoint{4.936500in}{1.789325in}}%
\pgfpathlineto{\pgfqpoint{4.937461in}{1.782644in}}%
\pgfpathlineto{\pgfqpoint{4.938422in}{1.784335in}}%
\pgfpathlineto{\pgfqpoint{4.941497in}{1.814410in}}%
\pgfpathlineto{\pgfqpoint{4.942073in}{1.812959in}}%
\pgfpathlineto{\pgfqpoint{4.942458in}{1.811143in}}%
\pgfpathlineto{\pgfqpoint{4.942842in}{1.811995in}}%
\pgfpathlineto{\pgfqpoint{4.943418in}{1.820882in}}%
\pgfpathlineto{\pgfqpoint{4.944187in}{1.816342in}}%
\pgfpathlineto{\pgfqpoint{4.944764in}{1.814492in}}%
\pgfpathlineto{\pgfqpoint{4.944956in}{1.818035in}}%
\pgfpathlineto{\pgfqpoint{4.945532in}{1.816853in}}%
\pgfpathlineto{\pgfqpoint{4.945917in}{1.817436in}}%
\pgfpathlineto{\pgfqpoint{4.947454in}{1.823763in}}%
\pgfpathlineto{\pgfqpoint{4.947838in}{1.818080in}}%
\pgfpathlineto{\pgfqpoint{4.948607in}{1.822679in}}%
\pgfpathlineto{\pgfqpoint{4.948799in}{1.822602in}}%
\pgfpathlineto{\pgfqpoint{4.948991in}{1.825113in}}%
\pgfpathlineto{\pgfqpoint{4.949184in}{1.819973in}}%
\pgfpathlineto{\pgfqpoint{4.949568in}{1.821108in}}%
\pgfpathlineto{\pgfqpoint{4.950529in}{1.817216in}}%
\pgfpathlineto{\pgfqpoint{4.951105in}{1.821424in}}%
\pgfpathlineto{\pgfqpoint{4.951490in}{1.817403in}}%
\pgfpathlineto{\pgfqpoint{4.952258in}{1.816232in}}%
\pgfpathlineto{\pgfqpoint{4.952450in}{1.816477in}}%
\pgfpathlineto{\pgfqpoint{4.952643in}{1.817661in}}%
\pgfpathlineto{\pgfqpoint{4.953027in}{1.814051in}}%
\pgfpathlineto{\pgfqpoint{4.953796in}{1.812905in}}%
\pgfpathlineto{\pgfqpoint{4.953411in}{1.814467in}}%
\pgfpathlineto{\pgfqpoint{4.954180in}{1.813684in}}%
\pgfpathlineto{\pgfqpoint{4.954372in}{1.814045in}}%
\pgfpathlineto{\pgfqpoint{4.954564in}{1.813329in}}%
\pgfpathlineto{\pgfqpoint{4.955141in}{1.814899in}}%
\pgfpathlineto{\pgfqpoint{4.956102in}{1.808691in}}%
\pgfpathlineto{\pgfqpoint{4.957062in}{1.814113in}}%
\pgfpathlineto{\pgfqpoint{4.957447in}{1.812746in}}%
\pgfpathlineto{\pgfqpoint{4.958792in}{1.806544in}}%
\pgfpathlineto{\pgfqpoint{4.958984in}{1.807119in}}%
\pgfpathlineto{\pgfqpoint{4.959176in}{1.806724in}}%
\pgfpathlineto{\pgfqpoint{4.959368in}{1.808624in}}%
\pgfpathlineto{\pgfqpoint{4.960329in}{1.812091in}}%
\pgfpathlineto{\pgfqpoint{4.959753in}{1.807299in}}%
\pgfpathlineto{\pgfqpoint{4.960521in}{1.810368in}}%
\pgfpathlineto{\pgfqpoint{4.962059in}{1.799966in}}%
\pgfpathlineto{\pgfqpoint{4.963020in}{1.804171in}}%
\pgfpathlineto{\pgfqpoint{4.963212in}{1.800805in}}%
\pgfpathlineto{\pgfqpoint{4.963980in}{1.801512in}}%
\pgfpathlineto{\pgfqpoint{4.964365in}{1.795695in}}%
\pgfpathlineto{\pgfqpoint{4.964557in}{1.797614in}}%
\pgfpathlineto{\pgfqpoint{4.964749in}{1.794173in}}%
\pgfpathlineto{\pgfqpoint{4.965518in}{1.796586in}}%
\pgfpathlineto{\pgfqpoint{4.966286in}{1.792945in}}%
\pgfpathlineto{\pgfqpoint{4.966479in}{1.795852in}}%
\pgfpathlineto{\pgfqpoint{4.968400in}{1.804166in}}%
\pgfpathlineto{\pgfqpoint{4.969169in}{1.803570in}}%
\pgfpathlineto{\pgfqpoint{4.969553in}{1.799258in}}%
\pgfpathlineto{\pgfqpoint{4.970322in}{1.802174in}}%
\pgfpathlineto{\pgfqpoint{4.970514in}{1.801536in}}%
\pgfpathlineto{\pgfqpoint{4.970706in}{1.802713in}}%
\pgfpathlineto{\pgfqpoint{4.971091in}{1.805436in}}%
\pgfpathlineto{\pgfqpoint{4.971667in}{1.803634in}}%
\pgfpathlineto{\pgfqpoint{4.972628in}{1.799937in}}%
\pgfpathlineto{\pgfqpoint{4.972052in}{1.805445in}}%
\pgfpathlineto{\pgfqpoint{4.972820in}{1.802197in}}%
\pgfpathlineto{\pgfqpoint{4.973012in}{1.803129in}}%
\pgfpathlineto{\pgfqpoint{4.973397in}{1.799515in}}%
\pgfpathlineto{\pgfqpoint{4.973589in}{1.799749in}}%
\pgfpathlineto{\pgfqpoint{4.975703in}{1.812865in}}%
\pgfpathlineto{\pgfqpoint{4.976856in}{1.807460in}}%
\pgfpathlineto{\pgfqpoint{4.977432in}{1.809332in}}%
\pgfpathlineto{\pgfqpoint{4.977817in}{1.817207in}}%
\pgfpathlineto{\pgfqpoint{4.978585in}{1.813452in}}%
\pgfpathlineto{\pgfqpoint{4.979546in}{1.808831in}}%
\pgfpathlineto{\pgfqpoint{4.979738in}{1.810880in}}%
\pgfpathlineto{\pgfqpoint{4.980123in}{1.810125in}}%
\pgfpathlineto{\pgfqpoint{4.980699in}{1.805896in}}%
\pgfpathlineto{\pgfqpoint{4.981276in}{1.809218in}}%
\pgfpathlineto{\pgfqpoint{4.982429in}{1.803654in}}%
\pgfpathlineto{\pgfqpoint{4.982044in}{1.809572in}}%
\pgfpathlineto{\pgfqpoint{4.982621in}{1.805326in}}%
\pgfpathlineto{\pgfqpoint{4.982813in}{1.804449in}}%
\pgfpathlineto{\pgfqpoint{4.983197in}{1.805725in}}%
\pgfpathlineto{\pgfqpoint{4.983389in}{1.805065in}}%
\pgfpathlineto{\pgfqpoint{4.984158in}{1.809182in}}%
\pgfpathlineto{\pgfqpoint{4.984542in}{1.806962in}}%
\pgfpathlineto{\pgfqpoint{4.984735in}{1.806415in}}%
\pgfpathlineto{\pgfqpoint{4.984927in}{1.801669in}}%
\pgfpathlineto{\pgfqpoint{4.985695in}{1.807628in}}%
\pgfpathlineto{\pgfqpoint{4.985888in}{1.804841in}}%
\pgfpathlineto{\pgfqpoint{4.986272in}{1.805898in}}%
\pgfpathlineto{\pgfqpoint{4.987425in}{1.799537in}}%
\pgfpathlineto{\pgfqpoint{4.987617in}{1.799011in}}%
\pgfpathlineto{\pgfqpoint{4.987809in}{1.800945in}}%
\pgfpathlineto{\pgfqpoint{4.988194in}{1.801605in}}%
\pgfpathlineto{\pgfqpoint{4.989347in}{1.797865in}}%
\pgfpathlineto{\pgfqpoint{4.988770in}{1.802366in}}%
\pgfpathlineto{\pgfqpoint{4.989539in}{1.797986in}}%
\pgfpathlineto{\pgfqpoint{4.990307in}{1.796768in}}%
\pgfpathlineto{\pgfqpoint{4.990500in}{1.799405in}}%
\pgfpathlineto{\pgfqpoint{4.990692in}{1.798096in}}%
\pgfpathlineto{\pgfqpoint{4.990884in}{1.800127in}}%
\pgfpathlineto{\pgfqpoint{4.991268in}{1.799389in}}%
\pgfpathlineto{\pgfqpoint{4.992229in}{1.804113in}}%
\pgfpathlineto{\pgfqpoint{4.992421in}{1.801671in}}%
\pgfpathlineto{\pgfqpoint{4.993574in}{1.809566in}}%
\pgfpathlineto{\pgfqpoint{4.992998in}{1.801201in}}%
\pgfpathlineto{\pgfqpoint{4.994343in}{1.806568in}}%
\pgfpathlineto{\pgfqpoint{4.995880in}{1.800221in}}%
\pgfpathlineto{\pgfqpoint{4.996073in}{1.800957in}}%
\pgfpathlineto{\pgfqpoint{4.996265in}{1.799111in}}%
\pgfpathlineto{\pgfqpoint{4.996457in}{1.797239in}}%
\pgfpathlineto{\pgfqpoint{4.996841in}{1.800188in}}%
\pgfpathlineto{\pgfqpoint{4.997994in}{1.809345in}}%
\pgfpathlineto{\pgfqpoint{4.998186in}{1.807570in}}%
\pgfpathlineto{\pgfqpoint{4.999724in}{1.814403in}}%
\pgfpathlineto{\pgfqpoint{4.999916in}{1.811286in}}%
\pgfpathlineto{\pgfqpoint{5.000685in}{1.815256in}}%
\pgfpathlineto{\pgfqpoint{5.001453in}{1.818867in}}%
\pgfpathlineto{\pgfqpoint{5.001645in}{1.814525in}}%
\pgfpathlineto{\pgfqpoint{5.002222in}{1.818750in}}%
\pgfpathlineto{\pgfqpoint{5.002798in}{1.814703in}}%
\pgfpathlineto{\pgfqpoint{5.002991in}{1.810687in}}%
\pgfpathlineto{\pgfqpoint{5.003759in}{1.817563in}}%
\pgfpathlineto{\pgfqpoint{5.003951in}{1.816608in}}%
\pgfpathlineto{\pgfqpoint{5.004144in}{1.818817in}}%
\pgfpathlineto{\pgfqpoint{5.005104in}{1.817456in}}%
\pgfpathlineto{\pgfqpoint{5.005489in}{1.822905in}}%
\pgfpathlineto{\pgfqpoint{5.006450in}{1.822652in}}%
\pgfpathlineto{\pgfqpoint{5.006834in}{1.825063in}}%
\pgfpathlineto{\pgfqpoint{5.007026in}{1.823428in}}%
\pgfpathlineto{\pgfqpoint{5.007603in}{1.815858in}}%
\pgfpathlineto{\pgfqpoint{5.008179in}{1.819442in}}%
\pgfpathlineto{\pgfqpoint{5.008371in}{1.819441in}}%
\pgfpathlineto{\pgfqpoint{5.008563in}{1.818291in}}%
\pgfpathlineto{\pgfqpoint{5.009140in}{1.820873in}}%
\pgfpathlineto{\pgfqpoint{5.009524in}{1.822620in}}%
\pgfpathlineto{\pgfqpoint{5.010869in}{1.828095in}}%
\pgfpathlineto{\pgfqpoint{5.010101in}{1.822078in}}%
\pgfpathlineto{\pgfqpoint{5.011062in}{1.827667in}}%
\pgfpathlineto{\pgfqpoint{5.012407in}{1.817319in}}%
\pgfpathlineto{\pgfqpoint{5.012983in}{1.823682in}}%
\pgfpathlineto{\pgfqpoint{5.013560in}{1.821517in}}%
\pgfpathlineto{\pgfqpoint{5.015481in}{1.814597in}}%
\pgfpathlineto{\pgfqpoint{5.016827in}{1.820434in}}%
\pgfpathlineto{\pgfqpoint{5.017403in}{1.825044in}}%
\pgfpathlineto{\pgfqpoint{5.017788in}{1.821702in}}%
\pgfpathlineto{\pgfqpoint{5.019133in}{1.810451in}}%
\pgfpathlineto{\pgfqpoint{5.020094in}{1.816508in}}%
\pgfpathlineto{\pgfqpoint{5.020670in}{1.815750in}}%
\pgfpathlineto{\pgfqpoint{5.023168in}{1.799535in}}%
\pgfpathlineto{\pgfqpoint{5.023937in}{1.801543in}}%
\pgfpathlineto{\pgfqpoint{5.025474in}{1.807826in}}%
\pgfpathlineto{\pgfqpoint{5.026819in}{1.802926in}}%
\pgfpathlineto{\pgfqpoint{5.027012in}{1.804065in}}%
\pgfpathlineto{\pgfqpoint{5.028165in}{1.811570in}}%
\pgfpathlineto{\pgfqpoint{5.028357in}{1.811216in}}%
\pgfpathlineto{\pgfqpoint{5.028741in}{1.810906in}}%
\pgfpathlineto{\pgfqpoint{5.028933in}{1.812970in}}%
\pgfpathlineto{\pgfqpoint{5.029318in}{1.805823in}}%
\pgfpathlineto{\pgfqpoint{5.029510in}{1.806176in}}%
\pgfpathlineto{\pgfqpoint{5.029894in}{1.804565in}}%
\pgfpathlineto{\pgfqpoint{5.030471in}{1.801512in}}%
\pgfpathlineto{\pgfqpoint{5.030855in}{1.804055in}}%
\pgfpathlineto{\pgfqpoint{5.031239in}{1.807628in}}%
\pgfpathlineto{\pgfqpoint{5.031816in}{1.802905in}}%
\pgfpathlineto{\pgfqpoint{5.032008in}{1.803790in}}%
\pgfpathlineto{\pgfqpoint{5.032392in}{1.800524in}}%
\pgfpathlineto{\pgfqpoint{5.032584in}{1.798738in}}%
\pgfpathlineto{\pgfqpoint{5.032969in}{1.802151in}}%
\pgfpathlineto{\pgfqpoint{5.033545in}{1.806210in}}%
\pgfpathlineto{\pgfqpoint{5.034122in}{1.804965in}}%
\pgfpathlineto{\pgfqpoint{5.034506in}{1.806259in}}%
\pgfpathlineto{\pgfqpoint{5.034698in}{1.804134in}}%
\pgfpathlineto{\pgfqpoint{5.035659in}{1.793447in}}%
\pgfpathlineto{\pgfqpoint{5.036812in}{1.795785in}}%
\pgfpathlineto{\pgfqpoint{5.038542in}{1.802176in}}%
\pgfpathlineto{\pgfqpoint{5.038734in}{1.798665in}}%
\pgfpathlineto{\pgfqpoint{5.039502in}{1.804053in}}%
\pgfpathlineto{\pgfqpoint{5.039695in}{1.804755in}}%
\pgfpathlineto{\pgfqpoint{5.039887in}{1.803905in}}%
\pgfpathlineto{\pgfqpoint{5.040079in}{1.803897in}}%
\pgfpathlineto{\pgfqpoint{5.040271in}{1.801535in}}%
\pgfpathlineto{\pgfqpoint{5.040848in}{1.805050in}}%
\pgfpathlineto{\pgfqpoint{5.041040in}{1.802483in}}%
\pgfpathlineto{\pgfqpoint{5.041424in}{1.805733in}}%
\pgfpathlineto{\pgfqpoint{5.041808in}{1.804152in}}%
\pgfpathlineto{\pgfqpoint{5.042962in}{1.796437in}}%
\pgfpathlineto{\pgfqpoint{5.043154in}{1.800922in}}%
\pgfpathlineto{\pgfqpoint{5.043922in}{1.799946in}}%
\pgfpathlineto{\pgfqpoint{5.044115in}{1.797357in}}%
\pgfpathlineto{\pgfqpoint{5.044499in}{1.802478in}}%
\pgfpathlineto{\pgfqpoint{5.045460in}{1.805863in}}%
\pgfpathlineto{\pgfqpoint{5.045652in}{1.804314in}}%
\pgfpathlineto{\pgfqpoint{5.046036in}{1.806353in}}%
\pgfpathlineto{\pgfqpoint{5.046228in}{1.807906in}}%
\pgfpathlineto{\pgfqpoint{5.046997in}{1.805792in}}%
\pgfpathlineto{\pgfqpoint{5.047189in}{1.807581in}}%
\pgfpathlineto{\pgfqpoint{5.048342in}{1.800706in}}%
\pgfpathlineto{\pgfqpoint{5.048534in}{1.802457in}}%
\pgfpathlineto{\pgfqpoint{5.048919in}{1.804244in}}%
\pgfpathlineto{\pgfqpoint{5.049111in}{1.801372in}}%
\pgfpathlineto{\pgfqpoint{5.050840in}{1.790008in}}%
\pgfpathlineto{\pgfqpoint{5.051225in}{1.790597in}}%
\pgfpathlineto{\pgfqpoint{5.052378in}{1.795351in}}%
\pgfpathlineto{\pgfqpoint{5.052570in}{1.792984in}}%
\pgfpathlineto{\pgfqpoint{5.052954in}{1.795090in}}%
\pgfpathlineto{\pgfqpoint{5.053339in}{1.792682in}}%
\pgfpathlineto{\pgfqpoint{5.053531in}{1.789567in}}%
\pgfpathlineto{\pgfqpoint{5.054107in}{1.795900in}}%
\pgfpathlineto{\pgfqpoint{5.054492in}{1.790615in}}%
\pgfpathlineto{\pgfqpoint{5.054876in}{1.787143in}}%
\pgfpathlineto{\pgfqpoint{5.056029in}{1.788761in}}%
\pgfpathlineto{\pgfqpoint{5.056413in}{1.788522in}}%
\pgfpathlineto{\pgfqpoint{5.056990in}{1.790004in}}%
\pgfpathlineto{\pgfqpoint{5.057182in}{1.787621in}}%
\pgfpathlineto{\pgfqpoint{5.057374in}{1.785605in}}%
\pgfpathlineto{\pgfqpoint{5.057951in}{1.785966in}}%
\pgfpathlineto{\pgfqpoint{5.058335in}{1.791015in}}%
\pgfpathlineto{\pgfqpoint{5.058719in}{1.785846in}}%
\pgfpathlineto{\pgfqpoint{5.059104in}{1.787203in}}%
\pgfpathlineto{\pgfqpoint{5.059296in}{1.790075in}}%
\pgfpathlineto{\pgfqpoint{5.059872in}{1.785777in}}%
\pgfpathlineto{\pgfqpoint{5.060257in}{1.787846in}}%
\pgfpathlineto{\pgfqpoint{5.060641in}{1.791153in}}%
\pgfpathlineto{\pgfqpoint{5.060833in}{1.786052in}}%
\pgfpathlineto{\pgfqpoint{5.061217in}{1.784714in}}%
\pgfpathlineto{\pgfqpoint{5.061410in}{1.787748in}}%
\pgfpathlineto{\pgfqpoint{5.062178in}{1.798894in}}%
\pgfpathlineto{\pgfqpoint{5.062947in}{1.795457in}}%
\pgfpathlineto{\pgfqpoint{5.063331in}{1.798079in}}%
\pgfpathlineto{\pgfqpoint{5.063716in}{1.795318in}}%
\pgfpathlineto{\pgfqpoint{5.064869in}{1.791567in}}%
\pgfpathlineto{\pgfqpoint{5.065061in}{1.791927in}}%
\pgfpathlineto{\pgfqpoint{5.065445in}{1.799161in}}%
\pgfpathlineto{\pgfqpoint{5.066214in}{1.794667in}}%
\pgfpathlineto{\pgfqpoint{5.066790in}{1.798580in}}%
\pgfpathlineto{\pgfqpoint{5.066983in}{1.797568in}}%
\pgfpathlineto{\pgfqpoint{5.067559in}{1.790831in}}%
\pgfpathlineto{\pgfqpoint{5.068136in}{1.795667in}}%
\pgfpathlineto{\pgfqpoint{5.069673in}{1.788877in}}%
\pgfpathlineto{\pgfqpoint{5.069865in}{1.789179in}}%
\pgfpathlineto{\pgfqpoint{5.071018in}{1.793342in}}%
\pgfpathlineto{\pgfqpoint{5.070634in}{1.787289in}}%
\pgfpathlineto{\pgfqpoint{5.071210in}{1.792854in}}%
\pgfpathlineto{\pgfqpoint{5.072555in}{1.788099in}}%
\pgfpathlineto{\pgfqpoint{5.072171in}{1.793521in}}%
\pgfpathlineto{\pgfqpoint{5.072940in}{1.788790in}}%
\pgfpathlineto{\pgfqpoint{5.073708in}{1.797973in}}%
\pgfpathlineto{\pgfqpoint{5.074477in}{1.797533in}}%
\pgfpathlineto{\pgfqpoint{5.074669in}{1.795496in}}%
\pgfpathlineto{\pgfqpoint{5.075054in}{1.798157in}}%
\pgfpathlineto{\pgfqpoint{5.075246in}{1.798064in}}%
\pgfpathlineto{\pgfqpoint{5.075438in}{1.802925in}}%
\pgfpathlineto{\pgfqpoint{5.076399in}{1.800383in}}%
\pgfpathlineto{\pgfqpoint{5.076975in}{1.798571in}}%
\pgfpathlineto{\pgfqpoint{5.077744in}{1.805679in}}%
\pgfpathlineto{\pgfqpoint{5.079089in}{1.801934in}}%
\pgfpathlineto{\pgfqpoint{5.079281in}{1.805743in}}%
\pgfpathlineto{\pgfqpoint{5.080050in}{1.799514in}}%
\pgfpathlineto{\pgfqpoint{5.081011in}{1.802444in}}%
\pgfpathlineto{\pgfqpoint{5.080819in}{1.798584in}}%
\pgfpathlineto{\pgfqpoint{5.081203in}{1.800637in}}%
\pgfpathlineto{\pgfqpoint{5.081587in}{1.800858in}}%
\pgfpathlineto{\pgfqpoint{5.082164in}{1.795994in}}%
\pgfpathlineto{\pgfqpoint{5.083509in}{1.790509in}}%
\pgfpathlineto{\pgfqpoint{5.083701in}{1.791867in}}%
\pgfpathlineto{\pgfqpoint{5.084085in}{1.790129in}}%
\pgfpathlineto{\pgfqpoint{5.084278in}{1.791134in}}%
\pgfpathlineto{\pgfqpoint{5.085238in}{1.785645in}}%
\pgfpathlineto{\pgfqpoint{5.085623in}{1.788517in}}%
\pgfpathlineto{\pgfqpoint{5.085815in}{1.788693in}}%
\pgfpathlineto{\pgfqpoint{5.086199in}{1.787989in}}%
\pgfpathlineto{\pgfqpoint{5.087352in}{1.792696in}}%
\pgfpathlineto{\pgfqpoint{5.088505in}{1.779183in}}%
\pgfpathlineto{\pgfqpoint{5.089274in}{1.779387in}}%
\pgfpathlineto{\pgfqpoint{5.089850in}{1.780392in}}%
\pgfpathlineto{\pgfqpoint{5.090043in}{1.778871in}}%
\pgfpathlineto{\pgfqpoint{5.090235in}{1.778172in}}%
\pgfpathlineto{\pgfqpoint{5.090427in}{1.781189in}}%
\pgfpathlineto{\pgfqpoint{5.090811in}{1.781852in}}%
\pgfpathlineto{\pgfqpoint{5.091772in}{1.786664in}}%
\pgfpathlineto{\pgfqpoint{5.092157in}{1.783530in}}%
\pgfpathlineto{\pgfqpoint{5.092541in}{1.786397in}}%
\pgfpathlineto{\pgfqpoint{5.093117in}{1.782852in}}%
\pgfpathlineto{\pgfqpoint{5.093502in}{1.785362in}}%
\pgfpathlineto{\pgfqpoint{5.093694in}{1.786932in}}%
\pgfpathlineto{\pgfqpoint{5.094078in}{1.783936in}}%
\pgfpathlineto{\pgfqpoint{5.094270in}{1.784652in}}%
\pgfpathlineto{\pgfqpoint{5.095039in}{1.778189in}}%
\pgfpathlineto{\pgfqpoint{5.095231in}{1.781954in}}%
\pgfpathlineto{\pgfqpoint{5.096384in}{1.786165in}}%
\pgfpathlineto{\pgfqpoint{5.096576in}{1.786188in}}%
\pgfpathlineto{\pgfqpoint{5.097922in}{1.792331in}}%
\pgfpathlineto{\pgfqpoint{5.098306in}{1.790754in}}%
\pgfpathlineto{\pgfqpoint{5.100420in}{1.782567in}}%
\pgfpathlineto{\pgfqpoint{5.100612in}{1.783496in}}%
\pgfpathlineto{\pgfqpoint{5.101381in}{1.785618in}}%
\pgfpathlineto{\pgfqpoint{5.101188in}{1.781179in}}%
\pgfpathlineto{\pgfqpoint{5.101573in}{1.783721in}}%
\pgfpathlineto{\pgfqpoint{5.101957in}{1.780787in}}%
\pgfpathlineto{\pgfqpoint{5.102534in}{1.783621in}}%
\pgfpathlineto{\pgfqpoint{5.103110in}{1.788120in}}%
\pgfpathlineto{\pgfqpoint{5.103687in}{1.786629in}}%
\pgfpathlineto{\pgfqpoint{5.104263in}{1.787576in}}%
\pgfpathlineto{\pgfqpoint{5.104455in}{1.783662in}}%
\pgfpathlineto{\pgfqpoint{5.104647in}{1.781464in}}%
\pgfpathlineto{\pgfqpoint{5.105032in}{1.786378in}}%
\pgfpathlineto{\pgfqpoint{5.105416in}{1.782977in}}%
\pgfpathlineto{\pgfqpoint{5.106569in}{1.788567in}}%
\pgfpathlineto{\pgfqpoint{5.106761in}{1.787355in}}%
\pgfpathlineto{\pgfqpoint{5.106953in}{1.785725in}}%
\pgfpathlineto{\pgfqpoint{5.107338in}{1.787800in}}%
\pgfpathlineto{\pgfqpoint{5.107914in}{1.785942in}}%
\pgfpathlineto{\pgfqpoint{5.108299in}{1.785499in}}%
\pgfpathlineto{\pgfqpoint{5.109259in}{1.788490in}}%
\pgfpathlineto{\pgfqpoint{5.109452in}{1.788151in}}%
\pgfpathlineto{\pgfqpoint{5.109644in}{1.789790in}}%
\pgfpathlineto{\pgfqpoint{5.109836in}{1.789369in}}%
\pgfpathlineto{\pgfqpoint{5.110412in}{1.788819in}}%
\pgfpathlineto{\pgfqpoint{5.111758in}{1.795368in}}%
\pgfpathlineto{\pgfqpoint{5.112718in}{1.788566in}}%
\pgfpathlineto{\pgfqpoint{5.113103in}{1.790101in}}%
\pgfpathlineto{\pgfqpoint{5.113487in}{1.791168in}}%
\pgfpathlineto{\pgfqpoint{5.113871in}{1.789820in}}%
\pgfpathlineto{\pgfqpoint{5.115409in}{1.783134in}}%
\pgfpathlineto{\pgfqpoint{5.115793in}{1.784409in}}%
\pgfpathlineto{\pgfqpoint{5.116370in}{1.790118in}}%
\pgfpathlineto{\pgfqpoint{5.116562in}{1.786739in}}%
\pgfpathlineto{\pgfqpoint{5.117523in}{1.777397in}}%
\pgfpathlineto{\pgfqpoint{5.117907in}{1.778335in}}%
\pgfpathlineto{\pgfqpoint{5.121750in}{1.756552in}}%
\pgfpathlineto{\pgfqpoint{5.121943in}{1.759614in}}%
\pgfpathlineto{\pgfqpoint{5.122135in}{1.761890in}}%
\pgfpathlineto{\pgfqpoint{5.122711in}{1.758012in}}%
\pgfpathlineto{\pgfqpoint{5.122903in}{1.758212in}}%
\pgfpathlineto{\pgfqpoint{5.123096in}{1.758386in}}%
\pgfpathlineto{\pgfqpoint{5.124633in}{1.767797in}}%
\pgfpathlineto{\pgfqpoint{5.126170in}{1.760842in}}%
\pgfpathlineto{\pgfqpoint{5.126362in}{1.761209in}}%
\pgfpathlineto{\pgfqpoint{5.126555in}{1.764403in}}%
\pgfpathlineto{\pgfqpoint{5.127323in}{1.758858in}}%
\pgfpathlineto{\pgfqpoint{5.127900in}{1.757555in}}%
\pgfpathlineto{\pgfqpoint{5.128476in}{1.753482in}}%
\pgfpathlineto{\pgfqpoint{5.129053in}{1.756227in}}%
\pgfpathlineto{\pgfqpoint{5.129245in}{1.757277in}}%
\pgfpathlineto{\pgfqpoint{5.129629in}{1.753242in}}%
\pgfpathlineto{\pgfqpoint{5.130206in}{1.749997in}}%
\pgfpathlineto{\pgfqpoint{5.131359in}{1.745607in}}%
\pgfpathlineto{\pgfqpoint{5.131551in}{1.747450in}}%
\pgfpathlineto{\pgfqpoint{5.132127in}{1.751992in}}%
\pgfpathlineto{\pgfqpoint{5.131935in}{1.747110in}}%
\pgfpathlineto{\pgfqpoint{5.132512in}{1.748382in}}%
\pgfpathlineto{\pgfqpoint{5.132896in}{1.744921in}}%
\pgfpathlineto{\pgfqpoint{5.133473in}{1.746857in}}%
\pgfpathlineto{\pgfqpoint{5.134241in}{1.755548in}}%
\pgfpathlineto{\pgfqpoint{5.135394in}{1.754930in}}%
\pgfpathlineto{\pgfqpoint{5.137124in}{1.748857in}}%
\pgfpathlineto{\pgfqpoint{5.137892in}{1.744756in}}%
\pgfpathlineto{\pgfqpoint{5.139238in}{1.738598in}}%
\pgfpathlineto{\pgfqpoint{5.139430in}{1.739746in}}%
\pgfpathlineto{\pgfqpoint{5.141159in}{1.750566in}}%
\pgfpathlineto{\pgfqpoint{5.141928in}{1.749978in}}%
\pgfpathlineto{\pgfqpoint{5.143273in}{1.744515in}}%
\pgfpathlineto{\pgfqpoint{5.143658in}{1.747082in}}%
\pgfpathlineto{\pgfqpoint{5.144618in}{1.751709in}}%
\pgfpathlineto{\pgfqpoint{5.145003in}{1.751471in}}%
\pgfpathlineto{\pgfqpoint{5.146540in}{1.745907in}}%
\pgfpathlineto{\pgfqpoint{5.148077in}{1.740028in}}%
\pgfpathlineto{\pgfqpoint{5.148462in}{1.744547in}}%
\pgfpathlineto{\pgfqpoint{5.149038in}{1.738928in}}%
\pgfpathlineto{\pgfqpoint{5.149423in}{1.740696in}}%
\pgfpathlineto{\pgfqpoint{5.149999in}{1.739201in}}%
\pgfpathlineto{\pgfqpoint{5.150191in}{1.737700in}}%
\pgfpathlineto{\pgfqpoint{5.150768in}{1.740912in}}%
\pgfpathlineto{\pgfqpoint{5.150960in}{1.742365in}}%
\pgfpathlineto{\pgfqpoint{5.151536in}{1.739469in}}%
\pgfpathlineto{\pgfqpoint{5.152882in}{1.728673in}}%
\pgfpathlineto{\pgfqpoint{5.153074in}{1.729021in}}%
\pgfpathlineto{\pgfqpoint{5.154419in}{1.721236in}}%
\pgfpathlineto{\pgfqpoint{5.154803in}{1.725967in}}%
\pgfpathlineto{\pgfqpoint{5.155572in}{1.731085in}}%
\pgfpathlineto{\pgfqpoint{5.156148in}{1.727445in}}%
\pgfpathlineto{\pgfqpoint{5.156341in}{1.724597in}}%
\pgfpathlineto{\pgfqpoint{5.157109in}{1.728717in}}%
\pgfpathlineto{\pgfqpoint{5.161721in}{1.697112in}}%
\pgfpathlineto{\pgfqpoint{5.162874in}{1.698976in}}%
\pgfpathlineto{\pgfqpoint{5.164220in}{1.705027in}}%
\pgfpathlineto{\pgfqpoint{5.164796in}{1.702450in}}%
\pgfpathlineto{\pgfqpoint{5.164988in}{1.705671in}}%
\pgfpathlineto{\pgfqpoint{5.165949in}{1.705150in}}%
\pgfpathlineto{\pgfqpoint{5.166141in}{1.707618in}}%
\pgfpathlineto{\pgfqpoint{5.166526in}{1.705864in}}%
\pgfpathlineto{\pgfqpoint{5.167102in}{1.708717in}}%
\pgfpathlineto{\pgfqpoint{5.171138in}{1.695256in}}%
\pgfpathlineto{\pgfqpoint{5.171714in}{1.698192in}}%
\pgfpathlineto{\pgfqpoint{5.172098in}{1.695886in}}%
\pgfpathlineto{\pgfqpoint{5.172291in}{1.695114in}}%
\pgfpathlineto{\pgfqpoint{5.172483in}{1.697063in}}%
\pgfpathlineto{\pgfqpoint{5.172675in}{1.696665in}}%
\pgfpathlineto{\pgfqpoint{5.173059in}{1.700511in}}%
\pgfpathlineto{\pgfqpoint{5.173444in}{1.694135in}}%
\pgfpathlineto{\pgfqpoint{5.173636in}{1.695440in}}%
\pgfpathlineto{\pgfqpoint{5.175173in}{1.706104in}}%
\pgfpathlineto{\pgfqpoint{5.176134in}{1.705112in}}%
\pgfpathlineto{\pgfqpoint{5.176903in}{1.700423in}}%
\pgfpathlineto{\pgfqpoint{5.177095in}{1.703266in}}%
\pgfpathlineto{\pgfqpoint{5.177671in}{1.708163in}}%
\pgfpathlineto{\pgfqpoint{5.178440in}{1.706184in}}%
\pgfpathlineto{\pgfqpoint{5.179016in}{1.702595in}}%
\pgfpathlineto{\pgfqpoint{5.179593in}{1.704636in}}%
\pgfpathlineto{\pgfqpoint{5.182283in}{1.723498in}}%
\pgfpathlineto{\pgfqpoint{5.183052in}{1.722273in}}%
\pgfpathlineto{\pgfqpoint{5.183821in}{1.717124in}}%
\pgfpathlineto{\pgfqpoint{5.184397in}{1.717616in}}%
\pgfpathlineto{\pgfqpoint{5.185934in}{1.724598in}}%
\pgfpathlineto{\pgfqpoint{5.186895in}{1.732004in}}%
\pgfpathlineto{\pgfqpoint{5.187472in}{1.738453in}}%
\pgfpathlineto{\pgfqpoint{5.188240in}{1.736371in}}%
\pgfpathlineto{\pgfqpoint{5.189009in}{1.737617in}}%
\pgfpathlineto{\pgfqpoint{5.188625in}{1.734715in}}%
\pgfpathlineto{\pgfqpoint{5.189201in}{1.735741in}}%
\pgfpathlineto{\pgfqpoint{5.189394in}{1.734528in}}%
\pgfpathlineto{\pgfqpoint{5.189778in}{1.738476in}}%
\pgfpathlineto{\pgfqpoint{5.190931in}{1.748688in}}%
\pgfpathlineto{\pgfqpoint{5.191123in}{1.746659in}}%
\pgfpathlineto{\pgfqpoint{5.193045in}{1.741547in}}%
\pgfpathlineto{\pgfqpoint{5.193813in}{1.743648in}}%
\pgfpathlineto{\pgfqpoint{5.193429in}{1.740659in}}%
\pgfpathlineto{\pgfqpoint{5.194006in}{1.741304in}}%
\pgfpathlineto{\pgfqpoint{5.196119in}{1.732265in}}%
\pgfpathlineto{\pgfqpoint{5.196312in}{1.733984in}}%
\pgfpathlineto{\pgfqpoint{5.197657in}{1.742530in}}%
\pgfpathlineto{\pgfqpoint{5.198041in}{1.738986in}}%
\pgfpathlineto{\pgfqpoint{5.198618in}{1.733721in}}%
\pgfpathlineto{\pgfqpoint{5.199002in}{1.740526in}}%
\pgfpathlineto{\pgfqpoint{5.200924in}{1.744946in}}%
\pgfpathlineto{\pgfqpoint{5.202461in}{1.730210in}}%
\pgfpathlineto{\pgfqpoint{5.203037in}{1.733268in}}%
\pgfpathlineto{\pgfqpoint{5.203422in}{1.735345in}}%
\pgfpathlineto{\pgfqpoint{5.203614in}{1.731725in}}%
\pgfpathlineto{\pgfqpoint{5.204383in}{1.730171in}}%
\pgfpathlineto{\pgfqpoint{5.204575in}{1.730645in}}%
\pgfpathlineto{\pgfqpoint{5.205343in}{1.732171in}}%
\pgfpathlineto{\pgfqpoint{5.205536in}{1.729987in}}%
\pgfpathlineto{\pgfqpoint{5.205920in}{1.724584in}}%
\pgfpathlineto{\pgfqpoint{5.206689in}{1.727736in}}%
\pgfpathlineto{\pgfqpoint{5.206881in}{1.727922in}}%
\pgfpathlineto{\pgfqpoint{5.207073in}{1.727575in}}%
\pgfpathlineto{\pgfqpoint{5.207457in}{1.727985in}}%
\pgfpathlineto{\pgfqpoint{5.209571in}{1.720018in}}%
\pgfpathlineto{\pgfqpoint{5.209763in}{1.722123in}}%
\pgfpathlineto{\pgfqpoint{5.210148in}{1.715663in}}%
\pgfpathlineto{\pgfqpoint{5.210916in}{1.712104in}}%
\pgfpathlineto{\pgfqpoint{5.211108in}{1.714759in}}%
\pgfpathlineto{\pgfqpoint{5.212261in}{1.719334in}}%
\pgfpathlineto{\pgfqpoint{5.213799in}{1.710838in}}%
\pgfpathlineto{\pgfqpoint{5.215336in}{1.716175in}}%
\pgfpathlineto{\pgfqpoint{5.215528in}{1.715794in}}%
\pgfpathlineto{\pgfqpoint{5.215913in}{1.715319in}}%
\pgfpathlineto{\pgfqpoint{5.216297in}{1.718917in}}%
\pgfpathlineto{\pgfqpoint{5.216874in}{1.722601in}}%
\pgfpathlineto{\pgfqpoint{5.217258in}{1.718253in}}%
\pgfpathlineto{\pgfqpoint{5.217450in}{1.717916in}}%
\pgfpathlineto{\pgfqpoint{5.217642in}{1.718441in}}%
\pgfpathlineto{\pgfqpoint{5.218987in}{1.707961in}}%
\pgfpathlineto{\pgfqpoint{5.220333in}{1.713427in}}%
\pgfpathlineto{\pgfqpoint{5.221486in}{1.711477in}}%
\pgfpathlineto{\pgfqpoint{5.223407in}{1.722491in}}%
\pgfpathlineto{\pgfqpoint{5.224945in}{1.709677in}}%
\pgfpathlineto{\pgfqpoint{5.225137in}{1.710240in}}%
\pgfpathlineto{\pgfqpoint{5.225521in}{1.709052in}}%
\pgfpathlineto{\pgfqpoint{5.228019in}{1.694597in}}%
\pgfpathlineto{\pgfqpoint{5.228596in}{1.697236in}}%
\pgfpathlineto{\pgfqpoint{5.228980in}{1.694508in}}%
\pgfpathlineto{\pgfqpoint{5.229557in}{1.692275in}}%
\pgfpathlineto{\pgfqpoint{5.229749in}{1.693573in}}%
\pgfpathlineto{\pgfqpoint{5.230133in}{1.697009in}}%
\pgfpathlineto{\pgfqpoint{5.230710in}{1.693012in}}%
\pgfpathlineto{\pgfqpoint{5.231094in}{1.693639in}}%
\pgfpathlineto{\pgfqpoint{5.231478in}{1.691869in}}%
\pgfpathlineto{\pgfqpoint{5.231670in}{1.689095in}}%
\pgfpathlineto{\pgfqpoint{5.232247in}{1.692227in}}%
\pgfpathlineto{\pgfqpoint{5.232631in}{1.691029in}}%
\pgfpathlineto{\pgfqpoint{5.232823in}{1.691235in}}%
\pgfpathlineto{\pgfqpoint{5.234553in}{1.680523in}}%
\pgfpathlineto{\pgfqpoint{5.236090in}{1.671109in}}%
\pgfpathlineto{\pgfqpoint{5.234937in}{1.680619in}}%
\pgfpathlineto{\pgfqpoint{5.236667in}{1.671568in}}%
\pgfpathlineto{\pgfqpoint{5.237435in}{1.681204in}}%
\pgfpathlineto{\pgfqpoint{5.237820in}{1.675235in}}%
\pgfpathlineto{\pgfqpoint{5.238204in}{1.673069in}}%
\pgfpathlineto{\pgfqpoint{5.238781in}{1.674615in}}%
\pgfpathlineto{\pgfqpoint{5.238973in}{1.676275in}}%
\pgfpathlineto{\pgfqpoint{5.239742in}{1.675455in}}%
\pgfpathlineto{\pgfqpoint{5.241087in}{1.669372in}}%
\pgfpathlineto{\pgfqpoint{5.241279in}{1.671824in}}%
\pgfpathlineto{\pgfqpoint{5.241855in}{1.666703in}}%
\pgfpathlineto{\pgfqpoint{5.242048in}{1.664654in}}%
\pgfpathlineto{\pgfqpoint{5.242624in}{1.668709in}}%
\pgfpathlineto{\pgfqpoint{5.243201in}{1.673783in}}%
\pgfpathlineto{\pgfqpoint{5.243777in}{1.670073in}}%
\pgfpathlineto{\pgfqpoint{5.243969in}{1.669345in}}%
\pgfpathlineto{\pgfqpoint{5.244354in}{1.670714in}}%
\pgfpathlineto{\pgfqpoint{5.244738in}{1.674728in}}%
\pgfpathlineto{\pgfqpoint{5.244930in}{1.671759in}}%
\pgfpathlineto{\pgfqpoint{5.246083in}{1.660907in}}%
\pgfpathlineto{\pgfqpoint{5.246275in}{1.662132in}}%
\pgfpathlineto{\pgfqpoint{5.247044in}{1.666157in}}%
\pgfpathlineto{\pgfqpoint{5.247620in}{1.664262in}}%
\pgfpathlineto{\pgfqpoint{5.250311in}{1.651793in}}%
\pgfpathlineto{\pgfqpoint{5.248005in}{1.666414in}}%
\pgfpathlineto{\pgfqpoint{5.250695in}{1.651960in}}%
\pgfpathlineto{\pgfqpoint{5.251464in}{1.657689in}}%
\pgfpathlineto{\pgfqpoint{5.251656in}{1.654723in}}%
\pgfpathlineto{\pgfqpoint{5.252809in}{1.649963in}}%
\pgfpathlineto{\pgfqpoint{5.253001in}{1.650908in}}%
\pgfpathlineto{\pgfqpoint{5.253578in}{1.648226in}}%
\pgfpathlineto{\pgfqpoint{5.254154in}{1.650687in}}%
\pgfpathlineto{\pgfqpoint{5.254731in}{1.653594in}}%
\pgfpathlineto{\pgfqpoint{5.254538in}{1.649236in}}%
\pgfpathlineto{\pgfqpoint{5.254923in}{1.652946in}}%
\pgfpathlineto{\pgfqpoint{5.256268in}{1.659666in}}%
\pgfpathlineto{\pgfqpoint{5.258382in}{1.648531in}}%
\pgfpathlineto{\pgfqpoint{5.258766in}{1.651435in}}%
\pgfpathlineto{\pgfqpoint{5.258958in}{1.649576in}}%
\pgfpathlineto{\pgfqpoint{5.259535in}{1.643762in}}%
\pgfpathlineto{\pgfqpoint{5.260303in}{1.644198in}}%
\pgfpathlineto{\pgfqpoint{5.261072in}{1.645271in}}%
\pgfpathlineto{\pgfqpoint{5.261264in}{1.643120in}}%
\pgfpathlineto{\pgfqpoint{5.261841in}{1.645981in}}%
\pgfpathlineto{\pgfqpoint{5.262033in}{1.643856in}}%
\pgfpathlineto{\pgfqpoint{5.262417in}{1.637079in}}%
\pgfpathlineto{\pgfqpoint{5.263378in}{1.637269in}}%
\pgfpathlineto{\pgfqpoint{5.263955in}{1.637917in}}%
\pgfpathlineto{\pgfqpoint{5.264723in}{1.633734in}}%
\pgfpathlineto{\pgfqpoint{5.264916in}{1.634972in}}%
\pgfpathlineto{\pgfqpoint{5.265492in}{1.632003in}}%
\pgfpathlineto{\pgfqpoint{5.265876in}{1.634592in}}%
\pgfpathlineto{\pgfqpoint{5.267414in}{1.629073in}}%
\pgfpathlineto{\pgfqpoint{5.267606in}{1.634287in}}%
\pgfpathlineto{\pgfqpoint{5.268567in}{1.631841in}}%
\pgfpathlineto{\pgfqpoint{5.269143in}{1.634529in}}%
\pgfpathlineto{\pgfqpoint{5.269528in}{1.632018in}}%
\pgfpathlineto{\pgfqpoint{5.270104in}{1.629024in}}%
\pgfpathlineto{\pgfqpoint{5.270296in}{1.632475in}}%
\pgfpathlineto{\pgfqpoint{5.270488in}{1.631980in}}%
\pgfpathlineto{\pgfqpoint{5.271257in}{1.642187in}}%
\pgfpathlineto{\pgfqpoint{5.271834in}{1.637978in}}%
\pgfpathlineto{\pgfqpoint{5.272026in}{1.633432in}}%
\pgfpathlineto{\pgfqpoint{5.272987in}{1.633932in}}%
\pgfpathlineto{\pgfqpoint{5.273371in}{1.629833in}}%
\pgfpathlineto{\pgfqpoint{5.274332in}{1.631618in}}%
\pgfpathlineto{\pgfqpoint{5.274524in}{1.631617in}}%
\pgfpathlineto{\pgfqpoint{5.276061in}{1.640525in}}%
\pgfpathlineto{\pgfqpoint{5.276253in}{1.639049in}}%
\pgfpathlineto{\pgfqpoint{5.276638in}{1.642348in}}%
\pgfpathlineto{\pgfqpoint{5.277022in}{1.640441in}}%
\pgfpathlineto{\pgfqpoint{5.277214in}{1.640455in}}%
\pgfpathlineto{\pgfqpoint{5.277983in}{1.642572in}}%
\pgfpathlineto{\pgfqpoint{5.277791in}{1.639890in}}%
\pgfpathlineto{\pgfqpoint{5.278175in}{1.640749in}}%
\pgfpathlineto{\pgfqpoint{5.278944in}{1.635185in}}%
\pgfpathlineto{\pgfqpoint{5.279328in}{1.637835in}}%
\pgfpathlineto{\pgfqpoint{5.281058in}{1.632845in}}%
\pgfpathlineto{\pgfqpoint{5.281250in}{1.633887in}}%
\pgfpathlineto{\pgfqpoint{5.282403in}{1.637845in}}%
\pgfpathlineto{\pgfqpoint{5.282595in}{1.637524in}}%
\pgfpathlineto{\pgfqpoint{5.282979in}{1.633489in}}%
\pgfpathlineto{\pgfqpoint{5.283364in}{1.638151in}}%
\pgfpathlineto{\pgfqpoint{5.283556in}{1.638054in}}%
\pgfpathlineto{\pgfqpoint{5.284324in}{1.637592in}}%
\pgfpathlineto{\pgfqpoint{5.285093in}{1.648358in}}%
\pgfpathlineto{\pgfqpoint{5.285285in}{1.647567in}}%
\pgfpathlineto{\pgfqpoint{5.285862in}{1.648935in}}%
\pgfpathlineto{\pgfqpoint{5.286054in}{1.648634in}}%
\pgfpathlineto{\pgfqpoint{5.286246in}{1.649301in}}%
\pgfpathlineto{\pgfqpoint{5.287399in}{1.635885in}}%
\pgfpathlineto{\pgfqpoint{5.287784in}{1.638149in}}%
\pgfpathlineto{\pgfqpoint{5.287976in}{1.636303in}}%
\pgfpathlineto{\pgfqpoint{5.288168in}{1.638959in}}%
\pgfpathlineto{\pgfqpoint{5.288552in}{1.637085in}}%
\pgfpathlineto{\pgfqpoint{5.288937in}{1.640796in}}%
\pgfpathlineto{\pgfqpoint{5.289705in}{1.640521in}}%
\pgfpathlineto{\pgfqpoint{5.289897in}{1.639629in}}%
\pgfpathlineto{\pgfqpoint{5.290282in}{1.642397in}}%
\pgfpathlineto{\pgfqpoint{5.290474in}{1.640144in}}%
\pgfpathlineto{\pgfqpoint{5.290666in}{1.642658in}}%
\pgfpathlineto{\pgfqpoint{5.291243in}{1.637794in}}%
\pgfpathlineto{\pgfqpoint{5.291627in}{1.640800in}}%
\pgfpathlineto{\pgfqpoint{5.292203in}{1.644319in}}%
\pgfpathlineto{\pgfqpoint{5.293164in}{1.649424in}}%
\pgfpathlineto{\pgfqpoint{5.293549in}{1.648917in}}%
\pgfpathlineto{\pgfqpoint{5.294125in}{1.644079in}}%
\pgfpathlineto{\pgfqpoint{5.295086in}{1.645800in}}%
\pgfpathlineto{\pgfqpoint{5.295470in}{1.648883in}}%
\pgfpathlineto{\pgfqpoint{5.295855in}{1.643260in}}%
\pgfpathlineto{\pgfqpoint{5.296047in}{1.642712in}}%
\pgfpathlineto{\pgfqpoint{5.296239in}{1.644909in}}%
\pgfpathlineto{\pgfqpoint{5.296431in}{1.644738in}}%
\pgfpathlineto{\pgfqpoint{5.297200in}{1.644493in}}%
\pgfpathlineto{\pgfqpoint{5.297776in}{1.648052in}}%
\pgfpathlineto{\pgfqpoint{5.297968in}{1.647403in}}%
\pgfpathlineto{\pgfqpoint{5.298161in}{1.648293in}}%
\pgfpathlineto{\pgfqpoint{5.298929in}{1.657984in}}%
\pgfpathlineto{\pgfqpoint{5.299506in}{1.655951in}}%
\pgfpathlineto{\pgfqpoint{5.299890in}{1.653329in}}%
\pgfpathlineto{\pgfqpoint{5.300274in}{1.657270in}}%
\pgfpathlineto{\pgfqpoint{5.301427in}{1.662595in}}%
\pgfpathlineto{\pgfqpoint{5.300851in}{1.655034in}}%
\pgfpathlineto{\pgfqpoint{5.302196in}{1.662248in}}%
\pgfpathlineto{\pgfqpoint{5.303349in}{1.653400in}}%
\pgfpathlineto{\pgfqpoint{5.303733in}{1.654929in}}%
\pgfpathlineto{\pgfqpoint{5.303926in}{1.657918in}}%
\pgfpathlineto{\pgfqpoint{5.303926in}{1.657918in}}%
\pgfusepath{stroke}%
\end{pgfscope}%
\begin{pgfscope}%
\pgfpathrectangle{\pgfqpoint{3.286364in}{0.660000in}}{\pgfqpoint{2.113636in}{2.100000in}}%
\pgfusepath{clip}%
\pgfsetroundcap%
\pgfsetroundjoin%
\pgfsetlinewidth{0.602250pt}%
\definecolor{currentstroke}{rgb}{0.596078,0.305882,0.639216}%
\pgfsetstrokecolor{currentstroke}%
\pgfsetdash{}{0pt}%
\pgfpathmoveto{\pgfqpoint{3.382438in}{1.789806in}}%
\pgfpathlineto{\pgfqpoint{3.383015in}{1.790467in}}%
\pgfpathlineto{\pgfqpoint{3.382822in}{1.788345in}}%
\pgfpathlineto{\pgfqpoint{3.383399in}{1.789432in}}%
\pgfpathlineto{\pgfqpoint{3.383591in}{1.787203in}}%
\pgfpathlineto{\pgfqpoint{3.384168in}{1.791903in}}%
\pgfpathlineto{\pgfqpoint{3.384360in}{1.791217in}}%
\pgfpathlineto{\pgfqpoint{3.384936in}{1.787227in}}%
\pgfpathlineto{\pgfqpoint{3.385513in}{1.787723in}}%
\pgfpathlineto{\pgfqpoint{3.385705in}{1.790019in}}%
\pgfpathlineto{\pgfqpoint{3.386474in}{1.788715in}}%
\pgfpathlineto{\pgfqpoint{3.387434in}{1.786613in}}%
\pgfpathlineto{\pgfqpoint{3.387627in}{1.789351in}}%
\pgfpathlineto{\pgfqpoint{3.388395in}{1.787244in}}%
\pgfpathlineto{\pgfqpoint{3.389164in}{1.780283in}}%
\pgfpathlineto{\pgfqpoint{3.389933in}{1.781680in}}%
\pgfpathlineto{\pgfqpoint{3.390317in}{1.783370in}}%
\pgfpathlineto{\pgfqpoint{3.390509in}{1.780196in}}%
\pgfpathlineto{\pgfqpoint{3.391086in}{1.784153in}}%
\pgfpathlineto{\pgfqpoint{3.392239in}{1.791173in}}%
\pgfpathlineto{\pgfqpoint{3.392623in}{1.790419in}}%
\pgfpathlineto{\pgfqpoint{3.393584in}{1.786936in}}%
\pgfpathlineto{\pgfqpoint{3.393776in}{1.788582in}}%
\pgfpathlineto{\pgfqpoint{3.395121in}{1.792298in}}%
\pgfpathlineto{\pgfqpoint{3.395890in}{1.790022in}}%
\pgfpathlineto{\pgfqpoint{3.396082in}{1.793125in}}%
\pgfpathlineto{\pgfqpoint{3.396851in}{1.796589in}}%
\pgfpathlineto{\pgfqpoint{3.397043in}{1.794011in}}%
\pgfpathlineto{\pgfqpoint{3.398580in}{1.785386in}}%
\pgfpathlineto{\pgfqpoint{3.398964in}{1.784086in}}%
\pgfpathlineto{\pgfqpoint{3.400117in}{1.771039in}}%
\pgfpathlineto{\pgfqpoint{3.400502in}{1.777655in}}%
\pgfpathlineto{\pgfqpoint{3.401078in}{1.770593in}}%
\pgfpathlineto{\pgfqpoint{3.401270in}{1.768984in}}%
\pgfpathlineto{\pgfqpoint{3.401655in}{1.771304in}}%
\pgfpathlineto{\pgfqpoint{3.401847in}{1.776480in}}%
\pgfpathlineto{\pgfqpoint{3.402808in}{1.773311in}}%
\pgfpathlineto{\pgfqpoint{3.403384in}{1.767680in}}%
\pgfpathlineto{\pgfqpoint{3.404537in}{1.769910in}}%
\pgfpathlineto{\pgfqpoint{3.404730in}{1.770862in}}%
\pgfpathlineto{\pgfqpoint{3.404922in}{1.767356in}}%
\pgfpathlineto{\pgfqpoint{3.406459in}{1.749610in}}%
\pgfpathlineto{\pgfqpoint{3.407036in}{1.755467in}}%
\pgfpathlineto{\pgfqpoint{3.408765in}{1.747257in}}%
\pgfpathlineto{\pgfqpoint{3.408957in}{1.748486in}}%
\pgfpathlineto{\pgfqpoint{3.409534in}{1.745497in}}%
\pgfpathlineto{\pgfqpoint{3.409726in}{1.743511in}}%
\pgfpathlineto{\pgfqpoint{3.410302in}{1.746723in}}%
\pgfpathlineto{\pgfqpoint{3.410495in}{1.745242in}}%
\pgfpathlineto{\pgfqpoint{3.411648in}{1.757827in}}%
\pgfpathlineto{\pgfqpoint{3.412224in}{1.756085in}}%
\pgfpathlineto{\pgfqpoint{3.412993in}{1.752302in}}%
\pgfpathlineto{\pgfqpoint{3.412608in}{1.758839in}}%
\pgfpathlineto{\pgfqpoint{3.413377in}{1.754081in}}%
\pgfpathlineto{\pgfqpoint{3.414530in}{1.760359in}}%
\pgfpathlineto{\pgfqpoint{3.415299in}{1.756978in}}%
\pgfpathlineto{\pgfqpoint{3.415875in}{1.757285in}}%
\pgfpathlineto{\pgfqpoint{3.416260in}{1.754735in}}%
\pgfpathlineto{\pgfqpoint{3.416452in}{1.751474in}}%
\pgfpathlineto{\pgfqpoint{3.417220in}{1.755134in}}%
\pgfpathlineto{\pgfqpoint{3.417797in}{1.759495in}}%
\pgfpathlineto{\pgfqpoint{3.420295in}{1.771022in}}%
\pgfpathlineto{\pgfqpoint{3.421064in}{1.775240in}}%
\pgfpathlineto{\pgfqpoint{3.421832in}{1.773545in}}%
\pgfpathlineto{\pgfqpoint{3.422409in}{1.774263in}}%
\pgfpathlineto{\pgfqpoint{3.422793in}{1.768795in}}%
\pgfpathlineto{\pgfqpoint{3.423370in}{1.774190in}}%
\pgfpathlineto{\pgfqpoint{3.425291in}{1.783942in}}%
\pgfpathlineto{\pgfqpoint{3.425484in}{1.785971in}}%
\pgfpathlineto{\pgfqpoint{3.426444in}{1.785347in}}%
\pgfpathlineto{\pgfqpoint{3.428174in}{1.779972in}}%
\pgfpathlineto{\pgfqpoint{3.429327in}{1.767757in}}%
\pgfpathlineto{\pgfqpoint{3.429904in}{1.771404in}}%
\pgfpathlineto{\pgfqpoint{3.430288in}{1.771150in}}%
\pgfpathlineto{\pgfqpoint{3.431057in}{1.773578in}}%
\pgfpathlineto{\pgfqpoint{3.431249in}{1.773767in}}%
\pgfpathlineto{\pgfqpoint{3.431441in}{1.776245in}}%
\pgfpathlineto{\pgfqpoint{3.432017in}{1.769858in}}%
\pgfpathlineto{\pgfqpoint{3.432402in}{1.766652in}}%
\pgfpathlineto{\pgfqpoint{3.433363in}{1.772916in}}%
\pgfpathlineto{\pgfqpoint{3.433747in}{1.771553in}}%
\pgfpathlineto{\pgfqpoint{3.434323in}{1.772225in}}%
\pgfpathlineto{\pgfqpoint{3.435669in}{1.779247in}}%
\pgfpathlineto{\pgfqpoint{3.435861in}{1.781800in}}%
\pgfpathlineto{\pgfqpoint{3.436629in}{1.779114in}}%
\pgfpathlineto{\pgfqpoint{3.437590in}{1.776411in}}%
\pgfpathlineto{\pgfqpoint{3.438935in}{1.785225in}}%
\pgfpathlineto{\pgfqpoint{3.439128in}{1.783830in}}%
\pgfpathlineto{\pgfqpoint{3.441049in}{1.799447in}}%
\pgfpathlineto{\pgfqpoint{3.441434in}{1.796801in}}%
\pgfpathlineto{\pgfqpoint{3.441818in}{1.796065in}}%
\pgfpathlineto{\pgfqpoint{3.442202in}{1.798927in}}%
\pgfpathlineto{\pgfqpoint{3.442394in}{1.800865in}}%
\pgfpathlineto{\pgfqpoint{3.442971in}{1.795966in}}%
\pgfpathlineto{\pgfqpoint{3.443547in}{1.787209in}}%
\pgfpathlineto{\pgfqpoint{3.444316in}{1.789875in}}%
\pgfpathlineto{\pgfqpoint{3.444508in}{1.789936in}}%
\pgfpathlineto{\pgfqpoint{3.445469in}{1.791339in}}%
\pgfpathlineto{\pgfqpoint{3.445085in}{1.788950in}}%
\pgfpathlineto{\pgfqpoint{3.445661in}{1.790680in}}%
\pgfpathlineto{\pgfqpoint{3.445853in}{1.789704in}}%
\pgfpathlineto{\pgfqpoint{3.446046in}{1.792075in}}%
\pgfpathlineto{\pgfqpoint{3.446238in}{1.791570in}}%
\pgfpathlineto{\pgfqpoint{3.448159in}{1.801349in}}%
\pgfpathlineto{\pgfqpoint{3.448352in}{1.801013in}}%
\pgfpathlineto{\pgfqpoint{3.448544in}{1.799256in}}%
\pgfpathlineto{\pgfqpoint{3.449120in}{1.803137in}}%
\pgfpathlineto{\pgfqpoint{3.449312in}{1.802953in}}%
\pgfpathlineto{\pgfqpoint{3.449697in}{1.805873in}}%
\pgfpathlineto{\pgfqpoint{3.450081in}{1.801468in}}%
\pgfpathlineto{\pgfqpoint{3.450273in}{1.800127in}}%
\pgfpathlineto{\pgfqpoint{3.450658in}{1.803500in}}%
\pgfpathlineto{\pgfqpoint{3.451042in}{1.801259in}}%
\pgfpathlineto{\pgfqpoint{3.452003in}{1.807713in}}%
\pgfpathlineto{\pgfqpoint{3.452387in}{1.804810in}}%
\pgfpathlineto{\pgfqpoint{3.452771in}{1.798915in}}%
\pgfpathlineto{\pgfqpoint{3.453348in}{1.800839in}}%
\pgfpathlineto{\pgfqpoint{3.453540in}{1.804954in}}%
\pgfpathlineto{\pgfqpoint{3.454501in}{1.804890in}}%
\pgfpathlineto{\pgfqpoint{3.454885in}{1.804539in}}%
\pgfpathlineto{\pgfqpoint{3.455654in}{1.811523in}}%
\pgfpathlineto{\pgfqpoint{3.456423in}{1.810561in}}%
\pgfpathlineto{\pgfqpoint{3.456807in}{1.808336in}}%
\pgfpathlineto{\pgfqpoint{3.457191in}{1.810359in}}%
\pgfpathlineto{\pgfqpoint{3.458344in}{1.816339in}}%
\pgfpathlineto{\pgfqpoint{3.458921in}{1.814187in}}%
\pgfpathlineto{\pgfqpoint{3.460843in}{1.804087in}}%
\pgfpathlineto{\pgfqpoint{3.461035in}{1.808385in}}%
\pgfpathlineto{\pgfqpoint{3.461803in}{1.804139in}}%
\pgfpathlineto{\pgfqpoint{3.462764in}{1.800643in}}%
\pgfpathlineto{\pgfqpoint{3.462188in}{1.804774in}}%
\pgfpathlineto{\pgfqpoint{3.463149in}{1.802480in}}%
\pgfpathlineto{\pgfqpoint{3.463917in}{1.806840in}}%
\pgfpathlineto{\pgfqpoint{3.464494in}{1.805016in}}%
\pgfpathlineto{\pgfqpoint{3.464878in}{1.808775in}}%
\pgfpathlineto{\pgfqpoint{3.465070in}{1.804068in}}%
\pgfpathlineto{\pgfqpoint{3.465455in}{1.805303in}}%
\pgfpathlineto{\pgfqpoint{3.466031in}{1.802885in}}%
\pgfpathlineto{\pgfqpoint{3.466223in}{1.799123in}}%
\pgfpathlineto{\pgfqpoint{3.466992in}{1.803522in}}%
\pgfpathlineto{\pgfqpoint{3.467184in}{1.804535in}}%
\pgfpathlineto{\pgfqpoint{3.467568in}{1.801015in}}%
\pgfpathlineto{\pgfqpoint{3.468145in}{1.802554in}}%
\pgfpathlineto{\pgfqpoint{3.469106in}{1.816327in}}%
\pgfpathlineto{\pgfqpoint{3.469490in}{1.811201in}}%
\pgfpathlineto{\pgfqpoint{3.469682in}{1.813196in}}%
\pgfpathlineto{\pgfqpoint{3.470643in}{1.812425in}}%
\pgfpathlineto{\pgfqpoint{3.470835in}{1.810552in}}%
\pgfpathlineto{\pgfqpoint{3.471412in}{1.813990in}}%
\pgfpathlineto{\pgfqpoint{3.471604in}{1.816159in}}%
\pgfpathlineto{\pgfqpoint{3.472180in}{1.812144in}}%
\pgfpathlineto{\pgfqpoint{3.472565in}{1.815452in}}%
\pgfpathlineto{\pgfqpoint{3.472949in}{1.816895in}}%
\pgfpathlineto{\pgfqpoint{3.473141in}{1.816104in}}%
\pgfpathlineto{\pgfqpoint{3.473910in}{1.809351in}}%
\pgfpathlineto{\pgfqpoint{3.474486in}{1.810476in}}%
\pgfpathlineto{\pgfqpoint{3.475447in}{1.814237in}}%
\pgfpathlineto{\pgfqpoint{3.476216in}{1.813922in}}%
\pgfpathlineto{\pgfqpoint{3.476408in}{1.812994in}}%
\pgfpathlineto{\pgfqpoint{3.476600in}{1.813358in}}%
\pgfpathlineto{\pgfqpoint{3.477561in}{1.817446in}}%
\pgfpathlineto{\pgfqpoint{3.477753in}{1.816467in}}%
\pgfpathlineto{\pgfqpoint{3.478138in}{1.818658in}}%
\pgfpathlineto{\pgfqpoint{3.478714in}{1.815779in}}%
\pgfpathlineto{\pgfqpoint{3.480059in}{1.806468in}}%
\pgfpathlineto{\pgfqpoint{3.480444in}{1.812958in}}%
\pgfpathlineto{\pgfqpoint{3.481212in}{1.809238in}}%
\pgfpathlineto{\pgfqpoint{3.482365in}{1.804749in}}%
\pgfpathlineto{\pgfqpoint{3.483134in}{1.805568in}}%
\pgfpathlineto{\pgfqpoint{3.483518in}{1.806836in}}%
\pgfpathlineto{\pgfqpoint{3.483711in}{1.804897in}}%
\pgfpathlineto{\pgfqpoint{3.484095in}{1.801835in}}%
\pgfpathlineto{\pgfqpoint{3.484479in}{1.807430in}}%
\pgfpathlineto{\pgfqpoint{3.485056in}{1.807323in}}%
\pgfpathlineto{\pgfqpoint{3.485248in}{1.805626in}}%
\pgfpathlineto{\pgfqpoint{3.486017in}{1.808341in}}%
\pgfpathlineto{\pgfqpoint{3.486209in}{1.809414in}}%
\pgfpathlineto{\pgfqpoint{3.486593in}{1.806064in}}%
\pgfpathlineto{\pgfqpoint{3.486785in}{1.806208in}}%
\pgfpathlineto{\pgfqpoint{3.486977in}{1.805136in}}%
\pgfpathlineto{\pgfqpoint{3.487170in}{1.805375in}}%
\pgfpathlineto{\pgfqpoint{3.487554in}{1.804229in}}%
\pgfpathlineto{\pgfqpoint{3.488130in}{1.813271in}}%
\pgfpathlineto{\pgfqpoint{3.488899in}{1.813122in}}%
\pgfpathlineto{\pgfqpoint{3.489091in}{1.813005in}}%
\pgfpathlineto{\pgfqpoint{3.490436in}{1.823446in}}%
\pgfpathlineto{\pgfqpoint{3.490629in}{1.821631in}}%
\pgfpathlineto{\pgfqpoint{3.491397in}{1.817892in}}%
\pgfpathlineto{\pgfqpoint{3.491589in}{1.819609in}}%
\pgfpathlineto{\pgfqpoint{3.492935in}{1.825201in}}%
\pgfpathlineto{\pgfqpoint{3.493127in}{1.824275in}}%
\pgfpathlineto{\pgfqpoint{3.494472in}{1.827477in}}%
\pgfpathlineto{\pgfqpoint{3.495241in}{1.831035in}}%
\pgfpathlineto{\pgfqpoint{3.495433in}{1.829492in}}%
\pgfpathlineto{\pgfqpoint{3.495817in}{1.822044in}}%
\pgfpathlineto{\pgfqpoint{3.496586in}{1.825620in}}%
\pgfpathlineto{\pgfqpoint{3.497739in}{1.830693in}}%
\pgfpathlineto{\pgfqpoint{3.497931in}{1.830347in}}%
\pgfpathlineto{\pgfqpoint{3.498123in}{1.827273in}}%
\pgfpathlineto{\pgfqpoint{3.498507in}{1.834364in}}%
\pgfpathlineto{\pgfqpoint{3.498700in}{1.833757in}}%
\pgfpathlineto{\pgfqpoint{3.500237in}{1.836357in}}%
\pgfpathlineto{\pgfqpoint{3.500621in}{1.836981in}}%
\pgfpathlineto{\pgfqpoint{3.501390in}{1.835675in}}%
\pgfpathlineto{\pgfqpoint{3.501582in}{1.837109in}}%
\pgfpathlineto{\pgfqpoint{3.501966in}{1.833112in}}%
\pgfpathlineto{\pgfqpoint{3.502351in}{1.834721in}}%
\pgfpathlineto{\pgfqpoint{3.502927in}{1.836776in}}%
\pgfpathlineto{\pgfqpoint{3.502735in}{1.833697in}}%
\pgfpathlineto{\pgfqpoint{3.503696in}{1.834733in}}%
\pgfpathlineto{\pgfqpoint{3.504273in}{1.831271in}}%
\pgfpathlineto{\pgfqpoint{3.505233in}{1.833090in}}%
\pgfpathlineto{\pgfqpoint{3.506194in}{1.835346in}}%
\pgfpathlineto{\pgfqpoint{3.506386in}{1.834631in}}%
\pgfpathlineto{\pgfqpoint{3.506963in}{1.832687in}}%
\pgfpathlineto{\pgfqpoint{3.507347in}{1.835104in}}%
\pgfpathlineto{\pgfqpoint{3.507539in}{1.836574in}}%
\pgfpathlineto{\pgfqpoint{3.508116in}{1.833478in}}%
\pgfpathlineto{\pgfqpoint{3.508308in}{1.834202in}}%
\pgfpathlineto{\pgfqpoint{3.508500in}{1.834255in}}%
\pgfpathlineto{\pgfqpoint{3.508692in}{1.832026in}}%
\pgfpathlineto{\pgfqpoint{3.509077in}{1.834968in}}%
\pgfpathlineto{\pgfqpoint{3.509653in}{1.833577in}}%
\pgfpathlineto{\pgfqpoint{3.511383in}{1.841388in}}%
\pgfpathlineto{\pgfqpoint{3.511959in}{1.838173in}}%
\pgfpathlineto{\pgfqpoint{3.512344in}{1.841204in}}%
\pgfpathlineto{\pgfqpoint{3.513689in}{1.848953in}}%
\pgfpathlineto{\pgfqpoint{3.513881in}{1.848462in}}%
\pgfpathlineto{\pgfqpoint{3.514650in}{1.845334in}}%
\pgfpathlineto{\pgfqpoint{3.514842in}{1.848664in}}%
\pgfpathlineto{\pgfqpoint{3.515034in}{1.846585in}}%
\pgfpathlineto{\pgfqpoint{3.515418in}{1.851923in}}%
\pgfpathlineto{\pgfqpoint{3.516187in}{1.849584in}}%
\pgfpathlineto{\pgfqpoint{3.516379in}{1.846984in}}%
\pgfpathlineto{\pgfqpoint{3.516956in}{1.852397in}}%
\pgfpathlineto{\pgfqpoint{3.517148in}{1.855529in}}%
\pgfpathlineto{\pgfqpoint{3.517532in}{1.847503in}}%
\pgfpathlineto{\pgfqpoint{3.517916in}{1.850573in}}%
\pgfpathlineto{\pgfqpoint{3.521375in}{1.834479in}}%
\pgfpathlineto{\pgfqpoint{3.521952in}{1.837092in}}%
\pgfpathlineto{\pgfqpoint{3.522528in}{1.840201in}}%
\pgfpathlineto{\pgfqpoint{3.522721in}{1.836763in}}%
\pgfpathlineto{\pgfqpoint{3.523297in}{1.839892in}}%
\pgfpathlineto{\pgfqpoint{3.523489in}{1.837025in}}%
\pgfpathlineto{\pgfqpoint{3.524450in}{1.838851in}}%
\pgfpathlineto{\pgfqpoint{3.525027in}{1.840040in}}%
\pgfpathlineto{\pgfqpoint{3.525603in}{1.836086in}}%
\pgfpathlineto{\pgfqpoint{3.526756in}{1.846898in}}%
\pgfpathlineto{\pgfqpoint{3.527141in}{1.844849in}}%
\pgfpathlineto{\pgfqpoint{3.529639in}{1.829499in}}%
\pgfpathlineto{\pgfqpoint{3.529831in}{1.831294in}}%
\pgfpathlineto{\pgfqpoint{3.530407in}{1.827948in}}%
\pgfpathlineto{\pgfqpoint{3.530600in}{1.829332in}}%
\pgfpathlineto{\pgfqpoint{3.530792in}{1.829278in}}%
\pgfpathlineto{\pgfqpoint{3.531368in}{1.833467in}}%
\pgfpathlineto{\pgfqpoint{3.532137in}{1.832927in}}%
\pgfpathlineto{\pgfqpoint{3.534059in}{1.824383in}}%
\pgfpathlineto{\pgfqpoint{3.534827in}{1.825262in}}%
\pgfpathlineto{\pgfqpoint{3.535212in}{1.825803in}}%
\pgfpathlineto{\pgfqpoint{3.536172in}{1.821595in}}%
\pgfpathlineto{\pgfqpoint{3.536557in}{1.827142in}}%
\pgfpathlineto{\pgfqpoint{3.537325in}{1.822979in}}%
\pgfpathlineto{\pgfqpoint{3.538094in}{1.815648in}}%
\pgfpathlineto{\pgfqpoint{3.538478in}{1.822540in}}%
\pgfpathlineto{\pgfqpoint{3.538863in}{1.822259in}}%
\pgfpathlineto{\pgfqpoint{3.539247in}{1.817547in}}%
\pgfpathlineto{\pgfqpoint{3.539824in}{1.820383in}}%
\pgfpathlineto{\pgfqpoint{3.540400in}{1.824812in}}%
\pgfpathlineto{\pgfqpoint{3.540977in}{1.821202in}}%
\pgfpathlineto{\pgfqpoint{3.541169in}{1.821427in}}%
\pgfpathlineto{\pgfqpoint{3.541361in}{1.819534in}}%
\pgfpathlineto{\pgfqpoint{3.542130in}{1.822854in}}%
\pgfpathlineto{\pgfqpoint{3.543090in}{1.830574in}}%
\pgfpathlineto{\pgfqpoint{3.543475in}{1.828338in}}%
\pgfpathlineto{\pgfqpoint{3.544051in}{1.825419in}}%
\pgfpathlineto{\pgfqpoint{3.544243in}{1.826600in}}%
\pgfpathlineto{\pgfqpoint{3.545396in}{1.835475in}}%
\pgfpathlineto{\pgfqpoint{3.545589in}{1.833083in}}%
\pgfpathlineto{\pgfqpoint{3.545973in}{1.831717in}}%
\pgfpathlineto{\pgfqpoint{3.546165in}{1.835629in}}%
\pgfpathlineto{\pgfqpoint{3.546357in}{1.835285in}}%
\pgfpathlineto{\pgfqpoint{3.546549in}{1.834958in}}%
\pgfpathlineto{\pgfqpoint{3.547510in}{1.838672in}}%
\pgfpathlineto{\pgfqpoint{3.547895in}{1.838201in}}%
\pgfpathlineto{\pgfqpoint{3.548279in}{1.835451in}}%
\pgfpathlineto{\pgfqpoint{3.548855in}{1.838469in}}%
\pgfpathlineto{\pgfqpoint{3.549240in}{1.838977in}}%
\pgfpathlineto{\pgfqpoint{3.550008in}{1.839956in}}%
\pgfpathlineto{\pgfqpoint{3.550777in}{1.830274in}}%
\pgfpathlineto{\pgfqpoint{3.550969in}{1.830502in}}%
\pgfpathlineto{\pgfqpoint{3.552122in}{1.835062in}}%
\pgfpathlineto{\pgfqpoint{3.552315in}{1.834636in}}%
\pgfpathlineto{\pgfqpoint{3.552699in}{1.832490in}}%
\pgfpathlineto{\pgfqpoint{3.552891in}{1.833420in}}%
\pgfpathlineto{\pgfqpoint{3.554428in}{1.842733in}}%
\pgfpathlineto{\pgfqpoint{3.554621in}{1.842792in}}%
\pgfpathlineto{\pgfqpoint{3.554813in}{1.844409in}}%
\pgfpathlineto{\pgfqpoint{3.555389in}{1.841286in}}%
\pgfpathlineto{\pgfqpoint{3.555581in}{1.841873in}}%
\pgfpathlineto{\pgfqpoint{3.555774in}{1.841505in}}%
\pgfpathlineto{\pgfqpoint{3.555966in}{1.841897in}}%
\pgfpathlineto{\pgfqpoint{3.557311in}{1.853741in}}%
\pgfpathlineto{\pgfqpoint{3.557695in}{1.850884in}}%
\pgfpathlineto{\pgfqpoint{3.558272in}{1.852980in}}%
\pgfpathlineto{\pgfqpoint{3.558656in}{1.850706in}}%
\pgfpathlineto{\pgfqpoint{3.558848in}{1.850734in}}%
\pgfpathlineto{\pgfqpoint{3.559040in}{1.852036in}}%
\pgfpathlineto{\pgfqpoint{3.559233in}{1.849840in}}%
\pgfpathlineto{\pgfqpoint{3.560578in}{1.839671in}}%
\pgfpathlineto{\pgfqpoint{3.560770in}{1.840690in}}%
\pgfpathlineto{\pgfqpoint{3.563845in}{1.860688in}}%
\pgfpathlineto{\pgfqpoint{3.564805in}{1.853938in}}%
\pgfpathlineto{\pgfqpoint{3.564998in}{1.859128in}}%
\pgfpathlineto{\pgfqpoint{3.566151in}{1.867185in}}%
\pgfpathlineto{\pgfqpoint{3.566343in}{1.863672in}}%
\pgfpathlineto{\pgfqpoint{3.567304in}{1.864271in}}%
\pgfpathlineto{\pgfqpoint{3.567880in}{1.859167in}}%
\pgfpathlineto{\pgfqpoint{3.568841in}{1.863250in}}%
\pgfpathlineto{\pgfqpoint{3.569033in}{1.861713in}}%
\pgfpathlineto{\pgfqpoint{3.570378in}{1.854667in}}%
\pgfpathlineto{\pgfqpoint{3.570763in}{1.857541in}}%
\pgfpathlineto{\pgfqpoint{3.571147in}{1.857173in}}%
\pgfpathlineto{\pgfqpoint{3.571339in}{1.858011in}}%
\pgfpathlineto{\pgfqpoint{3.572300in}{1.862974in}}%
\pgfpathlineto{\pgfqpoint{3.572684in}{1.860291in}}%
\pgfpathlineto{\pgfqpoint{3.573837in}{1.866498in}}%
\pgfpathlineto{\pgfqpoint{3.574029in}{1.864970in}}%
\pgfpathlineto{\pgfqpoint{3.575375in}{1.855615in}}%
\pgfpathlineto{\pgfqpoint{3.575567in}{1.856450in}}%
\pgfpathlineto{\pgfqpoint{3.575759in}{1.857609in}}%
\pgfpathlineto{\pgfqpoint{3.576143in}{1.853952in}}%
\pgfpathlineto{\pgfqpoint{3.576720in}{1.848881in}}%
\pgfpathlineto{\pgfqpoint{3.577296in}{1.853257in}}%
\pgfpathlineto{\pgfqpoint{3.577873in}{1.854793in}}%
\pgfpathlineto{\pgfqpoint{3.578065in}{1.852045in}}%
\pgfpathlineto{\pgfqpoint{3.578257in}{1.849275in}}%
\pgfpathlineto{\pgfqpoint{3.578834in}{1.856501in}}%
\pgfpathlineto{\pgfqpoint{3.579026in}{1.858632in}}%
\pgfpathlineto{\pgfqpoint{3.579602in}{1.852889in}}%
\pgfpathlineto{\pgfqpoint{3.580563in}{1.848147in}}%
\pgfpathlineto{\pgfqpoint{3.579987in}{1.853074in}}%
\pgfpathlineto{\pgfqpoint{3.581140in}{1.850057in}}%
\pgfpathlineto{\pgfqpoint{3.581332in}{1.849778in}}%
\pgfpathlineto{\pgfqpoint{3.582293in}{1.848516in}}%
\pgfpathlineto{\pgfqpoint{3.582485in}{1.851501in}}%
\pgfpathlineto{\pgfqpoint{3.584022in}{1.839862in}}%
\pgfpathlineto{\pgfqpoint{3.585560in}{1.848181in}}%
\pgfpathlineto{\pgfqpoint{3.586713in}{1.840476in}}%
\pgfpathlineto{\pgfqpoint{3.587866in}{1.844005in}}%
\pgfpathlineto{\pgfqpoint{3.588250in}{1.845928in}}%
\pgfpathlineto{\pgfqpoint{3.589979in}{1.858919in}}%
\pgfpathlineto{\pgfqpoint{3.590172in}{1.856299in}}%
\pgfpathlineto{\pgfqpoint{3.590748in}{1.855135in}}%
\pgfpathlineto{\pgfqpoint{3.590940in}{1.856553in}}%
\pgfpathlineto{\pgfqpoint{3.591325in}{1.858632in}}%
\pgfpathlineto{\pgfqpoint{3.591517in}{1.857882in}}%
\pgfpathlineto{\pgfqpoint{3.591709in}{1.853162in}}%
\pgfpathlineto{\pgfqpoint{3.592478in}{1.855862in}}%
\pgfpathlineto{\pgfqpoint{3.592670in}{1.857395in}}%
\pgfpathlineto{\pgfqpoint{3.593054in}{1.855545in}}%
\pgfpathlineto{\pgfqpoint{3.593438in}{1.855554in}}%
\pgfpathlineto{\pgfqpoint{3.593631in}{1.852226in}}%
\pgfpathlineto{\pgfqpoint{3.594207in}{1.856728in}}%
\pgfpathlineto{\pgfqpoint{3.594399in}{1.855428in}}%
\pgfpathlineto{\pgfqpoint{3.594591in}{1.856075in}}%
\pgfpathlineto{\pgfqpoint{3.595744in}{1.845506in}}%
\pgfpathlineto{\pgfqpoint{3.595937in}{1.848067in}}%
\pgfpathlineto{\pgfqpoint{3.596129in}{1.847769in}}%
\pgfpathlineto{\pgfqpoint{3.596705in}{1.850874in}}%
\pgfpathlineto{\pgfqpoint{3.597282in}{1.850634in}}%
\pgfpathlineto{\pgfqpoint{3.598050in}{1.842702in}}%
\pgfpathlineto{\pgfqpoint{3.599011in}{1.843154in}}%
\pgfpathlineto{\pgfqpoint{3.599780in}{1.845065in}}%
\pgfpathlineto{\pgfqpoint{3.600164in}{1.844624in}}%
\pgfpathlineto{\pgfqpoint{3.600357in}{1.842899in}}%
\pgfpathlineto{\pgfqpoint{3.600549in}{1.846418in}}%
\pgfpathlineto{\pgfqpoint{3.600933in}{1.845602in}}%
\pgfpathlineto{\pgfqpoint{3.601125in}{1.847089in}}%
\pgfpathlineto{\pgfqpoint{3.601702in}{1.844471in}}%
\pgfpathlineto{\pgfqpoint{3.601894in}{1.844026in}}%
\pgfpathlineto{\pgfqpoint{3.603239in}{1.850400in}}%
\pgfpathlineto{\pgfqpoint{3.603623in}{1.848618in}}%
\pgfpathlineto{\pgfqpoint{3.604008in}{1.851322in}}%
\pgfpathlineto{\pgfqpoint{3.604969in}{1.859043in}}%
\pgfpathlineto{\pgfqpoint{3.605353in}{1.858568in}}%
\pgfpathlineto{\pgfqpoint{3.606122in}{1.857639in}}%
\pgfpathlineto{\pgfqpoint{3.606314in}{1.859168in}}%
\pgfpathlineto{\pgfqpoint{3.607851in}{1.868201in}}%
\pgfpathlineto{\pgfqpoint{3.608428in}{1.868460in}}%
\pgfpathlineto{\pgfqpoint{3.609004in}{1.864309in}}%
\pgfpathlineto{\pgfqpoint{3.609196in}{1.866625in}}%
\pgfpathlineto{\pgfqpoint{3.609388in}{1.864070in}}%
\pgfpathlineto{\pgfqpoint{3.609965in}{1.866396in}}%
\pgfpathlineto{\pgfqpoint{3.611502in}{1.861797in}}%
\pgfpathlineto{\pgfqpoint{3.612271in}{1.858236in}}%
\pgfpathlineto{\pgfqpoint{3.612847in}{1.858324in}}%
\pgfpathlineto{\pgfqpoint{3.613424in}{1.858878in}}%
\pgfpathlineto{\pgfqpoint{3.614769in}{1.853964in}}%
\pgfpathlineto{\pgfqpoint{3.615153in}{1.857127in}}%
\pgfpathlineto{\pgfqpoint{3.615538in}{1.853339in}}%
\pgfpathlineto{\pgfqpoint{3.615730in}{1.853268in}}%
\pgfpathlineto{\pgfqpoint{3.617459in}{1.845895in}}%
\pgfpathlineto{\pgfqpoint{3.617652in}{1.845742in}}%
\pgfpathlineto{\pgfqpoint{3.618228in}{1.839728in}}%
\pgfpathlineto{\pgfqpoint{3.618612in}{1.844607in}}%
\pgfpathlineto{\pgfqpoint{3.618805in}{1.846947in}}%
\pgfpathlineto{\pgfqpoint{3.619381in}{1.841476in}}%
\pgfpathlineto{\pgfqpoint{3.619958in}{1.842997in}}%
\pgfpathlineto{\pgfqpoint{3.619765in}{1.840600in}}%
\pgfpathlineto{\pgfqpoint{3.620150in}{1.842133in}}%
\pgfpathlineto{\pgfqpoint{3.621111in}{1.845287in}}%
\pgfpathlineto{\pgfqpoint{3.621303in}{1.844594in}}%
\pgfpathlineto{\pgfqpoint{3.621687in}{1.841726in}}%
\pgfpathlineto{\pgfqpoint{3.622071in}{1.843275in}}%
\pgfpathlineto{\pgfqpoint{3.623032in}{1.847235in}}%
\pgfpathlineto{\pgfqpoint{3.623224in}{1.846895in}}%
\pgfpathlineto{\pgfqpoint{3.623609in}{1.843345in}}%
\pgfpathlineto{\pgfqpoint{3.623993in}{1.848790in}}%
\pgfpathlineto{\pgfqpoint{3.624377in}{1.845356in}}%
\pgfpathlineto{\pgfqpoint{3.624570in}{1.845052in}}%
\pgfpathlineto{\pgfqpoint{3.624762in}{1.846900in}}%
\pgfpathlineto{\pgfqpoint{3.625531in}{1.844179in}}%
\pgfpathlineto{\pgfqpoint{3.625723in}{1.843664in}}%
\pgfpathlineto{\pgfqpoint{3.627068in}{1.833383in}}%
\pgfpathlineto{\pgfqpoint{3.627260in}{1.833942in}}%
\pgfpathlineto{\pgfqpoint{3.628221in}{1.836788in}}%
\pgfpathlineto{\pgfqpoint{3.628413in}{1.831288in}}%
\pgfpathlineto{\pgfqpoint{3.629374in}{1.833779in}}%
\pgfpathlineto{\pgfqpoint{3.630527in}{1.838353in}}%
\pgfpathlineto{\pgfqpoint{3.631103in}{1.838218in}}%
\pgfpathlineto{\pgfqpoint{3.631296in}{1.835833in}}%
\pgfpathlineto{\pgfqpoint{3.631872in}{1.838638in}}%
\pgfpathlineto{\pgfqpoint{3.632064in}{1.842053in}}%
\pgfpathlineto{\pgfqpoint{3.633025in}{1.839843in}}%
\pgfpathlineto{\pgfqpoint{3.633217in}{1.839495in}}%
\pgfpathlineto{\pgfqpoint{3.633794in}{1.847048in}}%
\pgfpathlineto{\pgfqpoint{3.634370in}{1.844574in}}%
\pgfpathlineto{\pgfqpoint{3.634755in}{1.844774in}}%
\pgfpathlineto{\pgfqpoint{3.634947in}{1.847338in}}%
\pgfpathlineto{\pgfqpoint{3.635715in}{1.844489in}}%
\pgfpathlineto{\pgfqpoint{3.636676in}{1.849944in}}%
\pgfpathlineto{\pgfqpoint{3.637061in}{1.845480in}}%
\pgfpathlineto{\pgfqpoint{3.637253in}{1.845268in}}%
\pgfpathlineto{\pgfqpoint{3.638214in}{1.849575in}}%
\pgfpathlineto{\pgfqpoint{3.638406in}{1.847467in}}%
\pgfpathlineto{\pgfqpoint{3.639367in}{1.839308in}}%
\pgfpathlineto{\pgfqpoint{3.639751in}{1.844247in}}%
\pgfpathlineto{\pgfqpoint{3.640135in}{1.844518in}}%
\pgfpathlineto{\pgfqpoint{3.640904in}{1.840990in}}%
\pgfpathlineto{\pgfqpoint{3.641480in}{1.845883in}}%
\pgfpathlineto{\pgfqpoint{3.642057in}{1.842295in}}%
\pgfpathlineto{\pgfqpoint{3.642249in}{1.843075in}}%
\pgfpathlineto{\pgfqpoint{3.642441in}{1.841597in}}%
\pgfpathlineto{\pgfqpoint{3.643594in}{1.835799in}}%
\pgfpathlineto{\pgfqpoint{3.643786in}{1.835857in}}%
\pgfpathlineto{\pgfqpoint{3.645516in}{1.847416in}}%
\pgfpathlineto{\pgfqpoint{3.646477in}{1.843849in}}%
\pgfpathlineto{\pgfqpoint{3.647245in}{1.845236in}}%
\pgfpathlineto{\pgfqpoint{3.647630in}{1.842977in}}%
\pgfpathlineto{\pgfqpoint{3.648014in}{1.846213in}}%
\pgfpathlineto{\pgfqpoint{3.648783in}{1.849877in}}%
\pgfpathlineto{\pgfqpoint{3.649552in}{1.848516in}}%
\pgfpathlineto{\pgfqpoint{3.649744in}{1.849220in}}%
\pgfpathlineto{\pgfqpoint{3.650320in}{1.847946in}}%
\pgfpathlineto{\pgfqpoint{3.650705in}{1.848770in}}%
\pgfpathlineto{\pgfqpoint{3.650897in}{1.848641in}}%
\pgfpathlineto{\pgfqpoint{3.651089in}{1.851390in}}%
\pgfpathlineto{\pgfqpoint{3.651281in}{1.848471in}}%
\pgfpathlineto{\pgfqpoint{3.651858in}{1.849035in}}%
\pgfpathlineto{\pgfqpoint{3.652050in}{1.847635in}}%
\pgfpathlineto{\pgfqpoint{3.652626in}{1.849792in}}%
\pgfpathlineto{\pgfqpoint{3.652818in}{1.852583in}}%
\pgfpathlineto{\pgfqpoint{3.653395in}{1.845919in}}%
\pgfpathlineto{\pgfqpoint{3.653779in}{1.850386in}}%
\pgfpathlineto{\pgfqpoint{3.654548in}{1.847637in}}%
\pgfpathlineto{\pgfqpoint{3.654932in}{1.847802in}}%
\pgfpathlineto{\pgfqpoint{3.655124in}{1.848609in}}%
\pgfpathlineto{\pgfqpoint{3.655317in}{1.847288in}}%
\pgfpathlineto{\pgfqpoint{3.657238in}{1.833737in}}%
\pgfpathlineto{\pgfqpoint{3.658776in}{1.840687in}}%
\pgfpathlineto{\pgfqpoint{3.659160in}{1.839692in}}%
\pgfpathlineto{\pgfqpoint{3.659352in}{1.834805in}}%
\pgfpathlineto{\pgfqpoint{3.660121in}{1.843003in}}%
\pgfpathlineto{\pgfqpoint{3.660505in}{1.840348in}}%
\pgfpathlineto{\pgfqpoint{3.660697in}{1.843381in}}%
\pgfpathlineto{\pgfqpoint{3.660889in}{1.842676in}}%
\pgfpathlineto{\pgfqpoint{3.662427in}{1.849569in}}%
\pgfpathlineto{\pgfqpoint{3.661274in}{1.839455in}}%
\pgfpathlineto{\pgfqpoint{3.662619in}{1.847495in}}%
\pgfpathlineto{\pgfqpoint{3.663195in}{1.849146in}}%
\pgfpathlineto{\pgfqpoint{3.663388in}{1.846665in}}%
\pgfpathlineto{\pgfqpoint{3.664156in}{1.843205in}}%
\pgfpathlineto{\pgfqpoint{3.663772in}{1.846791in}}%
\pgfpathlineto{\pgfqpoint{3.664733in}{1.844291in}}%
\pgfpathlineto{\pgfqpoint{3.665694in}{1.851750in}}%
\pgfpathlineto{\pgfqpoint{3.666462in}{1.848536in}}%
\pgfpathlineto{\pgfqpoint{3.667039in}{1.848490in}}%
\pgfpathlineto{\pgfqpoint{3.667231in}{1.846432in}}%
\pgfpathlineto{\pgfqpoint{3.667807in}{1.850253in}}%
\pgfpathlineto{\pgfqpoint{3.668000in}{1.850061in}}%
\pgfpathlineto{\pgfqpoint{3.670306in}{1.869832in}}%
\pgfpathlineto{\pgfqpoint{3.671651in}{1.859193in}}%
\pgfpathlineto{\pgfqpoint{3.672035in}{1.859547in}}%
\pgfpathlineto{\pgfqpoint{3.672419in}{1.855457in}}%
\pgfpathlineto{\pgfqpoint{3.673188in}{1.856791in}}%
\pgfpathlineto{\pgfqpoint{3.674726in}{1.848523in}}%
\pgfpathlineto{\pgfqpoint{3.675302in}{1.850751in}}%
\pgfpathlineto{\pgfqpoint{3.675494in}{1.851536in}}%
\pgfpathlineto{\pgfqpoint{3.675879in}{1.848913in}}%
\pgfpathlineto{\pgfqpoint{3.676071in}{1.850234in}}%
\pgfpathlineto{\pgfqpoint{3.678377in}{1.843815in}}%
\pgfpathlineto{\pgfqpoint{3.679914in}{1.859140in}}%
\pgfpathlineto{\pgfqpoint{3.680491in}{1.858248in}}%
\pgfpathlineto{\pgfqpoint{3.681259in}{1.849605in}}%
\pgfpathlineto{\pgfqpoint{3.681836in}{1.852821in}}%
\pgfpathlineto{\pgfqpoint{3.682028in}{1.852606in}}%
\pgfpathlineto{\pgfqpoint{3.683181in}{1.848770in}}%
\pgfpathlineto{\pgfqpoint{3.683373in}{1.849109in}}%
\pgfpathlineto{\pgfqpoint{3.683565in}{1.849236in}}%
\pgfpathlineto{\pgfqpoint{3.683757in}{1.850818in}}%
\pgfpathlineto{\pgfqpoint{3.684334in}{1.846479in}}%
\pgfpathlineto{\pgfqpoint{3.685487in}{1.851250in}}%
\pgfpathlineto{\pgfqpoint{3.684718in}{1.845843in}}%
\pgfpathlineto{\pgfqpoint{3.685679in}{1.848924in}}%
\pgfpathlineto{\pgfqpoint{3.686256in}{1.844038in}}%
\pgfpathlineto{\pgfqpoint{3.686640in}{1.846426in}}%
\pgfpathlineto{\pgfqpoint{3.688177in}{1.855348in}}%
\pgfpathlineto{\pgfqpoint{3.688946in}{1.847406in}}%
\pgfpathlineto{\pgfqpoint{3.689330in}{1.852298in}}%
\pgfpathlineto{\pgfqpoint{3.690675in}{1.861275in}}%
\pgfpathlineto{\pgfqpoint{3.691252in}{1.864540in}}%
\pgfpathlineto{\pgfqpoint{3.692405in}{1.866761in}}%
\pgfpathlineto{\pgfqpoint{3.691828in}{1.861532in}}%
\pgfpathlineto{\pgfqpoint{3.692597in}{1.866093in}}%
\pgfpathlineto{\pgfqpoint{3.694327in}{1.860499in}}%
\pgfpathlineto{\pgfqpoint{3.693174in}{1.866264in}}%
\pgfpathlineto{\pgfqpoint{3.694711in}{1.860804in}}%
\pgfpathlineto{\pgfqpoint{3.695095in}{1.863276in}}%
\pgfpathlineto{\pgfqpoint{3.695672in}{1.860755in}}%
\pgfpathlineto{\pgfqpoint{3.696440in}{1.858420in}}%
\pgfpathlineto{\pgfqpoint{3.697017in}{1.858868in}}%
\pgfpathlineto{\pgfqpoint{3.698362in}{1.863271in}}%
\pgfpathlineto{\pgfqpoint{3.698554in}{1.862176in}}%
\pgfpathlineto{\pgfqpoint{3.700284in}{1.851078in}}%
\pgfpathlineto{\pgfqpoint{3.700668in}{1.852840in}}%
\pgfpathlineto{\pgfqpoint{3.701821in}{1.861757in}}%
\pgfpathlineto{\pgfqpoint{3.702398in}{1.858904in}}%
\pgfpathlineto{\pgfqpoint{3.702974in}{1.855026in}}%
\pgfpathlineto{\pgfqpoint{3.703359in}{1.860151in}}%
\pgfpathlineto{\pgfqpoint{3.703743in}{1.856079in}}%
\pgfpathlineto{\pgfqpoint{3.704127in}{1.853703in}}%
\pgfpathlineto{\pgfqpoint{3.704704in}{1.857690in}}%
\pgfpathlineto{\pgfqpoint{3.704896in}{1.858101in}}%
\pgfpathlineto{\pgfqpoint{3.705088in}{1.857435in}}%
\pgfpathlineto{\pgfqpoint{3.706818in}{1.843200in}}%
\pgfpathlineto{\pgfqpoint{3.707586in}{1.848494in}}%
\pgfpathlineto{\pgfqpoint{3.707778in}{1.847912in}}%
\pgfpathlineto{\pgfqpoint{3.708163in}{1.849539in}}%
\pgfpathlineto{\pgfqpoint{3.708355in}{1.850526in}}%
\pgfpathlineto{\pgfqpoint{3.708739in}{1.848165in}}%
\pgfpathlineto{\pgfqpoint{3.709700in}{1.842224in}}%
\pgfpathlineto{\pgfqpoint{3.709892in}{1.844149in}}%
\pgfpathlineto{\pgfqpoint{3.711045in}{1.852244in}}%
\pgfpathlineto{\pgfqpoint{3.711237in}{1.850765in}}%
\pgfpathlineto{\pgfqpoint{3.712775in}{1.844160in}}%
\pgfpathlineto{\pgfqpoint{3.713543in}{1.851017in}}%
\pgfpathlineto{\pgfqpoint{3.713928in}{1.845815in}}%
\pgfpathlineto{\pgfqpoint{3.714696in}{1.845231in}}%
\pgfpathlineto{\pgfqpoint{3.714312in}{1.848025in}}%
\pgfpathlineto{\pgfqpoint{3.714889in}{1.846319in}}%
\pgfpathlineto{\pgfqpoint{3.715081in}{1.847906in}}%
\pgfpathlineto{\pgfqpoint{3.715465in}{1.843355in}}%
\pgfpathlineto{\pgfqpoint{3.715657in}{1.844795in}}%
\pgfpathlineto{\pgfqpoint{3.716234in}{1.843199in}}%
\pgfpathlineto{\pgfqpoint{3.716426in}{1.841429in}}%
\pgfpathlineto{\pgfqpoint{3.717002in}{1.846378in}}%
\pgfpathlineto{\pgfqpoint{3.717387in}{1.849932in}}%
\pgfpathlineto{\pgfqpoint{3.717963in}{1.845921in}}%
\pgfpathlineto{\pgfqpoint{3.718732in}{1.842545in}}%
\pgfpathlineto{\pgfqpoint{3.718924in}{1.846204in}}%
\pgfpathlineto{\pgfqpoint{3.719116in}{1.844276in}}%
\pgfpathlineto{\pgfqpoint{3.719308in}{1.844822in}}%
\pgfpathlineto{\pgfqpoint{3.719501in}{1.842781in}}%
\pgfpathlineto{\pgfqpoint{3.719693in}{1.843282in}}%
\pgfpathlineto{\pgfqpoint{3.720077in}{1.840388in}}%
\pgfpathlineto{\pgfqpoint{3.720461in}{1.844651in}}%
\pgfpathlineto{\pgfqpoint{3.721038in}{1.849947in}}%
\pgfpathlineto{\pgfqpoint{3.721614in}{1.847101in}}%
\pgfpathlineto{\pgfqpoint{3.721807in}{1.846027in}}%
\pgfpathlineto{\pgfqpoint{3.722191in}{1.849071in}}%
\pgfpathlineto{\pgfqpoint{3.722383in}{1.849872in}}%
\pgfpathlineto{\pgfqpoint{3.722768in}{1.847551in}}%
\pgfpathlineto{\pgfqpoint{3.722960in}{1.848802in}}%
\pgfpathlineto{\pgfqpoint{3.723152in}{1.846845in}}%
\pgfpathlineto{\pgfqpoint{3.723728in}{1.850755in}}%
\pgfpathlineto{\pgfqpoint{3.723921in}{1.849294in}}%
\pgfpathlineto{\pgfqpoint{3.725650in}{1.856448in}}%
\pgfpathlineto{\pgfqpoint{3.726227in}{1.851355in}}%
\pgfpathlineto{\pgfqpoint{3.726995in}{1.852714in}}%
\pgfpathlineto{\pgfqpoint{3.727187in}{1.854656in}}%
\pgfpathlineto{\pgfqpoint{3.727572in}{1.849561in}}%
\pgfpathlineto{\pgfqpoint{3.727764in}{1.851262in}}%
\pgfpathlineto{\pgfqpoint{3.728148in}{1.850297in}}%
\pgfpathlineto{\pgfqpoint{3.728340in}{1.852340in}}%
\pgfpathlineto{\pgfqpoint{3.728725in}{1.852012in}}%
\pgfpathlineto{\pgfqpoint{3.728917in}{1.851627in}}%
\pgfpathlineto{\pgfqpoint{3.729109in}{1.851852in}}%
\pgfpathlineto{\pgfqpoint{3.729878in}{1.848598in}}%
\pgfpathlineto{\pgfqpoint{3.730262in}{1.853861in}}%
\pgfpathlineto{\pgfqpoint{3.730646in}{1.850249in}}%
\pgfpathlineto{\pgfqpoint{3.731607in}{1.850473in}}%
\pgfpathlineto{\pgfqpoint{3.732184in}{1.852591in}}%
\pgfpathlineto{\pgfqpoint{3.734298in}{1.869270in}}%
\pgfpathlineto{\pgfqpoint{3.734490in}{1.868933in}}%
\pgfpathlineto{\pgfqpoint{3.735066in}{1.866052in}}%
\pgfpathlineto{\pgfqpoint{3.735643in}{1.866122in}}%
\pgfpathlineto{\pgfqpoint{3.737757in}{1.850644in}}%
\pgfpathlineto{\pgfqpoint{3.739102in}{1.859577in}}%
\pgfpathlineto{\pgfqpoint{3.739486in}{1.857189in}}%
\pgfpathlineto{\pgfqpoint{3.739678in}{1.857060in}}%
\pgfpathlineto{\pgfqpoint{3.739870in}{1.859471in}}%
\pgfpathlineto{\pgfqpoint{3.740639in}{1.856766in}}%
\pgfpathlineto{\pgfqpoint{3.742753in}{1.852740in}}%
\pgfpathlineto{\pgfqpoint{3.742945in}{1.854266in}}%
\pgfpathlineto{\pgfqpoint{3.744675in}{1.864413in}}%
\pgfpathlineto{\pgfqpoint{3.744867in}{1.861365in}}%
\pgfpathlineto{\pgfqpoint{3.745059in}{1.860385in}}%
\pgfpathlineto{\pgfqpoint{3.745443in}{1.863650in}}%
\pgfpathlineto{\pgfqpoint{3.747365in}{1.874186in}}%
\pgfpathlineto{\pgfqpoint{3.747557in}{1.873787in}}%
\pgfpathlineto{\pgfqpoint{3.748134in}{1.867277in}}%
\pgfpathlineto{\pgfqpoint{3.748518in}{1.874029in}}%
\pgfpathlineto{\pgfqpoint{3.750248in}{1.884672in}}%
\pgfpathlineto{\pgfqpoint{3.750440in}{1.883657in}}%
\pgfpathlineto{\pgfqpoint{3.751016in}{1.880889in}}%
\pgfpathlineto{\pgfqpoint{3.751401in}{1.884870in}}%
\pgfpathlineto{\pgfqpoint{3.752361in}{1.887461in}}%
\pgfpathlineto{\pgfqpoint{3.753322in}{1.883463in}}%
\pgfpathlineto{\pgfqpoint{3.753514in}{1.886241in}}%
\pgfpathlineto{\pgfqpoint{3.754283in}{1.892962in}}%
\pgfpathlineto{\pgfqpoint{3.754860in}{1.889663in}}%
\pgfpathlineto{\pgfqpoint{3.755436in}{1.884823in}}%
\pgfpathlineto{\pgfqpoint{3.756205in}{1.886106in}}%
\pgfpathlineto{\pgfqpoint{3.756397in}{1.886208in}}%
\pgfpathlineto{\pgfqpoint{3.756781in}{1.884026in}}%
\pgfpathlineto{\pgfqpoint{3.757166in}{1.885944in}}%
\pgfpathlineto{\pgfqpoint{3.757934in}{1.884957in}}%
\pgfpathlineto{\pgfqpoint{3.758703in}{1.893320in}}%
\pgfpathlineto{\pgfqpoint{3.760240in}{1.883803in}}%
\pgfpathlineto{\pgfqpoint{3.761201in}{1.882004in}}%
\pgfpathlineto{\pgfqpoint{3.761393in}{1.883828in}}%
\pgfpathlineto{\pgfqpoint{3.761778in}{1.879030in}}%
\pgfpathlineto{\pgfqpoint{3.763123in}{1.873930in}}%
\pgfpathlineto{\pgfqpoint{3.763315in}{1.875263in}}%
\pgfpathlineto{\pgfqpoint{3.763699in}{1.871648in}}%
\pgfpathlineto{\pgfqpoint{3.763891in}{1.869293in}}%
\pgfpathlineto{\pgfqpoint{3.764084in}{1.872381in}}%
\pgfpathlineto{\pgfqpoint{3.764468in}{1.871466in}}%
\pgfpathlineto{\pgfqpoint{3.764660in}{1.874281in}}%
\pgfpathlineto{\pgfqpoint{3.765237in}{1.870610in}}%
\pgfpathlineto{\pgfqpoint{3.765813in}{1.869042in}}%
\pgfpathlineto{\pgfqpoint{3.766005in}{1.870977in}}%
\pgfpathlineto{\pgfqpoint{3.766774in}{1.876545in}}%
\pgfpathlineto{\pgfqpoint{3.767158in}{1.871527in}}%
\pgfpathlineto{\pgfqpoint{3.767350in}{1.870733in}}%
\pgfpathlineto{\pgfqpoint{3.767735in}{1.872455in}}%
\pgfpathlineto{\pgfqpoint{3.768119in}{1.872118in}}%
\pgfpathlineto{\pgfqpoint{3.768888in}{1.879970in}}%
\pgfpathlineto{\pgfqpoint{3.769464in}{1.879771in}}%
\pgfpathlineto{\pgfqpoint{3.769849in}{1.883501in}}%
\pgfpathlineto{\pgfqpoint{3.771194in}{1.870292in}}%
\pgfpathlineto{\pgfqpoint{3.771386in}{1.872902in}}%
\pgfpathlineto{\pgfqpoint{3.771578in}{1.872909in}}%
\pgfpathlineto{\pgfqpoint{3.773692in}{1.862124in}}%
\pgfpathlineto{\pgfqpoint{3.773884in}{1.864698in}}%
\pgfpathlineto{\pgfqpoint{3.774076in}{1.865150in}}%
\pgfpathlineto{\pgfqpoint{3.774269in}{1.863147in}}%
\pgfpathlineto{\pgfqpoint{3.774653in}{1.863734in}}%
\pgfpathlineto{\pgfqpoint{3.775229in}{1.858905in}}%
\pgfpathlineto{\pgfqpoint{3.776190in}{1.859240in}}%
\pgfpathlineto{\pgfqpoint{3.776767in}{1.862243in}}%
\pgfpathlineto{\pgfqpoint{3.777151in}{1.860751in}}%
\pgfpathlineto{\pgfqpoint{3.777728in}{1.857069in}}%
\pgfpathlineto{\pgfqpoint{3.778304in}{1.857383in}}%
\pgfpathlineto{\pgfqpoint{3.779457in}{1.861583in}}%
\pgfpathlineto{\pgfqpoint{3.779649in}{1.861475in}}%
\pgfpathlineto{\pgfqpoint{3.779841in}{1.860301in}}%
\pgfpathlineto{\pgfqpoint{3.780034in}{1.860491in}}%
\pgfpathlineto{\pgfqpoint{3.780226in}{1.864840in}}%
\pgfpathlineto{\pgfqpoint{3.781187in}{1.862688in}}%
\pgfpathlineto{\pgfqpoint{3.781763in}{1.859316in}}%
\pgfpathlineto{\pgfqpoint{3.782147in}{1.860907in}}%
\pgfpathlineto{\pgfqpoint{3.782916in}{1.856477in}}%
\pgfpathlineto{\pgfqpoint{3.783300in}{1.859844in}}%
\pgfpathlineto{\pgfqpoint{3.784069in}{1.859545in}}%
\pgfpathlineto{\pgfqpoint{3.786759in}{1.848580in}}%
\pgfpathlineto{\pgfqpoint{3.786952in}{1.851009in}}%
\pgfpathlineto{\pgfqpoint{3.787720in}{1.849031in}}%
\pgfpathlineto{\pgfqpoint{3.788681in}{1.840211in}}%
\pgfpathlineto{\pgfqpoint{3.789258in}{1.840348in}}%
\pgfpathlineto{\pgfqpoint{3.789450in}{1.840017in}}%
\pgfpathlineto{\pgfqpoint{3.790218in}{1.840329in}}%
\pgfpathlineto{\pgfqpoint{3.790795in}{1.834723in}}%
\pgfpathlineto{\pgfqpoint{3.791371in}{1.839237in}}%
\pgfpathlineto{\pgfqpoint{3.792140in}{1.836754in}}%
\pgfpathlineto{\pgfqpoint{3.792332in}{1.834117in}}%
\pgfpathlineto{\pgfqpoint{3.792909in}{1.838245in}}%
\pgfpathlineto{\pgfqpoint{3.793293in}{1.834833in}}%
\pgfpathlineto{\pgfqpoint{3.794638in}{1.840242in}}%
\pgfpathlineto{\pgfqpoint{3.796176in}{1.851003in}}%
\pgfpathlineto{\pgfqpoint{3.796560in}{1.850103in}}%
\pgfpathlineto{\pgfqpoint{3.797521in}{1.845482in}}%
\pgfpathlineto{\pgfqpoint{3.798097in}{1.846461in}}%
\pgfpathlineto{\pgfqpoint{3.798482in}{1.851292in}}%
\pgfpathlineto{\pgfqpoint{3.799058in}{1.846303in}}%
\pgfpathlineto{\pgfqpoint{3.800403in}{1.843024in}}%
\pgfpathlineto{\pgfqpoint{3.800596in}{1.845135in}}%
\pgfpathlineto{\pgfqpoint{3.800980in}{1.839746in}}%
\pgfpathlineto{\pgfqpoint{3.801172in}{1.840711in}}%
\pgfpathlineto{\pgfqpoint{3.801364in}{1.840891in}}%
\pgfpathlineto{\pgfqpoint{3.801749in}{1.843901in}}%
\pgfpathlineto{\pgfqpoint{3.802325in}{1.839610in}}%
\pgfpathlineto{\pgfqpoint{3.802517in}{1.842485in}}%
\pgfpathlineto{\pgfqpoint{3.802709in}{1.842840in}}%
\pgfpathlineto{\pgfqpoint{3.804247in}{1.831684in}}%
\pgfpathlineto{\pgfqpoint{3.805400in}{1.835203in}}%
\pgfpathlineto{\pgfqpoint{3.806745in}{1.826656in}}%
\pgfpathlineto{\pgfqpoint{3.806937in}{1.828695in}}%
\pgfpathlineto{\pgfqpoint{3.807129in}{1.831300in}}%
\pgfpathlineto{\pgfqpoint{3.807514in}{1.824754in}}%
\pgfpathlineto{\pgfqpoint{3.807898in}{1.828355in}}%
\pgfpathlineto{\pgfqpoint{3.809243in}{1.822637in}}%
\pgfpathlineto{\pgfqpoint{3.810780in}{1.830555in}}%
\pgfpathlineto{\pgfqpoint{3.811165in}{1.831478in}}%
\pgfpathlineto{\pgfqpoint{3.811357in}{1.830119in}}%
\pgfpathlineto{\pgfqpoint{3.811549in}{1.827921in}}%
\pgfpathlineto{\pgfqpoint{3.811933in}{1.834274in}}%
\pgfpathlineto{\pgfqpoint{3.812894in}{1.843041in}}%
\pgfpathlineto{\pgfqpoint{3.813086in}{1.841511in}}%
\pgfpathlineto{\pgfqpoint{3.813471in}{1.837790in}}%
\pgfpathlineto{\pgfqpoint{3.814239in}{1.840886in}}%
\pgfpathlineto{\pgfqpoint{3.815392in}{1.844986in}}%
\pgfpathlineto{\pgfqpoint{3.815008in}{1.840760in}}%
\pgfpathlineto{\pgfqpoint{3.815777in}{1.843787in}}%
\pgfpathlineto{\pgfqpoint{3.817506in}{1.852152in}}%
\pgfpathlineto{\pgfqpoint{3.818083in}{1.850831in}}%
\pgfpathlineto{\pgfqpoint{3.820197in}{1.843417in}}%
\pgfpathlineto{\pgfqpoint{3.820389in}{1.844079in}}%
\pgfpathlineto{\pgfqpoint{3.820581in}{1.844461in}}%
\pgfpathlineto{\pgfqpoint{3.820773in}{1.843232in}}%
\pgfpathlineto{\pgfqpoint{3.821734in}{1.841479in}}%
\pgfpathlineto{\pgfqpoint{3.821158in}{1.844628in}}%
\pgfpathlineto{\pgfqpoint{3.821926in}{1.842665in}}%
\pgfpathlineto{\pgfqpoint{3.822118in}{1.845946in}}%
\pgfpathlineto{\pgfqpoint{3.823079in}{1.843676in}}%
\pgfpathlineto{\pgfqpoint{3.824617in}{1.836843in}}%
\pgfpathlineto{\pgfqpoint{3.825193in}{1.837259in}}%
\pgfpathlineto{\pgfqpoint{3.825385in}{1.838437in}}%
\pgfpathlineto{\pgfqpoint{3.825577in}{1.834239in}}%
\pgfpathlineto{\pgfqpoint{3.826346in}{1.834734in}}%
\pgfpathlineto{\pgfqpoint{3.827499in}{1.829536in}}%
\pgfpathlineto{\pgfqpoint{3.827691in}{1.829846in}}%
\pgfpathlineto{\pgfqpoint{3.830382in}{1.808260in}}%
\pgfpathlineto{\pgfqpoint{3.830766in}{1.810173in}}%
\pgfpathlineto{\pgfqpoint{3.831919in}{1.815117in}}%
\pgfpathlineto{\pgfqpoint{3.832111in}{1.813990in}}%
\pgfpathlineto{\pgfqpoint{3.832880in}{1.807291in}}%
\pgfpathlineto{\pgfqpoint{3.833648in}{1.809145in}}%
\pgfpathlineto{\pgfqpoint{3.834033in}{1.806238in}}%
\pgfpathlineto{\pgfqpoint{3.834417in}{1.808241in}}%
\pgfpathlineto{\pgfqpoint{3.835378in}{1.813463in}}%
\pgfpathlineto{\pgfqpoint{3.835762in}{1.812851in}}%
\pgfpathlineto{\pgfqpoint{3.837684in}{1.797241in}}%
\pgfpathlineto{\pgfqpoint{3.838260in}{1.799518in}}%
\pgfpathlineto{\pgfqpoint{3.838645in}{1.797602in}}%
\pgfpathlineto{\pgfqpoint{3.839029in}{1.803748in}}%
\pgfpathlineto{\pgfqpoint{3.839606in}{1.796580in}}%
\pgfpathlineto{\pgfqpoint{3.839798in}{1.793794in}}%
\pgfpathlineto{\pgfqpoint{3.840759in}{1.795955in}}%
\pgfpathlineto{\pgfqpoint{3.842680in}{1.807455in}}%
\pgfpathlineto{\pgfqpoint{3.842872in}{1.806422in}}%
\pgfpathlineto{\pgfqpoint{3.844602in}{1.785276in}}%
\pgfpathlineto{\pgfqpoint{3.844986in}{1.789814in}}%
\pgfpathlineto{\pgfqpoint{3.845947in}{1.788016in}}%
\pgfpathlineto{\pgfqpoint{3.846139in}{1.785748in}}%
\pgfpathlineto{\pgfqpoint{3.846716in}{1.788379in}}%
\pgfpathlineto{\pgfqpoint{3.846908in}{1.791459in}}%
\pgfpathlineto{\pgfqpoint{3.847485in}{1.784670in}}%
\pgfpathlineto{\pgfqpoint{3.847677in}{1.787884in}}%
\pgfpathlineto{\pgfqpoint{3.848253in}{1.781842in}}%
\pgfpathlineto{\pgfqpoint{3.848830in}{1.785241in}}%
\pgfpathlineto{\pgfqpoint{3.849214in}{1.784173in}}%
\pgfpathlineto{\pgfqpoint{3.850175in}{1.788618in}}%
\pgfpathlineto{\pgfqpoint{3.850944in}{1.792361in}}%
\pgfpathlineto{\pgfqpoint{3.851520in}{1.785135in}}%
\pgfpathlineto{\pgfqpoint{3.851904in}{1.786106in}}%
\pgfpathlineto{\pgfqpoint{3.852481in}{1.783771in}}%
\pgfpathlineto{\pgfqpoint{3.852865in}{1.786789in}}%
\pgfpathlineto{\pgfqpoint{3.853057in}{1.785995in}}%
\pgfpathlineto{\pgfqpoint{3.853250in}{1.788353in}}%
\pgfpathlineto{\pgfqpoint{3.854018in}{1.794583in}}%
\pgfpathlineto{\pgfqpoint{3.854403in}{1.790666in}}%
\pgfpathlineto{\pgfqpoint{3.855171in}{1.779764in}}%
\pgfpathlineto{\pgfqpoint{3.856132in}{1.780863in}}%
\pgfpathlineto{\pgfqpoint{3.858630in}{1.802024in}}%
\pgfpathlineto{\pgfqpoint{3.859399in}{1.802222in}}%
\pgfpathlineto{\pgfqpoint{3.859015in}{1.801613in}}%
\pgfpathlineto{\pgfqpoint{3.859591in}{1.801946in}}%
\pgfpathlineto{\pgfqpoint{3.859975in}{1.803419in}}%
\pgfpathlineto{\pgfqpoint{3.860168in}{1.800800in}}%
\pgfpathlineto{\pgfqpoint{3.861897in}{1.811008in}}%
\pgfpathlineto{\pgfqpoint{3.863434in}{1.802873in}}%
\pgfpathlineto{\pgfqpoint{3.863627in}{1.803203in}}%
\pgfpathlineto{\pgfqpoint{3.865164in}{1.812851in}}%
\pgfpathlineto{\pgfqpoint{3.865356in}{1.811229in}}%
\pgfpathlineto{\pgfqpoint{3.867086in}{1.800417in}}%
\pgfpathlineto{\pgfqpoint{3.867470in}{1.801684in}}%
\pgfpathlineto{\pgfqpoint{3.868239in}{1.799658in}}%
\pgfpathlineto{\pgfqpoint{3.868623in}{1.800064in}}%
\pgfpathlineto{\pgfqpoint{3.871313in}{1.779536in}}%
\pgfpathlineto{\pgfqpoint{3.872851in}{1.787887in}}%
\pgfpathlineto{\pgfqpoint{3.873043in}{1.784803in}}%
\pgfpathlineto{\pgfqpoint{3.873812in}{1.789855in}}%
\pgfpathlineto{\pgfqpoint{3.874772in}{1.786089in}}%
\pgfpathlineto{\pgfqpoint{3.874965in}{1.786346in}}%
\pgfpathlineto{\pgfqpoint{3.876886in}{1.797919in}}%
\pgfpathlineto{\pgfqpoint{3.877463in}{1.794636in}}%
\pgfpathlineto{\pgfqpoint{3.877655in}{1.792233in}}%
\pgfpathlineto{\pgfqpoint{3.878039in}{1.797699in}}%
\pgfpathlineto{\pgfqpoint{3.878231in}{1.795817in}}%
\pgfpathlineto{\pgfqpoint{3.878424in}{1.799402in}}%
\pgfpathlineto{\pgfqpoint{3.879192in}{1.797977in}}%
\pgfpathlineto{\pgfqpoint{3.879384in}{1.793610in}}%
\pgfpathlineto{\pgfqpoint{3.880153in}{1.797308in}}%
\pgfpathlineto{\pgfqpoint{3.880345in}{1.797948in}}%
\pgfpathlineto{\pgfqpoint{3.880537in}{1.797402in}}%
\pgfpathlineto{\pgfqpoint{3.881690in}{1.791699in}}%
\pgfpathlineto{\pgfqpoint{3.883036in}{1.800011in}}%
\pgfpathlineto{\pgfqpoint{3.883228in}{1.799980in}}%
\pgfpathlineto{\pgfqpoint{3.884957in}{1.806484in}}%
\pgfpathlineto{\pgfqpoint{3.887071in}{1.794513in}}%
\pgfpathlineto{\pgfqpoint{3.887455in}{1.796630in}}%
\pgfpathlineto{\pgfqpoint{3.887840in}{1.793118in}}%
\pgfpathlineto{\pgfqpoint{3.888032in}{1.793749in}}%
\pgfpathlineto{\pgfqpoint{3.888416in}{1.789123in}}%
\pgfpathlineto{\pgfqpoint{3.889185in}{1.792373in}}%
\pgfpathlineto{\pgfqpoint{3.889761in}{1.798956in}}%
\pgfpathlineto{\pgfqpoint{3.890146in}{1.791537in}}%
\pgfpathlineto{\pgfqpoint{3.890722in}{1.792677in}}%
\pgfpathlineto{\pgfqpoint{3.891491in}{1.784429in}}%
\pgfpathlineto{\pgfqpoint{3.891683in}{1.786519in}}%
\pgfpathlineto{\pgfqpoint{3.891875in}{1.783587in}}%
\pgfpathlineto{\pgfqpoint{3.892452in}{1.784207in}}%
\pgfpathlineto{\pgfqpoint{3.892644in}{1.782228in}}%
\pgfpathlineto{\pgfqpoint{3.893220in}{1.786803in}}%
\pgfpathlineto{\pgfqpoint{3.893413in}{1.785266in}}%
\pgfpathlineto{\pgfqpoint{3.893797in}{1.790098in}}%
\pgfpathlineto{\pgfqpoint{3.894374in}{1.788519in}}%
\pgfpathlineto{\pgfqpoint{3.894566in}{1.785563in}}%
\pgfpathlineto{\pgfqpoint{3.894758in}{1.789134in}}%
\pgfpathlineto{\pgfqpoint{3.895527in}{1.787577in}}%
\pgfpathlineto{\pgfqpoint{3.896295in}{1.790560in}}%
\pgfpathlineto{\pgfqpoint{3.896487in}{1.786282in}}%
\pgfpathlineto{\pgfqpoint{3.897448in}{1.787029in}}%
\pgfpathlineto{\pgfqpoint{3.898025in}{1.783071in}}%
\pgfpathlineto{\pgfqpoint{3.898986in}{1.784027in}}%
\pgfpathlineto{\pgfqpoint{3.900523in}{1.796306in}}%
\pgfpathlineto{\pgfqpoint{3.900715in}{1.795529in}}%
\pgfpathlineto{\pgfqpoint{3.901099in}{1.792241in}}%
\pgfpathlineto{\pgfqpoint{3.901676in}{1.793550in}}%
\pgfpathlineto{\pgfqpoint{3.901868in}{1.795636in}}%
\pgfpathlineto{\pgfqpoint{3.902445in}{1.793154in}}%
\pgfpathlineto{\pgfqpoint{3.903982in}{1.785759in}}%
\pgfpathlineto{\pgfqpoint{3.903021in}{1.793584in}}%
\pgfpathlineto{\pgfqpoint{3.904174in}{1.786051in}}%
\pgfpathlineto{\pgfqpoint{3.904751in}{1.793755in}}%
\pgfpathlineto{\pgfqpoint{3.905519in}{1.791918in}}%
\pgfpathlineto{\pgfqpoint{3.905711in}{1.793406in}}%
\pgfpathlineto{\pgfqpoint{3.906096in}{1.790985in}}%
\pgfpathlineto{\pgfqpoint{3.906672in}{1.785375in}}%
\pgfpathlineto{\pgfqpoint{3.907249in}{1.789743in}}%
\pgfpathlineto{\pgfqpoint{3.908017in}{1.798797in}}%
\pgfpathlineto{\pgfqpoint{3.908210in}{1.803806in}}%
\pgfpathlineto{\pgfqpoint{3.909170in}{1.800322in}}%
\pgfpathlineto{\pgfqpoint{3.909363in}{1.800164in}}%
\pgfpathlineto{\pgfqpoint{3.910131in}{1.790176in}}%
\pgfpathlineto{\pgfqpoint{3.910900in}{1.793781in}}%
\pgfpathlineto{\pgfqpoint{3.911669in}{1.801563in}}%
\pgfpathlineto{\pgfqpoint{3.912053in}{1.796011in}}%
\pgfpathlineto{\pgfqpoint{3.913206in}{1.789711in}}%
\pgfpathlineto{\pgfqpoint{3.913398in}{1.785876in}}%
\pgfpathlineto{\pgfqpoint{3.914167in}{1.789345in}}%
\pgfpathlineto{\pgfqpoint{3.914359in}{1.789507in}}%
\pgfpathlineto{\pgfqpoint{3.914551in}{1.792120in}}%
\pgfpathlineto{\pgfqpoint{3.915512in}{1.791748in}}%
\pgfpathlineto{\pgfqpoint{3.916857in}{1.785022in}}%
\pgfpathlineto{\pgfqpoint{3.917049in}{1.785052in}}%
\pgfpathlineto{\pgfqpoint{3.917626in}{1.788334in}}%
\pgfpathlineto{\pgfqpoint{3.917818in}{1.785615in}}%
\pgfpathlineto{\pgfqpoint{3.918010in}{1.782663in}}%
\pgfpathlineto{\pgfqpoint{3.918587in}{1.786956in}}%
\pgfpathlineto{\pgfqpoint{3.919355in}{1.785490in}}%
\pgfpathlineto{\pgfqpoint{3.919932in}{1.789890in}}%
\pgfpathlineto{\pgfqpoint{3.920124in}{1.789009in}}%
\pgfpathlineto{\pgfqpoint{3.920316in}{1.791734in}}%
\pgfpathlineto{\pgfqpoint{3.920508in}{1.792653in}}%
\pgfpathlineto{\pgfqpoint{3.920893in}{1.791869in}}%
\pgfpathlineto{\pgfqpoint{3.922046in}{1.779746in}}%
\pgfpathlineto{\pgfqpoint{3.922430in}{1.779784in}}%
\pgfpathlineto{\pgfqpoint{3.923199in}{1.779257in}}%
\pgfpathlineto{\pgfqpoint{3.923391in}{1.781769in}}%
\pgfpathlineto{\pgfqpoint{3.925120in}{1.790514in}}%
\pgfpathlineto{\pgfqpoint{3.925313in}{1.789161in}}%
\pgfpathlineto{\pgfqpoint{3.925505in}{1.794947in}}%
\pgfpathlineto{\pgfqpoint{3.925697in}{1.792712in}}%
\pgfpathlineto{\pgfqpoint{3.927042in}{1.796133in}}%
\pgfpathlineto{\pgfqpoint{3.928003in}{1.790302in}}%
\pgfpathlineto{\pgfqpoint{3.928387in}{1.790395in}}%
\pgfpathlineto{\pgfqpoint{3.928579in}{1.792758in}}%
\pgfpathlineto{\pgfqpoint{3.929348in}{1.791551in}}%
\pgfpathlineto{\pgfqpoint{3.929732in}{1.791850in}}%
\pgfpathlineto{\pgfqpoint{3.930885in}{1.787857in}}%
\pgfpathlineto{\pgfqpoint{3.931462in}{1.791431in}}%
\pgfpathlineto{\pgfqpoint{3.932231in}{1.791019in}}%
\pgfpathlineto{\pgfqpoint{3.932615in}{1.789500in}}%
\pgfpathlineto{\pgfqpoint{3.932999in}{1.792786in}}%
\pgfpathlineto{\pgfqpoint{3.933960in}{1.799990in}}%
\pgfpathlineto{\pgfqpoint{3.933576in}{1.792413in}}%
\pgfpathlineto{\pgfqpoint{3.934344in}{1.796241in}}%
\pgfpathlineto{\pgfqpoint{3.935497in}{1.801374in}}%
\pgfpathlineto{\pgfqpoint{3.935882in}{1.798993in}}%
\pgfpathlineto{\pgfqpoint{3.936266in}{1.794229in}}%
\pgfpathlineto{\pgfqpoint{3.937227in}{1.795557in}}%
\pgfpathlineto{\pgfqpoint{3.938572in}{1.803437in}}%
\pgfpathlineto{\pgfqpoint{3.938956in}{1.799527in}}%
\pgfpathlineto{\pgfqpoint{3.939533in}{1.795883in}}%
\pgfpathlineto{\pgfqpoint{3.939917in}{1.798236in}}%
\pgfpathlineto{\pgfqpoint{3.940109in}{1.800192in}}%
\pgfpathlineto{\pgfqpoint{3.940494in}{1.796720in}}%
\pgfpathlineto{\pgfqpoint{3.940878in}{1.797872in}}%
\pgfpathlineto{\pgfqpoint{3.941262in}{1.798931in}}%
\pgfpathlineto{\pgfqpoint{3.941455in}{1.797378in}}%
\pgfpathlineto{\pgfqpoint{3.942031in}{1.794685in}}%
\pgfpathlineto{\pgfqpoint{3.942608in}{1.797132in}}%
\pgfpathlineto{\pgfqpoint{3.942992in}{1.798708in}}%
\pgfpathlineto{\pgfqpoint{3.943569in}{1.804814in}}%
\pgfpathlineto{\pgfqpoint{3.944145in}{1.803620in}}%
\pgfpathlineto{\pgfqpoint{3.944529in}{1.803292in}}%
\pgfpathlineto{\pgfqpoint{3.946067in}{1.807588in}}%
\pgfpathlineto{\pgfqpoint{3.946643in}{1.803780in}}%
\pgfpathlineto{\pgfqpoint{3.947220in}{1.806236in}}%
\pgfpathlineto{\pgfqpoint{3.947412in}{1.806375in}}%
\pgfpathlineto{\pgfqpoint{3.947604in}{1.804930in}}%
\pgfpathlineto{\pgfqpoint{3.948181in}{1.808881in}}%
\pgfpathlineto{\pgfqpoint{3.948565in}{1.810551in}}%
\pgfpathlineto{\pgfqpoint{3.948949in}{1.807401in}}%
\pgfpathlineto{\pgfqpoint{3.949141in}{1.807306in}}%
\pgfpathlineto{\pgfqpoint{3.949526in}{1.809874in}}%
\pgfpathlineto{\pgfqpoint{3.949910in}{1.805886in}}%
\pgfpathlineto{\pgfqpoint{3.950294in}{1.809448in}}%
\pgfpathlineto{\pgfqpoint{3.952024in}{1.794930in}}%
\pgfpathlineto{\pgfqpoint{3.952600in}{1.799149in}}%
\pgfpathlineto{\pgfqpoint{3.952985in}{1.799065in}}%
\pgfpathlineto{\pgfqpoint{3.953753in}{1.802519in}}%
\pgfpathlineto{\pgfqpoint{3.954138in}{1.797797in}}%
\pgfpathlineto{\pgfqpoint{3.954714in}{1.799643in}}%
\pgfpathlineto{\pgfqpoint{3.956252in}{1.809828in}}%
\pgfpathlineto{\pgfqpoint{3.956636in}{1.806671in}}%
\pgfpathlineto{\pgfqpoint{3.957020in}{1.810454in}}%
\pgfpathlineto{\pgfqpoint{3.957212in}{1.809691in}}%
\pgfpathlineto{\pgfqpoint{3.959326in}{1.822632in}}%
\pgfpathlineto{\pgfqpoint{3.959711in}{1.821308in}}%
\pgfpathlineto{\pgfqpoint{3.960671in}{1.813514in}}%
\pgfpathlineto{\pgfqpoint{3.960864in}{1.816453in}}%
\pgfpathlineto{\pgfqpoint{3.962593in}{1.820123in}}%
\pgfpathlineto{\pgfqpoint{3.963362in}{1.816470in}}%
\pgfpathlineto{\pgfqpoint{3.963746in}{1.819219in}}%
\pgfpathlineto{\pgfqpoint{3.964707in}{1.827015in}}%
\pgfpathlineto{\pgfqpoint{3.964899in}{1.824672in}}%
\pgfpathlineto{\pgfqpoint{3.965283in}{1.821297in}}%
\pgfpathlineto{\pgfqpoint{3.966052in}{1.822024in}}%
\pgfpathlineto{\pgfqpoint{3.967590in}{1.831281in}}%
\pgfpathlineto{\pgfqpoint{3.968358in}{1.836195in}}%
\pgfpathlineto{\pgfqpoint{3.968743in}{1.834579in}}%
\pgfpathlineto{\pgfqpoint{3.969896in}{1.829482in}}%
\pgfpathlineto{\pgfqpoint{3.970280in}{1.825650in}}%
\pgfpathlineto{\pgfqpoint{3.971049in}{1.827065in}}%
\pgfpathlineto{\pgfqpoint{3.971433in}{1.831570in}}%
\pgfpathlineto{\pgfqpoint{3.972202in}{1.828479in}}%
\pgfpathlineto{\pgfqpoint{3.973355in}{1.838665in}}%
\pgfpathlineto{\pgfqpoint{3.974123in}{1.837629in}}%
\pgfpathlineto{\pgfqpoint{3.974892in}{1.835313in}}%
\pgfpathlineto{\pgfqpoint{3.975084in}{1.838148in}}%
\pgfpathlineto{\pgfqpoint{3.975853in}{1.840901in}}%
\pgfpathlineto{\pgfqpoint{3.976237in}{1.840015in}}%
\pgfpathlineto{\pgfqpoint{3.977006in}{1.835188in}}%
\pgfpathlineto{\pgfqpoint{3.976621in}{1.840312in}}%
\pgfpathlineto{\pgfqpoint{3.977582in}{1.836674in}}%
\pgfpathlineto{\pgfqpoint{3.978351in}{1.843730in}}%
\pgfpathlineto{\pgfqpoint{3.978735in}{1.840911in}}%
\pgfpathlineto{\pgfqpoint{3.979504in}{1.835067in}}%
\pgfpathlineto{\pgfqpoint{3.980273in}{1.835873in}}%
\pgfpathlineto{\pgfqpoint{3.981233in}{1.841177in}}%
\pgfpathlineto{\pgfqpoint{3.981426in}{1.839390in}}%
\pgfpathlineto{\pgfqpoint{3.982002in}{1.833045in}}%
\pgfpathlineto{\pgfqpoint{3.982579in}{1.838553in}}%
\pgfpathlineto{\pgfqpoint{3.983732in}{1.844714in}}%
\pgfpathlineto{\pgfqpoint{3.983924in}{1.842487in}}%
\pgfpathlineto{\pgfqpoint{3.984308in}{1.839177in}}%
\pgfpathlineto{\pgfqpoint{3.984692in}{1.843647in}}%
\pgfpathlineto{\pgfqpoint{3.985077in}{1.841261in}}%
\pgfpathlineto{\pgfqpoint{3.985461in}{1.840087in}}%
\pgfpathlineto{\pgfqpoint{3.986038in}{1.842133in}}%
\pgfpathlineto{\pgfqpoint{3.986230in}{1.840053in}}%
\pgfpathlineto{\pgfqpoint{3.986806in}{1.844085in}}%
\pgfpathlineto{\pgfqpoint{3.987191in}{1.840595in}}%
\pgfpathlineto{\pgfqpoint{3.987575in}{1.840008in}}%
\pgfpathlineto{\pgfqpoint{3.989112in}{1.851328in}}%
\pgfpathlineto{\pgfqpoint{3.990650in}{1.847265in}}%
\pgfpathlineto{\pgfqpoint{3.989497in}{1.851994in}}%
\pgfpathlineto{\pgfqpoint{3.990842in}{1.847676in}}%
\pgfpathlineto{\pgfqpoint{3.991803in}{1.843597in}}%
\pgfpathlineto{\pgfqpoint{3.992956in}{1.829511in}}%
\pgfpathlineto{\pgfqpoint{3.993340in}{1.831765in}}%
\pgfpathlineto{\pgfqpoint{3.994877in}{1.847327in}}%
\pgfpathlineto{\pgfqpoint{3.995070in}{1.846284in}}%
\pgfpathlineto{\pgfqpoint{3.996030in}{1.835243in}}%
\pgfpathlineto{\pgfqpoint{3.997376in}{1.837551in}}%
\pgfpathlineto{\pgfqpoint{3.997760in}{1.838637in}}%
\pgfpathlineto{\pgfqpoint{3.998144in}{1.835687in}}%
\pgfpathlineto{\pgfqpoint{3.998336in}{1.834440in}}%
\pgfpathlineto{\pgfqpoint{3.998529in}{1.836042in}}%
\pgfpathlineto{\pgfqpoint{4.001027in}{1.862966in}}%
\pgfpathlineto{\pgfqpoint{4.003141in}{1.855302in}}%
\pgfpathlineto{\pgfqpoint{4.003333in}{1.856026in}}%
\pgfpathlineto{\pgfqpoint{4.003909in}{1.859295in}}%
\pgfpathlineto{\pgfqpoint{4.004101in}{1.855609in}}%
\pgfpathlineto{\pgfqpoint{4.004486in}{1.856649in}}%
\pgfpathlineto{\pgfqpoint{4.005831in}{1.850533in}}%
\pgfpathlineto{\pgfqpoint{4.004870in}{1.857938in}}%
\pgfpathlineto{\pgfqpoint{4.006215in}{1.854041in}}%
\pgfpathlineto{\pgfqpoint{4.006600in}{1.861792in}}%
\pgfpathlineto{\pgfqpoint{4.007560in}{1.861439in}}%
\pgfpathlineto{\pgfqpoint{4.009290in}{1.851364in}}%
\pgfpathlineto{\pgfqpoint{4.009674in}{1.854099in}}%
\pgfpathlineto{\pgfqpoint{4.009866in}{1.855704in}}%
\pgfpathlineto{\pgfqpoint{4.010059in}{1.850019in}}%
\pgfpathlineto{\pgfqpoint{4.010443in}{1.848728in}}%
\pgfpathlineto{\pgfqpoint{4.010827in}{1.850436in}}%
\pgfpathlineto{\pgfqpoint{4.011019in}{1.850358in}}%
\pgfpathlineto{\pgfqpoint{4.012941in}{1.862874in}}%
\pgfpathlineto{\pgfqpoint{4.013325in}{1.852636in}}%
\pgfpathlineto{\pgfqpoint{4.014094in}{1.857846in}}%
\pgfpathlineto{\pgfqpoint{4.015247in}{1.863458in}}%
\pgfpathlineto{\pgfqpoint{4.014478in}{1.856671in}}%
\pgfpathlineto{\pgfqpoint{4.015439in}{1.860354in}}%
\pgfpathlineto{\pgfqpoint{4.015824in}{1.861245in}}%
\pgfpathlineto{\pgfqpoint{4.016016in}{1.857417in}}%
\pgfpathlineto{\pgfqpoint{4.016785in}{1.863263in}}%
\pgfpathlineto{\pgfqpoint{4.016977in}{1.862864in}}%
\pgfpathlineto{\pgfqpoint{4.017169in}{1.859787in}}%
\pgfpathlineto{\pgfqpoint{4.017938in}{1.859906in}}%
\pgfpathlineto{\pgfqpoint{4.019475in}{1.867793in}}%
\pgfpathlineto{\pgfqpoint{4.019859in}{1.865894in}}%
\pgfpathlineto{\pgfqpoint{4.021589in}{1.856413in}}%
\pgfpathlineto{\pgfqpoint{4.021781in}{1.858215in}}%
\pgfpathlineto{\pgfqpoint{4.023895in}{1.869137in}}%
\pgfpathlineto{\pgfqpoint{4.024279in}{1.867240in}}%
\pgfpathlineto{\pgfqpoint{4.024663in}{1.867109in}}%
\pgfpathlineto{\pgfqpoint{4.025432in}{1.869182in}}%
\pgfpathlineto{\pgfqpoint{4.025624in}{1.866965in}}%
\pgfpathlineto{\pgfqpoint{4.026009in}{1.869824in}}%
\pgfpathlineto{\pgfqpoint{4.026201in}{1.869511in}}%
\pgfpathlineto{\pgfqpoint{4.027546in}{1.875626in}}%
\pgfpathlineto{\pgfqpoint{4.027930in}{1.873717in}}%
\pgfpathlineto{\pgfqpoint{4.028315in}{1.869740in}}%
\pgfpathlineto{\pgfqpoint{4.028891in}{1.872733in}}%
\pgfpathlineto{\pgfqpoint{4.029083in}{1.873321in}}%
\pgfpathlineto{\pgfqpoint{4.029275in}{1.871352in}}%
\pgfpathlineto{\pgfqpoint{4.029660in}{1.868720in}}%
\pgfpathlineto{\pgfqpoint{4.030044in}{1.875014in}}%
\pgfpathlineto{\pgfqpoint{4.031389in}{1.878165in}}%
\pgfpathlineto{\pgfqpoint{4.032158in}{1.868907in}}%
\pgfpathlineto{\pgfqpoint{4.032734in}{1.875134in}}%
\pgfpathlineto{\pgfqpoint{4.033311in}{1.868371in}}%
\pgfpathlineto{\pgfqpoint{4.034272in}{1.871701in}}%
\pgfpathlineto{\pgfqpoint{4.034464in}{1.876447in}}%
\pgfpathlineto{\pgfqpoint{4.035233in}{1.869115in}}%
\pgfpathlineto{\pgfqpoint{4.036001in}{1.861242in}}%
\pgfpathlineto{\pgfqpoint{4.036386in}{1.863380in}}%
\pgfpathlineto{\pgfqpoint{4.036962in}{1.867110in}}%
\pgfpathlineto{\pgfqpoint{4.037346in}{1.863984in}}%
\pgfpathlineto{\pgfqpoint{4.037539in}{1.863282in}}%
\pgfpathlineto{\pgfqpoint{4.037731in}{1.864404in}}%
\pgfpathlineto{\pgfqpoint{4.038115in}{1.868098in}}%
\pgfpathlineto{\pgfqpoint{4.039076in}{1.867252in}}%
\pgfpathlineto{\pgfqpoint{4.040421in}{1.862262in}}%
\pgfpathlineto{\pgfqpoint{4.041382in}{1.866786in}}%
\pgfpathlineto{\pgfqpoint{4.041574in}{1.865105in}}%
\pgfpathlineto{\pgfqpoint{4.042919in}{1.859177in}}%
\pgfpathlineto{\pgfqpoint{4.043304in}{1.855953in}}%
\pgfpathlineto{\pgfqpoint{4.043688in}{1.856267in}}%
\pgfpathlineto{\pgfqpoint{4.045033in}{1.865432in}}%
\pgfpathlineto{\pgfqpoint{4.047147in}{1.852776in}}%
\pgfpathlineto{\pgfqpoint{4.047339in}{1.853472in}}%
\pgfpathlineto{\pgfqpoint{4.048300in}{1.851459in}}%
\pgfpathlineto{\pgfqpoint{4.049453in}{1.859449in}}%
\pgfpathlineto{\pgfqpoint{4.049645in}{1.857289in}}%
\pgfpathlineto{\pgfqpoint{4.050606in}{1.846356in}}%
\pgfpathlineto{\pgfqpoint{4.051183in}{1.847495in}}%
\pgfpathlineto{\pgfqpoint{4.052528in}{1.851227in}}%
\pgfpathlineto{\pgfqpoint{4.052912in}{1.850399in}}%
\pgfpathlineto{\pgfqpoint{4.056371in}{1.828131in}}%
\pgfpathlineto{\pgfqpoint{4.056948in}{1.828858in}}%
\pgfpathlineto{\pgfqpoint{4.057332in}{1.829901in}}%
\pgfpathlineto{\pgfqpoint{4.057524in}{1.827689in}}%
\pgfpathlineto{\pgfqpoint{4.057716in}{1.827805in}}%
\pgfpathlineto{\pgfqpoint{4.059446in}{1.815699in}}%
\pgfpathlineto{\pgfqpoint{4.058101in}{1.828259in}}%
\pgfpathlineto{\pgfqpoint{4.059830in}{1.820927in}}%
\pgfpathlineto{\pgfqpoint{4.060407in}{1.825047in}}%
\pgfpathlineto{\pgfqpoint{4.060791in}{1.822336in}}%
\pgfpathlineto{\pgfqpoint{4.061175in}{1.818585in}}%
\pgfpathlineto{\pgfqpoint{4.061752in}{1.820830in}}%
\pgfpathlineto{\pgfqpoint{4.063097in}{1.827518in}}%
\pgfpathlineto{\pgfqpoint{4.063866in}{1.825304in}}%
\pgfpathlineto{\pgfqpoint{4.064250in}{1.822788in}}%
\pgfpathlineto{\pgfqpoint{4.064634in}{1.824964in}}%
\pgfpathlineto{\pgfqpoint{4.065211in}{1.833900in}}%
\pgfpathlineto{\pgfqpoint{4.065787in}{1.831911in}}%
\pgfpathlineto{\pgfqpoint{4.066940in}{1.823771in}}%
\pgfpathlineto{\pgfqpoint{4.067133in}{1.824104in}}%
\pgfpathlineto{\pgfqpoint{4.069439in}{1.838763in}}%
\pgfpathlineto{\pgfqpoint{4.069631in}{1.837122in}}%
\pgfpathlineto{\pgfqpoint{4.069823in}{1.833085in}}%
\pgfpathlineto{\pgfqpoint{4.070399in}{1.839595in}}%
\pgfpathlineto{\pgfqpoint{4.070592in}{1.839203in}}%
\pgfpathlineto{\pgfqpoint{4.072898in}{1.822478in}}%
\pgfpathlineto{\pgfqpoint{4.073090in}{1.824514in}}%
\pgfpathlineto{\pgfqpoint{4.073282in}{1.824460in}}%
\pgfpathlineto{\pgfqpoint{4.073858in}{1.815467in}}%
\pgfpathlineto{\pgfqpoint{4.074435in}{1.820722in}}%
\pgfpathlineto{\pgfqpoint{4.074819in}{1.825956in}}%
\pgfpathlineto{\pgfqpoint{4.075396in}{1.820339in}}%
\pgfpathlineto{\pgfqpoint{4.075588in}{1.822276in}}%
\pgfpathlineto{\pgfqpoint{4.076933in}{1.830454in}}%
\pgfpathlineto{\pgfqpoint{4.077317in}{1.825899in}}%
\pgfpathlineto{\pgfqpoint{4.077702in}{1.822761in}}%
\pgfpathlineto{\pgfqpoint{4.078086in}{1.828577in}}%
\pgfpathlineto{\pgfqpoint{4.078278in}{1.825662in}}%
\pgfpathlineto{\pgfqpoint{4.078855in}{1.824374in}}%
\pgfpathlineto{\pgfqpoint{4.079239in}{1.828092in}}%
\pgfpathlineto{\pgfqpoint{4.080200in}{1.822071in}}%
\pgfpathlineto{\pgfqpoint{4.080392in}{1.824231in}}%
\pgfpathlineto{\pgfqpoint{4.080584in}{1.826169in}}%
\pgfpathlineto{\pgfqpoint{4.080969in}{1.820874in}}%
\pgfpathlineto{\pgfqpoint{4.081161in}{1.821262in}}%
\pgfpathlineto{\pgfqpoint{4.081737in}{1.815027in}}%
\pgfpathlineto{\pgfqpoint{4.082506in}{1.817654in}}%
\pgfpathlineto{\pgfqpoint{4.082698in}{1.817093in}}%
\pgfpathlineto{\pgfqpoint{4.083659in}{1.823071in}}%
\pgfpathlineto{\pgfqpoint{4.083851in}{1.821888in}}%
\pgfpathlineto{\pgfqpoint{4.084812in}{1.806713in}}%
\pgfpathlineto{\pgfqpoint{4.085581in}{1.806858in}}%
\pgfpathlineto{\pgfqpoint{4.088655in}{1.824523in}}%
\pgfpathlineto{\pgfqpoint{4.088848in}{1.824140in}}%
\pgfpathlineto{\pgfqpoint{4.090961in}{1.807900in}}%
\pgfpathlineto{\pgfqpoint{4.091154in}{1.810878in}}%
\pgfpathlineto{\pgfqpoint{4.091346in}{1.809744in}}%
\pgfpathlineto{\pgfqpoint{4.091922in}{1.812539in}}%
\pgfpathlineto{\pgfqpoint{4.093267in}{1.819079in}}%
\pgfpathlineto{\pgfqpoint{4.093460in}{1.818758in}}%
\pgfpathlineto{\pgfqpoint{4.093652in}{1.814761in}}%
\pgfpathlineto{\pgfqpoint{4.094613in}{1.815318in}}%
\pgfpathlineto{\pgfqpoint{4.095573in}{1.828037in}}%
\pgfpathlineto{\pgfqpoint{4.095958in}{1.826865in}}%
\pgfpathlineto{\pgfqpoint{4.097303in}{1.821930in}}%
\pgfpathlineto{\pgfqpoint{4.097495in}{1.821963in}}%
\pgfpathlineto{\pgfqpoint{4.097687in}{1.822631in}}%
\pgfpathlineto{\pgfqpoint{4.097879in}{1.820698in}}%
\pgfpathlineto{\pgfqpoint{4.099225in}{1.813669in}}%
\pgfpathlineto{\pgfqpoint{4.099417in}{1.817140in}}%
\pgfpathlineto{\pgfqpoint{4.100378in}{1.816429in}}%
\pgfpathlineto{\pgfqpoint{4.100762in}{1.815967in}}%
\pgfpathlineto{\pgfqpoint{4.101146in}{1.816634in}}%
\pgfpathlineto{\pgfqpoint{4.101338in}{1.815668in}}%
\pgfpathlineto{\pgfqpoint{4.102684in}{1.799983in}}%
\pgfpathlineto{\pgfqpoint{4.102876in}{1.799316in}}%
\pgfpathlineto{\pgfqpoint{4.103260in}{1.801029in}}%
\pgfpathlineto{\pgfqpoint{4.103452in}{1.801586in}}%
\pgfpathlineto{\pgfqpoint{4.103837in}{1.800499in}}%
\pgfpathlineto{\pgfqpoint{4.104605in}{1.801533in}}%
\pgfpathlineto{\pgfqpoint{4.105182in}{1.797162in}}%
\pgfpathlineto{\pgfqpoint{4.105950in}{1.800835in}}%
\pgfpathlineto{\pgfqpoint{4.106143in}{1.799852in}}%
\pgfpathlineto{\pgfqpoint{4.106335in}{1.794893in}}%
\pgfpathlineto{\pgfqpoint{4.107296in}{1.798128in}}%
\pgfpathlineto{\pgfqpoint{4.107680in}{1.802051in}}%
\pgfpathlineto{\pgfqpoint{4.108256in}{1.798793in}}%
\pgfpathlineto{\pgfqpoint{4.108449in}{1.797733in}}%
\pgfpathlineto{\pgfqpoint{4.108833in}{1.800765in}}%
\pgfpathlineto{\pgfqpoint{4.109986in}{1.809470in}}%
\pgfpathlineto{\pgfqpoint{4.110370in}{1.806884in}}%
\pgfpathlineto{\pgfqpoint{4.111715in}{1.801115in}}%
\pgfpathlineto{\pgfqpoint{4.111908in}{1.802244in}}%
\pgfpathlineto{\pgfqpoint{4.113829in}{1.808092in}}%
\pgfpathlineto{\pgfqpoint{4.114214in}{1.806800in}}%
\pgfpathlineto{\pgfqpoint{4.114406in}{1.805416in}}%
\pgfpathlineto{\pgfqpoint{4.114982in}{1.807119in}}%
\pgfpathlineto{\pgfqpoint{4.116135in}{1.813499in}}%
\pgfpathlineto{\pgfqpoint{4.116520in}{1.808995in}}%
\pgfpathlineto{\pgfqpoint{4.117096in}{1.813073in}}%
\pgfpathlineto{\pgfqpoint{4.118249in}{1.819799in}}%
\pgfpathlineto{\pgfqpoint{4.119402in}{1.813734in}}%
\pgfpathlineto{\pgfqpoint{4.120363in}{1.819169in}}%
\pgfpathlineto{\pgfqpoint{4.120747in}{1.817681in}}%
\pgfpathlineto{\pgfqpoint{4.120940in}{1.818443in}}%
\pgfpathlineto{\pgfqpoint{4.121132in}{1.817330in}}%
\pgfpathlineto{\pgfqpoint{4.121516in}{1.813079in}}%
\pgfpathlineto{\pgfqpoint{4.122477in}{1.813581in}}%
\pgfpathlineto{\pgfqpoint{4.123630in}{1.810539in}}%
\pgfpathlineto{\pgfqpoint{4.124399in}{1.816705in}}%
\pgfpathlineto{\pgfqpoint{4.124783in}{1.813703in}}%
\pgfpathlineto{\pgfqpoint{4.125359in}{1.813076in}}%
\pgfpathlineto{\pgfqpoint{4.125936in}{1.815767in}}%
\pgfpathlineto{\pgfqpoint{4.126320in}{1.816502in}}%
\pgfpathlineto{\pgfqpoint{4.127473in}{1.810294in}}%
\pgfpathlineto{\pgfqpoint{4.128434in}{1.814184in}}%
\pgfpathlineto{\pgfqpoint{4.128050in}{1.809007in}}%
\pgfpathlineto{\pgfqpoint{4.128626in}{1.811490in}}%
\pgfpathlineto{\pgfqpoint{4.129011in}{1.812993in}}%
\pgfpathlineto{\pgfqpoint{4.129587in}{1.811488in}}%
\pgfpathlineto{\pgfqpoint{4.129971in}{1.810860in}}%
\pgfpathlineto{\pgfqpoint{4.130164in}{1.811840in}}%
\pgfpathlineto{\pgfqpoint{4.131701in}{1.820758in}}%
\pgfpathlineto{\pgfqpoint{4.131893in}{1.817261in}}%
\pgfpathlineto{\pgfqpoint{4.132085in}{1.815629in}}%
\pgfpathlineto{\pgfqpoint{4.132662in}{1.819799in}}%
\pgfpathlineto{\pgfqpoint{4.133238in}{1.818770in}}%
\pgfpathlineto{\pgfqpoint{4.133046in}{1.820160in}}%
\pgfpathlineto{\pgfqpoint{4.133430in}{1.819881in}}%
\pgfpathlineto{\pgfqpoint{4.133815in}{1.825937in}}%
\pgfpathlineto{\pgfqpoint{4.134583in}{1.821625in}}%
\pgfpathlineto{\pgfqpoint{4.135160in}{1.822375in}}%
\pgfpathlineto{\pgfqpoint{4.134968in}{1.820513in}}%
\pgfpathlineto{\pgfqpoint{4.135352in}{1.821307in}}%
\pgfpathlineto{\pgfqpoint{4.135929in}{1.817885in}}%
\pgfpathlineto{\pgfqpoint{4.136313in}{1.821660in}}%
\pgfpathlineto{\pgfqpoint{4.137082in}{1.826228in}}%
\pgfpathlineto{\pgfqpoint{4.137466in}{1.824317in}}%
\pgfpathlineto{\pgfqpoint{4.137658in}{1.823982in}}%
\pgfpathlineto{\pgfqpoint{4.137850in}{1.825864in}}%
\pgfpathlineto{\pgfqpoint{4.138811in}{1.831675in}}%
\pgfpathlineto{\pgfqpoint{4.139003in}{1.830433in}}%
\pgfpathlineto{\pgfqpoint{4.139196in}{1.827225in}}%
\pgfpathlineto{\pgfqpoint{4.139964in}{1.829866in}}%
\pgfpathlineto{\pgfqpoint{4.140733in}{1.832090in}}%
\pgfpathlineto{\pgfqpoint{4.140925in}{1.830941in}}%
\pgfpathlineto{\pgfqpoint{4.142270in}{1.825107in}}%
\pgfpathlineto{\pgfqpoint{4.141502in}{1.831810in}}%
\pgfpathlineto{\pgfqpoint{4.142462in}{1.826002in}}%
\pgfpathlineto{\pgfqpoint{4.142655in}{1.829422in}}%
\pgfpathlineto{\pgfqpoint{4.143231in}{1.823172in}}%
\pgfpathlineto{\pgfqpoint{4.143423in}{1.823616in}}%
\pgfpathlineto{\pgfqpoint{4.145921in}{1.807057in}}%
\pgfpathlineto{\pgfqpoint{4.146114in}{1.808655in}}%
\pgfpathlineto{\pgfqpoint{4.146498in}{1.803570in}}%
\pgfpathlineto{\pgfqpoint{4.146690in}{1.803952in}}%
\pgfpathlineto{\pgfqpoint{4.146882in}{1.802757in}}%
\pgfpathlineto{\pgfqpoint{4.147074in}{1.799644in}}%
\pgfpathlineto{\pgfqpoint{4.147651in}{1.807174in}}%
\pgfpathlineto{\pgfqpoint{4.148035in}{1.805955in}}%
\pgfpathlineto{\pgfqpoint{4.148227in}{1.808324in}}%
\pgfpathlineto{\pgfqpoint{4.148612in}{1.809014in}}%
\pgfpathlineto{\pgfqpoint{4.149380in}{1.805907in}}%
\pgfpathlineto{\pgfqpoint{4.149573in}{1.809963in}}%
\pgfpathlineto{\pgfqpoint{4.149765in}{1.808743in}}%
\pgfpathlineto{\pgfqpoint{4.150149in}{1.811995in}}%
\pgfpathlineto{\pgfqpoint{4.150533in}{1.810792in}}%
\pgfpathlineto{\pgfqpoint{4.150726in}{1.811472in}}%
\pgfpathlineto{\pgfqpoint{4.150918in}{1.807802in}}%
\pgfpathlineto{\pgfqpoint{4.151686in}{1.809029in}}%
\pgfpathlineto{\pgfqpoint{4.152071in}{1.815675in}}%
\pgfpathlineto{\pgfqpoint{4.152839in}{1.810460in}}%
\pgfpathlineto{\pgfqpoint{4.155914in}{1.797637in}}%
\pgfpathlineto{\pgfqpoint{4.156106in}{1.799870in}}%
\pgfpathlineto{\pgfqpoint{4.156875in}{1.797729in}}%
\pgfpathlineto{\pgfqpoint{4.157451in}{1.793724in}}%
\pgfpathlineto{\pgfqpoint{4.157644in}{1.798537in}}%
\pgfpathlineto{\pgfqpoint{4.158028in}{1.797393in}}%
\pgfpathlineto{\pgfqpoint{4.158220in}{1.797333in}}%
\pgfpathlineto{\pgfqpoint{4.159181in}{1.792650in}}%
\pgfpathlineto{\pgfqpoint{4.159373in}{1.794375in}}%
\pgfpathlineto{\pgfqpoint{4.159757in}{1.793742in}}%
\pgfpathlineto{\pgfqpoint{4.160142in}{1.794919in}}%
\pgfpathlineto{\pgfqpoint{4.160526in}{1.789617in}}%
\pgfpathlineto{\pgfqpoint{4.161487in}{1.791229in}}%
\pgfpathlineto{\pgfqpoint{4.162063in}{1.791879in}}%
\pgfpathlineto{\pgfqpoint{4.162256in}{1.790772in}}%
\pgfpathlineto{\pgfqpoint{4.162448in}{1.788663in}}%
\pgfpathlineto{\pgfqpoint{4.163024in}{1.793696in}}%
\pgfpathlineto{\pgfqpoint{4.163793in}{1.796488in}}%
\pgfpathlineto{\pgfqpoint{4.164177in}{1.794357in}}%
\pgfpathlineto{\pgfqpoint{4.164370in}{1.793352in}}%
\pgfpathlineto{\pgfqpoint{4.164562in}{1.796124in}}%
\pgfpathlineto{\pgfqpoint{4.166483in}{1.812457in}}%
\pgfpathlineto{\pgfqpoint{4.167829in}{1.799402in}}%
\pgfpathlineto{\pgfqpoint{4.168597in}{1.801199in}}%
\pgfpathlineto{\pgfqpoint{4.169366in}{1.800322in}}%
\pgfpathlineto{\pgfqpoint{4.169750in}{1.803794in}}%
\pgfpathlineto{\pgfqpoint{4.169942in}{1.803568in}}%
\pgfpathlineto{\pgfqpoint{4.170327in}{1.801278in}}%
\pgfpathlineto{\pgfqpoint{4.170711in}{1.804459in}}%
\pgfpathlineto{\pgfqpoint{4.172441in}{1.814452in}}%
\pgfpathlineto{\pgfqpoint{4.173017in}{1.810925in}}%
\pgfpathlineto{\pgfqpoint{4.173978in}{1.813232in}}%
\pgfpathlineto{\pgfqpoint{4.173401in}{1.809512in}}%
\pgfpathlineto{\pgfqpoint{4.174362in}{1.811197in}}%
\pgfpathlineto{\pgfqpoint{4.176476in}{1.798723in}}%
\pgfpathlineto{\pgfqpoint{4.174747in}{1.812192in}}%
\pgfpathlineto{\pgfqpoint{4.177437in}{1.799390in}}%
\pgfpathlineto{\pgfqpoint{4.177629in}{1.803986in}}%
\pgfpathlineto{\pgfqpoint{4.178590in}{1.803035in}}%
\pgfpathlineto{\pgfqpoint{4.180512in}{1.796469in}}%
\pgfpathlineto{\pgfqpoint{4.180896in}{1.798875in}}%
\pgfpathlineto{\pgfqpoint{4.182049in}{1.805075in}}%
\pgfpathlineto{\pgfqpoint{4.182433in}{1.803831in}}%
\pgfpathlineto{\pgfqpoint{4.183586in}{1.784129in}}%
\pgfpathlineto{\pgfqpoint{4.184163in}{1.789315in}}%
\pgfpathlineto{\pgfqpoint{4.185124in}{1.792045in}}%
\pgfpathlineto{\pgfqpoint{4.184547in}{1.787353in}}%
\pgfpathlineto{\pgfqpoint{4.185508in}{1.791754in}}%
\pgfpathlineto{\pgfqpoint{4.186084in}{1.793230in}}%
\pgfpathlineto{\pgfqpoint{4.186277in}{1.791902in}}%
\pgfpathlineto{\pgfqpoint{4.187238in}{1.786708in}}%
\pgfpathlineto{\pgfqpoint{4.187622in}{1.788160in}}%
\pgfpathlineto{\pgfqpoint{4.187814in}{1.788524in}}%
\pgfpathlineto{\pgfqpoint{4.188006in}{1.787382in}}%
\pgfpathlineto{\pgfqpoint{4.189159in}{1.777486in}}%
\pgfpathlineto{\pgfqpoint{4.189544in}{1.779198in}}%
\pgfpathlineto{\pgfqpoint{4.190697in}{1.778366in}}%
\pgfpathlineto{\pgfqpoint{4.191850in}{1.773646in}}%
\pgfpathlineto{\pgfqpoint{4.192042in}{1.776367in}}%
\pgfpathlineto{\pgfqpoint{4.192810in}{1.778063in}}%
\pgfpathlineto{\pgfqpoint{4.193195in}{1.775352in}}%
\pgfpathlineto{\pgfqpoint{4.193579in}{1.780039in}}%
\pgfpathlineto{\pgfqpoint{4.193771in}{1.778451in}}%
\pgfpathlineto{\pgfqpoint{4.196077in}{1.766731in}}%
\pgfpathlineto{\pgfqpoint{4.196846in}{1.755315in}}%
\pgfpathlineto{\pgfqpoint{4.197615in}{1.756526in}}%
\pgfpathlineto{\pgfqpoint{4.197807in}{1.756462in}}%
\pgfpathlineto{\pgfqpoint{4.197999in}{1.757292in}}%
\pgfpathlineto{\pgfqpoint{4.198191in}{1.762273in}}%
\pgfpathlineto{\pgfqpoint{4.198768in}{1.755694in}}%
\pgfpathlineto{\pgfqpoint{4.199152in}{1.758377in}}%
\pgfpathlineto{\pgfqpoint{4.199344in}{1.759404in}}%
\pgfpathlineto{\pgfqpoint{4.199728in}{1.755694in}}%
\pgfpathlineto{\pgfqpoint{4.199921in}{1.758642in}}%
\pgfpathlineto{\pgfqpoint{4.200305in}{1.752823in}}%
\pgfpathlineto{\pgfqpoint{4.200881in}{1.760721in}}%
\pgfpathlineto{\pgfqpoint{4.201074in}{1.761043in}}%
\pgfpathlineto{\pgfqpoint{4.201266in}{1.759978in}}%
\pgfpathlineto{\pgfqpoint{4.203187in}{1.754607in}}%
\pgfpathlineto{\pgfqpoint{4.203764in}{1.757609in}}%
\pgfpathlineto{\pgfqpoint{4.203956in}{1.754373in}}%
\pgfpathlineto{\pgfqpoint{4.204148in}{1.755493in}}%
\pgfpathlineto{\pgfqpoint{4.204917in}{1.751235in}}%
\pgfpathlineto{\pgfqpoint{4.205301in}{1.754504in}}%
\pgfpathlineto{\pgfqpoint{4.205686in}{1.761766in}}%
\pgfpathlineto{\pgfqpoint{4.206454in}{1.757057in}}%
\pgfpathlineto{\pgfqpoint{4.207223in}{1.755908in}}%
\pgfpathlineto{\pgfqpoint{4.207607in}{1.759374in}}%
\pgfpathlineto{\pgfqpoint{4.207992in}{1.753472in}}%
\pgfpathlineto{\pgfqpoint{4.208760in}{1.755016in}}%
\pgfpathlineto{\pgfqpoint{4.209913in}{1.760185in}}%
\pgfpathlineto{\pgfqpoint{4.210105in}{1.758801in}}%
\pgfpathlineto{\pgfqpoint{4.210490in}{1.758323in}}%
\pgfpathlineto{\pgfqpoint{4.211451in}{1.762797in}}%
\pgfpathlineto{\pgfqpoint{4.212796in}{1.759018in}}%
\pgfpathlineto{\pgfqpoint{4.213180in}{1.762892in}}%
\pgfpathlineto{\pgfqpoint{4.213565in}{1.756299in}}%
\pgfpathlineto{\pgfqpoint{4.214525in}{1.755377in}}%
\pgfpathlineto{\pgfqpoint{4.214141in}{1.757398in}}%
\pgfpathlineto{\pgfqpoint{4.214718in}{1.756332in}}%
\pgfpathlineto{\pgfqpoint{4.215294in}{1.754833in}}%
\pgfpathlineto{\pgfqpoint{4.215678in}{1.756384in}}%
\pgfpathlineto{\pgfqpoint{4.216255in}{1.758704in}}%
\pgfpathlineto{\pgfqpoint{4.216639in}{1.756064in}}%
\pgfpathlineto{\pgfqpoint{4.217600in}{1.747538in}}%
\pgfpathlineto{\pgfqpoint{4.217984in}{1.752616in}}%
\pgfpathlineto{\pgfqpoint{4.218177in}{1.752860in}}%
\pgfpathlineto{\pgfqpoint{4.218561in}{1.756067in}}%
\pgfpathlineto{\pgfqpoint{4.219137in}{1.752432in}}%
\pgfpathlineto{\pgfqpoint{4.219714in}{1.749891in}}%
\pgfpathlineto{\pgfqpoint{4.220675in}{1.746510in}}%
\pgfpathlineto{\pgfqpoint{4.220867in}{1.749228in}}%
\pgfpathlineto{\pgfqpoint{4.221443in}{1.749551in}}%
\pgfpathlineto{\pgfqpoint{4.223173in}{1.756246in}}%
\pgfpathlineto{\pgfqpoint{4.223365in}{1.755816in}}%
\pgfpathlineto{\pgfqpoint{4.223557in}{1.755522in}}%
\pgfpathlineto{\pgfqpoint{4.223942in}{1.755134in}}%
\pgfpathlineto{\pgfqpoint{4.225095in}{1.762344in}}%
\pgfpathlineto{\pgfqpoint{4.225287in}{1.761794in}}%
\pgfpathlineto{\pgfqpoint{4.225671in}{1.755941in}}%
\pgfpathlineto{\pgfqpoint{4.226440in}{1.757011in}}%
\pgfpathlineto{\pgfqpoint{4.228746in}{1.768873in}}%
\pgfpathlineto{\pgfqpoint{4.230091in}{1.763762in}}%
\pgfpathlineto{\pgfqpoint{4.231628in}{1.772942in}}%
\pgfpathlineto{\pgfqpoint{4.233358in}{1.765681in}}%
\pgfpathlineto{\pgfqpoint{4.235279in}{1.753243in}}%
\pgfpathlineto{\pgfqpoint{4.235856in}{1.756822in}}%
\pgfpathlineto{\pgfqpoint{4.236240in}{1.760976in}}%
\pgfpathlineto{\pgfqpoint{4.236625in}{1.755741in}}%
\pgfpathlineto{\pgfqpoint{4.237009in}{1.757625in}}%
\pgfpathlineto{\pgfqpoint{4.238162in}{1.751037in}}%
\pgfpathlineto{\pgfqpoint{4.238546in}{1.752627in}}%
\pgfpathlineto{\pgfqpoint{4.238739in}{1.755035in}}%
\pgfpathlineto{\pgfqpoint{4.239507in}{1.752435in}}%
\pgfpathlineto{\pgfqpoint{4.239699in}{1.753375in}}%
\pgfpathlineto{\pgfqpoint{4.240084in}{1.753451in}}%
\pgfpathlineto{\pgfqpoint{4.240468in}{1.751275in}}%
\pgfpathlineto{\pgfqpoint{4.240852in}{1.752299in}}%
\pgfpathlineto{\pgfqpoint{4.241045in}{1.755732in}}%
\pgfpathlineto{\pgfqpoint{4.241813in}{1.750444in}}%
\pgfpathlineto{\pgfqpoint{4.242005in}{1.746946in}}%
\pgfpathlineto{\pgfqpoint{4.242390in}{1.751526in}}%
\pgfpathlineto{\pgfqpoint{4.242966in}{1.747581in}}%
\pgfpathlineto{\pgfqpoint{4.246041in}{1.758372in}}%
\pgfpathlineto{\pgfqpoint{4.246233in}{1.755853in}}%
\pgfpathlineto{\pgfqpoint{4.247770in}{1.748703in}}%
\pgfpathlineto{\pgfqpoint{4.248155in}{1.748516in}}%
\pgfpathlineto{\pgfqpoint{4.249692in}{1.740454in}}%
\pgfpathlineto{\pgfqpoint{4.249884in}{1.741955in}}%
\pgfpathlineto{\pgfqpoint{4.250076in}{1.741820in}}%
\pgfpathlineto{\pgfqpoint{4.251037in}{1.746829in}}%
\pgfpathlineto{\pgfqpoint{4.251229in}{1.744944in}}%
\pgfpathlineto{\pgfqpoint{4.251422in}{1.744308in}}%
\pgfpathlineto{\pgfqpoint{4.251998in}{1.751682in}}%
\pgfpathlineto{\pgfqpoint{4.252575in}{1.746227in}}%
\pgfpathlineto{\pgfqpoint{4.252959in}{1.744301in}}%
\pgfpathlineto{\pgfqpoint{4.253343in}{1.746991in}}%
\pgfpathlineto{\pgfqpoint{4.253535in}{1.745315in}}%
\pgfpathlineto{\pgfqpoint{4.254304in}{1.746482in}}%
\pgfpathlineto{\pgfqpoint{4.254496in}{1.743345in}}%
\pgfpathlineto{\pgfqpoint{4.255073in}{1.747157in}}%
\pgfpathlineto{\pgfqpoint{4.256226in}{1.751757in}}%
\pgfpathlineto{\pgfqpoint{4.257187in}{1.743776in}}%
\pgfpathlineto{\pgfqpoint{4.257571in}{1.744480in}}%
\pgfpathlineto{\pgfqpoint{4.258724in}{1.740883in}}%
\pgfpathlineto{\pgfqpoint{4.260454in}{1.752710in}}%
\pgfpathlineto{\pgfqpoint{4.260646in}{1.750858in}}%
\pgfpathlineto{\pgfqpoint{4.260838in}{1.748807in}}%
\pgfpathlineto{\pgfqpoint{4.261414in}{1.754452in}}%
\pgfpathlineto{\pgfqpoint{4.261607in}{1.754400in}}%
\pgfpathlineto{\pgfqpoint{4.261799in}{1.755180in}}%
\pgfpathlineto{\pgfqpoint{4.262952in}{1.760116in}}%
\pgfpathlineto{\pgfqpoint{4.263336in}{1.759215in}}%
\pgfpathlineto{\pgfqpoint{4.264489in}{1.753282in}}%
\pgfpathlineto{\pgfqpoint{4.265066in}{1.755545in}}%
\pgfpathlineto{\pgfqpoint{4.266795in}{1.751414in}}%
\pgfpathlineto{\pgfqpoint{4.267179in}{1.754517in}}%
\pgfpathlineto{\pgfqpoint{4.267948in}{1.752222in}}%
\pgfpathlineto{\pgfqpoint{4.268909in}{1.755577in}}%
\pgfpathlineto{\pgfqpoint{4.269101in}{1.754321in}}%
\pgfpathlineto{\pgfqpoint{4.269870in}{1.751575in}}%
\pgfpathlineto{\pgfqpoint{4.270254in}{1.753968in}}%
\pgfpathlineto{\pgfqpoint{4.271407in}{1.748875in}}%
\pgfpathlineto{\pgfqpoint{4.271791in}{1.751302in}}%
\pgfpathlineto{\pgfqpoint{4.273137in}{1.759173in}}%
\pgfpathlineto{\pgfqpoint{4.273329in}{1.760190in}}%
\pgfpathlineto{\pgfqpoint{4.273521in}{1.759706in}}%
\pgfpathlineto{\pgfqpoint{4.273713in}{1.755435in}}%
\pgfpathlineto{\pgfqpoint{4.274674in}{1.758363in}}%
\pgfpathlineto{\pgfqpoint{4.276788in}{1.766846in}}%
\pgfpathlineto{\pgfqpoint{4.275058in}{1.757447in}}%
\pgfpathlineto{\pgfqpoint{4.277172in}{1.763621in}}%
\pgfpathlineto{\pgfqpoint{4.278133in}{1.760112in}}%
\pgfpathlineto{\pgfqpoint{4.278325in}{1.762125in}}%
\pgfpathlineto{\pgfqpoint{4.278517in}{1.762433in}}%
\pgfpathlineto{\pgfqpoint{4.278902in}{1.759071in}}%
\pgfpathlineto{\pgfqpoint{4.279478in}{1.760828in}}%
\pgfpathlineto{\pgfqpoint{4.279670in}{1.762457in}}%
\pgfpathlineto{\pgfqpoint{4.280247in}{1.758920in}}%
\pgfpathlineto{\pgfqpoint{4.280631in}{1.756746in}}%
\pgfpathlineto{\pgfqpoint{4.280823in}{1.760246in}}%
\pgfpathlineto{\pgfqpoint{4.281015in}{1.761043in}}%
\pgfpathlineto{\pgfqpoint{4.281208in}{1.759810in}}%
\pgfpathlineto{\pgfqpoint{4.281400in}{1.757865in}}%
\pgfpathlineto{\pgfqpoint{4.281592in}{1.760235in}}%
\pgfpathlineto{\pgfqpoint{4.281976in}{1.759488in}}%
\pgfpathlineto{\pgfqpoint{4.282553in}{1.767444in}}%
\pgfpathlineto{\pgfqpoint{4.283321in}{1.765606in}}%
\pgfpathlineto{\pgfqpoint{4.285051in}{1.753140in}}%
\pgfpathlineto{\pgfqpoint{4.288894in}{1.772388in}}%
\pgfpathlineto{\pgfqpoint{4.289279in}{1.771507in}}%
\pgfpathlineto{\pgfqpoint{4.291008in}{1.764923in}}%
\pgfpathlineto{\pgfqpoint{4.291393in}{1.759831in}}%
\pgfpathlineto{\pgfqpoint{4.291969in}{1.764647in}}%
\pgfpathlineto{\pgfqpoint{4.292546in}{1.763711in}}%
\pgfpathlineto{\pgfqpoint{4.293122in}{1.768794in}}%
\pgfpathlineto{\pgfqpoint{4.293891in}{1.766676in}}%
\pgfpathlineto{\pgfqpoint{4.296197in}{1.757468in}}%
\pgfpathlineto{\pgfqpoint{4.296773in}{1.758856in}}%
\pgfpathlineto{\pgfqpoint{4.296965in}{1.760437in}}%
\pgfpathlineto{\pgfqpoint{4.297350in}{1.757545in}}%
\pgfpathlineto{\pgfqpoint{4.297542in}{1.757871in}}%
\pgfpathlineto{\pgfqpoint{4.299079in}{1.749934in}}%
\pgfpathlineto{\pgfqpoint{4.299271in}{1.752683in}}%
\pgfpathlineto{\pgfqpoint{4.299656in}{1.752581in}}%
\pgfpathlineto{\pgfqpoint{4.299848in}{1.754982in}}%
\pgfpathlineto{\pgfqpoint{4.300809in}{1.754214in}}%
\pgfpathlineto{\pgfqpoint{4.301577in}{1.755190in}}%
\pgfpathlineto{\pgfqpoint{4.302154in}{1.751527in}}%
\pgfpathlineto{\pgfqpoint{4.303307in}{1.758218in}}%
\pgfpathlineto{\pgfqpoint{4.303499in}{1.756514in}}%
\pgfpathlineto{\pgfqpoint{4.304076in}{1.755621in}}%
\pgfpathlineto{\pgfqpoint{4.304268in}{1.757481in}}%
\pgfpathlineto{\pgfqpoint{4.304460in}{1.757786in}}%
\pgfpathlineto{\pgfqpoint{4.305805in}{1.773030in}}%
\pgfpathlineto{\pgfqpoint{4.305997in}{1.771501in}}%
\pgfpathlineto{\pgfqpoint{4.307342in}{1.767839in}}%
\pgfpathlineto{\pgfqpoint{4.307535in}{1.767959in}}%
\pgfpathlineto{\pgfqpoint{4.308111in}{1.771283in}}%
\pgfpathlineto{\pgfqpoint{4.308688in}{1.769436in}}%
\pgfpathlineto{\pgfqpoint{4.309649in}{1.764787in}}%
\pgfpathlineto{\pgfqpoint{4.310225in}{1.765007in}}%
\pgfpathlineto{\pgfqpoint{4.310609in}{1.764904in}}%
\pgfpathlineto{\pgfqpoint{4.311378in}{1.768485in}}%
\pgfpathlineto{\pgfqpoint{4.311570in}{1.767125in}}%
\pgfpathlineto{\pgfqpoint{4.311762in}{1.767760in}}%
\pgfpathlineto{\pgfqpoint{4.311955in}{1.772678in}}%
\pgfpathlineto{\pgfqpoint{4.312915in}{1.770449in}}%
\pgfpathlineto{\pgfqpoint{4.314068in}{1.761972in}}%
\pgfpathlineto{\pgfqpoint{4.314261in}{1.765365in}}%
\pgfpathlineto{\pgfqpoint{4.314645in}{1.762420in}}%
\pgfpathlineto{\pgfqpoint{4.315221in}{1.765002in}}%
\pgfpathlineto{\pgfqpoint{4.315798in}{1.766885in}}%
\pgfpathlineto{\pgfqpoint{4.316374in}{1.765298in}}%
\pgfpathlineto{\pgfqpoint{4.316759in}{1.764482in}}%
\pgfpathlineto{\pgfqpoint{4.316951in}{1.764956in}}%
\pgfpathlineto{\pgfqpoint{4.317335in}{1.768877in}}%
\pgfpathlineto{\pgfqpoint{4.317720in}{1.764553in}}%
\pgfpathlineto{\pgfqpoint{4.318104in}{1.766076in}}%
\pgfpathlineto{\pgfqpoint{4.318296in}{1.763007in}}%
\pgfpathlineto{\pgfqpoint{4.318873in}{1.768124in}}%
\pgfpathlineto{\pgfqpoint{4.319065in}{1.767382in}}%
\pgfpathlineto{\pgfqpoint{4.321947in}{1.782182in}}%
\pgfpathlineto{\pgfqpoint{4.322139in}{1.781694in}}%
\pgfpathlineto{\pgfqpoint{4.322332in}{1.785074in}}%
\pgfpathlineto{\pgfqpoint{4.322716in}{1.777598in}}%
\pgfpathlineto{\pgfqpoint{4.322908in}{1.779236in}}%
\pgfpathlineto{\pgfqpoint{4.324061in}{1.773767in}}%
\pgfpathlineto{\pgfqpoint{4.324445in}{1.775274in}}%
\pgfpathlineto{\pgfqpoint{4.326367in}{1.786367in}}%
\pgfpathlineto{\pgfqpoint{4.326751in}{1.788887in}}%
\pgfpathlineto{\pgfqpoint{4.327136in}{1.785692in}}%
\pgfpathlineto{\pgfqpoint{4.328865in}{1.775713in}}%
\pgfpathlineto{\pgfqpoint{4.329057in}{1.778293in}}%
\pgfpathlineto{\pgfqpoint{4.329826in}{1.774132in}}%
\pgfpathlineto{\pgfqpoint{4.331940in}{1.764555in}}%
\pgfpathlineto{\pgfqpoint{4.332901in}{1.768682in}}%
\pgfpathlineto{\pgfqpoint{4.333093in}{1.766713in}}%
\pgfpathlineto{\pgfqpoint{4.333285in}{1.764554in}}%
\pgfpathlineto{\pgfqpoint{4.334054in}{1.767149in}}%
\pgfpathlineto{\pgfqpoint{4.334246in}{1.764662in}}%
\pgfpathlineto{\pgfqpoint{4.334438in}{1.764301in}}%
\pgfpathlineto{\pgfqpoint{4.334823in}{1.765667in}}%
\pgfpathlineto{\pgfqpoint{4.335015in}{1.766143in}}%
\pgfpathlineto{\pgfqpoint{4.335399in}{1.764901in}}%
\pgfpathlineto{\pgfqpoint{4.338474in}{1.754080in}}%
\pgfpathlineto{\pgfqpoint{4.339242in}{1.755934in}}%
\pgfpathlineto{\pgfqpoint{4.339819in}{1.756206in}}%
\pgfpathlineto{\pgfqpoint{4.340011in}{1.757086in}}%
\pgfpathlineto{\pgfqpoint{4.340395in}{1.754185in}}%
\pgfpathlineto{\pgfqpoint{4.340588in}{1.752116in}}%
\pgfpathlineto{\pgfqpoint{4.341356in}{1.754992in}}%
\pgfpathlineto{\pgfqpoint{4.342509in}{1.762387in}}%
\pgfpathlineto{\pgfqpoint{4.343278in}{1.758562in}}%
\pgfpathlineto{\pgfqpoint{4.343662in}{1.757388in}}%
\pgfpathlineto{\pgfqpoint{4.344431in}{1.750024in}}%
\pgfpathlineto{\pgfqpoint{4.344815in}{1.751241in}}%
\pgfpathlineto{\pgfqpoint{4.346545in}{1.765482in}}%
\pgfpathlineto{\pgfqpoint{4.347121in}{1.762755in}}%
\pgfpathlineto{\pgfqpoint{4.347313in}{1.761424in}}%
\pgfpathlineto{\pgfqpoint{4.347890in}{1.763559in}}%
\pgfpathlineto{\pgfqpoint{4.348082in}{1.762753in}}%
\pgfpathlineto{\pgfqpoint{4.348274in}{1.766859in}}%
\pgfpathlineto{\pgfqpoint{4.348851in}{1.761241in}}%
\pgfpathlineto{\pgfqpoint{4.349043in}{1.764251in}}%
\pgfpathlineto{\pgfqpoint{4.350004in}{1.762274in}}%
\pgfpathlineto{\pgfqpoint{4.350196in}{1.763465in}}%
\pgfpathlineto{\pgfqpoint{4.350388in}{1.763425in}}%
\pgfpathlineto{\pgfqpoint{4.351541in}{1.756672in}}%
\pgfpathlineto{\pgfqpoint{4.352310in}{1.757091in}}%
\pgfpathlineto{\pgfqpoint{4.352694in}{1.755795in}}%
\pgfpathlineto{\pgfqpoint{4.353655in}{1.761505in}}%
\pgfpathlineto{\pgfqpoint{4.353847in}{1.760246in}}%
\pgfpathlineto{\pgfqpoint{4.354808in}{1.756760in}}%
\pgfpathlineto{\pgfqpoint{4.356537in}{1.770904in}}%
\pgfpathlineto{\pgfqpoint{4.356730in}{1.770789in}}%
\pgfpathlineto{\pgfqpoint{4.357306in}{1.767954in}}%
\pgfpathlineto{\pgfqpoint{4.358459in}{1.776043in}}%
\pgfpathlineto{\pgfqpoint{4.359036in}{1.770068in}}%
\pgfpathlineto{\pgfqpoint{4.359804in}{1.771905in}}%
\pgfpathlineto{\pgfqpoint{4.361150in}{1.781895in}}%
\pgfpathlineto{\pgfqpoint{4.361342in}{1.779410in}}%
\pgfpathlineto{\pgfqpoint{4.362495in}{1.774078in}}%
\pgfpathlineto{\pgfqpoint{4.362879in}{1.776878in}}%
\pgfpathlineto{\pgfqpoint{4.364224in}{1.781223in}}%
\pgfpathlineto{\pgfqpoint{4.363263in}{1.775155in}}%
\pgfpathlineto{\pgfqpoint{4.364416in}{1.778111in}}%
\pgfpathlineto{\pgfqpoint{4.366146in}{1.768345in}}%
\pgfpathlineto{\pgfqpoint{4.366338in}{1.770723in}}%
\pgfpathlineto{\pgfqpoint{4.366530in}{1.771545in}}%
\pgfpathlineto{\pgfqpoint{4.366722in}{1.769714in}}%
\pgfpathlineto{\pgfqpoint{4.367107in}{1.767193in}}%
\pgfpathlineto{\pgfqpoint{4.367683in}{1.770044in}}%
\pgfpathlineto{\pgfqpoint{4.369028in}{1.766748in}}%
\pgfpathlineto{\pgfqpoint{4.370566in}{1.782967in}}%
\pgfpathlineto{\pgfqpoint{4.370758in}{1.782046in}}%
\pgfpathlineto{\pgfqpoint{4.371527in}{1.780898in}}%
\pgfpathlineto{\pgfqpoint{4.371719in}{1.781125in}}%
\pgfpathlineto{\pgfqpoint{4.373448in}{1.791827in}}%
\pgfpathlineto{\pgfqpoint{4.372103in}{1.778239in}}%
\pgfpathlineto{\pgfqpoint{4.373833in}{1.788153in}}%
\pgfpathlineto{\pgfqpoint{4.374025in}{1.788423in}}%
\pgfpathlineto{\pgfqpoint{4.374986in}{1.796186in}}%
\pgfpathlineto{\pgfqpoint{4.375370in}{1.794019in}}%
\pgfpathlineto{\pgfqpoint{4.375946in}{1.797145in}}%
\pgfpathlineto{\pgfqpoint{4.376331in}{1.794771in}}%
\pgfpathlineto{\pgfqpoint{4.376523in}{1.793867in}}%
\pgfpathlineto{\pgfqpoint{4.376907in}{1.795634in}}%
\pgfpathlineto{\pgfqpoint{4.377484in}{1.800194in}}%
\pgfpathlineto{\pgfqpoint{4.378252in}{1.800095in}}%
\pgfpathlineto{\pgfqpoint{4.378637in}{1.804669in}}%
\pgfpathlineto{\pgfqpoint{4.379021in}{1.801120in}}%
\pgfpathlineto{\pgfqpoint{4.379982in}{1.791901in}}%
\pgfpathlineto{\pgfqpoint{4.380558in}{1.792720in}}%
\pgfpathlineto{\pgfqpoint{4.380751in}{1.797683in}}%
\pgfpathlineto{\pgfqpoint{4.381711in}{1.796777in}}%
\pgfpathlineto{\pgfqpoint{4.381904in}{1.794033in}}%
\pgfpathlineto{\pgfqpoint{4.382288in}{1.797847in}}%
\pgfpathlineto{\pgfqpoint{4.382672in}{1.797192in}}%
\pgfpathlineto{\pgfqpoint{4.383633in}{1.794949in}}%
\pgfpathlineto{\pgfqpoint{4.383057in}{1.798665in}}%
\pgfpathlineto{\pgfqpoint{4.383825in}{1.795668in}}%
\pgfpathlineto{\pgfqpoint{4.384210in}{1.798288in}}%
\pgfpathlineto{\pgfqpoint{4.384402in}{1.794470in}}%
\pgfpathlineto{\pgfqpoint{4.384786in}{1.795088in}}%
\pgfpathlineto{\pgfqpoint{4.384978in}{1.795347in}}%
\pgfpathlineto{\pgfqpoint{4.385555in}{1.791188in}}%
\pgfpathlineto{\pgfqpoint{4.385939in}{1.791733in}}%
\pgfpathlineto{\pgfqpoint{4.386324in}{1.795993in}}%
\pgfpathlineto{\pgfqpoint{4.387092in}{1.794334in}}%
\pgfpathlineto{\pgfqpoint{4.387861in}{1.798874in}}%
\pgfpathlineto{\pgfqpoint{4.388630in}{1.797729in}}%
\pgfpathlineto{\pgfqpoint{4.391128in}{1.791408in}}%
\pgfpathlineto{\pgfqpoint{4.392281in}{1.799005in}}%
\pgfpathlineto{\pgfqpoint{4.392473in}{1.796205in}}%
\pgfpathlineto{\pgfqpoint{4.393242in}{1.795155in}}%
\pgfpathlineto{\pgfqpoint{4.393434in}{1.796010in}}%
\pgfpathlineto{\pgfqpoint{4.393818in}{1.797364in}}%
\pgfpathlineto{\pgfqpoint{4.394587in}{1.790362in}}%
\pgfpathlineto{\pgfqpoint{4.394971in}{1.794972in}}%
\pgfpathlineto{\pgfqpoint{4.395548in}{1.793579in}}%
\pgfpathlineto{\pgfqpoint{4.395740in}{1.793797in}}%
\pgfpathlineto{\pgfqpoint{4.396316in}{1.797605in}}%
\pgfpathlineto{\pgfqpoint{4.396701in}{1.795191in}}%
\pgfpathlineto{\pgfqpoint{4.397469in}{1.789880in}}%
\pgfpathlineto{\pgfqpoint{4.398046in}{1.792242in}}%
\pgfpathlineto{\pgfqpoint{4.400544in}{1.804027in}}%
\pgfpathlineto{\pgfqpoint{4.401313in}{1.802238in}}%
\pgfpathlineto{\pgfqpoint{4.403426in}{1.788457in}}%
\pgfpathlineto{\pgfqpoint{4.404195in}{1.783522in}}%
\pgfpathlineto{\pgfqpoint{4.403811in}{1.789561in}}%
\pgfpathlineto{\pgfqpoint{4.404579in}{1.786840in}}%
\pgfpathlineto{\pgfqpoint{4.406309in}{1.801949in}}%
\pgfpathlineto{\pgfqpoint{4.406501in}{1.799827in}}%
\pgfpathlineto{\pgfqpoint{4.406693in}{1.799812in}}%
\pgfpathlineto{\pgfqpoint{4.407462in}{1.796024in}}%
\pgfpathlineto{\pgfqpoint{4.407846in}{1.798682in}}%
\pgfpathlineto{\pgfqpoint{4.408039in}{1.798953in}}%
\pgfpathlineto{\pgfqpoint{4.408231in}{1.798212in}}%
\pgfpathlineto{\pgfqpoint{4.408423in}{1.797451in}}%
\pgfpathlineto{\pgfqpoint{4.408807in}{1.799878in}}%
\pgfpathlineto{\pgfqpoint{4.409768in}{1.805938in}}%
\pgfpathlineto{\pgfqpoint{4.409960in}{1.801654in}}%
\pgfpathlineto{\pgfqpoint{4.411690in}{1.792590in}}%
\pgfpathlineto{\pgfqpoint{4.410345in}{1.801794in}}%
\pgfpathlineto{\pgfqpoint{4.411882in}{1.795814in}}%
\pgfpathlineto{\pgfqpoint{4.412458in}{1.799859in}}%
\pgfpathlineto{\pgfqpoint{4.412843in}{1.797761in}}%
\pgfpathlineto{\pgfqpoint{4.413804in}{1.793073in}}%
\pgfpathlineto{\pgfqpoint{4.413996in}{1.794961in}}%
\pgfpathlineto{\pgfqpoint{4.415917in}{1.805375in}}%
\pgfpathlineto{\pgfqpoint{4.416110in}{1.800738in}}%
\pgfpathlineto{\pgfqpoint{4.416686in}{1.805693in}}%
\pgfpathlineto{\pgfqpoint{4.416878in}{1.804457in}}%
\pgfpathlineto{\pgfqpoint{4.418031in}{1.807728in}}%
\pgfpathlineto{\pgfqpoint{4.418223in}{1.807455in}}%
\pgfpathlineto{\pgfqpoint{4.418416in}{1.809103in}}%
\pgfpathlineto{\pgfqpoint{4.418608in}{1.808711in}}%
\pgfpathlineto{\pgfqpoint{4.419953in}{1.811889in}}%
\pgfpathlineto{\pgfqpoint{4.420145in}{1.811704in}}%
\pgfpathlineto{\pgfqpoint{4.421106in}{1.801539in}}%
\pgfpathlineto{\pgfqpoint{4.421490in}{1.804134in}}%
\pgfpathlineto{\pgfqpoint{4.423220in}{1.808053in}}%
\pgfpathlineto{\pgfqpoint{4.423412in}{1.807485in}}%
\pgfpathlineto{\pgfqpoint{4.423604in}{1.809951in}}%
\pgfpathlineto{\pgfqpoint{4.423796in}{1.809548in}}%
\pgfpathlineto{\pgfqpoint{4.423988in}{1.809931in}}%
\pgfpathlineto{\pgfqpoint{4.424181in}{1.808938in}}%
\pgfpathlineto{\pgfqpoint{4.425526in}{1.802758in}}%
\pgfpathlineto{\pgfqpoint{4.425718in}{1.804348in}}%
\pgfpathlineto{\pgfqpoint{4.425910in}{1.804788in}}%
\pgfpathlineto{\pgfqpoint{4.427447in}{1.795448in}}%
\pgfpathlineto{\pgfqpoint{4.428985in}{1.809011in}}%
\pgfpathlineto{\pgfqpoint{4.429561in}{1.807383in}}%
\pgfpathlineto{\pgfqpoint{4.430138in}{1.807826in}}%
\pgfpathlineto{\pgfqpoint{4.431483in}{1.803618in}}%
\pgfpathlineto{\pgfqpoint{4.431675in}{1.800592in}}%
\pgfpathlineto{\pgfqpoint{4.432252in}{1.806669in}}%
\pgfpathlineto{\pgfqpoint{4.433597in}{1.813801in}}%
\pgfpathlineto{\pgfqpoint{4.434366in}{1.812277in}}%
\pgfpathlineto{\pgfqpoint{4.435903in}{1.809970in}}%
\pgfpathlineto{\pgfqpoint{4.437056in}{1.815364in}}%
\pgfpathlineto{\pgfqpoint{4.437248in}{1.812438in}}%
\pgfpathlineto{\pgfqpoint{4.438593in}{1.807032in}}%
\pgfpathlineto{\pgfqpoint{4.439170in}{1.810799in}}%
\pgfpathlineto{\pgfqpoint{4.439746in}{1.808633in}}%
\pgfpathlineto{\pgfqpoint{4.439938in}{1.808439in}}%
\pgfpathlineto{\pgfqpoint{4.440131in}{1.808916in}}%
\pgfpathlineto{\pgfqpoint{4.440899in}{1.813263in}}%
\pgfpathlineto{\pgfqpoint{4.441284in}{1.812711in}}%
\pgfpathlineto{\pgfqpoint{4.443013in}{1.807670in}}%
\pgfpathlineto{\pgfqpoint{4.441668in}{1.813727in}}%
\pgfpathlineto{\pgfqpoint{4.443205in}{1.809795in}}%
\pgfpathlineto{\pgfqpoint{4.443974in}{1.811147in}}%
\pgfpathlineto{\pgfqpoint{4.443782in}{1.809503in}}%
\pgfpathlineto{\pgfqpoint{4.444166in}{1.809582in}}%
\pgfpathlineto{\pgfqpoint{4.445511in}{1.801698in}}%
\pgfpathlineto{\pgfqpoint{4.445703in}{1.802767in}}%
\pgfpathlineto{\pgfqpoint{4.446088in}{1.800605in}}%
\pgfpathlineto{\pgfqpoint{4.446472in}{1.806277in}}%
\pgfpathlineto{\pgfqpoint{4.446664in}{1.806113in}}%
\pgfpathlineto{\pgfqpoint{4.447817in}{1.815195in}}%
\pgfpathlineto{\pgfqpoint{4.448009in}{1.813853in}}%
\pgfpathlineto{\pgfqpoint{4.448202in}{1.811962in}}%
\pgfpathlineto{\pgfqpoint{4.448394in}{1.814436in}}%
\pgfpathlineto{\pgfqpoint{4.448778in}{1.814192in}}%
\pgfpathlineto{\pgfqpoint{4.450315in}{1.826670in}}%
\pgfpathlineto{\pgfqpoint{4.450892in}{1.825457in}}%
\pgfpathlineto{\pgfqpoint{4.451468in}{1.823513in}}%
\pgfpathlineto{\pgfqpoint{4.451661in}{1.825978in}}%
\pgfpathlineto{\pgfqpoint{4.451853in}{1.829018in}}%
\pgfpathlineto{\pgfqpoint{4.452814in}{1.828318in}}%
\pgfpathlineto{\pgfqpoint{4.453582in}{1.834125in}}%
\pgfpathlineto{\pgfqpoint{4.453967in}{1.831015in}}%
\pgfpathlineto{\pgfqpoint{4.455120in}{1.826016in}}%
\pgfpathlineto{\pgfqpoint{4.458771in}{1.852155in}}%
\pgfpathlineto{\pgfqpoint{4.459347in}{1.847581in}}%
\pgfpathlineto{\pgfqpoint{4.461077in}{1.852543in}}%
\pgfpathlineto{\pgfqpoint{4.462038in}{1.848136in}}%
\pgfpathlineto{\pgfqpoint{4.462422in}{1.849445in}}%
\pgfpathlineto{\pgfqpoint{4.462614in}{1.848335in}}%
\pgfpathlineto{\pgfqpoint{4.463191in}{1.851319in}}%
\pgfpathlineto{\pgfqpoint{4.464344in}{1.857435in}}%
\pgfpathlineto{\pgfqpoint{4.464536in}{1.855456in}}%
\pgfpathlineto{\pgfqpoint{4.464920in}{1.855899in}}%
\pgfpathlineto{\pgfqpoint{4.465305in}{1.851886in}}%
\pgfpathlineto{\pgfqpoint{4.465881in}{1.855880in}}%
\pgfpathlineto{\pgfqpoint{4.466650in}{1.869982in}}%
\pgfpathlineto{\pgfqpoint{4.467226in}{1.862328in}}%
\pgfpathlineto{\pgfqpoint{4.467611in}{1.860789in}}%
\pgfpathlineto{\pgfqpoint{4.467803in}{1.865144in}}%
\pgfpathlineto{\pgfqpoint{4.467995in}{1.865345in}}%
\pgfpathlineto{\pgfqpoint{4.468956in}{1.861395in}}%
\pgfpathlineto{\pgfqpoint{4.469148in}{1.863627in}}%
\pgfpathlineto{\pgfqpoint{4.470109in}{1.868305in}}%
\pgfpathlineto{\pgfqpoint{4.470301in}{1.866357in}}%
\pgfpathlineto{\pgfqpoint{4.472799in}{1.877939in}}%
\pgfpathlineto{\pgfqpoint{4.472991in}{1.876322in}}%
\pgfpathlineto{\pgfqpoint{4.473760in}{1.867897in}}%
\pgfpathlineto{\pgfqpoint{4.474336in}{1.870752in}}%
\pgfpathlineto{\pgfqpoint{4.475297in}{1.880160in}}%
\pgfpathlineto{\pgfqpoint{4.475682in}{1.878453in}}%
\pgfpathlineto{\pgfqpoint{4.476066in}{1.881482in}}%
\pgfpathlineto{\pgfqpoint{4.476835in}{1.879834in}}%
\pgfpathlineto{\pgfqpoint{4.477219in}{1.878582in}}%
\pgfpathlineto{\pgfqpoint{4.477411in}{1.880328in}}%
\pgfpathlineto{\pgfqpoint{4.477603in}{1.878350in}}%
\pgfpathlineto{\pgfqpoint{4.478180in}{1.882500in}}%
\pgfpathlineto{\pgfqpoint{4.478372in}{1.879947in}}%
\pgfpathlineto{\pgfqpoint{4.478948in}{1.889869in}}%
\pgfpathlineto{\pgfqpoint{4.479909in}{1.887593in}}%
\pgfpathlineto{\pgfqpoint{4.480294in}{1.885909in}}%
\pgfpathlineto{\pgfqpoint{4.480486in}{1.888741in}}%
\pgfpathlineto{\pgfqpoint{4.481831in}{1.891566in}}%
\pgfpathlineto{\pgfqpoint{4.483368in}{1.904871in}}%
\pgfpathlineto{\pgfqpoint{4.484137in}{1.903764in}}%
\pgfpathlineto{\pgfqpoint{4.485290in}{1.898961in}}%
\pgfpathlineto{\pgfqpoint{4.485482in}{1.900285in}}%
\pgfpathlineto{\pgfqpoint{4.485867in}{1.897823in}}%
\pgfpathlineto{\pgfqpoint{4.486443in}{1.894207in}}%
\pgfpathlineto{\pgfqpoint{4.487020in}{1.895103in}}%
\pgfpathlineto{\pgfqpoint{4.487212in}{1.895133in}}%
\pgfpathlineto{\pgfqpoint{4.488173in}{1.898095in}}%
\pgfpathlineto{\pgfqpoint{4.488365in}{1.896111in}}%
\pgfpathlineto{\pgfqpoint{4.489518in}{1.890137in}}%
\pgfpathlineto{\pgfqpoint{4.489710in}{1.890507in}}%
\pgfpathlineto{\pgfqpoint{4.490286in}{1.892076in}}%
\pgfpathlineto{\pgfqpoint{4.490671in}{1.891317in}}%
\pgfpathlineto{\pgfqpoint{4.491439in}{1.891515in}}%
\pgfpathlineto{\pgfqpoint{4.492016in}{1.889327in}}%
\pgfpathlineto{\pgfqpoint{4.492400in}{1.889303in}}%
\pgfpathlineto{\pgfqpoint{4.493169in}{1.893808in}}%
\pgfpathlineto{\pgfqpoint{4.492977in}{1.888740in}}%
\pgfpathlineto{\pgfqpoint{4.493553in}{1.890299in}}%
\pgfpathlineto{\pgfqpoint{4.493938in}{1.888591in}}%
\pgfpathlineto{\pgfqpoint{4.494322in}{1.891710in}}%
\pgfpathlineto{\pgfqpoint{4.495667in}{1.894855in}}%
\pgfpathlineto{\pgfqpoint{4.496820in}{1.888419in}}%
\pgfpathlineto{\pgfqpoint{4.497012in}{1.890301in}}%
\pgfpathlineto{\pgfqpoint{4.497204in}{1.891165in}}%
\pgfpathlineto{\pgfqpoint{4.497397in}{1.889222in}}%
\pgfpathlineto{\pgfqpoint{4.497589in}{1.887279in}}%
\pgfpathlineto{\pgfqpoint{4.498165in}{1.891553in}}%
\pgfpathlineto{\pgfqpoint{4.498357in}{1.891826in}}%
\pgfpathlineto{\pgfqpoint{4.498550in}{1.889777in}}%
\pgfpathlineto{\pgfqpoint{4.499126in}{1.892914in}}%
\pgfpathlineto{\pgfqpoint{4.499318in}{1.892752in}}%
\pgfpathlineto{\pgfqpoint{4.499510in}{1.894114in}}%
\pgfpathlineto{\pgfqpoint{4.499895in}{1.890645in}}%
\pgfpathlineto{\pgfqpoint{4.500279in}{1.891839in}}%
\pgfpathlineto{\pgfqpoint{4.500471in}{1.889011in}}%
\pgfpathlineto{\pgfqpoint{4.500663in}{1.892683in}}%
\pgfpathlineto{\pgfqpoint{4.501240in}{1.890622in}}%
\pgfpathlineto{\pgfqpoint{4.502201in}{1.897639in}}%
\pgfpathlineto{\pgfqpoint{4.502585in}{1.895964in}}%
\pgfpathlineto{\pgfqpoint{4.502969in}{1.891691in}}%
\pgfpathlineto{\pgfqpoint{4.503930in}{1.892922in}}%
\pgfpathlineto{\pgfqpoint{4.505852in}{1.902141in}}%
\pgfpathlineto{\pgfqpoint{4.506429in}{1.895966in}}%
\pgfpathlineto{\pgfqpoint{4.507005in}{1.896122in}}%
\pgfpathlineto{\pgfqpoint{4.507197in}{1.899161in}}%
\pgfpathlineto{\pgfqpoint{4.507774in}{1.892441in}}%
\pgfpathlineto{\pgfqpoint{4.508158in}{1.891254in}}%
\pgfpathlineto{\pgfqpoint{4.509119in}{1.884234in}}%
\pgfpathlineto{\pgfqpoint{4.509311in}{1.889926in}}%
\pgfpathlineto{\pgfqpoint{4.510848in}{1.900926in}}%
\pgfpathlineto{\pgfqpoint{4.511041in}{1.901638in}}%
\pgfpathlineto{\pgfqpoint{4.511233in}{1.900862in}}%
\pgfpathlineto{\pgfqpoint{4.511617in}{1.896256in}}%
\pgfpathlineto{\pgfqpoint{4.512386in}{1.897977in}}%
\pgfpathlineto{\pgfqpoint{4.513154in}{1.901330in}}%
\pgfpathlineto{\pgfqpoint{4.514307in}{1.907179in}}%
\pgfpathlineto{\pgfqpoint{4.514500in}{1.906803in}}%
\pgfpathlineto{\pgfqpoint{4.514884in}{1.907430in}}%
\pgfpathlineto{\pgfqpoint{4.515268in}{1.903664in}}%
\pgfpathlineto{\pgfqpoint{4.516037in}{1.905351in}}%
\pgfpathlineto{\pgfqpoint{4.516229in}{1.905885in}}%
\pgfpathlineto{\pgfqpoint{4.516421in}{1.905379in}}%
\pgfpathlineto{\pgfqpoint{4.517382in}{1.900903in}}%
\pgfpathlineto{\pgfqpoint{4.517574in}{1.904057in}}%
\pgfpathlineto{\pgfqpoint{4.518919in}{1.899581in}}%
\pgfpathlineto{\pgfqpoint{4.519112in}{1.904525in}}%
\pgfpathlineto{\pgfqpoint{4.520072in}{1.901716in}}%
\pgfpathlineto{\pgfqpoint{4.521802in}{1.913133in}}%
\pgfpathlineto{\pgfqpoint{4.521994in}{1.912083in}}%
\pgfpathlineto{\pgfqpoint{4.523147in}{1.907512in}}%
\pgfpathlineto{\pgfqpoint{4.523724in}{1.912103in}}%
\pgfpathlineto{\pgfqpoint{4.524108in}{1.909178in}}%
\pgfpathlineto{\pgfqpoint{4.524300in}{1.906712in}}%
\pgfpathlineto{\pgfqpoint{4.524877in}{1.911465in}}%
\pgfpathlineto{\pgfqpoint{4.526606in}{1.922486in}}%
\pgfpathlineto{\pgfqpoint{4.528143in}{1.915448in}}%
\pgfpathlineto{\pgfqpoint{4.528336in}{1.916707in}}%
\pgfpathlineto{\pgfqpoint{4.528528in}{1.917163in}}%
\pgfpathlineto{\pgfqpoint{4.528912in}{1.915624in}}%
\pgfpathlineto{\pgfqpoint{4.529104in}{1.915571in}}%
\pgfpathlineto{\pgfqpoint{4.530065in}{1.918732in}}%
\pgfpathlineto{\pgfqpoint{4.529489in}{1.915189in}}%
\pgfpathlineto{\pgfqpoint{4.530450in}{1.917481in}}%
\pgfpathlineto{\pgfqpoint{4.530642in}{1.915311in}}%
\pgfpathlineto{\pgfqpoint{4.531026in}{1.919059in}}%
\pgfpathlineto{\pgfqpoint{4.531218in}{1.918991in}}%
\pgfpathlineto{\pgfqpoint{4.531410in}{1.920327in}}%
\pgfpathlineto{\pgfqpoint{4.531795in}{1.915658in}}%
\pgfpathlineto{\pgfqpoint{4.532756in}{1.908876in}}%
\pgfpathlineto{\pgfqpoint{4.533140in}{1.910872in}}%
\pgfpathlineto{\pgfqpoint{4.533524in}{1.913293in}}%
\pgfpathlineto{\pgfqpoint{4.534101in}{1.909661in}}%
\pgfpathlineto{\pgfqpoint{4.535446in}{1.903837in}}%
\pgfpathlineto{\pgfqpoint{4.536215in}{1.910961in}}%
\pgfpathlineto{\pgfqpoint{4.536791in}{1.909654in}}%
\pgfpathlineto{\pgfqpoint{4.537752in}{1.902023in}}%
\pgfpathlineto{\pgfqpoint{4.538136in}{1.903942in}}%
\pgfpathlineto{\pgfqpoint{4.538521in}{1.905472in}}%
\pgfpathlineto{\pgfqpoint{4.538905in}{1.910457in}}%
\pgfpathlineto{\pgfqpoint{4.539481in}{1.902622in}}%
\pgfpathlineto{\pgfqpoint{4.540827in}{1.893461in}}%
\pgfpathlineto{\pgfqpoint{4.541403in}{1.897223in}}%
\pgfpathlineto{\pgfqpoint{4.541980in}{1.902935in}}%
\pgfpathlineto{\pgfqpoint{4.542748in}{1.902034in}}%
\pgfpathlineto{\pgfqpoint{4.544093in}{1.895936in}}%
\pgfpathlineto{\pgfqpoint{4.544862in}{1.895847in}}%
\pgfpathlineto{\pgfqpoint{4.545439in}{1.897571in}}%
\pgfpathlineto{\pgfqpoint{4.546015in}{1.896610in}}%
\pgfpathlineto{\pgfqpoint{4.547168in}{1.889541in}}%
\pgfpathlineto{\pgfqpoint{4.547360in}{1.893050in}}%
\pgfpathlineto{\pgfqpoint{4.547745in}{1.895496in}}%
\pgfpathlineto{\pgfqpoint{4.548898in}{1.898212in}}%
\pgfpathlineto{\pgfqpoint{4.549666in}{1.892919in}}%
\pgfpathlineto{\pgfqpoint{4.550051in}{1.895153in}}%
\pgfpathlineto{\pgfqpoint{4.550627in}{1.891141in}}%
\pgfpathlineto{\pgfqpoint{4.551011in}{1.887607in}}%
\pgfpathlineto{\pgfqpoint{4.551780in}{1.888388in}}%
\pgfpathlineto{\pgfqpoint{4.551972in}{1.891817in}}%
\pgfpathlineto{\pgfqpoint{4.552549in}{1.889619in}}%
\pgfpathlineto{\pgfqpoint{4.552741in}{1.884928in}}%
\pgfpathlineto{\pgfqpoint{4.553125in}{1.892124in}}%
\pgfpathlineto{\pgfqpoint{4.553318in}{1.891866in}}%
\pgfpathlineto{\pgfqpoint{4.554086in}{1.898422in}}%
\pgfpathlineto{\pgfqpoint{4.554471in}{1.895708in}}%
\pgfpathlineto{\pgfqpoint{4.555047in}{1.891491in}}%
\pgfpathlineto{\pgfqpoint{4.555624in}{1.894615in}}%
\pgfpathlineto{\pgfqpoint{4.555816in}{1.894445in}}%
\pgfpathlineto{\pgfqpoint{4.556008in}{1.895651in}}%
\pgfpathlineto{\pgfqpoint{4.557545in}{1.907835in}}%
\pgfpathlineto{\pgfqpoint{4.557930in}{1.909771in}}%
\pgfpathlineto{\pgfqpoint{4.558506in}{1.907293in}}%
\pgfpathlineto{\pgfqpoint{4.558698in}{1.906908in}}%
\pgfpathlineto{\pgfqpoint{4.559851in}{1.913163in}}%
\pgfpathlineto{\pgfqpoint{4.559083in}{1.906189in}}%
\pgfpathlineto{\pgfqpoint{4.560043in}{1.912071in}}%
\pgfpathlineto{\pgfqpoint{4.560620in}{1.912801in}}%
\pgfpathlineto{\pgfqpoint{4.561004in}{1.909639in}}%
\pgfpathlineto{\pgfqpoint{4.561196in}{1.913366in}}%
\pgfpathlineto{\pgfqpoint{4.561773in}{1.906363in}}%
\pgfpathlineto{\pgfqpoint{4.562926in}{1.897958in}}%
\pgfpathlineto{\pgfqpoint{4.563310in}{1.892276in}}%
\pgfpathlineto{\pgfqpoint{4.564079in}{1.893169in}}%
\pgfpathlineto{\pgfqpoint{4.564848in}{1.897833in}}%
\pgfpathlineto{\pgfqpoint{4.565040in}{1.894499in}}%
\pgfpathlineto{\pgfqpoint{4.565616in}{1.885900in}}%
\pgfpathlineto{\pgfqpoint{4.566385in}{1.887064in}}%
\pgfpathlineto{\pgfqpoint{4.567538in}{1.893797in}}%
\pgfpathlineto{\pgfqpoint{4.566769in}{1.886926in}}%
\pgfpathlineto{\pgfqpoint{4.567922in}{1.889633in}}%
\pgfpathlineto{\pgfqpoint{4.568307in}{1.887940in}}%
\pgfpathlineto{\pgfqpoint{4.568883in}{1.891407in}}%
\pgfpathlineto{\pgfqpoint{4.569267in}{1.894492in}}%
\pgfpathlineto{\pgfqpoint{4.569652in}{1.890604in}}%
\pgfpathlineto{\pgfqpoint{4.570036in}{1.892469in}}%
\pgfpathlineto{\pgfqpoint{4.570228in}{1.889637in}}%
\pgfpathlineto{\pgfqpoint{4.570805in}{1.896037in}}%
\pgfpathlineto{\pgfqpoint{4.577146in}{1.927846in}}%
\pgfpathlineto{\pgfqpoint{4.577723in}{1.927413in}}%
\pgfpathlineto{\pgfqpoint{4.578492in}{1.923312in}}%
\pgfpathlineto{\pgfqpoint{4.578876in}{1.924584in}}%
\pgfpathlineto{\pgfqpoint{4.579068in}{1.927129in}}%
\pgfpathlineto{\pgfqpoint{4.579837in}{1.926952in}}%
\pgfpathlineto{\pgfqpoint{4.580413in}{1.928548in}}%
\pgfpathlineto{\pgfqpoint{4.580990in}{1.923481in}}%
\pgfpathlineto{\pgfqpoint{4.581758in}{1.926479in}}%
\pgfpathlineto{\pgfqpoint{4.581374in}{1.922785in}}%
\pgfpathlineto{\pgfqpoint{4.582143in}{1.923546in}}%
\pgfpathlineto{\pgfqpoint{4.583104in}{1.912385in}}%
\pgfpathlineto{\pgfqpoint{4.583488in}{1.916943in}}%
\pgfpathlineto{\pgfqpoint{4.585794in}{1.928710in}}%
\pgfpathlineto{\pgfqpoint{4.585986in}{1.926801in}}%
\pgfpathlineto{\pgfqpoint{4.586947in}{1.922950in}}%
\pgfpathlineto{\pgfqpoint{4.587139in}{1.924580in}}%
\pgfpathlineto{\pgfqpoint{4.587523in}{1.925117in}}%
\pgfpathlineto{\pgfqpoint{4.587716in}{1.927107in}}%
\pgfpathlineto{\pgfqpoint{4.588292in}{1.923679in}}%
\pgfpathlineto{\pgfqpoint{4.588676in}{1.922982in}}%
\pgfpathlineto{\pgfqpoint{4.589445in}{1.917400in}}%
\pgfpathlineto{\pgfqpoint{4.590022in}{1.919986in}}%
\pgfpathlineto{\pgfqpoint{4.591367in}{1.927688in}}%
\pgfpathlineto{\pgfqpoint{4.592135in}{1.923000in}}%
\pgfpathlineto{\pgfqpoint{4.592520in}{1.923169in}}%
\pgfpathlineto{\pgfqpoint{4.593096in}{1.927579in}}%
\pgfpathlineto{\pgfqpoint{4.593865in}{1.925032in}}%
\pgfpathlineto{\pgfqpoint{4.595210in}{1.917668in}}%
\pgfpathlineto{\pgfqpoint{4.595402in}{1.919534in}}%
\pgfpathlineto{\pgfqpoint{4.595594in}{1.916429in}}%
\pgfpathlineto{\pgfqpoint{4.596171in}{1.918686in}}%
\pgfpathlineto{\pgfqpoint{4.597132in}{1.916340in}}%
\pgfpathlineto{\pgfqpoint{4.597324in}{1.917239in}}%
\pgfpathlineto{\pgfqpoint{4.598861in}{1.922656in}}%
\pgfpathlineto{\pgfqpoint{4.600399in}{1.916119in}}%
\pgfpathlineto{\pgfqpoint{4.600591in}{1.917671in}}%
\pgfpathlineto{\pgfqpoint{4.601936in}{1.922305in}}%
\pgfpathlineto{\pgfqpoint{4.602513in}{1.924043in}}%
\pgfpathlineto{\pgfqpoint{4.602897in}{1.919754in}}%
\pgfpathlineto{\pgfqpoint{4.604050in}{1.927027in}}%
\pgfpathlineto{\pgfqpoint{4.604242in}{1.924071in}}%
\pgfpathlineto{\pgfqpoint{4.605972in}{1.916452in}}%
\pgfpathlineto{\pgfqpoint{4.606164in}{1.917259in}}%
\pgfpathlineto{\pgfqpoint{4.606356in}{1.917137in}}%
\pgfpathlineto{\pgfqpoint{4.606548in}{1.917698in}}%
\pgfpathlineto{\pgfqpoint{4.606740in}{1.915347in}}%
\pgfpathlineto{\pgfqpoint{4.607125in}{1.919436in}}%
\pgfpathlineto{\pgfqpoint{4.607317in}{1.919263in}}%
\pgfpathlineto{\pgfqpoint{4.607509in}{1.920175in}}%
\pgfpathlineto{\pgfqpoint{4.607701in}{1.917057in}}%
\pgfpathlineto{\pgfqpoint{4.607893in}{1.915884in}}%
\pgfpathlineto{\pgfqpoint{4.608085in}{1.919395in}}%
\pgfpathlineto{\pgfqpoint{4.609623in}{1.925649in}}%
\pgfpathlineto{\pgfqpoint{4.610968in}{1.921094in}}%
\pgfpathlineto{\pgfqpoint{4.613082in}{1.931395in}}%
\pgfpathlineto{\pgfqpoint{4.613274in}{1.929492in}}%
\pgfpathlineto{\pgfqpoint{4.613466in}{1.933283in}}%
\pgfpathlineto{\pgfqpoint{4.614235in}{1.929923in}}%
\pgfpathlineto{\pgfqpoint{4.614619in}{1.929524in}}%
\pgfpathlineto{\pgfqpoint{4.615964in}{1.931774in}}%
\pgfpathlineto{\pgfqpoint{4.616156in}{1.932571in}}%
\pgfpathlineto{\pgfqpoint{4.616349in}{1.931297in}}%
\pgfpathlineto{\pgfqpoint{4.616541in}{1.928166in}}%
\pgfpathlineto{\pgfqpoint{4.617117in}{1.931647in}}%
\pgfpathlineto{\pgfqpoint{4.617502in}{1.930324in}}%
\pgfpathlineto{\pgfqpoint{4.617886in}{1.930234in}}%
\pgfpathlineto{\pgfqpoint{4.618078in}{1.926782in}}%
\pgfpathlineto{\pgfqpoint{4.618462in}{1.931556in}}%
\pgfpathlineto{\pgfqpoint{4.619039in}{1.929425in}}%
\pgfpathlineto{\pgfqpoint{4.619231in}{1.928100in}}%
\pgfpathlineto{\pgfqpoint{4.619423in}{1.930562in}}%
\pgfpathlineto{\pgfqpoint{4.620192in}{1.928780in}}%
\pgfpathlineto{\pgfqpoint{4.620384in}{1.929211in}}%
\pgfpathlineto{\pgfqpoint{4.620768in}{1.927732in}}%
\pgfpathlineto{\pgfqpoint{4.620961in}{1.927860in}}%
\pgfpathlineto{\pgfqpoint{4.622498in}{1.915857in}}%
\pgfpathlineto{\pgfqpoint{4.624035in}{1.922627in}}%
\pgfpathlineto{\pgfqpoint{4.625573in}{1.927425in}}%
\pgfpathlineto{\pgfqpoint{4.625957in}{1.923750in}}%
\pgfpathlineto{\pgfqpoint{4.626534in}{1.928538in}}%
\pgfpathlineto{\pgfqpoint{4.626918in}{1.924104in}}%
\pgfpathlineto{\pgfqpoint{4.627687in}{1.925542in}}%
\pgfpathlineto{\pgfqpoint{4.627879in}{1.924631in}}%
\pgfpathlineto{\pgfqpoint{4.628263in}{1.919800in}}%
\pgfpathlineto{\pgfqpoint{4.628647in}{1.925325in}}%
\pgfpathlineto{\pgfqpoint{4.629224in}{1.921628in}}%
\pgfpathlineto{\pgfqpoint{4.630377in}{1.924299in}}%
\pgfpathlineto{\pgfqpoint{4.631338in}{1.919698in}}%
\pgfpathlineto{\pgfqpoint{4.631530in}{1.920905in}}%
\pgfpathlineto{\pgfqpoint{4.631914in}{1.928580in}}%
\pgfpathlineto{\pgfqpoint{4.632683in}{1.925264in}}%
\pgfpathlineto{\pgfqpoint{4.633067in}{1.923091in}}%
\pgfpathlineto{\pgfqpoint{4.634220in}{1.929746in}}%
\pgfpathlineto{\pgfqpoint{4.635181in}{1.922492in}}%
\pgfpathlineto{\pgfqpoint{4.635758in}{1.925265in}}%
\pgfpathlineto{\pgfqpoint{4.635950in}{1.925514in}}%
\pgfpathlineto{\pgfqpoint{4.636911in}{1.918727in}}%
\pgfpathlineto{\pgfqpoint{4.637295in}{1.920114in}}%
\pgfpathlineto{\pgfqpoint{4.637679in}{1.927283in}}%
\pgfpathlineto{\pgfqpoint{4.638640in}{1.924226in}}%
\pgfpathlineto{\pgfqpoint{4.639217in}{1.918512in}}%
\pgfpathlineto{\pgfqpoint{4.640562in}{1.920059in}}%
\pgfpathlineto{\pgfqpoint{4.641523in}{1.928804in}}%
\pgfpathlineto{\pgfqpoint{4.642291in}{1.927384in}}%
\pgfpathlineto{\pgfqpoint{4.642483in}{1.926331in}}%
\pgfpathlineto{\pgfqpoint{4.642676in}{1.929079in}}%
\pgfpathlineto{\pgfqpoint{4.644021in}{1.935403in}}%
\pgfpathlineto{\pgfqpoint{4.645366in}{1.927907in}}%
\pgfpathlineto{\pgfqpoint{4.645558in}{1.927869in}}%
\pgfpathlineto{\pgfqpoint{4.646903in}{1.920656in}}%
\pgfpathlineto{\pgfqpoint{4.647095in}{1.921594in}}%
\pgfpathlineto{\pgfqpoint{4.647864in}{1.921166in}}%
\pgfpathlineto{\pgfqpoint{4.648825in}{1.928842in}}%
\pgfpathlineto{\pgfqpoint{4.651131in}{1.919669in}}%
\pgfpathlineto{\pgfqpoint{4.651515in}{1.921864in}}%
\pgfpathlineto{\pgfqpoint{4.651900in}{1.918728in}}%
\pgfpathlineto{\pgfqpoint{4.652476in}{1.921236in}}%
\pgfpathlineto{\pgfqpoint{4.655359in}{1.908331in}}%
\pgfpathlineto{\pgfqpoint{4.653245in}{1.922074in}}%
\pgfpathlineto{\pgfqpoint{4.655551in}{1.909260in}}%
\pgfpathlineto{\pgfqpoint{4.657665in}{1.915914in}}%
\pgfpathlineto{\pgfqpoint{4.658049in}{1.910805in}}%
\pgfpathlineto{\pgfqpoint{4.658818in}{1.914809in}}%
\pgfpathlineto{\pgfqpoint{4.659394in}{1.914925in}}%
\pgfpathlineto{\pgfqpoint{4.660932in}{1.908699in}}%
\pgfpathlineto{\pgfqpoint{4.661316in}{1.909704in}}%
\pgfpathlineto{\pgfqpoint{4.661508in}{1.907983in}}%
\pgfpathlineto{\pgfqpoint{4.661700in}{1.907857in}}%
\pgfpathlineto{\pgfqpoint{4.662085in}{1.910261in}}%
\pgfpathlineto{\pgfqpoint{4.662661in}{1.908296in}}%
\pgfpathlineto{\pgfqpoint{4.662853in}{1.907642in}}%
\pgfpathlineto{\pgfqpoint{4.663045in}{1.908888in}}%
\pgfpathlineto{\pgfqpoint{4.663238in}{1.912164in}}%
\pgfpathlineto{\pgfqpoint{4.664006in}{1.906457in}}%
\pgfpathlineto{\pgfqpoint{4.664967in}{1.912622in}}%
\pgfpathlineto{\pgfqpoint{4.665351in}{1.909871in}}%
\pgfpathlineto{\pgfqpoint{4.667273in}{1.900253in}}%
\pgfpathlineto{\pgfqpoint{4.668618in}{1.905444in}}%
\pgfpathlineto{\pgfqpoint{4.668810in}{1.905955in}}%
\pgfpathlineto{\pgfqpoint{4.669003in}{1.904076in}}%
\pgfpathlineto{\pgfqpoint{4.669195in}{1.904634in}}%
\pgfpathlineto{\pgfqpoint{4.670540in}{1.900081in}}%
\pgfpathlineto{\pgfqpoint{4.670732in}{1.903555in}}%
\pgfpathlineto{\pgfqpoint{4.671693in}{1.901262in}}%
\pgfpathlineto{\pgfqpoint{4.671885in}{1.900141in}}%
\pgfpathlineto{\pgfqpoint{4.672462in}{1.902109in}}%
\pgfpathlineto{\pgfqpoint{4.672654in}{1.902485in}}%
\pgfpathlineto{\pgfqpoint{4.672846in}{1.899163in}}%
\pgfpathlineto{\pgfqpoint{4.673615in}{1.904213in}}%
\pgfpathlineto{\pgfqpoint{4.673807in}{1.904593in}}%
\pgfpathlineto{\pgfqpoint{4.673999in}{1.903743in}}%
\pgfpathlineto{\pgfqpoint{4.675729in}{1.890261in}}%
\pgfpathlineto{\pgfqpoint{4.675921in}{1.891932in}}%
\pgfpathlineto{\pgfqpoint{4.676113in}{1.889229in}}%
\pgfpathlineto{\pgfqpoint{4.676305in}{1.883588in}}%
\pgfpathlineto{\pgfqpoint{4.677266in}{1.885818in}}%
\pgfpathlineto{\pgfqpoint{4.677650in}{1.886914in}}%
\pgfpathlineto{\pgfqpoint{4.677842in}{1.885707in}}%
\pgfpathlineto{\pgfqpoint{4.678227in}{1.880818in}}%
\pgfpathlineto{\pgfqpoint{4.679188in}{1.882221in}}%
\pgfpathlineto{\pgfqpoint{4.680148in}{1.884206in}}%
\pgfpathlineto{\pgfqpoint{4.679764in}{1.880140in}}%
\pgfpathlineto{\pgfqpoint{4.680341in}{1.882557in}}%
\pgfpathlineto{\pgfqpoint{4.682262in}{1.872547in}}%
\pgfpathlineto{\pgfqpoint{4.682647in}{1.872739in}}%
\pgfpathlineto{\pgfqpoint{4.683992in}{1.880875in}}%
\pgfpathlineto{\pgfqpoint{4.684376in}{1.879875in}}%
\pgfpathlineto{\pgfqpoint{4.684760in}{1.877389in}}%
\pgfpathlineto{\pgfqpoint{4.684953in}{1.880495in}}%
\pgfpathlineto{\pgfqpoint{4.685529in}{1.878597in}}%
\pgfpathlineto{\pgfqpoint{4.686298in}{1.876186in}}%
\pgfpathlineto{\pgfqpoint{4.687259in}{1.867197in}}%
\pgfpathlineto{\pgfqpoint{4.688027in}{1.867345in}}%
\pgfpathlineto{\pgfqpoint{4.688412in}{1.867442in}}%
\pgfpathlineto{\pgfqpoint{4.688604in}{1.865255in}}%
\pgfpathlineto{\pgfqpoint{4.689565in}{1.876261in}}%
\pgfpathlineto{\pgfqpoint{4.689949in}{1.875387in}}%
\pgfpathlineto{\pgfqpoint{4.690333in}{1.871282in}}%
\pgfpathlineto{\pgfqpoint{4.690718in}{1.873888in}}%
\pgfpathlineto{\pgfqpoint{4.691102in}{1.877949in}}%
\pgfpathlineto{\pgfqpoint{4.691486in}{1.872219in}}%
\pgfpathlineto{\pgfqpoint{4.691871in}{1.867714in}}%
\pgfpathlineto{\pgfqpoint{4.692447in}{1.871363in}}%
\pgfpathlineto{\pgfqpoint{4.693792in}{1.879496in}}%
\pgfpathlineto{\pgfqpoint{4.694753in}{1.876404in}}%
\pgfpathlineto{\pgfqpoint{4.695330in}{1.875527in}}%
\pgfpathlineto{\pgfqpoint{4.696867in}{1.864451in}}%
\pgfpathlineto{\pgfqpoint{4.697251in}{1.864256in}}%
\pgfpathlineto{\pgfqpoint{4.698596in}{1.870356in}}%
\pgfpathlineto{\pgfqpoint{4.698789in}{1.870443in}}%
\pgfpathlineto{\pgfqpoint{4.699365in}{1.868801in}}%
\pgfpathlineto{\pgfqpoint{4.699557in}{1.872079in}}%
\pgfpathlineto{\pgfqpoint{4.699942in}{1.873573in}}%
\pgfpathlineto{\pgfqpoint{4.701287in}{1.885104in}}%
\pgfpathlineto{\pgfqpoint{4.701479in}{1.884119in}}%
\pgfpathlineto{\pgfqpoint{4.701863in}{1.886284in}}%
\pgfpathlineto{\pgfqpoint{4.703016in}{1.894089in}}%
\pgfpathlineto{\pgfqpoint{4.703209in}{1.891298in}}%
\pgfpathlineto{\pgfqpoint{4.704169in}{1.889815in}}%
\pgfpathlineto{\pgfqpoint{4.704362in}{1.890192in}}%
\pgfpathlineto{\pgfqpoint{4.704746in}{1.895652in}}%
\pgfpathlineto{\pgfqpoint{4.705322in}{1.893565in}}%
\pgfpathlineto{\pgfqpoint{4.706668in}{1.880541in}}%
\pgfpathlineto{\pgfqpoint{4.706860in}{1.883147in}}%
\pgfpathlineto{\pgfqpoint{4.707052in}{1.884790in}}%
\pgfpathlineto{\pgfqpoint{4.707436in}{1.880264in}}%
\pgfpathlineto{\pgfqpoint{4.708205in}{1.875624in}}%
\pgfpathlineto{\pgfqpoint{4.708589in}{1.876676in}}%
\pgfpathlineto{\pgfqpoint{4.709166in}{1.882297in}}%
\pgfpathlineto{\pgfqpoint{4.710127in}{1.881603in}}%
\pgfpathlineto{\pgfqpoint{4.711087in}{1.876458in}}%
\pgfpathlineto{\pgfqpoint{4.711472in}{1.880019in}}%
\pgfpathlineto{\pgfqpoint{4.711664in}{1.880596in}}%
\pgfpathlineto{\pgfqpoint{4.712048in}{1.878424in}}%
\pgfpathlineto{\pgfqpoint{4.712240in}{1.876350in}}%
\pgfpathlineto{\pgfqpoint{4.712625in}{1.878692in}}%
\pgfpathlineto{\pgfqpoint{4.712817in}{1.878653in}}%
\pgfpathlineto{\pgfqpoint{4.713586in}{1.883927in}}%
\pgfpathlineto{\pgfqpoint{4.713970in}{1.882101in}}%
\pgfpathlineto{\pgfqpoint{4.714162in}{1.880660in}}%
\pgfpathlineto{\pgfqpoint{4.714354in}{1.883888in}}%
\pgfpathlineto{\pgfqpoint{4.714931in}{1.882959in}}%
\pgfpathlineto{\pgfqpoint{4.716660in}{1.892998in}}%
\pgfpathlineto{\pgfqpoint{4.717045in}{1.893596in}}%
\pgfpathlineto{\pgfqpoint{4.717237in}{1.891396in}}%
\pgfpathlineto{\pgfqpoint{4.717813in}{1.887553in}}%
\pgfpathlineto{\pgfqpoint{4.718198in}{1.892184in}}%
\pgfpathlineto{\pgfqpoint{4.718774in}{1.899477in}}%
\pgfpathlineto{\pgfqpoint{4.719351in}{1.898040in}}%
\pgfpathlineto{\pgfqpoint{4.720119in}{1.892683in}}%
\pgfpathlineto{\pgfqpoint{4.720504in}{1.897408in}}%
\pgfpathlineto{\pgfqpoint{4.722810in}{1.888863in}}%
\pgfpathlineto{\pgfqpoint{4.723002in}{1.889080in}}%
\pgfpathlineto{\pgfqpoint{4.723963in}{1.896389in}}%
\pgfpathlineto{\pgfqpoint{4.724347in}{1.895602in}}%
\pgfpathlineto{\pgfqpoint{4.724539in}{1.894644in}}%
\pgfpathlineto{\pgfqpoint{4.724731in}{1.897959in}}%
\pgfpathlineto{\pgfqpoint{4.725308in}{1.902911in}}%
\pgfpathlineto{\pgfqpoint{4.726461in}{1.906732in}}%
\pgfpathlineto{\pgfqpoint{4.727230in}{1.894990in}}%
\pgfpathlineto{\pgfqpoint{4.727998in}{1.898643in}}%
\pgfpathlineto{\pgfqpoint{4.728190in}{1.897789in}}%
\pgfpathlineto{\pgfqpoint{4.728575in}{1.900481in}}%
\pgfpathlineto{\pgfqpoint{4.729920in}{1.909303in}}%
\pgfpathlineto{\pgfqpoint{4.730112in}{1.907239in}}%
\pgfpathlineto{\pgfqpoint{4.730881in}{1.910723in}}%
\pgfpathlineto{\pgfqpoint{4.731073in}{1.907699in}}%
\pgfpathlineto{\pgfqpoint{4.731265in}{1.908756in}}%
\pgfpathlineto{\pgfqpoint{4.731457in}{1.903836in}}%
\pgfpathlineto{\pgfqpoint{4.731649in}{1.905605in}}%
\pgfpathlineto{\pgfqpoint{4.731842in}{1.903880in}}%
\pgfpathlineto{\pgfqpoint{4.732418in}{1.908031in}}%
\pgfpathlineto{\pgfqpoint{4.732610in}{1.905800in}}%
\pgfpathlineto{\pgfqpoint{4.732995in}{1.910471in}}%
\pgfpathlineto{\pgfqpoint{4.733571in}{1.907129in}}%
\pgfpathlineto{\pgfqpoint{4.733763in}{1.906117in}}%
\pgfpathlineto{\pgfqpoint{4.734340in}{1.908954in}}%
\pgfpathlineto{\pgfqpoint{4.734532in}{1.907215in}}%
\pgfpathlineto{\pgfqpoint{4.736261in}{1.915129in}}%
\pgfpathlineto{\pgfqpoint{4.737414in}{1.910269in}}%
\pgfpathlineto{\pgfqpoint{4.737607in}{1.910520in}}%
\pgfpathlineto{\pgfqpoint{4.737799in}{1.909728in}}%
\pgfpathlineto{\pgfqpoint{4.737991in}{1.911447in}}%
\pgfpathlineto{\pgfqpoint{4.738183in}{1.913036in}}%
\pgfpathlineto{\pgfqpoint{4.738760in}{1.908931in}}%
\pgfpathlineto{\pgfqpoint{4.738952in}{1.908895in}}%
\pgfpathlineto{\pgfqpoint{4.740873in}{1.927683in}}%
\pgfpathlineto{\pgfqpoint{4.741066in}{1.926949in}}%
\pgfpathlineto{\pgfqpoint{4.742603in}{1.921702in}}%
\pgfpathlineto{\pgfqpoint{4.742795in}{1.922621in}}%
\pgfpathlineto{\pgfqpoint{4.742987in}{1.921416in}}%
\pgfpathlineto{\pgfqpoint{4.743564in}{1.913798in}}%
\pgfpathlineto{\pgfqpoint{4.744332in}{1.915703in}}%
\pgfpathlineto{\pgfqpoint{4.745101in}{1.923436in}}%
\pgfpathlineto{\pgfqpoint{4.745485in}{1.916460in}}%
\pgfpathlineto{\pgfqpoint{4.746062in}{1.917340in}}%
\pgfpathlineto{\pgfqpoint{4.747023in}{1.924515in}}%
\pgfpathlineto{\pgfqpoint{4.747215in}{1.920518in}}%
\pgfpathlineto{\pgfqpoint{4.747407in}{1.921014in}}%
\pgfpathlineto{\pgfqpoint{4.747599in}{1.919365in}}%
\pgfpathlineto{\pgfqpoint{4.747791in}{1.918833in}}%
\pgfpathlineto{\pgfqpoint{4.747984in}{1.921132in}}%
\pgfpathlineto{\pgfqpoint{4.748368in}{1.921556in}}%
\pgfpathlineto{\pgfqpoint{4.749905in}{1.931241in}}%
\pgfpathlineto{\pgfqpoint{4.750290in}{1.933360in}}%
\pgfpathlineto{\pgfqpoint{4.750482in}{1.930244in}}%
\pgfpathlineto{\pgfqpoint{4.750674in}{1.930517in}}%
\pgfpathlineto{\pgfqpoint{4.751827in}{1.931041in}}%
\pgfpathlineto{\pgfqpoint{4.752019in}{1.932167in}}%
\pgfpathlineto{\pgfqpoint{4.752596in}{1.930580in}}%
\pgfpathlineto{\pgfqpoint{4.752788in}{1.931543in}}%
\pgfpathlineto{\pgfqpoint{4.753557in}{1.929001in}}%
\pgfpathlineto{\pgfqpoint{4.753749in}{1.930158in}}%
\pgfpathlineto{\pgfqpoint{4.754133in}{1.932059in}}%
\pgfpathlineto{\pgfqpoint{4.754517in}{1.930016in}}%
\pgfpathlineto{\pgfqpoint{4.754902in}{1.930948in}}%
\pgfpathlineto{\pgfqpoint{4.758937in}{1.918758in}}%
\pgfpathlineto{\pgfqpoint{4.759129in}{1.922075in}}%
\pgfpathlineto{\pgfqpoint{4.760090in}{1.919584in}}%
\pgfpathlineto{\pgfqpoint{4.761243in}{1.924210in}}%
\pgfpathlineto{\pgfqpoint{4.760475in}{1.918665in}}%
\pgfpathlineto{\pgfqpoint{4.761435in}{1.922712in}}%
\pgfpathlineto{\pgfqpoint{4.763165in}{1.906564in}}%
\pgfpathlineto{\pgfqpoint{4.763934in}{1.910456in}}%
\pgfpathlineto{\pgfqpoint{4.764126in}{1.908546in}}%
\pgfpathlineto{\pgfqpoint{4.765471in}{1.904308in}}%
\pgfpathlineto{\pgfqpoint{4.765663in}{1.905634in}}%
\pgfpathlineto{\pgfqpoint{4.765855in}{1.902365in}}%
\pgfpathlineto{\pgfqpoint{4.767393in}{1.887348in}}%
\pgfpathlineto{\pgfqpoint{4.767969in}{1.890711in}}%
\pgfpathlineto{\pgfqpoint{4.768546in}{1.889585in}}%
\pgfpathlineto{\pgfqpoint{4.769699in}{1.893899in}}%
\pgfpathlineto{\pgfqpoint{4.770275in}{1.890726in}}%
\pgfpathlineto{\pgfqpoint{4.770659in}{1.891415in}}%
\pgfpathlineto{\pgfqpoint{4.770852in}{1.894454in}}%
\pgfpathlineto{\pgfqpoint{4.771812in}{1.892550in}}%
\pgfpathlineto{\pgfqpoint{4.772005in}{1.892612in}}%
\pgfpathlineto{\pgfqpoint{4.772965in}{1.896913in}}%
\pgfpathlineto{\pgfqpoint{4.773350in}{1.895421in}}%
\pgfpathlineto{\pgfqpoint{4.776617in}{1.870710in}}%
\pgfpathlineto{\pgfqpoint{4.776809in}{1.872546in}}%
\pgfpathlineto{\pgfqpoint{4.777962in}{1.881436in}}%
\pgfpathlineto{\pgfqpoint{4.779499in}{1.879477in}}%
\pgfpathlineto{\pgfqpoint{4.779691in}{1.879374in}}%
\pgfpathlineto{\pgfqpoint{4.779884in}{1.880179in}}%
\pgfpathlineto{\pgfqpoint{4.780076in}{1.877421in}}%
\pgfpathlineto{\pgfqpoint{4.780460in}{1.881671in}}%
\pgfpathlineto{\pgfqpoint{4.780844in}{1.880771in}}%
\pgfpathlineto{\pgfqpoint{4.781421in}{1.877400in}}%
\pgfpathlineto{\pgfqpoint{4.781613in}{1.882241in}}%
\pgfpathlineto{\pgfqpoint{4.781997in}{1.879598in}}%
\pgfpathlineto{\pgfqpoint{4.783727in}{1.867980in}}%
\pgfpathlineto{\pgfqpoint{4.784111in}{1.868981in}}%
\pgfpathlineto{\pgfqpoint{4.784303in}{1.868376in}}%
\pgfpathlineto{\pgfqpoint{4.784496in}{1.869711in}}%
\pgfpathlineto{\pgfqpoint{4.784880in}{1.871882in}}%
\pgfpathlineto{\pgfqpoint{4.785456in}{1.868420in}}%
\pgfpathlineto{\pgfqpoint{4.786994in}{1.860925in}}%
\pgfpathlineto{\pgfqpoint{4.788339in}{1.865141in}}%
\pgfpathlineto{\pgfqpoint{4.788531in}{1.862474in}}%
\pgfpathlineto{\pgfqpoint{4.788915in}{1.865328in}}%
\pgfpathlineto{\pgfqpoint{4.789108in}{1.863969in}}%
\pgfpathlineto{\pgfqpoint{4.790453in}{1.874201in}}%
\pgfpathlineto{\pgfqpoint{4.790645in}{1.872950in}}%
\pgfpathlineto{\pgfqpoint{4.790837in}{1.872818in}}%
\pgfpathlineto{\pgfqpoint{4.791029in}{1.870831in}}%
\pgfpathlineto{\pgfqpoint{4.791414in}{1.874962in}}%
\pgfpathlineto{\pgfqpoint{4.791798in}{1.873694in}}%
\pgfpathlineto{\pgfqpoint{4.791990in}{1.873702in}}%
\pgfpathlineto{\pgfqpoint{4.793527in}{1.868993in}}%
\pgfpathlineto{\pgfqpoint{4.793720in}{1.870441in}}%
\pgfpathlineto{\pgfqpoint{4.794104in}{1.867560in}}%
\pgfpathlineto{\pgfqpoint{4.795833in}{1.855156in}}%
\pgfpathlineto{\pgfqpoint{4.796410in}{1.857125in}}%
\pgfpathlineto{\pgfqpoint{4.797371in}{1.851353in}}%
\pgfpathlineto{\pgfqpoint{4.797755in}{1.852511in}}%
\pgfpathlineto{\pgfqpoint{4.798140in}{1.853693in}}%
\pgfpathlineto{\pgfqpoint{4.798908in}{1.856833in}}%
\pgfpathlineto{\pgfqpoint{4.799293in}{1.856390in}}%
\pgfpathlineto{\pgfqpoint{4.800638in}{1.848029in}}%
\pgfpathlineto{\pgfqpoint{4.800830in}{1.849343in}}%
\pgfpathlineto{\pgfqpoint{4.801791in}{1.850744in}}%
\pgfpathlineto{\pgfqpoint{4.801406in}{1.849195in}}%
\pgfpathlineto{\pgfqpoint{4.801983in}{1.850018in}}%
\pgfpathlineto{\pgfqpoint{4.802175in}{1.850618in}}%
\pgfpathlineto{\pgfqpoint{4.802944in}{1.857645in}}%
\pgfpathlineto{\pgfqpoint{4.803328in}{1.855627in}}%
\pgfpathlineto{\pgfqpoint{4.803905in}{1.849235in}}%
\pgfpathlineto{\pgfqpoint{4.804673in}{1.850678in}}%
\pgfpathlineto{\pgfqpoint{4.804865in}{1.852139in}}%
\pgfpathlineto{\pgfqpoint{4.805442in}{1.849167in}}%
\pgfpathlineto{\pgfqpoint{4.805634in}{1.849331in}}%
\pgfpathlineto{\pgfqpoint{4.806211in}{1.855318in}}%
\pgfpathlineto{\pgfqpoint{4.806787in}{1.851645in}}%
\pgfpathlineto{\pgfqpoint{4.806979in}{1.852327in}}%
\pgfpathlineto{\pgfqpoint{4.807364in}{1.850026in}}%
\pgfpathlineto{\pgfqpoint{4.807556in}{1.851698in}}%
\pgfpathlineto{\pgfqpoint{4.808709in}{1.845434in}}%
\pgfpathlineto{\pgfqpoint{4.808901in}{1.845885in}}%
\pgfpathlineto{\pgfqpoint{4.810246in}{1.851507in}}%
\pgfpathlineto{\pgfqpoint{4.810823in}{1.852906in}}%
\pgfpathlineto{\pgfqpoint{4.811783in}{1.844083in}}%
\pgfpathlineto{\pgfqpoint{4.812168in}{1.844163in}}%
\pgfpathlineto{\pgfqpoint{4.812360in}{1.843205in}}%
\pgfpathlineto{\pgfqpoint{4.813129in}{1.839764in}}%
\pgfpathlineto{\pgfqpoint{4.813513in}{1.841983in}}%
\pgfpathlineto{\pgfqpoint{4.813897in}{1.840855in}}%
\pgfpathlineto{\pgfqpoint{4.814089in}{1.842080in}}%
\pgfpathlineto{\pgfqpoint{4.815242in}{1.834531in}}%
\pgfpathlineto{\pgfqpoint{4.815627in}{1.835160in}}%
\pgfpathlineto{\pgfqpoint{4.816011in}{1.838248in}}%
\pgfpathlineto{\pgfqpoint{4.816395in}{1.833253in}}%
\pgfpathlineto{\pgfqpoint{4.816588in}{1.830177in}}%
\pgfpathlineto{\pgfqpoint{4.817356in}{1.835613in}}%
\pgfpathlineto{\pgfqpoint{4.817741in}{1.833808in}}%
\pgfpathlineto{\pgfqpoint{4.818125in}{1.829358in}}%
\pgfpathlineto{\pgfqpoint{4.818509in}{1.834523in}}%
\pgfpathlineto{\pgfqpoint{4.819662in}{1.837185in}}%
\pgfpathlineto{\pgfqpoint{4.821392in}{1.849434in}}%
\pgfpathlineto{\pgfqpoint{4.821584in}{1.848629in}}%
\pgfpathlineto{\pgfqpoint{4.822929in}{1.857460in}}%
\pgfpathlineto{\pgfqpoint{4.823121in}{1.857439in}}%
\pgfpathlineto{\pgfqpoint{4.823314in}{1.859072in}}%
\pgfpathlineto{\pgfqpoint{4.823890in}{1.854648in}}%
\pgfpathlineto{\pgfqpoint{4.824659in}{1.857978in}}%
\pgfpathlineto{\pgfqpoint{4.824851in}{1.861101in}}%
\pgfpathlineto{\pgfqpoint{4.825812in}{1.859951in}}%
\pgfpathlineto{\pgfqpoint{4.826196in}{1.861117in}}%
\pgfpathlineto{\pgfqpoint{4.826388in}{1.859551in}}%
\pgfpathlineto{\pgfqpoint{4.826580in}{1.859594in}}%
\pgfpathlineto{\pgfqpoint{4.827157in}{1.852728in}}%
\pgfpathlineto{\pgfqpoint{4.827926in}{1.856062in}}%
\pgfpathlineto{\pgfqpoint{4.828310in}{1.859060in}}%
\pgfpathlineto{\pgfqpoint{4.828886in}{1.856159in}}%
\pgfpathlineto{\pgfqpoint{4.829655in}{1.852904in}}%
\pgfpathlineto{\pgfqpoint{4.830424in}{1.853750in}}%
\pgfpathlineto{\pgfqpoint{4.831000in}{1.849753in}}%
\pgfpathlineto{\pgfqpoint{4.832730in}{1.855844in}}%
\pgfpathlineto{\pgfqpoint{4.832922in}{1.854766in}}%
\pgfpathlineto{\pgfqpoint{4.833114in}{1.853223in}}%
\pgfpathlineto{\pgfqpoint{4.833498in}{1.856904in}}%
\pgfpathlineto{\pgfqpoint{4.834267in}{1.859303in}}%
\pgfpathlineto{\pgfqpoint{4.834459in}{1.856893in}}%
\pgfpathlineto{\pgfqpoint{4.836765in}{1.845788in}}%
\pgfpathlineto{\pgfqpoint{4.838110in}{1.840155in}}%
\pgfpathlineto{\pgfqpoint{4.838303in}{1.842056in}}%
\pgfpathlineto{\pgfqpoint{4.839263in}{1.846339in}}%
\pgfpathlineto{\pgfqpoint{4.839648in}{1.844218in}}%
\pgfpathlineto{\pgfqpoint{4.840416in}{1.845576in}}%
\pgfpathlineto{\pgfqpoint{4.840609in}{1.844273in}}%
\pgfpathlineto{\pgfqpoint{4.842915in}{1.835565in}}%
\pgfpathlineto{\pgfqpoint{4.840993in}{1.844827in}}%
\pgfpathlineto{\pgfqpoint{4.844068in}{1.836648in}}%
\pgfpathlineto{\pgfqpoint{4.844260in}{1.841074in}}%
\pgfpathlineto{\pgfqpoint{4.845221in}{1.840588in}}%
\pgfpathlineto{\pgfqpoint{4.845797in}{1.837183in}}%
\pgfpathlineto{\pgfqpoint{4.846181in}{1.839762in}}%
\pgfpathlineto{\pgfqpoint{4.849064in}{1.852662in}}%
\pgfpathlineto{\pgfqpoint{4.850025in}{1.849649in}}%
\pgfpathlineto{\pgfqpoint{4.850409in}{1.854583in}}%
\pgfpathlineto{\pgfqpoint{4.850986in}{1.852419in}}%
\pgfpathlineto{\pgfqpoint{4.851562in}{1.846590in}}%
\pgfpathlineto{\pgfqpoint{4.852331in}{1.848671in}}%
\pgfpathlineto{\pgfqpoint{4.852715in}{1.847791in}}%
\pgfpathlineto{\pgfqpoint{4.853100in}{1.849361in}}%
\pgfpathlineto{\pgfqpoint{4.853292in}{1.852325in}}%
\pgfpathlineto{\pgfqpoint{4.853868in}{1.845603in}}%
\pgfpathlineto{\pgfqpoint{4.854253in}{1.842638in}}%
\pgfpathlineto{\pgfqpoint{4.854637in}{1.844946in}}%
\pgfpathlineto{\pgfqpoint{4.855213in}{1.853176in}}%
\pgfpathlineto{\pgfqpoint{4.855790in}{1.850257in}}%
\pgfpathlineto{\pgfqpoint{4.856366in}{1.845406in}}%
\pgfpathlineto{\pgfqpoint{4.856751in}{1.847727in}}%
\pgfpathlineto{\pgfqpoint{4.857327in}{1.851897in}}%
\pgfpathlineto{\pgfqpoint{4.857904in}{1.850881in}}%
\pgfpathlineto{\pgfqpoint{4.858288in}{1.849461in}}%
\pgfpathlineto{\pgfqpoint{4.858865in}{1.851846in}}%
\pgfpathlineto{\pgfqpoint{4.860594in}{1.862246in}}%
\pgfpathlineto{\pgfqpoint{4.861363in}{1.860006in}}%
\pgfpathlineto{\pgfqpoint{4.862516in}{1.854062in}}%
\pgfpathlineto{\pgfqpoint{4.862708in}{1.856185in}}%
\pgfpathlineto{\pgfqpoint{4.863092in}{1.858293in}}%
\pgfpathlineto{\pgfqpoint{4.863669in}{1.857294in}}%
\pgfpathlineto{\pgfqpoint{4.863861in}{1.856347in}}%
\pgfpathlineto{\pgfqpoint{4.864053in}{1.859268in}}%
\pgfpathlineto{\pgfqpoint{4.864437in}{1.858037in}}%
\pgfpathlineto{\pgfqpoint{4.867704in}{1.868568in}}%
\pgfpathlineto{\pgfqpoint{4.864822in}{1.857361in}}%
\pgfpathlineto{\pgfqpoint{4.867896in}{1.867960in}}%
\pgfpathlineto{\pgfqpoint{4.868089in}{1.866124in}}%
\pgfpathlineto{\pgfqpoint{4.868473in}{1.872994in}}%
\pgfpathlineto{\pgfqpoint{4.868857in}{1.871569in}}%
\pgfpathlineto{\pgfqpoint{4.869049in}{1.873256in}}%
\pgfpathlineto{\pgfqpoint{4.870010in}{1.882259in}}%
\pgfpathlineto{\pgfqpoint{4.870587in}{1.877030in}}%
\pgfpathlineto{\pgfqpoint{4.872124in}{1.883717in}}%
\pgfpathlineto{\pgfqpoint{4.872316in}{1.883547in}}%
\pgfpathlineto{\pgfqpoint{4.872893in}{1.876745in}}%
\pgfpathlineto{\pgfqpoint{4.873662in}{1.877566in}}%
\pgfpathlineto{\pgfqpoint{4.874046in}{1.875572in}}%
\pgfpathlineto{\pgfqpoint{4.874430in}{1.877197in}}%
\pgfpathlineto{\pgfqpoint{4.874622in}{1.879019in}}%
\pgfpathlineto{\pgfqpoint{4.875007in}{1.873983in}}%
\pgfpathlineto{\pgfqpoint{4.875199in}{1.873032in}}%
\pgfpathlineto{\pgfqpoint{4.875391in}{1.876664in}}%
\pgfpathlineto{\pgfqpoint{4.875583in}{1.876963in}}%
\pgfpathlineto{\pgfqpoint{4.875968in}{1.869632in}}%
\pgfpathlineto{\pgfqpoint{4.876736in}{1.873475in}}%
\pgfpathlineto{\pgfqpoint{4.878466in}{1.863311in}}%
\pgfpathlineto{\pgfqpoint{4.877121in}{1.873634in}}%
\pgfpathlineto{\pgfqpoint{4.878658in}{1.864143in}}%
\pgfpathlineto{\pgfqpoint{4.879234in}{1.865284in}}%
\pgfpathlineto{\pgfqpoint{4.879042in}{1.862544in}}%
\pgfpathlineto{\pgfqpoint{4.879427in}{1.862943in}}%
\pgfpathlineto{\pgfqpoint{4.880387in}{1.860016in}}%
\pgfpathlineto{\pgfqpoint{4.880580in}{1.864495in}}%
\pgfpathlineto{\pgfqpoint{4.881540in}{1.862311in}}%
\pgfpathlineto{\pgfqpoint{4.881733in}{1.862083in}}%
\pgfpathlineto{\pgfqpoint{4.882886in}{1.856623in}}%
\pgfpathlineto{\pgfqpoint{4.884423in}{1.862542in}}%
\pgfpathlineto{\pgfqpoint{4.884999in}{1.858256in}}%
\pgfpathlineto{\pgfqpoint{4.885576in}{1.860958in}}%
\pgfpathlineto{\pgfqpoint{4.885768in}{1.860846in}}%
\pgfpathlineto{\pgfqpoint{4.886921in}{1.856601in}}%
\pgfpathlineto{\pgfqpoint{4.886537in}{1.861085in}}%
\pgfpathlineto{\pgfqpoint{4.887113in}{1.857199in}}%
\pgfpathlineto{\pgfqpoint{4.887305in}{1.857613in}}%
\pgfpathlineto{\pgfqpoint{4.888458in}{1.850710in}}%
\pgfpathlineto{\pgfqpoint{4.888651in}{1.852730in}}%
\pgfpathlineto{\pgfqpoint{4.889419in}{1.855150in}}%
\pgfpathlineto{\pgfqpoint{4.889611in}{1.853481in}}%
\pgfpathlineto{\pgfqpoint{4.890380in}{1.851301in}}%
\pgfpathlineto{\pgfqpoint{4.890188in}{1.855468in}}%
\pgfpathlineto{\pgfqpoint{4.890764in}{1.853213in}}%
\pgfpathlineto{\pgfqpoint{4.891533in}{1.857013in}}%
\pgfpathlineto{\pgfqpoint{4.891725in}{1.855646in}}%
\pgfpathlineto{\pgfqpoint{4.892110in}{1.852680in}}%
\pgfpathlineto{\pgfqpoint{4.892494in}{1.854902in}}%
\pgfpathlineto{\pgfqpoint{4.894608in}{1.877615in}}%
\pgfpathlineto{\pgfqpoint{4.894800in}{1.876085in}}%
\pgfpathlineto{\pgfqpoint{4.896145in}{1.865114in}}%
\pgfpathlineto{\pgfqpoint{4.896337in}{1.867087in}}%
\pgfpathlineto{\pgfqpoint{4.897875in}{1.876171in}}%
\pgfpathlineto{\pgfqpoint{4.898067in}{1.874848in}}%
\pgfpathlineto{\pgfqpoint{4.898643in}{1.877790in}}%
\pgfpathlineto{\pgfqpoint{4.898836in}{1.877869in}}%
\pgfpathlineto{\pgfqpoint{4.899028in}{1.875851in}}%
\pgfpathlineto{\pgfqpoint{4.899412in}{1.880131in}}%
\pgfpathlineto{\pgfqpoint{4.899604in}{1.879917in}}%
\pgfpathlineto{\pgfqpoint{4.899796in}{1.881458in}}%
\pgfpathlineto{\pgfqpoint{4.900181in}{1.876396in}}%
\pgfpathlineto{\pgfqpoint{4.900373in}{1.874483in}}%
\pgfpathlineto{\pgfqpoint{4.900757in}{1.877591in}}%
\pgfpathlineto{\pgfqpoint{4.901334in}{1.882726in}}%
\pgfpathlineto{\pgfqpoint{4.901718in}{1.877110in}}%
\pgfpathlineto{\pgfqpoint{4.902871in}{1.873229in}}%
\pgfpathlineto{\pgfqpoint{4.904024in}{1.875010in}}%
\pgfpathlineto{\pgfqpoint{4.904216in}{1.878498in}}%
\pgfpathlineto{\pgfqpoint{4.904601in}{1.874809in}}%
\pgfpathlineto{\pgfqpoint{4.905177in}{1.876235in}}%
\pgfpathlineto{\pgfqpoint{4.905369in}{1.873804in}}%
\pgfpathlineto{\pgfqpoint{4.906138in}{1.874752in}}%
\pgfpathlineto{\pgfqpoint{4.907291in}{1.881073in}}%
\pgfpathlineto{\pgfqpoint{4.907675in}{1.877960in}}%
\pgfpathlineto{\pgfqpoint{4.907867in}{1.875706in}}%
\pgfpathlineto{\pgfqpoint{4.908444in}{1.880049in}}%
\pgfpathlineto{\pgfqpoint{4.908828in}{1.882787in}}%
\pgfpathlineto{\pgfqpoint{4.909405in}{1.880050in}}%
\pgfpathlineto{\pgfqpoint{4.909597in}{1.876230in}}%
\pgfpathlineto{\pgfqpoint{4.910366in}{1.882666in}}%
\pgfpathlineto{\pgfqpoint{4.911134in}{1.884507in}}%
\pgfpathlineto{\pgfqpoint{4.911326in}{1.882514in}}%
\pgfpathlineto{\pgfqpoint{4.911519in}{1.882157in}}%
\pgfpathlineto{\pgfqpoint{4.911711in}{1.884269in}}%
\pgfpathlineto{\pgfqpoint{4.912095in}{1.881851in}}%
\pgfpathlineto{\pgfqpoint{4.912672in}{1.883215in}}%
\pgfpathlineto{\pgfqpoint{4.912864in}{1.884624in}}%
\pgfpathlineto{\pgfqpoint{4.913056in}{1.880632in}}%
\pgfpathlineto{\pgfqpoint{4.913440in}{1.883806in}}%
\pgfpathlineto{\pgfqpoint{4.913825in}{1.880164in}}%
\pgfpathlineto{\pgfqpoint{4.914401in}{1.881786in}}%
\pgfpathlineto{\pgfqpoint{4.914978in}{1.885691in}}%
\pgfpathlineto{\pgfqpoint{4.915362in}{1.880391in}}%
\pgfpathlineto{\pgfqpoint{4.916323in}{1.874682in}}%
\pgfpathlineto{\pgfqpoint{4.916899in}{1.876368in}}%
\pgfpathlineto{\pgfqpoint{4.917860in}{1.883426in}}%
\pgfpathlineto{\pgfqpoint{4.918244in}{1.880259in}}%
\pgfpathlineto{\pgfqpoint{4.918437in}{1.882219in}}%
\pgfpathlineto{\pgfqpoint{4.919013in}{1.879183in}}%
\pgfpathlineto{\pgfqpoint{4.919205in}{1.880505in}}%
\pgfpathlineto{\pgfqpoint{4.919397in}{1.878433in}}%
\pgfpathlineto{\pgfqpoint{4.919974in}{1.881914in}}%
\pgfpathlineto{\pgfqpoint{4.920166in}{1.882235in}}%
\pgfpathlineto{\pgfqpoint{4.920358in}{1.881246in}}%
\pgfpathlineto{\pgfqpoint{4.921704in}{1.876614in}}%
\pgfpathlineto{\pgfqpoint{4.922664in}{1.881606in}}%
\pgfpathlineto{\pgfqpoint{4.923049in}{1.881085in}}%
\pgfpathlineto{\pgfqpoint{4.923433in}{1.882447in}}%
\pgfpathlineto{\pgfqpoint{4.923625in}{1.880623in}}%
\pgfpathlineto{\pgfqpoint{4.925163in}{1.866796in}}%
\pgfpathlineto{\pgfqpoint{4.925739in}{1.872374in}}%
\pgfpathlineto{\pgfqpoint{4.926316in}{1.868646in}}%
\pgfpathlineto{\pgfqpoint{4.927084in}{1.862319in}}%
\pgfpathlineto{\pgfqpoint{4.927276in}{1.865851in}}%
\pgfpathlineto{\pgfqpoint{4.927469in}{1.868567in}}%
\pgfpathlineto{\pgfqpoint{4.928045in}{1.865448in}}%
\pgfpathlineto{\pgfqpoint{4.929198in}{1.859339in}}%
\pgfpathlineto{\pgfqpoint{4.929390in}{1.860617in}}%
\pgfpathlineto{\pgfqpoint{4.929775in}{1.861740in}}%
\pgfpathlineto{\pgfqpoint{4.929967in}{1.860663in}}%
\pgfpathlineto{\pgfqpoint{4.930351in}{1.857818in}}%
\pgfpathlineto{\pgfqpoint{4.930735in}{1.861898in}}%
\pgfpathlineto{\pgfqpoint{4.930928in}{1.860348in}}%
\pgfpathlineto{\pgfqpoint{4.931696in}{1.861107in}}%
\pgfpathlineto{\pgfqpoint{4.931312in}{1.860062in}}%
\pgfpathlineto{\pgfqpoint{4.931888in}{1.860663in}}%
\pgfpathlineto{\pgfqpoint{4.933041in}{1.857403in}}%
\pgfpathlineto{\pgfqpoint{4.933426in}{1.858776in}}%
\pgfpathlineto{\pgfqpoint{4.934002in}{1.857232in}}%
\pgfpathlineto{\pgfqpoint{4.934194in}{1.857746in}}%
\pgfpathlineto{\pgfqpoint{4.935155in}{1.866072in}}%
\pgfpathlineto{\pgfqpoint{4.935347in}{1.863324in}}%
\pgfpathlineto{\pgfqpoint{4.936693in}{1.857346in}}%
\pgfpathlineto{\pgfqpoint{4.938038in}{1.867565in}}%
\pgfpathlineto{\pgfqpoint{4.938614in}{1.867162in}}%
\pgfpathlineto{\pgfqpoint{4.939767in}{1.862574in}}%
\pgfpathlineto{\pgfqpoint{4.939191in}{1.868005in}}%
\pgfpathlineto{\pgfqpoint{4.939959in}{1.863329in}}%
\pgfpathlineto{\pgfqpoint{4.940728in}{1.863840in}}%
\pgfpathlineto{\pgfqpoint{4.940920in}{1.857884in}}%
\pgfpathlineto{\pgfqpoint{4.941881in}{1.861764in}}%
\pgfpathlineto{\pgfqpoint{4.942650in}{1.867827in}}%
\pgfpathlineto{\pgfqpoint{4.943034in}{1.865285in}}%
\pgfpathlineto{\pgfqpoint{4.943995in}{1.858757in}}%
\pgfpathlineto{\pgfqpoint{4.944187in}{1.860936in}}%
\pgfpathlineto{\pgfqpoint{4.944379in}{1.861647in}}%
\pgfpathlineto{\pgfqpoint{4.944572in}{1.860571in}}%
\pgfpathlineto{\pgfqpoint{4.944956in}{1.856842in}}%
\pgfpathlineto{\pgfqpoint{4.945532in}{1.860306in}}%
\pgfpathlineto{\pgfqpoint{4.945725in}{1.860652in}}%
\pgfpathlineto{\pgfqpoint{4.945917in}{1.859890in}}%
\pgfpathlineto{\pgfqpoint{4.946109in}{1.857403in}}%
\pgfpathlineto{\pgfqpoint{4.946878in}{1.861157in}}%
\pgfpathlineto{\pgfqpoint{4.947070in}{1.859266in}}%
\pgfpathlineto{\pgfqpoint{4.947454in}{1.860899in}}%
\pgfpathlineto{\pgfqpoint{4.947646in}{1.857698in}}%
\pgfpathlineto{\pgfqpoint{4.949184in}{1.850811in}}%
\pgfpathlineto{\pgfqpoint{4.949376in}{1.851290in}}%
\pgfpathlineto{\pgfqpoint{4.949760in}{1.847747in}}%
\pgfpathlineto{\pgfqpoint{4.950337in}{1.850930in}}%
\pgfpathlineto{\pgfqpoint{4.952066in}{1.855354in}}%
\pgfpathlineto{\pgfqpoint{4.952643in}{1.851565in}}%
\pgfpathlineto{\pgfqpoint{4.953411in}{1.852985in}}%
\pgfpathlineto{\pgfqpoint{4.954564in}{1.849319in}}%
\pgfpathlineto{\pgfqpoint{4.954756in}{1.849702in}}%
\pgfpathlineto{\pgfqpoint{4.956102in}{1.855203in}}%
\pgfpathlineto{\pgfqpoint{4.957831in}{1.846225in}}%
\pgfpathlineto{\pgfqpoint{4.958408in}{1.849057in}}%
\pgfpathlineto{\pgfqpoint{4.958792in}{1.845601in}}%
\pgfpathlineto{\pgfqpoint{4.959561in}{1.849861in}}%
\pgfpathlineto{\pgfqpoint{4.960329in}{1.840903in}}%
\pgfpathlineto{\pgfqpoint{4.960714in}{1.842747in}}%
\pgfpathlineto{\pgfqpoint{4.962059in}{1.852672in}}%
\pgfpathlineto{\pgfqpoint{4.962827in}{1.851990in}}%
\pgfpathlineto{\pgfqpoint{4.963212in}{1.852101in}}%
\pgfpathlineto{\pgfqpoint{4.966479in}{1.869155in}}%
\pgfpathlineto{\pgfqpoint{4.967055in}{1.870518in}}%
\pgfpathlineto{\pgfqpoint{4.967439in}{1.867547in}}%
\pgfpathlineto{\pgfqpoint{4.968977in}{1.875813in}}%
\pgfpathlineto{\pgfqpoint{4.969169in}{1.877211in}}%
\pgfpathlineto{\pgfqpoint{4.969361in}{1.875034in}}%
\pgfpathlineto{\pgfqpoint{4.969746in}{1.870017in}}%
\pgfpathlineto{\pgfqpoint{4.970322in}{1.873527in}}%
\pgfpathlineto{\pgfqpoint{4.973205in}{1.894370in}}%
\pgfpathlineto{\pgfqpoint{4.973973in}{1.892754in}}%
\pgfpathlineto{\pgfqpoint{4.974742in}{1.890264in}}%
\pgfpathlineto{\pgfqpoint{4.974934in}{1.891876in}}%
\pgfpathlineto{\pgfqpoint{4.975895in}{1.891511in}}%
\pgfpathlineto{\pgfqpoint{4.976471in}{1.896914in}}%
\pgfpathlineto{\pgfqpoint{4.976664in}{1.895101in}}%
\pgfpathlineto{\pgfqpoint{4.977048in}{1.898065in}}%
\pgfpathlineto{\pgfqpoint{4.977432in}{1.897247in}}%
\pgfpathlineto{\pgfqpoint{4.978201in}{1.894539in}}%
\pgfpathlineto{\pgfqpoint{4.978585in}{1.895164in}}%
\pgfpathlineto{\pgfqpoint{4.979162in}{1.900009in}}%
\pgfpathlineto{\pgfqpoint{4.979738in}{1.896841in}}%
\pgfpathlineto{\pgfqpoint{4.980315in}{1.897878in}}%
\pgfpathlineto{\pgfqpoint{4.980123in}{1.895883in}}%
\pgfpathlineto{\pgfqpoint{4.980507in}{1.897081in}}%
\pgfpathlineto{\pgfqpoint{4.982429in}{1.888725in}}%
\pgfpathlineto{\pgfqpoint{4.980891in}{1.897319in}}%
\pgfpathlineto{\pgfqpoint{4.983005in}{1.890185in}}%
\pgfpathlineto{\pgfqpoint{4.983389in}{1.887407in}}%
\pgfpathlineto{\pgfqpoint{4.984350in}{1.892346in}}%
\pgfpathlineto{\pgfqpoint{4.985888in}{1.883067in}}%
\pgfpathlineto{\pgfqpoint{4.986080in}{1.881987in}}%
\pgfpathlineto{\pgfqpoint{4.986272in}{1.886568in}}%
\pgfpathlineto{\pgfqpoint{4.986464in}{1.884133in}}%
\pgfpathlineto{\pgfqpoint{4.988194in}{1.901082in}}%
\pgfpathlineto{\pgfqpoint{4.989154in}{1.900261in}}%
\pgfpathlineto{\pgfqpoint{4.991076in}{1.889030in}}%
\pgfpathlineto{\pgfqpoint{4.989923in}{1.900881in}}%
\pgfpathlineto{\pgfqpoint{4.991268in}{1.891129in}}%
\pgfpathlineto{\pgfqpoint{4.992613in}{1.897765in}}%
\pgfpathlineto{\pgfqpoint{4.992998in}{1.895771in}}%
\pgfpathlineto{\pgfqpoint{4.994535in}{1.889687in}}%
\pgfpathlineto{\pgfqpoint{4.996265in}{1.899752in}}%
\pgfpathlineto{\pgfqpoint{4.996457in}{1.899261in}}%
\pgfpathlineto{\pgfqpoint{4.996649in}{1.896952in}}%
\pgfpathlineto{\pgfqpoint{4.997033in}{1.901321in}}%
\pgfpathlineto{\pgfqpoint{4.997418in}{1.900951in}}%
\pgfpathlineto{\pgfqpoint{4.997610in}{1.899125in}}%
\pgfpathlineto{\pgfqpoint{4.997994in}{1.903107in}}%
\pgfpathlineto{\pgfqpoint{4.998186in}{1.900361in}}%
\pgfpathlineto{\pgfqpoint{4.999147in}{1.908368in}}%
\pgfpathlineto{\pgfqpoint{4.999339in}{1.907941in}}%
\pgfpathlineto{\pgfqpoint{5.000492in}{1.896078in}}%
\pgfpathlineto{\pgfqpoint{5.001261in}{1.897535in}}%
\pgfpathlineto{\pgfqpoint{5.001453in}{1.899387in}}%
\pgfpathlineto{\pgfqpoint{5.001838in}{1.895168in}}%
\pgfpathlineto{\pgfqpoint{5.002222in}{1.897080in}}%
\pgfpathlineto{\pgfqpoint{5.003183in}{1.886389in}}%
\pgfpathlineto{\pgfqpoint{5.003951in}{1.890517in}}%
\pgfpathlineto{\pgfqpoint{5.004720in}{1.895106in}}%
\pgfpathlineto{\pgfqpoint{5.004912in}{1.893809in}}%
\pgfpathlineto{\pgfqpoint{5.005489in}{1.886122in}}%
\pgfpathlineto{\pgfqpoint{5.006065in}{1.887504in}}%
\pgfpathlineto{\pgfqpoint{5.007410in}{1.893354in}}%
\pgfpathlineto{\pgfqpoint{5.007603in}{1.891924in}}%
\pgfpathlineto{\pgfqpoint{5.008179in}{1.892451in}}%
\pgfpathlineto{\pgfqpoint{5.009524in}{1.899783in}}%
\pgfpathlineto{\pgfqpoint{5.009716in}{1.898907in}}%
\pgfpathlineto{\pgfqpoint{5.009909in}{1.900088in}}%
\pgfpathlineto{\pgfqpoint{5.010101in}{1.898008in}}%
\pgfpathlineto{\pgfqpoint{5.010293in}{1.898761in}}%
\pgfpathlineto{\pgfqpoint{5.010485in}{1.894004in}}%
\pgfpathlineto{\pgfqpoint{5.011254in}{1.900480in}}%
\pgfpathlineto{\pgfqpoint{5.011830in}{1.904530in}}%
\pgfpathlineto{\pgfqpoint{5.012022in}{1.902458in}}%
\pgfpathlineto{\pgfqpoint{5.012215in}{1.898259in}}%
\pgfpathlineto{\pgfqpoint{5.012791in}{1.903725in}}%
\pgfpathlineto{\pgfqpoint{5.014328in}{1.914192in}}%
\pgfpathlineto{\pgfqpoint{5.014713in}{1.912871in}}%
\pgfpathlineto{\pgfqpoint{5.014905in}{1.914875in}}%
\pgfpathlineto{\pgfqpoint{5.015097in}{1.914660in}}%
\pgfpathlineto{\pgfqpoint{5.015481in}{1.914308in}}%
\pgfpathlineto{\pgfqpoint{5.016442in}{1.916937in}}%
\pgfpathlineto{\pgfqpoint{5.016634in}{1.917303in}}%
\pgfpathlineto{\pgfqpoint{5.017019in}{1.914333in}}%
\pgfpathlineto{\pgfqpoint{5.017403in}{1.919493in}}%
\pgfpathlineto{\pgfqpoint{5.017980in}{1.926518in}}%
\pgfpathlineto{\pgfqpoint{5.018364in}{1.922384in}}%
\pgfpathlineto{\pgfqpoint{5.020094in}{1.907635in}}%
\pgfpathlineto{\pgfqpoint{5.020286in}{1.909364in}}%
\pgfpathlineto{\pgfqpoint{5.021054in}{1.907657in}}%
\pgfpathlineto{\pgfqpoint{5.021631in}{1.911369in}}%
\pgfpathlineto{\pgfqpoint{5.022015in}{1.909586in}}%
\pgfpathlineto{\pgfqpoint{5.022976in}{1.901555in}}%
\pgfpathlineto{\pgfqpoint{5.023553in}{1.902061in}}%
\pgfpathlineto{\pgfqpoint{5.024129in}{1.900850in}}%
\pgfpathlineto{\pgfqpoint{5.024321in}{1.898753in}}%
\pgfpathlineto{\pgfqpoint{5.025090in}{1.900498in}}%
\pgfpathlineto{\pgfqpoint{5.025474in}{1.902987in}}%
\pgfpathlineto{\pgfqpoint{5.026051in}{1.899384in}}%
\pgfpathlineto{\pgfqpoint{5.026627in}{1.900577in}}%
\pgfpathlineto{\pgfqpoint{5.026819in}{1.900080in}}%
\pgfpathlineto{\pgfqpoint{5.027012in}{1.898315in}}%
\pgfpathlineto{\pgfqpoint{5.027396in}{1.902795in}}%
\pgfpathlineto{\pgfqpoint{5.027588in}{1.902274in}}%
\pgfpathlineto{\pgfqpoint{5.028165in}{1.899807in}}%
\pgfpathlineto{\pgfqpoint{5.030278in}{1.885411in}}%
\pgfpathlineto{\pgfqpoint{5.031239in}{1.885018in}}%
\pgfpathlineto{\pgfqpoint{5.031624in}{1.890791in}}%
\pgfpathlineto{\pgfqpoint{5.032200in}{1.888482in}}%
\pgfpathlineto{\pgfqpoint{5.032392in}{1.893701in}}%
\pgfpathlineto{\pgfqpoint{5.032969in}{1.891373in}}%
\pgfpathlineto{\pgfqpoint{5.033545in}{1.891614in}}%
\pgfpathlineto{\pgfqpoint{5.035083in}{1.900232in}}%
\pgfpathlineto{\pgfqpoint{5.035275in}{1.899565in}}%
\pgfpathlineto{\pgfqpoint{5.036043in}{1.905648in}}%
\pgfpathlineto{\pgfqpoint{5.036428in}{1.902626in}}%
\pgfpathlineto{\pgfqpoint{5.036620in}{1.900906in}}%
\pgfpathlineto{\pgfqpoint{5.037196in}{1.905712in}}%
\pgfpathlineto{\pgfqpoint{5.037581in}{1.901706in}}%
\pgfpathlineto{\pgfqpoint{5.038926in}{1.906900in}}%
\pgfpathlineto{\pgfqpoint{5.039118in}{1.906532in}}%
\pgfpathlineto{\pgfqpoint{5.039310in}{1.905548in}}%
\pgfpathlineto{\pgfqpoint{5.039502in}{1.907724in}}%
\pgfpathlineto{\pgfqpoint{5.041040in}{1.917305in}}%
\pgfpathlineto{\pgfqpoint{5.041232in}{1.917163in}}%
\pgfpathlineto{\pgfqpoint{5.041808in}{1.922201in}}%
\pgfpathlineto{\pgfqpoint{5.042193in}{1.918156in}}%
\pgfpathlineto{\pgfqpoint{5.042577in}{1.914003in}}%
\pgfpathlineto{\pgfqpoint{5.043346in}{1.916918in}}%
\pgfpathlineto{\pgfqpoint{5.044499in}{1.924894in}}%
\pgfpathlineto{\pgfqpoint{5.045075in}{1.923859in}}%
\pgfpathlineto{\pgfqpoint{5.045844in}{1.920995in}}%
\pgfpathlineto{\pgfqpoint{5.045460in}{1.926231in}}%
\pgfpathlineto{\pgfqpoint{5.046228in}{1.922685in}}%
\pgfpathlineto{\pgfqpoint{5.047189in}{1.926296in}}%
\pgfpathlineto{\pgfqpoint{5.046613in}{1.922588in}}%
\pgfpathlineto{\pgfqpoint{5.047574in}{1.923958in}}%
\pgfpathlineto{\pgfqpoint{5.048150in}{1.924818in}}%
\pgfpathlineto{\pgfqpoint{5.048342in}{1.922831in}}%
\pgfpathlineto{\pgfqpoint{5.048534in}{1.919840in}}%
\pgfpathlineto{\pgfqpoint{5.049303in}{1.924282in}}%
\pgfpathlineto{\pgfqpoint{5.049687in}{1.921800in}}%
\pgfpathlineto{\pgfqpoint{5.049880in}{1.922879in}}%
\pgfpathlineto{\pgfqpoint{5.051033in}{1.931857in}}%
\pgfpathlineto{\pgfqpoint{5.051225in}{1.930926in}}%
\pgfpathlineto{\pgfqpoint{5.051801in}{1.936760in}}%
\pgfpathlineto{\pgfqpoint{5.052570in}{1.934760in}}%
\pgfpathlineto{\pgfqpoint{5.052762in}{1.935839in}}%
\pgfpathlineto{\pgfqpoint{5.052954in}{1.934436in}}%
\pgfpathlineto{\pgfqpoint{5.053531in}{1.931006in}}%
\pgfpathlineto{\pgfqpoint{5.053723in}{1.934073in}}%
\pgfpathlineto{\pgfqpoint{5.054107in}{1.938317in}}%
\pgfpathlineto{\pgfqpoint{5.054876in}{1.935059in}}%
\pgfpathlineto{\pgfqpoint{5.055068in}{1.935039in}}%
\pgfpathlineto{\pgfqpoint{5.055260in}{1.933505in}}%
\pgfpathlineto{\pgfqpoint{5.055645in}{1.939113in}}%
\pgfpathlineto{\pgfqpoint{5.056029in}{1.936217in}}%
\pgfpathlineto{\pgfqpoint{5.056221in}{1.936295in}}%
\pgfpathlineto{\pgfqpoint{5.057374in}{1.922748in}}%
\pgfpathlineto{\pgfqpoint{5.058143in}{1.926991in}}%
\pgfpathlineto{\pgfqpoint{5.058527in}{1.923243in}}%
\pgfpathlineto{\pgfqpoint{5.059296in}{1.926102in}}%
\pgfpathlineto{\pgfqpoint{5.059680in}{1.925724in}}%
\pgfpathlineto{\pgfqpoint{5.060833in}{1.931174in}}%
\pgfpathlineto{\pgfqpoint{5.061025in}{1.931103in}}%
\pgfpathlineto{\pgfqpoint{5.062563in}{1.936437in}}%
\pgfpathlineto{\pgfqpoint{5.063716in}{1.928529in}}%
\pgfpathlineto{\pgfqpoint{5.063908in}{1.929472in}}%
\pgfpathlineto{\pgfqpoint{5.064292in}{1.926641in}}%
\pgfpathlineto{\pgfqpoint{5.064484in}{1.923444in}}%
\pgfpathlineto{\pgfqpoint{5.065061in}{1.929193in}}%
\pgfpathlineto{\pgfqpoint{5.065253in}{1.927718in}}%
\pgfpathlineto{\pgfqpoint{5.065445in}{1.926848in}}%
\pgfpathlineto{\pgfqpoint{5.065637in}{1.928924in}}%
\pgfpathlineto{\pgfqpoint{5.066406in}{1.934828in}}%
\pgfpathlineto{\pgfqpoint{5.066983in}{1.934638in}}%
\pgfpathlineto{\pgfqpoint{5.067175in}{1.933489in}}%
\pgfpathlineto{\pgfqpoint{5.067559in}{1.937739in}}%
\pgfpathlineto{\pgfqpoint{5.068520in}{1.933444in}}%
\pgfpathlineto{\pgfqpoint{5.068712in}{1.933662in}}%
\pgfpathlineto{\pgfqpoint{5.068904in}{1.928732in}}%
\pgfpathlineto{\pgfqpoint{5.069865in}{1.931228in}}%
\pgfpathlineto{\pgfqpoint{5.070249in}{1.931596in}}%
\pgfpathlineto{\pgfqpoint{5.070442in}{1.935488in}}%
\pgfpathlineto{\pgfqpoint{5.071210in}{1.929944in}}%
\pgfpathlineto{\pgfqpoint{5.071402in}{1.931095in}}%
\pgfpathlineto{\pgfqpoint{5.071979in}{1.928269in}}%
\pgfpathlineto{\pgfqpoint{5.072555in}{1.925581in}}%
\pgfpathlineto{\pgfqpoint{5.072363in}{1.929051in}}%
\pgfpathlineto{\pgfqpoint{5.073132in}{1.927751in}}%
\pgfpathlineto{\pgfqpoint{5.073324in}{1.927474in}}%
\pgfpathlineto{\pgfqpoint{5.073708in}{1.922453in}}%
\pgfpathlineto{\pgfqpoint{5.074477in}{1.923420in}}%
\pgfpathlineto{\pgfqpoint{5.074861in}{1.926257in}}%
\pgfpathlineto{\pgfqpoint{5.075438in}{1.924884in}}%
\pgfpathlineto{\pgfqpoint{5.076014in}{1.923142in}}%
\pgfpathlineto{\pgfqpoint{5.076207in}{1.927479in}}%
\pgfpathlineto{\pgfqpoint{5.076399in}{1.924760in}}%
\pgfpathlineto{\pgfqpoint{5.076591in}{1.926822in}}%
\pgfpathlineto{\pgfqpoint{5.076975in}{1.921757in}}%
\pgfpathlineto{\pgfqpoint{5.077167in}{1.922225in}}%
\pgfpathlineto{\pgfqpoint{5.078705in}{1.913620in}}%
\pgfpathlineto{\pgfqpoint{5.078897in}{1.916324in}}%
\pgfpathlineto{\pgfqpoint{5.079281in}{1.912852in}}%
\pgfpathlineto{\pgfqpoint{5.079473in}{1.914604in}}%
\pgfpathlineto{\pgfqpoint{5.081203in}{1.900331in}}%
\pgfpathlineto{\pgfqpoint{5.081395in}{1.902009in}}%
\pgfpathlineto{\pgfqpoint{5.081972in}{1.899379in}}%
\pgfpathlineto{\pgfqpoint{5.083701in}{1.888874in}}%
\pgfpathlineto{\pgfqpoint{5.083893in}{1.890288in}}%
\pgfpathlineto{\pgfqpoint{5.084278in}{1.893244in}}%
\pgfpathlineto{\pgfqpoint{5.085046in}{1.891014in}}%
\pgfpathlineto{\pgfqpoint{5.085431in}{1.893543in}}%
\pgfpathlineto{\pgfqpoint{5.085815in}{1.891231in}}%
\pgfpathlineto{\pgfqpoint{5.087352in}{1.885009in}}%
\pgfpathlineto{\pgfqpoint{5.087737in}{1.885980in}}%
\pgfpathlineto{\pgfqpoint{5.089658in}{1.891625in}}%
\pgfpathlineto{\pgfqpoint{5.090235in}{1.895832in}}%
\pgfpathlineto{\pgfqpoint{5.090619in}{1.889821in}}%
\pgfpathlineto{\pgfqpoint{5.091388in}{1.886763in}}%
\pgfpathlineto{\pgfqpoint{5.091580in}{1.887580in}}%
\pgfpathlineto{\pgfqpoint{5.091772in}{1.891651in}}%
\pgfpathlineto{\pgfqpoint{5.092733in}{1.889847in}}%
\pgfpathlineto{\pgfqpoint{5.093502in}{1.887290in}}%
\pgfpathlineto{\pgfqpoint{5.093886in}{1.888762in}}%
\pgfpathlineto{\pgfqpoint{5.094270in}{1.887399in}}%
\pgfpathlineto{\pgfqpoint{5.097729in}{1.904988in}}%
\pgfpathlineto{\pgfqpoint{5.099267in}{1.901276in}}%
\pgfpathlineto{\pgfqpoint{5.099459in}{1.899553in}}%
\pgfpathlineto{\pgfqpoint{5.100228in}{1.901027in}}%
\pgfpathlineto{\pgfqpoint{5.100612in}{1.902665in}}%
\pgfpathlineto{\pgfqpoint{5.101381in}{1.902294in}}%
\pgfpathlineto{\pgfqpoint{5.102726in}{1.899532in}}%
\pgfpathlineto{\pgfqpoint{5.101765in}{1.904169in}}%
\pgfpathlineto{\pgfqpoint{5.102918in}{1.899697in}}%
\pgfpathlineto{\pgfqpoint{5.103302in}{1.902430in}}%
\pgfpathlineto{\pgfqpoint{5.103687in}{1.898825in}}%
\pgfpathlineto{\pgfqpoint{5.103879in}{1.898846in}}%
\pgfpathlineto{\pgfqpoint{5.104647in}{1.902750in}}%
\pgfpathlineto{\pgfqpoint{5.105224in}{1.900894in}}%
\pgfpathlineto{\pgfqpoint{5.105608in}{1.898816in}}%
\pgfpathlineto{\pgfqpoint{5.105800in}{1.901445in}}%
\pgfpathlineto{\pgfqpoint{5.106185in}{1.900106in}}%
\pgfpathlineto{\pgfqpoint{5.106761in}{1.904225in}}%
\pgfpathlineto{\pgfqpoint{5.106953in}{1.900592in}}%
\pgfpathlineto{\pgfqpoint{5.107722in}{1.890571in}}%
\pgfpathlineto{\pgfqpoint{5.108106in}{1.896367in}}%
\pgfpathlineto{\pgfqpoint{5.111373in}{1.916868in}}%
\pgfpathlineto{\pgfqpoint{5.111758in}{1.913097in}}%
\pgfpathlineto{\pgfqpoint{5.112142in}{1.917963in}}%
\pgfpathlineto{\pgfqpoint{5.112334in}{1.917919in}}%
\pgfpathlineto{\pgfqpoint{5.113103in}{1.916022in}}%
\pgfpathlineto{\pgfqpoint{5.113679in}{1.922691in}}%
\pgfpathlineto{\pgfqpoint{5.114256in}{1.917811in}}%
\pgfpathlineto{\pgfqpoint{5.114448in}{1.917042in}}%
\pgfpathlineto{\pgfqpoint{5.114640in}{1.919600in}}%
\pgfpathlineto{\pgfqpoint{5.115601in}{1.922167in}}%
\pgfpathlineto{\pgfqpoint{5.115217in}{1.917952in}}%
\pgfpathlineto{\pgfqpoint{5.115793in}{1.922088in}}%
\pgfpathlineto{\pgfqpoint{5.117138in}{1.915575in}}%
\pgfpathlineto{\pgfqpoint{5.117523in}{1.920012in}}%
\pgfpathlineto{\pgfqpoint{5.118291in}{1.917998in}}%
\pgfpathlineto{\pgfqpoint{5.119060in}{1.922076in}}%
\pgfpathlineto{\pgfqpoint{5.120021in}{1.919517in}}%
\pgfpathlineto{\pgfqpoint{5.120982in}{1.917593in}}%
\pgfpathlineto{\pgfqpoint{5.122711in}{1.925195in}}%
\pgfpathlineto{\pgfqpoint{5.123864in}{1.921417in}}%
\pgfpathlineto{\pgfqpoint{5.124249in}{1.920691in}}%
\pgfpathlineto{\pgfqpoint{5.124825in}{1.928551in}}%
\pgfpathlineto{\pgfqpoint{5.125209in}{1.930436in}}%
\pgfpathlineto{\pgfqpoint{5.125786in}{1.928493in}}%
\pgfpathlineto{\pgfqpoint{5.126170in}{1.928041in}}%
\pgfpathlineto{\pgfqpoint{5.126362in}{1.931242in}}%
\pgfpathlineto{\pgfqpoint{5.126555in}{1.933429in}}%
\pgfpathlineto{\pgfqpoint{5.126939in}{1.929788in}}%
\pgfpathlineto{\pgfqpoint{5.127131in}{1.930743in}}%
\pgfpathlineto{\pgfqpoint{5.127708in}{1.929339in}}%
\pgfpathlineto{\pgfqpoint{5.128092in}{1.930956in}}%
\pgfpathlineto{\pgfqpoint{5.128284in}{1.930867in}}%
\pgfpathlineto{\pgfqpoint{5.129053in}{1.925850in}}%
\pgfpathlineto{\pgfqpoint{5.129821in}{1.926064in}}%
\pgfpathlineto{\pgfqpoint{5.131935in}{1.912950in}}%
\pgfpathlineto{\pgfqpoint{5.132320in}{1.915811in}}%
\pgfpathlineto{\pgfqpoint{5.132512in}{1.915697in}}%
\pgfpathlineto{\pgfqpoint{5.133280in}{1.909884in}}%
\pgfpathlineto{\pgfqpoint{5.133665in}{1.912097in}}%
\pgfpathlineto{\pgfqpoint{5.133857in}{1.912113in}}%
\pgfpathlineto{\pgfqpoint{5.134241in}{1.909275in}}%
\pgfpathlineto{\pgfqpoint{5.134626in}{1.912344in}}%
\pgfpathlineto{\pgfqpoint{5.135010in}{1.910725in}}%
\pgfpathlineto{\pgfqpoint{5.135586in}{1.914287in}}%
\pgfpathlineto{\pgfqpoint{5.135394in}{1.910436in}}%
\pgfpathlineto{\pgfqpoint{5.136163in}{1.914129in}}%
\pgfpathlineto{\pgfqpoint{5.136355in}{1.910206in}}%
\pgfpathlineto{\pgfqpoint{5.137124in}{1.916410in}}%
\pgfpathlineto{\pgfqpoint{5.137316in}{1.918572in}}%
\pgfpathlineto{\pgfqpoint{5.137892in}{1.914434in}}%
\pgfpathlineto{\pgfqpoint{5.138085in}{1.914244in}}%
\pgfpathlineto{\pgfqpoint{5.138469in}{1.916871in}}%
\pgfpathlineto{\pgfqpoint{5.138853in}{1.914292in}}%
\pgfpathlineto{\pgfqpoint{5.139238in}{1.910273in}}%
\pgfpathlineto{\pgfqpoint{5.139814in}{1.913129in}}%
\pgfpathlineto{\pgfqpoint{5.140391in}{1.915103in}}%
\pgfpathlineto{\pgfqpoint{5.140583in}{1.913183in}}%
\pgfpathlineto{\pgfqpoint{5.140775in}{1.910445in}}%
\pgfpathlineto{\pgfqpoint{5.141736in}{1.912586in}}%
\pgfpathlineto{\pgfqpoint{5.142505in}{1.908882in}}%
\pgfpathlineto{\pgfqpoint{5.143081in}{1.909182in}}%
\pgfpathlineto{\pgfqpoint{5.143273in}{1.909042in}}%
\pgfpathlineto{\pgfqpoint{5.144618in}{1.913933in}}%
\pgfpathlineto{\pgfqpoint{5.145195in}{1.911453in}}%
\pgfpathlineto{\pgfqpoint{5.145387in}{1.912524in}}%
\pgfpathlineto{\pgfqpoint{5.146348in}{1.920372in}}%
\pgfpathlineto{\pgfqpoint{5.146732in}{1.918777in}}%
\pgfpathlineto{\pgfqpoint{5.147117in}{1.914349in}}%
\pgfpathlineto{\pgfqpoint{5.147501in}{1.917359in}}%
\pgfpathlineto{\pgfqpoint{5.148654in}{1.927005in}}%
\pgfpathlineto{\pgfqpoint{5.148846in}{1.925361in}}%
\pgfpathlineto{\pgfqpoint{5.149807in}{1.929241in}}%
\pgfpathlineto{\pgfqpoint{5.150191in}{1.928007in}}%
\pgfpathlineto{\pgfqpoint{5.150383in}{1.926902in}}%
\pgfpathlineto{\pgfqpoint{5.150768in}{1.930251in}}%
\pgfpathlineto{\pgfqpoint{5.151344in}{1.931973in}}%
\pgfpathlineto{\pgfqpoint{5.151729in}{1.931605in}}%
\pgfpathlineto{\pgfqpoint{5.151921in}{1.928806in}}%
\pgfpathlineto{\pgfqpoint{5.152497in}{1.934561in}}%
\pgfpathlineto{\pgfqpoint{5.152689in}{1.932425in}}%
\pgfpathlineto{\pgfqpoint{5.152882in}{1.934657in}}%
\pgfpathlineto{\pgfqpoint{5.153458in}{1.928463in}}%
\pgfpathlineto{\pgfqpoint{5.153842in}{1.932080in}}%
\pgfpathlineto{\pgfqpoint{5.154611in}{1.930885in}}%
\pgfpathlineto{\pgfqpoint{5.155380in}{1.924510in}}%
\pgfpathlineto{\pgfqpoint{5.156341in}{1.926898in}}%
\pgfpathlineto{\pgfqpoint{5.157109in}{1.937209in}}%
\pgfpathlineto{\pgfqpoint{5.157686in}{1.934483in}}%
\pgfpathlineto{\pgfqpoint{5.157878in}{1.935342in}}%
\pgfpathlineto{\pgfqpoint{5.158262in}{1.933303in}}%
\pgfpathlineto{\pgfqpoint{5.158454in}{1.932645in}}%
\pgfpathlineto{\pgfqpoint{5.159223in}{1.921437in}}%
\pgfpathlineto{\pgfqpoint{5.159800in}{1.922076in}}%
\pgfpathlineto{\pgfqpoint{5.159992in}{1.922087in}}%
\pgfpathlineto{\pgfqpoint{5.161721in}{1.908194in}}%
\pgfpathlineto{\pgfqpoint{5.162874in}{1.919225in}}%
\pgfpathlineto{\pgfqpoint{5.163451in}{1.914410in}}%
\pgfpathlineto{\pgfqpoint{5.165180in}{1.909653in}}%
\pgfpathlineto{\pgfqpoint{5.165373in}{1.910877in}}%
\pgfpathlineto{\pgfqpoint{5.165949in}{1.907791in}}%
\pgfpathlineto{\pgfqpoint{5.166910in}{1.903806in}}%
\pgfpathlineto{\pgfqpoint{5.167102in}{1.904247in}}%
\pgfpathlineto{\pgfqpoint{5.168639in}{1.913237in}}%
\pgfpathlineto{\pgfqpoint{5.170561in}{1.906989in}}%
\pgfpathlineto{\pgfqpoint{5.171138in}{1.908708in}}%
\pgfpathlineto{\pgfqpoint{5.171330in}{1.910527in}}%
\pgfpathlineto{\pgfqpoint{5.171522in}{1.907177in}}%
\pgfpathlineto{\pgfqpoint{5.171906in}{1.909060in}}%
\pgfpathlineto{\pgfqpoint{5.172098in}{1.906540in}}%
\pgfpathlineto{\pgfqpoint{5.172867in}{1.910477in}}%
\pgfpathlineto{\pgfqpoint{5.174020in}{1.914764in}}%
\pgfpathlineto{\pgfqpoint{5.175942in}{1.907751in}}%
\pgfpathlineto{\pgfqpoint{5.176903in}{1.911218in}}%
\pgfpathlineto{\pgfqpoint{5.177095in}{1.909041in}}%
\pgfpathlineto{\pgfqpoint{5.177287in}{1.908736in}}%
\pgfpathlineto{\pgfqpoint{5.177479in}{1.910275in}}%
\pgfpathlineto{\pgfqpoint{5.177671in}{1.910751in}}%
\pgfpathlineto{\pgfqpoint{5.178056in}{1.915388in}}%
\pgfpathlineto{\pgfqpoint{5.178824in}{1.911862in}}%
\pgfpathlineto{\pgfqpoint{5.179016in}{1.910115in}}%
\pgfpathlineto{\pgfqpoint{5.179593in}{1.913512in}}%
\pgfpathlineto{\pgfqpoint{5.179977in}{1.913753in}}%
\pgfpathlineto{\pgfqpoint{5.181899in}{1.904569in}}%
\pgfpathlineto{\pgfqpoint{5.182091in}{1.904746in}}%
\pgfpathlineto{\pgfqpoint{5.182283in}{1.904452in}}%
\pgfpathlineto{\pgfqpoint{5.182475in}{1.905567in}}%
\pgfpathlineto{\pgfqpoint{5.183052in}{1.909969in}}%
\pgfpathlineto{\pgfqpoint{5.183628in}{1.906776in}}%
\pgfpathlineto{\pgfqpoint{5.184013in}{1.904961in}}%
\pgfpathlineto{\pgfqpoint{5.184397in}{1.901837in}}%
\pgfpathlineto{\pgfqpoint{5.185166in}{1.903744in}}%
\pgfpathlineto{\pgfqpoint{5.185742in}{1.908109in}}%
\pgfpathlineto{\pgfqpoint{5.185934in}{1.904740in}}%
\pgfpathlineto{\pgfqpoint{5.186127in}{1.902727in}}%
\pgfpathlineto{\pgfqpoint{5.186511in}{1.909103in}}%
\pgfpathlineto{\pgfqpoint{5.186703in}{1.909285in}}%
\pgfpathlineto{\pgfqpoint{5.188048in}{1.895958in}}%
\pgfpathlineto{\pgfqpoint{5.188240in}{1.897074in}}%
\pgfpathlineto{\pgfqpoint{5.189009in}{1.906475in}}%
\pgfpathlineto{\pgfqpoint{5.189778in}{1.904255in}}%
\pgfpathlineto{\pgfqpoint{5.190162in}{1.909252in}}%
\pgfpathlineto{\pgfqpoint{5.190931in}{1.905615in}}%
\pgfpathlineto{\pgfqpoint{5.191507in}{1.900261in}}%
\pgfpathlineto{\pgfqpoint{5.192084in}{1.903893in}}%
\pgfpathlineto{\pgfqpoint{5.193429in}{1.898237in}}%
\pgfpathlineto{\pgfqpoint{5.193621in}{1.900239in}}%
\pgfpathlineto{\pgfqpoint{5.194390in}{1.905570in}}%
\pgfpathlineto{\pgfqpoint{5.194774in}{1.903823in}}%
\pgfpathlineto{\pgfqpoint{5.195351in}{1.908499in}}%
\pgfpathlineto{\pgfqpoint{5.195735in}{1.906956in}}%
\pgfpathlineto{\pgfqpoint{5.196312in}{1.901064in}}%
\pgfpathlineto{\pgfqpoint{5.196888in}{1.905440in}}%
\pgfpathlineto{\pgfqpoint{5.200539in}{1.887827in}}%
\pgfpathlineto{\pgfqpoint{5.200731in}{1.889058in}}%
\pgfpathlineto{\pgfqpoint{5.201692in}{1.887435in}}%
\pgfpathlineto{\pgfqpoint{5.202077in}{1.889017in}}%
\pgfpathlineto{\pgfqpoint{5.202845in}{1.891208in}}%
\pgfpathlineto{\pgfqpoint{5.203230in}{1.890800in}}%
\pgfpathlineto{\pgfqpoint{5.204190in}{1.882739in}}%
\pgfpathlineto{\pgfqpoint{5.204767in}{1.884443in}}%
\pgfpathlineto{\pgfqpoint{5.204959in}{1.885714in}}%
\pgfpathlineto{\pgfqpoint{5.205728in}{1.883402in}}%
\pgfpathlineto{\pgfqpoint{5.205920in}{1.884991in}}%
\pgfpathlineto{\pgfqpoint{5.206304in}{1.884254in}}%
\pgfpathlineto{\pgfqpoint{5.207073in}{1.886795in}}%
\pgfpathlineto{\pgfqpoint{5.207265in}{1.883889in}}%
\pgfpathlineto{\pgfqpoint{5.208034in}{1.885645in}}%
\pgfpathlineto{\pgfqpoint{5.210148in}{1.896989in}}%
\pgfpathlineto{\pgfqpoint{5.210724in}{1.894998in}}%
\pgfpathlineto{\pgfqpoint{5.211108in}{1.893713in}}%
\pgfpathlineto{\pgfqpoint{5.212454in}{1.900791in}}%
\pgfpathlineto{\pgfqpoint{5.213222in}{1.902665in}}%
\pgfpathlineto{\pgfqpoint{5.213607in}{1.901711in}}%
\pgfpathlineto{\pgfqpoint{5.215721in}{1.912758in}}%
\pgfpathlineto{\pgfqpoint{5.216681in}{1.898350in}}%
\pgfpathlineto{\pgfqpoint{5.217066in}{1.901085in}}%
\pgfpathlineto{\pgfqpoint{5.218603in}{1.908417in}}%
\pgfpathlineto{\pgfqpoint{5.218795in}{1.908530in}}%
\pgfpathlineto{\pgfqpoint{5.219756in}{1.904940in}}%
\pgfpathlineto{\pgfqpoint{5.220140in}{1.905163in}}%
\pgfpathlineto{\pgfqpoint{5.220333in}{1.906159in}}%
\pgfpathlineto{\pgfqpoint{5.220717in}{1.904665in}}%
\pgfpathlineto{\pgfqpoint{5.223215in}{1.891888in}}%
\pgfpathlineto{\pgfqpoint{5.223407in}{1.891874in}}%
\pgfpathlineto{\pgfqpoint{5.223599in}{1.893071in}}%
\pgfpathlineto{\pgfqpoint{5.224176in}{1.890735in}}%
\pgfpathlineto{\pgfqpoint{5.224560in}{1.887445in}}%
\pgfpathlineto{\pgfqpoint{5.224945in}{1.888851in}}%
\pgfpathlineto{\pgfqpoint{5.225137in}{1.892524in}}%
\pgfpathlineto{\pgfqpoint{5.225905in}{1.889792in}}%
\pgfpathlineto{\pgfqpoint{5.226482in}{1.887946in}}%
\pgfpathlineto{\pgfqpoint{5.227058in}{1.878473in}}%
\pgfpathlineto{\pgfqpoint{5.227827in}{1.880058in}}%
\pgfpathlineto{\pgfqpoint{5.228019in}{1.881544in}}%
\pgfpathlineto{\pgfqpoint{5.228596in}{1.879571in}}%
\pgfpathlineto{\pgfqpoint{5.229172in}{1.874132in}}%
\pgfpathlineto{\pgfqpoint{5.229941in}{1.876577in}}%
\pgfpathlineto{\pgfqpoint{5.230517in}{1.879241in}}%
\pgfpathlineto{\pgfqpoint{5.230902in}{1.876109in}}%
\pgfpathlineto{\pgfqpoint{5.231094in}{1.877141in}}%
\pgfpathlineto{\pgfqpoint{5.232055in}{1.872929in}}%
\pgfpathlineto{\pgfqpoint{5.232439in}{1.874282in}}%
\pgfpathlineto{\pgfqpoint{5.234553in}{1.886115in}}%
\pgfpathlineto{\pgfqpoint{5.233016in}{1.873912in}}%
\pgfpathlineto{\pgfqpoint{5.234937in}{1.885504in}}%
\pgfpathlineto{\pgfqpoint{5.236475in}{1.877882in}}%
\pgfpathlineto{\pgfqpoint{5.236859in}{1.883197in}}%
\pgfpathlineto{\pgfqpoint{5.237628in}{1.879570in}}%
\pgfpathlineto{\pgfqpoint{5.239165in}{1.873278in}}%
\pgfpathlineto{\pgfqpoint{5.238204in}{1.881509in}}%
\pgfpathlineto{\pgfqpoint{5.239357in}{1.875490in}}%
\pgfpathlineto{\pgfqpoint{5.242432in}{1.885963in}}%
\pgfpathlineto{\pgfqpoint{5.243393in}{1.887059in}}%
\pgfpathlineto{\pgfqpoint{5.243585in}{1.884609in}}%
\pgfpathlineto{\pgfqpoint{5.243777in}{1.886287in}}%
\pgfpathlineto{\pgfqpoint{5.244161in}{1.883223in}}%
\pgfpathlineto{\pgfqpoint{5.244546in}{1.883414in}}%
\pgfpathlineto{\pgfqpoint{5.245507in}{1.888932in}}%
\pgfpathlineto{\pgfqpoint{5.245891in}{1.891133in}}%
\pgfpathlineto{\pgfqpoint{5.246275in}{1.888210in}}%
\pgfpathlineto{\pgfqpoint{5.248581in}{1.870994in}}%
\pgfpathlineto{\pgfqpoint{5.248966in}{1.872662in}}%
\pgfpathlineto{\pgfqpoint{5.250311in}{1.877899in}}%
\pgfpathlineto{\pgfqpoint{5.250503in}{1.877648in}}%
\pgfpathlineto{\pgfqpoint{5.250887in}{1.875677in}}%
\pgfpathlineto{\pgfqpoint{5.251079in}{1.871379in}}%
\pgfpathlineto{\pgfqpoint{5.251848in}{1.873492in}}%
\pgfpathlineto{\pgfqpoint{5.252617in}{1.877386in}}%
\pgfpathlineto{\pgfqpoint{5.253385in}{1.876843in}}%
\pgfpathlineto{\pgfqpoint{5.254923in}{1.880771in}}%
\pgfpathlineto{\pgfqpoint{5.254154in}{1.876340in}}%
\pgfpathlineto{\pgfqpoint{5.255307in}{1.879954in}}%
\pgfpathlineto{\pgfqpoint{5.255884in}{1.881187in}}%
\pgfpathlineto{\pgfqpoint{5.256076in}{1.878793in}}%
\pgfpathlineto{\pgfqpoint{5.256268in}{1.876023in}}%
\pgfpathlineto{\pgfqpoint{5.256844in}{1.879538in}}%
\pgfpathlineto{\pgfqpoint{5.257613in}{1.883857in}}%
\pgfpathlineto{\pgfqpoint{5.257997in}{1.881123in}}%
\pgfpathlineto{\pgfqpoint{5.258190in}{1.881342in}}%
\pgfpathlineto{\pgfqpoint{5.258766in}{1.872356in}}%
\pgfpathlineto{\pgfqpoint{5.259535in}{1.877118in}}%
\pgfpathlineto{\pgfqpoint{5.260688in}{1.870239in}}%
\pgfpathlineto{\pgfqpoint{5.261649in}{1.874000in}}%
\pgfpathlineto{\pgfqpoint{5.262033in}{1.879108in}}%
\pgfpathlineto{\pgfqpoint{5.262994in}{1.876237in}}%
\pgfpathlineto{\pgfqpoint{5.264723in}{1.869629in}}%
\pgfpathlineto{\pgfqpoint{5.265300in}{1.871029in}}%
\pgfpathlineto{\pgfqpoint{5.265684in}{1.869343in}}%
\pgfpathlineto{\pgfqpoint{5.266261in}{1.863580in}}%
\pgfpathlineto{\pgfqpoint{5.266645in}{1.870050in}}%
\pgfpathlineto{\pgfqpoint{5.267222in}{1.870897in}}%
\pgfpathlineto{\pgfqpoint{5.267606in}{1.872783in}}%
\pgfpathlineto{\pgfqpoint{5.267990in}{1.870263in}}%
\pgfpathlineto{\pgfqpoint{5.268759in}{1.864349in}}%
\pgfpathlineto{\pgfqpoint{5.269143in}{1.868160in}}%
\pgfpathlineto{\pgfqpoint{5.269335in}{1.867359in}}%
\pgfpathlineto{\pgfqpoint{5.269528in}{1.869721in}}%
\pgfpathlineto{\pgfqpoint{5.269720in}{1.870641in}}%
\pgfpathlineto{\pgfqpoint{5.269912in}{1.868154in}}%
\pgfpathlineto{\pgfqpoint{5.270488in}{1.864421in}}%
\pgfpathlineto{\pgfqpoint{5.271065in}{1.867419in}}%
\pgfpathlineto{\pgfqpoint{5.271449in}{1.865932in}}%
\pgfpathlineto{\pgfqpoint{5.271641in}{1.866349in}}%
\pgfpathlineto{\pgfqpoint{5.272218in}{1.874860in}}%
\pgfpathlineto{\pgfqpoint{5.272794in}{1.872933in}}%
\pgfpathlineto{\pgfqpoint{5.273947in}{1.866442in}}%
\pgfpathlineto{\pgfqpoint{5.275293in}{1.873588in}}%
\pgfpathlineto{\pgfqpoint{5.275485in}{1.873513in}}%
\pgfpathlineto{\pgfqpoint{5.275677in}{1.874644in}}%
\pgfpathlineto{\pgfqpoint{5.276253in}{1.871682in}}%
\pgfpathlineto{\pgfqpoint{5.276830in}{1.868316in}}%
\pgfpathlineto{\pgfqpoint{5.277022in}{1.869018in}}%
\pgfpathlineto{\pgfqpoint{5.277983in}{1.880885in}}%
\pgfpathlineto{\pgfqpoint{5.278367in}{1.877360in}}%
\pgfpathlineto{\pgfqpoint{5.279328in}{1.871471in}}%
\pgfpathlineto{\pgfqpoint{5.280289in}{1.872782in}}%
\pgfpathlineto{\pgfqpoint{5.281250in}{1.881278in}}%
\pgfpathlineto{\pgfqpoint{5.282211in}{1.877019in}}%
\pgfpathlineto{\pgfqpoint{5.282403in}{1.876044in}}%
\pgfpathlineto{\pgfqpoint{5.282595in}{1.877637in}}%
\pgfpathlineto{\pgfqpoint{5.283940in}{1.885136in}}%
\pgfpathlineto{\pgfqpoint{5.284132in}{1.884396in}}%
\pgfpathlineto{\pgfqpoint{5.285477in}{1.887852in}}%
\pgfpathlineto{\pgfqpoint{5.284901in}{1.883896in}}%
\pgfpathlineto{\pgfqpoint{5.285862in}{1.885693in}}%
\pgfpathlineto{\pgfqpoint{5.286054in}{1.883614in}}%
\pgfpathlineto{\pgfqpoint{5.286631in}{1.887032in}}%
\pgfpathlineto{\pgfqpoint{5.287015in}{1.885151in}}%
\pgfpathlineto{\pgfqpoint{5.287591in}{1.886782in}}%
\pgfpathlineto{\pgfqpoint{5.287784in}{1.885886in}}%
\pgfpathlineto{\pgfqpoint{5.289321in}{1.872926in}}%
\pgfpathlineto{\pgfqpoint{5.289513in}{1.874896in}}%
\pgfpathlineto{\pgfqpoint{5.291243in}{1.879257in}}%
\pgfpathlineto{\pgfqpoint{5.290090in}{1.873992in}}%
\pgfpathlineto{\pgfqpoint{5.291435in}{1.878944in}}%
\pgfpathlineto{\pgfqpoint{5.291627in}{1.878479in}}%
\pgfpathlineto{\pgfqpoint{5.292011in}{1.879483in}}%
\pgfpathlineto{\pgfqpoint{5.292203in}{1.879168in}}%
\pgfpathlineto{\pgfqpoint{5.292972in}{1.878324in}}%
\pgfpathlineto{\pgfqpoint{5.293741in}{1.886325in}}%
\pgfpathlineto{\pgfqpoint{5.293933in}{1.883060in}}%
\pgfpathlineto{\pgfqpoint{5.294894in}{1.884233in}}%
\pgfpathlineto{\pgfqpoint{5.295278in}{1.882171in}}%
\pgfpathlineto{\pgfqpoint{5.295470in}{1.882738in}}%
\pgfpathlineto{\pgfqpoint{5.297008in}{1.889188in}}%
\pgfpathlineto{\pgfqpoint{5.297392in}{1.884491in}}%
\pgfpathlineto{\pgfqpoint{5.298161in}{1.884625in}}%
\pgfpathlineto{\pgfqpoint{5.298353in}{1.884709in}}%
\pgfpathlineto{\pgfqpoint{5.298737in}{1.887843in}}%
\pgfpathlineto{\pgfqpoint{5.300082in}{1.879290in}}%
\pgfpathlineto{\pgfqpoint{5.300274in}{1.879243in}}%
\pgfpathlineto{\pgfqpoint{5.301427in}{1.871624in}}%
\pgfpathlineto{\pgfqpoint{5.301620in}{1.872627in}}%
\pgfpathlineto{\pgfqpoint{5.302965in}{1.868536in}}%
\pgfpathlineto{\pgfqpoint{5.303157in}{1.868965in}}%
\pgfpathlineto{\pgfqpoint{5.303926in}{1.873737in}}%
\pgfpathlineto{\pgfqpoint{5.303926in}{1.873737in}}%
\pgfusepath{stroke}%
\end{pgfscope}%
\begin{pgfscope}%
\pgfpathrectangle{\pgfqpoint{3.286364in}{0.660000in}}{\pgfqpoint{2.113636in}{2.100000in}}%
\pgfusepath{clip}%
\pgfsetroundcap%
\pgfsetroundjoin%
\pgfsetlinewidth{0.602250pt}%
\definecolor{currentstroke}{rgb}{0.600000,0.600000,0.600000}%
\pgfsetstrokecolor{currentstroke}%
\pgfsetdash{}{0pt}%
\pgfpathmoveto{\pgfqpoint{3.382438in}{1.826618in}}%
\pgfpathlineto{\pgfqpoint{3.382630in}{1.826235in}}%
\pgfpathlineto{\pgfqpoint{3.382822in}{1.828322in}}%
\pgfpathlineto{\pgfqpoint{3.383207in}{1.827504in}}%
\pgfpathlineto{\pgfqpoint{3.384936in}{1.814504in}}%
\pgfpathlineto{\pgfqpoint{3.385128in}{1.815004in}}%
\pgfpathlineto{\pgfqpoint{3.385513in}{1.819968in}}%
\pgfpathlineto{\pgfqpoint{3.386089in}{1.814262in}}%
\pgfpathlineto{\pgfqpoint{3.386474in}{1.817084in}}%
\pgfpathlineto{\pgfqpoint{3.387434in}{1.812495in}}%
\pgfpathlineto{\pgfqpoint{3.388203in}{1.811502in}}%
\pgfpathlineto{\pgfqpoint{3.388780in}{1.815101in}}%
\pgfpathlineto{\pgfqpoint{3.389740in}{1.806352in}}%
\pgfpathlineto{\pgfqpoint{3.390125in}{1.809803in}}%
\pgfpathlineto{\pgfqpoint{3.391278in}{1.814552in}}%
\pgfpathlineto{\pgfqpoint{3.391470in}{1.814040in}}%
\pgfpathlineto{\pgfqpoint{3.393199in}{1.807283in}}%
\pgfpathlineto{\pgfqpoint{3.392239in}{1.814971in}}%
\pgfpathlineto{\pgfqpoint{3.393584in}{1.809236in}}%
\pgfpathlineto{\pgfqpoint{3.394160in}{1.813251in}}%
\pgfpathlineto{\pgfqpoint{3.394545in}{1.809806in}}%
\pgfpathlineto{\pgfqpoint{3.396082in}{1.799812in}}%
\pgfpathlineto{\pgfqpoint{3.396851in}{1.808247in}}%
\pgfpathlineto{\pgfqpoint{3.397427in}{1.805973in}}%
\pgfpathlineto{\pgfqpoint{3.397619in}{1.805204in}}%
\pgfpathlineto{\pgfqpoint{3.398196in}{1.806298in}}%
\pgfpathlineto{\pgfqpoint{3.401463in}{1.815896in}}%
\pgfpathlineto{\pgfqpoint{3.398772in}{1.805897in}}%
\pgfpathlineto{\pgfqpoint{3.401655in}{1.814094in}}%
\pgfpathlineto{\pgfqpoint{3.404345in}{1.789931in}}%
\pgfpathlineto{\pgfqpoint{3.404922in}{1.791910in}}%
\pgfpathlineto{\pgfqpoint{3.405498in}{1.794421in}}%
\pgfpathlineto{\pgfqpoint{3.406267in}{1.793423in}}%
\pgfpathlineto{\pgfqpoint{3.407612in}{1.784802in}}%
\pgfpathlineto{\pgfqpoint{3.407036in}{1.794542in}}%
\pgfpathlineto{\pgfqpoint{3.408189in}{1.788045in}}%
\pgfpathlineto{\pgfqpoint{3.410110in}{1.798725in}}%
\pgfpathlineto{\pgfqpoint{3.410495in}{1.796389in}}%
\pgfpathlineto{\pgfqpoint{3.410687in}{1.794762in}}%
\pgfpathlineto{\pgfqpoint{3.411263in}{1.797047in}}%
\pgfpathlineto{\pgfqpoint{3.411455in}{1.797061in}}%
\pgfpathlineto{\pgfqpoint{3.412608in}{1.805347in}}%
\pgfpathlineto{\pgfqpoint{3.412801in}{1.799579in}}%
\pgfpathlineto{\pgfqpoint{3.413569in}{1.804036in}}%
\pgfpathlineto{\pgfqpoint{3.415683in}{1.812832in}}%
\pgfpathlineto{\pgfqpoint{3.415875in}{1.810305in}}%
\pgfpathlineto{\pgfqpoint{3.416836in}{1.806608in}}%
\pgfpathlineto{\pgfqpoint{3.417028in}{1.809850in}}%
\pgfpathlineto{\pgfqpoint{3.418373in}{1.813728in}}%
\pgfpathlineto{\pgfqpoint{3.418758in}{1.811005in}}%
\pgfpathlineto{\pgfqpoint{3.419334in}{1.809445in}}%
\pgfpathlineto{\pgfqpoint{3.419719in}{1.811527in}}%
\pgfpathlineto{\pgfqpoint{3.421832in}{1.819984in}}%
\pgfpathlineto{\pgfqpoint{3.422025in}{1.818965in}}%
\pgfpathlineto{\pgfqpoint{3.422217in}{1.819170in}}%
\pgfpathlineto{\pgfqpoint{3.422793in}{1.827584in}}%
\pgfpathlineto{\pgfqpoint{3.423562in}{1.826470in}}%
\pgfpathlineto{\pgfqpoint{3.425099in}{1.839230in}}%
\pgfpathlineto{\pgfqpoint{3.425291in}{1.834403in}}%
\pgfpathlineto{\pgfqpoint{3.425676in}{1.830394in}}%
\pgfpathlineto{\pgfqpoint{3.425868in}{1.834941in}}%
\pgfpathlineto{\pgfqpoint{3.426444in}{1.832811in}}%
\pgfpathlineto{\pgfqpoint{3.428366in}{1.827798in}}%
\pgfpathlineto{\pgfqpoint{3.427021in}{1.833661in}}%
\pgfpathlineto{\pgfqpoint{3.428750in}{1.828043in}}%
\pgfpathlineto{\pgfqpoint{3.429519in}{1.832129in}}%
\pgfpathlineto{\pgfqpoint{3.429904in}{1.829661in}}%
\pgfpathlineto{\pgfqpoint{3.431825in}{1.840737in}}%
\pgfpathlineto{\pgfqpoint{3.432017in}{1.845785in}}%
\pgfpathlineto{\pgfqpoint{3.432978in}{1.843673in}}%
\pgfpathlineto{\pgfqpoint{3.434131in}{1.846664in}}%
\pgfpathlineto{\pgfqpoint{3.433747in}{1.843015in}}%
\pgfpathlineto{\pgfqpoint{3.434323in}{1.846495in}}%
\pgfpathlineto{\pgfqpoint{3.434900in}{1.848234in}}%
\pgfpathlineto{\pgfqpoint{3.435284in}{1.845922in}}%
\pgfpathlineto{\pgfqpoint{3.435476in}{1.846967in}}%
\pgfpathlineto{\pgfqpoint{3.436822in}{1.840696in}}%
\pgfpathlineto{\pgfqpoint{3.437014in}{1.844359in}}%
\pgfpathlineto{\pgfqpoint{3.438359in}{1.855793in}}%
\pgfpathlineto{\pgfqpoint{3.438551in}{1.852024in}}%
\pgfpathlineto{\pgfqpoint{3.439896in}{1.857209in}}%
\pgfpathlineto{\pgfqpoint{3.440857in}{1.851617in}}%
\pgfpathlineto{\pgfqpoint{3.441241in}{1.853170in}}%
\pgfpathlineto{\pgfqpoint{3.441626in}{1.849789in}}%
\pgfpathlineto{\pgfqpoint{3.442394in}{1.852761in}}%
\pgfpathlineto{\pgfqpoint{3.443547in}{1.859603in}}%
\pgfpathlineto{\pgfqpoint{3.443740in}{1.857485in}}%
\pgfpathlineto{\pgfqpoint{3.444893in}{1.850108in}}%
\pgfpathlineto{\pgfqpoint{3.445085in}{1.852480in}}%
\pgfpathlineto{\pgfqpoint{3.445469in}{1.850056in}}%
\pgfpathlineto{\pgfqpoint{3.445853in}{1.846982in}}%
\pgfpathlineto{\pgfqpoint{3.446622in}{1.847255in}}%
\pgfpathlineto{\pgfqpoint{3.446814in}{1.847778in}}%
\pgfpathlineto{\pgfqpoint{3.447006in}{1.846158in}}%
\pgfpathlineto{\pgfqpoint{3.447199in}{1.847071in}}%
\pgfpathlineto{\pgfqpoint{3.447583in}{1.844192in}}%
\pgfpathlineto{\pgfqpoint{3.447967in}{1.847160in}}%
\pgfpathlineto{\pgfqpoint{3.448352in}{1.846123in}}%
\pgfpathlineto{\pgfqpoint{3.448736in}{1.843641in}}%
\pgfpathlineto{\pgfqpoint{3.449120in}{1.846954in}}%
\pgfpathlineto{\pgfqpoint{3.449312in}{1.845388in}}%
\pgfpathlineto{\pgfqpoint{3.449889in}{1.847852in}}%
\pgfpathlineto{\pgfqpoint{3.450273in}{1.844583in}}%
\pgfpathlineto{\pgfqpoint{3.450465in}{1.847633in}}%
\pgfpathlineto{\pgfqpoint{3.450658in}{1.846591in}}%
\pgfpathlineto{\pgfqpoint{3.450850in}{1.850221in}}%
\pgfpathlineto{\pgfqpoint{3.451234in}{1.848852in}}%
\pgfpathlineto{\pgfqpoint{3.452003in}{1.853095in}}%
\pgfpathlineto{\pgfqpoint{3.452195in}{1.850213in}}%
\pgfpathlineto{\pgfqpoint{3.452964in}{1.847336in}}%
\pgfpathlineto{\pgfqpoint{3.453348in}{1.848004in}}%
\pgfpathlineto{\pgfqpoint{3.454117in}{1.850775in}}%
\pgfpathlineto{\pgfqpoint{3.453732in}{1.847556in}}%
\pgfpathlineto{\pgfqpoint{3.454501in}{1.848087in}}%
\pgfpathlineto{\pgfqpoint{3.454693in}{1.848268in}}%
\pgfpathlineto{\pgfqpoint{3.454885in}{1.847303in}}%
\pgfpathlineto{\pgfqpoint{3.455462in}{1.843904in}}%
\pgfpathlineto{\pgfqpoint{3.455654in}{1.848356in}}%
\pgfpathlineto{\pgfqpoint{3.456807in}{1.849348in}}%
\pgfpathlineto{\pgfqpoint{3.458152in}{1.842232in}}%
\pgfpathlineto{\pgfqpoint{3.458344in}{1.843688in}}%
\pgfpathlineto{\pgfqpoint{3.458729in}{1.843998in}}%
\pgfpathlineto{\pgfqpoint{3.459305in}{1.840056in}}%
\pgfpathlineto{\pgfqpoint{3.460074in}{1.850899in}}%
\pgfpathlineto{\pgfqpoint{3.460650in}{1.848602in}}%
\pgfpathlineto{\pgfqpoint{3.461419in}{1.846458in}}%
\pgfpathlineto{\pgfqpoint{3.461227in}{1.849494in}}%
\pgfpathlineto{\pgfqpoint{3.461803in}{1.848076in}}%
\pgfpathlineto{\pgfqpoint{3.462188in}{1.846969in}}%
\pgfpathlineto{\pgfqpoint{3.462380in}{1.848678in}}%
\pgfpathlineto{\pgfqpoint{3.462572in}{1.848027in}}%
\pgfpathlineto{\pgfqpoint{3.462764in}{1.849056in}}%
\pgfpathlineto{\pgfqpoint{3.463149in}{1.846515in}}%
\pgfpathlineto{\pgfqpoint{3.463341in}{1.846502in}}%
\pgfpathlineto{\pgfqpoint{3.463917in}{1.842038in}}%
\pgfpathlineto{\pgfqpoint{3.464878in}{1.844156in}}%
\pgfpathlineto{\pgfqpoint{3.465070in}{1.845647in}}%
\pgfpathlineto{\pgfqpoint{3.465455in}{1.840665in}}%
\pgfpathlineto{\pgfqpoint{3.465839in}{1.843639in}}%
\pgfpathlineto{\pgfqpoint{3.467184in}{1.848271in}}%
\pgfpathlineto{\pgfqpoint{3.467376in}{1.847759in}}%
\pgfpathlineto{\pgfqpoint{3.467568in}{1.848112in}}%
\pgfpathlineto{\pgfqpoint{3.467761in}{1.850433in}}%
\pgfpathlineto{\pgfqpoint{3.468337in}{1.846438in}}%
\pgfpathlineto{\pgfqpoint{3.468914in}{1.841353in}}%
\pgfpathlineto{\pgfqpoint{3.469298in}{1.843482in}}%
\pgfpathlineto{\pgfqpoint{3.469490in}{1.846454in}}%
\pgfpathlineto{\pgfqpoint{3.470067in}{1.843662in}}%
\pgfpathlineto{\pgfqpoint{3.470259in}{1.840268in}}%
\pgfpathlineto{\pgfqpoint{3.471027in}{1.844796in}}%
\pgfpathlineto{\pgfqpoint{3.472373in}{1.853649in}}%
\pgfpathlineto{\pgfqpoint{3.472565in}{1.849169in}}%
\pgfpathlineto{\pgfqpoint{3.473333in}{1.850845in}}%
\pgfpathlineto{\pgfqpoint{3.473526in}{1.854124in}}%
\pgfpathlineto{\pgfqpoint{3.473910in}{1.847759in}}%
\pgfpathlineto{\pgfqpoint{3.474294in}{1.850627in}}%
\pgfpathlineto{\pgfqpoint{3.475447in}{1.845591in}}%
\pgfpathlineto{\pgfqpoint{3.476792in}{1.852659in}}%
\pgfpathlineto{\pgfqpoint{3.477561in}{1.849620in}}%
\pgfpathlineto{\pgfqpoint{3.477753in}{1.851731in}}%
\pgfpathlineto{\pgfqpoint{3.479099in}{1.860549in}}%
\pgfpathlineto{\pgfqpoint{3.479291in}{1.860134in}}%
\pgfpathlineto{\pgfqpoint{3.479675in}{1.859399in}}%
\pgfpathlineto{\pgfqpoint{3.480059in}{1.863311in}}%
\pgfpathlineto{\pgfqpoint{3.480828in}{1.861374in}}%
\pgfpathlineto{\pgfqpoint{3.481405in}{1.862831in}}%
\pgfpathlineto{\pgfqpoint{3.481981in}{1.859517in}}%
\pgfpathlineto{\pgfqpoint{3.482173in}{1.861657in}}%
\pgfpathlineto{\pgfqpoint{3.482558in}{1.857855in}}%
\pgfpathlineto{\pgfqpoint{3.482750in}{1.858160in}}%
\pgfpathlineto{\pgfqpoint{3.483326in}{1.853467in}}%
\pgfpathlineto{\pgfqpoint{3.483903in}{1.855129in}}%
\pgfpathlineto{\pgfqpoint{3.484095in}{1.855914in}}%
\pgfpathlineto{\pgfqpoint{3.484287in}{1.853101in}}%
\pgfpathlineto{\pgfqpoint{3.484864in}{1.849906in}}%
\pgfpathlineto{\pgfqpoint{3.485248in}{1.851396in}}%
\pgfpathlineto{\pgfqpoint{3.486977in}{1.862112in}}%
\pgfpathlineto{\pgfqpoint{3.487170in}{1.860940in}}%
\pgfpathlineto{\pgfqpoint{3.487554in}{1.861383in}}%
\pgfpathlineto{\pgfqpoint{3.488323in}{1.857838in}}%
\pgfpathlineto{\pgfqpoint{3.488707in}{1.860480in}}%
\pgfpathlineto{\pgfqpoint{3.489091in}{1.856726in}}%
\pgfpathlineto{\pgfqpoint{3.490821in}{1.850093in}}%
\pgfpathlineto{\pgfqpoint{3.491782in}{1.855868in}}%
\pgfpathlineto{\pgfqpoint{3.492166in}{1.855172in}}%
\pgfpathlineto{\pgfqpoint{3.492358in}{1.855023in}}%
\pgfpathlineto{\pgfqpoint{3.492550in}{1.852453in}}%
\pgfpathlineto{\pgfqpoint{3.493127in}{1.857464in}}%
\pgfpathlineto{\pgfqpoint{3.493319in}{1.856664in}}%
\pgfpathlineto{\pgfqpoint{3.493703in}{1.859689in}}%
\pgfpathlineto{\pgfqpoint{3.494280in}{1.856357in}}%
\pgfpathlineto{\pgfqpoint{3.494472in}{1.856886in}}%
\pgfpathlineto{\pgfqpoint{3.494664in}{1.857768in}}%
\pgfpathlineto{\pgfqpoint{3.495817in}{1.851084in}}%
\pgfpathlineto{\pgfqpoint{3.497162in}{1.861790in}}%
\pgfpathlineto{\pgfqpoint{3.497354in}{1.860388in}}%
\pgfpathlineto{\pgfqpoint{3.497547in}{1.856255in}}%
\pgfpathlineto{\pgfqpoint{3.498123in}{1.864631in}}%
\pgfpathlineto{\pgfqpoint{3.498315in}{1.863017in}}%
\pgfpathlineto{\pgfqpoint{3.498507in}{1.865416in}}%
\pgfpathlineto{\pgfqpoint{3.499468in}{1.864913in}}%
\pgfpathlineto{\pgfqpoint{3.499853in}{1.861659in}}%
\pgfpathlineto{\pgfqpoint{3.500429in}{1.864732in}}%
\pgfpathlineto{\pgfqpoint{3.500813in}{1.865487in}}%
\pgfpathlineto{\pgfqpoint{3.501198in}{1.864107in}}%
\pgfpathlineto{\pgfqpoint{3.501390in}{1.859692in}}%
\pgfpathlineto{\pgfqpoint{3.502159in}{1.866359in}}%
\pgfpathlineto{\pgfqpoint{3.502735in}{1.871134in}}%
\pgfpathlineto{\pgfqpoint{3.503312in}{1.869505in}}%
\pgfpathlineto{\pgfqpoint{3.504657in}{1.872524in}}%
\pgfpathlineto{\pgfqpoint{3.505233in}{1.873930in}}%
\pgfpathlineto{\pgfqpoint{3.505426in}{1.871892in}}%
\pgfpathlineto{\pgfqpoint{3.505618in}{1.870513in}}%
\pgfpathlineto{\pgfqpoint{3.506194in}{1.870956in}}%
\pgfpathlineto{\pgfqpoint{3.506771in}{1.873997in}}%
\pgfpathlineto{\pgfqpoint{3.507347in}{1.871536in}}%
\pgfpathlineto{\pgfqpoint{3.507732in}{1.872797in}}%
\pgfpathlineto{\pgfqpoint{3.508308in}{1.871030in}}%
\pgfpathlineto{\pgfqpoint{3.509845in}{1.863528in}}%
\pgfpathlineto{\pgfqpoint{3.510038in}{1.865715in}}%
\pgfpathlineto{\pgfqpoint{3.510230in}{1.865985in}}%
\pgfpathlineto{\pgfqpoint{3.510422in}{1.862568in}}%
\pgfpathlineto{\pgfqpoint{3.511383in}{1.864462in}}%
\pgfpathlineto{\pgfqpoint{3.511575in}{1.866288in}}%
\pgfpathlineto{\pgfqpoint{3.512151in}{1.861353in}}%
\pgfpathlineto{\pgfqpoint{3.512344in}{1.864297in}}%
\pgfpathlineto{\pgfqpoint{3.514650in}{1.852023in}}%
\pgfpathlineto{\pgfqpoint{3.514842in}{1.854142in}}%
\pgfpathlineto{\pgfqpoint{3.516956in}{1.871576in}}%
\pgfpathlineto{\pgfqpoint{3.517148in}{1.869611in}}%
\pgfpathlineto{\pgfqpoint{3.517532in}{1.873534in}}%
\pgfpathlineto{\pgfqpoint{3.518109in}{1.871147in}}%
\pgfpathlineto{\pgfqpoint{3.518301in}{1.871659in}}%
\pgfpathlineto{\pgfqpoint{3.518493in}{1.870100in}}%
\pgfpathlineto{\pgfqpoint{3.520030in}{1.862792in}}%
\pgfpathlineto{\pgfqpoint{3.520222in}{1.864995in}}%
\pgfpathlineto{\pgfqpoint{3.520799in}{1.862712in}}%
\pgfpathlineto{\pgfqpoint{3.521183in}{1.863489in}}%
\pgfpathlineto{\pgfqpoint{3.522528in}{1.869442in}}%
\pgfpathlineto{\pgfqpoint{3.523105in}{1.867915in}}%
\pgfpathlineto{\pgfqpoint{3.523297in}{1.865880in}}%
\pgfpathlineto{\pgfqpoint{3.523874in}{1.869640in}}%
\pgfpathlineto{\pgfqpoint{3.524642in}{1.876989in}}%
\pgfpathlineto{\pgfqpoint{3.525027in}{1.871920in}}%
\pgfpathlineto{\pgfqpoint{3.525987in}{1.864150in}}%
\pgfpathlineto{\pgfqpoint{3.526756in}{1.865658in}}%
\pgfpathlineto{\pgfqpoint{3.528101in}{1.875546in}}%
\pgfpathlineto{\pgfqpoint{3.528294in}{1.875230in}}%
\pgfpathlineto{\pgfqpoint{3.528486in}{1.874336in}}%
\pgfpathlineto{\pgfqpoint{3.528678in}{1.876613in}}%
\pgfpathlineto{\pgfqpoint{3.530600in}{1.883746in}}%
\pgfpathlineto{\pgfqpoint{3.530792in}{1.882258in}}%
\pgfpathlineto{\pgfqpoint{3.531368in}{1.886224in}}%
\pgfpathlineto{\pgfqpoint{3.531560in}{1.885937in}}%
\pgfpathlineto{\pgfqpoint{3.533098in}{1.897742in}}%
\pgfpathlineto{\pgfqpoint{3.533290in}{1.896928in}}%
\pgfpathlineto{\pgfqpoint{3.533482in}{1.899083in}}%
\pgfpathlineto{\pgfqpoint{3.536172in}{1.917424in}}%
\pgfpathlineto{\pgfqpoint{3.537518in}{1.913246in}}%
\pgfpathlineto{\pgfqpoint{3.538863in}{1.905740in}}%
\pgfpathlineto{\pgfqpoint{3.539247in}{1.910353in}}%
\pgfpathlineto{\pgfqpoint{3.540592in}{1.913162in}}%
\pgfpathlineto{\pgfqpoint{3.540784in}{1.912513in}}%
\pgfpathlineto{\pgfqpoint{3.540977in}{1.914819in}}%
\pgfpathlineto{\pgfqpoint{3.541169in}{1.915961in}}%
\pgfpathlineto{\pgfqpoint{3.541361in}{1.913226in}}%
\pgfpathlineto{\pgfqpoint{3.541553in}{1.913937in}}%
\pgfpathlineto{\pgfqpoint{3.542706in}{1.903547in}}%
\pgfpathlineto{\pgfqpoint{3.542898in}{1.903922in}}%
\pgfpathlineto{\pgfqpoint{3.543283in}{1.902903in}}%
\pgfpathlineto{\pgfqpoint{3.543667in}{1.907905in}}%
\pgfpathlineto{\pgfqpoint{3.544051in}{1.901935in}}%
\pgfpathlineto{\pgfqpoint{3.544243in}{1.898555in}}%
\pgfpathlineto{\pgfqpoint{3.545012in}{1.900391in}}%
\pgfpathlineto{\pgfqpoint{3.546357in}{1.908151in}}%
\pgfpathlineto{\pgfqpoint{3.546549in}{1.907253in}}%
\pgfpathlineto{\pgfqpoint{3.548471in}{1.902319in}}%
\pgfpathlineto{\pgfqpoint{3.546934in}{1.907742in}}%
\pgfpathlineto{\pgfqpoint{3.548663in}{1.902970in}}%
\pgfpathlineto{\pgfqpoint{3.549048in}{1.908386in}}%
\pgfpathlineto{\pgfqpoint{3.550008in}{1.905939in}}%
\pgfpathlineto{\pgfqpoint{3.550969in}{1.898429in}}%
\pgfpathlineto{\pgfqpoint{3.551546in}{1.902496in}}%
\pgfpathlineto{\pgfqpoint{3.551738in}{1.903522in}}%
\pgfpathlineto{\pgfqpoint{3.551930in}{1.899904in}}%
\pgfpathlineto{\pgfqpoint{3.552122in}{1.900900in}}%
\pgfpathlineto{\pgfqpoint{3.552315in}{1.896840in}}%
\pgfpathlineto{\pgfqpoint{3.553275in}{1.899957in}}%
\pgfpathlineto{\pgfqpoint{3.554813in}{1.905766in}}%
\pgfpathlineto{\pgfqpoint{3.555197in}{1.904421in}}%
\pgfpathlineto{\pgfqpoint{3.555966in}{1.899817in}}%
\pgfpathlineto{\pgfqpoint{3.556350in}{1.900736in}}%
\pgfpathlineto{\pgfqpoint{3.556542in}{1.904365in}}%
\pgfpathlineto{\pgfqpoint{3.557119in}{1.900556in}}%
\pgfpathlineto{\pgfqpoint{3.558656in}{1.889106in}}%
\pgfpathlineto{\pgfqpoint{3.559233in}{1.898485in}}%
\pgfpathlineto{\pgfqpoint{3.560193in}{1.892133in}}%
\pgfpathlineto{\pgfqpoint{3.560770in}{1.889161in}}%
\pgfpathlineto{\pgfqpoint{3.561154in}{1.887412in}}%
\pgfpathlineto{\pgfqpoint{3.561346in}{1.889412in}}%
\pgfpathlineto{\pgfqpoint{3.561731in}{1.889442in}}%
\pgfpathlineto{\pgfqpoint{3.561923in}{1.889865in}}%
\pgfpathlineto{\pgfqpoint{3.562115in}{1.887699in}}%
\pgfpathlineto{\pgfqpoint{3.562307in}{1.887377in}}%
\pgfpathlineto{\pgfqpoint{3.562692in}{1.891113in}}%
\pgfpathlineto{\pgfqpoint{3.563268in}{1.888246in}}%
\pgfpathlineto{\pgfqpoint{3.564998in}{1.877059in}}%
\pgfpathlineto{\pgfqpoint{3.564229in}{1.888879in}}%
\pgfpathlineto{\pgfqpoint{3.565574in}{1.880968in}}%
\pgfpathlineto{\pgfqpoint{3.566151in}{1.881527in}}%
\pgfpathlineto{\pgfqpoint{3.567496in}{1.875020in}}%
\pgfpathlineto{\pgfqpoint{3.567688in}{1.872301in}}%
\pgfpathlineto{\pgfqpoint{3.568457in}{1.876685in}}%
\pgfpathlineto{\pgfqpoint{3.569802in}{1.885451in}}%
\pgfpathlineto{\pgfqpoint{3.570378in}{1.884920in}}%
\pgfpathlineto{\pgfqpoint{3.572108in}{1.871760in}}%
\pgfpathlineto{\pgfqpoint{3.572300in}{1.869060in}}%
\pgfpathlineto{\pgfqpoint{3.572684in}{1.873081in}}%
\pgfpathlineto{\pgfqpoint{3.573837in}{1.876915in}}%
\pgfpathlineto{\pgfqpoint{3.574222in}{1.878589in}}%
\pgfpathlineto{\pgfqpoint{3.574414in}{1.872607in}}%
\pgfpathlineto{\pgfqpoint{3.574798in}{1.880059in}}%
\pgfpathlineto{\pgfqpoint{3.575182in}{1.876264in}}%
\pgfpathlineto{\pgfqpoint{3.575951in}{1.886056in}}%
\pgfpathlineto{\pgfqpoint{3.576528in}{1.882119in}}%
\pgfpathlineto{\pgfqpoint{3.577296in}{1.877759in}}%
\pgfpathlineto{\pgfqpoint{3.577681in}{1.879111in}}%
\pgfpathlineto{\pgfqpoint{3.578065in}{1.879017in}}%
\pgfpathlineto{\pgfqpoint{3.578257in}{1.874243in}}%
\pgfpathlineto{\pgfqpoint{3.579218in}{1.876140in}}%
\pgfpathlineto{\pgfqpoint{3.579410in}{1.878510in}}%
\pgfpathlineto{\pgfqpoint{3.580179in}{1.876958in}}%
\pgfpathlineto{\pgfqpoint{3.580371in}{1.876243in}}%
\pgfpathlineto{\pgfqpoint{3.580563in}{1.879184in}}%
\pgfpathlineto{\pgfqpoint{3.580755in}{1.878799in}}%
\pgfpathlineto{\pgfqpoint{3.580948in}{1.879058in}}%
\pgfpathlineto{\pgfqpoint{3.582293in}{1.876326in}}%
\pgfpathlineto{\pgfqpoint{3.582485in}{1.876203in}}%
\pgfpathlineto{\pgfqpoint{3.582677in}{1.876785in}}%
\pgfpathlineto{\pgfqpoint{3.582869in}{1.874543in}}%
\pgfpathlineto{\pgfqpoint{3.583446in}{1.878822in}}%
\pgfpathlineto{\pgfqpoint{3.583638in}{1.878474in}}%
\pgfpathlineto{\pgfqpoint{3.584407in}{1.885777in}}%
\pgfpathlineto{\pgfqpoint{3.586136in}{1.893804in}}%
\pgfpathlineto{\pgfqpoint{3.587866in}{1.885951in}}%
\pgfpathlineto{\pgfqpoint{3.588826in}{1.890549in}}%
\pgfpathlineto{\pgfqpoint{3.589211in}{1.890023in}}%
\pgfpathlineto{\pgfqpoint{3.590940in}{1.881690in}}%
\pgfpathlineto{\pgfqpoint{3.591132in}{1.885782in}}%
\pgfpathlineto{\pgfqpoint{3.593823in}{1.901131in}}%
\pgfpathlineto{\pgfqpoint{3.591517in}{1.885368in}}%
\pgfpathlineto{\pgfqpoint{3.594591in}{1.898183in}}%
\pgfpathlineto{\pgfqpoint{3.594976in}{1.897667in}}%
\pgfpathlineto{\pgfqpoint{3.595168in}{1.899742in}}%
\pgfpathlineto{\pgfqpoint{3.596129in}{1.896525in}}%
\pgfpathlineto{\pgfqpoint{3.596321in}{1.900844in}}%
\pgfpathlineto{\pgfqpoint{3.597090in}{1.896972in}}%
\pgfpathlineto{\pgfqpoint{3.597282in}{1.896723in}}%
\pgfpathlineto{\pgfqpoint{3.597474in}{1.898766in}}%
\pgfpathlineto{\pgfqpoint{3.597858in}{1.894778in}}%
\pgfpathlineto{\pgfqpoint{3.599203in}{1.889972in}}%
\pgfpathlineto{\pgfqpoint{3.599396in}{1.890363in}}%
\pgfpathlineto{\pgfqpoint{3.601125in}{1.879373in}}%
\pgfpathlineto{\pgfqpoint{3.602470in}{1.890110in}}%
\pgfpathlineto{\pgfqpoint{3.602663in}{1.889452in}}%
\pgfpathlineto{\pgfqpoint{3.603816in}{1.887790in}}%
\pgfpathlineto{\pgfqpoint{3.604392in}{1.892128in}}%
\pgfpathlineto{\pgfqpoint{3.604776in}{1.890587in}}%
\pgfpathlineto{\pgfqpoint{3.605353in}{1.894032in}}%
\pgfpathlineto{\pgfqpoint{3.605929in}{1.887991in}}%
\pgfpathlineto{\pgfqpoint{3.606314in}{1.887362in}}%
\pgfpathlineto{\pgfqpoint{3.606698in}{1.889595in}}%
\pgfpathlineto{\pgfqpoint{3.606890in}{1.885937in}}%
\pgfpathlineto{\pgfqpoint{3.607275in}{1.890924in}}%
\pgfpathlineto{\pgfqpoint{3.607659in}{1.890005in}}%
\pgfpathlineto{\pgfqpoint{3.609581in}{1.905182in}}%
\pgfpathlineto{\pgfqpoint{3.609773in}{1.904919in}}%
\pgfpathlineto{\pgfqpoint{3.609965in}{1.905231in}}%
\pgfpathlineto{\pgfqpoint{3.610157in}{1.903953in}}%
\pgfpathlineto{\pgfqpoint{3.611502in}{1.900464in}}%
\pgfpathlineto{\pgfqpoint{3.610541in}{1.904124in}}%
\pgfpathlineto{\pgfqpoint{3.611694in}{1.901819in}}%
\pgfpathlineto{\pgfqpoint{3.612271in}{1.903181in}}%
\pgfpathlineto{\pgfqpoint{3.612655in}{1.901033in}}%
\pgfpathlineto{\pgfqpoint{3.612847in}{1.901240in}}%
\pgfpathlineto{\pgfqpoint{3.614000in}{1.894089in}}%
\pgfpathlineto{\pgfqpoint{3.614193in}{1.895402in}}%
\pgfpathlineto{\pgfqpoint{3.614577in}{1.899150in}}%
\pgfpathlineto{\pgfqpoint{3.615538in}{1.897169in}}%
\pgfpathlineto{\pgfqpoint{3.615922in}{1.899359in}}%
\pgfpathlineto{\pgfqpoint{3.616114in}{1.898289in}}%
\pgfpathlineto{\pgfqpoint{3.616691in}{1.895651in}}%
\pgfpathlineto{\pgfqpoint{3.617075in}{1.898643in}}%
\pgfpathlineto{\pgfqpoint{3.618612in}{1.903298in}}%
\pgfpathlineto{\pgfqpoint{3.617844in}{1.897278in}}%
\pgfpathlineto{\pgfqpoint{3.618805in}{1.903193in}}%
\pgfpathlineto{\pgfqpoint{3.619189in}{1.900368in}}%
\pgfpathlineto{\pgfqpoint{3.619958in}{1.902503in}}%
\pgfpathlineto{\pgfqpoint{3.621879in}{1.896270in}}%
\pgfpathlineto{\pgfqpoint{3.625531in}{1.924556in}}%
\pgfpathlineto{\pgfqpoint{3.625723in}{1.926292in}}%
\pgfpathlineto{\pgfqpoint{3.626299in}{1.921529in}}%
\pgfpathlineto{\pgfqpoint{3.626491in}{1.923256in}}%
\pgfpathlineto{\pgfqpoint{3.628413in}{1.933852in}}%
\pgfpathlineto{\pgfqpoint{3.629182in}{1.932686in}}%
\pgfpathlineto{\pgfqpoint{3.629566in}{1.929632in}}%
\pgfpathlineto{\pgfqpoint{3.630143in}{1.931817in}}%
\pgfpathlineto{\pgfqpoint{3.631488in}{1.939134in}}%
\pgfpathlineto{\pgfqpoint{3.632256in}{1.933159in}}%
\pgfpathlineto{\pgfqpoint{3.632641in}{1.935686in}}%
\pgfpathlineto{\pgfqpoint{3.633409in}{1.934963in}}%
\pgfpathlineto{\pgfqpoint{3.633986in}{1.940186in}}%
\pgfpathlineto{\pgfqpoint{3.634370in}{1.935614in}}%
\pgfpathlineto{\pgfqpoint{3.635523in}{1.937295in}}%
\pgfpathlineto{\pgfqpoint{3.636484in}{1.941862in}}%
\pgfpathlineto{\pgfqpoint{3.636868in}{1.939858in}}%
\pgfpathlineto{\pgfqpoint{3.637061in}{1.940219in}}%
\pgfpathlineto{\pgfqpoint{3.637253in}{1.939835in}}%
\pgfpathlineto{\pgfqpoint{3.637637in}{1.936071in}}%
\pgfpathlineto{\pgfqpoint{3.638214in}{1.940428in}}%
\pgfpathlineto{\pgfqpoint{3.638982in}{1.943929in}}%
\pgfpathlineto{\pgfqpoint{3.639174in}{1.942545in}}%
\pgfpathlineto{\pgfqpoint{3.639367in}{1.940548in}}%
\pgfpathlineto{\pgfqpoint{3.640327in}{1.941110in}}%
\pgfpathlineto{\pgfqpoint{3.640904in}{1.941156in}}%
\pgfpathlineto{\pgfqpoint{3.642633in}{1.931052in}}%
\pgfpathlineto{\pgfqpoint{3.642826in}{1.931897in}}%
\pgfpathlineto{\pgfqpoint{3.643018in}{1.932341in}}%
\pgfpathlineto{\pgfqpoint{3.643210in}{1.931831in}}%
\pgfpathlineto{\pgfqpoint{3.643786in}{1.926326in}}%
\pgfpathlineto{\pgfqpoint{3.644555in}{1.927545in}}%
\pgfpathlineto{\pgfqpoint{3.645132in}{1.925895in}}%
\pgfpathlineto{\pgfqpoint{3.646092in}{1.920772in}}%
\pgfpathlineto{\pgfqpoint{3.646477in}{1.922422in}}%
\pgfpathlineto{\pgfqpoint{3.647053in}{1.917873in}}%
\pgfpathlineto{\pgfqpoint{3.647630in}{1.921163in}}%
\pgfpathlineto{\pgfqpoint{3.647822in}{1.922602in}}%
\pgfpathlineto{\pgfqpoint{3.648398in}{1.920112in}}%
\pgfpathlineto{\pgfqpoint{3.649167in}{1.916484in}}%
\pgfpathlineto{\pgfqpoint{3.649359in}{1.917593in}}%
\pgfpathlineto{\pgfqpoint{3.649936in}{1.922973in}}%
\pgfpathlineto{\pgfqpoint{3.650512in}{1.920458in}}%
\pgfpathlineto{\pgfqpoint{3.650705in}{1.919252in}}%
\pgfpathlineto{\pgfqpoint{3.651089in}{1.923822in}}%
\pgfpathlineto{\pgfqpoint{3.651281in}{1.922643in}}%
\pgfpathlineto{\pgfqpoint{3.651473in}{1.922943in}}%
\pgfpathlineto{\pgfqpoint{3.651858in}{1.921776in}}%
\pgfpathlineto{\pgfqpoint{3.653011in}{1.933116in}}%
\pgfpathlineto{\pgfqpoint{3.653203in}{1.928219in}}%
\pgfpathlineto{\pgfqpoint{3.656662in}{1.905658in}}%
\pgfpathlineto{\pgfqpoint{3.657430in}{1.912165in}}%
\pgfpathlineto{\pgfqpoint{3.657815in}{1.908956in}}%
\pgfpathlineto{\pgfqpoint{3.659160in}{1.903964in}}%
\pgfpathlineto{\pgfqpoint{3.659352in}{1.903413in}}%
\pgfpathlineto{\pgfqpoint{3.659736in}{1.904161in}}%
\pgfpathlineto{\pgfqpoint{3.659929in}{1.908994in}}%
\pgfpathlineto{\pgfqpoint{3.660697in}{1.907103in}}%
\pgfpathlineto{\pgfqpoint{3.662042in}{1.902242in}}%
\pgfpathlineto{\pgfqpoint{3.662427in}{1.900881in}}%
\pgfpathlineto{\pgfqpoint{3.662619in}{1.902962in}}%
\pgfpathlineto{\pgfqpoint{3.663003in}{1.907405in}}%
\pgfpathlineto{\pgfqpoint{3.663964in}{1.907273in}}%
\pgfpathlineto{\pgfqpoint{3.665117in}{1.901696in}}%
\pgfpathlineto{\pgfqpoint{3.664348in}{1.907599in}}%
\pgfpathlineto{\pgfqpoint{3.665501in}{1.904068in}}%
\pgfpathlineto{\pgfqpoint{3.665694in}{1.903318in}}%
\pgfpathlineto{\pgfqpoint{3.665886in}{1.904544in}}%
\pgfpathlineto{\pgfqpoint{3.666270in}{1.904171in}}%
\pgfpathlineto{\pgfqpoint{3.666462in}{1.906427in}}%
\pgfpathlineto{\pgfqpoint{3.667039in}{1.900616in}}%
\pgfpathlineto{\pgfqpoint{3.667231in}{1.899986in}}%
\pgfpathlineto{\pgfqpoint{3.667423in}{1.901721in}}%
\pgfpathlineto{\pgfqpoint{3.668576in}{1.911498in}}%
\pgfpathlineto{\pgfqpoint{3.668960in}{1.910774in}}%
\pgfpathlineto{\pgfqpoint{3.670690in}{1.894647in}}%
\pgfpathlineto{\pgfqpoint{3.671074in}{1.890990in}}%
\pgfpathlineto{\pgfqpoint{3.671843in}{1.893658in}}%
\pgfpathlineto{\pgfqpoint{3.672227in}{1.895214in}}%
\pgfpathlineto{\pgfqpoint{3.673380in}{1.889431in}}%
\pgfpathlineto{\pgfqpoint{3.673573in}{1.890329in}}%
\pgfpathlineto{\pgfqpoint{3.674726in}{1.886941in}}%
\pgfpathlineto{\pgfqpoint{3.674149in}{1.891511in}}%
\pgfpathlineto{\pgfqpoint{3.675110in}{1.888982in}}%
\pgfpathlineto{\pgfqpoint{3.675494in}{1.891917in}}%
\pgfpathlineto{\pgfqpoint{3.675879in}{1.888720in}}%
\pgfpathlineto{\pgfqpoint{3.676071in}{1.886707in}}%
\pgfpathlineto{\pgfqpoint{3.676455in}{1.889924in}}%
\pgfpathlineto{\pgfqpoint{3.676647in}{1.889926in}}%
\pgfpathlineto{\pgfqpoint{3.677032in}{1.890851in}}%
\pgfpathlineto{\pgfqpoint{3.677224in}{1.886650in}}%
\pgfpathlineto{\pgfqpoint{3.677992in}{1.892026in}}%
\pgfpathlineto{\pgfqpoint{3.678377in}{1.893301in}}%
\pgfpathlineto{\pgfqpoint{3.678761in}{1.890639in}}%
\pgfpathlineto{\pgfqpoint{3.680491in}{1.884631in}}%
\pgfpathlineto{\pgfqpoint{3.681259in}{1.891963in}}%
\pgfpathlineto{\pgfqpoint{3.681836in}{1.896139in}}%
\pgfpathlineto{\pgfqpoint{3.682028in}{1.891074in}}%
\pgfpathlineto{\pgfqpoint{3.682604in}{1.891573in}}%
\pgfpathlineto{\pgfqpoint{3.683373in}{1.885752in}}%
\pgfpathlineto{\pgfqpoint{3.683757in}{1.884640in}}%
\pgfpathlineto{\pgfqpoint{3.683950in}{1.884552in}}%
\pgfpathlineto{\pgfqpoint{3.684526in}{1.889496in}}%
\pgfpathlineto{\pgfqpoint{3.685103in}{1.887348in}}%
\pgfpathlineto{\pgfqpoint{3.685295in}{1.887157in}}%
\pgfpathlineto{\pgfqpoint{3.685487in}{1.884654in}}%
\pgfpathlineto{\pgfqpoint{3.686256in}{1.888633in}}%
\pgfpathlineto{\pgfqpoint{3.687793in}{1.901832in}}%
\pgfpathlineto{\pgfqpoint{3.689138in}{1.894731in}}%
\pgfpathlineto{\pgfqpoint{3.689330in}{1.894891in}}%
\pgfpathlineto{\pgfqpoint{3.691060in}{1.911758in}}%
\pgfpathlineto{\pgfqpoint{3.691252in}{1.908054in}}%
\pgfpathlineto{\pgfqpoint{3.692021in}{1.904329in}}%
\pgfpathlineto{\pgfqpoint{3.692597in}{1.906165in}}%
\pgfpathlineto{\pgfqpoint{3.692789in}{1.911035in}}%
\pgfpathlineto{\pgfqpoint{3.693750in}{1.908555in}}%
\pgfpathlineto{\pgfqpoint{3.694711in}{1.905130in}}%
\pgfpathlineto{\pgfqpoint{3.694903in}{1.905278in}}%
\pgfpathlineto{\pgfqpoint{3.695287in}{1.909818in}}%
\pgfpathlineto{\pgfqpoint{3.696056in}{1.906353in}}%
\pgfpathlineto{\pgfqpoint{3.697017in}{1.896452in}}%
\pgfpathlineto{\pgfqpoint{3.697401in}{1.899326in}}%
\pgfpathlineto{\pgfqpoint{3.697786in}{1.899608in}}%
\pgfpathlineto{\pgfqpoint{3.697978in}{1.896758in}}%
\pgfpathlineto{\pgfqpoint{3.698747in}{1.901597in}}%
\pgfpathlineto{\pgfqpoint{3.699515in}{1.893750in}}%
\pgfpathlineto{\pgfqpoint{3.699900in}{1.896217in}}%
\pgfpathlineto{\pgfqpoint{3.701437in}{1.909972in}}%
\pgfpathlineto{\pgfqpoint{3.702013in}{1.914727in}}%
\pgfpathlineto{\pgfqpoint{3.702398in}{1.910031in}}%
\pgfpathlineto{\pgfqpoint{3.702974in}{1.906331in}}%
\pgfpathlineto{\pgfqpoint{3.703551in}{1.908545in}}%
\pgfpathlineto{\pgfqpoint{3.703743in}{1.909772in}}%
\pgfpathlineto{\pgfqpoint{3.704319in}{1.906484in}}%
\pgfpathlineto{\pgfqpoint{3.704896in}{1.904128in}}%
\pgfpathlineto{\pgfqpoint{3.705857in}{1.898919in}}%
\pgfpathlineto{\pgfqpoint{3.706049in}{1.899719in}}%
\pgfpathlineto{\pgfqpoint{3.706241in}{1.900863in}}%
\pgfpathlineto{\pgfqpoint{3.706433in}{1.899158in}}%
\pgfpathlineto{\pgfqpoint{3.707010in}{1.899082in}}%
\pgfpathlineto{\pgfqpoint{3.707971in}{1.897747in}}%
\pgfpathlineto{\pgfqpoint{3.707394in}{1.900381in}}%
\pgfpathlineto{\pgfqpoint{3.708163in}{1.898295in}}%
\pgfpathlineto{\pgfqpoint{3.708931in}{1.904952in}}%
\pgfpathlineto{\pgfqpoint{3.710084in}{1.902997in}}%
\pgfpathlineto{\pgfqpoint{3.711045in}{1.893548in}}%
\pgfpathlineto{\pgfqpoint{3.711430in}{1.899275in}}%
\pgfpathlineto{\pgfqpoint{3.712198in}{1.896315in}}%
\pgfpathlineto{\pgfqpoint{3.712390in}{1.900124in}}%
\pgfpathlineto{\pgfqpoint{3.712583in}{1.900406in}}%
\pgfpathlineto{\pgfqpoint{3.713928in}{1.892244in}}%
\pgfpathlineto{\pgfqpoint{3.712967in}{1.901136in}}%
\pgfpathlineto{\pgfqpoint{3.714312in}{1.893040in}}%
\pgfpathlineto{\pgfqpoint{3.716426in}{1.904699in}}%
\pgfpathlineto{\pgfqpoint{3.716618in}{1.901414in}}%
\pgfpathlineto{\pgfqpoint{3.717387in}{1.904777in}}%
\pgfpathlineto{\pgfqpoint{3.717963in}{1.902190in}}%
\pgfpathlineto{\pgfqpoint{3.718155in}{1.903079in}}%
\pgfpathlineto{\pgfqpoint{3.718348in}{1.899391in}}%
\pgfpathlineto{\pgfqpoint{3.719116in}{1.903645in}}%
\pgfpathlineto{\pgfqpoint{3.719308in}{1.902804in}}%
\pgfpathlineto{\pgfqpoint{3.719501in}{1.906671in}}%
\pgfpathlineto{\pgfqpoint{3.720077in}{1.909773in}}%
\pgfpathlineto{\pgfqpoint{3.720269in}{1.908148in}}%
\pgfpathlineto{\pgfqpoint{3.721038in}{1.901873in}}%
\pgfpathlineto{\pgfqpoint{3.721614in}{1.902832in}}%
\pgfpathlineto{\pgfqpoint{3.721807in}{1.903657in}}%
\pgfpathlineto{\pgfqpoint{3.722191in}{1.896955in}}%
\pgfpathlineto{\pgfqpoint{3.722960in}{1.899679in}}%
\pgfpathlineto{\pgfqpoint{3.724497in}{1.909747in}}%
\pgfpathlineto{\pgfqpoint{3.724689in}{1.907259in}}%
\pgfpathlineto{\pgfqpoint{3.725458in}{1.911075in}}%
\pgfpathlineto{\pgfqpoint{3.725650in}{1.912538in}}%
\pgfpathlineto{\pgfqpoint{3.726034in}{1.910311in}}%
\pgfpathlineto{\pgfqpoint{3.726803in}{1.906368in}}%
\pgfpathlineto{\pgfqpoint{3.726419in}{1.910571in}}%
\pgfpathlineto{\pgfqpoint{3.727187in}{1.907830in}}%
\pgfpathlineto{\pgfqpoint{3.727380in}{1.910055in}}%
\pgfpathlineto{\pgfqpoint{3.727764in}{1.904473in}}%
\pgfpathlineto{\pgfqpoint{3.727956in}{1.905099in}}%
\pgfpathlineto{\pgfqpoint{3.728725in}{1.903665in}}%
\pgfpathlineto{\pgfqpoint{3.728340in}{1.906331in}}%
\pgfpathlineto{\pgfqpoint{3.728917in}{1.905576in}}%
\pgfpathlineto{\pgfqpoint{3.729109in}{1.906048in}}%
\pgfpathlineto{\pgfqpoint{3.729301in}{1.903805in}}%
\pgfpathlineto{\pgfqpoint{3.729493in}{1.902585in}}%
\pgfpathlineto{\pgfqpoint{3.729878in}{1.906720in}}%
\pgfpathlineto{\pgfqpoint{3.730070in}{1.906271in}}%
\pgfpathlineto{\pgfqpoint{3.730262in}{1.908138in}}%
\pgfpathlineto{\pgfqpoint{3.730454in}{1.907174in}}%
\pgfpathlineto{\pgfqpoint{3.731799in}{1.911705in}}%
\pgfpathlineto{\pgfqpoint{3.733337in}{1.902899in}}%
\pgfpathlineto{\pgfqpoint{3.733529in}{1.904964in}}%
\pgfpathlineto{\pgfqpoint{3.734105in}{1.908777in}}%
\pgfpathlineto{\pgfqpoint{3.734682in}{1.907181in}}%
\pgfpathlineto{\pgfqpoint{3.735258in}{1.905999in}}%
\pgfpathlineto{\pgfqpoint{3.735451in}{1.908660in}}%
\pgfpathlineto{\pgfqpoint{3.735835in}{1.906174in}}%
\pgfpathlineto{\pgfqpoint{3.736796in}{1.913946in}}%
\pgfpathlineto{\pgfqpoint{3.737564in}{1.910092in}}%
\pgfpathlineto{\pgfqpoint{3.738333in}{1.905308in}}%
\pgfpathlineto{\pgfqpoint{3.738717in}{1.908784in}}%
\pgfpathlineto{\pgfqpoint{3.740447in}{1.923839in}}%
\pgfpathlineto{\pgfqpoint{3.740639in}{1.923524in}}%
\pgfpathlineto{\pgfqpoint{3.741408in}{1.928649in}}%
\pgfpathlineto{\pgfqpoint{3.741792in}{1.926996in}}%
\pgfpathlineto{\pgfqpoint{3.741984in}{1.926119in}}%
\pgfpathlineto{\pgfqpoint{3.742369in}{1.928914in}}%
\pgfpathlineto{\pgfqpoint{3.743329in}{1.931051in}}%
\pgfpathlineto{\pgfqpoint{3.744867in}{1.939024in}}%
\pgfpathlineto{\pgfqpoint{3.743906in}{1.927910in}}%
\pgfpathlineto{\pgfqpoint{3.745059in}{1.938524in}}%
\pgfpathlineto{\pgfqpoint{3.745635in}{1.938180in}}%
\pgfpathlineto{\pgfqpoint{3.745443in}{1.938873in}}%
\pgfpathlineto{\pgfqpoint{3.746020in}{1.938901in}}%
\pgfpathlineto{\pgfqpoint{3.746789in}{1.943388in}}%
\pgfpathlineto{\pgfqpoint{3.747557in}{1.948832in}}%
\pgfpathlineto{\pgfqpoint{3.747942in}{1.945169in}}%
\pgfpathlineto{\pgfqpoint{3.748710in}{1.945575in}}%
\pgfpathlineto{\pgfqpoint{3.748518in}{1.944387in}}%
\pgfpathlineto{\pgfqpoint{3.749095in}{1.945337in}}%
\pgfpathlineto{\pgfqpoint{3.749287in}{1.945273in}}%
\pgfpathlineto{\pgfqpoint{3.750248in}{1.940719in}}%
\pgfpathlineto{\pgfqpoint{3.750440in}{1.942505in}}%
\pgfpathlineto{\pgfqpoint{3.750632in}{1.944574in}}%
\pgfpathlineto{\pgfqpoint{3.751401in}{1.942123in}}%
\pgfpathlineto{\pgfqpoint{3.751593in}{1.941603in}}%
\pgfpathlineto{\pgfqpoint{3.751785in}{1.943785in}}%
\pgfpathlineto{\pgfqpoint{3.752361in}{1.949559in}}%
\pgfpathlineto{\pgfqpoint{3.752938in}{1.947538in}}%
\pgfpathlineto{\pgfqpoint{3.756013in}{1.962374in}}%
\pgfpathlineto{\pgfqpoint{3.756205in}{1.959856in}}%
\pgfpathlineto{\pgfqpoint{3.756781in}{1.964094in}}%
\pgfpathlineto{\pgfqpoint{3.756973in}{1.964037in}}%
\pgfpathlineto{\pgfqpoint{3.758511in}{1.969910in}}%
\pgfpathlineto{\pgfqpoint{3.760240in}{1.959352in}}%
\pgfpathlineto{\pgfqpoint{3.760432in}{1.961482in}}%
\pgfpathlineto{\pgfqpoint{3.761009in}{1.956665in}}%
\pgfpathlineto{\pgfqpoint{3.761393in}{1.951622in}}%
\pgfpathlineto{\pgfqpoint{3.762162in}{1.955022in}}%
\pgfpathlineto{\pgfqpoint{3.763123in}{1.954244in}}%
\pgfpathlineto{\pgfqpoint{3.763507in}{1.960132in}}%
\pgfpathlineto{\pgfqpoint{3.764276in}{1.957674in}}%
\pgfpathlineto{\pgfqpoint{3.764660in}{1.958640in}}%
\pgfpathlineto{\pgfqpoint{3.766197in}{1.953228in}}%
\pgfpathlineto{\pgfqpoint{3.766390in}{1.954765in}}%
\pgfpathlineto{\pgfqpoint{3.768119in}{1.961893in}}%
\pgfpathlineto{\pgfqpoint{3.768696in}{1.960732in}}%
\pgfpathlineto{\pgfqpoint{3.769080in}{1.961836in}}%
\pgfpathlineto{\pgfqpoint{3.769656in}{1.956805in}}%
\pgfpathlineto{\pgfqpoint{3.770041in}{1.961982in}}%
\pgfpathlineto{\pgfqpoint{3.770425in}{1.963330in}}%
\pgfpathlineto{\pgfqpoint{3.771002in}{1.960900in}}%
\pgfpathlineto{\pgfqpoint{3.771770in}{1.956187in}}%
\pgfpathlineto{\pgfqpoint{3.772155in}{1.958570in}}%
\pgfpathlineto{\pgfqpoint{3.773116in}{1.961513in}}%
\pgfpathlineto{\pgfqpoint{3.773308in}{1.959451in}}%
\pgfpathlineto{\pgfqpoint{3.773500in}{1.957567in}}%
\pgfpathlineto{\pgfqpoint{3.773884in}{1.960872in}}%
\pgfpathlineto{\pgfqpoint{3.774269in}{1.959817in}}%
\pgfpathlineto{\pgfqpoint{3.775037in}{1.957726in}}%
\pgfpathlineto{\pgfqpoint{3.775229in}{1.960817in}}%
\pgfpathlineto{\pgfqpoint{3.775998in}{1.957895in}}%
\pgfpathlineto{\pgfqpoint{3.776190in}{1.958095in}}%
\pgfpathlineto{\pgfqpoint{3.776382in}{1.957890in}}%
\pgfpathlineto{\pgfqpoint{3.777151in}{1.964353in}}%
\pgfpathlineto{\pgfqpoint{3.777535in}{1.963390in}}%
\pgfpathlineto{\pgfqpoint{3.778881in}{1.960421in}}%
\pgfpathlineto{\pgfqpoint{3.777920in}{1.964006in}}%
\pgfpathlineto{\pgfqpoint{3.779073in}{1.961812in}}%
\pgfpathlineto{\pgfqpoint{3.779457in}{1.962464in}}%
\pgfpathlineto{\pgfqpoint{3.780418in}{1.957800in}}%
\pgfpathlineto{\pgfqpoint{3.780994in}{1.952521in}}%
\pgfpathlineto{\pgfqpoint{3.781379in}{1.957469in}}%
\pgfpathlineto{\pgfqpoint{3.783493in}{1.963646in}}%
\pgfpathlineto{\pgfqpoint{3.784646in}{1.958902in}}%
\pgfpathlineto{\pgfqpoint{3.784838in}{1.959790in}}%
\pgfpathlineto{\pgfqpoint{3.785030in}{1.961949in}}%
\pgfpathlineto{\pgfqpoint{3.785799in}{1.958735in}}%
\pgfpathlineto{\pgfqpoint{3.786375in}{1.960710in}}%
\pgfpathlineto{\pgfqpoint{3.786759in}{1.950411in}}%
\pgfpathlineto{\pgfqpoint{3.787720in}{1.953584in}}%
\pgfpathlineto{\pgfqpoint{3.788297in}{1.957937in}}%
\pgfpathlineto{\pgfqpoint{3.788681in}{1.955595in}}%
\pgfpathlineto{\pgfqpoint{3.790218in}{1.941652in}}%
\pgfpathlineto{\pgfqpoint{3.790795in}{1.944347in}}%
\pgfpathlineto{\pgfqpoint{3.792332in}{1.951806in}}%
\pgfpathlineto{\pgfqpoint{3.793677in}{1.945146in}}%
\pgfpathlineto{\pgfqpoint{3.793870in}{1.945638in}}%
\pgfpathlineto{\pgfqpoint{3.794062in}{1.943298in}}%
\pgfpathlineto{\pgfqpoint{3.794254in}{1.943816in}}%
\pgfpathlineto{\pgfqpoint{3.795599in}{1.937915in}}%
\pgfpathlineto{\pgfqpoint{3.796944in}{1.940978in}}%
\pgfpathlineto{\pgfqpoint{3.796176in}{1.936225in}}%
\pgfpathlineto{\pgfqpoint{3.797137in}{1.940319in}}%
\pgfpathlineto{\pgfqpoint{3.799058in}{1.927675in}}%
\pgfpathlineto{\pgfqpoint{3.799250in}{1.928633in}}%
\pgfpathlineto{\pgfqpoint{3.799635in}{1.931573in}}%
\pgfpathlineto{\pgfqpoint{3.800019in}{1.927945in}}%
\pgfpathlineto{\pgfqpoint{3.801749in}{1.917645in}}%
\pgfpathlineto{\pgfqpoint{3.801941in}{1.920396in}}%
\pgfpathlineto{\pgfqpoint{3.802133in}{1.915267in}}%
\pgfpathlineto{\pgfqpoint{3.802902in}{1.918513in}}%
\pgfpathlineto{\pgfqpoint{3.803094in}{1.919249in}}%
\pgfpathlineto{\pgfqpoint{3.803286in}{1.917266in}}%
\pgfpathlineto{\pgfqpoint{3.803670in}{1.913595in}}%
\pgfpathlineto{\pgfqpoint{3.804247in}{1.917835in}}%
\pgfpathlineto{\pgfqpoint{3.804631in}{1.914845in}}%
\pgfpathlineto{\pgfqpoint{3.805400in}{1.910841in}}%
\pgfpathlineto{\pgfqpoint{3.806361in}{1.911478in}}%
\pgfpathlineto{\pgfqpoint{3.807706in}{1.915008in}}%
\pgfpathlineto{\pgfqpoint{3.808090in}{1.916257in}}%
\pgfpathlineto{\pgfqpoint{3.808282in}{1.913301in}}%
\pgfpathlineto{\pgfqpoint{3.808474in}{1.913728in}}%
\pgfpathlineto{\pgfqpoint{3.809435in}{1.910034in}}%
\pgfpathlineto{\pgfqpoint{3.810204in}{1.912086in}}%
\pgfpathlineto{\pgfqpoint{3.810396in}{1.912772in}}%
\pgfpathlineto{\pgfqpoint{3.811933in}{1.900379in}}%
\pgfpathlineto{\pgfqpoint{3.812126in}{1.900576in}}%
\pgfpathlineto{\pgfqpoint{3.812318in}{1.896996in}}%
\pgfpathlineto{\pgfqpoint{3.813279in}{1.898327in}}%
\pgfpathlineto{\pgfqpoint{3.815200in}{1.902669in}}%
\pgfpathlineto{\pgfqpoint{3.816353in}{1.909645in}}%
\pgfpathlineto{\pgfqpoint{3.816545in}{1.908725in}}%
\pgfpathlineto{\pgfqpoint{3.817891in}{1.901779in}}%
\pgfpathlineto{\pgfqpoint{3.818083in}{1.901887in}}%
\pgfpathlineto{\pgfqpoint{3.818275in}{1.901297in}}%
\pgfpathlineto{\pgfqpoint{3.818659in}{1.898724in}}%
\pgfpathlineto{\pgfqpoint{3.818851in}{1.902776in}}%
\pgfpathlineto{\pgfqpoint{3.819044in}{1.903951in}}%
\pgfpathlineto{\pgfqpoint{3.819620in}{1.901837in}}%
\pgfpathlineto{\pgfqpoint{3.820005in}{1.900471in}}%
\pgfpathlineto{\pgfqpoint{3.820389in}{1.901377in}}%
\pgfpathlineto{\pgfqpoint{3.820965in}{1.904238in}}%
\pgfpathlineto{\pgfqpoint{3.821350in}{1.901486in}}%
\pgfpathlineto{\pgfqpoint{3.822695in}{1.894329in}}%
\pgfpathlineto{\pgfqpoint{3.823079in}{1.895962in}}%
\pgfpathlineto{\pgfqpoint{3.824040in}{1.902952in}}%
\pgfpathlineto{\pgfqpoint{3.823656in}{1.895426in}}%
\pgfpathlineto{\pgfqpoint{3.824617in}{1.902081in}}%
\pgfpathlineto{\pgfqpoint{3.825193in}{1.903013in}}%
\pgfpathlineto{\pgfqpoint{3.825385in}{1.902762in}}%
\pgfpathlineto{\pgfqpoint{3.826346in}{1.898857in}}%
\pgfpathlineto{\pgfqpoint{3.826538in}{1.900826in}}%
\pgfpathlineto{\pgfqpoint{3.826730in}{1.901273in}}%
\pgfpathlineto{\pgfqpoint{3.827691in}{1.899490in}}%
\pgfpathlineto{\pgfqpoint{3.828076in}{1.906118in}}%
\pgfpathlineto{\pgfqpoint{3.829036in}{1.910488in}}%
\pgfpathlineto{\pgfqpoint{3.829421in}{1.909707in}}%
\pgfpathlineto{\pgfqpoint{3.829805in}{1.911131in}}%
\pgfpathlineto{\pgfqpoint{3.829997in}{1.908788in}}%
\pgfpathlineto{\pgfqpoint{3.830189in}{1.909175in}}%
\pgfpathlineto{\pgfqpoint{3.830382in}{1.907089in}}%
\pgfpathlineto{\pgfqpoint{3.830958in}{1.912124in}}%
\pgfpathlineto{\pgfqpoint{3.831727in}{1.918079in}}%
\pgfpathlineto{\pgfqpoint{3.832111in}{1.916642in}}%
\pgfpathlineto{\pgfqpoint{3.834225in}{1.901596in}}%
\pgfpathlineto{\pgfqpoint{3.834609in}{1.900627in}}%
\pgfpathlineto{\pgfqpoint{3.835570in}{1.905271in}}%
\pgfpathlineto{\pgfqpoint{3.835762in}{1.900530in}}%
\pgfpathlineto{\pgfqpoint{3.836531in}{1.905944in}}%
\pgfpathlineto{\pgfqpoint{3.838068in}{1.903418in}}%
\pgfpathlineto{\pgfqpoint{3.838645in}{1.909795in}}%
\pgfpathlineto{\pgfqpoint{3.839413in}{1.907483in}}%
\pgfpathlineto{\pgfqpoint{3.839606in}{1.907695in}}%
\pgfpathlineto{\pgfqpoint{3.841527in}{1.918016in}}%
\pgfpathlineto{\pgfqpoint{3.842488in}{1.912695in}}%
\pgfpathlineto{\pgfqpoint{3.842872in}{1.914752in}}%
\pgfpathlineto{\pgfqpoint{3.843257in}{1.914141in}}%
\pgfpathlineto{\pgfqpoint{3.845179in}{1.903292in}}%
\pgfpathlineto{\pgfqpoint{3.846524in}{1.909957in}}%
\pgfpathlineto{\pgfqpoint{3.846908in}{1.908077in}}%
\pgfpathlineto{\pgfqpoint{3.847292in}{1.911436in}}%
\pgfpathlineto{\pgfqpoint{3.848830in}{1.917774in}}%
\pgfpathlineto{\pgfqpoint{3.849214in}{1.917046in}}%
\pgfpathlineto{\pgfqpoint{3.850559in}{1.907978in}}%
\pgfpathlineto{\pgfqpoint{3.850751in}{1.910467in}}%
\pgfpathlineto{\pgfqpoint{3.851712in}{1.912529in}}%
\pgfpathlineto{\pgfqpoint{3.851904in}{1.909915in}}%
\pgfpathlineto{\pgfqpoint{3.852865in}{1.911075in}}%
\pgfpathlineto{\pgfqpoint{3.853057in}{1.910941in}}%
\pgfpathlineto{\pgfqpoint{3.853250in}{1.909061in}}%
\pgfpathlineto{\pgfqpoint{3.853826in}{1.914052in}}%
\pgfpathlineto{\pgfqpoint{3.854210in}{1.917602in}}%
\pgfpathlineto{\pgfqpoint{3.855363in}{1.915679in}}%
\pgfpathlineto{\pgfqpoint{3.855556in}{1.915558in}}%
\pgfpathlineto{\pgfqpoint{3.855748in}{1.917138in}}%
\pgfpathlineto{\pgfqpoint{3.856132in}{1.912497in}}%
\pgfpathlineto{\pgfqpoint{3.856324in}{1.913498in}}%
\pgfpathlineto{\pgfqpoint{3.857093in}{1.910336in}}%
\pgfpathlineto{\pgfqpoint{3.857477in}{1.913187in}}%
\pgfpathlineto{\pgfqpoint{3.859975in}{1.926682in}}%
\pgfpathlineto{\pgfqpoint{3.860168in}{1.922350in}}%
\pgfpathlineto{\pgfqpoint{3.861128in}{1.925301in}}%
\pgfpathlineto{\pgfqpoint{3.861321in}{1.921461in}}%
\pgfpathlineto{\pgfqpoint{3.861513in}{1.925915in}}%
\pgfpathlineto{\pgfqpoint{3.862089in}{1.923142in}}%
\pgfpathlineto{\pgfqpoint{3.863242in}{1.932657in}}%
\pgfpathlineto{\pgfqpoint{3.863627in}{1.928442in}}%
\pgfpathlineto{\pgfqpoint{3.864395in}{1.925962in}}%
\pgfpathlineto{\pgfqpoint{3.864011in}{1.931428in}}%
\pgfpathlineto{\pgfqpoint{3.864587in}{1.927969in}}%
\pgfpathlineto{\pgfqpoint{3.865356in}{1.932360in}}%
\pgfpathlineto{\pgfqpoint{3.865740in}{1.929461in}}%
\pgfpathlineto{\pgfqpoint{3.867662in}{1.922405in}}%
\pgfpathlineto{\pgfqpoint{3.867854in}{1.923457in}}%
\pgfpathlineto{\pgfqpoint{3.868046in}{1.920589in}}%
\pgfpathlineto{\pgfqpoint{3.868623in}{1.918812in}}%
\pgfpathlineto{\pgfqpoint{3.868815in}{1.920851in}}%
\pgfpathlineto{\pgfqpoint{3.869007in}{1.920235in}}%
\pgfpathlineto{\pgfqpoint{3.870929in}{1.933413in}}%
\pgfpathlineto{\pgfqpoint{3.872659in}{1.923640in}}%
\pgfpathlineto{\pgfqpoint{3.872851in}{1.923822in}}%
\pgfpathlineto{\pgfqpoint{3.873043in}{1.922488in}}%
\pgfpathlineto{\pgfqpoint{3.873235in}{1.922424in}}%
\pgfpathlineto{\pgfqpoint{3.873427in}{1.923050in}}%
\pgfpathlineto{\pgfqpoint{3.873812in}{1.928374in}}%
\pgfpathlineto{\pgfqpoint{3.874388in}{1.925604in}}%
\pgfpathlineto{\pgfqpoint{3.874772in}{1.921904in}}%
\pgfpathlineto{\pgfqpoint{3.875541in}{1.923321in}}%
\pgfpathlineto{\pgfqpoint{3.875733in}{1.925260in}}%
\pgfpathlineto{\pgfqpoint{3.876118in}{1.919440in}}%
\pgfpathlineto{\pgfqpoint{3.876310in}{1.920720in}}%
\pgfpathlineto{\pgfqpoint{3.877271in}{1.913552in}}%
\pgfpathlineto{\pgfqpoint{3.877655in}{1.914176in}}%
\pgfpathlineto{\pgfqpoint{3.879192in}{1.923627in}}%
\pgfpathlineto{\pgfqpoint{3.879384in}{1.923887in}}%
\pgfpathlineto{\pgfqpoint{3.879577in}{1.922328in}}%
\pgfpathlineto{\pgfqpoint{3.879769in}{1.922718in}}%
\pgfpathlineto{\pgfqpoint{3.881306in}{1.915173in}}%
\pgfpathlineto{\pgfqpoint{3.881498in}{1.914954in}}%
\pgfpathlineto{\pgfqpoint{3.881690in}{1.916009in}}%
\pgfpathlineto{\pgfqpoint{3.882075in}{1.917970in}}%
\pgfpathlineto{\pgfqpoint{3.882267in}{1.916053in}}%
\pgfpathlineto{\pgfqpoint{3.882651in}{1.911168in}}%
\pgfpathlineto{\pgfqpoint{3.883420in}{1.913999in}}%
\pgfpathlineto{\pgfqpoint{3.884573in}{1.918360in}}%
\pgfpathlineto{\pgfqpoint{3.884765in}{1.916279in}}%
\pgfpathlineto{\pgfqpoint{3.884957in}{1.915100in}}%
\pgfpathlineto{\pgfqpoint{3.885342in}{1.917856in}}%
\pgfpathlineto{\pgfqpoint{3.885726in}{1.916859in}}%
\pgfpathlineto{\pgfqpoint{3.885918in}{1.916429in}}%
\pgfpathlineto{\pgfqpoint{3.887455in}{1.926664in}}%
\pgfpathlineto{\pgfqpoint{3.887648in}{1.926348in}}%
\pgfpathlineto{\pgfqpoint{3.888608in}{1.930730in}}%
\pgfpathlineto{\pgfqpoint{3.888416in}{1.926080in}}%
\pgfpathlineto{\pgfqpoint{3.889377in}{1.929132in}}%
\pgfpathlineto{\pgfqpoint{3.890146in}{1.930018in}}%
\pgfpathlineto{\pgfqpoint{3.890914in}{1.936710in}}%
\pgfpathlineto{\pgfqpoint{3.891491in}{1.935027in}}%
\pgfpathlineto{\pgfqpoint{3.891875in}{1.933713in}}%
\pgfpathlineto{\pgfqpoint{3.892067in}{1.935892in}}%
\pgfpathlineto{\pgfqpoint{3.892452in}{1.934147in}}%
\pgfpathlineto{\pgfqpoint{3.895334in}{1.950348in}}%
\pgfpathlineto{\pgfqpoint{3.896680in}{1.941925in}}%
\pgfpathlineto{\pgfqpoint{3.895719in}{1.950843in}}%
\pgfpathlineto{\pgfqpoint{3.897448in}{1.944791in}}%
\pgfpathlineto{\pgfqpoint{3.897640in}{1.944905in}}%
\pgfpathlineto{\pgfqpoint{3.899370in}{1.954321in}}%
\pgfpathlineto{\pgfqpoint{3.899562in}{1.952191in}}%
\pgfpathlineto{\pgfqpoint{3.900523in}{1.947003in}}%
\pgfpathlineto{\pgfqpoint{3.900715in}{1.949888in}}%
\pgfpathlineto{\pgfqpoint{3.902252in}{1.962020in}}%
\pgfpathlineto{\pgfqpoint{3.902637in}{1.959613in}}%
\pgfpathlineto{\pgfqpoint{3.904558in}{1.947812in}}%
\pgfpathlineto{\pgfqpoint{3.903213in}{1.960539in}}%
\pgfpathlineto{\pgfqpoint{3.905327in}{1.951515in}}%
\pgfpathlineto{\pgfqpoint{3.905519in}{1.952233in}}%
\pgfpathlineto{\pgfqpoint{3.905711in}{1.949159in}}%
\pgfpathlineto{\pgfqpoint{3.906672in}{1.944649in}}%
\pgfpathlineto{\pgfqpoint{3.907249in}{1.945402in}}%
\pgfpathlineto{\pgfqpoint{3.908402in}{1.951541in}}%
\pgfpathlineto{\pgfqpoint{3.907633in}{1.944521in}}%
\pgfpathlineto{\pgfqpoint{3.908786in}{1.947988in}}%
\pgfpathlineto{\pgfqpoint{3.909170in}{1.945092in}}%
\pgfpathlineto{\pgfqpoint{3.910323in}{1.936401in}}%
\pgfpathlineto{\pgfqpoint{3.910516in}{1.940025in}}%
\pgfpathlineto{\pgfqpoint{3.911669in}{1.950032in}}%
\pgfpathlineto{\pgfqpoint{3.912245in}{1.945925in}}%
\pgfpathlineto{\pgfqpoint{3.912822in}{1.941299in}}%
\pgfpathlineto{\pgfqpoint{3.913206in}{1.945746in}}%
\pgfpathlineto{\pgfqpoint{3.913398in}{1.947689in}}%
\pgfpathlineto{\pgfqpoint{3.913975in}{1.942738in}}%
\pgfpathlineto{\pgfqpoint{3.914935in}{1.931898in}}%
\pgfpathlineto{\pgfqpoint{3.915512in}{1.935044in}}%
\pgfpathlineto{\pgfqpoint{3.916473in}{1.925352in}}%
\pgfpathlineto{\pgfqpoint{3.917049in}{1.928148in}}%
\pgfpathlineto{\pgfqpoint{3.917626in}{1.938723in}}%
\pgfpathlineto{\pgfqpoint{3.918202in}{1.933817in}}%
\pgfpathlineto{\pgfqpoint{3.918779in}{1.931218in}}%
\pgfpathlineto{\pgfqpoint{3.919163in}{1.931877in}}%
\pgfpathlineto{\pgfqpoint{3.920893in}{1.944062in}}%
\pgfpathlineto{\pgfqpoint{3.923391in}{1.933656in}}%
\pgfpathlineto{\pgfqpoint{3.924352in}{1.935846in}}%
\pgfpathlineto{\pgfqpoint{3.924928in}{1.937910in}}%
\pgfpathlineto{\pgfqpoint{3.925313in}{1.937512in}}%
\pgfpathlineto{\pgfqpoint{3.926658in}{1.929377in}}%
\pgfpathlineto{\pgfqpoint{3.928003in}{1.935334in}}%
\pgfpathlineto{\pgfqpoint{3.928387in}{1.934490in}}%
\pgfpathlineto{\pgfqpoint{3.928579in}{1.931957in}}%
\pgfpathlineto{\pgfqpoint{3.929156in}{1.937519in}}%
\pgfpathlineto{\pgfqpoint{3.929348in}{1.938496in}}%
\pgfpathlineto{\pgfqpoint{3.929540in}{1.937134in}}%
\pgfpathlineto{\pgfqpoint{3.929732in}{1.937419in}}%
\pgfpathlineto{\pgfqpoint{3.930309in}{1.929229in}}%
\pgfpathlineto{\pgfqpoint{3.931078in}{1.932540in}}%
\pgfpathlineto{\pgfqpoint{3.931846in}{1.934386in}}%
\pgfpathlineto{\pgfqpoint{3.932038in}{1.933726in}}%
\pgfpathlineto{\pgfqpoint{3.933191in}{1.925823in}}%
\pgfpathlineto{\pgfqpoint{3.933576in}{1.927347in}}%
\pgfpathlineto{\pgfqpoint{3.934537in}{1.933265in}}%
\pgfpathlineto{\pgfqpoint{3.933960in}{1.927030in}}%
\pgfpathlineto{\pgfqpoint{3.935305in}{1.931900in}}%
\pgfpathlineto{\pgfqpoint{3.936266in}{1.929188in}}%
\pgfpathlineto{\pgfqpoint{3.936650in}{1.929557in}}%
\pgfpathlineto{\pgfqpoint{3.936843in}{1.931428in}}%
\pgfpathlineto{\pgfqpoint{3.937227in}{1.926964in}}%
\pgfpathlineto{\pgfqpoint{3.937419in}{1.925653in}}%
\pgfpathlineto{\pgfqpoint{3.937996in}{1.928927in}}%
\pgfpathlineto{\pgfqpoint{3.938188in}{1.932456in}}%
\pgfpathlineto{\pgfqpoint{3.939149in}{1.930852in}}%
\pgfpathlineto{\pgfqpoint{3.940109in}{1.930489in}}%
\pgfpathlineto{\pgfqpoint{3.940494in}{1.933881in}}%
\pgfpathlineto{\pgfqpoint{3.940878in}{1.933441in}}%
\pgfpathlineto{\pgfqpoint{3.942223in}{1.929497in}}%
\pgfpathlineto{\pgfqpoint{3.941262in}{1.935427in}}%
\pgfpathlineto{\pgfqpoint{3.942416in}{1.930407in}}%
\pgfpathlineto{\pgfqpoint{3.942608in}{1.935451in}}%
\pgfpathlineto{\pgfqpoint{3.943569in}{1.933090in}}%
\pgfpathlineto{\pgfqpoint{3.943761in}{1.935549in}}%
\pgfpathlineto{\pgfqpoint{3.944145in}{1.930238in}}%
\pgfpathlineto{\pgfqpoint{3.944337in}{1.927294in}}%
\pgfpathlineto{\pgfqpoint{3.944914in}{1.933198in}}%
\pgfpathlineto{\pgfqpoint{3.945106in}{1.932840in}}%
\pgfpathlineto{\pgfqpoint{3.945490in}{1.935161in}}%
\pgfpathlineto{\pgfqpoint{3.946067in}{1.934172in}}%
\pgfpathlineto{\pgfqpoint{3.947220in}{1.929986in}}%
\pgfpathlineto{\pgfqpoint{3.948949in}{1.944057in}}%
\pgfpathlineto{\pgfqpoint{3.949141in}{1.945171in}}%
\pgfpathlineto{\pgfqpoint{3.949718in}{1.943106in}}%
\pgfpathlineto{\pgfqpoint{3.949910in}{1.942789in}}%
\pgfpathlineto{\pgfqpoint{3.950102in}{1.942994in}}%
\pgfpathlineto{\pgfqpoint{3.951640in}{1.953760in}}%
\pgfpathlineto{\pgfqpoint{3.951832in}{1.953078in}}%
\pgfpathlineto{\pgfqpoint{3.953369in}{1.948382in}}%
\pgfpathlineto{\pgfqpoint{3.953561in}{1.948782in}}%
\pgfpathlineto{\pgfqpoint{3.953753in}{1.946016in}}%
\pgfpathlineto{\pgfqpoint{3.954138in}{1.949910in}}%
\pgfpathlineto{\pgfqpoint{3.954330in}{1.949105in}}%
\pgfpathlineto{\pgfqpoint{3.955867in}{1.963146in}}%
\pgfpathlineto{\pgfqpoint{3.956444in}{1.963258in}}%
\pgfpathlineto{\pgfqpoint{3.956828in}{1.960164in}}%
\pgfpathlineto{\pgfqpoint{3.957981in}{1.968856in}}%
\pgfpathlineto{\pgfqpoint{3.958365in}{1.968376in}}%
\pgfpathlineto{\pgfqpoint{3.959903in}{1.974717in}}%
\pgfpathlineto{\pgfqpoint{3.958942in}{1.967597in}}%
\pgfpathlineto{\pgfqpoint{3.960287in}{1.971844in}}%
\pgfpathlineto{\pgfqpoint{3.960864in}{1.971062in}}%
\pgfpathlineto{\pgfqpoint{3.961056in}{1.972529in}}%
\pgfpathlineto{\pgfqpoint{3.962209in}{1.981697in}}%
\pgfpathlineto{\pgfqpoint{3.962785in}{1.979896in}}%
\pgfpathlineto{\pgfqpoint{3.963362in}{1.976030in}}%
\pgfpathlineto{\pgfqpoint{3.964323in}{1.969270in}}%
\pgfpathlineto{\pgfqpoint{3.964515in}{1.971991in}}%
\pgfpathlineto{\pgfqpoint{3.964707in}{1.976003in}}%
\pgfpathlineto{\pgfqpoint{3.965283in}{1.969516in}}%
\pgfpathlineto{\pgfqpoint{3.966244in}{1.961069in}}%
\pgfpathlineto{\pgfqpoint{3.967013in}{1.965640in}}%
\pgfpathlineto{\pgfqpoint{3.967205in}{1.966422in}}%
\pgfpathlineto{\pgfqpoint{3.967590in}{1.964434in}}%
\pgfpathlineto{\pgfqpoint{3.967782in}{1.965464in}}%
\pgfpathlineto{\pgfqpoint{3.969127in}{1.959244in}}%
\pgfpathlineto{\pgfqpoint{3.968358in}{1.966231in}}%
\pgfpathlineto{\pgfqpoint{3.969319in}{1.961313in}}%
\pgfpathlineto{\pgfqpoint{3.969511in}{1.964708in}}%
\pgfpathlineto{\pgfqpoint{3.970088in}{1.958835in}}%
\pgfpathlineto{\pgfqpoint{3.970280in}{1.956866in}}%
\pgfpathlineto{\pgfqpoint{3.970664in}{1.963827in}}%
\pgfpathlineto{\pgfqpoint{3.972778in}{1.973640in}}%
\pgfpathlineto{\pgfqpoint{3.972970in}{1.970577in}}%
\pgfpathlineto{\pgfqpoint{3.974508in}{1.961539in}}%
\pgfpathlineto{\pgfqpoint{3.975276in}{1.961463in}}%
\pgfpathlineto{\pgfqpoint{3.975661in}{1.959484in}}%
\pgfpathlineto{\pgfqpoint{3.976237in}{1.960708in}}%
\pgfpathlineto{\pgfqpoint{3.980657in}{1.976442in}}%
\pgfpathlineto{\pgfqpoint{3.981233in}{1.976887in}}%
\pgfpathlineto{\pgfqpoint{3.982002in}{1.972077in}}%
\pgfpathlineto{\pgfqpoint{3.983155in}{1.977947in}}%
\pgfpathlineto{\pgfqpoint{3.983347in}{1.977537in}}%
\pgfpathlineto{\pgfqpoint{3.985461in}{1.959306in}}%
\pgfpathlineto{\pgfqpoint{3.985653in}{1.962038in}}%
\pgfpathlineto{\pgfqpoint{3.986038in}{1.956668in}}%
\pgfpathlineto{\pgfqpoint{3.986422in}{1.957551in}}%
\pgfpathlineto{\pgfqpoint{3.988344in}{1.962505in}}%
\pgfpathlineto{\pgfqpoint{3.988920in}{1.958118in}}%
\pgfpathlineto{\pgfqpoint{3.989304in}{1.959620in}}%
\pgfpathlineto{\pgfqpoint{3.991034in}{1.967938in}}%
\pgfpathlineto{\pgfqpoint{3.992956in}{1.960601in}}%
\pgfpathlineto{\pgfqpoint{3.993148in}{1.960884in}}%
\pgfpathlineto{\pgfqpoint{3.993724in}{1.954983in}}%
\pgfpathlineto{\pgfqpoint{3.994301in}{1.958890in}}%
\pgfpathlineto{\pgfqpoint{3.995454in}{1.970917in}}%
\pgfpathlineto{\pgfqpoint{3.996030in}{1.966641in}}%
\pgfpathlineto{\pgfqpoint{3.996223in}{1.966240in}}%
\pgfpathlineto{\pgfqpoint{3.998336in}{1.987843in}}%
\pgfpathlineto{\pgfqpoint{3.998529in}{1.988105in}}%
\pgfpathlineto{\pgfqpoint{3.999489in}{1.985382in}}%
\pgfpathlineto{\pgfqpoint{3.999682in}{1.986889in}}%
\pgfpathlineto{\pgfqpoint{4.000066in}{1.986226in}}%
\pgfpathlineto{\pgfqpoint{4.001411in}{1.995689in}}%
\pgfpathlineto{\pgfqpoint{4.001795in}{1.994933in}}%
\pgfpathlineto{\pgfqpoint{4.001988in}{1.993723in}}%
\pgfpathlineto{\pgfqpoint{4.002372in}{1.998341in}}%
\pgfpathlineto{\pgfqpoint{4.002756in}{2.000940in}}%
\pgfpathlineto{\pgfqpoint{4.003141in}{1.997443in}}%
\pgfpathlineto{\pgfqpoint{4.004101in}{1.993575in}}%
\pgfpathlineto{\pgfqpoint{4.003525in}{1.997902in}}%
\pgfpathlineto{\pgfqpoint{4.004678in}{1.996356in}}%
\pgfpathlineto{\pgfqpoint{4.004870in}{1.997402in}}%
\pgfpathlineto{\pgfqpoint{4.005062in}{1.996404in}}%
\pgfpathlineto{\pgfqpoint{4.005254in}{1.993197in}}%
\pgfpathlineto{\pgfqpoint{4.005639in}{1.998536in}}%
\pgfpathlineto{\pgfqpoint{4.006215in}{1.993354in}}%
\pgfpathlineto{\pgfqpoint{4.006792in}{1.992591in}}%
\pgfpathlineto{\pgfqpoint{4.006984in}{1.993964in}}%
\pgfpathlineto{\pgfqpoint{4.007176in}{1.996372in}}%
\pgfpathlineto{\pgfqpoint{4.007560in}{1.989971in}}%
\pgfpathlineto{\pgfqpoint{4.007945in}{1.994093in}}%
\pgfpathlineto{\pgfqpoint{4.008137in}{1.994356in}}%
\pgfpathlineto{\pgfqpoint{4.009674in}{1.985894in}}%
\pgfpathlineto{\pgfqpoint{4.010251in}{1.990215in}}%
\pgfpathlineto{\pgfqpoint{4.010827in}{1.988035in}}%
\pgfpathlineto{\pgfqpoint{4.011019in}{1.984639in}}%
\pgfpathlineto{\pgfqpoint{4.011596in}{1.989813in}}%
\pgfpathlineto{\pgfqpoint{4.011788in}{1.987873in}}%
\pgfpathlineto{\pgfqpoint{4.011980in}{1.987493in}}%
\pgfpathlineto{\pgfqpoint{4.012941in}{1.996133in}}%
\pgfpathlineto{\pgfqpoint{4.013325in}{1.994359in}}%
\pgfpathlineto{\pgfqpoint{4.013710in}{1.990525in}}%
\pgfpathlineto{\pgfqpoint{4.014286in}{1.992443in}}%
\pgfpathlineto{\pgfqpoint{4.017169in}{2.010213in}}%
\pgfpathlineto{\pgfqpoint{4.017361in}{2.009615in}}%
\pgfpathlineto{\pgfqpoint{4.019283in}{1.991919in}}%
\pgfpathlineto{\pgfqpoint{4.019667in}{1.989293in}}%
\pgfpathlineto{\pgfqpoint{4.020628in}{1.990695in}}%
\pgfpathlineto{\pgfqpoint{4.020820in}{1.991001in}}%
\pgfpathlineto{\pgfqpoint{4.021973in}{1.997700in}}%
\pgfpathlineto{\pgfqpoint{4.022165in}{1.997135in}}%
\pgfpathlineto{\pgfqpoint{4.022934in}{1.993876in}}%
\pgfpathlineto{\pgfqpoint{4.023126in}{1.995015in}}%
\pgfpathlineto{\pgfqpoint{4.023510in}{1.989446in}}%
\pgfpathlineto{\pgfqpoint{4.024087in}{1.990369in}}%
\pgfpathlineto{\pgfqpoint{4.025240in}{2.002812in}}%
\pgfpathlineto{\pgfqpoint{4.025624in}{2.000124in}}%
\pgfpathlineto{\pgfqpoint{4.026969in}{1.995431in}}%
\pgfpathlineto{\pgfqpoint{4.028122in}{2.002568in}}%
\pgfpathlineto{\pgfqpoint{4.028315in}{1.999418in}}%
\pgfpathlineto{\pgfqpoint{4.028507in}{1.997257in}}%
\pgfpathlineto{\pgfqpoint{4.028891in}{2.001924in}}%
\pgfpathlineto{\pgfqpoint{4.029083in}{2.000631in}}%
\pgfpathlineto{\pgfqpoint{4.031197in}{2.012346in}}%
\pgfpathlineto{\pgfqpoint{4.031581in}{2.006853in}}%
\pgfpathlineto{\pgfqpoint{4.032542in}{2.009279in}}%
\pgfpathlineto{\pgfqpoint{4.033503in}{2.016212in}}%
\pgfpathlineto{\pgfqpoint{4.034272in}{2.022779in}}%
\pgfpathlineto{\pgfqpoint{4.034656in}{2.018520in}}%
\pgfpathlineto{\pgfqpoint{4.035040in}{2.020608in}}%
\pgfpathlineto{\pgfqpoint{4.035233in}{2.018250in}}%
\pgfpathlineto{\pgfqpoint{4.035809in}{2.015225in}}%
\pgfpathlineto{\pgfqpoint{4.036193in}{2.018887in}}%
\pgfpathlineto{\pgfqpoint{4.038884in}{2.032260in}}%
\pgfpathlineto{\pgfqpoint{4.036770in}{2.017071in}}%
\pgfpathlineto{\pgfqpoint{4.039268in}{2.031740in}}%
\pgfpathlineto{\pgfqpoint{4.040229in}{2.023027in}}%
\pgfpathlineto{\pgfqpoint{4.040806in}{2.024757in}}%
\pgfpathlineto{\pgfqpoint{4.040998in}{2.025113in}}%
\pgfpathlineto{\pgfqpoint{4.041382in}{2.023878in}}%
\pgfpathlineto{\pgfqpoint{4.041574in}{2.021122in}}%
\pgfpathlineto{\pgfqpoint{4.042151in}{2.028586in}}%
\pgfpathlineto{\pgfqpoint{4.043688in}{2.040890in}}%
\pgfpathlineto{\pgfqpoint{4.044265in}{2.040332in}}%
\pgfpathlineto{\pgfqpoint{4.044457in}{2.040481in}}%
\pgfpathlineto{\pgfqpoint{4.045610in}{2.033150in}}%
\pgfpathlineto{\pgfqpoint{4.046186in}{2.033203in}}%
\pgfpathlineto{\pgfqpoint{4.048300in}{2.023544in}}%
\pgfpathlineto{\pgfqpoint{4.048684in}{2.025365in}}%
\pgfpathlineto{\pgfqpoint{4.049069in}{2.023378in}}%
\pgfpathlineto{\pgfqpoint{4.049261in}{2.024034in}}%
\pgfpathlineto{\pgfqpoint{4.049453in}{2.021589in}}%
\pgfpathlineto{\pgfqpoint{4.050030in}{2.028140in}}%
\pgfpathlineto{\pgfqpoint{4.051567in}{2.036294in}}%
\pgfpathlineto{\pgfqpoint{4.051759in}{2.034936in}}%
\pgfpathlineto{\pgfqpoint{4.052336in}{2.038327in}}%
\pgfpathlineto{\pgfqpoint{4.052528in}{2.037730in}}%
\pgfpathlineto{\pgfqpoint{4.053489in}{2.044450in}}%
\pgfpathlineto{\pgfqpoint{4.053873in}{2.043071in}}%
\pgfpathlineto{\pgfqpoint{4.054065in}{2.040020in}}%
\pgfpathlineto{\pgfqpoint{4.054642in}{2.045893in}}%
\pgfpathlineto{\pgfqpoint{4.054834in}{2.045821in}}%
\pgfpathlineto{\pgfqpoint{4.055026in}{2.046535in}}%
\pgfpathlineto{\pgfqpoint{4.055410in}{2.050337in}}%
\pgfpathlineto{\pgfqpoint{4.055987in}{2.045746in}}%
\pgfpathlineto{\pgfqpoint{4.056755in}{2.042872in}}%
\pgfpathlineto{\pgfqpoint{4.056948in}{2.038677in}}%
\pgfpathlineto{\pgfqpoint{4.057524in}{2.045052in}}%
\pgfpathlineto{\pgfqpoint{4.057716in}{2.047190in}}%
\pgfpathlineto{\pgfqpoint{4.058101in}{2.040694in}}%
\pgfpathlineto{\pgfqpoint{4.059061in}{2.037189in}}%
\pgfpathlineto{\pgfqpoint{4.059254in}{2.039437in}}%
\pgfpathlineto{\pgfqpoint{4.060407in}{2.049537in}}%
\pgfpathlineto{\pgfqpoint{4.061367in}{2.045554in}}%
\pgfpathlineto{\pgfqpoint{4.061752in}{2.046590in}}%
\pgfpathlineto{\pgfqpoint{4.061944in}{2.045399in}}%
\pgfpathlineto{\pgfqpoint{4.064250in}{2.032988in}}%
\pgfpathlineto{\pgfqpoint{4.064634in}{2.035926in}}%
\pgfpathlineto{\pgfqpoint{4.065403in}{2.034085in}}%
\pgfpathlineto{\pgfqpoint{4.066364in}{2.029842in}}%
\pgfpathlineto{\pgfqpoint{4.066748in}{2.031325in}}%
\pgfpathlineto{\pgfqpoint{4.067901in}{2.025300in}}%
\pgfpathlineto{\pgfqpoint{4.068093in}{2.027888in}}%
\pgfpathlineto{\pgfqpoint{4.068478in}{2.029325in}}%
\pgfpathlineto{\pgfqpoint{4.068670in}{2.028227in}}%
\pgfpathlineto{\pgfqpoint{4.070592in}{2.006444in}}%
\pgfpathlineto{\pgfqpoint{4.071360in}{2.002008in}}%
\pgfpathlineto{\pgfqpoint{4.071745in}{2.003349in}}%
\pgfpathlineto{\pgfqpoint{4.073474in}{2.016357in}}%
\pgfpathlineto{\pgfqpoint{4.073858in}{2.014712in}}%
\pgfpathlineto{\pgfqpoint{4.074435in}{2.016101in}}%
\pgfpathlineto{\pgfqpoint{4.075011in}{2.019106in}}%
\pgfpathlineto{\pgfqpoint{4.076549in}{2.037513in}}%
\pgfpathlineto{\pgfqpoint{4.077317in}{2.030750in}}%
\pgfpathlineto{\pgfqpoint{4.077894in}{2.033478in}}%
\pgfpathlineto{\pgfqpoint{4.078663in}{2.039594in}}%
\pgfpathlineto{\pgfqpoint{4.079047in}{2.038329in}}%
\pgfpathlineto{\pgfqpoint{4.080008in}{2.030581in}}%
\pgfpathlineto{\pgfqpoint{4.080392in}{2.032548in}}%
\pgfpathlineto{\pgfqpoint{4.081161in}{2.031740in}}%
\pgfpathlineto{\pgfqpoint{4.081545in}{2.034145in}}%
\pgfpathlineto{\pgfqpoint{4.081929in}{2.032609in}}%
\pgfpathlineto{\pgfqpoint{4.082314in}{2.035075in}}%
\pgfpathlineto{\pgfqpoint{4.083851in}{2.047206in}}%
\pgfpathlineto{\pgfqpoint{4.084428in}{2.053462in}}%
\pgfpathlineto{\pgfqpoint{4.085004in}{2.050346in}}%
\pgfpathlineto{\pgfqpoint{4.085196in}{2.050440in}}%
\pgfpathlineto{\pgfqpoint{4.086349in}{2.054199in}}%
\pgfpathlineto{\pgfqpoint{4.088079in}{2.035325in}}%
\pgfpathlineto{\pgfqpoint{4.088655in}{2.036286in}}%
\pgfpathlineto{\pgfqpoint{4.088848in}{2.034653in}}%
\pgfpathlineto{\pgfqpoint{4.089232in}{2.031429in}}%
\pgfpathlineto{\pgfqpoint{4.089808in}{2.035209in}}%
\pgfpathlineto{\pgfqpoint{4.091154in}{2.038598in}}%
\pgfpathlineto{\pgfqpoint{4.090385in}{2.032271in}}%
\pgfpathlineto{\pgfqpoint{4.091346in}{2.038564in}}%
\pgfpathlineto{\pgfqpoint{4.091922in}{2.043079in}}%
\pgfpathlineto{\pgfqpoint{4.092114in}{2.043979in}}%
\pgfpathlineto{\pgfqpoint{4.092499in}{2.041744in}}%
\pgfpathlineto{\pgfqpoint{4.092691in}{2.041781in}}%
\pgfpathlineto{\pgfqpoint{4.093267in}{2.038843in}}%
\pgfpathlineto{\pgfqpoint{4.093652in}{2.041248in}}%
\pgfpathlineto{\pgfqpoint{4.094613in}{2.047159in}}%
\pgfpathlineto{\pgfqpoint{4.095573in}{2.044656in}}%
\pgfpathlineto{\pgfqpoint{4.096150in}{2.040472in}}%
\pgfpathlineto{\pgfqpoint{4.096726in}{2.044273in}}%
\pgfpathlineto{\pgfqpoint{4.097111in}{2.042794in}}%
\pgfpathlineto{\pgfqpoint{4.098264in}{2.039532in}}%
\pgfpathlineto{\pgfqpoint{4.098456in}{2.040186in}}%
\pgfpathlineto{\pgfqpoint{4.098648in}{2.040575in}}%
\pgfpathlineto{\pgfqpoint{4.098840in}{2.039553in}}%
\pgfpathlineto{\pgfqpoint{4.100762in}{2.023770in}}%
\pgfpathlineto{\pgfqpoint{4.101146in}{2.027063in}}%
\pgfpathlineto{\pgfqpoint{4.102491in}{2.031165in}}%
\pgfpathlineto{\pgfqpoint{4.102684in}{2.030067in}}%
\pgfpathlineto{\pgfqpoint{4.103452in}{2.031660in}}%
\pgfpathlineto{\pgfqpoint{4.103837in}{2.028120in}}%
\pgfpathlineto{\pgfqpoint{4.105182in}{2.035939in}}%
\pgfpathlineto{\pgfqpoint{4.105374in}{2.035181in}}%
\pgfpathlineto{\pgfqpoint{4.105566in}{2.036270in}}%
\pgfpathlineto{\pgfqpoint{4.106143in}{2.033567in}}%
\pgfpathlineto{\pgfqpoint{4.106335in}{2.032200in}}%
\pgfpathlineto{\pgfqpoint{4.106911in}{2.035503in}}%
\pgfpathlineto{\pgfqpoint{4.107103in}{2.039400in}}%
\pgfpathlineto{\pgfqpoint{4.107488in}{2.034128in}}%
\pgfpathlineto{\pgfqpoint{4.108064in}{2.037888in}}%
\pgfpathlineto{\pgfqpoint{4.108449in}{2.034853in}}%
\pgfpathlineto{\pgfqpoint{4.109217in}{2.040792in}}%
\pgfpathlineto{\pgfqpoint{4.109794in}{2.037081in}}%
\pgfpathlineto{\pgfqpoint{4.110370in}{2.039733in}}%
\pgfpathlineto{\pgfqpoint{4.110947in}{2.038514in}}%
\pgfpathlineto{\pgfqpoint{4.111908in}{2.043186in}}%
\pgfpathlineto{\pgfqpoint{4.113445in}{2.036115in}}%
\pgfpathlineto{\pgfqpoint{4.113637in}{2.036429in}}%
\pgfpathlineto{\pgfqpoint{4.114982in}{2.041772in}}%
\pgfpathlineto{\pgfqpoint{4.117481in}{2.057438in}}%
\pgfpathlineto{\pgfqpoint{4.117865in}{2.056491in}}%
\pgfpathlineto{\pgfqpoint{4.119210in}{2.042278in}}%
\pgfpathlineto{\pgfqpoint{4.119787in}{2.045696in}}%
\pgfpathlineto{\pgfqpoint{4.120940in}{2.055536in}}%
\pgfpathlineto{\pgfqpoint{4.121132in}{2.051965in}}%
\pgfpathlineto{\pgfqpoint{4.121516in}{2.050633in}}%
\pgfpathlineto{\pgfqpoint{4.121900in}{2.051633in}}%
\pgfpathlineto{\pgfqpoint{4.122477in}{2.054250in}}%
\pgfpathlineto{\pgfqpoint{4.122861in}{2.053385in}}%
\pgfpathlineto{\pgfqpoint{4.123053in}{2.051500in}}%
\pgfpathlineto{\pgfqpoint{4.123630in}{2.055859in}}%
\pgfpathlineto{\pgfqpoint{4.123822in}{2.058339in}}%
\pgfpathlineto{\pgfqpoint{4.124783in}{2.057926in}}%
\pgfpathlineto{\pgfqpoint{4.126512in}{2.048082in}}%
\pgfpathlineto{\pgfqpoint{4.126705in}{2.049842in}}%
\pgfpathlineto{\pgfqpoint{4.127089in}{2.047852in}}%
\pgfpathlineto{\pgfqpoint{4.128242in}{2.057242in}}%
\pgfpathlineto{\pgfqpoint{4.129779in}{2.070693in}}%
\pgfpathlineto{\pgfqpoint{4.129971in}{2.070525in}}%
\pgfpathlineto{\pgfqpoint{4.130356in}{2.067217in}}%
\pgfpathlineto{\pgfqpoint{4.130932in}{2.070208in}}%
\pgfpathlineto{\pgfqpoint{4.131124in}{2.070713in}}%
\pgfpathlineto{\pgfqpoint{4.131317in}{2.068366in}}%
\pgfpathlineto{\pgfqpoint{4.131701in}{2.067488in}}%
\pgfpathlineto{\pgfqpoint{4.132470in}{2.062748in}}%
\pgfpathlineto{\pgfqpoint{4.132854in}{2.066758in}}%
\pgfpathlineto{\pgfqpoint{4.133238in}{2.069181in}}%
\pgfpathlineto{\pgfqpoint{4.133815in}{2.065768in}}%
\pgfpathlineto{\pgfqpoint{4.134199in}{2.066651in}}%
\pgfpathlineto{\pgfqpoint{4.134583in}{2.065996in}}%
\pgfpathlineto{\pgfqpoint{4.136121in}{2.059624in}}%
\pgfpathlineto{\pgfqpoint{4.134968in}{2.066159in}}%
\pgfpathlineto{\pgfqpoint{4.136313in}{2.059809in}}%
\pgfpathlineto{\pgfqpoint{4.136505in}{2.063657in}}%
\pgfpathlineto{\pgfqpoint{4.137274in}{2.057220in}}%
\pgfpathlineto{\pgfqpoint{4.138811in}{2.049526in}}%
\pgfpathlineto{\pgfqpoint{4.139196in}{2.051331in}}%
\pgfpathlineto{\pgfqpoint{4.140541in}{2.062607in}}%
\pgfpathlineto{\pgfqpoint{4.142270in}{2.060228in}}%
\pgfpathlineto{\pgfqpoint{4.142462in}{2.057754in}}%
\pgfpathlineto{\pgfqpoint{4.143231in}{2.062097in}}%
\pgfpathlineto{\pgfqpoint{4.144961in}{2.054506in}}%
\pgfpathlineto{\pgfqpoint{4.146498in}{2.060757in}}%
\pgfpathlineto{\pgfqpoint{4.146690in}{2.059606in}}%
\pgfpathlineto{\pgfqpoint{4.146882in}{2.061187in}}%
\pgfpathlineto{\pgfqpoint{4.147267in}{2.061133in}}%
\pgfpathlineto{\pgfqpoint{4.147459in}{2.062008in}}%
\pgfpathlineto{\pgfqpoint{4.148035in}{2.059809in}}%
\pgfpathlineto{\pgfqpoint{4.148420in}{2.061127in}}%
\pgfpathlineto{\pgfqpoint{4.150149in}{2.047412in}}%
\pgfpathlineto{\pgfqpoint{4.150341in}{2.048632in}}%
\pgfpathlineto{\pgfqpoint{4.150533in}{2.049839in}}%
\pgfpathlineto{\pgfqpoint{4.151110in}{2.047442in}}%
\pgfpathlineto{\pgfqpoint{4.151302in}{2.046579in}}%
\pgfpathlineto{\pgfqpoint{4.151686in}{2.047063in}}%
\pgfpathlineto{\pgfqpoint{4.153416in}{2.060177in}}%
\pgfpathlineto{\pgfqpoint{4.154185in}{2.058129in}}%
\pgfpathlineto{\pgfqpoint{4.153992in}{2.060919in}}%
\pgfpathlineto{\pgfqpoint{4.154569in}{2.058342in}}%
\pgfpathlineto{\pgfqpoint{4.156106in}{2.067516in}}%
\pgfpathlineto{\pgfqpoint{4.155145in}{2.057803in}}%
\pgfpathlineto{\pgfqpoint{4.156491in}{2.066249in}}%
\pgfpathlineto{\pgfqpoint{4.156875in}{2.064603in}}%
\pgfpathlineto{\pgfqpoint{4.157451in}{2.066173in}}%
\pgfpathlineto{\pgfqpoint{4.157836in}{2.065418in}}%
\pgfpathlineto{\pgfqpoint{4.158028in}{2.063108in}}%
\pgfpathlineto{\pgfqpoint{4.158604in}{2.066647in}}%
\pgfpathlineto{\pgfqpoint{4.158797in}{2.067288in}}%
\pgfpathlineto{\pgfqpoint{4.158989in}{2.065825in}}%
\pgfpathlineto{\pgfqpoint{4.159757in}{2.054789in}}%
\pgfpathlineto{\pgfqpoint{4.160718in}{2.057313in}}%
\pgfpathlineto{\pgfqpoint{4.160910in}{2.059375in}}%
\pgfpathlineto{\pgfqpoint{4.161487in}{2.055695in}}%
\pgfpathlineto{\pgfqpoint{4.162256in}{2.051166in}}%
\pgfpathlineto{\pgfqpoint{4.162640in}{2.053603in}}%
\pgfpathlineto{\pgfqpoint{4.163024in}{2.054295in}}%
\pgfpathlineto{\pgfqpoint{4.163409in}{2.056383in}}%
\pgfpathlineto{\pgfqpoint{4.163793in}{2.054549in}}%
\pgfpathlineto{\pgfqpoint{4.164562in}{2.051510in}}%
\pgfpathlineto{\pgfqpoint{4.164177in}{2.055017in}}%
\pgfpathlineto{\pgfqpoint{4.164946in}{2.051835in}}%
\pgfpathlineto{\pgfqpoint{4.166099in}{2.062903in}}%
\pgfpathlineto{\pgfqpoint{4.167252in}{2.061323in}}%
\pgfpathlineto{\pgfqpoint{4.168597in}{2.055713in}}%
\pgfpathlineto{\pgfqpoint{4.167829in}{2.061901in}}%
\pgfpathlineto{\pgfqpoint{4.168789in}{2.056440in}}%
\pgfpathlineto{\pgfqpoint{4.168982in}{2.056474in}}%
\pgfpathlineto{\pgfqpoint{4.169942in}{2.061110in}}%
\pgfpathlineto{\pgfqpoint{4.170135in}{2.060251in}}%
\pgfpathlineto{\pgfqpoint{4.170711in}{2.061910in}}%
\pgfpathlineto{\pgfqpoint{4.170903in}{2.061616in}}%
\pgfpathlineto{\pgfqpoint{4.172441in}{2.050196in}}%
\pgfpathlineto{\pgfqpoint{4.174170in}{2.060497in}}%
\pgfpathlineto{\pgfqpoint{4.174747in}{2.055953in}}%
\pgfpathlineto{\pgfqpoint{4.175323in}{2.057295in}}%
\pgfpathlineto{\pgfqpoint{4.175900in}{2.054619in}}%
\pgfpathlineto{\pgfqpoint{4.175707in}{2.057852in}}%
\pgfpathlineto{\pgfqpoint{4.176092in}{2.057332in}}%
\pgfpathlineto{\pgfqpoint{4.176284in}{2.059668in}}%
\pgfpathlineto{\pgfqpoint{4.176860in}{2.055168in}}%
\pgfpathlineto{\pgfqpoint{4.177053in}{2.056089in}}%
\pgfpathlineto{\pgfqpoint{4.177629in}{2.060211in}}%
\pgfpathlineto{\pgfqpoint{4.178206in}{2.062448in}}%
\pgfpathlineto{\pgfqpoint{4.178590in}{2.059499in}}%
\pgfpathlineto{\pgfqpoint{4.178782in}{2.061333in}}%
\pgfpathlineto{\pgfqpoint{4.179551in}{2.056063in}}%
\pgfpathlineto{\pgfqpoint{4.179935in}{2.049670in}}%
\pgfpathlineto{\pgfqpoint{4.180512in}{2.053455in}}%
\pgfpathlineto{\pgfqpoint{4.180896in}{2.052567in}}%
\pgfpathlineto{\pgfqpoint{4.181472in}{2.055248in}}%
\pgfpathlineto{\pgfqpoint{4.181665in}{2.053527in}}%
\pgfpathlineto{\pgfqpoint{4.182241in}{2.057190in}}%
\pgfpathlineto{\pgfqpoint{4.182818in}{2.055955in}}%
\pgfpathlineto{\pgfqpoint{4.183010in}{2.051187in}}%
\pgfpathlineto{\pgfqpoint{4.183971in}{2.054666in}}%
\pgfpathlineto{\pgfqpoint{4.184163in}{2.054068in}}%
\pgfpathlineto{\pgfqpoint{4.184355in}{2.055167in}}%
\pgfpathlineto{\pgfqpoint{4.185508in}{2.062381in}}%
\pgfpathlineto{\pgfqpoint{4.184739in}{2.054526in}}%
\pgfpathlineto{\pgfqpoint{4.185700in}{2.062198in}}%
\pgfpathlineto{\pgfqpoint{4.186469in}{2.063860in}}%
\pgfpathlineto{\pgfqpoint{4.187045in}{2.057112in}}%
\pgfpathlineto{\pgfqpoint{4.187814in}{2.062804in}}%
\pgfpathlineto{\pgfqpoint{4.188391in}{2.060602in}}%
\pgfpathlineto{\pgfqpoint{4.189159in}{2.056965in}}%
\pgfpathlineto{\pgfqpoint{4.189351in}{2.057925in}}%
\pgfpathlineto{\pgfqpoint{4.189736in}{2.061570in}}%
\pgfpathlineto{\pgfqpoint{4.190312in}{2.055524in}}%
\pgfpathlineto{\pgfqpoint{4.192042in}{2.068936in}}%
\pgfpathlineto{\pgfqpoint{4.192234in}{2.068149in}}%
\pgfpathlineto{\pgfqpoint{4.193003in}{2.062745in}}%
\pgfpathlineto{\pgfqpoint{4.193387in}{2.066267in}}%
\pgfpathlineto{\pgfqpoint{4.194156in}{2.072204in}}%
\pgfpathlineto{\pgfqpoint{4.194732in}{2.069590in}}%
\pgfpathlineto{\pgfqpoint{4.195501in}{2.064245in}}%
\pgfpathlineto{\pgfqpoint{4.195693in}{2.069392in}}%
\pgfpathlineto{\pgfqpoint{4.196846in}{2.077203in}}%
\pgfpathlineto{\pgfqpoint{4.197230in}{2.075059in}}%
\pgfpathlineto{\pgfqpoint{4.198191in}{2.073184in}}%
\pgfpathlineto{\pgfqpoint{4.197615in}{2.076029in}}%
\pgfpathlineto{\pgfqpoint{4.198575in}{2.074482in}}%
\pgfpathlineto{\pgfqpoint{4.198768in}{2.075543in}}%
\pgfpathlineto{\pgfqpoint{4.199152in}{2.072130in}}%
\pgfpathlineto{\pgfqpoint{4.199536in}{2.073034in}}%
\pgfpathlineto{\pgfqpoint{4.199728in}{2.071717in}}%
\pgfpathlineto{\pgfqpoint{4.199921in}{2.073397in}}%
\pgfpathlineto{\pgfqpoint{4.200497in}{2.069062in}}%
\pgfpathlineto{\pgfqpoint{4.200881in}{2.066372in}}%
\pgfpathlineto{\pgfqpoint{4.201650in}{2.061825in}}%
\pgfpathlineto{\pgfqpoint{4.201842in}{2.065150in}}%
\pgfpathlineto{\pgfqpoint{4.202227in}{2.073163in}}%
\pgfpathlineto{\pgfqpoint{4.202995in}{2.067634in}}%
\pgfpathlineto{\pgfqpoint{4.203572in}{2.057510in}}%
\pgfpathlineto{\pgfqpoint{4.204148in}{2.062220in}}%
\pgfpathlineto{\pgfqpoint{4.204917in}{2.058595in}}%
\pgfpathlineto{\pgfqpoint{4.205109in}{2.059970in}}%
\pgfpathlineto{\pgfqpoint{4.206262in}{2.064611in}}%
\pgfpathlineto{\pgfqpoint{4.206454in}{2.064292in}}%
\pgfpathlineto{\pgfqpoint{4.206646in}{2.064025in}}%
\pgfpathlineto{\pgfqpoint{4.206839in}{2.064967in}}%
\pgfpathlineto{\pgfqpoint{4.207992in}{2.069373in}}%
\pgfpathlineto{\pgfqpoint{4.208376in}{2.066911in}}%
\pgfpathlineto{\pgfqpoint{4.208760in}{2.073180in}}%
\pgfpathlineto{\pgfqpoint{4.209913in}{2.082099in}}%
\pgfpathlineto{\pgfqpoint{4.210298in}{2.078627in}}%
\pgfpathlineto{\pgfqpoint{4.211643in}{2.072259in}}%
\pgfpathlineto{\pgfqpoint{4.211835in}{2.072286in}}%
\pgfpathlineto{\pgfqpoint{4.212412in}{2.072370in}}%
\pgfpathlineto{\pgfqpoint{4.212219in}{2.071374in}}%
\pgfpathlineto{\pgfqpoint{4.212604in}{2.071635in}}%
\pgfpathlineto{\pgfqpoint{4.213565in}{2.067126in}}%
\pgfpathlineto{\pgfqpoint{4.213757in}{2.067491in}}%
\pgfpathlineto{\pgfqpoint{4.215102in}{2.075608in}}%
\pgfpathlineto{\pgfqpoint{4.215678in}{2.068918in}}%
\pgfpathlineto{\pgfqpoint{4.216447in}{2.071365in}}%
\pgfpathlineto{\pgfqpoint{4.216639in}{2.072578in}}%
\pgfpathlineto{\pgfqpoint{4.216831in}{2.069716in}}%
\pgfpathlineto{\pgfqpoint{4.217216in}{2.070268in}}%
\pgfpathlineto{\pgfqpoint{4.217408in}{2.068163in}}%
\pgfpathlineto{\pgfqpoint{4.217792in}{2.070645in}}%
\pgfpathlineto{\pgfqpoint{4.218369in}{2.069786in}}%
\pgfpathlineto{\pgfqpoint{4.218753in}{2.073381in}}%
\pgfpathlineto{\pgfqpoint{4.219330in}{2.069830in}}%
\pgfpathlineto{\pgfqpoint{4.219522in}{2.065745in}}%
\pgfpathlineto{\pgfqpoint{4.220290in}{2.068500in}}%
\pgfpathlineto{\pgfqpoint{4.222212in}{2.083022in}}%
\pgfpathlineto{\pgfqpoint{4.222404in}{2.082748in}}%
\pgfpathlineto{\pgfqpoint{4.222789in}{2.080676in}}%
\pgfpathlineto{\pgfqpoint{4.222981in}{2.084852in}}%
\pgfpathlineto{\pgfqpoint{4.223942in}{2.080559in}}%
\pgfpathlineto{\pgfqpoint{4.225095in}{2.075912in}}%
\pgfpathlineto{\pgfqpoint{4.225863in}{2.074639in}}%
\pgfpathlineto{\pgfqpoint{4.226440in}{2.081321in}}%
\pgfpathlineto{\pgfqpoint{4.227977in}{2.075840in}}%
\pgfpathlineto{\pgfqpoint{4.228554in}{2.081188in}}%
\pgfpathlineto{\pgfqpoint{4.228938in}{2.079417in}}%
\pgfpathlineto{\pgfqpoint{4.230283in}{2.073788in}}%
\pgfpathlineto{\pgfqpoint{4.231820in}{2.084612in}}%
\pgfpathlineto{\pgfqpoint{4.232781in}{2.080584in}}%
\pgfpathlineto{\pgfqpoint{4.233550in}{2.081130in}}%
\pgfpathlineto{\pgfqpoint{4.235087in}{2.090393in}}%
\pgfpathlineto{\pgfqpoint{4.235472in}{2.089537in}}%
\pgfpathlineto{\pgfqpoint{4.236048in}{2.078495in}}%
\pgfpathlineto{\pgfqpoint{4.237586in}{2.082583in}}%
\pgfpathlineto{\pgfqpoint{4.238546in}{2.079323in}}%
\pgfpathlineto{\pgfqpoint{4.238354in}{2.083209in}}%
\pgfpathlineto{\pgfqpoint{4.238931in}{2.081650in}}%
\pgfpathlineto{\pgfqpoint{4.239123in}{2.083625in}}%
\pgfpathlineto{\pgfqpoint{4.239507in}{2.080755in}}%
\pgfpathlineto{\pgfqpoint{4.239699in}{2.077282in}}%
\pgfpathlineto{\pgfqpoint{4.240084in}{2.085580in}}%
\pgfpathlineto{\pgfqpoint{4.240276in}{2.084698in}}%
\pgfpathlineto{\pgfqpoint{4.241429in}{2.095084in}}%
\pgfpathlineto{\pgfqpoint{4.241813in}{2.090856in}}%
\pgfpathlineto{\pgfqpoint{4.242198in}{2.097375in}}%
\pgfpathlineto{\pgfqpoint{4.242774in}{2.090547in}}%
\pgfpathlineto{\pgfqpoint{4.242966in}{2.090107in}}%
\pgfpathlineto{\pgfqpoint{4.243158in}{2.090440in}}%
\pgfpathlineto{\pgfqpoint{4.244696in}{2.096955in}}%
\pgfpathlineto{\pgfqpoint{4.245272in}{2.097992in}}%
\pgfpathlineto{\pgfqpoint{4.246041in}{2.095483in}}%
\pgfpathlineto{\pgfqpoint{4.246810in}{2.101380in}}%
\pgfpathlineto{\pgfqpoint{4.247578in}{2.099952in}}%
\pgfpathlineto{\pgfqpoint{4.248155in}{2.101511in}}%
\pgfpathlineto{\pgfqpoint{4.248923in}{2.102984in}}%
\pgfpathlineto{\pgfqpoint{4.249308in}{2.102376in}}%
\pgfpathlineto{\pgfqpoint{4.250653in}{2.095905in}}%
\pgfpathlineto{\pgfqpoint{4.252382in}{2.103962in}}%
\pgfpathlineto{\pgfqpoint{4.252767in}{2.099965in}}%
\pgfpathlineto{\pgfqpoint{4.253535in}{2.101851in}}%
\pgfpathlineto{\pgfqpoint{4.253728in}{2.099885in}}%
\pgfpathlineto{\pgfqpoint{4.254112in}{2.103166in}}%
\pgfpathlineto{\pgfqpoint{4.254496in}{2.102921in}}%
\pgfpathlineto{\pgfqpoint{4.255841in}{2.107732in}}%
\pgfpathlineto{\pgfqpoint{4.256034in}{2.107217in}}%
\pgfpathlineto{\pgfqpoint{4.256418in}{2.108853in}}%
\pgfpathlineto{\pgfqpoint{4.256610in}{2.111321in}}%
\pgfpathlineto{\pgfqpoint{4.257379in}{2.108523in}}%
\pgfpathlineto{\pgfqpoint{4.257955in}{2.107331in}}%
\pgfpathlineto{\pgfqpoint{4.258340in}{2.108345in}}%
\pgfpathlineto{\pgfqpoint{4.259493in}{2.113747in}}%
\pgfpathlineto{\pgfqpoint{4.259685in}{2.110990in}}%
\pgfpathlineto{\pgfqpoint{4.260069in}{2.109736in}}%
\pgfpathlineto{\pgfqpoint{4.260261in}{2.110311in}}%
\pgfpathlineto{\pgfqpoint{4.261414in}{2.114411in}}%
\pgfpathlineto{\pgfqpoint{4.261799in}{2.115441in}}%
\pgfpathlineto{\pgfqpoint{4.262375in}{2.113901in}}%
\pgfpathlineto{\pgfqpoint{4.263720in}{2.104487in}}%
\pgfpathlineto{\pgfqpoint{4.263913in}{2.105021in}}%
\pgfpathlineto{\pgfqpoint{4.264105in}{2.105990in}}%
\pgfpathlineto{\pgfqpoint{4.264297in}{2.105290in}}%
\pgfpathlineto{\pgfqpoint{4.264681in}{2.099442in}}%
\pgfpathlineto{\pgfqpoint{4.265450in}{2.100865in}}%
\pgfpathlineto{\pgfqpoint{4.266026in}{2.102934in}}%
\pgfpathlineto{\pgfqpoint{4.266219in}{2.100860in}}%
\pgfpathlineto{\pgfqpoint{4.266411in}{2.099534in}}%
\pgfpathlineto{\pgfqpoint{4.266795in}{2.102462in}}%
\pgfpathlineto{\pgfqpoint{4.266987in}{2.101757in}}%
\pgfpathlineto{\pgfqpoint{4.268909in}{2.109778in}}%
\pgfpathlineto{\pgfqpoint{4.269101in}{2.108043in}}%
\pgfpathlineto{\pgfqpoint{4.269293in}{2.104893in}}%
\pgfpathlineto{\pgfqpoint{4.270062in}{2.109795in}}%
\pgfpathlineto{\pgfqpoint{4.270254in}{2.107086in}}%
\pgfpathlineto{\pgfqpoint{4.270831in}{2.112264in}}%
\pgfpathlineto{\pgfqpoint{4.271791in}{2.109670in}}%
\pgfpathlineto{\pgfqpoint{4.272176in}{2.109363in}}%
\pgfpathlineto{\pgfqpoint{4.272560in}{2.115667in}}%
\pgfpathlineto{\pgfqpoint{4.273521in}{2.114420in}}%
\pgfpathlineto{\pgfqpoint{4.273713in}{2.114771in}}%
\pgfpathlineto{\pgfqpoint{4.273905in}{2.112810in}}%
\pgfpathlineto{\pgfqpoint{4.274097in}{2.112608in}}%
\pgfpathlineto{\pgfqpoint{4.274290in}{2.113861in}}%
\pgfpathlineto{\pgfqpoint{4.276211in}{2.127595in}}%
\pgfpathlineto{\pgfqpoint{4.276403in}{2.126923in}}%
\pgfpathlineto{\pgfqpoint{4.276596in}{2.127738in}}%
\pgfpathlineto{\pgfqpoint{4.276788in}{2.127786in}}%
\pgfpathlineto{\pgfqpoint{4.277172in}{2.130538in}}%
\pgfpathlineto{\pgfqpoint{4.277749in}{2.128790in}}%
\pgfpathlineto{\pgfqpoint{4.278325in}{2.129373in}}%
\pgfpathlineto{\pgfqpoint{4.279094in}{2.123544in}}%
\pgfpathlineto{\pgfqpoint{4.279286in}{2.124488in}}%
\pgfpathlineto{\pgfqpoint{4.279670in}{2.121533in}}%
\pgfpathlineto{\pgfqpoint{4.279862in}{2.122168in}}%
\pgfpathlineto{\pgfqpoint{4.280823in}{2.115513in}}%
\pgfpathlineto{\pgfqpoint{4.281015in}{2.118458in}}%
\pgfpathlineto{\pgfqpoint{4.281400in}{2.122811in}}%
\pgfpathlineto{\pgfqpoint{4.282168in}{2.120083in}}%
\pgfpathlineto{\pgfqpoint{4.282361in}{2.121940in}}%
\pgfpathlineto{\pgfqpoint{4.282553in}{2.119008in}}%
\pgfpathlineto{\pgfqpoint{4.282937in}{2.119210in}}%
\pgfpathlineto{\pgfqpoint{4.283129in}{2.115814in}}%
\pgfpathlineto{\pgfqpoint{4.283706in}{2.119331in}}%
\pgfpathlineto{\pgfqpoint{4.283898in}{2.117879in}}%
\pgfpathlineto{\pgfqpoint{4.284090in}{2.118788in}}%
\pgfpathlineto{\pgfqpoint{4.284282in}{2.116503in}}%
\pgfpathlineto{\pgfqpoint{4.285243in}{2.110493in}}%
\pgfpathlineto{\pgfqpoint{4.285435in}{2.115014in}}%
\pgfpathlineto{\pgfqpoint{4.285820in}{2.114194in}}%
\pgfpathlineto{\pgfqpoint{4.286012in}{2.115882in}}%
\pgfpathlineto{\pgfqpoint{4.287165in}{2.126529in}}%
\pgfpathlineto{\pgfqpoint{4.287357in}{2.125176in}}%
\pgfpathlineto{\pgfqpoint{4.287741in}{2.125643in}}%
\pgfpathlineto{\pgfqpoint{4.288702in}{2.119922in}}%
\pgfpathlineto{\pgfqpoint{4.290047in}{2.122611in}}%
\pgfpathlineto{\pgfqpoint{4.291200in}{2.116939in}}%
\pgfpathlineto{\pgfqpoint{4.291777in}{2.118146in}}%
\pgfpathlineto{\pgfqpoint{4.294083in}{2.109372in}}%
\pgfpathlineto{\pgfqpoint{4.294275in}{2.110640in}}%
\pgfpathlineto{\pgfqpoint{4.294467in}{2.113455in}}%
\pgfpathlineto{\pgfqpoint{4.295044in}{2.108248in}}%
\pgfpathlineto{\pgfqpoint{4.295236in}{2.110840in}}%
\pgfpathlineto{\pgfqpoint{4.295428in}{2.110814in}}%
\pgfpathlineto{\pgfqpoint{4.296197in}{2.119222in}}%
\pgfpathlineto{\pgfqpoint{4.296773in}{2.119080in}}%
\pgfpathlineto{\pgfqpoint{4.298118in}{2.111699in}}%
\pgfpathlineto{\pgfqpoint{4.298311in}{2.113858in}}%
\pgfpathlineto{\pgfqpoint{4.298887in}{2.108717in}}%
\pgfpathlineto{\pgfqpoint{4.299271in}{2.112335in}}%
\pgfpathlineto{\pgfqpoint{4.299656in}{2.108798in}}%
\pgfpathlineto{\pgfqpoint{4.299848in}{2.108041in}}%
\pgfpathlineto{\pgfqpoint{4.300040in}{2.111513in}}%
\pgfpathlineto{\pgfqpoint{4.300617in}{2.112893in}}%
\pgfpathlineto{\pgfqpoint{4.300809in}{2.110606in}}%
\pgfpathlineto{\pgfqpoint{4.301001in}{2.110008in}}%
\pgfpathlineto{\pgfqpoint{4.301193in}{2.111036in}}%
\pgfpathlineto{\pgfqpoint{4.301577in}{2.113931in}}%
\pgfpathlineto{\pgfqpoint{4.301962in}{2.110204in}}%
\pgfpathlineto{\pgfqpoint{4.302154in}{2.111670in}}%
\pgfpathlineto{\pgfqpoint{4.302346in}{2.110592in}}%
\pgfpathlineto{\pgfqpoint{4.302923in}{2.112783in}}%
\pgfpathlineto{\pgfqpoint{4.303115in}{2.111674in}}%
\pgfpathlineto{\pgfqpoint{4.303307in}{2.111630in}}%
\pgfpathlineto{\pgfqpoint{4.303499in}{2.110042in}}%
\pgfpathlineto{\pgfqpoint{4.304076in}{2.113223in}}%
\pgfpathlineto{\pgfqpoint{4.304460in}{2.116635in}}%
\pgfpathlineto{\pgfqpoint{4.305036in}{2.114533in}}%
\pgfpathlineto{\pgfqpoint{4.305229in}{2.112748in}}%
\pgfpathlineto{\pgfqpoint{4.305805in}{2.117435in}}%
\pgfpathlineto{\pgfqpoint{4.306189in}{2.120004in}}%
\pgfpathlineto{\pgfqpoint{4.306958in}{2.122708in}}%
\pgfpathlineto{\pgfqpoint{4.307150in}{2.122653in}}%
\pgfpathlineto{\pgfqpoint{4.309072in}{2.096820in}}%
\pgfpathlineto{\pgfqpoint{4.310609in}{2.088340in}}%
\pgfpathlineto{\pgfqpoint{4.310802in}{2.089090in}}%
\pgfpathlineto{\pgfqpoint{4.311378in}{2.093588in}}%
\pgfpathlineto{\pgfqpoint{4.311762in}{2.088453in}}%
\pgfpathlineto{\pgfqpoint{4.312915in}{2.092448in}}%
\pgfpathlineto{\pgfqpoint{4.313108in}{2.090712in}}%
\pgfpathlineto{\pgfqpoint{4.314645in}{2.081503in}}%
\pgfpathlineto{\pgfqpoint{4.315029in}{2.083584in}}%
\pgfpathlineto{\pgfqpoint{4.315221in}{2.084071in}}%
\pgfpathlineto{\pgfqpoint{4.315414in}{2.082205in}}%
\pgfpathlineto{\pgfqpoint{4.315606in}{2.083793in}}%
\pgfpathlineto{\pgfqpoint{4.315798in}{2.082359in}}%
\pgfpathlineto{\pgfqpoint{4.316374in}{2.084584in}}%
\pgfpathlineto{\pgfqpoint{4.316567in}{2.083052in}}%
\pgfpathlineto{\pgfqpoint{4.316951in}{2.081214in}}%
\pgfpathlineto{\pgfqpoint{4.318488in}{2.087964in}}%
\pgfpathlineto{\pgfqpoint{4.319065in}{2.084494in}}%
\pgfpathlineto{\pgfqpoint{4.319449in}{2.088158in}}%
\pgfpathlineto{\pgfqpoint{4.319641in}{2.084953in}}%
\pgfpathlineto{\pgfqpoint{4.319833in}{2.085681in}}%
\pgfpathlineto{\pgfqpoint{4.320218in}{2.083720in}}%
\pgfpathlineto{\pgfqpoint{4.320602in}{2.085416in}}%
\pgfpathlineto{\pgfqpoint{4.322716in}{2.070828in}}%
\pgfpathlineto{\pgfqpoint{4.322908in}{2.071848in}}%
\pgfpathlineto{\pgfqpoint{4.323100in}{2.067688in}}%
\pgfpathlineto{\pgfqpoint{4.323485in}{2.068570in}}%
\pgfpathlineto{\pgfqpoint{4.324445in}{2.075429in}}%
\pgfpathlineto{\pgfqpoint{4.324830in}{2.073911in}}%
\pgfpathlineto{\pgfqpoint{4.325214in}{2.071454in}}%
\pgfpathlineto{\pgfqpoint{4.325406in}{2.075831in}}%
\pgfpathlineto{\pgfqpoint{4.325983in}{2.073688in}}%
\pgfpathlineto{\pgfqpoint{4.327328in}{2.077278in}}%
\pgfpathlineto{\pgfqpoint{4.327520in}{2.077532in}}%
\pgfpathlineto{\pgfqpoint{4.327712in}{2.076851in}}%
\pgfpathlineto{\pgfqpoint{4.328289in}{2.078820in}}%
\pgfpathlineto{\pgfqpoint{4.329442in}{2.072572in}}%
\pgfpathlineto{\pgfqpoint{4.329634in}{2.074065in}}%
\pgfpathlineto{\pgfqpoint{4.329826in}{2.069802in}}%
\pgfpathlineto{\pgfqpoint{4.330018in}{2.068131in}}%
\pgfpathlineto{\pgfqpoint{4.330787in}{2.069135in}}%
\pgfpathlineto{\pgfqpoint{4.332516in}{2.076928in}}%
\pgfpathlineto{\pgfqpoint{4.334054in}{2.083139in}}%
\pgfpathlineto{\pgfqpoint{4.334438in}{2.078078in}}%
\pgfpathlineto{\pgfqpoint{4.335207in}{2.080264in}}%
\pgfpathlineto{\pgfqpoint{4.336360in}{2.084695in}}%
\pgfpathlineto{\pgfqpoint{4.336744in}{2.084137in}}%
\pgfpathlineto{\pgfqpoint{4.337321in}{2.077405in}}%
\pgfpathlineto{\pgfqpoint{4.337897in}{2.078488in}}%
\pgfpathlineto{\pgfqpoint{4.338089in}{2.081619in}}%
\pgfpathlineto{\pgfqpoint{4.338666in}{2.074345in}}%
\pgfpathlineto{\pgfqpoint{4.338858in}{2.073070in}}%
\pgfpathlineto{\pgfqpoint{4.339627in}{2.074636in}}%
\pgfpathlineto{\pgfqpoint{4.340780in}{2.081440in}}%
\pgfpathlineto{\pgfqpoint{4.340972in}{2.078450in}}%
\pgfpathlineto{\pgfqpoint{4.341164in}{2.076747in}}%
\pgfpathlineto{\pgfqpoint{4.341356in}{2.079439in}}%
\pgfpathlineto{\pgfqpoint{4.341741in}{2.078122in}}%
\pgfpathlineto{\pgfqpoint{4.342317in}{2.084092in}}%
\pgfpathlineto{\pgfqpoint{4.342894in}{2.080063in}}%
\pgfpathlineto{\pgfqpoint{4.343278in}{2.078403in}}%
\pgfpathlineto{\pgfqpoint{4.343662in}{2.078891in}}%
\pgfpathlineto{\pgfqpoint{4.345392in}{2.089139in}}%
\pgfpathlineto{\pgfqpoint{4.345584in}{2.085904in}}%
\pgfpathlineto{\pgfqpoint{4.346160in}{2.091304in}}%
\pgfpathlineto{\pgfqpoint{4.346353in}{2.090360in}}%
\pgfpathlineto{\pgfqpoint{4.346929in}{2.096728in}}%
\pgfpathlineto{\pgfqpoint{4.347506in}{2.095036in}}%
\pgfpathlineto{\pgfqpoint{4.347890in}{2.091153in}}%
\pgfpathlineto{\pgfqpoint{4.348274in}{2.095501in}}%
\pgfpathlineto{\pgfqpoint{4.348466in}{2.094739in}}%
\pgfpathlineto{\pgfqpoint{4.349812in}{2.099851in}}%
\pgfpathlineto{\pgfqpoint{4.350196in}{2.099747in}}%
\pgfpathlineto{\pgfqpoint{4.350965in}{2.103243in}}%
\pgfpathlineto{\pgfqpoint{4.351157in}{2.099891in}}%
\pgfpathlineto{\pgfqpoint{4.352502in}{2.094085in}}%
\pgfpathlineto{\pgfqpoint{4.353271in}{2.093079in}}%
\pgfpathlineto{\pgfqpoint{4.354424in}{2.100881in}}%
\pgfpathlineto{\pgfqpoint{4.355192in}{2.093990in}}%
\pgfpathlineto{\pgfqpoint{4.355961in}{2.094114in}}%
\pgfpathlineto{\pgfqpoint{4.357114in}{2.100247in}}%
\pgfpathlineto{\pgfqpoint{4.357306in}{2.099002in}}%
\pgfpathlineto{\pgfqpoint{4.357498in}{2.096348in}}%
\pgfpathlineto{\pgfqpoint{4.358267in}{2.099155in}}%
\pgfpathlineto{\pgfqpoint{4.358459in}{2.097835in}}%
\pgfpathlineto{\pgfqpoint{4.358651in}{2.098126in}}%
\pgfpathlineto{\pgfqpoint{4.358844in}{2.097724in}}%
\pgfpathlineto{\pgfqpoint{4.359997in}{2.109012in}}%
\pgfpathlineto{\pgfqpoint{4.360381in}{2.107746in}}%
\pgfpathlineto{\pgfqpoint{4.360573in}{2.108521in}}%
\pgfpathlineto{\pgfqpoint{4.360957in}{2.107736in}}%
\pgfpathlineto{\pgfqpoint{4.362110in}{2.102286in}}%
\pgfpathlineto{\pgfqpoint{4.362303in}{2.102379in}}%
\pgfpathlineto{\pgfqpoint{4.362495in}{2.105303in}}%
\pgfpathlineto{\pgfqpoint{4.363071in}{2.099861in}}%
\pgfpathlineto{\pgfqpoint{4.363456in}{2.095675in}}%
\pgfpathlineto{\pgfqpoint{4.364032in}{2.098952in}}%
\pgfpathlineto{\pgfqpoint{4.364993in}{2.106720in}}%
\pgfpathlineto{\pgfqpoint{4.365569in}{2.105618in}}%
\pgfpathlineto{\pgfqpoint{4.369413in}{2.087234in}}%
\pgfpathlineto{\pgfqpoint{4.369797in}{2.090930in}}%
\pgfpathlineto{\pgfqpoint{4.370566in}{2.093008in}}%
\pgfpathlineto{\pgfqpoint{4.370758in}{2.090743in}}%
\pgfpathlineto{\pgfqpoint{4.371334in}{2.087877in}}%
\pgfpathlineto{\pgfqpoint{4.371527in}{2.089641in}}%
\pgfpathlineto{\pgfqpoint{4.372487in}{2.097175in}}%
\pgfpathlineto{\pgfqpoint{4.372872in}{2.094575in}}%
\pgfpathlineto{\pgfqpoint{4.373256in}{2.097603in}}%
\pgfpathlineto{\pgfqpoint{4.373640in}{2.093005in}}%
\pgfpathlineto{\pgfqpoint{4.374217in}{2.089663in}}%
\pgfpathlineto{\pgfqpoint{4.374601in}{2.092842in}}%
\pgfpathlineto{\pgfqpoint{4.374793in}{2.094562in}}%
\pgfpathlineto{\pgfqpoint{4.375370in}{2.091518in}}%
\pgfpathlineto{\pgfqpoint{4.375562in}{2.091913in}}%
\pgfpathlineto{\pgfqpoint{4.375754in}{2.091619in}}%
\pgfpathlineto{\pgfqpoint{4.377676in}{2.104093in}}%
\pgfpathlineto{\pgfqpoint{4.378252in}{2.098744in}}%
\pgfpathlineto{\pgfqpoint{4.378829in}{2.100979in}}%
\pgfpathlineto{\pgfqpoint{4.379021in}{2.099898in}}%
\pgfpathlineto{\pgfqpoint{4.379405in}{2.102634in}}%
\pgfpathlineto{\pgfqpoint{4.379598in}{2.102106in}}%
\pgfpathlineto{\pgfqpoint{4.379982in}{2.104484in}}%
\pgfpathlineto{\pgfqpoint{4.380174in}{2.107426in}}%
\pgfpathlineto{\pgfqpoint{4.380751in}{2.101794in}}%
\pgfpathlineto{\pgfqpoint{4.380943in}{2.102033in}}%
\pgfpathlineto{\pgfqpoint{4.381135in}{2.101462in}}%
\pgfpathlineto{\pgfqpoint{4.381327in}{2.098762in}}%
\pgfpathlineto{\pgfqpoint{4.382096in}{2.102726in}}%
\pgfpathlineto{\pgfqpoint{4.384594in}{2.113895in}}%
\pgfpathlineto{\pgfqpoint{4.382480in}{2.102037in}}%
\pgfpathlineto{\pgfqpoint{4.384978in}{2.110263in}}%
\pgfpathlineto{\pgfqpoint{4.385171in}{2.108420in}}%
\pgfpathlineto{\pgfqpoint{4.385555in}{2.114950in}}%
\pgfpathlineto{\pgfqpoint{4.385747in}{2.112218in}}%
\pgfpathlineto{\pgfqpoint{4.385939in}{2.113336in}}%
\pgfpathlineto{\pgfqpoint{4.386131in}{2.109562in}}%
\pgfpathlineto{\pgfqpoint{4.386516in}{2.107377in}}%
\pgfpathlineto{\pgfqpoint{4.386900in}{2.108888in}}%
\pgfpathlineto{\pgfqpoint{4.387477in}{2.117074in}}%
\pgfpathlineto{\pgfqpoint{4.388245in}{2.114266in}}%
\pgfpathlineto{\pgfqpoint{4.388822in}{2.117814in}}%
\pgfpathlineto{\pgfqpoint{4.389590in}{2.116682in}}%
\pgfpathlineto{\pgfqpoint{4.390359in}{2.108973in}}%
\pgfpathlineto{\pgfqpoint{4.391896in}{2.097308in}}%
\pgfpathlineto{\pgfqpoint{4.392665in}{2.098990in}}%
\pgfpathlineto{\pgfqpoint{4.392281in}{2.096843in}}%
\pgfpathlineto{\pgfqpoint{4.392857in}{2.097423in}}%
\pgfpathlineto{\pgfqpoint{4.393049in}{2.097293in}}%
\pgfpathlineto{\pgfqpoint{4.393242in}{2.094954in}}%
\pgfpathlineto{\pgfqpoint{4.393434in}{2.097810in}}%
\pgfpathlineto{\pgfqpoint{4.394010in}{2.096669in}}%
\pgfpathlineto{\pgfqpoint{4.395163in}{2.101657in}}%
\pgfpathlineto{\pgfqpoint{4.394395in}{2.094908in}}%
\pgfpathlineto{\pgfqpoint{4.395355in}{2.100145in}}%
\pgfpathlineto{\pgfqpoint{4.395740in}{2.096857in}}%
\pgfpathlineto{\pgfqpoint{4.396893in}{2.097064in}}%
\pgfpathlineto{\pgfqpoint{4.397085in}{2.098508in}}%
\pgfpathlineto{\pgfqpoint{4.397469in}{2.096450in}}%
\pgfpathlineto{\pgfqpoint{4.397661in}{2.098152in}}%
\pgfpathlineto{\pgfqpoint{4.398814in}{2.093037in}}%
\pgfpathlineto{\pgfqpoint{4.399007in}{2.094317in}}%
\pgfpathlineto{\pgfqpoint{4.399583in}{2.092218in}}%
\pgfpathlineto{\pgfqpoint{4.399391in}{2.094767in}}%
\pgfpathlineto{\pgfqpoint{4.400160in}{2.092998in}}%
\pgfpathlineto{\pgfqpoint{4.400352in}{2.092928in}}%
\pgfpathlineto{\pgfqpoint{4.401313in}{2.097936in}}%
\pgfpathlineto{\pgfqpoint{4.401505in}{2.096018in}}%
\pgfpathlineto{\pgfqpoint{4.401697in}{2.093995in}}%
\pgfpathlineto{\pgfqpoint{4.402081in}{2.097922in}}%
\pgfpathlineto{\pgfqpoint{4.403619in}{2.106671in}}%
\pgfpathlineto{\pgfqpoint{4.405156in}{2.111824in}}%
\pgfpathlineto{\pgfqpoint{4.406501in}{2.116074in}}%
\pgfpathlineto{\pgfqpoint{4.407654in}{2.111565in}}%
\pgfpathlineto{\pgfqpoint{4.409384in}{2.095853in}}%
\pgfpathlineto{\pgfqpoint{4.409576in}{2.097984in}}%
\pgfpathlineto{\pgfqpoint{4.410345in}{2.099097in}}%
\pgfpathlineto{\pgfqpoint{4.410921in}{2.093416in}}%
\pgfpathlineto{\pgfqpoint{4.413227in}{2.106331in}}%
\pgfpathlineto{\pgfqpoint{4.411305in}{2.093294in}}%
\pgfpathlineto{\pgfqpoint{4.413419in}{2.105074in}}%
\pgfpathlineto{\pgfqpoint{4.415149in}{2.094463in}}%
\pgfpathlineto{\pgfqpoint{4.415533in}{2.098111in}}%
\pgfpathlineto{\pgfqpoint{4.415725in}{2.097347in}}%
\pgfpathlineto{\pgfqpoint{4.415917in}{2.099740in}}%
\pgfpathlineto{\pgfqpoint{4.418223in}{2.111039in}}%
\pgfpathlineto{\pgfqpoint{4.418992in}{2.105026in}}%
\pgfpathlineto{\pgfqpoint{4.419569in}{2.108243in}}%
\pgfpathlineto{\pgfqpoint{4.419953in}{2.107046in}}%
\pgfpathlineto{\pgfqpoint{4.421298in}{2.115676in}}%
\pgfpathlineto{\pgfqpoint{4.421682in}{2.119327in}}%
\pgfpathlineto{\pgfqpoint{4.422259in}{2.115586in}}%
\pgfpathlineto{\pgfqpoint{4.422643in}{2.118670in}}%
\pgfpathlineto{\pgfqpoint{4.423028in}{2.115584in}}%
\pgfpathlineto{\pgfqpoint{4.423604in}{2.118004in}}%
\pgfpathlineto{\pgfqpoint{4.423988in}{2.118877in}}%
\pgfpathlineto{\pgfqpoint{4.424181in}{2.116498in}}%
\pgfpathlineto{\pgfqpoint{4.424757in}{2.118267in}}%
\pgfpathlineto{\pgfqpoint{4.424949in}{2.117133in}}%
\pgfpathlineto{\pgfqpoint{4.425141in}{2.119817in}}%
\pgfpathlineto{\pgfqpoint{4.425334in}{2.118590in}}%
\pgfpathlineto{\pgfqpoint{4.425526in}{2.120675in}}%
\pgfpathlineto{\pgfqpoint{4.426294in}{2.118929in}}%
\pgfpathlineto{\pgfqpoint{4.427063in}{2.116118in}}%
\pgfpathlineto{\pgfqpoint{4.427255in}{2.118331in}}%
\pgfpathlineto{\pgfqpoint{4.428408in}{2.124625in}}%
\pgfpathlineto{\pgfqpoint{4.428793in}{2.122359in}}%
\pgfpathlineto{\pgfqpoint{4.430138in}{2.113369in}}%
\pgfpathlineto{\pgfqpoint{4.430522in}{2.118206in}}%
\pgfpathlineto{\pgfqpoint{4.432828in}{2.131263in}}%
\pgfpathlineto{\pgfqpoint{4.433020in}{2.130625in}}%
\pgfpathlineto{\pgfqpoint{4.433405in}{2.134577in}}%
\pgfpathlineto{\pgfqpoint{4.433789in}{2.129694in}}%
\pgfpathlineto{\pgfqpoint{4.436095in}{2.109123in}}%
\pgfpathlineto{\pgfqpoint{4.436864in}{2.112801in}}%
\pgfpathlineto{\pgfqpoint{4.437248in}{2.111049in}}%
\pgfpathlineto{\pgfqpoint{4.438593in}{2.107446in}}%
\pgfpathlineto{\pgfqpoint{4.438785in}{2.109816in}}%
\pgfpathlineto{\pgfqpoint{4.439554in}{2.106939in}}%
\pgfpathlineto{\pgfqpoint{4.440707in}{2.099238in}}%
\pgfpathlineto{\pgfqpoint{4.441284in}{2.100513in}}%
\pgfpathlineto{\pgfqpoint{4.441476in}{2.104822in}}%
\pgfpathlineto{\pgfqpoint{4.442052in}{2.099547in}}%
\pgfpathlineto{\pgfqpoint{4.442437in}{2.104441in}}%
\pgfpathlineto{\pgfqpoint{4.443397in}{2.091523in}}%
\pgfpathlineto{\pgfqpoint{4.443974in}{2.095230in}}%
\pgfpathlineto{\pgfqpoint{4.444166in}{2.095695in}}%
\pgfpathlineto{\pgfqpoint{4.444358in}{2.094756in}}%
\pgfpathlineto{\pgfqpoint{4.444743in}{2.087476in}}%
\pgfpathlineto{\pgfqpoint{4.445511in}{2.089282in}}%
\pgfpathlineto{\pgfqpoint{4.448586in}{2.101638in}}%
\pgfpathlineto{\pgfqpoint{4.448778in}{2.099668in}}%
\pgfpathlineto{\pgfqpoint{4.449355in}{2.103358in}}%
\pgfpathlineto{\pgfqpoint{4.449547in}{2.104116in}}%
\pgfpathlineto{\pgfqpoint{4.449739in}{2.100925in}}%
\pgfpathlineto{\pgfqpoint{4.450315in}{2.103591in}}%
\pgfpathlineto{\pgfqpoint{4.450700in}{2.100609in}}%
\pgfpathlineto{\pgfqpoint{4.451276in}{2.104722in}}%
\pgfpathlineto{\pgfqpoint{4.453006in}{2.119706in}}%
\pgfpathlineto{\pgfqpoint{4.454927in}{2.101643in}}%
\pgfpathlineto{\pgfqpoint{4.455504in}{2.103717in}}%
\pgfpathlineto{\pgfqpoint{4.455696in}{2.103544in}}%
\pgfpathlineto{\pgfqpoint{4.456273in}{2.109535in}}%
\pgfpathlineto{\pgfqpoint{4.456657in}{2.105602in}}%
\pgfpathlineto{\pgfqpoint{4.457041in}{2.101305in}}%
\pgfpathlineto{\pgfqpoint{4.457426in}{2.103087in}}%
\pgfpathlineto{\pgfqpoint{4.457618in}{2.108039in}}%
\pgfpathlineto{\pgfqpoint{4.458387in}{2.101506in}}%
\pgfpathlineto{\pgfqpoint{4.460885in}{2.092825in}}%
\pgfpathlineto{\pgfqpoint{4.461077in}{2.095727in}}%
\pgfpathlineto{\pgfqpoint{4.461846in}{2.091983in}}%
\pgfpathlineto{\pgfqpoint{4.462230in}{2.085243in}}%
\pgfpathlineto{\pgfqpoint{4.462999in}{2.086932in}}%
\pgfpathlineto{\pgfqpoint{4.464152in}{2.091633in}}%
\pgfpathlineto{\pgfqpoint{4.463383in}{2.086663in}}%
\pgfpathlineto{\pgfqpoint{4.464344in}{2.091166in}}%
\pgfpathlineto{\pgfqpoint{4.465305in}{2.087679in}}%
\pgfpathlineto{\pgfqpoint{4.465497in}{2.088329in}}%
\pgfpathlineto{\pgfqpoint{4.466842in}{2.098530in}}%
\pgfpathlineto{\pgfqpoint{4.467226in}{2.095776in}}%
\pgfpathlineto{\pgfqpoint{4.467418in}{2.095694in}}%
\pgfpathlineto{\pgfqpoint{4.469724in}{2.107386in}}%
\pgfpathlineto{\pgfqpoint{4.470301in}{2.105273in}}%
\pgfpathlineto{\pgfqpoint{4.470877in}{2.105749in}}%
\pgfpathlineto{\pgfqpoint{4.471070in}{2.106762in}}%
\pgfpathlineto{\pgfqpoint{4.471454in}{2.104161in}}%
\pgfpathlineto{\pgfqpoint{4.471646in}{2.105772in}}%
\pgfpathlineto{\pgfqpoint{4.471838in}{2.102784in}}%
\pgfpathlineto{\pgfqpoint{4.472799in}{2.105005in}}%
\pgfpathlineto{\pgfqpoint{4.473376in}{2.101186in}}%
\pgfpathlineto{\pgfqpoint{4.473952in}{2.102226in}}%
\pgfpathlineto{\pgfqpoint{4.474144in}{2.102750in}}%
\pgfpathlineto{\pgfqpoint{4.474336in}{2.101022in}}%
\pgfpathlineto{\pgfqpoint{4.474529in}{2.099339in}}%
\pgfpathlineto{\pgfqpoint{4.474721in}{2.101379in}}%
\pgfpathlineto{\pgfqpoint{4.475105in}{2.099790in}}%
\pgfpathlineto{\pgfqpoint{4.475682in}{2.107191in}}%
\pgfpathlineto{\pgfqpoint{4.476450in}{2.105130in}}%
\pgfpathlineto{\pgfqpoint{4.476642in}{2.105833in}}%
\pgfpathlineto{\pgfqpoint{4.477027in}{2.103269in}}%
\pgfpathlineto{\pgfqpoint{4.477219in}{2.099626in}}%
\pgfpathlineto{\pgfqpoint{4.477795in}{2.106795in}}%
\pgfpathlineto{\pgfqpoint{4.477988in}{2.107640in}}%
\pgfpathlineto{\pgfqpoint{4.478180in}{2.104877in}}%
\pgfpathlineto{\pgfqpoint{4.478948in}{2.106086in}}%
\pgfpathlineto{\pgfqpoint{4.479909in}{2.094149in}}%
\pgfpathlineto{\pgfqpoint{4.481062in}{2.085259in}}%
\pgfpathlineto{\pgfqpoint{4.481639in}{2.086647in}}%
\pgfpathlineto{\pgfqpoint{4.482023in}{2.083988in}}%
\pgfpathlineto{\pgfqpoint{4.482792in}{2.088040in}}%
\pgfpathlineto{\pgfqpoint{4.483945in}{2.084266in}}%
\pgfpathlineto{\pgfqpoint{4.485482in}{2.092545in}}%
\pgfpathlineto{\pgfqpoint{4.485867in}{2.089215in}}%
\pgfpathlineto{\pgfqpoint{4.486443in}{2.091582in}}%
\pgfpathlineto{\pgfqpoint{4.487788in}{2.096187in}}%
\pgfpathlineto{\pgfqpoint{4.489133in}{2.089172in}}%
\pgfpathlineto{\pgfqpoint{4.489710in}{2.088495in}}%
\pgfpathlineto{\pgfqpoint{4.490286in}{2.091286in}}%
\pgfpathlineto{\pgfqpoint{4.490479in}{2.087960in}}%
\pgfpathlineto{\pgfqpoint{4.491055in}{2.094259in}}%
\pgfpathlineto{\pgfqpoint{4.491247in}{2.095400in}}%
\pgfpathlineto{\pgfqpoint{4.491824in}{2.093396in}}%
\pgfpathlineto{\pgfqpoint{4.492208in}{2.089889in}}%
\pgfpathlineto{\pgfqpoint{4.492977in}{2.090479in}}%
\pgfpathlineto{\pgfqpoint{4.494130in}{2.097222in}}%
\pgfpathlineto{\pgfqpoint{4.494322in}{2.095815in}}%
\pgfpathlineto{\pgfqpoint{4.494706in}{2.093298in}}%
\pgfpathlineto{\pgfqpoint{4.495475in}{2.094155in}}%
\pgfpathlineto{\pgfqpoint{4.496051in}{2.097313in}}%
\pgfpathlineto{\pgfqpoint{4.496436in}{2.094599in}}%
\pgfpathlineto{\pgfqpoint{4.497204in}{2.093724in}}%
\pgfpathlineto{\pgfqpoint{4.496820in}{2.095866in}}%
\pgfpathlineto{\pgfqpoint{4.497397in}{2.095089in}}%
\pgfpathlineto{\pgfqpoint{4.499126in}{2.102357in}}%
\pgfpathlineto{\pgfqpoint{4.499510in}{2.100065in}}%
\pgfpathlineto{\pgfqpoint{4.499703in}{2.100814in}}%
\pgfpathlineto{\pgfqpoint{4.499895in}{2.099642in}}%
\pgfpathlineto{\pgfqpoint{4.500279in}{2.095112in}}%
\pgfpathlineto{\pgfqpoint{4.501048in}{2.097070in}}%
\pgfpathlineto{\pgfqpoint{4.501624in}{2.095600in}}%
\pgfpathlineto{\pgfqpoint{4.502009in}{2.094300in}}%
\pgfpathlineto{\pgfqpoint{4.502201in}{2.097205in}}%
\pgfpathlineto{\pgfqpoint{4.502969in}{2.098521in}}%
\pgfpathlineto{\pgfqpoint{4.503162in}{2.097163in}}%
\pgfpathlineto{\pgfqpoint{4.503738in}{2.095274in}}%
\pgfpathlineto{\pgfqpoint{4.503930in}{2.098108in}}%
\pgfpathlineto{\pgfqpoint{4.504122in}{2.098656in}}%
\pgfpathlineto{\pgfqpoint{4.504315in}{2.097988in}}%
\pgfpathlineto{\pgfqpoint{4.504891in}{2.099905in}}%
\pgfpathlineto{\pgfqpoint{4.505852in}{2.092487in}}%
\pgfpathlineto{\pgfqpoint{4.506044in}{2.093228in}}%
\pgfpathlineto{\pgfqpoint{4.506429in}{2.092684in}}%
\pgfpathlineto{\pgfqpoint{4.507389in}{2.089621in}}%
\pgfpathlineto{\pgfqpoint{4.507197in}{2.093595in}}%
\pgfpathlineto{\pgfqpoint{4.507582in}{2.091068in}}%
\pgfpathlineto{\pgfqpoint{4.508542in}{2.091325in}}%
\pgfpathlineto{\pgfqpoint{4.508735in}{2.092516in}}%
\pgfpathlineto{\pgfqpoint{4.508927in}{2.089331in}}%
\pgfpathlineto{\pgfqpoint{4.509311in}{2.089794in}}%
\pgfpathlineto{\pgfqpoint{4.511041in}{2.074489in}}%
\pgfpathlineto{\pgfqpoint{4.512001in}{2.078631in}}%
\pgfpathlineto{\pgfqpoint{4.512578in}{2.075935in}}%
\pgfpathlineto{\pgfqpoint{4.513731in}{2.079903in}}%
\pgfpathlineto{\pgfqpoint{4.514115in}{2.079400in}}%
\pgfpathlineto{\pgfqpoint{4.514307in}{2.076388in}}%
\pgfpathlineto{\pgfqpoint{4.514884in}{2.083206in}}%
\pgfpathlineto{\pgfqpoint{4.515076in}{2.080425in}}%
\pgfpathlineto{\pgfqpoint{4.515460in}{2.079120in}}%
\pgfpathlineto{\pgfqpoint{4.515845in}{2.080681in}}%
\pgfpathlineto{\pgfqpoint{4.516421in}{2.080100in}}%
\pgfpathlineto{\pgfqpoint{4.516998in}{2.086064in}}%
\pgfpathlineto{\pgfqpoint{4.517190in}{2.083840in}}%
\pgfpathlineto{\pgfqpoint{4.517574in}{2.090196in}}%
\pgfpathlineto{\pgfqpoint{4.517766in}{2.089505in}}%
\pgfpathlineto{\pgfqpoint{4.517959in}{2.089404in}}%
\pgfpathlineto{\pgfqpoint{4.518151in}{2.090472in}}%
\pgfpathlineto{\pgfqpoint{4.518535in}{2.086171in}}%
\pgfpathlineto{\pgfqpoint{4.519112in}{2.092131in}}%
\pgfpathlineto{\pgfqpoint{4.519688in}{2.091320in}}%
\pgfpathlineto{\pgfqpoint{4.520457in}{2.085331in}}%
\pgfpathlineto{\pgfqpoint{4.520841in}{2.088105in}}%
\pgfpathlineto{\pgfqpoint{4.521033in}{2.087876in}}%
\pgfpathlineto{\pgfqpoint{4.521418in}{2.088888in}}%
\pgfpathlineto{\pgfqpoint{4.521610in}{2.088454in}}%
\pgfpathlineto{\pgfqpoint{4.521994in}{2.084730in}}%
\pgfpathlineto{\pgfqpoint{4.522378in}{2.088842in}}%
\pgfpathlineto{\pgfqpoint{4.522571in}{2.087710in}}%
\pgfpathlineto{\pgfqpoint{4.522955in}{2.090794in}}%
\pgfpathlineto{\pgfqpoint{4.523339in}{2.086619in}}%
\pgfpathlineto{\pgfqpoint{4.523531in}{2.086674in}}%
\pgfpathlineto{\pgfqpoint{4.523724in}{2.085874in}}%
\pgfpathlineto{\pgfqpoint{4.524108in}{2.082854in}}%
\pgfpathlineto{\pgfqpoint{4.524877in}{2.084420in}}%
\pgfpathlineto{\pgfqpoint{4.525069in}{2.086075in}}%
\pgfpathlineto{\pgfqpoint{4.525453in}{2.080171in}}%
\pgfpathlineto{\pgfqpoint{4.525645in}{2.081632in}}%
\pgfpathlineto{\pgfqpoint{4.525837in}{2.081173in}}%
\pgfpathlineto{\pgfqpoint{4.526030in}{2.081909in}}%
\pgfpathlineto{\pgfqpoint{4.526798in}{2.084930in}}%
\pgfpathlineto{\pgfqpoint{4.526990in}{2.083461in}}%
\pgfpathlineto{\pgfqpoint{4.527951in}{2.076089in}}%
\pgfpathlineto{\pgfqpoint{4.528336in}{2.079266in}}%
\pgfpathlineto{\pgfqpoint{4.529297in}{2.082563in}}%
\pgfpathlineto{\pgfqpoint{4.528912in}{2.078834in}}%
\pgfpathlineto{\pgfqpoint{4.529489in}{2.081068in}}%
\pgfpathlineto{\pgfqpoint{4.530257in}{2.082107in}}%
\pgfpathlineto{\pgfqpoint{4.531603in}{2.075082in}}%
\pgfpathlineto{\pgfqpoint{4.531795in}{2.075036in}}%
\pgfpathlineto{\pgfqpoint{4.533716in}{2.066287in}}%
\pgfpathlineto{\pgfqpoint{4.533909in}{2.066229in}}%
\pgfpathlineto{\pgfqpoint{4.535062in}{2.059133in}}%
\pgfpathlineto{\pgfqpoint{4.535446in}{2.060946in}}%
\pgfpathlineto{\pgfqpoint{4.537368in}{2.077290in}}%
\pgfpathlineto{\pgfqpoint{4.538136in}{2.073989in}}%
\pgfpathlineto{\pgfqpoint{4.538328in}{2.072156in}}%
\pgfpathlineto{\pgfqpoint{4.539097in}{2.073328in}}%
\pgfpathlineto{\pgfqpoint{4.540827in}{2.080302in}}%
\pgfpathlineto{\pgfqpoint{4.541019in}{2.079933in}}%
\pgfpathlineto{\pgfqpoint{4.541787in}{2.076306in}}%
\pgfpathlineto{\pgfqpoint{4.542364in}{2.076728in}}%
\pgfpathlineto{\pgfqpoint{4.542940in}{2.079268in}}%
\pgfpathlineto{\pgfqpoint{4.543133in}{2.078162in}}%
\pgfpathlineto{\pgfqpoint{4.543517in}{2.075018in}}%
\pgfpathlineto{\pgfqpoint{4.544093in}{2.078145in}}%
\pgfpathlineto{\pgfqpoint{4.544478in}{2.080437in}}%
\pgfpathlineto{\pgfqpoint{4.544862in}{2.076436in}}%
\pgfpathlineto{\pgfqpoint{4.546592in}{2.065070in}}%
\pgfpathlineto{\pgfqpoint{4.546784in}{2.064495in}}%
\pgfpathlineto{\pgfqpoint{4.546976in}{2.066028in}}%
\pgfpathlineto{\pgfqpoint{4.548321in}{2.075588in}}%
\pgfpathlineto{\pgfqpoint{4.548513in}{2.074443in}}%
\pgfpathlineto{\pgfqpoint{4.549282in}{2.066189in}}%
\pgfpathlineto{\pgfqpoint{4.549666in}{2.071959in}}%
\pgfpathlineto{\pgfqpoint{4.549858in}{2.071644in}}%
\pgfpathlineto{\pgfqpoint{4.550435in}{2.064396in}}%
\pgfpathlineto{\pgfqpoint{4.551204in}{2.064496in}}%
\pgfpathlineto{\pgfqpoint{4.551780in}{2.063037in}}%
\pgfpathlineto{\pgfqpoint{4.551972in}{2.066023in}}%
\pgfpathlineto{\pgfqpoint{4.554663in}{2.080862in}}%
\pgfpathlineto{\pgfqpoint{4.555431in}{2.080584in}}%
\pgfpathlineto{\pgfqpoint{4.555624in}{2.079966in}}%
\pgfpathlineto{\pgfqpoint{4.556200in}{2.088280in}}%
\pgfpathlineto{\pgfqpoint{4.556969in}{2.085448in}}%
\pgfpathlineto{\pgfqpoint{4.557737in}{2.083554in}}%
\pgfpathlineto{\pgfqpoint{4.558122in}{2.084505in}}%
\pgfpathlineto{\pgfqpoint{4.558506in}{2.087566in}}%
\pgfpathlineto{\pgfqpoint{4.558890in}{2.084162in}}%
\pgfpathlineto{\pgfqpoint{4.559467in}{2.077817in}}%
\pgfpathlineto{\pgfqpoint{4.560043in}{2.079246in}}%
\pgfpathlineto{\pgfqpoint{4.560428in}{2.082510in}}%
\pgfpathlineto{\pgfqpoint{4.561196in}{2.081043in}}%
\pgfpathlineto{\pgfqpoint{4.562349in}{2.075288in}}%
\pgfpathlineto{\pgfqpoint{4.562542in}{2.075918in}}%
\pgfpathlineto{\pgfqpoint{4.563695in}{2.084823in}}%
\pgfpathlineto{\pgfqpoint{4.563887in}{2.083053in}}%
\pgfpathlineto{\pgfqpoint{4.564655in}{2.077700in}}%
\pgfpathlineto{\pgfqpoint{4.565232in}{2.078886in}}%
\pgfpathlineto{\pgfqpoint{4.565616in}{2.078000in}}%
\pgfpathlineto{\pgfqpoint{4.566577in}{2.083067in}}%
\pgfpathlineto{\pgfqpoint{4.567154in}{2.070298in}}%
\pgfpathlineto{\pgfqpoint{4.567922in}{2.077427in}}%
\pgfpathlineto{\pgfqpoint{4.568307in}{2.080390in}}%
\pgfpathlineto{\pgfqpoint{4.568691in}{2.075446in}}%
\pgfpathlineto{\pgfqpoint{4.568883in}{2.076722in}}%
\pgfpathlineto{\pgfqpoint{4.570228in}{2.085192in}}%
\pgfpathlineto{\pgfqpoint{4.570420in}{2.086817in}}%
\pgfpathlineto{\pgfqpoint{4.570805in}{2.083526in}}%
\pgfpathlineto{\pgfqpoint{4.570997in}{2.084249in}}%
\pgfpathlineto{\pgfqpoint{4.572150in}{2.077025in}}%
\pgfpathlineto{\pgfqpoint{4.572534in}{2.077993in}}%
\pgfpathlineto{\pgfqpoint{4.572919in}{2.075484in}}%
\pgfpathlineto{\pgfqpoint{4.573495in}{2.077399in}}%
\pgfpathlineto{\pgfqpoint{4.574072in}{2.081265in}}%
\pgfpathlineto{\pgfqpoint{4.574648in}{2.080405in}}%
\pgfpathlineto{\pgfqpoint{4.574840in}{2.080373in}}%
\pgfpathlineto{\pgfqpoint{4.575609in}{2.078050in}}%
\pgfpathlineto{\pgfqpoint{4.575801in}{2.078748in}}%
\pgfpathlineto{\pgfqpoint{4.576378in}{2.078479in}}%
\pgfpathlineto{\pgfqpoint{4.577146in}{2.080982in}}%
\pgfpathlineto{\pgfqpoint{4.577531in}{2.080612in}}%
\pgfpathlineto{\pgfqpoint{4.580798in}{2.067785in}}%
\pgfpathlineto{\pgfqpoint{4.582335in}{2.078638in}}%
\pgfpathlineto{\pgfqpoint{4.582527in}{2.076607in}}%
\pgfpathlineto{\pgfqpoint{4.582719in}{2.076215in}}%
\pgfpathlineto{\pgfqpoint{4.583104in}{2.077566in}}%
\pgfpathlineto{\pgfqpoint{4.584257in}{2.083650in}}%
\pgfpathlineto{\pgfqpoint{4.585025in}{2.082137in}}%
\pgfpathlineto{\pgfqpoint{4.585794in}{2.080570in}}%
\pgfpathlineto{\pgfqpoint{4.585410in}{2.082874in}}%
\pgfpathlineto{\pgfqpoint{4.585986in}{2.082468in}}%
\pgfpathlineto{\pgfqpoint{4.587523in}{2.091332in}}%
\pgfpathlineto{\pgfqpoint{4.587716in}{2.089384in}}%
\pgfpathlineto{\pgfqpoint{4.588869in}{2.086635in}}%
\pgfpathlineto{\pgfqpoint{4.589253in}{2.086753in}}%
\pgfpathlineto{\pgfqpoint{4.590790in}{2.091097in}}%
\pgfpathlineto{\pgfqpoint{4.591367in}{2.085160in}}%
\pgfpathlineto{\pgfqpoint{4.591943in}{2.088511in}}%
\pgfpathlineto{\pgfqpoint{4.592328in}{2.089135in}}%
\pgfpathlineto{\pgfqpoint{4.592904in}{2.093917in}}%
\pgfpathlineto{\pgfqpoint{4.593481in}{2.091254in}}%
\pgfpathlineto{\pgfqpoint{4.595018in}{2.082279in}}%
\pgfpathlineto{\pgfqpoint{4.595979in}{2.084700in}}%
\pgfpathlineto{\pgfqpoint{4.596171in}{2.083960in}}%
\pgfpathlineto{\pgfqpoint{4.596363in}{2.087530in}}%
\pgfpathlineto{\pgfqpoint{4.598093in}{2.083081in}}%
\pgfpathlineto{\pgfqpoint{4.598285in}{2.088408in}}%
\pgfpathlineto{\pgfqpoint{4.599053in}{2.079916in}}%
\pgfpathlineto{\pgfqpoint{4.599438in}{2.077273in}}%
\pgfpathlineto{\pgfqpoint{4.600014in}{2.081073in}}%
\pgfpathlineto{\pgfqpoint{4.601552in}{2.069836in}}%
\pgfpathlineto{\pgfqpoint{4.602128in}{2.070818in}}%
\pgfpathlineto{\pgfqpoint{4.602897in}{2.080274in}}%
\pgfpathlineto{\pgfqpoint{4.603666in}{2.075105in}}%
\pgfpathlineto{\pgfqpoint{4.606164in}{2.088276in}}%
\pgfpathlineto{\pgfqpoint{4.606740in}{2.084708in}}%
\pgfpathlineto{\pgfqpoint{4.606932in}{2.081411in}}%
\pgfpathlineto{\pgfqpoint{4.607893in}{2.081679in}}%
\pgfpathlineto{\pgfqpoint{4.608854in}{2.089882in}}%
\pgfpathlineto{\pgfqpoint{4.609046in}{2.088317in}}%
\pgfpathlineto{\pgfqpoint{4.609238in}{2.086860in}}%
\pgfpathlineto{\pgfqpoint{4.609815in}{2.089867in}}%
\pgfpathlineto{\pgfqpoint{4.610199in}{2.087814in}}%
\pgfpathlineto{\pgfqpoint{4.610391in}{2.088342in}}%
\pgfpathlineto{\pgfqpoint{4.610968in}{2.081864in}}%
\pgfpathlineto{\pgfqpoint{4.611544in}{2.084516in}}%
\pgfpathlineto{\pgfqpoint{4.612313in}{2.081598in}}%
\pgfpathlineto{\pgfqpoint{4.613082in}{2.083816in}}%
\pgfpathlineto{\pgfqpoint{4.613274in}{2.085037in}}%
\pgfpathlineto{\pgfqpoint{4.613658in}{2.081802in}}%
\pgfpathlineto{\pgfqpoint{4.614043in}{2.074983in}}%
\pgfpathlineto{\pgfqpoint{4.614811in}{2.080203in}}%
\pgfpathlineto{\pgfqpoint{4.615196in}{2.082739in}}%
\pgfpathlineto{\pgfqpoint{4.615388in}{2.079569in}}%
\pgfpathlineto{\pgfqpoint{4.616349in}{2.071902in}}%
\pgfpathlineto{\pgfqpoint{4.616925in}{2.072148in}}%
\pgfpathlineto{\pgfqpoint{4.617886in}{2.075686in}}%
\pgfpathlineto{\pgfqpoint{4.618078in}{2.077887in}}%
\pgfpathlineto{\pgfqpoint{4.618847in}{2.074082in}}%
\pgfpathlineto{\pgfqpoint{4.619615in}{2.077810in}}%
\pgfpathlineto{\pgfqpoint{4.620000in}{2.075089in}}%
\pgfpathlineto{\pgfqpoint{4.622114in}{2.070947in}}%
\pgfpathlineto{\pgfqpoint{4.623267in}{2.076903in}}%
\pgfpathlineto{\pgfqpoint{4.623459in}{2.076284in}}%
\pgfpathlineto{\pgfqpoint{4.624420in}{2.070005in}}%
\pgfpathlineto{\pgfqpoint{4.624612in}{2.074559in}}%
\pgfpathlineto{\pgfqpoint{4.624804in}{2.073952in}}%
\pgfpathlineto{\pgfqpoint{4.626341in}{2.083407in}}%
\pgfpathlineto{\pgfqpoint{4.627494in}{2.079097in}}%
\pgfpathlineto{\pgfqpoint{4.626918in}{2.084702in}}%
\pgfpathlineto{\pgfqpoint{4.628071in}{2.081550in}}%
\pgfpathlineto{\pgfqpoint{4.628455in}{2.085254in}}%
\pgfpathlineto{\pgfqpoint{4.628840in}{2.081474in}}%
\pgfpathlineto{\pgfqpoint{4.629032in}{2.081897in}}%
\pgfpathlineto{\pgfqpoint{4.629224in}{2.081844in}}%
\pgfpathlineto{\pgfqpoint{4.629800in}{2.077338in}}%
\pgfpathlineto{\pgfqpoint{4.630377in}{2.080438in}}%
\pgfpathlineto{\pgfqpoint{4.630953in}{2.080826in}}%
\pgfpathlineto{\pgfqpoint{4.631722in}{2.078155in}}%
\pgfpathlineto{\pgfqpoint{4.631914in}{2.078188in}}%
\pgfpathlineto{\pgfqpoint{4.634412in}{2.067835in}}%
\pgfpathlineto{\pgfqpoint{4.634605in}{2.071603in}}%
\pgfpathlineto{\pgfqpoint{4.635565in}{2.070003in}}%
\pgfpathlineto{\pgfqpoint{4.636142in}{2.068357in}}%
\pgfpathlineto{\pgfqpoint{4.636334in}{2.071217in}}%
\pgfpathlineto{\pgfqpoint{4.637103in}{2.074304in}}%
\pgfpathlineto{\pgfqpoint{4.636911in}{2.070719in}}%
\pgfpathlineto{\pgfqpoint{4.637487in}{2.074152in}}%
\pgfpathlineto{\pgfqpoint{4.637871in}{2.071086in}}%
\pgfpathlineto{\pgfqpoint{4.638256in}{2.075031in}}%
\pgfpathlineto{\pgfqpoint{4.638640in}{2.073263in}}%
\pgfpathlineto{\pgfqpoint{4.640370in}{2.066231in}}%
\pgfpathlineto{\pgfqpoint{4.640946in}{2.067043in}}%
\pgfpathlineto{\pgfqpoint{4.641138in}{2.070437in}}%
\pgfpathlineto{\pgfqpoint{4.641523in}{2.064337in}}%
\pgfpathlineto{\pgfqpoint{4.641907in}{2.066813in}}%
\pgfpathlineto{\pgfqpoint{4.644021in}{2.073191in}}%
\pgfpathlineto{\pgfqpoint{4.644213in}{2.071159in}}%
\pgfpathlineto{\pgfqpoint{4.644597in}{2.075648in}}%
\pgfpathlineto{\pgfqpoint{4.645366in}{2.081500in}}%
\pgfpathlineto{\pgfqpoint{4.646135in}{2.079159in}}%
\pgfpathlineto{\pgfqpoint{4.646519in}{2.077291in}}%
\pgfpathlineto{\pgfqpoint{4.646711in}{2.080363in}}%
\pgfpathlineto{\pgfqpoint{4.646903in}{2.077454in}}%
\pgfpathlineto{\pgfqpoint{4.647864in}{2.077861in}}%
\pgfpathlineto{\pgfqpoint{4.648056in}{2.078510in}}%
\pgfpathlineto{\pgfqpoint{4.648248in}{2.077958in}}%
\pgfpathlineto{\pgfqpoint{4.649786in}{2.063383in}}%
\pgfpathlineto{\pgfqpoint{4.650170in}{2.066244in}}%
\pgfpathlineto{\pgfqpoint{4.651515in}{2.075121in}}%
\pgfpathlineto{\pgfqpoint{4.653629in}{2.068212in}}%
\pgfpathlineto{\pgfqpoint{4.653821in}{2.071470in}}%
\pgfpathlineto{\pgfqpoint{4.654782in}{2.075153in}}%
\pgfpathlineto{\pgfqpoint{4.655167in}{2.072714in}}%
\pgfpathlineto{\pgfqpoint{4.656127in}{2.067337in}}%
\pgfpathlineto{\pgfqpoint{4.656512in}{2.068506in}}%
\pgfpathlineto{\pgfqpoint{4.656704in}{2.068644in}}%
\pgfpathlineto{\pgfqpoint{4.658049in}{2.061339in}}%
\pgfpathlineto{\pgfqpoint{4.658241in}{2.062425in}}%
\pgfpathlineto{\pgfqpoint{4.658433in}{2.065243in}}%
\pgfpathlineto{\pgfqpoint{4.658818in}{2.061495in}}%
\pgfpathlineto{\pgfqpoint{4.659202in}{2.062716in}}%
\pgfpathlineto{\pgfqpoint{4.659394in}{2.062272in}}%
\pgfpathlineto{\pgfqpoint{4.659586in}{2.062932in}}%
\pgfpathlineto{\pgfqpoint{4.660355in}{2.065933in}}%
\pgfpathlineto{\pgfqpoint{4.660547in}{2.063877in}}%
\pgfpathlineto{\pgfqpoint{4.662277in}{2.056220in}}%
\pgfpathlineto{\pgfqpoint{4.662469in}{2.056307in}}%
\pgfpathlineto{\pgfqpoint{4.663045in}{2.057941in}}%
\pgfpathlineto{\pgfqpoint{4.663430in}{2.057013in}}%
\pgfpathlineto{\pgfqpoint{4.665928in}{2.040581in}}%
\pgfpathlineto{\pgfqpoint{4.666504in}{2.041983in}}%
\pgfpathlineto{\pgfqpoint{4.667465in}{2.048610in}}%
\pgfpathlineto{\pgfqpoint{4.667850in}{2.046518in}}%
\pgfpathlineto{\pgfqpoint{4.668426in}{2.045870in}}%
\pgfpathlineto{\pgfqpoint{4.669195in}{2.049505in}}%
\pgfpathlineto{\pgfqpoint{4.669579in}{2.046554in}}%
\pgfpathlineto{\pgfqpoint{4.669963in}{2.051147in}}%
\pgfpathlineto{\pgfqpoint{4.670732in}{2.047922in}}%
\pgfpathlineto{\pgfqpoint{4.671309in}{2.044568in}}%
\pgfpathlineto{\pgfqpoint{4.671501in}{2.048571in}}%
\pgfpathlineto{\pgfqpoint{4.672269in}{2.056746in}}%
\pgfpathlineto{\pgfqpoint{4.672654in}{2.054209in}}%
\pgfpathlineto{\pgfqpoint{4.673422in}{2.051656in}}%
\pgfpathlineto{\pgfqpoint{4.673807in}{2.052794in}}%
\pgfpathlineto{\pgfqpoint{4.675729in}{2.067610in}}%
\pgfpathlineto{\pgfqpoint{4.675921in}{2.066485in}}%
\pgfpathlineto{\pgfqpoint{4.676497in}{2.064850in}}%
\pgfpathlineto{\pgfqpoint{4.676689in}{2.067044in}}%
\pgfpathlineto{\pgfqpoint{4.678995in}{2.076289in}}%
\pgfpathlineto{\pgfqpoint{4.679380in}{2.074236in}}%
\pgfpathlineto{\pgfqpoint{4.679764in}{2.077724in}}%
\pgfpathlineto{\pgfqpoint{4.680148in}{2.080647in}}%
\pgfpathlineto{\pgfqpoint{4.680341in}{2.078183in}}%
\pgfpathlineto{\pgfqpoint{4.681686in}{2.071212in}}%
\pgfpathlineto{\pgfqpoint{4.683031in}{2.069690in}}%
\pgfpathlineto{\pgfqpoint{4.684953in}{2.059417in}}%
\pgfpathlineto{\pgfqpoint{4.685337in}{2.066807in}}%
\pgfpathlineto{\pgfqpoint{4.685913in}{2.057305in}}%
\pgfpathlineto{\pgfqpoint{4.686298in}{2.057533in}}%
\pgfpathlineto{\pgfqpoint{4.688027in}{2.047899in}}%
\pgfpathlineto{\pgfqpoint{4.688219in}{2.049910in}}%
\pgfpathlineto{\pgfqpoint{4.688988in}{2.054780in}}%
\pgfpathlineto{\pgfqpoint{4.689372in}{2.054411in}}%
\pgfpathlineto{\pgfqpoint{4.689949in}{2.046890in}}%
\pgfpathlineto{\pgfqpoint{4.690718in}{2.049503in}}%
\pgfpathlineto{\pgfqpoint{4.690910in}{2.049535in}}%
\pgfpathlineto{\pgfqpoint{4.691294in}{2.044425in}}%
\pgfpathlineto{\pgfqpoint{4.691678in}{2.051056in}}%
\pgfpathlineto{\pgfqpoint{4.691871in}{2.050734in}}%
\pgfpathlineto{\pgfqpoint{4.693600in}{2.060433in}}%
\pgfpathlineto{\pgfqpoint{4.693984in}{2.054651in}}%
\pgfpathlineto{\pgfqpoint{4.694561in}{2.058095in}}%
\pgfpathlineto{\pgfqpoint{4.694945in}{2.060990in}}%
\pgfpathlineto{\pgfqpoint{4.695137in}{2.057103in}}%
\pgfpathlineto{\pgfqpoint{4.695330in}{2.057979in}}%
\pgfpathlineto{\pgfqpoint{4.696098in}{2.049398in}}%
\pgfpathlineto{\pgfqpoint{4.696483in}{2.053190in}}%
\pgfpathlineto{\pgfqpoint{4.696675in}{2.054733in}}%
\pgfpathlineto{\pgfqpoint{4.697251in}{2.050954in}}%
\pgfpathlineto{\pgfqpoint{4.697443in}{2.050219in}}%
\pgfpathlineto{\pgfqpoint{4.697636in}{2.054283in}}%
\pgfpathlineto{\pgfqpoint{4.698596in}{2.052062in}}%
\pgfpathlineto{\pgfqpoint{4.700134in}{2.040293in}}%
\pgfpathlineto{\pgfqpoint{4.700326in}{2.042423in}}%
\pgfpathlineto{\pgfqpoint{4.700710in}{2.039146in}}%
\pgfpathlineto{\pgfqpoint{4.700903in}{2.039960in}}%
\pgfpathlineto{\pgfqpoint{4.701095in}{2.038244in}}%
\pgfpathlineto{\pgfqpoint{4.701671in}{2.041507in}}%
\pgfpathlineto{\pgfqpoint{4.702056in}{2.038396in}}%
\pgfpathlineto{\pgfqpoint{4.702632in}{2.036595in}}%
\pgfpathlineto{\pgfqpoint{4.703209in}{2.040309in}}%
\pgfpathlineto{\pgfqpoint{4.703785in}{2.034485in}}%
\pgfpathlineto{\pgfqpoint{4.704554in}{2.037743in}}%
\pgfpathlineto{\pgfqpoint{4.704746in}{2.038518in}}%
\pgfpathlineto{\pgfqpoint{4.704938in}{2.036326in}}%
\pgfpathlineto{\pgfqpoint{4.705130in}{2.035442in}}%
\pgfpathlineto{\pgfqpoint{4.705322in}{2.038703in}}%
\pgfpathlineto{\pgfqpoint{4.705707in}{2.037356in}}%
\pgfpathlineto{\pgfqpoint{4.705899in}{2.037294in}}%
\pgfpathlineto{\pgfqpoint{4.706091in}{2.039680in}}%
\pgfpathlineto{\pgfqpoint{4.706668in}{2.034145in}}%
\pgfpathlineto{\pgfqpoint{4.707052in}{2.035358in}}%
\pgfpathlineto{\pgfqpoint{4.707244in}{2.037769in}}%
\pgfpathlineto{\pgfqpoint{4.707628in}{2.032959in}}%
\pgfpathlineto{\pgfqpoint{4.708013in}{2.034198in}}%
\pgfpathlineto{\pgfqpoint{4.708205in}{2.034685in}}%
\pgfpathlineto{\pgfqpoint{4.708781in}{2.033351in}}%
\pgfpathlineto{\pgfqpoint{4.709550in}{2.033775in}}%
\pgfpathlineto{\pgfqpoint{4.710127in}{2.029079in}}%
\pgfpathlineto{\pgfqpoint{4.711280in}{2.035179in}}%
\pgfpathlineto{\pgfqpoint{4.711472in}{2.034249in}}%
\pgfpathlineto{\pgfqpoint{4.711664in}{2.037283in}}%
\pgfpathlineto{\pgfqpoint{4.711856in}{2.036663in}}%
\pgfpathlineto{\pgfqpoint{4.713586in}{2.047230in}}%
\pgfpathlineto{\pgfqpoint{4.713778in}{2.045431in}}%
\pgfpathlineto{\pgfqpoint{4.713970in}{2.045629in}}%
\pgfpathlineto{\pgfqpoint{4.714162in}{2.045395in}}%
\pgfpathlineto{\pgfqpoint{4.715123in}{2.052880in}}%
\pgfpathlineto{\pgfqpoint{4.715507in}{2.051271in}}%
\pgfpathlineto{\pgfqpoint{4.717237in}{2.035509in}}%
\pgfpathlineto{\pgfqpoint{4.717813in}{2.040209in}}%
\pgfpathlineto{\pgfqpoint{4.719158in}{2.046089in}}%
\pgfpathlineto{\pgfqpoint{4.719927in}{2.040299in}}%
\pgfpathlineto{\pgfqpoint{4.720311in}{2.042211in}}%
\pgfpathlineto{\pgfqpoint{4.722041in}{2.050102in}}%
\pgfpathlineto{\pgfqpoint{4.722425in}{2.050731in}}%
\pgfpathlineto{\pgfqpoint{4.722617in}{2.050053in}}%
\pgfpathlineto{\pgfqpoint{4.722810in}{2.048163in}}%
\pgfpathlineto{\pgfqpoint{4.723386in}{2.050868in}}%
\pgfpathlineto{\pgfqpoint{4.724155in}{2.057410in}}%
\pgfpathlineto{\pgfqpoint{4.724347in}{2.060309in}}%
\pgfpathlineto{\pgfqpoint{4.724924in}{2.055572in}}%
\pgfpathlineto{\pgfqpoint{4.725116in}{2.056021in}}%
\pgfpathlineto{\pgfqpoint{4.727230in}{2.044088in}}%
\pgfpathlineto{\pgfqpoint{4.727422in}{2.041015in}}%
\pgfpathlineto{\pgfqpoint{4.728383in}{2.041963in}}%
\pgfpathlineto{\pgfqpoint{4.729728in}{2.039741in}}%
\pgfpathlineto{\pgfqpoint{4.729343in}{2.042668in}}%
\pgfpathlineto{\pgfqpoint{4.729920in}{2.040147in}}%
\pgfpathlineto{\pgfqpoint{4.730304in}{2.038731in}}%
\pgfpathlineto{\pgfqpoint{4.732226in}{2.048280in}}%
\pgfpathlineto{\pgfqpoint{4.732418in}{2.048305in}}%
\pgfpathlineto{\pgfqpoint{4.734148in}{2.063284in}}%
\pgfpathlineto{\pgfqpoint{4.734532in}{2.060171in}}%
\pgfpathlineto{\pgfqpoint{4.734916in}{2.062369in}}%
\pgfpathlineto{\pgfqpoint{4.735108in}{2.065093in}}%
\pgfpathlineto{\pgfqpoint{4.735493in}{2.058491in}}%
\pgfpathlineto{\pgfqpoint{4.735877in}{2.061391in}}%
\pgfpathlineto{\pgfqpoint{4.737414in}{2.054414in}}%
\pgfpathlineto{\pgfqpoint{4.737799in}{2.058219in}}%
\pgfpathlineto{\pgfqpoint{4.738183in}{2.050121in}}%
\pgfpathlineto{\pgfqpoint{4.738952in}{2.051674in}}%
\pgfpathlineto{\pgfqpoint{4.739336in}{2.050910in}}%
\pgfpathlineto{\pgfqpoint{4.740297in}{2.042875in}}%
\pgfpathlineto{\pgfqpoint{4.740489in}{2.046238in}}%
\pgfpathlineto{\pgfqpoint{4.740681in}{2.050196in}}%
\pgfpathlineto{\pgfqpoint{4.741642in}{2.048842in}}%
\pgfpathlineto{\pgfqpoint{4.742795in}{2.039078in}}%
\pgfpathlineto{\pgfqpoint{4.743179in}{2.042535in}}%
\pgfpathlineto{\pgfqpoint{4.746062in}{2.060121in}}%
\pgfpathlineto{\pgfqpoint{4.747407in}{2.051134in}}%
\pgfpathlineto{\pgfqpoint{4.747791in}{2.053848in}}%
\pgfpathlineto{\pgfqpoint{4.747984in}{2.053614in}}%
\pgfpathlineto{\pgfqpoint{4.749329in}{2.064266in}}%
\pgfpathlineto{\pgfqpoint{4.749521in}{2.062858in}}%
\pgfpathlineto{\pgfqpoint{4.752211in}{2.080676in}}%
\pgfpathlineto{\pgfqpoint{4.753172in}{2.076063in}}%
\pgfpathlineto{\pgfqpoint{4.753749in}{2.077815in}}%
\pgfpathlineto{\pgfqpoint{4.754133in}{2.079058in}}%
\pgfpathlineto{\pgfqpoint{4.754325in}{2.075777in}}%
\pgfpathlineto{\pgfqpoint{4.754902in}{2.080838in}}%
\pgfpathlineto{\pgfqpoint{4.756247in}{2.091072in}}%
\pgfpathlineto{\pgfqpoint{4.756823in}{2.090620in}}%
\pgfpathlineto{\pgfqpoint{4.758169in}{2.095437in}}%
\pgfpathlineto{\pgfqpoint{4.758553in}{2.091121in}}%
\pgfpathlineto{\pgfqpoint{4.759129in}{2.095993in}}%
\pgfpathlineto{\pgfqpoint{4.760282in}{2.105141in}}%
\pgfpathlineto{\pgfqpoint{4.760475in}{2.104023in}}%
\pgfpathlineto{\pgfqpoint{4.760667in}{2.101704in}}%
\pgfpathlineto{\pgfqpoint{4.761435in}{2.104138in}}%
\pgfpathlineto{\pgfqpoint{4.761628in}{2.103137in}}%
\pgfpathlineto{\pgfqpoint{4.761820in}{2.105563in}}%
\pgfpathlineto{\pgfqpoint{4.762588in}{2.101644in}}%
\pgfpathlineto{\pgfqpoint{4.763549in}{2.105598in}}%
\pgfpathlineto{\pgfqpoint{4.763934in}{2.102800in}}%
\pgfpathlineto{\pgfqpoint{4.765471in}{2.091698in}}%
\pgfpathlineto{\pgfqpoint{4.765663in}{2.092021in}}%
\pgfpathlineto{\pgfqpoint{4.766624in}{2.094608in}}%
\pgfpathlineto{\pgfqpoint{4.767008in}{2.100578in}}%
\pgfpathlineto{\pgfqpoint{4.767585in}{2.093755in}}%
\pgfpathlineto{\pgfqpoint{4.767777in}{2.095522in}}%
\pgfpathlineto{\pgfqpoint{4.768930in}{2.099225in}}%
\pgfpathlineto{\pgfqpoint{4.768546in}{2.095122in}}%
\pgfpathlineto{\pgfqpoint{4.769122in}{2.098990in}}%
\pgfpathlineto{\pgfqpoint{4.769506in}{2.097813in}}%
\pgfpathlineto{\pgfqpoint{4.769891in}{2.100170in}}%
\pgfpathlineto{\pgfqpoint{4.771236in}{2.105901in}}%
\pgfpathlineto{\pgfqpoint{4.772581in}{2.110896in}}%
\pgfpathlineto{\pgfqpoint{4.772965in}{2.111651in}}%
\pgfpathlineto{\pgfqpoint{4.773350in}{2.109575in}}%
\pgfpathlineto{\pgfqpoint{4.774311in}{2.114385in}}%
\pgfpathlineto{\pgfqpoint{4.775656in}{2.108212in}}%
\pgfpathlineto{\pgfqpoint{4.775848in}{2.109607in}}%
\pgfpathlineto{\pgfqpoint{4.776232in}{2.109131in}}%
\pgfpathlineto{\pgfqpoint{4.776617in}{2.103442in}}%
\pgfpathlineto{\pgfqpoint{4.777385in}{2.107860in}}%
\pgfpathlineto{\pgfqpoint{4.778346in}{2.104028in}}%
\pgfpathlineto{\pgfqpoint{4.778923in}{2.105836in}}%
\pgfpathlineto{\pgfqpoint{4.779691in}{2.109284in}}%
\pgfpathlineto{\pgfqpoint{4.779884in}{2.107686in}}%
\pgfpathlineto{\pgfqpoint{4.780076in}{2.105230in}}%
\pgfpathlineto{\pgfqpoint{4.780460in}{2.111793in}}%
\pgfpathlineto{\pgfqpoint{4.780652in}{2.111517in}}%
\pgfpathlineto{\pgfqpoint{4.781613in}{2.115870in}}%
\pgfpathlineto{\pgfqpoint{4.782190in}{2.115135in}}%
\pgfpathlineto{\pgfqpoint{4.782766in}{2.115311in}}%
\pgfpathlineto{\pgfqpoint{4.783535in}{2.110762in}}%
\pgfpathlineto{\pgfqpoint{4.783727in}{2.110995in}}%
\pgfpathlineto{\pgfqpoint{4.785072in}{2.115708in}}%
\pgfpathlineto{\pgfqpoint{4.786225in}{2.115639in}}%
\pgfpathlineto{\pgfqpoint{4.788147in}{2.130862in}}%
\pgfpathlineto{\pgfqpoint{4.788339in}{2.129319in}}%
\pgfpathlineto{\pgfqpoint{4.788915in}{2.128208in}}%
\pgfpathlineto{\pgfqpoint{4.789108in}{2.129914in}}%
\pgfpathlineto{\pgfqpoint{4.791221in}{2.135106in}}%
\pgfpathlineto{\pgfqpoint{4.791414in}{2.135018in}}%
\pgfpathlineto{\pgfqpoint{4.791798in}{2.132421in}}%
\pgfpathlineto{\pgfqpoint{4.792182in}{2.135075in}}%
\pgfpathlineto{\pgfqpoint{4.792374in}{2.136306in}}%
\pgfpathlineto{\pgfqpoint{4.792567in}{2.132273in}}%
\pgfpathlineto{\pgfqpoint{4.792759in}{2.133332in}}%
\pgfpathlineto{\pgfqpoint{4.793143in}{2.130798in}}%
\pgfpathlineto{\pgfqpoint{4.793527in}{2.133158in}}%
\pgfpathlineto{\pgfqpoint{4.793912in}{2.135329in}}%
\pgfpathlineto{\pgfqpoint{4.794296in}{2.130658in}}%
\pgfpathlineto{\pgfqpoint{4.794873in}{2.129219in}}%
\pgfpathlineto{\pgfqpoint{4.795065in}{2.131267in}}%
\pgfpathlineto{\pgfqpoint{4.795833in}{2.129107in}}%
\pgfpathlineto{\pgfqpoint{4.796218in}{2.130649in}}%
\pgfpathlineto{\pgfqpoint{4.796410in}{2.131608in}}%
\pgfpathlineto{\pgfqpoint{4.796794in}{2.128965in}}%
\pgfpathlineto{\pgfqpoint{4.797755in}{2.115650in}}%
\pgfpathlineto{\pgfqpoint{4.798140in}{2.121185in}}%
\pgfpathlineto{\pgfqpoint{4.798524in}{2.122385in}}%
\pgfpathlineto{\pgfqpoint{4.799100in}{2.121499in}}%
\pgfpathlineto{\pgfqpoint{4.799485in}{2.119282in}}%
\pgfpathlineto{\pgfqpoint{4.799869in}{2.120681in}}%
\pgfpathlineto{\pgfqpoint{4.800061in}{2.125721in}}%
\pgfpathlineto{\pgfqpoint{4.801022in}{2.124979in}}%
\pgfpathlineto{\pgfqpoint{4.802752in}{2.114181in}}%
\pgfpathlineto{\pgfqpoint{4.803712in}{2.119477in}}%
\pgfpathlineto{\pgfqpoint{4.804097in}{2.115200in}}%
\pgfpathlineto{\pgfqpoint{4.805058in}{2.112736in}}%
\pgfpathlineto{\pgfqpoint{4.805442in}{2.113969in}}%
\pgfpathlineto{\pgfqpoint{4.805634in}{2.115705in}}%
\pgfpathlineto{\pgfqpoint{4.806211in}{2.113058in}}%
\pgfpathlineto{\pgfqpoint{4.806403in}{2.113265in}}%
\pgfpathlineto{\pgfqpoint{4.806595in}{2.114070in}}%
\pgfpathlineto{\pgfqpoint{4.806787in}{2.113595in}}%
\pgfpathlineto{\pgfqpoint{4.808324in}{2.101942in}}%
\pgfpathlineto{\pgfqpoint{4.810054in}{2.108537in}}%
\pgfpathlineto{\pgfqpoint{4.811399in}{2.101390in}}%
\pgfpathlineto{\pgfqpoint{4.813705in}{2.114726in}}%
\pgfpathlineto{\pgfqpoint{4.814089in}{2.113550in}}%
\pgfpathlineto{\pgfqpoint{4.814474in}{2.114522in}}%
\pgfpathlineto{\pgfqpoint{4.816011in}{2.104688in}}%
\pgfpathlineto{\pgfqpoint{4.816395in}{2.096973in}}%
\pgfpathlineto{\pgfqpoint{4.817164in}{2.097680in}}%
\pgfpathlineto{\pgfqpoint{4.818509in}{2.106286in}}%
\pgfpathlineto{\pgfqpoint{4.818894in}{2.104353in}}%
\pgfpathlineto{\pgfqpoint{4.819854in}{2.101465in}}%
\pgfpathlineto{\pgfqpoint{4.820047in}{2.102558in}}%
\pgfpathlineto{\pgfqpoint{4.821392in}{2.105948in}}%
\pgfpathlineto{\pgfqpoint{4.822737in}{2.097685in}}%
\pgfpathlineto{\pgfqpoint{4.823121in}{2.102807in}}%
\pgfpathlineto{\pgfqpoint{4.823890in}{2.101872in}}%
\pgfpathlineto{\pgfqpoint{4.825427in}{2.091847in}}%
\pgfpathlineto{\pgfqpoint{4.826004in}{2.092608in}}%
\pgfpathlineto{\pgfqpoint{4.826580in}{2.093206in}}%
\pgfpathlineto{\pgfqpoint{4.826773in}{2.091879in}}%
\pgfpathlineto{\pgfqpoint{4.826965in}{2.091525in}}%
\pgfpathlineto{\pgfqpoint{4.828310in}{2.098206in}}%
\pgfpathlineto{\pgfqpoint{4.829271in}{2.090564in}}%
\pgfpathlineto{\pgfqpoint{4.829655in}{2.094135in}}%
\pgfpathlineto{\pgfqpoint{4.829847in}{2.093560in}}%
\pgfpathlineto{\pgfqpoint{4.830039in}{2.094101in}}%
\pgfpathlineto{\pgfqpoint{4.831577in}{2.103799in}}%
\pgfpathlineto{\pgfqpoint{4.832730in}{2.109557in}}%
\pgfpathlineto{\pgfqpoint{4.831961in}{2.103535in}}%
\pgfpathlineto{\pgfqpoint{4.833114in}{2.108535in}}%
\pgfpathlineto{\pgfqpoint{4.834651in}{2.101398in}}%
\pgfpathlineto{\pgfqpoint{4.834844in}{2.102432in}}%
\pgfpathlineto{\pgfqpoint{4.836573in}{2.110168in}}%
\pgfpathlineto{\pgfqpoint{4.836765in}{2.109935in}}%
\pgfpathlineto{\pgfqpoint{4.837150in}{2.108035in}}%
\pgfpathlineto{\pgfqpoint{4.837726in}{2.110590in}}%
\pgfpathlineto{\pgfqpoint{4.838879in}{2.115660in}}%
\pgfpathlineto{\pgfqpoint{4.840416in}{2.108523in}}%
\pgfpathlineto{\pgfqpoint{4.841377in}{2.117349in}}%
\pgfpathlineto{\pgfqpoint{4.841954in}{2.117261in}}%
\pgfpathlineto{\pgfqpoint{4.842146in}{2.118102in}}%
\pgfpathlineto{\pgfqpoint{4.842530in}{2.114983in}}%
\pgfpathlineto{\pgfqpoint{4.843299in}{2.106873in}}%
\pgfpathlineto{\pgfqpoint{4.843875in}{2.110101in}}%
\pgfpathlineto{\pgfqpoint{4.844260in}{2.111323in}}%
\pgfpathlineto{\pgfqpoint{4.844452in}{2.109793in}}%
\pgfpathlineto{\pgfqpoint{4.844836in}{2.110343in}}%
\pgfpathlineto{\pgfqpoint{4.845221in}{2.107489in}}%
\pgfpathlineto{\pgfqpoint{4.845797in}{2.109933in}}%
\pgfpathlineto{\pgfqpoint{4.847911in}{2.117747in}}%
\pgfpathlineto{\pgfqpoint{4.848103in}{2.116308in}}%
\pgfpathlineto{\pgfqpoint{4.849256in}{2.113096in}}%
\pgfpathlineto{\pgfqpoint{4.849448in}{2.113171in}}%
\pgfpathlineto{\pgfqpoint{4.850601in}{2.120276in}}%
\pgfpathlineto{\pgfqpoint{4.850794in}{2.117382in}}%
\pgfpathlineto{\pgfqpoint{4.850986in}{2.113114in}}%
\pgfpathlineto{\pgfqpoint{4.851754in}{2.118495in}}%
\pgfpathlineto{\pgfqpoint{4.852523in}{2.115084in}}%
\pgfpathlineto{\pgfqpoint{4.852907in}{2.118326in}}%
\pgfpathlineto{\pgfqpoint{4.854445in}{2.122343in}}%
\pgfpathlineto{\pgfqpoint{4.854637in}{2.122601in}}%
\pgfpathlineto{\pgfqpoint{4.854829in}{2.121520in}}%
\pgfpathlineto{\pgfqpoint{4.855021in}{2.121389in}}%
\pgfpathlineto{\pgfqpoint{4.855982in}{2.119225in}}%
\pgfpathlineto{\pgfqpoint{4.855406in}{2.122742in}}%
\pgfpathlineto{\pgfqpoint{4.856174in}{2.120519in}}%
\pgfpathlineto{\pgfqpoint{4.856751in}{2.119335in}}%
\pgfpathlineto{\pgfqpoint{4.857904in}{2.124627in}}%
\pgfpathlineto{\pgfqpoint{4.858096in}{2.122924in}}%
\pgfpathlineto{\pgfqpoint{4.858288in}{2.126297in}}%
\pgfpathlineto{\pgfqpoint{4.858672in}{2.124227in}}%
\pgfpathlineto{\pgfqpoint{4.858865in}{2.129550in}}%
\pgfpathlineto{\pgfqpoint{4.859825in}{2.127854in}}%
\pgfpathlineto{\pgfqpoint{4.861363in}{2.122361in}}%
\pgfpathlineto{\pgfqpoint{4.863092in}{2.139598in}}%
\pgfpathlineto{\pgfqpoint{4.863284in}{2.139288in}}%
\pgfpathlineto{\pgfqpoint{4.863477in}{2.139071in}}%
\pgfpathlineto{\pgfqpoint{4.864437in}{2.139635in}}%
\pgfpathlineto{\pgfqpoint{4.864630in}{2.135900in}}%
\pgfpathlineto{\pgfqpoint{4.865783in}{2.139735in}}%
\pgfpathlineto{\pgfqpoint{4.866551in}{2.134884in}}%
\pgfpathlineto{\pgfqpoint{4.866936in}{2.138642in}}%
\pgfpathlineto{\pgfqpoint{4.867128in}{2.137674in}}%
\pgfpathlineto{\pgfqpoint{4.867512in}{2.141344in}}%
\pgfpathlineto{\pgfqpoint{4.867704in}{2.139067in}}%
\pgfpathlineto{\pgfqpoint{4.868857in}{2.144252in}}%
\pgfpathlineto{\pgfqpoint{4.869434in}{2.143111in}}%
\pgfpathlineto{\pgfqpoint{4.870010in}{2.134795in}}%
\pgfpathlineto{\pgfqpoint{4.870587in}{2.139654in}}%
\pgfpathlineto{\pgfqpoint{4.870971in}{2.138596in}}%
\pgfpathlineto{\pgfqpoint{4.871163in}{2.140514in}}%
\pgfpathlineto{\pgfqpoint{4.872893in}{2.150567in}}%
\pgfpathlineto{\pgfqpoint{4.873085in}{2.149843in}}%
\pgfpathlineto{\pgfqpoint{4.873662in}{2.145891in}}%
\pgfpathlineto{\pgfqpoint{4.874238in}{2.146568in}}%
\pgfpathlineto{\pgfqpoint{4.874430in}{2.146162in}}%
\pgfpathlineto{\pgfqpoint{4.874622in}{2.147221in}}%
\pgfpathlineto{\pgfqpoint{4.875199in}{2.146685in}}%
\pgfpathlineto{\pgfqpoint{4.875775in}{2.151404in}}%
\pgfpathlineto{\pgfqpoint{4.875968in}{2.151587in}}%
\pgfpathlineto{\pgfqpoint{4.876160in}{2.148346in}}%
\pgfpathlineto{\pgfqpoint{4.877121in}{2.149638in}}%
\pgfpathlineto{\pgfqpoint{4.878850in}{2.158850in}}%
\pgfpathlineto{\pgfqpoint{4.879811in}{2.154409in}}%
\pgfpathlineto{\pgfqpoint{4.880003in}{2.155977in}}%
\pgfpathlineto{\pgfqpoint{4.881348in}{2.165121in}}%
\pgfpathlineto{\pgfqpoint{4.881925in}{2.160702in}}%
\pgfpathlineto{\pgfqpoint{4.883270in}{2.154923in}}%
\pgfpathlineto{\pgfqpoint{4.883654in}{2.157607in}}%
\pgfpathlineto{\pgfqpoint{4.884423in}{2.161137in}}%
\pgfpathlineto{\pgfqpoint{4.884807in}{2.160948in}}%
\pgfpathlineto{\pgfqpoint{4.885768in}{2.158940in}}%
\pgfpathlineto{\pgfqpoint{4.885192in}{2.161191in}}%
\pgfpathlineto{\pgfqpoint{4.885960in}{2.159183in}}%
\pgfpathlineto{\pgfqpoint{4.886152in}{2.159430in}}%
\pgfpathlineto{\pgfqpoint{4.886921in}{2.154380in}}%
\pgfpathlineto{\pgfqpoint{4.887113in}{2.157737in}}%
\pgfpathlineto{\pgfqpoint{4.887305in}{2.160156in}}%
\pgfpathlineto{\pgfqpoint{4.888074in}{2.157093in}}%
\pgfpathlineto{\pgfqpoint{4.889419in}{2.154779in}}%
\pgfpathlineto{\pgfqpoint{4.890572in}{2.160118in}}%
\pgfpathlineto{\pgfqpoint{4.890957in}{2.159686in}}%
\pgfpathlineto{\pgfqpoint{4.893070in}{2.170343in}}%
\pgfpathlineto{\pgfqpoint{4.893263in}{2.167819in}}%
\pgfpathlineto{\pgfqpoint{4.893455in}{2.167717in}}%
\pgfpathlineto{\pgfqpoint{4.893647in}{2.168957in}}%
\pgfpathlineto{\pgfqpoint{4.894031in}{2.166716in}}%
\pgfpathlineto{\pgfqpoint{4.894223in}{2.166941in}}%
\pgfpathlineto{\pgfqpoint{4.894416in}{2.165582in}}%
\pgfpathlineto{\pgfqpoint{4.895184in}{2.167978in}}%
\pgfpathlineto{\pgfqpoint{4.896914in}{2.179379in}}%
\pgfpathlineto{\pgfqpoint{4.897298in}{2.177320in}}%
\pgfpathlineto{\pgfqpoint{4.898259in}{2.178994in}}%
\pgfpathlineto{\pgfqpoint{4.899796in}{2.169696in}}%
\pgfpathlineto{\pgfqpoint{4.900949in}{2.173974in}}%
\pgfpathlineto{\pgfqpoint{4.901142in}{2.170864in}}%
\pgfpathlineto{\pgfqpoint{4.902487in}{2.161246in}}%
\pgfpathlineto{\pgfqpoint{4.901718in}{2.171703in}}%
\pgfpathlineto{\pgfqpoint{4.903063in}{2.164661in}}%
\pgfpathlineto{\pgfqpoint{4.904601in}{2.157415in}}%
\pgfpathlineto{\pgfqpoint{4.904985in}{2.159827in}}%
\pgfpathlineto{\pgfqpoint{4.906138in}{2.163146in}}%
\pgfpathlineto{\pgfqpoint{4.905369in}{2.159479in}}%
\pgfpathlineto{\pgfqpoint{4.906330in}{2.161912in}}%
\pgfpathlineto{\pgfqpoint{4.907099in}{2.163300in}}%
\pgfpathlineto{\pgfqpoint{4.907483in}{2.159637in}}%
\pgfpathlineto{\pgfqpoint{4.907675in}{2.161324in}}%
\pgfpathlineto{\pgfqpoint{4.907867in}{2.158219in}}%
\pgfpathlineto{\pgfqpoint{4.908060in}{2.158578in}}%
\pgfpathlineto{\pgfqpoint{4.908252in}{2.156505in}}%
\pgfpathlineto{\pgfqpoint{4.908444in}{2.160646in}}%
\pgfpathlineto{\pgfqpoint{4.908828in}{2.160647in}}%
\pgfpathlineto{\pgfqpoint{4.909020in}{2.162025in}}%
\pgfpathlineto{\pgfqpoint{4.909405in}{2.158595in}}%
\pgfpathlineto{\pgfqpoint{4.910366in}{2.156925in}}%
\pgfpathlineto{\pgfqpoint{4.910558in}{2.157792in}}%
\pgfpathlineto{\pgfqpoint{4.910750in}{2.158477in}}%
\pgfpathlineto{\pgfqpoint{4.911134in}{2.156417in}}%
\pgfpathlineto{\pgfqpoint{4.911326in}{2.152231in}}%
\pgfpathlineto{\pgfqpoint{4.912095in}{2.155500in}}%
\pgfpathlineto{\pgfqpoint{4.912287in}{2.156866in}}%
\pgfpathlineto{\pgfqpoint{4.912864in}{2.153263in}}%
\pgfpathlineto{\pgfqpoint{4.913248in}{2.152906in}}%
\pgfpathlineto{\pgfqpoint{4.914017in}{2.158703in}}%
\pgfpathlineto{\pgfqpoint{4.914593in}{2.158315in}}%
\pgfpathlineto{\pgfqpoint{4.914785in}{2.156470in}}%
\pgfpathlineto{\pgfqpoint{4.915170in}{2.159424in}}%
\pgfpathlineto{\pgfqpoint{4.915362in}{2.159260in}}%
\pgfpathlineto{\pgfqpoint{4.916899in}{2.168440in}}%
\pgfpathlineto{\pgfqpoint{4.917091in}{2.168104in}}%
\pgfpathlineto{\pgfqpoint{4.918244in}{2.161321in}}%
\pgfpathlineto{\pgfqpoint{4.919205in}{2.163933in}}%
\pgfpathlineto{\pgfqpoint{4.919397in}{2.164152in}}%
\pgfpathlineto{\pgfqpoint{4.919974in}{2.155322in}}%
\pgfpathlineto{\pgfqpoint{4.920743in}{2.156650in}}%
\pgfpathlineto{\pgfqpoint{4.922664in}{2.169379in}}%
\pgfpathlineto{\pgfqpoint{4.923241in}{2.172735in}}%
\pgfpathlineto{\pgfqpoint{4.923817in}{2.169749in}}%
\pgfpathlineto{\pgfqpoint{4.924010in}{2.170551in}}%
\pgfpathlineto{\pgfqpoint{4.924202in}{2.168509in}}%
\pgfpathlineto{\pgfqpoint{4.924586in}{2.168805in}}%
\pgfpathlineto{\pgfqpoint{4.925163in}{2.162582in}}%
\pgfpathlineto{\pgfqpoint{4.925547in}{2.168022in}}%
\pgfpathlineto{\pgfqpoint{4.926508in}{2.173163in}}%
\pgfpathlineto{\pgfqpoint{4.926700in}{2.171464in}}%
\pgfpathlineto{\pgfqpoint{4.927084in}{2.168826in}}%
\pgfpathlineto{\pgfqpoint{4.927469in}{2.173243in}}%
\pgfpathlineto{\pgfqpoint{4.928237in}{2.171986in}}%
\pgfpathlineto{\pgfqpoint{4.928429in}{2.173671in}}%
\pgfpathlineto{\pgfqpoint{4.929390in}{2.178052in}}%
\pgfpathlineto{\pgfqpoint{4.930351in}{2.168631in}}%
\pgfpathlineto{\pgfqpoint{4.930543in}{2.171484in}}%
\pgfpathlineto{\pgfqpoint{4.932465in}{2.179515in}}%
\pgfpathlineto{\pgfqpoint{4.932657in}{2.178258in}}%
\pgfpathlineto{\pgfqpoint{4.932849in}{2.182059in}}%
\pgfpathlineto{\pgfqpoint{4.934194in}{2.186560in}}%
\pgfpathlineto{\pgfqpoint{4.934387in}{2.185455in}}%
\pgfpathlineto{\pgfqpoint{4.934579in}{2.189871in}}%
\pgfpathlineto{\pgfqpoint{4.934771in}{2.190024in}}%
\pgfpathlineto{\pgfqpoint{4.935924in}{2.195079in}}%
\pgfpathlineto{\pgfqpoint{4.936116in}{2.195042in}}%
\pgfpathlineto{\pgfqpoint{4.936885in}{2.196366in}}%
\pgfpathlineto{\pgfqpoint{4.937269in}{2.192699in}}%
\pgfpathlineto{\pgfqpoint{4.938038in}{2.195642in}}%
\pgfpathlineto{\pgfqpoint{4.938230in}{2.192566in}}%
\pgfpathlineto{\pgfqpoint{4.940728in}{2.175947in}}%
\pgfpathlineto{\pgfqpoint{4.941112in}{2.179072in}}%
\pgfpathlineto{\pgfqpoint{4.941497in}{2.176996in}}%
\pgfpathlineto{\pgfqpoint{4.943226in}{2.165507in}}%
\pgfpathlineto{\pgfqpoint{4.943418in}{2.166439in}}%
\pgfpathlineto{\pgfqpoint{4.943803in}{2.168998in}}%
\pgfpathlineto{\pgfqpoint{4.944187in}{2.165833in}}%
\pgfpathlineto{\pgfqpoint{4.944379in}{2.164404in}}%
\pgfpathlineto{\pgfqpoint{4.944956in}{2.167954in}}%
\pgfpathlineto{\pgfqpoint{4.946109in}{2.171545in}}%
\pgfpathlineto{\pgfqpoint{4.946493in}{2.169811in}}%
\pgfpathlineto{\pgfqpoint{4.946685in}{2.172824in}}%
\pgfpathlineto{\pgfqpoint{4.947262in}{2.170973in}}%
\pgfpathlineto{\pgfqpoint{4.947646in}{2.171225in}}%
\pgfpathlineto{\pgfqpoint{4.948415in}{2.173142in}}%
\pgfpathlineto{\pgfqpoint{4.948607in}{2.170195in}}%
\pgfpathlineto{\pgfqpoint{4.948799in}{2.170212in}}%
\pgfpathlineto{\pgfqpoint{4.949568in}{2.173181in}}%
\pgfpathlineto{\pgfqpoint{4.949952in}{2.171077in}}%
\pgfpathlineto{\pgfqpoint{4.951297in}{2.166997in}}%
\pgfpathlineto{\pgfqpoint{4.951874in}{2.166883in}}%
\pgfpathlineto{\pgfqpoint{4.952450in}{2.169957in}}%
\pgfpathlineto{\pgfqpoint{4.952835in}{2.167065in}}%
\pgfpathlineto{\pgfqpoint{4.953411in}{2.167902in}}%
\pgfpathlineto{\pgfqpoint{4.954372in}{2.174983in}}%
\pgfpathlineto{\pgfqpoint{4.954756in}{2.173633in}}%
\pgfpathlineto{\pgfqpoint{4.954949in}{2.173875in}}%
\pgfpathlineto{\pgfqpoint{4.955141in}{2.177444in}}%
\pgfpathlineto{\pgfqpoint{4.955717in}{2.169310in}}%
\pgfpathlineto{\pgfqpoint{4.956102in}{2.174780in}}%
\pgfpathlineto{\pgfqpoint{4.956294in}{2.178968in}}%
\pgfpathlineto{\pgfqpoint{4.956678in}{2.174471in}}%
\pgfpathlineto{\pgfqpoint{4.956870in}{2.174509in}}%
\pgfpathlineto{\pgfqpoint{4.957639in}{2.170157in}}%
\pgfpathlineto{\pgfqpoint{4.957831in}{2.172842in}}%
\pgfpathlineto{\pgfqpoint{4.958023in}{2.174261in}}%
\pgfpathlineto{\pgfqpoint{4.958408in}{2.170938in}}%
\pgfpathlineto{\pgfqpoint{4.959561in}{2.166405in}}%
\pgfpathlineto{\pgfqpoint{4.959753in}{2.169599in}}%
\pgfpathlineto{\pgfqpoint{4.960714in}{2.167211in}}%
\pgfpathlineto{\pgfqpoint{4.961098in}{2.165271in}}%
\pgfpathlineto{\pgfqpoint{4.961290in}{2.167331in}}%
\pgfpathlineto{\pgfqpoint{4.961867in}{2.166523in}}%
\pgfpathlineto{\pgfqpoint{4.962059in}{2.166386in}}%
\pgfpathlineto{\pgfqpoint{4.963404in}{2.178905in}}%
\pgfpathlineto{\pgfqpoint{4.964173in}{2.176893in}}%
\pgfpathlineto{\pgfqpoint{4.964557in}{2.174481in}}%
\pgfpathlineto{\pgfqpoint{4.965326in}{2.175104in}}%
\pgfpathlineto{\pgfqpoint{4.965710in}{2.174725in}}%
\pgfpathlineto{\pgfqpoint{4.966671in}{2.183492in}}%
\pgfpathlineto{\pgfqpoint{4.966863in}{2.186642in}}%
\pgfpathlineto{\pgfqpoint{4.967824in}{2.185943in}}%
\pgfpathlineto{\pgfqpoint{4.968208in}{2.179681in}}%
\pgfpathlineto{\pgfqpoint{4.968592in}{2.184506in}}%
\pgfpathlineto{\pgfqpoint{4.969169in}{2.191053in}}%
\pgfpathlineto{\pgfqpoint{4.969746in}{2.187767in}}%
\pgfpathlineto{\pgfqpoint{4.970130in}{2.183266in}}%
\pgfpathlineto{\pgfqpoint{4.970706in}{2.186397in}}%
\pgfpathlineto{\pgfqpoint{4.970899in}{2.187384in}}%
\pgfpathlineto{\pgfqpoint{4.971091in}{2.186140in}}%
\pgfpathlineto{\pgfqpoint{4.972628in}{2.173585in}}%
\pgfpathlineto{\pgfqpoint{4.972820in}{2.176006in}}%
\pgfpathlineto{\pgfqpoint{4.973589in}{2.172447in}}%
\pgfpathlineto{\pgfqpoint{4.973781in}{2.171713in}}%
\pgfpathlineto{\pgfqpoint{4.973973in}{2.173065in}}%
\pgfpathlineto{\pgfqpoint{4.974165in}{2.172546in}}%
\pgfpathlineto{\pgfqpoint{4.975126in}{2.182879in}}%
\pgfpathlineto{\pgfqpoint{4.975511in}{2.181302in}}%
\pgfpathlineto{\pgfqpoint{4.975895in}{2.178385in}}%
\pgfpathlineto{\pgfqpoint{4.976279in}{2.180722in}}%
\pgfpathlineto{\pgfqpoint{4.977817in}{2.192601in}}%
\pgfpathlineto{\pgfqpoint{4.980507in}{2.176092in}}%
\pgfpathlineto{\pgfqpoint{4.980891in}{2.184649in}}%
\pgfpathlineto{\pgfqpoint{4.981852in}{2.183630in}}%
\pgfpathlineto{\pgfqpoint{4.982236in}{2.187031in}}%
\pgfpathlineto{\pgfqpoint{4.983005in}{2.186335in}}%
\pgfpathlineto{\pgfqpoint{4.983774in}{2.183010in}}%
\pgfpathlineto{\pgfqpoint{4.984158in}{2.185510in}}%
\pgfpathlineto{\pgfqpoint{4.984350in}{2.185428in}}%
\pgfpathlineto{\pgfqpoint{4.984542in}{2.187908in}}%
\pgfpathlineto{\pgfqpoint{4.985119in}{2.182557in}}%
\pgfpathlineto{\pgfqpoint{4.985503in}{2.178698in}}%
\pgfpathlineto{\pgfqpoint{4.986080in}{2.181556in}}%
\pgfpathlineto{\pgfqpoint{4.986464in}{2.184345in}}%
\pgfpathlineto{\pgfqpoint{4.986848in}{2.181199in}}%
\pgfpathlineto{\pgfqpoint{4.987041in}{2.181487in}}%
\pgfpathlineto{\pgfqpoint{4.988194in}{2.173355in}}%
\pgfpathlineto{\pgfqpoint{4.988386in}{2.175246in}}%
\pgfpathlineto{\pgfqpoint{4.988962in}{2.178933in}}%
\pgfpathlineto{\pgfqpoint{4.989731in}{2.186300in}}%
\pgfpathlineto{\pgfqpoint{4.990307in}{2.184190in}}%
\pgfpathlineto{\pgfqpoint{4.990500in}{2.182319in}}%
\pgfpathlineto{\pgfqpoint{4.990884in}{2.187774in}}%
\pgfpathlineto{\pgfqpoint{4.991845in}{2.198487in}}%
\pgfpathlineto{\pgfqpoint{4.992229in}{2.197006in}}%
\pgfpathlineto{\pgfqpoint{4.993382in}{2.189440in}}%
\pgfpathlineto{\pgfqpoint{4.994151in}{2.190919in}}%
\pgfpathlineto{\pgfqpoint{4.995304in}{2.194430in}}%
\pgfpathlineto{\pgfqpoint{4.994535in}{2.189998in}}%
\pgfpathlineto{\pgfqpoint{4.995496in}{2.193026in}}%
\pgfpathlineto{\pgfqpoint{4.996457in}{2.195002in}}%
\pgfpathlineto{\pgfqpoint{4.996649in}{2.193422in}}%
\pgfpathlineto{\pgfqpoint{4.997994in}{2.190360in}}%
\pgfpathlineto{\pgfqpoint{4.998186in}{2.191494in}}%
\pgfpathlineto{\pgfqpoint{4.999916in}{2.203406in}}%
\pgfpathlineto{\pgfqpoint{4.998571in}{2.189583in}}%
\pgfpathlineto{\pgfqpoint{5.000108in}{2.202717in}}%
\pgfpathlineto{\pgfqpoint{5.000492in}{2.202406in}}%
\pgfpathlineto{\pgfqpoint{5.000685in}{2.204482in}}%
\pgfpathlineto{\pgfqpoint{5.001261in}{2.199239in}}%
\pgfpathlineto{\pgfqpoint{5.002222in}{2.191900in}}%
\pgfpathlineto{\pgfqpoint{5.002798in}{2.194171in}}%
\pgfpathlineto{\pgfqpoint{5.003183in}{2.195964in}}%
\pgfpathlineto{\pgfqpoint{5.005104in}{2.185745in}}%
\pgfpathlineto{\pgfqpoint{5.007603in}{2.194633in}}%
\pgfpathlineto{\pgfqpoint{5.007987in}{2.193579in}}%
\pgfpathlineto{\pgfqpoint{5.009332in}{2.185986in}}%
\pgfpathlineto{\pgfqpoint{5.009524in}{2.189052in}}%
\pgfpathlineto{\pgfqpoint{5.010101in}{2.192663in}}%
\pgfpathlineto{\pgfqpoint{5.010293in}{2.191183in}}%
\pgfpathlineto{\pgfqpoint{5.010677in}{2.187710in}}%
\pgfpathlineto{\pgfqpoint{5.011062in}{2.190749in}}%
\pgfpathlineto{\pgfqpoint{5.012215in}{2.195841in}}%
\pgfpathlineto{\pgfqpoint{5.012791in}{2.192041in}}%
\pgfpathlineto{\pgfqpoint{5.013175in}{2.192089in}}%
\pgfpathlineto{\pgfqpoint{5.014521in}{2.200765in}}%
\pgfpathlineto{\pgfqpoint{5.014713in}{2.200745in}}%
\pgfpathlineto{\pgfqpoint{5.015097in}{2.197706in}}%
\pgfpathlineto{\pgfqpoint{5.015866in}{2.200402in}}%
\pgfpathlineto{\pgfqpoint{5.016058in}{2.199113in}}%
\pgfpathlineto{\pgfqpoint{5.016442in}{2.201000in}}%
\pgfpathlineto{\pgfqpoint{5.016827in}{2.200439in}}%
\pgfpathlineto{\pgfqpoint{5.017788in}{2.208394in}}%
\pgfpathlineto{\pgfqpoint{5.018172in}{2.206185in}}%
\pgfpathlineto{\pgfqpoint{5.019325in}{2.212396in}}%
\pgfpathlineto{\pgfqpoint{5.020094in}{2.210844in}}%
\pgfpathlineto{\pgfqpoint{5.021054in}{2.212762in}}%
\pgfpathlineto{\pgfqpoint{5.022592in}{2.225603in}}%
\pgfpathlineto{\pgfqpoint{5.023168in}{2.223476in}}%
\pgfpathlineto{\pgfqpoint{5.023745in}{2.226724in}}%
\pgfpathlineto{\pgfqpoint{5.023937in}{2.226126in}}%
\pgfpathlineto{\pgfqpoint{5.025282in}{2.220227in}}%
\pgfpathlineto{\pgfqpoint{5.026435in}{2.214049in}}%
\pgfpathlineto{\pgfqpoint{5.026627in}{2.214676in}}%
\pgfpathlineto{\pgfqpoint{5.027396in}{2.217347in}}%
\pgfpathlineto{\pgfqpoint{5.027780in}{2.216896in}}%
\pgfpathlineto{\pgfqpoint{5.028741in}{2.217234in}}%
\pgfpathlineto{\pgfqpoint{5.028933in}{2.220474in}}%
\pgfpathlineto{\pgfqpoint{5.029894in}{2.219619in}}%
\pgfpathlineto{\pgfqpoint{5.030855in}{2.222365in}}%
\pgfpathlineto{\pgfqpoint{5.030278in}{2.218785in}}%
\pgfpathlineto{\pgfqpoint{5.031047in}{2.220887in}}%
\pgfpathlineto{\pgfqpoint{5.032392in}{2.216172in}}%
\pgfpathlineto{\pgfqpoint{5.034122in}{2.224703in}}%
\pgfpathlineto{\pgfqpoint{5.032969in}{2.215783in}}%
\pgfpathlineto{\pgfqpoint{5.034890in}{2.223086in}}%
\pgfpathlineto{\pgfqpoint{5.035467in}{2.218804in}}%
\pgfpathlineto{\pgfqpoint{5.036043in}{2.221078in}}%
\pgfpathlineto{\pgfqpoint{5.036236in}{2.220992in}}%
\pgfpathlineto{\pgfqpoint{5.038926in}{2.235250in}}%
\pgfpathlineto{\pgfqpoint{5.039118in}{2.234181in}}%
\pgfpathlineto{\pgfqpoint{5.039310in}{2.232866in}}%
\pgfpathlineto{\pgfqpoint{5.040079in}{2.233116in}}%
\pgfpathlineto{\pgfqpoint{5.040271in}{2.233991in}}%
\pgfpathlineto{\pgfqpoint{5.040463in}{2.231936in}}%
\pgfpathlineto{\pgfqpoint{5.041808in}{2.223937in}}%
\pgfpathlineto{\pgfqpoint{5.042577in}{2.234222in}}%
\pgfpathlineto{\pgfqpoint{5.043154in}{2.232058in}}%
\pgfpathlineto{\pgfqpoint{5.043538in}{2.233585in}}%
\pgfpathlineto{\pgfqpoint{5.043730in}{2.231500in}}%
\pgfpathlineto{\pgfqpoint{5.045268in}{2.217187in}}%
\pgfpathlineto{\pgfqpoint{5.045844in}{2.216344in}}%
\pgfpathlineto{\pgfqpoint{5.046613in}{2.220009in}}%
\pgfpathlineto{\pgfqpoint{5.046997in}{2.218938in}}%
\pgfpathlineto{\pgfqpoint{5.047381in}{2.215600in}}%
\pgfpathlineto{\pgfqpoint{5.047958in}{2.219213in}}%
\pgfpathlineto{\pgfqpoint{5.048342in}{2.222606in}}%
\pgfpathlineto{\pgfqpoint{5.048727in}{2.219680in}}%
\pgfpathlineto{\pgfqpoint{5.049111in}{2.215824in}}%
\pgfpathlineto{\pgfqpoint{5.049687in}{2.220912in}}%
\pgfpathlineto{\pgfqpoint{5.050648in}{2.226879in}}%
\pgfpathlineto{\pgfqpoint{5.051033in}{2.225548in}}%
\pgfpathlineto{\pgfqpoint{5.051609in}{2.218447in}}%
\pgfpathlineto{\pgfqpoint{5.052378in}{2.221055in}}%
\pgfpathlineto{\pgfqpoint{5.053339in}{2.226027in}}%
\pgfpathlineto{\pgfqpoint{5.052762in}{2.220490in}}%
\pgfpathlineto{\pgfqpoint{5.053915in}{2.224377in}}%
\pgfpathlineto{\pgfqpoint{5.054107in}{2.222852in}}%
\pgfpathlineto{\pgfqpoint{5.054299in}{2.224429in}}%
\pgfpathlineto{\pgfqpoint{5.055068in}{2.223748in}}%
\pgfpathlineto{\pgfqpoint{5.055452in}{2.222487in}}%
\pgfpathlineto{\pgfqpoint{5.055645in}{2.222935in}}%
\pgfpathlineto{\pgfqpoint{5.056221in}{2.222739in}}%
\pgfpathlineto{\pgfqpoint{5.056798in}{2.227069in}}%
\pgfpathlineto{\pgfqpoint{5.057758in}{2.224882in}}%
\pgfpathlineto{\pgfqpoint{5.057566in}{2.227538in}}%
\pgfpathlineto{\pgfqpoint{5.057951in}{2.226032in}}%
\pgfpathlineto{\pgfqpoint{5.058143in}{2.226004in}}%
\pgfpathlineto{\pgfqpoint{5.059104in}{2.221883in}}%
\pgfpathlineto{\pgfqpoint{5.059296in}{2.223640in}}%
\pgfpathlineto{\pgfqpoint{5.059488in}{2.225116in}}%
\pgfpathlineto{\pgfqpoint{5.059872in}{2.222597in}}%
\pgfpathlineto{\pgfqpoint{5.060064in}{2.223956in}}%
\pgfpathlineto{\pgfqpoint{5.060257in}{2.221580in}}%
\pgfpathlineto{\pgfqpoint{5.060449in}{2.225409in}}%
\pgfpathlineto{\pgfqpoint{5.060833in}{2.224496in}}%
\pgfpathlineto{\pgfqpoint{5.061410in}{2.229825in}}%
\pgfpathlineto{\pgfqpoint{5.061986in}{2.225764in}}%
\pgfpathlineto{\pgfqpoint{5.063331in}{2.218945in}}%
\pgfpathlineto{\pgfqpoint{5.063716in}{2.222414in}}%
\pgfpathlineto{\pgfqpoint{5.064484in}{2.219869in}}%
\pgfpathlineto{\pgfqpoint{5.064676in}{2.216190in}}%
\pgfpathlineto{\pgfqpoint{5.065445in}{2.221435in}}%
\pgfpathlineto{\pgfqpoint{5.065637in}{2.221872in}}%
\pgfpathlineto{\pgfqpoint{5.065829in}{2.220191in}}%
\pgfpathlineto{\pgfqpoint{5.067367in}{2.215912in}}%
\pgfpathlineto{\pgfqpoint{5.066214in}{2.220591in}}%
\pgfpathlineto{\pgfqpoint{5.067559in}{2.216911in}}%
\pgfpathlineto{\pgfqpoint{5.068328in}{2.230200in}}%
\pgfpathlineto{\pgfqpoint{5.068904in}{2.224064in}}%
\pgfpathlineto{\pgfqpoint{5.069865in}{2.231529in}}%
\pgfpathlineto{\pgfqpoint{5.070442in}{2.227543in}}%
\pgfpathlineto{\pgfqpoint{5.071018in}{2.229144in}}%
\pgfpathlineto{\pgfqpoint{5.071979in}{2.219741in}}%
\pgfpathlineto{\pgfqpoint{5.072171in}{2.219989in}}%
\pgfpathlineto{\pgfqpoint{5.073516in}{2.211522in}}%
\pgfpathlineto{\pgfqpoint{5.073708in}{2.212072in}}%
\pgfpathlineto{\pgfqpoint{5.075630in}{2.225488in}}%
\pgfpathlineto{\pgfqpoint{5.075822in}{2.225222in}}%
\pgfpathlineto{\pgfqpoint{5.076207in}{2.219157in}}%
\pgfpathlineto{\pgfqpoint{5.076975in}{2.222767in}}%
\pgfpathlineto{\pgfqpoint{5.077167in}{2.224492in}}%
\pgfpathlineto{\pgfqpoint{5.077360in}{2.219049in}}%
\pgfpathlineto{\pgfqpoint{5.077936in}{2.222399in}}%
\pgfpathlineto{\pgfqpoint{5.078320in}{2.220235in}}%
\pgfpathlineto{\pgfqpoint{5.078705in}{2.222983in}}%
\pgfpathlineto{\pgfqpoint{5.080626in}{2.237410in}}%
\pgfpathlineto{\pgfqpoint{5.080819in}{2.234567in}}%
\pgfpathlineto{\pgfqpoint{5.081203in}{2.239697in}}%
\pgfpathlineto{\pgfqpoint{5.081587in}{2.244540in}}%
\pgfpathlineto{\pgfqpoint{5.082356in}{2.241062in}}%
\pgfpathlineto{\pgfqpoint{5.083125in}{2.231739in}}%
\pgfpathlineto{\pgfqpoint{5.083893in}{2.235530in}}%
\pgfpathlineto{\pgfqpoint{5.084278in}{2.234237in}}%
\pgfpathlineto{\pgfqpoint{5.084662in}{2.236464in}}%
\pgfpathlineto{\pgfqpoint{5.085046in}{2.237784in}}%
\pgfpathlineto{\pgfqpoint{5.085238in}{2.239235in}}%
\pgfpathlineto{\pgfqpoint{5.085815in}{2.235704in}}%
\pgfpathlineto{\pgfqpoint{5.086776in}{2.231856in}}%
\pgfpathlineto{\pgfqpoint{5.086968in}{2.233380in}}%
\pgfpathlineto{\pgfqpoint{5.087352in}{2.234377in}}%
\pgfpathlineto{\pgfqpoint{5.087737in}{2.232918in}}%
\pgfpathlineto{\pgfqpoint{5.088313in}{2.227895in}}%
\pgfpathlineto{\pgfqpoint{5.088890in}{2.232122in}}%
\pgfpathlineto{\pgfqpoint{5.089658in}{2.238064in}}%
\pgfpathlineto{\pgfqpoint{5.090043in}{2.234751in}}%
\pgfpathlineto{\pgfqpoint{5.090235in}{2.233128in}}%
\pgfpathlineto{\pgfqpoint{5.090619in}{2.238083in}}%
\pgfpathlineto{\pgfqpoint{5.090811in}{2.237344in}}%
\pgfpathlineto{\pgfqpoint{5.091004in}{2.237117in}}%
\pgfpathlineto{\pgfqpoint{5.091196in}{2.234358in}}%
\pgfpathlineto{\pgfqpoint{5.091772in}{2.238925in}}%
\pgfpathlineto{\pgfqpoint{5.093694in}{2.247205in}}%
\pgfpathlineto{\pgfqpoint{5.093886in}{2.245577in}}%
\pgfpathlineto{\pgfqpoint{5.094078in}{2.243925in}}%
\pgfpathlineto{\pgfqpoint{5.094655in}{2.247786in}}%
\pgfpathlineto{\pgfqpoint{5.095039in}{2.249026in}}%
\pgfpathlineto{\pgfqpoint{5.095231in}{2.246823in}}%
\pgfpathlineto{\pgfqpoint{5.095616in}{2.247188in}}%
\pgfpathlineto{\pgfqpoint{5.097153in}{2.237151in}}%
\pgfpathlineto{\pgfqpoint{5.097345in}{2.237194in}}%
\pgfpathlineto{\pgfqpoint{5.098690in}{2.227563in}}%
\pgfpathlineto{\pgfqpoint{5.101188in}{2.237942in}}%
\pgfpathlineto{\pgfqpoint{5.099459in}{2.226092in}}%
\pgfpathlineto{\pgfqpoint{5.101765in}{2.236259in}}%
\pgfpathlineto{\pgfqpoint{5.102341in}{2.231716in}}%
\pgfpathlineto{\pgfqpoint{5.102918in}{2.234693in}}%
\pgfpathlineto{\pgfqpoint{5.103494in}{2.237726in}}%
\pgfpathlineto{\pgfqpoint{5.103879in}{2.235381in}}%
\pgfpathlineto{\pgfqpoint{5.104647in}{2.233949in}}%
\pgfpathlineto{\pgfqpoint{5.104263in}{2.236067in}}%
\pgfpathlineto{\pgfqpoint{5.104840in}{2.235548in}}%
\pgfpathlineto{\pgfqpoint{5.105993in}{2.245195in}}%
\pgfpathlineto{\pgfqpoint{5.106377in}{2.243145in}}%
\pgfpathlineto{\pgfqpoint{5.107146in}{2.235473in}}%
\pgfpathlineto{\pgfqpoint{5.107530in}{2.241057in}}%
\pgfpathlineto{\pgfqpoint{5.108299in}{2.249798in}}%
\pgfpathlineto{\pgfqpoint{5.108683in}{2.245958in}}%
\pgfpathlineto{\pgfqpoint{5.111565in}{2.232581in}}%
\pgfpathlineto{\pgfqpoint{5.112142in}{2.230445in}}%
\pgfpathlineto{\pgfqpoint{5.112718in}{2.236289in}}%
\pgfpathlineto{\pgfqpoint{5.113295in}{2.233791in}}%
\pgfpathlineto{\pgfqpoint{5.113487in}{2.233953in}}%
\pgfpathlineto{\pgfqpoint{5.113871in}{2.241340in}}%
\pgfpathlineto{\pgfqpoint{5.114832in}{2.239800in}}%
\pgfpathlineto{\pgfqpoint{5.115024in}{2.239625in}}%
\pgfpathlineto{\pgfqpoint{5.115217in}{2.237102in}}%
\pgfpathlineto{\pgfqpoint{5.115793in}{2.241416in}}%
\pgfpathlineto{\pgfqpoint{5.115985in}{2.238537in}}%
\pgfpathlineto{\pgfqpoint{5.116754in}{2.244646in}}%
\pgfpathlineto{\pgfqpoint{5.119060in}{2.256761in}}%
\pgfpathlineto{\pgfqpoint{5.120021in}{2.249753in}}%
\pgfpathlineto{\pgfqpoint{5.120213in}{2.251609in}}%
\pgfpathlineto{\pgfqpoint{5.121558in}{2.257138in}}%
\pgfpathlineto{\pgfqpoint{5.121750in}{2.255647in}}%
\pgfpathlineto{\pgfqpoint{5.121943in}{2.258372in}}%
\pgfpathlineto{\pgfqpoint{5.122135in}{2.258321in}}%
\pgfpathlineto{\pgfqpoint{5.123480in}{2.263736in}}%
\pgfpathlineto{\pgfqpoint{5.124056in}{2.262067in}}%
\pgfpathlineto{\pgfqpoint{5.124249in}{2.264032in}}%
\pgfpathlineto{\pgfqpoint{5.124633in}{2.269064in}}%
\pgfpathlineto{\pgfqpoint{5.125209in}{2.264198in}}%
\pgfpathlineto{\pgfqpoint{5.126747in}{2.252258in}}%
\pgfpathlineto{\pgfqpoint{5.127131in}{2.254897in}}%
\pgfpathlineto{\pgfqpoint{5.127900in}{2.250931in}}%
\pgfpathlineto{\pgfqpoint{5.128284in}{2.253856in}}%
\pgfpathlineto{\pgfqpoint{5.129821in}{2.261424in}}%
\pgfpathlineto{\pgfqpoint{5.129053in}{2.252260in}}%
\pgfpathlineto{\pgfqpoint{5.130206in}{2.258763in}}%
\pgfpathlineto{\pgfqpoint{5.130398in}{2.257992in}}%
\pgfpathlineto{\pgfqpoint{5.130782in}{2.259414in}}%
\pgfpathlineto{\pgfqpoint{5.130974in}{2.260316in}}%
\pgfpathlineto{\pgfqpoint{5.131359in}{2.257634in}}%
\pgfpathlineto{\pgfqpoint{5.131551in}{2.258035in}}%
\pgfpathlineto{\pgfqpoint{5.133280in}{2.252055in}}%
\pgfpathlineto{\pgfqpoint{5.133473in}{2.252409in}}%
\pgfpathlineto{\pgfqpoint{5.133665in}{2.257246in}}%
\pgfpathlineto{\pgfqpoint{5.134433in}{2.252582in}}%
\pgfpathlineto{\pgfqpoint{5.134818in}{2.251807in}}%
\pgfpathlineto{\pgfqpoint{5.135971in}{2.264165in}}%
\pgfpathlineto{\pgfqpoint{5.136355in}{2.259190in}}%
\pgfpathlineto{\pgfqpoint{5.137316in}{2.256327in}}%
\pgfpathlineto{\pgfqpoint{5.137508in}{2.257848in}}%
\pgfpathlineto{\pgfqpoint{5.137700in}{2.258677in}}%
\pgfpathlineto{\pgfqpoint{5.137892in}{2.254881in}}%
\pgfpathlineto{\pgfqpoint{5.138277in}{2.253610in}}%
\pgfpathlineto{\pgfqpoint{5.139622in}{2.266394in}}%
\pgfpathlineto{\pgfqpoint{5.139814in}{2.265805in}}%
\pgfpathlineto{\pgfqpoint{5.140006in}{2.265545in}}%
\pgfpathlineto{\pgfqpoint{5.140199in}{2.266766in}}%
\pgfpathlineto{\pgfqpoint{5.143850in}{2.284531in}}%
\pgfpathlineto{\pgfqpoint{5.144234in}{2.282035in}}%
\pgfpathlineto{\pgfqpoint{5.144426in}{2.279056in}}%
\pgfpathlineto{\pgfqpoint{5.145387in}{2.280281in}}%
\pgfpathlineto{\pgfqpoint{5.145579in}{2.279815in}}%
\pgfpathlineto{\pgfqpoint{5.146156in}{2.289769in}}%
\pgfpathlineto{\pgfqpoint{5.146732in}{2.285697in}}%
\pgfpathlineto{\pgfqpoint{5.148077in}{2.278795in}}%
\pgfpathlineto{\pgfqpoint{5.149615in}{2.286576in}}%
\pgfpathlineto{\pgfqpoint{5.149807in}{2.284913in}}%
\pgfpathlineto{\pgfqpoint{5.149999in}{2.283339in}}%
\pgfpathlineto{\pgfqpoint{5.150576in}{2.284160in}}%
\pgfpathlineto{\pgfqpoint{5.151729in}{2.293215in}}%
\pgfpathlineto{\pgfqpoint{5.151921in}{2.291270in}}%
\pgfpathlineto{\pgfqpoint{5.152113in}{2.289191in}}%
\pgfpathlineto{\pgfqpoint{5.152497in}{2.291695in}}%
\pgfpathlineto{\pgfqpoint{5.152689in}{2.290733in}}%
\pgfpathlineto{\pgfqpoint{5.153074in}{2.294999in}}%
\pgfpathlineto{\pgfqpoint{5.153650in}{2.291133in}}%
\pgfpathlineto{\pgfqpoint{5.154227in}{2.286953in}}%
\pgfpathlineto{\pgfqpoint{5.154803in}{2.289736in}}%
\pgfpathlineto{\pgfqpoint{5.155380in}{2.296034in}}%
\pgfpathlineto{\pgfqpoint{5.155956in}{2.293383in}}%
\pgfpathlineto{\pgfqpoint{5.156725in}{2.290095in}}%
\pgfpathlineto{\pgfqpoint{5.156533in}{2.293606in}}%
\pgfpathlineto{\pgfqpoint{5.156917in}{2.293334in}}%
\pgfpathlineto{\pgfqpoint{5.157301in}{2.297248in}}%
\pgfpathlineto{\pgfqpoint{5.158262in}{2.296458in}}%
\pgfpathlineto{\pgfqpoint{5.158454in}{2.297885in}}%
\pgfpathlineto{\pgfqpoint{5.158839in}{2.295670in}}%
\pgfpathlineto{\pgfqpoint{5.159031in}{2.295690in}}%
\pgfpathlineto{\pgfqpoint{5.159607in}{2.290280in}}%
\pgfpathlineto{\pgfqpoint{5.160184in}{2.293859in}}%
\pgfpathlineto{\pgfqpoint{5.161337in}{2.296394in}}%
\pgfpathlineto{\pgfqpoint{5.161529in}{2.296089in}}%
\pgfpathlineto{\pgfqpoint{5.163066in}{2.310515in}}%
\pgfpathlineto{\pgfqpoint{5.163835in}{2.310337in}}%
\pgfpathlineto{\pgfqpoint{5.164412in}{2.313645in}}%
\pgfpathlineto{\pgfqpoint{5.164604in}{2.312023in}}%
\pgfpathlineto{\pgfqpoint{5.165180in}{2.314656in}}%
\pgfpathlineto{\pgfqpoint{5.165565in}{2.312442in}}%
\pgfpathlineto{\pgfqpoint{5.165757in}{2.312819in}}%
\pgfpathlineto{\pgfqpoint{5.167102in}{2.317409in}}%
\pgfpathlineto{\pgfqpoint{5.167486in}{2.318009in}}%
\pgfpathlineto{\pgfqpoint{5.168255in}{2.315233in}}%
\pgfpathlineto{\pgfqpoint{5.169792in}{2.322825in}}%
\pgfpathlineto{\pgfqpoint{5.170369in}{2.318255in}}%
\pgfpathlineto{\pgfqpoint{5.170945in}{2.322086in}}%
\pgfpathlineto{\pgfqpoint{5.171138in}{2.322117in}}%
\pgfpathlineto{\pgfqpoint{5.171714in}{2.327056in}}%
\pgfpathlineto{\pgfqpoint{5.172098in}{2.322597in}}%
\pgfpathlineto{\pgfqpoint{5.172291in}{2.320836in}}%
\pgfpathlineto{\pgfqpoint{5.172867in}{2.323472in}}%
\pgfpathlineto{\pgfqpoint{5.173251in}{2.321132in}}%
\pgfpathlineto{\pgfqpoint{5.173828in}{2.325954in}}%
\pgfpathlineto{\pgfqpoint{5.174212in}{2.324140in}}%
\pgfpathlineto{\pgfqpoint{5.174404in}{2.320041in}}%
\pgfpathlineto{\pgfqpoint{5.175173in}{2.323283in}}%
\pgfpathlineto{\pgfqpoint{5.176518in}{2.328982in}}%
\pgfpathlineto{\pgfqpoint{5.176710in}{2.329873in}}%
\pgfpathlineto{\pgfqpoint{5.176903in}{2.328591in}}%
\pgfpathlineto{\pgfqpoint{5.177095in}{2.329540in}}%
\pgfpathlineto{\pgfqpoint{5.177287in}{2.326699in}}%
\pgfpathlineto{\pgfqpoint{5.178056in}{2.330607in}}%
\pgfpathlineto{\pgfqpoint{5.178632in}{2.330202in}}%
\pgfpathlineto{\pgfqpoint{5.180746in}{2.339115in}}%
\pgfpathlineto{\pgfqpoint{5.180938in}{2.333234in}}%
\pgfpathlineto{\pgfqpoint{5.181899in}{2.334884in}}%
\pgfpathlineto{\pgfqpoint{5.182091in}{2.335803in}}%
\pgfpathlineto{\pgfqpoint{5.182668in}{2.333270in}}%
\pgfpathlineto{\pgfqpoint{5.183052in}{2.330918in}}%
\pgfpathlineto{\pgfqpoint{5.183821in}{2.331758in}}%
\pgfpathlineto{\pgfqpoint{5.184013in}{2.331841in}}%
\pgfpathlineto{\pgfqpoint{5.184205in}{2.330945in}}%
\pgfpathlineto{\pgfqpoint{5.185166in}{2.334249in}}%
\pgfpathlineto{\pgfqpoint{5.185358in}{2.333427in}}%
\pgfpathlineto{\pgfqpoint{5.185742in}{2.335562in}}%
\pgfpathlineto{\pgfqpoint{5.186319in}{2.332820in}}%
\pgfpathlineto{\pgfqpoint{5.186703in}{2.331112in}}%
\pgfpathlineto{\pgfqpoint{5.187087in}{2.334049in}}%
\pgfpathlineto{\pgfqpoint{5.188240in}{2.344732in}}%
\pgfpathlineto{\pgfqpoint{5.189009in}{2.342247in}}%
\pgfpathlineto{\pgfqpoint{5.192084in}{2.321224in}}%
\pgfpathlineto{\pgfqpoint{5.193429in}{2.325848in}}%
\pgfpathlineto{\pgfqpoint{5.194774in}{2.323542in}}%
\pgfpathlineto{\pgfqpoint{5.194006in}{2.326260in}}%
\pgfpathlineto{\pgfqpoint{5.194966in}{2.324005in}}%
\pgfpathlineto{\pgfqpoint{5.197465in}{2.318933in}}%
\pgfpathlineto{\pgfqpoint{5.197657in}{2.319210in}}%
\pgfpathlineto{\pgfqpoint{5.198425in}{2.311813in}}%
\pgfpathlineto{\pgfqpoint{5.198810in}{2.314828in}}%
\pgfpathlineto{\pgfqpoint{5.199002in}{2.316487in}}%
\pgfpathlineto{\pgfqpoint{5.199771in}{2.316228in}}%
\pgfpathlineto{\pgfqpoint{5.200539in}{2.312635in}}%
\pgfpathlineto{\pgfqpoint{5.200924in}{2.313967in}}%
\pgfpathlineto{\pgfqpoint{5.203230in}{2.324081in}}%
\pgfpathlineto{\pgfqpoint{5.203422in}{2.321334in}}%
\pgfpathlineto{\pgfqpoint{5.204959in}{2.313316in}}%
\pgfpathlineto{\pgfqpoint{5.207457in}{2.321696in}}%
\pgfpathlineto{\pgfqpoint{5.207649in}{2.319415in}}%
\pgfpathlineto{\pgfqpoint{5.208034in}{2.318033in}}%
\pgfpathlineto{\pgfqpoint{5.208226in}{2.315922in}}%
\pgfpathlineto{\pgfqpoint{5.208610in}{2.320889in}}%
\pgfpathlineto{\pgfqpoint{5.208802in}{2.323628in}}%
\pgfpathlineto{\pgfqpoint{5.209187in}{2.320600in}}%
\pgfpathlineto{\pgfqpoint{5.209571in}{2.321438in}}%
\pgfpathlineto{\pgfqpoint{5.210724in}{2.318047in}}%
\pgfpathlineto{\pgfqpoint{5.210916in}{2.319340in}}%
\pgfpathlineto{\pgfqpoint{5.211301in}{2.315763in}}%
\pgfpathlineto{\pgfqpoint{5.211493in}{2.313145in}}%
\pgfpathlineto{\pgfqpoint{5.211877in}{2.320073in}}%
\pgfpathlineto{\pgfqpoint{5.212261in}{2.322970in}}%
\pgfpathlineto{\pgfqpoint{5.212646in}{2.318238in}}%
\pgfpathlineto{\pgfqpoint{5.213415in}{2.317176in}}%
\pgfpathlineto{\pgfqpoint{5.213030in}{2.318416in}}%
\pgfpathlineto{\pgfqpoint{5.213607in}{2.317709in}}%
\pgfpathlineto{\pgfqpoint{5.213991in}{2.320130in}}%
\pgfpathlineto{\pgfqpoint{5.214568in}{2.317270in}}%
\pgfpathlineto{\pgfqpoint{5.214760in}{2.319258in}}%
\pgfpathlineto{\pgfqpoint{5.214952in}{2.316899in}}%
\pgfpathlineto{\pgfqpoint{5.215528in}{2.319772in}}%
\pgfpathlineto{\pgfqpoint{5.215913in}{2.318699in}}%
\pgfpathlineto{\pgfqpoint{5.216297in}{2.318433in}}%
\pgfpathlineto{\pgfqpoint{5.217450in}{2.321004in}}%
\pgfpathlineto{\pgfqpoint{5.218219in}{2.314372in}}%
\pgfpathlineto{\pgfqpoint{5.218795in}{2.315461in}}%
\pgfpathlineto{\pgfqpoint{5.220525in}{2.322340in}}%
\pgfpathlineto{\pgfqpoint{5.221293in}{2.316665in}}%
\pgfpathlineto{\pgfqpoint{5.221870in}{2.319155in}}%
\pgfpathlineto{\pgfqpoint{5.222446in}{2.323412in}}%
\pgfpathlineto{\pgfqpoint{5.223023in}{2.321148in}}%
\pgfpathlineto{\pgfqpoint{5.225905in}{2.330398in}}%
\pgfpathlineto{\pgfqpoint{5.227635in}{2.321748in}}%
\pgfpathlineto{\pgfqpoint{5.228980in}{2.325756in}}%
\pgfpathlineto{\pgfqpoint{5.229172in}{2.323637in}}%
\pgfpathlineto{\pgfqpoint{5.229557in}{2.331176in}}%
\pgfpathlineto{\pgfqpoint{5.229941in}{2.330591in}}%
\pgfpathlineto{\pgfqpoint{5.230710in}{2.333498in}}%
\pgfpathlineto{\pgfqpoint{5.230902in}{2.333322in}}%
\pgfpathlineto{\pgfqpoint{5.232247in}{2.328658in}}%
\pgfpathlineto{\pgfqpoint{5.231478in}{2.333742in}}%
\pgfpathlineto{\pgfqpoint{5.232439in}{2.329139in}}%
\pgfpathlineto{\pgfqpoint{5.233208in}{2.334772in}}%
\pgfpathlineto{\pgfqpoint{5.233592in}{2.331463in}}%
\pgfpathlineto{\pgfqpoint{5.234745in}{2.324119in}}%
\pgfpathlineto{\pgfqpoint{5.234937in}{2.327419in}}%
\pgfpathlineto{\pgfqpoint{5.236475in}{2.340920in}}%
\pgfpathlineto{\pgfqpoint{5.237243in}{2.337966in}}%
\pgfpathlineto{\pgfqpoint{5.237435in}{2.338226in}}%
\pgfpathlineto{\pgfqpoint{5.237628in}{2.337903in}}%
\pgfpathlineto{\pgfqpoint{5.237820in}{2.334034in}}%
\pgfpathlineto{\pgfqpoint{5.238396in}{2.339152in}}%
\pgfpathlineto{\pgfqpoint{5.238589in}{2.341186in}}%
\pgfpathlineto{\pgfqpoint{5.238973in}{2.338507in}}%
\pgfpathlineto{\pgfqpoint{5.239742in}{2.331263in}}%
\pgfpathlineto{\pgfqpoint{5.240126in}{2.331722in}}%
\pgfpathlineto{\pgfqpoint{5.241471in}{2.335904in}}%
\pgfpathlineto{\pgfqpoint{5.240510in}{2.330785in}}%
\pgfpathlineto{\pgfqpoint{5.241663in}{2.335793in}}%
\pgfpathlineto{\pgfqpoint{5.243393in}{2.331911in}}%
\pgfpathlineto{\pgfqpoint{5.243585in}{2.333429in}}%
\pgfpathlineto{\pgfqpoint{5.244161in}{2.330910in}}%
\pgfpathlineto{\pgfqpoint{5.244546in}{2.332884in}}%
\pgfpathlineto{\pgfqpoint{5.245699in}{2.329152in}}%
\pgfpathlineto{\pgfqpoint{5.245891in}{2.332653in}}%
\pgfpathlineto{\pgfqpoint{5.246660in}{2.329187in}}%
\pgfpathlineto{\pgfqpoint{5.247044in}{2.327313in}}%
\pgfpathlineto{\pgfqpoint{5.247428in}{2.325889in}}%
\pgfpathlineto{\pgfqpoint{5.247813in}{2.328592in}}%
\pgfpathlineto{\pgfqpoint{5.248773in}{2.327951in}}%
\pgfpathlineto{\pgfqpoint{5.249350in}{2.334148in}}%
\pgfpathlineto{\pgfqpoint{5.250311in}{2.332149in}}%
\pgfpathlineto{\pgfqpoint{5.250695in}{2.333255in}}%
\pgfpathlineto{\pgfqpoint{5.251079in}{2.335587in}}%
\pgfpathlineto{\pgfqpoint{5.251272in}{2.333273in}}%
\pgfpathlineto{\pgfqpoint{5.251656in}{2.328581in}}%
\pgfpathlineto{\pgfqpoint{5.252040in}{2.335360in}}%
\pgfpathlineto{\pgfqpoint{5.252232in}{2.334033in}}%
\pgfpathlineto{\pgfqpoint{5.252809in}{2.336638in}}%
\pgfpathlineto{\pgfqpoint{5.253001in}{2.333966in}}%
\pgfpathlineto{\pgfqpoint{5.253385in}{2.334613in}}%
\pgfpathlineto{\pgfqpoint{5.256268in}{2.320751in}}%
\pgfpathlineto{\pgfqpoint{5.256844in}{2.320583in}}%
\pgfpathlineto{\pgfqpoint{5.257421in}{2.313012in}}%
\pgfpathlineto{\pgfqpoint{5.257805in}{2.317454in}}%
\pgfpathlineto{\pgfqpoint{5.258190in}{2.325132in}}%
\pgfpathlineto{\pgfqpoint{5.258958in}{2.318940in}}%
\pgfpathlineto{\pgfqpoint{5.259343in}{2.317499in}}%
\pgfpathlineto{\pgfqpoint{5.259727in}{2.320643in}}%
\pgfpathlineto{\pgfqpoint{5.259919in}{2.318463in}}%
\pgfpathlineto{\pgfqpoint{5.260688in}{2.324203in}}%
\pgfpathlineto{\pgfqpoint{5.261264in}{2.321555in}}%
\pgfpathlineto{\pgfqpoint{5.261456in}{2.320074in}}%
\pgfpathlineto{\pgfqpoint{5.261649in}{2.325730in}}%
\pgfpathlineto{\pgfqpoint{5.261841in}{2.324145in}}%
\pgfpathlineto{\pgfqpoint{5.262033in}{2.327188in}}%
\pgfpathlineto{\pgfqpoint{5.262610in}{2.321300in}}%
\pgfpathlineto{\pgfqpoint{5.262802in}{2.322859in}}%
\pgfpathlineto{\pgfqpoint{5.263570in}{2.326756in}}%
\pgfpathlineto{\pgfqpoint{5.263955in}{2.323407in}}%
\pgfpathlineto{\pgfqpoint{5.264916in}{2.318929in}}%
\pgfpathlineto{\pgfqpoint{5.265108in}{2.322667in}}%
\pgfpathlineto{\pgfqpoint{5.266069in}{2.329398in}}%
\pgfpathlineto{\pgfqpoint{5.267606in}{2.339534in}}%
\pgfpathlineto{\pgfqpoint{5.267990in}{2.336564in}}%
\pgfpathlineto{\pgfqpoint{5.270296in}{2.331384in}}%
\pgfpathlineto{\pgfqpoint{5.268375in}{2.337385in}}%
\pgfpathlineto{\pgfqpoint{5.270873in}{2.331671in}}%
\pgfpathlineto{\pgfqpoint{5.272602in}{2.342191in}}%
\pgfpathlineto{\pgfqpoint{5.273179in}{2.335230in}}%
\pgfpathlineto{\pgfqpoint{5.273947in}{2.337058in}}%
\pgfpathlineto{\pgfqpoint{5.274140in}{2.338906in}}%
\pgfpathlineto{\pgfqpoint{5.274716in}{2.334423in}}%
\pgfpathlineto{\pgfqpoint{5.274908in}{2.335967in}}%
\pgfpathlineto{\pgfqpoint{5.276253in}{2.344948in}}%
\pgfpathlineto{\pgfqpoint{5.276638in}{2.340022in}}%
\pgfpathlineto{\pgfqpoint{5.277406in}{2.337467in}}%
\pgfpathlineto{\pgfqpoint{5.277022in}{2.340411in}}%
\pgfpathlineto{\pgfqpoint{5.277599in}{2.338397in}}%
\pgfpathlineto{\pgfqpoint{5.278559in}{2.343318in}}%
\pgfpathlineto{\pgfqpoint{5.278752in}{2.339966in}}%
\pgfpathlineto{\pgfqpoint{5.278944in}{2.341081in}}%
\pgfpathlineto{\pgfqpoint{5.279136in}{2.339660in}}%
\pgfpathlineto{\pgfqpoint{5.279520in}{2.329611in}}%
\pgfpathlineto{\pgfqpoint{5.280289in}{2.335622in}}%
\pgfpathlineto{\pgfqpoint{5.280481in}{2.336348in}}%
\pgfpathlineto{\pgfqpoint{5.282018in}{2.322828in}}%
\pgfpathlineto{\pgfqpoint{5.282979in}{2.317259in}}%
\pgfpathlineto{\pgfqpoint{5.283364in}{2.319678in}}%
\pgfpathlineto{\pgfqpoint{5.286438in}{2.336645in}}%
\pgfpathlineto{\pgfqpoint{5.286631in}{2.331748in}}%
\pgfpathlineto{\pgfqpoint{5.287591in}{2.333417in}}%
\pgfpathlineto{\pgfqpoint{5.288937in}{2.337982in}}%
\pgfpathlineto{\pgfqpoint{5.289321in}{2.334813in}}%
\pgfpathlineto{\pgfqpoint{5.289897in}{2.339148in}}%
\pgfpathlineto{\pgfqpoint{5.291243in}{2.345463in}}%
\pgfpathlineto{\pgfqpoint{5.291435in}{2.345706in}}%
\pgfpathlineto{\pgfqpoint{5.291627in}{2.345491in}}%
\pgfpathlineto{\pgfqpoint{5.292203in}{2.343698in}}%
\pgfpathlineto{\pgfqpoint{5.292588in}{2.343839in}}%
\pgfpathlineto{\pgfqpoint{5.294317in}{2.355701in}}%
\pgfpathlineto{\pgfqpoint{5.294509in}{2.354851in}}%
\pgfpathlineto{\pgfqpoint{5.295855in}{2.342905in}}%
\pgfpathlineto{\pgfqpoint{5.296815in}{2.344800in}}%
\pgfpathlineto{\pgfqpoint{5.297008in}{2.344407in}}%
\pgfpathlineto{\pgfqpoint{5.297584in}{2.339174in}}%
\pgfpathlineto{\pgfqpoint{5.298161in}{2.342401in}}%
\pgfpathlineto{\pgfqpoint{5.299698in}{2.334632in}}%
\pgfpathlineto{\pgfqpoint{5.298545in}{2.343547in}}%
\pgfpathlineto{\pgfqpoint{5.300467in}{2.337516in}}%
\pgfpathlineto{\pgfqpoint{5.301812in}{2.346565in}}%
\pgfpathlineto{\pgfqpoint{5.302196in}{2.343286in}}%
\pgfpathlineto{\pgfqpoint{5.302580in}{2.341452in}}%
\pgfpathlineto{\pgfqpoint{5.302773in}{2.342811in}}%
\pgfpathlineto{\pgfqpoint{5.303926in}{2.347172in}}%
\pgfpathlineto{\pgfqpoint{5.303926in}{2.347172in}}%
\pgfusepath{stroke}%
\end{pgfscope}%
\begin{pgfscope}%
\pgfpathrectangle{\pgfqpoint{3.286364in}{0.660000in}}{\pgfqpoint{2.113636in}{2.100000in}}%
\pgfusepath{clip}%
\pgfsetroundcap%
\pgfsetroundjoin%
\pgfsetlinewidth{0.602250pt}%
\definecolor{currentstroke}{rgb}{0.894118,0.101961,0.109804}%
\pgfsetstrokecolor{currentstroke}%
\pgfsetdash{}{0pt}%
\pgfpathmoveto{\pgfqpoint{3.382438in}{1.863430in}}%
\pgfpathlineto{\pgfqpoint{3.383207in}{1.856351in}}%
\pgfpathlineto{\pgfqpoint{3.383591in}{1.861416in}}%
\pgfpathlineto{\pgfqpoint{3.385513in}{1.866591in}}%
\pgfpathlineto{\pgfqpoint{3.383975in}{1.860437in}}%
\pgfpathlineto{\pgfqpoint{3.385705in}{1.865583in}}%
\pgfpathlineto{\pgfqpoint{3.386281in}{1.860204in}}%
\pgfpathlineto{\pgfqpoint{3.386474in}{1.866645in}}%
\pgfpathlineto{\pgfqpoint{3.386858in}{1.864187in}}%
\pgfpathlineto{\pgfqpoint{3.387434in}{1.866607in}}%
\pgfpathlineto{\pgfqpoint{3.387819in}{1.866236in}}%
\pgfpathlineto{\pgfqpoint{3.388395in}{1.859308in}}%
\pgfpathlineto{\pgfqpoint{3.388972in}{1.860369in}}%
\pgfpathlineto{\pgfqpoint{3.389356in}{1.862154in}}%
\pgfpathlineto{\pgfqpoint{3.389548in}{1.858449in}}%
\pgfpathlineto{\pgfqpoint{3.389933in}{1.860851in}}%
\pgfpathlineto{\pgfqpoint{3.390701in}{1.857109in}}%
\pgfpathlineto{\pgfqpoint{3.391278in}{1.861379in}}%
\pgfpathlineto{\pgfqpoint{3.391662in}{1.863393in}}%
\pgfpathlineto{\pgfqpoint{3.392239in}{1.863056in}}%
\pgfpathlineto{\pgfqpoint{3.392623in}{1.860565in}}%
\pgfpathlineto{\pgfqpoint{3.393007in}{1.861971in}}%
\pgfpathlineto{\pgfqpoint{3.393584in}{1.861533in}}%
\pgfpathlineto{\pgfqpoint{3.394352in}{1.866761in}}%
\pgfpathlineto{\pgfqpoint{3.394545in}{1.867576in}}%
\pgfpathlineto{\pgfqpoint{3.394737in}{1.866287in}}%
\pgfpathlineto{\pgfqpoint{3.395121in}{1.866497in}}%
\pgfpathlineto{\pgfqpoint{3.396658in}{1.856577in}}%
\pgfpathlineto{\pgfqpoint{3.396851in}{1.857445in}}%
\pgfpathlineto{\pgfqpoint{3.397043in}{1.858110in}}%
\pgfpathlineto{\pgfqpoint{3.397235in}{1.855407in}}%
\pgfpathlineto{\pgfqpoint{3.397427in}{1.854672in}}%
\pgfpathlineto{\pgfqpoint{3.398004in}{1.856277in}}%
\pgfpathlineto{\pgfqpoint{3.398388in}{1.855653in}}%
\pgfpathlineto{\pgfqpoint{3.399157in}{1.858916in}}%
\pgfpathlineto{\pgfqpoint{3.400694in}{1.849138in}}%
\pgfpathlineto{\pgfqpoint{3.402616in}{1.856027in}}%
\pgfpathlineto{\pgfqpoint{3.402808in}{1.851946in}}%
\pgfpathlineto{\pgfqpoint{3.403769in}{1.853964in}}%
\pgfpathlineto{\pgfqpoint{3.403961in}{1.853653in}}%
\pgfpathlineto{\pgfqpoint{3.404537in}{1.859980in}}%
\pgfpathlineto{\pgfqpoint{3.404922in}{1.853511in}}%
\pgfpathlineto{\pgfqpoint{3.405114in}{1.850532in}}%
\pgfpathlineto{\pgfqpoint{3.405498in}{1.854141in}}%
\pgfpathlineto{\pgfqpoint{3.405690in}{1.853881in}}%
\pgfpathlineto{\pgfqpoint{3.407228in}{1.860911in}}%
\pgfpathlineto{\pgfqpoint{3.407420in}{1.859517in}}%
\pgfpathlineto{\pgfqpoint{3.407804in}{1.862638in}}%
\pgfpathlineto{\pgfqpoint{3.408189in}{1.861033in}}%
\pgfpathlineto{\pgfqpoint{3.408957in}{1.863334in}}%
\pgfpathlineto{\pgfqpoint{3.409342in}{1.861928in}}%
\pgfpathlineto{\pgfqpoint{3.409534in}{1.862359in}}%
\pgfpathlineto{\pgfqpoint{3.410495in}{1.851695in}}%
\pgfpathlineto{\pgfqpoint{3.410687in}{1.857123in}}%
\pgfpathlineto{\pgfqpoint{3.412032in}{1.862157in}}%
\pgfpathlineto{\pgfqpoint{3.412416in}{1.862527in}}%
\pgfpathlineto{\pgfqpoint{3.412801in}{1.867188in}}%
\pgfpathlineto{\pgfqpoint{3.413185in}{1.858367in}}%
\pgfpathlineto{\pgfqpoint{3.413377in}{1.858552in}}%
\pgfpathlineto{\pgfqpoint{3.413954in}{1.861713in}}%
\pgfpathlineto{\pgfqpoint{3.414338in}{1.857317in}}%
\pgfpathlineto{\pgfqpoint{3.414530in}{1.859357in}}%
\pgfpathlineto{\pgfqpoint{3.414722in}{1.858432in}}%
\pgfpathlineto{\pgfqpoint{3.414914in}{1.861005in}}%
\pgfpathlineto{\pgfqpoint{3.416260in}{1.864690in}}%
\pgfpathlineto{\pgfqpoint{3.415683in}{1.860142in}}%
\pgfpathlineto{\pgfqpoint{3.416452in}{1.864120in}}%
\pgfpathlineto{\pgfqpoint{3.417797in}{1.869004in}}%
\pgfpathlineto{\pgfqpoint{3.418566in}{1.868575in}}%
\pgfpathlineto{\pgfqpoint{3.419526in}{1.864294in}}%
\pgfpathlineto{\pgfqpoint{3.419719in}{1.864862in}}%
\pgfpathlineto{\pgfqpoint{3.420295in}{1.871621in}}%
\pgfpathlineto{\pgfqpoint{3.420872in}{1.866915in}}%
\pgfpathlineto{\pgfqpoint{3.421640in}{1.876859in}}%
\pgfpathlineto{\pgfqpoint{3.422217in}{1.869404in}}%
\pgfpathlineto{\pgfqpoint{3.422409in}{1.868744in}}%
\pgfpathlineto{\pgfqpoint{3.422793in}{1.870543in}}%
\pgfpathlineto{\pgfqpoint{3.422985in}{1.870077in}}%
\pgfpathlineto{\pgfqpoint{3.425676in}{1.882722in}}%
\pgfpathlineto{\pgfqpoint{3.425868in}{1.881648in}}%
\pgfpathlineto{\pgfqpoint{3.426252in}{1.879914in}}%
\pgfpathlineto{\pgfqpoint{3.426444in}{1.879351in}}%
\pgfpathlineto{\pgfqpoint{3.426637in}{1.880926in}}%
\pgfpathlineto{\pgfqpoint{3.427790in}{1.885238in}}%
\pgfpathlineto{\pgfqpoint{3.427021in}{1.878823in}}%
\pgfpathlineto{\pgfqpoint{3.428174in}{1.883408in}}%
\pgfpathlineto{\pgfqpoint{3.429135in}{1.880471in}}%
\pgfpathlineto{\pgfqpoint{3.429327in}{1.882879in}}%
\pgfpathlineto{\pgfqpoint{3.430288in}{1.888220in}}%
\pgfpathlineto{\pgfqpoint{3.429711in}{1.881580in}}%
\pgfpathlineto{\pgfqpoint{3.430864in}{1.884456in}}%
\pgfpathlineto{\pgfqpoint{3.432210in}{1.878075in}}%
\pgfpathlineto{\pgfqpoint{3.432402in}{1.875675in}}%
\pgfpathlineto{\pgfqpoint{3.433170in}{1.879965in}}%
\pgfpathlineto{\pgfqpoint{3.433363in}{1.879471in}}%
\pgfpathlineto{\pgfqpoint{3.433555in}{1.880915in}}%
\pgfpathlineto{\pgfqpoint{3.433747in}{1.882119in}}%
\pgfpathlineto{\pgfqpoint{3.434131in}{1.880375in}}%
\pgfpathlineto{\pgfqpoint{3.434900in}{1.877261in}}%
\pgfpathlineto{\pgfqpoint{3.435284in}{1.878732in}}%
\pgfpathlineto{\pgfqpoint{3.436629in}{1.888740in}}%
\pgfpathlineto{\pgfqpoint{3.436822in}{1.887065in}}%
\pgfpathlineto{\pgfqpoint{3.438167in}{1.877422in}}%
\pgfpathlineto{\pgfqpoint{3.439512in}{1.879634in}}%
\pgfpathlineto{\pgfqpoint{3.439704in}{1.881061in}}%
\pgfpathlineto{\pgfqpoint{3.440088in}{1.876999in}}%
\pgfpathlineto{\pgfqpoint{3.441241in}{1.865921in}}%
\pgfpathlineto{\pgfqpoint{3.441818in}{1.869005in}}%
\pgfpathlineto{\pgfqpoint{3.443355in}{1.875916in}}%
\pgfpathlineto{\pgfqpoint{3.443547in}{1.876691in}}%
\pgfpathlineto{\pgfqpoint{3.443932in}{1.874575in}}%
\pgfpathlineto{\pgfqpoint{3.445085in}{1.867927in}}%
\pgfpathlineto{\pgfqpoint{3.445661in}{1.868655in}}%
\pgfpathlineto{\pgfqpoint{3.446046in}{1.871869in}}%
\pgfpathlineto{\pgfqpoint{3.446430in}{1.868444in}}%
\pgfpathlineto{\pgfqpoint{3.447967in}{1.860997in}}%
\pgfpathlineto{\pgfqpoint{3.448159in}{1.863832in}}%
\pgfpathlineto{\pgfqpoint{3.448928in}{1.860982in}}%
\pgfpathlineto{\pgfqpoint{3.449120in}{1.859702in}}%
\pgfpathlineto{\pgfqpoint{3.449312in}{1.861917in}}%
\pgfpathlineto{\pgfqpoint{3.449505in}{1.861716in}}%
\pgfpathlineto{\pgfqpoint{3.450850in}{1.870255in}}%
\pgfpathlineto{\pgfqpoint{3.451618in}{1.864910in}}%
\pgfpathlineto{\pgfqpoint{3.451811in}{1.868502in}}%
\pgfpathlineto{\pgfqpoint{3.452771in}{1.872415in}}%
\pgfpathlineto{\pgfqpoint{3.453348in}{1.867743in}}%
\pgfpathlineto{\pgfqpoint{3.453732in}{1.871330in}}%
\pgfpathlineto{\pgfqpoint{3.454885in}{1.878042in}}%
\pgfpathlineto{\pgfqpoint{3.455078in}{1.875282in}}%
\pgfpathlineto{\pgfqpoint{3.458152in}{1.864834in}}%
\pgfpathlineto{\pgfqpoint{3.461419in}{1.878694in}}%
\pgfpathlineto{\pgfqpoint{3.462188in}{1.874890in}}%
\pgfpathlineto{\pgfqpoint{3.462956in}{1.875173in}}%
\pgfpathlineto{\pgfqpoint{3.463149in}{1.874875in}}%
\pgfpathlineto{\pgfqpoint{3.463341in}{1.876233in}}%
\pgfpathlineto{\pgfqpoint{3.464686in}{1.883047in}}%
\pgfpathlineto{\pgfqpoint{3.465070in}{1.882762in}}%
\pgfpathlineto{\pgfqpoint{3.466031in}{1.879528in}}%
\pgfpathlineto{\pgfqpoint{3.466223in}{1.880415in}}%
\pgfpathlineto{\pgfqpoint{3.467761in}{1.886320in}}%
\pgfpathlineto{\pgfqpoint{3.468145in}{1.882667in}}%
\pgfpathlineto{\pgfqpoint{3.468337in}{1.880907in}}%
\pgfpathlineto{\pgfqpoint{3.468914in}{1.884475in}}%
\pgfpathlineto{\pgfqpoint{3.470835in}{1.895204in}}%
\pgfpathlineto{\pgfqpoint{3.469298in}{1.883267in}}%
\pgfpathlineto{\pgfqpoint{3.471027in}{1.893533in}}%
\pgfpathlineto{\pgfqpoint{3.471220in}{1.890527in}}%
\pgfpathlineto{\pgfqpoint{3.471796in}{1.894804in}}%
\pgfpathlineto{\pgfqpoint{3.473526in}{1.902036in}}%
\pgfpathlineto{\pgfqpoint{3.473718in}{1.900245in}}%
\pgfpathlineto{\pgfqpoint{3.473910in}{1.900026in}}%
\pgfpathlineto{\pgfqpoint{3.474102in}{1.900274in}}%
\pgfpathlineto{\pgfqpoint{3.475639in}{1.905625in}}%
\pgfpathlineto{\pgfqpoint{3.474679in}{1.898577in}}%
\pgfpathlineto{\pgfqpoint{3.475832in}{1.904046in}}%
\pgfpathlineto{\pgfqpoint{3.476600in}{1.900344in}}%
\pgfpathlineto{\pgfqpoint{3.476985in}{1.902735in}}%
\pgfpathlineto{\pgfqpoint{3.477561in}{1.905715in}}%
\pgfpathlineto{\pgfqpoint{3.477753in}{1.902488in}}%
\pgfpathlineto{\pgfqpoint{3.478330in}{1.902722in}}%
\pgfpathlineto{\pgfqpoint{3.478906in}{1.898953in}}%
\pgfpathlineto{\pgfqpoint{3.480828in}{1.906270in}}%
\pgfpathlineto{\pgfqpoint{3.482942in}{1.894377in}}%
\pgfpathlineto{\pgfqpoint{3.483134in}{1.895625in}}%
\pgfpathlineto{\pgfqpoint{3.483518in}{1.898181in}}%
\pgfpathlineto{\pgfqpoint{3.483903in}{1.892539in}}%
\pgfpathlineto{\pgfqpoint{3.484095in}{1.891708in}}%
\pgfpathlineto{\pgfqpoint{3.484287in}{1.893198in}}%
\pgfpathlineto{\pgfqpoint{3.486017in}{1.902151in}}%
\pgfpathlineto{\pgfqpoint{3.487554in}{1.907588in}}%
\pgfpathlineto{\pgfqpoint{3.487938in}{1.904580in}}%
\pgfpathlineto{\pgfqpoint{3.488130in}{1.902636in}}%
\pgfpathlineto{\pgfqpoint{3.488515in}{1.907436in}}%
\pgfpathlineto{\pgfqpoint{3.489860in}{1.909886in}}%
\pgfpathlineto{\pgfqpoint{3.492166in}{1.899553in}}%
\pgfpathlineto{\pgfqpoint{3.492935in}{1.906529in}}%
\pgfpathlineto{\pgfqpoint{3.493319in}{1.903384in}}%
\pgfpathlineto{\pgfqpoint{3.494088in}{1.908812in}}%
\pgfpathlineto{\pgfqpoint{3.494280in}{1.912662in}}%
\pgfpathlineto{\pgfqpoint{3.495048in}{1.906188in}}%
\pgfpathlineto{\pgfqpoint{3.495241in}{1.904422in}}%
\pgfpathlineto{\pgfqpoint{3.495625in}{1.908268in}}%
\pgfpathlineto{\pgfqpoint{3.495817in}{1.908152in}}%
\pgfpathlineto{\pgfqpoint{3.496394in}{1.905906in}}%
\pgfpathlineto{\pgfqpoint{3.497162in}{1.913555in}}%
\pgfpathlineto{\pgfqpoint{3.497931in}{1.909796in}}%
\pgfpathlineto{\pgfqpoint{3.498315in}{1.909851in}}%
\pgfpathlineto{\pgfqpoint{3.498507in}{1.910858in}}%
\pgfpathlineto{\pgfqpoint{3.498892in}{1.907605in}}%
\pgfpathlineto{\pgfqpoint{3.499084in}{1.907416in}}%
\pgfpathlineto{\pgfqpoint{3.499276in}{1.908520in}}%
\pgfpathlineto{\pgfqpoint{3.499853in}{1.910924in}}%
\pgfpathlineto{\pgfqpoint{3.500045in}{1.907770in}}%
\pgfpathlineto{\pgfqpoint{3.500621in}{1.910497in}}%
\pgfpathlineto{\pgfqpoint{3.501006in}{1.907650in}}%
\pgfpathlineto{\pgfqpoint{3.501390in}{1.911712in}}%
\pgfpathlineto{\pgfqpoint{3.501582in}{1.910352in}}%
\pgfpathlineto{\pgfqpoint{3.503696in}{1.919911in}}%
\pgfpathlineto{\pgfqpoint{3.504080in}{1.918326in}}%
\pgfpathlineto{\pgfqpoint{3.504273in}{1.915624in}}%
\pgfpathlineto{\pgfqpoint{3.504465in}{1.919276in}}%
\pgfpathlineto{\pgfqpoint{3.505233in}{1.917089in}}%
\pgfpathlineto{\pgfqpoint{3.505810in}{1.921873in}}%
\pgfpathlineto{\pgfqpoint{3.506386in}{1.918572in}}%
\pgfpathlineto{\pgfqpoint{3.506579in}{1.917722in}}%
\pgfpathlineto{\pgfqpoint{3.506771in}{1.920258in}}%
\pgfpathlineto{\pgfqpoint{3.506963in}{1.919042in}}%
\pgfpathlineto{\pgfqpoint{3.507732in}{1.924411in}}%
\pgfpathlineto{\pgfqpoint{3.508308in}{1.923497in}}%
\pgfpathlineto{\pgfqpoint{3.508885in}{1.921139in}}%
\pgfpathlineto{\pgfqpoint{3.509269in}{1.923524in}}%
\pgfpathlineto{\pgfqpoint{3.510422in}{1.927720in}}%
\pgfpathlineto{\pgfqpoint{3.510806in}{1.927476in}}%
\pgfpathlineto{\pgfqpoint{3.512151in}{1.920418in}}%
\pgfpathlineto{\pgfqpoint{3.512344in}{1.921019in}}%
\pgfpathlineto{\pgfqpoint{3.512920in}{1.930034in}}%
\pgfpathlineto{\pgfqpoint{3.513689in}{1.929901in}}%
\pgfpathlineto{\pgfqpoint{3.514073in}{1.931991in}}%
\pgfpathlineto{\pgfqpoint{3.514650in}{1.930744in}}%
\pgfpathlineto{\pgfqpoint{3.515034in}{1.928344in}}%
\pgfpathlineto{\pgfqpoint{3.515803in}{1.929880in}}%
\pgfpathlineto{\pgfqpoint{3.515995in}{1.929919in}}%
\pgfpathlineto{\pgfqpoint{3.517916in}{1.939023in}}%
\pgfpathlineto{\pgfqpoint{3.518301in}{1.936260in}}%
\pgfpathlineto{\pgfqpoint{3.518685in}{1.941024in}}%
\pgfpathlineto{\pgfqpoint{3.519646in}{1.947887in}}%
\pgfpathlineto{\pgfqpoint{3.520030in}{1.945812in}}%
\pgfpathlineto{\pgfqpoint{3.520222in}{1.945710in}}%
\pgfpathlineto{\pgfqpoint{3.521183in}{1.952000in}}%
\pgfpathlineto{\pgfqpoint{3.521568in}{1.948669in}}%
\pgfpathlineto{\pgfqpoint{3.522721in}{1.944422in}}%
\pgfpathlineto{\pgfqpoint{3.521952in}{1.949674in}}%
\pgfpathlineto{\pgfqpoint{3.523681in}{1.945129in}}%
\pgfpathlineto{\pgfqpoint{3.523874in}{1.947585in}}%
\pgfpathlineto{\pgfqpoint{3.524450in}{1.944308in}}%
\pgfpathlineto{\pgfqpoint{3.524834in}{1.945778in}}%
\pgfpathlineto{\pgfqpoint{3.525027in}{1.943979in}}%
\pgfpathlineto{\pgfqpoint{3.525603in}{1.948496in}}%
\pgfpathlineto{\pgfqpoint{3.525795in}{1.948464in}}%
\pgfpathlineto{\pgfqpoint{3.527141in}{1.959423in}}%
\pgfpathlineto{\pgfqpoint{3.529831in}{1.948535in}}%
\pgfpathlineto{\pgfqpoint{3.530023in}{1.948740in}}%
\pgfpathlineto{\pgfqpoint{3.530215in}{1.952244in}}%
\pgfpathlineto{\pgfqpoint{3.530792in}{1.946710in}}%
\pgfpathlineto{\pgfqpoint{3.530984in}{1.947536in}}%
\pgfpathlineto{\pgfqpoint{3.531176in}{1.947934in}}%
\pgfpathlineto{\pgfqpoint{3.532906in}{1.938264in}}%
\pgfpathlineto{\pgfqpoint{3.535788in}{1.948381in}}%
\pgfpathlineto{\pgfqpoint{3.535980in}{1.946399in}}%
\pgfpathlineto{\pgfqpoint{3.536172in}{1.946336in}}%
\pgfpathlineto{\pgfqpoint{3.538094in}{1.939235in}}%
\pgfpathlineto{\pgfqpoint{3.538286in}{1.939309in}}%
\pgfpathlineto{\pgfqpoint{3.541169in}{1.958145in}}%
\pgfpathlineto{\pgfqpoint{3.541937in}{1.950726in}}%
\pgfpathlineto{\pgfqpoint{3.542898in}{1.954180in}}%
\pgfpathlineto{\pgfqpoint{3.543283in}{1.955667in}}%
\pgfpathlineto{\pgfqpoint{3.543667in}{1.953589in}}%
\pgfpathlineto{\pgfqpoint{3.543859in}{1.952352in}}%
\pgfpathlineto{\pgfqpoint{3.544051in}{1.954655in}}%
\pgfpathlineto{\pgfqpoint{3.544243in}{1.952691in}}%
\pgfpathlineto{\pgfqpoint{3.545012in}{1.956905in}}%
\pgfpathlineto{\pgfqpoint{3.545396in}{1.955556in}}%
\pgfpathlineto{\pgfqpoint{3.546742in}{1.950456in}}%
\pgfpathlineto{\pgfqpoint{3.548855in}{1.968021in}}%
\pgfpathlineto{\pgfqpoint{3.549240in}{1.965289in}}%
\pgfpathlineto{\pgfqpoint{3.549624in}{1.967690in}}%
\pgfpathlineto{\pgfqpoint{3.551546in}{1.978532in}}%
\pgfpathlineto{\pgfqpoint{3.552507in}{1.969847in}}%
\pgfpathlineto{\pgfqpoint{3.552891in}{1.971814in}}%
\pgfpathlineto{\pgfqpoint{3.553083in}{1.970031in}}%
\pgfpathlineto{\pgfqpoint{3.553852in}{1.972674in}}%
\pgfpathlineto{\pgfqpoint{3.554044in}{1.970167in}}%
\pgfpathlineto{\pgfqpoint{3.554236in}{1.971683in}}%
\pgfpathlineto{\pgfqpoint{3.554813in}{1.968617in}}%
\pgfpathlineto{\pgfqpoint{3.555005in}{1.971550in}}%
\pgfpathlineto{\pgfqpoint{3.555389in}{1.966771in}}%
\pgfpathlineto{\pgfqpoint{3.555966in}{1.970916in}}%
\pgfpathlineto{\pgfqpoint{3.557887in}{1.975198in}}%
\pgfpathlineto{\pgfqpoint{3.558464in}{1.973914in}}%
\pgfpathlineto{\pgfqpoint{3.560386in}{1.983573in}}%
\pgfpathlineto{\pgfqpoint{3.561539in}{1.980475in}}%
\pgfpathlineto{\pgfqpoint{3.561731in}{1.982313in}}%
\pgfpathlineto{\pgfqpoint{3.561923in}{1.982684in}}%
\pgfpathlineto{\pgfqpoint{3.563652in}{1.967724in}}%
\pgfpathlineto{\pgfqpoint{3.563845in}{1.968669in}}%
\pgfpathlineto{\pgfqpoint{3.566727in}{1.989067in}}%
\pgfpathlineto{\pgfqpoint{3.566919in}{1.988336in}}%
\pgfpathlineto{\pgfqpoint{3.569033in}{1.976153in}}%
\pgfpathlineto{\pgfqpoint{3.569417in}{1.973682in}}%
\pgfpathlineto{\pgfqpoint{3.569994in}{1.969394in}}%
\pgfpathlineto{\pgfqpoint{3.570186in}{1.972174in}}%
\pgfpathlineto{\pgfqpoint{3.571531in}{1.977812in}}%
\pgfpathlineto{\pgfqpoint{3.571723in}{1.977731in}}%
\pgfpathlineto{\pgfqpoint{3.573261in}{1.972049in}}%
\pgfpathlineto{\pgfqpoint{3.573453in}{1.977257in}}%
\pgfpathlineto{\pgfqpoint{3.574414in}{1.975148in}}%
\pgfpathlineto{\pgfqpoint{3.575375in}{1.967185in}}%
\pgfpathlineto{\pgfqpoint{3.575567in}{1.969543in}}%
\pgfpathlineto{\pgfqpoint{3.576143in}{1.975046in}}%
\pgfpathlineto{\pgfqpoint{3.576912in}{1.973165in}}%
\pgfpathlineto{\pgfqpoint{3.577104in}{1.973306in}}%
\pgfpathlineto{\pgfqpoint{3.577681in}{1.977623in}}%
\pgfpathlineto{\pgfqpoint{3.578449in}{1.977225in}}%
\pgfpathlineto{\pgfqpoint{3.579026in}{1.979096in}}%
\pgfpathlineto{\pgfqpoint{3.579410in}{1.974186in}}%
\pgfpathlineto{\pgfqpoint{3.580563in}{1.980813in}}%
\pgfpathlineto{\pgfqpoint{3.580755in}{1.978845in}}%
\pgfpathlineto{\pgfqpoint{3.581716in}{1.979941in}}%
\pgfpathlineto{\pgfqpoint{3.582869in}{1.984941in}}%
\pgfpathlineto{\pgfqpoint{3.583061in}{1.984453in}}%
\pgfpathlineto{\pgfqpoint{3.583446in}{1.989023in}}%
\pgfpathlineto{\pgfqpoint{3.584407in}{1.988758in}}%
\pgfpathlineto{\pgfqpoint{3.584599in}{1.987255in}}%
\pgfpathlineto{\pgfqpoint{3.584983in}{1.990812in}}%
\pgfpathlineto{\pgfqpoint{3.587097in}{2.001661in}}%
\pgfpathlineto{\pgfqpoint{3.587289in}{1.999046in}}%
\pgfpathlineto{\pgfqpoint{3.587866in}{2.005656in}}%
\pgfpathlineto{\pgfqpoint{3.588250in}{2.009030in}}%
\pgfpathlineto{\pgfqpoint{3.589211in}{2.008595in}}%
\pgfpathlineto{\pgfqpoint{3.589403in}{2.005400in}}%
\pgfpathlineto{\pgfqpoint{3.590172in}{2.010122in}}%
\pgfpathlineto{\pgfqpoint{3.590364in}{2.009263in}}%
\pgfpathlineto{\pgfqpoint{3.590556in}{2.009959in}}%
\pgfpathlineto{\pgfqpoint{3.591709in}{2.016614in}}%
\pgfpathlineto{\pgfqpoint{3.591901in}{2.015681in}}%
\pgfpathlineto{\pgfqpoint{3.593823in}{2.005760in}}%
\pgfpathlineto{\pgfqpoint{3.594015in}{2.007106in}}%
\pgfpathlineto{\pgfqpoint{3.594399in}{2.004373in}}%
\pgfpathlineto{\pgfqpoint{3.594784in}{1.999409in}}%
\pgfpathlineto{\pgfqpoint{3.595360in}{2.004677in}}%
\pgfpathlineto{\pgfqpoint{3.596129in}{2.003654in}}%
\pgfpathlineto{\pgfqpoint{3.596321in}{2.005082in}}%
\pgfpathlineto{\pgfqpoint{3.596513in}{2.003791in}}%
\pgfpathlineto{\pgfqpoint{3.597090in}{2.004193in}}%
\pgfpathlineto{\pgfqpoint{3.598050in}{1.997678in}}%
\pgfpathlineto{\pgfqpoint{3.598243in}{2.000417in}}%
\pgfpathlineto{\pgfqpoint{3.598627in}{2.004710in}}%
\pgfpathlineto{\pgfqpoint{3.599011in}{1.999138in}}%
\pgfpathlineto{\pgfqpoint{3.599203in}{1.999682in}}%
\pgfpathlineto{\pgfqpoint{3.600164in}{2.005612in}}%
\pgfpathlineto{\pgfqpoint{3.600549in}{2.003361in}}%
\pgfpathlineto{\pgfqpoint{3.600741in}{2.001582in}}%
\pgfpathlineto{\pgfqpoint{3.601317in}{2.004908in}}%
\pgfpathlineto{\pgfqpoint{3.603239in}{2.011786in}}%
\pgfpathlineto{\pgfqpoint{3.602278in}{2.004391in}}%
\pgfpathlineto{\pgfqpoint{3.603431in}{2.011696in}}%
\pgfpathlineto{\pgfqpoint{3.604584in}{2.003739in}}%
\pgfpathlineto{\pgfqpoint{3.604776in}{2.005111in}}%
\pgfpathlineto{\pgfqpoint{3.604969in}{2.007653in}}%
\pgfpathlineto{\pgfqpoint{3.605545in}{2.004265in}}%
\pgfpathlineto{\pgfqpoint{3.605737in}{2.007275in}}%
\pgfpathlineto{\pgfqpoint{3.606506in}{1.998377in}}%
\pgfpathlineto{\pgfqpoint{3.607082in}{2.002461in}}%
\pgfpathlineto{\pgfqpoint{3.607659in}{1.998928in}}%
\pgfpathlineto{\pgfqpoint{3.608043in}{1.996426in}}%
\pgfpathlineto{\pgfqpoint{3.608620in}{1.999665in}}%
\pgfpathlineto{\pgfqpoint{3.608812in}{1.999607in}}%
\pgfpathlineto{\pgfqpoint{3.609773in}{2.004317in}}%
\pgfpathlineto{\pgfqpoint{3.609965in}{2.003367in}}%
\pgfpathlineto{\pgfqpoint{3.610349in}{2.000382in}}%
\pgfpathlineto{\pgfqpoint{3.610926in}{2.005237in}}%
\pgfpathlineto{\pgfqpoint{3.611118in}{2.005878in}}%
\pgfpathlineto{\pgfqpoint{3.611502in}{2.004691in}}%
\pgfpathlineto{\pgfqpoint{3.611887in}{2.000153in}}%
\pgfpathlineto{\pgfqpoint{3.612271in}{2.004713in}}%
\pgfpathlineto{\pgfqpoint{3.614193in}{2.015445in}}%
\pgfpathlineto{\pgfqpoint{3.614385in}{2.015330in}}%
\pgfpathlineto{\pgfqpoint{3.614769in}{2.015540in}}%
\pgfpathlineto{\pgfqpoint{3.615922in}{2.007138in}}%
\pgfpathlineto{\pgfqpoint{3.617267in}{2.010389in}}%
\pgfpathlineto{\pgfqpoint{3.617459in}{2.008228in}}%
\pgfpathlineto{\pgfqpoint{3.618036in}{2.013338in}}%
\pgfpathlineto{\pgfqpoint{3.618228in}{2.010458in}}%
\pgfpathlineto{\pgfqpoint{3.618997in}{2.009891in}}%
\pgfpathlineto{\pgfqpoint{3.619765in}{2.015148in}}%
\pgfpathlineto{\pgfqpoint{3.621303in}{2.009467in}}%
\pgfpathlineto{\pgfqpoint{3.623417in}{2.019068in}}%
\pgfpathlineto{\pgfqpoint{3.623609in}{2.018866in}}%
\pgfpathlineto{\pgfqpoint{3.623993in}{2.015205in}}%
\pgfpathlineto{\pgfqpoint{3.624570in}{2.020139in}}%
\pgfpathlineto{\pgfqpoint{3.625338in}{2.016914in}}%
\pgfpathlineto{\pgfqpoint{3.625531in}{2.019660in}}%
\pgfpathlineto{\pgfqpoint{3.627452in}{2.029447in}}%
\pgfpathlineto{\pgfqpoint{3.628221in}{2.026344in}}%
\pgfpathlineto{\pgfqpoint{3.628990in}{2.023936in}}%
\pgfpathlineto{\pgfqpoint{3.628605in}{2.027695in}}%
\pgfpathlineto{\pgfqpoint{3.629374in}{2.024613in}}%
\pgfpathlineto{\pgfqpoint{3.629566in}{2.026105in}}%
\pgfpathlineto{\pgfqpoint{3.629950in}{2.022591in}}%
\pgfpathlineto{\pgfqpoint{3.630335in}{2.023890in}}%
\pgfpathlineto{\pgfqpoint{3.630527in}{2.022603in}}%
\pgfpathlineto{\pgfqpoint{3.631296in}{2.024563in}}%
\pgfpathlineto{\pgfqpoint{3.631872in}{2.027372in}}%
\pgfpathlineto{\pgfqpoint{3.632256in}{2.023842in}}%
\pgfpathlineto{\pgfqpoint{3.633217in}{2.016568in}}%
\pgfpathlineto{\pgfqpoint{3.634178in}{2.011430in}}%
\pgfpathlineto{\pgfqpoint{3.634562in}{2.013568in}}%
\pgfpathlineto{\pgfqpoint{3.636676in}{2.026871in}}%
\pgfpathlineto{\pgfqpoint{3.636868in}{2.020924in}}%
\pgfpathlineto{\pgfqpoint{3.637637in}{2.022598in}}%
\pgfpathlineto{\pgfqpoint{3.638406in}{2.031243in}}%
\pgfpathlineto{\pgfqpoint{3.638982in}{2.030037in}}%
\pgfpathlineto{\pgfqpoint{3.639751in}{2.025680in}}%
\pgfpathlineto{\pgfqpoint{3.640520in}{2.027137in}}%
\pgfpathlineto{\pgfqpoint{3.640904in}{2.032373in}}%
\pgfpathlineto{\pgfqpoint{3.641288in}{2.026597in}}%
\pgfpathlineto{\pgfqpoint{3.641480in}{2.027864in}}%
\pgfpathlineto{\pgfqpoint{3.642441in}{2.026249in}}%
\pgfpathlineto{\pgfqpoint{3.642633in}{2.027439in}}%
\pgfpathlineto{\pgfqpoint{3.643402in}{2.025871in}}%
\pgfpathlineto{\pgfqpoint{3.643594in}{2.026952in}}%
\pgfpathlineto{\pgfqpoint{3.644171in}{2.022672in}}%
\pgfpathlineto{\pgfqpoint{3.644555in}{2.027517in}}%
\pgfpathlineto{\pgfqpoint{3.645132in}{2.032160in}}%
\pgfpathlineto{\pgfqpoint{3.645900in}{2.030819in}}%
\pgfpathlineto{\pgfqpoint{3.646285in}{2.028481in}}%
\pgfpathlineto{\pgfqpoint{3.646669in}{2.031302in}}%
\pgfpathlineto{\pgfqpoint{3.646861in}{2.031238in}}%
\pgfpathlineto{\pgfqpoint{3.648398in}{2.038509in}}%
\pgfpathlineto{\pgfqpoint{3.648591in}{2.037683in}}%
\pgfpathlineto{\pgfqpoint{3.649744in}{2.027738in}}%
\pgfpathlineto{\pgfqpoint{3.650128in}{2.028390in}}%
\pgfpathlineto{\pgfqpoint{3.650320in}{2.030157in}}%
\pgfpathlineto{\pgfqpoint{3.650897in}{2.026587in}}%
\pgfpathlineto{\pgfqpoint{3.652050in}{2.023863in}}%
\pgfpathlineto{\pgfqpoint{3.652242in}{2.025369in}}%
\pgfpathlineto{\pgfqpoint{3.652434in}{2.024028in}}%
\pgfpathlineto{\pgfqpoint{3.653587in}{2.018306in}}%
\pgfpathlineto{\pgfqpoint{3.654164in}{2.021314in}}%
\pgfpathlineto{\pgfqpoint{3.654356in}{2.018005in}}%
\pgfpathlineto{\pgfqpoint{3.655893in}{2.011749in}}%
\pgfpathlineto{\pgfqpoint{3.656085in}{2.014596in}}%
\pgfpathlineto{\pgfqpoint{3.656662in}{2.007373in}}%
\pgfpathlineto{\pgfqpoint{3.656854in}{2.006255in}}%
\pgfpathlineto{\pgfqpoint{3.657238in}{2.008505in}}%
\pgfpathlineto{\pgfqpoint{3.657430in}{2.008218in}}%
\pgfpathlineto{\pgfqpoint{3.657623in}{2.012366in}}%
\pgfpathlineto{\pgfqpoint{3.658199in}{2.006865in}}%
\pgfpathlineto{\pgfqpoint{3.659544in}{2.002401in}}%
\pgfpathlineto{\pgfqpoint{3.660697in}{2.006070in}}%
\pgfpathlineto{\pgfqpoint{3.660889in}{2.003321in}}%
\pgfpathlineto{\pgfqpoint{3.661274in}{2.007807in}}%
\pgfpathlineto{\pgfqpoint{3.661658in}{2.007425in}}%
\pgfpathlineto{\pgfqpoint{3.662042in}{2.009683in}}%
\pgfpathlineto{\pgfqpoint{3.662427in}{2.006509in}}%
\pgfpathlineto{\pgfqpoint{3.662619in}{2.007706in}}%
\pgfpathlineto{\pgfqpoint{3.663580in}{2.002726in}}%
\pgfpathlineto{\pgfqpoint{3.663772in}{2.005154in}}%
\pgfpathlineto{\pgfqpoint{3.663964in}{2.008908in}}%
\pgfpathlineto{\pgfqpoint{3.664733in}{2.004050in}}%
\pgfpathlineto{\pgfqpoint{3.665117in}{2.001744in}}%
\pgfpathlineto{\pgfqpoint{3.665501in}{2.003210in}}%
\pgfpathlineto{\pgfqpoint{3.666654in}{2.008848in}}%
\pgfpathlineto{\pgfqpoint{3.667807in}{2.004501in}}%
\pgfpathlineto{\pgfqpoint{3.667231in}{2.010277in}}%
\pgfpathlineto{\pgfqpoint{3.668192in}{2.005434in}}%
\pgfpathlineto{\pgfqpoint{3.668576in}{2.008307in}}%
\pgfpathlineto{\pgfqpoint{3.669537in}{2.006556in}}%
\pgfpathlineto{\pgfqpoint{3.673188in}{1.991473in}}%
\pgfpathlineto{\pgfqpoint{3.669921in}{2.007624in}}%
\pgfpathlineto{\pgfqpoint{3.674149in}{1.994590in}}%
\pgfpathlineto{\pgfqpoint{3.675302in}{1.998114in}}%
\pgfpathlineto{\pgfqpoint{3.675494in}{1.997630in}}%
\pgfpathlineto{\pgfqpoint{3.675686in}{1.997185in}}%
\pgfpathlineto{\pgfqpoint{3.676263in}{1.998308in}}%
\pgfpathlineto{\pgfqpoint{3.677224in}{2.002130in}}%
\pgfpathlineto{\pgfqpoint{3.677608in}{2.000454in}}%
\pgfpathlineto{\pgfqpoint{3.677800in}{1.998111in}}%
\pgfpathlineto{\pgfqpoint{3.678569in}{2.000006in}}%
\pgfpathlineto{\pgfqpoint{3.678953in}{2.000903in}}%
\pgfpathlineto{\pgfqpoint{3.679338in}{1.999520in}}%
\pgfpathlineto{\pgfqpoint{3.680491in}{1.992825in}}%
\pgfpathlineto{\pgfqpoint{3.680683in}{1.995269in}}%
\pgfpathlineto{\pgfqpoint{3.681451in}{1.999501in}}%
\pgfpathlineto{\pgfqpoint{3.682220in}{1.999161in}}%
\pgfpathlineto{\pgfqpoint{3.682412in}{1.996460in}}%
\pgfpathlineto{\pgfqpoint{3.682797in}{2.003149in}}%
\pgfpathlineto{\pgfqpoint{3.683373in}{2.005639in}}%
\pgfpathlineto{\pgfqpoint{3.684142in}{2.004892in}}%
\pgfpathlineto{\pgfqpoint{3.684334in}{2.004417in}}%
\pgfpathlineto{\pgfqpoint{3.684526in}{2.005273in}}%
\pgfpathlineto{\pgfqpoint{3.685871in}{2.016400in}}%
\pgfpathlineto{\pgfqpoint{3.686640in}{2.015642in}}%
\pgfpathlineto{\pgfqpoint{3.687216in}{2.015567in}}%
\pgfpathlineto{\pgfqpoint{3.687024in}{2.016345in}}%
\pgfpathlineto{\pgfqpoint{3.687409in}{2.016259in}}%
\pgfpathlineto{\pgfqpoint{3.687985in}{2.025426in}}%
\pgfpathlineto{\pgfqpoint{3.688946in}{2.018539in}}%
\pgfpathlineto{\pgfqpoint{3.689715in}{2.010961in}}%
\pgfpathlineto{\pgfqpoint{3.691060in}{2.005775in}}%
\pgfpathlineto{\pgfqpoint{3.691252in}{2.007289in}}%
\pgfpathlineto{\pgfqpoint{3.691828in}{2.004052in}}%
\pgfpathlineto{\pgfqpoint{3.692021in}{2.005019in}}%
\pgfpathlineto{\pgfqpoint{3.692213in}{2.000135in}}%
\pgfpathlineto{\pgfqpoint{3.693174in}{2.001388in}}%
\pgfpathlineto{\pgfqpoint{3.693366in}{2.002051in}}%
\pgfpathlineto{\pgfqpoint{3.693942in}{2.008870in}}%
\pgfpathlineto{\pgfqpoint{3.694519in}{2.007256in}}%
\pgfpathlineto{\pgfqpoint{3.694711in}{2.007252in}}%
\pgfpathlineto{\pgfqpoint{3.695287in}{2.004776in}}%
\pgfpathlineto{\pgfqpoint{3.695672in}{2.009023in}}%
\pgfpathlineto{\pgfqpoint{3.695864in}{2.006452in}}%
\pgfpathlineto{\pgfqpoint{3.697786in}{2.019322in}}%
\pgfpathlineto{\pgfqpoint{3.697978in}{2.017218in}}%
\pgfpathlineto{\pgfqpoint{3.698747in}{2.019677in}}%
\pgfpathlineto{\pgfqpoint{3.699323in}{2.027768in}}%
\pgfpathlineto{\pgfqpoint{3.700092in}{2.023953in}}%
\pgfpathlineto{\pgfqpoint{3.701629in}{2.018820in}}%
\pgfpathlineto{\pgfqpoint{3.702590in}{2.023134in}}%
\pgfpathlineto{\pgfqpoint{3.702782in}{2.021361in}}%
\pgfpathlineto{\pgfqpoint{3.703551in}{2.015796in}}%
\pgfpathlineto{\pgfqpoint{3.704127in}{2.018428in}}%
\pgfpathlineto{\pgfqpoint{3.704319in}{2.018750in}}%
\pgfpathlineto{\pgfqpoint{3.705472in}{2.014362in}}%
\pgfpathlineto{\pgfqpoint{3.705665in}{2.014953in}}%
\pgfpathlineto{\pgfqpoint{3.706818in}{2.012108in}}%
\pgfpathlineto{\pgfqpoint{3.707778in}{2.015409in}}%
\pgfpathlineto{\pgfqpoint{3.707971in}{2.015254in}}%
\pgfpathlineto{\pgfqpoint{3.709316in}{2.009747in}}%
\pgfpathlineto{\pgfqpoint{3.709508in}{2.009265in}}%
\pgfpathlineto{\pgfqpoint{3.709892in}{2.010956in}}%
\pgfpathlineto{\pgfqpoint{3.710084in}{2.011627in}}%
\pgfpathlineto{\pgfqpoint{3.710469in}{2.009036in}}%
\pgfpathlineto{\pgfqpoint{3.710661in}{2.009419in}}%
\pgfpathlineto{\pgfqpoint{3.710853in}{2.007863in}}%
\pgfpathlineto{\pgfqpoint{3.712967in}{1.997398in}}%
\pgfpathlineto{\pgfqpoint{3.713159in}{2.000661in}}%
\pgfpathlineto{\pgfqpoint{3.713736in}{2.008699in}}%
\pgfpathlineto{\pgfqpoint{3.714504in}{2.006952in}}%
\pgfpathlineto{\pgfqpoint{3.716234in}{2.000571in}}%
\pgfpathlineto{\pgfqpoint{3.718540in}{2.018322in}}%
\pgfpathlineto{\pgfqpoint{3.718924in}{2.016835in}}%
\pgfpathlineto{\pgfqpoint{3.719693in}{2.018238in}}%
\pgfpathlineto{\pgfqpoint{3.720654in}{2.010284in}}%
\pgfpathlineto{\pgfqpoint{3.721038in}{2.013103in}}%
\pgfpathlineto{\pgfqpoint{3.721807in}{2.011713in}}%
\pgfpathlineto{\pgfqpoint{3.722768in}{2.008528in}}%
\pgfpathlineto{\pgfqpoint{3.722960in}{2.009417in}}%
\pgfpathlineto{\pgfqpoint{3.723152in}{2.009836in}}%
\pgfpathlineto{\pgfqpoint{3.723344in}{2.007977in}}%
\pgfpathlineto{\pgfqpoint{3.723536in}{2.007210in}}%
\pgfpathlineto{\pgfqpoint{3.724497in}{2.015288in}}%
\pgfpathlineto{\pgfqpoint{3.724689in}{2.013150in}}%
\pgfpathlineto{\pgfqpoint{3.724881in}{2.011243in}}%
\pgfpathlineto{\pgfqpoint{3.725074in}{2.013655in}}%
\pgfpathlineto{\pgfqpoint{3.725650in}{2.012002in}}%
\pgfpathlineto{\pgfqpoint{3.726034in}{2.014844in}}%
\pgfpathlineto{\pgfqpoint{3.726611in}{2.013160in}}%
\pgfpathlineto{\pgfqpoint{3.726803in}{2.007819in}}%
\pgfpathlineto{\pgfqpoint{3.727764in}{2.008048in}}%
\pgfpathlineto{\pgfqpoint{3.728533in}{2.002448in}}%
\pgfpathlineto{\pgfqpoint{3.729878in}{1.997730in}}%
\pgfpathlineto{\pgfqpoint{3.729301in}{2.003944in}}%
\pgfpathlineto{\pgfqpoint{3.730070in}{1.999524in}}%
\pgfpathlineto{\pgfqpoint{3.730262in}{2.002252in}}%
\pgfpathlineto{\pgfqpoint{3.731031in}{2.001850in}}%
\pgfpathlineto{\pgfqpoint{3.732376in}{1.990254in}}%
\pgfpathlineto{\pgfqpoint{3.732952in}{1.992761in}}%
\pgfpathlineto{\pgfqpoint{3.733145in}{1.992493in}}%
\pgfpathlineto{\pgfqpoint{3.733337in}{1.989529in}}%
\pgfpathlineto{\pgfqpoint{3.734105in}{1.994646in}}%
\pgfpathlineto{\pgfqpoint{3.734298in}{1.997174in}}%
\pgfpathlineto{\pgfqpoint{3.734682in}{1.994433in}}%
\pgfpathlineto{\pgfqpoint{3.735258in}{1.996179in}}%
\pgfpathlineto{\pgfqpoint{3.735643in}{1.994581in}}%
\pgfpathlineto{\pgfqpoint{3.736796in}{1.998751in}}%
\pgfpathlineto{\pgfqpoint{3.737564in}{1.989186in}}%
\pgfpathlineto{\pgfqpoint{3.738333in}{1.991408in}}%
\pgfpathlineto{\pgfqpoint{3.738525in}{1.992725in}}%
\pgfpathlineto{\pgfqpoint{3.738910in}{1.991901in}}%
\pgfpathlineto{\pgfqpoint{3.739678in}{1.986666in}}%
\pgfpathlineto{\pgfqpoint{3.740063in}{1.987944in}}%
\pgfpathlineto{\pgfqpoint{3.740447in}{1.990482in}}%
\pgfpathlineto{\pgfqpoint{3.741023in}{1.987887in}}%
\pgfpathlineto{\pgfqpoint{3.741792in}{1.985289in}}%
\pgfpathlineto{\pgfqpoint{3.742561in}{1.992879in}}%
\pgfpathlineto{\pgfqpoint{3.742945in}{1.991254in}}%
\pgfpathlineto{\pgfqpoint{3.743906in}{1.987447in}}%
\pgfpathlineto{\pgfqpoint{3.744098in}{1.989834in}}%
\pgfpathlineto{\pgfqpoint{3.744290in}{1.987926in}}%
\pgfpathlineto{\pgfqpoint{3.744867in}{1.992892in}}%
\pgfpathlineto{\pgfqpoint{3.745251in}{1.989282in}}%
\pgfpathlineto{\pgfqpoint{3.745635in}{1.992029in}}%
\pgfpathlineto{\pgfqpoint{3.746981in}{1.997042in}}%
\pgfpathlineto{\pgfqpoint{3.747173in}{1.997003in}}%
\pgfpathlineto{\pgfqpoint{3.748326in}{1.993832in}}%
\pgfpathlineto{\pgfqpoint{3.747749in}{1.998748in}}%
\pgfpathlineto{\pgfqpoint{3.748518in}{1.994742in}}%
\pgfpathlineto{\pgfqpoint{3.748710in}{1.994037in}}%
\pgfpathlineto{\pgfqpoint{3.749479in}{2.002010in}}%
\pgfpathlineto{\pgfqpoint{3.749863in}{1.997230in}}%
\pgfpathlineto{\pgfqpoint{3.750055in}{1.997998in}}%
\pgfpathlineto{\pgfqpoint{3.750248in}{1.997400in}}%
\pgfpathlineto{\pgfqpoint{3.750440in}{1.993602in}}%
\pgfpathlineto{\pgfqpoint{3.751208in}{1.998833in}}%
\pgfpathlineto{\pgfqpoint{3.751401in}{1.998594in}}%
\pgfpathlineto{\pgfqpoint{3.752361in}{2.005913in}}%
\pgfpathlineto{\pgfqpoint{3.752746in}{2.005670in}}%
\pgfpathlineto{\pgfqpoint{3.753130in}{2.004703in}}%
\pgfpathlineto{\pgfqpoint{3.753322in}{2.007345in}}%
\pgfpathlineto{\pgfqpoint{3.753899in}{2.012435in}}%
\pgfpathlineto{\pgfqpoint{3.754667in}{2.010041in}}%
\pgfpathlineto{\pgfqpoint{3.754860in}{2.009971in}}%
\pgfpathlineto{\pgfqpoint{3.755052in}{2.008566in}}%
\pgfpathlineto{\pgfqpoint{3.755628in}{2.011294in}}%
\pgfpathlineto{\pgfqpoint{3.755820in}{2.009369in}}%
\pgfpathlineto{\pgfqpoint{3.756397in}{2.014637in}}%
\pgfpathlineto{\pgfqpoint{3.756589in}{2.012063in}}%
\pgfpathlineto{\pgfqpoint{3.757166in}{2.002109in}}%
\pgfpathlineto{\pgfqpoint{3.757934in}{2.005002in}}%
\pgfpathlineto{\pgfqpoint{3.758126in}{2.004507in}}%
\pgfpathlineto{\pgfqpoint{3.758319in}{2.006683in}}%
\pgfpathlineto{\pgfqpoint{3.758511in}{2.008603in}}%
\pgfpathlineto{\pgfqpoint{3.759279in}{2.005789in}}%
\pgfpathlineto{\pgfqpoint{3.759472in}{2.006301in}}%
\pgfpathlineto{\pgfqpoint{3.759664in}{2.003187in}}%
\pgfpathlineto{\pgfqpoint{3.760240in}{2.010136in}}%
\pgfpathlineto{\pgfqpoint{3.760432in}{2.006503in}}%
\pgfpathlineto{\pgfqpoint{3.760817in}{2.010876in}}%
\pgfpathlineto{\pgfqpoint{3.761778in}{2.010572in}}%
\pgfpathlineto{\pgfqpoint{3.762931in}{2.006280in}}%
\pgfpathlineto{\pgfqpoint{3.763123in}{2.008830in}}%
\pgfpathlineto{\pgfqpoint{3.763699in}{2.011531in}}%
\pgfpathlineto{\pgfqpoint{3.764084in}{2.007741in}}%
\pgfpathlineto{\pgfqpoint{3.764660in}{2.007432in}}%
\pgfpathlineto{\pgfqpoint{3.765429in}{2.012615in}}%
\pgfpathlineto{\pgfqpoint{3.765813in}{2.016352in}}%
\pgfpathlineto{\pgfqpoint{3.766582in}{2.014031in}}%
\pgfpathlineto{\pgfqpoint{3.768696in}{1.997723in}}%
\pgfpathlineto{\pgfqpoint{3.769849in}{2.001483in}}%
\pgfpathlineto{\pgfqpoint{3.769080in}{1.997332in}}%
\pgfpathlineto{\pgfqpoint{3.770425in}{2.001197in}}%
\pgfpathlineto{\pgfqpoint{3.770617in}{2.000051in}}%
\pgfpathlineto{\pgfqpoint{3.771002in}{2.002923in}}%
\pgfpathlineto{\pgfqpoint{3.771194in}{2.000676in}}%
\pgfpathlineto{\pgfqpoint{3.771386in}{2.003557in}}%
\pgfpathlineto{\pgfqpoint{3.771963in}{1.996541in}}%
\pgfpathlineto{\pgfqpoint{3.772923in}{2.003414in}}%
\pgfpathlineto{\pgfqpoint{3.773692in}{2.000537in}}%
\pgfpathlineto{\pgfqpoint{3.773884in}{2.000813in}}%
\pgfpathlineto{\pgfqpoint{3.774076in}{2.003486in}}%
\pgfpathlineto{\pgfqpoint{3.774653in}{1.999550in}}%
\pgfpathlineto{\pgfqpoint{3.774845in}{2.002144in}}%
\pgfpathlineto{\pgfqpoint{3.776382in}{1.997740in}}%
\pgfpathlineto{\pgfqpoint{3.776767in}{1.996624in}}%
\pgfpathlineto{\pgfqpoint{3.778496in}{2.009120in}}%
\pgfpathlineto{\pgfqpoint{3.778881in}{2.004796in}}%
\pgfpathlineto{\pgfqpoint{3.779457in}{2.009647in}}%
\pgfpathlineto{\pgfqpoint{3.780034in}{2.007133in}}%
\pgfpathlineto{\pgfqpoint{3.780226in}{2.008399in}}%
\pgfpathlineto{\pgfqpoint{3.782147in}{2.021433in}}%
\pgfpathlineto{\pgfqpoint{3.782340in}{2.017864in}}%
\pgfpathlineto{\pgfqpoint{3.783108in}{2.020282in}}%
\pgfpathlineto{\pgfqpoint{3.783300in}{2.021296in}}%
\pgfpathlineto{\pgfqpoint{3.783493in}{2.019039in}}%
\pgfpathlineto{\pgfqpoint{3.783685in}{2.016812in}}%
\pgfpathlineto{\pgfqpoint{3.784069in}{2.020858in}}%
\pgfpathlineto{\pgfqpoint{3.784453in}{2.019504in}}%
\pgfpathlineto{\pgfqpoint{3.784646in}{2.023834in}}%
\pgfpathlineto{\pgfqpoint{3.785414in}{2.020121in}}%
\pgfpathlineto{\pgfqpoint{3.786567in}{2.012355in}}%
\pgfpathlineto{\pgfqpoint{3.787144in}{2.014601in}}%
\pgfpathlineto{\pgfqpoint{3.788297in}{2.005552in}}%
\pgfpathlineto{\pgfqpoint{3.790218in}{1.992451in}}%
\pgfpathlineto{\pgfqpoint{3.790411in}{1.995609in}}%
\pgfpathlineto{\pgfqpoint{3.790987in}{1.991681in}}%
\pgfpathlineto{\pgfqpoint{3.791179in}{1.992988in}}%
\pgfpathlineto{\pgfqpoint{3.792332in}{1.988227in}}%
\pgfpathlineto{\pgfqpoint{3.792717in}{1.989815in}}%
\pgfpathlineto{\pgfqpoint{3.794062in}{1.996212in}}%
\pgfpathlineto{\pgfqpoint{3.794254in}{1.995368in}}%
\pgfpathlineto{\pgfqpoint{3.794446in}{1.997480in}}%
\pgfpathlineto{\pgfqpoint{3.794638in}{1.999424in}}%
\pgfpathlineto{\pgfqpoint{3.795215in}{1.995951in}}%
\pgfpathlineto{\pgfqpoint{3.795791in}{1.991527in}}%
\pgfpathlineto{\pgfqpoint{3.796176in}{1.995589in}}%
\pgfpathlineto{\pgfqpoint{3.796368in}{1.995606in}}%
\pgfpathlineto{\pgfqpoint{3.796560in}{1.992342in}}%
\pgfpathlineto{\pgfqpoint{3.797521in}{1.993639in}}%
\pgfpathlineto{\pgfqpoint{3.799250in}{1.984449in}}%
\pgfpathlineto{\pgfqpoint{3.799443in}{1.985334in}}%
\pgfpathlineto{\pgfqpoint{3.800019in}{1.984644in}}%
\pgfpathlineto{\pgfqpoint{3.800596in}{1.989101in}}%
\pgfpathlineto{\pgfqpoint{3.801749in}{1.982545in}}%
\pgfpathlineto{\pgfqpoint{3.801941in}{1.983563in}}%
\pgfpathlineto{\pgfqpoint{3.802325in}{1.984974in}}%
\pgfpathlineto{\pgfqpoint{3.802517in}{1.981093in}}%
\pgfpathlineto{\pgfqpoint{3.802709in}{1.981302in}}%
\pgfpathlineto{\pgfqpoint{3.804055in}{1.975449in}}%
\pgfpathlineto{\pgfqpoint{3.804247in}{1.975685in}}%
\pgfpathlineto{\pgfqpoint{3.804439in}{1.974466in}}%
\pgfpathlineto{\pgfqpoint{3.804823in}{1.968133in}}%
\pgfpathlineto{\pgfqpoint{3.805592in}{1.970924in}}%
\pgfpathlineto{\pgfqpoint{3.806553in}{1.976813in}}%
\pgfpathlineto{\pgfqpoint{3.807129in}{1.980400in}}%
\pgfpathlineto{\pgfqpoint{3.807514in}{1.975368in}}%
\pgfpathlineto{\pgfqpoint{3.808090in}{1.974578in}}%
\pgfpathlineto{\pgfqpoint{3.808474in}{1.977747in}}%
\pgfpathlineto{\pgfqpoint{3.810012in}{1.988100in}}%
\pgfpathlineto{\pgfqpoint{3.810780in}{1.986100in}}%
\pgfpathlineto{\pgfqpoint{3.811165in}{1.988249in}}%
\pgfpathlineto{\pgfqpoint{3.812318in}{1.995047in}}%
\pgfpathlineto{\pgfqpoint{3.811549in}{1.988053in}}%
\pgfpathlineto{\pgfqpoint{3.813086in}{1.992692in}}%
\pgfpathlineto{\pgfqpoint{3.813279in}{1.991548in}}%
\pgfpathlineto{\pgfqpoint{3.813471in}{1.994927in}}%
\pgfpathlineto{\pgfqpoint{3.813663in}{1.994711in}}%
\pgfpathlineto{\pgfqpoint{3.814047in}{1.998055in}}%
\pgfpathlineto{\pgfqpoint{3.814624in}{1.994407in}}%
\pgfpathlineto{\pgfqpoint{3.814816in}{1.992042in}}%
\pgfpathlineto{\pgfqpoint{3.815392in}{1.998416in}}%
\pgfpathlineto{\pgfqpoint{3.815585in}{1.997484in}}%
\pgfpathlineto{\pgfqpoint{3.815969in}{1.999727in}}%
\pgfpathlineto{\pgfqpoint{3.817698in}{2.011705in}}%
\pgfpathlineto{\pgfqpoint{3.818083in}{2.006073in}}%
\pgfpathlineto{\pgfqpoint{3.818851in}{2.006756in}}%
\pgfpathlineto{\pgfqpoint{3.820005in}{2.013253in}}%
\pgfpathlineto{\pgfqpoint{3.819620in}{2.004978in}}%
\pgfpathlineto{\pgfqpoint{3.820197in}{2.011854in}}%
\pgfpathlineto{\pgfqpoint{3.821158in}{2.010779in}}%
\pgfpathlineto{\pgfqpoint{3.820773in}{2.013216in}}%
\pgfpathlineto{\pgfqpoint{3.821350in}{2.011596in}}%
\pgfpathlineto{\pgfqpoint{3.822887in}{2.018234in}}%
\pgfpathlineto{\pgfqpoint{3.826346in}{2.027910in}}%
\pgfpathlineto{\pgfqpoint{3.826730in}{2.027555in}}%
\pgfpathlineto{\pgfqpoint{3.827691in}{2.033646in}}%
\pgfpathlineto{\pgfqpoint{3.828268in}{2.032436in}}%
\pgfpathlineto{\pgfqpoint{3.829036in}{2.036657in}}%
\pgfpathlineto{\pgfqpoint{3.829421in}{2.033521in}}%
\pgfpathlineto{\pgfqpoint{3.830766in}{2.026515in}}%
\pgfpathlineto{\pgfqpoint{3.831150in}{2.022529in}}%
\pgfpathlineto{\pgfqpoint{3.831727in}{2.025114in}}%
\pgfpathlineto{\pgfqpoint{3.831919in}{2.027092in}}%
\pgfpathlineto{\pgfqpoint{3.832880in}{2.025669in}}%
\pgfpathlineto{\pgfqpoint{3.833072in}{2.025557in}}%
\pgfpathlineto{\pgfqpoint{3.833841in}{2.027434in}}%
\pgfpathlineto{\pgfqpoint{3.834225in}{2.027345in}}%
\pgfpathlineto{\pgfqpoint{3.834609in}{2.024757in}}%
\pgfpathlineto{\pgfqpoint{3.835954in}{2.017972in}}%
\pgfpathlineto{\pgfqpoint{3.836147in}{2.021028in}}%
\pgfpathlineto{\pgfqpoint{3.836723in}{2.013931in}}%
\pgfpathlineto{\pgfqpoint{3.837684in}{2.013391in}}%
\pgfpathlineto{\pgfqpoint{3.838068in}{2.016934in}}%
\pgfpathlineto{\pgfqpoint{3.838453in}{2.015363in}}%
\pgfpathlineto{\pgfqpoint{3.838837in}{2.019171in}}%
\pgfpathlineto{\pgfqpoint{3.840182in}{2.022670in}}%
\pgfpathlineto{\pgfqpoint{3.840374in}{2.023412in}}%
\pgfpathlineto{\pgfqpoint{3.840566in}{2.020961in}}%
\pgfpathlineto{\pgfqpoint{3.840759in}{2.021157in}}%
\pgfpathlineto{\pgfqpoint{3.841143in}{2.019019in}}%
\pgfpathlineto{\pgfqpoint{3.841527in}{2.021257in}}%
\pgfpathlineto{\pgfqpoint{3.842680in}{2.027710in}}%
\pgfpathlineto{\pgfqpoint{3.842872in}{2.025451in}}%
\pgfpathlineto{\pgfqpoint{3.843641in}{2.028287in}}%
\pgfpathlineto{\pgfqpoint{3.844602in}{2.024089in}}%
\pgfpathlineto{\pgfqpoint{3.844794in}{2.025205in}}%
\pgfpathlineto{\pgfqpoint{3.844986in}{2.025621in}}%
\pgfpathlineto{\pgfqpoint{3.846332in}{2.018658in}}%
\pgfpathlineto{\pgfqpoint{3.850559in}{2.047768in}}%
\pgfpathlineto{\pgfqpoint{3.851520in}{2.041879in}}%
\pgfpathlineto{\pgfqpoint{3.851712in}{2.042969in}}%
\pgfpathlineto{\pgfqpoint{3.852481in}{2.048817in}}%
\pgfpathlineto{\pgfqpoint{3.852865in}{2.044150in}}%
\pgfpathlineto{\pgfqpoint{3.853057in}{2.044105in}}%
\pgfpathlineto{\pgfqpoint{3.853250in}{2.046545in}}%
\pgfpathlineto{\pgfqpoint{3.853634in}{2.042447in}}%
\pgfpathlineto{\pgfqpoint{3.854018in}{2.044052in}}%
\pgfpathlineto{\pgfqpoint{3.854210in}{2.042719in}}%
\pgfpathlineto{\pgfqpoint{3.854787in}{2.046238in}}%
\pgfpathlineto{\pgfqpoint{3.854979in}{2.046121in}}%
\pgfpathlineto{\pgfqpoint{3.855748in}{2.050284in}}%
\pgfpathlineto{\pgfqpoint{3.855940in}{2.047017in}}%
\pgfpathlineto{\pgfqpoint{3.856516in}{2.044008in}}%
\pgfpathlineto{\pgfqpoint{3.856709in}{2.046278in}}%
\pgfpathlineto{\pgfqpoint{3.857285in}{2.045849in}}%
\pgfpathlineto{\pgfqpoint{3.858246in}{2.053921in}}%
\pgfpathlineto{\pgfqpoint{3.858822in}{2.053942in}}%
\pgfpathlineto{\pgfqpoint{3.859591in}{2.047814in}}%
\pgfpathlineto{\pgfqpoint{3.859975in}{2.050059in}}%
\pgfpathlineto{\pgfqpoint{3.860552in}{2.054132in}}%
\pgfpathlineto{\pgfqpoint{3.860936in}{2.050614in}}%
\pgfpathlineto{\pgfqpoint{3.861321in}{2.051313in}}%
\pgfpathlineto{\pgfqpoint{3.861513in}{2.048831in}}%
\pgfpathlineto{\pgfqpoint{3.862281in}{2.041090in}}%
\pgfpathlineto{\pgfqpoint{3.862666in}{2.046011in}}%
\pgfpathlineto{\pgfqpoint{3.864011in}{2.039124in}}%
\pgfpathlineto{\pgfqpoint{3.864203in}{2.041612in}}%
\pgfpathlineto{\pgfqpoint{3.864587in}{2.040217in}}%
\pgfpathlineto{\pgfqpoint{3.865933in}{2.035525in}}%
\pgfpathlineto{\pgfqpoint{3.866701in}{2.034418in}}%
\pgfpathlineto{\pgfqpoint{3.866317in}{2.035915in}}%
\pgfpathlineto{\pgfqpoint{3.866893in}{2.035735in}}%
\pgfpathlineto{\pgfqpoint{3.867662in}{2.036772in}}%
\pgfpathlineto{\pgfqpoint{3.867278in}{2.035081in}}%
\pgfpathlineto{\pgfqpoint{3.867854in}{2.035329in}}%
\pgfpathlineto{\pgfqpoint{3.869392in}{2.027899in}}%
\pgfpathlineto{\pgfqpoint{3.869584in}{2.029670in}}%
\pgfpathlineto{\pgfqpoint{3.870160in}{2.026295in}}%
\pgfpathlineto{\pgfqpoint{3.871506in}{2.038712in}}%
\pgfpathlineto{\pgfqpoint{3.871698in}{2.039696in}}%
\pgfpathlineto{\pgfqpoint{3.871890in}{2.036676in}}%
\pgfpathlineto{\pgfqpoint{3.872274in}{2.038612in}}%
\pgfpathlineto{\pgfqpoint{3.872466in}{2.037050in}}%
\pgfpathlineto{\pgfqpoint{3.872659in}{2.038739in}}%
\pgfpathlineto{\pgfqpoint{3.872851in}{2.038621in}}%
\pgfpathlineto{\pgfqpoint{3.874004in}{2.047632in}}%
\pgfpathlineto{\pgfqpoint{3.874196in}{2.046248in}}%
\pgfpathlineto{\pgfqpoint{3.875157in}{2.051951in}}%
\pgfpathlineto{\pgfqpoint{3.875349in}{2.050783in}}%
\pgfpathlineto{\pgfqpoint{3.875733in}{2.051635in}}%
\pgfpathlineto{\pgfqpoint{3.876694in}{2.047638in}}%
\pgfpathlineto{\pgfqpoint{3.877463in}{2.044278in}}%
\pgfpathlineto{\pgfqpoint{3.877847in}{2.050017in}}%
\pgfpathlineto{\pgfqpoint{3.878424in}{2.042018in}}%
\pgfpathlineto{\pgfqpoint{3.879192in}{2.046519in}}%
\pgfpathlineto{\pgfqpoint{3.879384in}{2.045643in}}%
\pgfpathlineto{\pgfqpoint{3.879577in}{2.049008in}}%
\pgfpathlineto{\pgfqpoint{3.880922in}{2.056454in}}%
\pgfpathlineto{\pgfqpoint{3.881114in}{2.054776in}}%
\pgfpathlineto{\pgfqpoint{3.881306in}{2.052231in}}%
\pgfpathlineto{\pgfqpoint{3.881883in}{2.057274in}}%
\pgfpathlineto{\pgfqpoint{3.882075in}{2.057857in}}%
\pgfpathlineto{\pgfqpoint{3.882459in}{2.056301in}}%
\pgfpathlineto{\pgfqpoint{3.883420in}{2.048871in}}%
\pgfpathlineto{\pgfqpoint{3.883804in}{2.049627in}}%
\pgfpathlineto{\pgfqpoint{3.884381in}{2.047012in}}%
\pgfpathlineto{\pgfqpoint{3.884765in}{2.048602in}}%
\pgfpathlineto{\pgfqpoint{3.886110in}{2.054936in}}%
\pgfpathlineto{\pgfqpoint{3.886302in}{2.054801in}}%
\pgfpathlineto{\pgfqpoint{3.887071in}{2.047193in}}%
\pgfpathlineto{\pgfqpoint{3.887840in}{2.052459in}}%
\pgfpathlineto{\pgfqpoint{3.888032in}{2.054685in}}%
\pgfpathlineto{\pgfqpoint{3.888608in}{2.052243in}}%
\pgfpathlineto{\pgfqpoint{3.888801in}{2.053128in}}%
\pgfpathlineto{\pgfqpoint{3.890914in}{2.042618in}}%
\pgfpathlineto{\pgfqpoint{3.891107in}{2.043796in}}%
\pgfpathlineto{\pgfqpoint{3.891683in}{2.047601in}}%
\pgfpathlineto{\pgfqpoint{3.892644in}{2.051179in}}%
\pgfpathlineto{\pgfqpoint{3.893797in}{2.041222in}}%
\pgfpathlineto{\pgfqpoint{3.893989in}{2.043169in}}%
\pgfpathlineto{\pgfqpoint{3.894181in}{2.045767in}}%
\pgfpathlineto{\pgfqpoint{3.894950in}{2.042652in}}%
\pgfpathlineto{\pgfqpoint{3.895334in}{2.039977in}}%
\pgfpathlineto{\pgfqpoint{3.895719in}{2.042843in}}%
\pgfpathlineto{\pgfqpoint{3.895911in}{2.042372in}}%
\pgfpathlineto{\pgfqpoint{3.896680in}{2.048616in}}%
\pgfpathlineto{\pgfqpoint{3.897256in}{2.047523in}}%
\pgfpathlineto{\pgfqpoint{3.897640in}{2.047634in}}%
\pgfpathlineto{\pgfqpoint{3.898601in}{2.040353in}}%
\pgfpathlineto{\pgfqpoint{3.899178in}{2.050911in}}%
\pgfpathlineto{\pgfqpoint{3.899946in}{2.048474in}}%
\pgfpathlineto{\pgfqpoint{3.900331in}{2.053143in}}%
\pgfpathlineto{\pgfqpoint{3.901099in}{2.050410in}}%
\pgfpathlineto{\pgfqpoint{3.901292in}{2.050055in}}%
\pgfpathlineto{\pgfqpoint{3.902637in}{2.057631in}}%
\pgfpathlineto{\pgfqpoint{3.902829in}{2.057077in}}%
\pgfpathlineto{\pgfqpoint{3.903790in}{2.053266in}}%
\pgfpathlineto{\pgfqpoint{3.903982in}{2.056166in}}%
\pgfpathlineto{\pgfqpoint{3.904174in}{2.056695in}}%
\pgfpathlineto{\pgfqpoint{3.904366in}{2.055222in}}%
\pgfpathlineto{\pgfqpoint{3.904558in}{2.053649in}}%
\pgfpathlineto{\pgfqpoint{3.905135in}{2.055957in}}%
\pgfpathlineto{\pgfqpoint{3.905519in}{2.054890in}}%
\pgfpathlineto{\pgfqpoint{3.905711in}{2.055589in}}%
\pgfpathlineto{\pgfqpoint{3.906096in}{2.048551in}}%
\pgfpathlineto{\pgfqpoint{3.906864in}{2.052930in}}%
\pgfpathlineto{\pgfqpoint{3.907249in}{2.052351in}}%
\pgfpathlineto{\pgfqpoint{3.907825in}{2.059011in}}%
\pgfpathlineto{\pgfqpoint{3.908402in}{2.057458in}}%
\pgfpathlineto{\pgfqpoint{3.909170in}{2.052875in}}%
\pgfpathlineto{\pgfqpoint{3.909555in}{2.055257in}}%
\pgfpathlineto{\pgfqpoint{3.910131in}{2.058962in}}%
\pgfpathlineto{\pgfqpoint{3.910708in}{2.057021in}}%
\pgfpathlineto{\pgfqpoint{3.911092in}{2.053944in}}%
\pgfpathlineto{\pgfqpoint{3.911669in}{2.056676in}}%
\pgfpathlineto{\pgfqpoint{3.912629in}{2.065122in}}%
\pgfpathlineto{\pgfqpoint{3.913206in}{2.061297in}}%
\pgfpathlineto{\pgfqpoint{3.913975in}{2.054344in}}%
\pgfpathlineto{\pgfqpoint{3.914359in}{2.057922in}}%
\pgfpathlineto{\pgfqpoint{3.915704in}{2.071490in}}%
\pgfpathlineto{\pgfqpoint{3.916088in}{2.070945in}}%
\pgfpathlineto{\pgfqpoint{3.916665in}{2.065537in}}%
\pgfpathlineto{\pgfqpoint{3.917434in}{2.067996in}}%
\pgfpathlineto{\pgfqpoint{3.917626in}{2.069619in}}%
\pgfpathlineto{\pgfqpoint{3.918202in}{2.066901in}}%
\pgfpathlineto{\pgfqpoint{3.918395in}{2.067904in}}%
\pgfpathlineto{\pgfqpoint{3.919932in}{2.058442in}}%
\pgfpathlineto{\pgfqpoint{3.920893in}{2.062852in}}%
\pgfpathlineto{\pgfqpoint{3.921277in}{2.062109in}}%
\pgfpathlineto{\pgfqpoint{3.921661in}{2.064617in}}%
\pgfpathlineto{\pgfqpoint{3.923967in}{2.080225in}}%
\pgfpathlineto{\pgfqpoint{3.924160in}{2.077979in}}%
\pgfpathlineto{\pgfqpoint{3.924928in}{2.081778in}}%
\pgfpathlineto{\pgfqpoint{3.925313in}{2.079154in}}%
\pgfpathlineto{\pgfqpoint{3.926850in}{2.070593in}}%
\pgfpathlineto{\pgfqpoint{3.927811in}{2.073547in}}%
\pgfpathlineto{\pgfqpoint{3.927426in}{2.070303in}}%
\pgfpathlineto{\pgfqpoint{3.928003in}{2.071589in}}%
\pgfpathlineto{\pgfqpoint{3.928195in}{2.071063in}}%
\pgfpathlineto{\pgfqpoint{3.928579in}{2.072940in}}%
\pgfpathlineto{\pgfqpoint{3.929156in}{2.076658in}}%
\pgfpathlineto{\pgfqpoint{3.929732in}{2.073921in}}%
\pgfpathlineto{\pgfqpoint{3.930501in}{2.070377in}}%
\pgfpathlineto{\pgfqpoint{3.930693in}{2.071962in}}%
\pgfpathlineto{\pgfqpoint{3.931654in}{2.081302in}}%
\pgfpathlineto{\pgfqpoint{3.932038in}{2.078175in}}%
\pgfpathlineto{\pgfqpoint{3.932231in}{2.077185in}}%
\pgfpathlineto{\pgfqpoint{3.932423in}{2.081108in}}%
\pgfpathlineto{\pgfqpoint{3.932807in}{2.079158in}}%
\pgfpathlineto{\pgfqpoint{3.933191in}{2.077892in}}%
\pgfpathlineto{\pgfqpoint{3.936074in}{2.087191in}}%
\pgfpathlineto{\pgfqpoint{3.937803in}{2.098864in}}%
\pgfpathlineto{\pgfqpoint{3.937996in}{2.097042in}}%
\pgfpathlineto{\pgfqpoint{3.938380in}{2.099454in}}%
\pgfpathlineto{\pgfqpoint{3.938956in}{2.098133in}}%
\pgfpathlineto{\pgfqpoint{3.939149in}{2.098768in}}%
\pgfpathlineto{\pgfqpoint{3.939341in}{2.096232in}}%
\pgfpathlineto{\pgfqpoint{3.940109in}{2.092150in}}%
\pgfpathlineto{\pgfqpoint{3.940494in}{2.094782in}}%
\pgfpathlineto{\pgfqpoint{3.941647in}{2.097993in}}%
\pgfpathlineto{\pgfqpoint{3.941839in}{2.097361in}}%
\pgfpathlineto{\pgfqpoint{3.942031in}{2.098168in}}%
\pgfpathlineto{\pgfqpoint{3.942223in}{2.096048in}}%
\pgfpathlineto{\pgfqpoint{3.942608in}{2.091847in}}%
\pgfpathlineto{\pgfqpoint{3.943376in}{2.094791in}}%
\pgfpathlineto{\pgfqpoint{3.944722in}{2.097944in}}%
\pgfpathlineto{\pgfqpoint{3.945106in}{2.097509in}}%
\pgfpathlineto{\pgfqpoint{3.945298in}{2.099502in}}%
\pgfpathlineto{\pgfqpoint{3.945875in}{2.105306in}}%
\pgfpathlineto{\pgfqpoint{3.946451in}{2.104086in}}%
\pgfpathlineto{\pgfqpoint{3.946643in}{2.101471in}}%
\pgfpathlineto{\pgfqpoint{3.947220in}{2.104121in}}%
\pgfpathlineto{\pgfqpoint{3.948181in}{2.106678in}}%
\pgfpathlineto{\pgfqpoint{3.948373in}{2.105213in}}%
\pgfpathlineto{\pgfqpoint{3.948949in}{2.106899in}}%
\pgfpathlineto{\pgfqpoint{3.949141in}{2.105284in}}%
\pgfpathlineto{\pgfqpoint{3.949334in}{2.099405in}}%
\pgfpathlineto{\pgfqpoint{3.950102in}{2.107769in}}%
\pgfpathlineto{\pgfqpoint{3.951447in}{2.115560in}}%
\pgfpathlineto{\pgfqpoint{3.951640in}{2.114966in}}%
\pgfpathlineto{\pgfqpoint{3.952600in}{2.110924in}}%
\pgfpathlineto{\pgfqpoint{3.952793in}{2.113039in}}%
\pgfpathlineto{\pgfqpoint{3.953561in}{2.118861in}}%
\pgfpathlineto{\pgfqpoint{3.954330in}{2.116850in}}%
\pgfpathlineto{\pgfqpoint{3.954522in}{2.117462in}}%
\pgfpathlineto{\pgfqpoint{3.954714in}{2.116403in}}%
\pgfpathlineto{\pgfqpoint{3.955099in}{2.113140in}}%
\pgfpathlineto{\pgfqpoint{3.955483in}{2.117513in}}%
\pgfpathlineto{\pgfqpoint{3.955867in}{2.120199in}}%
\pgfpathlineto{\pgfqpoint{3.956636in}{2.118182in}}%
\pgfpathlineto{\pgfqpoint{3.956828in}{2.117176in}}%
\pgfpathlineto{\pgfqpoint{3.957020in}{2.120888in}}%
\pgfpathlineto{\pgfqpoint{3.957981in}{2.127232in}}%
\pgfpathlineto{\pgfqpoint{3.958558in}{2.124307in}}%
\pgfpathlineto{\pgfqpoint{3.959518in}{2.120527in}}%
\pgfpathlineto{\pgfqpoint{3.960479in}{2.128733in}}%
\pgfpathlineto{\pgfqpoint{3.960671in}{2.123206in}}%
\pgfpathlineto{\pgfqpoint{3.961056in}{2.118008in}}%
\pgfpathlineto{\pgfqpoint{3.961824in}{2.119194in}}%
\pgfpathlineto{\pgfqpoint{3.962017in}{2.120496in}}%
\pgfpathlineto{\pgfqpoint{3.962209in}{2.118212in}}%
\pgfpathlineto{\pgfqpoint{3.962401in}{2.114959in}}%
\pgfpathlineto{\pgfqpoint{3.963170in}{2.118251in}}%
\pgfpathlineto{\pgfqpoint{3.963554in}{2.119532in}}%
\pgfpathlineto{\pgfqpoint{3.964323in}{2.112988in}}%
\pgfpathlineto{\pgfqpoint{3.964515in}{2.110199in}}%
\pgfpathlineto{\pgfqpoint{3.965283in}{2.113796in}}%
\pgfpathlineto{\pgfqpoint{3.965668in}{2.114335in}}%
\pgfpathlineto{\pgfqpoint{3.967782in}{2.104006in}}%
\pgfpathlineto{\pgfqpoint{3.968743in}{2.107045in}}%
\pgfpathlineto{\pgfqpoint{3.968935in}{2.106607in}}%
\pgfpathlineto{\pgfqpoint{3.971241in}{2.092315in}}%
\pgfpathlineto{\pgfqpoint{3.971817in}{2.097613in}}%
\pgfpathlineto{\pgfqpoint{3.972970in}{2.102292in}}%
\pgfpathlineto{\pgfqpoint{3.973162in}{2.099766in}}%
\pgfpathlineto{\pgfqpoint{3.973355in}{2.096359in}}%
\pgfpathlineto{\pgfqpoint{3.974315in}{2.098914in}}%
\pgfpathlineto{\pgfqpoint{3.974892in}{2.100964in}}%
\pgfpathlineto{\pgfqpoint{3.975468in}{2.099953in}}%
\pgfpathlineto{\pgfqpoint{3.975661in}{2.099062in}}%
\pgfpathlineto{\pgfqpoint{3.975853in}{2.102293in}}%
\pgfpathlineto{\pgfqpoint{3.976621in}{2.102963in}}%
\pgfpathlineto{\pgfqpoint{3.978159in}{2.099575in}}%
\pgfpathlineto{\pgfqpoint{3.979312in}{2.108064in}}%
\pgfpathlineto{\pgfqpoint{3.979888in}{2.106322in}}%
\pgfpathlineto{\pgfqpoint{3.980465in}{2.101263in}}%
\pgfpathlineto{\pgfqpoint{3.981041in}{2.101565in}}%
\pgfpathlineto{\pgfqpoint{3.982386in}{2.110943in}}%
\pgfpathlineto{\pgfqpoint{3.982579in}{2.109834in}}%
\pgfpathlineto{\pgfqpoint{3.983155in}{2.111480in}}%
\pgfpathlineto{\pgfqpoint{3.983539in}{2.107727in}}%
\pgfpathlineto{\pgfqpoint{3.983732in}{2.110277in}}%
\pgfpathlineto{\pgfqpoint{3.984500in}{2.106226in}}%
\pgfpathlineto{\pgfqpoint{3.984885in}{2.104677in}}%
\pgfpathlineto{\pgfqpoint{3.985077in}{2.107342in}}%
\pgfpathlineto{\pgfqpoint{3.985653in}{2.108317in}}%
\pgfpathlineto{\pgfqpoint{3.985461in}{2.106991in}}%
\pgfpathlineto{\pgfqpoint{3.986038in}{2.107722in}}%
\pgfpathlineto{\pgfqpoint{3.986998in}{2.108310in}}%
\pgfpathlineto{\pgfqpoint{3.987767in}{2.101243in}}%
\pgfpathlineto{\pgfqpoint{3.988536in}{2.097213in}}%
\pgfpathlineto{\pgfqpoint{3.989112in}{2.098881in}}%
\pgfpathlineto{\pgfqpoint{3.989689in}{2.094585in}}%
\pgfpathlineto{\pgfqpoint{3.991034in}{2.090140in}}%
\pgfpathlineto{\pgfqpoint{3.991418in}{2.091653in}}%
\pgfpathlineto{\pgfqpoint{3.991803in}{2.094903in}}%
\pgfpathlineto{\pgfqpoint{3.992379in}{2.091316in}}%
\pgfpathlineto{\pgfqpoint{3.992571in}{2.092462in}}%
\pgfpathlineto{\pgfqpoint{3.994685in}{2.106254in}}%
\pgfpathlineto{\pgfqpoint{3.994877in}{2.107981in}}%
\pgfpathlineto{\pgfqpoint{3.995262in}{2.105489in}}%
\pgfpathlineto{\pgfqpoint{3.995646in}{2.105686in}}%
\pgfpathlineto{\pgfqpoint{3.995838in}{2.104110in}}%
\pgfpathlineto{\pgfqpoint{3.996415in}{2.106335in}}%
\pgfpathlineto{\pgfqpoint{3.997376in}{2.115853in}}%
\pgfpathlineto{\pgfqpoint{3.997760in}{2.112848in}}%
\pgfpathlineto{\pgfqpoint{3.999105in}{2.116822in}}%
\pgfpathlineto{\pgfqpoint{3.999297in}{2.116048in}}%
\pgfpathlineto{\pgfqpoint{3.999489in}{2.113761in}}%
\pgfpathlineto{\pgfqpoint{4.000066in}{2.116413in}}%
\pgfpathlineto{\pgfqpoint{4.000258in}{2.118852in}}%
\pgfpathlineto{\pgfqpoint{4.000835in}{2.114946in}}%
\pgfpathlineto{\pgfqpoint{4.001027in}{2.111333in}}%
\pgfpathlineto{\pgfqpoint{4.001411in}{2.116284in}}%
\pgfpathlineto{\pgfqpoint{4.001988in}{2.112857in}}%
\pgfpathlineto{\pgfqpoint{4.003525in}{2.104082in}}%
\pgfpathlineto{\pgfqpoint{4.003909in}{2.106134in}}%
\pgfpathlineto{\pgfqpoint{4.004870in}{2.102151in}}%
\pgfpathlineto{\pgfqpoint{4.004294in}{2.107625in}}%
\pgfpathlineto{\pgfqpoint{4.005447in}{2.102824in}}%
\pgfpathlineto{\pgfqpoint{4.006023in}{2.104540in}}%
\pgfpathlineto{\pgfqpoint{4.006407in}{2.102909in}}%
\pgfpathlineto{\pgfqpoint{4.006792in}{2.101647in}}%
\pgfpathlineto{\pgfqpoint{4.006984in}{2.102657in}}%
\pgfpathlineto{\pgfqpoint{4.007753in}{2.111497in}}%
\pgfpathlineto{\pgfqpoint{4.008137in}{2.105010in}}%
\pgfpathlineto{\pgfqpoint{4.008521in}{2.100565in}}%
\pgfpathlineto{\pgfqpoint{4.009098in}{2.106089in}}%
\pgfpathlineto{\pgfqpoint{4.010635in}{2.100213in}}%
\pgfpathlineto{\pgfqpoint{4.011019in}{2.102457in}}%
\pgfpathlineto{\pgfqpoint{4.011404in}{2.101751in}}%
\pgfpathlineto{\pgfqpoint{4.011980in}{2.096817in}}%
\pgfpathlineto{\pgfqpoint{4.011788in}{2.101883in}}%
\pgfpathlineto{\pgfqpoint{4.012749in}{2.096903in}}%
\pgfpathlineto{\pgfqpoint{4.012941in}{2.097344in}}%
\pgfpathlineto{\pgfqpoint{4.013133in}{2.093923in}}%
\pgfpathlineto{\pgfqpoint{4.013902in}{2.097835in}}%
\pgfpathlineto{\pgfqpoint{4.014094in}{2.097446in}}%
\pgfpathlineto{\pgfqpoint{4.016208in}{2.116123in}}%
\pgfpathlineto{\pgfqpoint{4.016592in}{2.113538in}}%
\pgfpathlineto{\pgfqpoint{4.018898in}{2.119400in}}%
\pgfpathlineto{\pgfqpoint{4.019091in}{2.119249in}}%
\pgfpathlineto{\pgfqpoint{4.019283in}{2.118322in}}%
\pgfpathlineto{\pgfqpoint{4.019475in}{2.121770in}}%
\pgfpathlineto{\pgfqpoint{4.021204in}{2.136704in}}%
\pgfpathlineto{\pgfqpoint{4.021589in}{2.135027in}}%
\pgfpathlineto{\pgfqpoint{4.021973in}{2.137402in}}%
\pgfpathlineto{\pgfqpoint{4.023318in}{2.143948in}}%
\pgfpathlineto{\pgfqpoint{4.023510in}{2.142178in}}%
\pgfpathlineto{\pgfqpoint{4.024087in}{2.142365in}}%
\pgfpathlineto{\pgfqpoint{4.024279in}{2.141126in}}%
\pgfpathlineto{\pgfqpoint{4.024856in}{2.144392in}}%
\pgfpathlineto{\pgfqpoint{4.025048in}{2.143102in}}%
\pgfpathlineto{\pgfqpoint{4.025240in}{2.144010in}}%
\pgfpathlineto{\pgfqpoint{4.025432in}{2.140738in}}%
\pgfpathlineto{\pgfqpoint{4.025624in}{2.142448in}}%
\pgfpathlineto{\pgfqpoint{4.026009in}{2.140161in}}%
\pgfpathlineto{\pgfqpoint{4.026393in}{2.143568in}}%
\pgfpathlineto{\pgfqpoint{4.026585in}{2.142770in}}%
\pgfpathlineto{\pgfqpoint{4.027162in}{2.141998in}}%
\pgfpathlineto{\pgfqpoint{4.028315in}{2.143742in}}%
\pgfpathlineto{\pgfqpoint{4.028699in}{2.141433in}}%
\pgfpathlineto{\pgfqpoint{4.028891in}{2.145173in}}%
\pgfpathlineto{\pgfqpoint{4.029852in}{2.151005in}}%
\pgfpathlineto{\pgfqpoint{4.030236in}{2.148489in}}%
\pgfpathlineto{\pgfqpoint{4.030428in}{2.149624in}}%
\pgfpathlineto{\pgfqpoint{4.030813in}{2.146324in}}%
\pgfpathlineto{\pgfqpoint{4.031197in}{2.148284in}}%
\pgfpathlineto{\pgfqpoint{4.032734in}{2.142755in}}%
\pgfpathlineto{\pgfqpoint{4.032927in}{2.143217in}}%
\pgfpathlineto{\pgfqpoint{4.033311in}{2.139023in}}%
\pgfpathlineto{\pgfqpoint{4.034080in}{2.141659in}}%
\pgfpathlineto{\pgfqpoint{4.034464in}{2.138592in}}%
\pgfpathlineto{\pgfqpoint{4.034848in}{2.139879in}}%
\pgfpathlineto{\pgfqpoint{4.036001in}{2.148846in}}%
\pgfpathlineto{\pgfqpoint{4.036193in}{2.148718in}}%
\pgfpathlineto{\pgfqpoint{4.036962in}{2.147333in}}%
\pgfpathlineto{\pgfqpoint{4.037346in}{2.151395in}}%
\pgfpathlineto{\pgfqpoint{4.037923in}{2.147767in}}%
\pgfpathlineto{\pgfqpoint{4.038115in}{2.144831in}}%
\pgfpathlineto{\pgfqpoint{4.038884in}{2.146868in}}%
\pgfpathlineto{\pgfqpoint{4.039268in}{2.148343in}}%
\pgfpathlineto{\pgfqpoint{4.039460in}{2.145725in}}%
\pgfpathlineto{\pgfqpoint{4.039652in}{2.144034in}}%
\pgfpathlineto{\pgfqpoint{4.040421in}{2.145954in}}%
\pgfpathlineto{\pgfqpoint{4.041382in}{2.147662in}}%
\pgfpathlineto{\pgfqpoint{4.041574in}{2.144633in}}%
\pgfpathlineto{\pgfqpoint{4.042343in}{2.146797in}}%
\pgfpathlineto{\pgfqpoint{4.044072in}{2.159485in}}%
\pgfpathlineto{\pgfqpoint{4.045225in}{2.154189in}}%
\pgfpathlineto{\pgfqpoint{4.045418in}{2.154952in}}%
\pgfpathlineto{\pgfqpoint{4.046763in}{2.160370in}}%
\pgfpathlineto{\pgfqpoint{4.048300in}{2.168802in}}%
\pgfpathlineto{\pgfqpoint{4.048492in}{2.167466in}}%
\pgfpathlineto{\pgfqpoint{4.050222in}{2.150072in}}%
\pgfpathlineto{\pgfqpoint{4.050990in}{2.153387in}}%
\pgfpathlineto{\pgfqpoint{4.050606in}{2.148310in}}%
\pgfpathlineto{\pgfqpoint{4.051183in}{2.150992in}}%
\pgfpathlineto{\pgfqpoint{4.051567in}{2.144665in}}%
\pgfpathlineto{\pgfqpoint{4.052336in}{2.147758in}}%
\pgfpathlineto{\pgfqpoint{4.052720in}{2.145529in}}%
\pgfpathlineto{\pgfqpoint{4.052912in}{2.141703in}}%
\pgfpathlineto{\pgfqpoint{4.053873in}{2.143715in}}%
\pgfpathlineto{\pgfqpoint{4.055602in}{2.157909in}}%
\pgfpathlineto{\pgfqpoint{4.055987in}{2.152634in}}%
\pgfpathlineto{\pgfqpoint{4.056755in}{2.154966in}}%
\pgfpathlineto{\pgfqpoint{4.057140in}{2.152834in}}%
\pgfpathlineto{\pgfqpoint{4.058293in}{2.158184in}}%
\pgfpathlineto{\pgfqpoint{4.058869in}{2.154926in}}%
\pgfpathlineto{\pgfqpoint{4.059446in}{2.156068in}}%
\pgfpathlineto{\pgfqpoint{4.060599in}{2.164106in}}%
\pgfpathlineto{\pgfqpoint{4.060791in}{2.162909in}}%
\pgfpathlineto{\pgfqpoint{4.060983in}{2.161741in}}%
\pgfpathlineto{\pgfqpoint{4.061560in}{2.163639in}}%
\pgfpathlineto{\pgfqpoint{4.061752in}{2.163166in}}%
\pgfpathlineto{\pgfqpoint{4.062328in}{2.168747in}}%
\pgfpathlineto{\pgfqpoint{4.063097in}{2.167255in}}%
\pgfpathlineto{\pgfqpoint{4.063289in}{2.167253in}}%
\pgfpathlineto{\pgfqpoint{4.063866in}{2.159169in}}%
\pgfpathlineto{\pgfqpoint{4.064250in}{2.163240in}}%
\pgfpathlineto{\pgfqpoint{4.064442in}{2.167085in}}%
\pgfpathlineto{\pgfqpoint{4.065403in}{2.164973in}}%
\pgfpathlineto{\pgfqpoint{4.066172in}{2.161263in}}%
\pgfpathlineto{\pgfqpoint{4.066556in}{2.164318in}}%
\pgfpathlineto{\pgfqpoint{4.069246in}{2.150134in}}%
\pgfpathlineto{\pgfqpoint{4.069439in}{2.150914in}}%
\pgfpathlineto{\pgfqpoint{4.069823in}{2.153496in}}%
\pgfpathlineto{\pgfqpoint{4.070015in}{2.151171in}}%
\pgfpathlineto{\pgfqpoint{4.070784in}{2.143035in}}%
\pgfpathlineto{\pgfqpoint{4.071168in}{2.146613in}}%
\pgfpathlineto{\pgfqpoint{4.073090in}{2.157545in}}%
\pgfpathlineto{\pgfqpoint{4.073282in}{2.156906in}}%
\pgfpathlineto{\pgfqpoint{4.074627in}{2.152683in}}%
\pgfpathlineto{\pgfqpoint{4.074819in}{2.152843in}}%
\pgfpathlineto{\pgfqpoint{4.076164in}{2.161655in}}%
\pgfpathlineto{\pgfqpoint{4.076741in}{2.161427in}}%
\pgfpathlineto{\pgfqpoint{4.078470in}{2.153048in}}%
\pgfpathlineto{\pgfqpoint{4.078663in}{2.155554in}}%
\pgfpathlineto{\pgfqpoint{4.078855in}{2.156053in}}%
\pgfpathlineto{\pgfqpoint{4.080584in}{2.144101in}}%
\pgfpathlineto{\pgfqpoint{4.080969in}{2.146072in}}%
\pgfpathlineto{\pgfqpoint{4.081545in}{2.144142in}}%
\pgfpathlineto{\pgfqpoint{4.081737in}{2.141197in}}%
\pgfpathlineto{\pgfqpoint{4.082314in}{2.145172in}}%
\pgfpathlineto{\pgfqpoint{4.082506in}{2.146835in}}%
\pgfpathlineto{\pgfqpoint{4.082890in}{2.141217in}}%
\pgfpathlineto{\pgfqpoint{4.083082in}{2.141067in}}%
\pgfpathlineto{\pgfqpoint{4.085004in}{2.132625in}}%
\pgfpathlineto{\pgfqpoint{4.085196in}{2.134072in}}%
\pgfpathlineto{\pgfqpoint{4.085388in}{2.131472in}}%
\pgfpathlineto{\pgfqpoint{4.085965in}{2.132815in}}%
\pgfpathlineto{\pgfqpoint{4.086157in}{2.129543in}}%
\pgfpathlineto{\pgfqpoint{4.086734in}{2.133299in}}%
\pgfpathlineto{\pgfqpoint{4.086926in}{2.135634in}}%
\pgfpathlineto{\pgfqpoint{4.087118in}{2.132929in}}%
\pgfpathlineto{\pgfqpoint{4.087502in}{2.134835in}}%
\pgfpathlineto{\pgfqpoint{4.087887in}{2.130697in}}%
\pgfpathlineto{\pgfqpoint{4.088655in}{2.131758in}}%
\pgfpathlineto{\pgfqpoint{4.089040in}{2.132025in}}%
\pgfpathlineto{\pgfqpoint{4.089232in}{2.129029in}}%
\pgfpathlineto{\pgfqpoint{4.090001in}{2.134303in}}%
\pgfpathlineto{\pgfqpoint{4.090961in}{2.133396in}}%
\pgfpathlineto{\pgfqpoint{4.091154in}{2.135649in}}%
\pgfpathlineto{\pgfqpoint{4.091346in}{2.134773in}}%
\pgfpathlineto{\pgfqpoint{4.091730in}{2.137721in}}%
\pgfpathlineto{\pgfqpoint{4.091922in}{2.138912in}}%
\pgfpathlineto{\pgfqpoint{4.092307in}{2.134682in}}%
\pgfpathlineto{\pgfqpoint{4.092691in}{2.137149in}}%
\pgfpathlineto{\pgfqpoint{4.092883in}{2.136951in}}%
\pgfpathlineto{\pgfqpoint{4.094228in}{2.126454in}}%
\pgfpathlineto{\pgfqpoint{4.094613in}{2.128920in}}%
\pgfpathlineto{\pgfqpoint{4.094997in}{2.132499in}}%
\pgfpathlineto{\pgfqpoint{4.095573in}{2.128713in}}%
\pgfpathlineto{\pgfqpoint{4.096534in}{2.125052in}}%
\pgfpathlineto{\pgfqpoint{4.096726in}{2.126475in}}%
\pgfpathlineto{\pgfqpoint{4.097303in}{2.129057in}}%
\pgfpathlineto{\pgfqpoint{4.098264in}{2.117676in}}%
\pgfpathlineto{\pgfqpoint{4.098840in}{2.115615in}}%
\pgfpathlineto{\pgfqpoint{4.099032in}{2.114525in}}%
\pgfpathlineto{\pgfqpoint{4.099417in}{2.117946in}}%
\pgfpathlineto{\pgfqpoint{4.101338in}{2.125822in}}%
\pgfpathlineto{\pgfqpoint{4.102107in}{2.120298in}}%
\pgfpathlineto{\pgfqpoint{4.102491in}{2.121972in}}%
\pgfpathlineto{\pgfqpoint{4.104797in}{2.132402in}}%
\pgfpathlineto{\pgfqpoint{4.104990in}{2.130758in}}%
\pgfpathlineto{\pgfqpoint{4.105374in}{2.135517in}}%
\pgfpathlineto{\pgfqpoint{4.105566in}{2.134802in}}%
\pgfpathlineto{\pgfqpoint{4.106719in}{2.137737in}}%
\pgfpathlineto{\pgfqpoint{4.108449in}{2.121823in}}%
\pgfpathlineto{\pgfqpoint{4.109986in}{2.130730in}}%
\pgfpathlineto{\pgfqpoint{4.110178in}{2.129337in}}%
\pgfpathlineto{\pgfqpoint{4.110755in}{2.131332in}}%
\pgfpathlineto{\pgfqpoint{4.110947in}{2.131224in}}%
\pgfpathlineto{\pgfqpoint{4.111908in}{2.134953in}}%
\pgfpathlineto{\pgfqpoint{4.112292in}{2.133419in}}%
\pgfpathlineto{\pgfqpoint{4.112484in}{2.133605in}}%
\pgfpathlineto{\pgfqpoint{4.112676in}{2.136406in}}%
\pgfpathlineto{\pgfqpoint{4.113253in}{2.131752in}}%
\pgfpathlineto{\pgfqpoint{4.114406in}{2.127496in}}%
\pgfpathlineto{\pgfqpoint{4.114598in}{2.128888in}}%
\pgfpathlineto{\pgfqpoint{4.115559in}{2.131702in}}%
\pgfpathlineto{\pgfqpoint{4.114982in}{2.128389in}}%
\pgfpathlineto{\pgfqpoint{4.115751in}{2.129778in}}%
\pgfpathlineto{\pgfqpoint{4.115943in}{2.129778in}}%
\pgfpathlineto{\pgfqpoint{4.116520in}{2.128197in}}%
\pgfpathlineto{\pgfqpoint{4.116904in}{2.129651in}}%
\pgfpathlineto{\pgfqpoint{4.117096in}{2.133979in}}%
\pgfpathlineto{\pgfqpoint{4.117865in}{2.127190in}}%
\pgfpathlineto{\pgfqpoint{4.118249in}{2.127588in}}%
\pgfpathlineto{\pgfqpoint{4.118634in}{2.125525in}}%
\pgfpathlineto{\pgfqpoint{4.119018in}{2.128630in}}%
\pgfpathlineto{\pgfqpoint{4.120555in}{2.143767in}}%
\pgfpathlineto{\pgfqpoint{4.120747in}{2.138576in}}%
\pgfpathlineto{\pgfqpoint{4.122093in}{2.133218in}}%
\pgfpathlineto{\pgfqpoint{4.123053in}{2.136033in}}%
\pgfpathlineto{\pgfqpoint{4.122477in}{2.131732in}}%
\pgfpathlineto{\pgfqpoint{4.123438in}{2.135545in}}%
\pgfpathlineto{\pgfqpoint{4.124783in}{2.128550in}}%
\pgfpathlineto{\pgfqpoint{4.124975in}{2.128874in}}%
\pgfpathlineto{\pgfqpoint{4.125359in}{2.127264in}}%
\pgfpathlineto{\pgfqpoint{4.125552in}{2.130613in}}%
\pgfpathlineto{\pgfqpoint{4.125744in}{2.130283in}}%
\pgfpathlineto{\pgfqpoint{4.126320in}{2.133993in}}%
\pgfpathlineto{\pgfqpoint{4.127089in}{2.132964in}}%
\pgfpathlineto{\pgfqpoint{4.128626in}{2.125777in}}%
\pgfpathlineto{\pgfqpoint{4.129971in}{2.135353in}}%
\pgfpathlineto{\pgfqpoint{4.130356in}{2.134767in}}%
\pgfpathlineto{\pgfqpoint{4.130548in}{2.134429in}}%
\pgfpathlineto{\pgfqpoint{4.130740in}{2.134760in}}%
\pgfpathlineto{\pgfqpoint{4.132470in}{2.141703in}}%
\pgfpathlineto{\pgfqpoint{4.134007in}{2.130656in}}%
\pgfpathlineto{\pgfqpoint{4.134583in}{2.131016in}}%
\pgfpathlineto{\pgfqpoint{4.134776in}{2.131769in}}%
\pgfpathlineto{\pgfqpoint{4.134968in}{2.130030in}}%
\pgfpathlineto{\pgfqpoint{4.135160in}{2.130124in}}%
\pgfpathlineto{\pgfqpoint{4.135352in}{2.127941in}}%
\pgfpathlineto{\pgfqpoint{4.135929in}{2.129930in}}%
\pgfpathlineto{\pgfqpoint{4.136121in}{2.133160in}}%
\pgfpathlineto{\pgfqpoint{4.136505in}{2.126523in}}%
\pgfpathlineto{\pgfqpoint{4.136697in}{2.127066in}}%
\pgfpathlineto{\pgfqpoint{4.137082in}{2.127486in}}%
\pgfpathlineto{\pgfqpoint{4.138427in}{2.134931in}}%
\pgfpathlineto{\pgfqpoint{4.138619in}{2.130898in}}%
\pgfpathlineto{\pgfqpoint{4.139388in}{2.136989in}}%
\pgfpathlineto{\pgfqpoint{4.139580in}{2.139935in}}%
\pgfpathlineto{\pgfqpoint{4.140349in}{2.134731in}}%
\pgfpathlineto{\pgfqpoint{4.140733in}{2.134179in}}%
\pgfpathlineto{\pgfqpoint{4.141309in}{2.138276in}}%
\pgfpathlineto{\pgfqpoint{4.142078in}{2.140875in}}%
\pgfpathlineto{\pgfqpoint{4.142270in}{2.139301in}}%
\pgfpathlineto{\pgfqpoint{4.142462in}{2.136435in}}%
\pgfpathlineto{\pgfqpoint{4.143039in}{2.142545in}}%
\pgfpathlineto{\pgfqpoint{4.143231in}{2.144580in}}%
\pgfpathlineto{\pgfqpoint{4.143615in}{2.141439in}}%
\pgfpathlineto{\pgfqpoint{4.144000in}{2.142228in}}%
\pgfpathlineto{\pgfqpoint{4.144576in}{2.145380in}}%
\pgfpathlineto{\pgfqpoint{4.144961in}{2.143460in}}%
\pgfpathlineto{\pgfqpoint{4.145537in}{2.137524in}}%
\pgfpathlineto{\pgfqpoint{4.146306in}{2.139477in}}%
\pgfpathlineto{\pgfqpoint{4.147267in}{2.141530in}}%
\pgfpathlineto{\pgfqpoint{4.146882in}{2.138481in}}%
\pgfpathlineto{\pgfqpoint{4.147459in}{2.141473in}}%
\pgfpathlineto{\pgfqpoint{4.148996in}{2.128047in}}%
\pgfpathlineto{\pgfqpoint{4.149188in}{2.128616in}}%
\pgfpathlineto{\pgfqpoint{4.149573in}{2.127263in}}%
\pgfpathlineto{\pgfqpoint{4.149765in}{2.129290in}}%
\pgfpathlineto{\pgfqpoint{4.151110in}{2.131492in}}%
\pgfpathlineto{\pgfqpoint{4.151494in}{2.130263in}}%
\pgfpathlineto{\pgfqpoint{4.151879in}{2.132028in}}%
\pgfpathlineto{\pgfqpoint{4.153032in}{2.140061in}}%
\pgfpathlineto{\pgfqpoint{4.153224in}{2.139230in}}%
\pgfpathlineto{\pgfqpoint{4.153800in}{2.131436in}}%
\pgfpathlineto{\pgfqpoint{4.154377in}{2.137842in}}%
\pgfpathlineto{\pgfqpoint{4.155722in}{2.142009in}}%
\pgfpathlineto{\pgfqpoint{4.155145in}{2.136254in}}%
\pgfpathlineto{\pgfqpoint{4.155914in}{2.141885in}}%
\pgfpathlineto{\pgfqpoint{4.156298in}{2.143517in}}%
\pgfpathlineto{\pgfqpoint{4.156491in}{2.142695in}}%
\pgfpathlineto{\pgfqpoint{4.157836in}{2.150002in}}%
\pgfpathlineto{\pgfqpoint{4.159950in}{2.136834in}}%
\pgfpathlineto{\pgfqpoint{4.160142in}{2.139227in}}%
\pgfpathlineto{\pgfqpoint{4.160334in}{2.139847in}}%
\pgfpathlineto{\pgfqpoint{4.160526in}{2.137336in}}%
\pgfpathlineto{\pgfqpoint{4.160718in}{2.139052in}}%
\pgfpathlineto{\pgfqpoint{4.162063in}{2.134969in}}%
\pgfpathlineto{\pgfqpoint{4.162448in}{2.135199in}}%
\pgfpathlineto{\pgfqpoint{4.162640in}{2.135627in}}%
\pgfpathlineto{\pgfqpoint{4.163985in}{2.142322in}}%
\pgfpathlineto{\pgfqpoint{4.164754in}{2.144989in}}%
\pgfpathlineto{\pgfqpoint{4.164946in}{2.141526in}}%
\pgfpathlineto{\pgfqpoint{4.165330in}{2.141111in}}%
\pgfpathlineto{\pgfqpoint{4.165907in}{2.142330in}}%
\pgfpathlineto{\pgfqpoint{4.167252in}{2.133907in}}%
\pgfpathlineto{\pgfqpoint{4.167444in}{2.134721in}}%
\pgfpathlineto{\pgfqpoint{4.167636in}{2.133080in}}%
\pgfpathlineto{\pgfqpoint{4.167829in}{2.133225in}}%
\pgfpathlineto{\pgfqpoint{4.168405in}{2.130604in}}%
\pgfpathlineto{\pgfqpoint{4.168597in}{2.131698in}}%
\pgfpathlineto{\pgfqpoint{4.169558in}{2.140111in}}%
\pgfpathlineto{\pgfqpoint{4.170135in}{2.139548in}}%
\pgfpathlineto{\pgfqpoint{4.170327in}{2.140299in}}%
\pgfpathlineto{\pgfqpoint{4.170711in}{2.138056in}}%
\pgfpathlineto{\pgfqpoint{4.172056in}{2.134836in}}%
\pgfpathlineto{\pgfqpoint{4.172441in}{2.128518in}}%
\pgfpathlineto{\pgfqpoint{4.173017in}{2.133851in}}%
\pgfpathlineto{\pgfqpoint{4.173594in}{2.136144in}}%
\pgfpathlineto{\pgfqpoint{4.173786in}{2.133717in}}%
\pgfpathlineto{\pgfqpoint{4.175515in}{2.125862in}}%
\pgfpathlineto{\pgfqpoint{4.175707in}{2.127471in}}%
\pgfpathlineto{\pgfqpoint{4.176668in}{2.130237in}}%
\pgfpathlineto{\pgfqpoint{4.178974in}{2.137305in}}%
\pgfpathlineto{\pgfqpoint{4.179166in}{2.135214in}}%
\pgfpathlineto{\pgfqpoint{4.179935in}{2.132712in}}%
\pgfpathlineto{\pgfqpoint{4.181088in}{2.144042in}}%
\pgfpathlineto{\pgfqpoint{4.181280in}{2.143557in}}%
\pgfpathlineto{\pgfqpoint{4.182818in}{2.133654in}}%
\pgfpathlineto{\pgfqpoint{4.183394in}{2.140968in}}%
\pgfpathlineto{\pgfqpoint{4.184355in}{2.138065in}}%
\pgfpathlineto{\pgfqpoint{4.184931in}{2.134637in}}%
\pgfpathlineto{\pgfqpoint{4.185124in}{2.138101in}}%
\pgfpathlineto{\pgfqpoint{4.185892in}{2.137007in}}%
\pgfpathlineto{\pgfqpoint{4.186277in}{2.141222in}}%
\pgfpathlineto{\pgfqpoint{4.187238in}{2.136644in}}%
\pgfpathlineto{\pgfqpoint{4.187430in}{2.138762in}}%
\pgfpathlineto{\pgfqpoint{4.187814in}{2.141659in}}%
\pgfpathlineto{\pgfqpoint{4.188583in}{2.139438in}}%
\pgfpathlineto{\pgfqpoint{4.189544in}{2.135880in}}%
\pgfpathlineto{\pgfqpoint{4.189928in}{2.138199in}}%
\pgfpathlineto{\pgfqpoint{4.190889in}{2.140626in}}%
\pgfpathlineto{\pgfqpoint{4.191081in}{2.136800in}}%
\pgfpathlineto{\pgfqpoint{4.191850in}{2.141247in}}%
\pgfpathlineto{\pgfqpoint{4.192234in}{2.140965in}}%
\pgfpathlineto{\pgfqpoint{4.197807in}{2.168860in}}%
\pgfpathlineto{\pgfqpoint{4.198960in}{2.163129in}}%
\pgfpathlineto{\pgfqpoint{4.199344in}{2.163289in}}%
\pgfpathlineto{\pgfqpoint{4.201266in}{2.173392in}}%
\pgfpathlineto{\pgfqpoint{4.201458in}{2.172302in}}%
\pgfpathlineto{\pgfqpoint{4.201842in}{2.172743in}}%
\pgfpathlineto{\pgfqpoint{4.202803in}{2.175455in}}%
\pgfpathlineto{\pgfqpoint{4.202995in}{2.173592in}}%
\pgfpathlineto{\pgfqpoint{4.203187in}{2.174524in}}%
\pgfpathlineto{\pgfqpoint{4.203764in}{2.172116in}}%
\pgfpathlineto{\pgfqpoint{4.203956in}{2.174196in}}%
\pgfpathlineto{\pgfqpoint{4.204340in}{2.171748in}}%
\pgfpathlineto{\pgfqpoint{4.204725in}{2.176693in}}%
\pgfpathlineto{\pgfqpoint{4.205301in}{2.173457in}}%
\pgfpathlineto{\pgfqpoint{4.205493in}{2.177306in}}%
\pgfpathlineto{\pgfqpoint{4.205686in}{2.179899in}}%
\pgfpathlineto{\pgfqpoint{4.206454in}{2.175990in}}%
\pgfpathlineto{\pgfqpoint{4.206839in}{2.180245in}}%
\pgfpathlineto{\pgfqpoint{4.207031in}{2.183742in}}%
\pgfpathlineto{\pgfqpoint{4.207799in}{2.179557in}}%
\pgfpathlineto{\pgfqpoint{4.207992in}{2.179609in}}%
\pgfpathlineto{\pgfqpoint{4.208184in}{2.180995in}}%
\pgfpathlineto{\pgfqpoint{4.208568in}{2.176488in}}%
\pgfpathlineto{\pgfqpoint{4.208952in}{2.174671in}}%
\pgfpathlineto{\pgfqpoint{4.209337in}{2.175828in}}%
\pgfpathlineto{\pgfqpoint{4.210298in}{2.184391in}}%
\pgfpathlineto{\pgfqpoint{4.210682in}{2.183165in}}%
\pgfpathlineto{\pgfqpoint{4.210874in}{2.180382in}}%
\pgfpathlineto{\pgfqpoint{4.211643in}{2.184496in}}%
\pgfpathlineto{\pgfqpoint{4.213372in}{2.177491in}}%
\pgfpathlineto{\pgfqpoint{4.213949in}{2.179134in}}%
\pgfpathlineto{\pgfqpoint{4.215678in}{2.192080in}}%
\pgfpathlineto{\pgfqpoint{4.215871in}{2.191423in}}%
\pgfpathlineto{\pgfqpoint{4.217024in}{2.187033in}}%
\pgfpathlineto{\pgfqpoint{4.217216in}{2.188848in}}%
\pgfpathlineto{\pgfqpoint{4.217408in}{2.187685in}}%
\pgfpathlineto{\pgfqpoint{4.217792in}{2.191343in}}%
\pgfpathlineto{\pgfqpoint{4.218177in}{2.188157in}}%
\pgfpathlineto{\pgfqpoint{4.219714in}{2.197133in}}%
\pgfpathlineto{\pgfqpoint{4.220098in}{2.195897in}}%
\pgfpathlineto{\pgfqpoint{4.220290in}{2.198254in}}%
\pgfpathlineto{\pgfqpoint{4.220483in}{2.196528in}}%
\pgfpathlineto{\pgfqpoint{4.221828in}{2.208895in}}%
\pgfpathlineto{\pgfqpoint{4.222981in}{2.208732in}}%
\pgfpathlineto{\pgfqpoint{4.223173in}{2.207612in}}%
\pgfpathlineto{\pgfqpoint{4.223749in}{2.209946in}}%
\pgfpathlineto{\pgfqpoint{4.223942in}{2.213468in}}%
\pgfpathlineto{\pgfqpoint{4.224710in}{2.209733in}}%
\pgfpathlineto{\pgfqpoint{4.226248in}{2.198440in}}%
\pgfpathlineto{\pgfqpoint{4.225287in}{2.210004in}}%
\pgfpathlineto{\pgfqpoint{4.226440in}{2.202348in}}%
\pgfpathlineto{\pgfqpoint{4.227401in}{2.208466in}}%
\pgfpathlineto{\pgfqpoint{4.227785in}{2.205312in}}%
\pgfpathlineto{\pgfqpoint{4.229322in}{2.201162in}}%
\pgfpathlineto{\pgfqpoint{4.229899in}{2.198734in}}%
\pgfpathlineto{\pgfqpoint{4.231436in}{2.188655in}}%
\pgfpathlineto{\pgfqpoint{4.231628in}{2.188799in}}%
\pgfpathlineto{\pgfqpoint{4.231820in}{2.187991in}}%
\pgfpathlineto{\pgfqpoint{4.232013in}{2.187451in}}%
\pgfpathlineto{\pgfqpoint{4.232205in}{2.191884in}}%
\pgfpathlineto{\pgfqpoint{4.233166in}{2.190135in}}%
\pgfpathlineto{\pgfqpoint{4.233358in}{2.189734in}}%
\pgfpathlineto{\pgfqpoint{4.233550in}{2.190538in}}%
\pgfpathlineto{\pgfqpoint{4.234319in}{2.194333in}}%
\pgfpathlineto{\pgfqpoint{4.234895in}{2.193226in}}%
\pgfpathlineto{\pgfqpoint{4.236625in}{2.202179in}}%
\pgfpathlineto{\pgfqpoint{4.237201in}{2.200769in}}%
\pgfpathlineto{\pgfqpoint{4.237970in}{2.196561in}}%
\pgfpathlineto{\pgfqpoint{4.238162in}{2.199863in}}%
\pgfpathlineto{\pgfqpoint{4.238739in}{2.204749in}}%
\pgfpathlineto{\pgfqpoint{4.239507in}{2.204616in}}%
\pgfpathlineto{\pgfqpoint{4.240084in}{2.204999in}}%
\pgfpathlineto{\pgfqpoint{4.240468in}{2.202430in}}%
\pgfpathlineto{\pgfqpoint{4.242005in}{2.208604in}}%
\pgfpathlineto{\pgfqpoint{4.242966in}{2.198947in}}%
\pgfpathlineto{\pgfqpoint{4.243927in}{2.199273in}}%
\pgfpathlineto{\pgfqpoint{4.244504in}{2.202589in}}%
\pgfpathlineto{\pgfqpoint{4.244888in}{2.201050in}}%
\pgfpathlineto{\pgfqpoint{4.245849in}{2.194052in}}%
\pgfpathlineto{\pgfqpoint{4.246233in}{2.196684in}}%
\pgfpathlineto{\pgfqpoint{4.246617in}{2.198092in}}%
\pgfpathlineto{\pgfqpoint{4.247002in}{2.203567in}}%
\pgfpathlineto{\pgfqpoint{4.247194in}{2.200766in}}%
\pgfpathlineto{\pgfqpoint{4.247578in}{2.195118in}}%
\pgfpathlineto{\pgfqpoint{4.248155in}{2.202050in}}%
\pgfpathlineto{\pgfqpoint{4.248347in}{2.198780in}}%
\pgfpathlineto{\pgfqpoint{4.249500in}{2.193510in}}%
\pgfpathlineto{\pgfqpoint{4.250461in}{2.196256in}}%
\pgfpathlineto{\pgfqpoint{4.250076in}{2.193441in}}%
\pgfpathlineto{\pgfqpoint{4.250653in}{2.194038in}}%
\pgfpathlineto{\pgfqpoint{4.251422in}{2.205404in}}%
\pgfpathlineto{\pgfqpoint{4.251998in}{2.199884in}}%
\pgfpathlineto{\pgfqpoint{4.253151in}{2.186825in}}%
\pgfpathlineto{\pgfqpoint{4.253343in}{2.187801in}}%
\pgfpathlineto{\pgfqpoint{4.254112in}{2.192963in}}%
\pgfpathlineto{\pgfqpoint{4.254688in}{2.191663in}}%
\pgfpathlineto{\pgfqpoint{4.256610in}{2.182793in}}%
\pgfpathlineto{\pgfqpoint{4.257955in}{2.189797in}}%
\pgfpathlineto{\pgfqpoint{4.258532in}{2.188704in}}%
\pgfpathlineto{\pgfqpoint{4.258724in}{2.184708in}}%
\pgfpathlineto{\pgfqpoint{4.259108in}{2.192066in}}%
\pgfpathlineto{\pgfqpoint{4.259493in}{2.191179in}}%
\pgfpathlineto{\pgfqpoint{4.260646in}{2.196851in}}%
\pgfpathlineto{\pgfqpoint{4.260838in}{2.195685in}}%
\pgfpathlineto{\pgfqpoint{4.261414in}{2.194637in}}%
\pgfpathlineto{\pgfqpoint{4.262760in}{2.180581in}}%
\pgfpathlineto{\pgfqpoint{4.263336in}{2.181638in}}%
\pgfpathlineto{\pgfqpoint{4.263528in}{2.181634in}}%
\pgfpathlineto{\pgfqpoint{4.263913in}{2.176743in}}%
\pgfpathlineto{\pgfqpoint{4.264489in}{2.183137in}}%
\pgfpathlineto{\pgfqpoint{4.265066in}{2.181044in}}%
\pgfpathlineto{\pgfqpoint{4.266219in}{2.177846in}}%
\pgfpathlineto{\pgfqpoint{4.266603in}{2.180994in}}%
\pgfpathlineto{\pgfqpoint{4.267372in}{2.179484in}}%
\pgfpathlineto{\pgfqpoint{4.267564in}{2.178989in}}%
\pgfpathlineto{\pgfqpoint{4.267756in}{2.179969in}}%
\pgfpathlineto{\pgfqpoint{4.268332in}{2.179789in}}%
\pgfpathlineto{\pgfqpoint{4.270446in}{2.192621in}}%
\pgfpathlineto{\pgfqpoint{4.270831in}{2.189072in}}%
\pgfpathlineto{\pgfqpoint{4.271984in}{2.184255in}}%
\pgfpathlineto{\pgfqpoint{4.272176in}{2.187311in}}%
\pgfpathlineto{\pgfqpoint{4.272560in}{2.181188in}}%
\pgfpathlineto{\pgfqpoint{4.272752in}{2.182946in}}%
\pgfpathlineto{\pgfqpoint{4.272944in}{2.181245in}}%
\pgfpathlineto{\pgfqpoint{4.273329in}{2.184736in}}%
\pgfpathlineto{\pgfqpoint{4.273521in}{2.186472in}}%
\pgfpathlineto{\pgfqpoint{4.273905in}{2.184507in}}%
\pgfpathlineto{\pgfqpoint{4.274097in}{2.184843in}}%
\pgfpathlineto{\pgfqpoint{4.274866in}{2.174514in}}%
\pgfpathlineto{\pgfqpoint{4.275443in}{2.177157in}}%
\pgfpathlineto{\pgfqpoint{4.276211in}{2.180459in}}%
\pgfpathlineto{\pgfqpoint{4.276788in}{2.179082in}}%
\pgfpathlineto{\pgfqpoint{4.277172in}{2.176259in}}%
\pgfpathlineto{\pgfqpoint{4.277556in}{2.179698in}}%
\pgfpathlineto{\pgfqpoint{4.277749in}{2.178075in}}%
\pgfpathlineto{\pgfqpoint{4.278133in}{2.182887in}}%
\pgfpathlineto{\pgfqpoint{4.278709in}{2.179513in}}%
\pgfpathlineto{\pgfqpoint{4.280439in}{2.172489in}}%
\pgfpathlineto{\pgfqpoint{4.280823in}{2.170903in}}%
\pgfpathlineto{\pgfqpoint{4.281208in}{2.172720in}}%
\pgfpathlineto{\pgfqpoint{4.281400in}{2.175145in}}%
\pgfpathlineto{\pgfqpoint{4.281976in}{2.174882in}}%
\pgfpathlineto{\pgfqpoint{4.283321in}{2.169271in}}%
\pgfpathlineto{\pgfqpoint{4.284667in}{2.166177in}}%
\pgfpathlineto{\pgfqpoint{4.286012in}{2.174244in}}%
\pgfpathlineto{\pgfqpoint{4.286588in}{2.172378in}}%
\pgfpathlineto{\pgfqpoint{4.286781in}{2.174925in}}%
\pgfpathlineto{\pgfqpoint{4.287357in}{2.176167in}}%
\pgfpathlineto{\pgfqpoint{4.287549in}{2.172634in}}%
\pgfpathlineto{\pgfqpoint{4.288126in}{2.178628in}}%
\pgfpathlineto{\pgfqpoint{4.288318in}{2.177084in}}%
\pgfpathlineto{\pgfqpoint{4.288510in}{2.179099in}}%
\pgfpathlineto{\pgfqpoint{4.289087in}{2.175823in}}%
\pgfpathlineto{\pgfqpoint{4.289279in}{2.175846in}}%
\pgfpathlineto{\pgfqpoint{4.289471in}{2.174582in}}%
\pgfpathlineto{\pgfqpoint{4.289855in}{2.176039in}}%
\pgfpathlineto{\pgfqpoint{4.291393in}{2.184323in}}%
\pgfpathlineto{\pgfqpoint{4.291585in}{2.183663in}}%
\pgfpathlineto{\pgfqpoint{4.291777in}{2.184209in}}%
\pgfpathlineto{\pgfqpoint{4.292930in}{2.195810in}}%
\pgfpathlineto{\pgfqpoint{4.293314in}{2.190061in}}%
\pgfpathlineto{\pgfqpoint{4.293699in}{2.188727in}}%
\pgfpathlineto{\pgfqpoint{4.294852in}{2.195395in}}%
\pgfpathlineto{\pgfqpoint{4.295620in}{2.192438in}}%
\pgfpathlineto{\pgfqpoint{4.295428in}{2.196161in}}%
\pgfpathlineto{\pgfqpoint{4.295812in}{2.195444in}}%
\pgfpathlineto{\pgfqpoint{4.296197in}{2.193414in}}%
\pgfpathlineto{\pgfqpoint{4.296389in}{2.196853in}}%
\pgfpathlineto{\pgfqpoint{4.296773in}{2.193259in}}%
\pgfpathlineto{\pgfqpoint{4.297158in}{2.196289in}}%
\pgfpathlineto{\pgfqpoint{4.299079in}{2.210035in}}%
\pgfpathlineto{\pgfqpoint{4.299271in}{2.207056in}}%
\pgfpathlineto{\pgfqpoint{4.299848in}{2.212779in}}%
\pgfpathlineto{\pgfqpoint{4.300232in}{2.208249in}}%
\pgfpathlineto{\pgfqpoint{4.300617in}{2.207423in}}%
\pgfpathlineto{\pgfqpoint{4.302730in}{2.219395in}}%
\pgfpathlineto{\pgfqpoint{4.302923in}{2.215615in}}%
\pgfpathlineto{\pgfqpoint{4.303883in}{2.216350in}}%
\pgfpathlineto{\pgfqpoint{4.305421in}{2.223398in}}%
\pgfpathlineto{\pgfqpoint{4.305613in}{2.219353in}}%
\pgfpathlineto{\pgfqpoint{4.305997in}{2.217807in}}%
\pgfpathlineto{\pgfqpoint{4.306574in}{2.218839in}}%
\pgfpathlineto{\pgfqpoint{4.306766in}{2.222341in}}%
\pgfpathlineto{\pgfqpoint{4.307342in}{2.219044in}}%
\pgfpathlineto{\pgfqpoint{4.307919in}{2.213598in}}%
\pgfpathlineto{\pgfqpoint{4.308303in}{2.218608in}}%
\pgfpathlineto{\pgfqpoint{4.310417in}{2.225398in}}%
\pgfpathlineto{\pgfqpoint{4.310609in}{2.223255in}}%
\pgfpathlineto{\pgfqpoint{4.311378in}{2.225963in}}%
\pgfpathlineto{\pgfqpoint{4.311762in}{2.227797in}}%
\pgfpathlineto{\pgfqpoint{4.311955in}{2.222831in}}%
\pgfpathlineto{\pgfqpoint{4.313300in}{2.218188in}}%
\pgfpathlineto{\pgfqpoint{4.314068in}{2.220934in}}%
\pgfpathlineto{\pgfqpoint{4.314645in}{2.218648in}}%
\pgfpathlineto{\pgfqpoint{4.315414in}{2.215536in}}%
\pgfpathlineto{\pgfqpoint{4.315029in}{2.219312in}}%
\pgfpathlineto{\pgfqpoint{4.315798in}{2.216855in}}%
\pgfpathlineto{\pgfqpoint{4.316374in}{2.222113in}}%
\pgfpathlineto{\pgfqpoint{4.316759in}{2.218674in}}%
\pgfpathlineto{\pgfqpoint{4.316951in}{2.217041in}}%
\pgfpathlineto{\pgfqpoint{4.317143in}{2.220639in}}%
\pgfpathlineto{\pgfqpoint{4.317720in}{2.217850in}}%
\pgfpathlineto{\pgfqpoint{4.318104in}{2.221340in}}%
\pgfpathlineto{\pgfqpoint{4.318873in}{2.220981in}}%
\pgfpathlineto{\pgfqpoint{4.319449in}{2.209463in}}%
\pgfpathlineto{\pgfqpoint{4.320410in}{2.211720in}}%
\pgfpathlineto{\pgfqpoint{4.321563in}{2.224534in}}%
\pgfpathlineto{\pgfqpoint{4.321947in}{2.222688in}}%
\pgfpathlineto{\pgfqpoint{4.322524in}{2.214137in}}%
\pgfpathlineto{\pgfqpoint{4.323292in}{2.217628in}}%
\pgfpathlineto{\pgfqpoint{4.324253in}{2.224268in}}%
\pgfpathlineto{\pgfqpoint{4.325022in}{2.222976in}}%
\pgfpathlineto{\pgfqpoint{4.325406in}{2.219375in}}%
\pgfpathlineto{\pgfqpoint{4.326175in}{2.221364in}}%
\pgfpathlineto{\pgfqpoint{4.327328in}{2.226860in}}%
\pgfpathlineto{\pgfqpoint{4.327712in}{2.226373in}}%
\pgfpathlineto{\pgfqpoint{4.327904in}{2.227691in}}%
\pgfpathlineto{\pgfqpoint{4.328289in}{2.222869in}}%
\pgfpathlineto{\pgfqpoint{4.328481in}{2.224843in}}%
\pgfpathlineto{\pgfqpoint{4.328865in}{2.226924in}}%
\pgfpathlineto{\pgfqpoint{4.329057in}{2.224512in}}%
\pgfpathlineto{\pgfqpoint{4.330403in}{2.218493in}}%
\pgfpathlineto{\pgfqpoint{4.330595in}{2.220732in}}%
\pgfpathlineto{\pgfqpoint{4.332132in}{2.230158in}}%
\pgfpathlineto{\pgfqpoint{4.332516in}{2.229203in}}%
\pgfpathlineto{\pgfqpoint{4.332709in}{2.228885in}}%
\pgfpathlineto{\pgfqpoint{4.332901in}{2.230134in}}%
\pgfpathlineto{\pgfqpoint{4.333862in}{2.234187in}}%
\pgfpathlineto{\pgfqpoint{4.334246in}{2.232694in}}%
\pgfpathlineto{\pgfqpoint{4.334438in}{2.230923in}}%
\pgfpathlineto{\pgfqpoint{4.334630in}{2.235894in}}%
\pgfpathlineto{\pgfqpoint{4.335207in}{2.232684in}}%
\pgfpathlineto{\pgfqpoint{4.335783in}{2.231170in}}%
\pgfpathlineto{\pgfqpoint{4.335976in}{2.234552in}}%
\pgfpathlineto{\pgfqpoint{4.336168in}{2.230352in}}%
\pgfpathlineto{\pgfqpoint{4.336936in}{2.233777in}}%
\pgfpathlineto{\pgfqpoint{4.337321in}{2.233058in}}%
\pgfpathlineto{\pgfqpoint{4.338474in}{2.238657in}}%
\pgfpathlineto{\pgfqpoint{4.338666in}{2.238979in}}%
\pgfpathlineto{\pgfqpoint{4.338858in}{2.237276in}}%
\pgfpathlineto{\pgfqpoint{4.339435in}{2.234840in}}%
\pgfpathlineto{\pgfqpoint{4.339627in}{2.237394in}}%
\pgfpathlineto{\pgfqpoint{4.340011in}{2.236767in}}%
\pgfpathlineto{\pgfqpoint{4.341164in}{2.242193in}}%
\pgfpathlineto{\pgfqpoint{4.341356in}{2.240971in}}%
\pgfpathlineto{\pgfqpoint{4.342701in}{2.247171in}}%
\pgfpathlineto{\pgfqpoint{4.342894in}{2.243458in}}%
\pgfpathlineto{\pgfqpoint{4.343470in}{2.240334in}}%
\pgfpathlineto{\pgfqpoint{4.344815in}{2.232043in}}%
\pgfpathlineto{\pgfqpoint{4.346737in}{2.246065in}}%
\pgfpathlineto{\pgfqpoint{4.347121in}{2.247078in}}%
\pgfpathlineto{\pgfqpoint{4.348851in}{2.254412in}}%
\pgfpathlineto{\pgfqpoint{4.349427in}{2.258630in}}%
\pgfpathlineto{\pgfqpoint{4.349812in}{2.252736in}}%
\pgfpathlineto{\pgfqpoint{4.350004in}{2.252332in}}%
\pgfpathlineto{\pgfqpoint{4.350772in}{2.256286in}}%
\pgfpathlineto{\pgfqpoint{4.351157in}{2.255703in}}%
\pgfpathlineto{\pgfqpoint{4.352310in}{2.251955in}}%
\pgfpathlineto{\pgfqpoint{4.351541in}{2.256161in}}%
\pgfpathlineto{\pgfqpoint{4.352502in}{2.252636in}}%
\pgfpathlineto{\pgfqpoint{4.352694in}{2.252467in}}%
\pgfpathlineto{\pgfqpoint{4.354039in}{2.246840in}}%
\pgfpathlineto{\pgfqpoint{4.354231in}{2.247876in}}%
\pgfpathlineto{\pgfqpoint{4.354424in}{2.245593in}}%
\pgfpathlineto{\pgfqpoint{4.354808in}{2.241379in}}%
\pgfpathlineto{\pgfqpoint{4.355384in}{2.243838in}}%
\pgfpathlineto{\pgfqpoint{4.355769in}{2.248230in}}%
\pgfpathlineto{\pgfqpoint{4.356345in}{2.242437in}}%
\pgfpathlineto{\pgfqpoint{4.356730in}{2.241726in}}%
\pgfpathlineto{\pgfqpoint{4.357306in}{2.243099in}}%
\pgfpathlineto{\pgfqpoint{4.357883in}{2.242708in}}%
\pgfpathlineto{\pgfqpoint{4.360189in}{2.247996in}}%
\pgfpathlineto{\pgfqpoint{4.360765in}{2.249960in}}%
\pgfpathlineto{\pgfqpoint{4.361918in}{2.240489in}}%
\pgfpathlineto{\pgfqpoint{4.362879in}{2.244578in}}%
\pgfpathlineto{\pgfqpoint{4.363071in}{2.242343in}}%
\pgfpathlineto{\pgfqpoint{4.364416in}{2.251910in}}%
\pgfpathlineto{\pgfqpoint{4.365185in}{2.249540in}}%
\pgfpathlineto{\pgfqpoint{4.365762in}{2.250708in}}%
\pgfpathlineto{\pgfqpoint{4.365954in}{2.251064in}}%
\pgfpathlineto{\pgfqpoint{4.368260in}{2.238270in}}%
\pgfpathlineto{\pgfqpoint{4.369797in}{2.243535in}}%
\pgfpathlineto{\pgfqpoint{4.369989in}{2.243106in}}%
\pgfpathlineto{\pgfqpoint{4.370950in}{2.239576in}}%
\pgfpathlineto{\pgfqpoint{4.371527in}{2.240228in}}%
\pgfpathlineto{\pgfqpoint{4.372103in}{2.243837in}}%
\pgfpathlineto{\pgfqpoint{4.372487in}{2.242073in}}%
\pgfpathlineto{\pgfqpoint{4.372680in}{2.240760in}}%
\pgfpathlineto{\pgfqpoint{4.373064in}{2.243770in}}%
\pgfpathlineto{\pgfqpoint{4.373256in}{2.247105in}}%
\pgfpathlineto{\pgfqpoint{4.374217in}{2.246962in}}%
\pgfpathlineto{\pgfqpoint{4.374409in}{2.247724in}}%
\pgfpathlineto{\pgfqpoint{4.374601in}{2.245664in}}%
\pgfpathlineto{\pgfqpoint{4.374793in}{2.244390in}}%
\pgfpathlineto{\pgfqpoint{4.375178in}{2.246555in}}%
\pgfpathlineto{\pgfqpoint{4.375370in}{2.249377in}}%
\pgfpathlineto{\pgfqpoint{4.375754in}{2.246383in}}%
\pgfpathlineto{\pgfqpoint{4.376139in}{2.246548in}}%
\pgfpathlineto{\pgfqpoint{4.376523in}{2.247279in}}%
\pgfpathlineto{\pgfqpoint{4.376907in}{2.250366in}}%
\pgfpathlineto{\pgfqpoint{4.377292in}{2.242895in}}%
\pgfpathlineto{\pgfqpoint{4.377676in}{2.243987in}}%
\pgfpathlineto{\pgfqpoint{4.377868in}{2.242867in}}%
\pgfpathlineto{\pgfqpoint{4.378060in}{2.241593in}}%
\pgfpathlineto{\pgfqpoint{4.378445in}{2.244364in}}%
\pgfpathlineto{\pgfqpoint{4.379405in}{2.248232in}}%
\pgfpathlineto{\pgfqpoint{4.379598in}{2.247255in}}%
\pgfpathlineto{\pgfqpoint{4.379982in}{2.245404in}}%
\pgfpathlineto{\pgfqpoint{4.380366in}{2.248849in}}%
\pgfpathlineto{\pgfqpoint{4.380558in}{2.246657in}}%
\pgfpathlineto{\pgfqpoint{4.380943in}{2.249936in}}%
\pgfpathlineto{\pgfqpoint{4.381327in}{2.245854in}}%
\pgfpathlineto{\pgfqpoint{4.383249in}{2.234245in}}%
\pgfpathlineto{\pgfqpoint{4.383633in}{2.237797in}}%
\pgfpathlineto{\pgfqpoint{4.384594in}{2.237550in}}%
\pgfpathlineto{\pgfqpoint{4.384786in}{2.238271in}}%
\pgfpathlineto{\pgfqpoint{4.384978in}{2.237666in}}%
\pgfpathlineto{\pgfqpoint{4.385171in}{2.234063in}}%
\pgfpathlineto{\pgfqpoint{4.385939in}{2.238352in}}%
\pgfpathlineto{\pgfqpoint{4.390551in}{2.276006in}}%
\pgfpathlineto{\pgfqpoint{4.391128in}{2.272457in}}%
\pgfpathlineto{\pgfqpoint{4.392281in}{2.261058in}}%
\pgfpathlineto{\pgfqpoint{4.392473in}{2.263668in}}%
\pgfpathlineto{\pgfqpoint{4.394779in}{2.284622in}}%
\pgfpathlineto{\pgfqpoint{4.395355in}{2.284364in}}%
\pgfpathlineto{\pgfqpoint{4.396701in}{2.291382in}}%
\pgfpathlineto{\pgfqpoint{4.396893in}{2.285521in}}%
\pgfpathlineto{\pgfqpoint{4.397854in}{2.287968in}}%
\pgfpathlineto{\pgfqpoint{4.399391in}{2.284434in}}%
\pgfpathlineto{\pgfqpoint{4.398238in}{2.288860in}}%
\pgfpathlineto{\pgfqpoint{4.399583in}{2.284931in}}%
\pgfpathlineto{\pgfqpoint{4.400352in}{2.287121in}}%
\pgfpathlineto{\pgfqpoint{4.399967in}{2.283240in}}%
\pgfpathlineto{\pgfqpoint{4.400928in}{2.285781in}}%
\pgfpathlineto{\pgfqpoint{4.401697in}{2.280127in}}%
\pgfpathlineto{\pgfqpoint{4.402081in}{2.280694in}}%
\pgfpathlineto{\pgfqpoint{4.402466in}{2.280282in}}%
\pgfpathlineto{\pgfqpoint{4.403619in}{2.287725in}}%
\pgfpathlineto{\pgfqpoint{4.404772in}{2.283182in}}%
\pgfpathlineto{\pgfqpoint{4.404964in}{2.284094in}}%
\pgfpathlineto{\pgfqpoint{4.406501in}{2.291339in}}%
\pgfpathlineto{\pgfqpoint{4.407270in}{2.287039in}}%
\pgfpathlineto{\pgfqpoint{4.407462in}{2.288710in}}%
\pgfpathlineto{\pgfqpoint{4.410345in}{2.316413in}}%
\pgfpathlineto{\pgfqpoint{4.410729in}{2.315701in}}%
\pgfpathlineto{\pgfqpoint{4.410921in}{2.317150in}}%
\pgfpathlineto{\pgfqpoint{4.412266in}{2.327647in}}%
\pgfpathlineto{\pgfqpoint{4.412458in}{2.327464in}}%
\pgfpathlineto{\pgfqpoint{4.412651in}{2.324452in}}%
\pgfpathlineto{\pgfqpoint{4.413419in}{2.326505in}}%
\pgfpathlineto{\pgfqpoint{4.414188in}{2.329693in}}%
\pgfpathlineto{\pgfqpoint{4.414957in}{2.320021in}}%
\pgfpathlineto{\pgfqpoint{4.415533in}{2.321103in}}%
\pgfpathlineto{\pgfqpoint{4.418416in}{2.305732in}}%
\pgfpathlineto{\pgfqpoint{4.418800in}{2.311163in}}%
\pgfpathlineto{\pgfqpoint{4.419376in}{2.315720in}}%
\pgfpathlineto{\pgfqpoint{4.419953in}{2.313181in}}%
\pgfpathlineto{\pgfqpoint{4.420337in}{2.311339in}}%
\pgfpathlineto{\pgfqpoint{4.420529in}{2.313569in}}%
\pgfpathlineto{\pgfqpoint{4.420722in}{2.315980in}}%
\pgfpathlineto{\pgfqpoint{4.421106in}{2.310443in}}%
\pgfpathlineto{\pgfqpoint{4.421490in}{2.306936in}}%
\pgfpathlineto{\pgfqpoint{4.422259in}{2.308439in}}%
\pgfpathlineto{\pgfqpoint{4.422643in}{2.309810in}}%
\pgfpathlineto{\pgfqpoint{4.422835in}{2.308781in}}%
\pgfpathlineto{\pgfqpoint{4.424181in}{2.305593in}}%
\pgfpathlineto{\pgfqpoint{4.424373in}{2.304575in}}%
\pgfpathlineto{\pgfqpoint{4.424757in}{2.307763in}}%
\pgfpathlineto{\pgfqpoint{4.424949in}{2.307400in}}%
\pgfpathlineto{\pgfqpoint{4.425141in}{2.307766in}}%
\pgfpathlineto{\pgfqpoint{4.426487in}{2.295223in}}%
\pgfpathlineto{\pgfqpoint{4.426679in}{2.297441in}}%
\pgfpathlineto{\pgfqpoint{4.426871in}{2.295199in}}%
\pgfpathlineto{\pgfqpoint{4.427255in}{2.297591in}}%
\pgfpathlineto{\pgfqpoint{4.427640in}{2.295441in}}%
\pgfpathlineto{\pgfqpoint{4.428408in}{2.295120in}}%
\pgfpathlineto{\pgfqpoint{4.429177in}{2.302209in}}%
\pgfpathlineto{\pgfqpoint{4.429561in}{2.299288in}}%
\pgfpathlineto{\pgfqpoint{4.430138in}{2.301378in}}%
\pgfpathlineto{\pgfqpoint{4.430330in}{2.302999in}}%
\pgfpathlineto{\pgfqpoint{4.430714in}{2.297374in}}%
\pgfpathlineto{\pgfqpoint{4.430906in}{2.300830in}}%
\pgfpathlineto{\pgfqpoint{4.431099in}{2.299144in}}%
\pgfpathlineto{\pgfqpoint{4.431675in}{2.302584in}}%
\pgfpathlineto{\pgfqpoint{4.431867in}{2.302633in}}%
\pgfpathlineto{\pgfqpoint{4.432060in}{2.301552in}}%
\pgfpathlineto{\pgfqpoint{4.433405in}{2.308975in}}%
\pgfpathlineto{\pgfqpoint{4.435326in}{2.304702in}}%
\pgfpathlineto{\pgfqpoint{4.435711in}{2.305128in}}%
\pgfpathlineto{\pgfqpoint{4.436287in}{2.304880in}}%
\pgfpathlineto{\pgfqpoint{4.436479in}{2.303777in}}%
\pgfpathlineto{\pgfqpoint{4.437056in}{2.306409in}}%
\pgfpathlineto{\pgfqpoint{4.437825in}{2.302239in}}%
\pgfpathlineto{\pgfqpoint{4.439938in}{2.286052in}}%
\pgfpathlineto{\pgfqpoint{4.440323in}{2.289470in}}%
\pgfpathlineto{\pgfqpoint{4.440515in}{2.289127in}}%
\pgfpathlineto{\pgfqpoint{4.441091in}{2.289004in}}%
\pgfpathlineto{\pgfqpoint{4.441668in}{2.292326in}}%
\pgfpathlineto{\pgfqpoint{4.442821in}{2.284909in}}%
\pgfpathlineto{\pgfqpoint{4.443397in}{2.288242in}}%
\pgfpathlineto{\pgfqpoint{4.444935in}{2.296100in}}%
\pgfpathlineto{\pgfqpoint{4.445127in}{2.295715in}}%
\pgfpathlineto{\pgfqpoint{4.448202in}{2.315832in}}%
\pgfpathlineto{\pgfqpoint{4.448778in}{2.310594in}}%
\pgfpathlineto{\pgfqpoint{4.449355in}{2.308920in}}%
\pgfpathlineto{\pgfqpoint{4.449547in}{2.311190in}}%
\pgfpathlineto{\pgfqpoint{4.451661in}{2.318471in}}%
\pgfpathlineto{\pgfqpoint{4.452045in}{2.318281in}}%
\pgfpathlineto{\pgfqpoint{4.453967in}{2.304563in}}%
\pgfpathlineto{\pgfqpoint{4.454159in}{2.306983in}}%
\pgfpathlineto{\pgfqpoint{4.454351in}{2.305912in}}%
\pgfpathlineto{\pgfqpoint{4.454735in}{2.308193in}}%
\pgfpathlineto{\pgfqpoint{4.456273in}{2.313946in}}%
\pgfpathlineto{\pgfqpoint{4.456465in}{2.313150in}}%
\pgfpathlineto{\pgfqpoint{4.456657in}{2.310234in}}%
\pgfpathlineto{\pgfqpoint{4.457426in}{2.314825in}}%
\pgfpathlineto{\pgfqpoint{4.457618in}{2.315000in}}%
\pgfpathlineto{\pgfqpoint{4.457810in}{2.318284in}}%
\pgfpathlineto{\pgfqpoint{4.458579in}{2.314813in}}%
\pgfpathlineto{\pgfqpoint{4.458771in}{2.314684in}}%
\pgfpathlineto{\pgfqpoint{4.459347in}{2.310174in}}%
\pgfpathlineto{\pgfqpoint{4.460116in}{2.311597in}}%
\pgfpathlineto{\pgfqpoint{4.461269in}{2.313212in}}%
\pgfpathlineto{\pgfqpoint{4.461653in}{2.319138in}}%
\pgfpathlineto{\pgfqpoint{4.462230in}{2.314637in}}%
\pgfpathlineto{\pgfqpoint{4.462806in}{2.311453in}}%
\pgfpathlineto{\pgfqpoint{4.463191in}{2.313551in}}%
\pgfpathlineto{\pgfqpoint{4.464536in}{2.322080in}}%
\pgfpathlineto{\pgfqpoint{4.465689in}{2.317011in}}%
\pgfpathlineto{\pgfqpoint{4.465881in}{2.317295in}}%
\pgfpathlineto{\pgfqpoint{4.467418in}{2.325358in}}%
\pgfpathlineto{\pgfqpoint{4.470685in}{2.313343in}}%
\pgfpathlineto{\pgfqpoint{4.471070in}{2.315147in}}%
\pgfpathlineto{\pgfqpoint{4.471454in}{2.310768in}}%
\pgfpathlineto{\pgfqpoint{4.472030in}{2.311967in}}%
\pgfpathlineto{\pgfqpoint{4.471838in}{2.309700in}}%
\pgfpathlineto{\pgfqpoint{4.472415in}{2.311607in}}%
\pgfpathlineto{\pgfqpoint{4.473568in}{2.305304in}}%
\pgfpathlineto{\pgfqpoint{4.473760in}{2.306054in}}%
\pgfpathlineto{\pgfqpoint{4.473952in}{2.309425in}}%
\pgfpathlineto{\pgfqpoint{4.474529in}{2.301815in}}%
\pgfpathlineto{\pgfqpoint{4.475874in}{2.297957in}}%
\pgfpathlineto{\pgfqpoint{4.477027in}{2.292682in}}%
\pgfpathlineto{\pgfqpoint{4.477603in}{2.294154in}}%
\pgfpathlineto{\pgfqpoint{4.477795in}{2.294817in}}%
\pgfpathlineto{\pgfqpoint{4.477988in}{2.293940in}}%
\pgfpathlineto{\pgfqpoint{4.478180in}{2.294408in}}%
\pgfpathlineto{\pgfqpoint{4.479525in}{2.285938in}}%
\pgfpathlineto{\pgfqpoint{4.480102in}{2.288809in}}%
\pgfpathlineto{\pgfqpoint{4.480678in}{2.287649in}}%
\pgfpathlineto{\pgfqpoint{4.480870in}{2.284255in}}%
\pgfpathlineto{\pgfqpoint{4.481447in}{2.289205in}}%
\pgfpathlineto{\pgfqpoint{4.481639in}{2.288231in}}%
\pgfpathlineto{\pgfqpoint{4.482215in}{2.292703in}}%
\pgfpathlineto{\pgfqpoint{4.482600in}{2.289757in}}%
\pgfpathlineto{\pgfqpoint{4.482984in}{2.284643in}}%
\pgfpathlineto{\pgfqpoint{4.483753in}{2.285920in}}%
\pgfpathlineto{\pgfqpoint{4.483945in}{2.286046in}}%
\pgfpathlineto{\pgfqpoint{4.484137in}{2.285602in}}%
\pgfpathlineto{\pgfqpoint{4.485098in}{2.289012in}}%
\pgfpathlineto{\pgfqpoint{4.485290in}{2.287491in}}%
\pgfpathlineto{\pgfqpoint{4.485482in}{2.288097in}}%
\pgfpathlineto{\pgfqpoint{4.485674in}{2.286078in}}%
\pgfpathlineto{\pgfqpoint{4.485867in}{2.286491in}}%
\pgfpathlineto{\pgfqpoint{4.486251in}{2.283827in}}%
\pgfpathlineto{\pgfqpoint{4.486443in}{2.286901in}}%
\pgfpathlineto{\pgfqpoint{4.487020in}{2.284357in}}%
\pgfpathlineto{\pgfqpoint{4.488365in}{2.292667in}}%
\pgfpathlineto{\pgfqpoint{4.488557in}{2.289837in}}%
\pgfpathlineto{\pgfqpoint{4.488749in}{2.289872in}}%
\pgfpathlineto{\pgfqpoint{4.489518in}{2.291020in}}%
\pgfpathlineto{\pgfqpoint{4.490094in}{2.286907in}}%
\pgfpathlineto{\pgfqpoint{4.490479in}{2.291442in}}%
\pgfpathlineto{\pgfqpoint{4.491055in}{2.288310in}}%
\pgfpathlineto{\pgfqpoint{4.491824in}{2.286690in}}%
\pgfpathlineto{\pgfqpoint{4.492016in}{2.287479in}}%
\pgfpathlineto{\pgfqpoint{4.493361in}{2.290388in}}%
\pgfpathlineto{\pgfqpoint{4.494706in}{2.283656in}}%
\pgfpathlineto{\pgfqpoint{4.494898in}{2.283831in}}%
\pgfpathlineto{\pgfqpoint{4.495475in}{2.283466in}}%
\pgfpathlineto{\pgfqpoint{4.496051in}{2.287031in}}%
\pgfpathlineto{\pgfqpoint{4.496628in}{2.288122in}}%
\pgfpathlineto{\pgfqpoint{4.497397in}{2.281193in}}%
\pgfpathlineto{\pgfqpoint{4.497781in}{2.283205in}}%
\pgfpathlineto{\pgfqpoint{4.497973in}{2.287928in}}%
\pgfpathlineto{\pgfqpoint{4.498742in}{2.281965in}}%
\pgfpathlineto{\pgfqpoint{4.502201in}{2.295042in}}%
\pgfpathlineto{\pgfqpoint{4.502969in}{2.292474in}}%
\pgfpathlineto{\pgfqpoint{4.503354in}{2.286695in}}%
\pgfpathlineto{\pgfqpoint{4.503930in}{2.292393in}}%
\pgfpathlineto{\pgfqpoint{4.504122in}{2.292282in}}%
\pgfpathlineto{\pgfqpoint{4.506621in}{2.307760in}}%
\pgfpathlineto{\pgfqpoint{4.506813in}{2.306030in}}%
\pgfpathlineto{\pgfqpoint{4.507005in}{2.304058in}}%
\pgfpathlineto{\pgfqpoint{4.507582in}{2.309316in}}%
\pgfpathlineto{\pgfqpoint{4.507966in}{2.305833in}}%
\pgfpathlineto{\pgfqpoint{4.508350in}{2.308909in}}%
\pgfpathlineto{\pgfqpoint{4.508735in}{2.314448in}}%
\pgfpathlineto{\pgfqpoint{4.509311in}{2.307967in}}%
\pgfpathlineto{\pgfqpoint{4.510272in}{2.304751in}}%
\pgfpathlineto{\pgfqpoint{4.510464in}{2.305318in}}%
\pgfpathlineto{\pgfqpoint{4.510656in}{2.307981in}}%
\pgfpathlineto{\pgfqpoint{4.511425in}{2.304750in}}%
\pgfpathlineto{\pgfqpoint{4.511617in}{2.306135in}}%
\pgfpathlineto{\pgfqpoint{4.514307in}{2.297767in}}%
\pgfpathlineto{\pgfqpoint{4.512194in}{2.306840in}}%
\pgfpathlineto{\pgfqpoint{4.514884in}{2.298832in}}%
\pgfpathlineto{\pgfqpoint{4.518343in}{2.320751in}}%
\pgfpathlineto{\pgfqpoint{4.518535in}{2.320066in}}%
\pgfpathlineto{\pgfqpoint{4.518919in}{2.317451in}}%
\pgfpathlineto{\pgfqpoint{4.519496in}{2.318860in}}%
\pgfpathlineto{\pgfqpoint{4.519688in}{2.320739in}}%
\pgfpathlineto{\pgfqpoint{4.520072in}{2.317545in}}%
\pgfpathlineto{\pgfqpoint{4.520265in}{2.318732in}}%
\pgfpathlineto{\pgfqpoint{4.520841in}{2.312822in}}%
\pgfpathlineto{\pgfqpoint{4.521418in}{2.316267in}}%
\pgfpathlineto{\pgfqpoint{4.521802in}{2.315485in}}%
\pgfpathlineto{\pgfqpoint{4.523147in}{2.324003in}}%
\pgfpathlineto{\pgfqpoint{4.523531in}{2.322115in}}%
\pgfpathlineto{\pgfqpoint{4.523724in}{2.323671in}}%
\pgfpathlineto{\pgfqpoint{4.524108in}{2.328278in}}%
\pgfpathlineto{\pgfqpoint{4.524877in}{2.324450in}}%
\pgfpathlineto{\pgfqpoint{4.526030in}{2.316819in}}%
\pgfpathlineto{\pgfqpoint{4.526222in}{2.317594in}}%
\pgfpathlineto{\pgfqpoint{4.526414in}{2.321423in}}%
\pgfpathlineto{\pgfqpoint{4.527183in}{2.316150in}}%
\pgfpathlineto{\pgfqpoint{4.527375in}{2.316073in}}%
\pgfpathlineto{\pgfqpoint{4.527759in}{2.319986in}}%
\pgfpathlineto{\pgfqpoint{4.528528in}{2.316968in}}%
\pgfpathlineto{\pgfqpoint{4.528720in}{2.316794in}}%
\pgfpathlineto{\pgfqpoint{4.529489in}{2.322489in}}%
\pgfpathlineto{\pgfqpoint{4.529873in}{2.319174in}}%
\pgfpathlineto{\pgfqpoint{4.530065in}{2.318227in}}%
\pgfpathlineto{\pgfqpoint{4.530257in}{2.320731in}}%
\pgfpathlineto{\pgfqpoint{4.530642in}{2.318848in}}%
\pgfpathlineto{\pgfqpoint{4.531026in}{2.323342in}}%
\pgfpathlineto{\pgfqpoint{4.531410in}{2.317560in}}%
\pgfpathlineto{\pgfqpoint{4.532179in}{2.312726in}}%
\pgfpathlineto{\pgfqpoint{4.532371in}{2.315529in}}%
\pgfpathlineto{\pgfqpoint{4.533140in}{2.321370in}}%
\pgfpathlineto{\pgfqpoint{4.533716in}{2.320536in}}%
\pgfpathlineto{\pgfqpoint{4.534485in}{2.317703in}}%
\pgfpathlineto{\pgfqpoint{4.535062in}{2.315291in}}%
\pgfpathlineto{\pgfqpoint{4.535254in}{2.319395in}}%
\pgfpathlineto{\pgfqpoint{4.535830in}{2.314414in}}%
\pgfpathlineto{\pgfqpoint{4.536022in}{2.315392in}}%
\pgfpathlineto{\pgfqpoint{4.536215in}{2.315286in}}%
\pgfpathlineto{\pgfqpoint{4.536407in}{2.318931in}}%
\pgfpathlineto{\pgfqpoint{4.536791in}{2.311002in}}%
\pgfpathlineto{\pgfqpoint{4.537175in}{2.313768in}}%
\pgfpathlineto{\pgfqpoint{4.538136in}{2.307679in}}%
\pgfpathlineto{\pgfqpoint{4.537560in}{2.315459in}}%
\pgfpathlineto{\pgfqpoint{4.538521in}{2.310223in}}%
\pgfpathlineto{\pgfqpoint{4.538905in}{2.315765in}}%
\pgfpathlineto{\pgfqpoint{4.539674in}{2.314674in}}%
\pgfpathlineto{\pgfqpoint{4.539866in}{2.311919in}}%
\pgfpathlineto{\pgfqpoint{4.540634in}{2.316107in}}%
\pgfpathlineto{\pgfqpoint{4.541787in}{2.311742in}}%
\pgfpathlineto{\pgfqpoint{4.542172in}{2.313911in}}%
\pgfpathlineto{\pgfqpoint{4.544862in}{2.323337in}}%
\pgfpathlineto{\pgfqpoint{4.545246in}{2.322308in}}%
\pgfpathlineto{\pgfqpoint{4.546976in}{2.313381in}}%
\pgfpathlineto{\pgfqpoint{4.546015in}{2.323463in}}%
\pgfpathlineto{\pgfqpoint{4.547552in}{2.315651in}}%
\pgfpathlineto{\pgfqpoint{4.550051in}{2.329429in}}%
\pgfpathlineto{\pgfqpoint{4.550243in}{2.329074in}}%
\pgfpathlineto{\pgfqpoint{4.550435in}{2.325511in}}%
\pgfpathlineto{\pgfqpoint{4.551011in}{2.329575in}}%
\pgfpathlineto{\pgfqpoint{4.551204in}{2.328757in}}%
\pgfpathlineto{\pgfqpoint{4.551972in}{2.333540in}}%
\pgfpathlineto{\pgfqpoint{4.552549in}{2.333329in}}%
\pgfpathlineto{\pgfqpoint{4.553318in}{2.345052in}}%
\pgfpathlineto{\pgfqpoint{4.553510in}{2.340264in}}%
\pgfpathlineto{\pgfqpoint{4.554086in}{2.345145in}}%
\pgfpathlineto{\pgfqpoint{4.554471in}{2.341637in}}%
\pgfpathlineto{\pgfqpoint{4.556200in}{2.347437in}}%
\pgfpathlineto{\pgfqpoint{4.556392in}{2.347419in}}%
\pgfpathlineto{\pgfqpoint{4.557161in}{2.347582in}}%
\pgfpathlineto{\pgfqpoint{4.557737in}{2.344312in}}%
\pgfpathlineto{\pgfqpoint{4.558314in}{2.344172in}}%
\pgfpathlineto{\pgfqpoint{4.559083in}{2.339551in}}%
\pgfpathlineto{\pgfqpoint{4.559467in}{2.342284in}}%
\pgfpathlineto{\pgfqpoint{4.559659in}{2.342303in}}%
\pgfpathlineto{\pgfqpoint{4.560620in}{2.345634in}}%
\pgfpathlineto{\pgfqpoint{4.560043in}{2.341384in}}%
\pgfpathlineto{\pgfqpoint{4.560812in}{2.344814in}}%
\pgfpathlineto{\pgfqpoint{4.563118in}{2.327937in}}%
\pgfpathlineto{\pgfqpoint{4.563887in}{2.326020in}}%
\pgfpathlineto{\pgfqpoint{4.564463in}{2.332439in}}%
\pgfpathlineto{\pgfqpoint{4.565040in}{2.326884in}}%
\pgfpathlineto{\pgfqpoint{4.565424in}{2.331292in}}%
\pgfpathlineto{\pgfqpoint{4.566385in}{2.335404in}}%
\pgfpathlineto{\pgfqpoint{4.566001in}{2.330877in}}%
\pgfpathlineto{\pgfqpoint{4.566769in}{2.333674in}}%
\pgfpathlineto{\pgfqpoint{4.566961in}{2.331451in}}%
\pgfpathlineto{\pgfqpoint{4.567346in}{2.337275in}}%
\pgfpathlineto{\pgfqpoint{4.567538in}{2.335709in}}%
\pgfpathlineto{\pgfqpoint{4.568691in}{2.346015in}}%
\pgfpathlineto{\pgfqpoint{4.569075in}{2.343025in}}%
\pgfpathlineto{\pgfqpoint{4.569267in}{2.344782in}}%
\pgfpathlineto{\pgfqpoint{4.569652in}{2.338451in}}%
\pgfpathlineto{\pgfqpoint{4.570036in}{2.337721in}}%
\pgfpathlineto{\pgfqpoint{4.570997in}{2.340886in}}%
\pgfpathlineto{\pgfqpoint{4.571381in}{2.335270in}}%
\pgfpathlineto{\pgfqpoint{4.572150in}{2.340115in}}%
\pgfpathlineto{\pgfqpoint{4.573111in}{2.342320in}}%
\pgfpathlineto{\pgfqpoint{4.572726in}{2.338192in}}%
\pgfpathlineto{\pgfqpoint{4.573303in}{2.341257in}}%
\pgfpathlineto{\pgfqpoint{4.573687in}{2.338966in}}%
\pgfpathlineto{\pgfqpoint{4.574456in}{2.340080in}}%
\pgfpathlineto{\pgfqpoint{4.576185in}{2.349689in}}%
\pgfpathlineto{\pgfqpoint{4.576378in}{2.347220in}}%
\pgfpathlineto{\pgfqpoint{4.576954in}{2.351795in}}%
\pgfpathlineto{\pgfqpoint{4.577146in}{2.350932in}}%
\pgfpathlineto{\pgfqpoint{4.577915in}{2.354049in}}%
\pgfpathlineto{\pgfqpoint{4.578299in}{2.351986in}}%
\pgfpathlineto{\pgfqpoint{4.578684in}{2.348901in}}%
\pgfpathlineto{\pgfqpoint{4.579837in}{2.343033in}}%
\pgfpathlineto{\pgfqpoint{4.580605in}{2.352210in}}%
\pgfpathlineto{\pgfqpoint{4.581182in}{2.349975in}}%
\pgfpathlineto{\pgfqpoint{4.581566in}{2.349049in}}%
\pgfpathlineto{\pgfqpoint{4.582527in}{2.346955in}}%
\pgfpathlineto{\pgfqpoint{4.582143in}{2.350054in}}%
\pgfpathlineto{\pgfqpoint{4.582719in}{2.347053in}}%
\pgfpathlineto{\pgfqpoint{4.582911in}{2.347416in}}%
\pgfpathlineto{\pgfqpoint{4.583104in}{2.351692in}}%
\pgfpathlineto{\pgfqpoint{4.583872in}{2.345586in}}%
\pgfpathlineto{\pgfqpoint{4.584449in}{2.346150in}}%
\pgfpathlineto{\pgfqpoint{4.585794in}{2.352022in}}%
\pgfpathlineto{\pgfqpoint{4.585986in}{2.350911in}}%
\pgfpathlineto{\pgfqpoint{4.586178in}{2.352812in}}%
\pgfpathlineto{\pgfqpoint{4.586755in}{2.358572in}}%
\pgfpathlineto{\pgfqpoint{4.587523in}{2.356946in}}%
\pgfpathlineto{\pgfqpoint{4.587716in}{2.356036in}}%
\pgfpathlineto{\pgfqpoint{4.587908in}{2.356741in}}%
\pgfpathlineto{\pgfqpoint{4.588484in}{2.361093in}}%
\pgfpathlineto{\pgfqpoint{4.588869in}{2.359625in}}%
\pgfpathlineto{\pgfqpoint{4.589637in}{2.354565in}}%
\pgfpathlineto{\pgfqpoint{4.590022in}{2.357779in}}%
\pgfpathlineto{\pgfqpoint{4.590406in}{2.361583in}}%
\pgfpathlineto{\pgfqpoint{4.591175in}{2.360578in}}%
\pgfpathlineto{\pgfqpoint{4.591367in}{2.360916in}}%
\pgfpathlineto{\pgfqpoint{4.591751in}{2.359639in}}%
\pgfpathlineto{\pgfqpoint{4.591943in}{2.359633in}}%
\pgfpathlineto{\pgfqpoint{4.592328in}{2.362279in}}%
\pgfpathlineto{\pgfqpoint{4.592712in}{2.357280in}}%
\pgfpathlineto{\pgfqpoint{4.593096in}{2.360566in}}%
\pgfpathlineto{\pgfqpoint{4.593673in}{2.354579in}}%
\pgfpathlineto{\pgfqpoint{4.594441in}{2.356302in}}%
\pgfpathlineto{\pgfqpoint{4.594634in}{2.356543in}}%
\pgfpathlineto{\pgfqpoint{4.596171in}{2.365825in}}%
\pgfpathlineto{\pgfqpoint{4.596555in}{2.365206in}}%
\pgfpathlineto{\pgfqpoint{4.598285in}{2.352560in}}%
\pgfpathlineto{\pgfqpoint{4.599438in}{2.361792in}}%
\pgfpathlineto{\pgfqpoint{4.599822in}{2.355937in}}%
\pgfpathlineto{\pgfqpoint{4.600399in}{2.353127in}}%
\pgfpathlineto{\pgfqpoint{4.601167in}{2.355540in}}%
\pgfpathlineto{\pgfqpoint{4.601552in}{2.357958in}}%
\pgfpathlineto{\pgfqpoint{4.601744in}{2.354625in}}%
\pgfpathlineto{\pgfqpoint{4.602705in}{2.343993in}}%
\pgfpathlineto{\pgfqpoint{4.603281in}{2.346919in}}%
\pgfpathlineto{\pgfqpoint{4.603858in}{2.345028in}}%
\pgfpathlineto{\pgfqpoint{4.604050in}{2.347217in}}%
\pgfpathlineto{\pgfqpoint{4.604242in}{2.348283in}}%
\pgfpathlineto{\pgfqpoint{4.605011in}{2.346444in}}%
\pgfpathlineto{\pgfqpoint{4.605587in}{2.342946in}}%
\pgfpathlineto{\pgfqpoint{4.605972in}{2.347417in}}%
\pgfpathlineto{\pgfqpoint{4.606164in}{2.345432in}}%
\pgfpathlineto{\pgfqpoint{4.606548in}{2.347015in}}%
\pgfpathlineto{\pgfqpoint{4.607125in}{2.347070in}}%
\pgfpathlineto{\pgfqpoint{4.608085in}{2.338362in}}%
\pgfpathlineto{\pgfqpoint{4.608662in}{2.341556in}}%
\pgfpathlineto{\pgfqpoint{4.608854in}{2.339122in}}%
\pgfpathlineto{\pgfqpoint{4.610007in}{2.328776in}}%
\pgfpathlineto{\pgfqpoint{4.610391in}{2.330556in}}%
\pgfpathlineto{\pgfqpoint{4.611352in}{2.334627in}}%
\pgfpathlineto{\pgfqpoint{4.610776in}{2.328700in}}%
\pgfpathlineto{\pgfqpoint{4.611737in}{2.333863in}}%
\pgfpathlineto{\pgfqpoint{4.613274in}{2.326417in}}%
\pgfpathlineto{\pgfqpoint{4.613466in}{2.326813in}}%
\pgfpathlineto{\pgfqpoint{4.613658in}{2.329203in}}%
\pgfpathlineto{\pgfqpoint{4.614427in}{2.326302in}}%
\pgfpathlineto{\pgfqpoint{4.614619in}{2.325324in}}%
\pgfpathlineto{\pgfqpoint{4.614811in}{2.328624in}}%
\pgfpathlineto{\pgfqpoint{4.615196in}{2.327262in}}%
\pgfpathlineto{\pgfqpoint{4.615388in}{2.328842in}}%
\pgfpathlineto{\pgfqpoint{4.615964in}{2.324731in}}%
\pgfpathlineto{\pgfqpoint{4.617309in}{2.317566in}}%
\pgfpathlineto{\pgfqpoint{4.617502in}{2.318498in}}%
\pgfpathlineto{\pgfqpoint{4.618847in}{2.326258in}}%
\pgfpathlineto{\pgfqpoint{4.619231in}{2.325304in}}%
\pgfpathlineto{\pgfqpoint{4.619423in}{2.323255in}}%
\pgfpathlineto{\pgfqpoint{4.620192in}{2.326213in}}%
\pgfpathlineto{\pgfqpoint{4.621729in}{2.331904in}}%
\pgfpathlineto{\pgfqpoint{4.621921in}{2.331292in}}%
\pgfpathlineto{\pgfqpoint{4.622114in}{2.329354in}}%
\pgfpathlineto{\pgfqpoint{4.622498in}{2.332010in}}%
\pgfpathlineto{\pgfqpoint{4.623074in}{2.330695in}}%
\pgfpathlineto{\pgfqpoint{4.623459in}{2.332535in}}%
\pgfpathlineto{\pgfqpoint{4.624420in}{2.328035in}}%
\pgfpathlineto{\pgfqpoint{4.625188in}{2.332119in}}%
\pgfpathlineto{\pgfqpoint{4.625380in}{2.330758in}}%
\pgfpathlineto{\pgfqpoint{4.625957in}{2.325809in}}%
\pgfpathlineto{\pgfqpoint{4.626534in}{2.327715in}}%
\pgfpathlineto{\pgfqpoint{4.627494in}{2.335272in}}%
\pgfpathlineto{\pgfqpoint{4.627687in}{2.333292in}}%
\pgfpathlineto{\pgfqpoint{4.629032in}{2.321464in}}%
\pgfpathlineto{\pgfqpoint{4.629224in}{2.322181in}}%
\pgfpathlineto{\pgfqpoint{4.630761in}{2.332152in}}%
\pgfpathlineto{\pgfqpoint{4.631146in}{2.329393in}}%
\pgfpathlineto{\pgfqpoint{4.631338in}{2.328478in}}%
\pgfpathlineto{\pgfqpoint{4.631530in}{2.331605in}}%
\pgfpathlineto{\pgfqpoint{4.632299in}{2.338156in}}%
\pgfpathlineto{\pgfqpoint{4.633259in}{2.335969in}}%
\pgfpathlineto{\pgfqpoint{4.633836in}{2.331828in}}%
\pgfpathlineto{\pgfqpoint{4.634220in}{2.335603in}}%
\pgfpathlineto{\pgfqpoint{4.634605in}{2.335425in}}%
\pgfpathlineto{\pgfqpoint{4.635565in}{2.342200in}}%
\pgfpathlineto{\pgfqpoint{4.636718in}{2.348054in}}%
\pgfpathlineto{\pgfqpoint{4.636911in}{2.347281in}}%
\pgfpathlineto{\pgfqpoint{4.638256in}{2.342786in}}%
\pgfpathlineto{\pgfqpoint{4.639024in}{2.349771in}}%
\pgfpathlineto{\pgfqpoint{4.639409in}{2.346609in}}%
\pgfpathlineto{\pgfqpoint{4.640177in}{2.347381in}}%
\pgfpathlineto{\pgfqpoint{4.640754in}{2.345293in}}%
\pgfpathlineto{\pgfqpoint{4.640946in}{2.345854in}}%
\pgfpathlineto{\pgfqpoint{4.642099in}{2.335110in}}%
\pgfpathlineto{\pgfqpoint{4.642291in}{2.335412in}}%
\pgfpathlineto{\pgfqpoint{4.643252in}{2.340709in}}%
\pgfpathlineto{\pgfqpoint{4.643444in}{2.339229in}}%
\pgfpathlineto{\pgfqpoint{4.644213in}{2.334037in}}%
\pgfpathlineto{\pgfqpoint{4.644597in}{2.337099in}}%
\pgfpathlineto{\pgfqpoint{4.645558in}{2.344104in}}%
\pgfpathlineto{\pgfqpoint{4.645942in}{2.341668in}}%
\pgfpathlineto{\pgfqpoint{4.646903in}{2.342436in}}%
\pgfpathlineto{\pgfqpoint{4.647288in}{2.338800in}}%
\pgfpathlineto{\pgfqpoint{4.647672in}{2.342073in}}%
\pgfpathlineto{\pgfqpoint{4.648248in}{2.338752in}}%
\pgfpathlineto{\pgfqpoint{4.649209in}{2.334819in}}%
\pgfpathlineto{\pgfqpoint{4.649401in}{2.337507in}}%
\pgfpathlineto{\pgfqpoint{4.649978in}{2.338977in}}%
\pgfpathlineto{\pgfqpoint{4.649786in}{2.336575in}}%
\pgfpathlineto{\pgfqpoint{4.650362in}{2.336975in}}%
\pgfpathlineto{\pgfqpoint{4.650554in}{2.334284in}}%
\pgfpathlineto{\pgfqpoint{4.651323in}{2.336113in}}%
\pgfpathlineto{\pgfqpoint{4.651708in}{2.337889in}}%
\pgfpathlineto{\pgfqpoint{4.652092in}{2.334794in}}%
\pgfpathlineto{\pgfqpoint{4.652284in}{2.335765in}}%
\pgfpathlineto{\pgfqpoint{4.652476in}{2.335637in}}%
\pgfpathlineto{\pgfqpoint{4.652668in}{2.337688in}}%
\pgfpathlineto{\pgfqpoint{4.653245in}{2.333379in}}%
\pgfpathlineto{\pgfqpoint{4.654398in}{2.324605in}}%
\pgfpathlineto{\pgfqpoint{4.654782in}{2.327628in}}%
\pgfpathlineto{\pgfqpoint{4.656127in}{2.338316in}}%
\pgfpathlineto{\pgfqpoint{4.656512in}{2.336392in}}%
\pgfpathlineto{\pgfqpoint{4.658241in}{2.328103in}}%
\pgfpathlineto{\pgfqpoint{4.659010in}{2.332717in}}%
\pgfpathlineto{\pgfqpoint{4.659394in}{2.330117in}}%
\pgfpathlineto{\pgfqpoint{4.661508in}{2.318827in}}%
\pgfpathlineto{\pgfqpoint{4.661700in}{2.320834in}}%
\pgfpathlineto{\pgfqpoint{4.662469in}{2.326632in}}%
\pgfpathlineto{\pgfqpoint{4.663045in}{2.323307in}}%
\pgfpathlineto{\pgfqpoint{4.663622in}{2.321632in}}%
\pgfpathlineto{\pgfqpoint{4.664006in}{2.322603in}}%
\pgfpathlineto{\pgfqpoint{4.664198in}{2.323835in}}%
\pgfpathlineto{\pgfqpoint{4.664583in}{2.320374in}}%
\pgfpathlineto{\pgfqpoint{4.665159in}{2.319570in}}%
\pgfpathlineto{\pgfqpoint{4.665351in}{2.321000in}}%
\pgfpathlineto{\pgfqpoint{4.665544in}{2.320101in}}%
\pgfpathlineto{\pgfqpoint{4.665736in}{2.320850in}}%
\pgfpathlineto{\pgfqpoint{4.666120in}{2.320037in}}%
\pgfpathlineto{\pgfqpoint{4.666697in}{2.314326in}}%
\pgfpathlineto{\pgfqpoint{4.667273in}{2.319543in}}%
\pgfpathlineto{\pgfqpoint{4.668426in}{2.312372in}}%
\pgfpathlineto{\pgfqpoint{4.669195in}{2.314833in}}%
\pgfpathlineto{\pgfqpoint{4.669387in}{2.314854in}}%
\pgfpathlineto{\pgfqpoint{4.670732in}{2.321650in}}%
\pgfpathlineto{\pgfqpoint{4.672269in}{2.315105in}}%
\pgfpathlineto{\pgfqpoint{4.673230in}{2.315632in}}%
\pgfpathlineto{\pgfqpoint{4.673615in}{2.312119in}}%
\pgfpathlineto{\pgfqpoint{4.673999in}{2.317037in}}%
\pgfpathlineto{\pgfqpoint{4.674768in}{2.313378in}}%
\pgfpathlineto{\pgfqpoint{4.674960in}{2.311086in}}%
\pgfpathlineto{\pgfqpoint{4.675344in}{2.316357in}}%
\pgfpathlineto{\pgfqpoint{4.675729in}{2.312941in}}%
\pgfpathlineto{\pgfqpoint{4.675921in}{2.316828in}}%
\pgfpathlineto{\pgfqpoint{4.676305in}{2.312000in}}%
\pgfpathlineto{\pgfqpoint{4.676882in}{2.316234in}}%
\pgfpathlineto{\pgfqpoint{4.677458in}{2.315326in}}%
\pgfpathlineto{\pgfqpoint{4.677650in}{2.316680in}}%
\pgfpathlineto{\pgfqpoint{4.677842in}{2.317165in}}%
\pgfpathlineto{\pgfqpoint{4.679188in}{2.304191in}}%
\pgfpathlineto{\pgfqpoint{4.679380in}{2.305095in}}%
\pgfpathlineto{\pgfqpoint{4.679956in}{2.306088in}}%
\pgfpathlineto{\pgfqpoint{4.680148in}{2.304582in}}%
\pgfpathlineto{\pgfqpoint{4.680341in}{2.300963in}}%
\pgfpathlineto{\pgfqpoint{4.680725in}{2.304836in}}%
\pgfpathlineto{\pgfqpoint{4.681301in}{2.301714in}}%
\pgfpathlineto{\pgfqpoint{4.681494in}{2.301330in}}%
\pgfpathlineto{\pgfqpoint{4.681878in}{2.302462in}}%
\pgfpathlineto{\pgfqpoint{4.682070in}{2.302078in}}%
\pgfpathlineto{\pgfqpoint{4.682839in}{2.306275in}}%
\pgfpathlineto{\pgfqpoint{4.683607in}{2.304377in}}%
\pgfpathlineto{\pgfqpoint{4.685145in}{2.293147in}}%
\pgfpathlineto{\pgfqpoint{4.685337in}{2.294986in}}%
\pgfpathlineto{\pgfqpoint{4.686682in}{2.300160in}}%
\pgfpathlineto{\pgfqpoint{4.687066in}{2.301465in}}%
\pgfpathlineto{\pgfqpoint{4.687259in}{2.299839in}}%
\pgfpathlineto{\pgfqpoint{4.688796in}{2.308094in}}%
\pgfpathlineto{\pgfqpoint{4.689180in}{2.309395in}}%
\pgfpathlineto{\pgfqpoint{4.689949in}{2.307419in}}%
\pgfpathlineto{\pgfqpoint{4.690141in}{2.309374in}}%
\pgfpathlineto{\pgfqpoint{4.690718in}{2.307525in}}%
\pgfpathlineto{\pgfqpoint{4.691102in}{2.303847in}}%
\pgfpathlineto{\pgfqpoint{4.691486in}{2.308677in}}%
\pgfpathlineto{\pgfqpoint{4.691871in}{2.310972in}}%
\pgfpathlineto{\pgfqpoint{4.692447in}{2.308062in}}%
\pgfpathlineto{\pgfqpoint{4.692639in}{2.310148in}}%
\pgfpathlineto{\pgfqpoint{4.692831in}{2.308711in}}%
\pgfpathlineto{\pgfqpoint{4.693216in}{2.310968in}}%
\pgfpathlineto{\pgfqpoint{4.694177in}{2.313952in}}%
\pgfpathlineto{\pgfqpoint{4.694369in}{2.311075in}}%
\pgfpathlineto{\pgfqpoint{4.695330in}{2.313081in}}%
\pgfpathlineto{\pgfqpoint{4.695906in}{2.311223in}}%
\pgfpathlineto{\pgfqpoint{4.697443in}{2.325226in}}%
\pgfpathlineto{\pgfqpoint{4.698212in}{2.320593in}}%
\pgfpathlineto{\pgfqpoint{4.698596in}{2.320817in}}%
\pgfpathlineto{\pgfqpoint{4.698789in}{2.323460in}}%
\pgfpathlineto{\pgfqpoint{4.699749in}{2.323037in}}%
\pgfpathlineto{\pgfqpoint{4.701095in}{2.327791in}}%
\pgfpathlineto{\pgfqpoint{4.702440in}{2.330985in}}%
\pgfpathlineto{\pgfqpoint{4.702632in}{2.330124in}}%
\pgfpathlineto{\pgfqpoint{4.704169in}{2.319461in}}%
\pgfpathlineto{\pgfqpoint{4.706091in}{2.329338in}}%
\pgfpathlineto{\pgfqpoint{4.707628in}{2.323900in}}%
\pgfpathlineto{\pgfqpoint{4.711472in}{2.343931in}}%
\pgfpathlineto{\pgfqpoint{4.712048in}{2.340886in}}%
\pgfpathlineto{\pgfqpoint{4.712433in}{2.343485in}}%
\pgfpathlineto{\pgfqpoint{4.712625in}{2.346218in}}%
\pgfpathlineto{\pgfqpoint{4.712817in}{2.342868in}}%
\pgfpathlineto{\pgfqpoint{4.713201in}{2.342870in}}%
\pgfpathlineto{\pgfqpoint{4.714162in}{2.332533in}}%
\pgfpathlineto{\pgfqpoint{4.714546in}{2.334004in}}%
\pgfpathlineto{\pgfqpoint{4.716468in}{2.344047in}}%
\pgfpathlineto{\pgfqpoint{4.716852in}{2.342315in}}%
\pgfpathlineto{\pgfqpoint{4.717621in}{2.336748in}}%
\pgfpathlineto{\pgfqpoint{4.718198in}{2.337799in}}%
\pgfpathlineto{\pgfqpoint{4.719351in}{2.341090in}}%
\pgfpathlineto{\pgfqpoint{4.719735in}{2.337756in}}%
\pgfpathlineto{\pgfqpoint{4.720311in}{2.340769in}}%
\pgfpathlineto{\pgfqpoint{4.721657in}{2.346571in}}%
\pgfpathlineto{\pgfqpoint{4.721080in}{2.339558in}}%
\pgfpathlineto{\pgfqpoint{4.722041in}{2.346128in}}%
\pgfpathlineto{\pgfqpoint{4.722617in}{2.347142in}}%
\pgfpathlineto{\pgfqpoint{4.724155in}{2.342611in}}%
\pgfpathlineto{\pgfqpoint{4.724731in}{2.342127in}}%
\pgfpathlineto{\pgfqpoint{4.724924in}{2.343836in}}%
\pgfpathlineto{\pgfqpoint{4.726461in}{2.334407in}}%
\pgfpathlineto{\pgfqpoint{4.726653in}{2.336777in}}%
\pgfpathlineto{\pgfqpoint{4.727230in}{2.330926in}}%
\pgfpathlineto{\pgfqpoint{4.727806in}{2.328234in}}%
\pgfpathlineto{\pgfqpoint{4.727998in}{2.328825in}}%
\pgfpathlineto{\pgfqpoint{4.728383in}{2.332583in}}%
\pgfpathlineto{\pgfqpoint{4.728959in}{2.329636in}}%
\pgfpathlineto{\pgfqpoint{4.729920in}{2.323416in}}%
\pgfpathlineto{\pgfqpoint{4.730304in}{2.324779in}}%
\pgfpathlineto{\pgfqpoint{4.731649in}{2.334271in}}%
\pgfpathlineto{\pgfqpoint{4.731842in}{2.332808in}}%
\pgfpathlineto{\pgfqpoint{4.732610in}{2.334354in}}%
\pgfpathlineto{\pgfqpoint{4.732802in}{2.334851in}}%
\pgfpathlineto{\pgfqpoint{4.733955in}{2.341828in}}%
\pgfpathlineto{\pgfqpoint{4.734148in}{2.341339in}}%
\pgfpathlineto{\pgfqpoint{4.734532in}{2.341432in}}%
\pgfpathlineto{\pgfqpoint{4.737030in}{2.326998in}}%
\pgfpathlineto{\pgfqpoint{4.737414in}{2.331071in}}%
\pgfpathlineto{\pgfqpoint{4.738183in}{2.329186in}}%
\pgfpathlineto{\pgfqpoint{4.738567in}{2.326701in}}%
\pgfpathlineto{\pgfqpoint{4.738760in}{2.328695in}}%
\pgfpathlineto{\pgfqpoint{4.740681in}{2.345644in}}%
\pgfpathlineto{\pgfqpoint{4.741258in}{2.342609in}}%
\pgfpathlineto{\pgfqpoint{4.741642in}{2.344444in}}%
\pgfpathlineto{\pgfqpoint{4.742603in}{2.349398in}}%
\pgfpathlineto{\pgfqpoint{4.742987in}{2.347335in}}%
\pgfpathlineto{\pgfqpoint{4.743756in}{2.345567in}}%
\pgfpathlineto{\pgfqpoint{4.743564in}{2.348703in}}%
\pgfpathlineto{\pgfqpoint{4.744140in}{2.345934in}}%
\pgfpathlineto{\pgfqpoint{4.744332in}{2.347556in}}%
\pgfpathlineto{\pgfqpoint{4.744717in}{2.342127in}}%
\pgfpathlineto{\pgfqpoint{4.745101in}{2.345351in}}%
\pgfpathlineto{\pgfqpoint{4.746062in}{2.349607in}}%
\pgfpathlineto{\pgfqpoint{4.747023in}{2.348178in}}%
\pgfpathlineto{\pgfqpoint{4.748176in}{2.340648in}}%
\pgfpathlineto{\pgfqpoint{4.748752in}{2.343229in}}%
\pgfpathlineto{\pgfqpoint{4.749521in}{2.348781in}}%
\pgfpathlineto{\pgfqpoint{4.749713in}{2.346193in}}%
\pgfpathlineto{\pgfqpoint{4.750674in}{2.342015in}}%
\pgfpathlineto{\pgfqpoint{4.750866in}{2.343815in}}%
\pgfpathlineto{\pgfqpoint{4.751251in}{2.342688in}}%
\pgfpathlineto{\pgfqpoint{4.751635in}{2.344716in}}%
\pgfpathlineto{\pgfqpoint{4.751827in}{2.344314in}}%
\pgfpathlineto{\pgfqpoint{4.752404in}{2.348279in}}%
\pgfpathlineto{\pgfqpoint{4.752788in}{2.344032in}}%
\pgfpathlineto{\pgfqpoint{4.752980in}{2.344065in}}%
\pgfpathlineto{\pgfqpoint{4.753364in}{2.342196in}}%
\pgfpathlineto{\pgfqpoint{4.753749in}{2.345626in}}%
\pgfpathlineto{\pgfqpoint{4.754133in}{2.342381in}}%
\pgfpathlineto{\pgfqpoint{4.754902in}{2.347172in}}%
\pgfpathlineto{\pgfqpoint{4.755286in}{2.346001in}}%
\pgfpathlineto{\pgfqpoint{4.756247in}{2.336910in}}%
\pgfpathlineto{\pgfqpoint{4.756631in}{2.339458in}}%
\pgfpathlineto{\pgfqpoint{4.757016in}{2.342358in}}%
\pgfpathlineto{\pgfqpoint{4.757592in}{2.339425in}}%
\pgfpathlineto{\pgfqpoint{4.757784in}{2.336715in}}%
\pgfpathlineto{\pgfqpoint{4.758361in}{2.343187in}}%
\pgfpathlineto{\pgfqpoint{4.758553in}{2.340157in}}%
\pgfpathlineto{\pgfqpoint{4.758745in}{2.340168in}}%
\pgfpathlineto{\pgfqpoint{4.760090in}{2.344613in}}%
\pgfpathlineto{\pgfqpoint{4.762204in}{2.353970in}}%
\pgfpathlineto{\pgfqpoint{4.762396in}{2.351770in}}%
\pgfpathlineto{\pgfqpoint{4.763165in}{2.352975in}}%
\pgfpathlineto{\pgfqpoint{4.763549in}{2.359281in}}%
\pgfpathlineto{\pgfqpoint{4.764318in}{2.356044in}}%
\pgfpathlineto{\pgfqpoint{4.765471in}{2.347692in}}%
\pgfpathlineto{\pgfqpoint{4.765663in}{2.349037in}}%
\pgfpathlineto{\pgfqpoint{4.765855in}{2.349211in}}%
\pgfpathlineto{\pgfqpoint{4.766047in}{2.348889in}}%
\pgfpathlineto{\pgfqpoint{4.766240in}{2.351986in}}%
\pgfpathlineto{\pgfqpoint{4.766624in}{2.345690in}}%
\pgfpathlineto{\pgfqpoint{4.767008in}{2.347050in}}%
\pgfpathlineto{\pgfqpoint{4.767585in}{2.351162in}}%
\pgfpathlineto{\pgfqpoint{4.768161in}{2.348174in}}%
\pgfpathlineto{\pgfqpoint{4.768930in}{2.344266in}}%
\pgfpathlineto{\pgfqpoint{4.769122in}{2.345533in}}%
\pgfpathlineto{\pgfqpoint{4.769506in}{2.344205in}}%
\pgfpathlineto{\pgfqpoint{4.770659in}{2.352682in}}%
\pgfpathlineto{\pgfqpoint{4.770852in}{2.353140in}}%
\pgfpathlineto{\pgfqpoint{4.771044in}{2.357865in}}%
\pgfpathlineto{\pgfqpoint{4.771812in}{2.352556in}}%
\pgfpathlineto{\pgfqpoint{4.772389in}{2.349769in}}%
\pgfpathlineto{\pgfqpoint{4.772773in}{2.351423in}}%
\pgfpathlineto{\pgfqpoint{4.773158in}{2.351285in}}%
\pgfpathlineto{\pgfqpoint{4.774119in}{2.353821in}}%
\pgfpathlineto{\pgfqpoint{4.774695in}{2.350074in}}%
\pgfpathlineto{\pgfqpoint{4.775079in}{2.350968in}}%
\pgfpathlineto{\pgfqpoint{4.775272in}{2.354334in}}%
\pgfpathlineto{\pgfqpoint{4.776040in}{2.350106in}}%
\pgfpathlineto{\pgfqpoint{4.776232in}{2.349373in}}%
\pgfpathlineto{\pgfqpoint{4.776425in}{2.351301in}}%
\pgfpathlineto{\pgfqpoint{4.777578in}{2.355854in}}%
\pgfpathlineto{\pgfqpoint{4.777001in}{2.350360in}}%
\pgfpathlineto{\pgfqpoint{4.777770in}{2.354847in}}%
\pgfpathlineto{\pgfqpoint{4.781421in}{2.343197in}}%
\pgfpathlineto{\pgfqpoint{4.782766in}{2.353573in}}%
\pgfpathlineto{\pgfqpoint{4.782958in}{2.353325in}}%
\pgfpathlineto{\pgfqpoint{4.783727in}{2.354269in}}%
\pgfpathlineto{\pgfqpoint{4.783919in}{2.352419in}}%
\pgfpathlineto{\pgfqpoint{4.784303in}{2.359309in}}%
\pgfpathlineto{\pgfqpoint{4.784496in}{2.361148in}}%
\pgfpathlineto{\pgfqpoint{4.785072in}{2.356755in}}%
\pgfpathlineto{\pgfqpoint{4.785264in}{2.352825in}}%
\pgfpathlineto{\pgfqpoint{4.786033in}{2.356443in}}%
\pgfpathlineto{\pgfqpoint{4.786417in}{2.354429in}}%
\pgfpathlineto{\pgfqpoint{4.786994in}{2.348750in}}%
\pgfpathlineto{\pgfqpoint{4.787570in}{2.351112in}}%
\pgfpathlineto{\pgfqpoint{4.787955in}{2.352260in}}%
\pgfpathlineto{\pgfqpoint{4.788339in}{2.350015in}}%
\pgfpathlineto{\pgfqpoint{4.788723in}{2.354442in}}%
\pgfpathlineto{\pgfqpoint{4.788915in}{2.349717in}}%
\pgfpathlineto{\pgfqpoint{4.789492in}{2.354793in}}%
\pgfpathlineto{\pgfqpoint{4.789684in}{2.354627in}}%
\pgfpathlineto{\pgfqpoint{4.789876in}{2.356932in}}%
\pgfpathlineto{\pgfqpoint{4.790453in}{2.351279in}}%
\pgfpathlineto{\pgfqpoint{4.790645in}{2.351162in}}%
\pgfpathlineto{\pgfqpoint{4.790837in}{2.351868in}}%
\pgfpathlineto{\pgfqpoint{4.791414in}{2.346217in}}%
\pgfpathlineto{\pgfqpoint{4.791990in}{2.349764in}}%
\pgfpathlineto{\pgfqpoint{4.792374in}{2.348299in}}%
\pgfpathlineto{\pgfqpoint{4.792567in}{2.350812in}}%
\pgfpathlineto{\pgfqpoint{4.792759in}{2.351143in}}%
\pgfpathlineto{\pgfqpoint{4.794488in}{2.343139in}}%
\pgfpathlineto{\pgfqpoint{4.794873in}{2.339716in}}%
\pgfpathlineto{\pgfqpoint{4.795449in}{2.342651in}}%
\pgfpathlineto{\pgfqpoint{4.796794in}{2.346349in}}%
\pgfpathlineto{\pgfqpoint{4.796986in}{2.344724in}}%
\pgfpathlineto{\pgfqpoint{4.797371in}{2.343791in}}%
\pgfpathlineto{\pgfqpoint{4.798716in}{2.350911in}}%
\pgfpathlineto{\pgfqpoint{4.800446in}{2.344366in}}%
\pgfpathlineto{\pgfqpoint{4.799100in}{2.353430in}}%
\pgfpathlineto{\pgfqpoint{4.800638in}{2.344552in}}%
\pgfpathlineto{\pgfqpoint{4.801022in}{2.343615in}}%
\pgfpathlineto{\pgfqpoint{4.802752in}{2.352733in}}%
\pgfpathlineto{\pgfqpoint{4.803136in}{2.355008in}}%
\pgfpathlineto{\pgfqpoint{4.806018in}{2.367769in}}%
\pgfpathlineto{\pgfqpoint{4.806595in}{2.366683in}}%
\pgfpathlineto{\pgfqpoint{4.807364in}{2.370129in}}%
\pgfpathlineto{\pgfqpoint{4.808324in}{2.373251in}}%
\pgfpathlineto{\pgfqpoint{4.808517in}{2.373033in}}%
\pgfpathlineto{\pgfqpoint{4.809093in}{2.369308in}}%
\pgfpathlineto{\pgfqpoint{4.809670in}{2.371792in}}%
\pgfpathlineto{\pgfqpoint{4.810054in}{2.367318in}}%
\pgfpathlineto{\pgfqpoint{4.810438in}{2.370777in}}%
\pgfpathlineto{\pgfqpoint{4.811015in}{2.378967in}}%
\pgfpathlineto{\pgfqpoint{4.811783in}{2.378050in}}%
\pgfpathlineto{\pgfqpoint{4.812168in}{2.376361in}}%
\pgfpathlineto{\pgfqpoint{4.812360in}{2.379460in}}%
\pgfpathlineto{\pgfqpoint{4.812744in}{2.382866in}}%
\pgfpathlineto{\pgfqpoint{4.812936in}{2.380780in}}%
\pgfpathlineto{\pgfqpoint{4.813705in}{2.371865in}}%
\pgfpathlineto{\pgfqpoint{4.814089in}{2.376905in}}%
\pgfpathlineto{\pgfqpoint{4.814282in}{2.376808in}}%
\pgfpathlineto{\pgfqpoint{4.814474in}{2.377594in}}%
\pgfpathlineto{\pgfqpoint{4.815627in}{2.384359in}}%
\pgfpathlineto{\pgfqpoint{4.815819in}{2.382181in}}%
\pgfpathlineto{\pgfqpoint{4.817164in}{2.387045in}}%
\pgfpathlineto{\pgfqpoint{4.817548in}{2.386288in}}%
\pgfpathlineto{\pgfqpoint{4.818125in}{2.391759in}}%
\pgfpathlineto{\pgfqpoint{4.818701in}{2.389789in}}%
\pgfpathlineto{\pgfqpoint{4.819278in}{2.386002in}}%
\pgfpathlineto{\pgfqpoint{4.819662in}{2.390719in}}%
\pgfpathlineto{\pgfqpoint{4.820047in}{2.387089in}}%
\pgfpathlineto{\pgfqpoint{4.821007in}{2.389457in}}%
\pgfpathlineto{\pgfqpoint{4.821200in}{2.388969in}}%
\pgfpathlineto{\pgfqpoint{4.822737in}{2.397692in}}%
\pgfpathlineto{\pgfqpoint{4.823506in}{2.388858in}}%
\pgfpathlineto{\pgfqpoint{4.824082in}{2.392778in}}%
\pgfpathlineto{\pgfqpoint{4.824467in}{2.397174in}}%
\pgfpathlineto{\pgfqpoint{4.825427in}{2.396631in}}%
\pgfpathlineto{\pgfqpoint{4.825620in}{2.397282in}}%
\pgfpathlineto{\pgfqpoint{4.825812in}{2.395540in}}%
\pgfpathlineto{\pgfqpoint{4.826196in}{2.390948in}}%
\pgfpathlineto{\pgfqpoint{4.827157in}{2.392731in}}%
\pgfpathlineto{\pgfqpoint{4.827541in}{2.394438in}}%
\pgfpathlineto{\pgfqpoint{4.827733in}{2.391085in}}%
\pgfpathlineto{\pgfqpoint{4.827926in}{2.392615in}}%
\pgfpathlineto{\pgfqpoint{4.828118in}{2.390060in}}%
\pgfpathlineto{\pgfqpoint{4.828694in}{2.393892in}}%
\pgfpathlineto{\pgfqpoint{4.829463in}{2.399065in}}%
\pgfpathlineto{\pgfqpoint{4.829847in}{2.397125in}}%
\pgfpathlineto{\pgfqpoint{4.830616in}{2.394161in}}%
\pgfpathlineto{\pgfqpoint{4.830808in}{2.395568in}}%
\pgfpathlineto{\pgfqpoint{4.831000in}{2.398383in}}%
\pgfpathlineto{\pgfqpoint{4.831577in}{2.393803in}}%
\pgfpathlineto{\pgfqpoint{4.831769in}{2.394782in}}%
\pgfpathlineto{\pgfqpoint{4.831961in}{2.393271in}}%
\pgfpathlineto{\pgfqpoint{4.832345in}{2.396575in}}%
\pgfpathlineto{\pgfqpoint{4.832730in}{2.395980in}}%
\pgfpathlineto{\pgfqpoint{4.833114in}{2.396070in}}%
\pgfpathlineto{\pgfqpoint{4.834075in}{2.393055in}}%
\pgfpathlineto{\pgfqpoint{4.833498in}{2.396323in}}%
\pgfpathlineto{\pgfqpoint{4.834267in}{2.395561in}}%
\pgfpathlineto{\pgfqpoint{4.836957in}{2.410242in}}%
\pgfpathlineto{\pgfqpoint{4.837150in}{2.408268in}}%
\pgfpathlineto{\pgfqpoint{4.838495in}{2.403386in}}%
\pgfpathlineto{\pgfqpoint{4.839456in}{2.411750in}}%
\pgfpathlineto{\pgfqpoint{4.839840in}{2.410444in}}%
\pgfpathlineto{\pgfqpoint{4.840993in}{2.404978in}}%
\pgfpathlineto{\pgfqpoint{4.841762in}{2.399470in}}%
\pgfpathlineto{\pgfqpoint{4.842146in}{2.403057in}}%
\pgfpathlineto{\pgfqpoint{4.842915in}{2.400723in}}%
\pgfpathlineto{\pgfqpoint{4.843107in}{2.401855in}}%
\pgfpathlineto{\pgfqpoint{4.843299in}{2.405595in}}%
\pgfpathlineto{\pgfqpoint{4.843683in}{2.401434in}}%
\pgfpathlineto{\pgfqpoint{4.844260in}{2.402859in}}%
\pgfpathlineto{\pgfqpoint{4.844836in}{2.406635in}}%
\pgfpathlineto{\pgfqpoint{4.845413in}{2.403426in}}%
\pgfpathlineto{\pgfqpoint{4.845605in}{2.404353in}}%
\pgfpathlineto{\pgfqpoint{4.845797in}{2.400966in}}%
\pgfpathlineto{\pgfqpoint{4.846950in}{2.397189in}}%
\pgfpathlineto{\pgfqpoint{4.848103in}{2.401718in}}%
\pgfpathlineto{\pgfqpoint{4.847335in}{2.396244in}}%
\pgfpathlineto{\pgfqpoint{4.848295in}{2.401559in}}%
\pgfpathlineto{\pgfqpoint{4.848488in}{2.399464in}}%
\pgfpathlineto{\pgfqpoint{4.848680in}{2.402153in}}%
\pgfpathlineto{\pgfqpoint{4.849256in}{2.400269in}}%
\pgfpathlineto{\pgfqpoint{4.849641in}{2.399531in}}%
\pgfpathlineto{\pgfqpoint{4.849833in}{2.401470in}}%
\pgfpathlineto{\pgfqpoint{4.850409in}{2.396402in}}%
\pgfpathlineto{\pgfqpoint{4.850986in}{2.400099in}}%
\pgfpathlineto{\pgfqpoint{4.851178in}{2.400051in}}%
\pgfpathlineto{\pgfqpoint{4.851754in}{2.400808in}}%
\pgfpathlineto{\pgfqpoint{4.852331in}{2.395548in}}%
\pgfpathlineto{\pgfqpoint{4.853100in}{2.402268in}}%
\pgfpathlineto{\pgfqpoint{4.853484in}{2.399305in}}%
\pgfpathlineto{\pgfqpoint{4.853868in}{2.393472in}}%
\pgfpathlineto{\pgfqpoint{4.854637in}{2.396683in}}%
\pgfpathlineto{\pgfqpoint{4.855021in}{2.396130in}}%
\pgfpathlineto{\pgfqpoint{4.855213in}{2.397491in}}%
\pgfpathlineto{\pgfqpoint{4.856174in}{2.392830in}}%
\pgfpathlineto{\pgfqpoint{4.857327in}{2.399349in}}%
\pgfpathlineto{\pgfqpoint{4.857519in}{2.399113in}}%
\pgfpathlineto{\pgfqpoint{4.857904in}{2.397647in}}%
\pgfpathlineto{\pgfqpoint{4.858096in}{2.400458in}}%
\pgfpathlineto{\pgfqpoint{4.858480in}{2.398881in}}%
\pgfpathlineto{\pgfqpoint{4.859249in}{2.404412in}}%
\pgfpathlineto{\pgfqpoint{4.859633in}{2.403073in}}%
\pgfpathlineto{\pgfqpoint{4.859825in}{2.400451in}}%
\pgfpathlineto{\pgfqpoint{4.860786in}{2.401939in}}%
\pgfpathlineto{\pgfqpoint{4.861555in}{2.398636in}}%
\pgfpathlineto{\pgfqpoint{4.862516in}{2.400342in}}%
\pgfpathlineto{\pgfqpoint{4.862900in}{2.403916in}}%
\pgfpathlineto{\pgfqpoint{4.863861in}{2.402533in}}%
\pgfpathlineto{\pgfqpoint{4.865206in}{2.407207in}}%
\pgfpathlineto{\pgfqpoint{4.865398in}{2.408816in}}%
\pgfpathlineto{\pgfqpoint{4.865590in}{2.405184in}}%
\pgfpathlineto{\pgfqpoint{4.865783in}{2.407487in}}%
\pgfpathlineto{\pgfqpoint{4.866167in}{2.402474in}}%
\pgfpathlineto{\pgfqpoint{4.867128in}{2.403329in}}%
\pgfpathlineto{\pgfqpoint{4.868473in}{2.406692in}}%
\pgfpathlineto{\pgfqpoint{4.868665in}{2.405442in}}%
\pgfpathlineto{\pgfqpoint{4.868857in}{2.405214in}}%
\pgfpathlineto{\pgfqpoint{4.869049in}{2.406947in}}%
\pgfpathlineto{\pgfqpoint{4.869242in}{2.404005in}}%
\pgfpathlineto{\pgfqpoint{4.869818in}{2.404004in}}%
\pgfpathlineto{\pgfqpoint{4.870010in}{2.401715in}}%
\pgfpathlineto{\pgfqpoint{4.870779in}{2.404443in}}%
\pgfpathlineto{\pgfqpoint{4.872893in}{2.417902in}}%
\pgfpathlineto{\pgfqpoint{4.873085in}{2.417801in}}%
\pgfpathlineto{\pgfqpoint{4.873469in}{2.418131in}}%
\pgfpathlineto{\pgfqpoint{4.874238in}{2.416928in}}%
\pgfpathlineto{\pgfqpoint{4.874430in}{2.417647in}}%
\pgfpathlineto{\pgfqpoint{4.874815in}{2.415318in}}%
\pgfpathlineto{\pgfqpoint{4.875583in}{2.415888in}}%
\pgfpathlineto{\pgfqpoint{4.876160in}{2.410302in}}%
\pgfpathlineto{\pgfqpoint{4.876736in}{2.414094in}}%
\pgfpathlineto{\pgfqpoint{4.877121in}{2.409069in}}%
\pgfpathlineto{\pgfqpoint{4.877313in}{2.410642in}}%
\pgfpathlineto{\pgfqpoint{4.877889in}{2.406262in}}%
\pgfpathlineto{\pgfqpoint{4.879811in}{2.399523in}}%
\pgfpathlineto{\pgfqpoint{4.880003in}{2.399950in}}%
\pgfpathlineto{\pgfqpoint{4.880195in}{2.401055in}}%
\pgfpathlineto{\pgfqpoint{4.880387in}{2.399330in}}%
\pgfpathlineto{\pgfqpoint{4.884231in}{2.357801in}}%
\pgfpathlineto{\pgfqpoint{4.884423in}{2.360012in}}%
\pgfpathlineto{\pgfqpoint{4.885192in}{2.356896in}}%
\pgfpathlineto{\pgfqpoint{4.885384in}{2.356489in}}%
\pgfpathlineto{\pgfqpoint{4.885768in}{2.357411in}}%
\pgfpathlineto{\pgfqpoint{4.887690in}{2.375630in}}%
\pgfpathlineto{\pgfqpoint{4.887882in}{2.375624in}}%
\pgfpathlineto{\pgfqpoint{4.888266in}{2.382164in}}%
\pgfpathlineto{\pgfqpoint{4.888843in}{2.377208in}}%
\pgfpathlineto{\pgfqpoint{4.890764in}{2.369035in}}%
\pgfpathlineto{\pgfqpoint{4.892110in}{2.374621in}}%
\pgfpathlineto{\pgfqpoint{4.892878in}{2.370327in}}%
\pgfpathlineto{\pgfqpoint{4.893070in}{2.370612in}}%
\pgfpathlineto{\pgfqpoint{4.893839in}{2.361064in}}%
\pgfpathlineto{\pgfqpoint{4.894416in}{2.365968in}}%
\pgfpathlineto{\pgfqpoint{4.897106in}{2.385184in}}%
\pgfpathlineto{\pgfqpoint{4.898067in}{2.382175in}}%
\pgfpathlineto{\pgfqpoint{4.898259in}{2.384184in}}%
\pgfpathlineto{\pgfqpoint{4.898451in}{2.382686in}}%
\pgfpathlineto{\pgfqpoint{4.898836in}{2.386014in}}%
\pgfpathlineto{\pgfqpoint{4.899028in}{2.387106in}}%
\pgfpathlineto{\pgfqpoint{4.899220in}{2.382086in}}%
\pgfpathlineto{\pgfqpoint{4.899796in}{2.384346in}}%
\pgfpathlineto{\pgfqpoint{4.899989in}{2.381785in}}%
\pgfpathlineto{\pgfqpoint{4.900757in}{2.378785in}}%
\pgfpathlineto{\pgfqpoint{4.901142in}{2.378928in}}%
\pgfpathlineto{\pgfqpoint{4.901718in}{2.382710in}}%
\pgfpathlineto{\pgfqpoint{4.902102in}{2.380827in}}%
\pgfpathlineto{\pgfqpoint{4.902487in}{2.376448in}}%
\pgfpathlineto{\pgfqpoint{4.903255in}{2.379862in}}%
\pgfpathlineto{\pgfqpoint{4.903448in}{2.379846in}}%
\pgfpathlineto{\pgfqpoint{4.904985in}{2.387895in}}%
\pgfpathlineto{\pgfqpoint{4.904024in}{2.379580in}}%
\pgfpathlineto{\pgfqpoint{4.905369in}{2.387316in}}%
\pgfpathlineto{\pgfqpoint{4.905561in}{2.385670in}}%
\pgfpathlineto{\pgfqpoint{4.906138in}{2.388912in}}%
\pgfpathlineto{\pgfqpoint{4.907099in}{2.393490in}}%
\pgfpathlineto{\pgfqpoint{4.908828in}{2.406722in}}%
\pgfpathlineto{\pgfqpoint{4.909405in}{2.405670in}}%
\pgfpathlineto{\pgfqpoint{4.909981in}{2.403498in}}%
\pgfpathlineto{\pgfqpoint{4.911519in}{2.392650in}}%
\pgfpathlineto{\pgfqpoint{4.914401in}{2.384707in}}%
\pgfpathlineto{\pgfqpoint{4.914593in}{2.386033in}}%
\pgfpathlineto{\pgfqpoint{4.916515in}{2.397530in}}%
\pgfpathlineto{\pgfqpoint{4.916707in}{2.396086in}}%
\pgfpathlineto{\pgfqpoint{4.916899in}{2.393819in}}%
\pgfpathlineto{\pgfqpoint{4.917668in}{2.397788in}}%
\pgfpathlineto{\pgfqpoint{4.917860in}{2.399829in}}%
\pgfpathlineto{\pgfqpoint{4.918437in}{2.397500in}}%
\pgfpathlineto{\pgfqpoint{4.918629in}{2.399433in}}%
\pgfpathlineto{\pgfqpoint{4.919205in}{2.395370in}}%
\pgfpathlineto{\pgfqpoint{4.919782in}{2.396036in}}%
\pgfpathlineto{\pgfqpoint{4.919974in}{2.397526in}}%
\pgfpathlineto{\pgfqpoint{4.920358in}{2.393908in}}%
\pgfpathlineto{\pgfqpoint{4.921319in}{2.394612in}}%
\pgfpathlineto{\pgfqpoint{4.921704in}{2.392831in}}%
\pgfpathlineto{\pgfqpoint{4.921896in}{2.395161in}}%
\pgfpathlineto{\pgfqpoint{4.922472in}{2.390675in}}%
\pgfpathlineto{\pgfqpoint{4.922664in}{2.390617in}}%
\pgfpathlineto{\pgfqpoint{4.923049in}{2.395601in}}%
\pgfpathlineto{\pgfqpoint{4.924010in}{2.394251in}}%
\pgfpathlineto{\pgfqpoint{4.924202in}{2.394306in}}%
\pgfpathlineto{\pgfqpoint{4.924586in}{2.391832in}}%
\pgfpathlineto{\pgfqpoint{4.924970in}{2.395452in}}%
\pgfpathlineto{\pgfqpoint{4.925163in}{2.393902in}}%
\pgfpathlineto{\pgfqpoint{4.926892in}{2.398955in}}%
\pgfpathlineto{\pgfqpoint{4.927084in}{2.398839in}}%
\pgfpathlineto{\pgfqpoint{4.927276in}{2.397046in}}%
\pgfpathlineto{\pgfqpoint{4.927661in}{2.400741in}}%
\pgfpathlineto{\pgfqpoint{4.928237in}{2.404047in}}%
\pgfpathlineto{\pgfqpoint{4.928622in}{2.401676in}}%
\pgfpathlineto{\pgfqpoint{4.930159in}{2.398423in}}%
\pgfpathlineto{\pgfqpoint{4.930543in}{2.404222in}}%
\pgfpathlineto{\pgfqpoint{4.931312in}{2.402958in}}%
\pgfpathlineto{\pgfqpoint{4.931504in}{2.401190in}}%
\pgfpathlineto{\pgfqpoint{4.931888in}{2.403869in}}%
\pgfpathlineto{\pgfqpoint{4.932081in}{2.403216in}}%
\pgfpathlineto{\pgfqpoint{4.932657in}{2.407839in}}%
\pgfpathlineto{\pgfqpoint{4.933041in}{2.406824in}}%
\pgfpathlineto{\pgfqpoint{4.933426in}{2.402046in}}%
\pgfpathlineto{\pgfqpoint{4.934194in}{2.405401in}}%
\pgfpathlineto{\pgfqpoint{4.934387in}{2.404954in}}%
\pgfpathlineto{\pgfqpoint{4.935924in}{2.393604in}}%
\pgfpathlineto{\pgfqpoint{4.936116in}{2.394631in}}%
\pgfpathlineto{\pgfqpoint{4.936308in}{2.390042in}}%
\pgfpathlineto{\pgfqpoint{4.937077in}{2.383295in}}%
\pgfpathlineto{\pgfqpoint{4.937461in}{2.385378in}}%
\pgfpathlineto{\pgfqpoint{4.938038in}{2.385983in}}%
\pgfpathlineto{\pgfqpoint{4.938614in}{2.383378in}}%
\pgfpathlineto{\pgfqpoint{4.940152in}{2.395278in}}%
\pgfpathlineto{\pgfqpoint{4.940728in}{2.389744in}}%
\pgfpathlineto{\pgfqpoint{4.941112in}{2.396248in}}%
\pgfpathlineto{\pgfqpoint{4.941305in}{2.394396in}}%
\pgfpathlineto{\pgfqpoint{4.942458in}{2.399187in}}%
\pgfpathlineto{\pgfqpoint{4.942650in}{2.399123in}}%
\pgfpathlineto{\pgfqpoint{4.943034in}{2.400380in}}%
\pgfpathlineto{\pgfqpoint{4.943226in}{2.396179in}}%
\pgfpathlineto{\pgfqpoint{4.943611in}{2.404848in}}%
\pgfpathlineto{\pgfqpoint{4.944379in}{2.400639in}}%
\pgfpathlineto{\pgfqpoint{4.945917in}{2.414373in}}%
\pgfpathlineto{\pgfqpoint{4.946685in}{2.410988in}}%
\pgfpathlineto{\pgfqpoint{4.947646in}{2.409448in}}%
\pgfpathlineto{\pgfqpoint{4.947262in}{2.412606in}}%
\pgfpathlineto{\pgfqpoint{4.947838in}{2.409957in}}%
\pgfpathlineto{\pgfqpoint{4.948991in}{2.414709in}}%
\pgfpathlineto{\pgfqpoint{4.949184in}{2.413078in}}%
\pgfpathlineto{\pgfqpoint{4.949568in}{2.408860in}}%
\pgfpathlineto{\pgfqpoint{4.949952in}{2.406293in}}%
\pgfpathlineto{\pgfqpoint{4.950337in}{2.409227in}}%
\pgfpathlineto{\pgfqpoint{4.950721in}{2.407439in}}%
\pgfpathlineto{\pgfqpoint{4.952066in}{2.414353in}}%
\pgfpathlineto{\pgfqpoint{4.952643in}{2.412382in}}%
\pgfpathlineto{\pgfqpoint{4.954564in}{2.402922in}}%
\pgfpathlineto{\pgfqpoint{4.953027in}{2.412852in}}%
\pgfpathlineto{\pgfqpoint{4.954756in}{2.403267in}}%
\pgfpathlineto{\pgfqpoint{4.955333in}{2.405809in}}%
\pgfpathlineto{\pgfqpoint{4.955141in}{2.402945in}}%
\pgfpathlineto{\pgfqpoint{4.955525in}{2.402859in}}%
\pgfpathlineto{\pgfqpoint{4.957639in}{2.390142in}}%
\pgfpathlineto{\pgfqpoint{4.960906in}{2.401536in}}%
\pgfpathlineto{\pgfqpoint{4.961098in}{2.400204in}}%
\pgfpathlineto{\pgfqpoint{4.961674in}{2.394033in}}%
\pgfpathlineto{\pgfqpoint{4.962059in}{2.395892in}}%
\pgfpathlineto{\pgfqpoint{4.963020in}{2.401551in}}%
\pgfpathlineto{\pgfqpoint{4.963212in}{2.399547in}}%
\pgfpathlineto{\pgfqpoint{4.963404in}{2.399637in}}%
\pgfpathlineto{\pgfqpoint{4.963788in}{2.396769in}}%
\pgfpathlineto{\pgfqpoint{4.964173in}{2.403652in}}%
\pgfpathlineto{\pgfqpoint{4.964941in}{2.410743in}}%
\pgfpathlineto{\pgfqpoint{4.966094in}{2.409490in}}%
\pgfpathlineto{\pgfqpoint{4.966479in}{2.405245in}}%
\pgfpathlineto{\pgfqpoint{4.967055in}{2.407574in}}%
\pgfpathlineto{\pgfqpoint{4.967439in}{2.409926in}}%
\pgfpathlineto{\pgfqpoint{4.967632in}{2.407387in}}%
\pgfpathlineto{\pgfqpoint{4.968016in}{2.402607in}}%
\pgfpathlineto{\pgfqpoint{4.968592in}{2.407443in}}%
\pgfpathlineto{\pgfqpoint{4.968785in}{2.407504in}}%
\pgfpathlineto{\pgfqpoint{4.968977in}{2.404215in}}%
\pgfpathlineto{\pgfqpoint{4.969553in}{2.408867in}}%
\pgfpathlineto{\pgfqpoint{4.969938in}{2.405765in}}%
\pgfpathlineto{\pgfqpoint{4.970130in}{2.406151in}}%
\pgfpathlineto{\pgfqpoint{4.970322in}{2.405376in}}%
\pgfpathlineto{\pgfqpoint{4.970514in}{2.403807in}}%
\pgfpathlineto{\pgfqpoint{4.970899in}{2.406397in}}%
\pgfpathlineto{\pgfqpoint{4.971091in}{2.404305in}}%
\pgfpathlineto{\pgfqpoint{4.971283in}{2.407142in}}%
\pgfpathlineto{\pgfqpoint{4.972052in}{2.403320in}}%
\pgfpathlineto{\pgfqpoint{4.972628in}{2.406600in}}%
\pgfpathlineto{\pgfqpoint{4.973973in}{2.413412in}}%
\pgfpathlineto{\pgfqpoint{4.974165in}{2.412272in}}%
\pgfpathlineto{\pgfqpoint{4.974742in}{2.412986in}}%
\pgfpathlineto{\pgfqpoint{4.975703in}{2.416793in}}%
\pgfpathlineto{\pgfqpoint{4.975318in}{2.412077in}}%
\pgfpathlineto{\pgfqpoint{4.975895in}{2.413836in}}%
\pgfpathlineto{\pgfqpoint{4.976471in}{2.410694in}}%
\pgfpathlineto{\pgfqpoint{4.976664in}{2.414832in}}%
\pgfpathlineto{\pgfqpoint{4.977432in}{2.418762in}}%
\pgfpathlineto{\pgfqpoint{4.978201in}{2.418323in}}%
\pgfpathlineto{\pgfqpoint{4.978393in}{2.417088in}}%
\pgfpathlineto{\pgfqpoint{4.978585in}{2.421767in}}%
\pgfpathlineto{\pgfqpoint{4.978777in}{2.422578in}}%
\pgfpathlineto{\pgfqpoint{4.978970in}{2.419802in}}%
\pgfpathlineto{\pgfqpoint{4.979162in}{2.421225in}}%
\pgfpathlineto{\pgfqpoint{4.979546in}{2.416727in}}%
\pgfpathlineto{\pgfqpoint{4.980315in}{2.417389in}}%
\pgfpathlineto{\pgfqpoint{4.980507in}{2.420853in}}%
\pgfpathlineto{\pgfqpoint{4.981468in}{2.418792in}}%
\pgfpathlineto{\pgfqpoint{4.981660in}{2.415619in}}%
\pgfpathlineto{\pgfqpoint{4.982429in}{2.420219in}}%
\pgfpathlineto{\pgfqpoint{4.982621in}{2.424228in}}%
\pgfpathlineto{\pgfqpoint{4.983197in}{2.419179in}}%
\pgfpathlineto{\pgfqpoint{4.983582in}{2.421905in}}%
\pgfpathlineto{\pgfqpoint{4.983774in}{2.421374in}}%
\pgfpathlineto{\pgfqpoint{4.984927in}{2.415944in}}%
\pgfpathlineto{\pgfqpoint{4.985119in}{2.416521in}}%
\pgfpathlineto{\pgfqpoint{4.985311in}{2.416134in}}%
\pgfpathlineto{\pgfqpoint{4.986848in}{2.403511in}}%
\pgfpathlineto{\pgfqpoint{4.987041in}{2.405032in}}%
\pgfpathlineto{\pgfqpoint{4.987233in}{2.403925in}}%
\pgfpathlineto{\pgfqpoint{4.987425in}{2.407890in}}%
\pgfpathlineto{\pgfqpoint{4.988770in}{2.413970in}}%
\pgfpathlineto{\pgfqpoint{4.988962in}{2.411316in}}%
\pgfpathlineto{\pgfqpoint{4.990500in}{2.397284in}}%
\pgfpathlineto{\pgfqpoint{4.991268in}{2.402456in}}%
\pgfpathlineto{\pgfqpoint{4.991653in}{2.401089in}}%
\pgfpathlineto{\pgfqpoint{4.991845in}{2.400767in}}%
\pgfpathlineto{\pgfqpoint{4.992037in}{2.402308in}}%
\pgfpathlineto{\pgfqpoint{4.993190in}{2.412626in}}%
\pgfpathlineto{\pgfqpoint{4.993959in}{2.410058in}}%
\pgfpathlineto{\pgfqpoint{4.994535in}{2.409190in}}%
\pgfpathlineto{\pgfqpoint{4.996265in}{2.399241in}}%
\pgfpathlineto{\pgfqpoint{4.996649in}{2.399243in}}%
\pgfpathlineto{\pgfqpoint{4.997033in}{2.395844in}}%
\pgfpathlineto{\pgfqpoint{4.997418in}{2.399016in}}%
\pgfpathlineto{\pgfqpoint{4.997994in}{2.403793in}}%
\pgfpathlineto{\pgfqpoint{4.998571in}{2.403067in}}%
\pgfpathlineto{\pgfqpoint{4.998955in}{2.403849in}}%
\pgfpathlineto{\pgfqpoint{5.000108in}{2.399019in}}%
\pgfpathlineto{\pgfqpoint{5.000300in}{2.399240in}}%
\pgfpathlineto{\pgfqpoint{5.000492in}{2.402552in}}%
\pgfpathlineto{\pgfqpoint{5.001453in}{2.402279in}}%
\pgfpathlineto{\pgfqpoint{5.001645in}{2.403136in}}%
\pgfpathlineto{\pgfqpoint{5.002222in}{2.401225in}}%
\pgfpathlineto{\pgfqpoint{5.003183in}{2.393655in}}%
\pgfpathlineto{\pgfqpoint{5.003375in}{2.397878in}}%
\pgfpathlineto{\pgfqpoint{5.004912in}{2.404764in}}%
\pgfpathlineto{\pgfqpoint{5.005104in}{2.403823in}}%
\pgfpathlineto{\pgfqpoint{5.005297in}{2.400829in}}%
\pgfpathlineto{\pgfqpoint{5.006065in}{2.404061in}}%
\pgfpathlineto{\pgfqpoint{5.006257in}{2.403360in}}%
\pgfpathlineto{\pgfqpoint{5.006450in}{2.405813in}}%
\pgfpathlineto{\pgfqpoint{5.007026in}{2.403807in}}%
\pgfpathlineto{\pgfqpoint{5.007218in}{2.405115in}}%
\pgfpathlineto{\pgfqpoint{5.007795in}{2.402544in}}%
\pgfpathlineto{\pgfqpoint{5.007987in}{2.402352in}}%
\pgfpathlineto{\pgfqpoint{5.009524in}{2.394247in}}%
\pgfpathlineto{\pgfqpoint{5.010101in}{2.393043in}}%
\pgfpathlineto{\pgfqpoint{5.010485in}{2.396429in}}%
\pgfpathlineto{\pgfqpoint{5.010677in}{2.394288in}}%
\pgfpathlineto{\pgfqpoint{5.011254in}{2.400226in}}%
\pgfpathlineto{\pgfqpoint{5.012022in}{2.398670in}}%
\pgfpathlineto{\pgfqpoint{5.012215in}{2.399035in}}%
\pgfpathlineto{\pgfqpoint{5.012599in}{2.404344in}}%
\pgfpathlineto{\pgfqpoint{5.013175in}{2.397445in}}%
\pgfpathlineto{\pgfqpoint{5.013368in}{2.398903in}}%
\pgfpathlineto{\pgfqpoint{5.013944in}{2.396352in}}%
\pgfpathlineto{\pgfqpoint{5.015289in}{2.391008in}}%
\pgfpathlineto{\pgfqpoint{5.016442in}{2.395159in}}%
\pgfpathlineto{\pgfqpoint{5.017211in}{2.386362in}}%
\pgfpathlineto{\pgfqpoint{5.017980in}{2.388761in}}%
\pgfpathlineto{\pgfqpoint{5.018556in}{2.388219in}}%
\pgfpathlineto{\pgfqpoint{5.019517in}{2.395002in}}%
\pgfpathlineto{\pgfqpoint{5.021054in}{2.385909in}}%
\pgfpathlineto{\pgfqpoint{5.022015in}{2.395065in}}%
\pgfpathlineto{\pgfqpoint{5.022207in}{2.392980in}}%
\pgfpathlineto{\pgfqpoint{5.022400in}{2.390891in}}%
\pgfpathlineto{\pgfqpoint{5.022784in}{2.394822in}}%
\pgfpathlineto{\pgfqpoint{5.022976in}{2.397876in}}%
\pgfpathlineto{\pgfqpoint{5.023553in}{2.391216in}}%
\pgfpathlineto{\pgfqpoint{5.023745in}{2.392546in}}%
\pgfpathlineto{\pgfqpoint{5.025090in}{2.385543in}}%
\pgfpathlineto{\pgfqpoint{5.025859in}{2.386096in}}%
\pgfpathlineto{\pgfqpoint{5.026051in}{2.388638in}}%
\pgfpathlineto{\pgfqpoint{5.026819in}{2.384817in}}%
\pgfpathlineto{\pgfqpoint{5.027396in}{2.383659in}}%
\pgfpathlineto{\pgfqpoint{5.028357in}{2.388548in}}%
\pgfpathlineto{\pgfqpoint{5.029318in}{2.389133in}}%
\pgfpathlineto{\pgfqpoint{5.029510in}{2.386237in}}%
\pgfpathlineto{\pgfqpoint{5.029702in}{2.388628in}}%
\pgfpathlineto{\pgfqpoint{5.030278in}{2.385021in}}%
\pgfpathlineto{\pgfqpoint{5.030471in}{2.387143in}}%
\pgfpathlineto{\pgfqpoint{5.031239in}{2.379629in}}%
\pgfpathlineto{\pgfqpoint{5.032008in}{2.382475in}}%
\pgfpathlineto{\pgfqpoint{5.032584in}{2.385210in}}%
\pgfpathlineto{\pgfqpoint{5.032777in}{2.383691in}}%
\pgfpathlineto{\pgfqpoint{5.033737in}{2.377486in}}%
\pgfpathlineto{\pgfqpoint{5.033930in}{2.379147in}}%
\pgfpathlineto{\pgfqpoint{5.034506in}{2.384213in}}%
\pgfpathlineto{\pgfqpoint{5.035275in}{2.381371in}}%
\pgfpathlineto{\pgfqpoint{5.036236in}{2.378908in}}%
\pgfpathlineto{\pgfqpoint{5.036428in}{2.380557in}}%
\pgfpathlineto{\pgfqpoint{5.036812in}{2.381667in}}%
\pgfpathlineto{\pgfqpoint{5.037004in}{2.382830in}}%
\pgfpathlineto{\pgfqpoint{5.037389in}{2.378677in}}%
\pgfpathlineto{\pgfqpoint{5.037581in}{2.380637in}}%
\pgfpathlineto{\pgfqpoint{5.037773in}{2.380159in}}%
\pgfpathlineto{\pgfqpoint{5.037965in}{2.381358in}}%
\pgfpathlineto{\pgfqpoint{5.039118in}{2.387449in}}%
\pgfpathlineto{\pgfqpoint{5.039310in}{2.386798in}}%
\pgfpathlineto{\pgfqpoint{5.040463in}{2.382669in}}%
\pgfpathlineto{\pgfqpoint{5.041616in}{2.371673in}}%
\pgfpathlineto{\pgfqpoint{5.042577in}{2.372831in}}%
\pgfpathlineto{\pgfqpoint{5.043538in}{2.378660in}}%
\pgfpathlineto{\pgfqpoint{5.043922in}{2.376585in}}%
\pgfpathlineto{\pgfqpoint{5.044307in}{2.371761in}}%
\pgfpathlineto{\pgfqpoint{5.044883in}{2.372500in}}%
\pgfpathlineto{\pgfqpoint{5.045844in}{2.389314in}}%
\pgfpathlineto{\pgfqpoint{5.046228in}{2.385176in}}%
\pgfpathlineto{\pgfqpoint{5.046421in}{2.381760in}}%
\pgfpathlineto{\pgfqpoint{5.046997in}{2.386795in}}%
\pgfpathlineto{\pgfqpoint{5.047381in}{2.384157in}}%
\pgfpathlineto{\pgfqpoint{5.047574in}{2.383515in}}%
\pgfpathlineto{\pgfqpoint{5.047766in}{2.384749in}}%
\pgfpathlineto{\pgfqpoint{5.047958in}{2.384524in}}%
\pgfpathlineto{\pgfqpoint{5.048150in}{2.387061in}}%
\pgfpathlineto{\pgfqpoint{5.048727in}{2.381815in}}%
\pgfpathlineto{\pgfqpoint{5.049111in}{2.380219in}}%
\pgfpathlineto{\pgfqpoint{5.049880in}{2.380864in}}%
\pgfpathlineto{\pgfqpoint{5.050264in}{2.383552in}}%
\pgfpathlineto{\pgfqpoint{5.050840in}{2.381472in}}%
\pgfpathlineto{\pgfqpoint{5.051417in}{2.383298in}}%
\pgfpathlineto{\pgfqpoint{5.051993in}{2.379860in}}%
\pgfpathlineto{\pgfqpoint{5.052186in}{2.381128in}}%
\pgfpathlineto{\pgfqpoint{5.052570in}{2.376934in}}%
\pgfpathlineto{\pgfqpoint{5.053146in}{2.380206in}}%
\pgfpathlineto{\pgfqpoint{5.053723in}{2.377137in}}%
\pgfpathlineto{\pgfqpoint{5.054107in}{2.378583in}}%
\pgfpathlineto{\pgfqpoint{5.054299in}{2.381908in}}%
\pgfpathlineto{\pgfqpoint{5.054876in}{2.376552in}}%
\pgfpathlineto{\pgfqpoint{5.055068in}{2.374577in}}%
\pgfpathlineto{\pgfqpoint{5.056029in}{2.374807in}}%
\pgfpathlineto{\pgfqpoint{5.056798in}{2.379271in}}%
\pgfpathlineto{\pgfqpoint{5.056990in}{2.380023in}}%
\pgfpathlineto{\pgfqpoint{5.057374in}{2.378512in}}%
\pgfpathlineto{\pgfqpoint{5.057758in}{2.379689in}}%
\pgfpathlineto{\pgfqpoint{5.057951in}{2.379002in}}%
\pgfpathlineto{\pgfqpoint{5.058335in}{2.380452in}}%
\pgfpathlineto{\pgfqpoint{5.059488in}{2.397431in}}%
\pgfpathlineto{\pgfqpoint{5.060064in}{2.394596in}}%
\pgfpathlineto{\pgfqpoint{5.060449in}{2.396114in}}%
\pgfpathlineto{\pgfqpoint{5.060641in}{2.392873in}}%
\pgfpathlineto{\pgfqpoint{5.061217in}{2.395360in}}%
\pgfpathlineto{\pgfqpoint{5.061410in}{2.394919in}}%
\pgfpathlineto{\pgfqpoint{5.061602in}{2.396157in}}%
\pgfpathlineto{\pgfqpoint{5.063523in}{2.403739in}}%
\pgfpathlineto{\pgfqpoint{5.063716in}{2.403017in}}%
\pgfpathlineto{\pgfqpoint{5.063908in}{2.399810in}}%
\pgfpathlineto{\pgfqpoint{5.064100in}{2.403723in}}%
\pgfpathlineto{\pgfqpoint{5.064676in}{2.402420in}}%
\pgfpathlineto{\pgfqpoint{5.066598in}{2.415295in}}%
\pgfpathlineto{\pgfqpoint{5.067751in}{2.414471in}}%
\pgfpathlineto{\pgfqpoint{5.068328in}{2.409967in}}%
\pgfpathlineto{\pgfqpoint{5.068904in}{2.411937in}}%
\pgfpathlineto{\pgfqpoint{5.069481in}{2.406701in}}%
\pgfpathlineto{\pgfqpoint{5.070442in}{2.407143in}}%
\pgfpathlineto{\pgfqpoint{5.071210in}{2.404447in}}%
\pgfpathlineto{\pgfqpoint{5.071402in}{2.407846in}}%
\pgfpathlineto{\pgfqpoint{5.071979in}{2.406747in}}%
\pgfpathlineto{\pgfqpoint{5.072363in}{2.401515in}}%
\pgfpathlineto{\pgfqpoint{5.073132in}{2.404846in}}%
\pgfpathlineto{\pgfqpoint{5.073708in}{2.401385in}}%
\pgfpathlineto{\pgfqpoint{5.074285in}{2.403331in}}%
\pgfpathlineto{\pgfqpoint{5.074861in}{2.408430in}}%
\pgfpathlineto{\pgfqpoint{5.075246in}{2.404905in}}%
\pgfpathlineto{\pgfqpoint{5.076591in}{2.399606in}}%
\pgfpathlineto{\pgfqpoint{5.076783in}{2.403635in}}%
\pgfpathlineto{\pgfqpoint{5.077360in}{2.396743in}}%
\pgfpathlineto{\pgfqpoint{5.077552in}{2.397467in}}%
\pgfpathlineto{\pgfqpoint{5.079089in}{2.387234in}}%
\pgfpathlineto{\pgfqpoint{5.079858in}{2.395493in}}%
\pgfpathlineto{\pgfqpoint{5.080819in}{2.394715in}}%
\pgfpathlineto{\pgfqpoint{5.081203in}{2.391300in}}%
\pgfpathlineto{\pgfqpoint{5.081587in}{2.395156in}}%
\pgfpathlineto{\pgfqpoint{5.081779in}{2.392987in}}%
\pgfpathlineto{\pgfqpoint{5.083317in}{2.399264in}}%
\pgfpathlineto{\pgfqpoint{5.083893in}{2.403539in}}%
\pgfpathlineto{\pgfqpoint{5.084278in}{2.401192in}}%
\pgfpathlineto{\pgfqpoint{5.085046in}{2.392828in}}%
\pgfpathlineto{\pgfqpoint{5.085431in}{2.396503in}}%
\pgfpathlineto{\pgfqpoint{5.085623in}{2.397677in}}%
\pgfpathlineto{\pgfqpoint{5.086007in}{2.395142in}}%
\pgfpathlineto{\pgfqpoint{5.086776in}{2.389185in}}%
\pgfpathlineto{\pgfqpoint{5.087160in}{2.393536in}}%
\pgfpathlineto{\pgfqpoint{5.087737in}{2.394741in}}%
\pgfpathlineto{\pgfqpoint{5.088121in}{2.391965in}}%
\pgfpathlineto{\pgfqpoint{5.088890in}{2.397540in}}%
\pgfpathlineto{\pgfqpoint{5.089274in}{2.396387in}}%
\pgfpathlineto{\pgfqpoint{5.090619in}{2.384390in}}%
\pgfpathlineto{\pgfqpoint{5.091388in}{2.386114in}}%
\pgfpathlineto{\pgfqpoint{5.092157in}{2.381413in}}%
\pgfpathlineto{\pgfqpoint{5.092733in}{2.388601in}}%
\pgfpathlineto{\pgfqpoint{5.093502in}{2.385409in}}%
\pgfpathlineto{\pgfqpoint{5.093694in}{2.384544in}}%
\pgfpathlineto{\pgfqpoint{5.093886in}{2.386697in}}%
\pgfpathlineto{\pgfqpoint{5.094078in}{2.387981in}}%
\pgfpathlineto{\pgfqpoint{5.094463in}{2.383774in}}%
\pgfpathlineto{\pgfqpoint{5.094847in}{2.382934in}}%
\pgfpathlineto{\pgfqpoint{5.095808in}{2.389141in}}%
\pgfpathlineto{\pgfqpoint{5.096000in}{2.388081in}}%
\pgfpathlineto{\pgfqpoint{5.096384in}{2.386041in}}%
\pgfpathlineto{\pgfqpoint{5.096961in}{2.388935in}}%
\pgfpathlineto{\pgfqpoint{5.098882in}{2.400464in}}%
\pgfpathlineto{\pgfqpoint{5.097345in}{2.388807in}}%
\pgfpathlineto{\pgfqpoint{5.099459in}{2.397239in}}%
\pgfpathlineto{\pgfqpoint{5.099651in}{2.397755in}}%
\pgfpathlineto{\pgfqpoint{5.099843in}{2.395696in}}%
\pgfpathlineto{\pgfqpoint{5.102149in}{2.386919in}}%
\pgfpathlineto{\pgfqpoint{5.102341in}{2.388840in}}%
\pgfpathlineto{\pgfqpoint{5.102726in}{2.387185in}}%
\pgfpathlineto{\pgfqpoint{5.103879in}{2.394042in}}%
\pgfpathlineto{\pgfqpoint{5.104071in}{2.391234in}}%
\pgfpathlineto{\pgfqpoint{5.104647in}{2.394860in}}%
\pgfpathlineto{\pgfqpoint{5.105032in}{2.393362in}}%
\pgfpathlineto{\pgfqpoint{5.105800in}{2.396097in}}%
\pgfpathlineto{\pgfqpoint{5.105993in}{2.394112in}}%
\pgfpathlineto{\pgfqpoint{5.106377in}{2.389508in}}%
\pgfpathlineto{\pgfqpoint{5.106953in}{2.393339in}}%
\pgfpathlineto{\pgfqpoint{5.107338in}{2.395522in}}%
\pgfpathlineto{\pgfqpoint{5.107722in}{2.391388in}}%
\pgfpathlineto{\pgfqpoint{5.107914in}{2.391120in}}%
\pgfpathlineto{\pgfqpoint{5.108106in}{2.391613in}}%
\pgfpathlineto{\pgfqpoint{5.108491in}{2.394742in}}%
\pgfpathlineto{\pgfqpoint{5.109259in}{2.392333in}}%
\pgfpathlineto{\pgfqpoint{5.109452in}{2.389978in}}%
\pgfpathlineto{\pgfqpoint{5.110220in}{2.392374in}}%
\pgfpathlineto{\pgfqpoint{5.110412in}{2.392839in}}%
\pgfpathlineto{\pgfqpoint{5.110605in}{2.390961in}}%
\pgfpathlineto{\pgfqpoint{5.111181in}{2.388617in}}%
\pgfpathlineto{\pgfqpoint{5.111565in}{2.391329in}}%
\pgfpathlineto{\pgfqpoint{5.111758in}{2.390206in}}%
\pgfpathlineto{\pgfqpoint{5.112718in}{2.401416in}}%
\pgfpathlineto{\pgfqpoint{5.113679in}{2.396725in}}%
\pgfpathlineto{\pgfqpoint{5.115409in}{2.388597in}}%
\pgfpathlineto{\pgfqpoint{5.115601in}{2.390596in}}%
\pgfpathlineto{\pgfqpoint{5.116178in}{2.387133in}}%
\pgfpathlineto{\pgfqpoint{5.116562in}{2.388662in}}%
\pgfpathlineto{\pgfqpoint{5.117523in}{2.399120in}}%
\pgfpathlineto{\pgfqpoint{5.117907in}{2.396965in}}%
\pgfpathlineto{\pgfqpoint{5.118099in}{2.394133in}}%
\pgfpathlineto{\pgfqpoint{5.118291in}{2.398496in}}%
\pgfpathlineto{\pgfqpoint{5.118868in}{2.397514in}}%
\pgfpathlineto{\pgfqpoint{5.119252in}{2.398736in}}%
\pgfpathlineto{\pgfqpoint{5.120597in}{2.394419in}}%
\pgfpathlineto{\pgfqpoint{5.120982in}{2.396047in}}%
\pgfpathlineto{\pgfqpoint{5.121558in}{2.393708in}}%
\pgfpathlineto{\pgfqpoint{5.121750in}{2.393140in}}%
\pgfpathlineto{\pgfqpoint{5.122711in}{2.401848in}}%
\pgfpathlineto{\pgfqpoint{5.122903in}{2.397687in}}%
\pgfpathlineto{\pgfqpoint{5.123288in}{2.402849in}}%
\pgfpathlineto{\pgfqpoint{5.124056in}{2.399600in}}%
\pgfpathlineto{\pgfqpoint{5.124249in}{2.398528in}}%
\pgfpathlineto{\pgfqpoint{5.124633in}{2.401321in}}%
\pgfpathlineto{\pgfqpoint{5.126939in}{2.412126in}}%
\pgfpathlineto{\pgfqpoint{5.128092in}{2.414714in}}%
\pgfpathlineto{\pgfqpoint{5.130014in}{2.400506in}}%
\pgfpathlineto{\pgfqpoint{5.130590in}{2.395958in}}%
\pgfpathlineto{\pgfqpoint{5.131551in}{2.396915in}}%
\pgfpathlineto{\pgfqpoint{5.132127in}{2.399674in}}%
\pgfpathlineto{\pgfqpoint{5.132512in}{2.399159in}}%
\pgfpathlineto{\pgfqpoint{5.133280in}{2.388646in}}%
\pgfpathlineto{\pgfqpoint{5.134049in}{2.390816in}}%
\pgfpathlineto{\pgfqpoint{5.135202in}{2.400705in}}%
\pgfpathlineto{\pgfqpoint{5.135971in}{2.400083in}}%
\pgfpathlineto{\pgfqpoint{5.136739in}{2.397145in}}%
\pgfpathlineto{\pgfqpoint{5.137124in}{2.397655in}}%
\pgfpathlineto{\pgfqpoint{5.137892in}{2.396396in}}%
\pgfpathlineto{\pgfqpoint{5.138853in}{2.392910in}}%
\pgfpathlineto{\pgfqpoint{5.139045in}{2.394323in}}%
\pgfpathlineto{\pgfqpoint{5.141352in}{2.388780in}}%
\pgfpathlineto{\pgfqpoint{5.142505in}{2.397821in}}%
\pgfpathlineto{\pgfqpoint{5.142697in}{2.395760in}}%
\pgfpathlineto{\pgfqpoint{5.143850in}{2.387689in}}%
\pgfpathlineto{\pgfqpoint{5.144042in}{2.387801in}}%
\pgfpathlineto{\pgfqpoint{5.144618in}{2.390010in}}%
\pgfpathlineto{\pgfqpoint{5.144811in}{2.386716in}}%
\pgfpathlineto{\pgfqpoint{5.145195in}{2.385256in}}%
\pgfpathlineto{\pgfqpoint{5.145579in}{2.385982in}}%
\pgfpathlineto{\pgfqpoint{5.146924in}{2.390664in}}%
\pgfpathlineto{\pgfqpoint{5.147693in}{2.391338in}}%
\pgfpathlineto{\pgfqpoint{5.148077in}{2.388639in}}%
\pgfpathlineto{\pgfqpoint{5.148270in}{2.390774in}}%
\pgfpathlineto{\pgfqpoint{5.148846in}{2.386853in}}%
\pgfpathlineto{\pgfqpoint{5.149038in}{2.387164in}}%
\pgfpathlineto{\pgfqpoint{5.149230in}{2.387710in}}%
\pgfpathlineto{\pgfqpoint{5.149423in}{2.387071in}}%
\pgfpathlineto{\pgfqpoint{5.151536in}{2.373863in}}%
\pgfpathlineto{\pgfqpoint{5.149999in}{2.388148in}}%
\pgfpathlineto{\pgfqpoint{5.151729in}{2.374411in}}%
\pgfpathlineto{\pgfqpoint{5.152882in}{2.376413in}}%
\pgfpathlineto{\pgfqpoint{5.153074in}{2.370614in}}%
\pgfpathlineto{\pgfqpoint{5.154035in}{2.373666in}}%
\pgfpathlineto{\pgfqpoint{5.156341in}{2.384778in}}%
\pgfpathlineto{\pgfqpoint{5.156725in}{2.381839in}}%
\pgfpathlineto{\pgfqpoint{5.156917in}{2.379833in}}%
\pgfpathlineto{\pgfqpoint{5.157686in}{2.383426in}}%
\pgfpathlineto{\pgfqpoint{5.157878in}{2.385700in}}%
\pgfpathlineto{\pgfqpoint{5.158647in}{2.383583in}}%
\pgfpathlineto{\pgfqpoint{5.158839in}{2.382751in}}%
\pgfpathlineto{\pgfqpoint{5.159031in}{2.385174in}}%
\pgfpathlineto{\pgfqpoint{5.159223in}{2.384759in}}%
\pgfpathlineto{\pgfqpoint{5.159992in}{2.388711in}}%
\pgfpathlineto{\pgfqpoint{5.160376in}{2.387312in}}%
\pgfpathlineto{\pgfqpoint{5.163066in}{2.370260in}}%
\pgfpathlineto{\pgfqpoint{5.163451in}{2.372636in}}%
\pgfpathlineto{\pgfqpoint{5.164220in}{2.379768in}}%
\pgfpathlineto{\pgfqpoint{5.164988in}{2.379683in}}%
\pgfpathlineto{\pgfqpoint{5.165180in}{2.378352in}}%
\pgfpathlineto{\pgfqpoint{5.165565in}{2.382429in}}%
\pgfpathlineto{\pgfqpoint{5.166333in}{2.383850in}}%
\pgfpathlineto{\pgfqpoint{5.165949in}{2.381381in}}%
\pgfpathlineto{\pgfqpoint{5.166526in}{2.381828in}}%
\pgfpathlineto{\pgfqpoint{5.166910in}{2.385692in}}%
\pgfpathlineto{\pgfqpoint{5.167679in}{2.383388in}}%
\pgfpathlineto{\pgfqpoint{5.167871in}{2.384520in}}%
\pgfpathlineto{\pgfqpoint{5.168063in}{2.381551in}}%
\pgfpathlineto{\pgfqpoint{5.168639in}{2.382644in}}%
\pgfpathlineto{\pgfqpoint{5.169408in}{2.376856in}}%
\pgfpathlineto{\pgfqpoint{5.169985in}{2.377709in}}%
\pgfpathlineto{\pgfqpoint{5.170753in}{2.382303in}}%
\pgfpathlineto{\pgfqpoint{5.171138in}{2.378731in}}%
\pgfpathlineto{\pgfqpoint{5.172291in}{2.373482in}}%
\pgfpathlineto{\pgfqpoint{5.172675in}{2.373946in}}%
\pgfpathlineto{\pgfqpoint{5.174020in}{2.377632in}}%
\pgfpathlineto{\pgfqpoint{5.174212in}{2.377397in}}%
\pgfpathlineto{\pgfqpoint{5.174981in}{2.386376in}}%
\pgfpathlineto{\pgfqpoint{5.175557in}{2.382251in}}%
\pgfpathlineto{\pgfqpoint{5.175750in}{2.381799in}}%
\pgfpathlineto{\pgfqpoint{5.176134in}{2.383003in}}%
\pgfpathlineto{\pgfqpoint{5.178248in}{2.393284in}}%
\pgfpathlineto{\pgfqpoint{5.176518in}{2.380544in}}%
\pgfpathlineto{\pgfqpoint{5.179016in}{2.391344in}}%
\pgfpathlineto{\pgfqpoint{5.180554in}{2.383686in}}%
\pgfpathlineto{\pgfqpoint{5.181515in}{2.387978in}}%
\pgfpathlineto{\pgfqpoint{5.182091in}{2.387799in}}%
\pgfpathlineto{\pgfqpoint{5.182283in}{2.386226in}}%
\pgfpathlineto{\pgfqpoint{5.182668in}{2.388651in}}%
\pgfpathlineto{\pgfqpoint{5.183052in}{2.388437in}}%
\pgfpathlineto{\pgfqpoint{5.183436in}{2.389655in}}%
\pgfpathlineto{\pgfqpoint{5.183628in}{2.386077in}}%
\pgfpathlineto{\pgfqpoint{5.184205in}{2.378381in}}%
\pgfpathlineto{\pgfqpoint{5.185358in}{2.380024in}}%
\pgfpathlineto{\pgfqpoint{5.185742in}{2.384429in}}%
\pgfpathlineto{\pgfqpoint{5.186511in}{2.383034in}}%
\pgfpathlineto{\pgfqpoint{5.186703in}{2.381548in}}%
\pgfpathlineto{\pgfqpoint{5.186895in}{2.387394in}}%
\pgfpathlineto{\pgfqpoint{5.187087in}{2.386215in}}%
\pgfpathlineto{\pgfqpoint{5.187472in}{2.387028in}}%
\pgfpathlineto{\pgfqpoint{5.187664in}{2.385709in}}%
\pgfpathlineto{\pgfqpoint{5.189201in}{2.375149in}}%
\pgfpathlineto{\pgfqpoint{5.189394in}{2.375401in}}%
\pgfpathlineto{\pgfqpoint{5.190162in}{2.380042in}}%
\pgfpathlineto{\pgfqpoint{5.190547in}{2.378453in}}%
\pgfpathlineto{\pgfqpoint{5.191123in}{2.375698in}}%
\pgfpathlineto{\pgfqpoint{5.191507in}{2.377413in}}%
\pgfpathlineto{\pgfqpoint{5.194390in}{2.392713in}}%
\pgfpathlineto{\pgfqpoint{5.194966in}{2.389713in}}%
\pgfpathlineto{\pgfqpoint{5.195159in}{2.390620in}}%
\pgfpathlineto{\pgfqpoint{5.196696in}{2.405675in}}%
\pgfpathlineto{\pgfqpoint{5.196888in}{2.405513in}}%
\pgfpathlineto{\pgfqpoint{5.197849in}{2.395844in}}%
\pgfpathlineto{\pgfqpoint{5.198233in}{2.399497in}}%
\pgfpathlineto{\pgfqpoint{5.199771in}{2.407035in}}%
\pgfpathlineto{\pgfqpoint{5.200155in}{2.405720in}}%
\pgfpathlineto{\pgfqpoint{5.200539in}{2.407243in}}%
\pgfpathlineto{\pgfqpoint{5.201308in}{2.406873in}}%
\pgfpathlineto{\pgfqpoint{5.201884in}{2.409962in}}%
\pgfpathlineto{\pgfqpoint{5.202269in}{2.411086in}}%
\pgfpathlineto{\pgfqpoint{5.202461in}{2.409154in}}%
\pgfpathlineto{\pgfqpoint{5.202845in}{2.409353in}}%
\pgfpathlineto{\pgfqpoint{5.203614in}{2.401270in}}%
\pgfpathlineto{\pgfqpoint{5.204383in}{2.390792in}}%
\pgfpathlineto{\pgfqpoint{5.204959in}{2.391734in}}%
\pgfpathlineto{\pgfqpoint{5.205151in}{2.390946in}}%
\pgfpathlineto{\pgfqpoint{5.205343in}{2.392689in}}%
\pgfpathlineto{\pgfqpoint{5.205728in}{2.400061in}}%
\pgfpathlineto{\pgfqpoint{5.206496in}{2.394056in}}%
\pgfpathlineto{\pgfqpoint{5.207265in}{2.395391in}}%
\pgfpathlineto{\pgfqpoint{5.207457in}{2.394834in}}%
\pgfpathlineto{\pgfqpoint{5.208034in}{2.393367in}}%
\pgfpathlineto{\pgfqpoint{5.208226in}{2.396159in}}%
\pgfpathlineto{\pgfqpoint{5.208995in}{2.399542in}}%
\pgfpathlineto{\pgfqpoint{5.209379in}{2.397685in}}%
\pgfpathlineto{\pgfqpoint{5.210148in}{2.389630in}}%
\pgfpathlineto{\pgfqpoint{5.211493in}{2.391057in}}%
\pgfpathlineto{\pgfqpoint{5.211877in}{2.395596in}}%
\pgfpathlineto{\pgfqpoint{5.212646in}{2.394689in}}%
\pgfpathlineto{\pgfqpoint{5.213607in}{2.389603in}}%
\pgfpathlineto{\pgfqpoint{5.213799in}{2.391574in}}%
\pgfpathlineto{\pgfqpoint{5.213991in}{2.391955in}}%
\pgfpathlineto{\pgfqpoint{5.214183in}{2.390793in}}%
\pgfpathlineto{\pgfqpoint{5.215336in}{2.385995in}}%
\pgfpathlineto{\pgfqpoint{5.215528in}{2.388117in}}%
\pgfpathlineto{\pgfqpoint{5.215913in}{2.385448in}}%
\pgfpathlineto{\pgfqpoint{5.216489in}{2.387958in}}%
\pgfpathlineto{\pgfqpoint{5.216874in}{2.392095in}}%
\pgfpathlineto{\pgfqpoint{5.217450in}{2.389137in}}%
\pgfpathlineto{\pgfqpoint{5.217834in}{2.389597in}}%
\pgfpathlineto{\pgfqpoint{5.219564in}{2.382970in}}%
\pgfpathlineto{\pgfqpoint{5.220140in}{2.381391in}}%
\pgfpathlineto{\pgfqpoint{5.219948in}{2.383671in}}%
\pgfpathlineto{\pgfqpoint{5.220333in}{2.383621in}}%
\pgfpathlineto{\pgfqpoint{5.221870in}{2.386920in}}%
\pgfpathlineto{\pgfqpoint{5.222062in}{2.384734in}}%
\pgfpathlineto{\pgfqpoint{5.222639in}{2.388785in}}%
\pgfpathlineto{\pgfqpoint{5.223215in}{2.388558in}}%
\pgfpathlineto{\pgfqpoint{5.224176in}{2.394649in}}%
\pgfpathlineto{\pgfqpoint{5.224560in}{2.392774in}}%
\pgfpathlineto{\pgfqpoint{5.224752in}{2.392920in}}%
\pgfpathlineto{\pgfqpoint{5.225521in}{2.399129in}}%
\pgfpathlineto{\pgfqpoint{5.225905in}{2.397482in}}%
\pgfpathlineto{\pgfqpoint{5.226098in}{2.396498in}}%
\pgfpathlineto{\pgfqpoint{5.226482in}{2.398722in}}%
\pgfpathlineto{\pgfqpoint{5.226866in}{2.401181in}}%
\pgfpathlineto{\pgfqpoint{5.227058in}{2.400471in}}%
\pgfpathlineto{\pgfqpoint{5.227827in}{2.393521in}}%
\pgfpathlineto{\pgfqpoint{5.228211in}{2.394554in}}%
\pgfpathlineto{\pgfqpoint{5.228788in}{2.397611in}}%
\pgfpathlineto{\pgfqpoint{5.229172in}{2.396810in}}%
\pgfpathlineto{\pgfqpoint{5.229364in}{2.394470in}}%
\pgfpathlineto{\pgfqpoint{5.229749in}{2.400873in}}%
\pgfpathlineto{\pgfqpoint{5.229941in}{2.401413in}}%
\pgfpathlineto{\pgfqpoint{5.230133in}{2.399356in}}%
\pgfpathlineto{\pgfqpoint{5.230325in}{2.399793in}}%
\pgfpathlineto{\pgfqpoint{5.230517in}{2.399562in}}%
\pgfpathlineto{\pgfqpoint{5.232247in}{2.383314in}}%
\pgfpathlineto{\pgfqpoint{5.232439in}{2.387295in}}%
\pgfpathlineto{\pgfqpoint{5.233208in}{2.383504in}}%
\pgfpathlineto{\pgfqpoint{5.234745in}{2.376649in}}%
\pgfpathlineto{\pgfqpoint{5.234937in}{2.376764in}}%
\pgfpathlineto{\pgfqpoint{5.235322in}{2.379454in}}%
\pgfpathlineto{\pgfqpoint{5.235706in}{2.378263in}}%
\pgfpathlineto{\pgfqpoint{5.236282in}{2.373697in}}%
\pgfpathlineto{\pgfqpoint{5.236667in}{2.377710in}}%
\pgfpathlineto{\pgfqpoint{5.236859in}{2.378303in}}%
\pgfpathlineto{\pgfqpoint{5.237243in}{2.376908in}}%
\pgfpathlineto{\pgfqpoint{5.237820in}{2.372097in}}%
\pgfpathlineto{\pgfqpoint{5.238204in}{2.376845in}}%
\pgfpathlineto{\pgfqpoint{5.239165in}{2.382074in}}%
\pgfpathlineto{\pgfqpoint{5.239357in}{2.380487in}}%
\pgfpathlineto{\pgfqpoint{5.239934in}{2.371413in}}%
\pgfpathlineto{\pgfqpoint{5.240510in}{2.376142in}}%
\pgfpathlineto{\pgfqpoint{5.241279in}{2.383134in}}%
\pgfpathlineto{\pgfqpoint{5.241855in}{2.379621in}}%
\pgfpathlineto{\pgfqpoint{5.242048in}{2.378794in}}%
\pgfpathlineto{\pgfqpoint{5.242240in}{2.381206in}}%
\pgfpathlineto{\pgfqpoint{5.242816in}{2.387654in}}%
\pgfpathlineto{\pgfqpoint{5.243201in}{2.384315in}}%
\pgfpathlineto{\pgfqpoint{5.244546in}{2.376699in}}%
\pgfpathlineto{\pgfqpoint{5.244738in}{2.378052in}}%
\pgfpathlineto{\pgfqpoint{5.245122in}{2.376143in}}%
\pgfpathlineto{\pgfqpoint{5.245507in}{2.376185in}}%
\pgfpathlineto{\pgfqpoint{5.246083in}{2.372843in}}%
\pgfpathlineto{\pgfqpoint{5.246660in}{2.375245in}}%
\pgfpathlineto{\pgfqpoint{5.249542in}{2.357627in}}%
\pgfpathlineto{\pgfqpoint{5.249926in}{2.357185in}}%
\pgfpathlineto{\pgfqpoint{5.250119in}{2.359655in}}%
\pgfpathlineto{\pgfqpoint{5.250695in}{2.357112in}}%
\pgfpathlineto{\pgfqpoint{5.250887in}{2.358178in}}%
\pgfpathlineto{\pgfqpoint{5.251079in}{2.356879in}}%
\pgfpathlineto{\pgfqpoint{5.251464in}{2.357809in}}%
\pgfpathlineto{\pgfqpoint{5.251656in}{2.362658in}}%
\pgfpathlineto{\pgfqpoint{5.252617in}{2.360957in}}%
\pgfpathlineto{\pgfqpoint{5.253193in}{2.361076in}}%
\pgfpathlineto{\pgfqpoint{5.253385in}{2.362962in}}%
\pgfpathlineto{\pgfqpoint{5.254154in}{2.360639in}}%
\pgfpathlineto{\pgfqpoint{5.254923in}{2.355206in}}%
\pgfpathlineto{\pgfqpoint{5.255307in}{2.359899in}}%
\pgfpathlineto{\pgfqpoint{5.256460in}{2.366894in}}%
\pgfpathlineto{\pgfqpoint{5.256652in}{2.362943in}}%
\pgfpathlineto{\pgfqpoint{5.256844in}{2.360930in}}%
\pgfpathlineto{\pgfqpoint{5.257229in}{2.365933in}}%
\pgfpathlineto{\pgfqpoint{5.257421in}{2.365653in}}%
\pgfpathlineto{\pgfqpoint{5.257805in}{2.363900in}}%
\pgfpathlineto{\pgfqpoint{5.258190in}{2.366844in}}%
\pgfpathlineto{\pgfqpoint{5.258382in}{2.368283in}}%
\pgfpathlineto{\pgfqpoint{5.258574in}{2.362911in}}%
\pgfpathlineto{\pgfqpoint{5.258766in}{2.364426in}}%
\pgfpathlineto{\pgfqpoint{5.261841in}{2.349597in}}%
\pgfpathlineto{\pgfqpoint{5.263378in}{2.358037in}}%
\pgfpathlineto{\pgfqpoint{5.264147in}{2.359715in}}%
\pgfpathlineto{\pgfqpoint{5.264339in}{2.356311in}}%
\pgfpathlineto{\pgfqpoint{5.264916in}{2.362362in}}%
\pgfpathlineto{\pgfqpoint{5.265108in}{2.361905in}}%
\pgfpathlineto{\pgfqpoint{5.265300in}{2.361823in}}%
\pgfpathlineto{\pgfqpoint{5.266453in}{2.369238in}}%
\pgfpathlineto{\pgfqpoint{5.266837in}{2.366501in}}%
\pgfpathlineto{\pgfqpoint{5.267222in}{2.365796in}}%
\pgfpathlineto{\pgfqpoint{5.267414in}{2.364116in}}%
\pgfpathlineto{\pgfqpoint{5.267990in}{2.368460in}}%
\pgfpathlineto{\pgfqpoint{5.269528in}{2.378433in}}%
\pgfpathlineto{\pgfqpoint{5.270104in}{2.376422in}}%
\pgfpathlineto{\pgfqpoint{5.270296in}{2.379144in}}%
\pgfpathlineto{\pgfqpoint{5.270681in}{2.377617in}}%
\pgfpathlineto{\pgfqpoint{5.270873in}{2.380498in}}%
\pgfpathlineto{\pgfqpoint{5.271449in}{2.374735in}}%
\pgfpathlineto{\pgfqpoint{5.271641in}{2.370656in}}%
\pgfpathlineto{\pgfqpoint{5.272410in}{2.377570in}}%
\pgfpathlineto{\pgfqpoint{5.272602in}{2.377412in}}%
\pgfpathlineto{\pgfqpoint{5.273179in}{2.373939in}}%
\pgfpathlineto{\pgfqpoint{5.273755in}{2.375159in}}%
\pgfpathlineto{\pgfqpoint{5.273947in}{2.374955in}}%
\pgfpathlineto{\pgfqpoint{5.274140in}{2.376366in}}%
\pgfpathlineto{\pgfqpoint{5.274524in}{2.377338in}}%
\pgfpathlineto{\pgfqpoint{5.274908in}{2.375240in}}%
\pgfpathlineto{\pgfqpoint{5.276253in}{2.387238in}}%
\pgfpathlineto{\pgfqpoint{5.276446in}{2.386064in}}%
\pgfpathlineto{\pgfqpoint{5.277599in}{2.374584in}}%
\pgfpathlineto{\pgfqpoint{5.278175in}{2.376824in}}%
\pgfpathlineto{\pgfqpoint{5.278367in}{2.378766in}}%
\pgfpathlineto{\pgfqpoint{5.278944in}{2.374341in}}%
\pgfpathlineto{\pgfqpoint{5.279136in}{2.371959in}}%
\pgfpathlineto{\pgfqpoint{5.279712in}{2.376756in}}%
\pgfpathlineto{\pgfqpoint{5.280481in}{2.382291in}}%
\pgfpathlineto{\pgfqpoint{5.282403in}{2.390169in}}%
\pgfpathlineto{\pgfqpoint{5.283940in}{2.381313in}}%
\pgfpathlineto{\pgfqpoint{5.284132in}{2.383183in}}%
\pgfpathlineto{\pgfqpoint{5.284324in}{2.382225in}}%
\pgfpathlineto{\pgfqpoint{5.284901in}{2.384611in}}%
\pgfpathlineto{\pgfqpoint{5.285477in}{2.383445in}}%
\pgfpathlineto{\pgfqpoint{5.285862in}{2.389246in}}%
\pgfpathlineto{\pgfqpoint{5.288168in}{2.405089in}}%
\pgfpathlineto{\pgfqpoint{5.288937in}{2.401979in}}%
\pgfpathlineto{\pgfqpoint{5.289129in}{2.399783in}}%
\pgfpathlineto{\pgfqpoint{5.289513in}{2.404893in}}%
\pgfpathlineto{\pgfqpoint{5.289897in}{2.401879in}}%
\pgfpathlineto{\pgfqpoint{5.290474in}{2.407407in}}%
\pgfpathlineto{\pgfqpoint{5.291050in}{2.404022in}}%
\pgfpathlineto{\pgfqpoint{5.291243in}{2.403283in}}%
\pgfpathlineto{\pgfqpoint{5.291627in}{2.405748in}}%
\pgfpathlineto{\pgfqpoint{5.291819in}{2.405279in}}%
\pgfpathlineto{\pgfqpoint{5.292396in}{2.406546in}}%
\pgfpathlineto{\pgfqpoint{5.292588in}{2.405985in}}%
\pgfpathlineto{\pgfqpoint{5.294317in}{2.400513in}}%
\pgfpathlineto{\pgfqpoint{5.296623in}{2.417645in}}%
\pgfpathlineto{\pgfqpoint{5.297008in}{2.414851in}}%
\pgfpathlineto{\pgfqpoint{5.298161in}{2.411223in}}%
\pgfpathlineto{\pgfqpoint{5.298545in}{2.413727in}}%
\pgfpathlineto{\pgfqpoint{5.299506in}{2.420698in}}%
\pgfpathlineto{\pgfqpoint{5.300851in}{2.429579in}}%
\pgfpathlineto{\pgfqpoint{5.301235in}{2.429297in}}%
\pgfpathlineto{\pgfqpoint{5.302388in}{2.426831in}}%
\pgfpathlineto{\pgfqpoint{5.302580in}{2.429372in}}%
\pgfpathlineto{\pgfqpoint{5.303157in}{2.425637in}}%
\pgfpathlineto{\pgfqpoint{5.303349in}{2.425656in}}%
\pgfpathlineto{\pgfqpoint{5.303733in}{2.421689in}}%
\pgfpathlineto{\pgfqpoint{5.303926in}{2.422869in}}%
\pgfusepath{stroke}%
\end{pgfscope}%
\begin{pgfscope}%
\pgfpathrectangle{\pgfqpoint{3.286364in}{0.660000in}}{\pgfqpoint{2.113636in}{2.100000in}}%
\pgfusepath{clip}%
\pgfsetroundcap%
\pgfsetroundjoin%
\pgfsetlinewidth{0.602250pt}%
\definecolor{currentstroke}{rgb}{0.870588,0.870588,0.000000}%
\pgfsetstrokecolor{currentstroke}%
\pgfsetdash{}{0pt}%
\pgfpathmoveto{\pgfqpoint{3.382438in}{1.900242in}}%
\pgfpathlineto{\pgfqpoint{3.382822in}{1.894768in}}%
\pgfpathlineto{\pgfqpoint{3.383591in}{1.899195in}}%
\pgfpathlineto{\pgfqpoint{3.384168in}{1.900686in}}%
\pgfpathlineto{\pgfqpoint{3.385321in}{1.903920in}}%
\pgfpathlineto{\pgfqpoint{3.385513in}{1.902297in}}%
\pgfpathlineto{\pgfqpoint{3.386089in}{1.906533in}}%
\pgfpathlineto{\pgfqpoint{3.386281in}{1.904170in}}%
\pgfpathlineto{\pgfqpoint{3.386474in}{1.905594in}}%
\pgfpathlineto{\pgfqpoint{3.386858in}{1.900178in}}%
\pgfpathlineto{\pgfqpoint{3.387050in}{1.900210in}}%
\pgfpathlineto{\pgfqpoint{3.387242in}{1.897909in}}%
\pgfpathlineto{\pgfqpoint{3.387819in}{1.901373in}}%
\pgfpathlineto{\pgfqpoint{3.388587in}{1.905020in}}%
\pgfpathlineto{\pgfqpoint{3.388203in}{1.901154in}}%
\pgfpathlineto{\pgfqpoint{3.388972in}{1.902472in}}%
\pgfpathlineto{\pgfqpoint{3.389548in}{1.898850in}}%
\pgfpathlineto{\pgfqpoint{3.390125in}{1.900992in}}%
\pgfpathlineto{\pgfqpoint{3.391086in}{1.897832in}}%
\pgfpathlineto{\pgfqpoint{3.391278in}{1.898229in}}%
\pgfpathlineto{\pgfqpoint{3.393584in}{1.912127in}}%
\pgfpathlineto{\pgfqpoint{3.393776in}{1.910338in}}%
\pgfpathlineto{\pgfqpoint{3.394160in}{1.913814in}}%
\pgfpathlineto{\pgfqpoint{3.394545in}{1.912918in}}%
\pgfpathlineto{\pgfqpoint{3.394737in}{1.913422in}}%
\pgfpathlineto{\pgfqpoint{3.395121in}{1.912175in}}%
\pgfpathlineto{\pgfqpoint{3.395890in}{1.912488in}}%
\pgfpathlineto{\pgfqpoint{3.396658in}{1.909792in}}%
\pgfpathlineto{\pgfqpoint{3.398196in}{1.917013in}}%
\pgfpathlineto{\pgfqpoint{3.398580in}{1.915089in}}%
\pgfpathlineto{\pgfqpoint{3.398964in}{1.919373in}}%
\pgfpathlineto{\pgfqpoint{3.399157in}{1.919595in}}%
\pgfpathlineto{\pgfqpoint{3.400886in}{1.913438in}}%
\pgfpathlineto{\pgfqpoint{3.402231in}{1.910047in}}%
\pgfpathlineto{\pgfqpoint{3.405114in}{1.919288in}}%
\pgfpathlineto{\pgfqpoint{3.405306in}{1.917152in}}%
\pgfpathlineto{\pgfqpoint{3.405498in}{1.917531in}}%
\pgfpathlineto{\pgfqpoint{3.405690in}{1.917223in}}%
\pgfpathlineto{\pgfqpoint{3.405883in}{1.921309in}}%
\pgfpathlineto{\pgfqpoint{3.406843in}{1.920441in}}%
\pgfpathlineto{\pgfqpoint{3.407228in}{1.916705in}}%
\pgfpathlineto{\pgfqpoint{3.407804in}{1.919457in}}%
\pgfpathlineto{\pgfqpoint{3.409342in}{1.924936in}}%
\pgfpathlineto{\pgfqpoint{3.409534in}{1.923613in}}%
\pgfpathlineto{\pgfqpoint{3.409726in}{1.922908in}}%
\pgfpathlineto{\pgfqpoint{3.409918in}{1.924907in}}%
\pgfpathlineto{\pgfqpoint{3.411071in}{1.935102in}}%
\pgfpathlineto{\pgfqpoint{3.411648in}{1.931084in}}%
\pgfpathlineto{\pgfqpoint{3.411840in}{1.930911in}}%
\pgfpathlineto{\pgfqpoint{3.412224in}{1.932120in}}%
\pgfpathlineto{\pgfqpoint{3.412993in}{1.928938in}}%
\pgfpathlineto{\pgfqpoint{3.414146in}{1.935518in}}%
\pgfpathlineto{\pgfqpoint{3.414722in}{1.934480in}}%
\pgfpathlineto{\pgfqpoint{3.415683in}{1.930685in}}%
\pgfpathlineto{\pgfqpoint{3.415875in}{1.931137in}}%
\pgfpathlineto{\pgfqpoint{3.417605in}{1.941240in}}%
\pgfpathlineto{\pgfqpoint{3.417797in}{1.940724in}}%
\pgfpathlineto{\pgfqpoint{3.420295in}{1.930317in}}%
\pgfpathlineto{\pgfqpoint{3.420487in}{1.932855in}}%
\pgfpathlineto{\pgfqpoint{3.421256in}{1.938449in}}%
\pgfpathlineto{\pgfqpoint{3.421448in}{1.933077in}}%
\pgfpathlineto{\pgfqpoint{3.422601in}{1.930139in}}%
\pgfpathlineto{\pgfqpoint{3.423178in}{1.931840in}}%
\pgfpathlineto{\pgfqpoint{3.424138in}{1.939577in}}%
\pgfpathlineto{\pgfqpoint{3.425291in}{1.936500in}}%
\pgfpathlineto{\pgfqpoint{3.425484in}{1.936236in}}%
\pgfpathlineto{\pgfqpoint{3.426637in}{1.941734in}}%
\pgfpathlineto{\pgfqpoint{3.425868in}{1.935262in}}%
\pgfpathlineto{\pgfqpoint{3.426829in}{1.940879in}}%
\pgfpathlineto{\pgfqpoint{3.427021in}{1.939422in}}%
\pgfpathlineto{\pgfqpoint{3.427405in}{1.942959in}}%
\pgfpathlineto{\pgfqpoint{3.427790in}{1.941547in}}%
\pgfpathlineto{\pgfqpoint{3.428174in}{1.943427in}}%
\pgfpathlineto{\pgfqpoint{3.428366in}{1.940747in}}%
\pgfpathlineto{\pgfqpoint{3.428558in}{1.941545in}}%
\pgfpathlineto{\pgfqpoint{3.429327in}{1.935966in}}%
\pgfpathlineto{\pgfqpoint{3.429711in}{1.936743in}}%
\pgfpathlineto{\pgfqpoint{3.430288in}{1.941629in}}%
\pgfpathlineto{\pgfqpoint{3.430864in}{1.940432in}}%
\pgfpathlineto{\pgfqpoint{3.431057in}{1.940278in}}%
\pgfpathlineto{\pgfqpoint{3.431825in}{1.942962in}}%
\pgfpathlineto{\pgfqpoint{3.432017in}{1.942277in}}%
\pgfpathlineto{\pgfqpoint{3.434131in}{1.930378in}}%
\pgfpathlineto{\pgfqpoint{3.436053in}{1.939859in}}%
\pgfpathlineto{\pgfqpoint{3.436437in}{1.934280in}}%
\pgfpathlineto{\pgfqpoint{3.437014in}{1.938920in}}%
\pgfpathlineto{\pgfqpoint{3.438359in}{1.945569in}}%
\pgfpathlineto{\pgfqpoint{3.439512in}{1.943764in}}%
\pgfpathlineto{\pgfqpoint{3.440088in}{1.942029in}}%
\pgfpathlineto{\pgfqpoint{3.440281in}{1.945183in}}%
\pgfpathlineto{\pgfqpoint{3.442202in}{1.953343in}}%
\pgfpathlineto{\pgfqpoint{3.443547in}{1.946673in}}%
\pgfpathlineto{\pgfqpoint{3.442587in}{1.953741in}}%
\pgfpathlineto{\pgfqpoint{3.443740in}{1.948178in}}%
\pgfpathlineto{\pgfqpoint{3.444508in}{1.953428in}}%
\pgfpathlineto{\pgfqpoint{3.444124in}{1.947976in}}%
\pgfpathlineto{\pgfqpoint{3.445085in}{1.951049in}}%
\pgfpathlineto{\pgfqpoint{3.446430in}{1.943612in}}%
\pgfpathlineto{\pgfqpoint{3.446814in}{1.944474in}}%
\pgfpathlineto{\pgfqpoint{3.447199in}{1.945005in}}%
\pgfpathlineto{\pgfqpoint{3.447391in}{1.943460in}}%
\pgfpathlineto{\pgfqpoint{3.447583in}{1.944752in}}%
\pgfpathlineto{\pgfqpoint{3.447775in}{1.943508in}}%
\pgfpathlineto{\pgfqpoint{3.448352in}{1.946564in}}%
\pgfpathlineto{\pgfqpoint{3.448928in}{1.950515in}}%
\pgfpathlineto{\pgfqpoint{3.449697in}{1.949561in}}%
\pgfpathlineto{\pgfqpoint{3.449889in}{1.948125in}}%
\pgfpathlineto{\pgfqpoint{3.450465in}{1.950696in}}%
\pgfpathlineto{\pgfqpoint{3.450850in}{1.949186in}}%
\pgfpathlineto{\pgfqpoint{3.451811in}{1.954371in}}%
\pgfpathlineto{\pgfqpoint{3.451426in}{1.948898in}}%
\pgfpathlineto{\pgfqpoint{3.452579in}{1.953042in}}%
\pgfpathlineto{\pgfqpoint{3.452771in}{1.953187in}}%
\pgfpathlineto{\pgfqpoint{3.452964in}{1.950187in}}%
\pgfpathlineto{\pgfqpoint{3.453540in}{1.955131in}}%
\pgfpathlineto{\pgfqpoint{3.453732in}{1.954528in}}%
\pgfpathlineto{\pgfqpoint{3.453925in}{1.954506in}}%
\pgfpathlineto{\pgfqpoint{3.454117in}{1.951957in}}%
\pgfpathlineto{\pgfqpoint{3.454309in}{1.956411in}}%
\pgfpathlineto{\pgfqpoint{3.454885in}{1.954417in}}%
\pgfpathlineto{\pgfqpoint{3.455270in}{1.952901in}}%
\pgfpathlineto{\pgfqpoint{3.455462in}{1.953611in}}%
\pgfpathlineto{\pgfqpoint{3.456423in}{1.956286in}}%
\pgfpathlineto{\pgfqpoint{3.456615in}{1.955036in}}%
\pgfpathlineto{\pgfqpoint{3.457191in}{1.956872in}}%
\pgfpathlineto{\pgfqpoint{3.458152in}{1.963768in}}%
\pgfpathlineto{\pgfqpoint{3.458537in}{1.958473in}}%
\pgfpathlineto{\pgfqpoint{3.458729in}{1.958726in}}%
\pgfpathlineto{\pgfqpoint{3.458921in}{1.957721in}}%
\pgfpathlineto{\pgfqpoint{3.459305in}{1.957670in}}%
\pgfpathlineto{\pgfqpoint{3.459882in}{1.954217in}}%
\pgfpathlineto{\pgfqpoint{3.460458in}{1.956174in}}%
\pgfpathlineto{\pgfqpoint{3.461419in}{1.960395in}}%
\pgfpathlineto{\pgfqpoint{3.461611in}{1.958479in}}%
\pgfpathlineto{\pgfqpoint{3.461803in}{1.958057in}}%
\pgfpathlineto{\pgfqpoint{3.463149in}{1.962841in}}%
\pgfpathlineto{\pgfqpoint{3.464494in}{1.967917in}}%
\pgfpathlineto{\pgfqpoint{3.464686in}{1.963738in}}%
\pgfpathlineto{\pgfqpoint{3.465070in}{1.963664in}}%
\pgfpathlineto{\pgfqpoint{3.465262in}{1.964626in}}%
\pgfpathlineto{\pgfqpoint{3.466223in}{1.972586in}}%
\pgfpathlineto{\pgfqpoint{3.467184in}{1.971607in}}%
\pgfpathlineto{\pgfqpoint{3.467761in}{1.972230in}}%
\pgfpathlineto{\pgfqpoint{3.468721in}{1.966373in}}%
\pgfpathlineto{\pgfqpoint{3.468914in}{1.965683in}}%
\pgfpathlineto{\pgfqpoint{3.469298in}{1.967821in}}%
\pgfpathlineto{\pgfqpoint{3.470835in}{1.974770in}}%
\pgfpathlineto{\pgfqpoint{3.472373in}{1.967116in}}%
\pgfpathlineto{\pgfqpoint{3.472565in}{1.967300in}}%
\pgfpathlineto{\pgfqpoint{3.473910in}{1.964349in}}%
\pgfpathlineto{\pgfqpoint{3.474486in}{1.969601in}}%
\pgfpathlineto{\pgfqpoint{3.474871in}{1.964907in}}%
\pgfpathlineto{\pgfqpoint{3.475255in}{1.959834in}}%
\pgfpathlineto{\pgfqpoint{3.475832in}{1.963839in}}%
\pgfpathlineto{\pgfqpoint{3.476216in}{1.966057in}}%
\pgfpathlineto{\pgfqpoint{3.476600in}{1.962889in}}%
\pgfpathlineto{\pgfqpoint{3.476985in}{1.965051in}}%
\pgfpathlineto{\pgfqpoint{3.477177in}{1.962782in}}%
\pgfpathlineto{\pgfqpoint{3.477561in}{1.965803in}}%
\pgfpathlineto{\pgfqpoint{3.478138in}{1.970909in}}%
\pgfpathlineto{\pgfqpoint{3.478522in}{1.966686in}}%
\pgfpathlineto{\pgfqpoint{3.478714in}{1.965400in}}%
\pgfpathlineto{\pgfqpoint{3.478906in}{1.967613in}}%
\pgfpathlineto{\pgfqpoint{3.479483in}{1.972842in}}%
\pgfpathlineto{\pgfqpoint{3.479867in}{1.968455in}}%
\pgfpathlineto{\pgfqpoint{3.481405in}{1.961260in}}%
\pgfpathlineto{\pgfqpoint{3.483134in}{1.971649in}}%
\pgfpathlineto{\pgfqpoint{3.483326in}{1.970918in}}%
\pgfpathlineto{\pgfqpoint{3.484095in}{1.975900in}}%
\pgfpathlineto{\pgfqpoint{3.485632in}{1.987461in}}%
\pgfpathlineto{\pgfqpoint{3.485824in}{1.986825in}}%
\pgfpathlineto{\pgfqpoint{3.487362in}{1.997372in}}%
\pgfpathlineto{\pgfqpoint{3.487554in}{1.996266in}}%
\pgfpathlineto{\pgfqpoint{3.487746in}{2.000042in}}%
\pgfpathlineto{\pgfqpoint{3.488323in}{1.997631in}}%
\pgfpathlineto{\pgfqpoint{3.488899in}{2.004481in}}%
\pgfpathlineto{\pgfqpoint{3.490244in}{1.997820in}}%
\pgfpathlineto{\pgfqpoint{3.490629in}{1.997255in}}%
\pgfpathlineto{\pgfqpoint{3.491013in}{1.998574in}}%
\pgfpathlineto{\pgfqpoint{3.491205in}{2.001832in}}%
\pgfpathlineto{\pgfqpoint{3.491589in}{1.998220in}}%
\pgfpathlineto{\pgfqpoint{3.492166in}{2.000235in}}%
\pgfpathlineto{\pgfqpoint{3.492358in}{1.999489in}}%
\pgfpathlineto{\pgfqpoint{3.493895in}{2.012168in}}%
\pgfpathlineto{\pgfqpoint{3.494088in}{2.013264in}}%
\pgfpathlineto{\pgfqpoint{3.494664in}{2.010177in}}%
\pgfpathlineto{\pgfqpoint{3.494856in}{2.009437in}}%
\pgfpathlineto{\pgfqpoint{3.495048in}{2.010328in}}%
\pgfpathlineto{\pgfqpoint{3.496586in}{2.015772in}}%
\pgfpathlineto{\pgfqpoint{3.496970in}{2.015576in}}%
\pgfpathlineto{\pgfqpoint{3.497354in}{2.020528in}}%
\pgfpathlineto{\pgfqpoint{3.497739in}{2.022806in}}%
\pgfpathlineto{\pgfqpoint{3.498315in}{2.018947in}}%
\pgfpathlineto{\pgfqpoint{3.498700in}{2.016902in}}%
\pgfpathlineto{\pgfqpoint{3.499276in}{2.019921in}}%
\pgfpathlineto{\pgfqpoint{3.499660in}{2.018950in}}%
\pgfpathlineto{\pgfqpoint{3.500045in}{2.020712in}}%
\pgfpathlineto{\pgfqpoint{3.500237in}{2.020298in}}%
\pgfpathlineto{\pgfqpoint{3.500813in}{2.022115in}}%
\pgfpathlineto{\pgfqpoint{3.501006in}{2.020763in}}%
\pgfpathlineto{\pgfqpoint{3.501966in}{2.012403in}}%
\pgfpathlineto{\pgfqpoint{3.502351in}{2.013949in}}%
\pgfpathlineto{\pgfqpoint{3.502543in}{2.014526in}}%
\pgfpathlineto{\pgfqpoint{3.502735in}{2.012231in}}%
\pgfpathlineto{\pgfqpoint{3.502927in}{2.011618in}}%
\pgfpathlineto{\pgfqpoint{3.504273in}{2.001004in}}%
\pgfpathlineto{\pgfqpoint{3.504465in}{2.002698in}}%
\pgfpathlineto{\pgfqpoint{3.504849in}{1.998418in}}%
\pgfpathlineto{\pgfqpoint{3.505041in}{1.999691in}}%
\pgfpathlineto{\pgfqpoint{3.505233in}{1.997774in}}%
\pgfpathlineto{\pgfqpoint{3.505618in}{2.003290in}}%
\pgfpathlineto{\pgfqpoint{3.506963in}{2.010217in}}%
\pgfpathlineto{\pgfqpoint{3.507155in}{2.008109in}}%
\pgfpathlineto{\pgfqpoint{3.507732in}{2.008678in}}%
\pgfpathlineto{\pgfqpoint{3.507924in}{2.007609in}}%
\pgfpathlineto{\pgfqpoint{3.508116in}{2.007629in}}%
\pgfpathlineto{\pgfqpoint{3.508308in}{2.007593in}}%
\pgfpathlineto{\pgfqpoint{3.509461in}{2.003770in}}%
\pgfpathlineto{\pgfqpoint{3.509653in}{2.005269in}}%
\pgfpathlineto{\pgfqpoint{3.511959in}{2.020089in}}%
\pgfpathlineto{\pgfqpoint{3.512920in}{2.018106in}}%
\pgfpathlineto{\pgfqpoint{3.513881in}{2.012655in}}%
\pgfpathlineto{\pgfqpoint{3.514265in}{2.013043in}}%
\pgfpathlineto{\pgfqpoint{3.515995in}{2.021553in}}%
\pgfpathlineto{\pgfqpoint{3.517532in}{2.012996in}}%
\pgfpathlineto{\pgfqpoint{3.517724in}{2.013891in}}%
\pgfpathlineto{\pgfqpoint{3.517916in}{2.017156in}}%
\pgfpathlineto{\pgfqpoint{3.518685in}{2.012489in}}%
\pgfpathlineto{\pgfqpoint{3.519069in}{2.010793in}}%
\pgfpathlineto{\pgfqpoint{3.519262in}{2.014822in}}%
\pgfpathlineto{\pgfqpoint{3.519454in}{2.014104in}}%
\pgfpathlineto{\pgfqpoint{3.520030in}{2.013559in}}%
\pgfpathlineto{\pgfqpoint{3.522528in}{2.028654in}}%
\pgfpathlineto{\pgfqpoint{3.523681in}{2.023781in}}%
\pgfpathlineto{\pgfqpoint{3.522913in}{2.030797in}}%
\pgfpathlineto{\pgfqpoint{3.524258in}{2.024517in}}%
\pgfpathlineto{\pgfqpoint{3.525987in}{2.035705in}}%
\pgfpathlineto{\pgfqpoint{3.526180in}{2.033957in}}%
\pgfpathlineto{\pgfqpoint{3.530407in}{2.055677in}}%
\pgfpathlineto{\pgfqpoint{3.530600in}{2.054816in}}%
\pgfpathlineto{\pgfqpoint{3.531368in}{2.056183in}}%
\pgfpathlineto{\pgfqpoint{3.532521in}{2.049342in}}%
\pgfpathlineto{\pgfqpoint{3.533866in}{2.054965in}}%
\pgfpathlineto{\pgfqpoint{3.534251in}{2.052191in}}%
\pgfpathlineto{\pgfqpoint{3.535212in}{2.045507in}}%
\pgfpathlineto{\pgfqpoint{3.535596in}{2.046108in}}%
\pgfpathlineto{\pgfqpoint{3.536365in}{2.051161in}}%
\pgfpathlineto{\pgfqpoint{3.536749in}{2.048902in}}%
\pgfpathlineto{\pgfqpoint{3.536941in}{2.049208in}}%
\pgfpathlineto{\pgfqpoint{3.537325in}{2.048217in}}%
\pgfpathlineto{\pgfqpoint{3.538094in}{2.044164in}}%
\pgfpathlineto{\pgfqpoint{3.538478in}{2.045601in}}%
\pgfpathlineto{\pgfqpoint{3.539631in}{2.053115in}}%
\pgfpathlineto{\pgfqpoint{3.539824in}{2.052743in}}%
\pgfpathlineto{\pgfqpoint{3.540208in}{2.048999in}}%
\pgfpathlineto{\pgfqpoint{3.542130in}{2.041128in}}%
\pgfpathlineto{\pgfqpoint{3.542322in}{2.041989in}}%
\pgfpathlineto{\pgfqpoint{3.542514in}{2.040301in}}%
\pgfpathlineto{\pgfqpoint{3.543090in}{2.037097in}}%
\pgfpathlineto{\pgfqpoint{3.543475in}{2.039871in}}%
\pgfpathlineto{\pgfqpoint{3.543667in}{2.040210in}}%
\pgfpathlineto{\pgfqpoint{3.544628in}{2.033881in}}%
\pgfpathlineto{\pgfqpoint{3.545012in}{2.035079in}}%
\pgfpathlineto{\pgfqpoint{3.545204in}{2.034685in}}%
\pgfpathlineto{\pgfqpoint{3.545973in}{2.041501in}}%
\pgfpathlineto{\pgfqpoint{3.546549in}{2.040829in}}%
\pgfpathlineto{\pgfqpoint{3.547126in}{2.041698in}}%
\pgfpathlineto{\pgfqpoint{3.547895in}{2.037510in}}%
\pgfpathlineto{\pgfqpoint{3.548855in}{2.043284in}}%
\pgfpathlineto{\pgfqpoint{3.549240in}{2.042302in}}%
\pgfpathlineto{\pgfqpoint{3.549432in}{2.040955in}}%
\pgfpathlineto{\pgfqpoint{3.549816in}{2.045606in}}%
\pgfpathlineto{\pgfqpoint{3.550008in}{2.044949in}}%
\pgfpathlineto{\pgfqpoint{3.551354in}{2.051607in}}%
\pgfpathlineto{\pgfqpoint{3.551738in}{2.052560in}}%
\pgfpathlineto{\pgfqpoint{3.551930in}{2.050413in}}%
\pgfpathlineto{\pgfqpoint{3.552315in}{2.052053in}}%
\pgfpathlineto{\pgfqpoint{3.553275in}{2.047423in}}%
\pgfpathlineto{\pgfqpoint{3.553468in}{2.049571in}}%
\pgfpathlineto{\pgfqpoint{3.553852in}{2.050437in}}%
\pgfpathlineto{\pgfqpoint{3.554044in}{2.048142in}}%
\pgfpathlineto{\pgfqpoint{3.554621in}{2.053601in}}%
\pgfpathlineto{\pgfqpoint{3.555197in}{2.055999in}}%
\pgfpathlineto{\pgfqpoint{3.556158in}{2.059889in}}%
\pgfpathlineto{\pgfqpoint{3.556350in}{2.055546in}}%
\pgfpathlineto{\pgfqpoint{3.557119in}{2.057818in}}%
\pgfpathlineto{\pgfqpoint{3.557887in}{2.064633in}}%
\pgfpathlineto{\pgfqpoint{3.558464in}{2.060717in}}%
\pgfpathlineto{\pgfqpoint{3.559040in}{2.060938in}}%
\pgfpathlineto{\pgfqpoint{3.560386in}{2.067873in}}%
\pgfpathlineto{\pgfqpoint{3.560578in}{2.068401in}}%
\pgfpathlineto{\pgfqpoint{3.561346in}{2.062074in}}%
\pgfpathlineto{\pgfqpoint{3.561731in}{2.063278in}}%
\pgfpathlineto{\pgfqpoint{3.562115in}{2.071031in}}%
\pgfpathlineto{\pgfqpoint{3.562884in}{2.068039in}}%
\pgfpathlineto{\pgfqpoint{3.564037in}{2.064458in}}%
\pgfpathlineto{\pgfqpoint{3.563652in}{2.069270in}}%
\pgfpathlineto{\pgfqpoint{3.564229in}{2.066863in}}%
\pgfpathlineto{\pgfqpoint{3.565190in}{2.070017in}}%
\pgfpathlineto{\pgfqpoint{3.565382in}{2.069663in}}%
\pgfpathlineto{\pgfqpoint{3.565574in}{2.066340in}}%
\pgfpathlineto{\pgfqpoint{3.566535in}{2.067612in}}%
\pgfpathlineto{\pgfqpoint{3.566727in}{2.067685in}}%
\pgfpathlineto{\pgfqpoint{3.567496in}{2.070949in}}%
\pgfpathlineto{\pgfqpoint{3.567880in}{2.068800in}}%
\pgfpathlineto{\pgfqpoint{3.568072in}{2.068494in}}%
\pgfpathlineto{\pgfqpoint{3.568264in}{2.069930in}}%
\pgfpathlineto{\pgfqpoint{3.569802in}{2.074965in}}%
\pgfpathlineto{\pgfqpoint{3.569994in}{2.074562in}}%
\pgfpathlineto{\pgfqpoint{3.570186in}{2.075407in}}%
\pgfpathlineto{\pgfqpoint{3.570378in}{2.074869in}}%
\pgfpathlineto{\pgfqpoint{3.571723in}{2.086876in}}%
\pgfpathlineto{\pgfqpoint{3.571916in}{2.086020in}}%
\pgfpathlineto{\pgfqpoint{3.572684in}{2.084652in}}%
\pgfpathlineto{\pgfqpoint{3.572876in}{2.087443in}}%
\pgfpathlineto{\pgfqpoint{3.573069in}{2.086139in}}%
\pgfpathlineto{\pgfqpoint{3.573261in}{2.087643in}}%
\pgfpathlineto{\pgfqpoint{3.573837in}{2.097067in}}%
\pgfpathlineto{\pgfqpoint{3.574414in}{2.089312in}}%
\pgfpathlineto{\pgfqpoint{3.575567in}{2.092959in}}%
\pgfpathlineto{\pgfqpoint{3.575951in}{2.089437in}}%
\pgfpathlineto{\pgfqpoint{3.576528in}{2.092168in}}%
\pgfpathlineto{\pgfqpoint{3.577681in}{2.100052in}}%
\pgfpathlineto{\pgfqpoint{3.577873in}{2.095752in}}%
\pgfpathlineto{\pgfqpoint{3.578449in}{2.090219in}}%
\pgfpathlineto{\pgfqpoint{3.578834in}{2.093315in}}%
\pgfpathlineto{\pgfqpoint{3.580371in}{2.101970in}}%
\pgfpathlineto{\pgfqpoint{3.581716in}{2.094302in}}%
\pgfpathlineto{\pgfqpoint{3.581908in}{2.094900in}}%
\pgfpathlineto{\pgfqpoint{3.582101in}{2.094954in}}%
\pgfpathlineto{\pgfqpoint{3.583446in}{2.087925in}}%
\pgfpathlineto{\pgfqpoint{3.583830in}{2.088686in}}%
\pgfpathlineto{\pgfqpoint{3.584022in}{2.090235in}}%
\pgfpathlineto{\pgfqpoint{3.584407in}{2.086538in}}%
\pgfpathlineto{\pgfqpoint{3.584983in}{2.079741in}}%
\pgfpathlineto{\pgfqpoint{3.585560in}{2.083646in}}%
\pgfpathlineto{\pgfqpoint{3.586713in}{2.089228in}}%
\pgfpathlineto{\pgfqpoint{3.586905in}{2.087926in}}%
\pgfpathlineto{\pgfqpoint{3.588058in}{2.086047in}}%
\pgfpathlineto{\pgfqpoint{3.587481in}{2.089113in}}%
\pgfpathlineto{\pgfqpoint{3.588250in}{2.086253in}}%
\pgfpathlineto{\pgfqpoint{3.589403in}{2.092903in}}%
\pgfpathlineto{\pgfqpoint{3.589787in}{2.090381in}}%
\pgfpathlineto{\pgfqpoint{3.590172in}{2.091412in}}%
\pgfpathlineto{\pgfqpoint{3.590364in}{2.090090in}}%
\pgfpathlineto{\pgfqpoint{3.590748in}{2.087347in}}%
\pgfpathlineto{\pgfqpoint{3.591132in}{2.091259in}}%
\pgfpathlineto{\pgfqpoint{3.591325in}{2.092539in}}%
\pgfpathlineto{\pgfqpoint{3.591901in}{2.089023in}}%
\pgfpathlineto{\pgfqpoint{3.594591in}{2.071030in}}%
\pgfpathlineto{\pgfqpoint{3.594976in}{2.071674in}}%
\pgfpathlineto{\pgfqpoint{3.596321in}{2.078049in}}%
\pgfpathlineto{\pgfqpoint{3.597090in}{2.085759in}}%
\pgfpathlineto{\pgfqpoint{3.597474in}{2.083977in}}%
\pgfpathlineto{\pgfqpoint{3.598435in}{2.075521in}}%
\pgfpathlineto{\pgfqpoint{3.598819in}{2.077714in}}%
\pgfpathlineto{\pgfqpoint{3.600164in}{2.074889in}}%
\pgfpathlineto{\pgfqpoint{3.599203in}{2.078092in}}%
\pgfpathlineto{\pgfqpoint{3.600357in}{2.075811in}}%
\pgfpathlineto{\pgfqpoint{3.601125in}{2.079475in}}%
\pgfpathlineto{\pgfqpoint{3.601510in}{2.078010in}}%
\pgfpathlineto{\pgfqpoint{3.602086in}{2.078138in}}%
\pgfpathlineto{\pgfqpoint{3.602278in}{2.077145in}}%
\pgfpathlineto{\pgfqpoint{3.602855in}{2.075240in}}%
\pgfpathlineto{\pgfqpoint{3.603431in}{2.077146in}}%
\pgfpathlineto{\pgfqpoint{3.603623in}{2.077134in}}%
\pgfpathlineto{\pgfqpoint{3.604584in}{2.080937in}}%
\pgfpathlineto{\pgfqpoint{3.604392in}{2.076288in}}%
\pgfpathlineto{\pgfqpoint{3.604776in}{2.078762in}}%
\pgfpathlineto{\pgfqpoint{3.606314in}{2.084236in}}%
\pgfpathlineto{\pgfqpoint{3.607275in}{2.088867in}}%
\pgfpathlineto{\pgfqpoint{3.607467in}{2.088240in}}%
\pgfpathlineto{\pgfqpoint{3.609388in}{2.078150in}}%
\pgfpathlineto{\pgfqpoint{3.609581in}{2.079577in}}%
\pgfpathlineto{\pgfqpoint{3.609773in}{2.077579in}}%
\pgfpathlineto{\pgfqpoint{3.610157in}{2.078374in}}%
\pgfpathlineto{\pgfqpoint{3.610349in}{2.076735in}}%
\pgfpathlineto{\pgfqpoint{3.611118in}{2.078562in}}%
\pgfpathlineto{\pgfqpoint{3.613232in}{2.092724in}}%
\pgfpathlineto{\pgfqpoint{3.614385in}{2.082527in}}%
\pgfpathlineto{\pgfqpoint{3.614577in}{2.083528in}}%
\pgfpathlineto{\pgfqpoint{3.615346in}{2.081588in}}%
\pgfpathlineto{\pgfqpoint{3.615538in}{2.083099in}}%
\pgfpathlineto{\pgfqpoint{3.616883in}{2.088792in}}%
\pgfpathlineto{\pgfqpoint{3.617267in}{2.083334in}}%
\pgfpathlineto{\pgfqpoint{3.617652in}{2.088878in}}%
\pgfpathlineto{\pgfqpoint{3.617844in}{2.088048in}}%
\pgfpathlineto{\pgfqpoint{3.618036in}{2.089328in}}%
\pgfpathlineto{\pgfqpoint{3.618228in}{2.087829in}}%
\pgfpathlineto{\pgfqpoint{3.618612in}{2.087849in}}%
\pgfpathlineto{\pgfqpoint{3.620150in}{2.080784in}}%
\pgfpathlineto{\pgfqpoint{3.624570in}{2.104478in}}%
\pgfpathlineto{\pgfqpoint{3.620534in}{2.079686in}}%
\pgfpathlineto{\pgfqpoint{3.625915in}{2.102798in}}%
\pgfpathlineto{\pgfqpoint{3.626684in}{2.098306in}}%
\pgfpathlineto{\pgfqpoint{3.628029in}{2.115194in}}%
\pgfpathlineto{\pgfqpoint{3.628221in}{2.114712in}}%
\pgfpathlineto{\pgfqpoint{3.628413in}{2.114809in}}%
\pgfpathlineto{\pgfqpoint{3.629950in}{2.109627in}}%
\pgfpathlineto{\pgfqpoint{3.630335in}{2.111037in}}%
\pgfpathlineto{\pgfqpoint{3.630527in}{2.106952in}}%
\pgfpathlineto{\pgfqpoint{3.630719in}{2.109857in}}%
\pgfpathlineto{\pgfqpoint{3.632064in}{2.100202in}}%
\pgfpathlineto{\pgfqpoint{3.632449in}{2.102616in}}%
\pgfpathlineto{\pgfqpoint{3.634370in}{2.120738in}}%
\pgfpathlineto{\pgfqpoint{3.634755in}{2.119312in}}%
\pgfpathlineto{\pgfqpoint{3.634947in}{2.121381in}}%
\pgfpathlineto{\pgfqpoint{3.635139in}{2.121358in}}%
\pgfpathlineto{\pgfqpoint{3.636100in}{2.126774in}}%
\pgfpathlineto{\pgfqpoint{3.636292in}{2.123253in}}%
\pgfpathlineto{\pgfqpoint{3.637445in}{2.119661in}}%
\pgfpathlineto{\pgfqpoint{3.639751in}{2.144170in}}%
\pgfpathlineto{\pgfqpoint{3.640135in}{2.150976in}}%
\pgfpathlineto{\pgfqpoint{3.641096in}{2.148823in}}%
\pgfpathlineto{\pgfqpoint{3.641288in}{2.149141in}}%
\pgfpathlineto{\pgfqpoint{3.641480in}{2.147255in}}%
\pgfpathlineto{\pgfqpoint{3.642057in}{2.152143in}}%
\pgfpathlineto{\pgfqpoint{3.642249in}{2.147992in}}%
\pgfpathlineto{\pgfqpoint{3.643018in}{2.149673in}}%
\pgfpathlineto{\pgfqpoint{3.642633in}{2.146509in}}%
\pgfpathlineto{\pgfqpoint{3.643210in}{2.147732in}}%
\pgfpathlineto{\pgfqpoint{3.643402in}{2.147910in}}%
\pgfpathlineto{\pgfqpoint{3.643594in}{2.149761in}}%
\pgfpathlineto{\pgfqpoint{3.644171in}{2.146095in}}%
\pgfpathlineto{\pgfqpoint{3.644363in}{2.147307in}}%
\pgfpathlineto{\pgfqpoint{3.645324in}{2.142801in}}%
\pgfpathlineto{\pgfqpoint{3.645516in}{2.145761in}}%
\pgfpathlineto{\pgfqpoint{3.646861in}{2.149794in}}%
\pgfpathlineto{\pgfqpoint{3.647053in}{2.150246in}}%
\pgfpathlineto{\pgfqpoint{3.647245in}{2.149341in}}%
\pgfpathlineto{\pgfqpoint{3.647438in}{2.149517in}}%
\pgfpathlineto{\pgfqpoint{3.648783in}{2.143353in}}%
\pgfpathlineto{\pgfqpoint{3.648975in}{2.142214in}}%
\pgfpathlineto{\pgfqpoint{3.649359in}{2.142660in}}%
\pgfpathlineto{\pgfqpoint{3.649744in}{2.147235in}}%
\pgfpathlineto{\pgfqpoint{3.650512in}{2.143536in}}%
\pgfpathlineto{\pgfqpoint{3.651089in}{2.146342in}}%
\pgfpathlineto{\pgfqpoint{3.651473in}{2.143140in}}%
\pgfpathlineto{\pgfqpoint{3.651858in}{2.142628in}}%
\pgfpathlineto{\pgfqpoint{3.652050in}{2.138880in}}%
\pgfpathlineto{\pgfqpoint{3.652818in}{2.144553in}}%
\pgfpathlineto{\pgfqpoint{3.653779in}{2.139347in}}%
\pgfpathlineto{\pgfqpoint{3.653971in}{2.142310in}}%
\pgfpathlineto{\pgfqpoint{3.654356in}{2.146209in}}%
\pgfpathlineto{\pgfqpoint{3.654740in}{2.144251in}}%
\pgfpathlineto{\pgfqpoint{3.655893in}{2.133283in}}%
\pgfpathlineto{\pgfqpoint{3.656085in}{2.134739in}}%
\pgfpathlineto{\pgfqpoint{3.656277in}{2.136788in}}%
\pgfpathlineto{\pgfqpoint{3.656854in}{2.131920in}}%
\pgfpathlineto{\pgfqpoint{3.657046in}{2.131889in}}%
\pgfpathlineto{\pgfqpoint{3.657430in}{2.124942in}}%
\pgfpathlineto{\pgfqpoint{3.658199in}{2.131237in}}%
\pgfpathlineto{\pgfqpoint{3.659160in}{2.140757in}}%
\pgfpathlineto{\pgfqpoint{3.659736in}{2.137215in}}%
\pgfpathlineto{\pgfqpoint{3.660313in}{2.135639in}}%
\pgfpathlineto{\pgfqpoint{3.660697in}{2.138560in}}%
\pgfpathlineto{\pgfqpoint{3.662619in}{2.119611in}}%
\pgfpathlineto{\pgfqpoint{3.663195in}{2.123226in}}%
\pgfpathlineto{\pgfqpoint{3.663772in}{2.127944in}}%
\pgfpathlineto{\pgfqpoint{3.664541in}{2.127769in}}%
\pgfpathlineto{\pgfqpoint{3.666847in}{2.124730in}}%
\pgfpathlineto{\pgfqpoint{3.667039in}{2.125478in}}%
\pgfpathlineto{\pgfqpoint{3.667231in}{2.123664in}}%
\pgfpathlineto{\pgfqpoint{3.667423in}{2.124375in}}%
\pgfpathlineto{\pgfqpoint{3.667615in}{2.122152in}}%
\pgfpathlineto{\pgfqpoint{3.668000in}{2.126756in}}%
\pgfpathlineto{\pgfqpoint{3.668384in}{2.132781in}}%
\pgfpathlineto{\pgfqpoint{3.669153in}{2.129922in}}%
\pgfpathlineto{\pgfqpoint{3.669345in}{2.129832in}}%
\pgfpathlineto{\pgfqpoint{3.670690in}{2.133575in}}%
\pgfpathlineto{\pgfqpoint{3.670882in}{2.132169in}}%
\pgfpathlineto{\pgfqpoint{3.671266in}{2.135817in}}%
\pgfpathlineto{\pgfqpoint{3.671651in}{2.133992in}}%
\pgfpathlineto{\pgfqpoint{3.672804in}{2.131838in}}%
\pgfpathlineto{\pgfqpoint{3.673380in}{2.127064in}}%
\pgfpathlineto{\pgfqpoint{3.673957in}{2.131455in}}%
\pgfpathlineto{\pgfqpoint{3.674341in}{2.136115in}}%
\pgfpathlineto{\pgfqpoint{3.674918in}{2.130947in}}%
\pgfpathlineto{\pgfqpoint{3.675110in}{2.133022in}}%
\pgfpathlineto{\pgfqpoint{3.675302in}{2.131962in}}%
\pgfpathlineto{\pgfqpoint{3.675494in}{2.133341in}}%
\pgfpathlineto{\pgfqpoint{3.676071in}{2.132748in}}%
\pgfpathlineto{\pgfqpoint{3.676263in}{2.137526in}}%
\pgfpathlineto{\pgfqpoint{3.677032in}{2.131635in}}%
\pgfpathlineto{\pgfqpoint{3.677800in}{2.128185in}}%
\pgfpathlineto{\pgfqpoint{3.677416in}{2.132140in}}%
\pgfpathlineto{\pgfqpoint{3.678185in}{2.129912in}}%
\pgfpathlineto{\pgfqpoint{3.678569in}{2.132143in}}%
\pgfpathlineto{\pgfqpoint{3.679145in}{2.129906in}}%
\pgfpathlineto{\pgfqpoint{3.679338in}{2.129435in}}%
\pgfpathlineto{\pgfqpoint{3.679530in}{2.133223in}}%
\pgfpathlineto{\pgfqpoint{3.680298in}{2.128687in}}%
\pgfpathlineto{\pgfqpoint{3.680491in}{2.130678in}}%
\pgfpathlineto{\pgfqpoint{3.681451in}{2.119292in}}%
\pgfpathlineto{\pgfqpoint{3.681836in}{2.123806in}}%
\pgfpathlineto{\pgfqpoint{3.683757in}{2.131194in}}%
\pgfpathlineto{\pgfqpoint{3.684142in}{2.128135in}}%
\pgfpathlineto{\pgfqpoint{3.685871in}{2.122592in}}%
\pgfpathlineto{\pgfqpoint{3.687024in}{2.133325in}}%
\pgfpathlineto{\pgfqpoint{3.687409in}{2.129549in}}%
\pgfpathlineto{\pgfqpoint{3.687985in}{2.122166in}}%
\pgfpathlineto{\pgfqpoint{3.688754in}{2.124349in}}%
\pgfpathlineto{\pgfqpoint{3.689715in}{2.129214in}}%
\pgfpathlineto{\pgfqpoint{3.690291in}{2.127277in}}%
\pgfpathlineto{\pgfqpoint{3.690675in}{2.125189in}}%
\pgfpathlineto{\pgfqpoint{3.691444in}{2.129445in}}%
\pgfpathlineto{\pgfqpoint{3.691828in}{2.126448in}}%
\pgfpathlineto{\pgfqpoint{3.692021in}{2.126115in}}%
\pgfpathlineto{\pgfqpoint{3.692213in}{2.127738in}}%
\pgfpathlineto{\pgfqpoint{3.692405in}{2.126859in}}%
\pgfpathlineto{\pgfqpoint{3.693366in}{2.134181in}}%
\pgfpathlineto{\pgfqpoint{3.693750in}{2.132391in}}%
\pgfpathlineto{\pgfqpoint{3.693942in}{2.132458in}}%
\pgfpathlineto{\pgfqpoint{3.695287in}{2.137655in}}%
\pgfpathlineto{\pgfqpoint{3.695480in}{2.137535in}}%
\pgfpathlineto{\pgfqpoint{3.695672in}{2.138067in}}%
\pgfpathlineto{\pgfqpoint{3.695864in}{2.137330in}}%
\pgfpathlineto{\pgfqpoint{3.696056in}{2.135720in}}%
\pgfpathlineto{\pgfqpoint{3.696633in}{2.139820in}}%
\pgfpathlineto{\pgfqpoint{3.697017in}{2.144637in}}%
\pgfpathlineto{\pgfqpoint{3.697593in}{2.142355in}}%
\pgfpathlineto{\pgfqpoint{3.698939in}{2.137532in}}%
\pgfpathlineto{\pgfqpoint{3.699131in}{2.138655in}}%
\pgfpathlineto{\pgfqpoint{3.699707in}{2.136529in}}%
\pgfpathlineto{\pgfqpoint{3.700092in}{2.134903in}}%
\pgfpathlineto{\pgfqpoint{3.700476in}{2.138291in}}%
\pgfpathlineto{\pgfqpoint{3.700860in}{2.137638in}}%
\pgfpathlineto{\pgfqpoint{3.701053in}{2.138588in}}%
\pgfpathlineto{\pgfqpoint{3.703359in}{2.148685in}}%
\pgfpathlineto{\pgfqpoint{3.704896in}{2.137923in}}%
\pgfpathlineto{\pgfqpoint{3.705857in}{2.144233in}}%
\pgfpathlineto{\pgfqpoint{3.706433in}{2.143746in}}%
\pgfpathlineto{\pgfqpoint{3.706625in}{2.142471in}}%
\pgfpathlineto{\pgfqpoint{3.706818in}{2.144475in}}%
\pgfpathlineto{\pgfqpoint{3.707010in}{2.143617in}}%
\pgfpathlineto{\pgfqpoint{3.707586in}{2.141322in}}%
\pgfpathlineto{\pgfqpoint{3.708355in}{2.148124in}}%
\pgfpathlineto{\pgfqpoint{3.711622in}{2.160992in}}%
\pgfpathlineto{\pgfqpoint{3.711814in}{2.159862in}}%
\pgfpathlineto{\pgfqpoint{3.712198in}{2.162135in}}%
\pgfpathlineto{\pgfqpoint{3.712390in}{2.163934in}}%
\pgfpathlineto{\pgfqpoint{3.712583in}{2.161376in}}%
\pgfpathlineto{\pgfqpoint{3.713159in}{2.161667in}}%
\pgfpathlineto{\pgfqpoint{3.713351in}{2.159194in}}%
\pgfpathlineto{\pgfqpoint{3.713928in}{2.162751in}}%
\pgfpathlineto{\pgfqpoint{3.714504in}{2.166618in}}%
\pgfpathlineto{\pgfqpoint{3.714889in}{2.162948in}}%
\pgfpathlineto{\pgfqpoint{3.715081in}{2.162323in}}%
\pgfpathlineto{\pgfqpoint{3.715273in}{2.163201in}}%
\pgfpathlineto{\pgfqpoint{3.715465in}{2.166648in}}%
\pgfpathlineto{\pgfqpoint{3.716042in}{2.160161in}}%
\pgfpathlineto{\pgfqpoint{3.716234in}{2.165178in}}%
\pgfpathlineto{\pgfqpoint{3.716426in}{2.163358in}}%
\pgfpathlineto{\pgfqpoint{3.716810in}{2.168683in}}%
\pgfpathlineto{\pgfqpoint{3.717195in}{2.166060in}}%
\pgfpathlineto{\pgfqpoint{3.717579in}{2.165436in}}%
\pgfpathlineto{\pgfqpoint{3.717771in}{2.168297in}}%
\pgfpathlineto{\pgfqpoint{3.718348in}{2.162780in}}%
\pgfpathlineto{\pgfqpoint{3.719116in}{2.158325in}}%
\pgfpathlineto{\pgfqpoint{3.719308in}{2.161683in}}%
\pgfpathlineto{\pgfqpoint{3.720269in}{2.165870in}}%
\pgfpathlineto{\pgfqpoint{3.720654in}{2.165149in}}%
\pgfpathlineto{\pgfqpoint{3.721038in}{2.166316in}}%
\pgfpathlineto{\pgfqpoint{3.721230in}{2.164542in}}%
\pgfpathlineto{\pgfqpoint{3.722191in}{2.158980in}}%
\pgfpathlineto{\pgfqpoint{3.722575in}{2.159103in}}%
\pgfpathlineto{\pgfqpoint{3.722768in}{2.159003in}}%
\pgfpathlineto{\pgfqpoint{3.724497in}{2.169317in}}%
\pgfpathlineto{\pgfqpoint{3.724689in}{2.168496in}}%
\pgfpathlineto{\pgfqpoint{3.726227in}{2.162993in}}%
\pgfpathlineto{\pgfqpoint{3.727572in}{2.170245in}}%
\pgfpathlineto{\pgfqpoint{3.727764in}{2.167339in}}%
\pgfpathlineto{\pgfqpoint{3.728340in}{2.174669in}}%
\pgfpathlineto{\pgfqpoint{3.728533in}{2.172126in}}%
\pgfpathlineto{\pgfqpoint{3.729301in}{2.179663in}}%
\pgfpathlineto{\pgfqpoint{3.729878in}{2.176335in}}%
\pgfpathlineto{\pgfqpoint{3.730454in}{2.173168in}}%
\pgfpathlineto{\pgfqpoint{3.730646in}{2.174576in}}%
\pgfpathlineto{\pgfqpoint{3.732376in}{2.163501in}}%
\pgfpathlineto{\pgfqpoint{3.733721in}{2.168964in}}%
\pgfpathlineto{\pgfqpoint{3.733913in}{2.167518in}}%
\pgfpathlineto{\pgfqpoint{3.735258in}{2.163378in}}%
\pgfpathlineto{\pgfqpoint{3.736411in}{2.170491in}}%
\pgfpathlineto{\pgfqpoint{3.736796in}{2.170050in}}%
\pgfpathlineto{\pgfqpoint{3.737180in}{2.172013in}}%
\pgfpathlineto{\pgfqpoint{3.737372in}{2.170033in}}%
\pgfpathlineto{\pgfqpoint{3.738333in}{2.166497in}}%
\pgfpathlineto{\pgfqpoint{3.738525in}{2.168752in}}%
\pgfpathlineto{\pgfqpoint{3.739870in}{2.175492in}}%
\pgfpathlineto{\pgfqpoint{3.740063in}{2.174184in}}%
\pgfpathlineto{\pgfqpoint{3.741216in}{2.186231in}}%
\pgfpathlineto{\pgfqpoint{3.741408in}{2.189276in}}%
\pgfpathlineto{\pgfqpoint{3.741792in}{2.184996in}}%
\pgfpathlineto{\pgfqpoint{3.742369in}{2.188348in}}%
\pgfpathlineto{\pgfqpoint{3.742945in}{2.187807in}}%
\pgfpathlineto{\pgfqpoint{3.743137in}{2.190636in}}%
\pgfpathlineto{\pgfqpoint{3.743329in}{2.187077in}}%
\pgfpathlineto{\pgfqpoint{3.744290in}{2.188449in}}%
\pgfpathlineto{\pgfqpoint{3.745635in}{2.192187in}}%
\pgfpathlineto{\pgfqpoint{3.746789in}{2.196963in}}%
\pgfpathlineto{\pgfqpoint{3.746981in}{2.195164in}}%
\pgfpathlineto{\pgfqpoint{3.747557in}{2.189217in}}%
\pgfpathlineto{\pgfqpoint{3.748326in}{2.191198in}}%
\pgfpathlineto{\pgfqpoint{3.748902in}{2.190323in}}%
\pgfpathlineto{\pgfqpoint{3.749095in}{2.192425in}}%
\pgfpathlineto{\pgfqpoint{3.749479in}{2.192151in}}%
\pgfpathlineto{\pgfqpoint{3.750055in}{2.199883in}}%
\pgfpathlineto{\pgfqpoint{3.750248in}{2.203213in}}%
\pgfpathlineto{\pgfqpoint{3.750824in}{2.197792in}}%
\pgfpathlineto{\pgfqpoint{3.751016in}{2.195300in}}%
\pgfpathlineto{\pgfqpoint{3.751593in}{2.197078in}}%
\pgfpathlineto{\pgfqpoint{3.752361in}{2.202641in}}%
\pgfpathlineto{\pgfqpoint{3.752554in}{2.201916in}}%
\pgfpathlineto{\pgfqpoint{3.754091in}{2.190067in}}%
\pgfpathlineto{\pgfqpoint{3.754860in}{2.185665in}}%
\pgfpathlineto{\pgfqpoint{3.755436in}{2.185864in}}%
\pgfpathlineto{\pgfqpoint{3.756973in}{2.199039in}}%
\pgfpathlineto{\pgfqpoint{3.757166in}{2.196109in}}%
\pgfpathlineto{\pgfqpoint{3.757742in}{2.194904in}}%
\pgfpathlineto{\pgfqpoint{3.757934in}{2.197701in}}%
\pgfpathlineto{\pgfqpoint{3.758511in}{2.191398in}}%
\pgfpathlineto{\pgfqpoint{3.759856in}{2.184356in}}%
\pgfpathlineto{\pgfqpoint{3.760240in}{2.186969in}}%
\pgfpathlineto{\pgfqpoint{3.760817in}{2.189093in}}%
\pgfpathlineto{\pgfqpoint{3.761201in}{2.186278in}}%
\pgfpathlineto{\pgfqpoint{3.761393in}{2.184287in}}%
\pgfpathlineto{\pgfqpoint{3.761778in}{2.186074in}}%
\pgfpathlineto{\pgfqpoint{3.762354in}{2.192407in}}%
\pgfpathlineto{\pgfqpoint{3.762931in}{2.190503in}}%
\pgfpathlineto{\pgfqpoint{3.763699in}{2.193103in}}%
\pgfpathlineto{\pgfqpoint{3.763891in}{2.190012in}}%
\pgfpathlineto{\pgfqpoint{3.764468in}{2.187174in}}%
\pgfpathlineto{\pgfqpoint{3.764276in}{2.190667in}}%
\pgfpathlineto{\pgfqpoint{3.764660in}{2.189466in}}%
\pgfpathlineto{\pgfqpoint{3.766005in}{2.194590in}}%
\pgfpathlineto{\pgfqpoint{3.766197in}{2.192970in}}%
\pgfpathlineto{\pgfqpoint{3.767158in}{2.193185in}}%
\pgfpathlineto{\pgfqpoint{3.767350in}{2.191149in}}%
\pgfpathlineto{\pgfqpoint{3.767543in}{2.193972in}}%
\pgfpathlineto{\pgfqpoint{3.767927in}{2.193482in}}%
\pgfpathlineto{\pgfqpoint{3.768696in}{2.196073in}}%
\pgfpathlineto{\pgfqpoint{3.769656in}{2.189914in}}%
\pgfpathlineto{\pgfqpoint{3.769849in}{2.191459in}}%
\pgfpathlineto{\pgfqpoint{3.770425in}{2.190605in}}%
\pgfpathlineto{\pgfqpoint{3.771002in}{2.194687in}}%
\pgfpathlineto{\pgfqpoint{3.771386in}{2.192068in}}%
\pgfpathlineto{\pgfqpoint{3.771770in}{2.194504in}}%
\pgfpathlineto{\pgfqpoint{3.773116in}{2.199743in}}%
\pgfpathlineto{\pgfqpoint{3.774845in}{2.190041in}}%
\pgfpathlineto{\pgfqpoint{3.775229in}{2.191469in}}%
\pgfpathlineto{\pgfqpoint{3.775422in}{2.194423in}}%
\pgfpathlineto{\pgfqpoint{3.775614in}{2.191129in}}%
\pgfpathlineto{\pgfqpoint{3.776190in}{2.193499in}}%
\pgfpathlineto{\pgfqpoint{3.776767in}{2.188668in}}%
\pgfpathlineto{\pgfqpoint{3.777343in}{2.192357in}}%
\pgfpathlineto{\pgfqpoint{3.778496in}{2.199270in}}%
\pgfpathlineto{\pgfqpoint{3.778688in}{2.195509in}}%
\pgfpathlineto{\pgfqpoint{3.778881in}{2.195954in}}%
\pgfpathlineto{\pgfqpoint{3.780418in}{2.189132in}}%
\pgfpathlineto{\pgfqpoint{3.782147in}{2.200021in}}%
\pgfpathlineto{\pgfqpoint{3.783300in}{2.197475in}}%
\pgfpathlineto{\pgfqpoint{3.783685in}{2.198045in}}%
\pgfpathlineto{\pgfqpoint{3.785222in}{2.207067in}}%
\pgfpathlineto{\pgfqpoint{3.785414in}{2.204697in}}%
\pgfpathlineto{\pgfqpoint{3.785799in}{2.201208in}}%
\pgfpathlineto{\pgfqpoint{3.786183in}{2.206653in}}%
\pgfpathlineto{\pgfqpoint{3.786375in}{2.206416in}}%
\pgfpathlineto{\pgfqpoint{3.787336in}{2.214975in}}%
\pgfpathlineto{\pgfqpoint{3.787912in}{2.211992in}}%
\pgfpathlineto{\pgfqpoint{3.788105in}{2.213159in}}%
\pgfpathlineto{\pgfqpoint{3.788489in}{2.209472in}}%
\pgfpathlineto{\pgfqpoint{3.789065in}{2.212002in}}%
\pgfpathlineto{\pgfqpoint{3.790026in}{2.215155in}}%
\pgfpathlineto{\pgfqpoint{3.790218in}{2.211739in}}%
\pgfpathlineto{\pgfqpoint{3.790795in}{2.219292in}}%
\pgfpathlineto{\pgfqpoint{3.791564in}{2.222697in}}%
\pgfpathlineto{\pgfqpoint{3.791756in}{2.220912in}}%
\pgfpathlineto{\pgfqpoint{3.793293in}{2.208476in}}%
\pgfpathlineto{\pgfqpoint{3.794254in}{2.216707in}}%
\pgfpathlineto{\pgfqpoint{3.794830in}{2.216059in}}%
\pgfpathlineto{\pgfqpoint{3.795023in}{2.215413in}}%
\pgfpathlineto{\pgfqpoint{3.795215in}{2.218442in}}%
\pgfpathlineto{\pgfqpoint{3.795599in}{2.224867in}}%
\pgfpathlineto{\pgfqpoint{3.796176in}{2.217900in}}%
\pgfpathlineto{\pgfqpoint{3.796944in}{2.216210in}}%
\pgfpathlineto{\pgfqpoint{3.796752in}{2.218060in}}%
\pgfpathlineto{\pgfqpoint{3.797137in}{2.217104in}}%
\pgfpathlineto{\pgfqpoint{3.797329in}{2.220776in}}%
\pgfpathlineto{\pgfqpoint{3.798290in}{2.219570in}}%
\pgfpathlineto{\pgfqpoint{3.798482in}{2.221358in}}%
\pgfpathlineto{\pgfqpoint{3.798866in}{2.218799in}}%
\pgfpathlineto{\pgfqpoint{3.800019in}{2.210250in}}%
\pgfpathlineto{\pgfqpoint{3.800211in}{2.212058in}}%
\pgfpathlineto{\pgfqpoint{3.800980in}{2.214839in}}%
\pgfpathlineto{\pgfqpoint{3.801364in}{2.219546in}}%
\pgfpathlineto{\pgfqpoint{3.801941in}{2.216018in}}%
\pgfpathlineto{\pgfqpoint{3.802325in}{2.211237in}}%
\pgfpathlineto{\pgfqpoint{3.802517in}{2.209388in}}%
\pgfpathlineto{\pgfqpoint{3.802709in}{2.213630in}}%
\pgfpathlineto{\pgfqpoint{3.802902in}{2.212922in}}%
\pgfpathlineto{\pgfqpoint{3.803094in}{2.215149in}}%
\pgfpathlineto{\pgfqpoint{3.803670in}{2.212169in}}%
\pgfpathlineto{\pgfqpoint{3.805015in}{2.205041in}}%
\pgfpathlineto{\pgfqpoint{3.805208in}{2.205815in}}%
\pgfpathlineto{\pgfqpoint{3.805592in}{2.209866in}}%
\pgfpathlineto{\pgfqpoint{3.805976in}{2.204142in}}%
\pgfpathlineto{\pgfqpoint{3.806168in}{2.203557in}}%
\pgfpathlineto{\pgfqpoint{3.806553in}{2.205800in}}%
\pgfpathlineto{\pgfqpoint{3.808859in}{2.221143in}}%
\pgfpathlineto{\pgfqpoint{3.809820in}{2.213362in}}%
\pgfpathlineto{\pgfqpoint{3.810204in}{2.213451in}}%
\pgfpathlineto{\pgfqpoint{3.811357in}{2.209899in}}%
\pgfpathlineto{\pgfqpoint{3.811741in}{2.212356in}}%
\pgfpathlineto{\pgfqpoint{3.813086in}{2.218263in}}%
\pgfpathlineto{\pgfqpoint{3.813279in}{2.218252in}}%
\pgfpathlineto{\pgfqpoint{3.814624in}{2.199209in}}%
\pgfpathlineto{\pgfqpoint{3.815008in}{2.203258in}}%
\pgfpathlineto{\pgfqpoint{3.816545in}{2.214296in}}%
\pgfpathlineto{\pgfqpoint{3.816738in}{2.211232in}}%
\pgfpathlineto{\pgfqpoint{3.818467in}{2.199127in}}%
\pgfpathlineto{\pgfqpoint{3.819044in}{2.197289in}}%
\pgfpathlineto{\pgfqpoint{3.819236in}{2.198480in}}%
\pgfpathlineto{\pgfqpoint{3.821350in}{2.213139in}}%
\pgfpathlineto{\pgfqpoint{3.821542in}{2.212746in}}%
\pgfpathlineto{\pgfqpoint{3.822311in}{2.205801in}}%
\pgfpathlineto{\pgfqpoint{3.822695in}{2.207730in}}%
\pgfpathlineto{\pgfqpoint{3.823464in}{2.214536in}}%
\pgfpathlineto{\pgfqpoint{3.824040in}{2.213447in}}%
\pgfpathlineto{\pgfqpoint{3.824424in}{2.210150in}}%
\pgfpathlineto{\pgfqpoint{3.825001in}{2.212024in}}%
\pgfpathlineto{\pgfqpoint{3.825577in}{2.213361in}}%
\pgfpathlineto{\pgfqpoint{3.825962in}{2.212344in}}%
\pgfpathlineto{\pgfqpoint{3.826538in}{2.210602in}}%
\pgfpathlineto{\pgfqpoint{3.827115in}{2.210870in}}%
\pgfpathlineto{\pgfqpoint{3.827691in}{2.213570in}}%
\pgfpathlineto{\pgfqpoint{3.828268in}{2.210847in}}%
\pgfpathlineto{\pgfqpoint{3.829036in}{2.211770in}}%
\pgfpathlineto{\pgfqpoint{3.829229in}{2.210610in}}%
\pgfpathlineto{\pgfqpoint{3.829421in}{2.213569in}}%
\pgfpathlineto{\pgfqpoint{3.829805in}{2.206478in}}%
\pgfpathlineto{\pgfqpoint{3.829997in}{2.206778in}}%
\pgfpathlineto{\pgfqpoint{3.831342in}{2.210508in}}%
\pgfpathlineto{\pgfqpoint{3.830382in}{2.205596in}}%
\pgfpathlineto{\pgfqpoint{3.831535in}{2.209166in}}%
\pgfpathlineto{\pgfqpoint{3.832111in}{2.203676in}}%
\pgfpathlineto{\pgfqpoint{3.832688in}{2.207480in}}%
\pgfpathlineto{\pgfqpoint{3.832880in}{2.204476in}}%
\pgfpathlineto{\pgfqpoint{3.833648in}{2.208523in}}%
\pgfpathlineto{\pgfqpoint{3.834225in}{2.202638in}}%
\pgfpathlineto{\pgfqpoint{3.834801in}{2.204832in}}%
\pgfpathlineto{\pgfqpoint{3.835954in}{2.211318in}}%
\pgfpathlineto{\pgfqpoint{3.837107in}{2.204516in}}%
\pgfpathlineto{\pgfqpoint{3.837300in}{2.206068in}}%
\pgfpathlineto{\pgfqpoint{3.837492in}{2.206493in}}%
\pgfpathlineto{\pgfqpoint{3.837684in}{2.206026in}}%
\pgfpathlineto{\pgfqpoint{3.838068in}{2.202522in}}%
\pgfpathlineto{\pgfqpoint{3.838453in}{2.205797in}}%
\pgfpathlineto{\pgfqpoint{3.839029in}{2.214654in}}%
\pgfpathlineto{\pgfqpoint{3.839606in}{2.209975in}}%
\pgfpathlineto{\pgfqpoint{3.839798in}{2.208942in}}%
\pgfpathlineto{\pgfqpoint{3.839990in}{2.211276in}}%
\pgfpathlineto{\pgfqpoint{3.840759in}{2.209725in}}%
\pgfpathlineto{\pgfqpoint{3.842104in}{2.219363in}}%
\pgfpathlineto{\pgfqpoint{3.842488in}{2.217688in}}%
\pgfpathlineto{\pgfqpoint{3.842680in}{2.217170in}}%
\pgfpathlineto{\pgfqpoint{3.844025in}{2.210288in}}%
\pgfpathlineto{\pgfqpoint{3.844410in}{2.208692in}}%
\pgfpathlineto{\pgfqpoint{3.844794in}{2.211176in}}%
\pgfpathlineto{\pgfqpoint{3.845371in}{2.215636in}}%
\pgfpathlineto{\pgfqpoint{3.845563in}{2.214543in}}%
\pgfpathlineto{\pgfqpoint{3.846908in}{2.223831in}}%
\pgfpathlineto{\pgfqpoint{3.845947in}{2.214063in}}%
\pgfpathlineto{\pgfqpoint{3.847100in}{2.221834in}}%
\pgfpathlineto{\pgfqpoint{3.847292in}{2.220207in}}%
\pgfpathlineto{\pgfqpoint{3.847869in}{2.223274in}}%
\pgfpathlineto{\pgfqpoint{3.848253in}{2.221227in}}%
\pgfpathlineto{\pgfqpoint{3.849214in}{2.232417in}}%
\pgfpathlineto{\pgfqpoint{3.849598in}{2.226796in}}%
\pgfpathlineto{\pgfqpoint{3.849791in}{2.227239in}}%
\pgfpathlineto{\pgfqpoint{3.849983in}{2.224449in}}%
\pgfpathlineto{\pgfqpoint{3.850751in}{2.226602in}}%
\pgfpathlineto{\pgfqpoint{3.851136in}{2.230881in}}%
\pgfpathlineto{\pgfqpoint{3.851904in}{2.229165in}}%
\pgfpathlineto{\pgfqpoint{3.853442in}{2.225131in}}%
\pgfpathlineto{\pgfqpoint{3.853826in}{2.226999in}}%
\pgfpathlineto{\pgfqpoint{3.854979in}{2.230737in}}%
\pgfpathlineto{\pgfqpoint{3.854210in}{2.226279in}}%
\pgfpathlineto{\pgfqpoint{3.855171in}{2.229119in}}%
\pgfpathlineto{\pgfqpoint{3.855556in}{2.223115in}}%
\pgfpathlineto{\pgfqpoint{3.856324in}{2.227204in}}%
\pgfpathlineto{\pgfqpoint{3.856901in}{2.223991in}}%
\pgfpathlineto{\pgfqpoint{3.857093in}{2.225616in}}%
\pgfpathlineto{\pgfqpoint{3.857285in}{2.228247in}}%
\pgfpathlineto{\pgfqpoint{3.857862in}{2.224289in}}%
\pgfpathlineto{\pgfqpoint{3.858054in}{2.224623in}}%
\pgfpathlineto{\pgfqpoint{3.858246in}{2.224237in}}%
\pgfpathlineto{\pgfqpoint{3.858438in}{2.224872in}}%
\pgfpathlineto{\pgfqpoint{3.858822in}{2.229177in}}%
\pgfpathlineto{\pgfqpoint{3.859207in}{2.224501in}}%
\pgfpathlineto{\pgfqpoint{3.860744in}{2.216573in}}%
\pgfpathlineto{\pgfqpoint{3.860936in}{2.219224in}}%
\pgfpathlineto{\pgfqpoint{3.861321in}{2.215992in}}%
\pgfpathlineto{\pgfqpoint{3.861705in}{2.218766in}}%
\pgfpathlineto{\pgfqpoint{3.861897in}{2.215222in}}%
\pgfpathlineto{\pgfqpoint{3.862666in}{2.221300in}}%
\pgfpathlineto{\pgfqpoint{3.863242in}{2.223544in}}%
\pgfpathlineto{\pgfqpoint{3.863434in}{2.220878in}}%
\pgfpathlineto{\pgfqpoint{3.864587in}{2.211331in}}%
\pgfpathlineto{\pgfqpoint{3.864972in}{2.214682in}}%
\pgfpathlineto{\pgfqpoint{3.866509in}{2.225697in}}%
\pgfpathlineto{\pgfqpoint{3.866893in}{2.225337in}}%
\pgfpathlineto{\pgfqpoint{3.868431in}{2.219596in}}%
\pgfpathlineto{\pgfqpoint{3.871313in}{2.235901in}}%
\pgfpathlineto{\pgfqpoint{3.873235in}{2.228584in}}%
\pgfpathlineto{\pgfqpoint{3.873619in}{2.232268in}}%
\pgfpathlineto{\pgfqpoint{3.873812in}{2.234777in}}%
\pgfpathlineto{\pgfqpoint{3.874196in}{2.229605in}}%
\pgfpathlineto{\pgfqpoint{3.875157in}{2.217020in}}%
\pgfpathlineto{\pgfqpoint{3.875925in}{2.221861in}}%
\pgfpathlineto{\pgfqpoint{3.876310in}{2.225258in}}%
\pgfpathlineto{\pgfqpoint{3.876694in}{2.220499in}}%
\pgfpathlineto{\pgfqpoint{3.876886in}{2.221366in}}%
\pgfpathlineto{\pgfqpoint{3.878424in}{2.223741in}}%
\pgfpathlineto{\pgfqpoint{3.878616in}{2.226269in}}%
\pgfpathlineto{\pgfqpoint{3.879384in}{2.222476in}}%
\pgfpathlineto{\pgfqpoint{3.881306in}{2.236517in}}%
\pgfpathlineto{\pgfqpoint{3.879769in}{2.220981in}}%
\pgfpathlineto{\pgfqpoint{3.882075in}{2.231474in}}%
\pgfpathlineto{\pgfqpoint{3.882843in}{2.228585in}}%
\pgfpathlineto{\pgfqpoint{3.883036in}{2.231675in}}%
\pgfpathlineto{\pgfqpoint{3.884957in}{2.241256in}}%
\pgfpathlineto{\pgfqpoint{3.886495in}{2.235077in}}%
\pgfpathlineto{\pgfqpoint{3.886879in}{2.236551in}}%
\pgfpathlineto{\pgfqpoint{3.887455in}{2.240513in}}%
\pgfpathlineto{\pgfqpoint{3.888032in}{2.237613in}}%
\pgfpathlineto{\pgfqpoint{3.888608in}{2.237633in}}%
\pgfpathlineto{\pgfqpoint{3.888801in}{2.240854in}}%
\pgfpathlineto{\pgfqpoint{3.889185in}{2.234622in}}%
\pgfpathlineto{\pgfqpoint{3.889569in}{2.235929in}}%
\pgfpathlineto{\pgfqpoint{3.890146in}{2.238022in}}%
\pgfpathlineto{\pgfqpoint{3.890722in}{2.236117in}}%
\pgfpathlineto{\pgfqpoint{3.890914in}{2.232488in}}%
\pgfpathlineto{\pgfqpoint{3.891683in}{2.238003in}}%
\pgfpathlineto{\pgfqpoint{3.892067in}{2.237050in}}%
\pgfpathlineto{\pgfqpoint{3.892452in}{2.238718in}}%
\pgfpathlineto{\pgfqpoint{3.892644in}{2.237440in}}%
\pgfpathlineto{\pgfqpoint{3.893605in}{2.242167in}}%
\pgfpathlineto{\pgfqpoint{3.893797in}{2.239229in}}%
\pgfpathlineto{\pgfqpoint{3.895142in}{2.234429in}}%
\pgfpathlineto{\pgfqpoint{3.895911in}{2.233721in}}%
\pgfpathlineto{\pgfqpoint{3.897064in}{2.224406in}}%
\pgfpathlineto{\pgfqpoint{3.897448in}{2.224556in}}%
\pgfpathlineto{\pgfqpoint{3.898025in}{2.224750in}}%
\pgfpathlineto{\pgfqpoint{3.898217in}{2.226164in}}%
\pgfpathlineto{\pgfqpoint{3.898601in}{2.221894in}}%
\pgfpathlineto{\pgfqpoint{3.898986in}{2.224965in}}%
\pgfpathlineto{\pgfqpoint{3.899370in}{2.227304in}}%
\pgfpathlineto{\pgfqpoint{3.899754in}{2.224029in}}%
\pgfpathlineto{\pgfqpoint{3.899946in}{2.226522in}}%
\pgfpathlineto{\pgfqpoint{3.900715in}{2.222350in}}%
\pgfpathlineto{\pgfqpoint{3.901676in}{2.224905in}}%
\pgfpathlineto{\pgfqpoint{3.901099in}{2.220323in}}%
\pgfpathlineto{\pgfqpoint{3.902252in}{2.224717in}}%
\pgfpathlineto{\pgfqpoint{3.902445in}{2.223988in}}%
\pgfpathlineto{\pgfqpoint{3.902637in}{2.226466in}}%
\pgfpathlineto{\pgfqpoint{3.903021in}{2.224638in}}%
\pgfpathlineto{\pgfqpoint{3.903405in}{2.230122in}}%
\pgfpathlineto{\pgfqpoint{3.903982in}{2.224503in}}%
\pgfpathlineto{\pgfqpoint{3.904174in}{2.224940in}}%
\pgfpathlineto{\pgfqpoint{3.904558in}{2.221691in}}%
\pgfpathlineto{\pgfqpoint{3.905327in}{2.222869in}}%
\pgfpathlineto{\pgfqpoint{3.905711in}{2.224276in}}%
\pgfpathlineto{\pgfqpoint{3.905904in}{2.221708in}}%
\pgfpathlineto{\pgfqpoint{3.906096in}{2.224704in}}%
\pgfpathlineto{\pgfqpoint{3.906288in}{2.219549in}}%
\pgfpathlineto{\pgfqpoint{3.906864in}{2.221851in}}%
\pgfpathlineto{\pgfqpoint{3.907057in}{2.222151in}}%
\pgfpathlineto{\pgfqpoint{3.907633in}{2.217919in}}%
\pgfpathlineto{\pgfqpoint{3.908017in}{2.220434in}}%
\pgfpathlineto{\pgfqpoint{3.908786in}{2.223833in}}%
\pgfpathlineto{\pgfqpoint{3.908402in}{2.219237in}}%
\pgfpathlineto{\pgfqpoint{3.908978in}{2.220355in}}%
\pgfpathlineto{\pgfqpoint{3.911476in}{2.208086in}}%
\pgfpathlineto{\pgfqpoint{3.913014in}{2.213477in}}%
\pgfpathlineto{\pgfqpoint{3.914743in}{2.203249in}}%
\pgfpathlineto{\pgfqpoint{3.914935in}{2.205476in}}%
\pgfpathlineto{\pgfqpoint{3.915512in}{2.201956in}}%
\pgfpathlineto{\pgfqpoint{3.917241in}{2.195307in}}%
\pgfpathlineto{\pgfqpoint{3.917434in}{2.196216in}}%
\pgfpathlineto{\pgfqpoint{3.919355in}{2.205155in}}%
\pgfpathlineto{\pgfqpoint{3.920508in}{2.201711in}}%
\pgfpathlineto{\pgfqpoint{3.920893in}{2.200863in}}%
\pgfpathlineto{\pgfqpoint{3.921661in}{2.204822in}}%
\pgfpathlineto{\pgfqpoint{3.923007in}{2.195264in}}%
\pgfpathlineto{\pgfqpoint{3.923199in}{2.195732in}}%
\pgfpathlineto{\pgfqpoint{3.924544in}{2.187832in}}%
\pgfpathlineto{\pgfqpoint{3.924736in}{2.187822in}}%
\pgfpathlineto{\pgfqpoint{3.924928in}{2.186280in}}%
\pgfpathlineto{\pgfqpoint{3.925313in}{2.192146in}}%
\pgfpathlineto{\pgfqpoint{3.925505in}{2.189200in}}%
\pgfpathlineto{\pgfqpoint{3.926273in}{2.193181in}}%
\pgfpathlineto{\pgfqpoint{3.926658in}{2.192838in}}%
\pgfpathlineto{\pgfqpoint{3.928195in}{2.184870in}}%
\pgfpathlineto{\pgfqpoint{3.929732in}{2.192006in}}%
\pgfpathlineto{\pgfqpoint{3.929925in}{2.192230in}}%
\pgfpathlineto{\pgfqpoint{3.930501in}{2.205221in}}%
\pgfpathlineto{\pgfqpoint{3.931078in}{2.197453in}}%
\pgfpathlineto{\pgfqpoint{3.931270in}{2.197713in}}%
\pgfpathlineto{\pgfqpoint{3.932231in}{2.191741in}}%
\pgfpathlineto{\pgfqpoint{3.932423in}{2.192933in}}%
\pgfpathlineto{\pgfqpoint{3.932615in}{2.193071in}}%
\pgfpathlineto{\pgfqpoint{3.933384in}{2.201456in}}%
\pgfpathlineto{\pgfqpoint{3.933960in}{2.200413in}}%
\pgfpathlineto{\pgfqpoint{3.934537in}{2.195915in}}%
\pgfpathlineto{\pgfqpoint{3.935305in}{2.197968in}}%
\pgfpathlineto{\pgfqpoint{3.935497in}{2.200825in}}%
\pgfpathlineto{\pgfqpoint{3.935882in}{2.195798in}}%
\pgfpathlineto{\pgfqpoint{3.936266in}{2.197055in}}%
\pgfpathlineto{\pgfqpoint{3.937611in}{2.190001in}}%
\pgfpathlineto{\pgfqpoint{3.937803in}{2.194199in}}%
\pgfpathlineto{\pgfqpoint{3.938572in}{2.188086in}}%
\pgfpathlineto{\pgfqpoint{3.938764in}{2.188175in}}%
\pgfpathlineto{\pgfqpoint{3.938956in}{2.187308in}}%
\pgfpathlineto{\pgfqpoint{3.939149in}{2.190064in}}%
\pgfpathlineto{\pgfqpoint{3.939341in}{2.186802in}}%
\pgfpathlineto{\pgfqpoint{3.939917in}{2.187409in}}%
\pgfpathlineto{\pgfqpoint{3.941455in}{2.180649in}}%
\pgfpathlineto{\pgfqpoint{3.941647in}{2.179192in}}%
\pgfpathlineto{\pgfqpoint{3.942223in}{2.180960in}}%
\pgfpathlineto{\pgfqpoint{3.942416in}{2.183042in}}%
\pgfpathlineto{\pgfqpoint{3.942800in}{2.180378in}}%
\pgfpathlineto{\pgfqpoint{3.944145in}{2.172128in}}%
\pgfpathlineto{\pgfqpoint{3.944337in}{2.176237in}}%
\pgfpathlineto{\pgfqpoint{3.944914in}{2.168265in}}%
\pgfpathlineto{\pgfqpoint{3.945106in}{2.168568in}}%
\pgfpathlineto{\pgfqpoint{3.946259in}{2.175269in}}%
\pgfpathlineto{\pgfqpoint{3.946451in}{2.172995in}}%
\pgfpathlineto{\pgfqpoint{3.947028in}{2.176197in}}%
\pgfpathlineto{\pgfqpoint{3.947220in}{2.174098in}}%
\pgfpathlineto{\pgfqpoint{3.948373in}{2.169784in}}%
\pgfpathlineto{\pgfqpoint{3.947604in}{2.174138in}}%
\pgfpathlineto{\pgfqpoint{3.948565in}{2.171226in}}%
\pgfpathlineto{\pgfqpoint{3.950679in}{2.158717in}}%
\pgfpathlineto{\pgfqpoint{3.951063in}{2.161286in}}%
\pgfpathlineto{\pgfqpoint{3.952793in}{2.170241in}}%
\pgfpathlineto{\pgfqpoint{3.953177in}{2.170858in}}%
\pgfpathlineto{\pgfqpoint{3.954714in}{2.161720in}}%
\pgfpathlineto{\pgfqpoint{3.955483in}{2.161620in}}%
\pgfpathlineto{\pgfqpoint{3.956636in}{2.170996in}}%
\pgfpathlineto{\pgfqpoint{3.957212in}{2.168536in}}%
\pgfpathlineto{\pgfqpoint{3.957405in}{2.172651in}}%
\pgfpathlineto{\pgfqpoint{3.957981in}{2.176399in}}%
\pgfpathlineto{\pgfqpoint{3.958365in}{2.172582in}}%
\pgfpathlineto{\pgfqpoint{3.958558in}{2.171212in}}%
\pgfpathlineto{\pgfqpoint{3.958942in}{2.176116in}}%
\pgfpathlineto{\pgfqpoint{3.960671in}{2.185552in}}%
\pgfpathlineto{\pgfqpoint{3.961824in}{2.187856in}}%
\pgfpathlineto{\pgfqpoint{3.962017in}{2.187415in}}%
\pgfpathlineto{\pgfqpoint{3.963938in}{2.180571in}}%
\pgfpathlineto{\pgfqpoint{3.965668in}{2.193288in}}%
\pgfpathlineto{\pgfqpoint{3.965860in}{2.192573in}}%
\pgfpathlineto{\pgfqpoint{3.966052in}{2.193069in}}%
\pgfpathlineto{\pgfqpoint{3.966629in}{2.197884in}}%
\pgfpathlineto{\pgfqpoint{3.967205in}{2.196930in}}%
\pgfpathlineto{\pgfqpoint{3.967782in}{2.192653in}}%
\pgfpathlineto{\pgfqpoint{3.968358in}{2.195831in}}%
\pgfpathlineto{\pgfqpoint{3.969703in}{2.203701in}}%
\pgfpathlineto{\pgfqpoint{3.971049in}{2.211120in}}%
\pgfpathlineto{\pgfqpoint{3.971625in}{2.204975in}}%
\pgfpathlineto{\pgfqpoint{3.972394in}{2.205409in}}%
\pgfpathlineto{\pgfqpoint{3.972778in}{2.204739in}}%
\pgfpathlineto{\pgfqpoint{3.973162in}{2.207128in}}%
\pgfpathlineto{\pgfqpoint{3.973547in}{2.209293in}}%
\pgfpathlineto{\pgfqpoint{3.974315in}{2.208701in}}%
\pgfpathlineto{\pgfqpoint{3.975276in}{2.206476in}}%
\pgfpathlineto{\pgfqpoint{3.975468in}{2.206804in}}%
\pgfpathlineto{\pgfqpoint{3.975853in}{2.208146in}}%
\pgfpathlineto{\pgfqpoint{3.976237in}{2.207932in}}%
\pgfpathlineto{\pgfqpoint{3.977198in}{2.201869in}}%
\pgfpathlineto{\pgfqpoint{3.977582in}{2.203118in}}%
\pgfpathlineto{\pgfqpoint{3.978159in}{2.205829in}}%
\pgfpathlineto{\pgfqpoint{3.979888in}{2.217910in}}%
\pgfpathlineto{\pgfqpoint{3.980273in}{2.217151in}}%
\pgfpathlineto{\pgfqpoint{3.980465in}{2.216304in}}%
\pgfpathlineto{\pgfqpoint{3.981618in}{2.228231in}}%
\pgfpathlineto{\pgfqpoint{3.981810in}{2.226106in}}%
\pgfpathlineto{\pgfqpoint{3.982002in}{2.223654in}}%
\pgfpathlineto{\pgfqpoint{3.982386in}{2.229011in}}%
\pgfpathlineto{\pgfqpoint{3.982963in}{2.230952in}}%
\pgfpathlineto{\pgfqpoint{3.983155in}{2.227511in}}%
\pgfpathlineto{\pgfqpoint{3.983347in}{2.227659in}}%
\pgfpathlineto{\pgfqpoint{3.983539in}{2.226726in}}%
\pgfpathlineto{\pgfqpoint{3.983732in}{2.220627in}}%
\pgfpathlineto{\pgfqpoint{3.984692in}{2.225384in}}%
\pgfpathlineto{\pgfqpoint{3.985653in}{2.230318in}}%
\pgfpathlineto{\pgfqpoint{3.986038in}{2.229101in}}%
\pgfpathlineto{\pgfqpoint{3.986422in}{2.225090in}}%
\pgfpathlineto{\pgfqpoint{3.986998in}{2.229376in}}%
\pgfpathlineto{\pgfqpoint{3.987383in}{2.228152in}}%
\pgfpathlineto{\pgfqpoint{3.989112in}{2.241660in}}%
\pgfpathlineto{\pgfqpoint{3.989689in}{2.240028in}}%
\pgfpathlineto{\pgfqpoint{3.989881in}{2.239051in}}%
\pgfpathlineto{\pgfqpoint{3.990073in}{2.240492in}}%
\pgfpathlineto{\pgfqpoint{3.990457in}{2.239975in}}%
\pgfpathlineto{\pgfqpoint{3.990842in}{2.245341in}}%
\pgfpathlineto{\pgfqpoint{3.991611in}{2.241358in}}%
\pgfpathlineto{\pgfqpoint{3.991995in}{2.238820in}}%
\pgfpathlineto{\pgfqpoint{3.992571in}{2.241919in}}%
\pgfpathlineto{\pgfqpoint{3.993148in}{2.242751in}}%
\pgfpathlineto{\pgfqpoint{3.993340in}{2.240282in}}%
\pgfpathlineto{\pgfqpoint{3.994685in}{2.245994in}}%
\pgfpathlineto{\pgfqpoint{3.994877in}{2.245131in}}%
\pgfpathlineto{\pgfqpoint{3.995262in}{2.245846in}}%
\pgfpathlineto{\pgfqpoint{3.995646in}{2.251653in}}%
\pgfpathlineto{\pgfqpoint{3.996030in}{2.245032in}}%
\pgfpathlineto{\pgfqpoint{3.996607in}{2.249581in}}%
\pgfpathlineto{\pgfqpoint{3.996799in}{2.247969in}}%
\pgfpathlineto{\pgfqpoint{3.996991in}{2.250301in}}%
\pgfpathlineto{\pgfqpoint{3.997376in}{2.249611in}}%
\pgfpathlineto{\pgfqpoint{3.997568in}{2.253578in}}%
\pgfpathlineto{\pgfqpoint{3.998529in}{2.252005in}}%
\pgfpathlineto{\pgfqpoint{4.000450in}{2.242851in}}%
\pgfpathlineto{\pgfqpoint{4.000642in}{2.245351in}}%
\pgfpathlineto{\pgfqpoint{4.001603in}{2.246756in}}%
\pgfpathlineto{\pgfqpoint{4.001411in}{2.244597in}}%
\pgfpathlineto{\pgfqpoint{4.001795in}{2.245949in}}%
\pgfpathlineto{\pgfqpoint{4.001988in}{2.245877in}}%
\pgfpathlineto{\pgfqpoint{4.002180in}{2.248345in}}%
\pgfpathlineto{\pgfqpoint{4.002756in}{2.243688in}}%
\pgfpathlineto{\pgfqpoint{4.002948in}{2.243688in}}%
\pgfpathlineto{\pgfqpoint{4.003909in}{2.247604in}}%
\pgfpathlineto{\pgfqpoint{4.004101in}{2.247209in}}%
\pgfpathlineto{\pgfqpoint{4.004486in}{2.245528in}}%
\pgfpathlineto{\pgfqpoint{4.004678in}{2.247553in}}%
\pgfpathlineto{\pgfqpoint{4.005639in}{2.251746in}}%
\pgfpathlineto{\pgfqpoint{4.005831in}{2.251589in}}%
\pgfpathlineto{\pgfqpoint{4.006600in}{2.248863in}}%
\pgfpathlineto{\pgfqpoint{4.006792in}{2.247421in}}%
\pgfpathlineto{\pgfqpoint{4.007368in}{2.249197in}}%
\pgfpathlineto{\pgfqpoint{4.009290in}{2.257430in}}%
\pgfpathlineto{\pgfqpoint{4.009674in}{2.255165in}}%
\pgfpathlineto{\pgfqpoint{4.011019in}{2.263365in}}%
\pgfpathlineto{\pgfqpoint{4.011212in}{2.263166in}}%
\pgfpathlineto{\pgfqpoint{4.011404in}{2.260403in}}%
\pgfpathlineto{\pgfqpoint{4.011980in}{2.266178in}}%
\pgfpathlineto{\pgfqpoint{4.012172in}{2.265001in}}%
\pgfpathlineto{\pgfqpoint{4.012557in}{2.267249in}}%
\pgfpathlineto{\pgfqpoint{4.012941in}{2.263614in}}%
\pgfpathlineto{\pgfqpoint{4.013325in}{2.266021in}}%
\pgfpathlineto{\pgfqpoint{4.013710in}{2.263444in}}%
\pgfpathlineto{\pgfqpoint{4.014478in}{2.263965in}}%
\pgfpathlineto{\pgfqpoint{4.014863in}{2.266668in}}%
\pgfpathlineto{\pgfqpoint{4.016016in}{2.274433in}}%
\pgfpathlineto{\pgfqpoint{4.016208in}{2.271702in}}%
\pgfpathlineto{\pgfqpoint{4.017169in}{2.273728in}}%
\pgfpathlineto{\pgfqpoint{4.016785in}{2.270302in}}%
\pgfpathlineto{\pgfqpoint{4.017553in}{2.273614in}}%
\pgfpathlineto{\pgfqpoint{4.018322in}{2.269835in}}%
\pgfpathlineto{\pgfqpoint{4.018706in}{2.271850in}}%
\pgfpathlineto{\pgfqpoint{4.019667in}{2.265848in}}%
\pgfpathlineto{\pgfqpoint{4.020244in}{2.268676in}}%
\pgfpathlineto{\pgfqpoint{4.021012in}{2.268416in}}%
\pgfpathlineto{\pgfqpoint{4.021204in}{2.269889in}}%
\pgfpathlineto{\pgfqpoint{4.022165in}{2.261209in}}%
\pgfpathlineto{\pgfqpoint{4.022934in}{2.262938in}}%
\pgfpathlineto{\pgfqpoint{4.023318in}{2.262791in}}%
\pgfpathlineto{\pgfqpoint{4.023703in}{2.254201in}}%
\pgfpathlineto{\pgfqpoint{4.024663in}{2.256122in}}%
\pgfpathlineto{\pgfqpoint{4.024856in}{2.255620in}}%
\pgfpathlineto{\pgfqpoint{4.026393in}{2.264324in}}%
\pgfpathlineto{\pgfqpoint{4.026585in}{2.262651in}}%
\pgfpathlineto{\pgfqpoint{4.026969in}{2.265965in}}%
\pgfpathlineto{\pgfqpoint{4.027162in}{2.264481in}}%
\pgfpathlineto{\pgfqpoint{4.027354in}{2.267272in}}%
\pgfpathlineto{\pgfqpoint{4.027930in}{2.262235in}}%
\pgfpathlineto{\pgfqpoint{4.028315in}{2.267006in}}%
\pgfpathlineto{\pgfqpoint{4.028507in}{2.267061in}}%
\pgfpathlineto{\pgfqpoint{4.028891in}{2.268568in}}%
\pgfpathlineto{\pgfqpoint{4.030428in}{2.259518in}}%
\pgfpathlineto{\pgfqpoint{4.030621in}{2.262523in}}%
\pgfpathlineto{\pgfqpoint{4.031389in}{2.259480in}}%
\pgfpathlineto{\pgfqpoint{4.031774in}{2.259736in}}%
\pgfpathlineto{\pgfqpoint{4.033503in}{2.245601in}}%
\pgfpathlineto{\pgfqpoint{4.034272in}{2.253415in}}%
\pgfpathlineto{\pgfqpoint{4.034848in}{2.252171in}}%
\pgfpathlineto{\pgfqpoint{4.035425in}{2.252840in}}%
\pgfpathlineto{\pgfqpoint{4.035233in}{2.251367in}}%
\pgfpathlineto{\pgfqpoint{4.035617in}{2.252048in}}%
\pgfpathlineto{\pgfqpoint{4.036193in}{2.253027in}}%
\pgfpathlineto{\pgfqpoint{4.036962in}{2.249225in}}%
\pgfpathlineto{\pgfqpoint{4.037539in}{2.253448in}}%
\pgfpathlineto{\pgfqpoint{4.037923in}{2.249566in}}%
\pgfpathlineto{\pgfqpoint{4.038884in}{2.243991in}}%
\pgfpathlineto{\pgfqpoint{4.039460in}{2.244157in}}%
\pgfpathlineto{\pgfqpoint{4.039845in}{2.248123in}}%
\pgfpathlineto{\pgfqpoint{4.040613in}{2.245284in}}%
\pgfpathlineto{\pgfqpoint{4.040806in}{2.245270in}}%
\pgfpathlineto{\pgfqpoint{4.042535in}{2.256200in}}%
\pgfpathlineto{\pgfqpoint{4.042727in}{2.255323in}}%
\pgfpathlineto{\pgfqpoint{4.043688in}{2.244553in}}%
\pgfpathlineto{\pgfqpoint{4.044072in}{2.245911in}}%
\pgfpathlineto{\pgfqpoint{4.044649in}{2.247224in}}%
\pgfpathlineto{\pgfqpoint{4.044457in}{2.245515in}}%
\pgfpathlineto{\pgfqpoint{4.044841in}{2.246581in}}%
\pgfpathlineto{\pgfqpoint{4.045225in}{2.242396in}}%
\pgfpathlineto{\pgfqpoint{4.045802in}{2.244838in}}%
\pgfpathlineto{\pgfqpoint{4.046186in}{2.250532in}}%
\pgfpathlineto{\pgfqpoint{4.047147in}{2.249243in}}%
\pgfpathlineto{\pgfqpoint{4.048877in}{2.239038in}}%
\pgfpathlineto{\pgfqpoint{4.049453in}{2.238199in}}%
\pgfpathlineto{\pgfqpoint{4.050222in}{2.244063in}}%
\pgfpathlineto{\pgfqpoint{4.051183in}{2.248617in}}%
\pgfpathlineto{\pgfqpoint{4.051567in}{2.246225in}}%
\pgfpathlineto{\pgfqpoint{4.053104in}{2.237142in}}%
\pgfpathlineto{\pgfqpoint{4.053296in}{2.236347in}}%
\pgfpathlineto{\pgfqpoint{4.053489in}{2.237728in}}%
\pgfpathlineto{\pgfqpoint{4.053681in}{2.237722in}}%
\pgfpathlineto{\pgfqpoint{4.055602in}{2.244106in}}%
\pgfpathlineto{\pgfqpoint{4.057332in}{2.238434in}}%
\pgfpathlineto{\pgfqpoint{4.057524in}{2.241369in}}%
\pgfpathlineto{\pgfqpoint{4.058485in}{2.240355in}}%
\pgfpathlineto{\pgfqpoint{4.059830in}{2.233094in}}%
\pgfpathlineto{\pgfqpoint{4.060791in}{2.237737in}}%
\pgfpathlineto{\pgfqpoint{4.060983in}{2.235565in}}%
\pgfpathlineto{\pgfqpoint{4.061175in}{2.232968in}}%
\pgfpathlineto{\pgfqpoint{4.061560in}{2.237267in}}%
\pgfpathlineto{\pgfqpoint{4.061752in}{2.236823in}}%
\pgfpathlineto{\pgfqpoint{4.062713in}{2.245394in}}%
\pgfpathlineto{\pgfqpoint{4.063481in}{2.242238in}}%
\pgfpathlineto{\pgfqpoint{4.066364in}{2.225281in}}%
\pgfpathlineto{\pgfqpoint{4.066556in}{2.226983in}}%
\pgfpathlineto{\pgfqpoint{4.066940in}{2.226907in}}%
\pgfpathlineto{\pgfqpoint{4.067901in}{2.226560in}}%
\pgfpathlineto{\pgfqpoint{4.068286in}{2.231033in}}%
\pgfpathlineto{\pgfqpoint{4.068670in}{2.231597in}}%
\pgfpathlineto{\pgfqpoint{4.068862in}{2.230841in}}%
\pgfpathlineto{\pgfqpoint{4.070399in}{2.238886in}}%
\pgfpathlineto{\pgfqpoint{4.071552in}{2.227482in}}%
\pgfpathlineto{\pgfqpoint{4.071937in}{2.228132in}}%
\pgfpathlineto{\pgfqpoint{4.072129in}{2.229354in}}%
\pgfpathlineto{\pgfqpoint{4.072513in}{2.228069in}}%
\pgfpathlineto{\pgfqpoint{4.073858in}{2.216685in}}%
\pgfpathlineto{\pgfqpoint{4.074435in}{2.220841in}}%
\pgfpathlineto{\pgfqpoint{4.074819in}{2.218949in}}%
\pgfpathlineto{\pgfqpoint{4.075972in}{2.209298in}}%
\pgfpathlineto{\pgfqpoint{4.076741in}{2.209819in}}%
\pgfpathlineto{\pgfqpoint{4.077510in}{2.213411in}}%
\pgfpathlineto{\pgfqpoint{4.078855in}{2.215953in}}%
\pgfpathlineto{\pgfqpoint{4.080008in}{2.214753in}}%
\pgfpathlineto{\pgfqpoint{4.080392in}{2.214232in}}%
\pgfpathlineto{\pgfqpoint{4.081929in}{2.203518in}}%
\pgfpathlineto{\pgfqpoint{4.083275in}{2.208107in}}%
\pgfpathlineto{\pgfqpoint{4.083467in}{2.204437in}}%
\pgfpathlineto{\pgfqpoint{4.084428in}{2.206550in}}%
\pgfpathlineto{\pgfqpoint{4.087118in}{2.223089in}}%
\pgfpathlineto{\pgfqpoint{4.088079in}{2.220805in}}%
\pgfpathlineto{\pgfqpoint{4.088271in}{2.220955in}}%
\pgfpathlineto{\pgfqpoint{4.089232in}{2.214129in}}%
\pgfpathlineto{\pgfqpoint{4.089616in}{2.216309in}}%
\pgfpathlineto{\pgfqpoint{4.090385in}{2.212546in}}%
\pgfpathlineto{\pgfqpoint{4.090769in}{2.214901in}}%
\pgfpathlineto{\pgfqpoint{4.092307in}{2.225439in}}%
\pgfpathlineto{\pgfqpoint{4.093460in}{2.218753in}}%
\pgfpathlineto{\pgfqpoint{4.093844in}{2.221732in}}%
\pgfpathlineto{\pgfqpoint{4.095189in}{2.225214in}}%
\pgfpathlineto{\pgfqpoint{4.095381in}{2.223038in}}%
\pgfpathlineto{\pgfqpoint{4.096150in}{2.226014in}}%
\pgfpathlineto{\pgfqpoint{4.096534in}{2.227440in}}%
\pgfpathlineto{\pgfqpoint{4.096726in}{2.226173in}}%
\pgfpathlineto{\pgfqpoint{4.097879in}{2.222701in}}%
\pgfpathlineto{\pgfqpoint{4.099225in}{2.227368in}}%
\pgfpathlineto{\pgfqpoint{4.100762in}{2.219547in}}%
\pgfpathlineto{\pgfqpoint{4.101146in}{2.219864in}}%
\pgfpathlineto{\pgfqpoint{4.101531in}{2.220979in}}%
\pgfpathlineto{\pgfqpoint{4.101723in}{2.219410in}}%
\pgfpathlineto{\pgfqpoint{4.102684in}{2.215595in}}%
\pgfpathlineto{\pgfqpoint{4.103068in}{2.216385in}}%
\pgfpathlineto{\pgfqpoint{4.103452in}{2.220951in}}%
\pgfpathlineto{\pgfqpoint{4.104029in}{2.217064in}}%
\pgfpathlineto{\pgfqpoint{4.105182in}{2.211409in}}%
\pgfpathlineto{\pgfqpoint{4.106527in}{2.226633in}}%
\pgfpathlineto{\pgfqpoint{4.106719in}{2.224581in}}%
\pgfpathlineto{\pgfqpoint{4.107680in}{2.230501in}}%
\pgfpathlineto{\pgfqpoint{4.108064in}{2.226502in}}%
\pgfpathlineto{\pgfqpoint{4.108256in}{2.224838in}}%
\pgfpathlineto{\pgfqpoint{4.108833in}{2.229231in}}%
\pgfpathlineto{\pgfqpoint{4.109025in}{2.229982in}}%
\pgfpathlineto{\pgfqpoint{4.109409in}{2.228079in}}%
\pgfpathlineto{\pgfqpoint{4.109602in}{2.228946in}}%
\pgfpathlineto{\pgfqpoint{4.109986in}{2.226203in}}%
\pgfpathlineto{\pgfqpoint{4.110562in}{2.228439in}}%
\pgfpathlineto{\pgfqpoint{4.110755in}{2.228725in}}%
\pgfpathlineto{\pgfqpoint{4.112100in}{2.236575in}}%
\pgfpathlineto{\pgfqpoint{4.112292in}{2.236230in}}%
\pgfpathlineto{\pgfqpoint{4.113061in}{2.233758in}}%
\pgfpathlineto{\pgfqpoint{4.113637in}{2.234493in}}%
\pgfpathlineto{\pgfqpoint{4.114406in}{2.238304in}}%
\pgfpathlineto{\pgfqpoint{4.114790in}{2.234955in}}%
\pgfpathlineto{\pgfqpoint{4.115751in}{2.237921in}}%
\pgfpathlineto{\pgfqpoint{4.116135in}{2.237249in}}%
\pgfpathlineto{\pgfqpoint{4.116904in}{2.230490in}}%
\pgfpathlineto{\pgfqpoint{4.117288in}{2.236362in}}%
\pgfpathlineto{\pgfqpoint{4.118634in}{2.242309in}}%
\pgfpathlineto{\pgfqpoint{4.119018in}{2.241540in}}%
\pgfpathlineto{\pgfqpoint{4.119594in}{2.237220in}}%
\pgfpathlineto{\pgfqpoint{4.120171in}{2.240316in}}%
\pgfpathlineto{\pgfqpoint{4.120747in}{2.243153in}}%
\pgfpathlineto{\pgfqpoint{4.120940in}{2.240346in}}%
\pgfpathlineto{\pgfqpoint{4.121324in}{2.238264in}}%
\pgfpathlineto{\pgfqpoint{4.121708in}{2.238959in}}%
\pgfpathlineto{\pgfqpoint{4.123246in}{2.246747in}}%
\pgfpathlineto{\pgfqpoint{4.124014in}{2.252265in}}%
\pgfpathlineto{\pgfqpoint{4.124591in}{2.248593in}}%
\pgfpathlineto{\pgfqpoint{4.124783in}{2.248636in}}%
\pgfpathlineto{\pgfqpoint{4.126320in}{2.238253in}}%
\pgfpathlineto{\pgfqpoint{4.126512in}{2.240360in}}%
\pgfpathlineto{\pgfqpoint{4.127281in}{2.244014in}}%
\pgfpathlineto{\pgfqpoint{4.127665in}{2.242540in}}%
\pgfpathlineto{\pgfqpoint{4.127858in}{2.241060in}}%
\pgfpathlineto{\pgfqpoint{4.128242in}{2.243668in}}%
\pgfpathlineto{\pgfqpoint{4.128818in}{2.241528in}}%
\pgfpathlineto{\pgfqpoint{4.129011in}{2.242588in}}%
\pgfpathlineto{\pgfqpoint{4.129203in}{2.242006in}}%
\pgfpathlineto{\pgfqpoint{4.129395in}{2.237814in}}%
\pgfpathlineto{\pgfqpoint{4.130356in}{2.240538in}}%
\pgfpathlineto{\pgfqpoint{4.130932in}{2.236496in}}%
\pgfpathlineto{\pgfqpoint{4.131317in}{2.241161in}}%
\pgfpathlineto{\pgfqpoint{4.131509in}{2.243042in}}%
\pgfpathlineto{\pgfqpoint{4.131701in}{2.241064in}}%
\pgfpathlineto{\pgfqpoint{4.132085in}{2.241513in}}%
\pgfpathlineto{\pgfqpoint{4.133430in}{2.230288in}}%
\pgfpathlineto{\pgfqpoint{4.133815in}{2.231692in}}%
\pgfpathlineto{\pgfqpoint{4.135160in}{2.234768in}}%
\pgfpathlineto{\pgfqpoint{4.135544in}{2.232873in}}%
\pgfpathlineto{\pgfqpoint{4.135929in}{2.235084in}}%
\pgfpathlineto{\pgfqpoint{4.136121in}{2.234897in}}%
\pgfpathlineto{\pgfqpoint{4.137466in}{2.241137in}}%
\pgfpathlineto{\pgfqpoint{4.137850in}{2.243004in}}%
\pgfpathlineto{\pgfqpoint{4.138235in}{2.240808in}}%
\pgfpathlineto{\pgfqpoint{4.138427in}{2.241511in}}%
\pgfpathlineto{\pgfqpoint{4.139964in}{2.234089in}}%
\pgfpathlineto{\pgfqpoint{4.140156in}{2.234055in}}%
\pgfpathlineto{\pgfqpoint{4.141886in}{2.226492in}}%
\pgfpathlineto{\pgfqpoint{4.142847in}{2.231490in}}%
\pgfpathlineto{\pgfqpoint{4.143231in}{2.230782in}}%
\pgfpathlineto{\pgfqpoint{4.144384in}{2.236521in}}%
\pgfpathlineto{\pgfqpoint{4.145345in}{2.233324in}}%
\pgfpathlineto{\pgfqpoint{4.145537in}{2.235921in}}%
\pgfpathlineto{\pgfqpoint{4.146690in}{2.232492in}}%
\pgfpathlineto{\pgfqpoint{4.146882in}{2.232668in}}%
\pgfpathlineto{\pgfqpoint{4.148420in}{2.242659in}}%
\pgfpathlineto{\pgfqpoint{4.148612in}{2.240997in}}%
\pgfpathlineto{\pgfqpoint{4.149380in}{2.243638in}}%
\pgfpathlineto{\pgfqpoint{4.151110in}{2.254558in}}%
\pgfpathlineto{\pgfqpoint{4.151302in}{2.254144in}}%
\pgfpathlineto{\pgfqpoint{4.152839in}{2.262463in}}%
\pgfpathlineto{\pgfqpoint{4.153032in}{2.262969in}}%
\pgfpathlineto{\pgfqpoint{4.154569in}{2.248918in}}%
\pgfpathlineto{\pgfqpoint{4.154761in}{2.249243in}}%
\pgfpathlineto{\pgfqpoint{4.154953in}{2.247546in}}%
\pgfpathlineto{\pgfqpoint{4.155530in}{2.247926in}}%
\pgfpathlineto{\pgfqpoint{4.156491in}{2.240394in}}%
\pgfpathlineto{\pgfqpoint{4.155914in}{2.248754in}}%
\pgfpathlineto{\pgfqpoint{4.157067in}{2.241549in}}%
\pgfpathlineto{\pgfqpoint{4.158797in}{2.253305in}}%
\pgfpathlineto{\pgfqpoint{4.159181in}{2.254631in}}%
\pgfpathlineto{\pgfqpoint{4.159373in}{2.252904in}}%
\pgfpathlineto{\pgfqpoint{4.159565in}{2.254127in}}%
\pgfpathlineto{\pgfqpoint{4.159757in}{2.252317in}}%
\pgfpathlineto{\pgfqpoint{4.160142in}{2.256109in}}%
\pgfpathlineto{\pgfqpoint{4.160718in}{2.252550in}}%
\pgfpathlineto{\pgfqpoint{4.161103in}{2.250320in}}%
\pgfpathlineto{\pgfqpoint{4.161487in}{2.252916in}}%
\pgfpathlineto{\pgfqpoint{4.161679in}{2.253599in}}%
\pgfpathlineto{\pgfqpoint{4.162063in}{2.251234in}}%
\pgfpathlineto{\pgfqpoint{4.162256in}{2.249170in}}%
\pgfpathlineto{\pgfqpoint{4.162640in}{2.253026in}}%
\pgfpathlineto{\pgfqpoint{4.162832in}{2.252698in}}%
\pgfpathlineto{\pgfqpoint{4.163409in}{2.259573in}}%
\pgfpathlineto{\pgfqpoint{4.164177in}{2.254495in}}%
\pgfpathlineto{\pgfqpoint{4.164370in}{2.252861in}}%
\pgfpathlineto{\pgfqpoint{4.164946in}{2.256515in}}%
\pgfpathlineto{\pgfqpoint{4.165330in}{2.256220in}}%
\pgfpathlineto{\pgfqpoint{4.165523in}{2.257971in}}%
\pgfpathlineto{\pgfqpoint{4.166099in}{2.254624in}}%
\pgfpathlineto{\pgfqpoint{4.166291in}{2.252232in}}%
\pgfpathlineto{\pgfqpoint{4.167252in}{2.253017in}}%
\pgfpathlineto{\pgfqpoint{4.167636in}{2.254347in}}%
\pgfpathlineto{\pgfqpoint{4.168021in}{2.251377in}}%
\pgfpathlineto{\pgfqpoint{4.168789in}{2.251586in}}%
\pgfpathlineto{\pgfqpoint{4.169174in}{2.249968in}}%
\pgfpathlineto{\pgfqpoint{4.172633in}{2.268667in}}%
\pgfpathlineto{\pgfqpoint{4.173978in}{2.261161in}}%
\pgfpathlineto{\pgfqpoint{4.174170in}{2.262162in}}%
\pgfpathlineto{\pgfqpoint{4.175515in}{2.269954in}}%
\pgfpathlineto{\pgfqpoint{4.175707in}{2.269433in}}%
\pgfpathlineto{\pgfqpoint{4.176476in}{2.273247in}}%
\pgfpathlineto{\pgfqpoint{4.176668in}{2.270367in}}%
\pgfpathlineto{\pgfqpoint{4.177053in}{2.265716in}}%
\pgfpathlineto{\pgfqpoint{4.177821in}{2.267032in}}%
\pgfpathlineto{\pgfqpoint{4.178013in}{2.267141in}}%
\pgfpathlineto{\pgfqpoint{4.178206in}{2.266517in}}%
\pgfpathlineto{\pgfqpoint{4.179551in}{2.260024in}}%
\pgfpathlineto{\pgfqpoint{4.179935in}{2.261006in}}%
\pgfpathlineto{\pgfqpoint{4.180512in}{2.256993in}}%
\pgfpathlineto{\pgfqpoint{4.180704in}{2.255119in}}%
\pgfpathlineto{\pgfqpoint{4.181472in}{2.258052in}}%
\pgfpathlineto{\pgfqpoint{4.181665in}{2.258419in}}%
\pgfpathlineto{\pgfqpoint{4.181857in}{2.258006in}}%
\pgfpathlineto{\pgfqpoint{4.183202in}{2.263403in}}%
\pgfpathlineto{\pgfqpoint{4.184355in}{2.271754in}}%
\pgfpathlineto{\pgfqpoint{4.184547in}{2.269546in}}%
\pgfpathlineto{\pgfqpoint{4.184931in}{2.266135in}}%
\pgfpathlineto{\pgfqpoint{4.185316in}{2.270474in}}%
\pgfpathlineto{\pgfqpoint{4.185700in}{2.267958in}}%
\pgfpathlineto{\pgfqpoint{4.186084in}{2.267031in}}%
\pgfpathlineto{\pgfqpoint{4.187238in}{2.270704in}}%
\pgfpathlineto{\pgfqpoint{4.187430in}{2.269199in}}%
\pgfpathlineto{\pgfqpoint{4.187622in}{2.271646in}}%
\pgfpathlineto{\pgfqpoint{4.188391in}{2.269851in}}%
\pgfpathlineto{\pgfqpoint{4.188583in}{2.270161in}}%
\pgfpathlineto{\pgfqpoint{4.188775in}{2.268380in}}%
\pgfpathlineto{\pgfqpoint{4.189159in}{2.264611in}}%
\pgfpathlineto{\pgfqpoint{4.189544in}{2.268588in}}%
\pgfpathlineto{\pgfqpoint{4.190889in}{2.274718in}}%
\pgfpathlineto{\pgfqpoint{4.191081in}{2.275429in}}%
\pgfpathlineto{\pgfqpoint{4.191273in}{2.272849in}}%
\pgfpathlineto{\pgfqpoint{4.192426in}{2.266553in}}%
\pgfpathlineto{\pgfqpoint{4.192618in}{2.268475in}}%
\pgfpathlineto{\pgfqpoint{4.193195in}{2.274945in}}%
\pgfpathlineto{\pgfqpoint{4.193963in}{2.274416in}}%
\pgfpathlineto{\pgfqpoint{4.194540in}{2.273145in}}%
\pgfpathlineto{\pgfqpoint{4.196846in}{2.284000in}}%
\pgfpathlineto{\pgfqpoint{4.197038in}{2.282966in}}%
\pgfpathlineto{\pgfqpoint{4.197422in}{2.285948in}}%
\pgfpathlineto{\pgfqpoint{4.197807in}{2.287114in}}%
\pgfpathlineto{\pgfqpoint{4.199152in}{2.293935in}}%
\pgfpathlineto{\pgfqpoint{4.199536in}{2.294761in}}%
\pgfpathlineto{\pgfqpoint{4.199728in}{2.298821in}}%
\pgfpathlineto{\pgfqpoint{4.200689in}{2.295647in}}%
\pgfpathlineto{\pgfqpoint{4.200881in}{2.294192in}}%
\pgfpathlineto{\pgfqpoint{4.201266in}{2.297986in}}%
\pgfpathlineto{\pgfqpoint{4.201842in}{2.301389in}}%
\pgfpathlineto{\pgfqpoint{4.202034in}{2.299401in}}%
\pgfpathlineto{\pgfqpoint{4.203572in}{2.283821in}}%
\pgfpathlineto{\pgfqpoint{4.203764in}{2.286100in}}%
\pgfpathlineto{\pgfqpoint{4.205301in}{2.299575in}}%
\pgfpathlineto{\pgfqpoint{4.206454in}{2.298631in}}%
\pgfpathlineto{\pgfqpoint{4.207992in}{2.317521in}}%
\pgfpathlineto{\pgfqpoint{4.208184in}{2.313980in}}%
\pgfpathlineto{\pgfqpoint{4.208952in}{2.315685in}}%
\pgfpathlineto{\pgfqpoint{4.209913in}{2.320886in}}%
\pgfpathlineto{\pgfqpoint{4.210298in}{2.318669in}}%
\pgfpathlineto{\pgfqpoint{4.211259in}{2.315160in}}%
\pgfpathlineto{\pgfqpoint{4.211451in}{2.317403in}}%
\pgfpathlineto{\pgfqpoint{4.212412in}{2.313930in}}%
\pgfpathlineto{\pgfqpoint{4.212604in}{2.314951in}}%
\pgfpathlineto{\pgfqpoint{4.213180in}{2.318624in}}%
\pgfpathlineto{\pgfqpoint{4.213565in}{2.314611in}}%
\pgfpathlineto{\pgfqpoint{4.213757in}{2.315559in}}%
\pgfpathlineto{\pgfqpoint{4.214525in}{2.312374in}}%
\pgfpathlineto{\pgfqpoint{4.214718in}{2.309574in}}%
\pgfpathlineto{\pgfqpoint{4.215486in}{2.310706in}}%
\pgfpathlineto{\pgfqpoint{4.215678in}{2.314339in}}%
\pgfpathlineto{\pgfqpoint{4.216447in}{2.307855in}}%
\pgfpathlineto{\pgfqpoint{4.217792in}{2.313686in}}%
\pgfpathlineto{\pgfqpoint{4.217984in}{2.312304in}}%
\pgfpathlineto{\pgfqpoint{4.218177in}{2.309834in}}%
\pgfpathlineto{\pgfqpoint{4.218369in}{2.313887in}}%
\pgfpathlineto{\pgfqpoint{4.218945in}{2.313207in}}%
\pgfpathlineto{\pgfqpoint{4.220675in}{2.321378in}}%
\pgfpathlineto{\pgfqpoint{4.221443in}{2.319813in}}%
\pgfpathlineto{\pgfqpoint{4.221059in}{2.321878in}}%
\pgfpathlineto{\pgfqpoint{4.221636in}{2.321695in}}%
\pgfpathlineto{\pgfqpoint{4.221828in}{2.322418in}}%
\pgfpathlineto{\pgfqpoint{4.222212in}{2.318557in}}%
\pgfpathlineto{\pgfqpoint{4.222596in}{2.325017in}}%
\pgfpathlineto{\pgfqpoint{4.222789in}{2.324651in}}%
\pgfpathlineto{\pgfqpoint{4.223557in}{2.326168in}}%
\pgfpathlineto{\pgfqpoint{4.223749in}{2.324101in}}%
\pgfpathlineto{\pgfqpoint{4.223942in}{2.324409in}}%
\pgfpathlineto{\pgfqpoint{4.224710in}{2.331970in}}%
\pgfpathlineto{\pgfqpoint{4.225095in}{2.326475in}}%
\pgfpathlineto{\pgfqpoint{4.225479in}{2.323409in}}%
\pgfpathlineto{\pgfqpoint{4.225863in}{2.327303in}}%
\pgfpathlineto{\pgfqpoint{4.226055in}{2.326801in}}%
\pgfpathlineto{\pgfqpoint{4.226632in}{2.329977in}}%
\pgfpathlineto{\pgfqpoint{4.226824in}{2.325343in}}%
\pgfpathlineto{\pgfqpoint{4.227016in}{2.324673in}}%
\pgfpathlineto{\pgfqpoint{4.227401in}{2.326936in}}%
\pgfpathlineto{\pgfqpoint{4.227785in}{2.325869in}}%
\pgfpathlineto{\pgfqpoint{4.227977in}{2.326156in}}%
\pgfpathlineto{\pgfqpoint{4.228361in}{2.319331in}}%
\pgfpathlineto{\pgfqpoint{4.228938in}{2.324540in}}%
\pgfpathlineto{\pgfqpoint{4.229707in}{2.334210in}}%
\pgfpathlineto{\pgfqpoint{4.230283in}{2.330960in}}%
\pgfpathlineto{\pgfqpoint{4.231244in}{2.333714in}}%
\pgfpathlineto{\pgfqpoint{4.231436in}{2.333215in}}%
\pgfpathlineto{\pgfqpoint{4.231628in}{2.333598in}}%
\pgfpathlineto{\pgfqpoint{4.232589in}{2.327893in}}%
\pgfpathlineto{\pgfqpoint{4.232781in}{2.330317in}}%
\pgfpathlineto{\pgfqpoint{4.234511in}{2.345049in}}%
\pgfpathlineto{\pgfqpoint{4.233166in}{2.329782in}}%
\pgfpathlineto{\pgfqpoint{4.235472in}{2.339032in}}%
\pgfpathlineto{\pgfqpoint{4.236817in}{2.336917in}}%
\pgfpathlineto{\pgfqpoint{4.237586in}{2.335403in}}%
\pgfpathlineto{\pgfqpoint{4.239507in}{2.317244in}}%
\pgfpathlineto{\pgfqpoint{4.239699in}{2.314611in}}%
\pgfpathlineto{\pgfqpoint{4.240468in}{2.315471in}}%
\pgfpathlineto{\pgfqpoint{4.242198in}{2.328116in}}%
\pgfpathlineto{\pgfqpoint{4.242390in}{2.326563in}}%
\pgfpathlineto{\pgfqpoint{4.244696in}{2.314504in}}%
\pgfpathlineto{\pgfqpoint{4.245464in}{2.320504in}}%
\pgfpathlineto{\pgfqpoint{4.245849in}{2.315897in}}%
\pgfpathlineto{\pgfqpoint{4.246041in}{2.313848in}}%
\pgfpathlineto{\pgfqpoint{4.246425in}{2.317105in}}%
\pgfpathlineto{\pgfqpoint{4.247002in}{2.314527in}}%
\pgfpathlineto{\pgfqpoint{4.247194in}{2.316844in}}%
\pgfpathlineto{\pgfqpoint{4.247770in}{2.313729in}}%
\pgfpathlineto{\pgfqpoint{4.247963in}{2.312589in}}%
\pgfpathlineto{\pgfqpoint{4.248155in}{2.317791in}}%
\pgfpathlineto{\pgfqpoint{4.248539in}{2.321615in}}%
\pgfpathlineto{\pgfqpoint{4.249116in}{2.317060in}}%
\pgfpathlineto{\pgfqpoint{4.250653in}{2.306599in}}%
\pgfpathlineto{\pgfqpoint{4.251229in}{2.310333in}}%
\pgfpathlineto{\pgfqpoint{4.251806in}{2.308115in}}%
\pgfpathlineto{\pgfqpoint{4.251998in}{2.306521in}}%
\pgfpathlineto{\pgfqpoint{4.252575in}{2.308241in}}%
\pgfpathlineto{\pgfqpoint{4.253151in}{2.310408in}}%
\pgfpathlineto{\pgfqpoint{4.253343in}{2.307046in}}%
\pgfpathlineto{\pgfqpoint{4.253728in}{2.308966in}}%
\pgfpathlineto{\pgfqpoint{4.254304in}{2.311655in}}%
\pgfpathlineto{\pgfqpoint{4.254496in}{2.314316in}}%
\pgfpathlineto{\pgfqpoint{4.254881in}{2.311798in}}%
\pgfpathlineto{\pgfqpoint{4.255649in}{2.302654in}}%
\pgfpathlineto{\pgfqpoint{4.256034in}{2.307146in}}%
\pgfpathlineto{\pgfqpoint{4.256226in}{2.308750in}}%
\pgfpathlineto{\pgfqpoint{4.256802in}{2.306754in}}%
\pgfpathlineto{\pgfqpoint{4.258147in}{2.299275in}}%
\pgfpathlineto{\pgfqpoint{4.258532in}{2.297215in}}%
\pgfpathlineto{\pgfqpoint{4.259108in}{2.299079in}}%
\pgfpathlineto{\pgfqpoint{4.259877in}{2.302422in}}%
\pgfpathlineto{\pgfqpoint{4.260069in}{2.299128in}}%
\pgfpathlineto{\pgfqpoint{4.260646in}{2.299631in}}%
\pgfpathlineto{\pgfqpoint{4.261030in}{2.296179in}}%
\pgfpathlineto{\pgfqpoint{4.261607in}{2.302623in}}%
\pgfpathlineto{\pgfqpoint{4.262375in}{2.300782in}}%
\pgfpathlineto{\pgfqpoint{4.262567in}{2.300898in}}%
\pgfpathlineto{\pgfqpoint{4.263720in}{2.298188in}}%
\pgfpathlineto{\pgfqpoint{4.263336in}{2.301183in}}%
\pgfpathlineto{\pgfqpoint{4.263913in}{2.298259in}}%
\pgfpathlineto{\pgfqpoint{4.264105in}{2.296499in}}%
\pgfpathlineto{\pgfqpoint{4.264873in}{2.299689in}}%
\pgfpathlineto{\pgfqpoint{4.265066in}{2.297410in}}%
\pgfpathlineto{\pgfqpoint{4.265258in}{2.297723in}}%
\pgfpathlineto{\pgfqpoint{4.265450in}{2.301292in}}%
\pgfpathlineto{\pgfqpoint{4.266219in}{2.295005in}}%
\pgfpathlineto{\pgfqpoint{4.267948in}{2.284328in}}%
\pgfpathlineto{\pgfqpoint{4.268332in}{2.275621in}}%
\pgfpathlineto{\pgfqpoint{4.269293in}{2.278348in}}%
\pgfpathlineto{\pgfqpoint{4.271407in}{2.294142in}}%
\pgfpathlineto{\pgfqpoint{4.271599in}{2.293670in}}%
\pgfpathlineto{\pgfqpoint{4.272560in}{2.293153in}}%
\pgfpathlineto{\pgfqpoint{4.272944in}{2.295040in}}%
\pgfpathlineto{\pgfqpoint{4.273905in}{2.295442in}}%
\pgfpathlineto{\pgfqpoint{4.274290in}{2.292330in}}%
\pgfpathlineto{\pgfqpoint{4.274482in}{2.287855in}}%
\pgfpathlineto{\pgfqpoint{4.275058in}{2.293608in}}%
\pgfpathlineto{\pgfqpoint{4.275443in}{2.290518in}}%
\pgfpathlineto{\pgfqpoint{4.275635in}{2.290432in}}%
\pgfpathlineto{\pgfqpoint{4.276211in}{2.285383in}}%
\pgfpathlineto{\pgfqpoint{4.276788in}{2.289569in}}%
\pgfpathlineto{\pgfqpoint{4.277172in}{2.291424in}}%
\pgfpathlineto{\pgfqpoint{4.277556in}{2.288691in}}%
\pgfpathlineto{\pgfqpoint{4.277941in}{2.285281in}}%
\pgfpathlineto{\pgfqpoint{4.278709in}{2.286691in}}%
\pgfpathlineto{\pgfqpoint{4.278902in}{2.288556in}}%
\pgfpathlineto{\pgfqpoint{4.279286in}{2.284389in}}%
\pgfpathlineto{\pgfqpoint{4.279670in}{2.281987in}}%
\pgfpathlineto{\pgfqpoint{4.280247in}{2.283941in}}%
\pgfpathlineto{\pgfqpoint{4.280439in}{2.284280in}}%
\pgfpathlineto{\pgfqpoint{4.281015in}{2.284398in}}%
\pgfpathlineto{\pgfqpoint{4.281592in}{2.280481in}}%
\pgfpathlineto{\pgfqpoint{4.284282in}{2.293942in}}%
\pgfpathlineto{\pgfqpoint{4.284859in}{2.292592in}}%
\pgfpathlineto{\pgfqpoint{4.285051in}{2.291165in}}%
\pgfpathlineto{\pgfqpoint{4.285820in}{2.291670in}}%
\pgfpathlineto{\pgfqpoint{4.286588in}{2.296392in}}%
\pgfpathlineto{\pgfqpoint{4.286781in}{2.295049in}}%
\pgfpathlineto{\pgfqpoint{4.287549in}{2.288349in}}%
\pgfpathlineto{\pgfqpoint{4.287934in}{2.293476in}}%
\pgfpathlineto{\pgfqpoint{4.288126in}{2.296516in}}%
\pgfpathlineto{\pgfqpoint{4.288510in}{2.290744in}}%
\pgfpathlineto{\pgfqpoint{4.289087in}{2.294754in}}%
\pgfpathlineto{\pgfqpoint{4.289279in}{2.296401in}}%
\pgfpathlineto{\pgfqpoint{4.289663in}{2.292930in}}%
\pgfpathlineto{\pgfqpoint{4.289855in}{2.290708in}}%
\pgfpathlineto{\pgfqpoint{4.290624in}{2.294548in}}%
\pgfpathlineto{\pgfqpoint{4.291008in}{2.292558in}}%
\pgfpathlineto{\pgfqpoint{4.291777in}{2.293644in}}%
\pgfpathlineto{\pgfqpoint{4.292546in}{2.297883in}}%
\pgfpathlineto{\pgfqpoint{4.292930in}{2.296001in}}%
\pgfpathlineto{\pgfqpoint{4.294275in}{2.300548in}}%
\pgfpathlineto{\pgfqpoint{4.293891in}{2.295686in}}%
\pgfpathlineto{\pgfqpoint{4.294467in}{2.299077in}}%
\pgfpathlineto{\pgfqpoint{4.295812in}{2.301096in}}%
\pgfpathlineto{\pgfqpoint{4.296005in}{2.299330in}}%
\pgfpathlineto{\pgfqpoint{4.296389in}{2.301965in}}%
\pgfpathlineto{\pgfqpoint{4.296965in}{2.307600in}}%
\pgfpathlineto{\pgfqpoint{4.297542in}{2.304048in}}%
\pgfpathlineto{\pgfqpoint{4.299464in}{2.297585in}}%
\pgfpathlineto{\pgfqpoint{4.300040in}{2.302842in}}%
\pgfpathlineto{\pgfqpoint{4.300424in}{2.299997in}}%
\pgfpathlineto{\pgfqpoint{4.300617in}{2.297893in}}%
\pgfpathlineto{\pgfqpoint{4.301385in}{2.300987in}}%
\pgfpathlineto{\pgfqpoint{4.301962in}{2.306563in}}%
\pgfpathlineto{\pgfqpoint{4.302730in}{2.304125in}}%
\pgfpathlineto{\pgfqpoint{4.304844in}{2.293469in}}%
\pgfpathlineto{\pgfqpoint{4.307535in}{2.312634in}}%
\pgfpathlineto{\pgfqpoint{4.307727in}{2.309113in}}%
\pgfpathlineto{\pgfqpoint{4.308495in}{2.313492in}}%
\pgfpathlineto{\pgfqpoint{4.308880in}{2.319524in}}%
\pgfpathlineto{\pgfqpoint{4.310033in}{2.316400in}}%
\pgfpathlineto{\pgfqpoint{4.310609in}{2.314425in}}%
\pgfpathlineto{\pgfqpoint{4.311186in}{2.314929in}}%
\pgfpathlineto{\pgfqpoint{4.314068in}{2.324914in}}%
\pgfpathlineto{\pgfqpoint{4.314645in}{2.319299in}}%
\pgfpathlineto{\pgfqpoint{4.315221in}{2.322579in}}%
\pgfpathlineto{\pgfqpoint{4.316951in}{2.329219in}}%
\pgfpathlineto{\pgfqpoint{4.317143in}{2.328309in}}%
\pgfpathlineto{\pgfqpoint{4.318104in}{2.320873in}}%
\pgfpathlineto{\pgfqpoint{4.318488in}{2.322713in}}%
\pgfpathlineto{\pgfqpoint{4.319641in}{2.329019in}}%
\pgfpathlineto{\pgfqpoint{4.320026in}{2.327063in}}%
\pgfpathlineto{\pgfqpoint{4.320410in}{2.325445in}}%
\pgfpathlineto{\pgfqpoint{4.320794in}{2.328729in}}%
\pgfpathlineto{\pgfqpoint{4.320986in}{2.327624in}}%
\pgfpathlineto{\pgfqpoint{4.321179in}{2.329478in}}%
\pgfpathlineto{\pgfqpoint{4.321755in}{2.324654in}}%
\pgfpathlineto{\pgfqpoint{4.321947in}{2.326341in}}%
\pgfpathlineto{\pgfqpoint{4.325598in}{2.347340in}}%
\pgfpathlineto{\pgfqpoint{4.326751in}{2.344426in}}%
\pgfpathlineto{\pgfqpoint{4.326944in}{2.343641in}}%
\pgfpathlineto{\pgfqpoint{4.327328in}{2.345332in}}%
\pgfpathlineto{\pgfqpoint{4.327712in}{2.347577in}}%
\pgfpathlineto{\pgfqpoint{4.328097in}{2.343592in}}%
\pgfpathlineto{\pgfqpoint{4.328289in}{2.344398in}}%
\pgfpathlineto{\pgfqpoint{4.329250in}{2.345830in}}%
\pgfpathlineto{\pgfqpoint{4.328865in}{2.343965in}}%
\pgfpathlineto{\pgfqpoint{4.329442in}{2.345365in}}%
\pgfpathlineto{\pgfqpoint{4.329634in}{2.344409in}}%
\pgfpathlineto{\pgfqpoint{4.330018in}{2.347322in}}%
\pgfpathlineto{\pgfqpoint{4.330403in}{2.345417in}}%
\pgfpathlineto{\pgfqpoint{4.331171in}{2.344646in}}%
\pgfpathlineto{\pgfqpoint{4.331556in}{2.346254in}}%
\pgfpathlineto{\pgfqpoint{4.332324in}{2.342426in}}%
\pgfpathlineto{\pgfqpoint{4.332516in}{2.342835in}}%
\pgfpathlineto{\pgfqpoint{4.334054in}{2.349297in}}%
\pgfpathlineto{\pgfqpoint{4.334246in}{2.349529in}}%
\pgfpathlineto{\pgfqpoint{4.334630in}{2.346478in}}%
\pgfpathlineto{\pgfqpoint{4.335015in}{2.353156in}}%
\pgfpathlineto{\pgfqpoint{4.335207in}{2.351385in}}%
\pgfpathlineto{\pgfqpoint{4.336360in}{2.347010in}}%
\pgfpathlineto{\pgfqpoint{4.335783in}{2.351935in}}%
\pgfpathlineto{\pgfqpoint{4.336744in}{2.350218in}}%
\pgfpathlineto{\pgfqpoint{4.337129in}{2.356299in}}%
\pgfpathlineto{\pgfqpoint{4.337897in}{2.351874in}}%
\pgfpathlineto{\pgfqpoint{4.338666in}{2.348122in}}%
\pgfpathlineto{\pgfqpoint{4.339050in}{2.350423in}}%
\pgfpathlineto{\pgfqpoint{4.339242in}{2.350025in}}%
\pgfpathlineto{\pgfqpoint{4.339627in}{2.351217in}}%
\pgfpathlineto{\pgfqpoint{4.340588in}{2.354362in}}%
\pgfpathlineto{\pgfqpoint{4.340395in}{2.350355in}}%
\pgfpathlineto{\pgfqpoint{4.340780in}{2.353809in}}%
\pgfpathlineto{\pgfqpoint{4.341164in}{2.350910in}}%
\pgfpathlineto{\pgfqpoint{4.341741in}{2.353841in}}%
\pgfpathlineto{\pgfqpoint{4.342125in}{2.353106in}}%
\pgfpathlineto{\pgfqpoint{4.342509in}{2.354974in}}%
\pgfpathlineto{\pgfqpoint{4.343086in}{2.352237in}}%
\pgfpathlineto{\pgfqpoint{4.343278in}{2.357197in}}%
\pgfpathlineto{\pgfqpoint{4.343662in}{2.358047in}}%
\pgfpathlineto{\pgfqpoint{4.344239in}{2.363154in}}%
\pgfpathlineto{\pgfqpoint{4.344815in}{2.360909in}}%
\pgfpathlineto{\pgfqpoint{4.345200in}{2.357737in}}%
\pgfpathlineto{\pgfqpoint{4.345968in}{2.358761in}}%
\pgfpathlineto{\pgfqpoint{4.347313in}{2.364722in}}%
\pgfpathlineto{\pgfqpoint{4.347506in}{2.363767in}}%
\pgfpathlineto{\pgfqpoint{4.347890in}{2.361976in}}%
\pgfpathlineto{\pgfqpoint{4.348082in}{2.363172in}}%
\pgfpathlineto{\pgfqpoint{4.348274in}{2.364826in}}%
\pgfpathlineto{\pgfqpoint{4.348851in}{2.360815in}}%
\pgfpathlineto{\pgfqpoint{4.349043in}{2.362281in}}%
\pgfpathlineto{\pgfqpoint{4.350196in}{2.364488in}}%
\pgfpathlineto{\pgfqpoint{4.349619in}{2.360024in}}%
\pgfpathlineto{\pgfqpoint{4.350388in}{2.363474in}}%
\pgfpathlineto{\pgfqpoint{4.351157in}{2.363847in}}%
\pgfpathlineto{\pgfqpoint{4.351925in}{2.366707in}}%
\pgfpathlineto{\pgfqpoint{4.352310in}{2.364486in}}%
\pgfpathlineto{\pgfqpoint{4.352502in}{2.364758in}}%
\pgfpathlineto{\pgfqpoint{4.352694in}{2.363598in}}%
\pgfpathlineto{\pgfqpoint{4.353078in}{2.364047in}}%
\pgfpathlineto{\pgfqpoint{4.354231in}{2.358354in}}%
\pgfpathlineto{\pgfqpoint{4.354616in}{2.358784in}}%
\pgfpathlineto{\pgfqpoint{4.355000in}{2.356057in}}%
\pgfpathlineto{\pgfqpoint{4.356537in}{2.348078in}}%
\pgfpathlineto{\pgfqpoint{4.356922in}{2.352076in}}%
\pgfpathlineto{\pgfqpoint{4.357691in}{2.356260in}}%
\pgfpathlineto{\pgfqpoint{4.358075in}{2.353799in}}%
\pgfpathlineto{\pgfqpoint{4.358267in}{2.353609in}}%
\pgfpathlineto{\pgfqpoint{4.358459in}{2.354284in}}%
\pgfpathlineto{\pgfqpoint{4.359420in}{2.350191in}}%
\pgfpathlineto{\pgfqpoint{4.359036in}{2.357471in}}%
\pgfpathlineto{\pgfqpoint{4.359612in}{2.351364in}}%
\pgfpathlineto{\pgfqpoint{4.361534in}{2.356805in}}%
\pgfpathlineto{\pgfqpoint{4.362879in}{2.347068in}}%
\pgfpathlineto{\pgfqpoint{4.364609in}{2.357386in}}%
\pgfpathlineto{\pgfqpoint{4.364801in}{2.358975in}}%
\pgfpathlineto{\pgfqpoint{4.365377in}{2.355090in}}%
\pgfpathlineto{\pgfqpoint{4.365569in}{2.355364in}}%
\pgfpathlineto{\pgfqpoint{4.365762in}{2.354621in}}%
\pgfpathlineto{\pgfqpoint{4.367107in}{2.359457in}}%
\pgfpathlineto{\pgfqpoint{4.367683in}{2.362335in}}%
\pgfpathlineto{\pgfqpoint{4.367491in}{2.358663in}}%
\pgfpathlineto{\pgfqpoint{4.368260in}{2.361810in}}%
\pgfpathlineto{\pgfqpoint{4.368644in}{2.357071in}}%
\pgfpathlineto{\pgfqpoint{4.369221in}{2.361896in}}%
\pgfpathlineto{\pgfqpoint{4.369413in}{2.361121in}}%
\pgfpathlineto{\pgfqpoint{4.369989in}{2.362618in}}%
\pgfpathlineto{\pgfqpoint{4.371334in}{2.369280in}}%
\pgfpathlineto{\pgfqpoint{4.371527in}{2.368149in}}%
\pgfpathlineto{\pgfqpoint{4.372103in}{2.363043in}}%
\pgfpathlineto{\pgfqpoint{4.372872in}{2.364804in}}%
\pgfpathlineto{\pgfqpoint{4.373256in}{2.369071in}}%
\pgfpathlineto{\pgfqpoint{4.374025in}{2.366962in}}%
\pgfpathlineto{\pgfqpoint{4.375370in}{2.352654in}}%
\pgfpathlineto{\pgfqpoint{4.375946in}{2.359628in}}%
\pgfpathlineto{\pgfqpoint{4.377292in}{2.371236in}}%
\pgfpathlineto{\pgfqpoint{4.379021in}{2.362161in}}%
\pgfpathlineto{\pgfqpoint{4.380174in}{2.366235in}}%
\pgfpathlineto{\pgfqpoint{4.380366in}{2.365693in}}%
\pgfpathlineto{\pgfqpoint{4.380751in}{2.363489in}}%
\pgfpathlineto{\pgfqpoint{4.381327in}{2.364938in}}%
\pgfpathlineto{\pgfqpoint{4.382865in}{2.375421in}}%
\pgfpathlineto{\pgfqpoint{4.381904in}{2.364723in}}%
\pgfpathlineto{\pgfqpoint{4.383057in}{2.373630in}}%
\pgfpathlineto{\pgfqpoint{4.383249in}{2.372639in}}%
\pgfpathlineto{\pgfqpoint{4.383633in}{2.376138in}}%
\pgfpathlineto{\pgfqpoint{4.383825in}{2.375740in}}%
\pgfpathlineto{\pgfqpoint{4.384018in}{2.376742in}}%
\pgfpathlineto{\pgfqpoint{4.384594in}{2.376529in}}%
\pgfpathlineto{\pgfqpoint{4.384978in}{2.378106in}}%
\pgfpathlineto{\pgfqpoint{4.385555in}{2.376362in}}%
\pgfpathlineto{\pgfqpoint{4.386324in}{2.372962in}}%
\pgfpathlineto{\pgfqpoint{4.386708in}{2.375133in}}%
\pgfpathlineto{\pgfqpoint{4.387477in}{2.379678in}}%
\pgfpathlineto{\pgfqpoint{4.388053in}{2.377200in}}%
\pgfpathlineto{\pgfqpoint{4.388822in}{2.374302in}}%
\pgfpathlineto{\pgfqpoint{4.389206in}{2.376002in}}%
\pgfpathlineto{\pgfqpoint{4.390167in}{2.380482in}}%
\pgfpathlineto{\pgfqpoint{4.390359in}{2.377547in}}%
\pgfpathlineto{\pgfqpoint{4.390551in}{2.374654in}}%
\pgfpathlineto{\pgfqpoint{4.390936in}{2.379432in}}%
\pgfpathlineto{\pgfqpoint{4.391128in}{2.382760in}}%
\pgfpathlineto{\pgfqpoint{4.391704in}{2.378800in}}%
\pgfpathlineto{\pgfqpoint{4.392089in}{2.380228in}}%
\pgfpathlineto{\pgfqpoint{4.392665in}{2.381964in}}%
\pgfpathlineto{\pgfqpoint{4.393049in}{2.380641in}}%
\pgfpathlineto{\pgfqpoint{4.395740in}{2.366979in}}%
\pgfpathlineto{\pgfqpoint{4.396124in}{2.368558in}}%
\pgfpathlineto{\pgfqpoint{4.397469in}{2.373208in}}%
\pgfpathlineto{\pgfqpoint{4.396508in}{2.368306in}}%
\pgfpathlineto{\pgfqpoint{4.397661in}{2.372223in}}%
\pgfpathlineto{\pgfqpoint{4.398046in}{2.372858in}}%
\pgfpathlineto{\pgfqpoint{4.399775in}{2.354224in}}%
\pgfpathlineto{\pgfqpoint{4.400352in}{2.355675in}}%
\pgfpathlineto{\pgfqpoint{4.400736in}{2.353706in}}%
\pgfpathlineto{\pgfqpoint{4.401313in}{2.345346in}}%
\pgfpathlineto{\pgfqpoint{4.402081in}{2.345851in}}%
\pgfpathlineto{\pgfqpoint{4.402273in}{2.349537in}}%
\pgfpathlineto{\pgfqpoint{4.403042in}{2.345445in}}%
\pgfpathlineto{\pgfqpoint{4.403811in}{2.346343in}}%
\pgfpathlineto{\pgfqpoint{4.404964in}{2.351203in}}%
\pgfpathlineto{\pgfqpoint{4.406117in}{2.342719in}}%
\pgfpathlineto{\pgfqpoint{4.406309in}{2.345880in}}%
\pgfpathlineto{\pgfqpoint{4.407078in}{2.349827in}}%
\pgfpathlineto{\pgfqpoint{4.407462in}{2.346751in}}%
\pgfpathlineto{\pgfqpoint{4.407654in}{2.346837in}}%
\pgfpathlineto{\pgfqpoint{4.409384in}{2.353428in}}%
\pgfpathlineto{\pgfqpoint{4.410345in}{2.349539in}}%
\pgfpathlineto{\pgfqpoint{4.409768in}{2.353856in}}%
\pgfpathlineto{\pgfqpoint{4.410537in}{2.352678in}}%
\pgfpathlineto{\pgfqpoint{4.412074in}{2.361196in}}%
\pgfpathlineto{\pgfqpoint{4.412458in}{2.360640in}}%
\pgfpathlineto{\pgfqpoint{4.412651in}{2.359433in}}%
\pgfpathlineto{\pgfqpoint{4.412843in}{2.363845in}}%
\pgfpathlineto{\pgfqpoint{4.413419in}{2.359906in}}%
\pgfpathlineto{\pgfqpoint{4.414188in}{2.363379in}}%
\pgfpathlineto{\pgfqpoint{4.414380in}{2.362466in}}%
\pgfpathlineto{\pgfqpoint{4.414572in}{2.359719in}}%
\pgfpathlineto{\pgfqpoint{4.415149in}{2.362531in}}%
\pgfpathlineto{\pgfqpoint{4.415341in}{2.360990in}}%
\pgfpathlineto{\pgfqpoint{4.417070in}{2.369814in}}%
\pgfpathlineto{\pgfqpoint{4.417263in}{2.367293in}}%
\pgfpathlineto{\pgfqpoint{4.417839in}{2.364045in}}%
\pgfpathlineto{\pgfqpoint{4.418223in}{2.366489in}}%
\pgfpathlineto{\pgfqpoint{4.418416in}{2.367345in}}%
\pgfpathlineto{\pgfqpoint{4.418992in}{2.366740in}}%
\pgfpathlineto{\pgfqpoint{4.419184in}{2.364633in}}%
\pgfpathlineto{\pgfqpoint{4.419569in}{2.369520in}}%
\pgfpathlineto{\pgfqpoint{4.419761in}{2.368248in}}%
\pgfpathlineto{\pgfqpoint{4.420914in}{2.373141in}}%
\pgfpathlineto{\pgfqpoint{4.421106in}{2.370974in}}%
\pgfpathlineto{\pgfqpoint{4.421490in}{2.369567in}}%
\pgfpathlineto{\pgfqpoint{4.421682in}{2.369976in}}%
\pgfpathlineto{\pgfqpoint{4.422259in}{2.367113in}}%
\pgfpathlineto{\pgfqpoint{4.422451in}{2.371662in}}%
\pgfpathlineto{\pgfqpoint{4.424181in}{2.384515in}}%
\pgfpathlineto{\pgfqpoint{4.426102in}{2.375864in}}%
\pgfpathlineto{\pgfqpoint{4.424565in}{2.385148in}}%
\pgfpathlineto{\pgfqpoint{4.426294in}{2.376047in}}%
\pgfpathlineto{\pgfqpoint{4.426871in}{2.375048in}}%
\pgfpathlineto{\pgfqpoint{4.427832in}{2.379004in}}%
\pgfpathlineto{\pgfqpoint{4.429177in}{2.377259in}}%
\pgfpathlineto{\pgfqpoint{4.428408in}{2.379799in}}%
\pgfpathlineto{\pgfqpoint{4.429561in}{2.377626in}}%
\pgfpathlineto{\pgfqpoint{4.429753in}{2.379743in}}%
\pgfpathlineto{\pgfqpoint{4.430330in}{2.377287in}}%
\pgfpathlineto{\pgfqpoint{4.430714in}{2.371005in}}%
\pgfpathlineto{\pgfqpoint{4.431291in}{2.377774in}}%
\pgfpathlineto{\pgfqpoint{4.432444in}{2.384133in}}%
\pgfpathlineto{\pgfqpoint{4.432828in}{2.381107in}}%
\pgfpathlineto{\pgfqpoint{4.433213in}{2.378468in}}%
\pgfpathlineto{\pgfqpoint{4.433789in}{2.379421in}}%
\pgfpathlineto{\pgfqpoint{4.434173in}{2.381530in}}%
\pgfpathlineto{\pgfqpoint{4.434558in}{2.377581in}}%
\pgfpathlineto{\pgfqpoint{4.434750in}{2.377747in}}%
\pgfpathlineto{\pgfqpoint{4.434942in}{2.376555in}}%
\pgfpathlineto{\pgfqpoint{4.435711in}{2.369840in}}%
\pgfpathlineto{\pgfqpoint{4.436287in}{2.370871in}}%
\pgfpathlineto{\pgfqpoint{4.436672in}{2.370509in}}%
\pgfpathlineto{\pgfqpoint{4.436864in}{2.371395in}}%
\pgfpathlineto{\pgfqpoint{4.437056in}{2.369475in}}%
\pgfpathlineto{\pgfqpoint{4.437440in}{2.372564in}}%
\pgfpathlineto{\pgfqpoint{4.437632in}{2.377047in}}%
\pgfpathlineto{\pgfqpoint{4.438401in}{2.372094in}}%
\pgfpathlineto{\pgfqpoint{4.438978in}{2.374154in}}%
\pgfpathlineto{\pgfqpoint{4.439170in}{2.376490in}}%
\pgfpathlineto{\pgfqpoint{4.439554in}{2.371663in}}%
\pgfpathlineto{\pgfqpoint{4.439746in}{2.374883in}}%
\pgfpathlineto{\pgfqpoint{4.440131in}{2.370556in}}%
\pgfpathlineto{\pgfqpoint{4.440899in}{2.372415in}}%
\pgfpathlineto{\pgfqpoint{4.441860in}{2.378877in}}%
\pgfpathlineto{\pgfqpoint{4.443397in}{2.389669in}}%
\pgfpathlineto{\pgfqpoint{4.444166in}{2.379389in}}%
\pgfpathlineto{\pgfqpoint{4.444743in}{2.386381in}}%
\pgfpathlineto{\pgfqpoint{4.446472in}{2.381074in}}%
\pgfpathlineto{\pgfqpoint{4.447241in}{2.384020in}}%
\pgfpathlineto{\pgfqpoint{4.447433in}{2.382021in}}%
\pgfpathlineto{\pgfqpoint{4.448394in}{2.375337in}}%
\pgfpathlineto{\pgfqpoint{4.448778in}{2.379707in}}%
\pgfpathlineto{\pgfqpoint{4.449162in}{2.380480in}}%
\pgfpathlineto{\pgfqpoint{4.450315in}{2.374937in}}%
\pgfpathlineto{\pgfqpoint{4.451853in}{2.384992in}}%
\pgfpathlineto{\pgfqpoint{4.452045in}{2.384529in}}%
\pgfpathlineto{\pgfqpoint{4.452237in}{2.387854in}}%
\pgfpathlineto{\pgfqpoint{4.452621in}{2.384432in}}%
\pgfpathlineto{\pgfqpoint{4.453006in}{2.386243in}}%
\pgfpathlineto{\pgfqpoint{4.454735in}{2.376774in}}%
\pgfpathlineto{\pgfqpoint{4.453390in}{2.387281in}}%
\pgfpathlineto{\pgfqpoint{4.455312in}{2.377534in}}%
\pgfpathlineto{\pgfqpoint{4.456081in}{2.379778in}}%
\pgfpathlineto{\pgfqpoint{4.456465in}{2.378723in}}%
\pgfpathlineto{\pgfqpoint{4.457041in}{2.379256in}}%
\pgfpathlineto{\pgfqpoint{4.458002in}{2.373946in}}%
\pgfpathlineto{\pgfqpoint{4.458963in}{2.365031in}}%
\pgfpathlineto{\pgfqpoint{4.459155in}{2.365502in}}%
\pgfpathlineto{\pgfqpoint{4.460693in}{2.372966in}}%
\pgfpathlineto{\pgfqpoint{4.460885in}{2.370213in}}%
\pgfpathlineto{\pgfqpoint{4.461461in}{2.377231in}}%
\pgfpathlineto{\pgfqpoint{4.461653in}{2.375033in}}%
\pgfpathlineto{\pgfqpoint{4.462038in}{2.375190in}}%
\pgfpathlineto{\pgfqpoint{4.462999in}{2.369002in}}%
\pgfpathlineto{\pgfqpoint{4.463383in}{2.364241in}}%
\pgfpathlineto{\pgfqpoint{4.464152in}{2.367058in}}%
\pgfpathlineto{\pgfqpoint{4.464920in}{2.366714in}}%
\pgfpathlineto{\pgfqpoint{4.465305in}{2.368419in}}%
\pgfpathlineto{\pgfqpoint{4.465689in}{2.366822in}}%
\pgfpathlineto{\pgfqpoint{4.466265in}{2.369000in}}%
\pgfpathlineto{\pgfqpoint{4.467611in}{2.371060in}}%
\pgfpathlineto{\pgfqpoint{4.467803in}{2.369426in}}%
\pgfpathlineto{\pgfqpoint{4.468379in}{2.372977in}}%
\pgfpathlineto{\pgfqpoint{4.469917in}{2.379373in}}%
\pgfpathlineto{\pgfqpoint{4.470109in}{2.378310in}}%
\pgfpathlineto{\pgfqpoint{4.470301in}{2.381089in}}%
\pgfpathlineto{\pgfqpoint{4.471262in}{2.387137in}}%
\pgfpathlineto{\pgfqpoint{4.471646in}{2.385011in}}%
\pgfpathlineto{\pgfqpoint{4.472415in}{2.388258in}}%
\pgfpathlineto{\pgfqpoint{4.472799in}{2.387329in}}%
\pgfpathlineto{\pgfqpoint{4.473568in}{2.382364in}}%
\pgfpathlineto{\pgfqpoint{4.473952in}{2.382747in}}%
\pgfpathlineto{\pgfqpoint{4.474721in}{2.390118in}}%
\pgfpathlineto{\pgfqpoint{4.475297in}{2.385693in}}%
\pgfpathlineto{\pgfqpoint{4.475489in}{2.385360in}}%
\pgfpathlineto{\pgfqpoint{4.475874in}{2.381615in}}%
\pgfpathlineto{\pgfqpoint{4.476258in}{2.387576in}}%
\pgfpathlineto{\pgfqpoint{4.476835in}{2.386819in}}%
\pgfpathlineto{\pgfqpoint{4.477603in}{2.389963in}}%
\pgfpathlineto{\pgfqpoint{4.477988in}{2.386604in}}%
\pgfpathlineto{\pgfqpoint{4.478756in}{2.388413in}}%
\pgfpathlineto{\pgfqpoint{4.478948in}{2.390034in}}%
\pgfpathlineto{\pgfqpoint{4.479525in}{2.387176in}}%
\pgfpathlineto{\pgfqpoint{4.479717in}{2.387551in}}%
\pgfpathlineto{\pgfqpoint{4.479909in}{2.387555in}}%
\pgfpathlineto{\pgfqpoint{4.480870in}{2.392355in}}%
\pgfpathlineto{\pgfqpoint{4.481255in}{2.390042in}}%
\pgfpathlineto{\pgfqpoint{4.481639in}{2.389227in}}%
\pgfpathlineto{\pgfqpoint{4.483945in}{2.395169in}}%
\pgfpathlineto{\pgfqpoint{4.484329in}{2.398410in}}%
\pgfpathlineto{\pgfqpoint{4.485098in}{2.396797in}}%
\pgfpathlineto{\pgfqpoint{4.486251in}{2.393373in}}%
\pgfpathlineto{\pgfqpoint{4.488557in}{2.418338in}}%
\pgfpathlineto{\pgfqpoint{4.490671in}{2.409460in}}%
\pgfpathlineto{\pgfqpoint{4.491055in}{2.410485in}}%
\pgfpathlineto{\pgfqpoint{4.492016in}{2.417039in}}%
\pgfpathlineto{\pgfqpoint{4.492208in}{2.414374in}}%
\pgfpathlineto{\pgfqpoint{4.493745in}{2.405372in}}%
\pgfpathlineto{\pgfqpoint{4.494322in}{2.407646in}}%
\pgfpathlineto{\pgfqpoint{4.495091in}{2.407467in}}%
\pgfpathlineto{\pgfqpoint{4.495475in}{2.406045in}}%
\pgfpathlineto{\pgfqpoint{4.496436in}{2.400731in}}%
\pgfpathlineto{\pgfqpoint{4.496628in}{2.402911in}}%
\pgfpathlineto{\pgfqpoint{4.497012in}{2.406887in}}%
\pgfpathlineto{\pgfqpoint{4.497781in}{2.404322in}}%
\pgfpathlineto{\pgfqpoint{4.498165in}{2.405981in}}%
\pgfpathlineto{\pgfqpoint{4.498357in}{2.403963in}}%
\pgfpathlineto{\pgfqpoint{4.498742in}{2.404536in}}%
\pgfpathlineto{\pgfqpoint{4.498934in}{2.402576in}}%
\pgfpathlineto{\pgfqpoint{4.499703in}{2.405072in}}%
\pgfpathlineto{\pgfqpoint{4.499895in}{2.405897in}}%
\pgfpathlineto{\pgfqpoint{4.500087in}{2.404002in}}%
\pgfpathlineto{\pgfqpoint{4.501048in}{2.399583in}}%
\pgfpathlineto{\pgfqpoint{4.501624in}{2.400637in}}%
\pgfpathlineto{\pgfqpoint{4.502777in}{2.405692in}}%
\pgfpathlineto{\pgfqpoint{4.503738in}{2.404702in}}%
\pgfpathlineto{\pgfqpoint{4.503546in}{2.406738in}}%
\pgfpathlineto{\pgfqpoint{4.503930in}{2.404972in}}%
\pgfpathlineto{\pgfqpoint{4.506044in}{2.421596in}}%
\pgfpathlineto{\pgfqpoint{4.506236in}{2.419926in}}%
\pgfpathlineto{\pgfqpoint{4.506813in}{2.423836in}}%
\pgfpathlineto{\pgfqpoint{4.507005in}{2.422394in}}%
\pgfpathlineto{\pgfqpoint{4.507774in}{2.425081in}}%
\pgfpathlineto{\pgfqpoint{4.507389in}{2.422133in}}%
\pgfpathlineto{\pgfqpoint{4.507966in}{2.423473in}}%
\pgfpathlineto{\pgfqpoint{4.508350in}{2.421058in}}%
\pgfpathlineto{\pgfqpoint{4.508735in}{2.425610in}}%
\pgfpathlineto{\pgfqpoint{4.511233in}{2.441006in}}%
\pgfpathlineto{\pgfqpoint{4.511425in}{2.441182in}}%
\pgfpathlineto{\pgfqpoint{4.512578in}{2.437408in}}%
\pgfpathlineto{\pgfqpoint{4.512962in}{2.441905in}}%
\pgfpathlineto{\pgfqpoint{4.513539in}{2.437767in}}%
\pgfpathlineto{\pgfqpoint{4.514884in}{2.432772in}}%
\pgfpathlineto{\pgfqpoint{4.515076in}{2.433641in}}%
\pgfpathlineto{\pgfqpoint{4.515845in}{2.437798in}}%
\pgfpathlineto{\pgfqpoint{4.516037in}{2.437587in}}%
\pgfpathlineto{\pgfqpoint{4.517382in}{2.432205in}}%
\pgfpathlineto{\pgfqpoint{4.519112in}{2.437849in}}%
\pgfpathlineto{\pgfqpoint{4.517766in}{2.430889in}}%
\pgfpathlineto{\pgfqpoint{4.519496in}{2.435776in}}%
\pgfpathlineto{\pgfqpoint{4.520072in}{2.432847in}}%
\pgfpathlineto{\pgfqpoint{4.519880in}{2.436146in}}%
\pgfpathlineto{\pgfqpoint{4.520265in}{2.434382in}}%
\pgfpathlineto{\pgfqpoint{4.521418in}{2.442679in}}%
\pgfpathlineto{\pgfqpoint{4.521610in}{2.442489in}}%
\pgfpathlineto{\pgfqpoint{4.521802in}{2.442100in}}%
\pgfpathlineto{\pgfqpoint{4.521994in}{2.442315in}}%
\pgfpathlineto{\pgfqpoint{4.523916in}{2.454538in}}%
\pgfpathlineto{\pgfqpoint{4.524108in}{2.453313in}}%
\pgfpathlineto{\pgfqpoint{4.524492in}{2.454426in}}%
\pgfpathlineto{\pgfqpoint{4.525453in}{2.457207in}}%
\pgfpathlineto{\pgfqpoint{4.526030in}{2.451615in}}%
\pgfpathlineto{\pgfqpoint{4.526414in}{2.457209in}}%
\pgfpathlineto{\pgfqpoint{4.526798in}{2.458432in}}%
\pgfpathlineto{\pgfqpoint{4.526990in}{2.462214in}}%
\pgfpathlineto{\pgfqpoint{4.527759in}{2.459003in}}%
\pgfpathlineto{\pgfqpoint{4.529104in}{2.453770in}}%
\pgfpathlineto{\pgfqpoint{4.529297in}{2.454577in}}%
\pgfpathlineto{\pgfqpoint{4.529489in}{2.453640in}}%
\pgfpathlineto{\pgfqpoint{4.529873in}{2.448992in}}%
\pgfpathlineto{\pgfqpoint{4.530834in}{2.450212in}}%
\pgfpathlineto{\pgfqpoint{4.531026in}{2.451091in}}%
\pgfpathlineto{\pgfqpoint{4.531218in}{2.448479in}}%
\pgfpathlineto{\pgfqpoint{4.531410in}{2.449626in}}%
\pgfpathlineto{\pgfqpoint{4.532371in}{2.443285in}}%
\pgfpathlineto{\pgfqpoint{4.532756in}{2.447813in}}%
\pgfpathlineto{\pgfqpoint{4.533140in}{2.447962in}}%
\pgfpathlineto{\pgfqpoint{4.533524in}{2.448755in}}%
\pgfpathlineto{\pgfqpoint{4.534485in}{2.444566in}}%
\pgfpathlineto{\pgfqpoint{4.536022in}{2.452440in}}%
\pgfpathlineto{\pgfqpoint{4.536215in}{2.449766in}}%
\pgfpathlineto{\pgfqpoint{4.536407in}{2.447119in}}%
\pgfpathlineto{\pgfqpoint{4.536791in}{2.452981in}}%
\pgfpathlineto{\pgfqpoint{4.537368in}{2.448907in}}%
\pgfpathlineto{\pgfqpoint{4.537560in}{2.448978in}}%
\pgfpathlineto{\pgfqpoint{4.538328in}{2.440951in}}%
\pgfpathlineto{\pgfqpoint{4.538713in}{2.445153in}}%
\pgfpathlineto{\pgfqpoint{4.539289in}{2.443904in}}%
\pgfpathlineto{\pgfqpoint{4.539097in}{2.445561in}}%
\pgfpathlineto{\pgfqpoint{4.539866in}{2.444271in}}%
\pgfpathlineto{\pgfqpoint{4.540442in}{2.449038in}}%
\pgfpathlineto{\pgfqpoint{4.540827in}{2.443350in}}%
\pgfpathlineto{\pgfqpoint{4.541403in}{2.446623in}}%
\pgfpathlineto{\pgfqpoint{4.541787in}{2.444113in}}%
\pgfpathlineto{\pgfqpoint{4.543325in}{2.438316in}}%
\pgfpathlineto{\pgfqpoint{4.543517in}{2.441115in}}%
\pgfpathlineto{\pgfqpoint{4.544093in}{2.433738in}}%
\pgfpathlineto{\pgfqpoint{4.544286in}{2.434905in}}%
\pgfpathlineto{\pgfqpoint{4.544670in}{2.432011in}}%
\pgfpathlineto{\pgfqpoint{4.545631in}{2.427090in}}%
\pgfpathlineto{\pgfqpoint{4.546015in}{2.428345in}}%
\pgfpathlineto{\pgfqpoint{4.546207in}{2.431832in}}%
\pgfpathlineto{\pgfqpoint{4.546592in}{2.426741in}}%
\pgfpathlineto{\pgfqpoint{4.546976in}{2.431052in}}%
\pgfpathlineto{\pgfqpoint{4.547360in}{2.427933in}}%
\pgfpathlineto{\pgfqpoint{4.547745in}{2.431465in}}%
\pgfpathlineto{\pgfqpoint{4.547937in}{2.430780in}}%
\pgfpathlineto{\pgfqpoint{4.548129in}{2.431472in}}%
\pgfpathlineto{\pgfqpoint{4.548513in}{2.428732in}}%
\pgfpathlineto{\pgfqpoint{4.548898in}{2.430146in}}%
\pgfpathlineto{\pgfqpoint{4.549666in}{2.423490in}}%
\pgfpathlineto{\pgfqpoint{4.550051in}{2.425905in}}%
\pgfpathlineto{\pgfqpoint{4.551396in}{2.434937in}}%
\pgfpathlineto{\pgfqpoint{4.551972in}{2.433115in}}%
\pgfpathlineto{\pgfqpoint{4.552357in}{2.433824in}}%
\pgfpathlineto{\pgfqpoint{4.553894in}{2.419111in}}%
\pgfpathlineto{\pgfqpoint{4.556392in}{2.407161in}}%
\pgfpathlineto{\pgfqpoint{4.556584in}{2.408792in}}%
\pgfpathlineto{\pgfqpoint{4.556777in}{2.411712in}}%
\pgfpathlineto{\pgfqpoint{4.556969in}{2.406550in}}%
\pgfpathlineto{\pgfqpoint{4.557353in}{2.408384in}}%
\pgfpathlineto{\pgfqpoint{4.559083in}{2.400662in}}%
\pgfpathlineto{\pgfqpoint{4.559275in}{2.400209in}}%
\pgfpathlineto{\pgfqpoint{4.559467in}{2.396766in}}%
\pgfpathlineto{\pgfqpoint{4.560428in}{2.398185in}}%
\pgfpathlineto{\pgfqpoint{4.560812in}{2.401810in}}%
\pgfpathlineto{\pgfqpoint{4.561196in}{2.396595in}}%
\pgfpathlineto{\pgfqpoint{4.561581in}{2.396231in}}%
\pgfpathlineto{\pgfqpoint{4.562734in}{2.402019in}}%
\pgfpathlineto{\pgfqpoint{4.562926in}{2.400282in}}%
\pgfpathlineto{\pgfqpoint{4.564079in}{2.404555in}}%
\pgfpathlineto{\pgfqpoint{4.564271in}{2.404071in}}%
\pgfpathlineto{\pgfqpoint{4.565232in}{2.393891in}}%
\pgfpathlineto{\pgfqpoint{4.565616in}{2.394871in}}%
\pgfpathlineto{\pgfqpoint{4.566577in}{2.399465in}}%
\pgfpathlineto{\pgfqpoint{4.567538in}{2.398437in}}%
\pgfpathlineto{\pgfqpoint{4.569652in}{2.389141in}}%
\pgfpathlineto{\pgfqpoint{4.570036in}{2.390977in}}%
\pgfpathlineto{\pgfqpoint{4.570228in}{2.396768in}}%
\pgfpathlineto{\pgfqpoint{4.571189in}{2.392787in}}%
\pgfpathlineto{\pgfqpoint{4.571573in}{2.390268in}}%
\pgfpathlineto{\pgfqpoint{4.571766in}{2.391393in}}%
\pgfpathlineto{\pgfqpoint{4.572919in}{2.397406in}}%
\pgfpathlineto{\pgfqpoint{4.573111in}{2.396982in}}%
\pgfpathlineto{\pgfqpoint{4.573687in}{2.396162in}}%
\pgfpathlineto{\pgfqpoint{4.574840in}{2.389011in}}%
\pgfpathlineto{\pgfqpoint{4.575032in}{2.391445in}}%
\pgfpathlineto{\pgfqpoint{4.575225in}{2.391895in}}%
\pgfpathlineto{\pgfqpoint{4.575417in}{2.389320in}}%
\pgfpathlineto{\pgfqpoint{4.575993in}{2.392729in}}%
\pgfpathlineto{\pgfqpoint{4.576570in}{2.395489in}}%
\pgfpathlineto{\pgfqpoint{4.576954in}{2.393901in}}%
\pgfpathlineto{\pgfqpoint{4.578299in}{2.390500in}}%
\pgfpathlineto{\pgfqpoint{4.578492in}{2.393357in}}%
\pgfpathlineto{\pgfqpoint{4.579068in}{2.388635in}}%
\pgfpathlineto{\pgfqpoint{4.580221in}{2.385188in}}%
\pgfpathlineto{\pgfqpoint{4.580413in}{2.385235in}}%
\pgfpathlineto{\pgfqpoint{4.580605in}{2.385829in}}%
\pgfpathlineto{\pgfqpoint{4.580798in}{2.384775in}}%
\pgfpathlineto{\pgfqpoint{4.581951in}{2.379377in}}%
\pgfpathlineto{\pgfqpoint{4.582143in}{2.381803in}}%
\pgfpathlineto{\pgfqpoint{4.582527in}{2.376565in}}%
\pgfpathlineto{\pgfqpoint{4.582911in}{2.377661in}}%
\pgfpathlineto{\pgfqpoint{4.583488in}{2.382047in}}%
\pgfpathlineto{\pgfqpoint{4.584064in}{2.378469in}}%
\pgfpathlineto{\pgfqpoint{4.585217in}{2.374228in}}%
\pgfpathlineto{\pgfqpoint{4.585410in}{2.375561in}}%
\pgfpathlineto{\pgfqpoint{4.586370in}{2.378757in}}%
\pgfpathlineto{\pgfqpoint{4.586563in}{2.378413in}}%
\pgfpathlineto{\pgfqpoint{4.587908in}{2.363346in}}%
\pgfpathlineto{\pgfqpoint{4.588676in}{2.365790in}}%
\pgfpathlineto{\pgfqpoint{4.589445in}{2.370432in}}%
\pgfpathlineto{\pgfqpoint{4.589829in}{2.368756in}}%
\pgfpathlineto{\pgfqpoint{4.590022in}{2.366151in}}%
\pgfpathlineto{\pgfqpoint{4.590598in}{2.371128in}}%
\pgfpathlineto{\pgfqpoint{4.590982in}{2.375150in}}%
\pgfpathlineto{\pgfqpoint{4.591751in}{2.371650in}}%
\pgfpathlineto{\pgfqpoint{4.592135in}{2.367795in}}%
\pgfpathlineto{\pgfqpoint{4.592712in}{2.373123in}}%
\pgfpathlineto{\pgfqpoint{4.595018in}{2.395041in}}%
\pgfpathlineto{\pgfqpoint{4.595402in}{2.394347in}}%
\pgfpathlineto{\pgfqpoint{4.595979in}{2.387263in}}%
\pgfpathlineto{\pgfqpoint{4.596747in}{2.389129in}}%
\pgfpathlineto{\pgfqpoint{4.597324in}{2.385602in}}%
\pgfpathlineto{\pgfqpoint{4.597708in}{2.388151in}}%
\pgfpathlineto{\pgfqpoint{4.597900in}{2.390048in}}%
\pgfpathlineto{\pgfqpoint{4.598093in}{2.385981in}}%
\pgfpathlineto{\pgfqpoint{4.598669in}{2.386749in}}%
\pgfpathlineto{\pgfqpoint{4.598861in}{2.388304in}}%
\pgfpathlineto{\pgfqpoint{4.599246in}{2.385704in}}%
\pgfpathlineto{\pgfqpoint{4.599438in}{2.385776in}}%
\pgfpathlineto{\pgfqpoint{4.600206in}{2.377782in}}%
\pgfpathlineto{\pgfqpoint{4.600591in}{2.383738in}}%
\pgfpathlineto{\pgfqpoint{4.601552in}{2.389296in}}%
\pgfpathlineto{\pgfqpoint{4.601936in}{2.387332in}}%
\pgfpathlineto{\pgfqpoint{4.602320in}{2.388023in}}%
\pgfpathlineto{\pgfqpoint{4.604050in}{2.375603in}}%
\pgfpathlineto{\pgfqpoint{4.605011in}{2.373897in}}%
\pgfpathlineto{\pgfqpoint{4.604819in}{2.376321in}}%
\pgfpathlineto{\pgfqpoint{4.605203in}{2.374917in}}%
\pgfpathlineto{\pgfqpoint{4.605587in}{2.373687in}}%
\pgfpathlineto{\pgfqpoint{4.607893in}{2.381884in}}%
\pgfpathlineto{\pgfqpoint{4.608085in}{2.381890in}}%
\pgfpathlineto{\pgfqpoint{4.609238in}{2.392274in}}%
\pgfpathlineto{\pgfqpoint{4.610007in}{2.390216in}}%
\pgfpathlineto{\pgfqpoint{4.610199in}{2.389521in}}%
\pgfpathlineto{\pgfqpoint{4.610584in}{2.390245in}}%
\pgfpathlineto{\pgfqpoint{4.611544in}{2.405527in}}%
\pgfpathlineto{\pgfqpoint{4.612121in}{2.399599in}}%
\pgfpathlineto{\pgfqpoint{4.612313in}{2.397384in}}%
\pgfpathlineto{\pgfqpoint{4.612697in}{2.404338in}}%
\pgfpathlineto{\pgfqpoint{4.612890in}{2.402863in}}%
\pgfpathlineto{\pgfqpoint{4.613082in}{2.403124in}}%
\pgfpathlineto{\pgfqpoint{4.613274in}{2.400874in}}%
\pgfpathlineto{\pgfqpoint{4.613658in}{2.405532in}}%
\pgfpathlineto{\pgfqpoint{4.613850in}{2.408664in}}%
\pgfpathlineto{\pgfqpoint{4.614427in}{2.402201in}}%
\pgfpathlineto{\pgfqpoint{4.615772in}{2.391172in}}%
\pgfpathlineto{\pgfqpoint{4.615964in}{2.393633in}}%
\pgfpathlineto{\pgfqpoint{4.616156in}{2.394013in}}%
\pgfpathlineto{\pgfqpoint{4.616349in}{2.393678in}}%
\pgfpathlineto{\pgfqpoint{4.617694in}{2.389802in}}%
\pgfpathlineto{\pgfqpoint{4.617886in}{2.389644in}}%
\pgfpathlineto{\pgfqpoint{4.619423in}{2.400001in}}%
\pgfpathlineto{\pgfqpoint{4.619615in}{2.398098in}}%
\pgfpathlineto{\pgfqpoint{4.620768in}{2.394310in}}%
\pgfpathlineto{\pgfqpoint{4.620000in}{2.398317in}}%
\pgfpathlineto{\pgfqpoint{4.620961in}{2.394928in}}%
\pgfpathlineto{\pgfqpoint{4.621153in}{2.396170in}}%
\pgfpathlineto{\pgfqpoint{4.621537in}{2.391695in}}%
\pgfpathlineto{\pgfqpoint{4.621729in}{2.394035in}}%
\pgfpathlineto{\pgfqpoint{4.622114in}{2.389513in}}%
\pgfpathlineto{\pgfqpoint{4.622882in}{2.392846in}}%
\pgfpathlineto{\pgfqpoint{4.624035in}{2.391239in}}%
\pgfpathlineto{\pgfqpoint{4.625188in}{2.397121in}}%
\pgfpathlineto{\pgfqpoint{4.625380in}{2.395461in}}%
\pgfpathlineto{\pgfqpoint{4.625957in}{2.397304in}}%
\pgfpathlineto{\pgfqpoint{4.626149in}{2.396543in}}%
\pgfpathlineto{\pgfqpoint{4.626726in}{2.397726in}}%
\pgfpathlineto{\pgfqpoint{4.626918in}{2.395815in}}%
\pgfpathlineto{\pgfqpoint{4.627302in}{2.395926in}}%
\pgfpathlineto{\pgfqpoint{4.627879in}{2.394029in}}%
\pgfpathlineto{\pgfqpoint{4.628455in}{2.398166in}}%
\pgfpathlineto{\pgfqpoint{4.630953in}{2.384365in}}%
\pgfpathlineto{\pgfqpoint{4.631722in}{2.388111in}}%
\pgfpathlineto{\pgfqpoint{4.632106in}{2.390860in}}%
\pgfpathlineto{\pgfqpoint{4.632299in}{2.388009in}}%
\pgfpathlineto{\pgfqpoint{4.632875in}{2.388898in}}%
\pgfpathlineto{\pgfqpoint{4.633067in}{2.390068in}}%
\pgfpathlineto{\pgfqpoint{4.633259in}{2.388021in}}%
\pgfpathlineto{\pgfqpoint{4.633452in}{2.388041in}}%
\pgfpathlineto{\pgfqpoint{4.635565in}{2.380169in}}%
\pgfpathlineto{\pgfqpoint{4.636142in}{2.381282in}}%
\pgfpathlineto{\pgfqpoint{4.635950in}{2.378965in}}%
\pgfpathlineto{\pgfqpoint{4.636526in}{2.379248in}}%
\pgfpathlineto{\pgfqpoint{4.637103in}{2.376638in}}%
\pgfpathlineto{\pgfqpoint{4.637295in}{2.383702in}}%
\pgfpathlineto{\pgfqpoint{4.638832in}{2.394020in}}%
\pgfpathlineto{\pgfqpoint{4.639409in}{2.392452in}}%
\pgfpathlineto{\pgfqpoint{4.639601in}{2.395908in}}%
\pgfpathlineto{\pgfqpoint{4.640562in}{2.395710in}}%
\pgfpathlineto{\pgfqpoint{4.640946in}{2.396752in}}%
\pgfpathlineto{\pgfqpoint{4.641138in}{2.395794in}}%
\pgfpathlineto{\pgfqpoint{4.641523in}{2.393834in}}%
\pgfpathlineto{\pgfqpoint{4.641907in}{2.397439in}}%
\pgfpathlineto{\pgfqpoint{4.642291in}{2.400706in}}%
\pgfpathlineto{\pgfqpoint{4.642483in}{2.397660in}}%
\pgfpathlineto{\pgfqpoint{4.643444in}{2.391239in}}%
\pgfpathlineto{\pgfqpoint{4.643636in}{2.392245in}}%
\pgfpathlineto{\pgfqpoint{4.644213in}{2.398998in}}%
\pgfpathlineto{\pgfqpoint{4.644982in}{2.398041in}}%
\pgfpathlineto{\pgfqpoint{4.646519in}{2.390775in}}%
\pgfpathlineto{\pgfqpoint{4.647480in}{2.398466in}}%
\pgfpathlineto{\pgfqpoint{4.647864in}{2.398408in}}%
\pgfpathlineto{\pgfqpoint{4.648248in}{2.394723in}}%
\pgfpathlineto{\pgfqpoint{4.648633in}{2.399426in}}%
\pgfpathlineto{\pgfqpoint{4.648825in}{2.397762in}}%
\pgfpathlineto{\pgfqpoint{4.650362in}{2.408139in}}%
\pgfpathlineto{\pgfqpoint{4.650554in}{2.405983in}}%
\pgfpathlineto{\pgfqpoint{4.651131in}{2.410724in}}%
\pgfpathlineto{\pgfqpoint{4.651323in}{2.408645in}}%
\pgfpathlineto{\pgfqpoint{4.652284in}{2.413888in}}%
\pgfpathlineto{\pgfqpoint{4.652476in}{2.413321in}}%
\pgfpathlineto{\pgfqpoint{4.653437in}{2.405166in}}%
\pgfpathlineto{\pgfqpoint{4.654014in}{2.408177in}}%
\pgfpathlineto{\pgfqpoint{4.654398in}{2.407366in}}%
\pgfpathlineto{\pgfqpoint{4.655935in}{2.416264in}}%
\pgfpathlineto{\pgfqpoint{4.656127in}{2.416399in}}%
\pgfpathlineto{\pgfqpoint{4.657280in}{2.427958in}}%
\pgfpathlineto{\pgfqpoint{4.657857in}{2.425595in}}%
\pgfpathlineto{\pgfqpoint{4.659010in}{2.417679in}}%
\pgfpathlineto{\pgfqpoint{4.659202in}{2.420871in}}%
\pgfpathlineto{\pgfqpoint{4.660355in}{2.426849in}}%
\pgfpathlineto{\pgfqpoint{4.660739in}{2.425968in}}%
\pgfpathlineto{\pgfqpoint{4.661316in}{2.423723in}}%
\pgfpathlineto{\pgfqpoint{4.661892in}{2.425465in}}%
\pgfpathlineto{\pgfqpoint{4.662085in}{2.425710in}}%
\pgfpathlineto{\pgfqpoint{4.662469in}{2.421942in}}%
\pgfpathlineto{\pgfqpoint{4.663238in}{2.425071in}}%
\pgfpathlineto{\pgfqpoint{4.666697in}{2.410194in}}%
\pgfpathlineto{\pgfqpoint{4.667657in}{2.414040in}}%
\pgfpathlineto{\pgfqpoint{4.668042in}{2.413480in}}%
\pgfpathlineto{\pgfqpoint{4.668234in}{2.414285in}}%
\pgfpathlineto{\pgfqpoint{4.668426in}{2.413868in}}%
\pgfpathlineto{\pgfqpoint{4.669003in}{2.404861in}}%
\pgfpathlineto{\pgfqpoint{4.669963in}{2.406024in}}%
\pgfpathlineto{\pgfqpoint{4.670156in}{2.406990in}}%
\pgfpathlineto{\pgfqpoint{4.670348in}{2.402303in}}%
\pgfpathlineto{\pgfqpoint{4.670924in}{2.398585in}}%
\pgfpathlineto{\pgfqpoint{4.671309in}{2.401663in}}%
\pgfpathlineto{\pgfqpoint{4.672077in}{2.409595in}}%
\pgfpathlineto{\pgfqpoint{4.672462in}{2.405112in}}%
\pgfpathlineto{\pgfqpoint{4.672654in}{2.403755in}}%
\pgfpathlineto{\pgfqpoint{4.673422in}{2.404327in}}%
\pgfpathlineto{\pgfqpoint{4.674960in}{2.416127in}}%
\pgfpathlineto{\pgfqpoint{4.675536in}{2.414688in}}%
\pgfpathlineto{\pgfqpoint{4.675729in}{2.415143in}}%
\pgfpathlineto{\pgfqpoint{4.675921in}{2.412789in}}%
\pgfpathlineto{\pgfqpoint{4.676113in}{2.412995in}}%
\pgfpathlineto{\pgfqpoint{4.676305in}{2.410116in}}%
\pgfpathlineto{\pgfqpoint{4.677074in}{2.412402in}}%
\pgfpathlineto{\pgfqpoint{4.677266in}{2.413094in}}%
\pgfpathlineto{\pgfqpoint{4.677458in}{2.410109in}}%
\pgfpathlineto{\pgfqpoint{4.677842in}{2.406293in}}%
\pgfpathlineto{\pgfqpoint{4.678611in}{2.406369in}}%
\pgfpathlineto{\pgfqpoint{4.679188in}{2.405255in}}%
\pgfpathlineto{\pgfqpoint{4.680341in}{2.411709in}}%
\pgfpathlineto{\pgfqpoint{4.682262in}{2.402636in}}%
\pgfpathlineto{\pgfqpoint{4.682839in}{2.401714in}}%
\pgfpathlineto{\pgfqpoint{4.682647in}{2.403449in}}%
\pgfpathlineto{\pgfqpoint{4.683223in}{2.402717in}}%
\pgfpathlineto{\pgfqpoint{4.683800in}{2.402226in}}%
\pgfpathlineto{\pgfqpoint{4.686490in}{2.416125in}}%
\pgfpathlineto{\pgfqpoint{4.687259in}{2.419036in}}%
\pgfpathlineto{\pgfqpoint{4.686874in}{2.415750in}}%
\pgfpathlineto{\pgfqpoint{4.687451in}{2.418089in}}%
\pgfpathlineto{\pgfqpoint{4.688796in}{2.414502in}}%
\pgfpathlineto{\pgfqpoint{4.689372in}{2.417662in}}%
\pgfpathlineto{\pgfqpoint{4.689565in}{2.414281in}}%
\pgfpathlineto{\pgfqpoint{4.690141in}{2.416906in}}%
\pgfpathlineto{\pgfqpoint{4.690333in}{2.416257in}}%
\pgfpathlineto{\pgfqpoint{4.690718in}{2.417232in}}%
\pgfpathlineto{\pgfqpoint{4.690910in}{2.417203in}}%
\pgfpathlineto{\pgfqpoint{4.691294in}{2.419526in}}%
\pgfpathlineto{\pgfqpoint{4.692639in}{2.425161in}}%
\pgfpathlineto{\pgfqpoint{4.693024in}{2.420968in}}%
\pgfpathlineto{\pgfqpoint{4.693600in}{2.423983in}}%
\pgfpathlineto{\pgfqpoint{4.693792in}{2.426736in}}%
\pgfpathlineto{\pgfqpoint{4.694753in}{2.426421in}}%
\pgfpathlineto{\pgfqpoint{4.695137in}{2.427332in}}%
\pgfpathlineto{\pgfqpoint{4.695330in}{2.426114in}}%
\pgfpathlineto{\pgfqpoint{4.696483in}{2.430595in}}%
\pgfpathlineto{\pgfqpoint{4.696675in}{2.428874in}}%
\pgfpathlineto{\pgfqpoint{4.696867in}{2.428114in}}%
\pgfpathlineto{\pgfqpoint{4.697059in}{2.431444in}}%
\pgfpathlineto{\pgfqpoint{4.697251in}{2.431613in}}%
\pgfpathlineto{\pgfqpoint{4.697443in}{2.429309in}}%
\pgfpathlineto{\pgfqpoint{4.698212in}{2.433054in}}%
\pgfpathlineto{\pgfqpoint{4.698404in}{2.432248in}}%
\pgfpathlineto{\pgfqpoint{4.698596in}{2.433995in}}%
\pgfpathlineto{\pgfqpoint{4.699942in}{2.442564in}}%
\pgfpathlineto{\pgfqpoint{4.700134in}{2.439872in}}%
\pgfpathlineto{\pgfqpoint{4.700326in}{2.439668in}}%
\pgfpathlineto{\pgfqpoint{4.700710in}{2.440829in}}%
\pgfpathlineto{\pgfqpoint{4.700903in}{2.441466in}}%
\pgfpathlineto{\pgfqpoint{4.701287in}{2.438938in}}%
\pgfpathlineto{\pgfqpoint{4.701479in}{2.438247in}}%
\pgfpathlineto{\pgfqpoint{4.701671in}{2.440611in}}%
\pgfpathlineto{\pgfqpoint{4.701863in}{2.442623in}}%
\pgfpathlineto{\pgfqpoint{4.702248in}{2.438413in}}%
\pgfpathlineto{\pgfqpoint{4.703016in}{2.433880in}}%
\pgfpathlineto{\pgfqpoint{4.703401in}{2.437850in}}%
\pgfpathlineto{\pgfqpoint{4.703593in}{2.438205in}}%
\pgfpathlineto{\pgfqpoint{4.704746in}{2.432240in}}%
\pgfpathlineto{\pgfqpoint{4.705515in}{2.435484in}}%
\pgfpathlineto{\pgfqpoint{4.705899in}{2.434542in}}%
\pgfpathlineto{\pgfqpoint{4.706475in}{2.436704in}}%
\pgfpathlineto{\pgfqpoint{4.707052in}{2.435090in}}%
\pgfpathlineto{\pgfqpoint{4.707628in}{2.440648in}}%
\pgfpathlineto{\pgfqpoint{4.709358in}{2.432360in}}%
\pgfpathlineto{\pgfqpoint{4.709742in}{2.435196in}}%
\pgfpathlineto{\pgfqpoint{4.710319in}{2.437631in}}%
\pgfpathlineto{\pgfqpoint{4.710511in}{2.434111in}}%
\pgfpathlineto{\pgfqpoint{4.710703in}{2.433972in}}%
\pgfpathlineto{\pgfqpoint{4.712240in}{2.444674in}}%
\pgfpathlineto{\pgfqpoint{4.712433in}{2.443796in}}%
\pgfpathlineto{\pgfqpoint{4.713009in}{2.446054in}}%
\pgfpathlineto{\pgfqpoint{4.714162in}{2.450503in}}%
\pgfpathlineto{\pgfqpoint{4.714354in}{2.449371in}}%
\pgfpathlineto{\pgfqpoint{4.714931in}{2.445916in}}%
\pgfpathlineto{\pgfqpoint{4.715315in}{2.448406in}}%
\pgfpathlineto{\pgfqpoint{4.715699in}{2.455194in}}%
\pgfpathlineto{\pgfqpoint{4.716468in}{2.454742in}}%
\pgfpathlineto{\pgfqpoint{4.717621in}{2.447163in}}%
\pgfpathlineto{\pgfqpoint{4.717813in}{2.447796in}}%
\pgfpathlineto{\pgfqpoint{4.718005in}{2.445836in}}%
\pgfpathlineto{\pgfqpoint{4.718582in}{2.449971in}}%
\pgfpathlineto{\pgfqpoint{4.719158in}{2.452853in}}%
\pgfpathlineto{\pgfqpoint{4.719351in}{2.452656in}}%
\pgfpathlineto{\pgfqpoint{4.720504in}{2.440771in}}%
\pgfpathlineto{\pgfqpoint{4.720696in}{2.446056in}}%
\pgfpathlineto{\pgfqpoint{4.721272in}{2.445114in}}%
\pgfpathlineto{\pgfqpoint{4.721849in}{2.449124in}}%
\pgfpathlineto{\pgfqpoint{4.723578in}{2.438502in}}%
\pgfpathlineto{\pgfqpoint{4.723963in}{2.440192in}}%
\pgfpathlineto{\pgfqpoint{4.725308in}{2.448382in}}%
\pgfpathlineto{\pgfqpoint{4.727230in}{2.440663in}}%
\pgfpathlineto{\pgfqpoint{4.729151in}{2.443342in}}%
\pgfpathlineto{\pgfqpoint{4.729728in}{2.446973in}}%
\pgfpathlineto{\pgfqpoint{4.730112in}{2.442640in}}%
\pgfpathlineto{\pgfqpoint{4.730304in}{2.444825in}}%
\pgfpathlineto{\pgfqpoint{4.730881in}{2.441535in}}%
\pgfpathlineto{\pgfqpoint{4.731457in}{2.443757in}}%
\pgfpathlineto{\pgfqpoint{4.732034in}{2.446956in}}%
\pgfpathlineto{\pgfqpoint{4.732226in}{2.442617in}}%
\pgfpathlineto{\pgfqpoint{4.732418in}{2.442730in}}%
\pgfpathlineto{\pgfqpoint{4.732610in}{2.443897in}}%
\pgfpathlineto{\pgfqpoint{4.732995in}{2.439700in}}%
\pgfpathlineto{\pgfqpoint{4.733955in}{2.443644in}}%
\pgfpathlineto{\pgfqpoint{4.733379in}{2.437781in}}%
\pgfpathlineto{\pgfqpoint{4.734340in}{2.441573in}}%
\pgfpathlineto{\pgfqpoint{4.734724in}{2.439278in}}%
\pgfpathlineto{\pgfqpoint{4.735301in}{2.441087in}}%
\pgfpathlineto{\pgfqpoint{4.735877in}{2.446954in}}%
\pgfpathlineto{\pgfqpoint{4.736646in}{2.444249in}}%
\pgfpathlineto{\pgfqpoint{4.737991in}{2.438663in}}%
\pgfpathlineto{\pgfqpoint{4.739144in}{2.442088in}}%
\pgfpathlineto{\pgfqpoint{4.740489in}{2.436679in}}%
\pgfpathlineto{\pgfqpoint{4.740681in}{2.435953in}}%
\pgfpathlineto{\pgfqpoint{4.740873in}{2.437344in}}%
\pgfpathlineto{\pgfqpoint{4.741834in}{2.442398in}}%
\pgfpathlineto{\pgfqpoint{4.742026in}{2.440216in}}%
\pgfpathlineto{\pgfqpoint{4.742219in}{2.437476in}}%
\pgfpathlineto{\pgfqpoint{4.742603in}{2.444752in}}%
\pgfpathlineto{\pgfqpoint{4.743372in}{2.449633in}}%
\pgfpathlineto{\pgfqpoint{4.743564in}{2.446889in}}%
\pgfpathlineto{\pgfqpoint{4.745101in}{2.436072in}}%
\pgfpathlineto{\pgfqpoint{4.746446in}{2.439176in}}%
\pgfpathlineto{\pgfqpoint{4.747791in}{2.430607in}}%
\pgfpathlineto{\pgfqpoint{4.748368in}{2.432893in}}%
\pgfpathlineto{\pgfqpoint{4.748945in}{2.431239in}}%
\pgfpathlineto{\pgfqpoint{4.749137in}{2.430764in}}%
\pgfpathlineto{\pgfqpoint{4.751058in}{2.445886in}}%
\pgfpathlineto{\pgfqpoint{4.751827in}{2.445278in}}%
\pgfpathlineto{\pgfqpoint{4.752596in}{2.440869in}}%
\pgfpathlineto{\pgfqpoint{4.752980in}{2.443780in}}%
\pgfpathlineto{\pgfqpoint{4.753172in}{2.445269in}}%
\pgfpathlineto{\pgfqpoint{4.753749in}{2.441408in}}%
\pgfpathlineto{\pgfqpoint{4.755094in}{2.438515in}}%
\pgfpathlineto{\pgfqpoint{4.755670in}{2.444889in}}%
\pgfpathlineto{\pgfqpoint{4.756055in}{2.442436in}}%
\pgfpathlineto{\pgfqpoint{4.757592in}{2.429695in}}%
\pgfpathlineto{\pgfqpoint{4.757784in}{2.430823in}}%
\pgfpathlineto{\pgfqpoint{4.759129in}{2.437328in}}%
\pgfpathlineto{\pgfqpoint{4.759322in}{2.436331in}}%
\pgfpathlineto{\pgfqpoint{4.759898in}{2.434827in}}%
\pgfpathlineto{\pgfqpoint{4.760090in}{2.435609in}}%
\pgfpathlineto{\pgfqpoint{4.761051in}{2.440613in}}%
\pgfpathlineto{\pgfqpoint{4.761435in}{2.440438in}}%
\pgfpathlineto{\pgfqpoint{4.762012in}{2.442679in}}%
\pgfpathlineto{\pgfqpoint{4.762588in}{2.441709in}}%
\pgfpathlineto{\pgfqpoint{4.763357in}{2.438881in}}%
\pgfpathlineto{\pgfqpoint{4.763549in}{2.439167in}}%
\pgfpathlineto{\pgfqpoint{4.765087in}{2.428709in}}%
\pgfpathlineto{\pgfqpoint{4.766240in}{2.434123in}}%
\pgfpathlineto{\pgfqpoint{4.766432in}{2.432661in}}%
\pgfpathlineto{\pgfqpoint{4.766816in}{2.435872in}}%
\pgfpathlineto{\pgfqpoint{4.767393in}{2.431498in}}%
\pgfpathlineto{\pgfqpoint{4.767777in}{2.429994in}}%
\pgfpathlineto{\pgfqpoint{4.768161in}{2.431749in}}%
\pgfpathlineto{\pgfqpoint{4.768353in}{2.430547in}}%
\pgfpathlineto{\pgfqpoint{4.768738in}{2.431680in}}%
\pgfpathlineto{\pgfqpoint{4.769314in}{2.424706in}}%
\pgfpathlineto{\pgfqpoint{4.769891in}{2.427810in}}%
\pgfpathlineto{\pgfqpoint{4.771620in}{2.430926in}}%
\pgfpathlineto{\pgfqpoint{4.770275in}{2.427108in}}%
\pgfpathlineto{\pgfqpoint{4.771812in}{2.428886in}}%
\pgfpathlineto{\pgfqpoint{4.773158in}{2.424979in}}%
\pgfpathlineto{\pgfqpoint{4.774503in}{2.429581in}}%
\pgfpathlineto{\pgfqpoint{4.774695in}{2.429206in}}%
\pgfpathlineto{\pgfqpoint{4.775079in}{2.426470in}}%
\pgfpathlineto{\pgfqpoint{4.775656in}{2.429415in}}%
\pgfpathlineto{\pgfqpoint{4.775848in}{2.427323in}}%
\pgfpathlineto{\pgfqpoint{4.776232in}{2.429707in}}%
\pgfpathlineto{\pgfqpoint{4.777578in}{2.436119in}}%
\pgfpathlineto{\pgfqpoint{4.777770in}{2.434538in}}%
\pgfpathlineto{\pgfqpoint{4.778154in}{2.438333in}}%
\pgfpathlineto{\pgfqpoint{4.780268in}{2.450866in}}%
\pgfpathlineto{\pgfqpoint{4.781037in}{2.448916in}}%
\pgfpathlineto{\pgfqpoint{4.782190in}{2.440031in}}%
\pgfpathlineto{\pgfqpoint{4.782574in}{2.440875in}}%
\pgfpathlineto{\pgfqpoint{4.782958in}{2.445880in}}%
\pgfpathlineto{\pgfqpoint{4.783919in}{2.444148in}}%
\pgfpathlineto{\pgfqpoint{4.784496in}{2.442557in}}%
\pgfpathlineto{\pgfqpoint{4.786225in}{2.454169in}}%
\pgfpathlineto{\pgfqpoint{4.786417in}{2.461090in}}%
\pgfpathlineto{\pgfqpoint{4.787378in}{2.459420in}}%
\pgfpathlineto{\pgfqpoint{4.787762in}{2.461938in}}%
\pgfpathlineto{\pgfqpoint{4.787955in}{2.458048in}}%
\pgfpathlineto{\pgfqpoint{4.788147in}{2.458328in}}%
\pgfpathlineto{\pgfqpoint{4.789684in}{2.454376in}}%
\pgfpathlineto{\pgfqpoint{4.790261in}{2.459330in}}%
\pgfpathlineto{\pgfqpoint{4.791029in}{2.456835in}}%
\pgfpathlineto{\pgfqpoint{4.791221in}{2.454249in}}%
\pgfpathlineto{\pgfqpoint{4.792182in}{2.455442in}}%
\pgfpathlineto{\pgfqpoint{4.792374in}{2.456501in}}%
\pgfpathlineto{\pgfqpoint{4.792759in}{2.454406in}}%
\pgfpathlineto{\pgfqpoint{4.793335in}{2.450406in}}%
\pgfpathlineto{\pgfqpoint{4.793720in}{2.453005in}}%
\pgfpathlineto{\pgfqpoint{4.794488in}{2.458741in}}%
\pgfpathlineto{\pgfqpoint{4.794873in}{2.456082in}}%
\pgfpathlineto{\pgfqpoint{4.795065in}{2.453297in}}%
\pgfpathlineto{\pgfqpoint{4.795641in}{2.457409in}}%
\pgfpathlineto{\pgfqpoint{4.795833in}{2.457097in}}%
\pgfpathlineto{\pgfqpoint{4.796026in}{2.457242in}}%
\pgfpathlineto{\pgfqpoint{4.796794in}{2.448505in}}%
\pgfpathlineto{\pgfqpoint{4.797371in}{2.454025in}}%
\pgfpathlineto{\pgfqpoint{4.798524in}{2.459314in}}%
\pgfpathlineto{\pgfqpoint{4.798716in}{2.457293in}}%
\pgfpathlineto{\pgfqpoint{4.799100in}{2.456030in}}%
\pgfpathlineto{\pgfqpoint{4.799485in}{2.458868in}}%
\pgfpathlineto{\pgfqpoint{4.799869in}{2.461573in}}%
\pgfpathlineto{\pgfqpoint{4.800638in}{2.461185in}}%
\pgfpathlineto{\pgfqpoint{4.801406in}{2.456324in}}%
\pgfpathlineto{\pgfqpoint{4.801791in}{2.460283in}}%
\pgfpathlineto{\pgfqpoint{4.803136in}{2.452168in}}%
\pgfpathlineto{\pgfqpoint{4.803328in}{2.453175in}}%
\pgfpathlineto{\pgfqpoint{4.803520in}{2.453921in}}%
\pgfpathlineto{\pgfqpoint{4.803712in}{2.451983in}}%
\pgfpathlineto{\pgfqpoint{4.804097in}{2.453256in}}%
\pgfpathlineto{\pgfqpoint{4.804289in}{2.450274in}}%
\pgfpathlineto{\pgfqpoint{4.805250in}{2.451944in}}%
\pgfpathlineto{\pgfqpoint{4.805634in}{2.452709in}}%
\pgfpathlineto{\pgfqpoint{4.806018in}{2.450792in}}%
\pgfpathlineto{\pgfqpoint{4.807940in}{2.444330in}}%
\pgfpathlineto{\pgfqpoint{4.809670in}{2.452574in}}%
\pgfpathlineto{\pgfqpoint{4.810246in}{2.449685in}}%
\pgfpathlineto{\pgfqpoint{4.810823in}{2.451219in}}%
\pgfpathlineto{\pgfqpoint{4.812360in}{2.464351in}}%
\pgfpathlineto{\pgfqpoint{4.812936in}{2.460251in}}%
\pgfpathlineto{\pgfqpoint{4.813129in}{2.460273in}}%
\pgfpathlineto{\pgfqpoint{4.816780in}{2.435554in}}%
\pgfpathlineto{\pgfqpoint{4.817356in}{2.437070in}}%
\pgfpathlineto{\pgfqpoint{4.817548in}{2.436798in}}%
\pgfpathlineto{\pgfqpoint{4.818894in}{2.429583in}}%
\pgfpathlineto{\pgfqpoint{4.820431in}{2.437875in}}%
\pgfpathlineto{\pgfqpoint{4.820623in}{2.435230in}}%
\pgfpathlineto{\pgfqpoint{4.820815in}{2.434413in}}%
\pgfpathlineto{\pgfqpoint{4.821200in}{2.436185in}}%
\pgfpathlineto{\pgfqpoint{4.821584in}{2.439292in}}%
\pgfpathlineto{\pgfqpoint{4.822353in}{2.437120in}}%
\pgfpathlineto{\pgfqpoint{4.822737in}{2.433660in}}%
\pgfpathlineto{\pgfqpoint{4.823121in}{2.438039in}}%
\pgfpathlineto{\pgfqpoint{4.823506in}{2.439835in}}%
\pgfpathlineto{\pgfqpoint{4.823698in}{2.442877in}}%
\pgfpathlineto{\pgfqpoint{4.824467in}{2.437743in}}%
\pgfpathlineto{\pgfqpoint{4.825235in}{2.433035in}}%
\pgfpathlineto{\pgfqpoint{4.824851in}{2.437852in}}%
\pgfpathlineto{\pgfqpoint{4.825620in}{2.437098in}}%
\pgfpathlineto{\pgfqpoint{4.825812in}{2.440956in}}%
\pgfpathlineto{\pgfqpoint{4.826773in}{2.438558in}}%
\pgfpathlineto{\pgfqpoint{4.827541in}{2.441028in}}%
\pgfpathlineto{\pgfqpoint{4.827733in}{2.437558in}}%
\pgfpathlineto{\pgfqpoint{4.828694in}{2.440545in}}%
\pgfpathlineto{\pgfqpoint{4.830424in}{2.451050in}}%
\pgfpathlineto{\pgfqpoint{4.833883in}{2.431822in}}%
\pgfpathlineto{\pgfqpoint{4.834075in}{2.433386in}}%
\pgfpathlineto{\pgfqpoint{4.835612in}{2.436225in}}%
\pgfpathlineto{\pgfqpoint{4.835804in}{2.435958in}}%
\pgfpathlineto{\pgfqpoint{4.836957in}{2.429478in}}%
\pgfpathlineto{\pgfqpoint{4.837534in}{2.435353in}}%
\pgfpathlineto{\pgfqpoint{4.837918in}{2.431760in}}%
\pgfpathlineto{\pgfqpoint{4.839648in}{2.420442in}}%
\pgfpathlineto{\pgfqpoint{4.840609in}{2.424545in}}%
\pgfpathlineto{\pgfqpoint{4.840801in}{2.421603in}}%
\pgfpathlineto{\pgfqpoint{4.842530in}{2.416292in}}%
\pgfpathlineto{\pgfqpoint{4.841569in}{2.422353in}}%
\pgfpathlineto{\pgfqpoint{4.842722in}{2.417824in}}%
\pgfpathlineto{\pgfqpoint{4.843683in}{2.425437in}}%
\pgfpathlineto{\pgfqpoint{4.844068in}{2.422767in}}%
\pgfpathlineto{\pgfqpoint{4.844260in}{2.420281in}}%
\pgfpathlineto{\pgfqpoint{4.844836in}{2.425944in}}%
\pgfpathlineto{\pgfqpoint{4.845028in}{2.425796in}}%
\pgfpathlineto{\pgfqpoint{4.845221in}{2.426957in}}%
\pgfpathlineto{\pgfqpoint{4.845797in}{2.432103in}}%
\pgfpathlineto{\pgfqpoint{4.846566in}{2.430651in}}%
\pgfpathlineto{\pgfqpoint{4.846758in}{2.429412in}}%
\pgfpathlineto{\pgfqpoint{4.847142in}{2.433643in}}%
\pgfpathlineto{\pgfqpoint{4.847335in}{2.431057in}}%
\pgfpathlineto{\pgfqpoint{4.848295in}{2.435607in}}%
\pgfpathlineto{\pgfqpoint{4.849641in}{2.427101in}}%
\pgfpathlineto{\pgfqpoint{4.849833in}{2.423820in}}%
\pgfpathlineto{\pgfqpoint{4.850409in}{2.428273in}}%
\pgfpathlineto{\pgfqpoint{4.850601in}{2.427823in}}%
\pgfpathlineto{\pgfqpoint{4.850794in}{2.429595in}}%
\pgfpathlineto{\pgfqpoint{4.851178in}{2.425303in}}%
\pgfpathlineto{\pgfqpoint{4.851947in}{2.420391in}}%
\pgfpathlineto{\pgfqpoint{4.852331in}{2.424058in}}%
\pgfpathlineto{\pgfqpoint{4.852523in}{2.423680in}}%
\pgfpathlineto{\pgfqpoint{4.852715in}{2.424770in}}%
\pgfpathlineto{\pgfqpoint{4.852907in}{2.425759in}}%
\pgfpathlineto{\pgfqpoint{4.853100in}{2.422408in}}%
\pgfpathlineto{\pgfqpoint{4.853676in}{2.420462in}}%
\pgfpathlineto{\pgfqpoint{4.854253in}{2.420700in}}%
\pgfpathlineto{\pgfqpoint{4.856174in}{2.433202in}}%
\pgfpathlineto{\pgfqpoint{4.856366in}{2.432461in}}%
\pgfpathlineto{\pgfqpoint{4.856943in}{2.435022in}}%
\pgfpathlineto{\pgfqpoint{4.857519in}{2.434664in}}%
\pgfpathlineto{\pgfqpoint{4.858672in}{2.430919in}}%
\pgfpathlineto{\pgfqpoint{4.858865in}{2.435138in}}%
\pgfpathlineto{\pgfqpoint{4.859441in}{2.430816in}}%
\pgfpathlineto{\pgfqpoint{4.859825in}{2.433204in}}%
\pgfpathlineto{\pgfqpoint{4.860018in}{2.433286in}}%
\pgfpathlineto{\pgfqpoint{4.861171in}{2.429201in}}%
\pgfpathlineto{\pgfqpoint{4.861363in}{2.430426in}}%
\pgfpathlineto{\pgfqpoint{4.861747in}{2.425933in}}%
\pgfpathlineto{\pgfqpoint{4.862324in}{2.427544in}}%
\pgfpathlineto{\pgfqpoint{4.863477in}{2.416790in}}%
\pgfpathlineto{\pgfqpoint{4.863861in}{2.418865in}}%
\pgfpathlineto{\pgfqpoint{4.864630in}{2.426386in}}%
\pgfpathlineto{\pgfqpoint{4.865206in}{2.425654in}}%
\pgfpathlineto{\pgfqpoint{4.865398in}{2.424686in}}%
\pgfpathlineto{\pgfqpoint{4.865590in}{2.426058in}}%
\pgfpathlineto{\pgfqpoint{4.865975in}{2.425997in}}%
\pgfpathlineto{\pgfqpoint{4.868665in}{2.435709in}}%
\pgfpathlineto{\pgfqpoint{4.868857in}{2.432813in}}%
\pgfpathlineto{\pgfqpoint{4.870395in}{2.424660in}}%
\pgfpathlineto{\pgfqpoint{4.870587in}{2.424920in}}%
\pgfpathlineto{\pgfqpoint{4.871932in}{2.433368in}}%
\pgfpathlineto{\pgfqpoint{4.872316in}{2.430212in}}%
\pgfpathlineto{\pgfqpoint{4.872509in}{2.427988in}}%
\pgfpathlineto{\pgfqpoint{4.873085in}{2.432553in}}%
\pgfpathlineto{\pgfqpoint{4.873277in}{2.432373in}}%
\pgfpathlineto{\pgfqpoint{4.874430in}{2.426843in}}%
\pgfpathlineto{\pgfqpoint{4.874622in}{2.427016in}}%
\pgfpathlineto{\pgfqpoint{4.875199in}{2.431790in}}%
\pgfpathlineto{\pgfqpoint{4.875391in}{2.426794in}}%
\pgfpathlineto{\pgfqpoint{4.876352in}{2.421206in}}%
\pgfpathlineto{\pgfqpoint{4.875968in}{2.426852in}}%
\pgfpathlineto{\pgfqpoint{4.876544in}{2.422266in}}%
\pgfpathlineto{\pgfqpoint{4.877505in}{2.428179in}}%
\pgfpathlineto{\pgfqpoint{4.877697in}{2.424063in}}%
\pgfpathlineto{\pgfqpoint{4.880195in}{2.438393in}}%
\pgfpathlineto{\pgfqpoint{4.880387in}{2.437923in}}%
\pgfpathlineto{\pgfqpoint{4.881348in}{2.432999in}}%
\pgfpathlineto{\pgfqpoint{4.880772in}{2.437996in}}%
\pgfpathlineto{\pgfqpoint{4.881925in}{2.434485in}}%
\pgfpathlineto{\pgfqpoint{4.882693in}{2.435781in}}%
\pgfpathlineto{\pgfqpoint{4.884999in}{2.450451in}}%
\pgfpathlineto{\pgfqpoint{4.885384in}{2.449412in}}%
\pgfpathlineto{\pgfqpoint{4.886345in}{2.451919in}}%
\pgfpathlineto{\pgfqpoint{4.886537in}{2.450183in}}%
\pgfpathlineto{\pgfqpoint{4.886921in}{2.447560in}}%
\pgfpathlineto{\pgfqpoint{4.887113in}{2.448272in}}%
\pgfpathlineto{\pgfqpoint{4.887690in}{2.456255in}}%
\pgfpathlineto{\pgfqpoint{4.888458in}{2.455257in}}%
\pgfpathlineto{\pgfqpoint{4.889227in}{2.462793in}}%
\pgfpathlineto{\pgfqpoint{4.889804in}{2.459672in}}%
\pgfpathlineto{\pgfqpoint{4.890572in}{2.461357in}}%
\pgfpathlineto{\pgfqpoint{4.890380in}{2.457990in}}%
\pgfpathlineto{\pgfqpoint{4.890764in}{2.460067in}}%
\pgfpathlineto{\pgfqpoint{4.890957in}{2.458099in}}%
\pgfpathlineto{\pgfqpoint{4.891533in}{2.459082in}}%
\pgfpathlineto{\pgfqpoint{4.892878in}{2.472007in}}%
\pgfpathlineto{\pgfqpoint{4.893070in}{2.471696in}}%
\pgfpathlineto{\pgfqpoint{4.894992in}{2.463706in}}%
\pgfpathlineto{\pgfqpoint{4.896145in}{2.468448in}}%
\pgfpathlineto{\pgfqpoint{4.896337in}{2.466186in}}%
\pgfpathlineto{\pgfqpoint{4.897298in}{2.469957in}}%
\pgfpathlineto{\pgfqpoint{4.897490in}{2.467806in}}%
\pgfpathlineto{\pgfqpoint{4.899412in}{2.453233in}}%
\pgfpathlineto{\pgfqpoint{4.899989in}{2.454293in}}%
\pgfpathlineto{\pgfqpoint{4.901334in}{2.459842in}}%
\pgfpathlineto{\pgfqpoint{4.901910in}{2.455427in}}%
\pgfpathlineto{\pgfqpoint{4.902487in}{2.456714in}}%
\pgfpathlineto{\pgfqpoint{4.903255in}{2.460146in}}%
\pgfpathlineto{\pgfqpoint{4.903448in}{2.458966in}}%
\pgfpathlineto{\pgfqpoint{4.903832in}{2.454640in}}%
\pgfpathlineto{\pgfqpoint{4.904601in}{2.457976in}}%
\pgfpathlineto{\pgfqpoint{4.905369in}{2.452863in}}%
\pgfpathlineto{\pgfqpoint{4.906522in}{2.453510in}}%
\pgfpathlineto{\pgfqpoint{4.906907in}{2.452953in}}%
\pgfpathlineto{\pgfqpoint{4.908060in}{2.461750in}}%
\pgfpathlineto{\pgfqpoint{4.909020in}{2.458193in}}%
\pgfpathlineto{\pgfqpoint{4.908636in}{2.464264in}}%
\pgfpathlineto{\pgfqpoint{4.909405in}{2.458415in}}%
\pgfpathlineto{\pgfqpoint{4.910750in}{2.469987in}}%
\pgfpathlineto{\pgfqpoint{4.910942in}{2.467734in}}%
\pgfpathlineto{\pgfqpoint{4.911326in}{2.464080in}}%
\pgfpathlineto{\pgfqpoint{4.911711in}{2.470419in}}%
\pgfpathlineto{\pgfqpoint{4.911903in}{2.469876in}}%
\pgfpathlineto{\pgfqpoint{4.915170in}{2.457738in}}%
\pgfpathlineto{\pgfqpoint{4.912479in}{2.470900in}}%
\pgfpathlineto{\pgfqpoint{4.915362in}{2.458232in}}%
\pgfpathlineto{\pgfqpoint{4.916707in}{2.465014in}}%
\pgfpathlineto{\pgfqpoint{4.917091in}{2.463704in}}%
\pgfpathlineto{\pgfqpoint{4.917860in}{2.462620in}}%
\pgfpathlineto{\pgfqpoint{4.917668in}{2.464766in}}%
\pgfpathlineto{\pgfqpoint{4.918244in}{2.462707in}}%
\pgfpathlineto{\pgfqpoint{4.918821in}{2.463948in}}%
\pgfpathlineto{\pgfqpoint{4.919013in}{2.461058in}}%
\pgfpathlineto{\pgfqpoint{4.919974in}{2.461419in}}%
\pgfpathlineto{\pgfqpoint{4.920358in}{2.463425in}}%
\pgfpathlineto{\pgfqpoint{4.920743in}{2.461076in}}%
\pgfpathlineto{\pgfqpoint{4.920935in}{2.461395in}}%
\pgfpathlineto{\pgfqpoint{4.921511in}{2.459420in}}%
\pgfpathlineto{\pgfqpoint{4.921319in}{2.462407in}}%
\pgfpathlineto{\pgfqpoint{4.921896in}{2.460016in}}%
\pgfpathlineto{\pgfqpoint{4.922088in}{2.462527in}}%
\pgfpathlineto{\pgfqpoint{4.922472in}{2.456748in}}%
\pgfpathlineto{\pgfqpoint{4.922664in}{2.457053in}}%
\pgfpathlineto{\pgfqpoint{4.923049in}{2.455407in}}%
\pgfpathlineto{\pgfqpoint{4.923241in}{2.457880in}}%
\pgfpathlineto{\pgfqpoint{4.925355in}{2.483696in}}%
\pgfpathlineto{\pgfqpoint{4.925547in}{2.482123in}}%
\pgfpathlineto{\pgfqpoint{4.925931in}{2.481402in}}%
\pgfpathlineto{\pgfqpoint{4.926123in}{2.479734in}}%
\pgfpathlineto{\pgfqpoint{4.926316in}{2.482039in}}%
\pgfpathlineto{\pgfqpoint{4.926508in}{2.481872in}}%
\pgfpathlineto{\pgfqpoint{4.926700in}{2.484470in}}%
\pgfpathlineto{\pgfqpoint{4.927469in}{2.482998in}}%
\pgfpathlineto{\pgfqpoint{4.927661in}{2.482366in}}%
\pgfpathlineto{\pgfqpoint{4.928045in}{2.484826in}}%
\pgfpathlineto{\pgfqpoint{4.928622in}{2.484611in}}%
\pgfpathlineto{\pgfqpoint{4.930351in}{2.493636in}}%
\pgfpathlineto{\pgfqpoint{4.931888in}{2.487075in}}%
\pgfpathlineto{\pgfqpoint{4.932273in}{2.482696in}}%
\pgfpathlineto{\pgfqpoint{4.932657in}{2.484705in}}%
\pgfpathlineto{\pgfqpoint{4.932849in}{2.488907in}}%
\pgfpathlineto{\pgfqpoint{4.933618in}{2.483726in}}%
\pgfpathlineto{\pgfqpoint{4.933810in}{2.485947in}}%
\pgfpathlineto{\pgfqpoint{4.935732in}{2.488452in}}%
\pgfpathlineto{\pgfqpoint{4.937269in}{2.484504in}}%
\pgfpathlineto{\pgfqpoint{4.939575in}{2.502289in}}%
\pgfpathlineto{\pgfqpoint{4.940152in}{2.504077in}}%
\pgfpathlineto{\pgfqpoint{4.940344in}{2.500849in}}%
\pgfpathlineto{\pgfqpoint{4.941112in}{2.504814in}}%
\pgfpathlineto{\pgfqpoint{4.941497in}{2.503946in}}%
\pgfpathlineto{\pgfqpoint{4.941689in}{2.508900in}}%
\pgfpathlineto{\pgfqpoint{4.942650in}{2.505351in}}%
\pgfpathlineto{\pgfqpoint{4.943034in}{2.504708in}}%
\pgfpathlineto{\pgfqpoint{4.944379in}{2.500026in}}%
\pgfpathlineto{\pgfqpoint{4.945725in}{2.505179in}}%
\pgfpathlineto{\pgfqpoint{4.944764in}{2.498779in}}%
\pgfpathlineto{\pgfqpoint{4.945917in}{2.502791in}}%
\pgfpathlineto{\pgfqpoint{4.946301in}{2.504108in}}%
\pgfpathlineto{\pgfqpoint{4.947070in}{2.502891in}}%
\pgfpathlineto{\pgfqpoint{4.947262in}{2.499924in}}%
\pgfpathlineto{\pgfqpoint{4.947838in}{2.504886in}}%
\pgfpathlineto{\pgfqpoint{4.948031in}{2.504694in}}%
\pgfpathlineto{\pgfqpoint{4.953027in}{2.532380in}}%
\pgfpathlineto{\pgfqpoint{4.954180in}{2.522822in}}%
\pgfpathlineto{\pgfqpoint{4.954372in}{2.525055in}}%
\pgfpathlineto{\pgfqpoint{4.954949in}{2.531185in}}%
\pgfpathlineto{\pgfqpoint{4.955525in}{2.527173in}}%
\pgfpathlineto{\pgfqpoint{4.956102in}{2.526498in}}%
\pgfpathlineto{\pgfqpoint{4.956294in}{2.528773in}}%
\pgfpathlineto{\pgfqpoint{4.956870in}{2.523027in}}%
\pgfpathlineto{\pgfqpoint{4.957062in}{2.526468in}}%
\pgfpathlineto{\pgfqpoint{4.957255in}{2.526838in}}%
\pgfpathlineto{\pgfqpoint{4.957639in}{2.532158in}}%
\pgfpathlineto{\pgfqpoint{4.958023in}{2.526699in}}%
\pgfpathlineto{\pgfqpoint{4.958408in}{2.528164in}}%
\pgfpathlineto{\pgfqpoint{4.959176in}{2.524893in}}%
\pgfpathlineto{\pgfqpoint{4.959561in}{2.525197in}}%
\pgfpathlineto{\pgfqpoint{4.959945in}{2.526520in}}%
\pgfpathlineto{\pgfqpoint{4.960137in}{2.523755in}}%
\pgfpathlineto{\pgfqpoint{4.960329in}{2.520857in}}%
\pgfpathlineto{\pgfqpoint{4.960906in}{2.524164in}}%
\pgfpathlineto{\pgfqpoint{4.961867in}{2.531633in}}%
\pgfpathlineto{\pgfqpoint{4.962443in}{2.530082in}}%
\pgfpathlineto{\pgfqpoint{4.964173in}{2.537526in}}%
\pgfpathlineto{\pgfqpoint{4.962827in}{2.529354in}}%
\pgfpathlineto{\pgfqpoint{4.964365in}{2.536176in}}%
\pgfpathlineto{\pgfqpoint{4.965326in}{2.536778in}}%
\pgfpathlineto{\pgfqpoint{4.965518in}{2.533745in}}%
\pgfpathlineto{\pgfqpoint{4.965902in}{2.536351in}}%
\pgfpathlineto{\pgfqpoint{4.966286in}{2.535778in}}%
\pgfpathlineto{\pgfqpoint{4.966479in}{2.531685in}}%
\pgfpathlineto{\pgfqpoint{4.967247in}{2.537181in}}%
\pgfpathlineto{\pgfqpoint{4.967824in}{2.541260in}}%
\pgfpathlineto{\pgfqpoint{4.968400in}{2.538110in}}%
\pgfpathlineto{\pgfqpoint{4.969361in}{2.535202in}}%
\pgfpathlineto{\pgfqpoint{4.968977in}{2.539505in}}%
\pgfpathlineto{\pgfqpoint{4.969553in}{2.536412in}}%
\pgfpathlineto{\pgfqpoint{4.970514in}{2.543544in}}%
\pgfpathlineto{\pgfqpoint{4.971667in}{2.547886in}}%
\pgfpathlineto{\pgfqpoint{4.971859in}{2.546638in}}%
\pgfpathlineto{\pgfqpoint{4.972628in}{2.547583in}}%
\pgfpathlineto{\pgfqpoint{4.973397in}{2.541982in}}%
\pgfpathlineto{\pgfqpoint{4.973589in}{2.543242in}}%
\pgfpathlineto{\pgfqpoint{4.973973in}{2.540483in}}%
\pgfpathlineto{\pgfqpoint{4.974550in}{2.537046in}}%
\pgfpathlineto{\pgfqpoint{4.974934in}{2.540486in}}%
\pgfpathlineto{\pgfqpoint{4.975126in}{2.542632in}}%
\pgfpathlineto{\pgfqpoint{4.975703in}{2.538494in}}%
\pgfpathlineto{\pgfqpoint{4.976087in}{2.534269in}}%
\pgfpathlineto{\pgfqpoint{4.976279in}{2.535131in}}%
\pgfpathlineto{\pgfqpoint{4.976856in}{2.542787in}}%
\pgfpathlineto{\pgfqpoint{4.977432in}{2.539267in}}%
\pgfpathlineto{\pgfqpoint{4.979354in}{2.545903in}}%
\pgfpathlineto{\pgfqpoint{4.980123in}{2.537859in}}%
\pgfpathlineto{\pgfqpoint{4.980699in}{2.539417in}}%
\pgfpathlineto{\pgfqpoint{4.980891in}{2.539981in}}%
\pgfpathlineto{\pgfqpoint{4.981083in}{2.539338in}}%
\pgfpathlineto{\pgfqpoint{4.981276in}{2.544758in}}%
\pgfpathlineto{\pgfqpoint{4.982236in}{2.541585in}}%
\pgfpathlineto{\pgfqpoint{4.983774in}{2.534785in}}%
\pgfpathlineto{\pgfqpoint{4.985503in}{2.550151in}}%
\pgfpathlineto{\pgfqpoint{4.985695in}{2.552817in}}%
\pgfpathlineto{\pgfqpoint{4.986272in}{2.548826in}}%
\pgfpathlineto{\pgfqpoint{4.986464in}{2.544061in}}%
\pgfpathlineto{\pgfqpoint{4.987425in}{2.547078in}}%
\pgfpathlineto{\pgfqpoint{4.987617in}{2.546163in}}%
\pgfpathlineto{\pgfqpoint{4.987809in}{2.550152in}}%
\pgfpathlineto{\pgfqpoint{4.988001in}{2.549361in}}%
\pgfpathlineto{\pgfqpoint{4.988770in}{2.546564in}}%
\pgfpathlineto{\pgfqpoint{4.988386in}{2.550515in}}%
\pgfpathlineto{\pgfqpoint{4.989154in}{2.548796in}}%
\pgfpathlineto{\pgfqpoint{4.989347in}{2.552120in}}%
\pgfpathlineto{\pgfqpoint{4.990115in}{2.546517in}}%
\pgfpathlineto{\pgfqpoint{4.991076in}{2.552372in}}%
\pgfpathlineto{\pgfqpoint{4.991653in}{2.550454in}}%
\pgfpathlineto{\pgfqpoint{4.991845in}{2.550561in}}%
\pgfpathlineto{\pgfqpoint{4.993767in}{2.560098in}}%
\pgfpathlineto{\pgfqpoint{4.994920in}{2.556256in}}%
\pgfpathlineto{\pgfqpoint{4.995112in}{2.558012in}}%
\pgfpathlineto{\pgfqpoint{4.995688in}{2.554072in}}%
\pgfpathlineto{\pgfqpoint{4.997033in}{2.558966in}}%
\pgfpathlineto{\pgfqpoint{4.998379in}{2.545802in}}%
\pgfpathlineto{\pgfqpoint{4.998763in}{2.547134in}}%
\pgfpathlineto{\pgfqpoint{4.999147in}{2.543433in}}%
\pgfpathlineto{\pgfqpoint{5.000685in}{2.537320in}}%
\pgfpathlineto{\pgfqpoint{5.000877in}{2.537644in}}%
\pgfpathlineto{\pgfqpoint{5.001069in}{2.537989in}}%
\pgfpathlineto{\pgfqpoint{5.002030in}{2.532346in}}%
\pgfpathlineto{\pgfqpoint{5.001838in}{2.538379in}}%
\pgfpathlineto{\pgfqpoint{5.002414in}{2.534869in}}%
\pgfpathlineto{\pgfqpoint{5.002606in}{2.535783in}}%
\pgfpathlineto{\pgfqpoint{5.002798in}{2.532658in}}%
\pgfpathlineto{\pgfqpoint{5.002991in}{2.535466in}}%
\pgfpathlineto{\pgfqpoint{5.004720in}{2.525838in}}%
\pgfpathlineto{\pgfqpoint{5.004912in}{2.525650in}}%
\pgfpathlineto{\pgfqpoint{5.005104in}{2.525859in}}%
\pgfpathlineto{\pgfqpoint{5.006642in}{2.535683in}}%
\pgfpathlineto{\pgfqpoint{5.006834in}{2.535904in}}%
\pgfpathlineto{\pgfqpoint{5.007603in}{2.532939in}}%
\pgfpathlineto{\pgfqpoint{5.007795in}{2.533423in}}%
\pgfpathlineto{\pgfqpoint{5.008948in}{2.539074in}}%
\pgfpathlineto{\pgfqpoint{5.010101in}{2.534896in}}%
\pgfpathlineto{\pgfqpoint{5.010293in}{2.536891in}}%
\pgfpathlineto{\pgfqpoint{5.011830in}{2.543994in}}%
\pgfpathlineto{\pgfqpoint{5.012599in}{2.547993in}}%
\pgfpathlineto{\pgfqpoint{5.013175in}{2.540345in}}%
\pgfpathlineto{\pgfqpoint{5.013368in}{2.540231in}}%
\pgfpathlineto{\pgfqpoint{5.014136in}{2.537564in}}%
\pgfpathlineto{\pgfqpoint{5.014328in}{2.540114in}}%
\pgfpathlineto{\pgfqpoint{5.014521in}{2.541862in}}%
\pgfpathlineto{\pgfqpoint{5.014905in}{2.536807in}}%
\pgfpathlineto{\pgfqpoint{5.015289in}{2.534235in}}%
\pgfpathlineto{\pgfqpoint{5.016250in}{2.534366in}}%
\pgfpathlineto{\pgfqpoint{5.016634in}{2.533152in}}%
\pgfpathlineto{\pgfqpoint{5.017403in}{2.538202in}}%
\pgfpathlineto{\pgfqpoint{5.019901in}{2.553439in}}%
\pgfpathlineto{\pgfqpoint{5.020094in}{2.551592in}}%
\pgfpathlineto{\pgfqpoint{5.021823in}{2.543386in}}%
\pgfpathlineto{\pgfqpoint{5.022015in}{2.546611in}}%
\pgfpathlineto{\pgfqpoint{5.022592in}{2.542612in}}%
\pgfpathlineto{\pgfqpoint{5.022784in}{2.544007in}}%
\pgfpathlineto{\pgfqpoint{5.024706in}{2.535585in}}%
\pgfpathlineto{\pgfqpoint{5.025474in}{2.541517in}}%
\pgfpathlineto{\pgfqpoint{5.026051in}{2.540441in}}%
\pgfpathlineto{\pgfqpoint{5.026243in}{2.537054in}}%
\pgfpathlineto{\pgfqpoint{5.027012in}{2.541480in}}%
\pgfpathlineto{\pgfqpoint{5.027588in}{2.541022in}}%
\pgfpathlineto{\pgfqpoint{5.028549in}{2.547080in}}%
\pgfpathlineto{\pgfqpoint{5.028741in}{2.545651in}}%
\pgfpathlineto{\pgfqpoint{5.029125in}{2.547134in}}%
\pgfpathlineto{\pgfqpoint{5.030471in}{2.552151in}}%
\pgfpathlineto{\pgfqpoint{5.031047in}{2.556974in}}%
\pgfpathlineto{\pgfqpoint{5.031816in}{2.555733in}}%
\pgfpathlineto{\pgfqpoint{5.032008in}{2.555414in}}%
\pgfpathlineto{\pgfqpoint{5.032584in}{2.551429in}}%
\pgfpathlineto{\pgfqpoint{5.032969in}{2.552903in}}%
\pgfpathlineto{\pgfqpoint{5.033161in}{2.555707in}}%
\pgfpathlineto{\pgfqpoint{5.033737in}{2.549329in}}%
\pgfpathlineto{\pgfqpoint{5.033930in}{2.551307in}}%
\pgfpathlineto{\pgfqpoint{5.034890in}{2.547533in}}%
\pgfpathlineto{\pgfqpoint{5.035083in}{2.548792in}}%
\pgfpathlineto{\pgfqpoint{5.035467in}{2.547690in}}%
\pgfpathlineto{\pgfqpoint{5.035851in}{2.549844in}}%
\pgfpathlineto{\pgfqpoint{5.036043in}{2.550773in}}%
\pgfpathlineto{\pgfqpoint{5.036428in}{2.548997in}}%
\pgfpathlineto{\pgfqpoint{5.037196in}{2.544153in}}%
\pgfpathlineto{\pgfqpoint{5.037389in}{2.546470in}}%
\pgfpathlineto{\pgfqpoint{5.037581in}{2.549501in}}%
\pgfpathlineto{\pgfqpoint{5.038349in}{2.549427in}}%
\pgfpathlineto{\pgfqpoint{5.038542in}{2.547066in}}%
\pgfpathlineto{\pgfqpoint{5.038926in}{2.550158in}}%
\pgfpathlineto{\pgfqpoint{5.039310in}{2.548109in}}%
\pgfpathlineto{\pgfqpoint{5.040271in}{2.555241in}}%
\pgfpathlineto{\pgfqpoint{5.040848in}{2.552382in}}%
\pgfpathlineto{\pgfqpoint{5.042001in}{2.549297in}}%
\pgfpathlineto{\pgfqpoint{5.043154in}{2.553413in}}%
\pgfpathlineto{\pgfqpoint{5.043346in}{2.552649in}}%
\pgfpathlineto{\pgfqpoint{5.044307in}{2.547846in}}%
\pgfpathlineto{\pgfqpoint{5.044691in}{2.548729in}}%
\pgfpathlineto{\pgfqpoint{5.045075in}{2.549099in}}%
\pgfpathlineto{\pgfqpoint{5.045268in}{2.547212in}}%
\pgfpathlineto{\pgfqpoint{5.045460in}{2.543530in}}%
\pgfpathlineto{\pgfqpoint{5.046228in}{2.548801in}}%
\pgfpathlineto{\pgfqpoint{5.048342in}{2.564421in}}%
\pgfpathlineto{\pgfqpoint{5.048534in}{2.561930in}}%
\pgfpathlineto{\pgfqpoint{5.048727in}{2.561829in}}%
\pgfpathlineto{\pgfqpoint{5.049687in}{2.558327in}}%
\pgfpathlineto{\pgfqpoint{5.049880in}{2.559239in}}%
\pgfpathlineto{\pgfqpoint{5.050072in}{2.563620in}}%
\pgfpathlineto{\pgfqpoint{5.051033in}{2.560956in}}%
\pgfpathlineto{\pgfqpoint{5.052186in}{2.557239in}}%
\pgfpathlineto{\pgfqpoint{5.052378in}{2.560019in}}%
\pgfpathlineto{\pgfqpoint{5.053146in}{2.556365in}}%
\pgfpathlineto{\pgfqpoint{5.053339in}{2.556576in}}%
\pgfpathlineto{\pgfqpoint{5.053531in}{2.554820in}}%
\pgfpathlineto{\pgfqpoint{5.053915in}{2.559584in}}%
\pgfpathlineto{\pgfqpoint{5.054107in}{2.559838in}}%
\pgfpathlineto{\pgfqpoint{5.054299in}{2.559037in}}%
\pgfpathlineto{\pgfqpoint{5.055837in}{2.544908in}}%
\pgfpathlineto{\pgfqpoint{5.056605in}{2.550660in}}%
\pgfpathlineto{\pgfqpoint{5.057566in}{2.549597in}}%
\pgfpathlineto{\pgfqpoint{5.059104in}{2.539983in}}%
\pgfpathlineto{\pgfqpoint{5.059296in}{2.541615in}}%
\pgfpathlineto{\pgfqpoint{5.059872in}{2.546454in}}%
\pgfpathlineto{\pgfqpoint{5.060449in}{2.542757in}}%
\pgfpathlineto{\pgfqpoint{5.060641in}{2.541155in}}%
\pgfpathlineto{\pgfqpoint{5.061025in}{2.545264in}}%
\pgfpathlineto{\pgfqpoint{5.062563in}{2.554563in}}%
\pgfpathlineto{\pgfqpoint{5.061602in}{2.544092in}}%
\pgfpathlineto{\pgfqpoint{5.062755in}{2.552591in}}%
\pgfpathlineto{\pgfqpoint{5.063331in}{2.551424in}}%
\pgfpathlineto{\pgfqpoint{5.064100in}{2.554596in}}%
\pgfpathlineto{\pgfqpoint{5.064292in}{2.550555in}}%
\pgfpathlineto{\pgfqpoint{5.065061in}{2.554304in}}%
\pgfpathlineto{\pgfqpoint{5.065253in}{2.554454in}}%
\pgfpathlineto{\pgfqpoint{5.065445in}{2.553484in}}%
\pgfpathlineto{\pgfqpoint{5.065637in}{2.552183in}}%
\pgfpathlineto{\pgfqpoint{5.065829in}{2.555139in}}%
\pgfpathlineto{\pgfqpoint{5.066406in}{2.553596in}}%
\pgfpathlineto{\pgfqpoint{5.066983in}{2.563551in}}%
\pgfpathlineto{\pgfqpoint{5.067559in}{2.559140in}}%
\pgfpathlineto{\pgfqpoint{5.069289in}{2.550948in}}%
\pgfpathlineto{\pgfqpoint{5.069481in}{2.551143in}}%
\pgfpathlineto{\pgfqpoint{5.069673in}{2.554819in}}%
\pgfpathlineto{\pgfqpoint{5.070249in}{2.548525in}}%
\pgfpathlineto{\pgfqpoint{5.070442in}{2.548577in}}%
\pgfpathlineto{\pgfqpoint{5.070634in}{2.548576in}}%
\pgfpathlineto{\pgfqpoint{5.070826in}{2.550747in}}%
\pgfpathlineto{\pgfqpoint{5.071402in}{2.545254in}}%
\pgfpathlineto{\pgfqpoint{5.072940in}{2.553409in}}%
\pgfpathlineto{\pgfqpoint{5.073324in}{2.551773in}}%
\pgfpathlineto{\pgfqpoint{5.074669in}{2.540976in}}%
\pgfpathlineto{\pgfqpoint{5.074861in}{2.541869in}}%
\pgfpathlineto{\pgfqpoint{5.076783in}{2.552728in}}%
\pgfpathlineto{\pgfqpoint{5.077552in}{2.551238in}}%
\pgfpathlineto{\pgfqpoint{5.078128in}{2.548297in}}%
\pgfpathlineto{\pgfqpoint{5.078513in}{2.549725in}}%
\pgfpathlineto{\pgfqpoint{5.079089in}{2.547849in}}%
\pgfpathlineto{\pgfqpoint{5.080050in}{2.554977in}}%
\pgfpathlineto{\pgfqpoint{5.081779in}{2.562897in}}%
\pgfpathlineto{\pgfqpoint{5.083701in}{2.550085in}}%
\pgfpathlineto{\pgfqpoint{5.084278in}{2.548723in}}%
\pgfpathlineto{\pgfqpoint{5.084470in}{2.551607in}}%
\pgfpathlineto{\pgfqpoint{5.084854in}{2.549703in}}%
\pgfpathlineto{\pgfqpoint{5.085046in}{2.549200in}}%
\pgfpathlineto{\pgfqpoint{5.085431in}{2.550599in}}%
\pgfpathlineto{\pgfqpoint{5.086007in}{2.553312in}}%
\pgfpathlineto{\pgfqpoint{5.086391in}{2.549630in}}%
\pgfpathlineto{\pgfqpoint{5.087352in}{2.545900in}}%
\pgfpathlineto{\pgfqpoint{5.086776in}{2.549947in}}%
\pgfpathlineto{\pgfqpoint{5.087544in}{2.547983in}}%
\pgfpathlineto{\pgfqpoint{5.088121in}{2.550343in}}%
\pgfpathlineto{\pgfqpoint{5.088505in}{2.547737in}}%
\pgfpathlineto{\pgfqpoint{5.088890in}{2.543185in}}%
\pgfpathlineto{\pgfqpoint{5.089658in}{2.545044in}}%
\pgfpathlineto{\pgfqpoint{5.091004in}{2.536644in}}%
\pgfpathlineto{\pgfqpoint{5.091388in}{2.539918in}}%
\pgfpathlineto{\pgfqpoint{5.092157in}{2.547558in}}%
\pgfpathlineto{\pgfqpoint{5.092733in}{2.545780in}}%
\pgfpathlineto{\pgfqpoint{5.093117in}{2.540078in}}%
\pgfpathlineto{\pgfqpoint{5.093886in}{2.543735in}}%
\pgfpathlineto{\pgfqpoint{5.094270in}{2.543805in}}%
\pgfpathlineto{\pgfqpoint{5.095039in}{2.536880in}}%
\pgfpathlineto{\pgfqpoint{5.095423in}{2.542674in}}%
\pgfpathlineto{\pgfqpoint{5.095616in}{2.543681in}}%
\pgfpathlineto{\pgfqpoint{5.096000in}{2.541007in}}%
\pgfpathlineto{\pgfqpoint{5.096192in}{2.540760in}}%
\pgfpathlineto{\pgfqpoint{5.096384in}{2.542478in}}%
\pgfpathlineto{\pgfqpoint{5.096769in}{2.538368in}}%
\pgfpathlineto{\pgfqpoint{5.097153in}{2.540213in}}%
\pgfpathlineto{\pgfqpoint{5.097345in}{2.540594in}}%
\pgfpathlineto{\pgfqpoint{5.098306in}{2.554085in}}%
\pgfpathlineto{\pgfqpoint{5.098882in}{2.551277in}}%
\pgfpathlineto{\pgfqpoint{5.100228in}{2.546836in}}%
\pgfpathlineto{\pgfqpoint{5.101188in}{2.553454in}}%
\pgfpathlineto{\pgfqpoint{5.101381in}{2.553010in}}%
\pgfpathlineto{\pgfqpoint{5.101957in}{2.550368in}}%
\pgfpathlineto{\pgfqpoint{5.102341in}{2.553992in}}%
\pgfpathlineto{\pgfqpoint{5.103302in}{2.557175in}}%
\pgfpathlineto{\pgfqpoint{5.103494in}{2.556579in}}%
\pgfpathlineto{\pgfqpoint{5.103879in}{2.553318in}}%
\pgfpathlineto{\pgfqpoint{5.104840in}{2.554870in}}%
\pgfpathlineto{\pgfqpoint{5.105032in}{2.554268in}}%
\pgfpathlineto{\pgfqpoint{5.105224in}{2.556522in}}%
\pgfpathlineto{\pgfqpoint{5.105416in}{2.555389in}}%
\pgfpathlineto{\pgfqpoint{5.105608in}{2.559000in}}%
\pgfpathlineto{\pgfqpoint{5.106569in}{2.558369in}}%
\pgfpathlineto{\pgfqpoint{5.106761in}{2.555400in}}%
\pgfpathlineto{\pgfqpoint{5.107338in}{2.561426in}}%
\pgfpathlineto{\pgfqpoint{5.107722in}{2.560995in}}%
\pgfpathlineto{\pgfqpoint{5.108683in}{2.558156in}}%
\pgfpathlineto{\pgfqpoint{5.108875in}{2.559377in}}%
\pgfpathlineto{\pgfqpoint{5.109452in}{2.560987in}}%
\pgfpathlineto{\pgfqpoint{5.109836in}{2.559506in}}%
\pgfpathlineto{\pgfqpoint{5.110220in}{2.556012in}}%
\pgfpathlineto{\pgfqpoint{5.110797in}{2.560258in}}%
\pgfpathlineto{\pgfqpoint{5.111181in}{2.559265in}}%
\pgfpathlineto{\pgfqpoint{5.111373in}{2.561871in}}%
\pgfpathlineto{\pgfqpoint{5.111950in}{2.566304in}}%
\pgfpathlineto{\pgfqpoint{5.112526in}{2.568233in}}%
\pgfpathlineto{\pgfqpoint{5.112718in}{2.565423in}}%
\pgfpathlineto{\pgfqpoint{5.113871in}{2.558342in}}%
\pgfpathlineto{\pgfqpoint{5.114064in}{2.558875in}}%
\pgfpathlineto{\pgfqpoint{5.114448in}{2.562545in}}%
\pgfpathlineto{\pgfqpoint{5.115217in}{2.561639in}}%
\pgfpathlineto{\pgfqpoint{5.115409in}{2.560360in}}%
\pgfpathlineto{\pgfqpoint{5.115793in}{2.564797in}}%
\pgfpathlineto{\pgfqpoint{5.119060in}{2.577833in}}%
\pgfpathlineto{\pgfqpoint{5.119444in}{2.578226in}}%
\pgfpathlineto{\pgfqpoint{5.119637in}{2.575005in}}%
\pgfpathlineto{\pgfqpoint{5.120213in}{2.579493in}}%
\pgfpathlineto{\pgfqpoint{5.120597in}{2.582422in}}%
\pgfpathlineto{\pgfqpoint{5.121174in}{2.578376in}}%
\pgfpathlineto{\pgfqpoint{5.121943in}{2.575387in}}%
\pgfpathlineto{\pgfqpoint{5.121558in}{2.578991in}}%
\pgfpathlineto{\pgfqpoint{5.122327in}{2.577232in}}%
\pgfpathlineto{\pgfqpoint{5.123288in}{2.583936in}}%
\pgfpathlineto{\pgfqpoint{5.123672in}{2.582457in}}%
\pgfpathlineto{\pgfqpoint{5.123864in}{2.581278in}}%
\pgfpathlineto{\pgfqpoint{5.124056in}{2.582481in}}%
\pgfpathlineto{\pgfqpoint{5.125017in}{2.589539in}}%
\pgfpathlineto{\pgfqpoint{5.125402in}{2.586935in}}%
\pgfpathlineto{\pgfqpoint{5.125594in}{2.586522in}}%
\pgfpathlineto{\pgfqpoint{5.126747in}{2.594162in}}%
\pgfpathlineto{\pgfqpoint{5.126939in}{2.593796in}}%
\pgfpathlineto{\pgfqpoint{5.131167in}{2.621219in}}%
\pgfpathlineto{\pgfqpoint{5.131359in}{2.618310in}}%
\pgfpathlineto{\pgfqpoint{5.131743in}{2.619245in}}%
\pgfpathlineto{\pgfqpoint{5.132320in}{2.617003in}}%
\pgfpathlineto{\pgfqpoint{5.132704in}{2.623664in}}%
\pgfpathlineto{\pgfqpoint{5.133857in}{2.620985in}}%
\pgfpathlineto{\pgfqpoint{5.134049in}{2.620573in}}%
\pgfpathlineto{\pgfqpoint{5.134626in}{2.613976in}}%
\pgfpathlineto{\pgfqpoint{5.135010in}{2.616416in}}%
\pgfpathlineto{\pgfqpoint{5.136739in}{2.629599in}}%
\pgfpathlineto{\pgfqpoint{5.136932in}{2.629691in}}%
\pgfpathlineto{\pgfqpoint{5.137316in}{2.626012in}}%
\pgfpathlineto{\pgfqpoint{5.137892in}{2.628217in}}%
\pgfpathlineto{\pgfqpoint{5.138277in}{2.629945in}}%
\pgfpathlineto{\pgfqpoint{5.138661in}{2.636817in}}%
\pgfpathlineto{\pgfqpoint{5.139430in}{2.632051in}}%
\pgfpathlineto{\pgfqpoint{5.139814in}{2.633797in}}%
\pgfpathlineto{\pgfqpoint{5.140006in}{2.630493in}}%
\pgfpathlineto{\pgfqpoint{5.140391in}{2.632993in}}%
\pgfpathlineto{\pgfqpoint{5.140967in}{2.630099in}}%
\pgfpathlineto{\pgfqpoint{5.141159in}{2.634965in}}%
\pgfpathlineto{\pgfqpoint{5.141544in}{2.632232in}}%
\pgfpathlineto{\pgfqpoint{5.141928in}{2.634947in}}%
\pgfpathlineto{\pgfqpoint{5.142312in}{2.630113in}}%
\pgfpathlineto{\pgfqpoint{5.142505in}{2.631550in}}%
\pgfpathlineto{\pgfqpoint{5.143850in}{2.626293in}}%
\pgfpathlineto{\pgfqpoint{5.144042in}{2.627158in}}%
\pgfpathlineto{\pgfqpoint{5.144811in}{2.627973in}}%
\pgfpathlineto{\pgfqpoint{5.145003in}{2.624482in}}%
\pgfpathlineto{\pgfqpoint{5.145579in}{2.630553in}}%
\pgfpathlineto{\pgfqpoint{5.145771in}{2.628833in}}%
\pgfpathlineto{\pgfqpoint{5.146732in}{2.631938in}}%
\pgfpathlineto{\pgfqpoint{5.146924in}{2.628823in}}%
\pgfpathlineto{\pgfqpoint{5.147117in}{2.633146in}}%
\pgfpathlineto{\pgfqpoint{5.147693in}{2.630124in}}%
\pgfpathlineto{\pgfqpoint{5.149230in}{2.640064in}}%
\pgfpathlineto{\pgfqpoint{5.150960in}{2.650182in}}%
\pgfpathlineto{\pgfqpoint{5.151536in}{2.651345in}}%
\pgfpathlineto{\pgfqpoint{5.151729in}{2.655784in}}%
\pgfpathlineto{\pgfqpoint{5.152497in}{2.652584in}}%
\pgfpathlineto{\pgfqpoint{5.153650in}{2.647865in}}%
\pgfpathlineto{\pgfqpoint{5.153842in}{2.648418in}}%
\pgfpathlineto{\pgfqpoint{5.154227in}{2.652946in}}%
\pgfpathlineto{\pgfqpoint{5.154995in}{2.649397in}}%
\pgfpathlineto{\pgfqpoint{5.155188in}{2.648185in}}%
\pgfpathlineto{\pgfqpoint{5.155380in}{2.650130in}}%
\pgfpathlineto{\pgfqpoint{5.156533in}{2.661149in}}%
\pgfpathlineto{\pgfqpoint{5.156917in}{2.657229in}}%
\pgfpathlineto{\pgfqpoint{5.157686in}{2.652929in}}%
\pgfpathlineto{\pgfqpoint{5.157878in}{2.657187in}}%
\pgfpathlineto{\pgfqpoint{5.159031in}{2.664545in}}%
\pgfpathlineto{\pgfqpoint{5.159607in}{2.660679in}}%
\pgfpathlineto{\pgfqpoint{5.160760in}{2.656688in}}%
\pgfpathlineto{\pgfqpoint{5.159992in}{2.662029in}}%
\pgfpathlineto{\pgfqpoint{5.160953in}{2.659231in}}%
\pgfpathlineto{\pgfqpoint{5.161913in}{2.663041in}}%
\pgfpathlineto{\pgfqpoint{5.162106in}{2.659698in}}%
\pgfpathlineto{\pgfqpoint{5.164412in}{2.639407in}}%
\pgfpathlineto{\pgfqpoint{5.164796in}{2.641262in}}%
\pgfpathlineto{\pgfqpoint{5.165180in}{2.643188in}}%
\pgfpathlineto{\pgfqpoint{5.165949in}{2.638817in}}%
\pgfpathlineto{\pgfqpoint{5.166141in}{2.640602in}}%
\pgfpathlineto{\pgfqpoint{5.166526in}{2.634855in}}%
\pgfpathlineto{\pgfqpoint{5.167294in}{2.636235in}}%
\pgfpathlineto{\pgfqpoint{5.167486in}{2.634428in}}%
\pgfpathlineto{\pgfqpoint{5.170177in}{2.609849in}}%
\pgfpathlineto{\pgfqpoint{5.170945in}{2.614961in}}%
\pgfpathlineto{\pgfqpoint{5.171906in}{2.622952in}}%
\pgfpathlineto{\pgfqpoint{5.172098in}{2.621724in}}%
\pgfpathlineto{\pgfqpoint{5.172291in}{2.619081in}}%
\pgfpathlineto{\pgfqpoint{5.172867in}{2.623268in}}%
\pgfpathlineto{\pgfqpoint{5.173251in}{2.620483in}}%
\pgfpathlineto{\pgfqpoint{5.174789in}{2.626767in}}%
\pgfpathlineto{\pgfqpoint{5.174981in}{2.622445in}}%
\pgfpathlineto{\pgfqpoint{5.175557in}{2.628399in}}%
\pgfpathlineto{\pgfqpoint{5.175750in}{2.627738in}}%
\pgfpathlineto{\pgfqpoint{5.175942in}{2.626355in}}%
\pgfpathlineto{\pgfqpoint{5.176326in}{2.628838in}}%
\pgfpathlineto{\pgfqpoint{5.176518in}{2.628513in}}%
\pgfpathlineto{\pgfqpoint{5.177479in}{2.635685in}}%
\pgfpathlineto{\pgfqpoint{5.178056in}{2.634259in}}%
\pgfpathlineto{\pgfqpoint{5.178248in}{2.634372in}}%
\pgfpathlineto{\pgfqpoint{5.179977in}{2.620605in}}%
\pgfpathlineto{\pgfqpoint{5.180362in}{2.622361in}}%
\pgfpathlineto{\pgfqpoint{5.180746in}{2.619521in}}%
\pgfpathlineto{\pgfqpoint{5.181707in}{2.612783in}}%
\pgfpathlineto{\pgfqpoint{5.182091in}{2.614295in}}%
\pgfpathlineto{\pgfqpoint{5.182475in}{2.611108in}}%
\pgfpathlineto{\pgfqpoint{5.182860in}{2.616684in}}%
\pgfpathlineto{\pgfqpoint{5.183436in}{2.618914in}}%
\pgfpathlineto{\pgfqpoint{5.184013in}{2.617732in}}%
\pgfpathlineto{\pgfqpoint{5.184589in}{2.615271in}}%
\pgfpathlineto{\pgfqpoint{5.184974in}{2.617641in}}%
\pgfpathlineto{\pgfqpoint{5.185166in}{2.621023in}}%
\pgfpathlineto{\pgfqpoint{5.185742in}{2.612998in}}%
\pgfpathlineto{\pgfqpoint{5.185934in}{2.615532in}}%
\pgfpathlineto{\pgfqpoint{5.186511in}{2.614355in}}%
\pgfpathlineto{\pgfqpoint{5.186703in}{2.611526in}}%
\pgfpathlineto{\pgfqpoint{5.186895in}{2.615236in}}%
\pgfpathlineto{\pgfqpoint{5.187472in}{2.613595in}}%
\pgfpathlineto{\pgfqpoint{5.187856in}{2.617667in}}%
\pgfpathlineto{\pgfqpoint{5.188240in}{2.611656in}}%
\pgfpathlineto{\pgfqpoint{5.188433in}{2.613461in}}%
\pgfpathlineto{\pgfqpoint{5.189009in}{2.614113in}}%
\pgfpathlineto{\pgfqpoint{5.189201in}{2.612726in}}%
\pgfpathlineto{\pgfqpoint{5.189778in}{2.611347in}}%
\pgfpathlineto{\pgfqpoint{5.189970in}{2.608441in}}%
\pgfpathlineto{\pgfqpoint{5.190162in}{2.611510in}}%
\pgfpathlineto{\pgfqpoint{5.190739in}{2.609187in}}%
\pgfpathlineto{\pgfqpoint{5.191892in}{2.615109in}}%
\pgfpathlineto{\pgfqpoint{5.192084in}{2.612837in}}%
\pgfpathlineto{\pgfqpoint{5.192853in}{2.615971in}}%
\pgfpathlineto{\pgfqpoint{5.193045in}{2.614328in}}%
\pgfpathlineto{\pgfqpoint{5.193813in}{2.610927in}}%
\pgfpathlineto{\pgfqpoint{5.193429in}{2.614801in}}%
\pgfpathlineto{\pgfqpoint{5.194390in}{2.612883in}}%
\pgfpathlineto{\pgfqpoint{5.195351in}{2.612515in}}%
\pgfpathlineto{\pgfqpoint{5.195543in}{2.615154in}}%
\pgfpathlineto{\pgfqpoint{5.196888in}{2.607656in}}%
\pgfpathlineto{\pgfqpoint{5.198041in}{2.613738in}}%
\pgfpathlineto{\pgfqpoint{5.198233in}{2.611984in}}%
\pgfpathlineto{\pgfqpoint{5.198425in}{2.607520in}}%
\pgfpathlineto{\pgfqpoint{5.199194in}{2.611893in}}%
\pgfpathlineto{\pgfqpoint{5.199578in}{2.613687in}}%
\pgfpathlineto{\pgfqpoint{5.199963in}{2.616135in}}%
\pgfpathlineto{\pgfqpoint{5.200539in}{2.614343in}}%
\pgfpathlineto{\pgfqpoint{5.201116in}{2.612333in}}%
\pgfpathlineto{\pgfqpoint{5.201500in}{2.614259in}}%
\pgfpathlineto{\pgfqpoint{5.201884in}{2.615600in}}%
\pgfpathlineto{\pgfqpoint{5.202269in}{2.612969in}}%
\pgfpathlineto{\pgfqpoint{5.202461in}{2.613677in}}%
\pgfpathlineto{\pgfqpoint{5.203998in}{2.605095in}}%
\pgfpathlineto{\pgfqpoint{5.204190in}{2.605405in}}%
\pgfpathlineto{\pgfqpoint{5.204575in}{2.604030in}}%
\pgfpathlineto{\pgfqpoint{5.204959in}{2.606748in}}%
\pgfpathlineto{\pgfqpoint{5.205343in}{2.606234in}}%
\pgfpathlineto{\pgfqpoint{5.205728in}{2.612613in}}%
\pgfpathlineto{\pgfqpoint{5.206496in}{2.611510in}}%
\pgfpathlineto{\pgfqpoint{5.207842in}{2.606469in}}%
\pgfpathlineto{\pgfqpoint{5.208034in}{2.607547in}}%
\pgfpathlineto{\pgfqpoint{5.208995in}{2.614947in}}%
\pgfpathlineto{\pgfqpoint{5.209379in}{2.610571in}}%
\pgfpathlineto{\pgfqpoint{5.210340in}{2.606076in}}%
\pgfpathlineto{\pgfqpoint{5.210532in}{2.608936in}}%
\pgfpathlineto{\pgfqpoint{5.210916in}{2.609955in}}%
\pgfpathlineto{\pgfqpoint{5.211877in}{2.601209in}}%
\pgfpathlineto{\pgfqpoint{5.212646in}{2.602725in}}%
\pgfpathlineto{\pgfqpoint{5.212838in}{2.601568in}}%
\pgfpathlineto{\pgfqpoint{5.213030in}{2.605187in}}%
\pgfpathlineto{\pgfqpoint{5.213607in}{2.602032in}}%
\pgfpathlineto{\pgfqpoint{5.214183in}{2.600739in}}%
\pgfpathlineto{\pgfqpoint{5.214952in}{2.608098in}}%
\pgfpathlineto{\pgfqpoint{5.216681in}{2.595216in}}%
\pgfpathlineto{\pgfqpoint{5.217066in}{2.599054in}}%
\pgfpathlineto{\pgfqpoint{5.218027in}{2.598300in}}%
\pgfpathlineto{\pgfqpoint{5.219180in}{2.596074in}}%
\pgfpathlineto{\pgfqpoint{5.220525in}{2.585749in}}%
\pgfpathlineto{\pgfqpoint{5.221293in}{2.593762in}}%
\pgfpathlineto{\pgfqpoint{5.221870in}{2.588605in}}%
\pgfpathlineto{\pgfqpoint{5.222254in}{2.589815in}}%
\pgfpathlineto{\pgfqpoint{5.222639in}{2.585494in}}%
\pgfpathlineto{\pgfqpoint{5.223023in}{2.586750in}}%
\pgfpathlineto{\pgfqpoint{5.223792in}{2.582510in}}%
\pgfpathlineto{\pgfqpoint{5.224176in}{2.587793in}}%
\pgfpathlineto{\pgfqpoint{5.224945in}{2.586143in}}%
\pgfpathlineto{\pgfqpoint{5.225521in}{2.582531in}}%
\pgfpathlineto{\pgfqpoint{5.225905in}{2.587503in}}%
\pgfpathlineto{\pgfqpoint{5.226098in}{2.587302in}}%
\pgfpathlineto{\pgfqpoint{5.226866in}{2.589221in}}%
\pgfpathlineto{\pgfqpoint{5.228211in}{2.582232in}}%
\pgfpathlineto{\pgfqpoint{5.228788in}{2.582582in}}%
\pgfpathlineto{\pgfqpoint{5.230133in}{2.575529in}}%
\pgfpathlineto{\pgfqpoint{5.230325in}{2.575360in}}%
\pgfpathlineto{\pgfqpoint{5.231094in}{2.567458in}}%
\pgfpathlineto{\pgfqpoint{5.231478in}{2.571136in}}%
\pgfpathlineto{\pgfqpoint{5.233784in}{2.587246in}}%
\pgfpathlineto{\pgfqpoint{5.233976in}{2.584784in}}%
\pgfpathlineto{\pgfqpoint{5.234553in}{2.588017in}}%
\pgfpathlineto{\pgfqpoint{5.235706in}{2.591741in}}%
\pgfpathlineto{\pgfqpoint{5.235898in}{2.591314in}}%
\pgfpathlineto{\pgfqpoint{5.237820in}{2.580913in}}%
\pgfpathlineto{\pgfqpoint{5.238589in}{2.582388in}}%
\pgfpathlineto{\pgfqpoint{5.239165in}{2.586083in}}%
\pgfpathlineto{\pgfqpoint{5.239742in}{2.584830in}}%
\pgfpathlineto{\pgfqpoint{5.241279in}{2.571750in}}%
\pgfpathlineto{\pgfqpoint{5.241471in}{2.573702in}}%
\pgfpathlineto{\pgfqpoint{5.241855in}{2.569144in}}%
\pgfpathlineto{\pgfqpoint{5.242816in}{2.570233in}}%
\pgfpathlineto{\pgfqpoint{5.243201in}{2.565304in}}%
\pgfpathlineto{\pgfqpoint{5.244354in}{2.571406in}}%
\pgfpathlineto{\pgfqpoint{5.244738in}{2.570496in}}%
\pgfpathlineto{\pgfqpoint{5.244930in}{2.569079in}}%
\pgfpathlineto{\pgfqpoint{5.245314in}{2.572727in}}%
\pgfpathlineto{\pgfqpoint{5.245507in}{2.570924in}}%
\pgfpathlineto{\pgfqpoint{5.246852in}{2.577978in}}%
\pgfpathlineto{\pgfqpoint{5.247044in}{2.576102in}}%
\pgfpathlineto{\pgfqpoint{5.247236in}{2.575813in}}%
\pgfpathlineto{\pgfqpoint{5.247620in}{2.577185in}}%
\pgfpathlineto{\pgfqpoint{5.249926in}{2.590710in}}%
\pgfpathlineto{\pgfqpoint{5.250311in}{2.588518in}}%
\pgfpathlineto{\pgfqpoint{5.250695in}{2.591470in}}%
\pgfpathlineto{\pgfqpoint{5.250887in}{2.592909in}}%
\pgfpathlineto{\pgfqpoint{5.251272in}{2.589139in}}%
\pgfpathlineto{\pgfqpoint{5.251848in}{2.592141in}}%
\pgfpathlineto{\pgfqpoint{5.252040in}{2.592230in}}%
\pgfpathlineto{\pgfqpoint{5.252809in}{2.593308in}}%
\pgfpathlineto{\pgfqpoint{5.253193in}{2.588161in}}%
\pgfpathlineto{\pgfqpoint{5.254731in}{2.594129in}}%
\pgfpathlineto{\pgfqpoint{5.253962in}{2.587627in}}%
\pgfpathlineto{\pgfqpoint{5.254923in}{2.593166in}}%
\pgfpathlineto{\pgfqpoint{5.255307in}{2.591283in}}%
\pgfpathlineto{\pgfqpoint{5.255884in}{2.593864in}}%
\pgfpathlineto{\pgfqpoint{5.256268in}{2.595982in}}%
\pgfpathlineto{\pgfqpoint{5.256652in}{2.592091in}}%
\pgfpathlineto{\pgfqpoint{5.257421in}{2.595998in}}%
\pgfpathlineto{\pgfqpoint{5.257805in}{2.595586in}}%
\pgfpathlineto{\pgfqpoint{5.259343in}{2.587703in}}%
\pgfpathlineto{\pgfqpoint{5.259535in}{2.587656in}}%
\pgfpathlineto{\pgfqpoint{5.261072in}{2.578189in}}%
\pgfpathlineto{\pgfqpoint{5.261456in}{2.579031in}}%
\pgfpathlineto{\pgfqpoint{5.261649in}{2.582174in}}%
\pgfpathlineto{\pgfqpoint{5.262610in}{2.580041in}}%
\pgfpathlineto{\pgfqpoint{5.262802in}{2.579833in}}%
\pgfpathlineto{\pgfqpoint{5.263570in}{2.572243in}}%
\pgfpathlineto{\pgfqpoint{5.264147in}{2.574151in}}%
\pgfpathlineto{\pgfqpoint{5.264339in}{2.574940in}}%
\pgfpathlineto{\pgfqpoint{5.264531in}{2.573405in}}%
\pgfpathlineto{\pgfqpoint{5.265300in}{2.571160in}}%
\pgfpathlineto{\pgfqpoint{5.265492in}{2.572647in}}%
\pgfpathlineto{\pgfqpoint{5.266453in}{2.575533in}}%
\pgfpathlineto{\pgfqpoint{5.265876in}{2.572253in}}%
\pgfpathlineto{\pgfqpoint{5.266645in}{2.574553in}}%
\pgfpathlineto{\pgfqpoint{5.267029in}{2.573494in}}%
\pgfpathlineto{\pgfqpoint{5.267798in}{2.575580in}}%
\pgfpathlineto{\pgfqpoint{5.269912in}{2.563759in}}%
\pgfpathlineto{\pgfqpoint{5.270104in}{2.564091in}}%
\pgfpathlineto{\pgfqpoint{5.270296in}{2.567266in}}%
\pgfpathlineto{\pgfqpoint{5.271257in}{2.566409in}}%
\pgfpathlineto{\pgfqpoint{5.271449in}{2.564907in}}%
\pgfpathlineto{\pgfqpoint{5.272026in}{2.568354in}}%
\pgfpathlineto{\pgfqpoint{5.272218in}{2.566355in}}%
\pgfpathlineto{\pgfqpoint{5.272602in}{2.569733in}}%
\pgfpathlineto{\pgfqpoint{5.273179in}{2.565937in}}%
\pgfpathlineto{\pgfqpoint{5.273563in}{2.561036in}}%
\pgfpathlineto{\pgfqpoint{5.274140in}{2.565630in}}%
\pgfpathlineto{\pgfqpoint{5.274332in}{2.565915in}}%
\pgfpathlineto{\pgfqpoint{5.274908in}{2.573983in}}%
\pgfpathlineto{\pgfqpoint{5.275677in}{2.573671in}}%
\pgfpathlineto{\pgfqpoint{5.275869in}{2.573281in}}%
\pgfpathlineto{\pgfqpoint{5.277406in}{2.584068in}}%
\pgfpathlineto{\pgfqpoint{5.277599in}{2.583654in}}%
\pgfpathlineto{\pgfqpoint{5.277983in}{2.585300in}}%
\pgfpathlineto{\pgfqpoint{5.278175in}{2.584901in}}%
\pgfpathlineto{\pgfqpoint{5.278367in}{2.584839in}}%
\pgfpathlineto{\pgfqpoint{5.279136in}{2.577806in}}%
\pgfpathlineto{\pgfqpoint{5.279520in}{2.581567in}}%
\pgfpathlineto{\pgfqpoint{5.280097in}{2.580888in}}%
\pgfpathlineto{\pgfqpoint{5.280289in}{2.581939in}}%
\pgfpathlineto{\pgfqpoint{5.280481in}{2.582840in}}%
\pgfpathlineto{\pgfqpoint{5.280865in}{2.579689in}}%
\pgfpathlineto{\pgfqpoint{5.281058in}{2.580232in}}%
\pgfpathlineto{\pgfqpoint{5.282018in}{2.586207in}}%
\pgfpathlineto{\pgfqpoint{5.282211in}{2.590679in}}%
\pgfpathlineto{\pgfqpoint{5.282979in}{2.587487in}}%
\pgfpathlineto{\pgfqpoint{5.283171in}{2.586695in}}%
\pgfpathlineto{\pgfqpoint{5.283364in}{2.588088in}}%
\pgfpathlineto{\pgfqpoint{5.283748in}{2.587170in}}%
\pgfpathlineto{\pgfqpoint{5.284517in}{2.591495in}}%
\pgfpathlineto{\pgfqpoint{5.285093in}{2.589626in}}%
\pgfpathlineto{\pgfqpoint{5.285477in}{2.587482in}}%
\pgfpathlineto{\pgfqpoint{5.285670in}{2.590270in}}%
\pgfpathlineto{\pgfqpoint{5.285862in}{2.590186in}}%
\pgfpathlineto{\pgfqpoint{5.286823in}{2.599251in}}%
\pgfpathlineto{\pgfqpoint{5.287399in}{2.596779in}}%
\pgfpathlineto{\pgfqpoint{5.287784in}{2.594045in}}%
\pgfpathlineto{\pgfqpoint{5.288360in}{2.596946in}}%
\pgfpathlineto{\pgfqpoint{5.288744in}{2.598477in}}%
\pgfpathlineto{\pgfqpoint{5.289129in}{2.596232in}}%
\pgfpathlineto{\pgfqpoint{5.290666in}{2.590574in}}%
\pgfpathlineto{\pgfqpoint{5.290858in}{2.590958in}}%
\pgfpathlineto{\pgfqpoint{5.292011in}{2.602444in}}%
\pgfpathlineto{\pgfqpoint{5.292396in}{2.600253in}}%
\pgfpathlineto{\pgfqpoint{5.292588in}{2.600306in}}%
\pgfpathlineto{\pgfqpoint{5.292780in}{2.599304in}}%
\pgfpathlineto{\pgfqpoint{5.293356in}{2.599770in}}%
\pgfpathlineto{\pgfqpoint{5.294125in}{2.593900in}}%
\pgfpathlineto{\pgfqpoint{5.295662in}{2.602142in}}%
\pgfpathlineto{\pgfqpoint{5.295855in}{2.600797in}}%
\pgfpathlineto{\pgfqpoint{5.296431in}{2.595892in}}%
\pgfpathlineto{\pgfqpoint{5.296815in}{2.598640in}}%
\pgfpathlineto{\pgfqpoint{5.297008in}{2.600795in}}%
\pgfpathlineto{\pgfqpoint{5.297584in}{2.596318in}}%
\pgfpathlineto{\pgfqpoint{5.297776in}{2.599264in}}%
\pgfpathlineto{\pgfqpoint{5.298737in}{2.595846in}}%
\pgfpathlineto{\pgfqpoint{5.298161in}{2.599794in}}%
\pgfpathlineto{\pgfqpoint{5.299121in}{2.597679in}}%
\pgfpathlineto{\pgfqpoint{5.299314in}{2.600132in}}%
\pgfpathlineto{\pgfqpoint{5.299890in}{2.596965in}}%
\pgfpathlineto{\pgfqpoint{5.300274in}{2.598476in}}%
\pgfpathlineto{\pgfqpoint{5.300659in}{2.597265in}}%
\pgfpathlineto{\pgfqpoint{5.302196in}{2.591244in}}%
\pgfpathlineto{\pgfqpoint{5.303541in}{2.598664in}}%
\pgfpathlineto{\pgfqpoint{5.303926in}{2.597837in}}%
\pgfusepath{stroke}%
\end{pgfscope}%
\begin{pgfscope}%
\pgfsetrectcap%
\pgfsetmiterjoin%
\pgfsetlinewidth{0.000000pt}%
\definecolor{currentstroke}{rgb}{1.000000,1.000000,1.000000}%
\pgfsetstrokecolor{currentstroke}%
\pgfsetdash{}{0pt}%
\pgfpathmoveto{\pgfqpoint{3.286364in}{0.660000in}}%
\pgfpathlineto{\pgfqpoint{3.286364in}{2.760000in}}%
\pgfusepath{}%
\end{pgfscope}%
\begin{pgfscope}%
\pgfsetrectcap%
\pgfsetmiterjoin%
\pgfsetlinewidth{0.000000pt}%
\definecolor{currentstroke}{rgb}{1.000000,1.000000,1.000000}%
\pgfsetstrokecolor{currentstroke}%
\pgfsetdash{}{0pt}%
\pgfpathmoveto{\pgfqpoint{5.400000in}{0.660000in}}%
\pgfpathlineto{\pgfqpoint{5.400000in}{2.760000in}}%
\pgfusepath{}%
\end{pgfscope}%
\begin{pgfscope}%
\pgfsetrectcap%
\pgfsetmiterjoin%
\pgfsetlinewidth{0.000000pt}%
\definecolor{currentstroke}{rgb}{1.000000,1.000000,1.000000}%
\pgfsetstrokecolor{currentstroke}%
\pgfsetdash{}{0pt}%
\pgfpathmoveto{\pgfqpoint{3.286364in}{0.660000in}}%
\pgfpathlineto{\pgfqpoint{5.400000in}{0.660000in}}%
\pgfusepath{}%
\end{pgfscope}%
\begin{pgfscope}%
\pgfsetrectcap%
\pgfsetmiterjoin%
\pgfsetlinewidth{0.000000pt}%
\definecolor{currentstroke}{rgb}{1.000000,1.000000,1.000000}%
\pgfsetstrokecolor{currentstroke}%
\pgfsetdash{}{0pt}%
\pgfpathmoveto{\pgfqpoint{3.286364in}{2.760000in}}%
\pgfpathlineto{\pgfqpoint{5.400000in}{2.760000in}}%
\pgfusepath{}%
\end{pgfscope}%
\end{pgfpicture}%
\makeatother%
\endgroup%

    \caption{Simulation of four different Dyson Brownian motions with different dimension.\label{fig:four_dyson}}
\end{figure}

The ``uniform spacing'' behavior of the Dyson process contrasts with the path simulations for the Wishart process in Figure \ref{fig:wishart_comparison}. In this figure, we have the path for a Wishart process of dimension nine with the corresponding finite variation path. As we can see, the spacing of the particles with the biggest values tends to grow faster. This can be explained because for the Wishart process, the interaction term is proportional to the value of both functions.

\begin{figure} 
    %% Creator: Matplotlib, PGF backend
%%
%% To include the figure in your LaTeX document, write
%%   \input{<filename>.pgf}
%%
%% Make sure the required packages are loaded in your preamble
%%   \usepackage{pgf}
%%
%% Also ensure that all the required font packages are loaded; for instance,
%% the lmodern package is sometimes necessary when using math font.
%%   \usepackage{lmodern}
%%
%% Figures using additional raster images can only be included by \input if
%% they are in the same directory as the main LaTeX file. For loading figures
%% from other directories you can use the `import` package
%%   \usepackage{import}
%%
%% and then include the figures with
%%   \import{<path to file>}{<filename>.pgf}
%%
%% Matplotlib used the following preamble
%%   
%%   \makeatletter\@ifpackageloaded{underscore}{}{\usepackage[strings]{underscore}}\makeatother
%%
\begingroup%
\makeatletter%
\begin{pgfpicture}%
\pgfpathrectangle{\pgfpointorigin}{\pgfqpoint{6.000000in}{4.000000in}}%
\pgfusepath{use as bounding box, clip}%
\begin{pgfscope}%
\pgfsetbuttcap%
\pgfsetmiterjoin%
\definecolor{currentfill}{rgb}{1.000000,1.000000,1.000000}%
\pgfsetfillcolor{currentfill}%
\pgfsetlinewidth{0.000000pt}%
\definecolor{currentstroke}{rgb}{1.000000,1.000000,1.000000}%
\pgfsetstrokecolor{currentstroke}%
\pgfsetdash{}{0pt}%
\pgfpathmoveto{\pgfqpoint{0.000000in}{0.000000in}}%
\pgfpathlineto{\pgfqpoint{6.000000in}{0.000000in}}%
\pgfpathlineto{\pgfqpoint{6.000000in}{4.000000in}}%
\pgfpathlineto{\pgfqpoint{0.000000in}{4.000000in}}%
\pgfpathlineto{\pgfqpoint{0.000000in}{0.000000in}}%
\pgfpathclose%
\pgfusepath{fill}%
\end{pgfscope}%
\begin{pgfscope}%
\pgfsetbuttcap%
\pgfsetmiterjoin%
\definecolor{currentfill}{rgb}{0.917647,0.917647,0.949020}%
\pgfsetfillcolor{currentfill}%
\pgfsetlinewidth{0.000000pt}%
\definecolor{currentstroke}{rgb}{0.000000,0.000000,0.000000}%
\pgfsetstrokecolor{currentstroke}%
\pgfsetstrokeopacity{0.000000}%
\pgfsetdash{}{0pt}%
\pgfpathmoveto{\pgfqpoint{0.750000in}{0.440000in}}%
\pgfpathlineto{\pgfqpoint{5.400000in}{0.440000in}}%
\pgfpathlineto{\pgfqpoint{5.400000in}{3.520000in}}%
\pgfpathlineto{\pgfqpoint{0.750000in}{3.520000in}}%
\pgfpathlineto{\pgfqpoint{0.750000in}{0.440000in}}%
\pgfpathclose%
\pgfusepath{fill}%
\end{pgfscope}%
\begin{pgfscope}%
\pgfpathrectangle{\pgfqpoint{0.750000in}{0.440000in}}{\pgfqpoint{4.650000in}{3.080000in}}%
\pgfusepath{clip}%
\pgfsetroundcap%
\pgfsetroundjoin%
\pgfsetlinewidth{1.003750pt}%
\definecolor{currentstroke}{rgb}{1.000000,1.000000,1.000000}%
\pgfsetstrokecolor{currentstroke}%
\pgfsetdash{}{0pt}%
\pgfpathmoveto{\pgfqpoint{0.961364in}{0.440000in}}%
\pgfpathlineto{\pgfqpoint{0.961364in}{3.520000in}}%
\pgfusepath{stroke}%
\end{pgfscope}%
\begin{pgfscope}%
\definecolor{textcolor}{rgb}{0.150000,0.150000,0.150000}%
\pgfsetstrokecolor{textcolor}%
\pgfsetfillcolor{textcolor}%
\pgftext[x=0.961364in,y=0.342778in,,top]{\color{textcolor}\rmfamily\fontsize{10.000000}{12.000000}\selectfont \(\displaystyle {0.00}\)}%
\end{pgfscope}%
\begin{pgfscope}%
\pgfpathrectangle{\pgfqpoint{0.750000in}{0.440000in}}{\pgfqpoint{4.650000in}{3.080000in}}%
\pgfusepath{clip}%
\pgfsetroundcap%
\pgfsetroundjoin%
\pgfsetlinewidth{1.003750pt}%
\definecolor{currentstroke}{rgb}{1.000000,1.000000,1.000000}%
\pgfsetstrokecolor{currentstroke}%
\pgfsetdash{}{0pt}%
\pgfpathmoveto{\pgfqpoint{1.489773in}{0.440000in}}%
\pgfpathlineto{\pgfqpoint{1.489773in}{3.520000in}}%
\pgfusepath{stroke}%
\end{pgfscope}%
\begin{pgfscope}%
\definecolor{textcolor}{rgb}{0.150000,0.150000,0.150000}%
\pgfsetstrokecolor{textcolor}%
\pgfsetfillcolor{textcolor}%
\pgftext[x=1.489773in,y=0.342778in,,top]{\color{textcolor}\rmfamily\fontsize{10.000000}{12.000000}\selectfont \(\displaystyle {0.25}\)}%
\end{pgfscope}%
\begin{pgfscope}%
\pgfpathrectangle{\pgfqpoint{0.750000in}{0.440000in}}{\pgfqpoint{4.650000in}{3.080000in}}%
\pgfusepath{clip}%
\pgfsetroundcap%
\pgfsetroundjoin%
\pgfsetlinewidth{1.003750pt}%
\definecolor{currentstroke}{rgb}{1.000000,1.000000,1.000000}%
\pgfsetstrokecolor{currentstroke}%
\pgfsetdash{}{0pt}%
\pgfpathmoveto{\pgfqpoint{2.018182in}{0.440000in}}%
\pgfpathlineto{\pgfqpoint{2.018182in}{3.520000in}}%
\pgfusepath{stroke}%
\end{pgfscope}%
\begin{pgfscope}%
\definecolor{textcolor}{rgb}{0.150000,0.150000,0.150000}%
\pgfsetstrokecolor{textcolor}%
\pgfsetfillcolor{textcolor}%
\pgftext[x=2.018182in,y=0.342778in,,top]{\color{textcolor}\rmfamily\fontsize{10.000000}{12.000000}\selectfont \(\displaystyle {0.50}\)}%
\end{pgfscope}%
\begin{pgfscope}%
\pgfpathrectangle{\pgfqpoint{0.750000in}{0.440000in}}{\pgfqpoint{4.650000in}{3.080000in}}%
\pgfusepath{clip}%
\pgfsetroundcap%
\pgfsetroundjoin%
\pgfsetlinewidth{1.003750pt}%
\definecolor{currentstroke}{rgb}{1.000000,1.000000,1.000000}%
\pgfsetstrokecolor{currentstroke}%
\pgfsetdash{}{0pt}%
\pgfpathmoveto{\pgfqpoint{2.546591in}{0.440000in}}%
\pgfpathlineto{\pgfqpoint{2.546591in}{3.520000in}}%
\pgfusepath{stroke}%
\end{pgfscope}%
\begin{pgfscope}%
\definecolor{textcolor}{rgb}{0.150000,0.150000,0.150000}%
\pgfsetstrokecolor{textcolor}%
\pgfsetfillcolor{textcolor}%
\pgftext[x=2.546591in,y=0.342778in,,top]{\color{textcolor}\rmfamily\fontsize{10.000000}{12.000000}\selectfont \(\displaystyle {0.75}\)}%
\end{pgfscope}%
\begin{pgfscope}%
\pgfpathrectangle{\pgfqpoint{0.750000in}{0.440000in}}{\pgfqpoint{4.650000in}{3.080000in}}%
\pgfusepath{clip}%
\pgfsetroundcap%
\pgfsetroundjoin%
\pgfsetlinewidth{1.003750pt}%
\definecolor{currentstroke}{rgb}{1.000000,1.000000,1.000000}%
\pgfsetstrokecolor{currentstroke}%
\pgfsetdash{}{0pt}%
\pgfpathmoveto{\pgfqpoint{3.075000in}{0.440000in}}%
\pgfpathlineto{\pgfqpoint{3.075000in}{3.520000in}}%
\pgfusepath{stroke}%
\end{pgfscope}%
\begin{pgfscope}%
\definecolor{textcolor}{rgb}{0.150000,0.150000,0.150000}%
\pgfsetstrokecolor{textcolor}%
\pgfsetfillcolor{textcolor}%
\pgftext[x=3.075000in,y=0.342778in,,top]{\color{textcolor}\rmfamily\fontsize{10.000000}{12.000000}\selectfont \(\displaystyle {1.00}\)}%
\end{pgfscope}%
\begin{pgfscope}%
\pgfpathrectangle{\pgfqpoint{0.750000in}{0.440000in}}{\pgfqpoint{4.650000in}{3.080000in}}%
\pgfusepath{clip}%
\pgfsetroundcap%
\pgfsetroundjoin%
\pgfsetlinewidth{1.003750pt}%
\definecolor{currentstroke}{rgb}{1.000000,1.000000,1.000000}%
\pgfsetstrokecolor{currentstroke}%
\pgfsetdash{}{0pt}%
\pgfpathmoveto{\pgfqpoint{3.603409in}{0.440000in}}%
\pgfpathlineto{\pgfqpoint{3.603409in}{3.520000in}}%
\pgfusepath{stroke}%
\end{pgfscope}%
\begin{pgfscope}%
\definecolor{textcolor}{rgb}{0.150000,0.150000,0.150000}%
\pgfsetstrokecolor{textcolor}%
\pgfsetfillcolor{textcolor}%
\pgftext[x=3.603409in,y=0.342778in,,top]{\color{textcolor}\rmfamily\fontsize{10.000000}{12.000000}\selectfont \(\displaystyle {1.25}\)}%
\end{pgfscope}%
\begin{pgfscope}%
\pgfpathrectangle{\pgfqpoint{0.750000in}{0.440000in}}{\pgfqpoint{4.650000in}{3.080000in}}%
\pgfusepath{clip}%
\pgfsetroundcap%
\pgfsetroundjoin%
\pgfsetlinewidth{1.003750pt}%
\definecolor{currentstroke}{rgb}{1.000000,1.000000,1.000000}%
\pgfsetstrokecolor{currentstroke}%
\pgfsetdash{}{0pt}%
\pgfpathmoveto{\pgfqpoint{4.131818in}{0.440000in}}%
\pgfpathlineto{\pgfqpoint{4.131818in}{3.520000in}}%
\pgfusepath{stroke}%
\end{pgfscope}%
\begin{pgfscope}%
\definecolor{textcolor}{rgb}{0.150000,0.150000,0.150000}%
\pgfsetstrokecolor{textcolor}%
\pgfsetfillcolor{textcolor}%
\pgftext[x=4.131818in,y=0.342778in,,top]{\color{textcolor}\rmfamily\fontsize{10.000000}{12.000000}\selectfont \(\displaystyle {1.50}\)}%
\end{pgfscope}%
\begin{pgfscope}%
\pgfpathrectangle{\pgfqpoint{0.750000in}{0.440000in}}{\pgfqpoint{4.650000in}{3.080000in}}%
\pgfusepath{clip}%
\pgfsetroundcap%
\pgfsetroundjoin%
\pgfsetlinewidth{1.003750pt}%
\definecolor{currentstroke}{rgb}{1.000000,1.000000,1.000000}%
\pgfsetstrokecolor{currentstroke}%
\pgfsetdash{}{0pt}%
\pgfpathmoveto{\pgfqpoint{4.660227in}{0.440000in}}%
\pgfpathlineto{\pgfqpoint{4.660227in}{3.520000in}}%
\pgfusepath{stroke}%
\end{pgfscope}%
\begin{pgfscope}%
\definecolor{textcolor}{rgb}{0.150000,0.150000,0.150000}%
\pgfsetstrokecolor{textcolor}%
\pgfsetfillcolor{textcolor}%
\pgftext[x=4.660227in,y=0.342778in,,top]{\color{textcolor}\rmfamily\fontsize{10.000000}{12.000000}\selectfont \(\displaystyle {1.75}\)}%
\end{pgfscope}%
\begin{pgfscope}%
\pgfpathrectangle{\pgfqpoint{0.750000in}{0.440000in}}{\pgfqpoint{4.650000in}{3.080000in}}%
\pgfusepath{clip}%
\pgfsetroundcap%
\pgfsetroundjoin%
\pgfsetlinewidth{1.003750pt}%
\definecolor{currentstroke}{rgb}{1.000000,1.000000,1.000000}%
\pgfsetstrokecolor{currentstroke}%
\pgfsetdash{}{0pt}%
\pgfpathmoveto{\pgfqpoint{5.188636in}{0.440000in}}%
\pgfpathlineto{\pgfqpoint{5.188636in}{3.520000in}}%
\pgfusepath{stroke}%
\end{pgfscope}%
\begin{pgfscope}%
\definecolor{textcolor}{rgb}{0.150000,0.150000,0.150000}%
\pgfsetstrokecolor{textcolor}%
\pgfsetfillcolor{textcolor}%
\pgftext[x=5.188636in,y=0.342778in,,top]{\color{textcolor}\rmfamily\fontsize{10.000000}{12.000000}\selectfont \(\displaystyle {2.00}\)}%
\end{pgfscope}%
\begin{pgfscope}%
\definecolor{textcolor}{rgb}{0.150000,0.150000,0.150000}%
\pgfsetstrokecolor{textcolor}%
\pgfsetfillcolor{textcolor}%
\pgftext[x=3.075000in,y=0.163766in,,top]{\color{textcolor}\rmfamily\fontsize{11.000000}{13.200000}\selectfont Time (\(\displaystyle t\))}%
\end{pgfscope}%
\begin{pgfscope}%
\pgfpathrectangle{\pgfqpoint{0.750000in}{0.440000in}}{\pgfqpoint{4.650000in}{3.080000in}}%
\pgfusepath{clip}%
\pgfsetroundcap%
\pgfsetroundjoin%
\pgfsetlinewidth{1.003750pt}%
\definecolor{currentstroke}{rgb}{1.000000,1.000000,1.000000}%
\pgfsetstrokecolor{currentstroke}%
\pgfsetdash{}{0pt}%
\pgfpathmoveto{\pgfqpoint{0.750000in}{0.579049in}}%
\pgfpathlineto{\pgfqpoint{5.400000in}{0.579049in}}%
\pgfusepath{stroke}%
\end{pgfscope}%
\begin{pgfscope}%
\definecolor{textcolor}{rgb}{0.150000,0.150000,0.150000}%
\pgfsetstrokecolor{textcolor}%
\pgfsetfillcolor{textcolor}%
\pgftext[x=0.583333in, y=0.530823in, left, base]{\color{textcolor}\rmfamily\fontsize{10.000000}{12.000000}\selectfont \(\displaystyle {0}\)}%
\end{pgfscope}%
\begin{pgfscope}%
\pgfpathrectangle{\pgfqpoint{0.750000in}{0.440000in}}{\pgfqpoint{4.650000in}{3.080000in}}%
\pgfusepath{clip}%
\pgfsetroundcap%
\pgfsetroundjoin%
\pgfsetlinewidth{1.003750pt}%
\definecolor{currentstroke}{rgb}{1.000000,1.000000,1.000000}%
\pgfsetstrokecolor{currentstroke}%
\pgfsetdash{}{0pt}%
\pgfpathmoveto{\pgfqpoint{0.750000in}{1.196934in}}%
\pgfpathlineto{\pgfqpoint{5.400000in}{1.196934in}}%
\pgfusepath{stroke}%
\end{pgfscope}%
\begin{pgfscope}%
\definecolor{textcolor}{rgb}{0.150000,0.150000,0.150000}%
\pgfsetstrokecolor{textcolor}%
\pgfsetfillcolor{textcolor}%
\pgftext[x=0.513888in, y=1.148709in, left, base]{\color{textcolor}\rmfamily\fontsize{10.000000}{12.000000}\selectfont \(\displaystyle {20}\)}%
\end{pgfscope}%
\begin{pgfscope}%
\pgfpathrectangle{\pgfqpoint{0.750000in}{0.440000in}}{\pgfqpoint{4.650000in}{3.080000in}}%
\pgfusepath{clip}%
\pgfsetroundcap%
\pgfsetroundjoin%
\pgfsetlinewidth{1.003750pt}%
\definecolor{currentstroke}{rgb}{1.000000,1.000000,1.000000}%
\pgfsetstrokecolor{currentstroke}%
\pgfsetdash{}{0pt}%
\pgfpathmoveto{\pgfqpoint{0.750000in}{1.814820in}}%
\pgfpathlineto{\pgfqpoint{5.400000in}{1.814820in}}%
\pgfusepath{stroke}%
\end{pgfscope}%
\begin{pgfscope}%
\definecolor{textcolor}{rgb}{0.150000,0.150000,0.150000}%
\pgfsetstrokecolor{textcolor}%
\pgfsetfillcolor{textcolor}%
\pgftext[x=0.513888in, y=1.766594in, left, base]{\color{textcolor}\rmfamily\fontsize{10.000000}{12.000000}\selectfont \(\displaystyle {40}\)}%
\end{pgfscope}%
\begin{pgfscope}%
\pgfpathrectangle{\pgfqpoint{0.750000in}{0.440000in}}{\pgfqpoint{4.650000in}{3.080000in}}%
\pgfusepath{clip}%
\pgfsetroundcap%
\pgfsetroundjoin%
\pgfsetlinewidth{1.003750pt}%
\definecolor{currentstroke}{rgb}{1.000000,1.000000,1.000000}%
\pgfsetstrokecolor{currentstroke}%
\pgfsetdash{}{0pt}%
\pgfpathmoveto{\pgfqpoint{0.750000in}{2.432705in}}%
\pgfpathlineto{\pgfqpoint{5.400000in}{2.432705in}}%
\pgfusepath{stroke}%
\end{pgfscope}%
\begin{pgfscope}%
\definecolor{textcolor}{rgb}{0.150000,0.150000,0.150000}%
\pgfsetstrokecolor{textcolor}%
\pgfsetfillcolor{textcolor}%
\pgftext[x=0.513888in, y=2.384480in, left, base]{\color{textcolor}\rmfamily\fontsize{10.000000}{12.000000}\selectfont \(\displaystyle {60}\)}%
\end{pgfscope}%
\begin{pgfscope}%
\pgfpathrectangle{\pgfqpoint{0.750000in}{0.440000in}}{\pgfqpoint{4.650000in}{3.080000in}}%
\pgfusepath{clip}%
\pgfsetroundcap%
\pgfsetroundjoin%
\pgfsetlinewidth{1.003750pt}%
\definecolor{currentstroke}{rgb}{1.000000,1.000000,1.000000}%
\pgfsetstrokecolor{currentstroke}%
\pgfsetdash{}{0pt}%
\pgfpathmoveto{\pgfqpoint{0.750000in}{3.050590in}}%
\pgfpathlineto{\pgfqpoint{5.400000in}{3.050590in}}%
\pgfusepath{stroke}%
\end{pgfscope}%
\begin{pgfscope}%
\definecolor{textcolor}{rgb}{0.150000,0.150000,0.150000}%
\pgfsetstrokecolor{textcolor}%
\pgfsetfillcolor{textcolor}%
\pgftext[x=0.513888in, y=3.002365in, left, base]{\color{textcolor}\rmfamily\fontsize{10.000000}{12.000000}\selectfont \(\displaystyle {80}\)}%
\end{pgfscope}%
\begin{pgfscope}%
\definecolor{textcolor}{rgb}{0.150000,0.150000,0.150000}%
\pgfsetstrokecolor{textcolor}%
\pgfsetfillcolor{textcolor}%
\pgftext[x=0.458333in,y=1.980000in,,bottom,rotate=90.000000]{\color{textcolor}\rmfamily\fontsize{11.000000}{13.200000}\selectfont Position in space (\(\displaystyle \lambda_i\))}%
\end{pgfscope}%
\begin{pgfscope}%
\pgfpathrectangle{\pgfqpoint{0.750000in}{0.440000in}}{\pgfqpoint{4.650000in}{3.080000in}}%
\pgfusepath{clip}%
\pgfsetroundcap%
\pgfsetroundjoin%
\pgfsetlinewidth{1.003750pt}%
\definecolor{currentstroke}{rgb}{0.215686,0.494118,0.721569}%
\pgfsetstrokecolor{currentstroke}%
\pgfsetdash{}{0pt}%
\pgfpathmoveto{\pgfqpoint{0.961364in}{0.582138in}}%
\pgfpathlineto{\pgfqpoint{1.046334in}{0.585778in}}%
\pgfpathlineto{\pgfqpoint{1.067577in}{0.588363in}}%
\pgfpathlineto{\pgfqpoint{1.088819in}{0.589519in}}%
\pgfpathlineto{\pgfqpoint{1.131304in}{0.594082in}}%
\pgfpathlineto{\pgfqpoint{1.173789in}{0.591722in}}%
\pgfpathlineto{\pgfqpoint{1.195032in}{0.595510in}}%
\pgfpathlineto{\pgfqpoint{1.216275in}{0.592360in}}%
\pgfpathlineto{\pgfqpoint{1.258760in}{0.591397in}}%
\pgfpathlineto{\pgfqpoint{1.280002in}{0.591360in}}%
\pgfpathlineto{\pgfqpoint{1.322487in}{0.588482in}}%
\pgfpathlineto{\pgfqpoint{1.428700in}{0.584053in}}%
\pgfpathlineto{\pgfqpoint{1.471185in}{0.585318in}}%
\pgfpathlineto{\pgfqpoint{1.492428in}{0.583536in}}%
\pgfpathlineto{\pgfqpoint{1.556156in}{0.586680in}}%
\pgfpathlineto{\pgfqpoint{1.577398in}{0.588331in}}%
\pgfpathlineto{\pgfqpoint{1.598641in}{0.587716in}}%
\pgfpathlineto{\pgfqpoint{1.704854in}{0.592143in}}%
\pgfpathlineto{\pgfqpoint{1.726096in}{0.590727in}}%
\pgfpathlineto{\pgfqpoint{1.747339in}{0.590560in}}%
\pgfpathlineto{\pgfqpoint{1.768582in}{0.592418in}}%
\pgfpathlineto{\pgfqpoint{1.789824in}{0.592829in}}%
\pgfpathlineto{\pgfqpoint{1.811067in}{0.594976in}}%
\pgfpathlineto{\pgfqpoint{1.832309in}{0.594101in}}%
\pgfpathlineto{\pgfqpoint{1.874794in}{0.589159in}}%
\pgfpathlineto{\pgfqpoint{1.896037in}{0.590358in}}%
\pgfpathlineto{\pgfqpoint{1.917280in}{0.587941in}}%
\pgfpathlineto{\pgfqpoint{1.938522in}{0.591835in}}%
\pgfpathlineto{\pgfqpoint{1.959765in}{0.591927in}}%
\pgfpathlineto{\pgfqpoint{1.981007in}{0.588869in}}%
\pgfpathlineto{\pgfqpoint{2.002250in}{0.587101in}}%
\pgfpathlineto{\pgfqpoint{2.023492in}{0.587925in}}%
\pgfpathlineto{\pgfqpoint{2.044735in}{0.591767in}}%
\pgfpathlineto{\pgfqpoint{2.065978in}{0.592475in}}%
\pgfpathlineto{\pgfqpoint{2.087220in}{0.590579in}}%
\pgfpathlineto{\pgfqpoint{2.108463in}{0.592564in}}%
\pgfpathlineto{\pgfqpoint{2.150948in}{0.592140in}}%
\pgfpathlineto{\pgfqpoint{2.172190in}{0.589577in}}%
\pgfpathlineto{\pgfqpoint{2.257161in}{0.585545in}}%
\pgfpathlineto{\pgfqpoint{2.278403in}{0.586654in}}%
\pgfpathlineto{\pgfqpoint{2.299646in}{0.589394in}}%
\pgfpathlineto{\pgfqpoint{2.342131in}{0.591487in}}%
\pgfpathlineto{\pgfqpoint{2.363374in}{0.589534in}}%
\pgfpathlineto{\pgfqpoint{2.384616in}{0.586337in}}%
\pgfpathlineto{\pgfqpoint{2.405859in}{0.585834in}}%
\pgfpathlineto{\pgfqpoint{2.427101in}{0.584119in}}%
\pgfpathlineto{\pgfqpoint{2.448344in}{0.585261in}}%
\pgfpathlineto{\pgfqpoint{2.490829in}{0.590055in}}%
\pgfpathlineto{\pgfqpoint{2.533314in}{0.587719in}}%
\pgfpathlineto{\pgfqpoint{2.554557in}{0.587735in}}%
\pgfpathlineto{\pgfqpoint{2.575799in}{0.590690in}}%
\pgfpathlineto{\pgfqpoint{2.639527in}{0.594090in}}%
\pgfpathlineto{\pgfqpoint{2.660770in}{0.588678in}}%
\pgfpathlineto{\pgfqpoint{2.703255in}{0.584905in}}%
\pgfpathlineto{\pgfqpoint{2.724497in}{0.586518in}}%
\pgfpathlineto{\pgfqpoint{2.766983in}{0.582135in}}%
\pgfpathlineto{\pgfqpoint{2.788225in}{0.583019in}}%
\pgfpathlineto{\pgfqpoint{2.809468in}{0.582319in}}%
\pgfpathlineto{\pgfqpoint{2.830710in}{0.584435in}}%
\pgfpathlineto{\pgfqpoint{2.873196in}{0.583815in}}%
\pgfpathlineto{\pgfqpoint{2.915681in}{0.583200in}}%
\pgfpathlineto{\pgfqpoint{2.979408in}{0.587054in}}%
\pgfpathlineto{\pgfqpoint{3.000651in}{0.585832in}}%
\pgfpathlineto{\pgfqpoint{3.043136in}{0.588674in}}%
\pgfpathlineto{\pgfqpoint{3.085621in}{0.586405in}}%
\pgfpathlineto{\pgfqpoint{3.106864in}{0.589216in}}%
\pgfpathlineto{\pgfqpoint{3.128106in}{0.590608in}}%
\pgfpathlineto{\pgfqpoint{3.149349in}{0.587527in}}%
\pgfpathlineto{\pgfqpoint{3.170592in}{0.589791in}}%
\pgfpathlineto{\pgfqpoint{3.191834in}{0.587852in}}%
\pgfpathlineto{\pgfqpoint{3.213077in}{0.588805in}}%
\pgfpathlineto{\pgfqpoint{3.234319in}{0.585334in}}%
\pgfpathlineto{\pgfqpoint{3.255562in}{0.587377in}}%
\pgfpathlineto{\pgfqpoint{3.298047in}{0.589633in}}%
\pgfpathlineto{\pgfqpoint{3.319290in}{0.588789in}}%
\pgfpathlineto{\pgfqpoint{3.361775in}{0.589113in}}%
\pgfpathlineto{\pgfqpoint{3.404260in}{0.590660in}}%
\pgfpathlineto{\pgfqpoint{3.425503in}{0.589336in}}%
\pgfpathlineto{\pgfqpoint{3.446745in}{0.591314in}}%
\pgfpathlineto{\pgfqpoint{3.531715in}{0.587523in}}%
\pgfpathlineto{\pgfqpoint{3.574201in}{0.589943in}}%
\pgfpathlineto{\pgfqpoint{3.637928in}{0.588144in}}%
\pgfpathlineto{\pgfqpoint{3.659171in}{0.591012in}}%
\pgfpathlineto{\pgfqpoint{3.680413in}{0.588114in}}%
\pgfpathlineto{\pgfqpoint{3.701656in}{0.589895in}}%
\pgfpathlineto{\pgfqpoint{3.722899in}{0.588962in}}%
\pgfpathlineto{\pgfqpoint{3.850354in}{0.590598in}}%
\pgfpathlineto{\pgfqpoint{3.871597in}{0.588011in}}%
\pgfpathlineto{\pgfqpoint{3.892839in}{0.588615in}}%
\pgfpathlineto{\pgfqpoint{3.914082in}{0.587479in}}%
\pgfpathlineto{\pgfqpoint{3.935324in}{0.584672in}}%
\pgfpathlineto{\pgfqpoint{4.020295in}{0.585735in}}%
\pgfpathlineto{\pgfqpoint{4.105265in}{0.585510in}}%
\pgfpathlineto{\pgfqpoint{4.126508in}{0.586230in}}%
\pgfpathlineto{\pgfqpoint{4.147750in}{0.588083in}}%
\pgfpathlineto{\pgfqpoint{4.168993in}{0.588556in}}%
\pgfpathlineto{\pgfqpoint{4.190235in}{0.590721in}}%
\pgfpathlineto{\pgfqpoint{4.232720in}{0.587555in}}%
\pgfpathlineto{\pgfqpoint{4.253963in}{0.589972in}}%
\pgfpathlineto{\pgfqpoint{4.275206in}{0.586969in}}%
\pgfpathlineto{\pgfqpoint{4.338933in}{0.585857in}}%
\pgfpathlineto{\pgfqpoint{4.360176in}{0.588579in}}%
\pgfpathlineto{\pgfqpoint{4.381418in}{0.589569in}}%
\pgfpathlineto{\pgfqpoint{4.402661in}{0.588903in}}%
\pgfpathlineto{\pgfqpoint{4.423904in}{0.590994in}}%
\pgfpathlineto{\pgfqpoint{4.445146in}{0.589883in}}%
\pgfpathlineto{\pgfqpoint{4.487631in}{0.589577in}}%
\pgfpathlineto{\pgfqpoint{4.593844in}{0.583857in}}%
\pgfpathlineto{\pgfqpoint{4.763785in}{0.584583in}}%
\pgfpathlineto{\pgfqpoint{4.806270in}{0.580951in}}%
\pgfpathlineto{\pgfqpoint{4.848755in}{0.581014in}}%
\pgfpathlineto{\pgfqpoint{4.869998in}{0.582229in}}%
\pgfpathlineto{\pgfqpoint{4.891240in}{0.581503in}}%
\pgfpathlineto{\pgfqpoint{4.912483in}{0.581981in}}%
\pgfpathlineto{\pgfqpoint{4.933725in}{0.580741in}}%
\pgfpathlineto{\pgfqpoint{4.954968in}{0.581259in}}%
\pgfpathlineto{\pgfqpoint{4.976211in}{0.580299in}}%
\pgfpathlineto{\pgfqpoint{4.997453in}{0.581954in}}%
\pgfpathlineto{\pgfqpoint{5.018696in}{0.582367in}}%
\pgfpathlineto{\pgfqpoint{5.061181in}{0.580419in}}%
\pgfpathlineto{\pgfqpoint{5.103666in}{0.580000in}}%
\pgfpathlineto{\pgfqpoint{5.146151in}{0.581076in}}%
\pgfpathlineto{\pgfqpoint{5.167394in}{0.581582in}}%
\pgfpathlineto{\pgfqpoint{5.188636in}{0.580836in}}%
\pgfpathlineto{\pgfqpoint{5.188636in}{0.580836in}}%
\pgfusepath{stroke}%
\end{pgfscope}%
\begin{pgfscope}%
\pgfpathrectangle{\pgfqpoint{0.750000in}{0.440000in}}{\pgfqpoint{4.650000in}{3.080000in}}%
\pgfusepath{clip}%
\pgfsetroundcap%
\pgfsetroundjoin%
\pgfsetlinewidth{1.003750pt}%
\definecolor{currentstroke}{rgb}{1.000000,0.498039,0.000000}%
\pgfsetstrokecolor{currentstroke}%
\pgfsetdash{}{0pt}%
\pgfpathmoveto{\pgfqpoint{0.961364in}{0.658988in}}%
\pgfpathlineto{\pgfqpoint{0.982606in}{0.662166in}}%
\pgfpathlineto{\pgfqpoint{1.003849in}{0.655790in}}%
\pgfpathlineto{\pgfqpoint{1.025091in}{0.656780in}}%
\pgfpathlineto{\pgfqpoint{1.046334in}{0.661529in}}%
\pgfpathlineto{\pgfqpoint{1.067577in}{0.659623in}}%
\pgfpathlineto{\pgfqpoint{1.088819in}{0.659942in}}%
\pgfpathlineto{\pgfqpoint{1.110062in}{0.665841in}}%
\pgfpathlineto{\pgfqpoint{1.131304in}{0.661359in}}%
\pgfpathlineto{\pgfqpoint{1.152547in}{0.673436in}}%
\pgfpathlineto{\pgfqpoint{1.173789in}{0.667396in}}%
\pgfpathlineto{\pgfqpoint{1.195032in}{0.672499in}}%
\pgfpathlineto{\pgfqpoint{1.216275in}{0.668742in}}%
\pgfpathlineto{\pgfqpoint{1.237517in}{0.671768in}}%
\pgfpathlineto{\pgfqpoint{1.258760in}{0.677662in}}%
\pgfpathlineto{\pgfqpoint{1.280002in}{0.673009in}}%
\pgfpathlineto{\pgfqpoint{1.301245in}{0.676373in}}%
\pgfpathlineto{\pgfqpoint{1.322487in}{0.675088in}}%
\pgfpathlineto{\pgfqpoint{1.343730in}{0.670075in}}%
\pgfpathlineto{\pgfqpoint{1.364973in}{0.668327in}}%
\pgfpathlineto{\pgfqpoint{1.386215in}{0.653584in}}%
\pgfpathlineto{\pgfqpoint{1.407458in}{0.644056in}}%
\pgfpathlineto{\pgfqpoint{1.428700in}{0.648644in}}%
\pgfpathlineto{\pgfqpoint{1.449943in}{0.637524in}}%
\pgfpathlineto{\pgfqpoint{1.471185in}{0.635680in}}%
\pgfpathlineto{\pgfqpoint{1.492428in}{0.635498in}}%
\pgfpathlineto{\pgfqpoint{1.556156in}{0.627357in}}%
\pgfpathlineto{\pgfqpoint{1.577398in}{0.628177in}}%
\pgfpathlineto{\pgfqpoint{1.598641in}{0.624685in}}%
\pgfpathlineto{\pgfqpoint{1.619884in}{0.627608in}}%
\pgfpathlineto{\pgfqpoint{1.641126in}{0.632376in}}%
\pgfpathlineto{\pgfqpoint{1.662369in}{0.628266in}}%
\pgfpathlineto{\pgfqpoint{1.726096in}{0.637703in}}%
\pgfpathlineto{\pgfqpoint{1.768582in}{0.636272in}}%
\pgfpathlineto{\pgfqpoint{1.789824in}{0.639125in}}%
\pgfpathlineto{\pgfqpoint{1.811067in}{0.639568in}}%
\pgfpathlineto{\pgfqpoint{1.832309in}{0.636445in}}%
\pgfpathlineto{\pgfqpoint{1.853552in}{0.640195in}}%
\pgfpathlineto{\pgfqpoint{1.874794in}{0.637061in}}%
\pgfpathlineto{\pgfqpoint{1.896037in}{0.643697in}}%
\pgfpathlineto{\pgfqpoint{1.917280in}{0.645952in}}%
\pgfpathlineto{\pgfqpoint{1.938522in}{0.640699in}}%
\pgfpathlineto{\pgfqpoint{1.959765in}{0.636970in}}%
\pgfpathlineto{\pgfqpoint{1.981007in}{0.640091in}}%
\pgfpathlineto{\pgfqpoint{2.002250in}{0.645461in}}%
\pgfpathlineto{\pgfqpoint{2.023492in}{0.642854in}}%
\pgfpathlineto{\pgfqpoint{2.044735in}{0.638531in}}%
\pgfpathlineto{\pgfqpoint{2.065978in}{0.638545in}}%
\pgfpathlineto{\pgfqpoint{2.087220in}{0.641327in}}%
\pgfpathlineto{\pgfqpoint{2.129705in}{0.638843in}}%
\pgfpathlineto{\pgfqpoint{2.150948in}{0.637715in}}%
\pgfpathlineto{\pgfqpoint{2.172190in}{0.634676in}}%
\pgfpathlineto{\pgfqpoint{2.193433in}{0.633918in}}%
\pgfpathlineto{\pgfqpoint{2.214676in}{0.628662in}}%
\pgfpathlineto{\pgfqpoint{2.235918in}{0.631527in}}%
\pgfpathlineto{\pgfqpoint{2.257161in}{0.626529in}}%
\pgfpathlineto{\pgfqpoint{2.299646in}{0.619625in}}%
\pgfpathlineto{\pgfqpoint{2.320889in}{0.619885in}}%
\pgfpathlineto{\pgfqpoint{2.342131in}{0.615290in}}%
\pgfpathlineto{\pgfqpoint{2.363374in}{0.617695in}}%
\pgfpathlineto{\pgfqpoint{2.384616in}{0.611363in}}%
\pgfpathlineto{\pgfqpoint{2.405859in}{0.616885in}}%
\pgfpathlineto{\pgfqpoint{2.448344in}{0.621014in}}%
\pgfpathlineto{\pgfqpoint{2.469587in}{0.622489in}}%
\pgfpathlineto{\pgfqpoint{2.490829in}{0.616310in}}%
\pgfpathlineto{\pgfqpoint{2.512072in}{0.614885in}}%
\pgfpathlineto{\pgfqpoint{2.533314in}{0.618404in}}%
\pgfpathlineto{\pgfqpoint{2.575799in}{0.616272in}}%
\pgfpathlineto{\pgfqpoint{2.618285in}{0.617903in}}%
\pgfpathlineto{\pgfqpoint{2.660770in}{0.617765in}}%
\pgfpathlineto{\pgfqpoint{2.682012in}{0.621079in}}%
\pgfpathlineto{\pgfqpoint{2.703255in}{0.622622in}}%
\pgfpathlineto{\pgfqpoint{2.745740in}{0.621817in}}%
\pgfpathlineto{\pgfqpoint{2.766983in}{0.624073in}}%
\pgfpathlineto{\pgfqpoint{2.788225in}{0.614256in}}%
\pgfpathlineto{\pgfqpoint{2.809468in}{0.617046in}}%
\pgfpathlineto{\pgfqpoint{2.830710in}{0.612976in}}%
\pgfpathlineto{\pgfqpoint{2.851953in}{0.618931in}}%
\pgfpathlineto{\pgfqpoint{2.873196in}{0.622193in}}%
\pgfpathlineto{\pgfqpoint{2.936923in}{0.618481in}}%
\pgfpathlineto{\pgfqpoint{2.979408in}{0.623218in}}%
\pgfpathlineto{\pgfqpoint{3.021894in}{0.634598in}}%
\pgfpathlineto{\pgfqpoint{3.043136in}{0.627413in}}%
\pgfpathlineto{\pgfqpoint{3.064379in}{0.628504in}}%
\pgfpathlineto{\pgfqpoint{3.085621in}{0.631179in}}%
\pgfpathlineto{\pgfqpoint{3.106864in}{0.625575in}}%
\pgfpathlineto{\pgfqpoint{3.128106in}{0.627857in}}%
\pgfpathlineto{\pgfqpoint{3.149349in}{0.622636in}}%
\pgfpathlineto{\pgfqpoint{3.170592in}{0.629310in}}%
\pgfpathlineto{\pgfqpoint{3.191834in}{0.625956in}}%
\pgfpathlineto{\pgfqpoint{3.213077in}{0.627187in}}%
\pgfpathlineto{\pgfqpoint{3.234319in}{0.624393in}}%
\pgfpathlineto{\pgfqpoint{3.298047in}{0.630821in}}%
\pgfpathlineto{\pgfqpoint{3.319290in}{0.639312in}}%
\pgfpathlineto{\pgfqpoint{3.340532in}{0.640876in}}%
\pgfpathlineto{\pgfqpoint{3.361775in}{0.646462in}}%
\pgfpathlineto{\pgfqpoint{3.404260in}{0.648910in}}%
\pgfpathlineto{\pgfqpoint{3.425503in}{0.643943in}}%
\pgfpathlineto{\pgfqpoint{3.446745in}{0.640723in}}%
\pgfpathlineto{\pgfqpoint{3.467988in}{0.640346in}}%
\pgfpathlineto{\pgfqpoint{3.489230in}{0.637294in}}%
\pgfpathlineto{\pgfqpoint{3.510473in}{0.631906in}}%
\pgfpathlineto{\pgfqpoint{3.531715in}{0.634224in}}%
\pgfpathlineto{\pgfqpoint{3.552958in}{0.631399in}}%
\pgfpathlineto{\pgfqpoint{3.595443in}{0.640064in}}%
\pgfpathlineto{\pgfqpoint{3.616686in}{0.634451in}}%
\pgfpathlineto{\pgfqpoint{3.637928in}{0.639857in}}%
\pgfpathlineto{\pgfqpoint{3.659171in}{0.641323in}}%
\pgfpathlineto{\pgfqpoint{3.680413in}{0.641435in}}%
\pgfpathlineto{\pgfqpoint{3.701656in}{0.639515in}}%
\pgfpathlineto{\pgfqpoint{3.722899in}{0.639946in}}%
\pgfpathlineto{\pgfqpoint{3.765384in}{0.636043in}}%
\pgfpathlineto{\pgfqpoint{3.807869in}{0.634447in}}%
\pgfpathlineto{\pgfqpoint{3.829111in}{0.636026in}}%
\pgfpathlineto{\pgfqpoint{3.850354in}{0.630931in}}%
\pgfpathlineto{\pgfqpoint{3.871597in}{0.627794in}}%
\pgfpathlineto{\pgfqpoint{3.914082in}{0.625036in}}%
\pgfpathlineto{\pgfqpoint{3.935324in}{0.629309in}}%
\pgfpathlineto{\pgfqpoint{3.956567in}{0.629320in}}%
\pgfpathlineto{\pgfqpoint{3.977810in}{0.621156in}}%
\pgfpathlineto{\pgfqpoint{3.999052in}{0.621897in}}%
\pgfpathlineto{\pgfqpoint{4.020295in}{0.626352in}}%
\pgfpathlineto{\pgfqpoint{4.041537in}{0.623757in}}%
\pgfpathlineto{\pgfqpoint{4.062780in}{0.627929in}}%
\pgfpathlineto{\pgfqpoint{4.190235in}{0.622896in}}%
\pgfpathlineto{\pgfqpoint{4.211478in}{0.627191in}}%
\pgfpathlineto{\pgfqpoint{4.232720in}{0.622429in}}%
\pgfpathlineto{\pgfqpoint{4.275206in}{0.619487in}}%
\pgfpathlineto{\pgfqpoint{4.296448in}{0.613637in}}%
\pgfpathlineto{\pgfqpoint{4.338933in}{0.620887in}}%
\pgfpathlineto{\pgfqpoint{4.360176in}{0.619083in}}%
\pgfpathlineto{\pgfqpoint{4.381418in}{0.620513in}}%
\pgfpathlineto{\pgfqpoint{4.402661in}{0.623800in}}%
\pgfpathlineto{\pgfqpoint{4.423904in}{0.617190in}}%
\pgfpathlineto{\pgfqpoint{4.445146in}{0.620508in}}%
\pgfpathlineto{\pgfqpoint{4.466389in}{0.620612in}}%
\pgfpathlineto{\pgfqpoint{4.508874in}{0.617123in}}%
\pgfpathlineto{\pgfqpoint{4.530116in}{0.617714in}}%
\pgfpathlineto{\pgfqpoint{4.551359in}{0.619866in}}%
\pgfpathlineto{\pgfqpoint{4.593844in}{0.619613in}}%
\pgfpathlineto{\pgfqpoint{4.615087in}{0.624487in}}%
\pgfpathlineto{\pgfqpoint{4.678815in}{0.624545in}}%
\pgfpathlineto{\pgfqpoint{4.700057in}{0.626991in}}%
\pgfpathlineto{\pgfqpoint{4.721300in}{0.627793in}}%
\pgfpathlineto{\pgfqpoint{4.742542in}{0.630405in}}%
\pgfpathlineto{\pgfqpoint{4.763785in}{0.625511in}}%
\pgfpathlineto{\pgfqpoint{4.785027in}{0.624642in}}%
\pgfpathlineto{\pgfqpoint{4.806270in}{0.627060in}}%
\pgfpathlineto{\pgfqpoint{4.827513in}{0.633648in}}%
\pgfpathlineto{\pgfqpoint{4.848755in}{0.628940in}}%
\pgfpathlineto{\pgfqpoint{4.869998in}{0.629447in}}%
\pgfpathlineto{\pgfqpoint{4.891240in}{0.619336in}}%
\pgfpathlineto{\pgfqpoint{4.912483in}{0.625085in}}%
\pgfpathlineto{\pgfqpoint{4.933725in}{0.625287in}}%
\pgfpathlineto{\pgfqpoint{4.954968in}{0.632582in}}%
\pgfpathlineto{\pgfqpoint{4.976211in}{0.638391in}}%
\pgfpathlineto{\pgfqpoint{4.997453in}{0.639759in}}%
\pgfpathlineto{\pgfqpoint{5.018696in}{0.643507in}}%
\pgfpathlineto{\pgfqpoint{5.039938in}{0.644787in}}%
\pgfpathlineto{\pgfqpoint{5.061181in}{0.641078in}}%
\pgfpathlineto{\pgfqpoint{5.082423in}{0.633113in}}%
\pgfpathlineto{\pgfqpoint{5.103666in}{0.632166in}}%
\pgfpathlineto{\pgfqpoint{5.124909in}{0.633041in}}%
\pgfpathlineto{\pgfqpoint{5.167394in}{0.627718in}}%
\pgfpathlineto{\pgfqpoint{5.188636in}{0.635495in}}%
\pgfpathlineto{\pgfqpoint{5.188636in}{0.635495in}}%
\pgfusepath{stroke}%
\end{pgfscope}%
\begin{pgfscope}%
\pgfpathrectangle{\pgfqpoint{0.750000in}{0.440000in}}{\pgfqpoint{4.650000in}{3.080000in}}%
\pgfusepath{clip}%
\pgfsetroundcap%
\pgfsetroundjoin%
\pgfsetlinewidth{1.003750pt}%
\definecolor{currentstroke}{rgb}{0.301961,0.686275,0.290196}%
\pgfsetstrokecolor{currentstroke}%
\pgfsetdash{}{0pt}%
\pgfpathmoveto{\pgfqpoint{0.961364in}{0.735837in}}%
\pgfpathlineto{\pgfqpoint{0.982606in}{0.737007in}}%
\pgfpathlineto{\pgfqpoint{1.003849in}{0.731670in}}%
\pgfpathlineto{\pgfqpoint{1.025091in}{0.736880in}}%
\pgfpathlineto{\pgfqpoint{1.046334in}{0.713306in}}%
\pgfpathlineto{\pgfqpoint{1.067577in}{0.707410in}}%
\pgfpathlineto{\pgfqpoint{1.088819in}{0.704242in}}%
\pgfpathlineto{\pgfqpoint{1.110062in}{0.713859in}}%
\pgfpathlineto{\pgfqpoint{1.131304in}{0.711303in}}%
\pgfpathlineto{\pgfqpoint{1.173789in}{0.703646in}}%
\pgfpathlineto{\pgfqpoint{1.195032in}{0.697566in}}%
\pgfpathlineto{\pgfqpoint{1.237517in}{0.713759in}}%
\pgfpathlineto{\pgfqpoint{1.258760in}{0.701079in}}%
\pgfpathlineto{\pgfqpoint{1.280002in}{0.708539in}}%
\pgfpathlineto{\pgfqpoint{1.301245in}{0.697734in}}%
\pgfpathlineto{\pgfqpoint{1.343730in}{0.703207in}}%
\pgfpathlineto{\pgfqpoint{1.364973in}{0.702434in}}%
\pgfpathlineto{\pgfqpoint{1.386215in}{0.712514in}}%
\pgfpathlineto{\pgfqpoint{1.428700in}{0.696674in}}%
\pgfpathlineto{\pgfqpoint{1.449943in}{0.703214in}}%
\pgfpathlineto{\pgfqpoint{1.471185in}{0.704277in}}%
\pgfpathlineto{\pgfqpoint{1.492428in}{0.707457in}}%
\pgfpathlineto{\pgfqpoint{1.513671in}{0.709242in}}%
\pgfpathlineto{\pgfqpoint{1.534913in}{0.713028in}}%
\pgfpathlineto{\pgfqpoint{1.556156in}{0.710701in}}%
\pgfpathlineto{\pgfqpoint{1.577398in}{0.712637in}}%
\pgfpathlineto{\pgfqpoint{1.598641in}{0.715830in}}%
\pgfpathlineto{\pgfqpoint{1.619884in}{0.713337in}}%
\pgfpathlineto{\pgfqpoint{1.641126in}{0.722048in}}%
\pgfpathlineto{\pgfqpoint{1.662369in}{0.715888in}}%
\pgfpathlineto{\pgfqpoint{1.683611in}{0.713825in}}%
\pgfpathlineto{\pgfqpoint{1.704854in}{0.715755in}}%
\pgfpathlineto{\pgfqpoint{1.747339in}{0.711619in}}%
\pgfpathlineto{\pgfqpoint{1.768582in}{0.714073in}}%
\pgfpathlineto{\pgfqpoint{1.789824in}{0.708833in}}%
\pgfpathlineto{\pgfqpoint{1.811067in}{0.708010in}}%
\pgfpathlineto{\pgfqpoint{1.874794in}{0.720352in}}%
\pgfpathlineto{\pgfqpoint{1.896037in}{0.714315in}}%
\pgfpathlineto{\pgfqpoint{1.917280in}{0.714600in}}%
\pgfpathlineto{\pgfqpoint{1.938522in}{0.709439in}}%
\pgfpathlineto{\pgfqpoint{1.959765in}{0.698347in}}%
\pgfpathlineto{\pgfqpoint{1.981007in}{0.695598in}}%
\pgfpathlineto{\pgfqpoint{2.002250in}{0.698808in}}%
\pgfpathlineto{\pgfqpoint{2.023492in}{0.704799in}}%
\pgfpathlineto{\pgfqpoint{2.044735in}{0.699586in}}%
\pgfpathlineto{\pgfqpoint{2.065978in}{0.700567in}}%
\pgfpathlineto{\pgfqpoint{2.087220in}{0.707657in}}%
\pgfpathlineto{\pgfqpoint{2.108463in}{0.706581in}}%
\pgfpathlineto{\pgfqpoint{2.129705in}{0.711353in}}%
\pgfpathlineto{\pgfqpoint{2.150948in}{0.725318in}}%
\pgfpathlineto{\pgfqpoint{2.172190in}{0.728326in}}%
\pgfpathlineto{\pgfqpoint{2.193433in}{0.729007in}}%
\pgfpathlineto{\pgfqpoint{2.214676in}{0.731476in}}%
\pgfpathlineto{\pgfqpoint{2.235918in}{0.727485in}}%
\pgfpathlineto{\pgfqpoint{2.257161in}{0.715842in}}%
\pgfpathlineto{\pgfqpoint{2.278403in}{0.724411in}}%
\pgfpathlineto{\pgfqpoint{2.299646in}{0.723201in}}%
\pgfpathlineto{\pgfqpoint{2.320889in}{0.725338in}}%
\pgfpathlineto{\pgfqpoint{2.342131in}{0.715618in}}%
\pgfpathlineto{\pgfqpoint{2.363374in}{0.720896in}}%
\pgfpathlineto{\pgfqpoint{2.384616in}{0.716178in}}%
\pgfpathlineto{\pgfqpoint{2.405859in}{0.705991in}}%
\pgfpathlineto{\pgfqpoint{2.427101in}{0.704273in}}%
\pgfpathlineto{\pgfqpoint{2.469587in}{0.713639in}}%
\pgfpathlineto{\pgfqpoint{2.490829in}{0.714233in}}%
\pgfpathlineto{\pgfqpoint{2.512072in}{0.719883in}}%
\pgfpathlineto{\pgfqpoint{2.533314in}{0.728340in}}%
\pgfpathlineto{\pgfqpoint{2.554557in}{0.721206in}}%
\pgfpathlineto{\pgfqpoint{2.575799in}{0.709969in}}%
\pgfpathlineto{\pgfqpoint{2.597042in}{0.711238in}}%
\pgfpathlineto{\pgfqpoint{2.618285in}{0.702657in}}%
\pgfpathlineto{\pgfqpoint{2.639527in}{0.692044in}}%
\pgfpathlineto{\pgfqpoint{2.660770in}{0.686988in}}%
\pgfpathlineto{\pgfqpoint{2.682012in}{0.688096in}}%
\pgfpathlineto{\pgfqpoint{2.703255in}{0.695353in}}%
\pgfpathlineto{\pgfqpoint{2.745740in}{0.698902in}}%
\pgfpathlineto{\pgfqpoint{2.766983in}{0.707694in}}%
\pgfpathlineto{\pgfqpoint{2.788225in}{0.719347in}}%
\pgfpathlineto{\pgfqpoint{2.809468in}{0.711904in}}%
\pgfpathlineto{\pgfqpoint{2.830710in}{0.714735in}}%
\pgfpathlineto{\pgfqpoint{2.851953in}{0.709148in}}%
\pgfpathlineto{\pgfqpoint{2.915681in}{0.725610in}}%
\pgfpathlineto{\pgfqpoint{2.936923in}{0.720543in}}%
\pgfpathlineto{\pgfqpoint{2.958166in}{0.718774in}}%
\pgfpathlineto{\pgfqpoint{2.979408in}{0.709618in}}%
\pgfpathlineto{\pgfqpoint{3.000651in}{0.711399in}}%
\pgfpathlineto{\pgfqpoint{3.021894in}{0.708633in}}%
\pgfpathlineto{\pgfqpoint{3.043136in}{0.712441in}}%
\pgfpathlineto{\pgfqpoint{3.064379in}{0.710279in}}%
\pgfpathlineto{\pgfqpoint{3.085621in}{0.712546in}}%
\pgfpathlineto{\pgfqpoint{3.106864in}{0.703629in}}%
\pgfpathlineto{\pgfqpoint{3.128106in}{0.692847in}}%
\pgfpathlineto{\pgfqpoint{3.149349in}{0.703123in}}%
\pgfpathlineto{\pgfqpoint{3.191834in}{0.684430in}}%
\pgfpathlineto{\pgfqpoint{3.213077in}{0.679678in}}%
\pgfpathlineto{\pgfqpoint{3.234319in}{0.686647in}}%
\pgfpathlineto{\pgfqpoint{3.255562in}{0.696435in}}%
\pgfpathlineto{\pgfqpoint{3.276804in}{0.701767in}}%
\pgfpathlineto{\pgfqpoint{3.298047in}{0.709272in}}%
\pgfpathlineto{\pgfqpoint{3.319290in}{0.704431in}}%
\pgfpathlineto{\pgfqpoint{3.340532in}{0.688090in}}%
\pgfpathlineto{\pgfqpoint{3.361775in}{0.690430in}}%
\pgfpathlineto{\pgfqpoint{3.383017in}{0.687397in}}%
\pgfpathlineto{\pgfqpoint{3.404260in}{0.682123in}}%
\pgfpathlineto{\pgfqpoint{3.425503in}{0.681539in}}%
\pgfpathlineto{\pgfqpoint{3.446745in}{0.677531in}}%
\pgfpathlineto{\pgfqpoint{3.467988in}{0.684923in}}%
\pgfpathlineto{\pgfqpoint{3.489230in}{0.684273in}}%
\pgfpathlineto{\pgfqpoint{3.510473in}{0.677651in}}%
\pgfpathlineto{\pgfqpoint{3.531715in}{0.672358in}}%
\pgfpathlineto{\pgfqpoint{3.552958in}{0.677067in}}%
\pgfpathlineto{\pgfqpoint{3.574201in}{0.673827in}}%
\pgfpathlineto{\pgfqpoint{3.595443in}{0.668121in}}%
\pgfpathlineto{\pgfqpoint{3.616686in}{0.676790in}}%
\pgfpathlineto{\pgfqpoint{3.659171in}{0.667840in}}%
\pgfpathlineto{\pgfqpoint{3.680413in}{0.669747in}}%
\pgfpathlineto{\pgfqpoint{3.701656in}{0.669710in}}%
\pgfpathlineto{\pgfqpoint{3.722899in}{0.677898in}}%
\pgfpathlineto{\pgfqpoint{3.786626in}{0.663434in}}%
\pgfpathlineto{\pgfqpoint{3.807869in}{0.665831in}}%
\pgfpathlineto{\pgfqpoint{3.829111in}{0.658815in}}%
\pgfpathlineto{\pgfqpoint{3.850354in}{0.662742in}}%
\pgfpathlineto{\pgfqpoint{3.871597in}{0.653768in}}%
\pgfpathlineto{\pgfqpoint{3.892839in}{0.654451in}}%
\pgfpathlineto{\pgfqpoint{3.914082in}{0.648140in}}%
\pgfpathlineto{\pgfqpoint{3.935324in}{0.650224in}}%
\pgfpathlineto{\pgfqpoint{3.956567in}{0.656629in}}%
\pgfpathlineto{\pgfqpoint{3.977810in}{0.668675in}}%
\pgfpathlineto{\pgfqpoint{3.999052in}{0.678319in}}%
\pgfpathlineto{\pgfqpoint{4.020295in}{0.681775in}}%
\pgfpathlineto{\pgfqpoint{4.062780in}{0.667793in}}%
\pgfpathlineto{\pgfqpoint{4.084022in}{0.670071in}}%
\pgfpathlineto{\pgfqpoint{4.105265in}{0.665824in}}%
\pgfpathlineto{\pgfqpoint{4.126508in}{0.674495in}}%
\pgfpathlineto{\pgfqpoint{4.147750in}{0.681628in}}%
\pgfpathlineto{\pgfqpoint{4.168993in}{0.675760in}}%
\pgfpathlineto{\pgfqpoint{4.190235in}{0.667182in}}%
\pgfpathlineto{\pgfqpoint{4.211478in}{0.666040in}}%
\pgfpathlineto{\pgfqpoint{4.232720in}{0.668139in}}%
\pgfpathlineto{\pgfqpoint{4.253963in}{0.677541in}}%
\pgfpathlineto{\pgfqpoint{4.275206in}{0.679311in}}%
\pgfpathlineto{\pgfqpoint{4.317691in}{0.679942in}}%
\pgfpathlineto{\pgfqpoint{4.338933in}{0.687272in}}%
\pgfpathlineto{\pgfqpoint{4.360176in}{0.691174in}}%
\pgfpathlineto{\pgfqpoint{4.381418in}{0.691066in}}%
\pgfpathlineto{\pgfqpoint{4.402661in}{0.683822in}}%
\pgfpathlineto{\pgfqpoint{4.445146in}{0.687754in}}%
\pgfpathlineto{\pgfqpoint{4.466389in}{0.691159in}}%
\pgfpathlineto{\pgfqpoint{4.487631in}{0.691956in}}%
\pgfpathlineto{\pgfqpoint{4.508874in}{0.698758in}}%
\pgfpathlineto{\pgfqpoint{4.530116in}{0.701483in}}%
\pgfpathlineto{\pgfqpoint{4.572602in}{0.691516in}}%
\pgfpathlineto{\pgfqpoint{4.593844in}{0.687121in}}%
\pgfpathlineto{\pgfqpoint{4.615087in}{0.684708in}}%
\pgfpathlineto{\pgfqpoint{4.636329in}{0.689582in}}%
\pgfpathlineto{\pgfqpoint{4.657572in}{0.681327in}}%
\pgfpathlineto{\pgfqpoint{4.678815in}{0.683940in}}%
\pgfpathlineto{\pgfqpoint{4.700057in}{0.682476in}}%
\pgfpathlineto{\pgfqpoint{4.721300in}{0.675292in}}%
\pgfpathlineto{\pgfqpoint{4.742542in}{0.670446in}}%
\pgfpathlineto{\pgfqpoint{4.763785in}{0.668034in}}%
\pgfpathlineto{\pgfqpoint{4.785027in}{0.660304in}}%
\pgfpathlineto{\pgfqpoint{4.806270in}{0.662918in}}%
\pgfpathlineto{\pgfqpoint{4.869998in}{0.684089in}}%
\pgfpathlineto{\pgfqpoint{4.891240in}{0.699666in}}%
\pgfpathlineto{\pgfqpoint{4.912483in}{0.709964in}}%
\pgfpathlineto{\pgfqpoint{4.933725in}{0.713649in}}%
\pgfpathlineto{\pgfqpoint{4.997453in}{0.711066in}}%
\pgfpathlineto{\pgfqpoint{5.018696in}{0.716447in}}%
\pgfpathlineto{\pgfqpoint{5.039938in}{0.725626in}}%
\pgfpathlineto{\pgfqpoint{5.061181in}{0.719412in}}%
\pgfpathlineto{\pgfqpoint{5.082423in}{0.728742in}}%
\pgfpathlineto{\pgfqpoint{5.103666in}{0.728208in}}%
\pgfpathlineto{\pgfqpoint{5.124909in}{0.726054in}}%
\pgfpathlineto{\pgfqpoint{5.146151in}{0.720196in}}%
\pgfpathlineto{\pgfqpoint{5.167394in}{0.727380in}}%
\pgfpathlineto{\pgfqpoint{5.188636in}{0.723279in}}%
\pgfpathlineto{\pgfqpoint{5.188636in}{0.723279in}}%
\pgfusepath{stroke}%
\end{pgfscope}%
\begin{pgfscope}%
\pgfpathrectangle{\pgfqpoint{0.750000in}{0.440000in}}{\pgfqpoint{4.650000in}{3.080000in}}%
\pgfusepath{clip}%
\pgfsetroundcap%
\pgfsetroundjoin%
\pgfsetlinewidth{1.003750pt}%
\definecolor{currentstroke}{rgb}{0.968627,0.505882,0.749020}%
\pgfsetstrokecolor{currentstroke}%
\pgfsetdash{}{0pt}%
\pgfpathmoveto{\pgfqpoint{0.961364in}{0.812687in}}%
\pgfpathlineto{\pgfqpoint{0.982606in}{0.803287in}}%
\pgfpathlineto{\pgfqpoint{1.003849in}{0.796437in}}%
\pgfpathlineto{\pgfqpoint{1.025091in}{0.796703in}}%
\pgfpathlineto{\pgfqpoint{1.046334in}{0.806473in}}%
\pgfpathlineto{\pgfqpoint{1.067577in}{0.812362in}}%
\pgfpathlineto{\pgfqpoint{1.088819in}{0.786794in}}%
\pgfpathlineto{\pgfqpoint{1.131304in}{0.785067in}}%
\pgfpathlineto{\pgfqpoint{1.152547in}{0.775751in}}%
\pgfpathlineto{\pgfqpoint{1.173789in}{0.792480in}}%
\pgfpathlineto{\pgfqpoint{1.195032in}{0.797868in}}%
\pgfpathlineto{\pgfqpoint{1.216275in}{0.799499in}}%
\pgfpathlineto{\pgfqpoint{1.237517in}{0.799004in}}%
\pgfpathlineto{\pgfqpoint{1.258760in}{0.786217in}}%
\pgfpathlineto{\pgfqpoint{1.280002in}{0.791943in}}%
\pgfpathlineto{\pgfqpoint{1.322487in}{0.807778in}}%
\pgfpathlineto{\pgfqpoint{1.343730in}{0.807864in}}%
\pgfpathlineto{\pgfqpoint{1.364973in}{0.797689in}}%
\pgfpathlineto{\pgfqpoint{1.386215in}{0.804930in}}%
\pgfpathlineto{\pgfqpoint{1.407458in}{0.803901in}}%
\pgfpathlineto{\pgfqpoint{1.428700in}{0.808465in}}%
\pgfpathlineto{\pgfqpoint{1.449943in}{0.818471in}}%
\pgfpathlineto{\pgfqpoint{1.471185in}{0.808919in}}%
\pgfpathlineto{\pgfqpoint{1.492428in}{0.823891in}}%
\pgfpathlineto{\pgfqpoint{1.513671in}{0.811594in}}%
\pgfpathlineto{\pgfqpoint{1.534913in}{0.802802in}}%
\pgfpathlineto{\pgfqpoint{1.556156in}{0.806177in}}%
\pgfpathlineto{\pgfqpoint{1.577398in}{0.799875in}}%
\pgfpathlineto{\pgfqpoint{1.598641in}{0.798807in}}%
\pgfpathlineto{\pgfqpoint{1.619884in}{0.809693in}}%
\pgfpathlineto{\pgfqpoint{1.662369in}{0.797088in}}%
\pgfpathlineto{\pgfqpoint{1.683611in}{0.798106in}}%
\pgfpathlineto{\pgfqpoint{1.704854in}{0.806369in}}%
\pgfpathlineto{\pgfqpoint{1.726096in}{0.792478in}}%
\pgfpathlineto{\pgfqpoint{1.768582in}{0.794265in}}%
\pgfpathlineto{\pgfqpoint{1.789824in}{0.809759in}}%
\pgfpathlineto{\pgfqpoint{1.832309in}{0.804060in}}%
\pgfpathlineto{\pgfqpoint{1.853552in}{0.810396in}}%
\pgfpathlineto{\pgfqpoint{1.874794in}{0.814302in}}%
\pgfpathlineto{\pgfqpoint{1.896037in}{0.805143in}}%
\pgfpathlineto{\pgfqpoint{1.917280in}{0.819394in}}%
\pgfpathlineto{\pgfqpoint{1.938522in}{0.808916in}}%
\pgfpathlineto{\pgfqpoint{2.002250in}{0.822472in}}%
\pgfpathlineto{\pgfqpoint{2.023492in}{0.838970in}}%
\pgfpathlineto{\pgfqpoint{2.044735in}{0.835866in}}%
\pgfpathlineto{\pgfqpoint{2.065978in}{0.829274in}}%
\pgfpathlineto{\pgfqpoint{2.087220in}{0.834902in}}%
\pgfpathlineto{\pgfqpoint{2.108463in}{0.820560in}}%
\pgfpathlineto{\pgfqpoint{2.129705in}{0.817481in}}%
\pgfpathlineto{\pgfqpoint{2.150948in}{0.800320in}}%
\pgfpathlineto{\pgfqpoint{2.172190in}{0.774584in}}%
\pgfpathlineto{\pgfqpoint{2.193433in}{0.774012in}}%
\pgfpathlineto{\pgfqpoint{2.214676in}{0.779917in}}%
\pgfpathlineto{\pgfqpoint{2.257161in}{0.784994in}}%
\pgfpathlineto{\pgfqpoint{2.278403in}{0.798157in}}%
\pgfpathlineto{\pgfqpoint{2.299646in}{0.805509in}}%
\pgfpathlineto{\pgfqpoint{2.320889in}{0.797556in}}%
\pgfpathlineto{\pgfqpoint{2.342131in}{0.785856in}}%
\pgfpathlineto{\pgfqpoint{2.363374in}{0.790209in}}%
\pgfpathlineto{\pgfqpoint{2.384616in}{0.800595in}}%
\pgfpathlineto{\pgfqpoint{2.427101in}{0.809717in}}%
\pgfpathlineto{\pgfqpoint{2.448344in}{0.812454in}}%
\pgfpathlineto{\pgfqpoint{2.512072in}{0.793732in}}%
\pgfpathlineto{\pgfqpoint{2.533314in}{0.794037in}}%
\pgfpathlineto{\pgfqpoint{2.575799in}{0.805905in}}%
\pgfpathlineto{\pgfqpoint{2.597042in}{0.807515in}}%
\pgfpathlineto{\pgfqpoint{2.618285in}{0.798698in}}%
\pgfpathlineto{\pgfqpoint{2.660770in}{0.825942in}}%
\pgfpathlineto{\pgfqpoint{2.682012in}{0.816835in}}%
\pgfpathlineto{\pgfqpoint{2.703255in}{0.832689in}}%
\pgfpathlineto{\pgfqpoint{2.724497in}{0.827249in}}%
\pgfpathlineto{\pgfqpoint{2.745740in}{0.830052in}}%
\pgfpathlineto{\pgfqpoint{2.766983in}{0.834802in}}%
\pgfpathlineto{\pgfqpoint{2.788225in}{0.820230in}}%
\pgfpathlineto{\pgfqpoint{2.809468in}{0.834679in}}%
\pgfpathlineto{\pgfqpoint{2.830710in}{0.820795in}}%
\pgfpathlineto{\pgfqpoint{2.851953in}{0.821143in}}%
\pgfpathlineto{\pgfqpoint{2.873196in}{0.811396in}}%
\pgfpathlineto{\pgfqpoint{2.894438in}{0.813563in}}%
\pgfpathlineto{\pgfqpoint{2.915681in}{0.823786in}}%
\pgfpathlineto{\pgfqpoint{2.936923in}{0.830569in}}%
\pgfpathlineto{\pgfqpoint{2.958166in}{0.826313in}}%
\pgfpathlineto{\pgfqpoint{2.979408in}{0.831652in}}%
\pgfpathlineto{\pgfqpoint{3.043136in}{0.856460in}}%
\pgfpathlineto{\pgfqpoint{3.085621in}{0.846723in}}%
\pgfpathlineto{\pgfqpoint{3.106864in}{0.858852in}}%
\pgfpathlineto{\pgfqpoint{3.128106in}{0.850733in}}%
\pgfpathlineto{\pgfqpoint{3.149349in}{0.854840in}}%
\pgfpathlineto{\pgfqpoint{3.170592in}{0.847955in}}%
\pgfpathlineto{\pgfqpoint{3.191834in}{0.848949in}}%
\pgfpathlineto{\pgfqpoint{3.213077in}{0.838842in}}%
\pgfpathlineto{\pgfqpoint{3.255562in}{0.852813in}}%
\pgfpathlineto{\pgfqpoint{3.276804in}{0.843162in}}%
\pgfpathlineto{\pgfqpoint{3.298047in}{0.856409in}}%
\pgfpathlineto{\pgfqpoint{3.340532in}{0.860594in}}%
\pgfpathlineto{\pgfqpoint{3.361775in}{0.865878in}}%
\pgfpathlineto{\pgfqpoint{3.383017in}{0.880876in}}%
\pgfpathlineto{\pgfqpoint{3.404260in}{0.873137in}}%
\pgfpathlineto{\pgfqpoint{3.446745in}{0.866389in}}%
\pgfpathlineto{\pgfqpoint{3.467988in}{0.848828in}}%
\pgfpathlineto{\pgfqpoint{3.489230in}{0.839484in}}%
\pgfpathlineto{\pgfqpoint{3.510473in}{0.836369in}}%
\pgfpathlineto{\pgfqpoint{3.552958in}{0.868296in}}%
\pgfpathlineto{\pgfqpoint{3.574201in}{0.869543in}}%
\pgfpathlineto{\pgfqpoint{3.595443in}{0.884396in}}%
\pgfpathlineto{\pgfqpoint{3.616686in}{0.878521in}}%
\pgfpathlineto{\pgfqpoint{3.637928in}{0.898571in}}%
\pgfpathlineto{\pgfqpoint{3.659171in}{0.915613in}}%
\pgfpathlineto{\pgfqpoint{3.680413in}{0.909268in}}%
\pgfpathlineto{\pgfqpoint{3.701656in}{0.895396in}}%
\pgfpathlineto{\pgfqpoint{3.722899in}{0.892294in}}%
\pgfpathlineto{\pgfqpoint{3.765384in}{0.889450in}}%
\pgfpathlineto{\pgfqpoint{3.786626in}{0.880302in}}%
\pgfpathlineto{\pgfqpoint{3.807869in}{0.883930in}}%
\pgfpathlineto{\pgfqpoint{3.829111in}{0.907129in}}%
\pgfpathlineto{\pgfqpoint{3.850354in}{0.913780in}}%
\pgfpathlineto{\pgfqpoint{3.871597in}{0.895012in}}%
\pgfpathlineto{\pgfqpoint{3.892839in}{0.891851in}}%
\pgfpathlineto{\pgfqpoint{3.914082in}{0.889919in}}%
\pgfpathlineto{\pgfqpoint{3.935324in}{0.872723in}}%
\pgfpathlineto{\pgfqpoint{3.956567in}{0.861537in}}%
\pgfpathlineto{\pgfqpoint{3.977810in}{0.856530in}}%
\pgfpathlineto{\pgfqpoint{3.999052in}{0.867081in}}%
\pgfpathlineto{\pgfqpoint{4.020295in}{0.869929in}}%
\pgfpathlineto{\pgfqpoint{4.041537in}{0.863529in}}%
\pgfpathlineto{\pgfqpoint{4.062780in}{0.869987in}}%
\pgfpathlineto{\pgfqpoint{4.084022in}{0.860187in}}%
\pgfpathlineto{\pgfqpoint{4.105265in}{0.858130in}}%
\pgfpathlineto{\pgfqpoint{4.126508in}{0.866581in}}%
\pgfpathlineto{\pgfqpoint{4.147750in}{0.862798in}}%
\pgfpathlineto{\pgfqpoint{4.168993in}{0.848462in}}%
\pgfpathlineto{\pgfqpoint{4.211478in}{0.847859in}}%
\pgfpathlineto{\pgfqpoint{4.232720in}{0.838268in}}%
\pgfpathlineto{\pgfqpoint{4.253963in}{0.849484in}}%
\pgfpathlineto{\pgfqpoint{4.275206in}{0.852294in}}%
\pgfpathlineto{\pgfqpoint{4.317691in}{0.827564in}}%
\pgfpathlineto{\pgfqpoint{4.338933in}{0.838243in}}%
\pgfpathlineto{\pgfqpoint{4.360176in}{0.834433in}}%
\pgfpathlineto{\pgfqpoint{4.381418in}{0.840650in}}%
\pgfpathlineto{\pgfqpoint{4.423904in}{0.861307in}}%
\pgfpathlineto{\pgfqpoint{4.466389in}{0.860156in}}%
\pgfpathlineto{\pgfqpoint{4.487631in}{0.847230in}}%
\pgfpathlineto{\pgfqpoint{4.508874in}{0.856225in}}%
\pgfpathlineto{\pgfqpoint{4.530116in}{0.843515in}}%
\pgfpathlineto{\pgfqpoint{4.551359in}{0.871172in}}%
\pgfpathlineto{\pgfqpoint{4.593844in}{0.873024in}}%
\pgfpathlineto{\pgfqpoint{4.615087in}{0.863214in}}%
\pgfpathlineto{\pgfqpoint{4.678815in}{0.849938in}}%
\pgfpathlineto{\pgfqpoint{4.700057in}{0.857224in}}%
\pgfpathlineto{\pgfqpoint{4.721300in}{0.849274in}}%
\pgfpathlineto{\pgfqpoint{4.742542in}{0.857308in}}%
\pgfpathlineto{\pgfqpoint{4.763785in}{0.867046in}}%
\pgfpathlineto{\pgfqpoint{4.785027in}{0.859819in}}%
\pgfpathlineto{\pgfqpoint{4.806270in}{0.864775in}}%
\pgfpathlineto{\pgfqpoint{4.848755in}{0.870998in}}%
\pgfpathlineto{\pgfqpoint{4.869998in}{0.895699in}}%
\pgfpathlineto{\pgfqpoint{4.891240in}{0.894408in}}%
\pgfpathlineto{\pgfqpoint{4.912483in}{0.898868in}}%
\pgfpathlineto{\pgfqpoint{4.933725in}{0.909530in}}%
\pgfpathlineto{\pgfqpoint{4.954968in}{0.910426in}}%
\pgfpathlineto{\pgfqpoint{4.997453in}{0.927106in}}%
\pgfpathlineto{\pgfqpoint{5.018696in}{0.934614in}}%
\pgfpathlineto{\pgfqpoint{5.039938in}{0.921772in}}%
\pgfpathlineto{\pgfqpoint{5.061181in}{0.929573in}}%
\pgfpathlineto{\pgfqpoint{5.082423in}{0.925441in}}%
\pgfpathlineto{\pgfqpoint{5.103666in}{0.924030in}}%
\pgfpathlineto{\pgfqpoint{5.124909in}{0.913640in}}%
\pgfpathlineto{\pgfqpoint{5.146151in}{0.901038in}}%
\pgfpathlineto{\pgfqpoint{5.167394in}{0.893575in}}%
\pgfpathlineto{\pgfqpoint{5.188636in}{0.897691in}}%
\pgfpathlineto{\pgfqpoint{5.188636in}{0.897691in}}%
\pgfusepath{stroke}%
\end{pgfscope}%
\begin{pgfscope}%
\pgfpathrectangle{\pgfqpoint{0.750000in}{0.440000in}}{\pgfqpoint{4.650000in}{3.080000in}}%
\pgfusepath{clip}%
\pgfsetroundcap%
\pgfsetroundjoin%
\pgfsetlinewidth{1.003750pt}%
\definecolor{currentstroke}{rgb}{0.650980,0.337255,0.156863}%
\pgfsetstrokecolor{currentstroke}%
\pgfsetdash{}{0pt}%
\pgfpathmoveto{\pgfqpoint{0.961364in}{0.889536in}}%
\pgfpathlineto{\pgfqpoint{0.982606in}{0.873464in}}%
\pgfpathlineto{\pgfqpoint{1.003849in}{0.864818in}}%
\pgfpathlineto{\pgfqpoint{1.025091in}{0.871550in}}%
\pgfpathlineto{\pgfqpoint{1.046334in}{0.855059in}}%
\pgfpathlineto{\pgfqpoint{1.067577in}{0.860013in}}%
\pgfpathlineto{\pgfqpoint{1.088819in}{0.874699in}}%
\pgfpathlineto{\pgfqpoint{1.110062in}{0.892087in}}%
\pgfpathlineto{\pgfqpoint{1.131304in}{0.873536in}}%
\pgfpathlineto{\pgfqpoint{1.152547in}{0.889553in}}%
\pgfpathlineto{\pgfqpoint{1.173789in}{0.871433in}}%
\pgfpathlineto{\pgfqpoint{1.195032in}{0.866383in}}%
\pgfpathlineto{\pgfqpoint{1.216275in}{0.838621in}}%
\pgfpathlineto{\pgfqpoint{1.237517in}{0.840588in}}%
\pgfpathlineto{\pgfqpoint{1.258760in}{0.844361in}}%
\pgfpathlineto{\pgfqpoint{1.280002in}{0.845059in}}%
\pgfpathlineto{\pgfqpoint{1.301245in}{0.851011in}}%
\pgfpathlineto{\pgfqpoint{1.322487in}{0.852396in}}%
\pgfpathlineto{\pgfqpoint{1.343730in}{0.859484in}}%
\pgfpathlineto{\pgfqpoint{1.364973in}{0.852069in}}%
\pgfpathlineto{\pgfqpoint{1.386215in}{0.842728in}}%
\pgfpathlineto{\pgfqpoint{1.407458in}{0.861757in}}%
\pgfpathlineto{\pgfqpoint{1.428700in}{0.866969in}}%
\pgfpathlineto{\pgfqpoint{1.449943in}{0.864539in}}%
\pgfpathlineto{\pgfqpoint{1.471185in}{0.880874in}}%
\pgfpathlineto{\pgfqpoint{1.492428in}{0.888961in}}%
\pgfpathlineto{\pgfqpoint{1.513671in}{0.891232in}}%
\pgfpathlineto{\pgfqpoint{1.534913in}{0.885679in}}%
\pgfpathlineto{\pgfqpoint{1.556156in}{0.892621in}}%
\pgfpathlineto{\pgfqpoint{1.577398in}{0.897425in}}%
\pgfpathlineto{\pgfqpoint{1.598641in}{0.890311in}}%
\pgfpathlineto{\pgfqpoint{1.619884in}{0.885325in}}%
\pgfpathlineto{\pgfqpoint{1.641126in}{0.890233in}}%
\pgfpathlineto{\pgfqpoint{1.662369in}{0.888404in}}%
\pgfpathlineto{\pgfqpoint{1.683611in}{0.921295in}}%
\pgfpathlineto{\pgfqpoint{1.704854in}{0.926737in}}%
\pgfpathlineto{\pgfqpoint{1.726096in}{0.924486in}}%
\pgfpathlineto{\pgfqpoint{1.747339in}{0.927947in}}%
\pgfpathlineto{\pgfqpoint{1.768582in}{0.924373in}}%
\pgfpathlineto{\pgfqpoint{1.789824in}{0.914204in}}%
\pgfpathlineto{\pgfqpoint{1.811067in}{0.920573in}}%
\pgfpathlineto{\pgfqpoint{1.832309in}{0.919280in}}%
\pgfpathlineto{\pgfqpoint{1.853552in}{0.907710in}}%
\pgfpathlineto{\pgfqpoint{1.874794in}{0.930433in}}%
\pgfpathlineto{\pgfqpoint{1.896037in}{0.928335in}}%
\pgfpathlineto{\pgfqpoint{1.917280in}{0.917443in}}%
\pgfpathlineto{\pgfqpoint{1.938522in}{0.913798in}}%
\pgfpathlineto{\pgfqpoint{1.959765in}{0.916039in}}%
\pgfpathlineto{\pgfqpoint{1.981007in}{0.919886in}}%
\pgfpathlineto{\pgfqpoint{2.002250in}{0.921001in}}%
\pgfpathlineto{\pgfqpoint{2.023492in}{0.918491in}}%
\pgfpathlineto{\pgfqpoint{2.044735in}{0.929937in}}%
\pgfpathlineto{\pgfqpoint{2.065978in}{0.917252in}}%
\pgfpathlineto{\pgfqpoint{2.087220in}{0.902828in}}%
\pgfpathlineto{\pgfqpoint{2.108463in}{0.911552in}}%
\pgfpathlineto{\pgfqpoint{2.129705in}{0.922769in}}%
\pgfpathlineto{\pgfqpoint{2.150948in}{0.920432in}}%
\pgfpathlineto{\pgfqpoint{2.172190in}{0.926557in}}%
\pgfpathlineto{\pgfqpoint{2.214676in}{0.947836in}}%
\pgfpathlineto{\pgfqpoint{2.235918in}{0.948684in}}%
\pgfpathlineto{\pgfqpoint{2.257161in}{0.945323in}}%
\pgfpathlineto{\pgfqpoint{2.278403in}{0.947276in}}%
\pgfpathlineto{\pgfqpoint{2.342131in}{0.974067in}}%
\pgfpathlineto{\pgfqpoint{2.384616in}{0.976377in}}%
\pgfpathlineto{\pgfqpoint{2.427101in}{0.967426in}}%
\pgfpathlineto{\pgfqpoint{2.448344in}{0.961120in}}%
\pgfpathlineto{\pgfqpoint{2.469587in}{0.959607in}}%
\pgfpathlineto{\pgfqpoint{2.490829in}{0.956256in}}%
\pgfpathlineto{\pgfqpoint{2.512072in}{0.965618in}}%
\pgfpathlineto{\pgfqpoint{2.533314in}{0.949869in}}%
\pgfpathlineto{\pgfqpoint{2.554557in}{0.957129in}}%
\pgfpathlineto{\pgfqpoint{2.575799in}{0.967021in}}%
\pgfpathlineto{\pgfqpoint{2.597042in}{0.981078in}}%
\pgfpathlineto{\pgfqpoint{2.618285in}{0.987368in}}%
\pgfpathlineto{\pgfqpoint{2.639527in}{0.987636in}}%
\pgfpathlineto{\pgfqpoint{2.682012in}{0.996968in}}%
\pgfpathlineto{\pgfqpoint{2.703255in}{0.990233in}}%
\pgfpathlineto{\pgfqpoint{2.745740in}{1.009975in}}%
\pgfpathlineto{\pgfqpoint{2.766983in}{0.996999in}}%
\pgfpathlineto{\pgfqpoint{2.788225in}{0.991338in}}%
\pgfpathlineto{\pgfqpoint{2.809468in}{0.999398in}}%
\pgfpathlineto{\pgfqpoint{2.830710in}{0.993562in}}%
\pgfpathlineto{\pgfqpoint{2.873196in}{1.022182in}}%
\pgfpathlineto{\pgfqpoint{2.915681in}{1.024475in}}%
\pgfpathlineto{\pgfqpoint{2.936923in}{1.030621in}}%
\pgfpathlineto{\pgfqpoint{2.958166in}{1.050375in}}%
\pgfpathlineto{\pgfqpoint{2.979408in}{1.030105in}}%
\pgfpathlineto{\pgfqpoint{3.021894in}{1.050598in}}%
\pgfpathlineto{\pgfqpoint{3.043136in}{1.040280in}}%
\pgfpathlineto{\pgfqpoint{3.064379in}{1.065894in}}%
\pgfpathlineto{\pgfqpoint{3.085621in}{1.076024in}}%
\pgfpathlineto{\pgfqpoint{3.106864in}{1.091169in}}%
\pgfpathlineto{\pgfqpoint{3.128106in}{1.073893in}}%
\pgfpathlineto{\pgfqpoint{3.149349in}{1.075947in}}%
\pgfpathlineto{\pgfqpoint{3.170592in}{1.059455in}}%
\pgfpathlineto{\pgfqpoint{3.191834in}{1.053444in}}%
\pgfpathlineto{\pgfqpoint{3.213077in}{1.075995in}}%
\pgfpathlineto{\pgfqpoint{3.234319in}{1.093203in}}%
\pgfpathlineto{\pgfqpoint{3.255562in}{1.101213in}}%
\pgfpathlineto{\pgfqpoint{3.298047in}{1.088062in}}%
\pgfpathlineto{\pgfqpoint{3.319290in}{1.086398in}}%
\pgfpathlineto{\pgfqpoint{3.340532in}{1.103372in}}%
\pgfpathlineto{\pgfqpoint{3.361775in}{1.092925in}}%
\pgfpathlineto{\pgfqpoint{3.383017in}{1.108884in}}%
\pgfpathlineto{\pgfqpoint{3.404260in}{1.115474in}}%
\pgfpathlineto{\pgfqpoint{3.425503in}{1.115014in}}%
\pgfpathlineto{\pgfqpoint{3.446745in}{1.101013in}}%
\pgfpathlineto{\pgfqpoint{3.467988in}{1.092919in}}%
\pgfpathlineto{\pgfqpoint{3.489230in}{1.071742in}}%
\pgfpathlineto{\pgfqpoint{3.531715in}{1.075271in}}%
\pgfpathlineto{\pgfqpoint{3.552958in}{1.084261in}}%
\pgfpathlineto{\pgfqpoint{3.574201in}{1.079737in}}%
\pgfpathlineto{\pgfqpoint{3.595443in}{1.066387in}}%
\pgfpathlineto{\pgfqpoint{3.616686in}{1.054904in}}%
\pgfpathlineto{\pgfqpoint{3.637928in}{1.041233in}}%
\pgfpathlineto{\pgfqpoint{3.659171in}{1.048731in}}%
\pgfpathlineto{\pgfqpoint{3.680413in}{1.066464in}}%
\pgfpathlineto{\pgfqpoint{3.701656in}{1.054643in}}%
\pgfpathlineto{\pgfqpoint{3.722899in}{1.038649in}}%
\pgfpathlineto{\pgfqpoint{3.744141in}{1.047561in}}%
\pgfpathlineto{\pgfqpoint{3.765384in}{1.049210in}}%
\pgfpathlineto{\pgfqpoint{3.786626in}{1.037322in}}%
\pgfpathlineto{\pgfqpoint{3.807869in}{1.019404in}}%
\pgfpathlineto{\pgfqpoint{3.850354in}{1.016970in}}%
\pgfpathlineto{\pgfqpoint{3.871597in}{1.023016in}}%
\pgfpathlineto{\pgfqpoint{3.892839in}{1.042655in}}%
\pgfpathlineto{\pgfqpoint{3.914082in}{1.036052in}}%
\pgfpathlineto{\pgfqpoint{3.935324in}{1.064393in}}%
\pgfpathlineto{\pgfqpoint{3.956567in}{1.048126in}}%
\pgfpathlineto{\pgfqpoint{3.977810in}{1.043797in}}%
\pgfpathlineto{\pgfqpoint{3.999052in}{1.032217in}}%
\pgfpathlineto{\pgfqpoint{4.020295in}{1.031116in}}%
\pgfpathlineto{\pgfqpoint{4.041537in}{1.041633in}}%
\pgfpathlineto{\pgfqpoint{4.062780in}{1.033037in}}%
\pgfpathlineto{\pgfqpoint{4.084022in}{1.027421in}}%
\pgfpathlineto{\pgfqpoint{4.105265in}{1.050384in}}%
\pgfpathlineto{\pgfqpoint{4.126508in}{1.049244in}}%
\pgfpathlineto{\pgfqpoint{4.147750in}{1.034065in}}%
\pgfpathlineto{\pgfqpoint{4.168993in}{1.046446in}}%
\pgfpathlineto{\pgfqpoint{4.190235in}{1.045515in}}%
\pgfpathlineto{\pgfqpoint{4.211478in}{1.047337in}}%
\pgfpathlineto{\pgfqpoint{4.232720in}{1.044360in}}%
\pgfpathlineto{\pgfqpoint{4.253963in}{1.065799in}}%
\pgfpathlineto{\pgfqpoint{4.275206in}{1.063372in}}%
\pgfpathlineto{\pgfqpoint{4.296448in}{1.076908in}}%
\pgfpathlineto{\pgfqpoint{4.317691in}{1.060071in}}%
\pgfpathlineto{\pgfqpoint{4.338933in}{1.069503in}}%
\pgfpathlineto{\pgfqpoint{4.360176in}{1.067261in}}%
\pgfpathlineto{\pgfqpoint{4.381418in}{1.072913in}}%
\pgfpathlineto{\pgfqpoint{4.402661in}{1.060638in}}%
\pgfpathlineto{\pgfqpoint{4.423904in}{1.046037in}}%
\pgfpathlineto{\pgfqpoint{4.445146in}{1.027935in}}%
\pgfpathlineto{\pgfqpoint{4.466389in}{1.028381in}}%
\pgfpathlineto{\pgfqpoint{4.487631in}{1.042905in}}%
\pgfpathlineto{\pgfqpoint{4.508874in}{1.051776in}}%
\pgfpathlineto{\pgfqpoint{4.551359in}{1.057569in}}%
\pgfpathlineto{\pgfqpoint{4.572602in}{1.061673in}}%
\pgfpathlineto{\pgfqpoint{4.593844in}{1.067398in}}%
\pgfpathlineto{\pgfqpoint{4.615087in}{1.088317in}}%
\pgfpathlineto{\pgfqpoint{4.636329in}{1.076375in}}%
\pgfpathlineto{\pgfqpoint{4.657572in}{1.074417in}}%
\pgfpathlineto{\pgfqpoint{4.700057in}{1.082586in}}%
\pgfpathlineto{\pgfqpoint{4.721300in}{1.072434in}}%
\pgfpathlineto{\pgfqpoint{4.742542in}{1.093576in}}%
\pgfpathlineto{\pgfqpoint{4.763785in}{1.111379in}}%
\pgfpathlineto{\pgfqpoint{4.785027in}{1.127059in}}%
\pgfpathlineto{\pgfqpoint{4.806270in}{1.131427in}}%
\pgfpathlineto{\pgfqpoint{4.848755in}{1.132644in}}%
\pgfpathlineto{\pgfqpoint{4.869998in}{1.125178in}}%
\pgfpathlineto{\pgfqpoint{4.891240in}{1.129966in}}%
\pgfpathlineto{\pgfqpoint{4.912483in}{1.132985in}}%
\pgfpathlineto{\pgfqpoint{4.954968in}{1.127702in}}%
\pgfpathlineto{\pgfqpoint{4.976211in}{1.150317in}}%
\pgfpathlineto{\pgfqpoint{4.997453in}{1.147204in}}%
\pgfpathlineto{\pgfqpoint{5.018696in}{1.189134in}}%
\pgfpathlineto{\pgfqpoint{5.039938in}{1.202060in}}%
\pgfpathlineto{\pgfqpoint{5.061181in}{1.196338in}}%
\pgfpathlineto{\pgfqpoint{5.082423in}{1.196108in}}%
\pgfpathlineto{\pgfqpoint{5.103666in}{1.201667in}}%
\pgfpathlineto{\pgfqpoint{5.124909in}{1.205021in}}%
\pgfpathlineto{\pgfqpoint{5.146151in}{1.201797in}}%
\pgfpathlineto{\pgfqpoint{5.167394in}{1.180814in}}%
\pgfpathlineto{\pgfqpoint{5.188636in}{1.181485in}}%
\pgfpathlineto{\pgfqpoint{5.188636in}{1.181485in}}%
\pgfusepath{stroke}%
\end{pgfscope}%
\begin{pgfscope}%
\pgfpathrectangle{\pgfqpoint{0.750000in}{0.440000in}}{\pgfqpoint{4.650000in}{3.080000in}}%
\pgfusepath{clip}%
\pgfsetroundcap%
\pgfsetroundjoin%
\pgfsetlinewidth{1.003750pt}%
\definecolor{currentstroke}{rgb}{0.596078,0.305882,0.639216}%
\pgfsetstrokecolor{currentstroke}%
\pgfsetdash{}{0pt}%
\pgfpathmoveto{\pgfqpoint{0.961364in}{0.966386in}}%
\pgfpathlineto{\pgfqpoint{0.982606in}{0.968238in}}%
\pgfpathlineto{\pgfqpoint{1.003849in}{0.995048in}}%
\pgfpathlineto{\pgfqpoint{1.025091in}{0.991968in}}%
\pgfpathlineto{\pgfqpoint{1.046334in}{0.982942in}}%
\pgfpathlineto{\pgfqpoint{1.067577in}{0.988580in}}%
\pgfpathlineto{\pgfqpoint{1.110062in}{0.983090in}}%
\pgfpathlineto{\pgfqpoint{1.131304in}{0.977878in}}%
\pgfpathlineto{\pgfqpoint{1.152547in}{0.963203in}}%
\pgfpathlineto{\pgfqpoint{1.173789in}{0.981603in}}%
\pgfpathlineto{\pgfqpoint{1.195032in}{0.968268in}}%
\pgfpathlineto{\pgfqpoint{1.216275in}{0.957982in}}%
\pgfpathlineto{\pgfqpoint{1.237517in}{0.957151in}}%
\pgfpathlineto{\pgfqpoint{1.258760in}{0.963993in}}%
\pgfpathlineto{\pgfqpoint{1.280002in}{0.955179in}}%
\pgfpathlineto{\pgfqpoint{1.301245in}{0.963745in}}%
\pgfpathlineto{\pgfqpoint{1.322487in}{0.970357in}}%
\pgfpathlineto{\pgfqpoint{1.343730in}{0.974775in}}%
\pgfpathlineto{\pgfqpoint{1.386215in}{0.968534in}}%
\pgfpathlineto{\pgfqpoint{1.407458in}{0.974620in}}%
\pgfpathlineto{\pgfqpoint{1.428700in}{0.989943in}}%
\pgfpathlineto{\pgfqpoint{1.449943in}{0.991100in}}%
\pgfpathlineto{\pgfqpoint{1.471185in}{0.978482in}}%
\pgfpathlineto{\pgfqpoint{1.492428in}{0.989009in}}%
\pgfpathlineto{\pgfqpoint{1.513671in}{0.996561in}}%
\pgfpathlineto{\pgfqpoint{1.534913in}{0.981947in}}%
\pgfpathlineto{\pgfqpoint{1.556156in}{0.990859in}}%
\pgfpathlineto{\pgfqpoint{1.577398in}{0.993644in}}%
\pgfpathlineto{\pgfqpoint{1.598641in}{0.983243in}}%
\pgfpathlineto{\pgfqpoint{1.619884in}{0.999580in}}%
\pgfpathlineto{\pgfqpoint{1.641126in}{0.988500in}}%
\pgfpathlineto{\pgfqpoint{1.662369in}{0.985481in}}%
\pgfpathlineto{\pgfqpoint{1.683611in}{0.999048in}}%
\pgfpathlineto{\pgfqpoint{1.704854in}{1.022735in}}%
\pgfpathlineto{\pgfqpoint{1.726096in}{1.025846in}}%
\pgfpathlineto{\pgfqpoint{1.747339in}{1.038029in}}%
\pgfpathlineto{\pgfqpoint{1.768582in}{1.039153in}}%
\pgfpathlineto{\pgfqpoint{1.789824in}{1.017337in}}%
\pgfpathlineto{\pgfqpoint{1.811067in}{1.027316in}}%
\pgfpathlineto{\pgfqpoint{1.853552in}{1.022327in}}%
\pgfpathlineto{\pgfqpoint{1.874794in}{1.025000in}}%
\pgfpathlineto{\pgfqpoint{1.896037in}{1.013960in}}%
\pgfpathlineto{\pgfqpoint{1.917280in}{1.020634in}}%
\pgfpathlineto{\pgfqpoint{1.959765in}{1.040836in}}%
\pgfpathlineto{\pgfqpoint{1.981007in}{1.057417in}}%
\pgfpathlineto{\pgfqpoint{2.023492in}{1.068919in}}%
\pgfpathlineto{\pgfqpoint{2.044735in}{1.066635in}}%
\pgfpathlineto{\pgfqpoint{2.065978in}{1.084402in}}%
\pgfpathlineto{\pgfqpoint{2.087220in}{1.075369in}}%
\pgfpathlineto{\pgfqpoint{2.108463in}{1.101751in}}%
\pgfpathlineto{\pgfqpoint{2.129705in}{1.095895in}}%
\pgfpathlineto{\pgfqpoint{2.150948in}{1.079500in}}%
\pgfpathlineto{\pgfqpoint{2.172190in}{1.072777in}}%
\pgfpathlineto{\pgfqpoint{2.193433in}{1.077388in}}%
\pgfpathlineto{\pgfqpoint{2.214676in}{1.084645in}}%
\pgfpathlineto{\pgfqpoint{2.235918in}{1.073903in}}%
\pgfpathlineto{\pgfqpoint{2.257161in}{1.079611in}}%
\pgfpathlineto{\pgfqpoint{2.278403in}{1.065204in}}%
\pgfpathlineto{\pgfqpoint{2.299646in}{1.075655in}}%
\pgfpathlineto{\pgfqpoint{2.320889in}{1.082477in}}%
\pgfpathlineto{\pgfqpoint{2.342131in}{1.091389in}}%
\pgfpathlineto{\pgfqpoint{2.363374in}{1.106842in}}%
\pgfpathlineto{\pgfqpoint{2.384616in}{1.114928in}}%
\pgfpathlineto{\pgfqpoint{2.405859in}{1.118760in}}%
\pgfpathlineto{\pgfqpoint{2.427101in}{1.119814in}}%
\pgfpathlineto{\pgfqpoint{2.448344in}{1.122874in}}%
\pgfpathlineto{\pgfqpoint{2.469587in}{1.114589in}}%
\pgfpathlineto{\pgfqpoint{2.490829in}{1.117528in}}%
\pgfpathlineto{\pgfqpoint{2.512072in}{1.114823in}}%
\pgfpathlineto{\pgfqpoint{2.533314in}{1.103793in}}%
\pgfpathlineto{\pgfqpoint{2.554557in}{1.110189in}}%
\pgfpathlineto{\pgfqpoint{2.575799in}{1.124410in}}%
\pgfpathlineto{\pgfqpoint{2.597042in}{1.142836in}}%
\pgfpathlineto{\pgfqpoint{2.618285in}{1.143220in}}%
\pgfpathlineto{\pgfqpoint{2.639527in}{1.159121in}}%
\pgfpathlineto{\pgfqpoint{2.660770in}{1.165810in}}%
\pgfpathlineto{\pgfqpoint{2.682012in}{1.179226in}}%
\pgfpathlineto{\pgfqpoint{2.703255in}{1.165880in}}%
\pgfpathlineto{\pgfqpoint{2.724497in}{1.188010in}}%
\pgfpathlineto{\pgfqpoint{2.745740in}{1.202643in}}%
\pgfpathlineto{\pgfqpoint{2.766983in}{1.208973in}}%
\pgfpathlineto{\pgfqpoint{2.788225in}{1.233828in}}%
\pgfpathlineto{\pgfqpoint{2.809468in}{1.264159in}}%
\pgfpathlineto{\pgfqpoint{2.830710in}{1.265643in}}%
\pgfpathlineto{\pgfqpoint{2.851953in}{1.248615in}}%
\pgfpathlineto{\pgfqpoint{2.873196in}{1.260221in}}%
\pgfpathlineto{\pgfqpoint{2.894438in}{1.281772in}}%
\pgfpathlineto{\pgfqpoint{2.915681in}{1.263457in}}%
\pgfpathlineto{\pgfqpoint{2.936923in}{1.292499in}}%
\pgfpathlineto{\pgfqpoint{2.958166in}{1.277782in}}%
\pgfpathlineto{\pgfqpoint{2.979408in}{1.272271in}}%
\pgfpathlineto{\pgfqpoint{3.000651in}{1.264848in}}%
\pgfpathlineto{\pgfqpoint{3.021894in}{1.259902in}}%
\pgfpathlineto{\pgfqpoint{3.043136in}{1.263811in}}%
\pgfpathlineto{\pgfqpoint{3.064379in}{1.221733in}}%
\pgfpathlineto{\pgfqpoint{3.085621in}{1.242206in}}%
\pgfpathlineto{\pgfqpoint{3.106864in}{1.216186in}}%
\pgfpathlineto{\pgfqpoint{3.128106in}{1.248423in}}%
\pgfpathlineto{\pgfqpoint{3.149349in}{1.251538in}}%
\pgfpathlineto{\pgfqpoint{3.170592in}{1.243456in}}%
\pgfpathlineto{\pgfqpoint{3.191834in}{1.230648in}}%
\pgfpathlineto{\pgfqpoint{3.234319in}{1.243836in}}%
\pgfpathlineto{\pgfqpoint{3.255562in}{1.278426in}}%
\pgfpathlineto{\pgfqpoint{3.276804in}{1.272766in}}%
\pgfpathlineto{\pgfqpoint{3.298047in}{1.269126in}}%
\pgfpathlineto{\pgfqpoint{3.319290in}{1.269825in}}%
\pgfpathlineto{\pgfqpoint{3.340532in}{1.261362in}}%
\pgfpathlineto{\pgfqpoint{3.361775in}{1.251673in}}%
\pgfpathlineto{\pgfqpoint{3.383017in}{1.245402in}}%
\pgfpathlineto{\pgfqpoint{3.404260in}{1.230326in}}%
\pgfpathlineto{\pgfqpoint{3.425503in}{1.245870in}}%
\pgfpathlineto{\pgfqpoint{3.446745in}{1.222080in}}%
\pgfpathlineto{\pgfqpoint{3.467988in}{1.234139in}}%
\pgfpathlineto{\pgfqpoint{3.489230in}{1.231806in}}%
\pgfpathlineto{\pgfqpoint{3.510473in}{1.219731in}}%
\pgfpathlineto{\pgfqpoint{3.531715in}{1.194630in}}%
\pgfpathlineto{\pgfqpoint{3.552958in}{1.180049in}}%
\pgfpathlineto{\pgfqpoint{3.574201in}{1.167207in}}%
\pgfpathlineto{\pgfqpoint{3.595443in}{1.158908in}}%
\pgfpathlineto{\pgfqpoint{3.616686in}{1.156120in}}%
\pgfpathlineto{\pgfqpoint{3.637928in}{1.163085in}}%
\pgfpathlineto{\pgfqpoint{3.659171in}{1.165365in}}%
\pgfpathlineto{\pgfqpoint{3.680413in}{1.189360in}}%
\pgfpathlineto{\pgfqpoint{3.701656in}{1.210036in}}%
\pgfpathlineto{\pgfqpoint{3.722899in}{1.225723in}}%
\pgfpathlineto{\pgfqpoint{3.744141in}{1.258807in}}%
\pgfpathlineto{\pgfqpoint{3.765384in}{1.265804in}}%
\pgfpathlineto{\pgfqpoint{3.786626in}{1.285312in}}%
\pgfpathlineto{\pgfqpoint{3.807869in}{1.275282in}}%
\pgfpathlineto{\pgfqpoint{3.829111in}{1.267586in}}%
\pgfpathlineto{\pgfqpoint{3.850354in}{1.284901in}}%
\pgfpathlineto{\pgfqpoint{3.892839in}{1.267133in}}%
\pgfpathlineto{\pgfqpoint{3.914082in}{1.287771in}}%
\pgfpathlineto{\pgfqpoint{3.935324in}{1.296624in}}%
\pgfpathlineto{\pgfqpoint{3.956567in}{1.318235in}}%
\pgfpathlineto{\pgfqpoint{3.977810in}{1.317470in}}%
\pgfpathlineto{\pgfqpoint{3.999052in}{1.330146in}}%
\pgfpathlineto{\pgfqpoint{4.020295in}{1.341515in}}%
\pgfpathlineto{\pgfqpoint{4.041537in}{1.345806in}}%
\pgfpathlineto{\pgfqpoint{4.062780in}{1.359794in}}%
\pgfpathlineto{\pgfqpoint{4.084022in}{1.362753in}}%
\pgfpathlineto{\pgfqpoint{4.105265in}{1.351338in}}%
\pgfpathlineto{\pgfqpoint{4.126508in}{1.342399in}}%
\pgfpathlineto{\pgfqpoint{4.147750in}{1.316750in}}%
\pgfpathlineto{\pgfqpoint{4.168993in}{1.343618in}}%
\pgfpathlineto{\pgfqpoint{4.190235in}{1.352663in}}%
\pgfpathlineto{\pgfqpoint{4.211478in}{1.366323in}}%
\pgfpathlineto{\pgfqpoint{4.232720in}{1.363678in}}%
\pgfpathlineto{\pgfqpoint{4.253963in}{1.369885in}}%
\pgfpathlineto{\pgfqpoint{4.275206in}{1.390490in}}%
\pgfpathlineto{\pgfqpoint{4.296448in}{1.400256in}}%
\pgfpathlineto{\pgfqpoint{4.317691in}{1.388607in}}%
\pgfpathlineto{\pgfqpoint{4.338933in}{1.408903in}}%
\pgfpathlineto{\pgfqpoint{4.360176in}{1.411061in}}%
\pgfpathlineto{\pgfqpoint{4.381418in}{1.410474in}}%
\pgfpathlineto{\pgfqpoint{4.402661in}{1.416697in}}%
\pgfpathlineto{\pgfqpoint{4.423904in}{1.421373in}}%
\pgfpathlineto{\pgfqpoint{4.445146in}{1.403079in}}%
\pgfpathlineto{\pgfqpoint{4.466389in}{1.397788in}}%
\pgfpathlineto{\pgfqpoint{4.487631in}{1.398390in}}%
\pgfpathlineto{\pgfqpoint{4.508874in}{1.389590in}}%
\pgfpathlineto{\pgfqpoint{4.530116in}{1.389187in}}%
\pgfpathlineto{\pgfqpoint{4.551359in}{1.401910in}}%
\pgfpathlineto{\pgfqpoint{4.572602in}{1.380049in}}%
\pgfpathlineto{\pgfqpoint{4.593844in}{1.409394in}}%
\pgfpathlineto{\pgfqpoint{4.615087in}{1.409248in}}%
\pgfpathlineto{\pgfqpoint{4.636329in}{1.449120in}}%
\pgfpathlineto{\pgfqpoint{4.657572in}{1.431519in}}%
\pgfpathlineto{\pgfqpoint{4.678815in}{1.439977in}}%
\pgfpathlineto{\pgfqpoint{4.700057in}{1.419635in}}%
\pgfpathlineto{\pgfqpoint{4.721300in}{1.424007in}}%
\pgfpathlineto{\pgfqpoint{4.742542in}{1.456798in}}%
\pgfpathlineto{\pgfqpoint{4.763785in}{1.451717in}}%
\pgfpathlineto{\pgfqpoint{4.785027in}{1.447911in}}%
\pgfpathlineto{\pgfqpoint{4.806270in}{1.457091in}}%
\pgfpathlineto{\pgfqpoint{4.827513in}{1.483652in}}%
\pgfpathlineto{\pgfqpoint{4.848755in}{1.497475in}}%
\pgfpathlineto{\pgfqpoint{4.869998in}{1.504594in}}%
\pgfpathlineto{\pgfqpoint{4.891240in}{1.515446in}}%
\pgfpathlineto{\pgfqpoint{4.912483in}{1.543710in}}%
\pgfpathlineto{\pgfqpoint{4.933725in}{1.579133in}}%
\pgfpathlineto{\pgfqpoint{4.954968in}{1.573412in}}%
\pgfpathlineto{\pgfqpoint{4.976211in}{1.563404in}}%
\pgfpathlineto{\pgfqpoint{4.997453in}{1.549794in}}%
\pgfpathlineto{\pgfqpoint{5.018696in}{1.559590in}}%
\pgfpathlineto{\pgfqpoint{5.039938in}{1.529211in}}%
\pgfpathlineto{\pgfqpoint{5.061181in}{1.525907in}}%
\pgfpathlineto{\pgfqpoint{5.082423in}{1.503373in}}%
\pgfpathlineto{\pgfqpoint{5.103666in}{1.496083in}}%
\pgfpathlineto{\pgfqpoint{5.124909in}{1.518198in}}%
\pgfpathlineto{\pgfqpoint{5.146151in}{1.523489in}}%
\pgfpathlineto{\pgfqpoint{5.167394in}{1.523285in}}%
\pgfpathlineto{\pgfqpoint{5.188636in}{1.532774in}}%
\pgfpathlineto{\pgfqpoint{5.188636in}{1.532774in}}%
\pgfusepath{stroke}%
\end{pgfscope}%
\begin{pgfscope}%
\pgfpathrectangle{\pgfqpoint{0.750000in}{0.440000in}}{\pgfqpoint{4.650000in}{3.080000in}}%
\pgfusepath{clip}%
\pgfsetroundcap%
\pgfsetroundjoin%
\pgfsetlinewidth{1.003750pt}%
\definecolor{currentstroke}{rgb}{0.600000,0.600000,0.600000}%
\pgfsetstrokecolor{currentstroke}%
\pgfsetdash{}{0pt}%
\pgfpathmoveto{\pgfqpoint{0.961364in}{1.043235in}}%
\pgfpathlineto{\pgfqpoint{0.982606in}{1.057974in}}%
\pgfpathlineto{\pgfqpoint{1.003849in}{1.066628in}}%
\pgfpathlineto{\pgfqpoint{1.025091in}{1.052204in}}%
\pgfpathlineto{\pgfqpoint{1.046334in}{1.058330in}}%
\pgfpathlineto{\pgfqpoint{1.067577in}{1.077964in}}%
\pgfpathlineto{\pgfqpoint{1.088819in}{1.066219in}}%
\pgfpathlineto{\pgfqpoint{1.110062in}{1.091789in}}%
\pgfpathlineto{\pgfqpoint{1.131304in}{1.094651in}}%
\pgfpathlineto{\pgfqpoint{1.152547in}{1.091719in}}%
\pgfpathlineto{\pgfqpoint{1.173789in}{1.106480in}}%
\pgfpathlineto{\pgfqpoint{1.195032in}{1.096799in}}%
\pgfpathlineto{\pgfqpoint{1.216275in}{1.091464in}}%
\pgfpathlineto{\pgfqpoint{1.237517in}{1.065410in}}%
\pgfpathlineto{\pgfqpoint{1.258760in}{1.078764in}}%
\pgfpathlineto{\pgfqpoint{1.280002in}{1.084857in}}%
\pgfpathlineto{\pgfqpoint{1.301245in}{1.073131in}}%
\pgfpathlineto{\pgfqpoint{1.322487in}{1.088831in}}%
\pgfpathlineto{\pgfqpoint{1.343730in}{1.119640in}}%
\pgfpathlineto{\pgfqpoint{1.364973in}{1.118730in}}%
\pgfpathlineto{\pgfqpoint{1.386215in}{1.122263in}}%
\pgfpathlineto{\pgfqpoint{1.407458in}{1.129592in}}%
\pgfpathlineto{\pgfqpoint{1.428700in}{1.140192in}}%
\pgfpathlineto{\pgfqpoint{1.449943in}{1.144269in}}%
\pgfpathlineto{\pgfqpoint{1.471185in}{1.159092in}}%
\pgfpathlineto{\pgfqpoint{1.492428in}{1.192974in}}%
\pgfpathlineto{\pgfqpoint{1.513671in}{1.196104in}}%
\pgfpathlineto{\pgfqpoint{1.534913in}{1.202323in}}%
\pgfpathlineto{\pgfqpoint{1.556156in}{1.204510in}}%
\pgfpathlineto{\pgfqpoint{1.577398in}{1.190181in}}%
\pgfpathlineto{\pgfqpoint{1.598641in}{1.180170in}}%
\pgfpathlineto{\pgfqpoint{1.619884in}{1.178176in}}%
\pgfpathlineto{\pgfqpoint{1.641126in}{1.186323in}}%
\pgfpathlineto{\pgfqpoint{1.662369in}{1.182937in}}%
\pgfpathlineto{\pgfqpoint{1.683611in}{1.170711in}}%
\pgfpathlineto{\pgfqpoint{1.704854in}{1.190149in}}%
\pgfpathlineto{\pgfqpoint{1.726096in}{1.180412in}}%
\pgfpathlineto{\pgfqpoint{1.747339in}{1.179059in}}%
\pgfpathlineto{\pgfqpoint{1.768582in}{1.164345in}}%
\pgfpathlineto{\pgfqpoint{1.789824in}{1.182551in}}%
\pgfpathlineto{\pgfqpoint{1.811067in}{1.184944in}}%
\pgfpathlineto{\pgfqpoint{1.832309in}{1.181179in}}%
\pgfpathlineto{\pgfqpoint{1.853552in}{1.173449in}}%
\pgfpathlineto{\pgfqpoint{1.874794in}{1.181829in}}%
\pgfpathlineto{\pgfqpoint{1.896037in}{1.197263in}}%
\pgfpathlineto{\pgfqpoint{1.917280in}{1.177279in}}%
\pgfpathlineto{\pgfqpoint{1.938522in}{1.198436in}}%
\pgfpathlineto{\pgfqpoint{1.959765in}{1.193785in}}%
\pgfpathlineto{\pgfqpoint{1.981007in}{1.207671in}}%
\pgfpathlineto{\pgfqpoint{2.002250in}{1.217081in}}%
\pgfpathlineto{\pgfqpoint{2.023492in}{1.244739in}}%
\pgfpathlineto{\pgfqpoint{2.044735in}{1.247491in}}%
\pgfpathlineto{\pgfqpoint{2.087220in}{1.282901in}}%
\pgfpathlineto{\pgfqpoint{2.108463in}{1.300434in}}%
\pgfpathlineto{\pgfqpoint{2.129705in}{1.311925in}}%
\pgfpathlineto{\pgfqpoint{2.172190in}{1.341419in}}%
\pgfpathlineto{\pgfqpoint{2.193433in}{1.336117in}}%
\pgfpathlineto{\pgfqpoint{2.214676in}{1.362094in}}%
\pgfpathlineto{\pgfqpoint{2.235918in}{1.385762in}}%
\pgfpathlineto{\pgfqpoint{2.257161in}{1.367292in}}%
\pgfpathlineto{\pgfqpoint{2.278403in}{1.373123in}}%
\pgfpathlineto{\pgfqpoint{2.320889in}{1.396572in}}%
\pgfpathlineto{\pgfqpoint{2.342131in}{1.403507in}}%
\pgfpathlineto{\pgfqpoint{2.363374in}{1.414566in}}%
\pgfpathlineto{\pgfqpoint{2.405859in}{1.387744in}}%
\pgfpathlineto{\pgfqpoint{2.427101in}{1.372528in}}%
\pgfpathlineto{\pgfqpoint{2.448344in}{1.377442in}}%
\pgfpathlineto{\pgfqpoint{2.469587in}{1.393278in}}%
\pgfpathlineto{\pgfqpoint{2.490829in}{1.394215in}}%
\pgfpathlineto{\pgfqpoint{2.512072in}{1.379917in}}%
\pgfpathlineto{\pgfqpoint{2.533314in}{1.377053in}}%
\pgfpathlineto{\pgfqpoint{2.554557in}{1.398977in}}%
\pgfpathlineto{\pgfqpoint{2.575799in}{1.387427in}}%
\pgfpathlineto{\pgfqpoint{2.597042in}{1.382095in}}%
\pgfpathlineto{\pgfqpoint{2.618285in}{1.365393in}}%
\pgfpathlineto{\pgfqpoint{2.639527in}{1.370860in}}%
\pgfpathlineto{\pgfqpoint{2.682012in}{1.397293in}}%
\pgfpathlineto{\pgfqpoint{2.703255in}{1.383667in}}%
\pgfpathlineto{\pgfqpoint{2.724497in}{1.388406in}}%
\pgfpathlineto{\pgfqpoint{2.745740in}{1.396409in}}%
\pgfpathlineto{\pgfqpoint{2.766983in}{1.380739in}}%
\pgfpathlineto{\pgfqpoint{2.788225in}{1.390029in}}%
\pgfpathlineto{\pgfqpoint{2.809468in}{1.388366in}}%
\pgfpathlineto{\pgfqpoint{2.830710in}{1.406107in}}%
\pgfpathlineto{\pgfqpoint{2.851953in}{1.425623in}}%
\pgfpathlineto{\pgfqpoint{2.873196in}{1.437260in}}%
\pgfpathlineto{\pgfqpoint{2.894438in}{1.440536in}}%
\pgfpathlineto{\pgfqpoint{2.915681in}{1.448085in}}%
\pgfpathlineto{\pgfqpoint{2.936923in}{1.454340in}}%
\pgfpathlineto{\pgfqpoint{2.958166in}{1.475325in}}%
\pgfpathlineto{\pgfqpoint{3.000651in}{1.448327in}}%
\pgfpathlineto{\pgfqpoint{3.021894in}{1.451804in}}%
\pgfpathlineto{\pgfqpoint{3.043136in}{1.451271in}}%
\pgfpathlineto{\pgfqpoint{3.064379in}{1.460927in}}%
\pgfpathlineto{\pgfqpoint{3.085621in}{1.482499in}}%
\pgfpathlineto{\pgfqpoint{3.106864in}{1.470265in}}%
\pgfpathlineto{\pgfqpoint{3.128106in}{1.482867in}}%
\pgfpathlineto{\pgfqpoint{3.149349in}{1.492328in}}%
\pgfpathlineto{\pgfqpoint{3.170592in}{1.493732in}}%
\pgfpathlineto{\pgfqpoint{3.191834in}{1.515968in}}%
\pgfpathlineto{\pgfqpoint{3.213077in}{1.541243in}}%
\pgfpathlineto{\pgfqpoint{3.234319in}{1.544833in}}%
\pgfpathlineto{\pgfqpoint{3.255562in}{1.526993in}}%
\pgfpathlineto{\pgfqpoint{3.276804in}{1.500443in}}%
\pgfpathlineto{\pgfqpoint{3.298047in}{1.511589in}}%
\pgfpathlineto{\pgfqpoint{3.319290in}{1.516202in}}%
\pgfpathlineto{\pgfqpoint{3.340532in}{1.494601in}}%
\pgfpathlineto{\pgfqpoint{3.361775in}{1.501265in}}%
\pgfpathlineto{\pgfqpoint{3.383017in}{1.479410in}}%
\pgfpathlineto{\pgfqpoint{3.404260in}{1.476828in}}%
\pgfpathlineto{\pgfqpoint{3.425503in}{1.468040in}}%
\pgfpathlineto{\pgfqpoint{3.446745in}{1.464366in}}%
\pgfpathlineto{\pgfqpoint{3.467988in}{1.481895in}}%
\pgfpathlineto{\pgfqpoint{3.489230in}{1.468086in}}%
\pgfpathlineto{\pgfqpoint{3.510473in}{1.484381in}}%
\pgfpathlineto{\pgfqpoint{3.531715in}{1.494038in}}%
\pgfpathlineto{\pgfqpoint{3.552958in}{1.465426in}}%
\pgfpathlineto{\pgfqpoint{3.574201in}{1.461194in}}%
\pgfpathlineto{\pgfqpoint{3.595443in}{1.479372in}}%
\pgfpathlineto{\pgfqpoint{3.616686in}{1.507448in}}%
\pgfpathlineto{\pgfqpoint{3.637928in}{1.509559in}}%
\pgfpathlineto{\pgfqpoint{3.659171in}{1.514984in}}%
\pgfpathlineto{\pgfqpoint{3.680413in}{1.506692in}}%
\pgfpathlineto{\pgfqpoint{3.701656in}{1.539212in}}%
\pgfpathlineto{\pgfqpoint{3.722899in}{1.514609in}}%
\pgfpathlineto{\pgfqpoint{3.744141in}{1.520376in}}%
\pgfpathlineto{\pgfqpoint{3.765384in}{1.516568in}}%
\pgfpathlineto{\pgfqpoint{3.786626in}{1.533946in}}%
\pgfpathlineto{\pgfqpoint{3.807869in}{1.568714in}}%
\pgfpathlineto{\pgfqpoint{3.850354in}{1.528869in}}%
\pgfpathlineto{\pgfqpoint{3.871597in}{1.515798in}}%
\pgfpathlineto{\pgfqpoint{3.892839in}{1.512256in}}%
\pgfpathlineto{\pgfqpoint{3.914082in}{1.536573in}}%
\pgfpathlineto{\pgfqpoint{3.935324in}{1.503951in}}%
\pgfpathlineto{\pgfqpoint{3.956567in}{1.488826in}}%
\pgfpathlineto{\pgfqpoint{3.999052in}{1.509944in}}%
\pgfpathlineto{\pgfqpoint{4.020295in}{1.531910in}}%
\pgfpathlineto{\pgfqpoint{4.041537in}{1.545552in}}%
\pgfpathlineto{\pgfqpoint{4.062780in}{1.557841in}}%
\pgfpathlineto{\pgfqpoint{4.084022in}{1.562001in}}%
\pgfpathlineto{\pgfqpoint{4.105265in}{1.572146in}}%
\pgfpathlineto{\pgfqpoint{4.126508in}{1.587892in}}%
\pgfpathlineto{\pgfqpoint{4.147750in}{1.584904in}}%
\pgfpathlineto{\pgfqpoint{4.168993in}{1.589742in}}%
\pgfpathlineto{\pgfqpoint{4.190235in}{1.613036in}}%
\pgfpathlineto{\pgfqpoint{4.211478in}{1.614866in}}%
\pgfpathlineto{\pgfqpoint{4.232720in}{1.629169in}}%
\pgfpathlineto{\pgfqpoint{4.253963in}{1.607891in}}%
\pgfpathlineto{\pgfqpoint{4.275206in}{1.629584in}}%
\pgfpathlineto{\pgfqpoint{4.296448in}{1.668195in}}%
\pgfpathlineto{\pgfqpoint{4.317691in}{1.670864in}}%
\pgfpathlineto{\pgfqpoint{4.338933in}{1.665616in}}%
\pgfpathlineto{\pgfqpoint{4.360176in}{1.651290in}}%
\pgfpathlineto{\pgfqpoint{4.381418in}{1.622506in}}%
\pgfpathlineto{\pgfqpoint{4.402661in}{1.629005in}}%
\pgfpathlineto{\pgfqpoint{4.423904in}{1.623747in}}%
\pgfpathlineto{\pgfqpoint{4.445146in}{1.616363in}}%
\pgfpathlineto{\pgfqpoint{4.466389in}{1.615138in}}%
\pgfpathlineto{\pgfqpoint{4.487631in}{1.594458in}}%
\pgfpathlineto{\pgfqpoint{4.508874in}{1.626933in}}%
\pgfpathlineto{\pgfqpoint{4.530116in}{1.641240in}}%
\pgfpathlineto{\pgfqpoint{4.551359in}{1.647172in}}%
\pgfpathlineto{\pgfqpoint{4.572602in}{1.627347in}}%
\pgfpathlineto{\pgfqpoint{4.593844in}{1.639176in}}%
\pgfpathlineto{\pgfqpoint{4.615087in}{1.672422in}}%
\pgfpathlineto{\pgfqpoint{4.636329in}{1.669519in}}%
\pgfpathlineto{\pgfqpoint{4.657572in}{1.672375in}}%
\pgfpathlineto{\pgfqpoint{4.678815in}{1.665192in}}%
\pgfpathlineto{\pgfqpoint{4.700057in}{1.700639in}}%
\pgfpathlineto{\pgfqpoint{4.721300in}{1.701093in}}%
\pgfpathlineto{\pgfqpoint{4.742542in}{1.718678in}}%
\pgfpathlineto{\pgfqpoint{4.763785in}{1.738574in}}%
\pgfpathlineto{\pgfqpoint{4.785027in}{1.763919in}}%
\pgfpathlineto{\pgfqpoint{4.806270in}{1.782381in}}%
\pgfpathlineto{\pgfqpoint{4.827513in}{1.782098in}}%
\pgfpathlineto{\pgfqpoint{4.848755in}{1.777160in}}%
\pgfpathlineto{\pgfqpoint{4.869998in}{1.787940in}}%
\pgfpathlineto{\pgfqpoint{4.891240in}{1.782628in}}%
\pgfpathlineto{\pgfqpoint{4.912483in}{1.789906in}}%
\pgfpathlineto{\pgfqpoint{4.933725in}{1.790495in}}%
\pgfpathlineto{\pgfqpoint{4.954968in}{1.805470in}}%
\pgfpathlineto{\pgfqpoint{4.976211in}{1.801217in}}%
\pgfpathlineto{\pgfqpoint{4.997453in}{1.817406in}}%
\pgfpathlineto{\pgfqpoint{5.018696in}{1.810619in}}%
\pgfpathlineto{\pgfqpoint{5.039938in}{1.794271in}}%
\pgfpathlineto{\pgfqpoint{5.082423in}{1.750932in}}%
\pgfpathlineto{\pgfqpoint{5.124909in}{1.769108in}}%
\pgfpathlineto{\pgfqpoint{5.146151in}{1.780796in}}%
\pgfpathlineto{\pgfqpoint{5.167394in}{1.778811in}}%
\pgfpathlineto{\pgfqpoint{5.188636in}{1.806578in}}%
\pgfpathlineto{\pgfqpoint{5.188636in}{1.806578in}}%
\pgfusepath{stroke}%
\end{pgfscope}%
\begin{pgfscope}%
\pgfpathrectangle{\pgfqpoint{0.750000in}{0.440000in}}{\pgfqpoint{4.650000in}{3.080000in}}%
\pgfusepath{clip}%
\pgfsetroundcap%
\pgfsetroundjoin%
\pgfsetlinewidth{1.003750pt}%
\definecolor{currentstroke}{rgb}{0.894118,0.101961,0.109804}%
\pgfsetstrokecolor{currentstroke}%
\pgfsetdash{}{0pt}%
\pgfpathmoveto{\pgfqpoint{0.961364in}{1.120085in}}%
\pgfpathlineto{\pgfqpoint{0.982606in}{1.125338in}}%
\pgfpathlineto{\pgfqpoint{1.003849in}{1.136696in}}%
\pgfpathlineto{\pgfqpoint{1.025091in}{1.164996in}}%
\pgfpathlineto{\pgfqpoint{1.046334in}{1.164145in}}%
\pgfpathlineto{\pgfqpoint{1.067577in}{1.181025in}}%
\pgfpathlineto{\pgfqpoint{1.110062in}{1.183014in}}%
\pgfpathlineto{\pgfqpoint{1.131304in}{1.194032in}}%
\pgfpathlineto{\pgfqpoint{1.152547in}{1.171412in}}%
\pgfpathlineto{\pgfqpoint{1.173789in}{1.176792in}}%
\pgfpathlineto{\pgfqpoint{1.195032in}{1.189739in}}%
\pgfpathlineto{\pgfqpoint{1.237517in}{1.206084in}}%
\pgfpathlineto{\pgfqpoint{1.258760in}{1.218661in}}%
\pgfpathlineto{\pgfqpoint{1.280002in}{1.239513in}}%
\pgfpathlineto{\pgfqpoint{1.301245in}{1.237339in}}%
\pgfpathlineto{\pgfqpoint{1.322487in}{1.255864in}}%
\pgfpathlineto{\pgfqpoint{1.343730in}{1.267435in}}%
\pgfpathlineto{\pgfqpoint{1.364973in}{1.273370in}}%
\pgfpathlineto{\pgfqpoint{1.386215in}{1.299896in}}%
\pgfpathlineto{\pgfqpoint{1.407458in}{1.312116in}}%
\pgfpathlineto{\pgfqpoint{1.428700in}{1.328196in}}%
\pgfpathlineto{\pgfqpoint{1.449943in}{1.313265in}}%
\pgfpathlineto{\pgfqpoint{1.471185in}{1.310890in}}%
\pgfpathlineto{\pgfqpoint{1.492428in}{1.296781in}}%
\pgfpathlineto{\pgfqpoint{1.513671in}{1.289267in}}%
\pgfpathlineto{\pgfqpoint{1.534913in}{1.297529in}}%
\pgfpathlineto{\pgfqpoint{1.556156in}{1.307783in}}%
\pgfpathlineto{\pgfqpoint{1.577398in}{1.309261in}}%
\pgfpathlineto{\pgfqpoint{1.598641in}{1.315392in}}%
\pgfpathlineto{\pgfqpoint{1.641126in}{1.370981in}}%
\pgfpathlineto{\pgfqpoint{1.662369in}{1.383805in}}%
\pgfpathlineto{\pgfqpoint{1.683611in}{1.397943in}}%
\pgfpathlineto{\pgfqpoint{1.704854in}{1.403602in}}%
\pgfpathlineto{\pgfqpoint{1.726096in}{1.389133in}}%
\pgfpathlineto{\pgfqpoint{1.747339in}{1.403767in}}%
\pgfpathlineto{\pgfqpoint{1.768582in}{1.409542in}}%
\pgfpathlineto{\pgfqpoint{1.789824in}{1.433417in}}%
\pgfpathlineto{\pgfqpoint{1.811067in}{1.440329in}}%
\pgfpathlineto{\pgfqpoint{1.832309in}{1.428796in}}%
\pgfpathlineto{\pgfqpoint{1.853552in}{1.439246in}}%
\pgfpathlineto{\pgfqpoint{1.874794in}{1.432844in}}%
\pgfpathlineto{\pgfqpoint{1.896037in}{1.459344in}}%
\pgfpathlineto{\pgfqpoint{1.917280in}{1.471190in}}%
\pgfpathlineto{\pgfqpoint{1.938522in}{1.480351in}}%
\pgfpathlineto{\pgfqpoint{1.959765in}{1.498282in}}%
\pgfpathlineto{\pgfqpoint{1.981007in}{1.503331in}}%
\pgfpathlineto{\pgfqpoint{2.002250in}{1.492495in}}%
\pgfpathlineto{\pgfqpoint{2.023492in}{1.505751in}}%
\pgfpathlineto{\pgfqpoint{2.044735in}{1.509630in}}%
\pgfpathlineto{\pgfqpoint{2.065978in}{1.518495in}}%
\pgfpathlineto{\pgfqpoint{2.087220in}{1.560158in}}%
\pgfpathlineto{\pgfqpoint{2.129705in}{1.601778in}}%
\pgfpathlineto{\pgfqpoint{2.150948in}{1.599684in}}%
\pgfpathlineto{\pgfqpoint{2.172190in}{1.591863in}}%
\pgfpathlineto{\pgfqpoint{2.193433in}{1.580970in}}%
\pgfpathlineto{\pgfqpoint{2.214676in}{1.606482in}}%
\pgfpathlineto{\pgfqpoint{2.235918in}{1.616187in}}%
\pgfpathlineto{\pgfqpoint{2.257161in}{1.615602in}}%
\pgfpathlineto{\pgfqpoint{2.278403in}{1.621458in}}%
\pgfpathlineto{\pgfqpoint{2.299646in}{1.628865in}}%
\pgfpathlineto{\pgfqpoint{2.320889in}{1.599217in}}%
\pgfpathlineto{\pgfqpoint{2.363374in}{1.617472in}}%
\pgfpathlineto{\pgfqpoint{2.384616in}{1.629338in}}%
\pgfpathlineto{\pgfqpoint{2.405859in}{1.628077in}}%
\pgfpathlineto{\pgfqpoint{2.427101in}{1.649664in}}%
\pgfpathlineto{\pgfqpoint{2.448344in}{1.646463in}}%
\pgfpathlineto{\pgfqpoint{2.469587in}{1.647873in}}%
\pgfpathlineto{\pgfqpoint{2.490829in}{1.636745in}}%
\pgfpathlineto{\pgfqpoint{2.512072in}{1.642710in}}%
\pgfpathlineto{\pgfqpoint{2.533314in}{1.631620in}}%
\pgfpathlineto{\pgfqpoint{2.554557in}{1.646287in}}%
\pgfpathlineto{\pgfqpoint{2.575799in}{1.668526in}}%
\pgfpathlineto{\pgfqpoint{2.597042in}{1.645957in}}%
\pgfpathlineto{\pgfqpoint{2.618285in}{1.638588in}}%
\pgfpathlineto{\pgfqpoint{2.639527in}{1.643301in}}%
\pgfpathlineto{\pgfqpoint{2.660770in}{1.666292in}}%
\pgfpathlineto{\pgfqpoint{2.682012in}{1.693478in}}%
\pgfpathlineto{\pgfqpoint{2.703255in}{1.679646in}}%
\pgfpathlineto{\pgfqpoint{2.724497in}{1.692590in}}%
\pgfpathlineto{\pgfqpoint{2.745740in}{1.679277in}}%
\pgfpathlineto{\pgfqpoint{2.766983in}{1.654914in}}%
\pgfpathlineto{\pgfqpoint{2.788225in}{1.677623in}}%
\pgfpathlineto{\pgfqpoint{2.809468in}{1.697968in}}%
\pgfpathlineto{\pgfqpoint{2.830710in}{1.708235in}}%
\pgfpathlineto{\pgfqpoint{2.851953in}{1.730417in}}%
\pgfpathlineto{\pgfqpoint{2.873196in}{1.735783in}}%
\pgfpathlineto{\pgfqpoint{2.894438in}{1.759623in}}%
\pgfpathlineto{\pgfqpoint{2.915681in}{1.696938in}}%
\pgfpathlineto{\pgfqpoint{2.936923in}{1.684570in}}%
\pgfpathlineto{\pgfqpoint{2.958166in}{1.657983in}}%
\pgfpathlineto{\pgfqpoint{2.979408in}{1.665185in}}%
\pgfpathlineto{\pgfqpoint{3.000651in}{1.680674in}}%
\pgfpathlineto{\pgfqpoint{3.021894in}{1.682191in}}%
\pgfpathlineto{\pgfqpoint{3.043136in}{1.685474in}}%
\pgfpathlineto{\pgfqpoint{3.064379in}{1.710573in}}%
\pgfpathlineto{\pgfqpoint{3.085621in}{1.727610in}}%
\pgfpathlineto{\pgfqpoint{3.106864in}{1.724542in}}%
\pgfpathlineto{\pgfqpoint{3.128106in}{1.751785in}}%
\pgfpathlineto{\pgfqpoint{3.149349in}{1.776097in}}%
\pgfpathlineto{\pgfqpoint{3.170592in}{1.783559in}}%
\pgfpathlineto{\pgfqpoint{3.191834in}{1.793872in}}%
\pgfpathlineto{\pgfqpoint{3.213077in}{1.796879in}}%
\pgfpathlineto{\pgfqpoint{3.234319in}{1.797251in}}%
\pgfpathlineto{\pgfqpoint{3.255562in}{1.801843in}}%
\pgfpathlineto{\pgfqpoint{3.276804in}{1.835203in}}%
\pgfpathlineto{\pgfqpoint{3.298047in}{1.865057in}}%
\pgfpathlineto{\pgfqpoint{3.319290in}{1.906781in}}%
\pgfpathlineto{\pgfqpoint{3.340532in}{1.899267in}}%
\pgfpathlineto{\pgfqpoint{3.361775in}{1.877156in}}%
\pgfpathlineto{\pgfqpoint{3.383017in}{1.878940in}}%
\pgfpathlineto{\pgfqpoint{3.404260in}{1.863011in}}%
\pgfpathlineto{\pgfqpoint{3.425503in}{1.866919in}}%
\pgfpathlineto{\pgfqpoint{3.446745in}{1.859403in}}%
\pgfpathlineto{\pgfqpoint{3.467988in}{1.872999in}}%
\pgfpathlineto{\pgfqpoint{3.489230in}{1.867402in}}%
\pgfpathlineto{\pgfqpoint{3.510473in}{1.872777in}}%
\pgfpathlineto{\pgfqpoint{3.531715in}{1.871823in}}%
\pgfpathlineto{\pgfqpoint{3.552958in}{1.896086in}}%
\pgfpathlineto{\pgfqpoint{3.574201in}{1.904520in}}%
\pgfpathlineto{\pgfqpoint{3.595443in}{1.920718in}}%
\pgfpathlineto{\pgfqpoint{3.637928in}{1.904388in}}%
\pgfpathlineto{\pgfqpoint{3.659171in}{1.898757in}}%
\pgfpathlineto{\pgfqpoint{3.680413in}{1.931541in}}%
\pgfpathlineto{\pgfqpoint{3.701656in}{1.929883in}}%
\pgfpathlineto{\pgfqpoint{3.722899in}{1.982455in}}%
\pgfpathlineto{\pgfqpoint{3.744141in}{1.989662in}}%
\pgfpathlineto{\pgfqpoint{3.765384in}{2.014214in}}%
\pgfpathlineto{\pgfqpoint{3.786626in}{2.044723in}}%
\pgfpathlineto{\pgfqpoint{3.807869in}{2.058193in}}%
\pgfpathlineto{\pgfqpoint{3.829111in}{2.062797in}}%
\pgfpathlineto{\pgfqpoint{3.850354in}{2.053148in}}%
\pgfpathlineto{\pgfqpoint{3.871597in}{2.085720in}}%
\pgfpathlineto{\pgfqpoint{3.892839in}{2.035973in}}%
\pgfpathlineto{\pgfqpoint{3.914082in}{2.082340in}}%
\pgfpathlineto{\pgfqpoint{3.935324in}{2.092624in}}%
\pgfpathlineto{\pgfqpoint{3.956567in}{2.104322in}}%
\pgfpathlineto{\pgfqpoint{3.977810in}{2.120423in}}%
\pgfpathlineto{\pgfqpoint{3.999052in}{2.152500in}}%
\pgfpathlineto{\pgfqpoint{4.020295in}{2.167633in}}%
\pgfpathlineto{\pgfqpoint{4.041537in}{2.185935in}}%
\pgfpathlineto{\pgfqpoint{4.062780in}{2.220105in}}%
\pgfpathlineto{\pgfqpoint{4.084022in}{2.235144in}}%
\pgfpathlineto{\pgfqpoint{4.105265in}{2.226175in}}%
\pgfpathlineto{\pgfqpoint{4.126508in}{2.248620in}}%
\pgfpathlineto{\pgfqpoint{4.147750in}{2.237483in}}%
\pgfpathlineto{\pgfqpoint{4.168993in}{2.224414in}}%
\pgfpathlineto{\pgfqpoint{4.190235in}{2.198925in}}%
\pgfpathlineto{\pgfqpoint{4.211478in}{2.234823in}}%
\pgfpathlineto{\pgfqpoint{4.232720in}{2.218129in}}%
\pgfpathlineto{\pgfqpoint{4.253963in}{2.220824in}}%
\pgfpathlineto{\pgfqpoint{4.275206in}{2.210825in}}%
\pgfpathlineto{\pgfqpoint{4.296448in}{2.182522in}}%
\pgfpathlineto{\pgfqpoint{4.317691in}{2.189477in}}%
\pgfpathlineto{\pgfqpoint{4.338933in}{2.158230in}}%
\pgfpathlineto{\pgfqpoint{4.360176in}{2.160762in}}%
\pgfpathlineto{\pgfqpoint{4.381418in}{2.192788in}}%
\pgfpathlineto{\pgfqpoint{4.402661in}{2.126306in}}%
\pgfpathlineto{\pgfqpoint{4.423904in}{2.123787in}}%
\pgfpathlineto{\pgfqpoint{4.445146in}{2.143770in}}%
\pgfpathlineto{\pgfqpoint{4.466389in}{2.160815in}}%
\pgfpathlineto{\pgfqpoint{4.487631in}{2.175521in}}%
\pgfpathlineto{\pgfqpoint{4.508874in}{2.206422in}}%
\pgfpathlineto{\pgfqpoint{4.530116in}{2.208028in}}%
\pgfpathlineto{\pgfqpoint{4.551359in}{2.181830in}}%
\pgfpathlineto{\pgfqpoint{4.572602in}{2.190412in}}%
\pgfpathlineto{\pgfqpoint{4.593844in}{2.186382in}}%
\pgfpathlineto{\pgfqpoint{4.615087in}{2.214176in}}%
\pgfpathlineto{\pgfqpoint{4.636329in}{2.216868in}}%
\pgfpathlineto{\pgfqpoint{4.657572in}{2.240497in}}%
\pgfpathlineto{\pgfqpoint{4.678815in}{2.281843in}}%
\pgfpathlineto{\pgfqpoint{4.700057in}{2.260234in}}%
\pgfpathlineto{\pgfqpoint{4.721300in}{2.271554in}}%
\pgfpathlineto{\pgfqpoint{4.742542in}{2.253281in}}%
\pgfpathlineto{\pgfqpoint{4.763785in}{2.283583in}}%
\pgfpathlineto{\pgfqpoint{4.806270in}{2.300672in}}%
\pgfpathlineto{\pgfqpoint{4.827513in}{2.300936in}}%
\pgfpathlineto{\pgfqpoint{4.848755in}{2.325750in}}%
\pgfpathlineto{\pgfqpoint{4.869998in}{2.309446in}}%
\pgfpathlineto{\pgfqpoint{4.912483in}{2.288428in}}%
\pgfpathlineto{\pgfqpoint{4.933725in}{2.237656in}}%
\pgfpathlineto{\pgfqpoint{4.954968in}{2.220804in}}%
\pgfpathlineto{\pgfqpoint{4.976211in}{2.227819in}}%
\pgfpathlineto{\pgfqpoint{4.997453in}{2.233552in}}%
\pgfpathlineto{\pgfqpoint{5.018696in}{2.257915in}}%
\pgfpathlineto{\pgfqpoint{5.039938in}{2.277899in}}%
\pgfpathlineto{\pgfqpoint{5.082423in}{2.210219in}}%
\pgfpathlineto{\pgfqpoint{5.103666in}{2.220388in}}%
\pgfpathlineto{\pgfqpoint{5.124909in}{2.250960in}}%
\pgfpathlineto{\pgfqpoint{5.146151in}{2.253698in}}%
\pgfpathlineto{\pgfqpoint{5.167394in}{2.251229in}}%
\pgfpathlineto{\pgfqpoint{5.188636in}{2.308351in}}%
\pgfpathlineto{\pgfqpoint{5.188636in}{2.308351in}}%
\pgfusepath{stroke}%
\end{pgfscope}%
\begin{pgfscope}%
\pgfpathrectangle{\pgfqpoint{0.750000in}{0.440000in}}{\pgfqpoint{4.650000in}{3.080000in}}%
\pgfusepath{clip}%
\pgfsetroundcap%
\pgfsetroundjoin%
\pgfsetlinewidth{1.003750pt}%
\definecolor{currentstroke}{rgb}{0.870588,0.870588,0.000000}%
\pgfsetstrokecolor{currentstroke}%
\pgfsetdash{}{0pt}%
\pgfpathmoveto{\pgfqpoint{0.961364in}{1.196934in}}%
\pgfpathlineto{\pgfqpoint{0.982606in}{1.206111in}}%
\pgfpathlineto{\pgfqpoint{1.025091in}{1.264766in}}%
\pgfpathlineto{\pgfqpoint{1.046334in}{1.292450in}}%
\pgfpathlineto{\pgfqpoint{1.067577in}{1.293133in}}%
\pgfpathlineto{\pgfqpoint{1.088819in}{1.298099in}}%
\pgfpathlineto{\pgfqpoint{1.110062in}{1.337522in}}%
\pgfpathlineto{\pgfqpoint{1.152547in}{1.387543in}}%
\pgfpathlineto{\pgfqpoint{1.173789in}{1.373932in}}%
\pgfpathlineto{\pgfqpoint{1.195032in}{1.390790in}}%
\pgfpathlineto{\pgfqpoint{1.216275in}{1.440715in}}%
\pgfpathlineto{\pgfqpoint{1.237517in}{1.485723in}}%
\pgfpathlineto{\pgfqpoint{1.258760in}{1.474713in}}%
\pgfpathlineto{\pgfqpoint{1.280002in}{1.480377in}}%
\pgfpathlineto{\pgfqpoint{1.301245in}{1.472526in}}%
\pgfpathlineto{\pgfqpoint{1.322487in}{1.484300in}}%
\pgfpathlineto{\pgfqpoint{1.343730in}{1.498346in}}%
\pgfpathlineto{\pgfqpoint{1.364973in}{1.504443in}}%
\pgfpathlineto{\pgfqpoint{1.386215in}{1.485423in}}%
\pgfpathlineto{\pgfqpoint{1.407458in}{1.504416in}}%
\pgfpathlineto{\pgfqpoint{1.449943in}{1.558199in}}%
\pgfpathlineto{\pgfqpoint{1.471185in}{1.540602in}}%
\pgfpathlineto{\pgfqpoint{1.492428in}{1.560248in}}%
\pgfpathlineto{\pgfqpoint{1.513671in}{1.547838in}}%
\pgfpathlineto{\pgfqpoint{1.534913in}{1.584277in}}%
\pgfpathlineto{\pgfqpoint{1.556156in}{1.574661in}}%
\pgfpathlineto{\pgfqpoint{1.577398in}{1.584929in}}%
\pgfpathlineto{\pgfqpoint{1.598641in}{1.589544in}}%
\pgfpathlineto{\pgfqpoint{1.619884in}{1.632741in}}%
\pgfpathlineto{\pgfqpoint{1.641126in}{1.626605in}}%
\pgfpathlineto{\pgfqpoint{1.662369in}{1.633611in}}%
\pgfpathlineto{\pgfqpoint{1.683611in}{1.657940in}}%
\pgfpathlineto{\pgfqpoint{1.704854in}{1.691623in}}%
\pgfpathlineto{\pgfqpoint{1.726096in}{1.677174in}}%
\pgfpathlineto{\pgfqpoint{1.768582in}{1.682869in}}%
\pgfpathlineto{\pgfqpoint{1.789824in}{1.693900in}}%
\pgfpathlineto{\pgfqpoint{1.811067in}{1.696790in}}%
\pgfpathlineto{\pgfqpoint{1.832309in}{1.692172in}}%
\pgfpathlineto{\pgfqpoint{1.853552in}{1.666500in}}%
\pgfpathlineto{\pgfqpoint{1.874794in}{1.718762in}}%
\pgfpathlineto{\pgfqpoint{1.896037in}{1.751191in}}%
\pgfpathlineto{\pgfqpoint{1.917280in}{1.730170in}}%
\pgfpathlineto{\pgfqpoint{1.938522in}{1.725408in}}%
\pgfpathlineto{\pgfqpoint{1.959765in}{1.731344in}}%
\pgfpathlineto{\pgfqpoint{1.981007in}{1.718997in}}%
\pgfpathlineto{\pgfqpoint{2.002250in}{1.737150in}}%
\pgfpathlineto{\pgfqpoint{2.023492in}{1.759843in}}%
\pgfpathlineto{\pgfqpoint{2.044735in}{1.767673in}}%
\pgfpathlineto{\pgfqpoint{2.065978in}{1.763392in}}%
\pgfpathlineto{\pgfqpoint{2.087220in}{1.785546in}}%
\pgfpathlineto{\pgfqpoint{2.108463in}{1.813206in}}%
\pgfpathlineto{\pgfqpoint{2.129705in}{1.777920in}}%
\pgfpathlineto{\pgfqpoint{2.150948in}{1.779289in}}%
\pgfpathlineto{\pgfqpoint{2.172190in}{1.760980in}}%
\pgfpathlineto{\pgfqpoint{2.193433in}{1.758635in}}%
\pgfpathlineto{\pgfqpoint{2.214676in}{1.792058in}}%
\pgfpathlineto{\pgfqpoint{2.235918in}{1.795515in}}%
\pgfpathlineto{\pgfqpoint{2.257161in}{1.802980in}}%
\pgfpathlineto{\pgfqpoint{2.278403in}{1.818084in}}%
\pgfpathlineto{\pgfqpoint{2.299646in}{1.869758in}}%
\pgfpathlineto{\pgfqpoint{2.342131in}{1.851390in}}%
\pgfpathlineto{\pgfqpoint{2.363374in}{1.858639in}}%
\pgfpathlineto{\pgfqpoint{2.384616in}{1.879749in}}%
\pgfpathlineto{\pgfqpoint{2.405859in}{1.873327in}}%
\pgfpathlineto{\pgfqpoint{2.427101in}{1.895517in}}%
\pgfpathlineto{\pgfqpoint{2.448344in}{1.904478in}}%
\pgfpathlineto{\pgfqpoint{2.469587in}{1.926919in}}%
\pgfpathlineto{\pgfqpoint{2.490829in}{1.964081in}}%
\pgfpathlineto{\pgfqpoint{2.512072in}{2.007963in}}%
\pgfpathlineto{\pgfqpoint{2.533314in}{2.023574in}}%
\pgfpathlineto{\pgfqpoint{2.554557in}{2.019791in}}%
\pgfpathlineto{\pgfqpoint{2.575799in}{2.040637in}}%
\pgfpathlineto{\pgfqpoint{2.597042in}{2.071399in}}%
\pgfpathlineto{\pgfqpoint{2.618285in}{2.052575in}}%
\pgfpathlineto{\pgfqpoint{2.639527in}{2.038499in}}%
\pgfpathlineto{\pgfqpoint{2.660770in}{2.083493in}}%
\pgfpathlineto{\pgfqpoint{2.682012in}{2.147971in}}%
\pgfpathlineto{\pgfqpoint{2.724497in}{2.158598in}}%
\pgfpathlineto{\pgfqpoint{2.745740in}{2.184708in}}%
\pgfpathlineto{\pgfqpoint{2.766983in}{2.187159in}}%
\pgfpathlineto{\pgfqpoint{2.788225in}{2.127290in}}%
\pgfpathlineto{\pgfqpoint{2.809468in}{2.109887in}}%
\pgfpathlineto{\pgfqpoint{2.830710in}{2.106405in}}%
\pgfpathlineto{\pgfqpoint{2.851953in}{2.154997in}}%
\pgfpathlineto{\pgfqpoint{2.873196in}{2.200484in}}%
\pgfpathlineto{\pgfqpoint{2.894438in}{2.232267in}}%
\pgfpathlineto{\pgfqpoint{2.915681in}{2.270564in}}%
\pgfpathlineto{\pgfqpoint{2.936923in}{2.285236in}}%
\pgfpathlineto{\pgfqpoint{2.958166in}{2.320698in}}%
\pgfpathlineto{\pgfqpoint{2.979408in}{2.313529in}}%
\pgfpathlineto{\pgfqpoint{3.000651in}{2.290273in}}%
\pgfpathlineto{\pgfqpoint{3.021894in}{2.261215in}}%
\pgfpathlineto{\pgfqpoint{3.043136in}{2.302341in}}%
\pgfpathlineto{\pgfqpoint{3.064379in}{2.330316in}}%
\pgfpathlineto{\pgfqpoint{3.085621in}{2.331876in}}%
\pgfpathlineto{\pgfqpoint{3.106864in}{2.395212in}}%
\pgfpathlineto{\pgfqpoint{3.128106in}{2.416580in}}%
\pgfpathlineto{\pgfqpoint{3.149349in}{2.404722in}}%
\pgfpathlineto{\pgfqpoint{3.170592in}{2.417179in}}%
\pgfpathlineto{\pgfqpoint{3.191834in}{2.423722in}}%
\pgfpathlineto{\pgfqpoint{3.213077in}{2.406734in}}%
\pgfpathlineto{\pgfqpoint{3.276804in}{2.322850in}}%
\pgfpathlineto{\pgfqpoint{3.298047in}{2.392813in}}%
\pgfpathlineto{\pgfqpoint{3.319290in}{2.370078in}}%
\pgfpathlineto{\pgfqpoint{3.340532in}{2.411418in}}%
\pgfpathlineto{\pgfqpoint{3.361775in}{2.413624in}}%
\pgfpathlineto{\pgfqpoint{3.383017in}{2.378444in}}%
\pgfpathlineto{\pgfqpoint{3.404260in}{2.417720in}}%
\pgfpathlineto{\pgfqpoint{3.425503in}{2.415968in}}%
\pgfpathlineto{\pgfqpoint{3.446745in}{2.427349in}}%
\pgfpathlineto{\pgfqpoint{3.467988in}{2.441731in}}%
\pgfpathlineto{\pgfqpoint{3.489230in}{2.462203in}}%
\pgfpathlineto{\pgfqpoint{3.510473in}{2.462313in}}%
\pgfpathlineto{\pgfqpoint{3.531715in}{2.510281in}}%
\pgfpathlineto{\pgfqpoint{3.552958in}{2.498609in}}%
\pgfpathlineto{\pgfqpoint{3.574201in}{2.496163in}}%
\pgfpathlineto{\pgfqpoint{3.595443in}{2.537927in}}%
\pgfpathlineto{\pgfqpoint{3.616686in}{2.559632in}}%
\pgfpathlineto{\pgfqpoint{3.659171in}{2.563778in}}%
\pgfpathlineto{\pgfqpoint{3.680413in}{2.545375in}}%
\pgfpathlineto{\pgfqpoint{3.701656in}{2.559561in}}%
\pgfpathlineto{\pgfqpoint{3.722899in}{2.545350in}}%
\pgfpathlineto{\pgfqpoint{3.744141in}{2.584775in}}%
\pgfpathlineto{\pgfqpoint{3.765384in}{2.569776in}}%
\pgfpathlineto{\pgfqpoint{3.786626in}{2.603543in}}%
\pgfpathlineto{\pgfqpoint{3.807869in}{2.582415in}}%
\pgfpathlineto{\pgfqpoint{3.829111in}{2.588245in}}%
\pgfpathlineto{\pgfqpoint{3.850354in}{2.582382in}}%
\pgfpathlineto{\pgfqpoint{3.871597in}{2.605277in}}%
\pgfpathlineto{\pgfqpoint{3.892839in}{2.604111in}}%
\pgfpathlineto{\pgfqpoint{3.914082in}{2.656257in}}%
\pgfpathlineto{\pgfqpoint{3.935324in}{2.589569in}}%
\pgfpathlineto{\pgfqpoint{3.956567in}{2.606173in}}%
\pgfpathlineto{\pgfqpoint{3.977810in}{2.527329in}}%
\pgfpathlineto{\pgfqpoint{3.999052in}{2.526713in}}%
\pgfpathlineto{\pgfqpoint{4.020295in}{2.521224in}}%
\pgfpathlineto{\pgfqpoint{4.041537in}{2.555152in}}%
\pgfpathlineto{\pgfqpoint{4.062780in}{2.529126in}}%
\pgfpathlineto{\pgfqpoint{4.084022in}{2.536193in}}%
\pgfpathlineto{\pgfqpoint{4.105265in}{2.555869in}}%
\pgfpathlineto{\pgfqpoint{4.126508in}{2.533970in}}%
\pgfpathlineto{\pgfqpoint{4.147750in}{2.565273in}}%
\pgfpathlineto{\pgfqpoint{4.168993in}{2.593611in}}%
\pgfpathlineto{\pgfqpoint{4.190235in}{2.543549in}}%
\pgfpathlineto{\pgfqpoint{4.211478in}{2.572793in}}%
\pgfpathlineto{\pgfqpoint{4.232720in}{2.543293in}}%
\pgfpathlineto{\pgfqpoint{4.253963in}{2.587419in}}%
\pgfpathlineto{\pgfqpoint{4.275206in}{2.623714in}}%
\pgfpathlineto{\pgfqpoint{4.296448in}{2.581724in}}%
\pgfpathlineto{\pgfqpoint{4.317691in}{2.591426in}}%
\pgfpathlineto{\pgfqpoint{4.338933in}{2.611578in}}%
\pgfpathlineto{\pgfqpoint{4.360176in}{2.613635in}}%
\pgfpathlineto{\pgfqpoint{4.381418in}{2.639281in}}%
\pgfpathlineto{\pgfqpoint{4.402661in}{2.669984in}}%
\pgfpathlineto{\pgfqpoint{4.423904in}{2.681169in}}%
\pgfpathlineto{\pgfqpoint{4.445146in}{2.736707in}}%
\pgfpathlineto{\pgfqpoint{4.466389in}{2.799576in}}%
\pgfpathlineto{\pgfqpoint{4.487631in}{2.806102in}}%
\pgfpathlineto{\pgfqpoint{4.508874in}{2.827771in}}%
\pgfpathlineto{\pgfqpoint{4.530116in}{2.821744in}}%
\pgfpathlineto{\pgfqpoint{4.572602in}{2.852588in}}%
\pgfpathlineto{\pgfqpoint{4.593844in}{2.921446in}}%
\pgfpathlineto{\pgfqpoint{4.615087in}{2.929043in}}%
\pgfpathlineto{\pgfqpoint{4.636329in}{2.945454in}}%
\pgfpathlineto{\pgfqpoint{4.657572in}{3.005226in}}%
\pgfpathlineto{\pgfqpoint{4.678815in}{3.019698in}}%
\pgfpathlineto{\pgfqpoint{4.700057in}{3.038566in}}%
\pgfpathlineto{\pgfqpoint{4.721300in}{3.022502in}}%
\pgfpathlineto{\pgfqpoint{4.742542in}{3.043986in}}%
\pgfpathlineto{\pgfqpoint{4.763785in}{3.048267in}}%
\pgfpathlineto{\pgfqpoint{4.785027in}{2.997346in}}%
\pgfpathlineto{\pgfqpoint{4.806270in}{3.064027in}}%
\pgfpathlineto{\pgfqpoint{4.827513in}{3.039723in}}%
\pgfpathlineto{\pgfqpoint{4.848755in}{3.025490in}}%
\pgfpathlineto{\pgfqpoint{4.891240in}{3.087355in}}%
\pgfpathlineto{\pgfqpoint{4.912483in}{3.100561in}}%
\pgfpathlineto{\pgfqpoint{4.933725in}{3.175479in}}%
\pgfpathlineto{\pgfqpoint{4.954968in}{3.184863in}}%
\pgfpathlineto{\pgfqpoint{4.976211in}{3.210371in}}%
\pgfpathlineto{\pgfqpoint{4.997453in}{3.246679in}}%
\pgfpathlineto{\pgfqpoint{5.018696in}{3.287869in}}%
\pgfpathlineto{\pgfqpoint{5.039938in}{3.273048in}}%
\pgfpathlineto{\pgfqpoint{5.082423in}{3.326081in}}%
\pgfpathlineto{\pgfqpoint{5.103666in}{3.317222in}}%
\pgfpathlineto{\pgfqpoint{5.124909in}{3.331722in}}%
\pgfpathlineto{\pgfqpoint{5.146151in}{3.360215in}}%
\pgfpathlineto{\pgfqpoint{5.167394in}{3.380000in}}%
\pgfpathlineto{\pgfqpoint{5.188636in}{3.371548in}}%
\pgfpathlineto{\pgfqpoint{5.188636in}{3.371548in}}%
\pgfusepath{stroke}%
\end{pgfscope}%
\begin{pgfscope}%
\pgfpathrectangle{\pgfqpoint{0.750000in}{0.440000in}}{\pgfqpoint{4.650000in}{3.080000in}}%
\pgfusepath{clip}%
\pgfsetroundcap%
\pgfsetroundjoin%
\pgfsetlinewidth{1.003750pt}%
\definecolor{currentstroke}{rgb}{0.215686,0.494118,0.721569}%
\pgfsetstrokecolor{currentstroke}%
\pgfsetdash{}{0pt}%
\pgfpathmoveto{\pgfqpoint{0.961364in}{0.582138in}}%
\pgfpathlineto{\pgfqpoint{1.407458in}{0.586158in}}%
\pgfpathlineto{\pgfqpoint{2.002250in}{0.589067in}}%
\pgfpathlineto{\pgfqpoint{3.064379in}{0.591540in}}%
\pgfpathlineto{\pgfqpoint{5.188636in}{0.595323in}}%
\pgfpathlineto{\pgfqpoint{5.188636in}{0.595323in}}%
\pgfusepath{stroke}%
\end{pgfscope}%
\begin{pgfscope}%
\pgfpathrectangle{\pgfqpoint{0.750000in}{0.440000in}}{\pgfqpoint{4.650000in}{3.080000in}}%
\pgfusepath{clip}%
\pgfsetroundcap%
\pgfsetroundjoin%
\pgfsetlinewidth{1.003750pt}%
\definecolor{currentstroke}{rgb}{1.000000,0.498039,0.000000}%
\pgfsetstrokecolor{currentstroke}%
\pgfsetdash{}{0pt}%
\pgfpathmoveto{\pgfqpoint{0.961364in}{0.658988in}}%
\pgfpathlineto{\pgfqpoint{1.173789in}{0.652853in}}%
\pgfpathlineto{\pgfqpoint{1.407458in}{0.648311in}}%
\pgfpathlineto{\pgfqpoint{1.683611in}{0.645147in}}%
\pgfpathlineto{\pgfqpoint{2.023492in}{0.643521in}}%
\pgfpathlineto{\pgfqpoint{2.469587in}{0.643788in}}%
\pgfpathlineto{\pgfqpoint{3.064379in}{0.646548in}}%
\pgfpathlineto{\pgfqpoint{3.914082in}{0.652886in}}%
\pgfpathlineto{\pgfqpoint{5.188636in}{0.664769in}}%
\pgfpathlineto{\pgfqpoint{5.188636in}{0.664769in}}%
\pgfusepath{stroke}%
\end{pgfscope}%
\begin{pgfscope}%
\pgfpathrectangle{\pgfqpoint{0.750000in}{0.440000in}}{\pgfqpoint{4.650000in}{3.080000in}}%
\pgfusepath{clip}%
\pgfsetroundcap%
\pgfsetroundjoin%
\pgfsetlinewidth{1.003750pt}%
\definecolor{currentstroke}{rgb}{0.301961,0.686275,0.290196}%
\pgfsetstrokecolor{currentstroke}%
\pgfsetdash{}{0pt}%
\pgfpathmoveto{\pgfqpoint{0.961364in}{0.735837in}}%
\pgfpathlineto{\pgfqpoint{1.110062in}{0.730519in}}%
\pgfpathlineto{\pgfqpoint{1.280002in}{0.726588in}}%
\pgfpathlineto{\pgfqpoint{1.492428in}{0.724005in}}%
\pgfpathlineto{\pgfqpoint{1.747339in}{0.723302in}}%
\pgfpathlineto{\pgfqpoint{2.044735in}{0.724739in}}%
\pgfpathlineto{\pgfqpoint{2.427101in}{0.728956in}}%
\pgfpathlineto{\pgfqpoint{2.894438in}{0.736443in}}%
\pgfpathlineto{\pgfqpoint{3.489230in}{0.748262in}}%
\pgfpathlineto{\pgfqpoint{4.275206in}{0.766199in}}%
\pgfpathlineto{\pgfqpoint{5.188636in}{0.788936in}}%
\pgfpathlineto{\pgfqpoint{5.188636in}{0.788936in}}%
\pgfusepath{stroke}%
\end{pgfscope}%
\begin{pgfscope}%
\pgfpathrectangle{\pgfqpoint{0.750000in}{0.440000in}}{\pgfqpoint{4.650000in}{3.080000in}}%
\pgfusepath{clip}%
\pgfsetroundcap%
\pgfsetroundjoin%
\pgfsetlinewidth{1.003750pt}%
\definecolor{currentstroke}{rgb}{0.968627,0.505882,0.749020}%
\pgfsetstrokecolor{currentstroke}%
\pgfsetdash{}{0pt}%
\pgfpathmoveto{\pgfqpoint{0.961364in}{0.812687in}}%
\pgfpathlineto{\pgfqpoint{1.088819in}{0.810372in}}%
\pgfpathlineto{\pgfqpoint{1.237517in}{0.809864in}}%
\pgfpathlineto{\pgfqpoint{1.428700in}{0.811563in}}%
\pgfpathlineto{\pgfqpoint{1.662369in}{0.816022in}}%
\pgfpathlineto{\pgfqpoint{1.959765in}{0.824155in}}%
\pgfpathlineto{\pgfqpoint{2.320889in}{0.836418in}}%
\pgfpathlineto{\pgfqpoint{2.766983in}{0.853882in}}%
\pgfpathlineto{\pgfqpoint{3.340532in}{0.878719in}}%
\pgfpathlineto{\pgfqpoint{4.062780in}{0.912360in}}%
\pgfpathlineto{\pgfqpoint{4.997453in}{0.958243in}}%
\pgfpathlineto{\pgfqpoint{5.188636in}{0.967850in}}%
\pgfpathlineto{\pgfqpoint{5.188636in}{0.967850in}}%
\pgfusepath{stroke}%
\end{pgfscope}%
\begin{pgfscope}%
\pgfpathrectangle{\pgfqpoint{0.750000in}{0.440000in}}{\pgfqpoint{4.650000in}{3.080000in}}%
\pgfusepath{clip}%
\pgfsetroundcap%
\pgfsetroundjoin%
\pgfsetlinewidth{1.003750pt}%
\definecolor{currentstroke}{rgb}{0.650980,0.337255,0.156863}%
\pgfsetstrokecolor{currentstroke}%
\pgfsetdash{}{0pt}%
\pgfpathmoveto{\pgfqpoint{0.961364in}{0.889536in}}%
\pgfpathlineto{\pgfqpoint{1.067577in}{0.891843in}}%
\pgfpathlineto{\pgfqpoint{1.216275in}{0.897472in}}%
\pgfpathlineto{\pgfqpoint{1.407458in}{0.907181in}}%
\pgfpathlineto{\pgfqpoint{1.662369in}{0.922624in}}%
\pgfpathlineto{\pgfqpoint{1.981007in}{0.944254in}}%
\pgfpathlineto{\pgfqpoint{2.405859in}{0.975443in}}%
\pgfpathlineto{\pgfqpoint{2.958166in}{1.018375in}}%
\pgfpathlineto{\pgfqpoint{3.659171in}{1.075193in}}%
\pgfpathlineto{\pgfqpoint{4.572602in}{1.151570in}}%
\pgfpathlineto{\pgfqpoint{5.188636in}{1.204043in}}%
\pgfpathlineto{\pgfqpoint{5.188636in}{1.204043in}}%
\pgfusepath{stroke}%
\end{pgfscope}%
\begin{pgfscope}%
\pgfpathrectangle{\pgfqpoint{0.750000in}{0.440000in}}{\pgfqpoint{4.650000in}{3.080000in}}%
\pgfusepath{clip}%
\pgfsetroundcap%
\pgfsetroundjoin%
\pgfsetlinewidth{1.003750pt}%
\definecolor{currentstroke}{rgb}{0.596078,0.305882,0.639216}%
\pgfsetstrokecolor{currentstroke}%
\pgfsetdash{}{0pt}%
\pgfpathmoveto{\pgfqpoint{0.961364in}{0.966386in}}%
\pgfpathlineto{\pgfqpoint{1.067577in}{0.975808in}}%
\pgfpathlineto{\pgfqpoint{1.216275in}{0.991382in}}%
\pgfpathlineto{\pgfqpoint{1.449943in}{1.018400in}}%
\pgfpathlineto{\pgfqpoint{1.811067in}{1.062677in}}%
\pgfpathlineto{\pgfqpoint{2.384616in}{1.135480in}}%
\pgfpathlineto{\pgfqpoint{3.234319in}{1.245715in}}%
\pgfpathlineto{\pgfqpoint{4.423904in}{1.402383in}}%
\pgfpathlineto{\pgfqpoint{5.188636in}{1.503959in}}%
\pgfpathlineto{\pgfqpoint{5.188636in}{1.503959in}}%
\pgfusepath{stroke}%
\end{pgfscope}%
\begin{pgfscope}%
\pgfpathrectangle{\pgfqpoint{0.750000in}{0.440000in}}{\pgfqpoint{4.650000in}{3.080000in}}%
\pgfusepath{clip}%
\pgfsetroundcap%
\pgfsetroundjoin%
\pgfsetlinewidth{1.003750pt}%
\definecolor{currentstroke}{rgb}{0.600000,0.600000,0.600000}%
\pgfsetstrokecolor{currentstroke}%
\pgfsetdash{}{0pt}%
\pgfpathmoveto{\pgfqpoint{0.961364in}{1.043235in}}%
\pgfpathlineto{\pgfqpoint{1.067577in}{1.063362in}}%
\pgfpathlineto{\pgfqpoint{1.322487in}{1.114852in}}%
\pgfpathlineto{\pgfqpoint{2.044735in}{1.261037in}}%
\pgfpathlineto{\pgfqpoint{2.830710in}{1.417438in}}%
\pgfpathlineto{\pgfqpoint{4.041537in}{1.655864in}}%
\pgfpathlineto{\pgfqpoint{5.188636in}{1.880648in}}%
\pgfpathlineto{\pgfqpoint{5.188636in}{1.880648in}}%
\pgfusepath{stroke}%
\end{pgfscope}%
\begin{pgfscope}%
\pgfpathrectangle{\pgfqpoint{0.750000in}{0.440000in}}{\pgfqpoint{4.650000in}{3.080000in}}%
\pgfusepath{clip}%
\pgfsetroundcap%
\pgfsetroundjoin%
\pgfsetlinewidth{1.003750pt}%
\definecolor{currentstroke}{rgb}{0.894118,0.101961,0.109804}%
\pgfsetstrokecolor{currentstroke}%
\pgfsetdash{}{0pt}%
\pgfpathmoveto{\pgfqpoint{0.961364in}{1.120085in}}%
\pgfpathlineto{\pgfqpoint{1.301245in}{1.234787in}}%
\pgfpathlineto{\pgfqpoint{1.513671in}{1.302794in}}%
\pgfpathlineto{\pgfqpoint{1.768582in}{1.381788in}}%
\pgfpathlineto{\pgfqpoint{2.065978in}{1.471542in}}%
\pgfpathlineto{\pgfqpoint{2.448344in}{1.584434in}}%
\pgfpathlineto{\pgfqpoint{2.915681in}{1.719924in}}%
\pgfpathlineto{\pgfqpoint{3.489230in}{1.883818in}}%
\pgfpathlineto{\pgfqpoint{4.232720in}{2.093829in}}%
\pgfpathlineto{\pgfqpoint{5.188636in}{2.361389in}}%
\pgfpathlineto{\pgfqpoint{5.188636in}{2.361389in}}%
\pgfusepath{stroke}%
\end{pgfscope}%
\begin{pgfscope}%
\pgfpathrectangle{\pgfqpoint{0.750000in}{0.440000in}}{\pgfqpoint{4.650000in}{3.080000in}}%
\pgfusepath{clip}%
\pgfsetroundcap%
\pgfsetroundjoin%
\pgfsetlinewidth{1.003750pt}%
\definecolor{currentstroke}{rgb}{0.870588,0.870588,0.000000}%
\pgfsetstrokecolor{currentstroke}%
\pgfsetdash{}{0pt}%
\pgfpathmoveto{\pgfqpoint{0.961364in}{1.196934in}}%
\pgfpathlineto{\pgfqpoint{1.003849in}{1.224070in}}%
\pgfpathlineto{\pgfqpoint{1.067577in}{1.261956in}}%
\pgfpathlineto{\pgfqpoint{1.152547in}{1.309207in}}%
\pgfpathlineto{\pgfqpoint{1.258760in}{1.364949in}}%
\pgfpathlineto{\pgfqpoint{1.386215in}{1.428652in}}%
\pgfpathlineto{\pgfqpoint{1.534913in}{1.499999in}}%
\pgfpathlineto{\pgfqpoint{1.704854in}{1.578813in}}%
\pgfpathlineto{\pgfqpoint{1.896037in}{1.665000in}}%
\pgfpathlineto{\pgfqpoint{2.129705in}{1.767772in}}%
\pgfpathlineto{\pgfqpoint{2.405859in}{1.886595in}}%
\pgfpathlineto{\pgfqpoint{2.724497in}{2.021136in}}%
\pgfpathlineto{\pgfqpoint{3.106864in}{2.179974in}}%
\pgfpathlineto{\pgfqpoint{3.552958in}{2.362695in}}%
\pgfpathlineto{\pgfqpoint{4.062780in}{2.569070in}}%
\pgfpathlineto{\pgfqpoint{4.678815in}{2.815950in}}%
\pgfpathlineto{\pgfqpoint{5.188636in}{3.018756in}}%
\pgfpathlineto{\pgfqpoint{5.188636in}{3.018756in}}%
\pgfusepath{stroke}%
\end{pgfscope}%
\begin{pgfscope}%
\pgfsetrectcap%
\pgfsetmiterjoin%
\pgfsetlinewidth{0.000000pt}%
\definecolor{currentstroke}{rgb}{1.000000,1.000000,1.000000}%
\pgfsetstrokecolor{currentstroke}%
\pgfsetdash{}{0pt}%
\pgfpathmoveto{\pgfqpoint{0.750000in}{0.440000in}}%
\pgfpathlineto{\pgfqpoint{0.750000in}{3.520000in}}%
\pgfusepath{}%
\end{pgfscope}%
\begin{pgfscope}%
\pgfsetrectcap%
\pgfsetmiterjoin%
\pgfsetlinewidth{0.000000pt}%
\definecolor{currentstroke}{rgb}{1.000000,1.000000,1.000000}%
\pgfsetstrokecolor{currentstroke}%
\pgfsetdash{}{0pt}%
\pgfpathmoveto{\pgfqpoint{5.400000in}{0.440000in}}%
\pgfpathlineto{\pgfqpoint{5.400000in}{3.520000in}}%
\pgfusepath{}%
\end{pgfscope}%
\begin{pgfscope}%
\pgfsetrectcap%
\pgfsetmiterjoin%
\pgfsetlinewidth{0.000000pt}%
\definecolor{currentstroke}{rgb}{1.000000,1.000000,1.000000}%
\pgfsetstrokecolor{currentstroke}%
\pgfsetdash{}{0pt}%
\pgfpathmoveto{\pgfqpoint{0.750000in}{0.440000in}}%
\pgfpathlineto{\pgfqpoint{5.400000in}{0.440000in}}%
\pgfusepath{}%
\end{pgfscope}%
\begin{pgfscope}%
\pgfsetrectcap%
\pgfsetmiterjoin%
\pgfsetlinewidth{0.000000pt}%
\definecolor{currentstroke}{rgb}{1.000000,1.000000,1.000000}%
\pgfsetstrokecolor{currentstroke}%
\pgfsetdash{}{0pt}%
\pgfpathmoveto{\pgfqpoint{0.750000in}{3.520000in}}%
\pgfpathlineto{\pgfqpoint{5.400000in}{3.520000in}}%
\pgfusepath{}%
\end{pgfscope}%
\end{pgfpicture}%
\makeatother%
\endgroup%

    \caption{Superposition of path simulations of a Wishart process and its finite dimensional part.\label{fig:wishart_comparison}}
\end{figure}

For the Jacobi process path simulations in Figure \ref{fig:jacobi}, we notice a more stable behavior. In Chapter \ref{ch:determinist} we will see that this process has an stationary distribution and thus it is expected that, given any initial condition, the process converges to it. Notice that in this simulation the convergence appears to have rather quickly, as the points appear to be evenly separated at time $t=0.05$.

\begin{figure} 
    \includegraphics[width=6in]{img/jacobi.pdf}
    \caption{Path simulation for a Jacobi process.\label{fig:jacobi}}
\end{figure}