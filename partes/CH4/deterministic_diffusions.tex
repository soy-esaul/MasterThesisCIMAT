\section{Deterministic eigenvalue processes for matrix-valued diffusions}

Now that we have shown the construction of a matrix-valued process whose eigenvalue is the deterministic Dyson Brownian motion, we generaliza the result to get processes with a deterministic spectrum that can follow the dynamics of any eigenvalue process with the form \eqref{eq:gen_dyson}.

% \begin{theorem}

%     Let $Z$ be a process with covariation $\d Z_{ij}\d Z_{kl} = (\delta_{ik}\delta_{jl} + \delta_{il}\delta_{jk} - 2\delta_{ij}\delta_{kl}\delta_{ik} )\d t$ and no finite variation part, which means $Z$ is a symmetric matrix with independent Brownian motions in its entries, except for the diagonal, where $Z_{ii} = 0$. Let $X$ be a matrix valued process such that  $X = \trans H \Lambda H$ and it satisfies the stochastic differential equation
    
    
%     \begin{equation}
%         \trans{H}\d X(t)H = g(X(t)) \d Z(t) h(X(t)) + h(X(t)) \d \trans{Z(t)} g(X(t)) + b(X(t))\d t, \label{eq:deterministic_matrix_diffusion}
%     \end{equation}

%     Then the eigenvalue process $\Lambda$ satisfies

%     \begin{equation}
%         \d \lambda_i = \biggl( b(\lambda_i) + \sum_{k\neq i} \frac{G(\lambda_i,\lambda_k)}{\lambda_i - \lambda_k} \biggr)\d t.
%     \end{equation}

% \end{theorem}

\begin{theorem} \label{thm:deterministic_diffusion}
    Let $B = (B(t), t\ge 0)$ be a Brownian motion in $\M_{p,p}(\R)$ and $Y(t) = QM\trans{Q}$ be a symmetric $p\times p$ matrix-valued stochastic process satisfying the stochastic differential equation

    \begin{equation}
        \d Y(t) = g(Y(t)) \d B(t) h(Y(t)) + h(Y(t)) \d \trans{B(t)} g(Y(t)) + b(Y(t))\d t, \label{eq:matrix_diffusion}
    \end{equation}

    where $g,h,b$ are real functions acting spectrally, and $Y(0)$ is a symmetric $p\times p$ matrix with $p$ different eigenvalues.

    Let $G(x,y) = g^2(x)h^2(y) + g^2(y)h^2(x)$, $\tau$ be defined as in \eqref{eq:collision_time}, and take a process $X = (X(t), t\ge 0)$ with diagonalization $X = H \Lambda \trans{H}$ such that $\trans H (\d \Lambda) H$ has the same off-diagonal entries as $\trans Q ( \d M) \circ Q$ and has diagonal entries equal to zero.
    
    Then, for $t < \tau$ the eigenvalue process $\Lambda(t)$ verifies the following system of stochastic differential equations:

    \begin{equation}
        \d \lambda_i = \biggl( b(\lambda_i) + \sum_{k\neq i} \frac{G(\lambda_i,\lambda_k)}{\lambda_i - \lambda_k} \biggr)\d t.
    \end{equation}
\end{theorem}


\begin{proof}
    We define again $L$ to be the stochastic logarithm of $H$, $L \coloneqq \trans H \circ \d H$ and using the same techniques as in Theorem \ref{thm:diffusion_real} we have that,

    \begin{equation*}
        \d \Lambda = \trans H \circ(\partial X) \circ H - (\partial L)\circ \Lambda + \Lambda \circ \partial L.
    \end{equation*}

    Using that $\Lambda \circ \partial L -  (\partial L)\circ \Lambda $ has zero diagonal, we get that $\d lambda_i = (\trans H \circ(\partial X) \circ H)_{ii}$ and by hypothesis, we know that the martingale part of this diagonal is zero. 

    Define $\d N \coloneqq \trans H \circ(\partial X) \circ H$. For $i\neq j$ we have that $\d L_{ij} = \d N_{ij}/(\lambda_j - \lambda_i)$.

    Finally, we compute the finite variation $\d F$ part of $\d N$,

    \begin{align*}
        \d F &= \trans H b(X) H \d t + \frac12 (\d \trans H \d X H + \trans H \d X H), \\ 
        &= b(\Lambda) \d t + \frac12( \trans{(\d N\d A)} + \d N \d A ).
    \end{align*}

    For $\d N \d A$ we find

    \begin{align*}
        (\d N \d A)_{ij} &= \sum_{k\neq j} \d N_{ik}\d A_{kj} = \sum_{k\neq j} \frac{\d N_{ik}\d N_{kj}}{\lambda_j - \lambda_k}.
    \end{align*}

    Now we use that the martingale part of  $\d N$ has the same entries as $\trans{Q} M Q$ and by the results in Theorem \ref{thm:diffusion_real} we know that 

    \begin{equation*}
        (\trans{Q} M Q)_{ik}(\trans{Q} M Q)_{kj} = \delta_{ij}G(\lambda_i,\lambda_k)\d t,
    \end{equation*}

    \noindent so substituting the last result we get

    \begin{equation*}
        \d \lambda_i = \d F_{ii} = b(\lambda_i)\d t + \sum_{k\neq j} \frac{G(\lambda_i,\lambda_k)\d t}{\lambda_j - \lambda_k},
    \end{equation*}

    \noindent which is the desired result.



\end{proof}


These results can be particularized for any matrix-valued diffusions. Especially, we are interested in the Wishart and Jacobi processes. We give the proofs for these as a corollary to last Theorem.

\subsection{Wishart process}

\begin{corollary}
    Let $Y = (Y(t), t \ge 0)$ be an $n\times n$ Wishart process with parameter $m$ and diagonalization $Y = QM\trans{Q}$. Let $X$ be an $n\times n$ self adjoint matrix process with diagonalization $X = H \Lambda \trans{H}$, such that the off-diagonal part of $\trans{H} \d X H $ and $\overset{d}{=} \trans{Q} \d Y Q$ coincide, and $(\trans{H} \d X H)_{ii} = 0$ for every $i \in [n]$. Then the eigenvalues of $X$, $\lambda_1 \ge \lambda_2 \ge \cdots \ge \lambda_n$ satisfy the following system of stochastic differential equations

    \begin{equation} \label{eq:deterministic_wishart}
        \d \lambda_i = \left( m +\sum_{k\neq i} \frac{ \abs{\lambda_i} + \abs{\lambda_k} }{\lambda_i - \lambda_k}\right)\d t.
    \end{equation}
\end{corollary}

\begin{proof}
    
\end{proof}


\subsection{Jacobi process}

\begin{corollary}
    Let $Y = (Y(t), t \ge 0)$ be an $n\times n$ matrix Jacobi process with parameters $n_1,n_2$ and diagonalization $Y = QM\trans{Q}$. Let $X$ be an $n\times n$ self adjoint matrix process with diagonalization $X = H \Lambda \trans{H}$, such that the off-diagonal part of $\trans{H} \d X H $ and $\overset{d}{=} \trans{Q} \d Y Q$ coincide, and $(\trans{H} \d X H)_{ii} = 0$ for every $i \in [n]$. Then the eigenvalues of $X$, $\lambda_1 \ge \lambda_2 \ge \cdots \ge \lambda_n$ satisfy the following system of stochastic differential equations

    \begin{equation} \label{eq:deterministic_jacobi}
        \d \lambda_i = \left( n_2 - (n_1 + n_2)\lambda_i + \sum_{k\neq i} \frac{\lambda_k(1-\lambda_i) + \lambda_i(1-\lambda_k)}{\lambda_i - \lambda_k} \right)\d t.
    \end{equation}

\end{corollary}

\section{Dynamical behavior of the deterministic eigenvalue processes}

\begin{lemma} \label{lemma:separation}
    Let $\lambda_1,\dots,\lambda_n$ be a system of $n$ functions moving according to \eqref{eq:deterministic_dyson} and $\lambda_i(0) > \lambda_j(0)$ and there is no $\lambda_k$ such that $\lambda_i(0) > \lambda_k(0) > \lambda_j(0)$. Then, $\lambda_i$ and $\lambda_k$ repel each other if and only if 

    \begin{equation} \label{eq:separation_condition}
        \frac{2}{(\lambda_i - \lambda_j)^2} > \sum_{k\neq i,j} \frac{1}{(\lambda_i - \lambda_k)(\lambda_j - \lambda_k)}
    \end{equation}
\end{lemma}

\begin{proof}

    The distance between the particles grows if and only if its derivative its positive, so 

    \begin{align*}
        \frac{\d }{\d t} (\lambda_i - \lambda_j) &= \sum_{k\neq i} \frac{1}{\lambda_i - \lambda_k} - \sum_{k\neq j} \frac{1}{\lambda_j - \lambda_k} = \sum_{k\neq i,j} \left( \frac{1}{\lambda_i - \lambda_k} - \frac{1}{\lambda_j - \lambda_k} \right) + \frac{2}{\lambda_i - \lambda_j},\\
        &= \sum_{k \neq i, j} \frac{\lambda_j - \lambda_k - \lambda_i + \lambda_k}{(\lambda_i- \lambda_k)(\lambda_j - \lambda_k)} + \frac{2}{\lambda_i - \lambda_j},= \frac{2}{(\lambda_i - \lambda_j)^2} - \sum_{k\neq i,j} \frac{1}{(\lambda_i - \lambda_k)(\lambda_j - \lambda_k)}.
    \end{align*}

    Comparing to zero, we get the desired result.
\end{proof}



\begin{theorem}
    If a system of functions satisfies \eqref{eq:deterministic_dyson}, then there ar no collisions.
\end{theorem}

\begin{proof}
    Write condition \eqref{eq:separation_condition} as

    \begin{equation*}
        \frac{2}{(\lambda_i - \lambda_j)^2} > \sum_{k\neq i,j} \frac{1}{(\lambda_i - \lambda_j + \lambda_j -\lambda_k)(\lambda_j - \lambda_k)},
    \end{equation*}

    \noindent and take $\lambda_i \to \lambda_j$. As $\lambda_i$ approaches $\lambda_k$, the left-hand side grows to $\infty$, while the right-hand side converges to the finite quantity

    \begin{equation*}
        \sum_{k\neq i,j} \frac{1}{(\lambda_j - \lambda_k)^2}.
    \end{equation*}

    Thus, before the collision occurs, the distance between $\lambda_i$ and $\lambda_j$ starts to grow.
\end{proof}

\begin{theorem} \label{thm:hermite_minimal_grows}
    Let $(\lambda_1, \lambda_2, \dots, \lambda_n)$ be a system of functions moving according to \eqref{eq:deterministic_dyson} and let $\lambda_i,\lambda_j$ be such that for a given $t_0$ it is satisfied $\lambda_i(t_0) > \lambda_j(t_0)$ and 
    
    \begin{equation*}
        \lambda_i(t_0) - \lambda_j(t_0) = \min_{k,l \in [n]} \abs{ \lambda_k(t_0) - \lambda_l(t_0)}.
    \end{equation*}

    Then, $\lambda_i$ and $\lambda_k$ repel. 
\end{theorem}

\begin{proof}
    Define the quotients $c_{ik}, c_{jk}$ as 

    \begin{equation*}
        c_{ik} = \frac{\lambda_i - \lambda_k}{\lambda_i - \lambda_j}, \qquad c_{jk} = \frac{\lambda_j - \lambda_k}{\lambda_i - \lambda_j}.
    \end{equation*}

    Notice that the condition that the distance between $\lambda_i$ and $\lambda_j$ is minimal means that there is no $\lambda_k$ between them and therefore for a fixed $k_0$, $c_{ik_0}$ and $c_{jk_0}$ have the same sign. We can write the right-hand side of \eqref{eq:separation_condition} as

    \begin{align*}
        \sum_{k\neq i,j} \frac{1}{(\lambda_i - \lambda_k)(\lambda_j - \lambda_k)} &= \sum_{k\neq i,j} \frac{1}{ c_{ik}(\lambda_i - \lambda_j) c_{jk}(\lambda_i - \lambda_j)} = \sum_{k\neq i,j} \frac{ (c_{ik}c_{jk})^{-1} }{(\lambda_i-\lambda_j)^2}.
    \end{align*}

    With this, condition \eqref{eq:separation_condition} can be written as

    \begin{align}
        \frac{2}{(\lambda_i - \lambda_j)^2} &> \sum_{k\neq i,j} \frac{ (c_{ik}c_{jk})^{-1} }{(\lambda_i-\lambda_j)^2},\\
        \Leftrightarrow 2 &> \sum_{k\neq i,j} \frac{1}{c_{ik}c_{jk}}. \label{eq:simplified_condition}
    \end{align}

    We know that the distance between $\lambda_i$ and $\lambda_j$ is minimal, in the worst-case scenario, all of the other distances are the same as $\lambda_i - \lambda_j$, so $\abs{c_{ik}},\abs{c_{ij}}>1$. Notice that if $\lambda_{i-1}$ is the function located immediately bellow $\lambda_i$, this would mean that in the worst case scenario $c_{i,i-1}=-1$ and $c_{j,i-1} = -2$. Similarly, for $\lambda_{i-2}$ we would have $c_{i,i-1} = -2, c_{j,i-2} =-3$. In general, $c_{i,i-l} = -l, c_{j,i-l} = -(l+1)$. Analogously, if $\lambda_{j+l}$ is the function located $l$ positions below $j$, then $c_{j,j+l} = l, c_{i,j+l} = l+1$.

    Now, for the arrangement of the functions, we have two extreme cases. If $\lambda_i$ and $\lambda_j$ are the functions in one of the extremes (i.e. the two biggest or two smallest ones), this would mean that the right-hand side of \eqref{eq:simplified_condition} can be written as

    \begin{align*}
        \sum_{k\neq i,j} \frac{1}{c_{ik}c_{jk}} &= \sum_{k=1}^{n-2} \frac{1}{k(k+1)} = \sum_{k=1}^{n-2} \frac{k+1 - k}{k(k+1)} = \sum_{k=1}^{n-2} \left( \frac{1}{k} - \frac{1}{k+1} \right),\\
        &= 1 - \frac{1}{n-1} = \frac{n-2}{n-1} < 2.
    \end{align*}

    So in this case, applying Lemma \ref{lemma:separation}, $\lambda_i,\lambda_j$ would repel each other. The other extreme case is when exactly half of the functions are located at each side of $\lambda_i$ and $\lambda_j$. Let us suppose first that $n$ is even, in this case we would have,

    \begin{align*}
        \sum_{k\neq i,j} \frac{1}{c_{ik}c_{jk}} &= 2 \sum_{k=1}^{\frac{n-2}{2}} \frac{1}{k(k+1)} = 2\left(1 - \frac{2}{n}\right) = 2 - \frac4n < 2.
    \end{align*}

    There is also separation in this case. Finally, if $n$ is even, we considr the previos sum for $\lfloor n/2 \rfloor$ and sum the term corresponding to the remaining function.

    \begin{equation*}
        \sum_{k\neq i,j} \frac{1}{c_{ik}c_{jk}} = 2 - \frac{4}{n} + \frac1{(\lfloor n/2\rfloor+1)(\lfloor n/2\rfloor +2)} < 2.
    \end{equation*}

    So, even in the worst-case scenario, criterion \eqref{eq:separation_condition} is satisfied, and we conclude that the minimal distance grows for every initial condition of the system.
\end{proof}

\begin{corollary}
    In a system governed by~\eqref{eq:deterministic_dyson}, for $n\ge3$ not all of the functions separate for every given initial condition. For $n =2$, the functions always repel each other.
\end{corollary}

\begin{proof}
    We prove the first part by providing a counterexample. Suppose that the separation between $\lambda_i$ and $\lambda_j$ is 1, while all of the other separations are $0.1$. The same computations that the proof of in Theorem \ref{thm:hermite_minimal_grows}, but with a re-scaling of the $c_{ik}c_{jk}$ lead to

    \begin{align*}
        \sum_{k\neq i,j} \frac{1}{c_{ik}c_{jk}} = \frac{1}{0.01}\left( 1 - \frac1{n-1}\right) = 100 - \frac{100}{n-1}.
    \end{align*}

    Clearly, for $n\ge 3$, condition~\eqref{eq:separation_condition} is satisfied.

    For the second part, we simply take that if $\lambda_i$ and $\lambda_j$ are the unique functions in the system, then 

    \begin{equation*}
        \sum_{k\neq i,j} \frac{1}{(\lambda_i - \lambda_k)(\lambda_j - \lambda_k)} = 0 < \frac{2}{(\lambda_i - \lambda_j)^2},
    \end{equation*}

    \noindent for every position of $\lambda_i, \lambda_j$.
\end{proof}

\begin{lemma} \label{lemma:separating_condition_wishart}
    Let $(\lambda_1, \lambda_2, \dots, \lambda_n)$ be a system of $n$ functions moving according to \eqref{eq:deterministic_wishart}, and let $\lambda_i, \lambda_j$ be such that $\lambda_i(t_0) > \lambda_j(t_0)$ and there is no $\lambda_k$ such that $\lambda_i(t_0) > \lambda_k(t_0) > \lambda_j(t_0)$. Then $\lambda_i$ and $\lambda_j$ repel each other if and only if 

    \begin{equation} \label{eq:separation_condition_wishart}
        \frac{\lambda_i + \lambda_j}{(\lambda_i - \lambda_j)^2} > \sum_{k\neq i,j} \frac{\lambda_k}{(\lambda_i-\lambda_k)(\lambda_j-\lambda_k)}.
    \end{equation}
\end{lemma}

\begin{proof}

    Take the derivative of the separation and use linearity with \eqref{eq:deterministic_wishart} to get

    \begin{align*}
        \frac{\d}{\d t}(\lambda_i - \lambda_j) &= \sum_{k\neq i} \frac{\lambda_i + \lambda_k}{\lambda_i - \lambda_k} - \sum_{k\neq j} \frac{\lambda_j + \lambda_k}{\lambda_j - \lambda_k}, \\
        &= \sum_{k \neq i,j} \frac{ (\lambda_i + \lambda_k)(\lambda_j - \lambda_k) - (\lambda_j + \lambda_k)(\lambda_i - \lambda_k) }{(\lambda_i - \lambda_k)(\lambda_j - \lambda_k)} + 2\frac{\lambda_i + \lambda_j}{\lambda_i - \lambda_j},\\
        &= \sum_{k\neq i,j} \frac{2\lambda_k(\lambda_j - \lambda_i)}{(\lambda_i-\lambda_k)(\lambda_j-\lambda_k)} + 2 \frac{\lambda_i + \lambda_j}{\lambda_i - \lambda_j}.
    \end{align*}

    Comparing to zero yields the result.
\end{proof}

\begin{corollary}
    In a system of $n$ functions satisfying \eqref{eq:deterministic_wishart} there is no collision of the functions.
\end{corollary}

\begin{proof}
    Use condition \eqref{eq:separation_condition_wishart} and let $\lambda_i \to \lambda_j$, then

    \begin{equation*}
        \frac{\lambda_i + \lambda_j}{(\lambda_i-\lambda_j)^2} \to \infty,
    \end{equation*}

    \noindent but

    \begin{align*}
        \sum_{k\neq i,j} \frac{\lambda_k}{(\lambda_i-\lambda_k)(\lambda_j-\lambda_k)} = \sum_{k\neq i,j} \frac{\lambda_k}{(\lambda_i-\lambda_j + \lambda_j -\lambda_k)(\lambda_j-\lambda_k)} \to \sum_{k\neq i,j} \frac{\lambda_k}{(\lambda_j-\lambda_k)^2} < \infty.
    \end{align*}

    So before the functions collide, the derivative of the separation is positive and they repel each other.
\end{proof}

\begin{theorem} \label{thm:laguerre_does_not_grow}
    Let $(\lambda_1, \lambda_2, \dots, \lambda_n)$ be a system of $n$ functions moving according to \eqref{eq:deterministic_wishart} and let $\lambda_i,\lambda_j$ be such that for a given $t_0$ it is satisfied $\lambda_i(t_0) > \lambda_j(t_0)$ and 
    
    \begin{equation*}
        \lambda_i(t_0) - \lambda_j(t_0) = \min_{k,l \in [n]} \abs{ \lambda_k(t_0) - \lambda_l(t_0)}.
    \end{equation*}

    Then, $\lambda_i$ and $\lambda_k$ do not necessarily repel. 
\end{theorem}

\begin{proof}
    We will prove providing an initial condition for which the minimal distance will have a negative derivative. Let $\lambda_j(t_0) = \min_{k \le n} \lambda_k(t_0)$ and so $\lambda_i(t_0) = \min_{k\neq j} \lambda_k(t_0)$. Similarly to the proof of Theorem \ref*{thm:hermite_minimal_grows}, let us define the quotients $c_{ik},c_{jk}$ as 

    \begin{equation*}
        c_{ik} = \frac{\lambda_i - \lambda_k}{\lambda_i - \lambda_j}, \qquad c_{jk} = \frac{\lambda_j - \lambda_k}{\lambda_i - \lambda_j}.
    \end{equation*}

    For every fixed $k$, the quotients $c_{ik}$ and $c_{jk}$ have the same sign and given the minimality of $\lambda_i - \lambda_j$ we have that $c_{ik}c_{jk}>1$. Using these quantities, the separation condition \eqref{eq:separation_condition_wishart} is reduced to 

    \begin{equation*}
        \lambda_i + \lambda_j > \sum_{k \neq i,j} \frac{\lambda_k}{c_{ik}c_{jk}}.
    \end{equation*}

    Suppose that $\lambda_i - \lambda_j =1$, and that $\lambda_k > \lambda_i$ for all $k \notin \{i,j\}$. Assume further that the separation between all the $(\lambda_k)_{k\neq i,j}$ is $1+\epsilon$ for some $\epsilon >0$. Then $c_{i,i-1}= 1+\epsilon, \abs{c_{j,i+1}}= 2 + \epsilon$ and in general $\abs{c_{i,i+l}}=l(1+\epsilon), \abs{c_{j,i+l}} = 1 + l(1+\epsilon)$. Furthermore, we have that $\lambda_{i+l}$ can be expressed as

    \begin{equation*}
         \lambda_{i+l} = \lambda_i + l(1+\epsilon) = \lambda_j + 1 + l(1+\epsilon).
    \end{equation*}

    With this, the separating condition for $\lambda_i$ and $\lambda_j$ can be written as

    \begin{equation*}
        2\lambda_j + 1 > \sum_{k=2}^{n} \frac{\lambda_j + 1 + k(1+\epsilon)}{k(1+\epsilon)(1+k(1+\epsilon))}. 
    \end{equation*}

    The left-hand side is fixed for fixed $\lambda_j(t_0)$. For the left-hand side we have

    \begin{align*}
        \sum_{k=2}^{n} \frac{\lambda_j + 1 + k(1+\epsilon)}{k(1+\epsilon)(1+k(1+\epsilon))} = \sum_{k=2}^n \frac{\lambda_j + 1}{k(1+\epsilon)(1+k(1+\epsilon))} + \sum_{k=2}^n \frac{1}{1+k(1+\epsilon)}.
    \end{align*}

    For the second element in the sum and $\epsilon$ sufficiently small we have 

    \begin{align*}
        \sum_{k=2}^n \frac{1}{1+k(1+\epsilon)} > \sum_{k=2}^n \frac{1}{1+2k} > \frac12\sum_{k=2}^n \frac{1}{1+k}.
    \end{align*}

    The last expression can be made arbitrarily big for $n$ big enough. We conclude that under these conditions, for a system with enough functions, the minimal distance can be made smaller with a specific initial condition.
    
\end{proof}

\section{Simulations}


\begin{figure}[h!]
    %% Creator: Matplotlib, PGF backend
%%
%% To include the figure in your LaTeX document, write
%%   \input{<filename>.pgf}
%%
%% Make sure the required packages are loaded in your preamble
%%   \usepackage{pgf}
%%
%% Also ensure that all the required font packages are loaded; for instance,
%% the lmodern package is sometimes necessary when using math font.
%%   \usepackage{lmodern}
%%
%% Figures using additional raster images can only be included by \input if
%% they are in the same directory as the main LaTeX file. For loading figures
%% from other directories you can use the `import` package
%%   \usepackage{import}
%%
%% and then include the figures with
%%   \import{<path to file>}{<filename>.pgf}
%%
%% Matplotlib used the following preamble
%%   
%%   \makeatletter\@ifpackageloaded{underscore}{}{\usepackage[strings]{underscore}}\makeatother
%%
\begingroup%
\makeatletter%
\begin{pgfpicture}%
\pgfpathrectangle{\pgfpointorigin}{\pgfqpoint{6.500000in}{2.500000in}}%
\pgfusepath{use as bounding box, clip}%
\begin{pgfscope}%
\pgfsetbuttcap%
\pgfsetmiterjoin%
\definecolor{currentfill}{rgb}{1.000000,1.000000,1.000000}%
\pgfsetfillcolor{currentfill}%
\pgfsetlinewidth{0.000000pt}%
\definecolor{currentstroke}{rgb}{1.000000,1.000000,1.000000}%
\pgfsetstrokecolor{currentstroke}%
\pgfsetdash{}{0pt}%
\pgfpathmoveto{\pgfqpoint{0.000000in}{0.000000in}}%
\pgfpathlineto{\pgfqpoint{6.500000in}{0.000000in}}%
\pgfpathlineto{\pgfqpoint{6.500000in}{2.500000in}}%
\pgfpathlineto{\pgfqpoint{0.000000in}{2.500000in}}%
\pgfpathlineto{\pgfqpoint{0.000000in}{0.000000in}}%
\pgfpathclose%
\pgfusepath{fill}%
\end{pgfscope}%
\begin{pgfscope}%
\pgfsetbuttcap%
\pgfsetmiterjoin%
\definecolor{currentfill}{rgb}{0.917647,0.917647,0.949020}%
\pgfsetfillcolor{currentfill}%
\pgfsetlinewidth{0.000000pt}%
\definecolor{currentstroke}{rgb}{0.000000,0.000000,0.000000}%
\pgfsetstrokecolor{currentstroke}%
\pgfsetstrokeopacity{0.000000}%
\pgfsetdash{}{0pt}%
\pgfpathmoveto{\pgfqpoint{0.812500in}{1.325000in}}%
\pgfpathlineto{\pgfqpoint{3.102273in}{1.325000in}}%
\pgfpathlineto{\pgfqpoint{3.102273in}{2.200000in}}%
\pgfpathlineto{\pgfqpoint{0.812500in}{2.200000in}}%
\pgfpathlineto{\pgfqpoint{0.812500in}{1.325000in}}%
\pgfpathclose%
\pgfusepath{fill}%
\end{pgfscope}%
\begin{pgfscope}%
\pgfpathrectangle{\pgfqpoint{0.812500in}{1.325000in}}{\pgfqpoint{2.289773in}{0.875000in}}%
\pgfusepath{clip}%
\pgfsetroundcap%
\pgfsetroundjoin%
\pgfsetlinewidth{1.003750pt}%
\definecolor{currentstroke}{rgb}{1.000000,1.000000,1.000000}%
\pgfsetstrokecolor{currentstroke}%
\pgfsetdash{}{0pt}%
\pgfpathmoveto{\pgfqpoint{0.916581in}{1.325000in}}%
\pgfpathlineto{\pgfqpoint{0.916581in}{2.200000in}}%
\pgfusepath{stroke}%
\end{pgfscope}%
\begin{pgfscope}%
\definecolor{textcolor}{rgb}{0.150000,0.150000,0.150000}%
\pgfsetstrokecolor{textcolor}%
\pgfsetfillcolor{textcolor}%
\pgftext[x=0.916581in,y=1.227778in,,top]{\color{textcolor}\rmfamily\fontsize{10.000000}{12.000000}\selectfont \(\displaystyle {0.0}\)}%
\end{pgfscope}%
\begin{pgfscope}%
\pgfpathrectangle{\pgfqpoint{0.812500in}{1.325000in}}{\pgfqpoint{2.289773in}{0.875000in}}%
\pgfusepath{clip}%
\pgfsetroundcap%
\pgfsetroundjoin%
\pgfsetlinewidth{1.003750pt}%
\definecolor{currentstroke}{rgb}{1.000000,1.000000,1.000000}%
\pgfsetstrokecolor{currentstroke}%
\pgfsetdash{}{0pt}%
\pgfpathmoveto{\pgfqpoint{1.436983in}{1.325000in}}%
\pgfpathlineto{\pgfqpoint{1.436983in}{2.200000in}}%
\pgfusepath{stroke}%
\end{pgfscope}%
\begin{pgfscope}%
\definecolor{textcolor}{rgb}{0.150000,0.150000,0.150000}%
\pgfsetstrokecolor{textcolor}%
\pgfsetfillcolor{textcolor}%
\pgftext[x=1.436983in,y=1.227778in,,top]{\color{textcolor}\rmfamily\fontsize{10.000000}{12.000000}\selectfont \(\displaystyle {2.5}\)}%
\end{pgfscope}%
\begin{pgfscope}%
\pgfpathrectangle{\pgfqpoint{0.812500in}{1.325000in}}{\pgfqpoint{2.289773in}{0.875000in}}%
\pgfusepath{clip}%
\pgfsetroundcap%
\pgfsetroundjoin%
\pgfsetlinewidth{1.003750pt}%
\definecolor{currentstroke}{rgb}{1.000000,1.000000,1.000000}%
\pgfsetstrokecolor{currentstroke}%
\pgfsetdash{}{0pt}%
\pgfpathmoveto{\pgfqpoint{1.957386in}{1.325000in}}%
\pgfpathlineto{\pgfqpoint{1.957386in}{2.200000in}}%
\pgfusepath{stroke}%
\end{pgfscope}%
\begin{pgfscope}%
\definecolor{textcolor}{rgb}{0.150000,0.150000,0.150000}%
\pgfsetstrokecolor{textcolor}%
\pgfsetfillcolor{textcolor}%
\pgftext[x=1.957386in,y=1.227778in,,top]{\color{textcolor}\rmfamily\fontsize{10.000000}{12.000000}\selectfont \(\displaystyle {5.0}\)}%
\end{pgfscope}%
\begin{pgfscope}%
\pgfpathrectangle{\pgfqpoint{0.812500in}{1.325000in}}{\pgfqpoint{2.289773in}{0.875000in}}%
\pgfusepath{clip}%
\pgfsetroundcap%
\pgfsetroundjoin%
\pgfsetlinewidth{1.003750pt}%
\definecolor{currentstroke}{rgb}{1.000000,1.000000,1.000000}%
\pgfsetstrokecolor{currentstroke}%
\pgfsetdash{}{0pt}%
\pgfpathmoveto{\pgfqpoint{2.477789in}{1.325000in}}%
\pgfpathlineto{\pgfqpoint{2.477789in}{2.200000in}}%
\pgfusepath{stroke}%
\end{pgfscope}%
\begin{pgfscope}%
\definecolor{textcolor}{rgb}{0.150000,0.150000,0.150000}%
\pgfsetstrokecolor{textcolor}%
\pgfsetfillcolor{textcolor}%
\pgftext[x=2.477789in,y=1.227778in,,top]{\color{textcolor}\rmfamily\fontsize{10.000000}{12.000000}\selectfont \(\displaystyle {7.5}\)}%
\end{pgfscope}%
\begin{pgfscope}%
\pgfpathrectangle{\pgfqpoint{0.812500in}{1.325000in}}{\pgfqpoint{2.289773in}{0.875000in}}%
\pgfusepath{clip}%
\pgfsetroundcap%
\pgfsetroundjoin%
\pgfsetlinewidth{1.003750pt}%
\definecolor{currentstroke}{rgb}{1.000000,1.000000,1.000000}%
\pgfsetstrokecolor{currentstroke}%
\pgfsetdash{}{0pt}%
\pgfpathmoveto{\pgfqpoint{2.998192in}{1.325000in}}%
\pgfpathlineto{\pgfqpoint{2.998192in}{2.200000in}}%
\pgfusepath{stroke}%
\end{pgfscope}%
\begin{pgfscope}%
\definecolor{textcolor}{rgb}{0.150000,0.150000,0.150000}%
\pgfsetstrokecolor{textcolor}%
\pgfsetfillcolor{textcolor}%
\pgftext[x=2.998192in,y=1.227778in,,top]{\color{textcolor}\rmfamily\fontsize{10.000000}{12.000000}\selectfont \(\displaystyle {10.0}\)}%
\end{pgfscope}%
\begin{pgfscope}%
\definecolor{textcolor}{rgb}{0.150000,0.150000,0.150000}%
\pgfsetstrokecolor{textcolor}%
\pgfsetfillcolor{textcolor}%
\pgftext[x=1.957386in,y=1.048766in,,top]{\color{textcolor}\rmfamily\fontsize{11.000000}{13.200000}\selectfont time (\(\displaystyle t\))}%
\end{pgfscope}%
\begin{pgfscope}%
\pgfpathrectangle{\pgfqpoint{0.812500in}{1.325000in}}{\pgfqpoint{2.289773in}{0.875000in}}%
\pgfusepath{clip}%
\pgfsetroundcap%
\pgfsetroundjoin%
\pgfsetlinewidth{1.003750pt}%
\definecolor{currentstroke}{rgb}{1.000000,1.000000,1.000000}%
\pgfsetstrokecolor{currentstroke}%
\pgfsetdash{}{0pt}%
\pgfpathmoveto{\pgfqpoint{0.812500in}{1.477238in}}%
\pgfpathlineto{\pgfqpoint{3.102273in}{1.477238in}}%
\pgfusepath{stroke}%
\end{pgfscope}%
\begin{pgfscope}%
\definecolor{textcolor}{rgb}{0.150000,0.150000,0.150000}%
\pgfsetstrokecolor{textcolor}%
\pgfsetfillcolor{textcolor}%
\pgftext[x=0.537808in, y=1.429013in, left, base]{\color{textcolor}\rmfamily\fontsize{10.000000}{12.000000}\selectfont \(\displaystyle {\ensuremath{-}5}\)}%
\end{pgfscope}%
\begin{pgfscope}%
\pgfpathrectangle{\pgfqpoint{0.812500in}{1.325000in}}{\pgfqpoint{2.289773in}{0.875000in}}%
\pgfusepath{clip}%
\pgfsetroundcap%
\pgfsetroundjoin%
\pgfsetlinewidth{1.003750pt}%
\definecolor{currentstroke}{rgb}{1.000000,1.000000,1.000000}%
\pgfsetstrokecolor{currentstroke}%
\pgfsetdash{}{0pt}%
\pgfpathmoveto{\pgfqpoint{0.812500in}{1.736567in}}%
\pgfpathlineto{\pgfqpoint{3.102273in}{1.736567in}}%
\pgfusepath{stroke}%
\end{pgfscope}%
\begin{pgfscope}%
\definecolor{textcolor}{rgb}{0.150000,0.150000,0.150000}%
\pgfsetstrokecolor{textcolor}%
\pgfsetfillcolor{textcolor}%
\pgftext[x=0.645833in, y=1.688342in, left, base]{\color{textcolor}\rmfamily\fontsize{10.000000}{12.000000}\selectfont \(\displaystyle {0}\)}%
\end{pgfscope}%
\begin{pgfscope}%
\pgfpathrectangle{\pgfqpoint{0.812500in}{1.325000in}}{\pgfqpoint{2.289773in}{0.875000in}}%
\pgfusepath{clip}%
\pgfsetroundcap%
\pgfsetroundjoin%
\pgfsetlinewidth{1.003750pt}%
\definecolor{currentstroke}{rgb}{1.000000,1.000000,1.000000}%
\pgfsetstrokecolor{currentstroke}%
\pgfsetdash{}{0pt}%
\pgfpathmoveto{\pgfqpoint{0.812500in}{1.995896in}}%
\pgfpathlineto{\pgfqpoint{3.102273in}{1.995896in}}%
\pgfusepath{stroke}%
\end{pgfscope}%
\begin{pgfscope}%
\definecolor{textcolor}{rgb}{0.150000,0.150000,0.150000}%
\pgfsetstrokecolor{textcolor}%
\pgfsetfillcolor{textcolor}%
\pgftext[x=0.645833in, y=1.947671in, left, base]{\color{textcolor}\rmfamily\fontsize{10.000000}{12.000000}\selectfont \(\displaystyle {5}\)}%
\end{pgfscope}%
\begin{pgfscope}%
\definecolor{textcolor}{rgb}{0.150000,0.150000,0.150000}%
\pgfsetstrokecolor{textcolor}%
\pgfsetfillcolor{textcolor}%
\pgftext[x=0.482252in,y=1.762500in,,bottom,rotate=90.000000]{\color{textcolor}\rmfamily\fontsize{11.000000}{13.200000}\selectfont Position}%
\end{pgfscope}%
\begin{pgfscope}%
\pgfpathrectangle{\pgfqpoint{0.812500in}{1.325000in}}{\pgfqpoint{2.289773in}{0.875000in}}%
\pgfusepath{clip}%
\pgfsetroundcap%
\pgfsetroundjoin%
\pgfsetlinewidth{1.756562pt}%
\definecolor{currentstroke}{rgb}{0.215686,0.494118,0.721569}%
\pgfsetstrokecolor{currentstroke}%
\pgfsetdash{}{0pt}%
\pgfpathmoveto{\pgfqpoint{0.916581in}{1.679515in}}%
\pgfpathlineto{\pgfqpoint{0.918664in}{1.673999in}}%
\pgfpathlineto{\pgfqpoint{0.926999in}{1.667252in}}%
\pgfpathlineto{\pgfqpoint{0.941585in}{1.659158in}}%
\pgfpathlineto{\pgfqpoint{0.964506in}{1.649296in}}%
\pgfpathlineto{\pgfqpoint{0.997845in}{1.637536in}}%
\pgfpathlineto{\pgfqpoint{1.043686in}{1.623848in}}%
\pgfpathlineto{\pgfqpoint{1.104113in}{1.608258in}}%
\pgfpathlineto{\pgfqpoint{1.179126in}{1.591279in}}%
\pgfpathlineto{\pgfqpoint{1.270809in}{1.572856in}}%
\pgfpathlineto{\pgfqpoint{1.381245in}{1.552995in}}%
\pgfpathlineto{\pgfqpoint{1.512517in}{1.531730in}}%
\pgfpathlineto{\pgfqpoint{1.664627in}{1.509405in}}%
\pgfpathlineto{\pgfqpoint{1.839658in}{1.485998in}}%
\pgfpathlineto{\pgfqpoint{2.041776in}{1.461269in}}%
\pgfpathlineto{\pgfqpoint{2.270983in}{1.435512in}}%
\pgfpathlineto{\pgfqpoint{2.531444in}{1.408534in}}%
\pgfpathlineto{\pgfqpoint{2.823162in}{1.380591in}}%
\pgfpathlineto{\pgfqpoint{2.998192in}{1.364773in}}%
\pgfpathlineto{\pgfqpoint{2.998192in}{1.364773in}}%
\pgfusepath{stroke}%
\end{pgfscope}%
\begin{pgfscope}%
\pgfpathrectangle{\pgfqpoint{0.812500in}{1.325000in}}{\pgfqpoint{2.289773in}{0.875000in}}%
\pgfusepath{clip}%
\pgfsetroundcap%
\pgfsetroundjoin%
\pgfsetlinewidth{1.756562pt}%
\definecolor{currentstroke}{rgb}{1.000000,0.498039,0.000000}%
\pgfsetstrokecolor{currentstroke}%
\pgfsetdash{}{0pt}%
\pgfpathmoveto{\pgfqpoint{0.916581in}{1.684701in}}%
\pgfpathlineto{\pgfqpoint{0.918664in}{1.689548in}}%
\pgfpathlineto{\pgfqpoint{0.924915in}{1.692870in}}%
\pgfpathlineto{\pgfqpoint{0.937418in}{1.696322in}}%
\pgfpathlineto{\pgfqpoint{0.958254in}{1.699273in}}%
\pgfpathlineto{\pgfqpoint{0.989510in}{1.701260in}}%
\pgfpathlineto{\pgfqpoint{1.039519in}{1.701951in}}%
\pgfpathlineto{\pgfqpoint{1.118699in}{1.700538in}}%
\pgfpathlineto{\pgfqpoint{1.264558in}{1.695329in}}%
\pgfpathlineto{\pgfqpoint{2.104287in}{1.662722in}}%
\pgfpathlineto{\pgfqpoint{2.598123in}{1.646816in}}%
\pgfpathlineto{\pgfqpoint{2.998192in}{1.635275in}}%
\pgfpathlineto{\pgfqpoint{2.998192in}{1.635275in}}%
\pgfusepath{stroke}%
\end{pgfscope}%
\begin{pgfscope}%
\pgfpathrectangle{\pgfqpoint{0.812500in}{1.325000in}}{\pgfqpoint{2.289773in}{0.875000in}}%
\pgfusepath{clip}%
\pgfsetroundcap%
\pgfsetroundjoin%
\pgfsetlinewidth{1.756562pt}%
\definecolor{currentstroke}{rgb}{0.301961,0.686275,0.290196}%
\pgfsetstrokecolor{currentstroke}%
\pgfsetdash{}{0pt}%
\pgfpathmoveto{\pgfqpoint{0.916581in}{1.840299in}}%
\pgfpathlineto{\pgfqpoint{0.918664in}{1.835452in}}%
\pgfpathlineto{\pgfqpoint{0.924915in}{1.832130in}}%
\pgfpathlineto{\pgfqpoint{0.937418in}{1.828678in}}%
\pgfpathlineto{\pgfqpoint{0.958254in}{1.825727in}}%
\pgfpathlineto{\pgfqpoint{0.989510in}{1.823740in}}%
\pgfpathlineto{\pgfqpoint{1.039519in}{1.823049in}}%
\pgfpathlineto{\pgfqpoint{1.118699in}{1.824462in}}%
\pgfpathlineto{\pgfqpoint{1.264558in}{1.829671in}}%
\pgfpathlineto{\pgfqpoint{2.104287in}{1.862278in}}%
\pgfpathlineto{\pgfqpoint{2.598123in}{1.878184in}}%
\pgfpathlineto{\pgfqpoint{2.998192in}{1.889725in}}%
\pgfpathlineto{\pgfqpoint{2.998192in}{1.889725in}}%
\pgfusepath{stroke}%
\end{pgfscope}%
\begin{pgfscope}%
\pgfpathrectangle{\pgfqpoint{0.812500in}{1.325000in}}{\pgfqpoint{2.289773in}{0.875000in}}%
\pgfusepath{clip}%
\pgfsetroundcap%
\pgfsetroundjoin%
\pgfsetlinewidth{1.756562pt}%
\definecolor{currentstroke}{rgb}{0.968627,0.505882,0.749020}%
\pgfsetstrokecolor{currentstroke}%
\pgfsetdash{}{0pt}%
\pgfpathmoveto{\pgfqpoint{0.916581in}{1.845485in}}%
\pgfpathlineto{\pgfqpoint{0.918664in}{1.851001in}}%
\pgfpathlineto{\pgfqpoint{0.926999in}{1.857748in}}%
\pgfpathlineto{\pgfqpoint{0.941585in}{1.865842in}}%
\pgfpathlineto{\pgfqpoint{0.964506in}{1.875704in}}%
\pgfpathlineto{\pgfqpoint{0.997845in}{1.887464in}}%
\pgfpathlineto{\pgfqpoint{1.043686in}{1.901152in}}%
\pgfpathlineto{\pgfqpoint{1.104113in}{1.916742in}}%
\pgfpathlineto{\pgfqpoint{1.179126in}{1.933721in}}%
\pgfpathlineto{\pgfqpoint{1.270809in}{1.952144in}}%
\pgfpathlineto{\pgfqpoint{1.381245in}{1.972005in}}%
\pgfpathlineto{\pgfqpoint{1.512517in}{1.993270in}}%
\pgfpathlineto{\pgfqpoint{1.664627in}{2.015595in}}%
\pgfpathlineto{\pgfqpoint{1.839658in}{2.039002in}}%
\pgfpathlineto{\pgfqpoint{2.041776in}{2.063731in}}%
\pgfpathlineto{\pgfqpoint{2.270983in}{2.089488in}}%
\pgfpathlineto{\pgfqpoint{2.531444in}{2.116466in}}%
\pgfpathlineto{\pgfqpoint{2.823162in}{2.144409in}}%
\pgfpathlineto{\pgfqpoint{2.998192in}{2.160227in}}%
\pgfpathlineto{\pgfqpoint{2.998192in}{2.160227in}}%
\pgfusepath{stroke}%
\end{pgfscope}%
\begin{pgfscope}%
\pgfsetrectcap%
\pgfsetmiterjoin%
\pgfsetlinewidth{0.000000pt}%
\definecolor{currentstroke}{rgb}{1.000000,1.000000,1.000000}%
\pgfsetstrokecolor{currentstroke}%
\pgfsetdash{}{0pt}%
\pgfpathmoveto{\pgfqpoint{0.812500in}{1.325000in}}%
\pgfpathlineto{\pgfqpoint{0.812500in}{2.200000in}}%
\pgfusepath{}%
\end{pgfscope}%
\begin{pgfscope}%
\pgfsetrectcap%
\pgfsetmiterjoin%
\pgfsetlinewidth{0.000000pt}%
\definecolor{currentstroke}{rgb}{1.000000,1.000000,1.000000}%
\pgfsetstrokecolor{currentstroke}%
\pgfsetdash{}{0pt}%
\pgfpathmoveto{\pgfqpoint{3.102273in}{1.325000in}}%
\pgfpathlineto{\pgfqpoint{3.102273in}{2.200000in}}%
\pgfusepath{}%
\end{pgfscope}%
\begin{pgfscope}%
\pgfsetrectcap%
\pgfsetmiterjoin%
\pgfsetlinewidth{0.000000pt}%
\definecolor{currentstroke}{rgb}{1.000000,1.000000,1.000000}%
\pgfsetstrokecolor{currentstroke}%
\pgfsetdash{}{0pt}%
\pgfpathmoveto{\pgfqpoint{0.812500in}{1.325000in}}%
\pgfpathlineto{\pgfqpoint{3.102273in}{1.325000in}}%
\pgfusepath{}%
\end{pgfscope}%
\begin{pgfscope}%
\pgfsetrectcap%
\pgfsetmiterjoin%
\pgfsetlinewidth{0.000000pt}%
\definecolor{currentstroke}{rgb}{1.000000,1.000000,1.000000}%
\pgfsetstrokecolor{currentstroke}%
\pgfsetdash{}{0pt}%
\pgfpathmoveto{\pgfqpoint{0.812500in}{2.200000in}}%
\pgfpathlineto{\pgfqpoint{3.102273in}{2.200000in}}%
\pgfusepath{}%
\end{pgfscope}%
\begin{pgfscope}%
\pgfsetbuttcap%
\pgfsetmiterjoin%
\definecolor{currentfill}{rgb}{0.917647,0.917647,0.949020}%
\pgfsetfillcolor{currentfill}%
\pgfsetlinewidth{0.000000pt}%
\definecolor{currentstroke}{rgb}{0.000000,0.000000,0.000000}%
\pgfsetstrokecolor{currentstroke}%
\pgfsetstrokeopacity{0.000000}%
\pgfsetdash{}{0pt}%
\pgfpathmoveto{\pgfqpoint{3.560227in}{1.325000in}}%
\pgfpathlineto{\pgfqpoint{5.850000in}{1.325000in}}%
\pgfpathlineto{\pgfqpoint{5.850000in}{2.200000in}}%
\pgfpathlineto{\pgfqpoint{3.560227in}{2.200000in}}%
\pgfpathlineto{\pgfqpoint{3.560227in}{1.325000in}}%
\pgfpathclose%
\pgfusepath{fill}%
\end{pgfscope}%
\begin{pgfscope}%
\pgfpathrectangle{\pgfqpoint{3.560227in}{1.325000in}}{\pgfqpoint{2.289773in}{0.875000in}}%
\pgfusepath{clip}%
\pgfsetroundcap%
\pgfsetroundjoin%
\pgfsetlinewidth{1.003750pt}%
\definecolor{currentstroke}{rgb}{1.000000,1.000000,1.000000}%
\pgfsetstrokecolor{currentstroke}%
\pgfsetdash{}{0pt}%
\pgfpathmoveto{\pgfqpoint{3.664308in}{1.325000in}}%
\pgfpathlineto{\pgfqpoint{3.664308in}{2.200000in}}%
\pgfusepath{stroke}%
\end{pgfscope}%
\begin{pgfscope}%
\definecolor{textcolor}{rgb}{0.150000,0.150000,0.150000}%
\pgfsetstrokecolor{textcolor}%
\pgfsetfillcolor{textcolor}%
\pgftext[x=3.664308in,y=1.227778in,,top]{\color{textcolor}\rmfamily\fontsize{10.000000}{12.000000}\selectfont \(\displaystyle {0.0}\)}%
\end{pgfscope}%
\begin{pgfscope}%
\pgfpathrectangle{\pgfqpoint{3.560227in}{1.325000in}}{\pgfqpoint{2.289773in}{0.875000in}}%
\pgfusepath{clip}%
\pgfsetroundcap%
\pgfsetroundjoin%
\pgfsetlinewidth{1.003750pt}%
\definecolor{currentstroke}{rgb}{1.000000,1.000000,1.000000}%
\pgfsetstrokecolor{currentstroke}%
\pgfsetdash{}{0pt}%
\pgfpathmoveto{\pgfqpoint{4.184711in}{1.325000in}}%
\pgfpathlineto{\pgfqpoint{4.184711in}{2.200000in}}%
\pgfusepath{stroke}%
\end{pgfscope}%
\begin{pgfscope}%
\definecolor{textcolor}{rgb}{0.150000,0.150000,0.150000}%
\pgfsetstrokecolor{textcolor}%
\pgfsetfillcolor{textcolor}%
\pgftext[x=4.184711in,y=1.227778in,,top]{\color{textcolor}\rmfamily\fontsize{10.000000}{12.000000}\selectfont \(\displaystyle {2.5}\)}%
\end{pgfscope}%
\begin{pgfscope}%
\pgfpathrectangle{\pgfqpoint{3.560227in}{1.325000in}}{\pgfqpoint{2.289773in}{0.875000in}}%
\pgfusepath{clip}%
\pgfsetroundcap%
\pgfsetroundjoin%
\pgfsetlinewidth{1.003750pt}%
\definecolor{currentstroke}{rgb}{1.000000,1.000000,1.000000}%
\pgfsetstrokecolor{currentstroke}%
\pgfsetdash{}{0pt}%
\pgfpathmoveto{\pgfqpoint{4.705114in}{1.325000in}}%
\pgfpathlineto{\pgfqpoint{4.705114in}{2.200000in}}%
\pgfusepath{stroke}%
\end{pgfscope}%
\begin{pgfscope}%
\definecolor{textcolor}{rgb}{0.150000,0.150000,0.150000}%
\pgfsetstrokecolor{textcolor}%
\pgfsetfillcolor{textcolor}%
\pgftext[x=4.705114in,y=1.227778in,,top]{\color{textcolor}\rmfamily\fontsize{10.000000}{12.000000}\selectfont \(\displaystyle {5.0}\)}%
\end{pgfscope}%
\begin{pgfscope}%
\pgfpathrectangle{\pgfqpoint{3.560227in}{1.325000in}}{\pgfqpoint{2.289773in}{0.875000in}}%
\pgfusepath{clip}%
\pgfsetroundcap%
\pgfsetroundjoin%
\pgfsetlinewidth{1.003750pt}%
\definecolor{currentstroke}{rgb}{1.000000,1.000000,1.000000}%
\pgfsetstrokecolor{currentstroke}%
\pgfsetdash{}{0pt}%
\pgfpathmoveto{\pgfqpoint{5.225517in}{1.325000in}}%
\pgfpathlineto{\pgfqpoint{5.225517in}{2.200000in}}%
\pgfusepath{stroke}%
\end{pgfscope}%
\begin{pgfscope}%
\definecolor{textcolor}{rgb}{0.150000,0.150000,0.150000}%
\pgfsetstrokecolor{textcolor}%
\pgfsetfillcolor{textcolor}%
\pgftext[x=5.225517in,y=1.227778in,,top]{\color{textcolor}\rmfamily\fontsize{10.000000}{12.000000}\selectfont \(\displaystyle {7.5}\)}%
\end{pgfscope}%
\begin{pgfscope}%
\pgfpathrectangle{\pgfqpoint{3.560227in}{1.325000in}}{\pgfqpoint{2.289773in}{0.875000in}}%
\pgfusepath{clip}%
\pgfsetroundcap%
\pgfsetroundjoin%
\pgfsetlinewidth{1.003750pt}%
\definecolor{currentstroke}{rgb}{1.000000,1.000000,1.000000}%
\pgfsetstrokecolor{currentstroke}%
\pgfsetdash{}{0pt}%
\pgfpathmoveto{\pgfqpoint{5.745919in}{1.325000in}}%
\pgfpathlineto{\pgfqpoint{5.745919in}{2.200000in}}%
\pgfusepath{stroke}%
\end{pgfscope}%
\begin{pgfscope}%
\definecolor{textcolor}{rgb}{0.150000,0.150000,0.150000}%
\pgfsetstrokecolor{textcolor}%
\pgfsetfillcolor{textcolor}%
\pgftext[x=5.745919in,y=1.227778in,,top]{\color{textcolor}\rmfamily\fontsize{10.000000}{12.000000}\selectfont \(\displaystyle {10.0}\)}%
\end{pgfscope}%
\begin{pgfscope}%
\definecolor{textcolor}{rgb}{0.150000,0.150000,0.150000}%
\pgfsetstrokecolor{textcolor}%
\pgfsetfillcolor{textcolor}%
\pgftext[x=4.705114in,y=1.048766in,,top]{\color{textcolor}\rmfamily\fontsize{11.000000}{13.200000}\selectfont time (\(\displaystyle t\))}%
\end{pgfscope}%
\begin{pgfscope}%
\pgfpathrectangle{\pgfqpoint{3.560227in}{1.325000in}}{\pgfqpoint{2.289773in}{0.875000in}}%
\pgfusepath{clip}%
\pgfsetroundcap%
\pgfsetroundjoin%
\pgfsetlinewidth{1.003750pt}%
\definecolor{currentstroke}{rgb}{1.000000,1.000000,1.000000}%
\pgfsetstrokecolor{currentstroke}%
\pgfsetdash{}{0pt}%
\pgfpathmoveto{\pgfqpoint{3.560227in}{1.325018in}}%
\pgfpathlineto{\pgfqpoint{5.850000in}{1.325018in}}%
\pgfusepath{stroke}%
\end{pgfscope}%
\begin{pgfscope}%
\definecolor{textcolor}{rgb}{0.150000,0.150000,0.150000}%
\pgfsetstrokecolor{textcolor}%
\pgfsetfillcolor{textcolor}%
\pgftext[x=3.216091in, y=1.276793in, left, base]{\color{textcolor}\rmfamily\fontsize{10.000000}{12.000000}\selectfont \(\displaystyle {\ensuremath{-}10}\)}%
\end{pgfscope}%
\begin{pgfscope}%
\pgfpathrectangle{\pgfqpoint{3.560227in}{1.325000in}}{\pgfqpoint{2.289773in}{0.875000in}}%
\pgfusepath{clip}%
\pgfsetroundcap%
\pgfsetroundjoin%
\pgfsetlinewidth{1.003750pt}%
\definecolor{currentstroke}{rgb}{1.000000,1.000000,1.000000}%
\pgfsetstrokecolor{currentstroke}%
\pgfsetdash{}{0pt}%
\pgfpathmoveto{\pgfqpoint{3.560227in}{1.762500in}}%
\pgfpathlineto{\pgfqpoint{5.850000in}{1.762500in}}%
\pgfusepath{stroke}%
\end{pgfscope}%
\begin{pgfscope}%
\definecolor{textcolor}{rgb}{0.150000,0.150000,0.150000}%
\pgfsetstrokecolor{textcolor}%
\pgfsetfillcolor{textcolor}%
\pgftext[x=3.393560in, y=1.714275in, left, base]{\color{textcolor}\rmfamily\fontsize{10.000000}{12.000000}\selectfont \(\displaystyle {0}\)}%
\end{pgfscope}%
\begin{pgfscope}%
\pgfpathrectangle{\pgfqpoint{3.560227in}{1.325000in}}{\pgfqpoint{2.289773in}{0.875000in}}%
\pgfusepath{clip}%
\pgfsetroundcap%
\pgfsetroundjoin%
\pgfsetlinewidth{1.003750pt}%
\definecolor{currentstroke}{rgb}{1.000000,1.000000,1.000000}%
\pgfsetstrokecolor{currentstroke}%
\pgfsetdash{}{0pt}%
\pgfpathmoveto{\pgfqpoint{3.560227in}{2.199982in}}%
\pgfpathlineto{\pgfqpoint{5.850000in}{2.199982in}}%
\pgfusepath{stroke}%
\end{pgfscope}%
\begin{pgfscope}%
\definecolor{textcolor}{rgb}{0.150000,0.150000,0.150000}%
\pgfsetstrokecolor{textcolor}%
\pgfsetfillcolor{textcolor}%
\pgftext[x=3.324116in, y=2.151757in, left, base]{\color{textcolor}\rmfamily\fontsize{10.000000}{12.000000}\selectfont \(\displaystyle {10}\)}%
\end{pgfscope}%
\begin{pgfscope}%
\definecolor{textcolor}{rgb}{0.150000,0.150000,0.150000}%
\pgfsetstrokecolor{textcolor}%
\pgfsetfillcolor{textcolor}%
\pgftext[x=3.160535in,y=1.762500in,,bottom,rotate=90.000000]{\color{textcolor}\rmfamily\fontsize{11.000000}{13.200000}\selectfont Position}%
\end{pgfscope}%
\begin{pgfscope}%
\pgfpathrectangle{\pgfqpoint{3.560227in}{1.325000in}}{\pgfqpoint{2.289773in}{0.875000in}}%
\pgfusepath{clip}%
\pgfsetroundcap%
\pgfsetroundjoin%
\pgfsetlinewidth{1.756562pt}%
\definecolor{currentstroke}{rgb}{0.215686,0.494118,0.721569}%
\pgfsetstrokecolor{currentstroke}%
\pgfsetdash{}{0pt}%
\pgfpathmoveto{\pgfqpoint{3.664308in}{1.718752in}}%
\pgfpathlineto{\pgfqpoint{3.678894in}{1.707354in}}%
\pgfpathlineto{\pgfqpoint{3.697647in}{1.695643in}}%
\pgfpathlineto{\pgfqpoint{3.722651in}{1.682731in}}%
\pgfpathlineto{\pgfqpoint{3.753907in}{1.669110in}}%
\pgfpathlineto{\pgfqpoint{3.793497in}{1.654318in}}%
\pgfpathlineto{\pgfqpoint{3.843506in}{1.638142in}}%
\pgfpathlineto{\pgfqpoint{3.903933in}{1.621057in}}%
\pgfpathlineto{\pgfqpoint{3.976862in}{1.602863in}}%
\pgfpathlineto{\pgfqpoint{4.062294in}{1.583899in}}%
\pgfpathlineto{\pgfqpoint{4.164395in}{1.563598in}}%
\pgfpathlineto{\pgfqpoint{4.283165in}{1.542331in}}%
\pgfpathlineto{\pgfqpoint{4.420689in}{1.520028in}}%
\pgfpathlineto{\pgfqpoint{4.579050in}{1.496657in}}%
\pgfpathlineto{\pgfqpoint{4.760332in}{1.472206in}}%
\pgfpathlineto{\pgfqpoint{4.966617in}{1.446679in}}%
\pgfpathlineto{\pgfqpoint{5.199991in}{1.420088in}}%
\pgfpathlineto{\pgfqpoint{5.462537in}{1.392452in}}%
\pgfpathlineto{\pgfqpoint{5.745919in}{1.364773in}}%
\pgfpathlineto{\pgfqpoint{5.745919in}{1.364773in}}%
\pgfusepath{stroke}%
\end{pgfscope}%
\begin{pgfscope}%
\pgfpathrectangle{\pgfqpoint{3.560227in}{1.325000in}}{\pgfqpoint{2.289773in}{0.875000in}}%
\pgfusepath{clip}%
\pgfsetroundcap%
\pgfsetroundjoin%
\pgfsetlinewidth{1.756562pt}%
\definecolor{currentstroke}{rgb}{1.000000,0.498039,0.000000}%
\pgfsetstrokecolor{currentstroke}%
\pgfsetdash{}{0pt}%
\pgfpathmoveto{\pgfqpoint{3.664308in}{1.740626in}}%
\pgfpathlineto{\pgfqpoint{3.693480in}{1.731590in}}%
\pgfpathlineto{\pgfqpoint{3.730986in}{1.722673in}}%
\pgfpathlineto{\pgfqpoint{3.780995in}{1.713198in}}%
\pgfpathlineto{\pgfqpoint{3.849757in}{1.702560in}}%
\pgfpathlineto{\pgfqpoint{3.939356in}{1.691037in}}%
\pgfpathlineto{\pgfqpoint{4.056043in}{1.678365in}}%
\pgfpathlineto{\pgfqpoint{4.203985in}{1.664632in}}%
\pgfpathlineto{\pgfqpoint{4.387350in}{1.649915in}}%
\pgfpathlineto{\pgfqpoint{4.614473in}{1.634004in}}%
\pgfpathlineto{\pgfqpoint{4.889521in}{1.617049in}}%
\pgfpathlineto{\pgfqpoint{5.218745in}{1.599053in}}%
\pgfpathlineto{\pgfqpoint{5.610479in}{1.579937in}}%
\pgfpathlineto{\pgfqpoint{5.745919in}{1.573778in}}%
\pgfpathlineto{\pgfqpoint{5.745919in}{1.573778in}}%
\pgfusepath{stroke}%
\end{pgfscope}%
\begin{pgfscope}%
\pgfpathrectangle{\pgfqpoint{3.560227in}{1.325000in}}{\pgfqpoint{2.289773in}{0.875000in}}%
\pgfusepath{clip}%
\pgfsetroundcap%
\pgfsetroundjoin%
\pgfsetlinewidth{1.756562pt}%
\definecolor{currentstroke}{rgb}{0.301961,0.686275,0.290196}%
\pgfsetstrokecolor{currentstroke}%
\pgfsetdash{}{0pt}%
\pgfpathmoveto{\pgfqpoint{3.664308in}{1.762500in}}%
\pgfpathlineto{\pgfqpoint{5.745919in}{1.762500in}}%
\pgfpathlineto{\pgfqpoint{5.745919in}{1.762500in}}%
\pgfusepath{stroke}%
\end{pgfscope}%
\begin{pgfscope}%
\pgfpathrectangle{\pgfqpoint{3.560227in}{1.325000in}}{\pgfqpoint{2.289773in}{0.875000in}}%
\pgfusepath{clip}%
\pgfsetroundcap%
\pgfsetroundjoin%
\pgfsetlinewidth{1.756562pt}%
\definecolor{currentstroke}{rgb}{0.968627,0.505882,0.749020}%
\pgfsetstrokecolor{currentstroke}%
\pgfsetdash{}{0pt}%
\pgfpathmoveto{\pgfqpoint{3.664308in}{1.784374in}}%
\pgfpathlineto{\pgfqpoint{3.693480in}{1.793410in}}%
\pgfpathlineto{\pgfqpoint{3.730986in}{1.802327in}}%
\pgfpathlineto{\pgfqpoint{3.780995in}{1.811802in}}%
\pgfpathlineto{\pgfqpoint{3.849757in}{1.822440in}}%
\pgfpathlineto{\pgfqpoint{3.939356in}{1.833963in}}%
\pgfpathlineto{\pgfqpoint{4.056043in}{1.846635in}}%
\pgfpathlineto{\pgfqpoint{4.203985in}{1.860368in}}%
\pgfpathlineto{\pgfqpoint{4.387350in}{1.875085in}}%
\pgfpathlineto{\pgfqpoint{4.614473in}{1.890996in}}%
\pgfpathlineto{\pgfqpoint{4.889521in}{1.907951in}}%
\pgfpathlineto{\pgfqpoint{5.218745in}{1.925947in}}%
\pgfpathlineto{\pgfqpoint{5.610479in}{1.945063in}}%
\pgfpathlineto{\pgfqpoint{5.745919in}{1.951222in}}%
\pgfpathlineto{\pgfqpoint{5.745919in}{1.951222in}}%
\pgfusepath{stroke}%
\end{pgfscope}%
\begin{pgfscope}%
\pgfpathrectangle{\pgfqpoint{3.560227in}{1.325000in}}{\pgfqpoint{2.289773in}{0.875000in}}%
\pgfusepath{clip}%
\pgfsetroundcap%
\pgfsetroundjoin%
\pgfsetlinewidth{1.756562pt}%
\definecolor{currentstroke}{rgb}{0.650980,0.337255,0.156863}%
\pgfsetstrokecolor{currentstroke}%
\pgfsetdash{}{0pt}%
\pgfpathmoveto{\pgfqpoint{3.664308in}{1.806248in}}%
\pgfpathlineto{\pgfqpoint{3.678894in}{1.817646in}}%
\pgfpathlineto{\pgfqpoint{3.697647in}{1.829357in}}%
\pgfpathlineto{\pgfqpoint{3.722651in}{1.842269in}}%
\pgfpathlineto{\pgfqpoint{3.753907in}{1.855890in}}%
\pgfpathlineto{\pgfqpoint{3.793497in}{1.870682in}}%
\pgfpathlineto{\pgfqpoint{3.843506in}{1.886858in}}%
\pgfpathlineto{\pgfqpoint{3.903933in}{1.903943in}}%
\pgfpathlineto{\pgfqpoint{3.976862in}{1.922137in}}%
\pgfpathlineto{\pgfqpoint{4.062294in}{1.941101in}}%
\pgfpathlineto{\pgfqpoint{4.164395in}{1.961402in}}%
\pgfpathlineto{\pgfqpoint{4.283165in}{1.982669in}}%
\pgfpathlineto{\pgfqpoint{4.420689in}{2.004972in}}%
\pgfpathlineto{\pgfqpoint{4.579050in}{2.028343in}}%
\pgfpathlineto{\pgfqpoint{4.760332in}{2.052794in}}%
\pgfpathlineto{\pgfqpoint{4.966617in}{2.078321in}}%
\pgfpathlineto{\pgfqpoint{5.199991in}{2.104912in}}%
\pgfpathlineto{\pgfqpoint{5.462537in}{2.132548in}}%
\pgfpathlineto{\pgfqpoint{5.745919in}{2.160227in}}%
\pgfpathlineto{\pgfqpoint{5.745919in}{2.160227in}}%
\pgfusepath{stroke}%
\end{pgfscope}%
\begin{pgfscope}%
\pgfsetrectcap%
\pgfsetmiterjoin%
\pgfsetlinewidth{0.000000pt}%
\definecolor{currentstroke}{rgb}{1.000000,1.000000,1.000000}%
\pgfsetstrokecolor{currentstroke}%
\pgfsetdash{}{0pt}%
\pgfpathmoveto{\pgfqpoint{3.560227in}{1.325000in}}%
\pgfpathlineto{\pgfqpoint{3.560227in}{2.200000in}}%
\pgfusepath{}%
\end{pgfscope}%
\begin{pgfscope}%
\pgfsetrectcap%
\pgfsetmiterjoin%
\pgfsetlinewidth{0.000000pt}%
\definecolor{currentstroke}{rgb}{1.000000,1.000000,1.000000}%
\pgfsetstrokecolor{currentstroke}%
\pgfsetdash{}{0pt}%
\pgfpathmoveto{\pgfqpoint{5.850000in}{1.325000in}}%
\pgfpathlineto{\pgfqpoint{5.850000in}{2.200000in}}%
\pgfusepath{}%
\end{pgfscope}%
\begin{pgfscope}%
\pgfsetrectcap%
\pgfsetmiterjoin%
\pgfsetlinewidth{0.000000pt}%
\definecolor{currentstroke}{rgb}{1.000000,1.000000,1.000000}%
\pgfsetstrokecolor{currentstroke}%
\pgfsetdash{}{0pt}%
\pgfpathmoveto{\pgfqpoint{3.560227in}{1.325000in}}%
\pgfpathlineto{\pgfqpoint{5.850000in}{1.325000in}}%
\pgfusepath{}%
\end{pgfscope}%
\begin{pgfscope}%
\pgfsetrectcap%
\pgfsetmiterjoin%
\pgfsetlinewidth{0.000000pt}%
\definecolor{currentstroke}{rgb}{1.000000,1.000000,1.000000}%
\pgfsetstrokecolor{currentstroke}%
\pgfsetdash{}{0pt}%
\pgfpathmoveto{\pgfqpoint{3.560227in}{2.200000in}}%
\pgfpathlineto{\pgfqpoint{5.850000in}{2.200000in}}%
\pgfusepath{}%
\end{pgfscope}%
\begin{pgfscope}%
\pgfsetbuttcap%
\pgfsetmiterjoin%
\definecolor{currentfill}{rgb}{0.917647,0.917647,0.949020}%
\pgfsetfillcolor{currentfill}%
\pgfsetlinewidth{0.000000pt}%
\definecolor{currentstroke}{rgb}{0.000000,0.000000,0.000000}%
\pgfsetstrokecolor{currentstroke}%
\pgfsetstrokeopacity{0.000000}%
\pgfsetdash{}{0pt}%
\pgfpathmoveto{\pgfqpoint{0.812500in}{0.275000in}}%
\pgfpathlineto{\pgfqpoint{3.102273in}{0.275000in}}%
\pgfpathlineto{\pgfqpoint{3.102273in}{1.150000in}}%
\pgfpathlineto{\pgfqpoint{0.812500in}{1.150000in}}%
\pgfpathlineto{\pgfqpoint{0.812500in}{0.275000in}}%
\pgfpathclose%
\pgfusepath{fill}%
\end{pgfscope}%
\begin{pgfscope}%
\pgfpathrectangle{\pgfqpoint{0.812500in}{0.275000in}}{\pgfqpoint{2.289773in}{0.875000in}}%
\pgfusepath{clip}%
\pgfsetroundcap%
\pgfsetroundjoin%
\pgfsetlinewidth{1.003750pt}%
\definecolor{currentstroke}{rgb}{1.000000,1.000000,1.000000}%
\pgfsetstrokecolor{currentstroke}%
\pgfsetdash{}{0pt}%
\pgfpathmoveto{\pgfqpoint{0.916581in}{0.275000in}}%
\pgfpathlineto{\pgfqpoint{0.916581in}{1.150000in}}%
\pgfusepath{stroke}%
\end{pgfscope}%
\begin{pgfscope}%
\definecolor{textcolor}{rgb}{0.150000,0.150000,0.150000}%
\pgfsetstrokecolor{textcolor}%
\pgfsetfillcolor{textcolor}%
\pgftext[x=0.916581in,y=0.177778in,,top]{\color{textcolor}\rmfamily\fontsize{10.000000}{12.000000}\selectfont \(\displaystyle {0.0}\)}%
\end{pgfscope}%
\begin{pgfscope}%
\pgfpathrectangle{\pgfqpoint{0.812500in}{0.275000in}}{\pgfqpoint{2.289773in}{0.875000in}}%
\pgfusepath{clip}%
\pgfsetroundcap%
\pgfsetroundjoin%
\pgfsetlinewidth{1.003750pt}%
\definecolor{currentstroke}{rgb}{1.000000,1.000000,1.000000}%
\pgfsetstrokecolor{currentstroke}%
\pgfsetdash{}{0pt}%
\pgfpathmoveto{\pgfqpoint{1.436983in}{0.275000in}}%
\pgfpathlineto{\pgfqpoint{1.436983in}{1.150000in}}%
\pgfusepath{stroke}%
\end{pgfscope}%
\begin{pgfscope}%
\definecolor{textcolor}{rgb}{0.150000,0.150000,0.150000}%
\pgfsetstrokecolor{textcolor}%
\pgfsetfillcolor{textcolor}%
\pgftext[x=1.436983in,y=0.177778in,,top]{\color{textcolor}\rmfamily\fontsize{10.000000}{12.000000}\selectfont \(\displaystyle {2.5}\)}%
\end{pgfscope}%
\begin{pgfscope}%
\pgfpathrectangle{\pgfqpoint{0.812500in}{0.275000in}}{\pgfqpoint{2.289773in}{0.875000in}}%
\pgfusepath{clip}%
\pgfsetroundcap%
\pgfsetroundjoin%
\pgfsetlinewidth{1.003750pt}%
\definecolor{currentstroke}{rgb}{1.000000,1.000000,1.000000}%
\pgfsetstrokecolor{currentstroke}%
\pgfsetdash{}{0pt}%
\pgfpathmoveto{\pgfqpoint{1.957386in}{0.275000in}}%
\pgfpathlineto{\pgfqpoint{1.957386in}{1.150000in}}%
\pgfusepath{stroke}%
\end{pgfscope}%
\begin{pgfscope}%
\definecolor{textcolor}{rgb}{0.150000,0.150000,0.150000}%
\pgfsetstrokecolor{textcolor}%
\pgfsetfillcolor{textcolor}%
\pgftext[x=1.957386in,y=0.177778in,,top]{\color{textcolor}\rmfamily\fontsize{10.000000}{12.000000}\selectfont \(\displaystyle {5.0}\)}%
\end{pgfscope}%
\begin{pgfscope}%
\pgfpathrectangle{\pgfqpoint{0.812500in}{0.275000in}}{\pgfqpoint{2.289773in}{0.875000in}}%
\pgfusepath{clip}%
\pgfsetroundcap%
\pgfsetroundjoin%
\pgfsetlinewidth{1.003750pt}%
\definecolor{currentstroke}{rgb}{1.000000,1.000000,1.000000}%
\pgfsetstrokecolor{currentstroke}%
\pgfsetdash{}{0pt}%
\pgfpathmoveto{\pgfqpoint{2.477789in}{0.275000in}}%
\pgfpathlineto{\pgfqpoint{2.477789in}{1.150000in}}%
\pgfusepath{stroke}%
\end{pgfscope}%
\begin{pgfscope}%
\definecolor{textcolor}{rgb}{0.150000,0.150000,0.150000}%
\pgfsetstrokecolor{textcolor}%
\pgfsetfillcolor{textcolor}%
\pgftext[x=2.477789in,y=0.177778in,,top]{\color{textcolor}\rmfamily\fontsize{10.000000}{12.000000}\selectfont \(\displaystyle {7.5}\)}%
\end{pgfscope}%
\begin{pgfscope}%
\pgfpathrectangle{\pgfqpoint{0.812500in}{0.275000in}}{\pgfqpoint{2.289773in}{0.875000in}}%
\pgfusepath{clip}%
\pgfsetroundcap%
\pgfsetroundjoin%
\pgfsetlinewidth{1.003750pt}%
\definecolor{currentstroke}{rgb}{1.000000,1.000000,1.000000}%
\pgfsetstrokecolor{currentstroke}%
\pgfsetdash{}{0pt}%
\pgfpathmoveto{\pgfqpoint{2.998192in}{0.275000in}}%
\pgfpathlineto{\pgfqpoint{2.998192in}{1.150000in}}%
\pgfusepath{stroke}%
\end{pgfscope}%
\begin{pgfscope}%
\definecolor{textcolor}{rgb}{0.150000,0.150000,0.150000}%
\pgfsetstrokecolor{textcolor}%
\pgfsetfillcolor{textcolor}%
\pgftext[x=2.998192in,y=0.177778in,,top]{\color{textcolor}\rmfamily\fontsize{10.000000}{12.000000}\selectfont \(\displaystyle {10.0}\)}%
\end{pgfscope}%
\begin{pgfscope}%
\definecolor{textcolor}{rgb}{0.150000,0.150000,0.150000}%
\pgfsetstrokecolor{textcolor}%
\pgfsetfillcolor{textcolor}%
\pgftext[x=1.957386in,y=-0.001234in,,top]{\color{textcolor}\rmfamily\fontsize{11.000000}{13.200000}\selectfont time (\(\displaystyle t\))}%
\end{pgfscope}%
\begin{pgfscope}%
\pgfpathrectangle{\pgfqpoint{0.812500in}{0.275000in}}{\pgfqpoint{2.289773in}{0.875000in}}%
\pgfusepath{clip}%
\pgfsetroundcap%
\pgfsetroundjoin%
\pgfsetlinewidth{1.003750pt}%
\definecolor{currentstroke}{rgb}{1.000000,1.000000,1.000000}%
\pgfsetstrokecolor{currentstroke}%
\pgfsetdash{}{0pt}%
\pgfpathmoveto{\pgfqpoint{0.812500in}{0.384603in}}%
\pgfpathlineto{\pgfqpoint{3.102273in}{0.384603in}}%
\pgfusepath{stroke}%
\end{pgfscope}%
\begin{pgfscope}%
\definecolor{textcolor}{rgb}{0.150000,0.150000,0.150000}%
\pgfsetstrokecolor{textcolor}%
\pgfsetfillcolor{textcolor}%
\pgftext[x=0.468363in, y=0.336378in, left, base]{\color{textcolor}\rmfamily\fontsize{10.000000}{12.000000}\selectfont \(\displaystyle {\ensuremath{-}10}\)}%
\end{pgfscope}%
\begin{pgfscope}%
\pgfpathrectangle{\pgfqpoint{0.812500in}{0.275000in}}{\pgfqpoint{2.289773in}{0.875000in}}%
\pgfusepath{clip}%
\pgfsetroundcap%
\pgfsetroundjoin%
\pgfsetlinewidth{1.003750pt}%
\definecolor{currentstroke}{rgb}{1.000000,1.000000,1.000000}%
\pgfsetstrokecolor{currentstroke}%
\pgfsetdash{}{0pt}%
\pgfpathmoveto{\pgfqpoint{0.812500in}{0.712500in}}%
\pgfpathlineto{\pgfqpoint{3.102273in}{0.712500in}}%
\pgfusepath{stroke}%
\end{pgfscope}%
\begin{pgfscope}%
\definecolor{textcolor}{rgb}{0.150000,0.150000,0.150000}%
\pgfsetstrokecolor{textcolor}%
\pgfsetfillcolor{textcolor}%
\pgftext[x=0.645833in, y=0.664275in, left, base]{\color{textcolor}\rmfamily\fontsize{10.000000}{12.000000}\selectfont \(\displaystyle {0}\)}%
\end{pgfscope}%
\begin{pgfscope}%
\pgfpathrectangle{\pgfqpoint{0.812500in}{0.275000in}}{\pgfqpoint{2.289773in}{0.875000in}}%
\pgfusepath{clip}%
\pgfsetroundcap%
\pgfsetroundjoin%
\pgfsetlinewidth{1.003750pt}%
\definecolor{currentstroke}{rgb}{1.000000,1.000000,1.000000}%
\pgfsetstrokecolor{currentstroke}%
\pgfsetdash{}{0pt}%
\pgfpathmoveto{\pgfqpoint{0.812500in}{1.040397in}}%
\pgfpathlineto{\pgfqpoint{3.102273in}{1.040397in}}%
\pgfusepath{stroke}%
\end{pgfscope}%
\begin{pgfscope}%
\definecolor{textcolor}{rgb}{0.150000,0.150000,0.150000}%
\pgfsetstrokecolor{textcolor}%
\pgfsetfillcolor{textcolor}%
\pgftext[x=0.576388in, y=0.992171in, left, base]{\color{textcolor}\rmfamily\fontsize{10.000000}{12.000000}\selectfont \(\displaystyle {10}\)}%
\end{pgfscope}%
\begin{pgfscope}%
\definecolor{textcolor}{rgb}{0.150000,0.150000,0.150000}%
\pgfsetstrokecolor{textcolor}%
\pgfsetfillcolor{textcolor}%
\pgftext[x=0.412808in,y=0.712500in,,bottom,rotate=90.000000]{\color{textcolor}\rmfamily\fontsize{11.000000}{13.200000}\selectfont Position}%
\end{pgfscope}%
\begin{pgfscope}%
\pgfpathrectangle{\pgfqpoint{0.812500in}{0.275000in}}{\pgfqpoint{2.289773in}{0.875000in}}%
\pgfusepath{clip}%
\pgfsetroundcap%
\pgfsetroundjoin%
\pgfsetlinewidth{1.756562pt}%
\definecolor{currentstroke}{rgb}{0.215686,0.494118,0.721569}%
\pgfsetstrokecolor{currentstroke}%
\pgfsetdash{}{0pt}%
\pgfpathmoveto{\pgfqpoint{0.916581in}{0.630526in}}%
\pgfpathlineto{\pgfqpoint{0.947836in}{0.617236in}}%
\pgfpathlineto{\pgfqpoint{0.987426in}{0.602908in}}%
\pgfpathlineto{\pgfqpoint{1.037435in}{0.587254in}}%
\pgfpathlineto{\pgfqpoint{1.097862in}{0.570699in}}%
\pgfpathlineto{\pgfqpoint{1.172875in}{0.552561in}}%
\pgfpathlineto{\pgfqpoint{1.260390in}{0.533740in}}%
\pgfpathlineto{\pgfqpoint{1.364575in}{0.513665in}}%
\pgfpathlineto{\pgfqpoint{1.485429in}{0.492675in}}%
\pgfpathlineto{\pgfqpoint{1.627121in}{0.470383in}}%
\pgfpathlineto{\pgfqpoint{1.789649in}{0.447116in}}%
\pgfpathlineto{\pgfqpoint{1.975098in}{0.422847in}}%
\pgfpathlineto{\pgfqpoint{2.187635in}{0.397327in}}%
\pgfpathlineto{\pgfqpoint{2.427260in}{0.370836in}}%
\pgfpathlineto{\pgfqpoint{2.698140in}{0.343173in}}%
\pgfpathlineto{\pgfqpoint{2.998192in}{0.314773in}}%
\pgfpathlineto{\pgfqpoint{2.998192in}{0.314773in}}%
\pgfusepath{stroke}%
\end{pgfscope}%
\begin{pgfscope}%
\pgfpathrectangle{\pgfqpoint{0.812500in}{0.275000in}}{\pgfqpoint{2.289773in}{0.875000in}}%
\pgfusepath{clip}%
\pgfsetroundcap%
\pgfsetroundjoin%
\pgfsetlinewidth{1.756562pt}%
\definecolor{currentstroke}{rgb}{1.000000,0.498039,0.000000}%
\pgfsetstrokecolor{currentstroke}%
\pgfsetdash{}{0pt}%
\pgfpathmoveto{\pgfqpoint{0.916581in}{0.657851in}}%
\pgfpathlineto{\pgfqpoint{0.979091in}{0.643831in}}%
\pgfpathlineto{\pgfqpoint{1.047853in}{0.630811in}}%
\pgfpathlineto{\pgfqpoint{1.133285in}{0.616972in}}%
\pgfpathlineto{\pgfqpoint{1.239553in}{0.602104in}}%
\pgfpathlineto{\pgfqpoint{1.368742in}{0.586344in}}%
\pgfpathlineto{\pgfqpoint{1.525020in}{0.569570in}}%
\pgfpathlineto{\pgfqpoint{1.712552in}{0.551728in}}%
\pgfpathlineto{\pgfqpoint{1.935508in}{0.532802in}}%
\pgfpathlineto{\pgfqpoint{2.198053in}{0.512800in}}%
\pgfpathlineto{\pgfqpoint{2.504356in}{0.491743in}}%
\pgfpathlineto{\pgfqpoint{2.860668in}{0.469532in}}%
\pgfpathlineto{\pgfqpoint{2.998192in}{0.461484in}}%
\pgfpathlineto{\pgfqpoint{2.998192in}{0.461484in}}%
\pgfusepath{stroke}%
\end{pgfscope}%
\begin{pgfscope}%
\pgfpathrectangle{\pgfqpoint{0.812500in}{0.275000in}}{\pgfqpoint{2.289773in}{0.875000in}}%
\pgfusepath{clip}%
\pgfsetroundcap%
\pgfsetroundjoin%
\pgfsetlinewidth{1.756562pt}%
\definecolor{currentstroke}{rgb}{0.301961,0.686275,0.290196}%
\pgfsetstrokecolor{currentstroke}%
\pgfsetdash{}{0pt}%
\pgfpathmoveto{\pgfqpoint{0.916581in}{0.685175in}}%
\pgfpathlineto{\pgfqpoint{1.020765in}{0.674799in}}%
\pgfpathlineto{\pgfqpoint{1.141620in}{0.665181in}}%
\pgfpathlineto{\pgfqpoint{1.302064in}{0.654745in}}%
\pgfpathlineto{\pgfqpoint{1.514601in}{0.643265in}}%
\pgfpathlineto{\pgfqpoint{1.787565in}{0.630838in}}%
\pgfpathlineto{\pgfqpoint{2.135542in}{0.617314in}}%
\pgfpathlineto{\pgfqpoint{2.568951in}{0.602772in}}%
\pgfpathlineto{\pgfqpoint{2.998192in}{0.590059in}}%
\pgfpathlineto{\pgfqpoint{2.998192in}{0.590059in}}%
\pgfusepath{stroke}%
\end{pgfscope}%
\begin{pgfscope}%
\pgfpathrectangle{\pgfqpoint{0.812500in}{0.275000in}}{\pgfqpoint{2.289773in}{0.875000in}}%
\pgfusepath{clip}%
\pgfsetroundcap%
\pgfsetroundjoin%
\pgfsetlinewidth{1.756562pt}%
\definecolor{currentstroke}{rgb}{0.968627,0.505882,0.749020}%
\pgfsetstrokecolor{currentstroke}%
\pgfsetdash{}{0pt}%
\pgfpathmoveto{\pgfqpoint{0.916581in}{0.712500in}}%
\pgfpathlineto{\pgfqpoint{2.998192in}{0.712500in}}%
\pgfpathlineto{\pgfqpoint{2.998192in}{0.712500in}}%
\pgfusepath{stroke}%
\end{pgfscope}%
\begin{pgfscope}%
\pgfpathrectangle{\pgfqpoint{0.812500in}{0.275000in}}{\pgfqpoint{2.289773in}{0.875000in}}%
\pgfusepath{clip}%
\pgfsetroundcap%
\pgfsetroundjoin%
\pgfsetlinewidth{1.756562pt}%
\definecolor{currentstroke}{rgb}{0.650980,0.337255,0.156863}%
\pgfsetstrokecolor{currentstroke}%
\pgfsetdash{}{0pt}%
\pgfpathmoveto{\pgfqpoint{0.916581in}{0.739825in}}%
\pgfpathlineto{\pgfqpoint{1.020765in}{0.750201in}}%
\pgfpathlineto{\pgfqpoint{1.141620in}{0.759819in}}%
\pgfpathlineto{\pgfqpoint{1.302064in}{0.770255in}}%
\pgfpathlineto{\pgfqpoint{1.514601in}{0.781735in}}%
\pgfpathlineto{\pgfqpoint{1.787565in}{0.794162in}}%
\pgfpathlineto{\pgfqpoint{2.135542in}{0.807686in}}%
\pgfpathlineto{\pgfqpoint{2.568951in}{0.822228in}}%
\pgfpathlineto{\pgfqpoint{2.998192in}{0.834941in}}%
\pgfpathlineto{\pgfqpoint{2.998192in}{0.834941in}}%
\pgfusepath{stroke}%
\end{pgfscope}%
\begin{pgfscope}%
\pgfpathrectangle{\pgfqpoint{0.812500in}{0.275000in}}{\pgfqpoint{2.289773in}{0.875000in}}%
\pgfusepath{clip}%
\pgfsetroundcap%
\pgfsetroundjoin%
\pgfsetlinewidth{1.756562pt}%
\definecolor{currentstroke}{rgb}{0.596078,0.305882,0.639216}%
\pgfsetstrokecolor{currentstroke}%
\pgfsetdash{}{0pt}%
\pgfpathmoveto{\pgfqpoint{0.916581in}{0.767149in}}%
\pgfpathlineto{\pgfqpoint{0.979091in}{0.781169in}}%
\pgfpathlineto{\pgfqpoint{1.047853in}{0.794189in}}%
\pgfpathlineto{\pgfqpoint{1.133285in}{0.808028in}}%
\pgfpathlineto{\pgfqpoint{1.239553in}{0.822896in}}%
\pgfpathlineto{\pgfqpoint{1.368742in}{0.838656in}}%
\pgfpathlineto{\pgfqpoint{1.525020in}{0.855430in}}%
\pgfpathlineto{\pgfqpoint{1.712552in}{0.873272in}}%
\pgfpathlineto{\pgfqpoint{1.935508in}{0.892198in}}%
\pgfpathlineto{\pgfqpoint{2.198053in}{0.912200in}}%
\pgfpathlineto{\pgfqpoint{2.504356in}{0.933257in}}%
\pgfpathlineto{\pgfqpoint{2.860668in}{0.955468in}}%
\pgfpathlineto{\pgfqpoint{2.998192in}{0.963516in}}%
\pgfpathlineto{\pgfqpoint{2.998192in}{0.963516in}}%
\pgfusepath{stroke}%
\end{pgfscope}%
\begin{pgfscope}%
\pgfpathrectangle{\pgfqpoint{0.812500in}{0.275000in}}{\pgfqpoint{2.289773in}{0.875000in}}%
\pgfusepath{clip}%
\pgfsetroundcap%
\pgfsetroundjoin%
\pgfsetlinewidth{1.756562pt}%
\definecolor{currentstroke}{rgb}{0.600000,0.600000,0.600000}%
\pgfsetstrokecolor{currentstroke}%
\pgfsetdash{}{0pt}%
\pgfpathmoveto{\pgfqpoint{0.916581in}{0.794474in}}%
\pgfpathlineto{\pgfqpoint{0.947836in}{0.807764in}}%
\pgfpathlineto{\pgfqpoint{0.987426in}{0.822092in}}%
\pgfpathlineto{\pgfqpoint{1.037435in}{0.837746in}}%
\pgfpathlineto{\pgfqpoint{1.097862in}{0.854301in}}%
\pgfpathlineto{\pgfqpoint{1.172875in}{0.872439in}}%
\pgfpathlineto{\pgfqpoint{1.260390in}{0.891260in}}%
\pgfpathlineto{\pgfqpoint{1.364575in}{0.911335in}}%
\pgfpathlineto{\pgfqpoint{1.485429in}{0.932325in}}%
\pgfpathlineto{\pgfqpoint{1.627121in}{0.954617in}}%
\pgfpathlineto{\pgfqpoint{1.789649in}{0.977884in}}%
\pgfpathlineto{\pgfqpoint{1.975098in}{1.002153in}}%
\pgfpathlineto{\pgfqpoint{2.187635in}{1.027673in}}%
\pgfpathlineto{\pgfqpoint{2.427260in}{1.054164in}}%
\pgfpathlineto{\pgfqpoint{2.698140in}{1.081827in}}%
\pgfpathlineto{\pgfqpoint{2.998192in}{1.110227in}}%
\pgfpathlineto{\pgfqpoint{2.998192in}{1.110227in}}%
\pgfusepath{stroke}%
\end{pgfscope}%
\begin{pgfscope}%
\pgfsetrectcap%
\pgfsetmiterjoin%
\pgfsetlinewidth{0.000000pt}%
\definecolor{currentstroke}{rgb}{1.000000,1.000000,1.000000}%
\pgfsetstrokecolor{currentstroke}%
\pgfsetdash{}{0pt}%
\pgfpathmoveto{\pgfqpoint{0.812500in}{0.275000in}}%
\pgfpathlineto{\pgfqpoint{0.812500in}{1.150000in}}%
\pgfusepath{}%
\end{pgfscope}%
\begin{pgfscope}%
\pgfsetrectcap%
\pgfsetmiterjoin%
\pgfsetlinewidth{0.000000pt}%
\definecolor{currentstroke}{rgb}{1.000000,1.000000,1.000000}%
\pgfsetstrokecolor{currentstroke}%
\pgfsetdash{}{0pt}%
\pgfpathmoveto{\pgfqpoint{3.102273in}{0.275000in}}%
\pgfpathlineto{\pgfqpoint{3.102273in}{1.150000in}}%
\pgfusepath{}%
\end{pgfscope}%
\begin{pgfscope}%
\pgfsetrectcap%
\pgfsetmiterjoin%
\pgfsetlinewidth{0.000000pt}%
\definecolor{currentstroke}{rgb}{1.000000,1.000000,1.000000}%
\pgfsetstrokecolor{currentstroke}%
\pgfsetdash{}{0pt}%
\pgfpathmoveto{\pgfqpoint{0.812500in}{0.275000in}}%
\pgfpathlineto{\pgfqpoint{3.102273in}{0.275000in}}%
\pgfusepath{}%
\end{pgfscope}%
\begin{pgfscope}%
\pgfsetrectcap%
\pgfsetmiterjoin%
\pgfsetlinewidth{0.000000pt}%
\definecolor{currentstroke}{rgb}{1.000000,1.000000,1.000000}%
\pgfsetstrokecolor{currentstroke}%
\pgfsetdash{}{0pt}%
\pgfpathmoveto{\pgfqpoint{0.812500in}{1.150000in}}%
\pgfpathlineto{\pgfqpoint{3.102273in}{1.150000in}}%
\pgfusepath{}%
\end{pgfscope}%
\begin{pgfscope}%
\pgfsetbuttcap%
\pgfsetmiterjoin%
\definecolor{currentfill}{rgb}{0.917647,0.917647,0.949020}%
\pgfsetfillcolor{currentfill}%
\pgfsetlinewidth{0.000000pt}%
\definecolor{currentstroke}{rgb}{0.000000,0.000000,0.000000}%
\pgfsetstrokecolor{currentstroke}%
\pgfsetstrokeopacity{0.000000}%
\pgfsetdash{}{0pt}%
\pgfpathmoveto{\pgfqpoint{3.560227in}{0.275000in}}%
\pgfpathlineto{\pgfqpoint{5.850000in}{0.275000in}}%
\pgfpathlineto{\pgfqpoint{5.850000in}{1.150000in}}%
\pgfpathlineto{\pgfqpoint{3.560227in}{1.150000in}}%
\pgfpathlineto{\pgfqpoint{3.560227in}{0.275000in}}%
\pgfpathclose%
\pgfusepath{fill}%
\end{pgfscope}%
\begin{pgfscope}%
\pgfpathrectangle{\pgfqpoint{3.560227in}{0.275000in}}{\pgfqpoint{2.289773in}{0.875000in}}%
\pgfusepath{clip}%
\pgfsetroundcap%
\pgfsetroundjoin%
\pgfsetlinewidth{1.003750pt}%
\definecolor{currentstroke}{rgb}{1.000000,1.000000,1.000000}%
\pgfsetstrokecolor{currentstroke}%
\pgfsetdash{}{0pt}%
\pgfpathmoveto{\pgfqpoint{3.664308in}{0.275000in}}%
\pgfpathlineto{\pgfqpoint{3.664308in}{1.150000in}}%
\pgfusepath{stroke}%
\end{pgfscope}%
\begin{pgfscope}%
\definecolor{textcolor}{rgb}{0.150000,0.150000,0.150000}%
\pgfsetstrokecolor{textcolor}%
\pgfsetfillcolor{textcolor}%
\pgftext[x=3.664308in,y=0.177778in,,top]{\color{textcolor}\rmfamily\fontsize{10.000000}{12.000000}\selectfont \(\displaystyle {0.0}\)}%
\end{pgfscope}%
\begin{pgfscope}%
\pgfpathrectangle{\pgfqpoint{3.560227in}{0.275000in}}{\pgfqpoint{2.289773in}{0.875000in}}%
\pgfusepath{clip}%
\pgfsetroundcap%
\pgfsetroundjoin%
\pgfsetlinewidth{1.003750pt}%
\definecolor{currentstroke}{rgb}{1.000000,1.000000,1.000000}%
\pgfsetstrokecolor{currentstroke}%
\pgfsetdash{}{0pt}%
\pgfpathmoveto{\pgfqpoint{4.184711in}{0.275000in}}%
\pgfpathlineto{\pgfqpoint{4.184711in}{1.150000in}}%
\pgfusepath{stroke}%
\end{pgfscope}%
\begin{pgfscope}%
\definecolor{textcolor}{rgb}{0.150000,0.150000,0.150000}%
\pgfsetstrokecolor{textcolor}%
\pgfsetfillcolor{textcolor}%
\pgftext[x=4.184711in,y=0.177778in,,top]{\color{textcolor}\rmfamily\fontsize{10.000000}{12.000000}\selectfont \(\displaystyle {2.5}\)}%
\end{pgfscope}%
\begin{pgfscope}%
\pgfpathrectangle{\pgfqpoint{3.560227in}{0.275000in}}{\pgfqpoint{2.289773in}{0.875000in}}%
\pgfusepath{clip}%
\pgfsetroundcap%
\pgfsetroundjoin%
\pgfsetlinewidth{1.003750pt}%
\definecolor{currentstroke}{rgb}{1.000000,1.000000,1.000000}%
\pgfsetstrokecolor{currentstroke}%
\pgfsetdash{}{0pt}%
\pgfpathmoveto{\pgfqpoint{4.705114in}{0.275000in}}%
\pgfpathlineto{\pgfqpoint{4.705114in}{1.150000in}}%
\pgfusepath{stroke}%
\end{pgfscope}%
\begin{pgfscope}%
\definecolor{textcolor}{rgb}{0.150000,0.150000,0.150000}%
\pgfsetstrokecolor{textcolor}%
\pgfsetfillcolor{textcolor}%
\pgftext[x=4.705114in,y=0.177778in,,top]{\color{textcolor}\rmfamily\fontsize{10.000000}{12.000000}\selectfont \(\displaystyle {5.0}\)}%
\end{pgfscope}%
\begin{pgfscope}%
\pgfpathrectangle{\pgfqpoint{3.560227in}{0.275000in}}{\pgfqpoint{2.289773in}{0.875000in}}%
\pgfusepath{clip}%
\pgfsetroundcap%
\pgfsetroundjoin%
\pgfsetlinewidth{1.003750pt}%
\definecolor{currentstroke}{rgb}{1.000000,1.000000,1.000000}%
\pgfsetstrokecolor{currentstroke}%
\pgfsetdash{}{0pt}%
\pgfpathmoveto{\pgfqpoint{5.225517in}{0.275000in}}%
\pgfpathlineto{\pgfqpoint{5.225517in}{1.150000in}}%
\pgfusepath{stroke}%
\end{pgfscope}%
\begin{pgfscope}%
\definecolor{textcolor}{rgb}{0.150000,0.150000,0.150000}%
\pgfsetstrokecolor{textcolor}%
\pgfsetfillcolor{textcolor}%
\pgftext[x=5.225517in,y=0.177778in,,top]{\color{textcolor}\rmfamily\fontsize{10.000000}{12.000000}\selectfont \(\displaystyle {7.5}\)}%
\end{pgfscope}%
\begin{pgfscope}%
\pgfpathrectangle{\pgfqpoint{3.560227in}{0.275000in}}{\pgfqpoint{2.289773in}{0.875000in}}%
\pgfusepath{clip}%
\pgfsetroundcap%
\pgfsetroundjoin%
\pgfsetlinewidth{1.003750pt}%
\definecolor{currentstroke}{rgb}{1.000000,1.000000,1.000000}%
\pgfsetstrokecolor{currentstroke}%
\pgfsetdash{}{0pt}%
\pgfpathmoveto{\pgfqpoint{5.745919in}{0.275000in}}%
\pgfpathlineto{\pgfqpoint{5.745919in}{1.150000in}}%
\pgfusepath{stroke}%
\end{pgfscope}%
\begin{pgfscope}%
\definecolor{textcolor}{rgb}{0.150000,0.150000,0.150000}%
\pgfsetstrokecolor{textcolor}%
\pgfsetfillcolor{textcolor}%
\pgftext[x=5.745919in,y=0.177778in,,top]{\color{textcolor}\rmfamily\fontsize{10.000000}{12.000000}\selectfont \(\displaystyle {10.0}\)}%
\end{pgfscope}%
\begin{pgfscope}%
\definecolor{textcolor}{rgb}{0.150000,0.150000,0.150000}%
\pgfsetstrokecolor{textcolor}%
\pgfsetfillcolor{textcolor}%
\pgftext[x=4.705114in,y=-0.001234in,,top]{\color{textcolor}\rmfamily\fontsize{11.000000}{13.200000}\selectfont time (\(\displaystyle t\))}%
\end{pgfscope}%
\begin{pgfscope}%
\pgfpathrectangle{\pgfqpoint{3.560227in}{0.275000in}}{\pgfqpoint{2.289773in}{0.875000in}}%
\pgfusepath{clip}%
\pgfsetroundcap%
\pgfsetroundjoin%
\pgfsetlinewidth{1.003750pt}%
\definecolor{currentstroke}{rgb}{1.000000,1.000000,1.000000}%
\pgfsetstrokecolor{currentstroke}%
\pgfsetdash{}{0pt}%
\pgfpathmoveto{\pgfqpoint{3.560227in}{0.438214in}}%
\pgfpathlineto{\pgfqpoint{5.850000in}{0.438214in}}%
\pgfusepath{stroke}%
\end{pgfscope}%
\begin{pgfscope}%
\definecolor{textcolor}{rgb}{0.150000,0.150000,0.150000}%
\pgfsetstrokecolor{textcolor}%
\pgfsetfillcolor{textcolor}%
\pgftext[x=3.216091in, y=0.389989in, left, base]{\color{textcolor}\rmfamily\fontsize{10.000000}{12.000000}\selectfont \(\displaystyle {\ensuremath{-}10}\)}%
\end{pgfscope}%
\begin{pgfscope}%
\pgfpathrectangle{\pgfqpoint{3.560227in}{0.275000in}}{\pgfqpoint{2.289773in}{0.875000in}}%
\pgfusepath{clip}%
\pgfsetroundcap%
\pgfsetroundjoin%
\pgfsetlinewidth{1.003750pt}%
\definecolor{currentstroke}{rgb}{1.000000,1.000000,1.000000}%
\pgfsetstrokecolor{currentstroke}%
\pgfsetdash{}{0pt}%
\pgfpathmoveto{\pgfqpoint{3.560227in}{0.712500in}}%
\pgfpathlineto{\pgfqpoint{5.850000in}{0.712500in}}%
\pgfusepath{stroke}%
\end{pgfscope}%
\begin{pgfscope}%
\definecolor{textcolor}{rgb}{0.150000,0.150000,0.150000}%
\pgfsetstrokecolor{textcolor}%
\pgfsetfillcolor{textcolor}%
\pgftext[x=3.393560in, y=0.664275in, left, base]{\color{textcolor}\rmfamily\fontsize{10.000000}{12.000000}\selectfont \(\displaystyle {0}\)}%
\end{pgfscope}%
\begin{pgfscope}%
\pgfpathrectangle{\pgfqpoint{3.560227in}{0.275000in}}{\pgfqpoint{2.289773in}{0.875000in}}%
\pgfusepath{clip}%
\pgfsetroundcap%
\pgfsetroundjoin%
\pgfsetlinewidth{1.003750pt}%
\definecolor{currentstroke}{rgb}{1.000000,1.000000,1.000000}%
\pgfsetstrokecolor{currentstroke}%
\pgfsetdash{}{0pt}%
\pgfpathmoveto{\pgfqpoint{3.560227in}{0.986786in}}%
\pgfpathlineto{\pgfqpoint{5.850000in}{0.986786in}}%
\pgfusepath{stroke}%
\end{pgfscope}%
\begin{pgfscope}%
\definecolor{textcolor}{rgb}{0.150000,0.150000,0.150000}%
\pgfsetstrokecolor{textcolor}%
\pgfsetfillcolor{textcolor}%
\pgftext[x=3.324116in, y=0.938560in, left, base]{\color{textcolor}\rmfamily\fontsize{10.000000}{12.000000}\selectfont \(\displaystyle {10}\)}%
\end{pgfscope}%
\begin{pgfscope}%
\definecolor{textcolor}{rgb}{0.150000,0.150000,0.150000}%
\pgfsetstrokecolor{textcolor}%
\pgfsetfillcolor{textcolor}%
\pgftext[x=3.160535in,y=0.712500in,,bottom,rotate=90.000000]{\color{textcolor}\rmfamily\fontsize{11.000000}{13.200000}\selectfont Position}%
\end{pgfscope}%
\begin{pgfscope}%
\pgfpathrectangle{\pgfqpoint{3.560227in}{0.275000in}}{\pgfqpoint{2.289773in}{0.875000in}}%
\pgfusepath{clip}%
\pgfsetroundcap%
\pgfsetroundjoin%
\pgfsetlinewidth{1.756562pt}%
\definecolor{currentstroke}{rgb}{0.215686,0.494118,0.721569}%
\pgfsetstrokecolor{currentstroke}%
\pgfsetdash{}{0pt}%
\pgfpathmoveto{\pgfqpoint{3.664308in}{0.643929in}}%
\pgfpathlineto{\pgfqpoint{3.687208in}{0.632073in}}%
\pgfpathlineto{\pgfqpoint{3.718019in}{0.618735in}}%
\pgfpathlineto{\pgfqpoint{3.757365in}{0.604211in}}%
\pgfpathlineto{\pgfqpoint{3.806496in}{0.588525in}}%
\pgfpathlineto{\pgfqpoint{3.866661in}{0.571724in}}%
\pgfpathlineto{\pgfqpoint{3.939108in}{0.553866in}}%
\pgfpathlineto{\pgfqpoint{4.025504in}{0.534921in}}%
\pgfpathlineto{\pgfqpoint{4.127513in}{0.514888in}}%
\pgfpathlineto{\pgfqpoint{4.246801in}{0.493784in}}%
\pgfpathlineto{\pgfqpoint{4.385242in}{0.471605in}}%
\pgfpathlineto{\pgfqpoint{4.544709in}{0.448358in}}%
\pgfpathlineto{\pgfqpoint{4.727285in}{0.424036in}}%
\pgfpathlineto{\pgfqpoint{4.935259in}{0.398621in}}%
\pgfpathlineto{\pgfqpoint{5.170713in}{0.372131in}}%
\pgfpathlineto{\pgfqpoint{5.436145in}{0.344550in}}%
\pgfpathlineto{\pgfqpoint{5.733845in}{0.315892in}}%
\pgfpathlineto{\pgfqpoint{5.745919in}{0.314773in}}%
\pgfpathlineto{\pgfqpoint{5.745919in}{0.314773in}}%
\pgfusepath{stroke}%
\end{pgfscope}%
\begin{pgfscope}%
\pgfpathrectangle{\pgfqpoint{3.560227in}{0.275000in}}{\pgfqpoint{2.289773in}{0.875000in}}%
\pgfusepath{clip}%
\pgfsetroundcap%
\pgfsetroundjoin%
\pgfsetlinewidth{1.756562pt}%
\definecolor{currentstroke}{rgb}{1.000000,0.498039,0.000000}%
\pgfsetstrokecolor{currentstroke}%
\pgfsetdash{}{0pt}%
\pgfpathmoveto{\pgfqpoint{3.664308in}{0.661071in}}%
\pgfpathlineto{\pgfqpoint{3.707193in}{0.647837in}}%
\pgfpathlineto{\pgfqpoint{3.756324in}{0.635147in}}%
\pgfpathlineto{\pgfqpoint{3.817738in}{0.621666in}}%
\pgfpathlineto{\pgfqpoint{3.894349in}{0.607205in}}%
\pgfpathlineto{\pgfqpoint{3.988655in}{0.591740in}}%
\pgfpathlineto{\pgfqpoint{4.103572in}{0.575219in}}%
\pgfpathlineto{\pgfqpoint{4.241805in}{0.557659in}}%
\pgfpathlineto{\pgfqpoint{4.406685in}{0.539018in}}%
\pgfpathlineto{\pgfqpoint{4.601335in}{0.519311in}}%
\pgfpathlineto{\pgfqpoint{4.829086in}{0.498543in}}%
\pgfpathlineto{\pgfqpoint{5.093893in}{0.476684in}}%
\pgfpathlineto{\pgfqpoint{5.399296in}{0.453756in}}%
\pgfpathlineto{\pgfqpoint{5.745919in}{0.429978in}}%
\pgfpathlineto{\pgfqpoint{5.745919in}{0.429978in}}%
\pgfusepath{stroke}%
\end{pgfscope}%
\begin{pgfscope}%
\pgfpathrectangle{\pgfqpoint{3.560227in}{0.275000in}}{\pgfqpoint{2.289773in}{0.875000in}}%
\pgfusepath{clip}%
\pgfsetroundcap%
\pgfsetroundjoin%
\pgfsetlinewidth{1.756562pt}%
\definecolor{currentstroke}{rgb}{0.301961,0.686275,0.290196}%
\pgfsetstrokecolor{currentstroke}%
\pgfsetdash{}{0pt}%
\pgfpathmoveto{\pgfqpoint{3.664308in}{0.678214in}}%
\pgfpathlineto{\pgfqpoint{3.728428in}{0.666503in}}%
\pgfpathlineto{\pgfqpoint{3.800667in}{0.655749in}}%
\pgfpathlineto{\pgfqpoint{3.893933in}{0.644216in}}%
\pgfpathlineto{\pgfqpoint{4.013845in}{0.631725in}}%
\pgfpathlineto{\pgfqpoint{4.165818in}{0.618220in}}%
\pgfpathlineto{\pgfqpoint{4.355472in}{0.603680in}}%
\pgfpathlineto{\pgfqpoint{4.589052in}{0.588079in}}%
\pgfpathlineto{\pgfqpoint{4.873012in}{0.571414in}}%
\pgfpathlineto{\pgfqpoint{5.214223in}{0.553684in}}%
\pgfpathlineto{\pgfqpoint{5.620178in}{0.534878in}}%
\pgfpathlineto{\pgfqpoint{5.745919in}{0.529444in}}%
\pgfpathlineto{\pgfqpoint{5.745919in}{0.529444in}}%
\pgfusepath{stroke}%
\end{pgfscope}%
\begin{pgfscope}%
\pgfpathrectangle{\pgfqpoint{3.560227in}{0.275000in}}{\pgfqpoint{2.289773in}{0.875000in}}%
\pgfusepath{clip}%
\pgfsetroundcap%
\pgfsetroundjoin%
\pgfsetlinewidth{1.756562pt}%
\definecolor{currentstroke}{rgb}{0.968627,0.505882,0.749020}%
\pgfsetstrokecolor{currentstroke}%
\pgfsetdash{}{0pt}%
\pgfpathmoveto{\pgfqpoint{3.664308in}{0.695357in}}%
\pgfpathlineto{\pgfqpoint{3.763402in}{0.687071in}}%
\pgfpathlineto{\pgfqpoint{3.886854in}{0.679224in}}%
\pgfpathlineto{\pgfqpoint{4.061103in}{0.670506in}}%
\pgfpathlineto{\pgfqpoint{4.302177in}{0.660790in}}%
\pgfpathlineto{\pgfqpoint{4.626733in}{0.650042in}}%
\pgfpathlineto{\pgfqpoint{5.053090in}{0.638246in}}%
\pgfpathlineto{\pgfqpoint{5.601441in}{0.625390in}}%
\pgfpathlineto{\pgfqpoint{5.745919in}{0.622308in}}%
\pgfpathlineto{\pgfqpoint{5.745919in}{0.622308in}}%
\pgfusepath{stroke}%
\end{pgfscope}%
\begin{pgfscope}%
\pgfpathrectangle{\pgfqpoint{3.560227in}{0.275000in}}{\pgfqpoint{2.289773in}{0.875000in}}%
\pgfusepath{clip}%
\pgfsetroundcap%
\pgfsetroundjoin%
\pgfsetlinewidth{1.756562pt}%
\definecolor{currentstroke}{rgb}{0.650980,0.337255,0.156863}%
\pgfsetstrokecolor{currentstroke}%
\pgfsetdash{}{0pt}%
\pgfpathmoveto{\pgfqpoint{3.664308in}{0.712500in}}%
\pgfpathlineto{\pgfqpoint{5.745919in}{0.712500in}}%
\pgfpathlineto{\pgfqpoint{5.745919in}{0.712500in}}%
\pgfusepath{stroke}%
\end{pgfscope}%
\begin{pgfscope}%
\pgfpathrectangle{\pgfqpoint{3.560227in}{0.275000in}}{\pgfqpoint{2.289773in}{0.875000in}}%
\pgfusepath{clip}%
\pgfsetroundcap%
\pgfsetroundjoin%
\pgfsetlinewidth{1.756562pt}%
\definecolor{currentstroke}{rgb}{0.596078,0.305882,0.639216}%
\pgfsetstrokecolor{currentstroke}%
\pgfsetdash{}{0pt}%
\pgfpathmoveto{\pgfqpoint{3.664308in}{0.729643in}}%
\pgfpathlineto{\pgfqpoint{3.763402in}{0.737929in}}%
\pgfpathlineto{\pgfqpoint{3.886854in}{0.745776in}}%
\pgfpathlineto{\pgfqpoint{4.061103in}{0.754494in}}%
\pgfpathlineto{\pgfqpoint{4.302177in}{0.764210in}}%
\pgfpathlineto{\pgfqpoint{4.626733in}{0.774958in}}%
\pgfpathlineto{\pgfqpoint{5.053090in}{0.786754in}}%
\pgfpathlineto{\pgfqpoint{5.601441in}{0.799610in}}%
\pgfpathlineto{\pgfqpoint{5.745919in}{0.802692in}}%
\pgfpathlineto{\pgfqpoint{5.745919in}{0.802692in}}%
\pgfusepath{stroke}%
\end{pgfscope}%
\begin{pgfscope}%
\pgfpathrectangle{\pgfqpoint{3.560227in}{0.275000in}}{\pgfqpoint{2.289773in}{0.875000in}}%
\pgfusepath{clip}%
\pgfsetroundcap%
\pgfsetroundjoin%
\pgfsetlinewidth{1.756562pt}%
\definecolor{currentstroke}{rgb}{0.600000,0.600000,0.600000}%
\pgfsetstrokecolor{currentstroke}%
\pgfsetdash{}{0pt}%
\pgfpathmoveto{\pgfqpoint{3.664308in}{0.746786in}}%
\pgfpathlineto{\pgfqpoint{3.728428in}{0.758497in}}%
\pgfpathlineto{\pgfqpoint{3.800667in}{0.769251in}}%
\pgfpathlineto{\pgfqpoint{3.893933in}{0.780784in}}%
\pgfpathlineto{\pgfqpoint{4.013845in}{0.793275in}}%
\pgfpathlineto{\pgfqpoint{4.165818in}{0.806780in}}%
\pgfpathlineto{\pgfqpoint{4.355472in}{0.821320in}}%
\pgfpathlineto{\pgfqpoint{4.589052in}{0.836921in}}%
\pgfpathlineto{\pgfqpoint{4.873012in}{0.853586in}}%
\pgfpathlineto{\pgfqpoint{5.214223in}{0.871316in}}%
\pgfpathlineto{\pgfqpoint{5.620178in}{0.890122in}}%
\pgfpathlineto{\pgfqpoint{5.745919in}{0.895556in}}%
\pgfpathlineto{\pgfqpoint{5.745919in}{0.895556in}}%
\pgfusepath{stroke}%
\end{pgfscope}%
\begin{pgfscope}%
\pgfpathrectangle{\pgfqpoint{3.560227in}{0.275000in}}{\pgfqpoint{2.289773in}{0.875000in}}%
\pgfusepath{clip}%
\pgfsetroundcap%
\pgfsetroundjoin%
\pgfsetlinewidth{1.756562pt}%
\definecolor{currentstroke}{rgb}{0.894118,0.101961,0.109804}%
\pgfsetstrokecolor{currentstroke}%
\pgfsetdash{}{0pt}%
\pgfpathmoveto{\pgfqpoint{3.664308in}{0.763929in}}%
\pgfpathlineto{\pgfqpoint{3.707193in}{0.777163in}}%
\pgfpathlineto{\pgfqpoint{3.756324in}{0.789853in}}%
\pgfpathlineto{\pgfqpoint{3.817738in}{0.803334in}}%
\pgfpathlineto{\pgfqpoint{3.894349in}{0.817795in}}%
\pgfpathlineto{\pgfqpoint{3.988655in}{0.833260in}}%
\pgfpathlineto{\pgfqpoint{4.103572in}{0.849781in}}%
\pgfpathlineto{\pgfqpoint{4.241805in}{0.867341in}}%
\pgfpathlineto{\pgfqpoint{4.406685in}{0.885982in}}%
\pgfpathlineto{\pgfqpoint{4.601335in}{0.905689in}}%
\pgfpathlineto{\pgfqpoint{4.829086in}{0.926457in}}%
\pgfpathlineto{\pgfqpoint{5.093893in}{0.948316in}}%
\pgfpathlineto{\pgfqpoint{5.399296in}{0.971244in}}%
\pgfpathlineto{\pgfqpoint{5.745919in}{0.995022in}}%
\pgfpathlineto{\pgfqpoint{5.745919in}{0.995022in}}%
\pgfusepath{stroke}%
\end{pgfscope}%
\begin{pgfscope}%
\pgfpathrectangle{\pgfqpoint{3.560227in}{0.275000in}}{\pgfqpoint{2.289773in}{0.875000in}}%
\pgfusepath{clip}%
\pgfsetroundcap%
\pgfsetroundjoin%
\pgfsetlinewidth{1.756562pt}%
\definecolor{currentstroke}{rgb}{0.870588,0.870588,0.000000}%
\pgfsetstrokecolor{currentstroke}%
\pgfsetdash{}{0pt}%
\pgfpathmoveto{\pgfqpoint{3.664308in}{0.781071in}}%
\pgfpathlineto{\pgfqpoint{3.687208in}{0.792927in}}%
\pgfpathlineto{\pgfqpoint{3.718019in}{0.806265in}}%
\pgfpathlineto{\pgfqpoint{3.757365in}{0.820789in}}%
\pgfpathlineto{\pgfqpoint{3.806496in}{0.836475in}}%
\pgfpathlineto{\pgfqpoint{3.866661in}{0.853276in}}%
\pgfpathlineto{\pgfqpoint{3.939108in}{0.871134in}}%
\pgfpathlineto{\pgfqpoint{4.025504in}{0.890079in}}%
\pgfpathlineto{\pgfqpoint{4.127513in}{0.910112in}}%
\pgfpathlineto{\pgfqpoint{4.246801in}{0.931216in}}%
\pgfpathlineto{\pgfqpoint{4.385242in}{0.953395in}}%
\pgfpathlineto{\pgfqpoint{4.544709in}{0.976642in}}%
\pgfpathlineto{\pgfqpoint{4.727285in}{1.000964in}}%
\pgfpathlineto{\pgfqpoint{4.935259in}{1.026379in}}%
\pgfpathlineto{\pgfqpoint{5.170713in}{1.052869in}}%
\pgfpathlineto{\pgfqpoint{5.436145in}{1.080450in}}%
\pgfpathlineto{\pgfqpoint{5.733845in}{1.109108in}}%
\pgfpathlineto{\pgfqpoint{5.745919in}{1.110227in}}%
\pgfpathlineto{\pgfqpoint{5.745919in}{1.110227in}}%
\pgfusepath{stroke}%
\end{pgfscope}%
\begin{pgfscope}%
\pgfsetrectcap%
\pgfsetmiterjoin%
\pgfsetlinewidth{0.000000pt}%
\definecolor{currentstroke}{rgb}{1.000000,1.000000,1.000000}%
\pgfsetstrokecolor{currentstroke}%
\pgfsetdash{}{0pt}%
\pgfpathmoveto{\pgfqpoint{3.560227in}{0.275000in}}%
\pgfpathlineto{\pgfqpoint{3.560227in}{1.150000in}}%
\pgfusepath{}%
\end{pgfscope}%
\begin{pgfscope}%
\pgfsetrectcap%
\pgfsetmiterjoin%
\pgfsetlinewidth{0.000000pt}%
\definecolor{currentstroke}{rgb}{1.000000,1.000000,1.000000}%
\pgfsetstrokecolor{currentstroke}%
\pgfsetdash{}{0pt}%
\pgfpathmoveto{\pgfqpoint{5.850000in}{0.275000in}}%
\pgfpathlineto{\pgfqpoint{5.850000in}{1.150000in}}%
\pgfusepath{}%
\end{pgfscope}%
\begin{pgfscope}%
\pgfsetrectcap%
\pgfsetmiterjoin%
\pgfsetlinewidth{0.000000pt}%
\definecolor{currentstroke}{rgb}{1.000000,1.000000,1.000000}%
\pgfsetstrokecolor{currentstroke}%
\pgfsetdash{}{0pt}%
\pgfpathmoveto{\pgfqpoint{3.560227in}{0.275000in}}%
\pgfpathlineto{\pgfqpoint{5.850000in}{0.275000in}}%
\pgfusepath{}%
\end{pgfscope}%
\begin{pgfscope}%
\pgfsetrectcap%
\pgfsetmiterjoin%
\pgfsetlinewidth{0.000000pt}%
\definecolor{currentstroke}{rgb}{1.000000,1.000000,1.000000}%
\pgfsetstrokecolor{currentstroke}%
\pgfsetdash{}{0pt}%
\pgfpathmoveto{\pgfqpoint{3.560227in}{1.150000in}}%
\pgfpathlineto{\pgfqpoint{5.850000in}{1.150000in}}%
\pgfusepath{}%
\end{pgfscope}%
\end{pgfpicture}%
\makeatother%
\endgroup%

\end{figure}

\begin{figure}[h!]
    %% Creator: Matplotlib, PGF backend
%%
%% To include the figure in your LaTeX document, write
%%   \input{<filename>.pgf}
%%
%% Make sure the required packages are loaded in your preamble
%%   \usepackage{pgf}
%%
%% Also ensure that all the required font packages are loaded; for instance,
%% the lmodern package is sometimes necessary when using math font.
%%   \usepackage{lmodern}
%%
%% Figures using additional raster images can only be included by \input if
%% they are in the same directory as the main LaTeX file. For loading figures
%% from other directories you can use the `import` package
%%   \usepackage{import}
%%
%% and then include the figures with
%%   \import{<path to file>}{<filename>.pgf}
%%
%% Matplotlib used the following preamble
%%   
%%   \makeatletter\@ifpackageloaded{underscore}{}{\usepackage[strings]{underscore}}\makeatother
%%
\begingroup%
\makeatletter%
\begin{pgfpicture}%
\pgfpathrectangle{\pgfpointorigin}{\pgfqpoint{6.000000in}{4.000000in}}%
\pgfusepath{use as bounding box, clip}%
\begin{pgfscope}%
\pgfsetbuttcap%
\pgfsetmiterjoin%
\definecolor{currentfill}{rgb}{1.000000,1.000000,1.000000}%
\pgfsetfillcolor{currentfill}%
\pgfsetlinewidth{0.000000pt}%
\definecolor{currentstroke}{rgb}{1.000000,1.000000,1.000000}%
\pgfsetstrokecolor{currentstroke}%
\pgfsetdash{}{0pt}%
\pgfpathmoveto{\pgfqpoint{0.000000in}{0.000000in}}%
\pgfpathlineto{\pgfqpoint{6.000000in}{0.000000in}}%
\pgfpathlineto{\pgfqpoint{6.000000in}{4.000000in}}%
\pgfpathlineto{\pgfqpoint{0.000000in}{4.000000in}}%
\pgfpathlineto{\pgfqpoint{0.000000in}{0.000000in}}%
\pgfpathclose%
\pgfusepath{fill}%
\end{pgfscope}%
\begin{pgfscope}%
\pgfsetbuttcap%
\pgfsetmiterjoin%
\definecolor{currentfill}{rgb}{0.917647,0.917647,0.949020}%
\pgfsetfillcolor{currentfill}%
\pgfsetlinewidth{0.000000pt}%
\definecolor{currentstroke}{rgb}{0.000000,0.000000,0.000000}%
\pgfsetstrokecolor{currentstroke}%
\pgfsetstrokeopacity{0.000000}%
\pgfsetdash{}{0pt}%
\pgfpathmoveto{\pgfqpoint{0.750000in}{0.440000in}}%
\pgfpathlineto{\pgfqpoint{5.400000in}{0.440000in}}%
\pgfpathlineto{\pgfqpoint{5.400000in}{3.520000in}}%
\pgfpathlineto{\pgfqpoint{0.750000in}{3.520000in}}%
\pgfpathlineto{\pgfqpoint{0.750000in}{0.440000in}}%
\pgfpathclose%
\pgfusepath{fill}%
\end{pgfscope}%
\begin{pgfscope}%
\pgfpathrectangle{\pgfqpoint{0.750000in}{0.440000in}}{\pgfqpoint{4.650000in}{3.080000in}}%
\pgfusepath{clip}%
\pgfsetroundcap%
\pgfsetroundjoin%
\pgfsetlinewidth{1.003750pt}%
\definecolor{currentstroke}{rgb}{1.000000,1.000000,1.000000}%
\pgfsetstrokecolor{currentstroke}%
\pgfsetdash{}{0pt}%
\pgfpathmoveto{\pgfqpoint{0.961364in}{0.440000in}}%
\pgfpathlineto{\pgfqpoint{0.961364in}{3.520000in}}%
\pgfusepath{stroke}%
\end{pgfscope}%
\begin{pgfscope}%
\definecolor{textcolor}{rgb}{0.150000,0.150000,0.150000}%
\pgfsetstrokecolor{textcolor}%
\pgfsetfillcolor{textcolor}%
\pgftext[x=0.961364in,y=0.342778in,,top]{\color{textcolor}\rmfamily\fontsize{10.000000}{12.000000}\selectfont \(\displaystyle {0.00}\)}%
\end{pgfscope}%
\begin{pgfscope}%
\pgfpathrectangle{\pgfqpoint{0.750000in}{0.440000in}}{\pgfqpoint{4.650000in}{3.080000in}}%
\pgfusepath{clip}%
\pgfsetroundcap%
\pgfsetroundjoin%
\pgfsetlinewidth{1.003750pt}%
\definecolor{currentstroke}{rgb}{1.000000,1.000000,1.000000}%
\pgfsetstrokecolor{currentstroke}%
\pgfsetdash{}{0pt}%
\pgfpathmoveto{\pgfqpoint{1.489773in}{0.440000in}}%
\pgfpathlineto{\pgfqpoint{1.489773in}{3.520000in}}%
\pgfusepath{stroke}%
\end{pgfscope}%
\begin{pgfscope}%
\definecolor{textcolor}{rgb}{0.150000,0.150000,0.150000}%
\pgfsetstrokecolor{textcolor}%
\pgfsetfillcolor{textcolor}%
\pgftext[x=1.489773in,y=0.342778in,,top]{\color{textcolor}\rmfamily\fontsize{10.000000}{12.000000}\selectfont \(\displaystyle {0.25}\)}%
\end{pgfscope}%
\begin{pgfscope}%
\pgfpathrectangle{\pgfqpoint{0.750000in}{0.440000in}}{\pgfqpoint{4.650000in}{3.080000in}}%
\pgfusepath{clip}%
\pgfsetroundcap%
\pgfsetroundjoin%
\pgfsetlinewidth{1.003750pt}%
\definecolor{currentstroke}{rgb}{1.000000,1.000000,1.000000}%
\pgfsetstrokecolor{currentstroke}%
\pgfsetdash{}{0pt}%
\pgfpathmoveto{\pgfqpoint{2.018182in}{0.440000in}}%
\pgfpathlineto{\pgfqpoint{2.018182in}{3.520000in}}%
\pgfusepath{stroke}%
\end{pgfscope}%
\begin{pgfscope}%
\definecolor{textcolor}{rgb}{0.150000,0.150000,0.150000}%
\pgfsetstrokecolor{textcolor}%
\pgfsetfillcolor{textcolor}%
\pgftext[x=2.018182in,y=0.342778in,,top]{\color{textcolor}\rmfamily\fontsize{10.000000}{12.000000}\selectfont \(\displaystyle {0.50}\)}%
\end{pgfscope}%
\begin{pgfscope}%
\pgfpathrectangle{\pgfqpoint{0.750000in}{0.440000in}}{\pgfqpoint{4.650000in}{3.080000in}}%
\pgfusepath{clip}%
\pgfsetroundcap%
\pgfsetroundjoin%
\pgfsetlinewidth{1.003750pt}%
\definecolor{currentstroke}{rgb}{1.000000,1.000000,1.000000}%
\pgfsetstrokecolor{currentstroke}%
\pgfsetdash{}{0pt}%
\pgfpathmoveto{\pgfqpoint{2.546591in}{0.440000in}}%
\pgfpathlineto{\pgfqpoint{2.546591in}{3.520000in}}%
\pgfusepath{stroke}%
\end{pgfscope}%
\begin{pgfscope}%
\definecolor{textcolor}{rgb}{0.150000,0.150000,0.150000}%
\pgfsetstrokecolor{textcolor}%
\pgfsetfillcolor{textcolor}%
\pgftext[x=2.546591in,y=0.342778in,,top]{\color{textcolor}\rmfamily\fontsize{10.000000}{12.000000}\selectfont \(\displaystyle {0.75}\)}%
\end{pgfscope}%
\begin{pgfscope}%
\pgfpathrectangle{\pgfqpoint{0.750000in}{0.440000in}}{\pgfqpoint{4.650000in}{3.080000in}}%
\pgfusepath{clip}%
\pgfsetroundcap%
\pgfsetroundjoin%
\pgfsetlinewidth{1.003750pt}%
\definecolor{currentstroke}{rgb}{1.000000,1.000000,1.000000}%
\pgfsetstrokecolor{currentstroke}%
\pgfsetdash{}{0pt}%
\pgfpathmoveto{\pgfqpoint{3.075000in}{0.440000in}}%
\pgfpathlineto{\pgfqpoint{3.075000in}{3.520000in}}%
\pgfusepath{stroke}%
\end{pgfscope}%
\begin{pgfscope}%
\definecolor{textcolor}{rgb}{0.150000,0.150000,0.150000}%
\pgfsetstrokecolor{textcolor}%
\pgfsetfillcolor{textcolor}%
\pgftext[x=3.075000in,y=0.342778in,,top]{\color{textcolor}\rmfamily\fontsize{10.000000}{12.000000}\selectfont \(\displaystyle {1.00}\)}%
\end{pgfscope}%
\begin{pgfscope}%
\pgfpathrectangle{\pgfqpoint{0.750000in}{0.440000in}}{\pgfqpoint{4.650000in}{3.080000in}}%
\pgfusepath{clip}%
\pgfsetroundcap%
\pgfsetroundjoin%
\pgfsetlinewidth{1.003750pt}%
\definecolor{currentstroke}{rgb}{1.000000,1.000000,1.000000}%
\pgfsetstrokecolor{currentstroke}%
\pgfsetdash{}{0pt}%
\pgfpathmoveto{\pgfqpoint{3.603409in}{0.440000in}}%
\pgfpathlineto{\pgfqpoint{3.603409in}{3.520000in}}%
\pgfusepath{stroke}%
\end{pgfscope}%
\begin{pgfscope}%
\definecolor{textcolor}{rgb}{0.150000,0.150000,0.150000}%
\pgfsetstrokecolor{textcolor}%
\pgfsetfillcolor{textcolor}%
\pgftext[x=3.603409in,y=0.342778in,,top]{\color{textcolor}\rmfamily\fontsize{10.000000}{12.000000}\selectfont \(\displaystyle {1.25}\)}%
\end{pgfscope}%
\begin{pgfscope}%
\pgfpathrectangle{\pgfqpoint{0.750000in}{0.440000in}}{\pgfqpoint{4.650000in}{3.080000in}}%
\pgfusepath{clip}%
\pgfsetroundcap%
\pgfsetroundjoin%
\pgfsetlinewidth{1.003750pt}%
\definecolor{currentstroke}{rgb}{1.000000,1.000000,1.000000}%
\pgfsetstrokecolor{currentstroke}%
\pgfsetdash{}{0pt}%
\pgfpathmoveto{\pgfqpoint{4.131818in}{0.440000in}}%
\pgfpathlineto{\pgfqpoint{4.131818in}{3.520000in}}%
\pgfusepath{stroke}%
\end{pgfscope}%
\begin{pgfscope}%
\definecolor{textcolor}{rgb}{0.150000,0.150000,0.150000}%
\pgfsetstrokecolor{textcolor}%
\pgfsetfillcolor{textcolor}%
\pgftext[x=4.131818in,y=0.342778in,,top]{\color{textcolor}\rmfamily\fontsize{10.000000}{12.000000}\selectfont \(\displaystyle {1.50}\)}%
\end{pgfscope}%
\begin{pgfscope}%
\pgfpathrectangle{\pgfqpoint{0.750000in}{0.440000in}}{\pgfqpoint{4.650000in}{3.080000in}}%
\pgfusepath{clip}%
\pgfsetroundcap%
\pgfsetroundjoin%
\pgfsetlinewidth{1.003750pt}%
\definecolor{currentstroke}{rgb}{1.000000,1.000000,1.000000}%
\pgfsetstrokecolor{currentstroke}%
\pgfsetdash{}{0pt}%
\pgfpathmoveto{\pgfqpoint{4.660227in}{0.440000in}}%
\pgfpathlineto{\pgfqpoint{4.660227in}{3.520000in}}%
\pgfusepath{stroke}%
\end{pgfscope}%
\begin{pgfscope}%
\definecolor{textcolor}{rgb}{0.150000,0.150000,0.150000}%
\pgfsetstrokecolor{textcolor}%
\pgfsetfillcolor{textcolor}%
\pgftext[x=4.660227in,y=0.342778in,,top]{\color{textcolor}\rmfamily\fontsize{10.000000}{12.000000}\selectfont \(\displaystyle {1.75}\)}%
\end{pgfscope}%
\begin{pgfscope}%
\pgfpathrectangle{\pgfqpoint{0.750000in}{0.440000in}}{\pgfqpoint{4.650000in}{3.080000in}}%
\pgfusepath{clip}%
\pgfsetroundcap%
\pgfsetroundjoin%
\pgfsetlinewidth{1.003750pt}%
\definecolor{currentstroke}{rgb}{1.000000,1.000000,1.000000}%
\pgfsetstrokecolor{currentstroke}%
\pgfsetdash{}{0pt}%
\pgfpathmoveto{\pgfqpoint{5.188636in}{0.440000in}}%
\pgfpathlineto{\pgfqpoint{5.188636in}{3.520000in}}%
\pgfusepath{stroke}%
\end{pgfscope}%
\begin{pgfscope}%
\definecolor{textcolor}{rgb}{0.150000,0.150000,0.150000}%
\pgfsetstrokecolor{textcolor}%
\pgfsetfillcolor{textcolor}%
\pgftext[x=5.188636in,y=0.342778in,,top]{\color{textcolor}\rmfamily\fontsize{10.000000}{12.000000}\selectfont \(\displaystyle {2.00}\)}%
\end{pgfscope}%
\begin{pgfscope}%
\definecolor{textcolor}{rgb}{0.150000,0.150000,0.150000}%
\pgfsetstrokecolor{textcolor}%
\pgfsetfillcolor{textcolor}%
\pgftext[x=3.075000in,y=0.163766in,,top]{\color{textcolor}\rmfamily\fontsize{11.000000}{13.200000}\selectfont Time (\(\displaystyle t\))}%
\end{pgfscope}%
\begin{pgfscope}%
\pgfpathrectangle{\pgfqpoint{0.750000in}{0.440000in}}{\pgfqpoint{4.650000in}{3.080000in}}%
\pgfusepath{clip}%
\pgfsetroundcap%
\pgfsetroundjoin%
\pgfsetlinewidth{1.003750pt}%
\definecolor{currentstroke}{rgb}{1.000000,1.000000,1.000000}%
\pgfsetstrokecolor{currentstroke}%
\pgfsetdash{}{0pt}%
\pgfpathmoveto{\pgfqpoint{0.750000in}{0.576450in}}%
\pgfpathlineto{\pgfqpoint{5.400000in}{0.576450in}}%
\pgfusepath{stroke}%
\end{pgfscope}%
\begin{pgfscope}%
\definecolor{textcolor}{rgb}{0.150000,0.150000,0.150000}%
\pgfsetstrokecolor{textcolor}%
\pgfsetfillcolor{textcolor}%
\pgftext[x=0.583333in, y=0.528225in, left, base]{\color{textcolor}\rmfamily\fontsize{10.000000}{12.000000}\selectfont \(\displaystyle {0}\)}%
\end{pgfscope}%
\begin{pgfscope}%
\pgfpathrectangle{\pgfqpoint{0.750000in}{0.440000in}}{\pgfqpoint{4.650000in}{3.080000in}}%
\pgfusepath{clip}%
\pgfsetroundcap%
\pgfsetroundjoin%
\pgfsetlinewidth{1.003750pt}%
\definecolor{currentstroke}{rgb}{1.000000,1.000000,1.000000}%
\pgfsetstrokecolor{currentstroke}%
\pgfsetdash{}{0pt}%
\pgfpathmoveto{\pgfqpoint{0.750000in}{0.931466in}}%
\pgfpathlineto{\pgfqpoint{5.400000in}{0.931466in}}%
\pgfusepath{stroke}%
\end{pgfscope}%
\begin{pgfscope}%
\definecolor{textcolor}{rgb}{0.150000,0.150000,0.150000}%
\pgfsetstrokecolor{textcolor}%
\pgfsetfillcolor{textcolor}%
\pgftext[x=0.513888in, y=0.883241in, left, base]{\color{textcolor}\rmfamily\fontsize{10.000000}{12.000000}\selectfont \(\displaystyle {10}\)}%
\end{pgfscope}%
\begin{pgfscope}%
\pgfpathrectangle{\pgfqpoint{0.750000in}{0.440000in}}{\pgfqpoint{4.650000in}{3.080000in}}%
\pgfusepath{clip}%
\pgfsetroundcap%
\pgfsetroundjoin%
\pgfsetlinewidth{1.003750pt}%
\definecolor{currentstroke}{rgb}{1.000000,1.000000,1.000000}%
\pgfsetstrokecolor{currentstroke}%
\pgfsetdash{}{0pt}%
\pgfpathmoveto{\pgfqpoint{0.750000in}{1.286483in}}%
\pgfpathlineto{\pgfqpoint{5.400000in}{1.286483in}}%
\pgfusepath{stroke}%
\end{pgfscope}%
\begin{pgfscope}%
\definecolor{textcolor}{rgb}{0.150000,0.150000,0.150000}%
\pgfsetstrokecolor{textcolor}%
\pgfsetfillcolor{textcolor}%
\pgftext[x=0.513888in, y=1.238258in, left, base]{\color{textcolor}\rmfamily\fontsize{10.000000}{12.000000}\selectfont \(\displaystyle {20}\)}%
\end{pgfscope}%
\begin{pgfscope}%
\pgfpathrectangle{\pgfqpoint{0.750000in}{0.440000in}}{\pgfqpoint{4.650000in}{3.080000in}}%
\pgfusepath{clip}%
\pgfsetroundcap%
\pgfsetroundjoin%
\pgfsetlinewidth{1.003750pt}%
\definecolor{currentstroke}{rgb}{1.000000,1.000000,1.000000}%
\pgfsetstrokecolor{currentstroke}%
\pgfsetdash{}{0pt}%
\pgfpathmoveto{\pgfqpoint{0.750000in}{1.641500in}}%
\pgfpathlineto{\pgfqpoint{5.400000in}{1.641500in}}%
\pgfusepath{stroke}%
\end{pgfscope}%
\begin{pgfscope}%
\definecolor{textcolor}{rgb}{0.150000,0.150000,0.150000}%
\pgfsetstrokecolor{textcolor}%
\pgfsetfillcolor{textcolor}%
\pgftext[x=0.513888in, y=1.593274in, left, base]{\color{textcolor}\rmfamily\fontsize{10.000000}{12.000000}\selectfont \(\displaystyle {30}\)}%
\end{pgfscope}%
\begin{pgfscope}%
\pgfpathrectangle{\pgfqpoint{0.750000in}{0.440000in}}{\pgfqpoint{4.650000in}{3.080000in}}%
\pgfusepath{clip}%
\pgfsetroundcap%
\pgfsetroundjoin%
\pgfsetlinewidth{1.003750pt}%
\definecolor{currentstroke}{rgb}{1.000000,1.000000,1.000000}%
\pgfsetstrokecolor{currentstroke}%
\pgfsetdash{}{0pt}%
\pgfpathmoveto{\pgfqpoint{0.750000in}{1.996516in}}%
\pgfpathlineto{\pgfqpoint{5.400000in}{1.996516in}}%
\pgfusepath{stroke}%
\end{pgfscope}%
\begin{pgfscope}%
\definecolor{textcolor}{rgb}{0.150000,0.150000,0.150000}%
\pgfsetstrokecolor{textcolor}%
\pgfsetfillcolor{textcolor}%
\pgftext[x=0.513888in, y=1.948291in, left, base]{\color{textcolor}\rmfamily\fontsize{10.000000}{12.000000}\selectfont \(\displaystyle {40}\)}%
\end{pgfscope}%
\begin{pgfscope}%
\pgfpathrectangle{\pgfqpoint{0.750000in}{0.440000in}}{\pgfqpoint{4.650000in}{3.080000in}}%
\pgfusepath{clip}%
\pgfsetroundcap%
\pgfsetroundjoin%
\pgfsetlinewidth{1.003750pt}%
\definecolor{currentstroke}{rgb}{1.000000,1.000000,1.000000}%
\pgfsetstrokecolor{currentstroke}%
\pgfsetdash{}{0pt}%
\pgfpathmoveto{\pgfqpoint{0.750000in}{2.351533in}}%
\pgfpathlineto{\pgfqpoint{5.400000in}{2.351533in}}%
\pgfusepath{stroke}%
\end{pgfscope}%
\begin{pgfscope}%
\definecolor{textcolor}{rgb}{0.150000,0.150000,0.150000}%
\pgfsetstrokecolor{textcolor}%
\pgfsetfillcolor{textcolor}%
\pgftext[x=0.513888in, y=2.303308in, left, base]{\color{textcolor}\rmfamily\fontsize{10.000000}{12.000000}\selectfont \(\displaystyle {50}\)}%
\end{pgfscope}%
\begin{pgfscope}%
\pgfpathrectangle{\pgfqpoint{0.750000in}{0.440000in}}{\pgfqpoint{4.650000in}{3.080000in}}%
\pgfusepath{clip}%
\pgfsetroundcap%
\pgfsetroundjoin%
\pgfsetlinewidth{1.003750pt}%
\definecolor{currentstroke}{rgb}{1.000000,1.000000,1.000000}%
\pgfsetstrokecolor{currentstroke}%
\pgfsetdash{}{0pt}%
\pgfpathmoveto{\pgfqpoint{0.750000in}{2.706549in}}%
\pgfpathlineto{\pgfqpoint{5.400000in}{2.706549in}}%
\pgfusepath{stroke}%
\end{pgfscope}%
\begin{pgfscope}%
\definecolor{textcolor}{rgb}{0.150000,0.150000,0.150000}%
\pgfsetstrokecolor{textcolor}%
\pgfsetfillcolor{textcolor}%
\pgftext[x=0.513888in, y=2.658324in, left, base]{\color{textcolor}\rmfamily\fontsize{10.000000}{12.000000}\selectfont \(\displaystyle {60}\)}%
\end{pgfscope}%
\begin{pgfscope}%
\pgfpathrectangle{\pgfqpoint{0.750000in}{0.440000in}}{\pgfqpoint{4.650000in}{3.080000in}}%
\pgfusepath{clip}%
\pgfsetroundcap%
\pgfsetroundjoin%
\pgfsetlinewidth{1.003750pt}%
\definecolor{currentstroke}{rgb}{1.000000,1.000000,1.000000}%
\pgfsetstrokecolor{currentstroke}%
\pgfsetdash{}{0pt}%
\pgfpathmoveto{\pgfqpoint{0.750000in}{3.061566in}}%
\pgfpathlineto{\pgfqpoint{5.400000in}{3.061566in}}%
\pgfusepath{stroke}%
\end{pgfscope}%
\begin{pgfscope}%
\definecolor{textcolor}{rgb}{0.150000,0.150000,0.150000}%
\pgfsetstrokecolor{textcolor}%
\pgfsetfillcolor{textcolor}%
\pgftext[x=0.513888in, y=3.013341in, left, base]{\color{textcolor}\rmfamily\fontsize{10.000000}{12.000000}\selectfont \(\displaystyle {70}\)}%
\end{pgfscope}%
\begin{pgfscope}%
\pgfpathrectangle{\pgfqpoint{0.750000in}{0.440000in}}{\pgfqpoint{4.650000in}{3.080000in}}%
\pgfusepath{clip}%
\pgfsetroundcap%
\pgfsetroundjoin%
\pgfsetlinewidth{1.003750pt}%
\definecolor{currentstroke}{rgb}{1.000000,1.000000,1.000000}%
\pgfsetstrokecolor{currentstroke}%
\pgfsetdash{}{0pt}%
\pgfpathmoveto{\pgfqpoint{0.750000in}{3.416583in}}%
\pgfpathlineto{\pgfqpoint{5.400000in}{3.416583in}}%
\pgfusepath{stroke}%
\end{pgfscope}%
\begin{pgfscope}%
\definecolor{textcolor}{rgb}{0.150000,0.150000,0.150000}%
\pgfsetstrokecolor{textcolor}%
\pgfsetfillcolor{textcolor}%
\pgftext[x=0.513888in, y=3.368357in, left, base]{\color{textcolor}\rmfamily\fontsize{10.000000}{12.000000}\selectfont \(\displaystyle {80}\)}%
\end{pgfscope}%
\begin{pgfscope}%
\definecolor{textcolor}{rgb}{0.150000,0.150000,0.150000}%
\pgfsetstrokecolor{textcolor}%
\pgfsetfillcolor{textcolor}%
\pgftext[x=0.458333in,y=1.980000in,,bottom,rotate=90.000000]{\color{textcolor}\rmfamily\fontsize{11.000000}{13.200000}\selectfont Position in space (\(\displaystyle \lambda_i\))}%
\end{pgfscope}%
\begin{pgfscope}%
\pgfpathrectangle{\pgfqpoint{0.750000in}{0.440000in}}{\pgfqpoint{4.650000in}{3.080000in}}%
\pgfusepath{clip}%
\pgfsetroundcap%
\pgfsetroundjoin%
\pgfsetlinewidth{1.756562pt}%
\definecolor{currentstroke}{rgb}{0.215686,0.494118,0.721569}%
\pgfsetstrokecolor{currentstroke}%
\pgfsetdash{}{0pt}%
\pgfpathmoveto{\pgfqpoint{0.961364in}{0.580000in}}%
\pgfpathlineto{\pgfqpoint{1.386215in}{0.584451in}}%
\pgfpathlineto{\pgfqpoint{1.917280in}{0.587613in}}%
\pgfpathlineto{\pgfqpoint{2.788225in}{0.590213in}}%
\pgfpathlineto{\pgfqpoint{5.188636in}{0.595151in}}%
\pgfpathlineto{\pgfqpoint{5.188636in}{0.595151in}}%
\pgfusepath{stroke}%
\end{pgfscope}%
\begin{pgfscope}%
\pgfpathrectangle{\pgfqpoint{0.750000in}{0.440000in}}{\pgfqpoint{4.650000in}{3.080000in}}%
\pgfusepath{clip}%
\pgfsetroundcap%
\pgfsetroundjoin%
\pgfsetlinewidth{1.756562pt}%
\definecolor{currentstroke}{rgb}{1.000000,0.498039,0.000000}%
\pgfsetstrokecolor{currentstroke}%
\pgfsetdash{}{0pt}%
\pgfpathmoveto{\pgfqpoint{0.961364in}{0.668310in}}%
\pgfpathlineto{\pgfqpoint{1.152547in}{0.661857in}}%
\pgfpathlineto{\pgfqpoint{1.364973in}{0.656824in}}%
\pgfpathlineto{\pgfqpoint{1.619884in}{0.653046in}}%
\pgfpathlineto{\pgfqpoint{1.917280in}{0.650877in}}%
\pgfpathlineto{\pgfqpoint{2.299646in}{0.650448in}}%
\pgfpathlineto{\pgfqpoint{2.788225in}{0.652260in}}%
\pgfpathlineto{\pgfqpoint{3.467988in}{0.657195in}}%
\pgfpathlineto{\pgfqpoint{4.445146in}{0.666722in}}%
\pgfpathlineto{\pgfqpoint{5.188636in}{0.674954in}}%
\pgfpathlineto{\pgfqpoint{5.188636in}{0.674954in}}%
\pgfusepath{stroke}%
\end{pgfscope}%
\begin{pgfscope}%
\pgfpathrectangle{\pgfqpoint{0.750000in}{0.440000in}}{\pgfqpoint{4.650000in}{3.080000in}}%
\pgfusepath{clip}%
\pgfsetroundcap%
\pgfsetroundjoin%
\pgfsetlinewidth{1.756562pt}%
\definecolor{currentstroke}{rgb}{0.301961,0.686275,0.290196}%
\pgfsetstrokecolor{currentstroke}%
\pgfsetdash{}{0pt}%
\pgfpathmoveto{\pgfqpoint{0.961364in}{0.756621in}}%
\pgfpathlineto{\pgfqpoint{1.110062in}{0.750509in}}%
\pgfpathlineto{\pgfqpoint{1.280002in}{0.745992in}}%
\pgfpathlineto{\pgfqpoint{1.471185in}{0.743212in}}%
\pgfpathlineto{\pgfqpoint{1.704854in}{0.742185in}}%
\pgfpathlineto{\pgfqpoint{1.981007in}{0.743328in}}%
\pgfpathlineto{\pgfqpoint{2.320889in}{0.747126in}}%
\pgfpathlineto{\pgfqpoint{2.724497in}{0.753924in}}%
\pgfpathlineto{\pgfqpoint{3.234319in}{0.764801in}}%
\pgfpathlineto{\pgfqpoint{3.892839in}{0.781197in}}%
\pgfpathlineto{\pgfqpoint{4.763785in}{0.805271in}}%
\pgfpathlineto{\pgfqpoint{5.188636in}{0.817639in}}%
\pgfpathlineto{\pgfqpoint{5.188636in}{0.817639in}}%
\pgfusepath{stroke}%
\end{pgfscope}%
\begin{pgfscope}%
\pgfpathrectangle{\pgfqpoint{0.750000in}{0.440000in}}{\pgfqpoint{4.650000in}{3.080000in}}%
\pgfusepath{clip}%
\pgfsetroundcap%
\pgfsetroundjoin%
\pgfsetlinewidth{1.756562pt}%
\definecolor{currentstroke}{rgb}{0.968627,0.505882,0.749020}%
\pgfsetstrokecolor{currentstroke}%
\pgfsetdash{}{0pt}%
\pgfpathmoveto{\pgfqpoint{0.961364in}{0.844931in}}%
\pgfpathlineto{\pgfqpoint{1.088819in}{0.842271in}}%
\pgfpathlineto{\pgfqpoint{1.237517in}{0.841688in}}%
\pgfpathlineto{\pgfqpoint{1.407458in}{0.843307in}}%
\pgfpathlineto{\pgfqpoint{1.619884in}{0.847659in}}%
\pgfpathlineto{\pgfqpoint{1.874794in}{0.855187in}}%
\pgfpathlineto{\pgfqpoint{2.193433in}{0.866958in}}%
\pgfpathlineto{\pgfqpoint{2.575799in}{0.883384in}}%
\pgfpathlineto{\pgfqpoint{3.043136in}{0.905707in}}%
\pgfpathlineto{\pgfqpoint{3.637928in}{0.936438in}}%
\pgfpathlineto{\pgfqpoint{4.381418in}{0.977172in}}%
\pgfpathlineto{\pgfqpoint{5.188636in}{1.023235in}}%
\pgfpathlineto{\pgfqpoint{5.188636in}{1.023235in}}%
\pgfusepath{stroke}%
\end{pgfscope}%
\begin{pgfscope}%
\pgfpathrectangle{\pgfqpoint{0.750000in}{0.440000in}}{\pgfqpoint{4.650000in}{3.080000in}}%
\pgfusepath{clip}%
\pgfsetroundcap%
\pgfsetroundjoin%
\pgfsetlinewidth{1.756562pt}%
\definecolor{currentstroke}{rgb}{0.650980,0.337255,0.156863}%
\pgfsetstrokecolor{currentstroke}%
\pgfsetdash{}{0pt}%
\pgfpathmoveto{\pgfqpoint{0.961364in}{0.933242in}}%
\pgfpathlineto{\pgfqpoint{1.067577in}{0.935892in}}%
\pgfpathlineto{\pgfqpoint{1.195032in}{0.941290in}}%
\pgfpathlineto{\pgfqpoint{1.364973in}{0.950837in}}%
\pgfpathlineto{\pgfqpoint{1.577398in}{0.965080in}}%
\pgfpathlineto{\pgfqpoint{1.853552in}{0.985910in}}%
\pgfpathlineto{\pgfqpoint{2.214676in}{1.015539in}}%
\pgfpathlineto{\pgfqpoint{2.660770in}{1.054429in}}%
\pgfpathlineto{\pgfqpoint{3.234319in}{1.106733in}}%
\pgfpathlineto{\pgfqpoint{3.956567in}{1.174894in}}%
\pgfpathlineto{\pgfqpoint{4.891240in}{1.265454in}}%
\pgfpathlineto{\pgfqpoint{5.188636in}{1.294652in}}%
\pgfpathlineto{\pgfqpoint{5.188636in}{1.294652in}}%
\pgfusepath{stroke}%
\end{pgfscope}%
\begin{pgfscope}%
\pgfpathrectangle{\pgfqpoint{0.750000in}{0.440000in}}{\pgfqpoint{4.650000in}{3.080000in}}%
\pgfusepath{clip}%
\pgfsetroundcap%
\pgfsetroundjoin%
\pgfsetlinewidth{1.756562pt}%
\definecolor{currentstroke}{rgb}{0.596078,0.305882,0.639216}%
\pgfsetstrokecolor{currentstroke}%
\pgfsetdash{}{0pt}%
\pgfpathmoveto{\pgfqpoint{0.961364in}{1.021552in}}%
\pgfpathlineto{\pgfqpoint{1.067577in}{1.032379in}}%
\pgfpathlineto{\pgfqpoint{1.216275in}{1.050276in}}%
\pgfpathlineto{\pgfqpoint{1.428700in}{1.078414in}}%
\pgfpathlineto{\pgfqpoint{1.747339in}{1.123077in}}%
\pgfpathlineto{\pgfqpoint{2.235918in}{1.193972in}}%
\pgfpathlineto{\pgfqpoint{2.958166in}{1.301134in}}%
\pgfpathlineto{\pgfqpoint{3.977810in}{1.454797in}}%
\pgfpathlineto{\pgfqpoint{5.188636in}{1.639296in}}%
\pgfpathlineto{\pgfqpoint{5.188636in}{1.639296in}}%
\pgfusepath{stroke}%
\end{pgfscope}%
\begin{pgfscope}%
\pgfpathrectangle{\pgfqpoint{0.750000in}{0.440000in}}{\pgfqpoint{4.650000in}{3.080000in}}%
\pgfusepath{clip}%
\pgfsetroundcap%
\pgfsetroundjoin%
\pgfsetlinewidth{1.756562pt}%
\definecolor{currentstroke}{rgb}{0.600000,0.600000,0.600000}%
\pgfsetstrokecolor{currentstroke}%
\pgfsetdash{}{0pt}%
\pgfpathmoveto{\pgfqpoint{0.961364in}{1.109862in}}%
\pgfpathlineto{\pgfqpoint{1.067577in}{1.132991in}}%
\pgfpathlineto{\pgfqpoint{1.301245in}{1.187182in}}%
\pgfpathlineto{\pgfqpoint{2.002250in}{1.350354in}}%
\pgfpathlineto{\pgfqpoint{2.703255in}{1.510872in}}%
\pgfpathlineto{\pgfqpoint{3.744141in}{1.746742in}}%
\pgfpathlineto{\pgfqpoint{5.188636in}{2.072162in}}%
\pgfpathlineto{\pgfqpoint{5.188636in}{2.072162in}}%
\pgfusepath{stroke}%
\end{pgfscope}%
\begin{pgfscope}%
\pgfpathrectangle{\pgfqpoint{0.750000in}{0.440000in}}{\pgfqpoint{4.650000in}{3.080000in}}%
\pgfusepath{clip}%
\pgfsetroundcap%
\pgfsetroundjoin%
\pgfsetlinewidth{1.756562pt}%
\definecolor{currentstroke}{rgb}{0.894118,0.101961,0.109804}%
\pgfsetstrokecolor{currentstroke}%
\pgfsetdash{}{0pt}%
\pgfpathmoveto{\pgfqpoint{0.961364in}{1.198173in}}%
\pgfpathlineto{\pgfqpoint{1.280002in}{1.322001in}}%
\pgfpathlineto{\pgfqpoint{1.471185in}{1.392714in}}%
\pgfpathlineto{\pgfqpoint{1.704854in}{1.476454in}}%
\pgfpathlineto{\pgfqpoint{1.981007in}{1.572816in}}%
\pgfpathlineto{\pgfqpoint{2.320889in}{1.688814in}}%
\pgfpathlineto{\pgfqpoint{2.724497in}{1.824075in}}%
\pgfpathlineto{\pgfqpoint{3.213077in}{1.985424in}}%
\pgfpathlineto{\pgfqpoint{3.829111in}{2.186439in}}%
\pgfpathlineto{\pgfqpoint{4.615087in}{2.440419in}}%
\pgfpathlineto{\pgfqpoint{5.188636in}{2.624598in}}%
\pgfpathlineto{\pgfqpoint{5.188636in}{2.624598in}}%
\pgfusepath{stroke}%
\end{pgfscope}%
\begin{pgfscope}%
\pgfpathrectangle{\pgfqpoint{0.750000in}{0.440000in}}{\pgfqpoint{4.650000in}{3.080000in}}%
\pgfusepath{clip}%
\pgfsetroundcap%
\pgfsetroundjoin%
\pgfsetlinewidth{1.756562pt}%
\definecolor{currentstroke}{rgb}{0.870588,0.870588,0.000000}%
\pgfsetstrokecolor{currentstroke}%
\pgfsetdash{}{0pt}%
\pgfpathmoveto{\pgfqpoint{0.961364in}{1.286483in}}%
\pgfpathlineto{\pgfqpoint{1.003849in}{1.317666in}}%
\pgfpathlineto{\pgfqpoint{1.067577in}{1.361201in}}%
\pgfpathlineto{\pgfqpoint{1.152547in}{1.415499in}}%
\pgfpathlineto{\pgfqpoint{1.237517in}{1.467006in}}%
\pgfpathlineto{\pgfqpoint{1.343730in}{1.528712in}}%
\pgfpathlineto{\pgfqpoint{1.471185in}{1.599978in}}%
\pgfpathlineto{\pgfqpoint{1.619884in}{1.680380in}}%
\pgfpathlineto{\pgfqpoint{1.789824in}{1.769649in}}%
\pgfpathlineto{\pgfqpoint{1.981007in}{1.867624in}}%
\pgfpathlineto{\pgfqpoint{2.214676in}{1.984769in}}%
\pgfpathlineto{\pgfqpoint{2.490829in}{2.120493in}}%
\pgfpathlineto{\pgfqpoint{2.809468in}{2.274411in}}%
\pgfpathlineto{\pgfqpoint{3.170592in}{2.446297in}}%
\pgfpathlineto{\pgfqpoint{3.595443in}{2.645963in}}%
\pgfpathlineto{\pgfqpoint{4.084022in}{2.873080in}}%
\pgfpathlineto{\pgfqpoint{4.657572in}{3.137209in}}%
\pgfpathlineto{\pgfqpoint{5.188636in}{3.380000in}}%
\pgfpathlineto{\pgfqpoint{5.188636in}{3.380000in}}%
\pgfusepath{stroke}%
\end{pgfscope}%
\begin{pgfscope}%
\pgfsetrectcap%
\pgfsetmiterjoin%
\pgfsetlinewidth{0.000000pt}%
\definecolor{currentstroke}{rgb}{1.000000,1.000000,1.000000}%
\pgfsetstrokecolor{currentstroke}%
\pgfsetdash{}{0pt}%
\pgfpathmoveto{\pgfqpoint{0.750000in}{0.440000in}}%
\pgfpathlineto{\pgfqpoint{0.750000in}{3.520000in}}%
\pgfusepath{}%
\end{pgfscope}%
\begin{pgfscope}%
\pgfsetrectcap%
\pgfsetmiterjoin%
\pgfsetlinewidth{0.000000pt}%
\definecolor{currentstroke}{rgb}{1.000000,1.000000,1.000000}%
\pgfsetstrokecolor{currentstroke}%
\pgfsetdash{}{0pt}%
\pgfpathmoveto{\pgfqpoint{5.400000in}{0.440000in}}%
\pgfpathlineto{\pgfqpoint{5.400000in}{3.520000in}}%
\pgfusepath{}%
\end{pgfscope}%
\begin{pgfscope}%
\pgfsetrectcap%
\pgfsetmiterjoin%
\pgfsetlinewidth{0.000000pt}%
\definecolor{currentstroke}{rgb}{1.000000,1.000000,1.000000}%
\pgfsetstrokecolor{currentstroke}%
\pgfsetdash{}{0pt}%
\pgfpathmoveto{\pgfqpoint{0.750000in}{0.440000in}}%
\pgfpathlineto{\pgfqpoint{5.400000in}{0.440000in}}%
\pgfusepath{}%
\end{pgfscope}%
\begin{pgfscope}%
\pgfsetrectcap%
\pgfsetmiterjoin%
\pgfsetlinewidth{0.000000pt}%
\definecolor{currentstroke}{rgb}{1.000000,1.000000,1.000000}%
\pgfsetstrokecolor{currentstroke}%
\pgfsetdash{}{0pt}%
\pgfpathmoveto{\pgfqpoint{0.750000in}{3.520000in}}%
\pgfpathlineto{\pgfqpoint{5.400000in}{3.520000in}}%
\pgfusepath{}%
\end{pgfscope}%
\end{pgfpicture}%
\makeatother%
\endgroup%

\end{figure}

\begin{figure}[h!]
    %% Creator: Matplotlib, PGF backend
%%
%% To include the figure in your LaTeX document, write
%%   \input{<filename>.pgf}
%%
%% Make sure the required packages are loaded in your preamble
%%   \usepackage{pgf}
%%
%% Also ensure that all the required font packages are loaded; for instance,
%% the lmodern package is sometimes necessary when using math font.
%%   \usepackage{lmodern}
%%
%% Figures using additional raster images can only be included by \input if
%% they are in the same directory as the main LaTeX file. For loading figures
%% from other directories you can use the `import` package
%%   \usepackage{import}
%%
%% and then include the figures with
%%   \import{<path to file>}{<filename>.pgf}
%%
%% Matplotlib used the following preamble
%%   
%%   \makeatletter\@ifpackageloaded{underscore}{}{\usepackage[strings]{underscore}}\makeatother
%%
\begingroup%
\makeatletter%
\begin{pgfpicture}%
\pgfpathrectangle{\pgfpointorigin}{\pgfqpoint{6.000000in}{4.000000in}}%
\pgfusepath{use as bounding box, clip}%
\begin{pgfscope}%
\pgfsetbuttcap%
\pgfsetmiterjoin%
\definecolor{currentfill}{rgb}{1.000000,1.000000,1.000000}%
\pgfsetfillcolor{currentfill}%
\pgfsetlinewidth{0.000000pt}%
\definecolor{currentstroke}{rgb}{1.000000,1.000000,1.000000}%
\pgfsetstrokecolor{currentstroke}%
\pgfsetdash{}{0pt}%
\pgfpathmoveto{\pgfqpoint{0.000000in}{0.000000in}}%
\pgfpathlineto{\pgfqpoint{6.000000in}{0.000000in}}%
\pgfpathlineto{\pgfqpoint{6.000000in}{4.000000in}}%
\pgfpathlineto{\pgfqpoint{0.000000in}{4.000000in}}%
\pgfpathlineto{\pgfqpoint{0.000000in}{0.000000in}}%
\pgfpathclose%
\pgfusepath{fill}%
\end{pgfscope}%
\begin{pgfscope}%
\pgfsetbuttcap%
\pgfsetmiterjoin%
\definecolor{currentfill}{rgb}{0.917647,0.917647,0.949020}%
\pgfsetfillcolor{currentfill}%
\pgfsetlinewidth{0.000000pt}%
\definecolor{currentstroke}{rgb}{0.000000,0.000000,0.000000}%
\pgfsetstrokecolor{currentstroke}%
\pgfsetstrokeopacity{0.000000}%
\pgfsetdash{}{0pt}%
\pgfpathmoveto{\pgfqpoint{0.750000in}{0.440000in}}%
\pgfpathlineto{\pgfqpoint{5.400000in}{0.440000in}}%
\pgfpathlineto{\pgfqpoint{5.400000in}{3.520000in}}%
\pgfpathlineto{\pgfqpoint{0.750000in}{3.520000in}}%
\pgfpathlineto{\pgfqpoint{0.750000in}{0.440000in}}%
\pgfpathclose%
\pgfusepath{fill}%
\end{pgfscope}%
\begin{pgfscope}%
\pgfpathrectangle{\pgfqpoint{0.750000in}{0.440000in}}{\pgfqpoint{4.650000in}{3.080000in}}%
\pgfusepath{clip}%
\pgfsetroundcap%
\pgfsetroundjoin%
\pgfsetlinewidth{1.003750pt}%
\definecolor{currentstroke}{rgb}{1.000000,1.000000,1.000000}%
\pgfsetstrokecolor{currentstroke}%
\pgfsetdash{}{0pt}%
\pgfpathmoveto{\pgfqpoint{0.961364in}{0.440000in}}%
\pgfpathlineto{\pgfqpoint{0.961364in}{3.520000in}}%
\pgfusepath{stroke}%
\end{pgfscope}%
\begin{pgfscope}%
\definecolor{textcolor}{rgb}{0.150000,0.150000,0.150000}%
\pgfsetstrokecolor{textcolor}%
\pgfsetfillcolor{textcolor}%
\pgftext[x=0.961364in,y=0.342778in,,top]{\color{textcolor}\rmfamily\fontsize{10.000000}{12.000000}\selectfont \(\displaystyle {0.00}\)}%
\end{pgfscope}%
\begin{pgfscope}%
\pgfpathrectangle{\pgfqpoint{0.750000in}{0.440000in}}{\pgfqpoint{4.650000in}{3.080000in}}%
\pgfusepath{clip}%
\pgfsetroundcap%
\pgfsetroundjoin%
\pgfsetlinewidth{1.003750pt}%
\definecolor{currentstroke}{rgb}{1.000000,1.000000,1.000000}%
\pgfsetstrokecolor{currentstroke}%
\pgfsetdash{}{0pt}%
\pgfpathmoveto{\pgfqpoint{1.806818in}{0.440000in}}%
\pgfpathlineto{\pgfqpoint{1.806818in}{3.520000in}}%
\pgfusepath{stroke}%
\end{pgfscope}%
\begin{pgfscope}%
\definecolor{textcolor}{rgb}{0.150000,0.150000,0.150000}%
\pgfsetstrokecolor{textcolor}%
\pgfsetfillcolor{textcolor}%
\pgftext[x=1.806818in,y=0.342778in,,top]{\color{textcolor}\rmfamily\fontsize{10.000000}{12.000000}\selectfont \(\displaystyle {0.01}\)}%
\end{pgfscope}%
\begin{pgfscope}%
\pgfpathrectangle{\pgfqpoint{0.750000in}{0.440000in}}{\pgfqpoint{4.650000in}{3.080000in}}%
\pgfusepath{clip}%
\pgfsetroundcap%
\pgfsetroundjoin%
\pgfsetlinewidth{1.003750pt}%
\definecolor{currentstroke}{rgb}{1.000000,1.000000,1.000000}%
\pgfsetstrokecolor{currentstroke}%
\pgfsetdash{}{0pt}%
\pgfpathmoveto{\pgfqpoint{2.652273in}{0.440000in}}%
\pgfpathlineto{\pgfqpoint{2.652273in}{3.520000in}}%
\pgfusepath{stroke}%
\end{pgfscope}%
\begin{pgfscope}%
\definecolor{textcolor}{rgb}{0.150000,0.150000,0.150000}%
\pgfsetstrokecolor{textcolor}%
\pgfsetfillcolor{textcolor}%
\pgftext[x=2.652273in,y=0.342778in,,top]{\color{textcolor}\rmfamily\fontsize{10.000000}{12.000000}\selectfont \(\displaystyle {0.02}\)}%
\end{pgfscope}%
\begin{pgfscope}%
\pgfpathrectangle{\pgfqpoint{0.750000in}{0.440000in}}{\pgfqpoint{4.650000in}{3.080000in}}%
\pgfusepath{clip}%
\pgfsetroundcap%
\pgfsetroundjoin%
\pgfsetlinewidth{1.003750pt}%
\definecolor{currentstroke}{rgb}{1.000000,1.000000,1.000000}%
\pgfsetstrokecolor{currentstroke}%
\pgfsetdash{}{0pt}%
\pgfpathmoveto{\pgfqpoint{3.497727in}{0.440000in}}%
\pgfpathlineto{\pgfqpoint{3.497727in}{3.520000in}}%
\pgfusepath{stroke}%
\end{pgfscope}%
\begin{pgfscope}%
\definecolor{textcolor}{rgb}{0.150000,0.150000,0.150000}%
\pgfsetstrokecolor{textcolor}%
\pgfsetfillcolor{textcolor}%
\pgftext[x=3.497727in,y=0.342778in,,top]{\color{textcolor}\rmfamily\fontsize{10.000000}{12.000000}\selectfont \(\displaystyle {0.03}\)}%
\end{pgfscope}%
\begin{pgfscope}%
\pgfpathrectangle{\pgfqpoint{0.750000in}{0.440000in}}{\pgfqpoint{4.650000in}{3.080000in}}%
\pgfusepath{clip}%
\pgfsetroundcap%
\pgfsetroundjoin%
\pgfsetlinewidth{1.003750pt}%
\definecolor{currentstroke}{rgb}{1.000000,1.000000,1.000000}%
\pgfsetstrokecolor{currentstroke}%
\pgfsetdash{}{0pt}%
\pgfpathmoveto{\pgfqpoint{4.343182in}{0.440000in}}%
\pgfpathlineto{\pgfqpoint{4.343182in}{3.520000in}}%
\pgfusepath{stroke}%
\end{pgfscope}%
\begin{pgfscope}%
\definecolor{textcolor}{rgb}{0.150000,0.150000,0.150000}%
\pgfsetstrokecolor{textcolor}%
\pgfsetfillcolor{textcolor}%
\pgftext[x=4.343182in,y=0.342778in,,top]{\color{textcolor}\rmfamily\fontsize{10.000000}{12.000000}\selectfont \(\displaystyle {0.04}\)}%
\end{pgfscope}%
\begin{pgfscope}%
\pgfpathrectangle{\pgfqpoint{0.750000in}{0.440000in}}{\pgfqpoint{4.650000in}{3.080000in}}%
\pgfusepath{clip}%
\pgfsetroundcap%
\pgfsetroundjoin%
\pgfsetlinewidth{1.003750pt}%
\definecolor{currentstroke}{rgb}{1.000000,1.000000,1.000000}%
\pgfsetstrokecolor{currentstroke}%
\pgfsetdash{}{0pt}%
\pgfpathmoveto{\pgfqpoint{5.188636in}{0.440000in}}%
\pgfpathlineto{\pgfqpoint{5.188636in}{3.520000in}}%
\pgfusepath{stroke}%
\end{pgfscope}%
\begin{pgfscope}%
\definecolor{textcolor}{rgb}{0.150000,0.150000,0.150000}%
\pgfsetstrokecolor{textcolor}%
\pgfsetfillcolor{textcolor}%
\pgftext[x=5.188636in,y=0.342778in,,top]{\color{textcolor}\rmfamily\fontsize{10.000000}{12.000000}\selectfont \(\displaystyle {0.05}\)}%
\end{pgfscope}%
\begin{pgfscope}%
\definecolor{textcolor}{rgb}{0.150000,0.150000,0.150000}%
\pgfsetstrokecolor{textcolor}%
\pgfsetfillcolor{textcolor}%
\pgftext[x=3.075000in,y=0.163766in,,top]{\color{textcolor}\rmfamily\fontsize{11.000000}{13.200000}\selectfont Time (\(\displaystyle t\))}%
\end{pgfscope}%
\begin{pgfscope}%
\pgfpathrectangle{\pgfqpoint{0.750000in}{0.440000in}}{\pgfqpoint{4.650000in}{3.080000in}}%
\pgfusepath{clip}%
\pgfsetroundcap%
\pgfsetroundjoin%
\pgfsetlinewidth{1.003750pt}%
\definecolor{currentstroke}{rgb}{1.000000,1.000000,1.000000}%
\pgfsetstrokecolor{currentstroke}%
\pgfsetdash{}{0pt}%
\pgfpathmoveto{\pgfqpoint{0.750000in}{1.044914in}}%
\pgfpathlineto{\pgfqpoint{5.400000in}{1.044914in}}%
\pgfusepath{stroke}%
\end{pgfscope}%
\begin{pgfscope}%
\definecolor{textcolor}{rgb}{0.150000,0.150000,0.150000}%
\pgfsetstrokecolor{textcolor}%
\pgfsetfillcolor{textcolor}%
\pgftext[x=0.475308in, y=0.996689in, left, base]{\color{textcolor}\rmfamily\fontsize{10.000000}{12.000000}\selectfont \(\displaystyle {0.2}\)}%
\end{pgfscope}%
\begin{pgfscope}%
\pgfpathrectangle{\pgfqpoint{0.750000in}{0.440000in}}{\pgfqpoint{4.650000in}{3.080000in}}%
\pgfusepath{clip}%
\pgfsetroundcap%
\pgfsetroundjoin%
\pgfsetlinewidth{1.003750pt}%
\definecolor{currentstroke}{rgb}{1.000000,1.000000,1.000000}%
\pgfsetstrokecolor{currentstroke}%
\pgfsetdash{}{0pt}%
\pgfpathmoveto{\pgfqpoint{0.750000in}{1.683247in}}%
\pgfpathlineto{\pgfqpoint{5.400000in}{1.683247in}}%
\pgfusepath{stroke}%
\end{pgfscope}%
\begin{pgfscope}%
\definecolor{textcolor}{rgb}{0.150000,0.150000,0.150000}%
\pgfsetstrokecolor{textcolor}%
\pgfsetfillcolor{textcolor}%
\pgftext[x=0.475308in, y=1.635022in, left, base]{\color{textcolor}\rmfamily\fontsize{10.000000}{12.000000}\selectfont \(\displaystyle {0.4}\)}%
\end{pgfscope}%
\begin{pgfscope}%
\pgfpathrectangle{\pgfqpoint{0.750000in}{0.440000in}}{\pgfqpoint{4.650000in}{3.080000in}}%
\pgfusepath{clip}%
\pgfsetroundcap%
\pgfsetroundjoin%
\pgfsetlinewidth{1.003750pt}%
\definecolor{currentstroke}{rgb}{1.000000,1.000000,1.000000}%
\pgfsetstrokecolor{currentstroke}%
\pgfsetdash{}{0pt}%
\pgfpathmoveto{\pgfqpoint{0.750000in}{2.321580in}}%
\pgfpathlineto{\pgfqpoint{5.400000in}{2.321580in}}%
\pgfusepath{stroke}%
\end{pgfscope}%
\begin{pgfscope}%
\definecolor{textcolor}{rgb}{0.150000,0.150000,0.150000}%
\pgfsetstrokecolor{textcolor}%
\pgfsetfillcolor{textcolor}%
\pgftext[x=0.475308in, y=2.273355in, left, base]{\color{textcolor}\rmfamily\fontsize{10.000000}{12.000000}\selectfont \(\displaystyle {0.6}\)}%
\end{pgfscope}%
\begin{pgfscope}%
\pgfpathrectangle{\pgfqpoint{0.750000in}{0.440000in}}{\pgfqpoint{4.650000in}{3.080000in}}%
\pgfusepath{clip}%
\pgfsetroundcap%
\pgfsetroundjoin%
\pgfsetlinewidth{1.003750pt}%
\definecolor{currentstroke}{rgb}{1.000000,1.000000,1.000000}%
\pgfsetstrokecolor{currentstroke}%
\pgfsetdash{}{0pt}%
\pgfpathmoveto{\pgfqpoint{0.750000in}{2.959913in}}%
\pgfpathlineto{\pgfqpoint{5.400000in}{2.959913in}}%
\pgfusepath{stroke}%
\end{pgfscope}%
\begin{pgfscope}%
\definecolor{textcolor}{rgb}{0.150000,0.150000,0.150000}%
\pgfsetstrokecolor{textcolor}%
\pgfsetfillcolor{textcolor}%
\pgftext[x=0.475308in, y=2.911688in, left, base]{\color{textcolor}\rmfamily\fontsize{10.000000}{12.000000}\selectfont \(\displaystyle {0.8}\)}%
\end{pgfscope}%
\begin{pgfscope}%
\definecolor{textcolor}{rgb}{0.150000,0.150000,0.150000}%
\pgfsetstrokecolor{textcolor}%
\pgfsetfillcolor{textcolor}%
\pgftext[x=0.419752in,y=1.980000in,,bottom,rotate=90.000000]{\color{textcolor}\rmfamily\fontsize{11.000000}{13.200000}\selectfont Position in space (\(\displaystyle \lambda_i\))}%
\end{pgfscope}%
\begin{pgfscope}%
\pgfpathrectangle{\pgfqpoint{0.750000in}{0.440000in}}{\pgfqpoint{4.650000in}{3.080000in}}%
\pgfusepath{clip}%
\pgfsetroundcap%
\pgfsetroundjoin%
\pgfsetlinewidth{1.756562pt}%
\definecolor{currentstroke}{rgb}{0.215686,0.494118,0.721569}%
\pgfsetstrokecolor{currentstroke}%
\pgfsetdash{}{0pt}%
\pgfpathmoveto{\pgfqpoint{0.961364in}{1.364081in}}%
\pgfpathlineto{\pgfqpoint{1.047635in}{1.223137in}}%
\pgfpathlineto{\pgfqpoint{1.133905in}{1.143916in}}%
\pgfpathlineto{\pgfqpoint{1.220176in}{1.081514in}}%
\pgfpathlineto{\pgfqpoint{1.306447in}{1.030204in}}%
\pgfpathlineto{\pgfqpoint{1.392718in}{0.986693in}}%
\pgfpathlineto{\pgfqpoint{1.478989in}{0.949049in}}%
\pgfpathlineto{\pgfqpoint{1.565260in}{0.916021in}}%
\pgfpathlineto{\pgfqpoint{1.651531in}{0.886743in}}%
\pgfpathlineto{\pgfqpoint{1.737801in}{0.860584in}}%
\pgfpathlineto{\pgfqpoint{1.824072in}{0.837065in}}%
\pgfpathlineto{\pgfqpoint{1.910343in}{0.815813in}}%
\pgfpathlineto{\pgfqpoint{1.996614in}{0.796530in}}%
\pgfpathlineto{\pgfqpoint{2.082885in}{0.778972in}}%
\pgfpathlineto{\pgfqpoint{2.169156in}{0.762938in}}%
\pgfpathlineto{\pgfqpoint{2.255427in}{0.748260in}}%
\pgfpathlineto{\pgfqpoint{2.341698in}{0.734793in}}%
\pgfpathlineto{\pgfqpoint{2.427968in}{0.722415in}}%
\pgfpathlineto{\pgfqpoint{2.514239in}{0.711018in}}%
\pgfpathlineto{\pgfqpoint{2.600510in}{0.700511in}}%
\pgfpathlineto{\pgfqpoint{2.686781in}{0.690810in}}%
\pgfpathlineto{\pgfqpoint{2.773052in}{0.681845in}}%
\pgfpathlineto{\pgfqpoint{2.859323in}{0.673551in}}%
\pgfpathlineto{\pgfqpoint{2.945594in}{0.665871in}}%
\pgfpathlineto{\pgfqpoint{3.031865in}{0.658754in}}%
\pgfpathlineto{\pgfqpoint{3.118135in}{0.652154in}}%
\pgfpathlineto{\pgfqpoint{3.204406in}{0.646030in}}%
\pgfpathlineto{\pgfqpoint{3.290677in}{0.640344in}}%
\pgfpathlineto{\pgfqpoint{3.376948in}{0.635063in}}%
\pgfpathlineto{\pgfqpoint{3.463219in}{0.630154in}}%
\pgfpathlineto{\pgfqpoint{3.549490in}{0.625591in}}%
\pgfpathlineto{\pgfqpoint{3.635761in}{0.621348in}}%
\pgfpathlineto{\pgfqpoint{3.722032in}{0.617400in}}%
\pgfpathlineto{\pgfqpoint{3.808302in}{0.613727in}}%
\pgfpathlineto{\pgfqpoint{3.894573in}{0.610308in}}%
\pgfpathlineto{\pgfqpoint{3.980844in}{0.607126in}}%
\pgfpathlineto{\pgfqpoint{4.067115in}{0.604164in}}%
\pgfpathlineto{\pgfqpoint{4.153386in}{0.601406in}}%
\pgfpathlineto{\pgfqpoint{4.239657in}{0.598839in}}%
\pgfpathlineto{\pgfqpoint{4.325928in}{0.596448in}}%
\pgfpathlineto{\pgfqpoint{4.412199in}{0.594222in}}%
\pgfpathlineto{\pgfqpoint{4.498469in}{0.592150in}}%
\pgfpathlineto{\pgfqpoint{4.584740in}{0.590221in}}%
\pgfpathlineto{\pgfqpoint{4.671011in}{0.588425in}}%
\pgfpathlineto{\pgfqpoint{4.757282in}{0.586755in}}%
\pgfpathlineto{\pgfqpoint{4.843553in}{0.585200in}}%
\pgfpathlineto{\pgfqpoint{4.929824in}{0.583755in}}%
\pgfpathlineto{\pgfqpoint{5.016095in}{0.582410in}}%
\pgfpathlineto{\pgfqpoint{5.102365in}{0.581161in}}%
\pgfpathlineto{\pgfqpoint{5.188636in}{0.580000in}}%
\pgfusepath{stroke}%
\end{pgfscope}%
\begin{pgfscope}%
\pgfpathrectangle{\pgfqpoint{0.750000in}{0.440000in}}{\pgfqpoint{4.650000in}{3.080000in}}%
\pgfusepath{clip}%
\pgfsetroundcap%
\pgfsetroundjoin%
\pgfsetlinewidth{1.756562pt}%
\definecolor{currentstroke}{rgb}{1.000000,0.498039,0.000000}%
\pgfsetstrokecolor{currentstroke}%
\pgfsetdash{}{0pt}%
\pgfpathmoveto{\pgfqpoint{0.961364in}{1.443872in}}%
\pgfpathlineto{\pgfqpoint{1.047635in}{1.359039in}}%
\pgfpathlineto{\pgfqpoint{1.133905in}{1.296087in}}%
\pgfpathlineto{\pgfqpoint{1.220176in}{1.249209in}}%
\pgfpathlineto{\pgfqpoint{1.306447in}{1.210562in}}%
\pgfpathlineto{\pgfqpoint{1.392718in}{1.177556in}}%
\pgfpathlineto{\pgfqpoint{1.478989in}{1.148747in}}%
\pgfpathlineto{\pgfqpoint{1.565260in}{1.123227in}}%
\pgfpathlineto{\pgfqpoint{1.651531in}{1.100375in}}%
\pgfpathlineto{\pgfqpoint{1.737801in}{1.079745in}}%
\pgfpathlineto{\pgfqpoint{1.824072in}{1.061003in}}%
\pgfpathlineto{\pgfqpoint{1.910343in}{1.043888in}}%
\pgfpathlineto{\pgfqpoint{1.996614in}{1.028194in}}%
\pgfpathlineto{\pgfqpoint{2.082885in}{1.013752in}}%
\pgfpathlineto{\pgfqpoint{2.169156in}{1.000423in}}%
\pgfpathlineto{\pgfqpoint{2.255427in}{0.988093in}}%
\pgfpathlineto{\pgfqpoint{2.341698in}{0.976661in}}%
\pgfpathlineto{\pgfqpoint{2.427968in}{0.966043in}}%
\pgfpathlineto{\pgfqpoint{2.514239in}{0.956167in}}%
\pgfpathlineto{\pgfqpoint{2.600510in}{0.946968in}}%
\pgfpathlineto{\pgfqpoint{2.686781in}{0.938389in}}%
\pgfpathlineto{\pgfqpoint{2.773052in}{0.930383in}}%
\pgfpathlineto{\pgfqpoint{2.859323in}{0.922903in}}%
\pgfpathlineto{\pgfqpoint{2.945594in}{0.915910in}}%
\pgfpathlineto{\pgfqpoint{3.031865in}{0.909370in}}%
\pgfpathlineto{\pgfqpoint{3.118135in}{0.903249in}}%
\pgfpathlineto{\pgfqpoint{3.204406in}{0.897519in}}%
\pgfpathlineto{\pgfqpoint{3.290677in}{0.892153in}}%
\pgfpathlineto{\pgfqpoint{3.376948in}{0.887127in}}%
\pgfpathlineto{\pgfqpoint{3.463219in}{0.882418in}}%
\pgfpathlineto{\pgfqpoint{3.549490in}{0.878007in}}%
\pgfpathlineto{\pgfqpoint{3.635761in}{0.873875in}}%
\pgfpathlineto{\pgfqpoint{3.722032in}{0.870004in}}%
\pgfpathlineto{\pgfqpoint{3.808302in}{0.866378in}}%
\pgfpathlineto{\pgfqpoint{3.894573in}{0.862983in}}%
\pgfpathlineto{\pgfqpoint{3.980844in}{0.859805in}}%
\pgfpathlineto{\pgfqpoint{4.067115in}{0.856831in}}%
\pgfpathlineto{\pgfqpoint{4.153386in}{0.854049in}}%
\pgfpathlineto{\pgfqpoint{4.239657in}{0.851448in}}%
\pgfpathlineto{\pgfqpoint{4.325928in}{0.849017in}}%
\pgfpathlineto{\pgfqpoint{4.412199in}{0.846747in}}%
\pgfpathlineto{\pgfqpoint{4.498469in}{0.844629in}}%
\pgfpathlineto{\pgfqpoint{4.584740in}{0.842655in}}%
\pgfpathlineto{\pgfqpoint{4.671011in}{0.840815in}}%
\pgfpathlineto{\pgfqpoint{4.757282in}{0.839103in}}%
\pgfpathlineto{\pgfqpoint{4.843553in}{0.837511in}}%
\pgfpathlineto{\pgfqpoint{4.929824in}{0.836034in}}%
\pgfpathlineto{\pgfqpoint{5.016095in}{0.834664in}}%
\pgfpathlineto{\pgfqpoint{5.102365in}{0.833395in}}%
\pgfpathlineto{\pgfqpoint{5.188636in}{0.832223in}}%
\pgfusepath{stroke}%
\end{pgfscope}%
\begin{pgfscope}%
\pgfpathrectangle{\pgfqpoint{0.750000in}{0.440000in}}{\pgfqpoint{4.650000in}{3.080000in}}%
\pgfusepath{clip}%
\pgfsetroundcap%
\pgfsetroundjoin%
\pgfsetlinewidth{1.756562pt}%
\definecolor{currentstroke}{rgb}{0.301961,0.686275,0.290196}%
\pgfsetstrokecolor{currentstroke}%
\pgfsetdash{}{0pt}%
\pgfpathmoveto{\pgfqpoint{0.961364in}{1.523664in}}%
\pgfpathlineto{\pgfqpoint{1.047635in}{1.472470in}}%
\pgfpathlineto{\pgfqpoint{1.133905in}{1.432382in}}%
\pgfpathlineto{\pgfqpoint{1.220176in}{1.401971in}}%
\pgfpathlineto{\pgfqpoint{1.306447in}{1.377171in}}%
\pgfpathlineto{\pgfqpoint{1.392718in}{1.356082in}}%
\pgfpathlineto{\pgfqpoint{1.478989in}{1.337728in}}%
\pgfpathlineto{\pgfqpoint{1.565260in}{1.321505in}}%
\pgfpathlineto{\pgfqpoint{1.651531in}{1.307009in}}%
\pgfpathlineto{\pgfqpoint{1.737801in}{1.293950in}}%
\pgfpathlineto{\pgfqpoint{1.824072in}{1.282111in}}%
\pgfpathlineto{\pgfqpoint{1.910343in}{1.271326in}}%
\pgfpathlineto{\pgfqpoint{1.996614in}{1.261460in}}%
\pgfpathlineto{\pgfqpoint{2.082885in}{1.252407in}}%
\pgfpathlineto{\pgfqpoint{2.169156in}{1.244078in}}%
\pgfpathlineto{\pgfqpoint{2.255427in}{1.236398in}}%
\pgfpathlineto{\pgfqpoint{2.341698in}{1.229306in}}%
\pgfpathlineto{\pgfqpoint{2.427968in}{1.222746in}}%
\pgfpathlineto{\pgfqpoint{2.514239in}{1.216672in}}%
\pgfpathlineto{\pgfqpoint{2.600510in}{1.211044in}}%
\pgfpathlineto{\pgfqpoint{2.686781in}{1.205826in}}%
\pgfpathlineto{\pgfqpoint{2.773052in}{1.200985in}}%
\pgfpathlineto{\pgfqpoint{2.859323in}{1.196494in}}%
\pgfpathlineto{\pgfqpoint{2.945594in}{1.192327in}}%
\pgfpathlineto{\pgfqpoint{3.031865in}{1.188462in}}%
\pgfpathlineto{\pgfqpoint{3.118135in}{1.184878in}}%
\pgfpathlineto{\pgfqpoint{3.204406in}{1.181555in}}%
\pgfpathlineto{\pgfqpoint{3.290677in}{1.178478in}}%
\pgfpathlineto{\pgfqpoint{3.376948in}{1.175630in}}%
\pgfpathlineto{\pgfqpoint{3.463219in}{1.172997in}}%
\pgfpathlineto{\pgfqpoint{3.549490in}{1.170566in}}%
\pgfpathlineto{\pgfqpoint{3.635761in}{1.168324in}}%
\pgfpathlineto{\pgfqpoint{3.722032in}{1.166261in}}%
\pgfpathlineto{\pgfqpoint{3.808302in}{1.164365in}}%
\pgfpathlineto{\pgfqpoint{3.894573in}{1.162628in}}%
\pgfpathlineto{\pgfqpoint{3.980844in}{1.161039in}}%
\pgfpathlineto{\pgfqpoint{4.067115in}{1.159590in}}%
\pgfpathlineto{\pgfqpoint{4.153386in}{1.158275in}}%
\pgfpathlineto{\pgfqpoint{4.239657in}{1.157084in}}%
\pgfpathlineto{\pgfqpoint{4.325928in}{1.156011in}}%
\pgfpathlineto{\pgfqpoint{4.412199in}{1.155050in}}%
\pgfpathlineto{\pgfqpoint{4.498469in}{1.154194in}}%
\pgfpathlineto{\pgfqpoint{4.584740in}{1.153437in}}%
\pgfpathlineto{\pgfqpoint{4.671011in}{1.152775in}}%
\pgfpathlineto{\pgfqpoint{4.757282in}{1.152201in}}%
\pgfpathlineto{\pgfqpoint{4.843553in}{1.151711in}}%
\pgfpathlineto{\pgfqpoint{4.929824in}{1.151301in}}%
\pgfpathlineto{\pgfqpoint{5.016095in}{1.150965in}}%
\pgfpathlineto{\pgfqpoint{5.102365in}{1.150700in}}%
\pgfpathlineto{\pgfqpoint{5.188636in}{1.150502in}}%
\pgfusepath{stroke}%
\end{pgfscope}%
\begin{pgfscope}%
\pgfpathrectangle{\pgfqpoint{0.750000in}{0.440000in}}{\pgfqpoint{4.650000in}{3.080000in}}%
\pgfusepath{clip}%
\pgfsetroundcap%
\pgfsetroundjoin%
\pgfsetlinewidth{1.756562pt}%
\definecolor{currentstroke}{rgb}{0.968627,0.505882,0.749020}%
\pgfsetstrokecolor{currentstroke}%
\pgfsetdash{}{0pt}%
\pgfpathmoveto{\pgfqpoint{0.961364in}{1.603456in}}%
\pgfpathlineto{\pgfqpoint{1.047635in}{1.580117in}}%
\pgfpathlineto{\pgfqpoint{1.133905in}{1.562695in}}%
\pgfpathlineto{\pgfqpoint{1.220176in}{1.549878in}}%
\pgfpathlineto{\pgfqpoint{1.306447in}{1.539876in}}%
\pgfpathlineto{\pgfqpoint{1.392718in}{1.531723in}}%
\pgfpathlineto{\pgfqpoint{1.478989in}{1.524916in}}%
\pgfpathlineto{\pgfqpoint{1.565260in}{1.519152in}}%
\pgfpathlineto{\pgfqpoint{1.651531in}{1.514228in}}%
\pgfpathlineto{\pgfqpoint{1.737801in}{1.509999in}}%
\pgfpathlineto{\pgfqpoint{1.824072in}{1.506358in}}%
\pgfpathlineto{\pgfqpoint{1.910343in}{1.503222in}}%
\pgfpathlineto{\pgfqpoint{1.996614in}{1.500524in}}%
\pgfpathlineto{\pgfqpoint{2.082885in}{1.498213in}}%
\pgfpathlineto{\pgfqpoint{2.169156in}{1.496243in}}%
\pgfpathlineto{\pgfqpoint{2.255427in}{1.494578in}}%
\pgfpathlineto{\pgfqpoint{2.341698in}{1.493188in}}%
\pgfpathlineto{\pgfqpoint{2.427968in}{1.492046in}}%
\pgfpathlineto{\pgfqpoint{2.514239in}{1.491127in}}%
\pgfpathlineto{\pgfqpoint{2.600510in}{1.490414in}}%
\pgfpathlineto{\pgfqpoint{2.686781in}{1.489887in}}%
\pgfpathlineto{\pgfqpoint{2.773052in}{1.489531in}}%
\pgfpathlineto{\pgfqpoint{2.859323in}{1.489332in}}%
\pgfpathlineto{\pgfqpoint{2.945594in}{1.489277in}}%
\pgfpathlineto{\pgfqpoint{3.031865in}{1.489356in}}%
\pgfpathlineto{\pgfqpoint{3.118135in}{1.489558in}}%
\pgfpathlineto{\pgfqpoint{3.204406in}{1.489874in}}%
\pgfpathlineto{\pgfqpoint{3.290677in}{1.490294in}}%
\pgfpathlineto{\pgfqpoint{3.376948in}{1.490812in}}%
\pgfpathlineto{\pgfqpoint{3.463219in}{1.491421in}}%
\pgfpathlineto{\pgfqpoint{3.549490in}{1.492113in}}%
\pgfpathlineto{\pgfqpoint{3.635761in}{1.492882in}}%
\pgfpathlineto{\pgfqpoint{3.722032in}{1.493724in}}%
\pgfpathlineto{\pgfqpoint{3.808302in}{1.494632in}}%
\pgfpathlineto{\pgfqpoint{3.894573in}{1.495602in}}%
\pgfpathlineto{\pgfqpoint{3.980844in}{1.496630in}}%
\pgfpathlineto{\pgfqpoint{4.067115in}{1.497710in}}%
\pgfpathlineto{\pgfqpoint{4.153386in}{1.498840in}}%
\pgfpathlineto{\pgfqpoint{4.239657in}{1.500015in}}%
\pgfpathlineto{\pgfqpoint{4.325928in}{1.501232in}}%
\pgfpathlineto{\pgfqpoint{4.412199in}{1.502487in}}%
\pgfpathlineto{\pgfqpoint{4.498469in}{1.503778in}}%
\pgfpathlineto{\pgfqpoint{4.584740in}{1.505102in}}%
\pgfpathlineto{\pgfqpoint{4.671011in}{1.506455in}}%
\pgfpathlineto{\pgfqpoint{4.757282in}{1.507835in}}%
\pgfpathlineto{\pgfqpoint{4.843553in}{1.509241in}}%
\pgfpathlineto{\pgfqpoint{4.929824in}{1.510668in}}%
\pgfpathlineto{\pgfqpoint{5.016095in}{1.512116in}}%
\pgfpathlineto{\pgfqpoint{5.102365in}{1.513582in}}%
\pgfpathlineto{\pgfqpoint{5.188636in}{1.515065in}}%
\pgfusepath{stroke}%
\end{pgfscope}%
\begin{pgfscope}%
\pgfpathrectangle{\pgfqpoint{0.750000in}{0.440000in}}{\pgfqpoint{4.650000in}{3.080000in}}%
\pgfusepath{clip}%
\pgfsetroundcap%
\pgfsetroundjoin%
\pgfsetlinewidth{1.756562pt}%
\definecolor{currentstroke}{rgb}{0.650980,0.337255,0.156863}%
\pgfsetstrokecolor{currentstroke}%
\pgfsetdash{}{0pt}%
\pgfpathmoveto{\pgfqpoint{0.961364in}{1.683247in}}%
\pgfpathlineto{\pgfqpoint{1.047635in}{1.686439in}}%
\pgfpathlineto{\pgfqpoint{1.133905in}{1.691748in}}%
\pgfpathlineto{\pgfqpoint{1.220176in}{1.697296in}}%
\pgfpathlineto{\pgfqpoint{1.306447in}{1.702856in}}%
\pgfpathlineto{\pgfqpoint{1.392718in}{1.708399in}}%
\pgfpathlineto{\pgfqpoint{1.478989in}{1.713914in}}%
\pgfpathlineto{\pgfqpoint{1.565260in}{1.719396in}}%
\pgfpathlineto{\pgfqpoint{1.651531in}{1.724841in}}%
\pgfpathlineto{\pgfqpoint{1.737801in}{1.730245in}}%
\pgfpathlineto{\pgfqpoint{1.824072in}{1.735608in}}%
\pgfpathlineto{\pgfqpoint{1.910343in}{1.740927in}}%
\pgfpathlineto{\pgfqpoint{1.996614in}{1.746202in}}%
\pgfpathlineto{\pgfqpoint{2.082885in}{1.751432in}}%
\pgfpathlineto{\pgfqpoint{2.169156in}{1.756617in}}%
\pgfpathlineto{\pgfqpoint{2.255427in}{1.761754in}}%
\pgfpathlineto{\pgfqpoint{2.341698in}{1.766845in}}%
\pgfpathlineto{\pgfqpoint{2.427968in}{1.771888in}}%
\pgfpathlineto{\pgfqpoint{2.514239in}{1.776884in}}%
\pgfpathlineto{\pgfqpoint{2.600510in}{1.781832in}}%
\pgfpathlineto{\pgfqpoint{2.686781in}{1.786732in}}%
\pgfpathlineto{\pgfqpoint{2.773052in}{1.791583in}}%
\pgfpathlineto{\pgfqpoint{2.859323in}{1.796386in}}%
\pgfpathlineto{\pgfqpoint{2.945594in}{1.801140in}}%
\pgfpathlineto{\pgfqpoint{3.031865in}{1.805845in}}%
\pgfpathlineto{\pgfqpoint{3.118135in}{1.810502in}}%
\pgfpathlineto{\pgfqpoint{3.204406in}{1.815109in}}%
\pgfpathlineto{\pgfqpoint{3.290677in}{1.819667in}}%
\pgfpathlineto{\pgfqpoint{3.376948in}{1.824176in}}%
\pgfpathlineto{\pgfqpoint{3.463219in}{1.828636in}}%
\pgfpathlineto{\pgfqpoint{3.549490in}{1.833047in}}%
\pgfpathlineto{\pgfqpoint{3.635761in}{1.837409in}}%
\pgfpathlineto{\pgfqpoint{3.722032in}{1.841721in}}%
\pgfpathlineto{\pgfqpoint{3.808302in}{1.845985in}}%
\pgfpathlineto{\pgfqpoint{3.894573in}{1.850200in}}%
\pgfpathlineto{\pgfqpoint{3.980844in}{1.854365in}}%
\pgfpathlineto{\pgfqpoint{4.067115in}{1.858482in}}%
\pgfpathlineto{\pgfqpoint{4.153386in}{1.862550in}}%
\pgfpathlineto{\pgfqpoint{4.239657in}{1.866569in}}%
\pgfpathlineto{\pgfqpoint{4.325928in}{1.870540in}}%
\pgfpathlineto{\pgfqpoint{4.412199in}{1.874462in}}%
\pgfpathlineto{\pgfqpoint{4.498469in}{1.878336in}}%
\pgfpathlineto{\pgfqpoint{4.584740in}{1.882162in}}%
\pgfpathlineto{\pgfqpoint{4.671011in}{1.885940in}}%
\pgfpathlineto{\pgfqpoint{4.757282in}{1.889670in}}%
\pgfpathlineto{\pgfqpoint{4.843553in}{1.893353in}}%
\pgfpathlineto{\pgfqpoint{4.929824in}{1.896987in}}%
\pgfpathlineto{\pgfqpoint{5.016095in}{1.900575in}}%
\pgfpathlineto{\pgfqpoint{5.102365in}{1.904116in}}%
\pgfpathlineto{\pgfqpoint{5.188636in}{1.907609in}}%
\pgfusepath{stroke}%
\end{pgfscope}%
\begin{pgfscope}%
\pgfpathrectangle{\pgfqpoint{0.750000in}{0.440000in}}{\pgfqpoint{4.650000in}{3.080000in}}%
\pgfusepath{clip}%
\pgfsetroundcap%
\pgfsetroundjoin%
\pgfsetlinewidth{1.756562pt}%
\definecolor{currentstroke}{rgb}{0.596078,0.305882,0.639216}%
\pgfsetstrokecolor{currentstroke}%
\pgfsetdash{}{0pt}%
\pgfpathmoveto{\pgfqpoint{0.961364in}{1.763039in}}%
\pgfpathlineto{\pgfqpoint{1.047635in}{1.793910in}}%
\pgfpathlineto{\pgfqpoint{1.133905in}{1.822828in}}%
\pgfpathlineto{\pgfqpoint{1.220176in}{1.847407in}}%
\pgfpathlineto{\pgfqpoint{1.306447in}{1.869134in}}%
\pgfpathlineto{\pgfqpoint{1.392718in}{1.888940in}}%
\pgfpathlineto{\pgfqpoint{1.478989in}{1.907310in}}%
\pgfpathlineto{\pgfqpoint{1.565260in}{1.924540in}}%
\pgfpathlineto{\pgfqpoint{1.651531in}{1.940826in}}%
\pgfpathlineto{\pgfqpoint{1.737801in}{1.956310in}}%
\pgfpathlineto{\pgfqpoint{1.824072in}{1.971097in}}%
\pgfpathlineto{\pgfqpoint{1.910343in}{1.985269in}}%
\pgfpathlineto{\pgfqpoint{1.996614in}{1.998890in}}%
\pgfpathlineto{\pgfqpoint{2.082885in}{2.012012in}}%
\pgfpathlineto{\pgfqpoint{2.169156in}{2.024679in}}%
\pgfpathlineto{\pgfqpoint{2.255427in}{2.036926in}}%
\pgfpathlineto{\pgfqpoint{2.341698in}{2.048786in}}%
\pgfpathlineto{\pgfqpoint{2.427968in}{2.060284in}}%
\pgfpathlineto{\pgfqpoint{2.514239in}{2.071443in}}%
\pgfpathlineto{\pgfqpoint{2.600510in}{2.082284in}}%
\pgfpathlineto{\pgfqpoint{2.686781in}{2.092824in}}%
\pgfpathlineto{\pgfqpoint{2.773052in}{2.103080in}}%
\pgfpathlineto{\pgfqpoint{2.859323in}{2.113065in}}%
\pgfpathlineto{\pgfqpoint{2.945594in}{2.122792in}}%
\pgfpathlineto{\pgfqpoint{3.031865in}{2.132273in}}%
\pgfpathlineto{\pgfqpoint{3.118135in}{2.141519in}}%
\pgfpathlineto{\pgfqpoint{3.204406in}{2.150539in}}%
\pgfpathlineto{\pgfqpoint{3.290677in}{2.159341in}}%
\pgfpathlineto{\pgfqpoint{3.376948in}{2.167935in}}%
\pgfpathlineto{\pgfqpoint{3.463219in}{2.176327in}}%
\pgfpathlineto{\pgfqpoint{3.549490in}{2.184525in}}%
\pgfpathlineto{\pgfqpoint{3.635761in}{2.192535in}}%
\pgfpathlineto{\pgfqpoint{3.722032in}{2.200364in}}%
\pgfpathlineto{\pgfqpoint{3.808302in}{2.208016in}}%
\pgfpathlineto{\pgfqpoint{3.894573in}{2.215498in}}%
\pgfpathlineto{\pgfqpoint{3.980844in}{2.222814in}}%
\pgfpathlineto{\pgfqpoint{4.067115in}{2.229970in}}%
\pgfpathlineto{\pgfqpoint{4.153386in}{2.236969in}}%
\pgfpathlineto{\pgfqpoint{4.239657in}{2.243816in}}%
\pgfpathlineto{\pgfqpoint{4.325928in}{2.250516in}}%
\pgfpathlineto{\pgfqpoint{4.412199in}{2.257071in}}%
\pgfpathlineto{\pgfqpoint{4.498469in}{2.263486in}}%
\pgfpathlineto{\pgfqpoint{4.584740in}{2.269764in}}%
\pgfpathlineto{\pgfqpoint{4.671011in}{2.275908in}}%
\pgfpathlineto{\pgfqpoint{4.757282in}{2.281922in}}%
\pgfpathlineto{\pgfqpoint{4.843553in}{2.287809in}}%
\pgfpathlineto{\pgfqpoint{4.929824in}{2.293572in}}%
\pgfpathlineto{\pgfqpoint{5.016095in}{2.299214in}}%
\pgfpathlineto{\pgfqpoint{5.102365in}{2.304736in}}%
\pgfpathlineto{\pgfqpoint{5.188636in}{2.310143in}}%
\pgfusepath{stroke}%
\end{pgfscope}%
\begin{pgfscope}%
\pgfpathrectangle{\pgfqpoint{0.750000in}{0.440000in}}{\pgfqpoint{4.650000in}{3.080000in}}%
\pgfusepath{clip}%
\pgfsetroundcap%
\pgfsetroundjoin%
\pgfsetlinewidth{1.756562pt}%
\definecolor{currentstroke}{rgb}{0.600000,0.600000,0.600000}%
\pgfsetstrokecolor{currentstroke}%
\pgfsetdash{}{0pt}%
\pgfpathmoveto{\pgfqpoint{0.961364in}{1.842830in}}%
\pgfpathlineto{\pgfqpoint{1.047635in}{1.905259in}}%
\pgfpathlineto{\pgfqpoint{1.133905in}{1.959512in}}%
\pgfpathlineto{\pgfqpoint{1.220176in}{2.003721in}}%
\pgfpathlineto{\pgfqpoint{1.306447in}{2.042117in}}%
\pgfpathlineto{\pgfqpoint{1.392718in}{2.076604in}}%
\pgfpathlineto{\pgfqpoint{1.478989in}{2.108161in}}%
\pgfpathlineto{\pgfqpoint{1.565260in}{2.137391in}}%
\pgfpathlineto{\pgfqpoint{1.651531in}{2.164701in}}%
\pgfpathlineto{\pgfqpoint{1.737801in}{2.190380in}}%
\pgfpathlineto{\pgfqpoint{1.824072in}{2.214649in}}%
\pgfpathlineto{\pgfqpoint{1.910343in}{2.237676in}}%
\pgfpathlineto{\pgfqpoint{1.996614in}{2.259596in}}%
\pgfpathlineto{\pgfqpoint{2.082885in}{2.280518in}}%
\pgfpathlineto{\pgfqpoint{2.169156in}{2.300534in}}%
\pgfpathlineto{\pgfqpoint{2.255427in}{2.319720in}}%
\pgfpathlineto{\pgfqpoint{2.341698in}{2.338141in}}%
\pgfpathlineto{\pgfqpoint{2.427968in}{2.355852in}}%
\pgfpathlineto{\pgfqpoint{2.514239in}{2.372904in}}%
\pgfpathlineto{\pgfqpoint{2.600510in}{2.389339in}}%
\pgfpathlineto{\pgfqpoint{2.686781in}{2.405195in}}%
\pgfpathlineto{\pgfqpoint{2.773052in}{2.420505in}}%
\pgfpathlineto{\pgfqpoint{2.859323in}{2.435301in}}%
\pgfpathlineto{\pgfqpoint{2.945594in}{2.449609in}}%
\pgfpathlineto{\pgfqpoint{3.031865in}{2.463455in}}%
\pgfpathlineto{\pgfqpoint{3.118135in}{2.476860in}}%
\pgfpathlineto{\pgfqpoint{3.204406in}{2.489847in}}%
\pgfpathlineto{\pgfqpoint{3.290677in}{2.502434in}}%
\pgfpathlineto{\pgfqpoint{3.376948in}{2.514638in}}%
\pgfpathlineto{\pgfqpoint{3.463219in}{2.526476in}}%
\pgfpathlineto{\pgfqpoint{3.549490in}{2.537963in}}%
\pgfpathlineto{\pgfqpoint{3.635761in}{2.549113in}}%
\pgfpathlineto{\pgfqpoint{3.722032in}{2.559939in}}%
\pgfpathlineto{\pgfqpoint{3.808302in}{2.570453in}}%
\pgfpathlineto{\pgfqpoint{3.894573in}{2.580668in}}%
\pgfpathlineto{\pgfqpoint{3.980844in}{2.590593in}}%
\pgfpathlineto{\pgfqpoint{4.067115in}{2.600239in}}%
\pgfpathlineto{\pgfqpoint{4.153386in}{2.609617in}}%
\pgfpathlineto{\pgfqpoint{4.239657in}{2.618734in}}%
\pgfpathlineto{\pgfqpoint{4.325928in}{2.627599in}}%
\pgfpathlineto{\pgfqpoint{4.412199in}{2.636222in}}%
\pgfpathlineto{\pgfqpoint{4.498469in}{2.644609in}}%
\pgfpathlineto{\pgfqpoint{4.584740in}{2.652768in}}%
\pgfpathlineto{\pgfqpoint{4.671011in}{2.660707in}}%
\pgfpathlineto{\pgfqpoint{4.757282in}{2.668431in}}%
\pgfpathlineto{\pgfqpoint{4.843553in}{2.675948in}}%
\pgfpathlineto{\pgfqpoint{4.929824in}{2.683264in}}%
\pgfpathlineto{\pgfqpoint{5.016095in}{2.690385in}}%
\pgfpathlineto{\pgfqpoint{5.102365in}{2.697315in}}%
\pgfpathlineto{\pgfqpoint{5.188636in}{2.704062in}}%
\pgfusepath{stroke}%
\end{pgfscope}%
\begin{pgfscope}%
\pgfpathrectangle{\pgfqpoint{0.750000in}{0.440000in}}{\pgfqpoint{4.650000in}{3.080000in}}%
\pgfusepath{clip}%
\pgfsetroundcap%
\pgfsetroundjoin%
\pgfsetlinewidth{1.756562pt}%
\definecolor{currentstroke}{rgb}{0.894118,0.101961,0.109804}%
\pgfsetstrokecolor{currentstroke}%
\pgfsetdash{}{0pt}%
\pgfpathmoveto{\pgfqpoint{0.961364in}{1.922622in}}%
\pgfpathlineto{\pgfqpoint{1.047635in}{2.026040in}}%
\pgfpathlineto{\pgfqpoint{1.133905in}{2.107585in}}%
\pgfpathlineto{\pgfqpoint{1.220176in}{2.171870in}}%
\pgfpathlineto{\pgfqpoint{1.306447in}{2.227439in}}%
\pgfpathlineto{\pgfqpoint{1.392718in}{2.276946in}}%
\pgfpathlineto{\pgfqpoint{1.478989in}{2.321864in}}%
\pgfpathlineto{\pgfqpoint{1.565260in}{2.363114in}}%
\pgfpathlineto{\pgfqpoint{1.651531in}{2.401330in}}%
\pgfpathlineto{\pgfqpoint{1.737801in}{2.436966in}}%
\pgfpathlineto{\pgfqpoint{1.824072in}{2.470368in}}%
\pgfpathlineto{\pgfqpoint{1.910343in}{2.501804in}}%
\pgfpathlineto{\pgfqpoint{1.996614in}{2.531489in}}%
\pgfpathlineto{\pgfqpoint{2.082885in}{2.559599in}}%
\pgfpathlineto{\pgfqpoint{2.169156in}{2.586280in}}%
\pgfpathlineto{\pgfqpoint{2.255427in}{2.611655in}}%
\pgfpathlineto{\pgfqpoint{2.341698in}{2.635831in}}%
\pgfpathlineto{\pgfqpoint{2.427968in}{2.658898in}}%
\pgfpathlineto{\pgfqpoint{2.514239in}{2.680937in}}%
\pgfpathlineto{\pgfqpoint{2.600510in}{2.702017in}}%
\pgfpathlineto{\pgfqpoint{2.686781in}{2.722202in}}%
\pgfpathlineto{\pgfqpoint{2.773052in}{2.741547in}}%
\pgfpathlineto{\pgfqpoint{2.859323in}{2.760103in}}%
\pgfpathlineto{\pgfqpoint{2.945594in}{2.777914in}}%
\pgfpathlineto{\pgfqpoint{3.031865in}{2.795022in}}%
\pgfpathlineto{\pgfqpoint{3.118135in}{2.811466in}}%
\pgfpathlineto{\pgfqpoint{3.204406in}{2.827278in}}%
\pgfpathlineto{\pgfqpoint{3.290677in}{2.842492in}}%
\pgfpathlineto{\pgfqpoint{3.376948in}{2.857136in}}%
\pgfpathlineto{\pgfqpoint{3.463219in}{2.871238in}}%
\pgfpathlineto{\pgfqpoint{3.549490in}{2.884823in}}%
\pgfpathlineto{\pgfqpoint{3.635761in}{2.897914in}}%
\pgfpathlineto{\pgfqpoint{3.722032in}{2.910534in}}%
\pgfpathlineto{\pgfqpoint{3.808302in}{2.922703in}}%
\pgfpathlineto{\pgfqpoint{3.894573in}{2.934440in}}%
\pgfpathlineto{\pgfqpoint{3.980844in}{2.945763in}}%
\pgfpathlineto{\pgfqpoint{4.067115in}{2.956690in}}%
\pgfpathlineto{\pgfqpoint{4.153386in}{2.967237in}}%
\pgfpathlineto{\pgfqpoint{4.239657in}{2.977419in}}%
\pgfpathlineto{\pgfqpoint{4.325928in}{2.987249in}}%
\pgfpathlineto{\pgfqpoint{4.412199in}{2.996743in}}%
\pgfpathlineto{\pgfqpoint{4.498469in}{3.005912in}}%
\pgfpathlineto{\pgfqpoint{4.584740in}{3.014770in}}%
\pgfpathlineto{\pgfqpoint{4.671011in}{3.023327in}}%
\pgfpathlineto{\pgfqpoint{4.757282in}{3.031596in}}%
\pgfpathlineto{\pgfqpoint{4.843553in}{3.039586in}}%
\pgfpathlineto{\pgfqpoint{4.929824in}{3.047308in}}%
\pgfpathlineto{\pgfqpoint{5.016095in}{3.054771in}}%
\pgfpathlineto{\pgfqpoint{5.102365in}{3.061985in}}%
\pgfpathlineto{\pgfqpoint{5.188636in}{3.068959in}}%
\pgfusepath{stroke}%
\end{pgfscope}%
\begin{pgfscope}%
\pgfpathrectangle{\pgfqpoint{0.750000in}{0.440000in}}{\pgfqpoint{4.650000in}{3.080000in}}%
\pgfusepath{clip}%
\pgfsetroundcap%
\pgfsetroundjoin%
\pgfsetlinewidth{1.756562pt}%
\definecolor{currentstroke}{rgb}{0.870588,0.870588,0.000000}%
\pgfsetstrokecolor{currentstroke}%
\pgfsetdash{}{0pt}%
\pgfpathmoveto{\pgfqpoint{0.961364in}{2.002414in}}%
\pgfpathlineto{\pgfqpoint{1.047635in}{2.177499in}}%
\pgfpathlineto{\pgfqpoint{1.133905in}{2.280275in}}%
\pgfpathlineto{\pgfqpoint{1.220176in}{2.365740in}}%
\pgfpathlineto{\pgfqpoint{1.306447in}{2.439328in}}%
\pgfpathlineto{\pgfqpoint{1.392718in}{2.504349in}}%
\pgfpathlineto{\pgfqpoint{1.478989in}{2.562769in}}%
\pgfpathlineto{\pgfqpoint{1.565260in}{2.615867in}}%
\pgfpathlineto{\pgfqpoint{1.651531in}{2.664535in}}%
\pgfpathlineto{\pgfqpoint{1.737801in}{2.709429in}}%
\pgfpathlineto{\pgfqpoint{1.824072in}{2.751048in}}%
\pgfpathlineto{\pgfqpoint{1.910343in}{2.789787in}}%
\pgfpathlineto{\pgfqpoint{1.996614in}{2.825962in}}%
\pgfpathlineto{\pgfqpoint{2.082885in}{2.859835in}}%
\pgfpathlineto{\pgfqpoint{2.169156in}{2.891625in}}%
\pgfpathlineto{\pgfqpoint{2.255427in}{2.921519in}}%
\pgfpathlineto{\pgfqpoint{2.341698in}{2.949675in}}%
\pgfpathlineto{\pgfqpoint{2.427968in}{2.976234in}}%
\pgfpathlineto{\pgfqpoint{2.514239in}{3.001317in}}%
\pgfpathlineto{\pgfqpoint{2.600510in}{3.025032in}}%
\pgfpathlineto{\pgfqpoint{2.686781in}{3.047476in}}%
\pgfpathlineto{\pgfqpoint{2.773052in}{3.068735in}}%
\pgfpathlineto{\pgfqpoint{2.859323in}{3.088886in}}%
\pgfpathlineto{\pgfqpoint{2.945594in}{3.108001in}}%
\pgfpathlineto{\pgfqpoint{3.031865in}{3.126143in}}%
\pgfpathlineto{\pgfqpoint{3.118135in}{3.143372in}}%
\pgfpathlineto{\pgfqpoint{3.204406in}{3.159741in}}%
\pgfpathlineto{\pgfqpoint{3.290677in}{3.175300in}}%
\pgfpathlineto{\pgfqpoint{3.376948in}{3.190094in}}%
\pgfpathlineto{\pgfqpoint{3.463219in}{3.204168in}}%
\pgfpathlineto{\pgfqpoint{3.549490in}{3.217558in}}%
\pgfpathlineto{\pgfqpoint{3.635761in}{3.230304in}}%
\pgfpathlineto{\pgfqpoint{3.722032in}{3.242438in}}%
\pgfpathlineto{\pgfqpoint{3.808302in}{3.253994in}}%
\pgfpathlineto{\pgfqpoint{3.894573in}{3.265000in}}%
\pgfpathlineto{\pgfqpoint{3.980844in}{3.275486in}}%
\pgfpathlineto{\pgfqpoint{4.067115in}{3.285477in}}%
\pgfpathlineto{\pgfqpoint{4.153386in}{3.294998in}}%
\pgfpathlineto{\pgfqpoint{4.239657in}{3.304074in}}%
\pgfpathlineto{\pgfqpoint{4.325928in}{3.312726in}}%
\pgfpathlineto{\pgfqpoint{4.412199in}{3.320974in}}%
\pgfpathlineto{\pgfqpoint{4.498469in}{3.328839in}}%
\pgfpathlineto{\pgfqpoint{4.584740in}{3.336339in}}%
\pgfpathlineto{\pgfqpoint{4.671011in}{3.343492in}}%
\pgfpathlineto{\pgfqpoint{4.757282in}{3.350315in}}%
\pgfpathlineto{\pgfqpoint{4.843553in}{3.356823in}}%
\pgfpathlineto{\pgfqpoint{4.929824in}{3.363032in}}%
\pgfpathlineto{\pgfqpoint{5.016095in}{3.368955in}}%
\pgfpathlineto{\pgfqpoint{5.102365in}{3.374607in}}%
\pgfpathlineto{\pgfqpoint{5.188636in}{3.380000in}}%
\pgfusepath{stroke}%
\end{pgfscope}%
\begin{pgfscope}%
\pgfsetrectcap%
\pgfsetmiterjoin%
\pgfsetlinewidth{0.000000pt}%
\definecolor{currentstroke}{rgb}{1.000000,1.000000,1.000000}%
\pgfsetstrokecolor{currentstroke}%
\pgfsetdash{}{0pt}%
\pgfpathmoveto{\pgfqpoint{0.750000in}{0.440000in}}%
\pgfpathlineto{\pgfqpoint{0.750000in}{3.520000in}}%
\pgfusepath{}%
\end{pgfscope}%
\begin{pgfscope}%
\pgfsetrectcap%
\pgfsetmiterjoin%
\pgfsetlinewidth{0.000000pt}%
\definecolor{currentstroke}{rgb}{1.000000,1.000000,1.000000}%
\pgfsetstrokecolor{currentstroke}%
\pgfsetdash{}{0pt}%
\pgfpathmoveto{\pgfqpoint{5.400000in}{0.440000in}}%
\pgfpathlineto{\pgfqpoint{5.400000in}{3.520000in}}%
\pgfusepath{}%
\end{pgfscope}%
\begin{pgfscope}%
\pgfsetrectcap%
\pgfsetmiterjoin%
\pgfsetlinewidth{0.000000pt}%
\definecolor{currentstroke}{rgb}{1.000000,1.000000,1.000000}%
\pgfsetstrokecolor{currentstroke}%
\pgfsetdash{}{0pt}%
\pgfpathmoveto{\pgfqpoint{0.750000in}{0.440000in}}%
\pgfpathlineto{\pgfqpoint{5.400000in}{0.440000in}}%
\pgfusepath{}%
\end{pgfscope}%
\begin{pgfscope}%
\pgfsetrectcap%
\pgfsetmiterjoin%
\pgfsetlinewidth{0.000000pt}%
\definecolor{currentstroke}{rgb}{1.000000,1.000000,1.000000}%
\pgfsetstrokecolor{currentstroke}%
\pgfsetdash{}{0pt}%
\pgfpathmoveto{\pgfqpoint{0.750000in}{3.520000in}}%
\pgfpathlineto{\pgfqpoint{5.400000in}{3.520000in}}%
\pgfusepath{}%
\end{pgfscope}%
\end{pgfpicture}%
\makeatother%
\endgroup%

\end{figure}