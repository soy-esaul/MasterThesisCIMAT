\section{Deterministic eigenvalue processes for matrix-valued diffusions}

Now that we have shown the construction of a matrix-valued process whose eigenvalue is the deterministic Dyson Brownian motion, we generaliza the result to get processes with a deterministic spectrum that can follow the dynamics of any eigenvalue process with the form \eqref{eq:gen_dyson}.

% \begin{theorem}

%     Let $Z$ be a process with covariation $\d Z_{ij}\d Z_{kl} = (\delta_{ik}\delta_{jl} + \delta_{il}\delta_{jk} - 2\delta_{ij}\delta_{kl}\delta_{ik} )\d t$ and no finite variation part, which means $Z$ is a symmetric matrix with independent Brownian motions in its entries, except for the diagonal, where $Z_{ii} = 0$. Let $X$ be a matrix valued process such that  $X = \trans H \Lambda H$ and it satisfies the stochastic differential equation
    
    
%     \begin{equation}
%         \trans{H}\d X(t)H = g(X(t)) \d Z(t) h(X(t)) + h(X(t)) \d \trans{Z(t)} g(X(t)) + b(X(t))\d t, \label{eq:deterministic_matrix_diffusion}
%     \end{equation}

%     Then the eigenvalue process $\Lambda$ satisfies

%     \begin{equation}
%         \d \lambda_i = \biggl( b(\lambda_i) + \sum_{k\neq i} \frac{G(\lambda_i,\lambda_k)}{\lambda_i - \lambda_k} \biggr)\d t.
%     \end{equation}

% \end{theorem}

\begin{theorem} \label{thm:deterministic_diffusion}
    Let $B = (B(t), t\ge 0)$ be a Brownian motion in $\M_{p,p}(\R)$ and $Y(t) = QM\trans{Q}$ be a symmetric $p\times p$ matrix-valued stochastic process satisfying the stochastic differential equation

    \begin{equation}
        \d Y(t) = g(Y(t)) \d B(t) h(Y(t)) + h(Y(t)) \d \trans{B(t)} g(Y(t)) + b(Y(t))\d t, \label{eq:matrix_diffusion}
    \end{equation}

    where $g,h,b$ are real functions acting spectrally, and $Y(0)$ is a symmetric $p\times p$ matrix with $p$ different eigenvalues.

    Let $G(x,y) = g^2(x)h^2(y) + g^2(y)h^2(x)$, $\tau$ be defined as in \eqref{eq:collision_time}, and take a process $X = (X(t), t\ge 0)$ with diagonalization $X = H \Lambda \trans{H}$ such that $\trans H (\d \Lambda) H$ has the same off-diagonal entries as $\trans Q ( \d M) \circ Q$ and has diagonal entries equal to zero.
    
    Then, for $t < \tau$ the eigenvalue process $\Lambda(t)$ verifies the following system of stochastic differential equations:

    \begin{equation}
        \d \lambda_i = \biggl( b(\lambda_i) + \sum_{k\neq i} \frac{G(\lambda_i,\lambda_k)}{\lambda_i - \lambda_k} \biggr)\d t.
    \end{equation}
\end{theorem}


\begin{proof}
    We define again $L$ to be the stochastic logarithm of $H$, $L \coloneqq \trans H \circ \d H$ and using the same techniques as in Theorem \ref{thm:diffusion_real} we have that,

    \begin{equation*}
        \d \Lambda = \trans H \circ(\partial X) \circ H - (\partial L)\circ \Lambda + \Lambda \circ \partial L.
    \end{equation*}

    Using that $\Lambda \circ \partial L -  (\partial L)\circ \Lambda $ has zero diagonal, we get that $\d lambda_i = (\trans H \circ(\partial X) \circ H)_{ii}$ and by hypothesis, we know that the martingale part of this diagonal is zero. 

    Define $\d N \coloneqq \trans H \circ(\partial X) \circ H$. For $i\neq j$ we have that $\d L_{ij} = \d N_{ij}/(\lambda_j - \lambda_i)$.

    Finally, we compute the finite variation $\d F$ part of $\d N$,

    \begin{align*}
        \d F &= \trans H b(X) H \d t + \frac12 (\d \trans H \d X H + \trans H \d X H), \\ 
        &= b(\Lambda) \d t + \frac12( \trans{(\d N\d A)} + \d N \d A ).
    \end{align*}

    For $\d N \d A$ we find

    \begin{align*}
        (\d N \d A)_{ij} &= \sum_{k\neq j} \d N_{ik}\d A_{kj} = \sum_{k\neq j} \frac{\d N_{ik}\d N_{kj}}{\lambda_j - \lambda_k}.
    \end{align*}

    Now we use that the martingale part of  $\d N$ has the same entries as $\trans{Q} M Q$ and by the results in Theorem \ref{thm:diffusion_real} we know that 

    \begin{equation*}
        (\trans{Q} M Q)_{ik}(\trans{Q} M Q)_{kj} = \delta_{ij}G(\lambda_i,\lambda_k)\d t,
    \end{equation*}

    \noindent so substituting the last result we get

    \begin{equation*}
        \d \lambda_i = \d F_{ii} = b(\lambda_i)\d t + \sum_{k\neq j} \frac{G(\lambda_i,\lambda_k)\d t}{\lambda_j - \lambda_k},
    \end{equation*}

    \noindent which is the desired result.



\end{proof}


These results can be particularized for any matrix-valued diffusions. Especially, we are interested in the Wishart and Jacobi processes. We give the proofs for these as a corollary to last Theorem.

\subsection{Wishart process}

\begin{corollary}
    Let $Y = (Y(t), t \ge 0)$ be an $n\times n$ Wishart process with parameter $m$ and diagonalization $Y = QM\trans{Q}$. Let $X$ be an $n\times n$ self adjoint matrix process with diagonalization $X = H \Lambda \trans{H}$, such that the off-diagonal part of $\trans{H} \d X H $ and $\overset{d}{=} \trans{Q} \d Y Q$ coincide, and $(\trans{H} \d X H)_{ii} = 0$ for every $i \in [n]$. Then the eigenvalues of $X$, $\lambda_1 \ge \lambda_2 \ge \cdots \ge \lambda_n$ satisfy the following system of stochastic differential equations

    \begin{equation} \label{eq:deterministic_wishart}
        \d \lambda_i = \left( m +\sum_{k\neq i} \frac{ \abs{\lambda_i} + \abs{\lambda_k} }{\lambda_i - \lambda_k}\right)\d t.
    \end{equation}
\end{corollary}

\begin{proof}
    
\end{proof}


\subsection{Jacobi process}

\begin{corollary}
    Let $Y = (Y(t), t \ge 0)$ be an $n\times n$ matrix Jacobi process with parameters $n_1,n_2$ and diagonalization $Y = QM\trans{Q}$. Let $X$ be an $n\times n$ self adjoint matrix process with diagonalization $X = H \Lambda \trans{H}$, such that the off-diagonal part of $\trans{H} \d X H $ and $\overset{d}{=} \trans{Q} \d Y Q$ coincide, and $(\trans{H} \d X H)_{ii} = 0$ for every $i \in [n]$. Then the eigenvalues of $X$, $\lambda_1 \ge \lambda_2 \ge \cdots \ge \lambda_n$ satisfy the following system of stochastic differential equations

    \begin{equation} \label{eq:deterministic_jacobi}
        \d \lambda_i = \left( n_2 - (n_1 + n_2)\lambda_i + \sum_{k\neq i} \frac{\lambda_k(1-\lambda_i) + \lambda_i(1-\lambda_k)}{\lambda_i - \lambda_k} \right)\d t.
    \end{equation}

\end{corollary}

\section{Dynamical behavior of the deterministic eigenvalue processes}

\begin{lemma} \label{lemma:separation}
    Let $\lambda_1,\dots,\lambda_n$ be a system of $n$ functions moving according to \eqref{eq:deterministic_dyson} and $\lambda_i(0) > \lambda_j(0)$ and there is no $\lambda_k$ such that $\lambda_i(0) > \lambda_k(0) > \lambda_j(0)$. Then, $\lambda_i$ and $\lambda_k$ repel each other if and only if 

    \begin{equation} \label{eq:separation_condition}
        \frac{2}{(\lambda_i - \lambda_j)^2} > \sum_{k\neq i,j} \frac{1}{(\lambda_i - \lambda_k)(\lambda_j - \lambda_k)}
    \end{equation}
\end{lemma}

\begin{proof}

    The distance between the particles grows if and only if its derivative its positive, so 

    \begin{align*}
        \frac{\d }{\d t} (\lambda_i - \lambda_j) &= \sum_{k\neq i} \frac{1}{\lambda_i - \lambda_k} - \sum_{k\neq j} \frac{1}{\lambda_j - \lambda_k} = \sum_{k\neq i,j} \left( \frac{1}{\lambda_i - \lambda_k} - \frac{1}{\lambda_j - \lambda_k} \right) + \frac{2}{\lambda_i - \lambda_j},\\
        &= \sum_{k \neq i, j} \frac{\lambda_j - \lambda_k - \lambda_i + \lambda_k}{(\lambda_i- \lambda_k)(\lambda_j - \lambda_k)} + \frac{2}{\lambda_i - \lambda_j},= \frac{2}{(\lambda_i - \lambda_j)^2} - \sum_{k\neq i,j} \frac{1}{(\lambda_i - \lambda_k)(\lambda_j - \lambda_k)}.
    \end{align*}

    Comparing to zero, we get the desired result.
\end{proof}



\begin{theorem}
    If a system of functions satisfies \eqref{eq:deterministic_dyson}, then there ar no collisions.
\end{theorem}

\begin{proof}
    Write condition \eqref{eq:separation_condition} as

    \begin{equation*}
        \frac{2}{(\lambda_i - \lambda_j)^2} > \sum_{k\neq i,j} \frac{1}{(\lambda_i - \lambda_j + \lambda_j -\lambda_k)(\lambda_j - \lambda_k)},
    \end{equation*}

    \noindent and take $\lambda_i \to \lambda_j$. As $\lambda_i$ approaches $\lambda_k$, the left-hand side grows to $\infty$, while the right-hand side converges to the finite quantity

    \begin{equation*}
        \sum_{k\neq i,j} \frac{1}{(\lambda_j - \lambda_k)^2}.
    \end{equation*}

    Thus, before the collision occurs, the distance between $\lambda_i$ and $\lambda_j$ starts to grow.
\end{proof}

\begin{theorem} \label{thm:hermite_minimal_grows}
    Let $(\lambda_1, \lambda_2, \dots, \lambda_n)$ be a system of functions moving according to \eqref{eq:deterministic_dyson} and let $\lambda_i,\lambda_j$ be such that for a given $t_0$ it is satisfied $\lambda_i(t_0) > \lambda_j(t_0)$ and 
    
    \begin{equation*}
        \lambda_i(t_0) - \lambda_j(t_0) = \min_{k,l \in [n]} \abs{ \lambda_k(t_0) - \lambda_l(t_0)}.
    \end{equation*}

    Then, $\lambda_i$ and $\lambda_k$ repel. 
\end{theorem}

\begin{proof}
    Define the quotients $c_{ik}, c_{jk}$ as 

    \begin{equation*}
        c_{ik} = \frac{\lambda_i - \lambda_k}{\lambda_i - \lambda_j}, \qquad c_{jk} = \frac{\lambda_j - \lambda_k}{\lambda_i - \lambda_j}.
    \end{equation*}

    Notice that the condition that the distance between $\lambda_i$ and $\lambda_j$ is minimal means that there is no $\lambda_k$ between them and therefore for a fixed $k_0$, $c_{ik_0}$ and $c_{jk_0}$ have the same sign. We can write the right-hand side of \eqref{eq:separation_condition} as

    \begin{align*}
        \sum_{k\neq i,j} \frac{1}{(\lambda_i - \lambda_k)(\lambda_j - \lambda_k)} &= \sum_{k\neq i,j} \frac{1}{ c_{ik}(\lambda_i - \lambda_j) c_{jk}(\lambda_i - \lambda_j)} = \sum_{k\neq i,j} \frac{ (c_{ik}c_{jk})^{-1} }{(\lambda_i-\lambda_j)^2}.
    \end{align*}

    With this, condition \eqref{eq:separation_condition} can be written as

    \begin{align}
        \frac{2}{(\lambda_i - \lambda_j)^2} &> \sum_{k\neq i,j} \frac{ (c_{ik}c_{jk})^{-1} }{(\lambda_i-\lambda_j)^2},\\
        \Leftrightarrow 2 &> \sum_{k\neq i,j} \frac{1}{c_{ik}c_{jk}}. \label{eq:simplified_condition}
    \end{align}

    We know that the distance between $\lambda_i$ and $\lambda_j$ is minimal, in the worst-case scenario, all of the other distances are the same as $\lambda_i - \lambda_j$, so $\abs{c_{ik}},\abs{c_{ij}}>1$. Notice that if $\lambda_{i-1}$ is the function located immediately bellow $\lambda_i$, this would mean that in the worst case scenario $c_{i,i-1}=-1$ and $c_{j,i-1} = -2$. Similarly, for $\lambda_{i-2}$ we would have $c_{i,i-1} = -2, c_{j,i-2} =-3$. In general, $c_{i,i-l} = -l, c_{j,i-l} = -(l+1)$. Analogously, if $\lambda_{j+l}$ is the function located $l$ positions below $j$, then $c_{j,j+l} = l, c_{i,j+l} = l+1$.

    Now, for the arrangement of the functions, we have two extreme cases. If $\lambda_i$ and $\lambda_j$ are the functions in one of the extremes (i.e. the two biggest or two smallest ones), this would mean that the right-hand side of \eqref{eq:simplified_condition} can be written as

    \begin{align*}
        \sum_{k\neq i,j} \frac{1}{c_{ik}c_{jk}} &= \sum_{k=1}^{n-2} \frac{1}{k(k+1)} = \sum_{k=1}^{n-2} \frac{k+1 - k}{k(k+1)} = \sum_{k=1}^{n-2} \left( \frac{1}{k} - \frac{1}{k+1} \right),\\
        &= 1 - \frac{1}{n-1} = \frac{n-2}{n-1} < 2.
    \end{align*}

    So in this case, applying Lemma \ref{lemma:separation}, $\lambda_i,\lambda_j$ would repel each other. The other extreme case is when exactly half of the functions are located at each side of $\lambda_i$ and $\lambda_j$. Let us suppose first that $n$ is even, in this case we would have,

    \begin{align*}
        \sum_{k\neq i,j} \frac{1}{c_{ik}c_{jk}} &= 2 \sum_{k=1}^{\frac{n-2}{2}} \frac{1}{k(k+1)} = 2\left(1 - \frac{2}{n}\right) = 2 - \frac4n < 2.
    \end{align*}

    There is also separation in this case. Finally, if $n$ is even, we considr the previos sum for $\lfloor n/2 \rfloor$ and sum the term corresponding to the remaining function.

    \begin{equation*}
        \sum_{k\neq i,j} \frac{1}{c_{ik}c_{jk}} = 2 - \frac{4}{n} + \frac1{(\lfloor n/2\rfloor+1)(\lfloor n/2\rfloor +2)} < 2.
    \end{equation*}

    So, even in the worst-case scenario, criterion \eqref{eq:separation_condition} is satisfied, and we conclude that the minimal distance grows for every initial condition of the system.
\end{proof}

\begin{corollary}
    In a system governed by~\eqref{eq:deterministic_dyson}, for $n\ge3$ not all of the functions separate for every given initial condition. For $n =2$, the functions always repel each other.
\end{corollary}

\begin{proof}
    We prove the first part by providing a counterexample. Suppose that the separation between $\lambda_i$ and $\lambda_j$ is 1, while all of the other separations are $0.1$. The same computations that the proof of in Theorem \ref{thm:hermite_minimal_grows}, but with a re-scaling of the $c_{ik}c_{jk}$ lead to

    \begin{align*}
        \sum_{k\neq i,j} \frac{1}{c_{ik}c_{jk}} = \frac{1}{0.01}\left( 1 - \frac1{n-1}\right) = 100 - \frac{100}{n-1}.
    \end{align*}

    Clearly, for $n\ge 3$, condition~\eqref{eq:separation_condition} is satisfied.

    For the second part, we simply take that if $\lambda_i$ and $\lambda_j$ are the unique functions in the system, then 

    \begin{equation*}
        \sum_{k\neq i,j} \frac{1}{(\lambda_i - \lambda_k)(\lambda_j - \lambda_k)} = 0 < \frac{2}{(\lambda_i - \lambda_j)^2},
    \end{equation*}

    \noindent for every position of $\lambda_i, \lambda_j$.
\end{proof}

\begin{lemma} \label{lemma:separating_condition_wishart}
    Let $(\lambda_1, \lambda_2, \dots, \lambda_n)$ be a system of $n$ functions moving according to \eqref{eq:deterministic_wishart}, and let $\lambda_i, \lambda_j$ be such that $\lambda_i(t_0) > \lambda_j(t_0)$ and there is no $\lambda_k$ such that $\lambda_i(t_0) > \lambda_k(t_0) > \lambda_j(t_0)$. Then $\lambda_i$ and $\lambda_j$ repel each other if and only if 

    \begin{equation} \label{eq:separation_condition_wishart}
        \frac{\lambda_i + \lambda_j}{(\lambda_i - \lambda_j)^2} > \sum_{k\neq i,j} \frac{\lambda_k}{(\lambda_i-\lambda_k)(\lambda_j-\lambda_k)}.
    \end{equation}
\end{lemma}

\begin{proof}

    Take the derivative of the separation and use linearity with \eqref{eq:deterministic_wishart} to get

    \begin{align*}
        \frac{\d}{\d t}(\lambda_i - \lambda_j) &= \sum_{k\neq i} \frac{\lambda_i + \lambda_k}{\lambda_i - \lambda_k} - \sum_{k\neq j} \frac{\lambda_j + \lambda_k}{\lambda_j - \lambda_k}, \\
        &= \sum_{k \neq i,j} \frac{ (\lambda_i + \lambda_k)(\lambda_j - \lambda_k) - (\lambda_j + \lambda_k)(\lambda_i - \lambda_k) }{(\lambda_i - \lambda_k)(\lambda_j - \lambda_k)} + 2\frac{\lambda_i + \lambda_j}{\lambda_i - \lambda_j},\\
        &= \sum_{k\neq i,j} \frac{2\lambda_k(\lambda_j - \lambda_i)}{(\lambda_i-\lambda_k)(\lambda_j-\lambda_k)} + 2 \frac{\lambda_i + \lambda_j}{\lambda_i - \lambda_j}.
    \end{align*}

    Comparing to zero yields the result.
\end{proof}

\begin{corollary}
    In a system of $n$ functions satisfying \eqref{eq:deterministic_wishart} there is no collision of the functions.
\end{corollary}

\begin{proof}
    Use condition \eqref{eq:separation_condition_wishart} and let $\lambda_i \to \lambda_j$, then

    \begin{equation*}
        \frac{\lambda_i + \lambda_j}{(\lambda_i-\lambda_j)^2} \to \infty,
    \end{equation*}

    \noindent but

    \begin{align*}
        \sum_{k\neq i,j} \frac{\lambda_k}{(\lambda_i-\lambda_k)(\lambda_j-\lambda_k)} = \sum_{k\neq i,j} \frac{\lambda_k}{(\lambda_i-\lambda_j + \lambda_j -\lambda_k)(\lambda_j-\lambda_k)} \to \sum_{k\neq i,j} \frac{\lambda_k}{(\lambda_j-\lambda_k)^2} < \infty.
    \end{align*}

    So before the functions collide, the derivative of the separation is positive and they repel each other.
\end{proof}

\begin{theorem} \label{thm:laguerre_does_not_grow}
    Let $(\lambda_1, \lambda_2, \dots, \lambda_n)$ be a system of $n$ functions moving according to \eqref{eq:deterministic_wishart} and let $\lambda_i,\lambda_j$ be such that for a given $t_0$ it is satisfied $\lambda_i(t_0) > \lambda_j(t_0)$ and 
    
    \begin{equation*}
        \lambda_i(t_0) - \lambda_j(t_0) = \min_{k,l \in [n]} \abs{ \lambda_k(t_0) - \lambda_l(t_0)}.
    \end{equation*}

    Then, $\lambda_i$ and $\lambda_k$ do not necessarily repel. 
\end{theorem}

\begin{proof}
    We will prove providing an initial condition for which the minimal distance will have a negative derivative. Let $\lambda_j(t_0) = \min_{k \le n} \lambda_k(t_0)$ and so $\lambda_i(t_0) = \min_{k\neq j} \lambda_k(t_0)$. Similarly to the proof of Theorem \ref*{thm:hermite_minimal_grows}, let us define the quotients $c_{ik},c_{jk}$ as 

    \begin{equation*}
        c_{ik} = \frac{\lambda_i - \lambda_k}{\lambda_i - \lambda_j}, \qquad c_{jk} = \frac{\lambda_j - \lambda_k}{\lambda_i - \lambda_j}.
    \end{equation*}

    For every fixed $k$, the quotients $c_{ik}$ and $c_{jk}$ have the same sign and given the minimality of $\lambda_i - \lambda_j$ we have that $c_{ik}c_{jk}>1$. Using these quantities, the separation condition \eqref{eq:separation_condition_wishart} is reduced to 

    \begin{equation*}
        \lambda_i + \lambda_j > \sum_{k \neq i,j} \frac{\lambda_k}{c_{ik}c_{jk}}.
    \end{equation*}

    Suppose that $\lambda_i - \lambda_j =1$, and that $\lambda_k > \lambda_i$ for all $k \notin \{i,j\}$. Assume further that the separation between all the $(\lambda_k)_{k\neq i,j}$ is $1+\epsilon$ for some $\epsilon >0$. Then $c_{i,i-1}= 1+\epsilon, \abs{c_{j,i+1}}= 2 + \epsilon$ and in general $\abs{c_{i,i+l}}=l(1+\epsilon), \abs{c_{j,i+l}} = 1 + l(1+\epsilon)$. Furthermore, we have that $\lambda_{i+l}$ can be expressed as

    \begin{equation*}
         \lambda_{i+l} = \lambda_i + l(1+\epsilon) = \lambda_j + 1 + l(1+\epsilon).
    \end{equation*}

    With this, the separating condition for $\lambda_i$ and $\lambda_j$ can be written as

    \begin{equation*}
        2\lambda_j + 1 > \sum_{k=2}^{n} \frac{\lambda_j + 1 + k(1+\epsilon)}{k(1+\epsilon)(1+k(1+\epsilon))}. 
    \end{equation*}

    The left-hand side is fixed for fixed $\lambda_j(t_0)$. For the left-hand side we have

    \begin{align*}
        \sum_{k=2}^{n} \frac{\lambda_j + 1 + k(1+\epsilon)}{k(1+\epsilon)(1+k(1+\epsilon))} = \sum_{k=2}^n \frac{\lambda_j + 1}{k(1+\epsilon)(1+k(1+\epsilon))} + \sum_{k=2}^n \frac{1}{1+k(1+\epsilon)}.
    \end{align*}

    For the second element in the sum and $\epsilon$ sufficiently small we have 

    \begin{align*}
        \sum_{k=2}^n \frac{1}{1+k(1+\epsilon)} > \sum_{k=2}^n \frac{1}{1+2k} > \frac12\sum_{k=2}^n \frac{1}{1+k}.
    \end{align*}

    The last expression can be made arbitrarily big for $n$ big enough. We conclude that under these conditions, for a system with enough functions, the minimal distance can be made smaller with a specific initial condition.
    
\end{proof}

\section{Simulations}


\begin{figure}[h!]
    %% Creator: Matplotlib, PGF backend
%%
%% To include the figure in your LaTeX document, write
%%   \input{<filename>.pgf}
%%
%% Make sure the required packages are loaded in your preamble
%%   \usepackage{pgf}
%%
%% Also ensure that all the required font packages are loaded; for instance,
%% the lmodern package is sometimes necessary when using math font.
%%   \usepackage{lmodern}
%%
%% Figures using additional raster images can only be included by \input if
%% they are in the same directory as the main LaTeX file. For loading figures
%% from other directories you can use the `import` package
%%   \usepackage{import}
%%
%% and then include the figures with
%%   \import{<path to file>}{<filename>.pgf}
%%
%% Matplotlib used the following preamble
%%   
%%   \makeatletter\@ifpackageloaded{underscore}{}{\usepackage[strings]{underscore}}\makeatother
%%
\begingroup%
\makeatletter%
\begin{pgfpicture}%
\pgfpathrectangle{\pgfpointorigin}{\pgfqpoint{6.000000in}{6.000000in}}%
\pgfusepath{use as bounding box, clip}%
\begin{pgfscope}%
\pgfsetbuttcap%
\pgfsetmiterjoin%
\definecolor{currentfill}{rgb}{1.000000,1.000000,1.000000}%
\pgfsetfillcolor{currentfill}%
\pgfsetlinewidth{0.000000pt}%
\definecolor{currentstroke}{rgb}{1.000000,1.000000,1.000000}%
\pgfsetstrokecolor{currentstroke}%
\pgfsetdash{}{0pt}%
\pgfpathmoveto{\pgfqpoint{0.000000in}{0.000000in}}%
\pgfpathlineto{\pgfqpoint{6.000000in}{0.000000in}}%
\pgfpathlineto{\pgfqpoint{6.000000in}{6.000000in}}%
\pgfpathlineto{\pgfqpoint{0.000000in}{6.000000in}}%
\pgfpathlineto{\pgfqpoint{0.000000in}{0.000000in}}%
\pgfpathclose%
\pgfusepath{fill}%
\end{pgfscope}%
\begin{pgfscope}%
\pgfsetbuttcap%
\pgfsetmiterjoin%
\definecolor{currentfill}{rgb}{0.917647,0.917647,0.949020}%
\pgfsetfillcolor{currentfill}%
\pgfsetlinewidth{0.000000pt}%
\definecolor{currentstroke}{rgb}{0.000000,0.000000,0.000000}%
\pgfsetstrokecolor{currentstroke}%
\pgfsetstrokeopacity{0.000000}%
\pgfsetdash{}{0pt}%
\pgfpathmoveto{\pgfqpoint{0.750000in}{3.180000in}}%
\pgfpathlineto{\pgfqpoint{2.863636in}{3.180000in}}%
\pgfpathlineto{\pgfqpoint{2.863636in}{5.280000in}}%
\pgfpathlineto{\pgfqpoint{0.750000in}{5.280000in}}%
\pgfpathlineto{\pgfqpoint{0.750000in}{3.180000in}}%
\pgfpathclose%
\pgfusepath{fill}%
\end{pgfscope}%
\begin{pgfscope}%
\pgfpathrectangle{\pgfqpoint{0.750000in}{3.180000in}}{\pgfqpoint{2.113636in}{2.100000in}}%
\pgfusepath{clip}%
\pgfsetroundcap%
\pgfsetroundjoin%
\pgfsetlinewidth{1.003750pt}%
\definecolor{currentstroke}{rgb}{1.000000,1.000000,1.000000}%
\pgfsetstrokecolor{currentstroke}%
\pgfsetdash{}{0pt}%
\pgfpathmoveto{\pgfqpoint{0.846074in}{3.180000in}}%
\pgfpathlineto{\pgfqpoint{0.846074in}{5.280000in}}%
\pgfusepath{stroke}%
\end{pgfscope}%
\begin{pgfscope}%
\definecolor{textcolor}{rgb}{0.150000,0.150000,0.150000}%
\pgfsetstrokecolor{textcolor}%
\pgfsetfillcolor{textcolor}%
\pgftext[x=0.846074in,y=3.082778in,,top]{\color{textcolor}\rmfamily\fontsize{10.000000}{12.000000}\selectfont \(\displaystyle {0.0}\)}%
\end{pgfscope}%
\begin{pgfscope}%
\pgfpathrectangle{\pgfqpoint{0.750000in}{3.180000in}}{\pgfqpoint{2.113636in}{2.100000in}}%
\pgfusepath{clip}%
\pgfsetroundcap%
\pgfsetroundjoin%
\pgfsetlinewidth{1.003750pt}%
\definecolor{currentstroke}{rgb}{1.000000,1.000000,1.000000}%
\pgfsetstrokecolor{currentstroke}%
\pgfsetdash{}{0pt}%
\pgfpathmoveto{\pgfqpoint{1.326446in}{3.180000in}}%
\pgfpathlineto{\pgfqpoint{1.326446in}{5.280000in}}%
\pgfusepath{stroke}%
\end{pgfscope}%
\begin{pgfscope}%
\definecolor{textcolor}{rgb}{0.150000,0.150000,0.150000}%
\pgfsetstrokecolor{textcolor}%
\pgfsetfillcolor{textcolor}%
\pgftext[x=1.326446in,y=3.082778in,,top]{\color{textcolor}\rmfamily\fontsize{10.000000}{12.000000}\selectfont \(\displaystyle {2.5}\)}%
\end{pgfscope}%
\begin{pgfscope}%
\pgfpathrectangle{\pgfqpoint{0.750000in}{3.180000in}}{\pgfqpoint{2.113636in}{2.100000in}}%
\pgfusepath{clip}%
\pgfsetroundcap%
\pgfsetroundjoin%
\pgfsetlinewidth{1.003750pt}%
\definecolor{currentstroke}{rgb}{1.000000,1.000000,1.000000}%
\pgfsetstrokecolor{currentstroke}%
\pgfsetdash{}{0pt}%
\pgfpathmoveto{\pgfqpoint{1.806818in}{3.180000in}}%
\pgfpathlineto{\pgfqpoint{1.806818in}{5.280000in}}%
\pgfusepath{stroke}%
\end{pgfscope}%
\begin{pgfscope}%
\definecolor{textcolor}{rgb}{0.150000,0.150000,0.150000}%
\pgfsetstrokecolor{textcolor}%
\pgfsetfillcolor{textcolor}%
\pgftext[x=1.806818in,y=3.082778in,,top]{\color{textcolor}\rmfamily\fontsize{10.000000}{12.000000}\selectfont \(\displaystyle {5.0}\)}%
\end{pgfscope}%
\begin{pgfscope}%
\pgfpathrectangle{\pgfqpoint{0.750000in}{3.180000in}}{\pgfqpoint{2.113636in}{2.100000in}}%
\pgfusepath{clip}%
\pgfsetroundcap%
\pgfsetroundjoin%
\pgfsetlinewidth{1.003750pt}%
\definecolor{currentstroke}{rgb}{1.000000,1.000000,1.000000}%
\pgfsetstrokecolor{currentstroke}%
\pgfsetdash{}{0pt}%
\pgfpathmoveto{\pgfqpoint{2.287190in}{3.180000in}}%
\pgfpathlineto{\pgfqpoint{2.287190in}{5.280000in}}%
\pgfusepath{stroke}%
\end{pgfscope}%
\begin{pgfscope}%
\definecolor{textcolor}{rgb}{0.150000,0.150000,0.150000}%
\pgfsetstrokecolor{textcolor}%
\pgfsetfillcolor{textcolor}%
\pgftext[x=2.287190in,y=3.082778in,,top]{\color{textcolor}\rmfamily\fontsize{10.000000}{12.000000}\selectfont \(\displaystyle {7.5}\)}%
\end{pgfscope}%
\begin{pgfscope}%
\pgfpathrectangle{\pgfqpoint{0.750000in}{3.180000in}}{\pgfqpoint{2.113636in}{2.100000in}}%
\pgfusepath{clip}%
\pgfsetroundcap%
\pgfsetroundjoin%
\pgfsetlinewidth{1.003750pt}%
\definecolor{currentstroke}{rgb}{1.000000,1.000000,1.000000}%
\pgfsetstrokecolor{currentstroke}%
\pgfsetdash{}{0pt}%
\pgfpathmoveto{\pgfqpoint{2.767562in}{3.180000in}}%
\pgfpathlineto{\pgfqpoint{2.767562in}{5.280000in}}%
\pgfusepath{stroke}%
\end{pgfscope}%
\begin{pgfscope}%
\definecolor{textcolor}{rgb}{0.150000,0.150000,0.150000}%
\pgfsetstrokecolor{textcolor}%
\pgfsetfillcolor{textcolor}%
\pgftext[x=2.767562in,y=3.082778in,,top]{\color{textcolor}\rmfamily\fontsize{10.000000}{12.000000}\selectfont \(\displaystyle {10.0}\)}%
\end{pgfscope}%
\begin{pgfscope}%
\definecolor{textcolor}{rgb}{0.150000,0.150000,0.150000}%
\pgfsetstrokecolor{textcolor}%
\pgfsetfillcolor{textcolor}%
\pgftext[x=1.806818in,y=2.903766in,,top]{\color{textcolor}\rmfamily\fontsize{11.000000}{13.200000}\selectfont time (\(\displaystyle t\))}%
\end{pgfscope}%
\begin{pgfscope}%
\pgfpathrectangle{\pgfqpoint{0.750000in}{3.180000in}}{\pgfqpoint{2.113636in}{2.100000in}}%
\pgfusepath{clip}%
\pgfsetroundcap%
\pgfsetroundjoin%
\pgfsetlinewidth{1.003750pt}%
\definecolor{currentstroke}{rgb}{1.000000,1.000000,1.000000}%
\pgfsetstrokecolor{currentstroke}%
\pgfsetdash{}{0pt}%
\pgfpathmoveto{\pgfqpoint{0.750000in}{3.234177in}}%
\pgfpathlineto{\pgfqpoint{2.863636in}{3.234177in}}%
\pgfusepath{stroke}%
\end{pgfscope}%
\begin{pgfscope}%
\definecolor{textcolor}{rgb}{0.150000,0.150000,0.150000}%
\pgfsetstrokecolor{textcolor}%
\pgfsetfillcolor{textcolor}%
\pgftext[x=0.367283in, y=3.185952in, left, base]{\color{textcolor}\rmfamily\fontsize{10.000000}{12.000000}\selectfont \(\displaystyle {\ensuremath{-}7.5}\)}%
\end{pgfscope}%
\begin{pgfscope}%
\pgfpathrectangle{\pgfqpoint{0.750000in}{3.180000in}}{\pgfqpoint{2.113636in}{2.100000in}}%
\pgfusepath{clip}%
\pgfsetroundcap%
\pgfsetroundjoin%
\pgfsetlinewidth{1.003750pt}%
\definecolor{currentstroke}{rgb}{1.000000,1.000000,1.000000}%
\pgfsetstrokecolor{currentstroke}%
\pgfsetdash{}{0pt}%
\pgfpathmoveto{\pgfqpoint{0.750000in}{3.545372in}}%
\pgfpathlineto{\pgfqpoint{2.863636in}{3.545372in}}%
\pgfusepath{stroke}%
\end{pgfscope}%
\begin{pgfscope}%
\definecolor{textcolor}{rgb}{0.150000,0.150000,0.150000}%
\pgfsetstrokecolor{textcolor}%
\pgfsetfillcolor{textcolor}%
\pgftext[x=0.367283in, y=3.497146in, left, base]{\color{textcolor}\rmfamily\fontsize{10.000000}{12.000000}\selectfont \(\displaystyle {\ensuremath{-}5.0}\)}%
\end{pgfscope}%
\begin{pgfscope}%
\pgfpathrectangle{\pgfqpoint{0.750000in}{3.180000in}}{\pgfqpoint{2.113636in}{2.100000in}}%
\pgfusepath{clip}%
\pgfsetroundcap%
\pgfsetroundjoin%
\pgfsetlinewidth{1.003750pt}%
\definecolor{currentstroke}{rgb}{1.000000,1.000000,1.000000}%
\pgfsetstrokecolor{currentstroke}%
\pgfsetdash{}{0pt}%
\pgfpathmoveto{\pgfqpoint{0.750000in}{3.856566in}}%
\pgfpathlineto{\pgfqpoint{2.863636in}{3.856566in}}%
\pgfusepath{stroke}%
\end{pgfscope}%
\begin{pgfscope}%
\definecolor{textcolor}{rgb}{0.150000,0.150000,0.150000}%
\pgfsetstrokecolor{textcolor}%
\pgfsetfillcolor{textcolor}%
\pgftext[x=0.367283in, y=3.808341in, left, base]{\color{textcolor}\rmfamily\fontsize{10.000000}{12.000000}\selectfont \(\displaystyle {\ensuremath{-}2.5}\)}%
\end{pgfscope}%
\begin{pgfscope}%
\pgfpathrectangle{\pgfqpoint{0.750000in}{3.180000in}}{\pgfqpoint{2.113636in}{2.100000in}}%
\pgfusepath{clip}%
\pgfsetroundcap%
\pgfsetroundjoin%
\pgfsetlinewidth{1.003750pt}%
\definecolor{currentstroke}{rgb}{1.000000,1.000000,1.000000}%
\pgfsetstrokecolor{currentstroke}%
\pgfsetdash{}{0pt}%
\pgfpathmoveto{\pgfqpoint{0.750000in}{4.167761in}}%
\pgfpathlineto{\pgfqpoint{2.863636in}{4.167761in}}%
\pgfusepath{stroke}%
\end{pgfscope}%
\begin{pgfscope}%
\definecolor{textcolor}{rgb}{0.150000,0.150000,0.150000}%
\pgfsetstrokecolor{textcolor}%
\pgfsetfillcolor{textcolor}%
\pgftext[x=0.475308in, y=4.119536in, left, base]{\color{textcolor}\rmfamily\fontsize{10.000000}{12.000000}\selectfont \(\displaystyle {0.0}\)}%
\end{pgfscope}%
\begin{pgfscope}%
\pgfpathrectangle{\pgfqpoint{0.750000in}{3.180000in}}{\pgfqpoint{2.113636in}{2.100000in}}%
\pgfusepath{clip}%
\pgfsetroundcap%
\pgfsetroundjoin%
\pgfsetlinewidth{1.003750pt}%
\definecolor{currentstroke}{rgb}{1.000000,1.000000,1.000000}%
\pgfsetstrokecolor{currentstroke}%
\pgfsetdash{}{0pt}%
\pgfpathmoveto{\pgfqpoint{0.750000in}{4.478956in}}%
\pgfpathlineto{\pgfqpoint{2.863636in}{4.478956in}}%
\pgfusepath{stroke}%
\end{pgfscope}%
\begin{pgfscope}%
\definecolor{textcolor}{rgb}{0.150000,0.150000,0.150000}%
\pgfsetstrokecolor{textcolor}%
\pgfsetfillcolor{textcolor}%
\pgftext[x=0.475308in, y=4.430730in, left, base]{\color{textcolor}\rmfamily\fontsize{10.000000}{12.000000}\selectfont \(\displaystyle {2.5}\)}%
\end{pgfscope}%
\begin{pgfscope}%
\pgfpathrectangle{\pgfqpoint{0.750000in}{3.180000in}}{\pgfqpoint{2.113636in}{2.100000in}}%
\pgfusepath{clip}%
\pgfsetroundcap%
\pgfsetroundjoin%
\pgfsetlinewidth{1.003750pt}%
\definecolor{currentstroke}{rgb}{1.000000,1.000000,1.000000}%
\pgfsetstrokecolor{currentstroke}%
\pgfsetdash{}{0pt}%
\pgfpathmoveto{\pgfqpoint{0.750000in}{4.790150in}}%
\pgfpathlineto{\pgfqpoint{2.863636in}{4.790150in}}%
\pgfusepath{stroke}%
\end{pgfscope}%
\begin{pgfscope}%
\definecolor{textcolor}{rgb}{0.150000,0.150000,0.150000}%
\pgfsetstrokecolor{textcolor}%
\pgfsetfillcolor{textcolor}%
\pgftext[x=0.475308in, y=4.741925in, left, base]{\color{textcolor}\rmfamily\fontsize{10.000000}{12.000000}\selectfont \(\displaystyle {5.0}\)}%
\end{pgfscope}%
\begin{pgfscope}%
\pgfpathrectangle{\pgfqpoint{0.750000in}{3.180000in}}{\pgfqpoint{2.113636in}{2.100000in}}%
\pgfusepath{clip}%
\pgfsetroundcap%
\pgfsetroundjoin%
\pgfsetlinewidth{1.003750pt}%
\definecolor{currentstroke}{rgb}{1.000000,1.000000,1.000000}%
\pgfsetstrokecolor{currentstroke}%
\pgfsetdash{}{0pt}%
\pgfpathmoveto{\pgfqpoint{0.750000in}{5.101345in}}%
\pgfpathlineto{\pgfqpoint{2.863636in}{5.101345in}}%
\pgfusepath{stroke}%
\end{pgfscope}%
\begin{pgfscope}%
\definecolor{textcolor}{rgb}{0.150000,0.150000,0.150000}%
\pgfsetstrokecolor{textcolor}%
\pgfsetfillcolor{textcolor}%
\pgftext[x=0.475308in, y=5.053120in, left, base]{\color{textcolor}\rmfamily\fontsize{10.000000}{12.000000}\selectfont \(\displaystyle {7.5}\)}%
\end{pgfscope}%
\begin{pgfscope}%
\definecolor{textcolor}{rgb}{0.150000,0.150000,0.150000}%
\pgfsetstrokecolor{textcolor}%
\pgfsetfillcolor{textcolor}%
\pgftext[x=0.311727in,y=4.230000in,,bottom,rotate=90.000000]{\color{textcolor}\rmfamily\fontsize{11.000000}{13.200000}\selectfont Position}%
\end{pgfscope}%
\begin{pgfscope}%
\pgfpathrectangle{\pgfqpoint{0.750000in}{3.180000in}}{\pgfqpoint{2.113636in}{2.100000in}}%
\pgfusepath{clip}%
\pgfsetroundcap%
\pgfsetroundjoin%
\pgfsetlinewidth{1.756562pt}%
\definecolor{currentstroke}{rgb}{0.215686,0.494118,0.721569}%
\pgfsetstrokecolor{currentstroke}%
\pgfsetdash{}{0pt}%
\pgfpathmoveto{\pgfqpoint{0.846074in}{4.030835in}}%
\pgfpathlineto{\pgfqpoint{0.847998in}{4.017597in}}%
\pgfpathlineto{\pgfqpoint{0.853768in}{4.004812in}}%
\pgfpathlineto{\pgfqpoint{0.863385in}{3.989626in}}%
\pgfpathlineto{\pgfqpoint{0.876849in}{3.972746in}}%
\pgfpathlineto{\pgfqpoint{0.896083in}{3.952602in}}%
\pgfpathlineto{\pgfqpoint{0.919164in}{3.931716in}}%
\pgfpathlineto{\pgfqpoint{0.948015in}{3.908608in}}%
\pgfpathlineto{\pgfqpoint{0.982637in}{3.883732in}}%
\pgfpathlineto{\pgfqpoint{1.024952in}{3.856218in}}%
\pgfpathlineto{\pgfqpoint{1.073037in}{3.827704in}}%
\pgfpathlineto{\pgfqpoint{1.128816in}{3.797334in}}%
\pgfpathlineto{\pgfqpoint{1.190365in}{3.766386in}}%
\pgfpathlineto{\pgfqpoint{1.259608in}{3.734068in}}%
\pgfpathlineto{\pgfqpoint{1.336544in}{3.700617in}}%
\pgfpathlineto{\pgfqpoint{1.423098in}{3.665467in}}%
\pgfpathlineto{\pgfqpoint{1.517345in}{3.629618in}}%
\pgfpathlineto{\pgfqpoint{1.621209in}{3.592513in}}%
\pgfpathlineto{\pgfqpoint{1.734690in}{3.554352in}}%
\pgfpathlineto{\pgfqpoint{1.859712in}{3.514709in}}%
\pgfpathlineto{\pgfqpoint{1.994351in}{3.474373in}}%
\pgfpathlineto{\pgfqpoint{2.140530in}{3.432916in}}%
\pgfpathlineto{\pgfqpoint{2.300173in}{3.389998in}}%
\pgfpathlineto{\pgfqpoint{2.471357in}{3.346309in}}%
\pgfpathlineto{\pgfqpoint{2.656004in}{3.301501in}}%
\pgfpathlineto{\pgfqpoint{2.767562in}{3.275455in}}%
\pgfpathlineto{\pgfqpoint{2.767562in}{3.275455in}}%
\pgfusepath{stroke}%
\end{pgfscope}%
\begin{pgfscope}%
\pgfpathrectangle{\pgfqpoint{0.750000in}{3.180000in}}{\pgfqpoint{2.113636in}{2.100000in}}%
\pgfusepath{clip}%
\pgfsetroundcap%
\pgfsetroundjoin%
\pgfsetlinewidth{1.756562pt}%
\definecolor{currentstroke}{rgb}{1.000000,0.498039,0.000000}%
\pgfsetstrokecolor{currentstroke}%
\pgfsetdash{}{0pt}%
\pgfpathmoveto{\pgfqpoint{0.846074in}{4.043283in}}%
\pgfpathlineto{\pgfqpoint{0.847998in}{4.054915in}}%
\pgfpathlineto{\pgfqpoint{0.853768in}{4.062888in}}%
\pgfpathlineto{\pgfqpoint{0.861462in}{4.068966in}}%
\pgfpathlineto{\pgfqpoint{0.873002in}{4.074623in}}%
\pgfpathlineto{\pgfqpoint{0.888389in}{4.079181in}}%
\pgfpathlineto{\pgfqpoint{0.907624in}{4.082401in}}%
\pgfpathlineto{\pgfqpoint{0.932628in}{4.084310in}}%
\pgfpathlineto{\pgfqpoint{0.967249in}{4.084573in}}%
\pgfpathlineto{\pgfqpoint{1.015335in}{4.082468in}}%
\pgfpathlineto{\pgfqpoint{1.084577in}{4.076978in}}%
\pgfpathlineto{\pgfqpoint{1.201905in}{4.065134in}}%
\pgfpathlineto{\pgfqpoint{1.698145in}{4.013328in}}%
\pgfpathlineto{\pgfqpoint{1.963576in}{3.988644in}}%
\pgfpathlineto{\pgfqpoint{2.254011in}{3.963903in}}%
\pgfpathlineto{\pgfqpoint{2.579068in}{3.938488in}}%
\pgfpathlineto{\pgfqpoint{2.767562in}{3.924660in}}%
\pgfpathlineto{\pgfqpoint{2.767562in}{3.924660in}}%
\pgfusepath{stroke}%
\end{pgfscope}%
\begin{pgfscope}%
\pgfpathrectangle{\pgfqpoint{0.750000in}{3.180000in}}{\pgfqpoint{2.113636in}{2.100000in}}%
\pgfusepath{clip}%
\pgfsetroundcap%
\pgfsetroundjoin%
\pgfsetlinewidth{1.756562pt}%
\definecolor{currentstroke}{rgb}{0.301961,0.686275,0.290196}%
\pgfsetstrokecolor{currentstroke}%
\pgfsetdash{}{0pt}%
\pgfpathmoveto{\pgfqpoint{0.846074in}{4.416717in}}%
\pgfpathlineto{\pgfqpoint{0.847998in}{4.405085in}}%
\pgfpathlineto{\pgfqpoint{0.853768in}{4.397112in}}%
\pgfpathlineto{\pgfqpoint{0.861462in}{4.391034in}}%
\pgfpathlineto{\pgfqpoint{0.873002in}{4.385377in}}%
\pgfpathlineto{\pgfqpoint{0.888389in}{4.380819in}}%
\pgfpathlineto{\pgfqpoint{0.907624in}{4.377599in}}%
\pgfpathlineto{\pgfqpoint{0.932628in}{4.375690in}}%
\pgfpathlineto{\pgfqpoint{0.967249in}{4.375427in}}%
\pgfpathlineto{\pgfqpoint{1.015335in}{4.377532in}}%
\pgfpathlineto{\pgfqpoint{1.084577in}{4.383022in}}%
\pgfpathlineto{\pgfqpoint{1.201905in}{4.394866in}}%
\pgfpathlineto{\pgfqpoint{1.698145in}{4.446672in}}%
\pgfpathlineto{\pgfqpoint{1.963576in}{4.471356in}}%
\pgfpathlineto{\pgfqpoint{2.254011in}{4.496097in}}%
\pgfpathlineto{\pgfqpoint{2.579068in}{4.521512in}}%
\pgfpathlineto{\pgfqpoint{2.767562in}{4.535340in}}%
\pgfpathlineto{\pgfqpoint{2.767562in}{4.535340in}}%
\pgfusepath{stroke}%
\end{pgfscope}%
\begin{pgfscope}%
\pgfpathrectangle{\pgfqpoint{0.750000in}{3.180000in}}{\pgfqpoint{2.113636in}{2.100000in}}%
\pgfusepath{clip}%
\pgfsetroundcap%
\pgfsetroundjoin%
\pgfsetlinewidth{1.756562pt}%
\definecolor{currentstroke}{rgb}{0.968627,0.505882,0.749020}%
\pgfsetstrokecolor{currentstroke}%
\pgfsetdash{}{0pt}%
\pgfpathmoveto{\pgfqpoint{0.846074in}{4.429165in}}%
\pgfpathlineto{\pgfqpoint{0.847998in}{4.442403in}}%
\pgfpathlineto{\pgfqpoint{0.853768in}{4.455188in}}%
\pgfpathlineto{\pgfqpoint{0.863385in}{4.470374in}}%
\pgfpathlineto{\pgfqpoint{0.876849in}{4.487254in}}%
\pgfpathlineto{\pgfqpoint{0.896083in}{4.507398in}}%
\pgfpathlineto{\pgfqpoint{0.919164in}{4.528284in}}%
\pgfpathlineto{\pgfqpoint{0.948015in}{4.551392in}}%
\pgfpathlineto{\pgfqpoint{0.982637in}{4.576268in}}%
\pgfpathlineto{\pgfqpoint{1.024952in}{4.603782in}}%
\pgfpathlineto{\pgfqpoint{1.073037in}{4.632296in}}%
\pgfpathlineto{\pgfqpoint{1.128816in}{4.662666in}}%
\pgfpathlineto{\pgfqpoint{1.190365in}{4.693614in}}%
\pgfpathlineto{\pgfqpoint{1.259608in}{4.725932in}}%
\pgfpathlineto{\pgfqpoint{1.336544in}{4.759383in}}%
\pgfpathlineto{\pgfqpoint{1.423098in}{4.794533in}}%
\pgfpathlineto{\pgfqpoint{1.517345in}{4.830382in}}%
\pgfpathlineto{\pgfqpoint{1.621209in}{4.867487in}}%
\pgfpathlineto{\pgfqpoint{1.734690in}{4.905648in}}%
\pgfpathlineto{\pgfqpoint{1.859712in}{4.945291in}}%
\pgfpathlineto{\pgfqpoint{1.994351in}{4.985627in}}%
\pgfpathlineto{\pgfqpoint{2.140530in}{5.027084in}}%
\pgfpathlineto{\pgfqpoint{2.300173in}{5.070002in}}%
\pgfpathlineto{\pgfqpoint{2.471357in}{5.113691in}}%
\pgfpathlineto{\pgfqpoint{2.656004in}{5.158499in}}%
\pgfpathlineto{\pgfqpoint{2.767562in}{5.184545in}}%
\pgfpathlineto{\pgfqpoint{2.767562in}{5.184545in}}%
\pgfusepath{stroke}%
\end{pgfscope}%
\begin{pgfscope}%
\pgfsetrectcap%
\pgfsetmiterjoin%
\pgfsetlinewidth{0.000000pt}%
\definecolor{currentstroke}{rgb}{1.000000,1.000000,1.000000}%
\pgfsetstrokecolor{currentstroke}%
\pgfsetdash{}{0pt}%
\pgfpathmoveto{\pgfqpoint{0.750000in}{3.180000in}}%
\pgfpathlineto{\pgfqpoint{0.750000in}{5.280000in}}%
\pgfusepath{}%
\end{pgfscope}%
\begin{pgfscope}%
\pgfsetrectcap%
\pgfsetmiterjoin%
\pgfsetlinewidth{0.000000pt}%
\definecolor{currentstroke}{rgb}{1.000000,1.000000,1.000000}%
\pgfsetstrokecolor{currentstroke}%
\pgfsetdash{}{0pt}%
\pgfpathmoveto{\pgfqpoint{2.863636in}{3.180000in}}%
\pgfpathlineto{\pgfqpoint{2.863636in}{5.280000in}}%
\pgfusepath{}%
\end{pgfscope}%
\begin{pgfscope}%
\pgfsetrectcap%
\pgfsetmiterjoin%
\pgfsetlinewidth{0.000000pt}%
\definecolor{currentstroke}{rgb}{1.000000,1.000000,1.000000}%
\pgfsetstrokecolor{currentstroke}%
\pgfsetdash{}{0pt}%
\pgfpathmoveto{\pgfqpoint{0.750000in}{3.180000in}}%
\pgfpathlineto{\pgfqpoint{2.863636in}{3.180000in}}%
\pgfusepath{}%
\end{pgfscope}%
\begin{pgfscope}%
\pgfsetrectcap%
\pgfsetmiterjoin%
\pgfsetlinewidth{0.000000pt}%
\definecolor{currentstroke}{rgb}{1.000000,1.000000,1.000000}%
\pgfsetstrokecolor{currentstroke}%
\pgfsetdash{}{0pt}%
\pgfpathmoveto{\pgfqpoint{0.750000in}{5.280000in}}%
\pgfpathlineto{\pgfqpoint{2.863636in}{5.280000in}}%
\pgfusepath{}%
\end{pgfscope}%
\begin{pgfscope}%
\pgfsetbuttcap%
\pgfsetmiterjoin%
\definecolor{currentfill}{rgb}{0.917647,0.917647,0.949020}%
\pgfsetfillcolor{currentfill}%
\pgfsetlinewidth{0.000000pt}%
\definecolor{currentstroke}{rgb}{0.000000,0.000000,0.000000}%
\pgfsetstrokecolor{currentstroke}%
\pgfsetstrokeopacity{0.000000}%
\pgfsetdash{}{0pt}%
\pgfpathmoveto{\pgfqpoint{3.286364in}{3.180000in}}%
\pgfpathlineto{\pgfqpoint{5.400000in}{3.180000in}}%
\pgfpathlineto{\pgfqpoint{5.400000in}{5.280000in}}%
\pgfpathlineto{\pgfqpoint{3.286364in}{5.280000in}}%
\pgfpathlineto{\pgfqpoint{3.286364in}{3.180000in}}%
\pgfpathclose%
\pgfusepath{fill}%
\end{pgfscope}%
\begin{pgfscope}%
\pgfpathrectangle{\pgfqpoint{3.286364in}{3.180000in}}{\pgfqpoint{2.113636in}{2.100000in}}%
\pgfusepath{clip}%
\pgfsetroundcap%
\pgfsetroundjoin%
\pgfsetlinewidth{1.003750pt}%
\definecolor{currentstroke}{rgb}{1.000000,1.000000,1.000000}%
\pgfsetstrokecolor{currentstroke}%
\pgfsetdash{}{0pt}%
\pgfpathmoveto{\pgfqpoint{3.382438in}{3.180000in}}%
\pgfpathlineto{\pgfqpoint{3.382438in}{5.280000in}}%
\pgfusepath{stroke}%
\end{pgfscope}%
\begin{pgfscope}%
\definecolor{textcolor}{rgb}{0.150000,0.150000,0.150000}%
\pgfsetstrokecolor{textcolor}%
\pgfsetfillcolor{textcolor}%
\pgftext[x=3.382438in,y=3.082778in,,top]{\color{textcolor}\rmfamily\fontsize{10.000000}{12.000000}\selectfont \(\displaystyle {0.0}\)}%
\end{pgfscope}%
\begin{pgfscope}%
\pgfpathrectangle{\pgfqpoint{3.286364in}{3.180000in}}{\pgfqpoint{2.113636in}{2.100000in}}%
\pgfusepath{clip}%
\pgfsetroundcap%
\pgfsetroundjoin%
\pgfsetlinewidth{1.003750pt}%
\definecolor{currentstroke}{rgb}{1.000000,1.000000,1.000000}%
\pgfsetstrokecolor{currentstroke}%
\pgfsetdash{}{0pt}%
\pgfpathmoveto{\pgfqpoint{3.862810in}{3.180000in}}%
\pgfpathlineto{\pgfqpoint{3.862810in}{5.280000in}}%
\pgfusepath{stroke}%
\end{pgfscope}%
\begin{pgfscope}%
\definecolor{textcolor}{rgb}{0.150000,0.150000,0.150000}%
\pgfsetstrokecolor{textcolor}%
\pgfsetfillcolor{textcolor}%
\pgftext[x=3.862810in,y=3.082778in,,top]{\color{textcolor}\rmfamily\fontsize{10.000000}{12.000000}\selectfont \(\displaystyle {2.5}\)}%
\end{pgfscope}%
\begin{pgfscope}%
\pgfpathrectangle{\pgfqpoint{3.286364in}{3.180000in}}{\pgfqpoint{2.113636in}{2.100000in}}%
\pgfusepath{clip}%
\pgfsetroundcap%
\pgfsetroundjoin%
\pgfsetlinewidth{1.003750pt}%
\definecolor{currentstroke}{rgb}{1.000000,1.000000,1.000000}%
\pgfsetstrokecolor{currentstroke}%
\pgfsetdash{}{0pt}%
\pgfpathmoveto{\pgfqpoint{4.343182in}{3.180000in}}%
\pgfpathlineto{\pgfqpoint{4.343182in}{5.280000in}}%
\pgfusepath{stroke}%
\end{pgfscope}%
\begin{pgfscope}%
\definecolor{textcolor}{rgb}{0.150000,0.150000,0.150000}%
\pgfsetstrokecolor{textcolor}%
\pgfsetfillcolor{textcolor}%
\pgftext[x=4.343182in,y=3.082778in,,top]{\color{textcolor}\rmfamily\fontsize{10.000000}{12.000000}\selectfont \(\displaystyle {5.0}\)}%
\end{pgfscope}%
\begin{pgfscope}%
\pgfpathrectangle{\pgfqpoint{3.286364in}{3.180000in}}{\pgfqpoint{2.113636in}{2.100000in}}%
\pgfusepath{clip}%
\pgfsetroundcap%
\pgfsetroundjoin%
\pgfsetlinewidth{1.003750pt}%
\definecolor{currentstroke}{rgb}{1.000000,1.000000,1.000000}%
\pgfsetstrokecolor{currentstroke}%
\pgfsetdash{}{0pt}%
\pgfpathmoveto{\pgfqpoint{4.823554in}{3.180000in}}%
\pgfpathlineto{\pgfqpoint{4.823554in}{5.280000in}}%
\pgfusepath{stroke}%
\end{pgfscope}%
\begin{pgfscope}%
\definecolor{textcolor}{rgb}{0.150000,0.150000,0.150000}%
\pgfsetstrokecolor{textcolor}%
\pgfsetfillcolor{textcolor}%
\pgftext[x=4.823554in,y=3.082778in,,top]{\color{textcolor}\rmfamily\fontsize{10.000000}{12.000000}\selectfont \(\displaystyle {7.5}\)}%
\end{pgfscope}%
\begin{pgfscope}%
\pgfpathrectangle{\pgfqpoint{3.286364in}{3.180000in}}{\pgfqpoint{2.113636in}{2.100000in}}%
\pgfusepath{clip}%
\pgfsetroundcap%
\pgfsetroundjoin%
\pgfsetlinewidth{1.003750pt}%
\definecolor{currentstroke}{rgb}{1.000000,1.000000,1.000000}%
\pgfsetstrokecolor{currentstroke}%
\pgfsetdash{}{0pt}%
\pgfpathmoveto{\pgfqpoint{5.303926in}{3.180000in}}%
\pgfpathlineto{\pgfqpoint{5.303926in}{5.280000in}}%
\pgfusepath{stroke}%
\end{pgfscope}%
\begin{pgfscope}%
\definecolor{textcolor}{rgb}{0.150000,0.150000,0.150000}%
\pgfsetstrokecolor{textcolor}%
\pgfsetfillcolor{textcolor}%
\pgftext[x=5.303926in,y=3.082778in,,top]{\color{textcolor}\rmfamily\fontsize{10.000000}{12.000000}\selectfont \(\displaystyle {10.0}\)}%
\end{pgfscope}%
\begin{pgfscope}%
\definecolor{textcolor}{rgb}{0.150000,0.150000,0.150000}%
\pgfsetstrokecolor{textcolor}%
\pgfsetfillcolor{textcolor}%
\pgftext[x=4.343182in,y=2.903766in,,top]{\color{textcolor}\rmfamily\fontsize{11.000000}{13.200000}\selectfont time (\(\displaystyle t\))}%
\end{pgfscope}%
\begin{pgfscope}%
\pgfpathrectangle{\pgfqpoint{3.286364in}{3.180000in}}{\pgfqpoint{2.113636in}{2.100000in}}%
\pgfusepath{clip}%
\pgfsetroundcap%
\pgfsetroundjoin%
\pgfsetlinewidth{1.003750pt}%
\definecolor{currentstroke}{rgb}{1.000000,1.000000,1.000000}%
\pgfsetstrokecolor{currentstroke}%
\pgfsetdash{}{0pt}%
\pgfpathmoveto{\pgfqpoint{3.286364in}{3.180043in}}%
\pgfpathlineto{\pgfqpoint{5.400000in}{3.180043in}}%
\pgfusepath{stroke}%
\end{pgfscope}%
\begin{pgfscope}%
\definecolor{textcolor}{rgb}{0.150000,0.150000,0.150000}%
\pgfsetstrokecolor{textcolor}%
\pgfsetfillcolor{textcolor}%
\pgftext[x=2.942227in, y=3.131817in, left, base]{\color{textcolor}\rmfamily\fontsize{10.000000}{12.000000}\selectfont \(\displaystyle {\ensuremath{-}10}\)}%
\end{pgfscope}%
\begin{pgfscope}%
\pgfpathrectangle{\pgfqpoint{3.286364in}{3.180000in}}{\pgfqpoint{2.113636in}{2.100000in}}%
\pgfusepath{clip}%
\pgfsetroundcap%
\pgfsetroundjoin%
\pgfsetlinewidth{1.003750pt}%
\definecolor{currentstroke}{rgb}{1.000000,1.000000,1.000000}%
\pgfsetstrokecolor{currentstroke}%
\pgfsetdash{}{0pt}%
\pgfpathmoveto{\pgfqpoint{3.286364in}{3.705021in}}%
\pgfpathlineto{\pgfqpoint{5.400000in}{3.705021in}}%
\pgfusepath{stroke}%
\end{pgfscope}%
\begin{pgfscope}%
\definecolor{textcolor}{rgb}{0.150000,0.150000,0.150000}%
\pgfsetstrokecolor{textcolor}%
\pgfsetfillcolor{textcolor}%
\pgftext[x=3.011672in, y=3.656796in, left, base]{\color{textcolor}\rmfamily\fontsize{10.000000}{12.000000}\selectfont \(\displaystyle {\ensuremath{-}5}\)}%
\end{pgfscope}%
\begin{pgfscope}%
\pgfpathrectangle{\pgfqpoint{3.286364in}{3.180000in}}{\pgfqpoint{2.113636in}{2.100000in}}%
\pgfusepath{clip}%
\pgfsetroundcap%
\pgfsetroundjoin%
\pgfsetlinewidth{1.003750pt}%
\definecolor{currentstroke}{rgb}{1.000000,1.000000,1.000000}%
\pgfsetstrokecolor{currentstroke}%
\pgfsetdash{}{0pt}%
\pgfpathmoveto{\pgfqpoint{3.286364in}{4.230000in}}%
\pgfpathlineto{\pgfqpoint{5.400000in}{4.230000in}}%
\pgfusepath{stroke}%
\end{pgfscope}%
\begin{pgfscope}%
\definecolor{textcolor}{rgb}{0.150000,0.150000,0.150000}%
\pgfsetstrokecolor{textcolor}%
\pgfsetfillcolor{textcolor}%
\pgftext[x=3.119697in, y=4.181775in, left, base]{\color{textcolor}\rmfamily\fontsize{10.000000}{12.000000}\selectfont \(\displaystyle {0}\)}%
\end{pgfscope}%
\begin{pgfscope}%
\pgfpathrectangle{\pgfqpoint{3.286364in}{3.180000in}}{\pgfqpoint{2.113636in}{2.100000in}}%
\pgfusepath{clip}%
\pgfsetroundcap%
\pgfsetroundjoin%
\pgfsetlinewidth{1.003750pt}%
\definecolor{currentstroke}{rgb}{1.000000,1.000000,1.000000}%
\pgfsetstrokecolor{currentstroke}%
\pgfsetdash{}{0pt}%
\pgfpathmoveto{\pgfqpoint{3.286364in}{4.754979in}}%
\pgfpathlineto{\pgfqpoint{5.400000in}{4.754979in}}%
\pgfusepath{stroke}%
\end{pgfscope}%
\begin{pgfscope}%
\definecolor{textcolor}{rgb}{0.150000,0.150000,0.150000}%
\pgfsetstrokecolor{textcolor}%
\pgfsetfillcolor{textcolor}%
\pgftext[x=3.119697in, y=4.706753in, left, base]{\color{textcolor}\rmfamily\fontsize{10.000000}{12.000000}\selectfont \(\displaystyle {5}\)}%
\end{pgfscope}%
\begin{pgfscope}%
\pgfpathrectangle{\pgfqpoint{3.286364in}{3.180000in}}{\pgfqpoint{2.113636in}{2.100000in}}%
\pgfusepath{clip}%
\pgfsetroundcap%
\pgfsetroundjoin%
\pgfsetlinewidth{1.003750pt}%
\definecolor{currentstroke}{rgb}{1.000000,1.000000,1.000000}%
\pgfsetstrokecolor{currentstroke}%
\pgfsetdash{}{0pt}%
\pgfpathmoveto{\pgfqpoint{3.286364in}{5.279957in}}%
\pgfpathlineto{\pgfqpoint{5.400000in}{5.279957in}}%
\pgfusepath{stroke}%
\end{pgfscope}%
\begin{pgfscope}%
\definecolor{textcolor}{rgb}{0.150000,0.150000,0.150000}%
\pgfsetstrokecolor{textcolor}%
\pgfsetfillcolor{textcolor}%
\pgftext[x=3.050252in, y=5.231732in, left, base]{\color{textcolor}\rmfamily\fontsize{10.000000}{12.000000}\selectfont \(\displaystyle {10}\)}%
\end{pgfscope}%
\begin{pgfscope}%
\definecolor{textcolor}{rgb}{0.150000,0.150000,0.150000}%
\pgfsetstrokecolor{textcolor}%
\pgfsetfillcolor{textcolor}%
\pgftext[x=2.886672in,y=4.230000in,,bottom,rotate=90.000000]{\color{textcolor}\rmfamily\fontsize{11.000000}{13.200000}\selectfont Position}%
\end{pgfscope}%
\begin{pgfscope}%
\pgfpathrectangle{\pgfqpoint{3.286364in}{3.180000in}}{\pgfqpoint{2.113636in}{2.100000in}}%
\pgfusepath{clip}%
\pgfsetroundcap%
\pgfsetroundjoin%
\pgfsetlinewidth{1.756562pt}%
\definecolor{currentstroke}{rgb}{0.215686,0.494118,0.721569}%
\pgfsetstrokecolor{currentstroke}%
\pgfsetdash{}{0pt}%
\pgfpathmoveto{\pgfqpoint{3.382438in}{4.125004in}}%
\pgfpathlineto{\pgfqpoint{3.393978in}{4.101173in}}%
\pgfpathlineto{\pgfqpoint{3.407442in}{4.078308in}}%
\pgfpathlineto{\pgfqpoint{3.424753in}{4.053351in}}%
\pgfpathlineto{\pgfqpoint{3.445911in}{4.027063in}}%
\pgfpathlineto{\pgfqpoint{3.470915in}{3.999891in}}%
\pgfpathlineto{\pgfqpoint{3.499766in}{3.972109in}}%
\pgfpathlineto{\pgfqpoint{3.532464in}{3.943892in}}%
\pgfpathlineto{\pgfqpoint{3.569009in}{3.915354in}}%
\pgfpathlineto{\pgfqpoint{3.611324in}{3.885264in}}%
\pgfpathlineto{\pgfqpoint{3.657486in}{3.855195in}}%
\pgfpathlineto{\pgfqpoint{3.709418in}{3.824029in}}%
\pgfpathlineto{\pgfqpoint{3.767120in}{3.792001in}}%
\pgfpathlineto{\pgfqpoint{3.832516in}{3.758331in}}%
\pgfpathlineto{\pgfqpoint{3.903682in}{3.724236in}}%
\pgfpathlineto{\pgfqpoint{3.982542in}{3.688963in}}%
\pgfpathlineto{\pgfqpoint{4.069096in}{3.652728in}}%
\pgfpathlineto{\pgfqpoint{4.163343in}{3.615701in}}%
\pgfpathlineto{\pgfqpoint{4.267207in}{3.577328in}}%
\pgfpathlineto{\pgfqpoint{4.380688in}{3.537831in}}%
\pgfpathlineto{\pgfqpoint{4.503787in}{3.497394in}}%
\pgfpathlineto{\pgfqpoint{4.636502in}{3.456162in}}%
\pgfpathlineto{\pgfqpoint{4.780758in}{3.413706in}}%
\pgfpathlineto{\pgfqpoint{4.936554in}{3.370207in}}%
\pgfpathlineto{\pgfqpoint{5.103891in}{3.325815in}}%
\pgfpathlineto{\pgfqpoint{5.284692in}{3.280181in}}%
\pgfpathlineto{\pgfqpoint{5.303926in}{3.275455in}}%
\pgfpathlineto{\pgfqpoint{5.303926in}{3.275455in}}%
\pgfusepath{stroke}%
\end{pgfscope}%
\begin{pgfscope}%
\pgfpathrectangle{\pgfqpoint{3.286364in}{3.180000in}}{\pgfqpoint{2.113636in}{2.100000in}}%
\pgfusepath{clip}%
\pgfsetroundcap%
\pgfsetroundjoin%
\pgfsetlinewidth{1.756562pt}%
\definecolor{currentstroke}{rgb}{1.000000,0.498039,0.000000}%
\pgfsetstrokecolor{currentstroke}%
\pgfsetdash{}{0pt}%
\pgfpathmoveto{\pgfqpoint{3.382438in}{4.177502in}}%
\pgfpathlineto{\pgfqpoint{3.401672in}{4.161421in}}%
\pgfpathlineto{\pgfqpoint{3.422830in}{4.146859in}}%
\pgfpathlineto{\pgfqpoint{3.449757in}{4.131290in}}%
\pgfpathlineto{\pgfqpoint{3.482455in}{4.115155in}}%
\pgfpathlineto{\pgfqpoint{3.522847in}{4.097907in}}%
\pgfpathlineto{\pgfqpoint{3.570932in}{4.079933in}}%
\pgfpathlineto{\pgfqpoint{3.628635in}{4.060869in}}%
\pgfpathlineto{\pgfqpoint{3.695954in}{4.041046in}}%
\pgfpathlineto{\pgfqpoint{3.774814in}{4.020195in}}%
\pgfpathlineto{\pgfqpoint{3.867138in}{3.998158in}}%
\pgfpathlineto{\pgfqpoint{3.972925in}{3.975243in}}%
\pgfpathlineto{\pgfqpoint{4.094100in}{3.951302in}}%
\pgfpathlineto{\pgfqpoint{4.232586in}{3.926246in}}%
\pgfpathlineto{\pgfqpoint{4.390305in}{3.900020in}}%
\pgfpathlineto{\pgfqpoint{4.567259in}{3.872883in}}%
\pgfpathlineto{\pgfqpoint{4.767294in}{3.844501in}}%
\pgfpathlineto{\pgfqpoint{4.990410in}{3.815130in}}%
\pgfpathlineto{\pgfqpoint{5.238530in}{3.784738in}}%
\pgfpathlineto{\pgfqpoint{5.303926in}{3.777067in}}%
\pgfpathlineto{\pgfqpoint{5.303926in}{3.777067in}}%
\pgfusepath{stroke}%
\end{pgfscope}%
\begin{pgfscope}%
\pgfpathrectangle{\pgfqpoint{3.286364in}{3.180000in}}{\pgfqpoint{2.113636in}{2.100000in}}%
\pgfusepath{clip}%
\pgfsetroundcap%
\pgfsetroundjoin%
\pgfsetlinewidth{1.756562pt}%
\definecolor{currentstroke}{rgb}{0.301961,0.686275,0.290196}%
\pgfsetstrokecolor{currentstroke}%
\pgfsetdash{}{0pt}%
\pgfpathmoveto{\pgfqpoint{3.382438in}{4.230000in}}%
\pgfpathlineto{\pgfqpoint{5.303926in}{4.230000in}}%
\pgfpathlineto{\pgfqpoint{5.303926in}{4.230000in}}%
\pgfusepath{stroke}%
\end{pgfscope}%
\begin{pgfscope}%
\pgfpathrectangle{\pgfqpoint{3.286364in}{3.180000in}}{\pgfqpoint{2.113636in}{2.100000in}}%
\pgfusepath{clip}%
\pgfsetroundcap%
\pgfsetroundjoin%
\pgfsetlinewidth{1.756562pt}%
\definecolor{currentstroke}{rgb}{0.968627,0.505882,0.749020}%
\pgfsetstrokecolor{currentstroke}%
\pgfsetdash{}{0pt}%
\pgfpathmoveto{\pgfqpoint{3.382438in}{4.282498in}}%
\pgfpathlineto{\pgfqpoint{3.401672in}{4.298579in}}%
\pgfpathlineto{\pgfqpoint{3.422830in}{4.313141in}}%
\pgfpathlineto{\pgfqpoint{3.449757in}{4.328710in}}%
\pgfpathlineto{\pgfqpoint{3.482455in}{4.344845in}}%
\pgfpathlineto{\pgfqpoint{3.522847in}{4.362093in}}%
\pgfpathlineto{\pgfqpoint{3.570932in}{4.380067in}}%
\pgfpathlineto{\pgfqpoint{3.628635in}{4.399131in}}%
\pgfpathlineto{\pgfqpoint{3.695954in}{4.418954in}}%
\pgfpathlineto{\pgfqpoint{3.774814in}{4.439805in}}%
\pgfpathlineto{\pgfqpoint{3.867138in}{4.461842in}}%
\pgfpathlineto{\pgfqpoint{3.972925in}{4.484757in}}%
\pgfpathlineto{\pgfqpoint{4.094100in}{4.508698in}}%
\pgfpathlineto{\pgfqpoint{4.232586in}{4.533754in}}%
\pgfpathlineto{\pgfqpoint{4.390305in}{4.559980in}}%
\pgfpathlineto{\pgfqpoint{4.567259in}{4.587117in}}%
\pgfpathlineto{\pgfqpoint{4.767294in}{4.615499in}}%
\pgfpathlineto{\pgfqpoint{4.990410in}{4.644870in}}%
\pgfpathlineto{\pgfqpoint{5.238530in}{4.675262in}}%
\pgfpathlineto{\pgfqpoint{5.303926in}{4.682933in}}%
\pgfpathlineto{\pgfqpoint{5.303926in}{4.682933in}}%
\pgfusepath{stroke}%
\end{pgfscope}%
\begin{pgfscope}%
\pgfpathrectangle{\pgfqpoint{3.286364in}{3.180000in}}{\pgfqpoint{2.113636in}{2.100000in}}%
\pgfusepath{clip}%
\pgfsetroundcap%
\pgfsetroundjoin%
\pgfsetlinewidth{1.756562pt}%
\definecolor{currentstroke}{rgb}{0.650980,0.337255,0.156863}%
\pgfsetstrokecolor{currentstroke}%
\pgfsetdash{}{0pt}%
\pgfpathmoveto{\pgfqpoint{3.382438in}{4.334996in}}%
\pgfpathlineto{\pgfqpoint{3.393978in}{4.358827in}}%
\pgfpathlineto{\pgfqpoint{3.407442in}{4.381692in}}%
\pgfpathlineto{\pgfqpoint{3.424753in}{4.406649in}}%
\pgfpathlineto{\pgfqpoint{3.445911in}{4.432937in}}%
\pgfpathlineto{\pgfqpoint{3.470915in}{4.460109in}}%
\pgfpathlineto{\pgfqpoint{3.499766in}{4.487891in}}%
\pgfpathlineto{\pgfqpoint{3.532464in}{4.516108in}}%
\pgfpathlineto{\pgfqpoint{3.569009in}{4.544646in}}%
\pgfpathlineto{\pgfqpoint{3.611324in}{4.574736in}}%
\pgfpathlineto{\pgfqpoint{3.657486in}{4.604805in}}%
\pgfpathlineto{\pgfqpoint{3.709418in}{4.635971in}}%
\pgfpathlineto{\pgfqpoint{3.767120in}{4.667999in}}%
\pgfpathlineto{\pgfqpoint{3.832516in}{4.701669in}}%
\pgfpathlineto{\pgfqpoint{3.903682in}{4.735764in}}%
\pgfpathlineto{\pgfqpoint{3.982542in}{4.771037in}}%
\pgfpathlineto{\pgfqpoint{4.069096in}{4.807272in}}%
\pgfpathlineto{\pgfqpoint{4.163343in}{4.844299in}}%
\pgfpathlineto{\pgfqpoint{4.267207in}{4.882672in}}%
\pgfpathlineto{\pgfqpoint{4.380688in}{4.922169in}}%
\pgfpathlineto{\pgfqpoint{4.503787in}{4.962606in}}%
\pgfpathlineto{\pgfqpoint{4.636502in}{5.003838in}}%
\pgfpathlineto{\pgfqpoint{4.780758in}{5.046294in}}%
\pgfpathlineto{\pgfqpoint{4.936554in}{5.089793in}}%
\pgfpathlineto{\pgfqpoint{5.103891in}{5.134185in}}%
\pgfpathlineto{\pgfqpoint{5.284692in}{5.179819in}}%
\pgfpathlineto{\pgfqpoint{5.303926in}{5.184545in}}%
\pgfpathlineto{\pgfqpoint{5.303926in}{5.184545in}}%
\pgfusepath{stroke}%
\end{pgfscope}%
\begin{pgfscope}%
\pgfsetrectcap%
\pgfsetmiterjoin%
\pgfsetlinewidth{0.000000pt}%
\definecolor{currentstroke}{rgb}{1.000000,1.000000,1.000000}%
\pgfsetstrokecolor{currentstroke}%
\pgfsetdash{}{0pt}%
\pgfpathmoveto{\pgfqpoint{3.286364in}{3.180000in}}%
\pgfpathlineto{\pgfqpoint{3.286364in}{5.280000in}}%
\pgfusepath{}%
\end{pgfscope}%
\begin{pgfscope}%
\pgfsetrectcap%
\pgfsetmiterjoin%
\pgfsetlinewidth{0.000000pt}%
\definecolor{currentstroke}{rgb}{1.000000,1.000000,1.000000}%
\pgfsetstrokecolor{currentstroke}%
\pgfsetdash{}{0pt}%
\pgfpathmoveto{\pgfqpoint{5.400000in}{3.180000in}}%
\pgfpathlineto{\pgfqpoint{5.400000in}{5.280000in}}%
\pgfusepath{}%
\end{pgfscope}%
\begin{pgfscope}%
\pgfsetrectcap%
\pgfsetmiterjoin%
\pgfsetlinewidth{0.000000pt}%
\definecolor{currentstroke}{rgb}{1.000000,1.000000,1.000000}%
\pgfsetstrokecolor{currentstroke}%
\pgfsetdash{}{0pt}%
\pgfpathmoveto{\pgfqpoint{3.286364in}{3.180000in}}%
\pgfpathlineto{\pgfqpoint{5.400000in}{3.180000in}}%
\pgfusepath{}%
\end{pgfscope}%
\begin{pgfscope}%
\pgfsetrectcap%
\pgfsetmiterjoin%
\pgfsetlinewidth{0.000000pt}%
\definecolor{currentstroke}{rgb}{1.000000,1.000000,1.000000}%
\pgfsetstrokecolor{currentstroke}%
\pgfsetdash{}{0pt}%
\pgfpathmoveto{\pgfqpoint{3.286364in}{5.280000in}}%
\pgfpathlineto{\pgfqpoint{5.400000in}{5.280000in}}%
\pgfusepath{}%
\end{pgfscope}%
\begin{pgfscope}%
\pgfsetbuttcap%
\pgfsetmiterjoin%
\definecolor{currentfill}{rgb}{0.917647,0.917647,0.949020}%
\pgfsetfillcolor{currentfill}%
\pgfsetlinewidth{0.000000pt}%
\definecolor{currentstroke}{rgb}{0.000000,0.000000,0.000000}%
\pgfsetstrokecolor{currentstroke}%
\pgfsetstrokeopacity{0.000000}%
\pgfsetdash{}{0pt}%
\pgfpathmoveto{\pgfqpoint{0.750000in}{0.660000in}}%
\pgfpathlineto{\pgfqpoint{2.863636in}{0.660000in}}%
\pgfpathlineto{\pgfqpoint{2.863636in}{2.760000in}}%
\pgfpathlineto{\pgfqpoint{0.750000in}{2.760000in}}%
\pgfpathlineto{\pgfqpoint{0.750000in}{0.660000in}}%
\pgfpathclose%
\pgfusepath{fill}%
\end{pgfscope}%
\begin{pgfscope}%
\pgfpathrectangle{\pgfqpoint{0.750000in}{0.660000in}}{\pgfqpoint{2.113636in}{2.100000in}}%
\pgfusepath{clip}%
\pgfsetroundcap%
\pgfsetroundjoin%
\pgfsetlinewidth{1.003750pt}%
\definecolor{currentstroke}{rgb}{1.000000,1.000000,1.000000}%
\pgfsetstrokecolor{currentstroke}%
\pgfsetdash{}{0pt}%
\pgfpathmoveto{\pgfqpoint{0.846074in}{0.660000in}}%
\pgfpathlineto{\pgfqpoint{0.846074in}{2.760000in}}%
\pgfusepath{stroke}%
\end{pgfscope}%
\begin{pgfscope}%
\definecolor{textcolor}{rgb}{0.150000,0.150000,0.150000}%
\pgfsetstrokecolor{textcolor}%
\pgfsetfillcolor{textcolor}%
\pgftext[x=0.846074in,y=0.562778in,,top]{\color{textcolor}\rmfamily\fontsize{10.000000}{12.000000}\selectfont \(\displaystyle {0.0}\)}%
\end{pgfscope}%
\begin{pgfscope}%
\pgfpathrectangle{\pgfqpoint{0.750000in}{0.660000in}}{\pgfqpoint{2.113636in}{2.100000in}}%
\pgfusepath{clip}%
\pgfsetroundcap%
\pgfsetroundjoin%
\pgfsetlinewidth{1.003750pt}%
\definecolor{currentstroke}{rgb}{1.000000,1.000000,1.000000}%
\pgfsetstrokecolor{currentstroke}%
\pgfsetdash{}{0pt}%
\pgfpathmoveto{\pgfqpoint{1.326446in}{0.660000in}}%
\pgfpathlineto{\pgfqpoint{1.326446in}{2.760000in}}%
\pgfusepath{stroke}%
\end{pgfscope}%
\begin{pgfscope}%
\definecolor{textcolor}{rgb}{0.150000,0.150000,0.150000}%
\pgfsetstrokecolor{textcolor}%
\pgfsetfillcolor{textcolor}%
\pgftext[x=1.326446in,y=0.562778in,,top]{\color{textcolor}\rmfamily\fontsize{10.000000}{12.000000}\selectfont \(\displaystyle {2.5}\)}%
\end{pgfscope}%
\begin{pgfscope}%
\pgfpathrectangle{\pgfqpoint{0.750000in}{0.660000in}}{\pgfqpoint{2.113636in}{2.100000in}}%
\pgfusepath{clip}%
\pgfsetroundcap%
\pgfsetroundjoin%
\pgfsetlinewidth{1.003750pt}%
\definecolor{currentstroke}{rgb}{1.000000,1.000000,1.000000}%
\pgfsetstrokecolor{currentstroke}%
\pgfsetdash{}{0pt}%
\pgfpathmoveto{\pgfqpoint{1.806818in}{0.660000in}}%
\pgfpathlineto{\pgfqpoint{1.806818in}{2.760000in}}%
\pgfusepath{stroke}%
\end{pgfscope}%
\begin{pgfscope}%
\definecolor{textcolor}{rgb}{0.150000,0.150000,0.150000}%
\pgfsetstrokecolor{textcolor}%
\pgfsetfillcolor{textcolor}%
\pgftext[x=1.806818in,y=0.562778in,,top]{\color{textcolor}\rmfamily\fontsize{10.000000}{12.000000}\selectfont \(\displaystyle {5.0}\)}%
\end{pgfscope}%
\begin{pgfscope}%
\pgfpathrectangle{\pgfqpoint{0.750000in}{0.660000in}}{\pgfqpoint{2.113636in}{2.100000in}}%
\pgfusepath{clip}%
\pgfsetroundcap%
\pgfsetroundjoin%
\pgfsetlinewidth{1.003750pt}%
\definecolor{currentstroke}{rgb}{1.000000,1.000000,1.000000}%
\pgfsetstrokecolor{currentstroke}%
\pgfsetdash{}{0pt}%
\pgfpathmoveto{\pgfqpoint{2.287190in}{0.660000in}}%
\pgfpathlineto{\pgfqpoint{2.287190in}{2.760000in}}%
\pgfusepath{stroke}%
\end{pgfscope}%
\begin{pgfscope}%
\definecolor{textcolor}{rgb}{0.150000,0.150000,0.150000}%
\pgfsetstrokecolor{textcolor}%
\pgfsetfillcolor{textcolor}%
\pgftext[x=2.287190in,y=0.562778in,,top]{\color{textcolor}\rmfamily\fontsize{10.000000}{12.000000}\selectfont \(\displaystyle {7.5}\)}%
\end{pgfscope}%
\begin{pgfscope}%
\pgfpathrectangle{\pgfqpoint{0.750000in}{0.660000in}}{\pgfqpoint{2.113636in}{2.100000in}}%
\pgfusepath{clip}%
\pgfsetroundcap%
\pgfsetroundjoin%
\pgfsetlinewidth{1.003750pt}%
\definecolor{currentstroke}{rgb}{1.000000,1.000000,1.000000}%
\pgfsetstrokecolor{currentstroke}%
\pgfsetdash{}{0pt}%
\pgfpathmoveto{\pgfqpoint{2.767562in}{0.660000in}}%
\pgfpathlineto{\pgfqpoint{2.767562in}{2.760000in}}%
\pgfusepath{stroke}%
\end{pgfscope}%
\begin{pgfscope}%
\definecolor{textcolor}{rgb}{0.150000,0.150000,0.150000}%
\pgfsetstrokecolor{textcolor}%
\pgfsetfillcolor{textcolor}%
\pgftext[x=2.767562in,y=0.562778in,,top]{\color{textcolor}\rmfamily\fontsize{10.000000}{12.000000}\selectfont \(\displaystyle {10.0}\)}%
\end{pgfscope}%
\begin{pgfscope}%
\definecolor{textcolor}{rgb}{0.150000,0.150000,0.150000}%
\pgfsetstrokecolor{textcolor}%
\pgfsetfillcolor{textcolor}%
\pgftext[x=1.806818in,y=0.383766in,,top]{\color{textcolor}\rmfamily\fontsize{11.000000}{13.200000}\selectfont time (\(\displaystyle t\))}%
\end{pgfscope}%
\begin{pgfscope}%
\pgfpathrectangle{\pgfqpoint{0.750000in}{0.660000in}}{\pgfqpoint{2.113636in}{2.100000in}}%
\pgfusepath{clip}%
\pgfsetroundcap%
\pgfsetroundjoin%
\pgfsetlinewidth{1.003750pt}%
\definecolor{currentstroke}{rgb}{1.000000,1.000000,1.000000}%
\pgfsetstrokecolor{currentstroke}%
\pgfsetdash{}{0pt}%
\pgfpathmoveto{\pgfqpoint{0.750000in}{0.923048in}}%
\pgfpathlineto{\pgfqpoint{2.863636in}{0.923048in}}%
\pgfusepath{stroke}%
\end{pgfscope}%
\begin{pgfscope}%
\definecolor{textcolor}{rgb}{0.150000,0.150000,0.150000}%
\pgfsetstrokecolor{textcolor}%
\pgfsetfillcolor{textcolor}%
\pgftext[x=0.405863in, y=0.874823in, left, base]{\color{textcolor}\rmfamily\fontsize{10.000000}{12.000000}\selectfont \(\displaystyle {\ensuremath{-}10}\)}%
\end{pgfscope}%
\begin{pgfscope}%
\pgfpathrectangle{\pgfqpoint{0.750000in}{0.660000in}}{\pgfqpoint{2.113636in}{2.100000in}}%
\pgfusepath{clip}%
\pgfsetroundcap%
\pgfsetroundjoin%
\pgfsetlinewidth{1.003750pt}%
\definecolor{currentstroke}{rgb}{1.000000,1.000000,1.000000}%
\pgfsetstrokecolor{currentstroke}%
\pgfsetdash{}{0pt}%
\pgfpathmoveto{\pgfqpoint{0.750000in}{1.316524in}}%
\pgfpathlineto{\pgfqpoint{2.863636in}{1.316524in}}%
\pgfusepath{stroke}%
\end{pgfscope}%
\begin{pgfscope}%
\definecolor{textcolor}{rgb}{0.150000,0.150000,0.150000}%
\pgfsetstrokecolor{textcolor}%
\pgfsetfillcolor{textcolor}%
\pgftext[x=0.475308in, y=1.268299in, left, base]{\color{textcolor}\rmfamily\fontsize{10.000000}{12.000000}\selectfont \(\displaystyle {\ensuremath{-}5}\)}%
\end{pgfscope}%
\begin{pgfscope}%
\pgfpathrectangle{\pgfqpoint{0.750000in}{0.660000in}}{\pgfqpoint{2.113636in}{2.100000in}}%
\pgfusepath{clip}%
\pgfsetroundcap%
\pgfsetroundjoin%
\pgfsetlinewidth{1.003750pt}%
\definecolor{currentstroke}{rgb}{1.000000,1.000000,1.000000}%
\pgfsetstrokecolor{currentstroke}%
\pgfsetdash{}{0pt}%
\pgfpathmoveto{\pgfqpoint{0.750000in}{1.710000in}}%
\pgfpathlineto{\pgfqpoint{2.863636in}{1.710000in}}%
\pgfusepath{stroke}%
\end{pgfscope}%
\begin{pgfscope}%
\definecolor{textcolor}{rgb}{0.150000,0.150000,0.150000}%
\pgfsetstrokecolor{textcolor}%
\pgfsetfillcolor{textcolor}%
\pgftext[x=0.583333in, y=1.661775in, left, base]{\color{textcolor}\rmfamily\fontsize{10.000000}{12.000000}\selectfont \(\displaystyle {0}\)}%
\end{pgfscope}%
\begin{pgfscope}%
\pgfpathrectangle{\pgfqpoint{0.750000in}{0.660000in}}{\pgfqpoint{2.113636in}{2.100000in}}%
\pgfusepath{clip}%
\pgfsetroundcap%
\pgfsetroundjoin%
\pgfsetlinewidth{1.003750pt}%
\definecolor{currentstroke}{rgb}{1.000000,1.000000,1.000000}%
\pgfsetstrokecolor{currentstroke}%
\pgfsetdash{}{0pt}%
\pgfpathmoveto{\pgfqpoint{0.750000in}{2.103476in}}%
\pgfpathlineto{\pgfqpoint{2.863636in}{2.103476in}}%
\pgfusepath{stroke}%
\end{pgfscope}%
\begin{pgfscope}%
\definecolor{textcolor}{rgb}{0.150000,0.150000,0.150000}%
\pgfsetstrokecolor{textcolor}%
\pgfsetfillcolor{textcolor}%
\pgftext[x=0.583333in, y=2.055251in, left, base]{\color{textcolor}\rmfamily\fontsize{10.000000}{12.000000}\selectfont \(\displaystyle {5}\)}%
\end{pgfscope}%
\begin{pgfscope}%
\pgfpathrectangle{\pgfqpoint{0.750000in}{0.660000in}}{\pgfqpoint{2.113636in}{2.100000in}}%
\pgfusepath{clip}%
\pgfsetroundcap%
\pgfsetroundjoin%
\pgfsetlinewidth{1.003750pt}%
\definecolor{currentstroke}{rgb}{1.000000,1.000000,1.000000}%
\pgfsetstrokecolor{currentstroke}%
\pgfsetdash{}{0pt}%
\pgfpathmoveto{\pgfqpoint{0.750000in}{2.496952in}}%
\pgfpathlineto{\pgfqpoint{2.863636in}{2.496952in}}%
\pgfusepath{stroke}%
\end{pgfscope}%
\begin{pgfscope}%
\definecolor{textcolor}{rgb}{0.150000,0.150000,0.150000}%
\pgfsetstrokecolor{textcolor}%
\pgfsetfillcolor{textcolor}%
\pgftext[x=0.513888in, y=2.448727in, left, base]{\color{textcolor}\rmfamily\fontsize{10.000000}{12.000000}\selectfont \(\displaystyle {10}\)}%
\end{pgfscope}%
\begin{pgfscope}%
\definecolor{textcolor}{rgb}{0.150000,0.150000,0.150000}%
\pgfsetstrokecolor{textcolor}%
\pgfsetfillcolor{textcolor}%
\pgftext[x=0.350308in,y=1.710000in,,bottom,rotate=90.000000]{\color{textcolor}\rmfamily\fontsize{11.000000}{13.200000}\selectfont Position}%
\end{pgfscope}%
\begin{pgfscope}%
\pgfpathrectangle{\pgfqpoint{0.750000in}{0.660000in}}{\pgfqpoint{2.113636in}{2.100000in}}%
\pgfusepath{clip}%
\pgfsetroundcap%
\pgfsetroundjoin%
\pgfsetlinewidth{1.756562pt}%
\definecolor{currentstroke}{rgb}{0.215686,0.494118,0.721569}%
\pgfsetstrokecolor{currentstroke}%
\pgfsetdash{}{0pt}%
\pgfpathmoveto{\pgfqpoint{0.846074in}{1.513262in}}%
\pgfpathlineto{\pgfqpoint{0.867232in}{1.489357in}}%
\pgfpathlineto{\pgfqpoint{0.892236in}{1.464427in}}%
\pgfpathlineto{\pgfqpoint{0.923011in}{1.437077in}}%
\pgfpathlineto{\pgfqpoint{0.957632in}{1.409409in}}%
\pgfpathlineto{\pgfqpoint{0.998024in}{1.380146in}}%
\pgfpathlineto{\pgfqpoint{1.042262in}{1.350885in}}%
\pgfpathlineto{\pgfqpoint{1.092271in}{1.320490in}}%
\pgfpathlineto{\pgfqpoint{1.148050in}{1.289202in}}%
\pgfpathlineto{\pgfqpoint{1.211522in}{1.256241in}}%
\pgfpathlineto{\pgfqpoint{1.280765in}{1.222838in}}%
\pgfpathlineto{\pgfqpoint{1.357702in}{1.188241in}}%
\pgfpathlineto{\pgfqpoint{1.442332in}{1.152670in}}%
\pgfpathlineto{\pgfqpoint{1.534656in}{1.116300in}}%
\pgfpathlineto{\pgfqpoint{1.636596in}{1.078578in}}%
\pgfpathlineto{\pgfqpoint{1.748154in}{1.039732in}}%
\pgfpathlineto{\pgfqpoint{1.869329in}{0.999945in}}%
\pgfpathlineto{\pgfqpoint{2.000121in}{0.959368in}}%
\pgfpathlineto{\pgfqpoint{2.142453in}{0.917571in}}%
\pgfpathlineto{\pgfqpoint{2.296326in}{0.874738in}}%
\pgfpathlineto{\pgfqpoint{2.461740in}{0.831020in}}%
\pgfpathlineto{\pgfqpoint{2.640617in}{0.786070in}}%
\pgfpathlineto{\pgfqpoint{2.767562in}{0.755455in}}%
\pgfpathlineto{\pgfqpoint{2.767562in}{0.755455in}}%
\pgfusepath{stroke}%
\end{pgfscope}%
\begin{pgfscope}%
\pgfpathrectangle{\pgfqpoint{0.750000in}{0.660000in}}{\pgfqpoint{2.113636in}{2.100000in}}%
\pgfusepath{clip}%
\pgfsetroundcap%
\pgfsetroundjoin%
\pgfsetlinewidth{1.756562pt}%
\definecolor{currentstroke}{rgb}{1.000000,0.498039,0.000000}%
\pgfsetstrokecolor{currentstroke}%
\pgfsetdash{}{0pt}%
\pgfpathmoveto{\pgfqpoint{0.846074in}{1.578841in}}%
\pgfpathlineto{\pgfqpoint{0.886466in}{1.554688in}}%
\pgfpathlineto{\pgfqpoint{0.926858in}{1.533242in}}%
\pgfpathlineto{\pgfqpoint{0.973020in}{1.511336in}}%
\pgfpathlineto{\pgfqpoint{1.024952in}{1.489168in}}%
\pgfpathlineto{\pgfqpoint{1.084577in}{1.466127in}}%
\pgfpathlineto{\pgfqpoint{1.153820in}{1.441804in}}%
\pgfpathlineto{\pgfqpoint{1.232680in}{1.416523in}}%
\pgfpathlineto{\pgfqpoint{1.321157in}{1.390515in}}%
\pgfpathlineto{\pgfqpoint{1.421174in}{1.363448in}}%
\pgfpathlineto{\pgfqpoint{1.534656in}{1.335088in}}%
\pgfpathlineto{\pgfqpoint{1.661601in}{1.305710in}}%
\pgfpathlineto{\pgfqpoint{1.802010in}{1.275520in}}%
\pgfpathlineto{\pgfqpoint{1.959729in}{1.243932in}}%
\pgfpathlineto{\pgfqpoint{2.132836in}{1.211560in}}%
\pgfpathlineto{\pgfqpoint{2.325177in}{1.177895in}}%
\pgfpathlineto{\pgfqpoint{2.536753in}{1.143168in}}%
\pgfpathlineto{\pgfqpoint{2.767562in}{1.107562in}}%
\pgfpathlineto{\pgfqpoint{2.767562in}{1.107562in}}%
\pgfusepath{stroke}%
\end{pgfscope}%
\begin{pgfscope}%
\pgfpathrectangle{\pgfqpoint{0.750000in}{0.660000in}}{\pgfqpoint{2.113636in}{2.100000in}}%
\pgfusepath{clip}%
\pgfsetroundcap%
\pgfsetroundjoin%
\pgfsetlinewidth{1.756562pt}%
\definecolor{currentstroke}{rgb}{0.301961,0.686275,0.290196}%
\pgfsetstrokecolor{currentstroke}%
\pgfsetdash{}{0pt}%
\pgfpathmoveto{\pgfqpoint{0.846074in}{1.644421in}}%
\pgfpathlineto{\pgfqpoint{0.909547in}{1.627355in}}%
\pgfpathlineto{\pgfqpoint{0.974943in}{1.612232in}}%
\pgfpathlineto{\pgfqpoint{1.051879in}{1.596795in}}%
\pgfpathlineto{\pgfqpoint{1.144203in}{1.580584in}}%
\pgfpathlineto{\pgfqpoint{1.255761in}{1.563314in}}%
\pgfpathlineto{\pgfqpoint{1.388476in}{1.545079in}}%
\pgfpathlineto{\pgfqpoint{1.546196in}{1.525723in}}%
\pgfpathlineto{\pgfqpoint{1.730843in}{1.505365in}}%
\pgfpathlineto{\pgfqpoint{1.946265in}{1.483913in}}%
\pgfpathlineto{\pgfqpoint{2.194386in}{1.461485in}}%
\pgfpathlineto{\pgfqpoint{2.480974in}{1.437868in}}%
\pgfpathlineto{\pgfqpoint{2.767562in}{1.416141in}}%
\pgfpathlineto{\pgfqpoint{2.767562in}{1.416141in}}%
\pgfusepath{stroke}%
\end{pgfscope}%
\begin{pgfscope}%
\pgfpathrectangle{\pgfqpoint{0.750000in}{0.660000in}}{\pgfqpoint{2.113636in}{2.100000in}}%
\pgfusepath{clip}%
\pgfsetroundcap%
\pgfsetroundjoin%
\pgfsetlinewidth{1.756562pt}%
\definecolor{currentstroke}{rgb}{0.968627,0.505882,0.749020}%
\pgfsetstrokecolor{currentstroke}%
\pgfsetdash{}{0pt}%
\pgfpathmoveto{\pgfqpoint{0.846074in}{1.710000in}}%
\pgfpathlineto{\pgfqpoint{2.767562in}{1.710000in}}%
\pgfpathlineto{\pgfqpoint{2.767562in}{1.710000in}}%
\pgfusepath{stroke}%
\end{pgfscope}%
\begin{pgfscope}%
\pgfpathrectangle{\pgfqpoint{0.750000in}{0.660000in}}{\pgfqpoint{2.113636in}{2.100000in}}%
\pgfusepath{clip}%
\pgfsetroundcap%
\pgfsetroundjoin%
\pgfsetlinewidth{1.756562pt}%
\definecolor{currentstroke}{rgb}{0.650980,0.337255,0.156863}%
\pgfsetstrokecolor{currentstroke}%
\pgfsetdash{}{0pt}%
\pgfpathmoveto{\pgfqpoint{0.846074in}{1.775579in}}%
\pgfpathlineto{\pgfqpoint{0.909547in}{1.792645in}}%
\pgfpathlineto{\pgfqpoint{0.974943in}{1.807768in}}%
\pgfpathlineto{\pgfqpoint{1.051879in}{1.823205in}}%
\pgfpathlineto{\pgfqpoint{1.144203in}{1.839416in}}%
\pgfpathlineto{\pgfqpoint{1.255761in}{1.856686in}}%
\pgfpathlineto{\pgfqpoint{1.388476in}{1.874921in}}%
\pgfpathlineto{\pgfqpoint{1.546196in}{1.894277in}}%
\pgfpathlineto{\pgfqpoint{1.730843in}{1.914635in}}%
\pgfpathlineto{\pgfqpoint{1.946265in}{1.936087in}}%
\pgfpathlineto{\pgfqpoint{2.194386in}{1.958515in}}%
\pgfpathlineto{\pgfqpoint{2.480974in}{1.982132in}}%
\pgfpathlineto{\pgfqpoint{2.767562in}{2.003859in}}%
\pgfpathlineto{\pgfqpoint{2.767562in}{2.003859in}}%
\pgfusepath{stroke}%
\end{pgfscope}%
\begin{pgfscope}%
\pgfpathrectangle{\pgfqpoint{0.750000in}{0.660000in}}{\pgfqpoint{2.113636in}{2.100000in}}%
\pgfusepath{clip}%
\pgfsetroundcap%
\pgfsetroundjoin%
\pgfsetlinewidth{1.756562pt}%
\definecolor{currentstroke}{rgb}{0.596078,0.305882,0.639216}%
\pgfsetstrokecolor{currentstroke}%
\pgfsetdash{}{0pt}%
\pgfpathmoveto{\pgfqpoint{0.846074in}{1.841159in}}%
\pgfpathlineto{\pgfqpoint{0.886466in}{1.865312in}}%
\pgfpathlineto{\pgfqpoint{0.926858in}{1.886758in}}%
\pgfpathlineto{\pgfqpoint{0.973020in}{1.908664in}}%
\pgfpathlineto{\pgfqpoint{1.024952in}{1.930832in}}%
\pgfpathlineto{\pgfqpoint{1.084577in}{1.953873in}}%
\pgfpathlineto{\pgfqpoint{1.153820in}{1.978196in}}%
\pgfpathlineto{\pgfqpoint{1.232680in}{2.003477in}}%
\pgfpathlineto{\pgfqpoint{1.321157in}{2.029485in}}%
\pgfpathlineto{\pgfqpoint{1.421174in}{2.056552in}}%
\pgfpathlineto{\pgfqpoint{1.534656in}{2.084912in}}%
\pgfpathlineto{\pgfqpoint{1.661601in}{2.114290in}}%
\pgfpathlineto{\pgfqpoint{1.802010in}{2.144480in}}%
\pgfpathlineto{\pgfqpoint{1.959729in}{2.176068in}}%
\pgfpathlineto{\pgfqpoint{2.132836in}{2.208440in}}%
\pgfpathlineto{\pgfqpoint{2.325177in}{2.242105in}}%
\pgfpathlineto{\pgfqpoint{2.536753in}{2.276832in}}%
\pgfpathlineto{\pgfqpoint{2.767562in}{2.312438in}}%
\pgfpathlineto{\pgfqpoint{2.767562in}{2.312438in}}%
\pgfusepath{stroke}%
\end{pgfscope}%
\begin{pgfscope}%
\pgfpathrectangle{\pgfqpoint{0.750000in}{0.660000in}}{\pgfqpoint{2.113636in}{2.100000in}}%
\pgfusepath{clip}%
\pgfsetroundcap%
\pgfsetroundjoin%
\pgfsetlinewidth{1.756562pt}%
\definecolor{currentstroke}{rgb}{0.600000,0.600000,0.600000}%
\pgfsetstrokecolor{currentstroke}%
\pgfsetdash{}{0pt}%
\pgfpathmoveto{\pgfqpoint{0.846074in}{1.906738in}}%
\pgfpathlineto{\pgfqpoint{0.867232in}{1.930643in}}%
\pgfpathlineto{\pgfqpoint{0.892236in}{1.955573in}}%
\pgfpathlineto{\pgfqpoint{0.923011in}{1.982923in}}%
\pgfpathlineto{\pgfqpoint{0.957632in}{2.010591in}}%
\pgfpathlineto{\pgfqpoint{0.998024in}{2.039854in}}%
\pgfpathlineto{\pgfqpoint{1.042262in}{2.069115in}}%
\pgfpathlineto{\pgfqpoint{1.092271in}{2.099510in}}%
\pgfpathlineto{\pgfqpoint{1.148050in}{2.130798in}}%
\pgfpathlineto{\pgfqpoint{1.211522in}{2.163759in}}%
\pgfpathlineto{\pgfqpoint{1.280765in}{2.197162in}}%
\pgfpathlineto{\pgfqpoint{1.357702in}{2.231759in}}%
\pgfpathlineto{\pgfqpoint{1.442332in}{2.267330in}}%
\pgfpathlineto{\pgfqpoint{1.534656in}{2.303700in}}%
\pgfpathlineto{\pgfqpoint{1.636596in}{2.341422in}}%
\pgfpathlineto{\pgfqpoint{1.748154in}{2.380268in}}%
\pgfpathlineto{\pgfqpoint{1.869329in}{2.420055in}}%
\pgfpathlineto{\pgfqpoint{2.000121in}{2.460632in}}%
\pgfpathlineto{\pgfqpoint{2.142453in}{2.502429in}}%
\pgfpathlineto{\pgfqpoint{2.296326in}{2.545262in}}%
\pgfpathlineto{\pgfqpoint{2.461740in}{2.588980in}}%
\pgfpathlineto{\pgfqpoint{2.640617in}{2.633930in}}%
\pgfpathlineto{\pgfqpoint{2.767562in}{2.664545in}}%
\pgfpathlineto{\pgfqpoint{2.767562in}{2.664545in}}%
\pgfusepath{stroke}%
\end{pgfscope}%
\begin{pgfscope}%
\pgfsetrectcap%
\pgfsetmiterjoin%
\pgfsetlinewidth{0.000000pt}%
\definecolor{currentstroke}{rgb}{1.000000,1.000000,1.000000}%
\pgfsetstrokecolor{currentstroke}%
\pgfsetdash{}{0pt}%
\pgfpathmoveto{\pgfqpoint{0.750000in}{0.660000in}}%
\pgfpathlineto{\pgfqpoint{0.750000in}{2.760000in}}%
\pgfusepath{}%
\end{pgfscope}%
\begin{pgfscope}%
\pgfsetrectcap%
\pgfsetmiterjoin%
\pgfsetlinewidth{0.000000pt}%
\definecolor{currentstroke}{rgb}{1.000000,1.000000,1.000000}%
\pgfsetstrokecolor{currentstroke}%
\pgfsetdash{}{0pt}%
\pgfpathmoveto{\pgfqpoint{2.863636in}{0.660000in}}%
\pgfpathlineto{\pgfqpoint{2.863636in}{2.760000in}}%
\pgfusepath{}%
\end{pgfscope}%
\begin{pgfscope}%
\pgfsetrectcap%
\pgfsetmiterjoin%
\pgfsetlinewidth{0.000000pt}%
\definecolor{currentstroke}{rgb}{1.000000,1.000000,1.000000}%
\pgfsetstrokecolor{currentstroke}%
\pgfsetdash{}{0pt}%
\pgfpathmoveto{\pgfqpoint{0.750000in}{0.660000in}}%
\pgfpathlineto{\pgfqpoint{2.863636in}{0.660000in}}%
\pgfusepath{}%
\end{pgfscope}%
\begin{pgfscope}%
\pgfsetrectcap%
\pgfsetmiterjoin%
\pgfsetlinewidth{0.000000pt}%
\definecolor{currentstroke}{rgb}{1.000000,1.000000,1.000000}%
\pgfsetstrokecolor{currentstroke}%
\pgfsetdash{}{0pt}%
\pgfpathmoveto{\pgfqpoint{0.750000in}{2.760000in}}%
\pgfpathlineto{\pgfqpoint{2.863636in}{2.760000in}}%
\pgfusepath{}%
\end{pgfscope}%
\begin{pgfscope}%
\pgfsetbuttcap%
\pgfsetmiterjoin%
\definecolor{currentfill}{rgb}{0.917647,0.917647,0.949020}%
\pgfsetfillcolor{currentfill}%
\pgfsetlinewidth{0.000000pt}%
\definecolor{currentstroke}{rgb}{0.000000,0.000000,0.000000}%
\pgfsetstrokecolor{currentstroke}%
\pgfsetstrokeopacity{0.000000}%
\pgfsetdash{}{0pt}%
\pgfpathmoveto{\pgfqpoint{3.286364in}{0.660000in}}%
\pgfpathlineto{\pgfqpoint{5.400000in}{0.660000in}}%
\pgfpathlineto{\pgfqpoint{5.400000in}{2.760000in}}%
\pgfpathlineto{\pgfqpoint{3.286364in}{2.760000in}}%
\pgfpathlineto{\pgfqpoint{3.286364in}{0.660000in}}%
\pgfpathclose%
\pgfusepath{fill}%
\end{pgfscope}%
\begin{pgfscope}%
\pgfpathrectangle{\pgfqpoint{3.286364in}{0.660000in}}{\pgfqpoint{2.113636in}{2.100000in}}%
\pgfusepath{clip}%
\pgfsetroundcap%
\pgfsetroundjoin%
\pgfsetlinewidth{1.003750pt}%
\definecolor{currentstroke}{rgb}{1.000000,1.000000,1.000000}%
\pgfsetstrokecolor{currentstroke}%
\pgfsetdash{}{0pt}%
\pgfpathmoveto{\pgfqpoint{3.382438in}{0.660000in}}%
\pgfpathlineto{\pgfqpoint{3.382438in}{2.760000in}}%
\pgfusepath{stroke}%
\end{pgfscope}%
\begin{pgfscope}%
\definecolor{textcolor}{rgb}{0.150000,0.150000,0.150000}%
\pgfsetstrokecolor{textcolor}%
\pgfsetfillcolor{textcolor}%
\pgftext[x=3.382438in,y=0.562778in,,top]{\color{textcolor}\rmfamily\fontsize{10.000000}{12.000000}\selectfont \(\displaystyle {0.0}\)}%
\end{pgfscope}%
\begin{pgfscope}%
\pgfpathrectangle{\pgfqpoint{3.286364in}{0.660000in}}{\pgfqpoint{2.113636in}{2.100000in}}%
\pgfusepath{clip}%
\pgfsetroundcap%
\pgfsetroundjoin%
\pgfsetlinewidth{1.003750pt}%
\definecolor{currentstroke}{rgb}{1.000000,1.000000,1.000000}%
\pgfsetstrokecolor{currentstroke}%
\pgfsetdash{}{0pt}%
\pgfpathmoveto{\pgfqpoint{3.862810in}{0.660000in}}%
\pgfpathlineto{\pgfqpoint{3.862810in}{2.760000in}}%
\pgfusepath{stroke}%
\end{pgfscope}%
\begin{pgfscope}%
\definecolor{textcolor}{rgb}{0.150000,0.150000,0.150000}%
\pgfsetstrokecolor{textcolor}%
\pgfsetfillcolor{textcolor}%
\pgftext[x=3.862810in,y=0.562778in,,top]{\color{textcolor}\rmfamily\fontsize{10.000000}{12.000000}\selectfont \(\displaystyle {2.5}\)}%
\end{pgfscope}%
\begin{pgfscope}%
\pgfpathrectangle{\pgfqpoint{3.286364in}{0.660000in}}{\pgfqpoint{2.113636in}{2.100000in}}%
\pgfusepath{clip}%
\pgfsetroundcap%
\pgfsetroundjoin%
\pgfsetlinewidth{1.003750pt}%
\definecolor{currentstroke}{rgb}{1.000000,1.000000,1.000000}%
\pgfsetstrokecolor{currentstroke}%
\pgfsetdash{}{0pt}%
\pgfpathmoveto{\pgfqpoint{4.343182in}{0.660000in}}%
\pgfpathlineto{\pgfqpoint{4.343182in}{2.760000in}}%
\pgfusepath{stroke}%
\end{pgfscope}%
\begin{pgfscope}%
\definecolor{textcolor}{rgb}{0.150000,0.150000,0.150000}%
\pgfsetstrokecolor{textcolor}%
\pgfsetfillcolor{textcolor}%
\pgftext[x=4.343182in,y=0.562778in,,top]{\color{textcolor}\rmfamily\fontsize{10.000000}{12.000000}\selectfont \(\displaystyle {5.0}\)}%
\end{pgfscope}%
\begin{pgfscope}%
\pgfpathrectangle{\pgfqpoint{3.286364in}{0.660000in}}{\pgfqpoint{2.113636in}{2.100000in}}%
\pgfusepath{clip}%
\pgfsetroundcap%
\pgfsetroundjoin%
\pgfsetlinewidth{1.003750pt}%
\definecolor{currentstroke}{rgb}{1.000000,1.000000,1.000000}%
\pgfsetstrokecolor{currentstroke}%
\pgfsetdash{}{0pt}%
\pgfpathmoveto{\pgfqpoint{4.823554in}{0.660000in}}%
\pgfpathlineto{\pgfqpoint{4.823554in}{2.760000in}}%
\pgfusepath{stroke}%
\end{pgfscope}%
\begin{pgfscope}%
\definecolor{textcolor}{rgb}{0.150000,0.150000,0.150000}%
\pgfsetstrokecolor{textcolor}%
\pgfsetfillcolor{textcolor}%
\pgftext[x=4.823554in,y=0.562778in,,top]{\color{textcolor}\rmfamily\fontsize{10.000000}{12.000000}\selectfont \(\displaystyle {7.5}\)}%
\end{pgfscope}%
\begin{pgfscope}%
\pgfpathrectangle{\pgfqpoint{3.286364in}{0.660000in}}{\pgfqpoint{2.113636in}{2.100000in}}%
\pgfusepath{clip}%
\pgfsetroundcap%
\pgfsetroundjoin%
\pgfsetlinewidth{1.003750pt}%
\definecolor{currentstroke}{rgb}{1.000000,1.000000,1.000000}%
\pgfsetstrokecolor{currentstroke}%
\pgfsetdash{}{0pt}%
\pgfpathmoveto{\pgfqpoint{5.303926in}{0.660000in}}%
\pgfpathlineto{\pgfqpoint{5.303926in}{2.760000in}}%
\pgfusepath{stroke}%
\end{pgfscope}%
\begin{pgfscope}%
\definecolor{textcolor}{rgb}{0.150000,0.150000,0.150000}%
\pgfsetstrokecolor{textcolor}%
\pgfsetfillcolor{textcolor}%
\pgftext[x=5.303926in,y=0.562778in,,top]{\color{textcolor}\rmfamily\fontsize{10.000000}{12.000000}\selectfont \(\displaystyle {10.0}\)}%
\end{pgfscope}%
\begin{pgfscope}%
\definecolor{textcolor}{rgb}{0.150000,0.150000,0.150000}%
\pgfsetstrokecolor{textcolor}%
\pgfsetfillcolor{textcolor}%
\pgftext[x=4.343182in,y=0.383766in,,top]{\color{textcolor}\rmfamily\fontsize{11.000000}{13.200000}\selectfont time (\(\displaystyle t\))}%
\end{pgfscope}%
\begin{pgfscope}%
\pgfpathrectangle{\pgfqpoint{3.286364in}{0.660000in}}{\pgfqpoint{2.113636in}{2.100000in}}%
\pgfusepath{clip}%
\pgfsetroundcap%
\pgfsetroundjoin%
\pgfsetlinewidth{1.003750pt}%
\definecolor{currentstroke}{rgb}{1.000000,1.000000,1.000000}%
\pgfsetstrokecolor{currentstroke}%
\pgfsetdash{}{0pt}%
\pgfpathmoveto{\pgfqpoint{3.286364in}{0.722572in}}%
\pgfpathlineto{\pgfqpoint{5.400000in}{0.722572in}}%
\pgfusepath{stroke}%
\end{pgfscope}%
\begin{pgfscope}%
\definecolor{textcolor}{rgb}{0.150000,0.150000,0.150000}%
\pgfsetstrokecolor{textcolor}%
\pgfsetfillcolor{textcolor}%
\pgftext[x=2.942227in, y=0.674347in, left, base]{\color{textcolor}\rmfamily\fontsize{10.000000}{12.000000}\selectfont \(\displaystyle {\ensuremath{-}15}\)}%
\end{pgfscope}%
\begin{pgfscope}%
\pgfpathrectangle{\pgfqpoint{3.286364in}{0.660000in}}{\pgfqpoint{2.113636in}{2.100000in}}%
\pgfusepath{clip}%
\pgfsetroundcap%
\pgfsetroundjoin%
\pgfsetlinewidth{1.003750pt}%
\definecolor{currentstroke}{rgb}{1.000000,1.000000,1.000000}%
\pgfsetstrokecolor{currentstroke}%
\pgfsetdash{}{0pt}%
\pgfpathmoveto{\pgfqpoint{3.286364in}{1.051715in}}%
\pgfpathlineto{\pgfqpoint{5.400000in}{1.051715in}}%
\pgfusepath{stroke}%
\end{pgfscope}%
\begin{pgfscope}%
\definecolor{textcolor}{rgb}{0.150000,0.150000,0.150000}%
\pgfsetstrokecolor{textcolor}%
\pgfsetfillcolor{textcolor}%
\pgftext[x=2.942227in, y=1.003490in, left, base]{\color{textcolor}\rmfamily\fontsize{10.000000}{12.000000}\selectfont \(\displaystyle {\ensuremath{-}10}\)}%
\end{pgfscope}%
\begin{pgfscope}%
\pgfpathrectangle{\pgfqpoint{3.286364in}{0.660000in}}{\pgfqpoint{2.113636in}{2.100000in}}%
\pgfusepath{clip}%
\pgfsetroundcap%
\pgfsetroundjoin%
\pgfsetlinewidth{1.003750pt}%
\definecolor{currentstroke}{rgb}{1.000000,1.000000,1.000000}%
\pgfsetstrokecolor{currentstroke}%
\pgfsetdash{}{0pt}%
\pgfpathmoveto{\pgfqpoint{3.286364in}{1.380857in}}%
\pgfpathlineto{\pgfqpoint{5.400000in}{1.380857in}}%
\pgfusepath{stroke}%
\end{pgfscope}%
\begin{pgfscope}%
\definecolor{textcolor}{rgb}{0.150000,0.150000,0.150000}%
\pgfsetstrokecolor{textcolor}%
\pgfsetfillcolor{textcolor}%
\pgftext[x=3.011672in, y=1.332632in, left, base]{\color{textcolor}\rmfamily\fontsize{10.000000}{12.000000}\selectfont \(\displaystyle {\ensuremath{-}5}\)}%
\end{pgfscope}%
\begin{pgfscope}%
\pgfpathrectangle{\pgfqpoint{3.286364in}{0.660000in}}{\pgfqpoint{2.113636in}{2.100000in}}%
\pgfusepath{clip}%
\pgfsetroundcap%
\pgfsetroundjoin%
\pgfsetlinewidth{1.003750pt}%
\definecolor{currentstroke}{rgb}{1.000000,1.000000,1.000000}%
\pgfsetstrokecolor{currentstroke}%
\pgfsetdash{}{0pt}%
\pgfpathmoveto{\pgfqpoint{3.286364in}{1.710000in}}%
\pgfpathlineto{\pgfqpoint{5.400000in}{1.710000in}}%
\pgfusepath{stroke}%
\end{pgfscope}%
\begin{pgfscope}%
\definecolor{textcolor}{rgb}{0.150000,0.150000,0.150000}%
\pgfsetstrokecolor{textcolor}%
\pgfsetfillcolor{textcolor}%
\pgftext[x=3.119697in, y=1.661775in, left, base]{\color{textcolor}\rmfamily\fontsize{10.000000}{12.000000}\selectfont \(\displaystyle {0}\)}%
\end{pgfscope}%
\begin{pgfscope}%
\pgfpathrectangle{\pgfqpoint{3.286364in}{0.660000in}}{\pgfqpoint{2.113636in}{2.100000in}}%
\pgfusepath{clip}%
\pgfsetroundcap%
\pgfsetroundjoin%
\pgfsetlinewidth{1.003750pt}%
\definecolor{currentstroke}{rgb}{1.000000,1.000000,1.000000}%
\pgfsetstrokecolor{currentstroke}%
\pgfsetdash{}{0pt}%
\pgfpathmoveto{\pgfqpoint{3.286364in}{2.039143in}}%
\pgfpathlineto{\pgfqpoint{5.400000in}{2.039143in}}%
\pgfusepath{stroke}%
\end{pgfscope}%
\begin{pgfscope}%
\definecolor{textcolor}{rgb}{0.150000,0.150000,0.150000}%
\pgfsetstrokecolor{textcolor}%
\pgfsetfillcolor{textcolor}%
\pgftext[x=3.119697in, y=1.990917in, left, base]{\color{textcolor}\rmfamily\fontsize{10.000000}{12.000000}\selectfont \(\displaystyle {5}\)}%
\end{pgfscope}%
\begin{pgfscope}%
\pgfpathrectangle{\pgfqpoint{3.286364in}{0.660000in}}{\pgfqpoint{2.113636in}{2.100000in}}%
\pgfusepath{clip}%
\pgfsetroundcap%
\pgfsetroundjoin%
\pgfsetlinewidth{1.003750pt}%
\definecolor{currentstroke}{rgb}{1.000000,1.000000,1.000000}%
\pgfsetstrokecolor{currentstroke}%
\pgfsetdash{}{0pt}%
\pgfpathmoveto{\pgfqpoint{3.286364in}{2.368285in}}%
\pgfpathlineto{\pgfqpoint{5.400000in}{2.368285in}}%
\pgfusepath{stroke}%
\end{pgfscope}%
\begin{pgfscope}%
\definecolor{textcolor}{rgb}{0.150000,0.150000,0.150000}%
\pgfsetstrokecolor{textcolor}%
\pgfsetfillcolor{textcolor}%
\pgftext[x=3.050252in, y=2.320060in, left, base]{\color{textcolor}\rmfamily\fontsize{10.000000}{12.000000}\selectfont \(\displaystyle {10}\)}%
\end{pgfscope}%
\begin{pgfscope}%
\pgfpathrectangle{\pgfqpoint{3.286364in}{0.660000in}}{\pgfqpoint{2.113636in}{2.100000in}}%
\pgfusepath{clip}%
\pgfsetroundcap%
\pgfsetroundjoin%
\pgfsetlinewidth{1.003750pt}%
\definecolor{currentstroke}{rgb}{1.000000,1.000000,1.000000}%
\pgfsetstrokecolor{currentstroke}%
\pgfsetdash{}{0pt}%
\pgfpathmoveto{\pgfqpoint{3.286364in}{2.697428in}}%
\pgfpathlineto{\pgfqpoint{5.400000in}{2.697428in}}%
\pgfusepath{stroke}%
\end{pgfscope}%
\begin{pgfscope}%
\definecolor{textcolor}{rgb}{0.150000,0.150000,0.150000}%
\pgfsetstrokecolor{textcolor}%
\pgfsetfillcolor{textcolor}%
\pgftext[x=3.050252in, y=2.649203in, left, base]{\color{textcolor}\rmfamily\fontsize{10.000000}{12.000000}\selectfont \(\displaystyle {15}\)}%
\end{pgfscope}%
\begin{pgfscope}%
\definecolor{textcolor}{rgb}{0.150000,0.150000,0.150000}%
\pgfsetstrokecolor{textcolor}%
\pgfsetfillcolor{textcolor}%
\pgftext[x=2.886672in,y=1.710000in,,bottom,rotate=90.000000]{\color{textcolor}\rmfamily\fontsize{11.000000}{13.200000}\selectfont Position}%
\end{pgfscope}%
\begin{pgfscope}%
\pgfpathrectangle{\pgfqpoint{3.286364in}{0.660000in}}{\pgfqpoint{2.113636in}{2.100000in}}%
\pgfusepath{clip}%
\pgfsetroundcap%
\pgfsetroundjoin%
\pgfsetlinewidth{1.756562pt}%
\definecolor{currentstroke}{rgb}{0.215686,0.494118,0.721569}%
\pgfsetstrokecolor{currentstroke}%
\pgfsetdash{}{0pt}%
\pgfpathmoveto{\pgfqpoint{3.382438in}{1.545429in}}%
\pgfpathlineto{\pgfqpoint{3.399157in}{1.522502in}}%
\pgfpathlineto{\pgfqpoint{3.419719in}{1.498154in}}%
\pgfpathlineto{\pgfqpoint{3.444124in}{1.472743in}}%
\pgfpathlineto{\pgfqpoint{3.472757in}{1.446194in}}%
\pgfpathlineto{\pgfqpoint{3.505810in}{1.418640in}}%
\pgfpathlineto{\pgfqpoint{3.543475in}{1.390189in}}%
\pgfpathlineto{\pgfqpoint{3.586136in}{1.360796in}}%
\pgfpathlineto{\pgfqpoint{3.634178in}{1.330442in}}%
\pgfpathlineto{\pgfqpoint{3.687985in}{1.299121in}}%
\pgfpathlineto{\pgfqpoint{3.747942in}{1.266835in}}%
\pgfpathlineto{\pgfqpoint{3.814432in}{1.233594in}}%
\pgfpathlineto{\pgfqpoint{3.888032in}{1.199324in}}%
\pgfpathlineto{\pgfqpoint{3.969127in}{1.164055in}}%
\pgfpathlineto{\pgfqpoint{4.058101in}{1.127818in}}%
\pgfpathlineto{\pgfqpoint{4.155530in}{1.090568in}}%
\pgfpathlineto{\pgfqpoint{4.261991in}{1.052276in}}%
\pgfpathlineto{\pgfqpoint{4.378060in}{1.012922in}}%
\pgfpathlineto{\pgfqpoint{4.504122in}{0.972558in}}%
\pgfpathlineto{\pgfqpoint{4.640946in}{0.931113in}}%
\pgfpathlineto{\pgfqpoint{4.788915in}{0.888644in}}%
\pgfpathlineto{\pgfqpoint{4.948799in}{0.845097in}}%
\pgfpathlineto{\pgfqpoint{5.120982in}{0.800529in}}%
\pgfpathlineto{\pgfqpoint{5.303926in}{0.755455in}}%
\pgfpathlineto{\pgfqpoint{5.303926in}{0.755455in}}%
\pgfusepath{stroke}%
\end{pgfscope}%
\begin{pgfscope}%
\pgfpathrectangle{\pgfqpoint{3.286364in}{0.660000in}}{\pgfqpoint{2.113636in}{2.100000in}}%
\pgfusepath{clip}%
\pgfsetroundcap%
\pgfsetroundjoin%
\pgfsetlinewidth{1.756562pt}%
\definecolor{currentstroke}{rgb}{1.000000,0.498039,0.000000}%
\pgfsetstrokecolor{currentstroke}%
\pgfsetdash{}{0pt}%
\pgfpathmoveto{\pgfqpoint{3.382438in}{1.586572in}}%
\pgfpathlineto{\pgfqpoint{3.411071in}{1.563032in}}%
\pgfpathlineto{\pgfqpoint{3.440857in}{1.541531in}}%
\pgfpathlineto{\pgfqpoint{3.475063in}{1.519652in}}%
\pgfpathlineto{\pgfqpoint{3.514457in}{1.497143in}}%
\pgfpathlineto{\pgfqpoint{3.560001in}{1.473732in}}%
\pgfpathlineto{\pgfqpoint{3.612271in}{1.449414in}}%
\pgfpathlineto{\pgfqpoint{3.672035in}{1.424115in}}%
\pgfpathlineto{\pgfqpoint{3.739870in}{1.397861in}}%
\pgfpathlineto{\pgfqpoint{3.816545in}{1.370615in}}%
\pgfpathlineto{\pgfqpoint{3.902829in}{1.342354in}}%
\pgfpathlineto{\pgfqpoint{3.999489in}{1.313073in}}%
\pgfpathlineto{\pgfqpoint{4.107488in}{1.282721in}}%
\pgfpathlineto{\pgfqpoint{4.227593in}{1.251318in}}%
\pgfpathlineto{\pgfqpoint{4.360573in}{1.218883in}}%
\pgfpathlineto{\pgfqpoint{4.507389in}{1.185396in}}%
\pgfpathlineto{\pgfqpoint{4.669003in}{1.150847in}}%
\pgfpathlineto{\pgfqpoint{4.846374in}{1.115235in}}%
\pgfpathlineto{\pgfqpoint{5.040655in}{1.078527in}}%
\pgfpathlineto{\pgfqpoint{5.252809in}{1.040737in}}%
\pgfpathlineto{\pgfqpoint{5.303926in}{1.031947in}}%
\pgfpathlineto{\pgfqpoint{5.303926in}{1.031947in}}%
\pgfusepath{stroke}%
\end{pgfscope}%
\begin{pgfscope}%
\pgfpathrectangle{\pgfqpoint{3.286364in}{0.660000in}}{\pgfqpoint{2.113636in}{2.100000in}}%
\pgfusepath{clip}%
\pgfsetroundcap%
\pgfsetroundjoin%
\pgfsetlinewidth{1.756562pt}%
\definecolor{currentstroke}{rgb}{0.301961,0.686275,0.290196}%
\pgfsetstrokecolor{currentstroke}%
\pgfsetdash{}{0pt}%
\pgfpathmoveto{\pgfqpoint{3.382438in}{1.627714in}}%
\pgfpathlineto{\pgfqpoint{3.423178in}{1.607750in}}%
\pgfpathlineto{\pgfqpoint{3.463725in}{1.590488in}}%
\pgfpathlineto{\pgfqpoint{3.510998in}{1.572851in}}%
\pgfpathlineto{\pgfqpoint{3.566727in}{1.554494in}}%
\pgfpathlineto{\pgfqpoint{3.632449in}{1.535243in}}%
\pgfpathlineto{\pgfqpoint{3.709700in}{1.514990in}}%
\pgfpathlineto{\pgfqpoint{3.799635in}{1.493765in}}%
\pgfpathlineto{\pgfqpoint{3.903982in}{1.471478in}}%
\pgfpathlineto{\pgfqpoint{4.024087in}{1.448154in}}%
\pgfpathlineto{\pgfqpoint{4.161487in}{1.423785in}}%
\pgfpathlineto{\pgfqpoint{4.317912in}{1.398348in}}%
\pgfpathlineto{\pgfqpoint{4.494898in}{1.371864in}}%
\pgfpathlineto{\pgfqpoint{4.694369in}{1.344304in}}%
\pgfpathlineto{\pgfqpoint{4.918244in}{1.315659in}}%
\pgfpathlineto{\pgfqpoint{5.168447in}{1.285926in}}%
\pgfpathlineto{\pgfqpoint{5.303926in}{1.270666in}}%
\pgfpathlineto{\pgfqpoint{5.303926in}{1.270666in}}%
\pgfusepath{stroke}%
\end{pgfscope}%
\begin{pgfscope}%
\pgfpathrectangle{\pgfqpoint{3.286364in}{0.660000in}}{\pgfqpoint{2.113636in}{2.100000in}}%
\pgfusepath{clip}%
\pgfsetroundcap%
\pgfsetroundjoin%
\pgfsetlinewidth{1.756562pt}%
\definecolor{currentstroke}{rgb}{0.968627,0.505882,0.749020}%
\pgfsetstrokecolor{currentstroke}%
\pgfsetdash{}{0pt}%
\pgfpathmoveto{\pgfqpoint{3.382438in}{1.668857in}}%
\pgfpathlineto{\pgfqpoint{3.442587in}{1.655159in}}%
\pgfpathlineto{\pgfqpoint{3.506194in}{1.643130in}}%
\pgfpathlineto{\pgfqpoint{3.585560in}{1.630475in}}%
\pgfpathlineto{\pgfqpoint{3.685295in}{1.616911in}}%
\pgfpathlineto{\pgfqpoint{3.809435in}{1.602353in}}%
\pgfpathlineto{\pgfqpoint{3.962017in}{1.586775in}}%
\pgfpathlineto{\pgfqpoint{4.147459in}{1.570148in}}%
\pgfpathlineto{\pgfqpoint{4.370374in}{1.552460in}}%
\pgfpathlineto{\pgfqpoint{4.635565in}{1.533711in}}%
\pgfpathlineto{\pgfqpoint{4.948415in}{1.513880in}}%
\pgfpathlineto{\pgfqpoint{5.303926in}{1.493538in}}%
\pgfpathlineto{\pgfqpoint{5.303926in}{1.493538in}}%
\pgfusepath{stroke}%
\end{pgfscope}%
\begin{pgfscope}%
\pgfpathrectangle{\pgfqpoint{3.286364in}{0.660000in}}{\pgfqpoint{2.113636in}{2.100000in}}%
\pgfusepath{clip}%
\pgfsetroundcap%
\pgfsetroundjoin%
\pgfsetlinewidth{1.756562pt}%
\definecolor{currentstroke}{rgb}{0.650980,0.337255,0.156863}%
\pgfsetstrokecolor{currentstroke}%
\pgfsetdash{}{0pt}%
\pgfpathmoveto{\pgfqpoint{3.382438in}{1.710000in}}%
\pgfpathlineto{\pgfqpoint{5.303926in}{1.710000in}}%
\pgfpathlineto{\pgfqpoint{5.303926in}{1.710000in}}%
\pgfusepath{stroke}%
\end{pgfscope}%
\begin{pgfscope}%
\pgfpathrectangle{\pgfqpoint{3.286364in}{0.660000in}}{\pgfqpoint{2.113636in}{2.100000in}}%
\pgfusepath{clip}%
\pgfsetroundcap%
\pgfsetroundjoin%
\pgfsetlinewidth{1.756562pt}%
\definecolor{currentstroke}{rgb}{0.596078,0.305882,0.639216}%
\pgfsetstrokecolor{currentstroke}%
\pgfsetdash{}{0pt}%
\pgfpathmoveto{\pgfqpoint{3.382438in}{1.751143in}}%
\pgfpathlineto{\pgfqpoint{3.442587in}{1.764841in}}%
\pgfpathlineto{\pgfqpoint{3.506194in}{1.776870in}}%
\pgfpathlineto{\pgfqpoint{3.585560in}{1.789525in}}%
\pgfpathlineto{\pgfqpoint{3.685295in}{1.803089in}}%
\pgfpathlineto{\pgfqpoint{3.809435in}{1.817647in}}%
\pgfpathlineto{\pgfqpoint{3.962017in}{1.833225in}}%
\pgfpathlineto{\pgfqpoint{4.147459in}{1.849852in}}%
\pgfpathlineto{\pgfqpoint{4.370374in}{1.867540in}}%
\pgfpathlineto{\pgfqpoint{4.635565in}{1.886289in}}%
\pgfpathlineto{\pgfqpoint{4.948415in}{1.906120in}}%
\pgfpathlineto{\pgfqpoint{5.303926in}{1.926462in}}%
\pgfpathlineto{\pgfqpoint{5.303926in}{1.926462in}}%
\pgfusepath{stroke}%
\end{pgfscope}%
\begin{pgfscope}%
\pgfpathrectangle{\pgfqpoint{3.286364in}{0.660000in}}{\pgfqpoint{2.113636in}{2.100000in}}%
\pgfusepath{clip}%
\pgfsetroundcap%
\pgfsetroundjoin%
\pgfsetlinewidth{1.756562pt}%
\definecolor{currentstroke}{rgb}{0.600000,0.600000,0.600000}%
\pgfsetstrokecolor{currentstroke}%
\pgfsetdash{}{0pt}%
\pgfpathmoveto{\pgfqpoint{3.382438in}{1.792286in}}%
\pgfpathlineto{\pgfqpoint{3.423178in}{1.812250in}}%
\pgfpathlineto{\pgfqpoint{3.463725in}{1.829512in}}%
\pgfpathlineto{\pgfqpoint{3.510998in}{1.847149in}}%
\pgfpathlineto{\pgfqpoint{3.566727in}{1.865506in}}%
\pgfpathlineto{\pgfqpoint{3.632449in}{1.884757in}}%
\pgfpathlineto{\pgfqpoint{3.709700in}{1.905010in}}%
\pgfpathlineto{\pgfqpoint{3.799635in}{1.926235in}}%
\pgfpathlineto{\pgfqpoint{3.903982in}{1.948522in}}%
\pgfpathlineto{\pgfqpoint{4.024087in}{1.971846in}}%
\pgfpathlineto{\pgfqpoint{4.161487in}{1.996215in}}%
\pgfpathlineto{\pgfqpoint{4.317912in}{2.021652in}}%
\pgfpathlineto{\pgfqpoint{4.494898in}{2.048136in}}%
\pgfpathlineto{\pgfqpoint{4.694369in}{2.075696in}}%
\pgfpathlineto{\pgfqpoint{4.918244in}{2.104341in}}%
\pgfpathlineto{\pgfqpoint{5.168447in}{2.134074in}}%
\pgfpathlineto{\pgfqpoint{5.303926in}{2.149334in}}%
\pgfpathlineto{\pgfqpoint{5.303926in}{2.149334in}}%
\pgfusepath{stroke}%
\end{pgfscope}%
\begin{pgfscope}%
\pgfpathrectangle{\pgfqpoint{3.286364in}{0.660000in}}{\pgfqpoint{2.113636in}{2.100000in}}%
\pgfusepath{clip}%
\pgfsetroundcap%
\pgfsetroundjoin%
\pgfsetlinewidth{1.756562pt}%
\definecolor{currentstroke}{rgb}{0.894118,0.101961,0.109804}%
\pgfsetstrokecolor{currentstroke}%
\pgfsetdash{}{0pt}%
\pgfpathmoveto{\pgfqpoint{3.382438in}{1.833428in}}%
\pgfpathlineto{\pgfqpoint{3.411071in}{1.856968in}}%
\pgfpathlineto{\pgfqpoint{3.440857in}{1.878469in}}%
\pgfpathlineto{\pgfqpoint{3.475063in}{1.900348in}}%
\pgfpathlineto{\pgfqpoint{3.514457in}{1.922857in}}%
\pgfpathlineto{\pgfqpoint{3.560001in}{1.946268in}}%
\pgfpathlineto{\pgfqpoint{3.612271in}{1.970586in}}%
\pgfpathlineto{\pgfqpoint{3.672035in}{1.995885in}}%
\pgfpathlineto{\pgfqpoint{3.739870in}{2.022139in}}%
\pgfpathlineto{\pgfqpoint{3.816545in}{2.049385in}}%
\pgfpathlineto{\pgfqpoint{3.902829in}{2.077646in}}%
\pgfpathlineto{\pgfqpoint{3.999489in}{2.106927in}}%
\pgfpathlineto{\pgfqpoint{4.107488in}{2.137279in}}%
\pgfpathlineto{\pgfqpoint{4.227593in}{2.168682in}}%
\pgfpathlineto{\pgfqpoint{4.360573in}{2.201117in}}%
\pgfpathlineto{\pgfqpoint{4.507389in}{2.234604in}}%
\pgfpathlineto{\pgfqpoint{4.669003in}{2.269153in}}%
\pgfpathlineto{\pgfqpoint{4.846374in}{2.304765in}}%
\pgfpathlineto{\pgfqpoint{5.040655in}{2.341473in}}%
\pgfpathlineto{\pgfqpoint{5.252809in}{2.379263in}}%
\pgfpathlineto{\pgfqpoint{5.303926in}{2.388053in}}%
\pgfpathlineto{\pgfqpoint{5.303926in}{2.388053in}}%
\pgfusepath{stroke}%
\end{pgfscope}%
\begin{pgfscope}%
\pgfpathrectangle{\pgfqpoint{3.286364in}{0.660000in}}{\pgfqpoint{2.113636in}{2.100000in}}%
\pgfusepath{clip}%
\pgfsetroundcap%
\pgfsetroundjoin%
\pgfsetlinewidth{1.756562pt}%
\definecolor{currentstroke}{rgb}{0.870588,0.870588,0.000000}%
\pgfsetstrokecolor{currentstroke}%
\pgfsetdash{}{0pt}%
\pgfpathmoveto{\pgfqpoint{3.382438in}{1.874571in}}%
\pgfpathlineto{\pgfqpoint{3.399157in}{1.897498in}}%
\pgfpathlineto{\pgfqpoint{3.419719in}{1.921846in}}%
\pgfpathlineto{\pgfqpoint{3.444124in}{1.947257in}}%
\pgfpathlineto{\pgfqpoint{3.472757in}{1.973806in}}%
\pgfpathlineto{\pgfqpoint{3.505810in}{2.001360in}}%
\pgfpathlineto{\pgfqpoint{3.543475in}{2.029811in}}%
\pgfpathlineto{\pgfqpoint{3.586136in}{2.059204in}}%
\pgfpathlineto{\pgfqpoint{3.634178in}{2.089558in}}%
\pgfpathlineto{\pgfqpoint{3.687985in}{2.120879in}}%
\pgfpathlineto{\pgfqpoint{3.747942in}{2.153165in}}%
\pgfpathlineto{\pgfqpoint{3.814432in}{2.186406in}}%
\pgfpathlineto{\pgfqpoint{3.888032in}{2.220676in}}%
\pgfpathlineto{\pgfqpoint{3.969127in}{2.255945in}}%
\pgfpathlineto{\pgfqpoint{4.058101in}{2.292182in}}%
\pgfpathlineto{\pgfqpoint{4.155530in}{2.329432in}}%
\pgfpathlineto{\pgfqpoint{4.261991in}{2.367724in}}%
\pgfpathlineto{\pgfqpoint{4.378060in}{2.407078in}}%
\pgfpathlineto{\pgfqpoint{4.504122in}{2.447442in}}%
\pgfpathlineto{\pgfqpoint{4.640946in}{2.488887in}}%
\pgfpathlineto{\pgfqpoint{4.788915in}{2.531356in}}%
\pgfpathlineto{\pgfqpoint{4.948799in}{2.574903in}}%
\pgfpathlineto{\pgfqpoint{5.120982in}{2.619471in}}%
\pgfpathlineto{\pgfqpoint{5.303926in}{2.664545in}}%
\pgfpathlineto{\pgfqpoint{5.303926in}{2.664545in}}%
\pgfusepath{stroke}%
\end{pgfscope}%
\begin{pgfscope}%
\pgfsetrectcap%
\pgfsetmiterjoin%
\pgfsetlinewidth{0.000000pt}%
\definecolor{currentstroke}{rgb}{1.000000,1.000000,1.000000}%
\pgfsetstrokecolor{currentstroke}%
\pgfsetdash{}{0pt}%
\pgfpathmoveto{\pgfqpoint{3.286364in}{0.660000in}}%
\pgfpathlineto{\pgfqpoint{3.286364in}{2.760000in}}%
\pgfusepath{}%
\end{pgfscope}%
\begin{pgfscope}%
\pgfsetrectcap%
\pgfsetmiterjoin%
\pgfsetlinewidth{0.000000pt}%
\definecolor{currentstroke}{rgb}{1.000000,1.000000,1.000000}%
\pgfsetstrokecolor{currentstroke}%
\pgfsetdash{}{0pt}%
\pgfpathmoveto{\pgfqpoint{5.400000in}{0.660000in}}%
\pgfpathlineto{\pgfqpoint{5.400000in}{2.760000in}}%
\pgfusepath{}%
\end{pgfscope}%
\begin{pgfscope}%
\pgfsetrectcap%
\pgfsetmiterjoin%
\pgfsetlinewidth{0.000000pt}%
\definecolor{currentstroke}{rgb}{1.000000,1.000000,1.000000}%
\pgfsetstrokecolor{currentstroke}%
\pgfsetdash{}{0pt}%
\pgfpathmoveto{\pgfqpoint{3.286364in}{0.660000in}}%
\pgfpathlineto{\pgfqpoint{5.400000in}{0.660000in}}%
\pgfusepath{}%
\end{pgfscope}%
\begin{pgfscope}%
\pgfsetrectcap%
\pgfsetmiterjoin%
\pgfsetlinewidth{0.000000pt}%
\definecolor{currentstroke}{rgb}{1.000000,1.000000,1.000000}%
\pgfsetstrokecolor{currentstroke}%
\pgfsetdash{}{0pt}%
\pgfpathmoveto{\pgfqpoint{3.286364in}{2.760000in}}%
\pgfpathlineto{\pgfqpoint{5.400000in}{2.760000in}}%
\pgfusepath{}%
\end{pgfscope}%
\end{pgfpicture}%
\makeatother%
\endgroup%

\end{figure}

\begin{figure}[h!]
    %% Creator: Matplotlib, PGF backend
%%
%% To include the figure in your LaTeX document, write
%%   \input{<filename>.pgf}
%%
%% Make sure the required packages are loaded in your preamble
%%   \usepackage{pgf}
%%
%% Also ensure that all the required font packages are loaded; for instance,
%% the lmodern package is sometimes necessary when using math font.
%%   \usepackage{lmodern}
%%
%% Figures using additional raster images can only be included by \input if
%% they are in the same directory as the main LaTeX file. For loading figures
%% from other directories you can use the `import` package
%%   \usepackage{import}
%%
%% and then include the figures with
%%   \import{<path to file>}{<filename>.pgf}
%%
%% Matplotlib used the following preamble
%%   
%%   \makeatletter\@ifpackageloaded{underscore}{}{\usepackage[strings]{underscore}}\makeatother
%%
\begingroup%
\makeatletter%
\begin{pgfpicture}%
\pgfpathrectangle{\pgfpointorigin}{\pgfqpoint{6.000000in}{2.500000in}}%
\pgfusepath{use as bounding box, clip}%
\begin{pgfscope}%
\pgfsetbuttcap%
\pgfsetmiterjoin%
\definecolor{currentfill}{rgb}{1.000000,1.000000,1.000000}%
\pgfsetfillcolor{currentfill}%
\pgfsetlinewidth{0.000000pt}%
\definecolor{currentstroke}{rgb}{1.000000,1.000000,1.000000}%
\pgfsetstrokecolor{currentstroke}%
\pgfsetdash{}{0pt}%
\pgfpathmoveto{\pgfqpoint{0.000000in}{0.000000in}}%
\pgfpathlineto{\pgfqpoint{6.000000in}{0.000000in}}%
\pgfpathlineto{\pgfqpoint{6.000000in}{2.500000in}}%
\pgfpathlineto{\pgfqpoint{0.000000in}{2.500000in}}%
\pgfpathlineto{\pgfqpoint{0.000000in}{0.000000in}}%
\pgfpathclose%
\pgfusepath{fill}%
\end{pgfscope}%
\begin{pgfscope}%
\pgfsetbuttcap%
\pgfsetmiterjoin%
\definecolor{currentfill}{rgb}{0.917647,0.917647,0.949020}%
\pgfsetfillcolor{currentfill}%
\pgfsetlinewidth{0.000000pt}%
\definecolor{currentstroke}{rgb}{0.000000,0.000000,0.000000}%
\pgfsetstrokecolor{currentstroke}%
\pgfsetstrokeopacity{0.000000}%
\pgfsetdash{}{0pt}%
\pgfpathmoveto{\pgfqpoint{0.750000in}{0.275000in}}%
\pgfpathlineto{\pgfqpoint{5.400000in}{0.275000in}}%
\pgfpathlineto{\pgfqpoint{5.400000in}{2.200000in}}%
\pgfpathlineto{\pgfqpoint{0.750000in}{2.200000in}}%
\pgfpathlineto{\pgfqpoint{0.750000in}{0.275000in}}%
\pgfpathclose%
\pgfusepath{fill}%
\end{pgfscope}%
\begin{pgfscope}%
\pgfpathrectangle{\pgfqpoint{0.750000in}{0.275000in}}{\pgfqpoint{4.650000in}{1.925000in}}%
\pgfusepath{clip}%
\pgfsetroundcap%
\pgfsetroundjoin%
\pgfsetlinewidth{1.003750pt}%
\definecolor{currentstroke}{rgb}{1.000000,1.000000,1.000000}%
\pgfsetstrokecolor{currentstroke}%
\pgfsetdash{}{0pt}%
\pgfpathmoveto{\pgfqpoint{0.961364in}{0.275000in}}%
\pgfpathlineto{\pgfqpoint{0.961364in}{2.200000in}}%
\pgfusepath{stroke}%
\end{pgfscope}%
\begin{pgfscope}%
\definecolor{textcolor}{rgb}{0.150000,0.150000,0.150000}%
\pgfsetstrokecolor{textcolor}%
\pgfsetfillcolor{textcolor}%
\pgftext[x=0.961364in,y=0.177778in,,top]{\color{textcolor}\rmfamily\fontsize{10.000000}{12.000000}\selectfont \(\displaystyle {0.00}\)}%
\end{pgfscope}%
\begin{pgfscope}%
\pgfpathrectangle{\pgfqpoint{0.750000in}{0.275000in}}{\pgfqpoint{4.650000in}{1.925000in}}%
\pgfusepath{clip}%
\pgfsetroundcap%
\pgfsetroundjoin%
\pgfsetlinewidth{1.003750pt}%
\definecolor{currentstroke}{rgb}{1.000000,1.000000,1.000000}%
\pgfsetstrokecolor{currentstroke}%
\pgfsetdash{}{0pt}%
\pgfpathmoveto{\pgfqpoint{1.489773in}{0.275000in}}%
\pgfpathlineto{\pgfqpoint{1.489773in}{2.200000in}}%
\pgfusepath{stroke}%
\end{pgfscope}%
\begin{pgfscope}%
\definecolor{textcolor}{rgb}{0.150000,0.150000,0.150000}%
\pgfsetstrokecolor{textcolor}%
\pgfsetfillcolor{textcolor}%
\pgftext[x=1.489773in,y=0.177778in,,top]{\color{textcolor}\rmfamily\fontsize{10.000000}{12.000000}\selectfont \(\displaystyle {0.25}\)}%
\end{pgfscope}%
\begin{pgfscope}%
\pgfpathrectangle{\pgfqpoint{0.750000in}{0.275000in}}{\pgfqpoint{4.650000in}{1.925000in}}%
\pgfusepath{clip}%
\pgfsetroundcap%
\pgfsetroundjoin%
\pgfsetlinewidth{1.003750pt}%
\definecolor{currentstroke}{rgb}{1.000000,1.000000,1.000000}%
\pgfsetstrokecolor{currentstroke}%
\pgfsetdash{}{0pt}%
\pgfpathmoveto{\pgfqpoint{2.018182in}{0.275000in}}%
\pgfpathlineto{\pgfqpoint{2.018182in}{2.200000in}}%
\pgfusepath{stroke}%
\end{pgfscope}%
\begin{pgfscope}%
\definecolor{textcolor}{rgb}{0.150000,0.150000,0.150000}%
\pgfsetstrokecolor{textcolor}%
\pgfsetfillcolor{textcolor}%
\pgftext[x=2.018182in,y=0.177778in,,top]{\color{textcolor}\rmfamily\fontsize{10.000000}{12.000000}\selectfont \(\displaystyle {0.50}\)}%
\end{pgfscope}%
\begin{pgfscope}%
\pgfpathrectangle{\pgfqpoint{0.750000in}{0.275000in}}{\pgfqpoint{4.650000in}{1.925000in}}%
\pgfusepath{clip}%
\pgfsetroundcap%
\pgfsetroundjoin%
\pgfsetlinewidth{1.003750pt}%
\definecolor{currentstroke}{rgb}{1.000000,1.000000,1.000000}%
\pgfsetstrokecolor{currentstroke}%
\pgfsetdash{}{0pt}%
\pgfpathmoveto{\pgfqpoint{2.546591in}{0.275000in}}%
\pgfpathlineto{\pgfqpoint{2.546591in}{2.200000in}}%
\pgfusepath{stroke}%
\end{pgfscope}%
\begin{pgfscope}%
\definecolor{textcolor}{rgb}{0.150000,0.150000,0.150000}%
\pgfsetstrokecolor{textcolor}%
\pgfsetfillcolor{textcolor}%
\pgftext[x=2.546591in,y=0.177778in,,top]{\color{textcolor}\rmfamily\fontsize{10.000000}{12.000000}\selectfont \(\displaystyle {0.75}\)}%
\end{pgfscope}%
\begin{pgfscope}%
\pgfpathrectangle{\pgfqpoint{0.750000in}{0.275000in}}{\pgfqpoint{4.650000in}{1.925000in}}%
\pgfusepath{clip}%
\pgfsetroundcap%
\pgfsetroundjoin%
\pgfsetlinewidth{1.003750pt}%
\definecolor{currentstroke}{rgb}{1.000000,1.000000,1.000000}%
\pgfsetstrokecolor{currentstroke}%
\pgfsetdash{}{0pt}%
\pgfpathmoveto{\pgfqpoint{3.075000in}{0.275000in}}%
\pgfpathlineto{\pgfqpoint{3.075000in}{2.200000in}}%
\pgfusepath{stroke}%
\end{pgfscope}%
\begin{pgfscope}%
\definecolor{textcolor}{rgb}{0.150000,0.150000,0.150000}%
\pgfsetstrokecolor{textcolor}%
\pgfsetfillcolor{textcolor}%
\pgftext[x=3.075000in,y=0.177778in,,top]{\color{textcolor}\rmfamily\fontsize{10.000000}{12.000000}\selectfont \(\displaystyle {1.00}\)}%
\end{pgfscope}%
\begin{pgfscope}%
\pgfpathrectangle{\pgfqpoint{0.750000in}{0.275000in}}{\pgfqpoint{4.650000in}{1.925000in}}%
\pgfusepath{clip}%
\pgfsetroundcap%
\pgfsetroundjoin%
\pgfsetlinewidth{1.003750pt}%
\definecolor{currentstroke}{rgb}{1.000000,1.000000,1.000000}%
\pgfsetstrokecolor{currentstroke}%
\pgfsetdash{}{0pt}%
\pgfpathmoveto{\pgfqpoint{3.603409in}{0.275000in}}%
\pgfpathlineto{\pgfqpoint{3.603409in}{2.200000in}}%
\pgfusepath{stroke}%
\end{pgfscope}%
\begin{pgfscope}%
\definecolor{textcolor}{rgb}{0.150000,0.150000,0.150000}%
\pgfsetstrokecolor{textcolor}%
\pgfsetfillcolor{textcolor}%
\pgftext[x=3.603409in,y=0.177778in,,top]{\color{textcolor}\rmfamily\fontsize{10.000000}{12.000000}\selectfont \(\displaystyle {1.25}\)}%
\end{pgfscope}%
\begin{pgfscope}%
\pgfpathrectangle{\pgfqpoint{0.750000in}{0.275000in}}{\pgfqpoint{4.650000in}{1.925000in}}%
\pgfusepath{clip}%
\pgfsetroundcap%
\pgfsetroundjoin%
\pgfsetlinewidth{1.003750pt}%
\definecolor{currentstroke}{rgb}{1.000000,1.000000,1.000000}%
\pgfsetstrokecolor{currentstroke}%
\pgfsetdash{}{0pt}%
\pgfpathmoveto{\pgfqpoint{4.131818in}{0.275000in}}%
\pgfpathlineto{\pgfqpoint{4.131818in}{2.200000in}}%
\pgfusepath{stroke}%
\end{pgfscope}%
\begin{pgfscope}%
\definecolor{textcolor}{rgb}{0.150000,0.150000,0.150000}%
\pgfsetstrokecolor{textcolor}%
\pgfsetfillcolor{textcolor}%
\pgftext[x=4.131818in,y=0.177778in,,top]{\color{textcolor}\rmfamily\fontsize{10.000000}{12.000000}\selectfont \(\displaystyle {1.50}\)}%
\end{pgfscope}%
\begin{pgfscope}%
\pgfpathrectangle{\pgfqpoint{0.750000in}{0.275000in}}{\pgfqpoint{4.650000in}{1.925000in}}%
\pgfusepath{clip}%
\pgfsetroundcap%
\pgfsetroundjoin%
\pgfsetlinewidth{1.003750pt}%
\definecolor{currentstroke}{rgb}{1.000000,1.000000,1.000000}%
\pgfsetstrokecolor{currentstroke}%
\pgfsetdash{}{0pt}%
\pgfpathmoveto{\pgfqpoint{4.660227in}{0.275000in}}%
\pgfpathlineto{\pgfqpoint{4.660227in}{2.200000in}}%
\pgfusepath{stroke}%
\end{pgfscope}%
\begin{pgfscope}%
\definecolor{textcolor}{rgb}{0.150000,0.150000,0.150000}%
\pgfsetstrokecolor{textcolor}%
\pgfsetfillcolor{textcolor}%
\pgftext[x=4.660227in,y=0.177778in,,top]{\color{textcolor}\rmfamily\fontsize{10.000000}{12.000000}\selectfont \(\displaystyle {1.75}\)}%
\end{pgfscope}%
\begin{pgfscope}%
\pgfpathrectangle{\pgfqpoint{0.750000in}{0.275000in}}{\pgfqpoint{4.650000in}{1.925000in}}%
\pgfusepath{clip}%
\pgfsetroundcap%
\pgfsetroundjoin%
\pgfsetlinewidth{1.003750pt}%
\definecolor{currentstroke}{rgb}{1.000000,1.000000,1.000000}%
\pgfsetstrokecolor{currentstroke}%
\pgfsetdash{}{0pt}%
\pgfpathmoveto{\pgfqpoint{5.188636in}{0.275000in}}%
\pgfpathlineto{\pgfqpoint{5.188636in}{2.200000in}}%
\pgfusepath{stroke}%
\end{pgfscope}%
\begin{pgfscope}%
\definecolor{textcolor}{rgb}{0.150000,0.150000,0.150000}%
\pgfsetstrokecolor{textcolor}%
\pgfsetfillcolor{textcolor}%
\pgftext[x=5.188636in,y=0.177778in,,top]{\color{textcolor}\rmfamily\fontsize{10.000000}{12.000000}\selectfont \(\displaystyle {2.00}\)}%
\end{pgfscope}%
\begin{pgfscope}%
\definecolor{textcolor}{rgb}{0.150000,0.150000,0.150000}%
\pgfsetstrokecolor{textcolor}%
\pgfsetfillcolor{textcolor}%
\pgftext[x=3.075000in,y=-0.001234in,,top]{\color{textcolor}\rmfamily\fontsize{11.000000}{13.200000}\selectfont Time (\(\displaystyle t\))}%
\end{pgfscope}%
\begin{pgfscope}%
\pgfpathrectangle{\pgfqpoint{0.750000in}{0.275000in}}{\pgfqpoint{4.650000in}{1.925000in}}%
\pgfusepath{clip}%
\pgfsetroundcap%
\pgfsetroundjoin%
\pgfsetlinewidth{1.003750pt}%
\definecolor{currentstroke}{rgb}{1.000000,1.000000,1.000000}%
\pgfsetstrokecolor{currentstroke}%
\pgfsetdash{}{0pt}%
\pgfpathmoveto{\pgfqpoint{0.750000in}{0.360281in}}%
\pgfpathlineto{\pgfqpoint{5.400000in}{0.360281in}}%
\pgfusepath{stroke}%
\end{pgfscope}%
\begin{pgfscope}%
\definecolor{textcolor}{rgb}{0.150000,0.150000,0.150000}%
\pgfsetstrokecolor{textcolor}%
\pgfsetfillcolor{textcolor}%
\pgftext[x=0.583333in, y=0.312056in, left, base]{\color{textcolor}\rmfamily\fontsize{10.000000}{12.000000}\selectfont \(\displaystyle {0}\)}%
\end{pgfscope}%
\begin{pgfscope}%
\pgfpathrectangle{\pgfqpoint{0.750000in}{0.275000in}}{\pgfqpoint{4.650000in}{1.925000in}}%
\pgfusepath{clip}%
\pgfsetroundcap%
\pgfsetroundjoin%
\pgfsetlinewidth{1.003750pt}%
\definecolor{currentstroke}{rgb}{1.000000,1.000000,1.000000}%
\pgfsetstrokecolor{currentstroke}%
\pgfsetdash{}{0pt}%
\pgfpathmoveto{\pgfqpoint{0.750000in}{0.804052in}}%
\pgfpathlineto{\pgfqpoint{5.400000in}{0.804052in}}%
\pgfusepath{stroke}%
\end{pgfscope}%
\begin{pgfscope}%
\definecolor{textcolor}{rgb}{0.150000,0.150000,0.150000}%
\pgfsetstrokecolor{textcolor}%
\pgfsetfillcolor{textcolor}%
\pgftext[x=0.513888in, y=0.755827in, left, base]{\color{textcolor}\rmfamily\fontsize{10.000000}{12.000000}\selectfont \(\displaystyle {20}\)}%
\end{pgfscope}%
\begin{pgfscope}%
\pgfpathrectangle{\pgfqpoint{0.750000in}{0.275000in}}{\pgfqpoint{4.650000in}{1.925000in}}%
\pgfusepath{clip}%
\pgfsetroundcap%
\pgfsetroundjoin%
\pgfsetlinewidth{1.003750pt}%
\definecolor{currentstroke}{rgb}{1.000000,1.000000,1.000000}%
\pgfsetstrokecolor{currentstroke}%
\pgfsetdash{}{0pt}%
\pgfpathmoveto{\pgfqpoint{0.750000in}{1.247823in}}%
\pgfpathlineto{\pgfqpoint{5.400000in}{1.247823in}}%
\pgfusepath{stroke}%
\end{pgfscope}%
\begin{pgfscope}%
\definecolor{textcolor}{rgb}{0.150000,0.150000,0.150000}%
\pgfsetstrokecolor{textcolor}%
\pgfsetfillcolor{textcolor}%
\pgftext[x=0.513888in, y=1.199597in, left, base]{\color{textcolor}\rmfamily\fontsize{10.000000}{12.000000}\selectfont \(\displaystyle {40}\)}%
\end{pgfscope}%
\begin{pgfscope}%
\pgfpathrectangle{\pgfqpoint{0.750000in}{0.275000in}}{\pgfqpoint{4.650000in}{1.925000in}}%
\pgfusepath{clip}%
\pgfsetroundcap%
\pgfsetroundjoin%
\pgfsetlinewidth{1.003750pt}%
\definecolor{currentstroke}{rgb}{1.000000,1.000000,1.000000}%
\pgfsetstrokecolor{currentstroke}%
\pgfsetdash{}{0pt}%
\pgfpathmoveto{\pgfqpoint{0.750000in}{1.691593in}}%
\pgfpathlineto{\pgfqpoint{5.400000in}{1.691593in}}%
\pgfusepath{stroke}%
\end{pgfscope}%
\begin{pgfscope}%
\definecolor{textcolor}{rgb}{0.150000,0.150000,0.150000}%
\pgfsetstrokecolor{textcolor}%
\pgfsetfillcolor{textcolor}%
\pgftext[x=0.513888in, y=1.643368in, left, base]{\color{textcolor}\rmfamily\fontsize{10.000000}{12.000000}\selectfont \(\displaystyle {60}\)}%
\end{pgfscope}%
\begin{pgfscope}%
\pgfpathrectangle{\pgfqpoint{0.750000in}{0.275000in}}{\pgfqpoint{4.650000in}{1.925000in}}%
\pgfusepath{clip}%
\pgfsetroundcap%
\pgfsetroundjoin%
\pgfsetlinewidth{1.003750pt}%
\definecolor{currentstroke}{rgb}{1.000000,1.000000,1.000000}%
\pgfsetstrokecolor{currentstroke}%
\pgfsetdash{}{0pt}%
\pgfpathmoveto{\pgfqpoint{0.750000in}{2.135364in}}%
\pgfpathlineto{\pgfqpoint{5.400000in}{2.135364in}}%
\pgfusepath{stroke}%
\end{pgfscope}%
\begin{pgfscope}%
\definecolor{textcolor}{rgb}{0.150000,0.150000,0.150000}%
\pgfsetstrokecolor{textcolor}%
\pgfsetfillcolor{textcolor}%
\pgftext[x=0.513888in, y=2.087139in, left, base]{\color{textcolor}\rmfamily\fontsize{10.000000}{12.000000}\selectfont \(\displaystyle {80}\)}%
\end{pgfscope}%
\begin{pgfscope}%
\definecolor{textcolor}{rgb}{0.150000,0.150000,0.150000}%
\pgfsetstrokecolor{textcolor}%
\pgfsetfillcolor{textcolor}%
\pgftext[x=0.458333in,y=1.237500in,,bottom,rotate=90.000000]{\color{textcolor}\rmfamily\fontsize{11.000000}{13.200000}\selectfont Position in space (\(\displaystyle \lambda_i\))}%
\end{pgfscope}%
\begin{pgfscope}%
\pgfpathrectangle{\pgfqpoint{0.750000in}{0.275000in}}{\pgfqpoint{4.650000in}{1.925000in}}%
\pgfusepath{clip}%
\pgfsetroundcap%
\pgfsetroundjoin%
\pgfsetlinewidth{1.756562pt}%
\definecolor{currentstroke}{rgb}{0.215686,0.494118,0.721569}%
\pgfsetstrokecolor{currentstroke}%
\pgfsetdash{}{0pt}%
\pgfpathmoveto{\pgfqpoint{0.961364in}{0.362500in}}%
\pgfpathlineto{\pgfqpoint{1.492428in}{0.365784in}}%
\pgfpathlineto{\pgfqpoint{2.257161in}{0.368028in}}%
\pgfpathlineto{\pgfqpoint{4.275206in}{0.370770in}}%
\pgfpathlineto{\pgfqpoint{5.188636in}{0.371969in}}%
\pgfpathlineto{\pgfqpoint{5.188636in}{0.371969in}}%
\pgfusepath{stroke}%
\end{pgfscope}%
\begin{pgfscope}%
\pgfpathrectangle{\pgfqpoint{0.750000in}{0.275000in}}{\pgfqpoint{4.650000in}{1.925000in}}%
\pgfusepath{clip}%
\pgfsetroundcap%
\pgfsetroundjoin%
\pgfsetlinewidth{1.756562pt}%
\definecolor{currentstroke}{rgb}{1.000000,0.498039,0.000000}%
\pgfsetstrokecolor{currentstroke}%
\pgfsetdash{}{0pt}%
\pgfpathmoveto{\pgfqpoint{0.961364in}{0.417694in}}%
\pgfpathlineto{\pgfqpoint{1.216275in}{0.412584in}}%
\pgfpathlineto{\pgfqpoint{1.513671in}{0.408978in}}%
\pgfpathlineto{\pgfqpoint{1.874794in}{0.406918in}}%
\pgfpathlineto{\pgfqpoint{2.342131in}{0.406574in}}%
\pgfpathlineto{\pgfqpoint{3.000651in}{0.408484in}}%
\pgfpathlineto{\pgfqpoint{3.999052in}{0.413833in}}%
\pgfpathlineto{\pgfqpoint{5.188636in}{0.421846in}}%
\pgfpathlineto{\pgfqpoint{5.188636in}{0.421846in}}%
\pgfusepath{stroke}%
\end{pgfscope}%
\begin{pgfscope}%
\pgfpathrectangle{\pgfqpoint{0.750000in}{0.275000in}}{\pgfqpoint{4.650000in}{1.925000in}}%
\pgfusepath{clip}%
\pgfsetroundcap%
\pgfsetroundjoin%
\pgfsetlinewidth{1.756562pt}%
\definecolor{currentstroke}{rgb}{0.301961,0.686275,0.290196}%
\pgfsetstrokecolor{currentstroke}%
\pgfsetdash{}{0pt}%
\pgfpathmoveto{\pgfqpoint{0.961364in}{0.472888in}}%
\pgfpathlineto{\pgfqpoint{1.152547in}{0.468227in}}%
\pgfpathlineto{\pgfqpoint{1.364973in}{0.465312in}}%
\pgfpathlineto{\pgfqpoint{1.641126in}{0.463908in}}%
\pgfpathlineto{\pgfqpoint{1.981007in}{0.464580in}}%
\pgfpathlineto{\pgfqpoint{2.405859in}{0.467740in}}%
\pgfpathlineto{\pgfqpoint{2.958166in}{0.474159in}}%
\pgfpathlineto{\pgfqpoint{3.701656in}{0.485145in}}%
\pgfpathlineto{\pgfqpoint{4.742542in}{0.502913in}}%
\pgfpathlineto{\pgfqpoint{5.188636in}{0.511024in}}%
\pgfpathlineto{\pgfqpoint{5.188636in}{0.511024in}}%
\pgfusepath{stroke}%
\end{pgfscope}%
\begin{pgfscope}%
\pgfpathrectangle{\pgfqpoint{0.750000in}{0.275000in}}{\pgfqpoint{4.650000in}{1.925000in}}%
\pgfusepath{clip}%
\pgfsetroundcap%
\pgfsetroundjoin%
\pgfsetlinewidth{1.756562pt}%
\definecolor{currentstroke}{rgb}{0.968627,0.505882,0.749020}%
\pgfsetstrokecolor{currentstroke}%
\pgfsetdash{}{0pt}%
\pgfpathmoveto{\pgfqpoint{0.961364in}{0.528082in}}%
\pgfpathlineto{\pgfqpoint{1.110062in}{0.526277in}}%
\pgfpathlineto{\pgfqpoint{1.301245in}{0.526284in}}%
\pgfpathlineto{\pgfqpoint{1.534913in}{0.528541in}}%
\pgfpathlineto{\pgfqpoint{1.853552in}{0.534053in}}%
\pgfpathlineto{\pgfqpoint{2.257161in}{0.543467in}}%
\pgfpathlineto{\pgfqpoint{2.766983in}{0.557669in}}%
\pgfpathlineto{\pgfqpoint{3.446745in}{0.578962in}}%
\pgfpathlineto{\pgfqpoint{4.338933in}{0.609246in}}%
\pgfpathlineto{\pgfqpoint{5.188636in}{0.639522in}}%
\pgfpathlineto{\pgfqpoint{5.188636in}{0.639522in}}%
\pgfusepath{stroke}%
\end{pgfscope}%
\begin{pgfscope}%
\pgfpathrectangle{\pgfqpoint{0.750000in}{0.275000in}}{\pgfqpoint{4.650000in}{1.925000in}}%
\pgfusepath{clip}%
\pgfsetroundcap%
\pgfsetroundjoin%
\pgfsetlinewidth{1.756562pt}%
\definecolor{currentstroke}{rgb}{0.650980,0.337255,0.156863}%
\pgfsetstrokecolor{currentstroke}%
\pgfsetdash{}{0pt}%
\pgfpathmoveto{\pgfqpoint{0.961364in}{0.583276in}}%
\pgfpathlineto{\pgfqpoint{1.110062in}{0.585920in}}%
\pgfpathlineto{\pgfqpoint{1.301245in}{0.591884in}}%
\pgfpathlineto{\pgfqpoint{1.556156in}{0.602232in}}%
\pgfpathlineto{\pgfqpoint{1.917280in}{0.619360in}}%
\pgfpathlineto{\pgfqpoint{2.405859in}{0.644975in}}%
\pgfpathlineto{\pgfqpoint{3.064379in}{0.681893in}}%
\pgfpathlineto{\pgfqpoint{3.956567in}{0.734309in}}%
\pgfpathlineto{\pgfqpoint{5.146151in}{0.806545in}}%
\pgfpathlineto{\pgfqpoint{5.188636in}{0.809158in}}%
\pgfpathlineto{\pgfqpoint{5.188636in}{0.809158in}}%
\pgfusepath{stroke}%
\end{pgfscope}%
\begin{pgfscope}%
\pgfpathrectangle{\pgfqpoint{0.750000in}{0.275000in}}{\pgfqpoint{4.650000in}{1.925000in}}%
\pgfusepath{clip}%
\pgfsetroundcap%
\pgfsetroundjoin%
\pgfsetlinewidth{1.756562pt}%
\definecolor{currentstroke}{rgb}{0.596078,0.305882,0.639216}%
\pgfsetstrokecolor{currentstroke}%
\pgfsetdash{}{0pt}%
\pgfpathmoveto{\pgfqpoint{0.961364in}{0.638470in}}%
\pgfpathlineto{\pgfqpoint{1.088819in}{0.646741in}}%
\pgfpathlineto{\pgfqpoint{1.280002in}{0.661558in}}%
\pgfpathlineto{\pgfqpoint{1.598641in}{0.688748in}}%
\pgfpathlineto{\pgfqpoint{2.150948in}{0.738448in}}%
\pgfpathlineto{\pgfqpoint{3.043136in}{0.821158in}}%
\pgfpathlineto{\pgfqpoint{4.402661in}{0.949586in}}%
\pgfpathlineto{\pgfqpoint{5.188636in}{1.024560in}}%
\pgfpathlineto{\pgfqpoint{5.188636in}{1.024560in}}%
\pgfusepath{stroke}%
\end{pgfscope}%
\begin{pgfscope}%
\pgfpathrectangle{\pgfqpoint{0.750000in}{0.275000in}}{\pgfqpoint{4.650000in}{1.925000in}}%
\pgfusepath{clip}%
\pgfsetroundcap%
\pgfsetroundjoin%
\pgfsetlinewidth{1.756562pt}%
\definecolor{currentstroke}{rgb}{0.600000,0.600000,0.600000}%
\pgfsetstrokecolor{currentstroke}%
\pgfsetdash{}{0pt}%
\pgfpathmoveto{\pgfqpoint{0.961364in}{0.693664in}}%
\pgfpathlineto{\pgfqpoint{1.088819in}{0.711135in}}%
\pgfpathlineto{\pgfqpoint{1.492428in}{0.769977in}}%
\pgfpathlineto{\pgfqpoint{2.235918in}{0.877558in}}%
\pgfpathlineto{\pgfqpoint{3.276804in}{1.025672in}}%
\pgfpathlineto{\pgfqpoint{5.188636in}{1.295101in}}%
\pgfpathlineto{\pgfqpoint{5.188636in}{1.295101in}}%
\pgfusepath{stroke}%
\end{pgfscope}%
\begin{pgfscope}%
\pgfpathrectangle{\pgfqpoint{0.750000in}{0.275000in}}{\pgfqpoint{4.650000in}{1.925000in}}%
\pgfusepath{clip}%
\pgfsetroundcap%
\pgfsetroundjoin%
\pgfsetlinewidth{1.756562pt}%
\definecolor{currentstroke}{rgb}{0.894118,0.101961,0.109804}%
\pgfsetstrokecolor{currentstroke}%
\pgfsetdash{}{0pt}%
\pgfpathmoveto{\pgfqpoint{0.961364in}{0.748858in}}%
\pgfpathlineto{\pgfqpoint{1.322487in}{0.836205in}}%
\pgfpathlineto{\pgfqpoint{1.577398in}{0.894427in}}%
\pgfpathlineto{\pgfqpoint{1.896037in}{0.964629in}}%
\pgfpathlineto{\pgfqpoint{2.299646in}{1.051018in}}%
\pgfpathlineto{\pgfqpoint{2.809468in}{1.157685in}}%
\pgfpathlineto{\pgfqpoint{3.467988in}{1.293044in}}%
\pgfpathlineto{\pgfqpoint{4.338933in}{1.469642in}}%
\pgfpathlineto{\pgfqpoint{5.188636in}{1.640374in}}%
\pgfpathlineto{\pgfqpoint{5.188636in}{1.640374in}}%
\pgfusepath{stroke}%
\end{pgfscope}%
\begin{pgfscope}%
\pgfpathrectangle{\pgfqpoint{0.750000in}{0.275000in}}{\pgfqpoint{4.650000in}{1.925000in}}%
\pgfusepath{clip}%
\pgfsetroundcap%
\pgfsetroundjoin%
\pgfsetlinewidth{1.756562pt}%
\definecolor{currentstroke}{rgb}{0.870588,0.870588,0.000000}%
\pgfsetstrokecolor{currentstroke}%
\pgfsetdash{}{0pt}%
\pgfpathmoveto{\pgfqpoint{0.961364in}{0.804052in}}%
\pgfpathlineto{\pgfqpoint{1.025091in}{0.832831in}}%
\pgfpathlineto{\pgfqpoint{1.110062in}{0.867985in}}%
\pgfpathlineto{\pgfqpoint{1.216275in}{0.908960in}}%
\pgfpathlineto{\pgfqpoint{1.343730in}{0.955445in}}%
\pgfpathlineto{\pgfqpoint{1.492428in}{1.007265in}}%
\pgfpathlineto{\pgfqpoint{1.683611in}{1.071337in}}%
\pgfpathlineto{\pgfqpoint{1.917280in}{1.147006in}}%
\pgfpathlineto{\pgfqpoint{2.193433in}{1.233886in}}%
\pgfpathlineto{\pgfqpoint{2.512072in}{1.331771in}}%
\pgfpathlineto{\pgfqpoint{2.894438in}{1.446919in}}%
\pgfpathlineto{\pgfqpoint{3.361775in}{1.585276in}}%
\pgfpathlineto{\pgfqpoint{3.935324in}{1.752622in}}%
\pgfpathlineto{\pgfqpoint{4.615087in}{1.948574in}}%
\pgfpathlineto{\pgfqpoint{5.188636in}{2.112500in}}%
\pgfpathlineto{\pgfqpoint{5.188636in}{2.112500in}}%
\pgfusepath{stroke}%
\end{pgfscope}%
\begin{pgfscope}%
\pgfsetrectcap%
\pgfsetmiterjoin%
\pgfsetlinewidth{0.000000pt}%
\definecolor{currentstroke}{rgb}{1.000000,1.000000,1.000000}%
\pgfsetstrokecolor{currentstroke}%
\pgfsetdash{}{0pt}%
\pgfpathmoveto{\pgfqpoint{0.750000in}{0.275000in}}%
\pgfpathlineto{\pgfqpoint{0.750000in}{2.200000in}}%
\pgfusepath{}%
\end{pgfscope}%
\begin{pgfscope}%
\pgfsetrectcap%
\pgfsetmiterjoin%
\pgfsetlinewidth{0.000000pt}%
\definecolor{currentstroke}{rgb}{1.000000,1.000000,1.000000}%
\pgfsetstrokecolor{currentstroke}%
\pgfsetdash{}{0pt}%
\pgfpathmoveto{\pgfqpoint{5.400000in}{0.275000in}}%
\pgfpathlineto{\pgfqpoint{5.400000in}{2.200000in}}%
\pgfusepath{}%
\end{pgfscope}%
\begin{pgfscope}%
\pgfsetrectcap%
\pgfsetmiterjoin%
\pgfsetlinewidth{0.000000pt}%
\definecolor{currentstroke}{rgb}{1.000000,1.000000,1.000000}%
\pgfsetstrokecolor{currentstroke}%
\pgfsetdash{}{0pt}%
\pgfpathmoveto{\pgfqpoint{0.750000in}{0.275000in}}%
\pgfpathlineto{\pgfqpoint{5.400000in}{0.275000in}}%
\pgfusepath{}%
\end{pgfscope}%
\begin{pgfscope}%
\pgfsetrectcap%
\pgfsetmiterjoin%
\pgfsetlinewidth{0.000000pt}%
\definecolor{currentstroke}{rgb}{1.000000,1.000000,1.000000}%
\pgfsetstrokecolor{currentstroke}%
\pgfsetdash{}{0pt}%
\pgfpathmoveto{\pgfqpoint{0.750000in}{2.200000in}}%
\pgfpathlineto{\pgfqpoint{5.400000in}{2.200000in}}%
\pgfusepath{}%
\end{pgfscope}%
\end{pgfpicture}%
\makeatother%
\endgroup%

\end{figure}

\begin{figure}[h!]
    %% Creator: Matplotlib, PGF backend
%%
%% To include the figure in your LaTeX document, write
%%   \input{<filename>.pgf}
%%
%% Make sure the required packages are loaded in your preamble
%%   \usepackage{pgf}
%%
%% Also ensure that all the required font packages are loaded; for instance,
%% the lmodern package is sometimes necessary when using math font.
%%   \usepackage{lmodern}
%%
%% Figures using additional raster images can only be included by \input if
%% they are in the same directory as the main LaTeX file. For loading figures
%% from other directories you can use the `import` package
%%   \usepackage{import}
%%
%% and then include the figures with
%%   \import{<path to file>}{<filename>.pgf}
%%
%% Matplotlib used the following preamble
%%   
%%   \makeatletter\@ifpackageloaded{underscore}{}{\usepackage[strings]{underscore}}\makeatother
%%
\begingroup%
\makeatletter%
\begin{pgfpicture}%
\pgfpathrectangle{\pgfpointorigin}{\pgfqpoint{6.000000in}{2.500000in}}%
\pgfusepath{use as bounding box, clip}%
\begin{pgfscope}%
\pgfsetbuttcap%
\pgfsetmiterjoin%
\definecolor{currentfill}{rgb}{1.000000,1.000000,1.000000}%
\pgfsetfillcolor{currentfill}%
\pgfsetlinewidth{0.000000pt}%
\definecolor{currentstroke}{rgb}{1.000000,1.000000,1.000000}%
\pgfsetstrokecolor{currentstroke}%
\pgfsetdash{}{0pt}%
\pgfpathmoveto{\pgfqpoint{0.000000in}{0.000000in}}%
\pgfpathlineto{\pgfqpoint{6.000000in}{0.000000in}}%
\pgfpathlineto{\pgfqpoint{6.000000in}{2.500000in}}%
\pgfpathlineto{\pgfqpoint{0.000000in}{2.500000in}}%
\pgfpathlineto{\pgfqpoint{0.000000in}{0.000000in}}%
\pgfpathclose%
\pgfusepath{fill}%
\end{pgfscope}%
\begin{pgfscope}%
\pgfsetbuttcap%
\pgfsetmiterjoin%
\definecolor{currentfill}{rgb}{0.917647,0.917647,0.949020}%
\pgfsetfillcolor{currentfill}%
\pgfsetlinewidth{0.000000pt}%
\definecolor{currentstroke}{rgb}{0.000000,0.000000,0.000000}%
\pgfsetstrokecolor{currentstroke}%
\pgfsetstrokeopacity{0.000000}%
\pgfsetdash{}{0pt}%
\pgfpathmoveto{\pgfqpoint{0.750000in}{0.275000in}}%
\pgfpathlineto{\pgfqpoint{5.400000in}{0.275000in}}%
\pgfpathlineto{\pgfqpoint{5.400000in}{2.200000in}}%
\pgfpathlineto{\pgfqpoint{0.750000in}{2.200000in}}%
\pgfpathlineto{\pgfqpoint{0.750000in}{0.275000in}}%
\pgfpathclose%
\pgfusepath{fill}%
\end{pgfscope}%
\begin{pgfscope}%
\pgfpathrectangle{\pgfqpoint{0.750000in}{0.275000in}}{\pgfqpoint{4.650000in}{1.925000in}}%
\pgfusepath{clip}%
\pgfsetroundcap%
\pgfsetroundjoin%
\pgfsetlinewidth{1.003750pt}%
\definecolor{currentstroke}{rgb}{1.000000,1.000000,1.000000}%
\pgfsetstrokecolor{currentstroke}%
\pgfsetdash{}{0pt}%
\pgfpathmoveto{\pgfqpoint{0.961364in}{0.275000in}}%
\pgfpathlineto{\pgfqpoint{0.961364in}{2.200000in}}%
\pgfusepath{stroke}%
\end{pgfscope}%
\begin{pgfscope}%
\definecolor{textcolor}{rgb}{0.150000,0.150000,0.150000}%
\pgfsetstrokecolor{textcolor}%
\pgfsetfillcolor{textcolor}%
\pgftext[x=0.961364in,y=0.177778in,,top]{\color{textcolor}\rmfamily\fontsize{10.000000}{12.000000}\selectfont \(\displaystyle {0.00}\)}%
\end{pgfscope}%
\begin{pgfscope}%
\pgfpathrectangle{\pgfqpoint{0.750000in}{0.275000in}}{\pgfqpoint{4.650000in}{1.925000in}}%
\pgfusepath{clip}%
\pgfsetroundcap%
\pgfsetroundjoin%
\pgfsetlinewidth{1.003750pt}%
\definecolor{currentstroke}{rgb}{1.000000,1.000000,1.000000}%
\pgfsetstrokecolor{currentstroke}%
\pgfsetdash{}{0pt}%
\pgfpathmoveto{\pgfqpoint{1.806818in}{0.275000in}}%
\pgfpathlineto{\pgfqpoint{1.806818in}{2.200000in}}%
\pgfusepath{stroke}%
\end{pgfscope}%
\begin{pgfscope}%
\definecolor{textcolor}{rgb}{0.150000,0.150000,0.150000}%
\pgfsetstrokecolor{textcolor}%
\pgfsetfillcolor{textcolor}%
\pgftext[x=1.806818in,y=0.177778in,,top]{\color{textcolor}\rmfamily\fontsize{10.000000}{12.000000}\selectfont \(\displaystyle {0.01}\)}%
\end{pgfscope}%
\begin{pgfscope}%
\pgfpathrectangle{\pgfqpoint{0.750000in}{0.275000in}}{\pgfqpoint{4.650000in}{1.925000in}}%
\pgfusepath{clip}%
\pgfsetroundcap%
\pgfsetroundjoin%
\pgfsetlinewidth{1.003750pt}%
\definecolor{currentstroke}{rgb}{1.000000,1.000000,1.000000}%
\pgfsetstrokecolor{currentstroke}%
\pgfsetdash{}{0pt}%
\pgfpathmoveto{\pgfqpoint{2.652273in}{0.275000in}}%
\pgfpathlineto{\pgfqpoint{2.652273in}{2.200000in}}%
\pgfusepath{stroke}%
\end{pgfscope}%
\begin{pgfscope}%
\definecolor{textcolor}{rgb}{0.150000,0.150000,0.150000}%
\pgfsetstrokecolor{textcolor}%
\pgfsetfillcolor{textcolor}%
\pgftext[x=2.652273in,y=0.177778in,,top]{\color{textcolor}\rmfamily\fontsize{10.000000}{12.000000}\selectfont \(\displaystyle {0.02}\)}%
\end{pgfscope}%
\begin{pgfscope}%
\pgfpathrectangle{\pgfqpoint{0.750000in}{0.275000in}}{\pgfqpoint{4.650000in}{1.925000in}}%
\pgfusepath{clip}%
\pgfsetroundcap%
\pgfsetroundjoin%
\pgfsetlinewidth{1.003750pt}%
\definecolor{currentstroke}{rgb}{1.000000,1.000000,1.000000}%
\pgfsetstrokecolor{currentstroke}%
\pgfsetdash{}{0pt}%
\pgfpathmoveto{\pgfqpoint{3.497727in}{0.275000in}}%
\pgfpathlineto{\pgfqpoint{3.497727in}{2.200000in}}%
\pgfusepath{stroke}%
\end{pgfscope}%
\begin{pgfscope}%
\definecolor{textcolor}{rgb}{0.150000,0.150000,0.150000}%
\pgfsetstrokecolor{textcolor}%
\pgfsetfillcolor{textcolor}%
\pgftext[x=3.497727in,y=0.177778in,,top]{\color{textcolor}\rmfamily\fontsize{10.000000}{12.000000}\selectfont \(\displaystyle {0.03}\)}%
\end{pgfscope}%
\begin{pgfscope}%
\pgfpathrectangle{\pgfqpoint{0.750000in}{0.275000in}}{\pgfqpoint{4.650000in}{1.925000in}}%
\pgfusepath{clip}%
\pgfsetroundcap%
\pgfsetroundjoin%
\pgfsetlinewidth{1.003750pt}%
\definecolor{currentstroke}{rgb}{1.000000,1.000000,1.000000}%
\pgfsetstrokecolor{currentstroke}%
\pgfsetdash{}{0pt}%
\pgfpathmoveto{\pgfqpoint{4.343182in}{0.275000in}}%
\pgfpathlineto{\pgfqpoint{4.343182in}{2.200000in}}%
\pgfusepath{stroke}%
\end{pgfscope}%
\begin{pgfscope}%
\definecolor{textcolor}{rgb}{0.150000,0.150000,0.150000}%
\pgfsetstrokecolor{textcolor}%
\pgfsetfillcolor{textcolor}%
\pgftext[x=4.343182in,y=0.177778in,,top]{\color{textcolor}\rmfamily\fontsize{10.000000}{12.000000}\selectfont \(\displaystyle {0.04}\)}%
\end{pgfscope}%
\begin{pgfscope}%
\pgfpathrectangle{\pgfqpoint{0.750000in}{0.275000in}}{\pgfqpoint{4.650000in}{1.925000in}}%
\pgfusepath{clip}%
\pgfsetroundcap%
\pgfsetroundjoin%
\pgfsetlinewidth{1.003750pt}%
\definecolor{currentstroke}{rgb}{1.000000,1.000000,1.000000}%
\pgfsetstrokecolor{currentstroke}%
\pgfsetdash{}{0pt}%
\pgfpathmoveto{\pgfqpoint{5.188636in}{0.275000in}}%
\pgfpathlineto{\pgfqpoint{5.188636in}{2.200000in}}%
\pgfusepath{stroke}%
\end{pgfscope}%
\begin{pgfscope}%
\definecolor{textcolor}{rgb}{0.150000,0.150000,0.150000}%
\pgfsetstrokecolor{textcolor}%
\pgfsetfillcolor{textcolor}%
\pgftext[x=5.188636in,y=0.177778in,,top]{\color{textcolor}\rmfamily\fontsize{10.000000}{12.000000}\selectfont \(\displaystyle {0.05}\)}%
\end{pgfscope}%
\begin{pgfscope}%
\definecolor{textcolor}{rgb}{0.150000,0.150000,0.150000}%
\pgfsetstrokecolor{textcolor}%
\pgfsetfillcolor{textcolor}%
\pgftext[x=3.075000in,y=-0.001234in,,top]{\color{textcolor}\rmfamily\fontsize{11.000000}{13.200000}\selectfont Time (\(\displaystyle t\))}%
\end{pgfscope}%
\begin{pgfscope}%
\pgfpathrectangle{\pgfqpoint{0.750000in}{0.275000in}}{\pgfqpoint{4.650000in}{1.925000in}}%
\pgfusepath{clip}%
\pgfsetroundcap%
\pgfsetroundjoin%
\pgfsetlinewidth{1.003750pt}%
\definecolor{currentstroke}{rgb}{1.000000,1.000000,1.000000}%
\pgfsetstrokecolor{currentstroke}%
\pgfsetdash{}{0pt}%
\pgfpathmoveto{\pgfqpoint{0.750000in}{0.653071in}}%
\pgfpathlineto{\pgfqpoint{5.400000in}{0.653071in}}%
\pgfusepath{stroke}%
\end{pgfscope}%
\begin{pgfscope}%
\definecolor{textcolor}{rgb}{0.150000,0.150000,0.150000}%
\pgfsetstrokecolor{textcolor}%
\pgfsetfillcolor{textcolor}%
\pgftext[x=0.475308in, y=0.604846in, left, base]{\color{textcolor}\rmfamily\fontsize{10.000000}{12.000000}\selectfont \(\displaystyle {0.2}\)}%
\end{pgfscope}%
\begin{pgfscope}%
\pgfpathrectangle{\pgfqpoint{0.750000in}{0.275000in}}{\pgfqpoint{4.650000in}{1.925000in}}%
\pgfusepath{clip}%
\pgfsetroundcap%
\pgfsetroundjoin%
\pgfsetlinewidth{1.003750pt}%
\definecolor{currentstroke}{rgb}{1.000000,1.000000,1.000000}%
\pgfsetstrokecolor{currentstroke}%
\pgfsetdash{}{0pt}%
\pgfpathmoveto{\pgfqpoint{0.750000in}{1.052030in}}%
\pgfpathlineto{\pgfqpoint{5.400000in}{1.052030in}}%
\pgfusepath{stroke}%
\end{pgfscope}%
\begin{pgfscope}%
\definecolor{textcolor}{rgb}{0.150000,0.150000,0.150000}%
\pgfsetstrokecolor{textcolor}%
\pgfsetfillcolor{textcolor}%
\pgftext[x=0.475308in, y=1.003804in, left, base]{\color{textcolor}\rmfamily\fontsize{10.000000}{12.000000}\selectfont \(\displaystyle {0.4}\)}%
\end{pgfscope}%
\begin{pgfscope}%
\pgfpathrectangle{\pgfqpoint{0.750000in}{0.275000in}}{\pgfqpoint{4.650000in}{1.925000in}}%
\pgfusepath{clip}%
\pgfsetroundcap%
\pgfsetroundjoin%
\pgfsetlinewidth{1.003750pt}%
\definecolor{currentstroke}{rgb}{1.000000,1.000000,1.000000}%
\pgfsetstrokecolor{currentstroke}%
\pgfsetdash{}{0pt}%
\pgfpathmoveto{\pgfqpoint{0.750000in}{1.450988in}}%
\pgfpathlineto{\pgfqpoint{5.400000in}{1.450988in}}%
\pgfusepath{stroke}%
\end{pgfscope}%
\begin{pgfscope}%
\definecolor{textcolor}{rgb}{0.150000,0.150000,0.150000}%
\pgfsetstrokecolor{textcolor}%
\pgfsetfillcolor{textcolor}%
\pgftext[x=0.475308in, y=1.402762in, left, base]{\color{textcolor}\rmfamily\fontsize{10.000000}{12.000000}\selectfont \(\displaystyle {0.6}\)}%
\end{pgfscope}%
\begin{pgfscope}%
\pgfpathrectangle{\pgfqpoint{0.750000in}{0.275000in}}{\pgfqpoint{4.650000in}{1.925000in}}%
\pgfusepath{clip}%
\pgfsetroundcap%
\pgfsetroundjoin%
\pgfsetlinewidth{1.003750pt}%
\definecolor{currentstroke}{rgb}{1.000000,1.000000,1.000000}%
\pgfsetstrokecolor{currentstroke}%
\pgfsetdash{}{0pt}%
\pgfpathmoveto{\pgfqpoint{0.750000in}{1.849946in}}%
\pgfpathlineto{\pgfqpoint{5.400000in}{1.849946in}}%
\pgfusepath{stroke}%
\end{pgfscope}%
\begin{pgfscope}%
\definecolor{textcolor}{rgb}{0.150000,0.150000,0.150000}%
\pgfsetstrokecolor{textcolor}%
\pgfsetfillcolor{textcolor}%
\pgftext[x=0.475308in, y=1.801720in, left, base]{\color{textcolor}\rmfamily\fontsize{10.000000}{12.000000}\selectfont \(\displaystyle {0.8}\)}%
\end{pgfscope}%
\begin{pgfscope}%
\definecolor{textcolor}{rgb}{0.150000,0.150000,0.150000}%
\pgfsetstrokecolor{textcolor}%
\pgfsetfillcolor{textcolor}%
\pgftext[x=0.419752in,y=1.237500in,,bottom,rotate=90.000000]{\color{textcolor}\rmfamily\fontsize{11.000000}{13.200000}\selectfont Position in space (\(\displaystyle \lambda_i\))}%
\end{pgfscope}%
\begin{pgfscope}%
\pgfpathrectangle{\pgfqpoint{0.750000in}{0.275000in}}{\pgfqpoint{4.650000in}{1.925000in}}%
\pgfusepath{clip}%
\pgfsetroundcap%
\pgfsetroundjoin%
\pgfsetlinewidth{1.756562pt}%
\definecolor{currentstroke}{rgb}{0.215686,0.494118,0.721569}%
\pgfsetstrokecolor{currentstroke}%
\pgfsetdash{}{0pt}%
\pgfpathmoveto{\pgfqpoint{0.961364in}{0.852550in}}%
\pgfpathlineto{\pgfqpoint{1.047635in}{0.764461in}}%
\pgfpathlineto{\pgfqpoint{1.133905in}{0.714947in}}%
\pgfpathlineto{\pgfqpoint{1.220176in}{0.675946in}}%
\pgfpathlineto{\pgfqpoint{1.306447in}{0.643877in}}%
\pgfpathlineto{\pgfqpoint{1.392718in}{0.616683in}}%
\pgfpathlineto{\pgfqpoint{1.478989in}{0.593155in}}%
\pgfpathlineto{\pgfqpoint{1.565260in}{0.572513in}}%
\pgfpathlineto{\pgfqpoint{1.651531in}{0.554215in}}%
\pgfpathlineto{\pgfqpoint{1.737801in}{0.537865in}}%
\pgfpathlineto{\pgfqpoint{1.824072in}{0.523166in}}%
\pgfpathlineto{\pgfqpoint{1.910343in}{0.509883in}}%
\pgfpathlineto{\pgfqpoint{1.996614in}{0.497831in}}%
\pgfpathlineto{\pgfqpoint{2.082885in}{0.486857in}}%
\pgfpathlineto{\pgfqpoint{2.169156in}{0.476836in}}%
\pgfpathlineto{\pgfqpoint{2.255427in}{0.467662in}}%
\pgfpathlineto{\pgfqpoint{2.341698in}{0.459246in}}%
\pgfpathlineto{\pgfqpoint{2.427968in}{0.451509in}}%
\pgfpathlineto{\pgfqpoint{2.514239in}{0.444386in}}%
\pgfpathlineto{\pgfqpoint{2.600510in}{0.437819in}}%
\pgfpathlineto{\pgfqpoint{2.686781in}{0.431756in}}%
\pgfpathlineto{\pgfqpoint{2.773052in}{0.426153in}}%
\pgfpathlineto{\pgfqpoint{2.859323in}{0.420969in}}%
\pgfpathlineto{\pgfqpoint{2.945594in}{0.416169in}}%
\pgfpathlineto{\pgfqpoint{3.031865in}{0.411721in}}%
\pgfpathlineto{\pgfqpoint{3.118135in}{0.407596in}}%
\pgfpathlineto{\pgfqpoint{3.204406in}{0.403769in}}%
\pgfpathlineto{\pgfqpoint{3.290677in}{0.400215in}}%
\pgfpathlineto{\pgfqpoint{3.376948in}{0.396914in}}%
\pgfpathlineto{\pgfqpoint{3.463219in}{0.393846in}}%
\pgfpathlineto{\pgfqpoint{3.549490in}{0.390994in}}%
\pgfpathlineto{\pgfqpoint{3.635761in}{0.388342in}}%
\pgfpathlineto{\pgfqpoint{3.722032in}{0.385875in}}%
\pgfpathlineto{\pgfqpoint{3.808302in}{0.383579in}}%
\pgfpathlineto{\pgfqpoint{3.894573in}{0.381443in}}%
\pgfpathlineto{\pgfqpoint{3.980844in}{0.379454in}}%
\pgfpathlineto{\pgfqpoint{4.067115in}{0.377603in}}%
\pgfpathlineto{\pgfqpoint{4.153386in}{0.375879in}}%
\pgfpathlineto{\pgfqpoint{4.239657in}{0.374274in}}%
\pgfpathlineto{\pgfqpoint{4.325928in}{0.372780in}}%
\pgfpathlineto{\pgfqpoint{4.412199in}{0.371389in}}%
\pgfpathlineto{\pgfqpoint{4.498469in}{0.370094in}}%
\pgfpathlineto{\pgfqpoint{4.584740in}{0.368888in}}%
\pgfpathlineto{\pgfqpoint{4.671011in}{0.367766in}}%
\pgfpathlineto{\pgfqpoint{4.757282in}{0.366722in}}%
\pgfpathlineto{\pgfqpoint{4.843553in}{0.365750in}}%
\pgfpathlineto{\pgfqpoint{4.929824in}{0.364847in}}%
\pgfpathlineto{\pgfqpoint{5.016095in}{0.364006in}}%
\pgfpathlineto{\pgfqpoint{5.102365in}{0.363226in}}%
\pgfpathlineto{\pgfqpoint{5.188636in}{0.362500in}}%
\pgfusepath{stroke}%
\end{pgfscope}%
\begin{pgfscope}%
\pgfpathrectangle{\pgfqpoint{0.750000in}{0.275000in}}{\pgfqpoint{4.650000in}{1.925000in}}%
\pgfusepath{clip}%
\pgfsetroundcap%
\pgfsetroundjoin%
\pgfsetlinewidth{1.756562pt}%
\definecolor{currentstroke}{rgb}{1.000000,0.498039,0.000000}%
\pgfsetstrokecolor{currentstroke}%
\pgfsetdash{}{0pt}%
\pgfpathmoveto{\pgfqpoint{0.961364in}{0.902420in}}%
\pgfpathlineto{\pgfqpoint{1.047635in}{0.849399in}}%
\pgfpathlineto{\pgfqpoint{1.133905in}{0.810054in}}%
\pgfpathlineto{\pgfqpoint{1.220176in}{0.780756in}}%
\pgfpathlineto{\pgfqpoint{1.306447in}{0.756601in}}%
\pgfpathlineto{\pgfqpoint{1.392718in}{0.735973in}}%
\pgfpathlineto{\pgfqpoint{1.478989in}{0.717967in}}%
\pgfpathlineto{\pgfqpoint{1.565260in}{0.702017in}}%
\pgfpathlineto{\pgfqpoint{1.651531in}{0.687734in}}%
\pgfpathlineto{\pgfqpoint{1.737801in}{0.674841in}}%
\pgfpathlineto{\pgfqpoint{1.824072in}{0.663127in}}%
\pgfpathlineto{\pgfqpoint{1.910343in}{0.652430in}}%
\pgfpathlineto{\pgfqpoint{1.996614in}{0.642621in}}%
\pgfpathlineto{\pgfqpoint{2.082885in}{0.633595in}}%
\pgfpathlineto{\pgfqpoint{2.169156in}{0.625265in}}%
\pgfpathlineto{\pgfqpoint{2.255427in}{0.617558in}}%
\pgfpathlineto{\pgfqpoint{2.341698in}{0.610413in}}%
\pgfpathlineto{\pgfqpoint{2.427968in}{0.603777in}}%
\pgfpathlineto{\pgfqpoint{2.514239in}{0.597604in}}%
\pgfpathlineto{\pgfqpoint{2.600510in}{0.591855in}}%
\pgfpathlineto{\pgfqpoint{2.686781in}{0.586493in}}%
\pgfpathlineto{\pgfqpoint{2.773052in}{0.581489in}}%
\pgfpathlineto{\pgfqpoint{2.859323in}{0.576814in}}%
\pgfpathlineto{\pgfqpoint{2.945594in}{0.572444in}}%
\pgfpathlineto{\pgfqpoint{3.031865in}{0.568356in}}%
\pgfpathlineto{\pgfqpoint{3.118135in}{0.564531in}}%
\pgfpathlineto{\pgfqpoint{3.204406in}{0.560949in}}%
\pgfpathlineto{\pgfqpoint{3.290677in}{0.557596in}}%
\pgfpathlineto{\pgfqpoint{3.376948in}{0.554454in}}%
\pgfpathlineto{\pgfqpoint{3.463219in}{0.551512in}}%
\pgfpathlineto{\pgfqpoint{3.549490in}{0.548755in}}%
\pgfpathlineto{\pgfqpoint{3.635761in}{0.546172in}}%
\pgfpathlineto{\pgfqpoint{3.722032in}{0.543752in}}%
\pgfpathlineto{\pgfqpoint{3.808302in}{0.541486in}}%
\pgfpathlineto{\pgfqpoint{3.894573in}{0.539365in}}%
\pgfpathlineto{\pgfqpoint{3.980844in}{0.537378in}}%
\pgfpathlineto{\pgfqpoint{4.067115in}{0.535519in}}%
\pgfpathlineto{\pgfqpoint{4.153386in}{0.533780in}}%
\pgfpathlineto{\pgfqpoint{4.239657in}{0.532155in}}%
\pgfpathlineto{\pgfqpoint{4.325928in}{0.530636in}}%
\pgfpathlineto{\pgfqpoint{4.412199in}{0.529217in}}%
\pgfpathlineto{\pgfqpoint{4.498469in}{0.527893in}}%
\pgfpathlineto{\pgfqpoint{4.584740in}{0.526659in}}%
\pgfpathlineto{\pgfqpoint{4.671011in}{0.525509in}}%
\pgfpathlineto{\pgfqpoint{4.757282in}{0.524439in}}%
\pgfpathlineto{\pgfqpoint{4.843553in}{0.523445in}}%
\pgfpathlineto{\pgfqpoint{4.929824in}{0.522521in}}%
\pgfpathlineto{\pgfqpoint{5.016095in}{0.521665in}}%
\pgfpathlineto{\pgfqpoint{5.102365in}{0.520872in}}%
\pgfpathlineto{\pgfqpoint{5.188636in}{0.520139in}}%
\pgfusepath{stroke}%
\end{pgfscope}%
\begin{pgfscope}%
\pgfpathrectangle{\pgfqpoint{0.750000in}{0.275000in}}{\pgfqpoint{4.650000in}{1.925000in}}%
\pgfusepath{clip}%
\pgfsetroundcap%
\pgfsetroundjoin%
\pgfsetlinewidth{1.756562pt}%
\definecolor{currentstroke}{rgb}{0.301961,0.686275,0.290196}%
\pgfsetstrokecolor{currentstroke}%
\pgfsetdash{}{0pt}%
\pgfpathmoveto{\pgfqpoint{0.961364in}{0.952290in}}%
\pgfpathlineto{\pgfqpoint{1.047635in}{0.920294in}}%
\pgfpathlineto{\pgfqpoint{1.133905in}{0.895239in}}%
\pgfpathlineto{\pgfqpoint{1.220176in}{0.876232in}}%
\pgfpathlineto{\pgfqpoint{1.306447in}{0.860732in}}%
\pgfpathlineto{\pgfqpoint{1.392718in}{0.847552in}}%
\pgfpathlineto{\pgfqpoint{1.478989in}{0.836080in}}%
\pgfpathlineto{\pgfqpoint{1.565260in}{0.825941in}}%
\pgfpathlineto{\pgfqpoint{1.651531in}{0.816881in}}%
\pgfpathlineto{\pgfqpoint{1.737801in}{0.808719in}}%
\pgfpathlineto{\pgfqpoint{1.824072in}{0.801320in}}%
\pgfpathlineto{\pgfqpoint{1.910343in}{0.794579in}}%
\pgfpathlineto{\pgfqpoint{1.996614in}{0.788412in}}%
\pgfpathlineto{\pgfqpoint{2.082885in}{0.782754in}}%
\pgfpathlineto{\pgfqpoint{2.169156in}{0.777549in}}%
\pgfpathlineto{\pgfqpoint{2.255427in}{0.772749in}}%
\pgfpathlineto{\pgfqpoint{2.341698in}{0.768316in}}%
\pgfpathlineto{\pgfqpoint{2.427968in}{0.764216in}}%
\pgfpathlineto{\pgfqpoint{2.514239in}{0.760420in}}%
\pgfpathlineto{\pgfqpoint{2.600510in}{0.756903in}}%
\pgfpathlineto{\pgfqpoint{2.686781in}{0.753641in}}%
\pgfpathlineto{\pgfqpoint{2.773052in}{0.750616in}}%
\pgfpathlineto{\pgfqpoint{2.859323in}{0.747809in}}%
\pgfpathlineto{\pgfqpoint{2.945594in}{0.745205in}}%
\pgfpathlineto{\pgfqpoint{3.031865in}{0.742789in}}%
\pgfpathlineto{\pgfqpoint{3.118135in}{0.740549in}}%
\pgfpathlineto{\pgfqpoint{3.204406in}{0.738472in}}%
\pgfpathlineto{\pgfqpoint{3.290677in}{0.736549in}}%
\pgfpathlineto{\pgfqpoint{3.376948in}{0.734769in}}%
\pgfpathlineto{\pgfqpoint{3.463219in}{0.733123in}}%
\pgfpathlineto{\pgfqpoint{3.549490in}{0.731604in}}%
\pgfpathlineto{\pgfqpoint{3.635761in}{0.730203in}}%
\pgfpathlineto{\pgfqpoint{3.722032in}{0.728913in}}%
\pgfpathlineto{\pgfqpoint{3.808302in}{0.727728in}}%
\pgfpathlineto{\pgfqpoint{3.894573in}{0.726642in}}%
\pgfpathlineto{\pgfqpoint{3.980844in}{0.725649in}}%
\pgfpathlineto{\pgfqpoint{4.067115in}{0.724744in}}%
\pgfpathlineto{\pgfqpoint{4.153386in}{0.723922in}}%
\pgfpathlineto{\pgfqpoint{4.239657in}{0.723177in}}%
\pgfpathlineto{\pgfqpoint{4.325928in}{0.722507in}}%
\pgfpathlineto{\pgfqpoint{4.412199in}{0.721906in}}%
\pgfpathlineto{\pgfqpoint{4.498469in}{0.721371in}}%
\pgfpathlineto{\pgfqpoint{4.584740in}{0.720898in}}%
\pgfpathlineto{\pgfqpoint{4.671011in}{0.720484in}}%
\pgfpathlineto{\pgfqpoint{4.757282in}{0.720126in}}%
\pgfpathlineto{\pgfqpoint{4.843553in}{0.719820in}}%
\pgfpathlineto{\pgfqpoint{4.929824in}{0.719563in}}%
\pgfpathlineto{\pgfqpoint{5.016095in}{0.719353in}}%
\pgfpathlineto{\pgfqpoint{5.102365in}{0.719188in}}%
\pgfpathlineto{\pgfqpoint{5.188636in}{0.719064in}}%
\pgfusepath{stroke}%
\end{pgfscope}%
\begin{pgfscope}%
\pgfpathrectangle{\pgfqpoint{0.750000in}{0.275000in}}{\pgfqpoint{4.650000in}{1.925000in}}%
\pgfusepath{clip}%
\pgfsetroundcap%
\pgfsetroundjoin%
\pgfsetlinewidth{1.756562pt}%
\definecolor{currentstroke}{rgb}{0.968627,0.505882,0.749020}%
\pgfsetstrokecolor{currentstroke}%
\pgfsetdash{}{0pt}%
\pgfpathmoveto{\pgfqpoint{0.961364in}{1.002160in}}%
\pgfpathlineto{\pgfqpoint{1.047635in}{0.987573in}}%
\pgfpathlineto{\pgfqpoint{1.133905in}{0.976685in}}%
\pgfpathlineto{\pgfqpoint{1.220176in}{0.968674in}}%
\pgfpathlineto{\pgfqpoint{1.306447in}{0.962423in}}%
\pgfpathlineto{\pgfqpoint{1.392718in}{0.957327in}}%
\pgfpathlineto{\pgfqpoint{1.478989in}{0.953072in}}%
\pgfpathlineto{\pgfqpoint{1.565260in}{0.949470in}}%
\pgfpathlineto{\pgfqpoint{1.651531in}{0.946392in}}%
\pgfpathlineto{\pgfqpoint{1.737801in}{0.943750in}}%
\pgfpathlineto{\pgfqpoint{1.824072in}{0.941474in}}%
\pgfpathlineto{\pgfqpoint{1.910343in}{0.939514in}}%
\pgfpathlineto{\pgfqpoint{1.996614in}{0.937828in}}%
\pgfpathlineto{\pgfqpoint{2.082885in}{0.936383in}}%
\pgfpathlineto{\pgfqpoint{2.169156in}{0.935152in}}%
\pgfpathlineto{\pgfqpoint{2.255427in}{0.934112in}}%
\pgfpathlineto{\pgfqpoint{2.341698in}{0.933243in}}%
\pgfpathlineto{\pgfqpoint{2.427968in}{0.932528in}}%
\pgfpathlineto{\pgfqpoint{2.514239in}{0.931955in}}%
\pgfpathlineto{\pgfqpoint{2.600510in}{0.931509in}}%
\pgfpathlineto{\pgfqpoint{2.686781in}{0.931179in}}%
\pgfpathlineto{\pgfqpoint{2.773052in}{0.930957in}}%
\pgfpathlineto{\pgfqpoint{2.859323in}{0.930832in}}%
\pgfpathlineto{\pgfqpoint{2.945594in}{0.930798in}}%
\pgfpathlineto{\pgfqpoint{3.031865in}{0.930848in}}%
\pgfpathlineto{\pgfqpoint{3.118135in}{0.930974in}}%
\pgfpathlineto{\pgfqpoint{3.204406in}{0.931171in}}%
\pgfpathlineto{\pgfqpoint{3.290677in}{0.931434in}}%
\pgfpathlineto{\pgfqpoint{3.376948in}{0.931758in}}%
\pgfpathlineto{\pgfqpoint{3.463219in}{0.932138in}}%
\pgfpathlineto{\pgfqpoint{3.549490in}{0.932570in}}%
\pgfpathlineto{\pgfqpoint{3.635761in}{0.933051in}}%
\pgfpathlineto{\pgfqpoint{3.722032in}{0.933577in}}%
\pgfpathlineto{\pgfqpoint{3.808302in}{0.934145in}}%
\pgfpathlineto{\pgfqpoint{3.894573in}{0.934751in}}%
\pgfpathlineto{\pgfqpoint{3.980844in}{0.935394in}}%
\pgfpathlineto{\pgfqpoint{4.067115in}{0.936069in}}%
\pgfpathlineto{\pgfqpoint{4.153386in}{0.936775in}}%
\pgfpathlineto{\pgfqpoint{4.239657in}{0.937509in}}%
\pgfpathlineto{\pgfqpoint{4.325928in}{0.938270in}}%
\pgfpathlineto{\pgfqpoint{4.412199in}{0.939054in}}%
\pgfpathlineto{\pgfqpoint{4.498469in}{0.939861in}}%
\pgfpathlineto{\pgfqpoint{4.584740in}{0.940688in}}%
\pgfpathlineto{\pgfqpoint{4.671011in}{0.941534in}}%
\pgfpathlineto{\pgfqpoint{4.757282in}{0.942397in}}%
\pgfpathlineto{\pgfqpoint{4.843553in}{0.943275in}}%
\pgfpathlineto{\pgfqpoint{4.929824in}{0.944168in}}%
\pgfpathlineto{\pgfqpoint{5.016095in}{0.945073in}}%
\pgfpathlineto{\pgfqpoint{5.102365in}{0.945989in}}%
\pgfpathlineto{\pgfqpoint{5.188636in}{0.946915in}}%
\pgfusepath{stroke}%
\end{pgfscope}%
\begin{pgfscope}%
\pgfpathrectangle{\pgfqpoint{0.750000in}{0.275000in}}{\pgfqpoint{4.650000in}{1.925000in}}%
\pgfusepath{clip}%
\pgfsetroundcap%
\pgfsetroundjoin%
\pgfsetlinewidth{1.756562pt}%
\definecolor{currentstroke}{rgb}{0.650980,0.337255,0.156863}%
\pgfsetstrokecolor{currentstroke}%
\pgfsetdash{}{0pt}%
\pgfpathmoveto{\pgfqpoint{0.961364in}{1.052030in}}%
\pgfpathlineto{\pgfqpoint{1.047635in}{1.054024in}}%
\pgfpathlineto{\pgfqpoint{1.133905in}{1.057342in}}%
\pgfpathlineto{\pgfqpoint{1.220176in}{1.060810in}}%
\pgfpathlineto{\pgfqpoint{1.306447in}{1.064285in}}%
\pgfpathlineto{\pgfqpoint{1.392718in}{1.067749in}}%
\pgfpathlineto{\pgfqpoint{1.478989in}{1.071196in}}%
\pgfpathlineto{\pgfqpoint{1.565260in}{1.074623in}}%
\pgfpathlineto{\pgfqpoint{1.651531in}{1.078025in}}%
\pgfpathlineto{\pgfqpoint{1.737801in}{1.081403in}}%
\pgfpathlineto{\pgfqpoint{1.824072in}{1.084755in}}%
\pgfpathlineto{\pgfqpoint{1.910343in}{1.088080in}}%
\pgfpathlineto{\pgfqpoint{1.996614in}{1.091376in}}%
\pgfpathlineto{\pgfqpoint{2.082885in}{1.094645in}}%
\pgfpathlineto{\pgfqpoint{2.169156in}{1.097885in}}%
\pgfpathlineto{\pgfqpoint{2.255427in}{1.101096in}}%
\pgfpathlineto{\pgfqpoint{2.341698in}{1.104278in}}%
\pgfpathlineto{\pgfqpoint{2.427968in}{1.107430in}}%
\pgfpathlineto{\pgfqpoint{2.514239in}{1.110553in}}%
\pgfpathlineto{\pgfqpoint{2.600510in}{1.113645in}}%
\pgfpathlineto{\pgfqpoint{2.686781in}{1.116707in}}%
\pgfpathlineto{\pgfqpoint{2.773052in}{1.119740in}}%
\pgfpathlineto{\pgfqpoint{2.859323in}{1.122741in}}%
\pgfpathlineto{\pgfqpoint{2.945594in}{1.125713in}}%
\pgfpathlineto{\pgfqpoint{3.031865in}{1.128653in}}%
\pgfpathlineto{\pgfqpoint{3.118135in}{1.131563in}}%
\pgfpathlineto{\pgfqpoint{3.204406in}{1.134443in}}%
\pgfpathlineto{\pgfqpoint{3.290677in}{1.137292in}}%
\pgfpathlineto{\pgfqpoint{3.376948in}{1.140110in}}%
\pgfpathlineto{\pgfqpoint{3.463219in}{1.142898in}}%
\pgfpathlineto{\pgfqpoint{3.549490in}{1.145654in}}%
\pgfpathlineto{\pgfqpoint{3.635761in}{1.148380in}}%
\pgfpathlineto{\pgfqpoint{3.722032in}{1.151076in}}%
\pgfpathlineto{\pgfqpoint{3.808302in}{1.153741in}}%
\pgfpathlineto{\pgfqpoint{3.894573in}{1.156375in}}%
\pgfpathlineto{\pgfqpoint{3.980844in}{1.158978in}}%
\pgfpathlineto{\pgfqpoint{4.067115in}{1.161551in}}%
\pgfpathlineto{\pgfqpoint{4.153386in}{1.164094in}}%
\pgfpathlineto{\pgfqpoint{4.239657in}{1.166606in}}%
\pgfpathlineto{\pgfqpoint{4.325928in}{1.169088in}}%
\pgfpathlineto{\pgfqpoint{4.412199in}{1.171539in}}%
\pgfpathlineto{\pgfqpoint{4.498469in}{1.173960in}}%
\pgfpathlineto{\pgfqpoint{4.584740in}{1.176351in}}%
\pgfpathlineto{\pgfqpoint{4.671011in}{1.178713in}}%
\pgfpathlineto{\pgfqpoint{4.757282in}{1.181044in}}%
\pgfpathlineto{\pgfqpoint{4.843553in}{1.183345in}}%
\pgfpathlineto{\pgfqpoint{4.929824in}{1.185617in}}%
\pgfpathlineto{\pgfqpoint{5.016095in}{1.187859in}}%
\pgfpathlineto{\pgfqpoint{5.102365in}{1.190072in}}%
\pgfpathlineto{\pgfqpoint{5.188636in}{1.192256in}}%
\pgfusepath{stroke}%
\end{pgfscope}%
\begin{pgfscope}%
\pgfpathrectangle{\pgfqpoint{0.750000in}{0.275000in}}{\pgfqpoint{4.650000in}{1.925000in}}%
\pgfusepath{clip}%
\pgfsetroundcap%
\pgfsetroundjoin%
\pgfsetlinewidth{1.756562pt}%
\definecolor{currentstroke}{rgb}{0.596078,0.305882,0.639216}%
\pgfsetstrokecolor{currentstroke}%
\pgfsetdash{}{0pt}%
\pgfpathmoveto{\pgfqpoint{0.961364in}{1.101899in}}%
\pgfpathlineto{\pgfqpoint{1.047635in}{1.121194in}}%
\pgfpathlineto{\pgfqpoint{1.133905in}{1.139267in}}%
\pgfpathlineto{\pgfqpoint{1.220176in}{1.154629in}}%
\pgfpathlineto{\pgfqpoint{1.306447in}{1.168209in}}%
\pgfpathlineto{\pgfqpoint{1.392718in}{1.180587in}}%
\pgfpathlineto{\pgfqpoint{1.478989in}{1.192069in}}%
\pgfpathlineto{\pgfqpoint{1.565260in}{1.202837in}}%
\pgfpathlineto{\pgfqpoint{1.651531in}{1.213016in}}%
\pgfpathlineto{\pgfqpoint{1.737801in}{1.222694in}}%
\pgfpathlineto{\pgfqpoint{1.824072in}{1.231936in}}%
\pgfpathlineto{\pgfqpoint{1.910343in}{1.240793in}}%
\pgfpathlineto{\pgfqpoint{1.996614in}{1.249306in}}%
\pgfpathlineto{\pgfqpoint{2.082885in}{1.257507in}}%
\pgfpathlineto{\pgfqpoint{2.169156in}{1.265424in}}%
\pgfpathlineto{\pgfqpoint{2.255427in}{1.273079in}}%
\pgfpathlineto{\pgfqpoint{2.341698in}{1.280491in}}%
\pgfpathlineto{\pgfqpoint{2.427968in}{1.287677in}}%
\pgfpathlineto{\pgfqpoint{2.514239in}{1.294652in}}%
\pgfpathlineto{\pgfqpoint{2.600510in}{1.301428in}}%
\pgfpathlineto{\pgfqpoint{2.686781in}{1.308015in}}%
\pgfpathlineto{\pgfqpoint{2.773052in}{1.314425in}}%
\pgfpathlineto{\pgfqpoint{2.859323in}{1.320666in}}%
\pgfpathlineto{\pgfqpoint{2.945594in}{1.326745in}}%
\pgfpathlineto{\pgfqpoint{3.031865in}{1.332671in}}%
\pgfpathlineto{\pgfqpoint{3.118135in}{1.338449in}}%
\pgfpathlineto{\pgfqpoint{3.204406in}{1.344087in}}%
\pgfpathlineto{\pgfqpoint{3.290677in}{1.349588in}}%
\pgfpathlineto{\pgfqpoint{3.376948in}{1.354959in}}%
\pgfpathlineto{\pgfqpoint{3.463219in}{1.360205in}}%
\pgfpathlineto{\pgfqpoint{3.549490in}{1.365328in}}%
\pgfpathlineto{\pgfqpoint{3.635761in}{1.370335in}}%
\pgfpathlineto{\pgfqpoint{3.722032in}{1.375227in}}%
\pgfpathlineto{\pgfqpoint{3.808302in}{1.380010in}}%
\pgfpathlineto{\pgfqpoint{3.894573in}{1.384686in}}%
\pgfpathlineto{\pgfqpoint{3.980844in}{1.389259in}}%
\pgfpathlineto{\pgfqpoint{4.067115in}{1.393731in}}%
\pgfpathlineto{\pgfqpoint{4.153386in}{1.398106in}}%
\pgfpathlineto{\pgfqpoint{4.239657in}{1.402385in}}%
\pgfpathlineto{\pgfqpoint{4.325928in}{1.406572in}}%
\pgfpathlineto{\pgfqpoint{4.412199in}{1.410669in}}%
\pgfpathlineto{\pgfqpoint{4.498469in}{1.414679in}}%
\pgfpathlineto{\pgfqpoint{4.584740in}{1.418602in}}%
\pgfpathlineto{\pgfqpoint{4.671011in}{1.422443in}}%
\pgfpathlineto{\pgfqpoint{4.757282in}{1.426201in}}%
\pgfpathlineto{\pgfqpoint{4.843553in}{1.429881in}}%
\pgfpathlineto{\pgfqpoint{4.929824in}{1.433483in}}%
\pgfpathlineto{\pgfqpoint{5.016095in}{1.437008in}}%
\pgfpathlineto{\pgfqpoint{5.102365in}{1.440460in}}%
\pgfpathlineto{\pgfqpoint{5.188636in}{1.443840in}}%
\pgfusepath{stroke}%
\end{pgfscope}%
\begin{pgfscope}%
\pgfpathrectangle{\pgfqpoint{0.750000in}{0.275000in}}{\pgfqpoint{4.650000in}{1.925000in}}%
\pgfusepath{clip}%
\pgfsetroundcap%
\pgfsetroundjoin%
\pgfsetlinewidth{1.756562pt}%
\definecolor{currentstroke}{rgb}{0.600000,0.600000,0.600000}%
\pgfsetstrokecolor{currentstroke}%
\pgfsetdash{}{0pt}%
\pgfpathmoveto{\pgfqpoint{0.961364in}{1.151769in}}%
\pgfpathlineto{\pgfqpoint{1.047635in}{1.190787in}}%
\pgfpathlineto{\pgfqpoint{1.133905in}{1.224695in}}%
\pgfpathlineto{\pgfqpoint{1.220176in}{1.252326in}}%
\pgfpathlineto{\pgfqpoint{1.306447in}{1.276323in}}%
\pgfpathlineto{\pgfqpoint{1.392718in}{1.297877in}}%
\pgfpathlineto{\pgfqpoint{1.478989in}{1.317601in}}%
\pgfpathlineto{\pgfqpoint{1.565260in}{1.335870in}}%
\pgfpathlineto{\pgfqpoint{1.651531in}{1.352938in}}%
\pgfpathlineto{\pgfqpoint{1.737801in}{1.368988in}}%
\pgfpathlineto{\pgfqpoint{1.824072in}{1.384156in}}%
\pgfpathlineto{\pgfqpoint{1.910343in}{1.398547in}}%
\pgfpathlineto{\pgfqpoint{1.996614in}{1.412247in}}%
\pgfpathlineto{\pgfqpoint{2.082885in}{1.425324in}}%
\pgfpathlineto{\pgfqpoint{2.169156in}{1.437834in}}%
\pgfpathlineto{\pgfqpoint{2.255427in}{1.449825in}}%
\pgfpathlineto{\pgfqpoint{2.341698in}{1.461338in}}%
\pgfpathlineto{\pgfqpoint{2.427968in}{1.472408in}}%
\pgfpathlineto{\pgfqpoint{2.514239in}{1.483065in}}%
\pgfpathlineto{\pgfqpoint{2.600510in}{1.493337in}}%
\pgfpathlineto{\pgfqpoint{2.686781in}{1.503247in}}%
\pgfpathlineto{\pgfqpoint{2.773052in}{1.512816in}}%
\pgfpathlineto{\pgfqpoint{2.859323in}{1.522063in}}%
\pgfpathlineto{\pgfqpoint{2.945594in}{1.531006in}}%
\pgfpathlineto{\pgfqpoint{3.031865in}{1.539659in}}%
\pgfpathlineto{\pgfqpoint{3.118135in}{1.548038in}}%
\pgfpathlineto{\pgfqpoint{3.204406in}{1.556154in}}%
\pgfpathlineto{\pgfqpoint{3.290677in}{1.564021in}}%
\pgfpathlineto{\pgfqpoint{3.376948in}{1.571649in}}%
\pgfpathlineto{\pgfqpoint{3.463219in}{1.579047in}}%
\pgfpathlineto{\pgfqpoint{3.549490in}{1.586227in}}%
\pgfpathlineto{\pgfqpoint{3.635761in}{1.593195in}}%
\pgfpathlineto{\pgfqpoint{3.722032in}{1.599962in}}%
\pgfpathlineto{\pgfqpoint{3.808302in}{1.606533in}}%
\pgfpathlineto{\pgfqpoint{3.894573in}{1.612917in}}%
\pgfpathlineto{\pgfqpoint{3.980844in}{1.619121in}}%
\pgfpathlineto{\pgfqpoint{4.067115in}{1.625150in}}%
\pgfpathlineto{\pgfqpoint{4.153386in}{1.631010in}}%
\pgfpathlineto{\pgfqpoint{4.239657in}{1.636709in}}%
\pgfpathlineto{\pgfqpoint{4.325928in}{1.642250in}}%
\pgfpathlineto{\pgfqpoint{4.412199in}{1.647639in}}%
\pgfpathlineto{\pgfqpoint{4.498469in}{1.652881in}}%
\pgfpathlineto{\pgfqpoint{4.584740in}{1.657980in}}%
\pgfpathlineto{\pgfqpoint{4.671011in}{1.662942in}}%
\pgfpathlineto{\pgfqpoint{4.757282in}{1.667770in}}%
\pgfpathlineto{\pgfqpoint{4.843553in}{1.672468in}}%
\pgfpathlineto{\pgfqpoint{4.929824in}{1.677040in}}%
\pgfpathlineto{\pgfqpoint{5.016095in}{1.681490in}}%
\pgfpathlineto{\pgfqpoint{5.102365in}{1.685822in}}%
\pgfpathlineto{\pgfqpoint{5.188636in}{1.690039in}}%
\pgfusepath{stroke}%
\end{pgfscope}%
\begin{pgfscope}%
\pgfpathrectangle{\pgfqpoint{0.750000in}{0.275000in}}{\pgfqpoint{4.650000in}{1.925000in}}%
\pgfusepath{clip}%
\pgfsetroundcap%
\pgfsetroundjoin%
\pgfsetlinewidth{1.756562pt}%
\definecolor{currentstroke}{rgb}{0.894118,0.101961,0.109804}%
\pgfsetstrokecolor{currentstroke}%
\pgfsetdash{}{0pt}%
\pgfpathmoveto{\pgfqpoint{0.961364in}{1.201639in}}%
\pgfpathlineto{\pgfqpoint{1.047635in}{1.266275in}}%
\pgfpathlineto{\pgfqpoint{1.133905in}{1.317240in}}%
\pgfpathlineto{\pgfqpoint{1.220176in}{1.357419in}}%
\pgfpathlineto{\pgfqpoint{1.306447in}{1.392149in}}%
\pgfpathlineto{\pgfqpoint{1.392718in}{1.423091in}}%
\pgfpathlineto{\pgfqpoint{1.478989in}{1.451165in}}%
\pgfpathlineto{\pgfqpoint{1.565260in}{1.476946in}}%
\pgfpathlineto{\pgfqpoint{1.651531in}{1.500831in}}%
\pgfpathlineto{\pgfqpoint{1.737801in}{1.523104in}}%
\pgfpathlineto{\pgfqpoint{1.824072in}{1.543980in}}%
\pgfpathlineto{\pgfqpoint{1.910343in}{1.563627in}}%
\pgfpathlineto{\pgfqpoint{1.996614in}{1.582180in}}%
\pgfpathlineto{\pgfqpoint{2.082885in}{1.599749in}}%
\pgfpathlineto{\pgfqpoint{2.169156in}{1.616425in}}%
\pgfpathlineto{\pgfqpoint{2.255427in}{1.632284in}}%
\pgfpathlineto{\pgfqpoint{2.341698in}{1.647394in}}%
\pgfpathlineto{\pgfqpoint{2.427968in}{1.661811in}}%
\pgfpathlineto{\pgfqpoint{2.514239in}{1.675585in}}%
\pgfpathlineto{\pgfqpoint{2.600510in}{1.688761in}}%
\pgfpathlineto{\pgfqpoint{2.686781in}{1.701376in}}%
\pgfpathlineto{\pgfqpoint{2.773052in}{1.713467in}}%
\pgfpathlineto{\pgfqpoint{2.859323in}{1.725064in}}%
\pgfpathlineto{\pgfqpoint{2.945594in}{1.736196in}}%
\pgfpathlineto{\pgfqpoint{3.031865in}{1.746889in}}%
\pgfpathlineto{\pgfqpoint{3.118135in}{1.757166in}}%
\pgfpathlineto{\pgfqpoint{3.204406in}{1.767049in}}%
\pgfpathlineto{\pgfqpoint{3.290677in}{1.776557in}}%
\pgfpathlineto{\pgfqpoint{3.376948in}{1.785710in}}%
\pgfpathlineto{\pgfqpoint{3.463219in}{1.794524in}}%
\pgfpathlineto{\pgfqpoint{3.549490in}{1.803014in}}%
\pgfpathlineto{\pgfqpoint{3.635761in}{1.811196in}}%
\pgfpathlineto{\pgfqpoint{3.722032in}{1.819084in}}%
\pgfpathlineto{\pgfqpoint{3.808302in}{1.826689in}}%
\pgfpathlineto{\pgfqpoint{3.894573in}{1.834025in}}%
\pgfpathlineto{\pgfqpoint{3.980844in}{1.841102in}}%
\pgfpathlineto{\pgfqpoint{4.067115in}{1.847932in}}%
\pgfpathlineto{\pgfqpoint{4.153386in}{1.854523in}}%
\pgfpathlineto{\pgfqpoint{4.239657in}{1.860887in}}%
\pgfpathlineto{\pgfqpoint{4.325928in}{1.867031in}}%
\pgfpathlineto{\pgfqpoint{4.412199in}{1.872964in}}%
\pgfpathlineto{\pgfqpoint{4.498469in}{1.878695in}}%
\pgfpathlineto{\pgfqpoint{4.584740in}{1.884231in}}%
\pgfpathlineto{\pgfqpoint{4.671011in}{1.889580in}}%
\pgfpathlineto{\pgfqpoint{4.757282in}{1.894747in}}%
\pgfpathlineto{\pgfqpoint{4.843553in}{1.899741in}}%
\pgfpathlineto{\pgfqpoint{4.929824in}{1.904567in}}%
\pgfpathlineto{\pgfqpoint{5.016095in}{1.909232in}}%
\pgfpathlineto{\pgfqpoint{5.102365in}{1.913741in}}%
\pgfpathlineto{\pgfqpoint{5.188636in}{1.918099in}}%
\pgfusepath{stroke}%
\end{pgfscope}%
\begin{pgfscope}%
\pgfpathrectangle{\pgfqpoint{0.750000in}{0.275000in}}{\pgfqpoint{4.650000in}{1.925000in}}%
\pgfusepath{clip}%
\pgfsetroundcap%
\pgfsetroundjoin%
\pgfsetlinewidth{1.756562pt}%
\definecolor{currentstroke}{rgb}{0.870588,0.870588,0.000000}%
\pgfsetstrokecolor{currentstroke}%
\pgfsetdash{}{0pt}%
\pgfpathmoveto{\pgfqpoint{0.961364in}{1.251509in}}%
\pgfpathlineto{\pgfqpoint{1.047635in}{1.360937in}}%
\pgfpathlineto{\pgfqpoint{1.133905in}{1.425172in}}%
\pgfpathlineto{\pgfqpoint{1.220176in}{1.478588in}}%
\pgfpathlineto{\pgfqpoint{1.306447in}{1.524580in}}%
\pgfpathlineto{\pgfqpoint{1.392718in}{1.565218in}}%
\pgfpathlineto{\pgfqpoint{1.478989in}{1.601731in}}%
\pgfpathlineto{\pgfqpoint{1.565260in}{1.634917in}}%
\pgfpathlineto{\pgfqpoint{1.651531in}{1.665335in}}%
\pgfpathlineto{\pgfqpoint{1.737801in}{1.693393in}}%
\pgfpathlineto{\pgfqpoint{1.824072in}{1.719405in}}%
\pgfpathlineto{\pgfqpoint{1.910343in}{1.743617in}}%
\pgfpathlineto{\pgfqpoint{1.996614in}{1.766226in}}%
\pgfpathlineto{\pgfqpoint{2.082885in}{1.787397in}}%
\pgfpathlineto{\pgfqpoint{2.169156in}{1.807266in}}%
\pgfpathlineto{\pgfqpoint{2.255427in}{1.825949in}}%
\pgfpathlineto{\pgfqpoint{2.341698in}{1.843547in}}%
\pgfpathlineto{\pgfqpoint{2.427968in}{1.860146in}}%
\pgfpathlineto{\pgfqpoint{2.514239in}{1.875823in}}%
\pgfpathlineto{\pgfqpoint{2.600510in}{1.890645in}}%
\pgfpathlineto{\pgfqpoint{2.686781in}{1.904673in}}%
\pgfpathlineto{\pgfqpoint{2.773052in}{1.917959in}}%
\pgfpathlineto{\pgfqpoint{2.859323in}{1.930554in}}%
\pgfpathlineto{\pgfqpoint{2.945594in}{1.942501in}}%
\pgfpathlineto{\pgfqpoint{3.031865in}{1.953840in}}%
\pgfpathlineto{\pgfqpoint{3.118135in}{1.964608in}}%
\pgfpathlineto{\pgfqpoint{3.204406in}{1.974838in}}%
\pgfpathlineto{\pgfqpoint{3.290677in}{1.984562in}}%
\pgfpathlineto{\pgfqpoint{3.376948in}{1.993809in}}%
\pgfpathlineto{\pgfqpoint{3.463219in}{2.002605in}}%
\pgfpathlineto{\pgfqpoint{3.549490in}{2.010974in}}%
\pgfpathlineto{\pgfqpoint{3.635761in}{2.018940in}}%
\pgfpathlineto{\pgfqpoint{3.722032in}{2.026524in}}%
\pgfpathlineto{\pgfqpoint{3.808302in}{2.033746in}}%
\pgfpathlineto{\pgfqpoint{3.894573in}{2.040625in}}%
\pgfpathlineto{\pgfqpoint{3.980844in}{2.047179in}}%
\pgfpathlineto{\pgfqpoint{4.067115in}{2.053423in}}%
\pgfpathlineto{\pgfqpoint{4.153386in}{2.059374in}}%
\pgfpathlineto{\pgfqpoint{4.239657in}{2.065046in}}%
\pgfpathlineto{\pgfqpoint{4.325928in}{2.070453in}}%
\pgfpathlineto{\pgfqpoint{4.412199in}{2.075609in}}%
\pgfpathlineto{\pgfqpoint{4.498469in}{2.080524in}}%
\pgfpathlineto{\pgfqpoint{4.584740in}{2.085212in}}%
\pgfpathlineto{\pgfqpoint{4.671011in}{2.089683in}}%
\pgfpathlineto{\pgfqpoint{4.757282in}{2.093947in}}%
\pgfpathlineto{\pgfqpoint{4.843553in}{2.098015in}}%
\pgfpathlineto{\pgfqpoint{4.929824in}{2.101895in}}%
\pgfpathlineto{\pgfqpoint{5.016095in}{2.105597in}}%
\pgfpathlineto{\pgfqpoint{5.102365in}{2.109129in}}%
\pgfpathlineto{\pgfqpoint{5.188636in}{2.112500in}}%
\pgfusepath{stroke}%
\end{pgfscope}%
\begin{pgfscope}%
\pgfsetrectcap%
\pgfsetmiterjoin%
\pgfsetlinewidth{0.000000pt}%
\definecolor{currentstroke}{rgb}{1.000000,1.000000,1.000000}%
\pgfsetstrokecolor{currentstroke}%
\pgfsetdash{}{0pt}%
\pgfpathmoveto{\pgfqpoint{0.750000in}{0.275000in}}%
\pgfpathlineto{\pgfqpoint{0.750000in}{2.200000in}}%
\pgfusepath{}%
\end{pgfscope}%
\begin{pgfscope}%
\pgfsetrectcap%
\pgfsetmiterjoin%
\pgfsetlinewidth{0.000000pt}%
\definecolor{currentstroke}{rgb}{1.000000,1.000000,1.000000}%
\pgfsetstrokecolor{currentstroke}%
\pgfsetdash{}{0pt}%
\pgfpathmoveto{\pgfqpoint{5.400000in}{0.275000in}}%
\pgfpathlineto{\pgfqpoint{5.400000in}{2.200000in}}%
\pgfusepath{}%
\end{pgfscope}%
\begin{pgfscope}%
\pgfsetrectcap%
\pgfsetmiterjoin%
\pgfsetlinewidth{0.000000pt}%
\definecolor{currentstroke}{rgb}{1.000000,1.000000,1.000000}%
\pgfsetstrokecolor{currentstroke}%
\pgfsetdash{}{0pt}%
\pgfpathmoveto{\pgfqpoint{0.750000in}{0.275000in}}%
\pgfpathlineto{\pgfqpoint{5.400000in}{0.275000in}}%
\pgfusepath{}%
\end{pgfscope}%
\begin{pgfscope}%
\pgfsetrectcap%
\pgfsetmiterjoin%
\pgfsetlinewidth{0.000000pt}%
\definecolor{currentstroke}{rgb}{1.000000,1.000000,1.000000}%
\pgfsetstrokecolor{currentstroke}%
\pgfsetdash{}{0pt}%
\pgfpathmoveto{\pgfqpoint{0.750000in}{2.200000in}}%
\pgfpathlineto{\pgfqpoint{5.400000in}{2.200000in}}%
\pgfusepath{}%
\end{pgfscope}%
\end{pgfpicture}%
\makeatother%
\endgroup%

\end{figure}