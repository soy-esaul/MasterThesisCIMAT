\section{Deterministic eigenvalue processes for matrix-valued diffusions}

Now that we have shown the construction of a matrix-valued process whose eigenvalue is the deterministic Dyson Brownian motion, we generaliza the result to get processes with a deterministic spectrum that can follow the dynamics of any eigenvalue process with the form \eqref{eq:gen_dyson}.

% \begin{theorem}

%     Let $Z$ be a process with covariation $\d Z_{ij}\d Z_{kl} = (\delta_{ik}\delta_{jl} + \delta_{il}\delta_{jk} - 2\delta_{ij}\delta_{kl}\delta_{ik} )\d t$ and no finite variation part, which means $Z$ is a symmetric matrix with independent Brownian motions in its entries, except for the diagonal, where $Z_{ii} = 0$. Let $X$ be a matrix valued process such that  $X = \trans H \Lambda H$ and it satisfies the stochastic differential equation
    
    
%     \begin{equation}
%         \trans{H}\d X(t)H = g(X(t)) \d Z(t) h(X(t)) + h(X(t)) \d \trans{Z(t)} g(X(t)) + b(X(t))\d t, \label{eq:deterministic_matrix_diffusion}
%     \end{equation}

%     Then the eigenvalue process $\Lambda$ satisfies

%     \begin{equation}
%         \d \lambda_i = \biggl( b(\lambda_i) + \sum_{k\neq i} \frac{G(\lambda_i,\lambda_k)}{\lambda_i - \lambda_k} \biggr)\d t.
%     \end{equation}

% \end{theorem}

\begin{theorem} \label{thm:deterministic_diffusion}
    Let $B = (B(t), t\ge 0)$ be a Brownian motion in $\M_{p,p}(\R)$ and $Y(t) = QM\trans{Q}$ be a symmetric $p\times p$ matrix-valued stochastic process satisfying the stochastic differential equation

    \begin{equation}
        \d Y(t) = g(Y(t)) \d B(t) h(Y(t)) + h(Y(t)) \d \trans{B(t)} g(Y(t)) + b(Y(t))\d t, \label{eq:matrix_diffusion}
    \end{equation}

    where $g,h,b$ are real functions acting spectrally, and $Y(0)$ is a symmetric $p\times p$ matrix with $p$ different eigenvalues.

    Let $G(x,y) = g^2(x)h^2(y) + g^2(y)h^2(x)$, $\tau$ be defined as in \eqref{eq:collision_time}, and take a process $X = (X(t), t\ge 0)$ with diagonalization $X = H \Lambda \trans{H}$ such that $\trans H (\d \Lambda) H$ has the same off-diagonal entries as $\trans Q ( \d M) \circ Q$ and has diagonal entries equal to zero.
    
    Then, for $t < \tau$ the eigenvalue process $\Lambda(t)$ verifies the following system of stochastic differential equations:

    \begin{equation}
        \d \lambda_i = \biggl( b(\lambda_i) + \sum_{k\neq i} \frac{G(\lambda_i,\lambda_k)}{\lambda_i - \lambda_k} \biggr)\d t.
    \end{equation}
\end{theorem}


\begin{proof}
    We define again $L$ to be the stochastic logarithm of $H$, $L \coloneqq \trans H \circ \d H$ and using the same techniques as in Theorem \ref{thm:diffusion_real} we have that,

    \begin{equation*}
        \d \Lambda = \trans H \circ(\partial X) \circ H - (\partial L)\circ \Lambda + \Lambda \circ \partial L.
    \end{equation*}

    Using that $\Lambda \circ \partial L -  (\partial L)\circ \Lambda $ has zero diagonal, we get that $\d lambda_i = (\trans H \circ(\partial X) \circ H)_{ii}$ and by hypothesis, we know that the martingale part of this diagonal is zero. 

    Define $\d N \coloneqq \trans H \circ(\partial X) \circ H$. For $i\neq j$ we have that $\d L_{ij} = \d N_{ij}/(\lambda_j - \lambda_i)$.

    Finally, we compute the finite variation $\d F$ part of $\d N$,

    \begin{align*}
        \d F &= \trans H b(X) H \d t + \frac12 (\d \trans H \d X H + \trans H \d X H), \\ 
        &= b(\Lambda) \d t + \frac12( \trans{(\d N\d A)} + \d N \d A ).
    \end{align*}

    For $\d N \d A$ we find

    \begin{align*}
        (\d N \d A)_{ij} &= \sum_{k\neq j} \d N_{ik}\d A_{kj} = \sum_{k\neq j} \frac{\d N_{ik}\d N_{kj}}{\lambda_j - \lambda_k}.
    \end{align*}

    Now we use that the martingale part of  $\d N$ has the same entries as $\trans{Q} M Q$ and by the results in Theorem \ref{thm:diffusion_real} we know that 

    \begin{equation*}
        (\trans{Q} M Q)_{ik}(\trans{Q} M Q)_{kj} = \delta_{ij}G(\lambda_i,\lambda_k)\d t,
    \end{equation*}

    \noindent so substituting the last result we get

    \begin{equation*}
        \d \lambda_i = \d F_{ii} = b(\lambda_i)\d t + \sum_{k\neq j} \frac{G(\lambda_i,\lambda_k)\d t}{\lambda_j - \lambda_k},
    \end{equation*}

    \noindent which is the desired result.



\end{proof}


These results can be particularized for any matrix-valued diffusions. Especially, we are interested in the Wishart and Jacobi processes. We give the proofs for these as a corollary to last Theorem.

\subsection{Wishart process}

\begin{corollary}
    Let $Y = (Y(t), t \ge 0)$ be an $n\times n$ Wishart process with parameter $m$ and diagonalization $Y = QM\trans{Q}$. Let $X$ be an $n\times n$ self adjoint matrix process with diagonalization $X = H \Lambda \trans{H}$, such that the off-diagonal part of $\trans{H} \d X H $ and $\overset{d}{=} \trans{Q} \d Y Q$ coincide, and $(\trans{H} \d X H)_{ii} = 0$ for every $i \in [n]$. Then the eigenvalues of $X$, $\lambda_1 \ge \lambda_2 \ge \cdots \ge \lambda_n$ satisfy the following system of stochastic differential equations

    \begin{equation*}
        \d \lambda_i = \left( m +\sum_{k\neq i} \frac{ \abs{\lambda_i} + \abs{\lambda_k} }{\lambda_i - \lambda_k}\right)\d t.
    \end{equation*}
\end{corollary}

\begin{proof}
    
\end{proof}


\subsection{Jacobi process}

\begin{corollary}
    Let $Y = (Y(t), t \ge 0)$ be an $n\times n$ matrix Jacobi process with parameters $n_1,n_2$ and diagonalization $Y = QM\trans{Q}$. Let $X$ be an $n\times n$ self adjoint matrix process with diagonalization $X = H \Lambda \trans{H}$, such that the off-diagonal part of $\trans{H} \d X H $ and $\overset{d}{=} \trans{Q} \d Y Q$ coincide, and $(\trans{H} \d X H)_{ii} = 0$ for every $i \in [n]$. Then the eigenvalues of $X$, $\lambda_1 \ge \lambda_2 \ge \cdots \ge \lambda_n$ satisfy the following system of stochastic differential equations

    \begin{equation} \label{eq:deterministic_jacobi}
        \d \lambda_i = \left( n_2 - (n_1 + n_2)\lambda_i + \sum_{k\neq i} \frac{\lambda_k(1-\lambda_i) + \lambda_i(1-\lambda_k)}{\lambda_i - \lambda_k} \right)\d t.
    \end{equation}

\end{corollary}

\section{Dynamical behavior of the deterministic eigenvalue processes}

\section{Simulations}