\section{Deterministic Dyson Brownian motion}

In this section we prove that a given matrix-valued stochastic process has a deterministic spectrum and follows the dynamics of the finite variation part in the Dyson Brownian motion. The proof uses the same techniques as the former results for the stochastic differential equations of eigenvalue processes. 

\begin{theorem}

Let $Z$ be a process with covariation $\d Z_{ij}\d Z_{kl} = (\delta_{ik}\delta_{jl} + \delta_{il}\delta_{jk} - 2\delta_{ij}\delta_{kl}\delta_{ik} )\d t$ and no finite variation part, which means $Z$ is a symmetric matrix with independent Brownian motions in its entries, except for the diagonal, where $Z_{ii} = 0$. Let $X$ be a matrix valued process such that  $X = \trans H \Lambda H$ and that satisfies the stochastic differential equation


\begin{equation*}
     \trans{H} \d X H = \d Z.
\end{equation*}

Then the eigenvalue process $\Lambda$ satisfies

\begin{equation} \label{eq:deterministic_dyson}
    \d \lambda_i = \sum_{k\neq i} \frac{\d t}{\lambda_i - \lambda_k}.
\end{equation}

\end{theorem}

\begin{proof}

Define $\d A = \trans{H}\partial H$ and $\d N = \trans{H} \partial Z H$.


The same procedure as in \ref{thm:diffusion_real} leads to 

\begin{equation*}
    \d \Lambda = \d N + \Lambda \d A - \d A \Lambda.
\end{equation*}

We conclude that $\d \lambda_i = \d N_{ii}$ and for $i\neq j$, 

\begin{align*}
    0 &= \d N_{ij} + \lambda_i \d A_{ij} - \lambda_j \d A_{ij},\\
    \Rightarrow \d A_{ij} &= \frac{\d N_{ij}}{\lambda_j - \lambda_i}.
\end{align*}

The quadratic covariation of $N$ is the same as the one for $Z$ because they only differ in a finite variation term, so

\begin{align*}
    \d N_{ij}\d N_{kl} &= \d \bigl< (\trans{H} \d X H)_{ij}, (\trans{H}\d X H)_{kl} \bigr>(t) %= \sum_{pqrs} \d \bigl< \trans{H}_{ip} \d X_{pq} H_{qj}, \trans{H}_{kr}\d X_{rs} H_{sl} \bigr>,\\
    % &= \sum_{pqrs} \trans{H}_{ip}  H_{qj}\trans{H}_{kr} H_{sl} \d X_{pq}\d X_{rs} = \sum_{pqrs} \trans{H}_{ip}  H_{qj}\trans{H}_{kr} H_{sl}\bigl( \delta_{pr}\delta_{qs} + \delta_{ps}\delta_{qr} - 2 \delta_{pq}\delta_{rs}\delta_{pr} \bigr)\d t,\\
    % &= \Biggl(\sum_{pqrs}\trans{H}_{ip}  H_{qj}\trans{H}_{kr} H_{sl}\delta_{pr}\delta_{qs} + \sum_{pqrs}\trans{H}_{ip}  H_{qj}\trans{H}_{kr} H_{sl}\delta_{ps}\delta_{qr} - 2 \sum_{pqrs}\trans{H}_{ip}  H_{qj}\trans{H}_{kr} H_{sl}\delta_{pq}\delta_{rs}\delta_{pr}\Biggr) \d t,\\
    % &= \sum_{pq} \trans{H}_{ip}H_{rk}\trans{H}_{jq}H_{ql}\d t + \sum_{pq}\trans{H}_{ip}H_{pl}\trans{H}_{kq}H_{qj} \d t - 2 \sum_{p} \trans{H}_{ip}H_{pj}\trans{H}_{kp}H_{pl} \d t,\\
    % &= \bigl(\delta_{ik}\delta_{jl} + \delta_{il}\delta_{kj} - 2\delta_{ij}\delta_{kl}\delta_{ik} \bigr)\d t.
    = \d \bigl< Z_{ij},Z_{kl} \bigr> = \bigl(\delta_{ik}\delta_{jl} + \delta_{il}\delta_{kj} - 2\delta_{ij}\delta_{kl}\delta_{ik} \bigr)\d t.
\end{align*}

Particularly, we have that $\d N_{ii}\d N_{jj} = 0$ for every choice of $j$ and $i$. Thus every entry in the diagonal of $N$ is a finite variation process and so it is $\lambda_i$. Let us finally compute the finite variation part $F$ of $N$.

\begin{align*}
    \d F &= \frac12 \bigl( \trans{H}\d X \d H + \d \trans{H}\d X H \bigr),\\
         &= \frac12 \bigl( \trans{H}\d X H \trans{H} \d H + \d \trans{H}H\trans{H}\d X H \bigr),\\
         &= \frac12 \bigl( \d N \d A + \trans{(\d N \d A)} \bigr).
\end{align*}

For $\d N \d A$ we have

\begin{align*}
    (\d N \d A)_{ij} &= \sum_{k\neq j} \d N_{ik}\d A_{kj} = \sum_{k\neq j} \frac{\d N_{ik}\d A_{kj}}{\lambda_j - \lambda_k},\\
    &= \sum_{k\neq j} \frac{ \delta_{ik}\delta_{kj} + \delta_{ij}\delta_{kk} - 2\delta_{ik}\delta_{kj}\delta_{ij} }{\lambda_j - \lambda_k}\d t = \delta_{ij}\sum_{k\neq j} \frac{ \d t }{\lambda_j - \lambda_k}.
\end{align*}

Then $F$ is diagonal with $\d F_{ii} = \sum_{k\neq i} \frac{\d t}{\lambda_i - \lambda_k}$. We conclude that

\begin{equation*}
    \d \lambda_i = \sum_{k\neq i} \frac{\d t}{\lambda_i - \lambda_k}.
\end{equation*}

\end{proof}