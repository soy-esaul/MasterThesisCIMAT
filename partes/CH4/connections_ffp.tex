\section{Connections with finite free probability}

In this section we relate finite free probability to eigenvalue processes. To do so, we first find the expected characteristic polynomial of the random matrices. We show that in some cases they are well-known polynomials whose convolutions satisfy good properties. We start with the case of a self-adjoint Brownian matrix and then get similar results for the Wishart and Jacobi processes.

Before proceeding with the processes, let us state a lemma that will help us to find the expected characteristic polynomials by using a tridiagonal model.

\begin{lemma} \label{lemma:chpol_tridiag}
    Let $A$ be an $n\times n$ tridiagonal symmetric matrix with diagonal elements $\{a_i\}_{1\le i \le n}$ and subdiagonal elements $\{ b_j \}_{2 \le j \le n}$, such that

    \begin{equation*}
        A = \begin{bmatrix}
            a_n   & b_{n-1} & 0     & \cdots & 0 \\ 
            b_{n-1} & a_{n-1}   & b_{n-2} & \cdots & 0 \\
            0     & b_{n-2} & a_{n-2}  & \cdots & 0 \\
            \vdots & \vdots & \vdots & \ddots & \vdots \\ 
            0      & \cdots & 0      & b_1 & a_1 
        \end{bmatrix}.
    \end{equation*}
    
    
    Then its characteristic polynomial $\chi^n_z(A) = \det(zI_n - A)$ satisfies the following recursion

    \begin{equation*}
        \chi^n_z(A) = (z-a_n)\chi^{n-1}_z(A) - b_{n-1}^2\chi^{n-2}_z(A), \label{eq:recursion}
    \end{equation*}

    \noindent where $\chi_z^{n-1}(A)$ represents the characteristic polynomial of the $(n-1)\times(n-1)$ lower-right block of $A$. Notice that, in particular, this block is also tridiagonal.
\end{lemma}

\begin{proof}
    Let us write

    \begin{equation*}
        A = \begin{bmatrix}
            a_n   & b_{n-1} \trans{e_1} \\
            b_{n-1}e_1 & B 
        \end{bmatrix},
    \end{equation*}

    \noindent with $e_1$ the first element in the canonical base of $\R^{n-1}$, and write $B$ as

    \begin{equation*}
        B = \begin{bmatrix}
            a_{n-1}   & b_{n-2} \trans{{e_1^{(n-2)}}} \\
            b_{n-2} e_1^{(n-2)} & C 
        \end{bmatrix}.
    \end{equation*}

    The determinant of $zI_n - A$ is 

    \begin{align*}
        \det \left( zI_n - A \right) &= (zI_n - A)_{11}\det(zI_{n-1}-B) - (zI_n - A)_{12}(xI_n - A)_{21}\det(xI_n - C)\\ 
        &= (z - a_n)\chi^{n-1}_z(A) - b_{n-1}^2\chi^{n-2}_z(A).
    \end{align*}
\end{proof}

With this lemma and the tools developed in the former chapters, we are ready to study the expected characteristic polynomial of some matrix-valued stochastic processes. In the next subsection, we start with the Dyson Brownian motion.

\subsection{Dyson Brownian motion}

In Chapter \ref{ch:finite_free}  we saw that the tools devloped by Marcus and collaborators \cite{article:finitefree,article:arizmendi_perales,article:finitefree} allow us to express expected characteristic polynomials of sums and products of random matrices as convolutions of the original polynomials. However, we still did not see any tool useful for finding such polynomials when we can not express them in terms of another characteristic polynomial. This topic has been covered in the literature several times \cite{book:percy_deift_orthogonal,edelman1988eigenvalues,article:aomoto1987jacobi_selberg_integrals} but we will use the approach by Ioanna Dumitriu and Alan Edelman found in \cite{article:dumitriu_edelman} because it requires to introduce less technical concepts. Once we find the expected characteristic polynomials by these means, we will relate them to Finite Free Probability Theory using the previous developed tools.

As a first step, we will make use of the invariance under orthogonal (or unitary) transformations of the GOE (or GUE) in order to find a tridiagonal model with the same matrix distribution.


\begin{lemma} \label{lemma:tridiag}
    Let $A_{\beta}$ denote an $n\times n$ matrix from the GOE, GUE or GSE, for $\beta = 1,2,4$, respectively. Then the eigenvalues of the tridiagonal matrix $H_\beta$ have the same joint law as the eigenvalues of $A_\beta$, with $H_\beta$ defined as

    \begin{align} \label{eq:tridiag_hermite}
        H_\beta &= \frac1{\sqrt2}\begin{bmatrix}
            N_1   & \xi_2 & 0     & \cdots & 0 \\ 
            \xi_2 & N_2   & \xi_3 & \cdots & 0 \\
            0     & \xi_3 & N_3  & \cdots & 0 \\
            \vdots & \vdots & \vdots & \ddots & \vdots \\ 
            0      & \cdots & 0      & \xi_{n} & N_n 
        \end{bmatrix}.
    \end{align}

    The diagonal entries $N_i, 0 \le i \le n$ are independent normal random variables with mean 0 and variance 2, while the subdiagonal entries $\xi_i$ are independent random variables distributed as

    \[ \xi_i \sim \chi_{\beta(n+1-i)}. \]

    Note that in this case $\chi_\nu$ denotes the chi distribution which is the squared root of a chi-squared random variable or the absolute value of a normal random variable in the case $\nu$ is integer.

    We call $H_\beta$ the tridiagonal $\beta$-Hermite ensemble.
\end{lemma}

\begin{proof}
    Write $A_{\beta}$ as 

    \begin{align*}
        A_{\beta} &= \begin{bmatrix}
            N_1 & \trans{\vec x} \\ 
            \vec x & B_{\beta}
        \end{bmatrix},
    \end{align*}

    \noindent with $N_1$ a normal random variable, $\vec x$ an $n-1$-dimensional gaussian vector with independent entries in $\R, \C$ or $\H$, depending on $\beta$, and $B_{\beta}$ an $(n-1)\times(n-1)$ matrix from the GOE, GUE or GSE, respectively. All of the elements are independent from each other.

    Now we take $H$ to be any $(n-1)\times(n-1)$ orthogonal (or unitary, symplectic, according to $\beta$) matrix such that $H \trans{\vec x} = \norm{\vec x}_2 e_1$, where $e_1 = (1,0,\cdots,0)$ is the first element in the canonical basis of $\R^{n-1}$. Then we have

    \begin{align*}
        \begin{bmatrix}
            1 & \trans{\vec0} \\ 
            \vec0 & H
        \end{bmatrix} 
        \begin{bmatrix}
            N_1 & \trans{\vec x} \\ 
            \vec x & B_{\beta}
        \end{bmatrix}  
        \begin{bmatrix}
            1 & \trans{\vec 0} \\ 
            \vec 0 & \trans H
        \end{bmatrix} &= \begin{bmatrix}
            N_1 & \trans{\vec x} \\ 
            \norm{\vec x}_2e_1 & H B_{\beta}
        \end{bmatrix}  
        \begin{bmatrix}
            1 & \trans{\vec0} \\ 
            \vec0 & \trans H
        \end{bmatrix} 
        =
        \begin{bmatrix}
            N_1 & \norm{x}_2\trans{e_1} \\ 
            \norm{x}_2 e_1 & H B_\beta \trans{H}
        \end{bmatrix}.
    \end{align*}

    Now we can find the distribution of each of the blocks of the new matrix. The variable $N_1$ has not changed, it is a standard normal variable. The term $\norm{\vec x}_2$ is the norm of a Gaussian vector of length $n-1$ with real (complex or quaternionic) entries, non-correlated and with variance $1/2$, so it is distributed like a $\frac1{\sqrt{2}}\chi_{\beta(n-1)}$ random variable, where $\beta$ indicates the number of normal variables in each entry of the matrix. Since $B_\beta$ is a GOE (GUE, GSE), it is invariant under orthogonal (unitary, symplectic) transformations and thus $H B_{\beta} \trans{H}$ is a GOE (GUE, GSE).

    The matrix $ \begin{bmatrix}
        1 & \trans{\vec0} \\ 
        \vec0 & H
    \end{bmatrix}$ is orthogonal (unitary, symplectic), so the eigenvalue distribution of $A_\beta$ remains unchanged under this transformation. By repeating the procedure with $B_\beta$, we find the tridiagonal matrix \eqref{eq:tridiag_hermite}, which finishes the proof.
\end{proof}


Since $H_\beta$ has the same distribution as an invariant Gaussian ensemble, the expected characteristic polynomials are the same, and we can find them using that $H_\beta$ is tridiagonal and Lemma \ref{lemma:chpol_tridiag}.



\begin{theorem}
    Let $X$ be an $n\times n$ GUE, then its expected characteristic polynomial, $P_n(z)$ is the $n$th Hermite polynomial.
\end{theorem}

\begin{proof}
    Let $A_2$ be a tridiagonalization of $X$. Due to Lemma \ref{lemma:tridiag} we know that the eigenvalues of $A_2$ have the same joint distribution of those of $X$, so the expected characteristic polynomial must coincide. We can use Lemma \ref{lemma:chpol_tridiag} to find the expected characteristic polynomial of $A$. Let $Q_n$ denote its characteristic polynomial, then

    \begin{align*}
        P_n(z) &= \mathbb E\left( \left(z - \frac{N_1}{\sqrt2}\right)Q_{n-1}(z) - \frac{\xi_{2(n-1)}^2}{2}Q_{n-2}(z) \right),\\ 
               &= zP_{n-1}(z) - (n-1)P_{n-2}(z).
    \end{align*}

    So $P_n(z)$ satisfies the recursion that determines the Hermite polynomials. We need to check the initial conditions $P_1(z)$ and $P_2(z)$. For $P_1(z)$ the condition is trivial

    \begin{equation*}
        P_1(z) = \mathbb E \left( z - \frac{N_1}{\sqrt{2}} \right) = z.
    \end{equation*}

    For $P_2(z)$, we have

    \begin{align*}
        P_2(z) &= \mathbb E \left( \det \begin{bmatrix}
        z - N_1 & \xi_{2} \\ 
        \xi_2 & z - N_2
        \end{bmatrix} \right) = \mathbb E \left[ \left(z - \frac{N_1}{\sqrt{2}}\right)^2 - \frac{\xi_2^2}2 \right],\\ 
        &= z^2 - 1.
    \end{align*}

    So $P_1(z) = H_1(z)$ and $P_2(z) = H_2(z)$. Using the recursion, we can conclude that $P_n(z) = H_n(z)$.
\end{proof}

Recall that a self adjoint Brownian motion has the same law as $\sqrt{t}$ times a Gaussian invariante ensemble. Once again, we use an equality in distribution and Lemma \ref{lemma:chpol_tridiag} in order to find the expected characteristic polynomial of a self adjoint Brownian motion.

\begin{corollary} \label{corollary:brownian_expected}
    Let $B(t)$ be an $n\times n$ self-adjoint complex Brownian matrix, then its expected characteristic polynomial $P_n(z,t)$ is the $n$th generalized Hermite polynomial with variance $t$, $H_n(z,t)$, i.e. the Hermite polynomials which are orthogonal with respect to a Gaussian random variable of variance $t$.
\end{corollary}

\begin{proof}

    We will prove the result by showing that $P_n(x,t)$ satisfies the following recursion
    
    \begin{align*}
        P_n(z,t) &= x P_{n-1}(z,t) - t(n-1)P_{n-2}(z,t),
    \end{align*}

    \noindent with initial conditions $P_1(z,t) = z$ and $P_2(z,t) = z^2-t$.

    For any given $t$, $B(t)$ has the same law as $\sqrt{t}A_2$ with $A_2$ a GUE. By Lemma \ref{lemma:tridiag}, the expected characteristic polynomial is the same as that of $\sqrt{t}H_2$ with $H_2$ a $2$-Hermite polynomial. Let $Q_n(x,t)$ be the characteristic polyomial of $\sqrt{t}H_2$, applying Lemma \ref{lemma:chpol_tridiag} we have

    \begin{align*}
        P_n(x,t) &= \mathbb E \left(  \left( x - \sqrt{\frac t2}N_1 \right) Q_{n-1}(x,t) - \frac t2 \xi_{2(n-1)}^2 Q_{n-2}(x,t)\right), \\ 
        &= x P_{n-1}(x,t) - t(n-1)P_{n-2}(x,t).
    \end{align*}

    Now we find the first two polynomials,

    \begin{align*}
        P_1(x,t) &= \mathbb E \left( x - \sqrt{\frac t2}N_1 \right) = x,\\
        P_2(x,t) &= \mathbb E \left( \det\begin{bmatrix}
            x - \sqrt{\frac t2}N_1 & \sqrt{\frac t2} \xi_2 \\ 
            \sqrt{\frac t2} \xi_2  & x - \sqrt{\frac t2}N_2                         
        \end{bmatrix} \right),\\ 
        &= \mathbb E \left( \left(x - \sqrt{\frac t2}N_1\right)\left( x - \sqrt{\frac t2}N_2 \right) - \frac t2 \xi_2^2 \right),\\ 
        &= x^2 - t.
    \end{align*}

    With this, we have found a recursion and initial conditions that are enough to uniquely determine the expected characteristic polynomial of a self-adjoint complex Brownian matrix.
\end{proof}

    Now that we know that the expected characteristic polynomial of a self adjoint Brownian motion is a Hermite polynomial with variance $t$, we are ready to give a much easier proof of Proposition \ref{prop:convolution_of_hermites} using the tools of Finite Free Probability Theory.

    \begin{proof}[Alternate proof of Proposition \ref{prop:convolution_of_hermites}]
        Let $H_n(z,t_1)$ and $H_n(z,t_2)$ be the expected characteristic polynomials of two independent self-adjoints Brownian motions $B_1 = (B_1(t), t\ge 0)$ and  $B_2 = (B_2 (t), t\ge0)$ evaluated at the times $t_1$ and $t_2$, respectively. By the Lévy property of the Brownian motion, we know that $B \coloneqq B_1 + B_2$ is another self-adjoint Brownian motion, evaluated at the time $t_1 + t_2$. By Theorem \ref{thm:symmad}, and using the invariance under unitary transformations of $B_1, B_2$, the convolution of the Hermite polynomials is the expected characteristic polynomial of $B$, and Corollary \ref{corollary:brownian_expected} allows us to conclude that this polynomial is $H_n(z,t_1 + t_2)$.
    \end{proof}

    The Hermite polynomials not only are the expected characteristic polynomials for our Gaussian invariant ensembles, they somehow also carry all the ``non-stochastic information'' of the eigenvalues in the sense that their roots satisfy \eqref{eq:deterministic_dyson}. We will prove this by first showing that the Hermite polynomials are ``harmonic'' in some sense, i.e., they are a solution for the martingale problem of a Brownian motion in reversed time. \todo{Revisar que esto sea cierto}

\begin{theorem} \label{thm:heat_burgers}
    The Hermite polynomials $H_n(z,t)$ solve the differential equation

    \begin{equation} \label{eq:calor}
      \partial_t [H_n(z,t)] + \frac12 \partial_{zz}[H_n(z,t)] = 0,
    \end{equation}

    \noindent and the Cauchy transform $G_{H_n}(y)$ of the empirical measure associated to its roots $\left\{ z_j(t) \right\}_{j\in [n]}$, 
    
    \begin{equation*}
        G_{H_n}(z,t) \coloneqq \frac1n \sum_{j=1}^n \frac{1}{z_j(t) - z},
    \end{equation*}

    \noindent solves the viscous Burgers equation with diffusion coefficient $-1/2$,

    \begin{equation*} \label{eq:burgers_12}
        {\partial_t}[\partial G_{H_n}(z,t)] + nG_{H_n}(z,t)\partial_z[G_{H_n}(z,t)] = -\frac12  \partial_{zz}[G_{H_n}(z,t)].
    \end{equation*}
    
\end{theorem}

\begin{proof}

    Lets us write $H_n(z,t)= H_n$, $G_{H_n}(z,t) = G$ and $\frac{\partial f}{\partial z} = \partial_z f$ to simplify the notation.

    First, we prove that $H$ satisfies \eqref{eq:calor}. Write $H_n(z,t) = \exp\left\{-\frac{t \partial_z^2}{2}\right\}(z^n)$, and

    \begin{align*}
        \partial_t H_n &= \partial_t \exp\left\{-\frac{t \partial_z^2}{2}\right\}(z^n) 
        = \partial_t \sum_{j=0}^{\lfloor n/2 \rfloor} (-1)^j \frac{t^j \partial_z^{2j}(z^n) }{2^j j!},\\ 
        &= \sum_{j=0}^{\lfloor n/2 \rfloor} (-1)^j \frac{\partial_t (t^j) \partial_z^{2j}(z^n) }{2^j j!} = \sum_{j=0}^{\lfloor n/2 \rfloor} (-1)^j \frac{j(t^{j-1}) \partial_z^{2j}(z^n) }{2^j j!},\\ 
        &= \sum_{j=1}^{\lfloor n/2 \rfloor} (-1)^j \frac{j(t^{j-1}) \partial_z^{2j}(z^n) }{2^j j!}= -\frac12\sum_{j=1}^{\lfloor n/2 \rfloor} (-1)^{j-1} \frac{(t^{j-1}) \partial_z^{2(j-1)}\partial_{zz}(z^n) }{2^{j-1} (j-1)!}, \\ 
        &= -\frac12\partial_{zz} \sum_{k=0}^{\lfloor n/2 - 1\rfloor} (-1)^{k} \frac{(t^{k}) \partial_z^{2k}(z^n) }{2^{k} k!} = - \frac12 \partial_{zz} H_n.
    \end{align*}

    
    
    For the Cauchy transform part, we recall from Lemma \ref{lemma:cauchy_empirical_polynomial} that since $H_n$ is monic on $z$, then  $G= \frac{\partial_z H_n}{nH_n}$.

    Now, let us show that $G$ satisfies \eqref{eq:burgers_12}, for this we compute $\partial_t G$, $G\partial_z G$ and $\partial_{zz}G$,

    \begin{align*}
        \partial_t G &= \frac1n\partial_t \left( \frac{\partial_z H_n}{H_n} \right) = \frac1n\frac{ H_n \partial_t \partial_z H_n - \partial_z H_n \partial_t H_n }{H_n^2},\\ 
        \partial_z G &= \frac1n\partial_z \left( \frac{\partial_z H_n}{H_n} \right) = \frac1n\frac{H_n \partial_{zz}H_n - (\partial_z H_n)^2}{H_n^2},\\ 
        \partial_{zz} G &= \frac1n\partial_z \left( \frac{H_n \partial_{zz}H_n - (\partial_z H_n)^2}{H_n^2} \right), \\ 
        &= \frac1n \frac{ H_n^2\partial_z H_n \partial_{zz}H_n + H_n^3\partial_{zzz}H_n - 2H_n^2\partial_z H_n \partial_{zz}H_n }{H_n^4} \\
        & \phantom{separadoasídem}-\frac1n\frac{2H_n^2\partial_z H_n \partial_{zz}H_n - 2 H_n (\partial_z H_n)^3}{H_n^4},\\
        &= \frac1n\frac{-3H_n\partial_z H_n \partial_{zz}H_n + H_n^2 \partial_{zzz}H_n + 2(\partial_z H_n)^3}{H_n^3} \\
        G\partial_z G &= \frac{\partial_z H_n}{nH_n}\left( \frac{H_n \partial_{zz}H_n - (\partial_z H_n)^2}{H_n^2}  \right) = \frac{1}{n^2}\frac{H_n\partial_z H_n \partial_{zz}H_n - (\partial_z H_n)^3}{H_n^3}.
    \end{align*}

    Finally, we can use the above results to find

    \begin{align*}
        \partial_t G + n G\partial_z G &= \frac1n\left(\frac{ H_n \partial_t \partial_z H_n - \partial_z H_n \partial_t H_n }{H_n^2} + \frac{H_n\partial_z H_n \partial_{zz}H_n - (\partial_z H_n)^3}{H_n^3} \right),\\ 
        &= \frac1n\left( \frac{ -\frac12H_n^2 \partial_{zzz} H_n + \frac12H_n\partial_z H_n \partial_{zz} H_n + H_n\partial_z H_n \partial_{zz}H_n - (\partial_z H_n)^3}{H_n^3} \right), \\ 
        &= \frac1n\left( \frac{\frac32 H_n^2 \partial_{z}H_n\partial_{zz}H_n - \frac12H_n^2\partial_{zzz}H_n - (\partial_z H_n)^3}{H_n^3} \right) = -\frac12 \partial_{zz}G.
    \end{align*}
    % \begin{align*}
    %     \partial_t s &= \partial_t \left( \frac{-\partial_z P}{P} \right) = \frac{ \partial_z P \partial_t P - P \partial_t \partial_z P }{P^2},\\ 
    %     \partial_z s &= \partial_z \left( \frac{-\partial_z P}{P} \right) = \frac{(\partial_z P)^2 - P \partial_{zz}P}{P^2},\\ 
    %     \partial_{zz} s &= \partial_z \left( \frac{(\partial_z P)^2 - P \partial_{zz}P}{P^2} \right), \\ 
    %     &= \frac{\partial_z(\partial_z P)^2 P^2 - 2P(\partial_z P)^2\partial_{z}P}{P^4} - \frac{ P^2\partial_z P \partial_{zz}P + P^3\partial_{zzz}P - 2P^2(\partial_z P)\partial_{zz}P }{P^4}, \\ 
    %     &= \frac{ 2P^2\partial_z P \partial_{zz}P - 2P(\partial_z P)^2\partial_{z}P}{P^4} + \frac{ P^2\partial_z P \partial_{zz}P - P^3\partial_{zzz}P}{P^4}, \\
    %     &= \frac{ 3P\partial_z P \partial_{zz}P - 2(\partial_z P)^3 - P^2\partial_{zzz}P}{P^3}, \\ 
    %     %&=\frac{ \partial_{zz}P\partial_z P + \partial_z (\partial_z P)^2 - P \partial_{zzz}P }{P^2} - 2\frac{(\partial_z P)^3}{P^2},
    %     s\partial_z s &= -\frac{\partial_z P}{P} \left( \frac{(\partial_z P)^2 - P \partial_{zz}P}{P^2} \right) = \frac{P \partial_z P \partial_{zz}P - (\partial_z P)^3}{P^3}
    % \end{align*}
\end{proof}

Using Theorem \ref{thm:heat_burgers} its is easy to prove that the roots of the time dependent Hermite polynomials evolve according to \eqref{eq:deterministic_dyson}.

\begin{corollary}
    The roots $\left\{z_i(t)\right\}_{i\le n}$ of the Hermite polynomials $H_n(z,t)$ satisfy the deterministic Dyson's equation,

    \begin{equation*}
        \d z_i = \sum_{k\neq i} \frac{\d t}{z_{i} - z_k}.
    \end{equation*}
\end{corollary}

\begin{proof}
    Let $z_i(t)$ be the roots of $H_n(z,t)$, this means

    \begin{align*}
        H_n(z_i(t),t) &= 0, \qquad \Rightarrow \partial_t [H_n(z_i(t),t)] = 0.
    \end{align*}

    By the chain rule and the heat kind equation \eqref{eq:calor} in Theorem \ref{thm:heat_burgers} we have that

    \begin{align*}
        0 &= \partial_t [H_n(z_i(t),t)] = (\partial_t [z_i])\partial_z [H_n(z_i(t),t)] - \frac12 \partial_{zz} [H_n(z_i(t),t)],
        \intertext{using the Leibniz rule}
        \frac{\d}{\d t} z_i(t) &= \frac{\partial_{zz} [H_n(z_i(t),t)]}{2\partial_z [H_n(z_i(t),t)]} = \frac{2 \sum_{k\neq i} \prod_{j\neq i, j\neq k} (z_i - z_j)}{2 \prod_{j\neq i} (z_i - z_j)} = \sum_{k\neq i} \frac{1}{z_i - z_k}.
    \end{align*}
\end{proof}

As a final part for this subsection, let us prove that if you have an arbitrary (real valued) initial condition for a self adjoint Brownian motion, then its expected characteristic polynomial will also satisfy \eqref{eq:calor}. Notice that as a consequence, this means that its roots will move according to \eqref{eq:deterministic_dyson}.

\begin{theorem}
    Let $A$ be an $n\times n$ fixed self-adjoint matrix and $W$ an $n\times n$ self-adjoint Brownian matrix, then $q_A(z,t)$ defined as
    
    \begin{equation*}
        q_A(z,t) \coloneqq \E{ \chi_z( A + W) },
    \end{equation*}
    
    \noindent satisfies the following differential equation
    
    \begin{equation*}
        \partial_t [q_A(z,t)] + \frac12 \partial_{zz} [q_A(z,t)] = 0.
    \end{equation*}
    \end{theorem}
    
    \begin{proof}
        
        Let $p(z) = \E{\chi_z(A)}$ and $r(z,t) = \E{\chi_z(W)}$. Corollary \ref{corollary:brownian_expected} tells us that $r(z,t)$ is the $n$th Hermite polynomial $H_n(z,t)$. Write these polynomials as

        \begin{align*}
            p(z)     &= \sum_{j=0}^n a_j z^j,\\ 
            H_d(z,t) &= \sum_{j=0}^n b_j t^{j/2} z^{d-j}.
        \end{align*}
        
        Notice that if $n$ is odd all of the $b_j$ are zero for even $j$  and if $n$ is even, all of the $b_j$ are zero for odd $j$. Further, using the explicit expression for the coefficients we have the following recursion for $b_j$

        \begin{align}
            b_{j} &= \frac{n! (-1)^{j/2}}{2^{j/2} (j/2)! (n-j)!},\\ 
            b_{j-2} &= \frac{n! (-1)^{\frac{j-2}{2}}}{2^{\frac{j-2}{2}} \left(\frac{j-2}{2}\right)! (n-j+2)!},\\ 
            \Rightarrow b_{j} &= \frac{n! (-1)^{\frac{j-2}{2}}}{2^{\frac{j-2}{2}} \left(\frac{j-2}{2}\right)! (n-j+2)!} \frac{(-1)(n-j+2)(n-j+1)}{j},\\  
            &= b_{j-2} \frac{(-1)(n-j+2)(n-j+1)}{j}. \label{eq:hermite_coef_recurr}
        \end{align}

        Using the invariance of $W$ under unitary transforms and Theorem \ref{thm:symmad} we have that $q_A(z,t) = H_n(z,t) \boxplus_n p(z)$. By definition, 

        \begin{equation*}
            q_A(z,t) = H_n(z,t) \boxplus_n p(z) \coloneqq \sum_{k=0}^n z^{n-k}(-1)^k \sum_{i=0}^k c_{k,i,n} b_i a_{k-i} t^{i/2},
        \end{equation*} 

        \noindent with $c_{k,i,n} = \frac{(n-i)!(n-k+i)!}{n!(n-k)!}$.

        % In order to prove the differential equation, we compute the derivatives of each power. The coefficient of $z^{d-k}$ is $e_k = (-1)^k \sum_{i=0}^k c_{k,i,d} b_i a_{k-i} t^{i/2}$, then

        % \begin{align*}
        %     \partial_{zz} e_k z^{d-k} &= (d-k)(d-k-1)e_k z^{d-k-2}.
        % \end{align*}

        % So $\partial_{zz} q_A(z,t)$ is a polynomial of order $d-k-2$ with 
        
        % \begin{equation*}
        %     \partial_{zz} q_A(z,t) = \sum_{k=0}^{d-2} z^{d-2-k}(d-k)(d-k-1)e_k = \sum_{k=0}^{d-2} z^{d-2-k}(d-k)(d-k-1) (-1)^k \sum_{i=0}^k c_{k,i,d} b_i a_{k-i} t^{i/2}.
        % \end{equation*}

        % Now for the derivative with respect to $t$,

        % \begin{align*}
        %     \partial_t e_k z^{d-k} &= \partial_t \left((-1)^k \sum_{i=0}^k c_{k,i,d} b_i a_{k-i} t^{i/2} z^{d-k}\right) = (-1)^k \sum_{i=0}^k c_{k,i,d} b_i a_{k-i} \frac{it^{(i-2)/2}}{2}z^{d-k}.
        % \end{align*}

        We compute first $\partial_{zz} [q_A(z,t)]$,

        \begin{align*}
            \partial_{zz} [q_A(z,t)] %&= \partial_{zz} \left[ \sum_{k=0}^d z^{d-k}(-1)^k \sum_{i=0}^k c_{k,i,d} b_i a_{k-i} t^{i/2} \right],\\ 
            &= \partial_{zz} \left[ \sum_{k=0}^n z^{n-k}(-1)^k \sum_{i=0}^k \frac{(n-i)!(n-k+i)!}{n!(n-k)!} b_i a_{k-i} t^{i/2} \right],\\
            &= \sum_{k=0}^{n-2} (n-k)(n-k-1)z^{n-k-2}(-1)^k \sum_{i=0}^k \frac{(n-i)!(n-k+i)!}{n!(n-k)!} b_i a_{k-i} t^{i/2}, \\
            &= \sum_{k=0}^{n-2} z^{n-k-2}(-1)^k \sum_{i=0}^k \frac{(n-i)!(n-k+i)!}{n!(n-k-2)!} b_i a_{k-i} t^{i/2}, \\
            &= \sum_{k=2}^{n} z^{n-k}(-1)^k \sum_{i=0}^{k-2} \frac{(n-i)!(n-(k-2)+i)!}{n!(n-(k-2)-2)!} b_i a_{k-2-i}t^{i/2},\\ 
            &= \sum_{k=2}^{n} z^{n-k}(-1)^k \sum_{i=0}^{k-2} \frac{(n-i)!(n-k+i+2)!}{n!(n-k)!} b_i a_{k-2-i}t^{i/2}.
        \end{align*}

        Now the derivative with respect to $t$ is

        % \begin{align*}
        %     \partial_t q_A(z,t) &= \partial_t \left[ \sum_{k=0}^d z^{d-k}(-1)^k \sum_{i=0}^k c_{k,i,d} b_i a_{k-i} t^{i/2} \right],\\ 
        %     &= \sum_{k=0}^d z^{d-k}(-1)^k \sum_{i=0}^k \frac{(d-i)!(d-k+i)!}{d!(d-k)!} b_i a_{k-i} \frac{i}{2}t^{\frac{i-2}{2}},\\ 
        %     &= \sum_{k=0}^d z^{d-k}(-1)^k \sum_{j=0}^{k-2} \frac{(d-j-2)!(d-k+j+2)!}{d!(d-k)!} b_{j+2}a_{k-j-2}\left( \frac{j+2}{2} \right)t^{j/2},\\ 
        %     &= \sum_{k=0}^d z^{d-k}(-1)^k \sum_{j=0}^{k-2} \frac{(d-j)!(d-k+j+2)!}{d!(d-k)!} b_ja_{k-j-2}\frac{(j+2)t^{j/2}}2,\\
        %     \intertext{using \eqref{eq:hermite_coef_recurr} for $b_{j+2}$} 
        %     &= \frac12 \sum_{k=0}^d z^{d-k}(-1)^{k+1} \sum_{j=0}^{k-2} \frac{(d-j)!(d-k+j+2)!}{d!(d-k)!} b_ja_{k-j-2}t^{j/2}, \\ 
        %     &= -\frac12 \sum_{k=2}^d z^{d-k}(-1)^{k} \sum_{j=0}^{k-2} \frac{(d-j)!(d-k+j+2)!}{d!(d-k)!} b_ja_{k-j-2}t^{j/2} = -\frac12 \partial_{zz} q_A(z,t).
        % \end{align*}

        \begin{align*}
            \partial_t [q_A(z,t)] &= \partial_t \left[ \sum_{k=0}^n z^{n-k}(-1)^k \sum_{i=0}^k c_{k,i,n} b_i a_{k-i} t^{i/2} \right],\\ 
            &= \sum_{k=0}^n z^{n-k}(-1)^k \sum_{i=0}^k \frac{(n-i)!(n-k+i)!}{n!(n-k)!} b_i a_{k-i} \frac{i}{2}t^{\frac{i-2}{2}},\\ 
            &= \sum_{k=0}^n z^{n-k}(-1)^k \sum_{j=0}^{k-2} \frac{(n-j-2)!(n-k+j+2)!}{n!(n-k)!} b_{j+2}a_{k-j-2}\left( \frac{j+2}{2} \right)t^{j/2},\\ 
            &= \sum_{k=0}^n z^{n-k}(-1)^k \sum_{j=0}^{k-2} \frac{(n-j)!(n-k+j+2)!}{n!(n-k)!} b_ja_{k-j-2}\frac{(j+2)t^{j/2}}2,\\
            \intertext{using \eqref{eq:hermite_coef_recurr} for $b_{j+2}$} 
            &= \frac12 \sum_{k=0}^n z^{n-k}(-1)^{k+1} \sum_{j=0}^{k-2} \frac{(n-j)!(n-k+j+2)!}{n!(n-k)!} b_ja_{k-j-2}t^{j/2}, \\ 
            &= -\frac12 \sum_{k=2}^n z^{n-k}(-1)^{k} \sum_{j=0}^{k-2} \frac{(n-j)!(n-k+j+2)!}{n!(n-k)!} b_ja_{k-j-2}t^{j/2} = -\frac12 \partial_{zz} q_A(z,t).
        \end{align*}
    \end{proof}
    
    We conclude with this theorem that no matter the initial condition of the self-adjoint Brownian motion, the roots of its expected characteristic polynomial will mode according to the deterministic Dyson equation. In the next subsection we get similar results for the Wishart process.

\subsection{Wishart process} \label{subsec:wishart_laguerre}

    Similarly to the Brownian motion case, we will first construct a tridiagonal model that will help us to get the expected characteristic polynomial.

\begin{lemma}
    Let $W_\beta$ be an $n\times n$ matrix from the  $\beta$-Wishart ensemble ($\beta \in \{1,2,4\}$). The eigenvalues of $W_\beta$ have the same joint law as those of the tridiagonal matrix $L_\beta = \trans{B_\beta}  B_\beta$ with $B_\beta$ is an $m\times n$ bidiagonal matrix defined as 

    \begin{align} \label{eq:tridiag_laguerre}
        B_\beta &= \begin{bmatrix}
            \xi_{n\beta}   & 0 & 0     & \cdots & 0 \\ 
            \xi_{\beta(m-1)} & \xi_{n\beta - \beta}   & 0 & \cdots & 0 \\
            % 0     & \ddots & \ddots  & \cdots & 0 \\
            \vdots & \vdots & \vdots & \ddots & \vdots \\ 
            0      & \cdots & 0      & \xi_{\beta} & \xi_{n\beta - \beta(m-1)} 
        \end{bmatrix}.
    \end{align}

    The variables $\xi_j$ being independent random variables with distribution $\xi_j \sim \chi_j$. We call $L_\beta$ the tridiagonal $\beta$-Laguerre ensemble.
\end{lemma}

\begin{proof}
    Take $G$ to be an $m\times n$ matrix with independent standard $\beta$-Gaussian random variables as entries (real if $\beta=1$, complex if $\beta=2$ and quaternions if $\beta=4$). Write $G$ as 

    \begin{equation*}
        \begin{bmatrix}
            \trans{\vec x}\\
            G_1
        \end{bmatrix}.
    \end{equation*}

    We have that $\vec x$ is a vector distributed as a multivariate $\beta$-normal variable with mean $\vec 0$ and covariance matrix $\Sigma = I$, while $G_1$ is an $(m-1)\times n$ matrix of independent standard $\beta$-Gaussian random variables. 

    Take $R$ to be a ``right reflector'' corresponding to $\vec x$ independent of $G_1$, which means $\trans{\vec x}R = \norm{\vec x}_2 e_1$. Due to being a reflector, $R$ is orthogonal (unitary, simplectic) and this means that $G_1 R$ is an $(m-1)\times n$ matrix with independent standard $\beta$-Gaussian matrices as entries.

    Now take $G_1R = [\vec y, G_2]$ with $\vec y$ being a $\beta$-Gaussian vector with mean $\vec 0$ and covariance matrix $\Sigma = I$. Then $G_2$ is an $(m-1)\times (n_1)$ matrix of independent standard $\beta$-Gaussian random variables. Let $L$ be a left reflector corresponding to $\vec y$ ($Ly = \norm{\vec y}_2 e_1$) independent of $G_2$. Again by the orthogonality (unitarity, simplecticity) of $L$, $LG_2$ is still an $(m-1)\times(n-1)$ matrix of independent standard $\beta$-Gaussian random variables. This means that

    \begin{equation*}
        \begin{bmatrix}
            1 & 0 \\
            0 & L
        \end{bmatrix} G R = \begin{bmatrix}
            \norm{\vec x}_2 & 0 \\
            \norm{\vec y}_2 e_1 & LG_2
        \end{bmatrix}.
    \end{equation*}

    We proceed with this procedure now for $LG_2$. The product by an orthogonal (unitary, simplectic) matrix does not affect the singular values of a matrix, so the singular values of the bidiagonal matrix $B_\beta$ and the original matrix $G$ are the same. The eigenvalues of $W = G\trans{G}$ are the squares of the singular values of $G$ and the same happens with the eigenvalues of $L_\beta = B_\beta\trans{B_\beta}$ which is a tridiagonal matrix.

    The distribution of the entries follows from the definition of the complex and simplectic normal distributions as sums of real normal random variables.
\end{proof}

    Another use of Lemma \ref{lemma:chpol_tridiag} together with the last Lemma help us to prove that the Laguerre polynomials are the expected characteristic polynomials of the Wishart ensemble.


\begin{theorem}
    Let $R$ be an $n\times n$ matrix and $R_{ij}$ be independent Gaussian random variables with mean 0 and variance 1, then

    \begin{equation*}
        E\left[ \chi_x\left(\trans{R}R\right) \right] = L_n(z),
    \end{equation*}
    \todo{Esto lo podría hacer en general, con una recursión más general.}
    \noindent where $L_n(x) = \left( 1 - \partial_z \right)^n [z^n]$ is the $n$th standard Laguerre polynomial.
\end{theorem}

\begin{proof}
    We use the tridiagonal model defined before together with Lemma \ref{lemma:chpol_tridiag}. Let $P_n(z)$ be the characteristic polynomial of $L_\beta$ with $L_\beta$ being a tridiagonal $\beta$-Laguerre ensemble, then

    \begin{equation*}
        P_n(z) = (z - (L_\beta)_{11})P_{n-1}(z) - (L_\beta)_{12}^2 P_{n-2}(z).
    \end{equation*}

    The corresponding entries are

    \begin{equation*}
        (L_\beta)_{11} = \xi^2_{n\beta} + \xi^2_{\beta(m-1)}, \qquad (L_\beta)_{12} = \xi_{\beta(n-1)}\xi_{\beta(m-1)},
    \end{equation*}

    \noindent with this the expected characteristic polynomial satisfies the recursion

    \begin{align} \label{eq:laguerre_recursion}
        \E{P_n(z)} &= \E{ (z - \xi^2_{n\beta} - \xi^2_{\beta(m-1)})P_{n-1}(z) - \xi^2_{\beta(n-1)}\xi_{\beta(m-1)}^2 P_{n-2}(z) },\\ 
        &= (z - \beta(n - m + 1))\E{P_{n-1}(z)} - \beta^2(n-1)(m-1)\E{P_{n-2}(z)}.
    \end{align}

    Which is satisfied by the Laguerre polynomials. By finding the cases $n=1$ and $n=2$, then using a recursive argument, we finish the proof.

    The case $n=1$ is trivial because in this case $G$ is an univariate Gaussian random variable, so $\trans{G}G = G^2$ is a chi-squared random variable and $P_1(z) = z - 1$. For the case $n=2$ we have

    \begin{align*}
        P_2(z) &= \det[zI - \trans{G}G] = \det\begin{bmatrix}
            z- g_1^2 - g_2^2 & -g_1g_3 - g_2 g_4 \\
            - g_1g_3 - g_2 g_4 & z - g_3^2 - g_4^2
        \end{bmatrix}, \\
        &= z^2 - z(g_1^2 + g_2^2 + g_3^2 + g_4^2) +g_1^2 g_3^2 + g_2^2g_4^2 -2g_1g_2g_3g_4.
    \end{align*}

    Taking expectation leads to $P_z(z) = z^2 - 4z + 2$. The rest of the polynomials are found using recursion \eqref{eq:laguerre_recursion}.
\end{proof}

    The dynamical case can be found using the static case and the fact that a Wishart matrix has the same law as $t\trans GG$, for $G$ a matrix of independent standard normal variables.

\begin{theorem}
    Let $B$ be an $n\times n$ matrix and $B_{ij}$ be independent standard Brownian motions, then 

    \begin{equation*}
        E\left[ \chi_z\left( \trans{B}B\right) \right] = L_n(z,t),
    \end{equation*}

    \noindent where $L_n(z,t) = \left( 1 - t\partial_z \right)^n z^n$ is the $n$th Laguerre polynomial with variance $t$.
\end{theorem}

\begin{proof}
    Let $G$ be an $n\times n$ matrix of independent standard Gaussian random variables, so $B \overset{d}{=} \sqrt{t}G$ and this means $\trans{B}B \overset{d}{=} t \trans G G$, it follows that

    \begin{align*}
        \E{\det[zI - \trans B B]} &= \E{\det[zI - t \trans G G]} = t^n \E{\frac zt I - \trans GG]}, \\
        &= t^n \left(1 - \frac{\d }{\d (z/t)}\right)^n\left[ \frac{z}{t}^n\right] = t^n\sum_{k=0}^n (-1)^k \binom nk \frac{n!}{(n-k)!} \left(\frac zt\right)^{n-k},\\
        &= \sum_{k=0}^n (-t)^k \binom nk \frac{n!}{(n-k)!} z^{n-k} = \left( 1 - t\frac{\d}{\d z} \right)^n [z^n].
    \end{align*}
\end{proof}

    As defined before, the time dependant Laguerre polynomials are generated by the operator $(1-t\partial_z)^{n}[z^n]$. On the other hand, the Finite Free Poisson distribution is represented by the roots of the polynomial $(1-\frac1n\partial_z)^{n}[z^n]$, this is the expected characteristic polynomial of a Wishart process rescales by $1/n$ and evaluated at time $t=1$. It turns out that we can find a heat kind equation for every polynomial satisfying $p(z,t)=(1-ct\partial_z)^n[z^n]$ with $c \in\R$. Every such equation would imply a unique dynamics for the polynomial roots which is basically a rescaling of the space-time relationship.

\begin{theorem}
    Let $c\in \R$ and $P(t,z) \coloneqq \left( 1 - ct\partial_z \right)^n [z^n]$, with $t\ge 0, z \in \C$, then $P(t,z)$ satisfies the following differential equation

    \begin{equation*} 
        c\partial_z P(t,z) + cz  \partial_{zz}P(t,z) + \partial_t P(t,z) = 0.
     \end{equation*}

\end{theorem}

 \begin{proof}
    By definition of $P(t,z)$,

    \begin{align*}
        P(t,z) &= \left( 1 - ct\partial_z \right)^n [z^n] = \sum_{k=0}^n \binom{n}{k}\left( -ct \right)^k \partial_z^k [z^n ] = \sum_{k=0}^n \binom{n}{k}\left( -ct \right)^k \frac{n!}{(n-k)!}z^{n-k}.
    \end{align*}

    Now we find the derivatives

    \begin{align*}
        \partial_z [P(t,z)] &= \partial_z\left[ \sum_{k=0}^n \binom{n}{k} \frac{\left( -ct \right)^k n!}{(n-k)!}z^{n-k} \right] = \sum_{k=0}^{n-1} \binom{n}{k} \frac{\left( -ct \right)^k n!}{(n-k-1)!}z^{n-k-1},\\ 
        \partial_{zz} [P(t,z)] &= \partial_z\left[\sum_{k=0}^{n-1} \binom{n}{k} \frac{\left(-ct\right)^k n!}{(n-k-1)!}z^{n-k-1}\right] = \sum_{k=0}^{n-2} \binom{n}{k} \frac{\left(-ct\right)^k n!}{(n-k-2)!}z^{n-k-2},\\ 
        \partial_t [P(t,z)] &= \partial_t \left[ \sum_{k=0}^n \binom{n}{k} \frac{\left( -ct \right)^k n!}{(n-k)!}z^{n-k} \right] = \sum_{k=1}^n \binom{n}{k}k \frac{(-c)^kt^{k-1} n!}{(n-k)!}z^{n-k},\\ 
        &= \sum_{k=0}^{n-1} \binom{n}{k+1}(k+1)(-c)^{k+1}t^{k} \frac{n!}{(n-k-1)!}z^{n-k-1}.
    \end{align*}

    For the sum we can separate the last terms with index $n-1$. Thus we have

    \begin{align*}
        \left(c\partial_z + cz\partial_{zz} + \partial_t\right) [P] &= \sum_{k=0}^{n-2} \left[ c\binom{n}{k}\left( -ct \right)^k \frac{n!}{(n-k-1)!}z^{n-k-1}\right.\\
        &\phantom{esp} + cz\binom{n}{k}\left( -ct \right)^k \frac{n!}{(n-k-2)!}z^{n-k-2}\\ &\phantom{espac} + \left.\binom{n}{k+1}(k+1)(-c)^{k+1}t^{k} \frac{n!}{(n-k-1)!}z^{n-k-1}\right] \\
        &\phantom{espaciooo}+ cn(-ct)^{n-1}n!+n(-c)^nt^{n-1}n!,
        \intertext{reagruping the terms we find}
        &= \sum_{k=0}^{n-2} \binom{n}{k} (-ct)^k n!\left( \frac{(n-k-1)cz^{n-k-1}-c(n-k-1)z^{n-k-1}}{(n-k-1)!} \right), \\ 
        &=\sum_{k=0}^{n-2} \binom{n}{k} \frac{ c^{k+1}(-t)^kn!z^{n-k-1}}{(n-k-1)!}\left( 1 + n-k-1 -n +k \right) = 0.
    \end{align*}
 \end{proof}

 Now, analogously to the Hermite polynomials and the Dyson Brownian motion, we prove the dynamical behavior for the roots of the polynomial defined by $(1-ct\partial_z)^n[z^n]$. In this case, the dynamics will depend on the constant $c$. For a negative $c$, the time will be reversed and in general for a bigger $c$ the movement will be faster. 

 \begin{theorem}
    Let $P(t,z)$ be a monic polynomial with degree $n$ satisfying the equation

     \begin{equation*} 
        c\partial_z [P(t,z)] + cz  \partial_{zz}[P(t,z)] + \partial_t [P(t,z)] = 0.
     \end{equation*}

    Then its roots $\left(z_i(t)\right)_{i=1}^n$ satisfy the equation of motion

    \begin{equation*}
        \d z_i= c \left( \sum_{k\neq i} \frac{z_i + z_k}{z_i - z_k}  + n\right){\d t}.
    \end{equation*}
\end{theorem}

\begin{proof}
    Let $z_i(t)$ be such that $P(t,z_i(t)) = 0$ for every $t$, this means in particular that $\partial_z P(t,z_i(t)) = 0$, so

    \begin{align*}
        0 &= \partial_t [P(t,z_i(t))] = \partial_t [z_i(t)] \partial_z [P(t,z)]|_{z=z_i} + \partial_t [P(t,z)]|_{z=z_i}, \\ 
        &= \partial_t [z_i(t)] \partial_z [P(t,z)]|_{z=z_i} - \left[c\partial_z [P(t,z)] - cz \partial_{zz}[P(t,z)]\right]_{z=z_i},
        \intertext{now we use that $P(t,z)$ is monic and the Leibnitz rule to get}
        \frac{\d}{\d t} z_i(t) &= c\left[\frac{\partial_z [P(t,z)] + z  \partial_{zz}[P(t,z)]}{\partial_z [P(t,z)]}\right]_{z=z_i} = c\left[ 1 +  \frac{2z_i\sum_{k\neq i}\prod_{j\neq i,j\neq k}(z_i - z_j)}{\prod_{j\neq i}(z_i - z_j)} \right],\\
        &= c\left[ 1 + \sum_{k\neq i}\frac{2z_i}{z_i - z_k}\right],\\
        &= c\left[ 1 + \sum_{k\neq i}\frac{2z_i}{z_i - z_k} - \sum_{k\neq i}\frac{z_i-z_k}{z_i - z_k} + \sum_{k\neq i}\frac{z_i-z_k}{z_i - z_k} \right],\\ 
        &= c\left[ 1 + \sum_{k\neq i}\frac{2z_i-z_i + z_k}{z_i - z_k} + n-1 \right] = c\left[\sum_{k\neq i}\frac{z_i + z_k}{z_i - z_k} + n\right].
    \end{align*}
\end{proof}

    In total analogy to Proposition \ref{prop:convolution_of_hermites}, we have a similar result for the Laguerre polynomials using asymmetric additive convolution.

\begin{proposition}
    Let $L_n(z,t_1) = (1-ct_1\partial_z)^n[z^n]$ and $L_n(z,t_2) = (1-ct_2\partial_z)^n[z^n]$ be two Laguerre polynomials of degree $n$, then their asymmetric additive convolution $L_n(z,t_1) \aac L_n(z,t_2)$ is a Laguerre polynomial of degree $n$ with variance $t_1+t_2$, $L_n(z,t_1+t_2)$. 
\end{proposition}

\begin{proof}
    \begin{align*}
        L_n(z,t_1) \aac L_n(z,t_2) &= \sum_{k=0}^n z^{n-k}(-1)^k \sum_{j=0}^k \left(\frac{(n-j)!(n-k+j)!}{n!(n-k)!}\right)^2\binom{n}{j}\frac{(ct_1)^jn!}{(n-j)!}\binom{n}{k-j}\frac{(ct_1)^{k-j} n!}{(n-k+j)!}\\ 
        &= \sum_{k=0}^n z^{n-k}(-1)^k \sum_{j=0}^k \frac{((n-j)!)^2((n-k+j)!)^2(n!)^4(ct_1)^j(ct_2)^{k-j}}{(n!)^2((n-k)!)^2j!(k-j)!((n-j)!)^2((n-k+j)!)^2},\\ 
        &= \sum_{k=0}^n \binom{n}{k} \frac{(-1)^k z^{n-k}n!}{(n-k)!}\sum_{j=0}^k \binom{k}{j}(ct_1)^j(ct_2)^{k-j}, \\  
        &=  \sum_{k=0}^n \binom{n}{k} \frac{(-1)^k z^{n-k}n!}{(n-k)!}[c(t_1 + t_2)]^k = L_n(z,t_1 + t_2).
    \end{align*}
\end{proof}

Finally, just like in the Dyson Brownian motion case, we state a Theorem that allows us to conclude that, no matter the initial position of the roots, if convoluted with a Laguerre polynomial, they will always move according to the deterministic Wishart process.

\begin{theorem}
    Let $L_n(z,t)$ be a Laguerre polynomial of order $n$ with variance $t$ and let $p(z)$ be any monic polynomial. Then the asymmetric additive convolution of $L_n(z,t)$ and $p(z)$, $L_n(z,t) \aac p(z)$ satisfies the following differential equation

    \begin{equation*}
        c\partial_z [L_n(z,t) \aac p(z)] + cz \partial_{zz}[L_n(z,t) \aac p(z)] + \partial_t [L_n(z,t) \aac p(z)] = 0.
    \end{equation*}
\end{theorem}

\begin{proof}
    We find the derivatives first

    \begin{align*}
        \partial_z[L_n(z,t) \aac p(z)] &= \partial_z \left[ \sum_{k=0}^n z^{n-k}(-1)^k \sum_{j=0}^k \left(\frac{(n-j)!(n-k+j)!}{n!(n-k)!}\right)^2 \binom{n}{j}\frac{(ct_1)^jn!}{(n-j)!}b_{k-j}\right],\\ 
        &= \sum_{k=0}^{n-1} (n-k)z^{n-k-1}(-1)^k \sum_{j=0}^k \left(\frac{(n-j)!(n-k+j)!}{n!(n-k)!}\right)^2 \binom{n}{j}\frac{(ct_1)^jn!}{(n-j)!}b_{k-j},
        \intertext{we use the first derivative to find the second one}
        \partial_{zz}[L_n(z,t) \aac p(z)] &= \partial_{zz} \left[ \sum_{k=0}^n z^{n-k}(-1)^k \sum_{j=0}^k \left(\frac{(n-j)!(n-k+j)!}{n!(n-k)!}\right)^2 \binom{n}{j}\frac{(ct_1)^jn!}{(n-j)!}b_{k-j}\right],\\ 
        &= \sum_{k=0}^{n-2} (n-k)(n-k-1)z^{n-k-2}(-1)^k \sum_{j=0}^k \left(\frac{(n-j)!(n-k+j)!}{n!(n-k)!}\right)^2 \binom{n}{j}\frac{(ct_1)^jn!}{(n-j)!}b_{k-j},
        \intertext{finally, for the derivative in time}
        \partial_t[L_n(z,t) \aac p(z)] &= \partial_t \left[ \sum_{k=0}^n z^{n-k}(-1)^k \sum_{j=0}^k \left(\frac{(n-j)!(n-k+j)!}{n!(n-k)!}\right)^2 \binom{n}{j}\frac{(ct_1)^jn!}{(n-j)!}b_{k-j}\right],\\
        &= \sum_{k=0}^n z^{n-k}(-1)^k \sum_{j=1}^k \left(\frac{(n-j)!(n-k+j)!}{n!(n-k)!}\right)^2 \binom{n}{j}\frac{c^j jt_1^{j-1}n!}{(n-j)!}b_{k-j},\\ 
        &= \sum_{k=1}^n z^{n-k}(-1)^k \sum_{j=0}^{k-1} \left(\frac{(n-j-1)!(n-k+j+1)!}{n!(n-k)!}\right)^2 \binom{n}{j+1}\frac{c^{j+1} (j+1)t_1^{j}n!}{(n-j-1)!}b_{k-j-1},\\ 
        \intertext{with a shift in the coefficients we can have the same index as in the two other derivatives}
        &= \sum_{k=0}^{n-1} z^{n-k-1}(-1)^{k+1} \sum_{j=0}^{k} \left(\frac{(n-j-1)!(n-k+j)!}{n!(n-k-1)!}\right)^2 \binom{n}{j+1}\frac{c^{j+1} (j+1)t_1^{j}n!}{(n-j-1)!}b_{k-j},\\ 
        &= \sum_{k=0}^{n-1} (n-k)^2 z^{n-k-1}(-1)^{k+1} \sum_{j=0}^{k} \left(\frac{(n-j)!(n-k+j)!}{n!(n-k)!}\right)^2 \binom{n}{j}\frac{c^{j+1}t_1^{j}n!}{(n-j)!}b_{k-j},\\ 
    \end{align*}

    Now we can sum all over the same set onf indexes up to $n-2$ and sum the two terms wth index $n-1$ outside

    \begin{align*}
        (c&\partial_z + cz\partial_{zz} + \partial_t )[L_n(z,t)\aac p(z)] \\  &= \sum_{k=0}^{n-2} z^{n-k-1}(-1)^k \sum_{j=0}^{k} \left(\frac{(n-j)!(n-k+j)!}{n!(n-k-1)!}\right)^2 \binom{n}{j}\frac{c^{j+1}t_1^{j}n!}{(n-j)!}b_{k-j}\Bigl((n-k)^2 - (n-k)^2\Bigr)\\ 
        &\phantom{espacio}+ (-1)^{n-1}\sum_{j=0}^{n-1} \left( \frac{(j+1)!}{n!} \right)^2\binom{n}{j}\frac{c^{j+1}t_1^{j}n!}{(n-j)!}b_{n-j+1}\\
        &\phantom{espaciooooooo} + (-1)^{n}\sum_{j=0}^{n-1} \left( \frac{(j+1)!}{n!} \right)^2\binom{n}{j}\frac{c^{j+1}t_1^{j}n!}{(n-j)!}b_{n-j+1} = 0.
    \end{align*}
\end{proof}

All the results in this section were analogous to the Hermite polynomials relationship to the Dyson Brownian motion. As we will see in the next section, the results are not as easy to generalize for the matrix Jacobi process. Thus we will need to make a more extensive use of Finite Free Probability Theory.

\subsection{Jacobi process}

For the last process we study in this thesis, we will use a slightly different approach. Although some useful matrix models have been proposed for the Jacobi ensamble \cite{article:jacobi_matrix_model,article:edelman_sutton_jacobi}, we will make use of the definition of the Jacobi matrix in terms of Wishart matrices and what we already know about the Laguerre relationship to Wishart. As we will see, the Jacobi process is stationary, which makes it impossible to define something as the ``dynamical Jacobi polynomials'' (at least as we did with Hermite and Laguerre), however, it is possible to recover the deterministic dynamics when we have an initial law that does not make the process stationary. 

Before proceeding with the proper Jacobi process, let us find the expected characteristic polynomial of a static Jacobi matrix. This process is more challenging than the two previous processes, but the static case is greatly simplified by using the tools of Finite Free Probability theory. 

Recall from Chapter \ref{ch:eigen_processes} that there are essentially two definitions for the Jacobi matrix, one coming form the Multivariate Analysis of Variance and other coming from the upper left corner of a Haar unitary Brownian matrix. For the static case, We work with the MANOVA case. 

\begin{theorem}
    Let $A,B$ be two $n\times n$ random matrices with i.i.d. standard Gaussian entries. Define the matrix $M$ as

    \begin{equation*}
        M \coloneqq (\trans AA + \trans BB)^{-1}\trans AA,
    \end{equation*}

    then the expected characteristic polynomial of $M$, $\chi_z(M)$ is the $n$th Legendre polynomial,

    \begin{equation*}
        \chi_z(M) = \frac{n!n!}{(2n)!}\sum_{k=0}^n z^k(-1)^{n-k} \binom nk \binom{n+k}k.
    \end{equation*}
    \todo{Revisar definición correcta de los Legendre}
\end{theorem}

\begin{proof}

Denote $p_M(z) \coloneqq \chi_z(M)$. To obtain \( p_M(z) \), we can first find that of its inverse, \( p_{M^{-1}}(z) \). We have

\begin{equation*}
    M^{-1} = (A^TA)^{-1}(A^TA + B^TB) = I + (A^TA)^{-1}B^TB.
\end{equation*}

Under the assumptions that \( I \) and \( (A^TA)^{-1}B^TB \) are normal and that \( (A^TA)^{-1}B^TB \) is invariant under conjugation by an orthogonal matrix, it holds that

\begin{equation*}
    p_{M^{-1}}(z) = p_I(z) \boxplus_n p_{(A^TA)^{-1}B^TB}(z).
\end{equation*}

Furthermore, under the assumption that \( (A^TA)^{-1} \) and \( B^TB \) are normal and invariant under conjugation by orthogonal matrices, it holds that

\begin{equation*}
    p_{(A^TA)^{-1}B^TB}(z) = p_{(A^TA)^{-1}} \boxtimes p_{B^TB}(z).
\end{equation*}

We also know that \( p_{B^TB}(z) = L_n(z) \) and \( p_{(A^TA)^{-1}}(z)=\frac{z^n}{L_n(0)} L_n(1/z) \). With this, we can find \( p_{M^{-1}}(z) \), by noticing that $L_n(0)= (-1)^nn!$,


\begin{align*}
    p_{(A^TA)^{-1}B^TB}(z) &= p_{(A^TA)^{-1}} \boxtimes_n p_{B^TB}(z) = (\frac{(-1)^n}{n!}z^n p_{A^T A}(1/z) \boxtimes p_{B^TB}(z)),\\
    &= \frac{(-1)^n}{n!}\sum_{k=0}^n z^{n-k} (-1)^k \frac{a_kb_k}{\binom{n}{k}},\\ 
    &= \frac{(-1)^n}{n!}\sum_{k=0}^n z^{n-k}(-1)^k \frac{\binom{n}{k}\frac{n!}{(n-k)!}\binom{n}{k}(-1)^{n-2k}\frac{n!}{k!}}{\binom{n}{k}}, \\
    &= \frac{(-1)^n}{n!}\sum_{k=0}^n z^{n-k}(-1)^k \binom{n}{k}\frac{n!}{(n-k)!}(-1)^{n-2k}\frac{n!}{k!},\\
    &= \sum_{k=0}^n z^{n-k}(-1)^k \binom{n}{k}^2 (-1)^{2n-2k}.
\end{align*}

Now we calculate \( p_{M^{-1}}(z) \), remembering that \( p_I(z) = (z-1)^n \)

\begin{align*}
    p_I \boxplus_n p_{(A^TA)^{-1}B^TB} &= \sum_{k=0}^n z^{n-k}(-1)^k \sum_{j=0}^k \frac{(n-j)!(n-k+j)!}{n!(n-k)!}\binom{n}{j}\binom{n}{k-j}^2(-1)^{2n-2(k-j)},\\ 
    &= \sum_{k=0}^n z^{n-k}(-1)^{2n-k}\binom{n}{k} \sum_{j=0}^k \frac{n!n!k!}{j!(k-j)!(k-j)!(n-k+j)!},\\ 
    &= \sum_{k=0}^n z^{n-k}(-1)^{2n-k}\binom{n}{k} \sum_{j=0}^k \binom{n}{k-j}\binom{k}{j},\\  
    &= \sum_{k=0}^n z^{n-k}(-1)^{2n-k}\binom{n}{k} \binom{n+k}{k}.
\end{align*}

Finally, we use \( p_{M}(z) = \frac{z^n}{p_{M^{-1}}(0)} p_{M^{-1}}(1/z) \) to obtain

\begin{align*}
    p_{M}(z) &= \frac{n!n!z^n}{(2n)!} \sum_{k=0}^n z^{k-n}(-1)^{n-k}n!\binom{n}{k} \binom{n+k}{k} = \frac{n!n!}{(2n)!}\sum_{k=0}^n z^{k}(-1)^{n-k}n!\binom{n}{k} \binom{n+k}{k}.
\end{align*}

This corresponds to the \(n\)-th Legendre polynomial evaluated at \( 1-2z \), which is precisely the desired expected characteristic polynomial.
\end{proof}

Now we will approach the dynamical version. Let $N_1,N_2$ be two $n\times n$ random matrices with i.i.d standard Gaussian entries and $B_1, B_2$ be two independent Brownian motions in $\M_{n,n}(\R)$. Notice that 

\begin{align*}
    (\trans{B_1} B_1 + \trans{B_2} B_2)^{-1} \trans{B_1}B_1 &\overset{d}= (t\trans{N_1} N_1 + t\trans{N_2} N_2)^{-1} t\trans{N_1}N_1 = (\trans{N_1} N_1 + \trans{N_2} N_2)^{-1} \trans{N_1}N_1,\\
    (\trans{B_1} B_1 + \trans{B_2} B_2)^{-1/2} \trans{B_1}B_1& (\trans{B_1} B_1 + \trans{B_2} B_2)^{-1/2} \\ &\overset{d}=(t\trans{N_1} N_1 + t\trans{N_2} N_2)^{-1/2} t\trans{N_1}N_1 (t\trans{N_1} N_1 + t\trans{N_2} N_2)^{-1/2},\\
    &=(\trans{N_1} N_1 + \trans{N_2} N_2)^{-1/2} \trans{N_1}N_1 (\trans{N_1} N_1 + \trans{N_2} N_2)^{-1/2}.
\end{align*}

Thus, in both definitions of the Jacobi matrix, if we substitute the Gaussian random variables with Brownian motions, we get the same distribution. This means that the matrix Jacobi process is stationary with the law of the zeroes of the associated Jacobi polynomial. Then the expected characteristic polynomial doest not depend on $t$ and we can not formulate something as the time dependent Jacobi polynomials or Jacobi polynomials with variance. We will see, however, that if we start the process with another distribution, we can recover the deterministic Jacobi dynamics. All of the following content is taken from \cite{article:marcus_finite_free_point_processes}, where the author manages to generalize the results we got in the previous subsections together with the Jacobi case. We will work with the Jacobi matrix definition in \cite{doumerc2005matrices} where the Jacobi process can be expressed by the following matrix

\begin{equation*}
    W = (\trans{M_1}M_1 + \trans{M_2}M_2)^{-1}\trans{M_1}M_1(\trans{M_1}M_1 + \trans{M_2}M_2)^{-1},
\end{equation*}

\noindent where $M_1, M_2$ are independent Gaussian matrices with dimensions $n_1\times k$ and $n_2 \times k$, respectively.  We can write $W_i = \trans{M_i}M_i$ and $\Delta=\det\left[(W_1+W_2)^{-1}\right]$ for a shorter notation. Now to compute the characteristic polynomial we have,

\begin{align*}
    p_W(z) &= \det\left[ zI - (W_1+W_2)^{-\frac12}W_1(W_1+W_2)^{-\frac12} \right], \\ 
    &= \Delta\det\left[ z(W_1+W_2) - (W_1+W_2)^{\frac12}W_1(W_1+W_2)^{-\frac12} \right],\\ 
    &= \Delta\det\left[ z(W_1+W_2)(W_1+W_2)^{\frac12}(W_1+W_2)^{-\frac12} - (W_1+W_2)^{\frac12}W_1(W_1+W_2)^{-\frac12} \right], \\
    &= \Delta\det\left[(W_1+W_2)^{\frac12}\right]\det\left[ z(W_1 + W_2) - W_1\right]\det\left[(W_1+W_2)^{-\frac12}\right],\\ 
    &= \Delta\det\left[ (z-1)W_1 + zW_2 \right].
\end{align*}

Thus finding the expected characteristic polynomial (up to a normalization constant) can be done without the need to use the inverse of $W_1 + W_2$.



Replicating the procedure in \cite{article:marcus_finite_free_point_processes}, we will first replicate the former results of the static Wishart and Jacobi matrices with an operator that generalized characteristic polynomials, then we will show that the Wishart process can be built in the base of this operator by adding a time parameter. Finally, we will extend this same procedure to the Jacobi process. In the process of building these generalizations we will make extensive use of Finite Free Probability Theory.

\begin{definition}[Generalized characteristic polynomial]
    Let $A, B$ be two random independent Gaussian matrices in $\mathcal M_{n_1,k}(\mathbb R)$ and $\mathcal M_{n_2,k}(\mathbb R)$ respectively. We define the generalized characteristic polynomial $p_{A,B}(x,y,z)$ as

    \begin{equation*}
        p_{A,B}(x,y,z) \coloneqq \det\left[ xI + y \trans{A}A + z \trans{B}B \right].
    \end{equation*}

    Similarly, we define the reciprocal generalized reciprocal polynomial as $q_{A,B}(x,y,z) \coloneqq y^{n_1}z^{n_2}P_{A,B}(x,y,z)$.
\end{definition}

    Notice that this generalizes the characteristic polynomial of the (static and dynamical) Wishart and the static Jacobi cases. 

    Similarly to the univariate cases (Theorem \ref{thm:multiplicative_operators}), we have a multiplicative Theorem in terms of the operators that generate the new multivariate polynomials.

    \begin{theorem} \label{thm:multivariate_operators}
        Let $F$ and $G$ be two-variable polynomials and $A,B \in \mathcal M_{k,n_1}, C,D \in \mathcal M_{k,n_2}$ such that $C$ and $D$ are invariant under product by signed permutation matrices. Suppose that

         \begin{align*}
            \E{q_{A,B}(x,y,z)} &= F(\partial_x \partial_y,\partial_x\partial_z)[x^ky^{n_1}z^{n_2}],\\ 
            \E{q_{C,D}(x,y,z)} &= G(\partial_x \partial_y,\partial_x\partial_z)[x^ky^{n_1}z^{n_2}].
         \end{align*}

         Then $\E{q_{A+C, B+D}(x,y,z)} = F(\partial_x \partial_y,\partial_x\partial_z)G(\partial_x \partial_y,\partial_x\partial_z)[x^ky^{n_1}z^{n_2}]$
    \end{theorem}

    \begin{proof}
        If $F',G'$ are the linear differential operators (multivariate polynomials on $\partial_x\partial_y, \partial_x\partial_z$) generating $p_{A,B}$ and $p_{C,D}$ and they satisfy the former multiplicative property, then it can be extended to $F,G$ by linearity. Thus we prove for $F', G'$. 


        By definition, we have that 
        
        \begin{align*}
            \E{ p_{A, B}(x,y,z) } &= \E{ \det[ x I + y \trans A A + z \trans BB ] } = \E{\chi_x(-y\trans AA -z\trans BB)},\\
            \E{ p_{C, D}(x,y,z) } &= \E{ \det[ x I + y \trans CC + z \trans DD ] } = \E{\chi_x(-y\trans CC -z\trans DD)}.
        \end{align*}

        Notice that by Theorem \ref{thm:symmad} we have that

        \begin{align*}
            \E{p_{A+C, B+D}(x,y,z)} &= \E{\chi_x(-y\trans AA -z\trans BB)} \boxplus_n \E{\chi_x(-y\trans CC -z\trans DD)}.
        \end{align*}

        Applying Theorem \ref{thm:multiplicative_operators} we get to the desired result by considering the derivatives in $y,z$ as constant coefficients when evaluating in $x$.
    \end{proof}

    With the next Theorem we get a limit for the products of differential operators, similarly to what we did in Chapter \ref{ch:finite_free} when proving the finite free limit theorems. We will use this Theorem to prove properties of the generalized characteristic polynomial of Gaussian matrices, by see these matrices as limits.

    \begin{theorem} \label{thm:exp_operator}
        Let $\{A_i\}_{i=0}^\infty$ and $\{B_i\}_{i=0}^\infty$ be sequences of matrices invariant under transformation by signed permutation matrices such that

        \begin{align*}
            \E{Tr(\trans A_iA_i)} &= \sigma_1 n_1 k,\\
            \E{Tr(\trans B_iB_i)} &= \sigma_2 n_2 k.
        \end{align*}

        If we define $C_m, D_m$ as

        \begin{align*}
            C_m &\coloneqq \sum_{i=0}^m \frac{A_i}{\sqrt m}, \\
            D_m &\coloneqq \sum_{i=0}^m \frac{D_i}{\sqrt m}.
        \end{align*}

        Then the matrices $C_m, D_m$ satisfy

        \begin{equation*}
            \lim_{m\to\infty} \E{q_{C,D}(x,y,z)} = e^{\sigma_1 \partial_x \partial_y + \sigma_2 \partial_x \partial_z}[x^k y^{n_1} z^{n_2}].
        \end{equation*}
    \end{theorem}

    \begin{proof}
        First we use the following expansion for the determinant,

        \begin{equation*}
            \det( { A} + h { B} ) = \det({ A})  + h \, Tr \left( \mbox{adj} ({A}) \, { B} \right) + O \left(h^2\right).
        \end{equation*}
        \todo{Probar esto}

        This means for $q_{A_i/\sqrt m, B_i/\sqrt m}(x,y,z)$ that

        \begin{align*}
            \E{q_{A_i/\sqrt m, B_i/\sqrt m}(x,y,z)} &= y^{n_1}z^{n_2}\E{ \det[ xI + \frac{1}{m}\left( y^{-1}\trans A_i A_i + z^{-1}\trans B_i BI \right) ] }, \\ 
            &= y^{n_1}z^{n_2}\left[ x^k + \frac{x^{k-1}y^{-1}}{m}\E{Tr(\trans A_iA_i)} + \frac{x^{k-1}z^{-1}}{m}\E{Tr(\trans B_i B_i)} + O\left(\frac{1}{m^2}\right))\right],\\ 
            &= x^k y^{n_1}z^{n-2} + \frac{\sigma_1 n_1 k}{m}x^{k-1}y^{n_1-1}z^{n-2} + \frac{\sigma_2 n_2 k}{m}x^{k-1}y^{n_1}z^{n_2-1} + O \left(\frac{1}{m^2}\right), \\
            &= \left( 1 + \frac{\sigma_1}{m}\partial_y\partial_x + \frac{\sigma_2}{m}\partial_z\partial_x + O \left(\frac{1}{m^2}\right)\right)[x^k y^{n_1} z^{n_2}]
        \end{align*}

    Once we have found $q_{A_i/\sqrt m, B_i/\sqrt{m}}(x,y,z)$ as a polynomial on the operators $\partial_x \partial_y$ and $\partial_x \partial_z$, we apply Theorem \ref{thm:multivariate_operators} to get, for $q_{C_m,D_m}(x,y,z)$,

    \begin{align*}
        \E{ q_{C_m,D_m}(x,y,z) } &= \left( 1 + \frac{\sigma_1}{m}\partial_y\partial_x + \frac{\sigma_2}{m}\partial_z\partial_x + O \left(\frac{1}{m^2}\right)\right)^m[x^k y^{n_1} z^{n_2}]
    \end{align*}

    Tanking $m\to \infty$ in the last expression we get

    \begin{align*}
        \lim_{m\to\infty} \E{q_{C_m,D_m}(x,y,z)} &= \lim_{m\to\infty} \left( 1 + \frac{\sigma_1}{m}\partial_y\partial_x + \frac{\sigma_2}{m}\partial_z\partial_x + O \left(\frac{1}{m^2}\right)\right)^m[x^k y^{n_1} z^{n_2}],\\
        &= e^{\sigma_1 \partial_x\partial_y + \sigma_2 \partial_x \partial_z}[x^k y^{n_1}z^{n_2}],
    \end{align*}

    \noindent which is the desired result.
    \end{proof}

    Applying the previous Theorem, we find that the differential operator that generates the generalized characteristic polynomial of Gaussian matrices, has an exponential form.

    \begin{corollary}
        Let $A$ and $B$ be equal in law to $\sigma_1 N_1$ and $\sigma_2 N_2$ where $N_1$ and $N_2$ are matrices with all of the entries being independent standard normal random variables, then
        
        \begin{equation*}
            \E{ q_{A,B}(x,y,z) } = e^{\sigma_1 \partial_x \partial_y + \sigma_2 \partial_x \partial_z}[x^k y^{n_1} z^{n_2}].
        \end{equation*}
        \end{corollary}

        \begin{proof}
            %By Theorem \ref{thm:exp_operator} 
            Let $\{A_i\}_{i\in \N}$ and $\{B_i\}_{i\in \N}$ be as in Theorem \ref{thm:exp_operator} and define $C_m, D_m$ in the same way. On the one hand we have, as shown previously

            \begin{equation*}
                \lim_{m\to \infty} \E{ q_{C_m,D_m}(x,y,z)} =  e^{\sigma_1 \partial_x \partial_y + \sigma_2 \partial_x \partial_z}[x^k y^{n_1} z^{n_2}].
            \end{equation*}

            Due to the existence of the second moment of the entries, on the other hand, we have that $C_m, D_m$ converge in law to a Gaussian independent matrix whose entries have variance $\sigma_1^2$ and $\sigma_2^2$, respectively, i.e.

            \begin{equation*}
                \lim_{m\to \infty} \E{ q_{C_m,D_m}(x,y,z)} = \E{ q_{\sigma_1 N_1, \sigma_2 N_2}(x,y,z)}.
            \end{equation*}        
        \end{proof}

        With this we have an operator that allows us to compute $q_{A,B}(x,y,z)$ for $A,B$ independent Gaussian matrices with zero mean and variances $\sigma_1,\sigma_2$, respectively. Our goal is to be able to use this to study matrix-valued stochastic processes by letting the variances vary linearly in time. Hence, the entries of the matrices have the same law as a standard Brownian motion starting at zero. An interesting question is how this behavior would be affected if we start our processes at points different than zero. The answer is given by Theorem \ref{thm:multivariate_operators} and noticing that any polynomial $r(x,y,z)$ of orders $k,n_1,n_2$ can be seen as a polynomial differential operator acting on $x^ky^{n_1}y^{n_2}$, so if $S\in \M_{k,n_1}(\R),T\in \M_{k,n_2}(\R)$ are fixed matrices and $A, B$ are Gaussian independent matrices with variances $\sigma_1$ and $\sigma_2$, respectively, then

        \begin{equation*}
            \E{q_{S+A,T+B}(x,y,z)} = e^{\sigma_1 \partial_x \partial_y + \sigma_2 \partial_x \partial_z}[q_{S,T}(x,y,z)].
        \end{equation*}
        
        
        
        We are interested in $p_{A,B}(x,y,)$, the reciprocal polynomial in $y$ and $z$.  Let us denote by $R^z_{k}(\cdot)$ the reciprocal polynomial operator of order $k$ in the variable $z$, which means, for $r(x,y,z)$ a polynomial,

        \begin{equation*}
            R^z_k \left(r(x,y,z)\right) = z^kr\left(x,y,\frac1z\right).
        \end{equation*}

        Now notice that for $A,B$ Gaussian independent matrices with variances $\sigma_1,\sigma_2$ and given matrices $S,T$ we have $q_{S+A,T+B}(x,y,z)= R^z_{n_2}(R^y_{n_1}(p_{S+A,T+B}(x,y,z))$, so

        \begin{align*}
            \E{p_{A,B}(x,y,z)} &= R^z_{n_2}\left( R^y_{n_1}\left(\E{ q_{S+A,T+B}(x,y,z) } \right) \right), \\
            &= R^z_{n_2}\left( R^y_{n_1}\left(e^{\sigma_1 \partial_x\partial_y + \sigma_2 \partial_x\partial_z} q_{S,T}(x,y,z) \right) \right), \\
            &= R^z_{n_2}\left( R^y_{n_1}\left(e^{\sigma_1 \partial_x\partial_y + \sigma_2 \partial_x\partial_z} R^z_{n_2}(R^y_{n_1}(p_{S,T}(x,y,z) \right) \right),\\ 
            &= \left( R^z_{n_2}\circ R^{y}_{n_1}\circ e^{\sigma_1\partial_x\partial_y + \sigma_2 \partial_x\partial_y} \circ R^z_{n_2}\circ R^{y}_{n_1} \right)[p_{S,T}(x,y,z)].
        \end{align*}

    We will find an expression for this operator applied to $x^iy^jz^l$ with $j\le n_1, l \le n_2$. Then it is extended linearly to a general polynomial.

    \begin{align*}
        ( R^z_{n_2}\circ R^{y}_{n_1}&\circ e^{\sigma_1\partial_x\partial_y + \sigma_2 \partial_x\partial_y} \circ R^z_{n_2}\circ R^{y}_{n_1})[x^i y^j z^l]\\ 
        &=( R^z_{n_2}\circ R^{y}_{n_1}\circ e^{\sigma_1\partial_x\partial_y + \sigma_2 \partial_x\partial_y})[x^i y^{n_1-j}z^{n_2-l}],\\ 
        &= ( R^z_{n_2}\circ R^{y}_{n_1}\circ e^{\sigma_2\partial_x\partial_z}) \sum_{r=0}^\infty \frac{\sigma_1^r}{r!} \partial_x^r \partial_y^r [x^i y^{n_1-j}z^{n_2-l}], \\
        &= ( R^z_{n_2}\circ R^{y}_{n_1}\circ e^{\sigma_2\partial_x\partial_z}) \sum_{r=0}^\infty \frac{\sigma_1^r}{r!} \partial^r_x [x^i] \frac{(n_1-j)!}{(n_1 - j - r)!}y^{n_1-j-r}z^{n_2-l},\\ 
        &= ( R^z_{n_2}\circ R^{y}_{n_1}\circ e^{\sigma_2\partial_x\partial_z})z^{-l}\sum_{r=0}^\infty \binom{n_1-j}{r}y^{n_1 - j -r} \sigma_1^r\partial_x^r[x^i],
\intertext{doing the same procedure for $e^{\sigma_2\partial_x\partial_z}$ we get }
        &= ( R^z_{n_2}\circ R^{y}_{n_1}\circ e^{\sigma_2\partial_x\partial_z})z^{n_2-l}\left(y + \sigma_1\partial_x\right)^{n_1 - j}[x^i],\\
        &= R^z_{n_2}\circ R^{y}_{n_1}\left\{ (z+\sigma_2\partial_x)^{n_2-l} \left(y + \sigma_1\partial_x\right)^{n_1 - j}[x^i]\right\},\\
        &= (1+z\sigma_2\partial_x)^{n_2-l} \left(1 + y\sigma_1\partial_x\right)^{n_1 - j}[x^i y^j z^l].
    \end{align*}


    % With this, we can conclude that 

    % \begin{equation*}
    %     \E{p_{S+A,T+B}} = (1+z\sigma_2\partial_x)^{n_2-l} \left(1 + y\sigma_1\partial_x\right)^{n_1 - j}[p_{S,T}(x,y,z)].
    % \end{equation*}

    In the case when we start the processes at the origin we have that 

    \begin{equation*}
        p_{S,T}(x,y,z) = \det[ xI + y 0_{n_1,k} + z 0_{n_2,k}] =  x^k \det[I] = x^k,
    \end{equation*}

    \noindent which leads to


    \begin{equation} \label{eq:generalized_operator}
        \E{p_{A,B}(x,y,z)} = (1+z\sigma_2\partial_x)^{n_2-l} \left(1 + y\sigma_1\partial_x\right)^{n_1 - j}[x^k].
    \end{equation}

    If we let $\sigma_1 = \sigma_2 = 1$ and replace $y=-1, z=0$ in \eqref{eq:generalized_operator}, for $n_1 = k$, we recover the static Wishart matrix and we can conclude that

    \begin{equation} \label{eq:wishart_finite_free}
        \E{\det[xI - \trans{A}A]} = (1 - \partial_x)^{n_1}[x^k],
    \end{equation}

    \noindent which is exactly the associated Laguerre polynomial previously defined in Section \ref{sec:polynomials}. Thus, we have successfully replicated the result in Subsection \ref{subsec:wishart_laguerre}.

    Keeping the unitary variances and replacing $x=0, y=(x-1), z=x$, we get the static Jacobi matrix, as shown previously, thus 

    \begin{equation*}
        \E{\det[(x-1)W_1 + xW_2]} = (1+x\partial_x)^{n_2} \left(1 + (x-1)\partial_x\right)^{n_1}[x^k] = P_k^{n_2 - k, n_1-k}(2x-1).
    \end{equation*}
    \todo{Sustituir esta forma en el operador del polinomio de Jacobi.}
    This is the Jacobi polynomial of order $k$ with parameters $n_2 - k$ and $n_1-k$ evaluated in $2x-1$. When we normalize to make it monic, we get the result appearing in \cite{edelman1988eigenvalues}.

    Now let $\sigma_1= \sigma_2 = t$, then $A,B$ are equal in law to standard Brownian motions in $\M_{n_1,k}(\R)$ and $\M_{n_2,k}(\R)$, respectively. We should recover the expected characteristic polynomial of the Wishart and Jacobi processes. The Wishart case is evident by replacing $-1$ with $-t$ in \eqref{eq:wishart_finite_free} and compare it to equation \eqref{eq:laguerre_with_variance}. For the Jacobi process with initial condition $p(z)$, we have

    \begin{equation} \label{eq:dynamic_jacobi}
        \E{\det[(z-1)W_1(t) + zW_2(t)]} = (1+xt\partial_x)^{n_2} \left(1 + (x-1)t\partial_x\right)^{n_1}[p(z)].
    \end{equation}

    In analogy to what we found for the Dyson Brownian motion and the Wishart process, we need to verify if the roots of the polynomial defined in \eqref{eq:dynamic_jacobi} satisfy a differential equation given by the finite variation part of \eqref{eq:jacobi}. Let us work in full generality and define for polynomials of the form $x^i y^j z^l$, the operator

    \begin{equation*}
        Q_{n_1,n_2}^t [x^i y^j z^l] = (1+zt\partial_x)^{n_2-l} \left(1 + yt\partial_x\right)^{n_1 - j}[x^i y^j z^l],
    \end{equation*}

    \noindent that is later extended to general polynomials linearly. 

    Denote the time-dependant polynomial $Q_{n_1,n_2}^t[p(x,y,z)]$ by $\hat p(x,y,z,t)$. In analogy to what we have done with the Hernmite and Laguerre cases, we want to find for every $t$ the values $x(t), y(t), z(t)$ such that $\hat p(x(t),y(t),z(t),t)=0$. By derivating with respect to $t$, we get

    \begin{align*}
       0 &= \partial_t [\hat p(x(t),y(t),z(t),t)]\\ &= [(\partial_t x(t))\partial_x \hat p(x,y,z,t) + (\partial_t y(t))\partial_y  \hat p(x,y,z,t) \\
       &\phantom{espaciooooooooo}+ (\partial_t z(t))\partial_z \hat p(x,y,z,t) + \partial_t \hat p(x,y,z,t)]_{(x,y,z)=(x(t),y(t),z(t))}.
    \end{align*}

    Now, again in full generality, let us find some properties satisfied by the time-dependant polynomial with arbitrary parameters.

    \begin{lemma}
        The time-dependant polynomial $\hat p(x,y,z,t)$ satisfies

        \begin{equation} \label{eq:general_harmonic_roots}
            \partial_t[\hat p(x,y,z,t)]_{t=0} = (n_1y + n_2z)\partial_x[\hat p(x,y,z,t)]_{t=0} - y^2\partial_y\partial_x[\hat p(x,y,z,t)]_{t=0} - z^2\partial_z\partial_x[\hat p(x,y,z,t)]_{t=0}.
        \end{equation}
    \end{lemma}

    \begin{proof}
        A general polynomial $p$ on $x,y,z$, can be written as 

        \begin{equation*}
            p(x,y,z) = \sum_{i,j,l} c_{ijl}x^i y^j z^l,
        \end{equation*}

        \noindent where $c_{ijl}$ is the coefficiente associated to the term $x^i y^j z^l$. When we apply $Q_{n_1,n_2}^t$ to $p(x,y,z)$ we have

        \begin{align*}
            Q_{n_1,n_2}^t [p(x,y,z)] &= (1+zt\partial_x)^{n_2-l} \left(1 + yt\partial_x\right)^{n_1 - j}\left[\sum_{i,j,l}c_{ijl}x^i y^j z^l\right],\\
            &= \sum_{i,j,l}c_{ijl}(1+zt\partial_x)^{n_2-l} \left(1 + yt\partial_x\right)^{n_1 - j}\left[x^i y^j z^l\right]\\
            &= \sum_{i,j,l}c_{ijl}\sum_{r=0}^{n_2-l}\binom{n_2-l}{r}(zt\partial_x)^r\sum_{s=0}^{n_1-j}\binom{n_1-j}{s}(yt\partial_x)^{n_1-j}[x^iy^jk^l],\\
            &= \sum_{i,j,l} c_{ijl}\left\{ (n_2-l)zt\partial_x + (n_1-j)yt\partial_x + O(t^2) \right\}[x^iy^jk^l],\\ 
            &= \sum_{i,j,l} c_{ijl}\left\{ (n_2-l)itx^{i-1}y^j z^{l+1} + (n_1-j)itx^{i-1}y^{j+1}z^l \right\} + O(t^2).
        \end{align*}

        Then, differentiating in $t$ and evaluating at $t=0$ gives us

        \begin{align*}
            \left.\partial_t \left\{ Q_{n_1,n_2}^t [p(x,y,z)] \right\}\right|_{t=0} &= \left.\partial_t \left\{ \sum_{i,j,l} c_{ijl}\left\{ (n_2-l)itx^{i-1}y^j z^{l+1} + (n_1-j)itx^{i-1}y^{j+1}z^l \right\} + O(t^2) \right\}\right|_{t=0}, \\
            &= \sum_{ijl} c_{ijl}\{ (n_2-l)ix^{i-1}y^jz^{l+1} + (n_1-j)ix^{i-1}y^{j+1}z^l\}.
        \end{align*}

        We can re-write the last two terms in every summand as

        \begin{align*}
            (n_2-l)ix^{i-1}y^jz^{l+1} &= n_2z\partial_x[x^iy^jz^l] - z^2\partial_x\partial_z[x^iy^jz^l],\\
            (n_1-j)ix^{i-1}y^{j+1}z^{l} &= n_1y\partial_x[x^iy^jz^l] - y^2\partial_x\partial_y[x^iy^jz^l].
        \end{align*}

        Extending by linearity to $p$ we have that

        \begin{equation*}
            \partial_t[\hat p]_{t=0} = (n_1y + n_2z)\partial_x[p] - y^2\partial_y\partial_x[p] - z^2\partial_z\partial_x[p].
        \end{equation*}

        This finishes the proof since $p(x,y,z) = \hat p(x,y,z,t)|_{t=0}$.
    \end{proof}

    This last result together with \eqref{eq:general_harmonic_roots} give us that

    \begin{align*}
        \partial_t[x(t)]\partial_x[p(x(t),y(t),z(t))] &+ \partial_t[y(t)]\partial_y[p(x(t),y(t),z(t))] + \partial_t[z(t)]\partial_z[p(x(t),y(t),z(t)) \\
        &= - (n_1y + n_2z)\partial_x[p] + y^2\partial_y\partial_x[p] + z^2\partial_z\partial_x[p].
    \end{align*}

    As a last step, we will use that $\hat p$  is homogeneous in order to find more convenient expressions for the derivatives. We state this result as a general Lemma for homogeneous polynomials.

    \begin{lemma}
        Let $r(x,y,z)$ be a homogeneous polynomial of order $k$ in $x,y,z$, then

         \begin{align*}
            y^2\partial_y\partial_x r + z^\partial_z\partial_x r &= (y+z)(k-1)\partial_x r - yz (\partial_z\partial_x r + \partial_y\partial_x r) - (y+z)x\partial_{xx} r.
        \end{align*}
    \end{lemma}

    \begin{proof}
        The fact that $r$ is homogeneous with degree $k$ implies that $\partial_x r$ is homogeneus with degree $k-1$, then

        \begin{align*}
            z\partial_z\partial_x r + y \partial_y\partial_x r + x \partial_{xx}r = (k-1)\partial_x r.
        \end{align*}

        The last equality implies

        \begin{align*}
            y^2 \partial_y \partial_x r &= y(k-1)\partial_x r - yz\partial_z\partial_x r - yx\partial_{xx}r,\\
            z^2 \partial_z \partial_x r &= z(k-1)\partial_x r - zy\partial_y\partial_x r - zx\partial_{xx}r.
        \end{align*}

        And summing up these two expressions leads to

        \begin{align*}
            y^2\partial_y\partial_x r + z^2\partial_z\partial_x r &= (y+z)(k-1)\partial_x r - yz (\partial_z\partial_x r + \partial_y\partial_x r) - (y+z)x\partial_{xx} r.
        \end{align*}
    \end{proof}

    Then we have for $\hat p$

    \begin{align*}
        (\partial_t x)(\partial_x \hat p) &+ (\partial_t y)(\partial_y \hat p) + (\partial_t z)(\partial_z \hat p)|_{t=0} \\ 
        &= -(n_1y + n_2 z)\partial_x \hat p + (y+z)(k-1)\partial_x \hat p - yz (\partial_z\partial_x \hat p + \partial_y\partial_x \hat p) - (y+z)x\partial_{xx} \hat p|_{t=0}.
    \end{align*}

    With this, we are ready to prove that, for an arbitrary initial condition, we can recover the deterministic Jacobi dynamics.

    \begin{theorem}
        Let $J = (J(t), t \ge 0)$ be an $n\times n$ matrix valued Jacobi process with parameters $n_1 - k$ and $n_2 - k$, respectively. Let $A \in \M_{n,n}$ be any matrix with roots in $[0,1]$. Then the roots of the expected characteristic polynomial of $A + J(t)$ satisfy the deterministic Jacobi equation \eqref{eq:deterministic_jacobi}.
    \end{theorem}

    \begin{proof}

    In the above results, we can substitute $y = w, z = (w-1)$ and define the polynomial $j(x,w,t)$ as

    \begin{equation*}
        j(x,w,t) = \hat p(x(t), w(t), w(t)-1, t).
    \end{equation*}

    Using the above results, we have for $j(w,t)$

    \begin{align*}
        (\partial_t x)&(\partial_x j) + (\partial_t w)(\partial_w j)|_{t=0} \\
        &= -(n_1w + n_2w-n_2))\partial_x j + (2w-1)(k-1)\partial_x j - w(w-1)(\partial_x\partial_w j) - (2w-1)x\partial_{xx} j|_{t=0}.
    \end{align*}

    Letting $x=0$, we find the evolution equation for the matrix Jacobi process

    \begin{align*}
        \left.(\partial_t w)\frac{\partial_w j}{\partial_x j}\right|_{t=0} &= \left.-(n_1w + n_2w) + (2w-1)(k-1) - w(w-1)\frac{\partial_x\partial_w j}{\partial_x j}\right|_{t=0},\\
        \Rightarrow \partial_t w &= -[n_1w + n_2 (w-1) +(2w-1)(k-1)]\frac{\partial_x j}{\partial_w j} - w(w-1)\frac{\partial_x\partial_w j}{\partial_w j},\\ 
        &= [-(n_1-k+1)w - (n_2 -k+1)(w-1)]\frac{\partial_x j}{\partial_w j} - w(w-1)\frac{\partial_x\partial_w j}{\partial_w j}.
    \end{align*}

    To further simplify the last expression we need some hypothesis about the relationship between $\partial_x j$ and $\partial_w j$. We recover this from the definition as the generalized characteristic polynomial.

    \begin{align*}
        j(x,w,t) &= \E{ \det[ xI + (u-1)\trans A_1(t) A_1(t) + u \trans A_2(t) A_2(t) ] }, \\
        &= \E{ \det[ xI + (u-1)\trans A_1(t) A_1(t) + u (I - \trans A_1(t) A_1(t)) ] }, \\
        &= \E{ \det[ (x+u)I - \trans A_1(t) A_1(t) +(u-u)\trans A_1(t) A_1(t) ] }, \\
        &= \E{ \det[ (x+u)I - \trans A_1(t) A_1(t)] }.
    \end{align*}

    So this means that the derivatives of $j$ with respect to $x$ are the same as its derivatives with respect to $w$ and then we can equate $\partial_w j$ to $\partial_x j$ to get

    \begin{align*}
        \partial_t w_i &= -(n_1-k+1)w_i - (n_2 -k+1)(w_i-1) - w_i(w_i-1)\frac{\partial_{xx} j}{\partial_x j},\\ 
        &=-(n_1-k+1)w_i - (n_2 -k+1)(w_i-1) - w_i(w_i-1)\sum_{j \neq i} \frac{2}{w_i - w_j},\\ 
        &= n_2 - (n_1 + n_2)w_i + (w_i + w_i -1)(k-1) - \sum_{j\neq i} \frac{2w_i(w_i -1)}{w_i - w_j},\\
        &= n_2 - (n_1 + n_2)w_i + (w_i + w_i -1)\sum_{j \neq i}\frac{w_i - w_j}{w_i - w_j} - \sum_{j\neq i} \frac{2w_i(w_i -1)}{w_i - w_j},\\ 
        &= n_2 - (n_1 + n_2)w_i + \sum_{j \neq i}\frac{ (w_i + w_i -1)(w_i - w_j) - 2 w_i(w_i-1) }{w_i - w_j},\\ 
        &= n_2 - (n_1 + n_2)w_i + \sum_{j \neq i}\frac{ w_i^2 + w_i(w_i-1) - w_iw_j -w_j(w_i-1) -2w_i(w_i-1) }{w_i - w_j},\\
        &=  n_2 - (n_1 + n_2)w_i + \sum_{j \neq i}\frac{ - w_iw_j - w_j(w_i-1) +w_i }{w_i - w_j},\\
        &= n_2 - (n_1 + n_2)w_i + \sum_{j \neq i}\frac{w_j(1-w_i) +w_i(1-w_j)}{w_i - w_j}.
    \end{align*}

\end{proof}
    %Notice that by definition, $p(x,y,z)$ is a homogeneous polynomial, and this allows us to 
    
    % Poner casos estáticos junto al Hermite
    % Determinar dinámica de las raíces
    % Escribir que la distancia más pequeña crece (por lo menos en el caso Hermite)

